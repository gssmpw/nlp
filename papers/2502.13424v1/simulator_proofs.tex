
%\section{Missing proofs from Section~\ref{sec:main-simulation}}
%\section{Proofs from Section~\ref{sec:main-simulation} -- Analysis of Main Algorithm}
\section{Details of Section~\ref{sec:main-simulation} -- Analysis of the \alg Algorithm}
\label{sec:proofs-main-simulation}



\begin{figure}[thbp]
\centering
\begin{subfigure}[htbp]{0.30\textwidth}
\centering
\vspace*{-10ex}
\includegraphics[width=\linewidth]{beep001.jpeg}
\caption{Some part of a beeping network.}
\label{subfig:bn}
\end{subfigure}
\hspace{0.1in}
\begin{subfigure}[htbp]{0.30\textwidth}
\centering
\includegraphics[width=\linewidth]{beep002.jpeg}
\caption{Announcing super-round of some phase $j$: $\{c,g\}$ announce, $\{a,b,d,e,f\}$ hear noise, but only $\{b,d,e,f\}$ receive an extended-ID $\langle c\rangle$ and become $c$-$responsive$.}
\label{subfig:announce}
\end{subfigure}
\hspace{0.1in}
\begin{subfigure}[htbp]{0.30\textwidth}
\centering
\vspace*{-3ex}
\includegraphics[width=\linewidth]{beep004.jpeg}
\caption{Responding $3$ super-rounds within some sub-phase $a'$: $\{b,e\}$ respond and $c$ receives $\langle b\rangle\langle c\rangle\langle m_{b,c}\rangle$ and $\langle e\rangle\langle c\rangle\langle m_{e,c}\rangle$.}
\label{subfig:resp1}
\end{subfigure}
\\
\begin{subfigure}[htbp]{0.30\textwidth}
\centering
\vspace*{3ex}
\includegraphics[width=\linewidth]{beep005.jpeg}
\caption{Confirming $3$ super-rounds during sub-phase $a'$: $c$ confirms, $\{b,e\}$ receive $\langle c\rangle\langle b\rangle\langle m_{c,b}\rangle$ and $\langle c\rangle\langle e\rangle\langle m_{c,e}\rangle$ respectively. After this $\{b,e\}$ abandon the $c$-$responsive$ status and $\{\{b,c\},\{e,c\}\}$ are marked as realized.}
\label{subfig:conf1}
\end{subfigure}
\hspace{0.1in}
\begin{subfigure}[htbp]{0.30\textwidth}
\centering
\vspace*{-8ex}
\includegraphics[width=\linewidth]{beep006.jpeg}
\caption{Responding $3$ super-rounds within some sub-phase $a''$: $d$ responds and $c$ receives $\langle d\rangle\langle c\rangle\langle m_{d,c}\rangle$.}
\label{subfig:resp2}
\end{subfigure}
\hspace{0.1in}
\begin{subfigure}[htbp]{0.30\textwidth}
\centering
\vspace*{-3ex}
\includegraphics[width=\linewidth]{beep007.jpeg}
\caption{Confirming $3$ super-rounds during sub-phase $a''$: $c$ confirms, $d$ receives $\langle c\rangle\langle d\rangle\langle m_{c,d}\rangle$. After this $d$ abandons the $c$-$responsive$ status and $\{d,c\}$ is marked as realized.}
\label{subfig:conf2}
\end{subfigure}
\\
\begin{subfigure}[htbp]{0.30\textwidth}
\centering
\includegraphics[width=\linewidth]{beep008.jpeg}
\caption{Responding $3$ super-rounds within some sub-phase $a'''$: $f$ responds and $c$ receives $\langle f\rangle\langle c\rangle\langle m_{f,c}\rangle$.}
\label{subfig:resp3}
\end{subfigure}
\hspace{0.1in}
\begin{subfigure}[htbp]{0.30\textwidth}
\centering
\vspace*{5ex}
\includegraphics[width=\linewidth]{beep009.jpeg}
\caption{Confirming $3$ super-rounds during sub-phase $a'''$: $c$ confirms, $f$ receives $\langle c\rangle\langle f\rangle\langle m_{c,f}\rangle$. After this $f$ abandons the $c$-$responsive$ status and $\{f,c\}$ is marked as realized.}
\label{subfig:conf3}
\end{subfigure}
\hspace{0.1in}
\begin{subfigure}[htbp]{0.30\textwidth}
\centering
\includegraphics[width=\linewidth]{beep010.jpeg}
\caption{By the end of phase $j$ links $\{\{b,c\},\{d,c\},\{e,c\},\{f,c\}\}$ have been realized.}
\label{subfig:end}
\end{subfigure}
\caption{Illustration of \alg algorithm -- consecutive handshakes between the announcer $c$ and its responders during a phase.}
\label{fig:alg}
\end{figure}


\begin{proof}[Proof of Lemma~\ref{lem:correct-receiving}]
The proof is by contradiction -- suppose that in some super-round a node $w$ 
%stays silent and 
receives an extended-ID $z$ but the claim of the lemma does not hold. 
Without lost of generality, we may assume that this is the first such super-round.

Recall that the definition of receiving an extended-ID requires that node $w$ has been silent in this super-round. Note that if exactly one neighbor of node $w$ has been beeping during the super-round, it must have been an extended-ID of some node (by specification of the algorithm), and therefore node $w$ receives this extended-ID (as other neighbors do not beep at all).
Similarly, we argue that at least one neighbor of node $w$ must have been beeping (some extended-ID) in the super-round, as otherwise node $w$ would not have received any beep (and so, also no extended-ID) in the considered super-round.

In the remainder we focus on the complementary case that at least two neighbors of node $w$ have been beeping in the super-round, each of them some extended-ID (again, by specification of the algorithm, a node beeps only some extended-ID or stay silent during any super-round).

First, suppose that some two neighbors, $v_1,v_2$, beeped different extended-IDs, say $z_1\ne z_2$, respectively.
It means that node $w$ received more than $\log n$ beeps during the super-round: $\log n$ beeps coming from one of the extended-IDs and at least one more because the extended IDs of different nodes differ by at least one position. Hence, the received sequence of beeps does not form any extended-ID, as it must always have $\log n$ bits $1$ corresponding to the beeps. This contradicts the fact that $w$ receives an extended-ID in the considered super-round.

Second, suppose that all extended-IDs beeped by (at least two) neighbors on node $w$ are the same. If this happens in an announcing, or a first responding, or a first confirming super-round, it is a contradiction because all nodes that beep in such super-rounds beep their own extended-IDs, which are pairwise different.
%
If this happens in a second confirming super-round, it means that these two neighbors $v_1,v_2$ belong to the same set $\mF_{\Delta,k_i}(j)$, for some phase number $j$, and both of them received an extended-ID of $z$ in the preceding responding super-round. By the fact that the considered super-round is the first when the lemma's claim does not hold, we get that in this preceding responding super-round, both $v_1,v_2$ received the extended-ID of $z$ when $z$ was their unique beeping neighbor (beeping its own extended-ID, by the specification of the responding super-rounds).
This, however, implies that in the beginning of the current phase $j$, i.e., during its announcing super-round, both $v_1,v_2$ beeped their extended-IDs and, again by the choice of the current contradictory super-round, their neighbor $w$ could not have received any extended-ID -- this is a contradiction with the fact that $z$ was transmitting in the responding super-round preceding the considered (contradictory) super-round. More precisely, only $(j,\cdot)$-responsive nodes can transmit in responding super-rounds, but $z$ is not $(j,\cdot)$-responsive because it had not received any extended-IDs in the first (announcing) super-round of the current phase.

The last sub-case of the above scenario, when all extended-IDs beeped by (at least two) neighbors on node $w$ are the same, is as follows. 
If this situation happens in a second responding super-round, it means 
that these two neighbors $v_1,v_2$ are both $(j,z)$-responsive and beep extended-ID of $z$. This is, however, acceptable due to the exception in the lemma's statement.

This completes the proof of the lemma.
\end{proof}


\begin{proof}[Proof of Lemma~\ref{lem:correct-realization}]
It is enough to show that points (a) and (b) of the definition of link realization occurred in the last four super-rounds (two responding and two confirming) and also that the other node, $w$, (locally) marks link $\{v,w\}$ as realized at the same time when $v$ does.

If node $v$ marked the link $\{v,w\}$ as realized, it could be because of one of two reasons. 

First, it is in the set $\mF_{\Delta,k_i}(j)$, where $i$ is the number of the current epoch and $j$ is the number of the current phase and received an extended-ID of $w$ followed by its own extended-ID in the preceding two responding super-rounds. This satisfies point (a) of the definition of link realization, as both were beeped by node $w$, by the algorithm specification, and by Lemma~\ref{lem:correct-receiving}. Note that the exception in that lemma does not really apply here because if there were two or more neighbors beeping the same extended-ID (of $v$) in the second responding round, they would also be beeping their own extended-IDs in the first responding round, which could contradict the fact that $v$ received a single extended-ID at that super round.

This also means that $v$ has beeped its own extended-ID followed by extended-ID of $w$ in the last two confirming super-rounds, which must have been received by $w$ because $w$ is $(j,v)$-responsive (because only such nodes could have beeped in the preceding responding super-rounds) and thus its only neighbor in set $\mF_{\Delta,k_i}(j)$ (only such nodes are allowed to beep in confirming super-rounds) is $v$; here we use Fact~\ref{fact:single-beeping}. Hence, $w$ also marks link 
$\{v,w\}$ as realized at the end of the two confirming super-rounds; by the algorithm's specification, point (c) of the definition also holds in this case.

Second, it is $(j,z)$-responsive and received an extended-ID of $z$ followed by its own extended-ID in the current two confirming super-rounds. This satisfies point (b) of the definition, as both were beeped by node $z$, by the specification of the algorithm and Lemma~\ref{lem:correct-receiving} (exception in that lemma does not apply here because we now consider only confirming super-rounds).

This also means that $v$ beeped its own extended-ID followed by extended-ID of $z$ in the preceding two responding super-rounds (because $v$ is $(j,z)$-responsive and only such nodes could beep in the preceding responding super-rounds), which must have been received by $z$ (otherwise, by the specification of the algorithm, $z$ would not beep its extended-ID followed by the extended-ID of $v$ in the last two confirming super-rounds). Hence, $z$ also marks link $\{v,z\}$ as realized at the end of the two confirming super-rounds by the algorithm's specification, and point (c) of the definition also holds in this case.
\end{proof}



\begin{proof}[Proof of Lemma~\ref{lem:subphase-progress}]
The lemma follows from the definition of the $(n,k_i/2^{a-2},k_i/2^{a-1})$-avoiding selector $\mF_{k_i/2^{a-2},k_i/2^{a-1}}$ used throughout sub-phase $a$ of phase $j$ of epoch $i$. 
By specification of the sub-phase, only nodes $w$ such that $w$ is $(j,v)$-responsive and it does not marked link $\{v,w\}$ as realized take active part in sub-phase $a$ (in the sense that only those nodes can beep extended-IDs of itself followed by $v$ in pairs of responding super-rounds), while other neighbors of $v$ do not beep at all. The latter statement needs more justification -- in the beginning of the current phase, in the announcing super-round, $v$ must have beeped because some nodes have become $(j,v)$-responsive in this phase (w.l.o.g. we may assume that at least one node has become $(j,v)$-responsive, because otherwise the lemma trivially holds), therefore, by Lemma~\ref{lem:correct-receiving}, other neighbors of $v$ could not receive another announcement and become $(j,v')$-responsive, for some $v'\ne v$, and thus by the description of the algorithm -- they stay silent throughout the whole phase. 

By lemma assumption, there are at most $\Delta/2^{i+a-2}=k_i/2^{a-2}$ $(j,v)$-responsive nodes $w$ that have not marked link $\{v,w\}$ as realized by the beginning of the sub-phase. Hence, at least half of them will be in a singleton intersection with some set $\mF_{k_i/2^{a-2},k_i/2^{a-1}}(b)$, by Definition~\ref{def:avoid-selector} and Fact~\ref{fact:avoiding-selectors}, in which case $v$ receives their beeping in the corresponding pair of the responding super-rounds. Consequently, $v$ beeps back its own extended-ID and the extended-ID of $w$ in the following two confirming super-rounds. 

Node $w$ receives those beepings, as there is no other neighbor of $w$ who is allowed to beep in these two rounds -- indeed if there was, it would belong to set $\mF_{\Delta,k_i}(j)$ and thus it would have been beeping in the announcing super-round of this phase, preventing (together with neighbor $v$ of $w$) node $w$ from receiving anything in that super-round (by Lemma~\ref{lem:correct-receiving}), which contradicts the fact that $w$ must have received an extended-ID of $v$ in that round to become $(j,v)$-responsive (as assumed). Therefore, by the description of the algorithm, $w$ marks link $\{v,w\}$ as realized. This completes the proof that the number of $(j,v)$-responsive neighbors $w$ of $v$ who remain without realizing link $\{v,w\}$ becomes less than $\Delta/2^{i+a-1}$ at the end of the considered sub-phase. 
\end{proof}


\begin{proof}[Proof of Lemma~\ref{lem:phase-progress}]
It follows directly from the fact that a phase, after its announcing super-round, iterates sub-phases $a=1,\ldots,\log k_i$. Each subsequent sub-phase halves the number of not-realized links $\{v,w\}$, for $(j,v)$-responsive nodes $w$ and each announcing node $v$, c.f., Lemma~\ref{lem:subphase-progress}, starting from the assumed $2k_i$ maximum number of $(j,v)$-responding nodes (recall that $(j,v)$-responding nodes form a subset of those to whom links are not realized, hence there are at most $2k_i$ of them in the beginning).
\end{proof}


\begin{proof}[Proof of Lemma~\ref{lem:epoch-invariant}]
The proof is by induction on epoch number $i$.
Obviously, the invariant holds at the beginning of the first epoch, i.e., $\kappa_1\le k_1$, where $\kappa_i$ was defined as the sharp upper bound on the maximum number of not realized links at a node at the end of epoch $i$ and $k_i$ is the parameter used in the algorithm for epoch $i$. 

Consider epoch $i\ge 1$.
We have to prove that:
assuming that $\kappa_{i'}\le k_{i'}$, for any $1\le i' < i$, we also have $\kappa_i \le k_i$.
Technically we can assume that $\kappa_0=k_0=\Delta$.

Consider a node $w$. 
By the inductive assumption, it has at most $k_{i-1}$ neighbors $v$ such that link $\{v,w\}$ has not been marked by $w$ as realized.
By Definition~\ref{def:avoid-selector} and Fact~\ref{fact:avoiding-selectors} applied to $(n,\Delta,\Delta-k_i)$-avoiding selector $\mF_{\Delta,k_i}$, 
which sets are used for announcing super-rounds (and later for confirming super-rounds), the number of neighbors $v$ of node $w$ from whom node $w$ has not received their extended-ID during the announcing super-round is smaller than $k_i$. By Lemma~\ref{lem:phase-progress}, all such nodes $w$ realize their links, and by Lemma~\ref{lem:correct-realization}, also node $v$ realizes these links during the considered phase. Hence, the number of non-realized links incident to any node $w$ drops below $k_i$ by the end of epoch $k_i$.
\end{proof}


