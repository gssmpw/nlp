\section{More Examples of High Norm Tokens}\label{sec:more_high_norm}

\cref{fig:more_llm_norm} shows more examples of high-norm tokens in the LLM series.
Specifically, it displays a plethora of model variations in the LLaMA2, LLaMA3, LLaMA3.1, LLaMA3.2, Phi3, and Qwen2 series.
Without exception, they all exhibit the high-norm phenomenon, especially in the initial token.
The patterns of the high-norm look similar if one model is fine-tuned from another, which echos our findings about the stability of high-norm tokens in \cref{sec:signature}.

\section{More Examples of Token Norms in Mistral and BERT}\label{sec:more_bert}

\cref{fig:more_bert} shows the norms of the first few tokens and the token `\texttt{.}' in the sentence `\texttt{The quick brown fox jumps over the lazy dog.}' for more models in the Mistral, BERT, RoBERTa, and DistilBERT families.

\begin{figure}[t]
    \centering
    \begin{subfigure}[t]{0.24\linewidth}
        \includegraphics[width=\textwidth]{figures/paper/mistral_7b_norm_3d.pdf}
        \caption{Mistral-7B-v0.1}\label{fig:mistral_7b_norm_3d}
    \end{subfigure}
    \begin{subfigure}[t]{0.24\linewidth}
        \includegraphics[width=\textwidth]{figures/paper/mistral_7b_instruct_norm_3d.pdf}
        \caption{Mistral-7B-Instruct-v0.1}\label{fig:mistral_7b_instruct_norm_3d}
    \end{subfigure}
    \begin{subfigure}[t]{0.24\linewidth}
        \includegraphics[width=\textwidth]{figures/paper/mistral2_7b_instruct_norm_3d.pdf}
        \caption{Mistral-7B-Instruct-v0.2}\label{fig:mistral2_7b_instruct_norm_3d}
    \end{subfigure}
    \begin{subfigure}[t]{0.24\linewidth}
        \includegraphics[width=\textwidth]{figures/paper/mistral3_7b_instruct_norm_3d.pdf}
        \caption{Mistral-7B-Instruct-v0.3}\label{fig:mistral3_7b_instruct_norm_3d}
    \end{subfigure}\\
    \begin{subfigure}[t]{0.24\linewidth}
        \includegraphics[width=\textwidth]{figures/paper/bert_base_cased_norm_3d.pdf}
        \caption{BERT-Base-Cased}\label{fig:bert_base_cased_norm_3d}
    \end{subfigure}
    \begin{subfigure}[t]{0.24\linewidth}
        \includegraphics[width=\textwidth]{figures/paper/bert_base_uncased_norm_3d.pdf}
        \caption{BERT-Base-Uncased}\label{fig:bert_base_uncased_norm_3d}
    \end{subfigure}
    \begin{subfigure}[t]{0.24\linewidth}
        \includegraphics[width=\textwidth]{figures/paper/bert_base_chinese_norm_3d.pdf}
        \caption{BERT-Base-Chinese}\label{fig:bert_base_chinese_norm_3d}
    \end{subfigure}
    \begin{subfigure}[t]{0.24\linewidth}
        \includegraphics[width=\textwidth]{figures/paper/bert_large_uncased_norm_3d.pdf}
        \caption{BERT-Large-Uncased}\label{fig:bert_large_uncased_norm_3d}
    \end{subfigure}\\
    \begin{subfigure}[t]{0.24\linewidth}
        \includegraphics[width=\textwidth]{figures/paper/roberta_base_norm_3d.pdf}
        \caption{RoBERTa-Base}\label{fig:roberta_base_norm_3d}
    \end{subfigure}
    \begin{subfigure}[t]{0.24\linewidth}
        \includegraphics[width=\textwidth]{figures/paper/roberta_large_norm_3d.pdf}
        \caption{RoBERTa-Large}\label{fig:roberta_large_norm_3d}
    \end{subfigure}
    \begin{subfigure}[t]{0.24\linewidth}
        \includegraphics[width=\textwidth]{figures/paper/distilbert_base_cased_norm_3d.pdf}
        \caption{DistilBERT-Base-Cased}\label{fig:distilbert_base_cased_norm_3d}
    \end{subfigure}
    \begin{subfigure}[t]{0.24\linewidth}
        \includegraphics[width=\textwidth]{figures/paper/distilbert_base_uncased_norm_3d.pdf}
        \caption{DistilBERT-Base-Uncased}\label{fig:distilbert_base_uncased_norm_3d}
    \end{subfigure}
    \caption{(Continuation of \cref{fig:factors}).
    The example token norms in Mistral, BERT, RoBERTa and DistilBERT models.
    }\label{fig:more_bert}
\end{figure}


\begin{figure*}[!t]
    \centering
    \begin{subfigure}[t]{0.23\textwidth}
        \includegraphics[width=\textwidth]{figures/paper/llama2_7b_chat_norm_3d.pdf}
        \caption{LLama2-7B-Chat}\label{fig:llama2_7b_chat_norm_3d}
    \end{subfigure}
    \begin{subfigure}[t]{0.23\textwidth}
        \includegraphics[width=\textwidth]{figures/paper/llama2_7b_code_norm_3d.pdf}
        \caption{LLama2-7B-Code}\label{fig:llama2_7b_code_norm_3d}
    \end{subfigure}
    \begin{subfigure}[t]{0.23\textwidth}
        \includegraphics[width=\textwidth]{figures/paper/llama2_7b_code_python_norm_3d.pdf}
        \caption{LLama2-7B-Code-Python}\label{fig:llama2_7b_code_python_norm_3d}
    \end{subfigure}
    \begin{subfigure}[t]{0.23\textwidth}
        \includegraphics[width=\textwidth]{figures/paper/llama2_7b_code_instruct_norm_3d.pdf}
        \caption{LLama2-7B-Code-Instruct}\label{fig:llama2_7b_code_instruct_norm_3d}
    \end{subfigure}\\
    \begin{subfigure}[t]{0.23\textwidth}
        \includegraphics[width=\textwidth]{figures/paper/llama2_13b_norm_3d.pdf}
        \caption{LLama2-13B}\label{fig:llama2_13b_norm_3d}
    \end{subfigure}
    \begin{subfigure}[t]{0.23\textwidth}
        \includegraphics[width=\textwidth]{figures/paper/llama2_70b_norm_3d.pdf}
        \caption{LLama2-70B}\label{fig:llama2_70b_norm_3d}
    \end{subfigure}
    \begin{subfigure}[t]{0.23\textwidth}
        \includegraphics[width=\textwidth]{figures/paper/llama3_8b_norm_3d.pdf}
        \caption{LLama3-8B}\label{fig:llama3_8b_norm_3d}
    \end{subfigure}
    \begin{subfigure}[t]{0.23\textwidth}
        \includegraphics[width=\textwidth]{figures/paper/llama3_8b_guard_norm_3d.pdf}
        \caption{LLama3-8B-Guard}\label{fig:llama3_8b_guard_norm_3d}
    \end{subfigure}\\
    \begin{subfigure}[t]{0.23\textwidth}
        \includegraphics[width=\textwidth]{figures/paper/llama3_8b_instruct_norm_3d.pdf}
        \caption{LLama3-8B-Instruct}\label{fig:llama3_8b_instruct_norm_3d}
    \end{subfigure}
    \begin{subfigure}[t]{0.23\textwidth}
        \includegraphics[width=\textwidth]{figures/paper/llama3_8b_norm_3d.pdf}
        \caption{LLama3.1-8B}\label{fig:llama31_8b_norm_3d}
    \end{subfigure}
    \begin{subfigure}[t]{0.23\textwidth}
        \includegraphics[width=\textwidth]{figures/paper/llama31_8b_guard_norm_3d.pdf}
        \caption{LLama3.1-8B-Guard}\label{fig:llama31_8b_guard_norm_3d}
    \end{subfigure}
    \begin{subfigure}[t]{0.23\textwidth}
        \includegraphics[width=\textwidth]{figures/paper/llama31_8b_instruct_norm_3d.pdf}
        \caption{LLama3.1-8B-Instruct}\label{fig:llama31_8b_instruct_norm_3d}
    \end{subfigure}\\
    \begin{subfigure}[t]{0.23\textwidth}
        \includegraphics[width=\textwidth]{figures/paper/llama32_1b_norm_3d.pdf}
        \caption{LLama3.2-1B}\label{fig:llama32_1b_norm_3d}
    \end{subfigure}
    \begin{subfigure}[t]{0.23\textwidth}
        \includegraphics[width=\textwidth]{figures/paper/llama32_1b_instruct_norm_3d.pdf}
        \caption{LLama3.2-1B-Instruct}\label{fig:llama32_1b_instruct_norm_3d}
    \end{subfigure}
    \begin{subfigure}[t]{0.23\textwidth}
        \includegraphics[width=\textwidth]{figures/paper/llama32_3b_norm_3d.pdf}
        \caption{LLama3.2-3B}\label{fig:llama32_3b_norm_3d}
    \end{subfigure}
    \begin{subfigure}[t]{0.23\textwidth}
        \includegraphics[width=\textwidth]{figures/paper/llama32_3b_instruct_norm_3d.pdf}
        \caption{LLama3.2-3B-Instruct}\label{fig:llama32_3b_instruct_norm_3d}
    \end{subfigure}\\
    \begin{subfigure}[t]{0.23\textwidth}
        \includegraphics[width=\textwidth]{figures/paper/phi3_mini_norm_3d.pdf}
        \caption{Phi3-Mini}\label{fig:phi3_mini_norm_3d}
    \end{subfigure}
    \begin{subfigure}[t]{0.23\textwidth}
        \includegraphics[width=\textwidth]{figures/paper/phi3_mini_128k_norm_3d.pdf}
        \caption{Phi3-Mini-128K}\label{fig:phi3_mini_128k_norm_3d}
    \end{subfigure}
    \begin{subfigure}[t]{0.23\textwidth}
        \includegraphics[width=\textwidth]{figures/paper/phi3_medium_128k_norm_3d.pdf}
        \caption{Phi3-Medium-128K}\label{fig:phi3_medium_128k_norm_3d}
    \end{subfigure}
    \begin{subfigure}[t]{0.23\textwidth}
        \includegraphics[width=\textwidth]{figures/paper/phi35_mini_norm_3d.pdf}
        \caption{Phi3.5-Mini}\label{fig:phi35_mini_norm_3d}
    \end{subfigure}\\
    \begin{subfigure}[t]{0.23\textwidth}
        \includegraphics[width=\textwidth]{figures/paper/qwen2_7b_norm_3d.pdf}
        \caption{Qwen2-7B}\label{fig:qwen2_7b_norm_3d}
    \end{subfigure}
    \begin{subfigure}[t]{0.23\textwidth}
        \includegraphics[width=\textwidth]{figures/paper/qwen2_7b_instruct_norm_3d.pdf}
        \caption{Qwen2-7B-Instruct}\label{fig:qwen2_7b_instruct_norm_3d}
    \end{subfigure}
    \begin{subfigure}[t]{0.23\textwidth}
        \includegraphics[width=\textwidth]{figures/paper/qwen2_7b_math_norm_3d.pdf}
        \caption{Qwen2-7B-Math}\label{fig:qwen2_7b_math_norm_3d}
    \end{subfigure}
    \begin{subfigure}[t]{0.23\textwidth}
        \includegraphics[width=\textwidth]{figures/paper/qwen2_7b_math_instruct_norm_3d.pdf}
        \caption{Qwen2-7B-Math-Instruct}\label{fig:qwen2_7b_math_instruct_norm_3d}
    \end{subfigure}
    \\
    \caption{(Continuation of \cref{fig:llm_norm}). High norm tokens in various LLM series.
        % \(x\)-axis is the layer id, \(y\)-axis are tokens, and \(z\)-axis is the norm.
        Each subfigure plots the norm of the first few tokens and the token `\texttt{.}' in the sentence `\texttt{The quick brown fox jumps over the lazy dog.}'.
        Here, the token `\texttt{The}' is the initial token, and it has a high norm in all models.
        The \(x\)-axis is the layer id, the \(y\)-axis shows different tokens, and the \(z\)-axis is the norm.
        Layer 0 is the input embedding layer, and the others are transformer layers.
        % The last row shows the high-norm defective tokens in DINOv2 and the absence of high-norm tokens in BERT-Large and RoBERTa-Large.
    }\label{fig:more_llm_norm}
\end{figure*}

\section{More High-Norm Direction Statistics}\label{sec:high_norm_statistics}

\cref{tab:more_high_norm_statistics} shows the average pairwise angles between all high-norm tokens for more LLMs.
The thresholds are determined by inspecting the visualization of the token norms as in \cref{fig:llm_norm}.
The results confirm that all the high-norm tokens in one model have very similar directions across all layers and all tokens.

\begin{table}[t]
    \caption{Average pairwise angles between all high-norm tokens for each LLM\@.
        High norm tokens are collected across all layers, and all tokens from 1000 rows in the WikiText2-v1 dataset.
    }\label{tab:more_high_norm_statistics}
    \begin{center}
        \begin{small}
            \begin{sc}
                \begin{tabular}{lccccr}
                    \toprule
                    Model        & Threshold & Mean Pairwise Angle (degree) \\
                    \midrule
                    LLaMA2-7B    & 500       & 3.12                         \\
                    Phi3-Medium  & 2500      & 5.31                         \\
                    MPT-7B       & 1500      & 2.81                         \\
                    Pythia-160M  & 200       & 8.01                         \\
                    Vicuna1.5-7B & 400       & 4.15                         \\
                    Falcon2-11B  & 3000      & 4.11                         \\
                    GPT2-Medium  & 3000      & 1.49                         \\
                    Qwen2.5-1.5B & 8000      & 1.00                         \\
                    \bottomrule
                \end{tabular}
            \end{sc}
        \end{small}
    \end{center}
\end{table}

\section{More Examples of Explosion Direction Prediction}\label{sec:more_explosion}

\cref{fig:more_llm_angle} shows the angles between the predicted layer-wise singular defect directions and the empirical high-norm direction for more models.
Notice that in some models, such as Phi3-Medium, there are multiple explosion layers and decrease layers, and the angles between the predicted directions and the empirical high-norm directions are close in those layers.

\begin{figure*}[!t]
    \centering
    \begin{subfigure}[t]{0.49\textwidth}
        \includegraphics[width=\textwidth]{figures/paper/llama2_7b_chat_angle.pdf}
        \caption{LLaMA2-7B-Chat}\label{fig:llama2_7b_chat_angle}
    \end{subfigure}
    \begin{subfigure}[t]{0.49\textwidth}
        \includegraphics[width=\textwidth]{figures/paper/llama2_7b_code_angle.pdf}
        \caption{LLaMA2-7B-Code}\label{fig:llama2_7b_code_angle}
    \end{subfigure}\\
    \begin{subfigure}[t]{0.49\textwidth}
        \includegraphics[width=\textwidth]{figures/paper/llama2_13b_angle.pdf}
        \caption{LLaMA2-13B}\label{fig:llama2_13b_angle}
    \end{subfigure}
    \begin{subfigure}[t]{0.49\textwidth}
        \includegraphics[width=\textwidth]{figures/paper/llama2_13b_chat_angle.pdf}
        \caption{LLaMA2-13B-Chat}\label{fig:llama2_13b_chat_angle}
    \end{subfigure}\\
    \begin{subfigure}[t]{0.49\textwidth}
        \includegraphics[width=\textwidth]{figures/paper/llama3_8b_angle.pdf}
        \caption{LLaMA3-8B}\label{fig:llama3_8b_angle}
    \end{subfigure}
    \begin{subfigure}[t]{0.49\textwidth}
        \includegraphics[width=\textwidth]{figures/paper/llama3_8b_instruct_angle.pdf}
        \caption{LLaMA3-8B-Instruct}\label{fig:llama3_8b_instruct_angle}
    \end{subfigure}\\
    \begin{subfigure}[t]{0.49\textwidth}
        \includegraphics[width=\textwidth]{figures/paper/phi3_mini_angle.pdf}
        \caption{Phi3-Mini}\label{fig:phi3_mini_angle}
    \end{subfigure}
    \begin{subfigure}[t]{0.49\textwidth}
        \includegraphics[width=\textwidth]{figures/paper/phi3.5_mini_angle.pdf}
        \caption{Phi3.5-Mini}\label{fig:phi3.5_mini_angle}
    \end{subfigure}\\
    \begin{subfigure}[t]{0.49\textwidth}
        \includegraphics[width=\textwidth]{figures/paper/phi3_mini_128k_angle.pdf}
        \caption{Phi3-Mini-128k}\label{fig:phi3_mini_128k_angle}
    \end{subfigure}
    \begin{subfigure}[t]{0.49\textwidth}
        \includegraphics[width=\textwidth]{figures/paper/phi3_medium_angle.pdf}
        \caption{Phi3-Medium}\label{fig:phi3_medium_angle}
    \end{subfigure}\\
    \begin{subfigure}[t]{0.49\textwidth}
        \includegraphics[width=\textwidth]{figures/paper/qwen2_7b_angle.pdf}
        \caption{Qwen2-7B}\label{fig:qwen2_7b_angle}
    \end{subfigure}
    \begin{subfigure}[t]{0.49\textwidth}
        \includegraphics[width=\textwidth]{figures/paper/qwen2_7b_instruct_angle.pdf}
        \caption{Qwen2-7B-Instruct}\label{fig:qwen2_7b_instruct_angle}
    \end{subfigure}
    \caption{(Continuation of \cref{fig:llama2_angle}).
    Acute angles between layer-wise singular defect directions and empirical high-norm direction for more models.
    We can see that small angles align with either the explosion or the decay layer.
    }\label{fig:more_llm_angle}
\end{figure*}

\section{More Examples of Eigenvalues in Decay Layer}\label{sec:more_diminish}

\cref{fig:more_llm_eig} shows the minimum angle between the eigenvectors of the linear approximation of the layer residual and the empirical high-norm direction for more models.
Their corresponding eigenvalues are also plotted.


\begin{figure*}[!t]
    \centering
    \begin{subfigure}[t]{0.49\textwidth}
        \includegraphics[width=\textwidth]{figures/paper/llama2_7b_chat_eig.pdf}
        \caption{LLaMA2-7B-Chat}\label{fig:llama2_7b_chat_eig}
    \end{subfigure}
    \begin{subfigure}[t]{0.49\textwidth}
        \includegraphics[width=\textwidth]{figures/paper/llama2_7b_code_eig.pdf}
        \caption{LLaMA2-7B-Code}\label{fig:llama2_7b_code_eig}
    \end{subfigure}\\
    \begin{subfigure}[t]{0.49\textwidth}
        \includegraphics[width=\textwidth]{figures/paper/llama2_13b_eig.pdf}
        \caption{LLaMA2-13B}\label{fig:llama2_13b_eig}
    \end{subfigure}
    \begin{subfigure}[t]{0.49\textwidth}
        \includegraphics[width=\textwidth]{figures/paper/llama2_13b_chat_eig.pdf}
        \caption{LLaMA2-13B-Chat}\label{fig:llama2_13b_chat_eig}
    \end{subfigure}\\
    \begin{subfigure}[t]{0.49\textwidth}
        \includegraphics[width=\textwidth]{figures/paper/llama3_8b_eig.pdf}
        \caption{LLaMA3-8B}\label{fig:llama3_8b_eig}
    \end{subfigure}
    \begin{subfigure}[t]{0.49\textwidth}
        \includegraphics[width=\textwidth]{figures/paper/llama3_8b_instruct_eig.pdf}
        \caption{LLaMA3-8B-Instruct}\label{fig:llama3_8b_instruct_eig}
    \end{subfigure}\\
    \begin{subfigure}[t]{0.49\textwidth}
        \includegraphics[width=\textwidth]{figures/paper/phi3_mini_eig.pdf}
        \caption{Phi3-Mini}\label{fig:phi3_mini_eig}
    \end{subfigure}
    \begin{subfigure}[t]{0.49\textwidth}
        \includegraphics[width=\textwidth]{figures/paper/phi3.5_mini_eig.pdf}
        \caption{Phi3.5-Mini}\label{fig:phi3.5_mini_eig}
    \end{subfigure}\\
    \begin{subfigure}[t]{0.49\textwidth}
        \includegraphics[width=\textwidth]{figures/paper/phi3_mini_128k_eig.pdf}
        \caption{Phi3-Mini-128k}\label{fig:phi3_mini_128k_eig}
    \end{subfigure}
    \begin{subfigure}[t]{0.49\textwidth}
        \includegraphics[width=\textwidth]{figures/paper/phi3_medium_eig.pdf}
        \caption{Phi3-Medium}\label{fig:phi3_medium_eig}
    \end{subfigure}\\
    \begin{subfigure}[t]{0.49\textwidth}
        \includegraphics[width=\textwidth]{figures/paper/qwen2_7b_eig.pdf}
        \caption{Qwen2-7B}\label{fig:qwen2_7b_eig}
    \end{subfigure}
    \begin{subfigure}[t]{0.49\textwidth}
        \includegraphics[width=\textwidth]{figures/paper/qwen2_7b_instruct_eig.pdf}
        \caption{Qwen2-7B-Instruct}\label{fig:qwen2_7b_instruct_eig}
    \end{subfigure}
    \caption{(Continuation of \cref{fig:llama2_7b_negeig}).
    For each layer, the minimum angles between the eigenvectors of \(R\) and the empirical high-norm direction are shown in \textcolor{red}{red}, and the corresponding eigenvalues are shown in \textcolor{blue}{blue}.
    Numbers for the explosion and decay layers are annotated.
    }\label{fig:more_llm_eig}
\end{figure*}

\section{More Examples of Exploding Path}\label{sec:more_development}

\cref{fig:more_llm_1st_type} shows the norm of the initial token after the explosion layer.
All trained tokens are plotted.
Besides, a number of random input embeddings are also used as initial tokens, and their norms after the explosion layer are plotted.


\begin{figure*}[!t]
    \centering
    \begin{subfigure}[t]{0.47\textwidth}
        \includegraphics[width=\textwidth]{figures/paper/llama2_7b_chat_1st_type.png}
        \caption{LLaMA2-7B-Chat}\label{fig:llama2_7b_chat_1st_type}
    \end{subfigure}
    \begin{subfigure}[t]{0.47\textwidth}
        \includegraphics[width=\textwidth]{figures/paper/llama2_7b_code_1st_type.png}
        \caption{LLaMA2-7B-Code}\label{fig:llama2_7b_code_1st_type}
    \end{subfigure}\\
    \begin{subfigure}[t]{0.47\textwidth}
        \includegraphics[width=\textwidth]{figures/paper/llama2_13b_1st_type.png}
        \caption{LLaMA2-13B}\label{fig:llama2_13b_1st_type}
    \end{subfigure}
    \begin{subfigure}[t]{0.47\textwidth}
        \includegraphics[width=\textwidth]{figures/paper/llama2_13b_chat_1st_type.png}
        \caption{LLaMA2-13B-Chat}\label{fig:llama2_13b_chat_1st_type}
    \end{subfigure}\\
    \begin{subfigure}[t]{0.47\textwidth}
        \includegraphics[width=\textwidth]{figures/paper/llama3_8b_1st_type.png}
        \caption{LLaMA3-8B}\label{fig:llama3_8b_1st_type}
    \end{subfigure}
    \begin{subfigure}[t]{0.47\textwidth}
        \includegraphics[width=\textwidth]{figures/paper/llama3_8b_instruct_1st_type.png}
        \caption{LLaMA3-8B-Instruct}\label{fig:llama3_8b_instruct_1st_type}
    \end{subfigure}\\
    \begin{subfigure}[t]{0.47\textwidth}
        \includegraphics[width=\textwidth]{figures/paper/phi3_mini_1st_type.png}
        \caption{Phi3-Mini}\label{fig:phi3_mini_1st_type}
    \end{subfigure}
    \begin{subfigure}[t]{0.47\textwidth}
        \includegraphics[width=\textwidth]{figures/paper/phi3.5_mini_1st_type.png}
        \caption{Phi3.5-Mini}\label{fig:phi3.5_mini_1st_type}
    \end{subfigure}\\
    \begin{subfigure}[t]{0.47\textwidth}
        \includegraphics[width=\textwidth]{figures/paper/phi3_mini_128k_1st_type.png}
        \caption{Phi3-Mini-128k}\label{fig:phi3_mini_128k_1st_type}
    \end{subfigure}
    \begin{subfigure}[t]{0.47\textwidth}
        \includegraphics[width=\textwidth]{figures/paper/phi3_medium_1st_type.png}
        \caption{Phi3-Medium}\label{fig:phi3_medium_1st_type}
    \end{subfigure}\\
    \begin{subfigure}[t]{0.47\textwidth}
        \includegraphics[width=\textwidth]{figures/paper/qwen2_7b_1st_type.png}
        \caption{Qwen2-7B}\label{fig:qwen2_7b_1st_type}
    \end{subfigure}
    \begin{subfigure}[t]{0.47\textwidth}
        \includegraphics[width=\textwidth]{figures/paper/qwen2_7b_instruct_1st_type.png}
        \caption{Qwen2-7B-Instruct}\label{fig:qwen2_7b_instruct_1st_type}
    \end{subfigure}
    \caption{(Continuation of \cref{fig:llama2_7b_withattn}).
    Norm of all trained tokens after the explosion layer when they are used as the initial token in an input sequence.
    We also plot the norm of a number of random input embeddings (sorted by their output norms) after the explosion layer.
    }\label{fig:more_llm_1st_type}
\end{figure*}

\section{More Examples of Explosion Subspace}\label{sec:more_subspace}

\cref{fig:more_llm_ffn_output} shows that the leading right singular vector of the FFN module in the explosion layer ignites the explosion of the token norms for more models.
\cref{fig:more_llm_subspace_coef} shows the coefficients of tokens projected to the explosion subspace just before the FFN in the explosion layer of LLMs for more models.

\begin{figure*}[!t]
    \centering
    \begin{subfigure}[t]{0.49\textwidth}
        \includegraphics[width=\textwidth]{figures/paper/llama2_7b_chat_ffn_output.pdf}
        \caption{LLaMA2-7B-Chat}\label{fig:llama2_7b_chat_ffn_output}
    \end{subfigure}
    \begin{subfigure}[t]{0.49\textwidth}
        \includegraphics[width=\textwidth]{figures/paper/llama2_7b_code_ffn_output.pdf}
        \caption{LLaMA2-7B-Code}\label{fig:llama2_7b_code_ffn_output}
    \end{subfigure}\\
    \begin{subfigure}[t]{0.49\textwidth}
        \includegraphics[width=\textwidth]{figures/paper/llama2_13b_ffn_output.pdf}
        \caption{LLaMA2-13B}\label{fig:llama2_13b_ffn_output}
    \end{subfigure}
    \begin{subfigure}[t]{0.49\textwidth}
        \includegraphics[width=\textwidth]{figures/paper/llama2_13b_chat_ffn_output.pdf}
        \caption{LLaMA2-13B-Chat}\label{fig:llama2_13b_chat_ffn_output}
    \end{subfigure}\\
    \begin{subfigure}[t]{0.49\textwidth}
        \includegraphics[width=\textwidth]{figures/paper/llama3_8b_ffn_output.pdf}
        \caption{LLaMA3-8B}\label{fig:llama3_8b_ffn_output}
    \end{subfigure}
    \begin{subfigure}[t]{0.49\textwidth}
        \includegraphics[width=\textwidth]{figures/paper/llama3_8b_instruct_ffn_output.pdf}
        \caption{LLaMA3-8B-Instruct}\label{fig:llama3_8b_instruct_ffn_output}
    \end{subfigure}\\
    \begin{subfigure}[t]{0.49\textwidth}
        \includegraphics[width=\textwidth]{figures/paper/phi3_mini_ffn_output.pdf}
        \caption{Phi3-Mini}\label{fig:phi3_mini_ffn_output}
    \end{subfigure}
    \begin{subfigure}[t]{0.49\textwidth}
        \includegraphics[width=\textwidth]{figures/paper/phi3.5_mini_ffn_output.pdf}
        \caption{Phi3.5-Mini}\label{fig:phi3.5_mini_ffn_output}
    \end{subfigure}\\
    \begin{subfigure}[t]{0.49\textwidth}
        \includegraphics[width=\textwidth]{figures/paper/phi3_mini_128k_ffn_output.pdf}
        \caption{Phi3-Mini-128k}\label{fig:phi3_mini_128k_ffn_output}
    \end{subfigure}
    \begin{subfigure}[t]{0.49\textwidth}
        \includegraphics[width=\textwidth]{figures/paper/phi3_medium_ffn_output.pdf}
        \caption{Phi3-Medium}\label{fig:phi3_medium_ffn_output}
    \end{subfigure}\\
    \begin{subfigure}[t]{0.49\textwidth}
        \includegraphics[width=\textwidth]{figures/paper/qwen2_7b_ffn_output.pdf}
        \caption{Qwen2-7B}\label{fig:qwen2_7b_ffn_output}
    \end{subfigure}
    \begin{subfigure}[t]{0.49\textwidth}
        \includegraphics[width=\textwidth]{figures/paper/qwen2_7b_instruct_ffn_output.pdf}
        \caption{Qwen2-7B-Instruct}\label{fig:qwen2_7b_instruct_ffn_output}
    \end{subfigure}
    \caption{(Continuation of \cref{fig:llama2_7b_ffn_output}).
    Norm of output tokens of FFN in the explosion layer using the right singular vectors of \(F\) as inputs to FFN.
    We can see that the leading right singular vector of the FFN module in the explosion layer ignites the explosion of the token norms.
    }\label{fig:more_llm_ffn_output}
\end{figure*}


\begin{figure*}[!t]
    \centering
    \begin{subfigure}[t]{0.47\textwidth}
        \includegraphics[width=\textwidth]{figures/paper/llama2_7b_chat_subspace_coef.png}
        \caption{LLaMA2-7B-Chat}\label{fig:llama2_7b_chat_subspace_coef}
    \end{subfigure}
    \begin{subfigure}[t]{0.47\textwidth}
        \includegraphics[width=\textwidth]{figures/paper/llama2_7b_code_subspace_coef.png}
        \caption{LLaMA2-7B-Code}\label{fig:llama2_7b_code_subspace_coef}
    \end{subfigure}\\
    \begin{subfigure}[t]{0.47\textwidth}
        \includegraphics[width=\textwidth]{figures/paper/llama2_13b_subspace_coef.png}
        \caption{LLaMA2-13B}\label{fig:llama2_13b_subspace_coef}
    \end{subfigure}
    \begin{subfigure}[t]{0.47\textwidth}
        \includegraphics[width=\textwidth]{figures/paper/llama2_13b_chat_subspace_coef.png}
        \caption{LLaMA2-13B-Chat}\label{fig:llama2_13b_chat_subspace_coef}
    \end{subfigure}\\
    \begin{subfigure}[t]{0.47\textwidth}
        \includegraphics[width=\textwidth]{figures/paper/llama3_8b_subspace_coef.png}
        \caption{LLaMA3-8B}\label{fig:llama3_8b_subspace_coef}
    \end{subfigure}
    \begin{subfigure}[t]{0.47\textwidth}
        \includegraphics[width=\textwidth]{figures/paper/llama3_8b_instruct_subspace_coef.png}
        \caption{LLaMA3-8B-Instruct}\label{fig:llama3_8b_instruct_subspace_coef}
    \end{subfigure}\\
    \begin{subfigure}[t]{0.47\textwidth}
        \includegraphics[width=\textwidth]{figures/paper/phi3_mini_subspace_coef.png}
        \caption{Phi3-Mini}\label{fig:phi3_mini_subspace_coef}
    \end{subfigure}
    \begin{subfigure}[t]{0.47\textwidth}
        \includegraphics[width=\textwidth]{figures/paper/phi3.5_mini_subspace_coef.png}
        \caption{Phi3.5-Mini}\label{fig:phi3.5_mini_subspace_coef}
    \end{subfigure}\\
    \begin{subfigure}[t]{0.47\textwidth}
        \includegraphics[width=\textwidth]{figures/paper/phi3_mini_128k_subspace_coef.png}
        \caption{Phi3-Mini-128k}\label{fig:phi3_mini_128k_subspace_coef}
    \end{subfigure}
    \begin{subfigure}[t]{0.47\textwidth}
        \includegraphics[width=\textwidth]{figures/paper/phi3_medium_subspace_coef.png}
        \caption{Phi3-Medium}\label{fig:phi3_medium_subspace_coef}
    \end{subfigure}\\
    \begin{subfigure}[t]{0.47\textwidth}
        \includegraphics[width=\textwidth]{figures/paper/qwen2_7b_subspace_coef.png}
        \caption{Qwen2-7B}\label{fig:qwen2_7b_subspace_coef}
    \end{subfigure}
    \begin{subfigure}[t]{0.47\textwidth}
        \includegraphics[width=\textwidth]{figures/paper/qwen2_7b_instruct_subspace_coef.png}
        \caption{Qwen2-7B-Instruct}\label{fig:qwen2_7b_instruct_subspace_coef}
    \end{subfigure}
    \caption{(Continuation of \cref{fig:llama2_7b_subspace_coef}).
    Coefficient of tokens projected to the explosion subspace just before the FFN in the explosion layer of LLMs.
    The initial tokens have a high component in the explosion subspace.
    }\label{fig:more_llm_subspace_coef}
\end{figure*}
