\section{Literature Review}
\label{sec:literature}
VRP was first formulated as an integer linear program by Dantzig and Ramser \cite{dantzig1959truck} in 1959. The seminal heuristic algorithm to solve it was developed by Clark and Wright \cite{clarke1964scheduling} in 1964. The proof of the NP-hardness of VRP was provided by Lenstra and Kan \cite{lenstra1981complexity} in 1981. In addition to these milestones, extensive studies in VRP models (e.g. \cite{golden1984fleet}
\cite{kolen1987vehicle} \cite{savelsbergh1995general}
\cite{renaud1996tabu} \cite{erdougan2012green}
\cite{schneider2014electric}), construction heuristics (e.g. \cite{gillett1974heuristic} \cite{fisher1981generalized} \cite{bramel1995location}), meta-heuristics (e.g. \cite{homberger1999two} \cite{bell2004ant} \cite{pisinger2007general}), branch-and-bounds approaches (i.e., \cite{desrochers1992new} \cite{fischetti1994branch} \cite{fukasawa2006robust}), and learning-based approaches (e.g. \cite{nazari2018reinforcement} \cite{kool2018attention} \cite{li2021learning}) have been conducted and published over the past decades. Numerous benchmark data sets (e.g. \cite{solomon1987algorithms} \cite{augerat1995computational} \cite{uchoa2017new}) have also been created to support the performance tests of newly-developed VRP algorithms. The most recent research on VRP follows a trend toward designing a general-purpose algorithm by the hybridization of various frameworks to solve a wide range of VRP variants efficiently and on a larger scale. The representative work is the unified hybrid genetic search (UHGS) developed by Vidal et al. \cite{VIDAL2014658}. More comprehensive literature reviews from different aspects of VRP are referred to these papers (\cite{braysy2005vehicle_1} \cite{braysy2005vehicle_2} \cite{montoya2015literature} \cite{kocc2016thirty} \cite{lin2014survey} \cite{braekers2016vehicle}) and this book \cite{toth2014vehicle}.

A key feature of the VRP-SA is its mixed fleet of AVs and HDVs, which can be modeled by the family of Heterogeneous Fleet Vehicle Routing Problems (HFVRP) involving only two types of vehicles. This topic is covered in a chapter by Toth and Vigo \cite{toth2014vehicle}. The HFVRP family can be further refined into different categories depending on fleet constraints (limited or unlimited) and cost structures (vehicle-dependent or vehicle-independent). Research on exact approaches for the HFVRP is relatively limited. The most effective exact approach currently available is the branch-and-cut-and-price algorithm developed by Baldacci and Mingozzi \cite{baldacci2009unified}. In contrast, most existing approaches are heuristic-based, such as the multi-start adaptive memory programming (MAMP) by Li et al. \cite{LI20101111}, the variable neighborhood search (VNS) by Imran et al. \cite{IMRAN2009509}, and the memetic algorithms by Prins \cite{PRINS2009916}. Currently, the UHGS approach \cite{VIDAL2014658} is recognized as the leading method for the HFVRP.

One recent work by Molina et al. \cite{MOLINA2020103745} also discussed the VRP with a limited number of resources available. However, their focus is on scenarios where resource limitations—such as drivers, vehicles, or other equipment—prevent the servicing of all customers. This contrasts with our work, where limited resources (e.g., remote control) shared across the entire AV fleet can influence the routing decisions of individual vehicles when serving multiple customers.