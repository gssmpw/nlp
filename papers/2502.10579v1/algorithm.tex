\section{Query Relevant Subgraph}

As you can see in Figure~\ref{fig:example}, let us consider that we have three snapshots of a graph named $G_{i}$, $G_{i+1}$, and $G_{i+2}$. We added 7 edges and removed 15 edges from snapshot $G_{i}$ to get the snapshot $G_{i+1}$. Similarly, we added 11 edges and removed 18 edges from $G_{i+1}$ to find $G_{i+2}$. Our goal is to find the arrays of results for each snapshot using our query and our specific algorithm. In this example, our algorithm is the Single Source Shortest Path (SSSP), which finds the shortest path between a specific source node and all the other nodes. The specific node in this case is node a. First, we need to find the \emph{CommonGraph} of these three snapshots, which is a graph containing the common edges of these three snapshots. After identifying the \emph{CommonGraph}, we should run the sampling algorithm to identify the important edges for our query. Then we should run the SSSP algorithm from vertex $a$ to all the other vertices on the \emph{CommonGraph} to find the important edges that contributed to the answer of our query. Therefore, we will mark all the edges that have contributed to the results of each vertex and add them to the edges of our sampled \emph{CommonGraph}. In the sampling process, some nodes are not accessible from the source node, and they remain completely untouched. The value of these nodes is set to infinity for the $SSSP$ algorithm. For example, if you look at Figure~\ref{fig:example}, the values of nodes $s$, $p$, and $t$ are infinity, indicating that they are not reachable from the source node $a$. Therefore, they form isolated islands. In this situation, we should sample all the edges of these islands because, when we add edges to obtain the results of each snapshot, we might add an edge that connects these islands to the source node a. By sampling all the edges in the islands, we ensure that the correctness of the sampled \emph{CommonGraph} is maintained. Therefore, we build the sampled \emph{CommonGraph}, which has significantly fewer edges compared to the CommonGraph. For example, in Figure~\ref{fig:example}, the \emph{CommonGraph} has 38 edges, while the sampled \emph{CommonGraph} has only 23 edges. The smaller \emph{CommonGraph} offers two benefits for us. First, the time for the initial processing of the \emph{CommonGraph} is eliminated because we have already solved the query on it and obtained the array of results. Second, the time for adding the missing edges to get the results of each snapshot is reduced due to the significantly lower number of edges on the \emph{CommonGraph}, resulting in faster convergence. 



\begin{figure}[!t]
    \centering
    \includegraphics[width=0.9\columnwidth]{diagrams/Picture1.pdf}
    \caption{CommonGraph $G_{CG}$ for Snapshots $G_i$, $G_{i+1}$ and $G_{i+2}$; and Solution for Query SSSP(a) on $G_{CG}$.}
    \label{fig1}
\end{figure}

\begin{figure*}[!t]
    \begin{minipage}{3.25in}
    \includegraphics[width=0.9\columnwidth]{diagrams/Picture2.pdf}
    \caption{Classifying Edges for SSSP(a): (Red) Contributing Edges; (Pink) Unreachable Edges; and (Black) Non-Contributing Edges.}
    \label{fig2}
    \end{minipage}
    \begin{minipage}{0.2in}
    $\;\;$
    \end{minipage}
    \begin{minipage}{3.25in}   
    \includegraphics[width=0.9\columnwidth]{diagrams/Picture3.pdf}
    \vspace{0.15in}
    \caption{$G_{QRG}^{SSSP(a)}$: Query Relevant Graph for SSSP(a).}
    \label{fig3}
    \end{minipage}
\end{figure*}

\begin{figure}[!t]
    \centering
    \includegraphics[width=0.85\columnwidth]{diagrams/Picture4.pdf}
    \caption{$G_{QRG}^SSSP(a)$ + Edges from $G_i$ not present in $G_{CG}$.}
    \label{fig4}
\end{figure}


\begin{figure*}[!t]
    \begin{minipage}{3.25in}
    \includegraphics[width=0.9\columnwidth]{diagrams/Picture5.pdf}
    \caption{Solution for SSSP(a) on $G_{QRG}^{SSSP(a)}$ + Edges from $G_i$ not present in $G_{CG}$.}
    \label{fig5}
    \end{minipage}
    \begin{minipage}{0.2in}
    $\;\;$
    \end{minipage}
    \begin{minipage}{3.25in}   
    \includegraphics[width=0.9\columnwidth]{diagrams/Picture6.pdf}
    \vspace{0.15in}
    \caption{Solution for SSSP(a) on $G_i$.}
    \label{fig6}
    \end{minipage}
\end{figure*}





Adding the missing edges to the sampled CommonGraph does not render anything incorrect because if we add a new edge to the sampled CommonGraph, and it is shorter than our old shortest path, we will consider it as the new shortest path. This is theoretically correct because all the other paths that we removed were greater than the old shortest path. We can prove it through contradiction. 

\textcolor{blue}{You can see the algorithm for edge sampling in the CommonGraph in Algorithm~\ref{qrc_algorithm}. The algorithm has two inputs: \emph{CommonGraph} and query. The output of the algorithm is the sampled \emph{CommonGraph}. The algorithm consists of two parts. The first part involves sampling edges from the \emph{CommonGraph} according to the given query and algorithm. Afterward, we add the missing edges related to the islands. To achieve this, we begin by evaluating the query on the \emph{CommonGraph}. For each edge $e(u, v)$ in the edge list of the \emph{CommonGraph} (representing an edge from vertex $u$ to vertex $v$), we determine whether this edge updates the value of vertex $v$. If an edge from vertex $u$ to vertex $v$ updates the value of vertex $v$, we check if applying the algorithm with the weight of the edge between $u$ and $v$ to the value of vertex $u$ is equal to the value of vertex $v$. If it is equal, then it indicates that the edge from vertex u to vertex $v$ contributes to the value of vertex $v$, and thus we add it to the edge list of the sampled \emph{CommonGraph}. Next, we proceed to the second part of the algorithm. In this part, we traverse all the vertices of the \emph{CommonGraph} and check whether they are untouched or touched. If a vertex is untouched, it implies that an island exists, and we should sample all the outgoing edges from that island.}

\begin{algorithm}[t] 
    \caption{Training process with CMG-Modu  strategy} \label{al:gm}
    \begin{algorithmic}[1]
        \REQUIRE{Paired histopathology features and molecular markers features $\{\mathbf{U}_h, \mathbf{U}_m \}$ for up-to-date brain tumor diagnosis; Model parameters $\Theta = \{\theta^h, \theta^m\}$; Multi-modal classifier $\mathcal{D}$; Task label $Y$; Histopathology-based grading label $T$.}
        
        \ENSURE{The optimal model parameters $\Theta^*$}
        
        %\textbf{The CMG-Modu learning strategy}
        \WHILE{$\Theta$ doesn't reach convergence} 
        \FOR{each $\mathbf{u}_h$ and $\mathbf{u}_m$ $\in$ $\{\mathbf{U}_h, \mathbf{U}_m \}$}
        \STATE Concatenate $\mathbf{u}_h$ and $\mathbf{u}_m$ to get features $\mathbf{u}_c$
        \STATE Minimize ${\mathcal{L}_{\rm{CE}}}(Y, \mathcal{D}(\mathbf{u}_c ; \theta^h, \theta^m)$

        \IF {T belongs to high grade}
        \STATE Calibrate molecular gradients using Eq. \eqref{e:gm1}
        \ELSE
        \STATE Calibrate histopathology gradients using Eq. \eqref{e:gm1}
        \ENDIF
        
        \ENDFOR
        \ENDWHILE
        \STATE The $\mathcal{D}$ is well-trained by CMG-Modu.

    \end{algorithmic}
\end{algorithm}

\vspace{0.075in}
\noindent
\textbf{Theorem 1}: 
Given a graph $G$, edges present in the CommonGraph of $G$ ($G_c$), that are not present the Query Relevant Graph for query $Q$ ($G_{qrg}^Q$) are not needed to obtain precise query results for all snapshots.
%Adding a new edge between two nodes $u$ and $v$, which is shorter than their current shortest path, creates a new shortest path for those two nodes.

\hspace{0.5in}
\includegraphics[width=0.275\textwidth]{diagrams/theorem1.pdf}

\vspace{0.075in}
\noindent
\textbf{Proof}: We can prove it by contradiction. Let us consider that we add a new edge (red) between two nodes $u$ and $v$. The distance between $u$ and $v$, denoted as $dist(u,v)$, is smaller than the sum of the distances between u and an intermediate node $m$ ($dist(u,m)$) plus the distance between $m$ and $v$ ($dist(m,v)$):

\vspace{0.05in}
$\;\;$ $dist(u,v)$ $\;<\;$ $dist(u,m)+dist(m,v)$ \hspace{0.8in} (1) 
\vspace{0.05in}
\\
\noindent 
Therefore, $dist(u,v)$ becomes the new shortest path. And also $dist(u,m) + dist(m,v)$ is shorter than any other path between two nodes u and v except dist(u,v); Thus:

\vspace{0.05in}
$\;\;$ $dist(u,m)+dist(m,v)$ $\;<\;$ $dist(u,n)+dist(n,v)$ \hspace{0.15in} (2) 
\vspace{0.05in}
\\
\noindent 
Now, let us assume we have a shorter shortest path from node $n$ between nodes $u$ and $v$ in the \emph{CommonGraph}, which we did not sample (contradiction to our theorem). This implies that:

\vspace{0.05in}
$\;\;$ $dist(u,n)+dist(n,v)$ $\;<\;$ $dist(u,v)$ \hspace{0.87in} (3)
\vspace{0.05in}
\\
\noindent
based on equation (1), $dist(u,v)$ is the shorter than $dist(u,m) + dist(m,v)$ and if we replace it:

\vspace{0.05in}
$\;\;$ $dist(u,n)+dist(n,v)$ $\;<\;$ $dist(u,m)+dist(m,v)$ \hspace{0.15in} (4)
\vspace{0.05in}
\\
\noindent
However, this contradicts our earlier conclusion because based on the equation (2), $dist(u,n) + dist(n,v)$ can not be less than $dist(u,m) + dist(m,v)$.



