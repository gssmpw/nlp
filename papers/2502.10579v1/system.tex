



% \vspace{-0.1in}
\section{Our Approach Based Upon Identifying Unchanged Vertex Values (UVVs)}
%In the preceding section we showed how the Common Graph enables elimination of redundant computation by performing substantially performing computation shared across snapshots once, and then utilizing its result to bootstrap the incremental computations. However, we have observed that the size of the Common Graph is comparable to size of the evolving graph. Both being large, all computations suffer from memory performance degradation due to large memory footprint. In this section we describe our initial approach for reducing the memory footprint.

From our motivating study, we observed that the query results computed for different snapshots are substantially the same, i.e., the addition and deletion of edges frequently causes changes to property values of a small subset of vertices. In our discussion, we use UVVs to refer to vertices with Unchanged Vertex Values. The example in Figure~\ref{alg1} illustrates the presence of UVVs -- 6 of the 10 total shortest path values computed from source vertex $s$ are the same for the two snapshots, these are the ones marked green. %We have further observed that the query results computed using the Common Graph also overlap many of the results for individual snapshots. 

\vspace{0.125in}
\begin{tabular} {|p{7.5cm}|} \hline
\textsf{Put differently, we observe that, for a specific query $Q$, there are many vertices whose property values remain unchanged across all snapshots.} \\ \hline
\end{tabular}
\vspace{0.15in}

\begin{figure}[!t]
    \centering
    \includegraphics[width=0.99\linewidth]{diagrams/Pic1.pdf}
    \vspace{-0.1in}
    \caption{Results of query SSSP(s) for two consecutive snapshots. The second snapshot is obtained by deleting red edges from the first snapshot and also adding the blue edges to the second snapshot. Note that the shortest path values for vertices marked in green are identical for both snapshots.}
    \label{alg1}
    \vspace{-0.175in}
\end{figure}

This observation motivates us to identify UVVs and then use them to eliminate wasteful work performed during incremental computations, including computations that attempt to update UVV vertices and edge traversals that lead to UVV vertices. By identifying UVVs and shrinking the size of the graph over which incremental computations for a given query $Q$ are performed, we will affect this optimization. The reduced graph is named Q-Relevant Subgraph (QRS) for query $Q$. Next we describe the steps of our algorithm and illustrate it using an example.

Let us consider the shortest path query in this discussion, though our approach is applicable to other vertex specific path-based monotonic algorithms. Furthermore, without loss of generality, assume that all vertices are present in all snapshots. Only the set of edges present differs across the snapshots due to batches of updates in the form of edge additions and deletions that are performed as the graph evolves.

\vspace{0.1in}
Our approach for identifying UVVs and generating the query specific Q-Relevant Subgraph is as follows.

\paragraph{Step 1: Bounding Vertex Values for a Query} 
Let $E_0$, $E_1$, $\cdots$ $E_n$ denote the sets of edges corresponding to the evolving graph's snapshots $G_0$, $G_1$, $\cdots$ $G_n$. We consider two graphs that are derived from the above snapshots as follows:

\begin{itemize}\itemsep2pt
    \item \textbf{Intersection Graph} $G_\cap$: This is the graph that contains edges that are common to all the snapshots, i.e., $E_\cap = E_0 \cap \cdots \cap E_n$.
    \item \textbf{Union Graph} $G_\cup$: This is the graph that contains all edges present across all the snapshots, i.e., $E_\cup = E_0 \cup \cdots \cup E_n$.
\end{itemize}
%We can also construct a graph that contains all the edges present in the snapshots. We refer to this graph as the Union Graph $G_{\cup}$ as opposed to the Common Graph that can be referred to as the Intersection Graph $G_{\cap}$.


\begin{figure}[!h]
    \centering
    \includegraphics[width=0.99\linewidth]{diagrams/Pic21.pdf}
    \vspace{-0.1in}
    \caption{The Union Graph $G_{\cup}$ provides upperbounds on path lengths across all snapshots while the Intersection Graph $G_{\cap}$ provides the lowerbounds. Query Relevant Graph obtained by reducing $G_{\cap}$ and the query results used to bootstrap incremental computations.}
    \label{alg2}
    \vspace{-0.15in}
%\end{figure*}
% \vspace{0.15in}
% %\begin{figure*}[!t]
%     \centering
%     \includegraphics[width=0.6\linewidth]{diagrams/Pic3.pdf}
%     \vspace{-0.15in}
%     \caption{Query Relevant Graph obtained by reducing $G_{\cap}$ and the query results used to bootstrap incremental computations.}
%     \label{alg3}
%     \vspace{-0.1in}
\end{figure}



We evaluate the shortest path query for source vertex $s$ on both $G_\cap$ and $G_\cup$. Let us denote the shortest path value computed for some vertex $v$ corresponding to  $G_\cap$ and $G_\cup$ by $Val_\cap(s,v)$ and $Val_\cup(s,v)$. The following theorem captures the relationship between the shortest path value of $v$ for any snapshot $G_i$.



\vspace{0.1in}
\textbf{Theorem 1.} Given source vertex $s$, the shortest path values from $s$ to vertex $v$ for $G_\cap$ and $G_\cup$, that is,  $Val_\cap(s,v)$ and $Val_\cup(s,v)$, represent the \emph{upperbound} and \emph{lowerbound} over the shortest path value of vertex $v$ across all snapshots $G_0$, $G_1$ $\cdots$ $G_n$.

\textbf{Proof.} 
We observe that the Intersection Graph $G_\cap$ contains only a subset of paths from any snapshot $G_i$ because $E_\cap = E_0 \cap \cdots \cap E_n$. Therefore, the shortest path value of vertex $v$ corresponding to snapshot $G_i$, denoted by $Val_i(s,v)$, is bounded by $Val_\cap(s,v)$ as follows:

\vspace{-0.15in}
\[  Val_i(s,v) \leq Val_\cap(s,v) \]

Similarly, we observe that the Union Graph $G_\cup$ contains a superset of paths from any snapshot $G_i$ because $E_\cup = E_0 \cup \cdots \cup E_n$. Therefore, the shortest path value of vertex $v$ corresponding to snapshot $G_i$, denoted by $Val_i(s,v)$, is bounded $Val_\cup(s,v)$ as follows:

\vspace{-0.15in}
\[ Val_\cup(s,v) \leq Val_i(s,v) \]

Therefore we conclude the following:
\[ Val_\cup(s,v) \leq Val_i(s,v) \leq Val_\cap(s,v) \]

Note that the above result holds even when vertex $v$ is not reachable from $s$ in $G_i$ or $G_\cap$ (i.e., shortest path value is $\infty$).

Finally, in an evolving graph, an edge between a pair of nodes may be added and deleted a number of times. This type of edge will not be present in $G_{\cap}$ since it is not present in all the snapshots. However, it will be present in $G_{\cup}$. The weight of this edge can be set to the minimum of all weights encountered for this edge to obtain a safe lowerbound. 

% \vspace{-0.05in}
\begin{table}[!t]
\vspace{0.05in}
\caption{Upper and Lower bounds for different algorithms.}
\label{UpperboundLowerbound}
\vspace{-0.1in}
\small
\centering
\begin{tabular}{|l|c|} \hline

Alg.
%& \textsc{Needed} $( e(u,v) )$ \\ 
& Upperbound and Lowerbound for $Val_i(s,v)$ \\ \hline \hline

BFS
%& $Val(v) < min( Val(u)+1, val(v))$ \\
& $Val_\cup(s,v) \leq Val_i(s,v) \leq Val_\cap(s,v)$ \\ \hline

SSWP
%& $Val(v) < min( Val(u), wt(u,v) )$ \\
& $Val_\cap(s,v) \leq Val_i(s,v) \leq Val_\cup(s,v)$ \\ \hline

SSNP
%& $Val(v) > max( Val(u), wt(u,v) )$ \\
& $Val_\cup(s,v) \leq Val_i(s,v) \leq Val_\cap(s,v)$ \\ \hline

SSSP
%& $Val(v) > Val(u) + wt(u,v)$ \\
& $Val_\cup(s,v) \leq Val_i(s,v) \leq Val_\cap(s,v)$ \\ \hline
 
Viterbi
%& $Val(v) < Val(u)/wt(u,v)$ \\
& $Val_\cap(s,v) \leq Val_i(s,v) \leq Val_\cup(s,v)$ \\ \hline

\end{tabular}
\vspace{-0.15in}
\end{table}

The results of computing the lower and upper bounds for our example using the intersection and union graphs are shown in Figure~\ref{alg2}. 

In our discussion we presented the upper and lower bounds for $Val_i(s, v)$ for the SSSP algorithm. In Table~\ref{UpperboundLowerbound}, we present the bounds for various benchmarks that we use in our evaluation (see Table~\ref{benchmarks}). The upper and lower bounds for the SSSP, SSNP, and BFS algorithms are similar because we take the minimum of all possible results for a node -- the intersection graph gives us the upperbound and the union graph the lowerbound. On the other hand, for the SSWP and Viterbi algorithms, since we take the maximum of all possible results, the union graph gives us the upperbound, and the intersection graph gives us the lowerbound.


\paragraph{Step 2: Identifying UVVs -- Vertices Whose Values Remain the Same Across All Snapshots} So far we have observed that by solving a shortest path query on both $G_{\cup}$ and $G_{\cap}$ we can bound the path lengths for any vertex across all snapshots. This is illustrated in the example shown in Figure~\ref{alg2}. It is interesting to note that for many vertices, the lowerbound and upperbound match precisely. This implies that the \textbf{shortest path lengths for these vertices remain unchanged across all snapshots}. Consequently, we already have their results, and now we need to perform incremental computations to update the results of only a subset of vertices in each snapshot.

\vspace{0.05in}
\textbf{Theorem 2.} Given a shortest path query with source vertex $s$ and some vertex $v$:

\vspace{-0.15in}
\[ if \;\; Val_\cup(s,v) = Val_\cap(s,v) \;\; then\]

\vspace{-0.25in}
\[ \forall i \;[0\ldots n] \;\; Val_i(s,v) = Val_\cup(s,v) = Val_\cap(s,v), \]

\noindent
i.e., the query result value for vertex $v$ is the same for all snapshots and equal to $Val_\cup(s,v)$ (or $Val_\cap(s,v)$).

\vspace{0.1in}
\textbf{Proof.} Since $Val_\cup(s,v) = Val_\cap(s,v)$, the shortest path from $s$ to $v$ of the same length is present both in $G_\cap$ and $G_\cup$. Moreover, the presence of shortest path in $G_\cap$ implies that it is also present in all the snapshots because $G_\cap$ is the subgraph of each snapshot. 



Furthermore, there cannot be any path from $s$ to $v$ in any snapshot $G_i$ that is of a shorter path length than $Val_\cap(s,v)$ even though $G_i$ contains edges that are not present in $G_\cap$. This is because if such a path existed, it would also be present in $G_\cup$ and that would contradict the fact that $Val_\cup(s,v) = Val_\cap(s,v)$.


\begin{figure*}[!t]
    \centering
    \includegraphics[width=0.62\linewidth]{diagrams/Pic4-new.pdf}
    \vspace{-0.15in}
    \caption{Carrying out incremental computations via edge additions to obtain final query results for each snapshot. Note that for vertex $c$, the final query result is the same (10) for both snapshots while its lower and upper bounds were 9 and 11.}
    \label{alg4}
    \vspace{-0.15in}
\end{figure*}

\vspace{0.05in}
Note that when $Val_\cup(s,v) = Val_\cap(s,v)$ we have already found the shortest path value from $s$ to $v$ for all snapshots. However, when $Val_\cup(s,v) \neq Val_\cap(s,v)$, it does not imply that $Val_i(s,v)$ cannot be the same for all snapshots. 

Our algorithm is safe but not complete, that is, for a given query it does not identify all vertices for which the shortest path value remains the same across all snapshots.
In Figure~\ref{alg2} we could not identify that its value remains unchanged. This is because neither $G_{\cap}$ nor $G_{\cup}$ provide the value 10, rather they provided values 11 and 9. Yet, in Figure~\ref{alg4}, note that the shortest path value for vertex $c$ is 10 in both snapshots. 




\vspace{-0.05in}
\paragraph{Step 3: Deriving Q-Relevant Subgraph} Before performing incremental computations, we can substantially reduce the size of the Intersection Graph as follows. 

\vspace{0.075in}
\begin{tabular} {|p{7.5cm}|} \hline
\textsf{For each vertex whose property value has already
been determined, that is, its lowerbound and upperbound are found to be equal, the set of its incoming edges can be removed from the graph.} \\ \hline
\end{tabular}

\vspace{0.1in}
\noindent
This is because for the vertices whose results are already known, no vertex updates are needed and the incoming edges are responsible for causing their updates can be safely eliminated. In Figure~\ref{alg2}, we show the resulting Q-Relevant Subgraph obtained after reducing the Intersection Graph $G_{\cap}$ by eliminating incoming edges of vertices with precise vertex values. The shortest path values of vertices that bootstrap the next incremental computation phase correspond to the shortest path values obtained by solving the query on $G_\cap$.

\vspace{-0.075in}
\paragraph{Step 4: Incremental Computations for Snapshots} Starting from the Q-Relevant Subgraph, and initial results computed from the Intersection Graph, we perform incremental computations to obtain precise results for both snapshots (see Figure~\ref{alg4}). Note when a batch of additions is used for incremental computation, the batch can also be further reduced by eliminating the edges whose sink is a node with already known precise solution. In Figure~\ref{alg4}, edge from $d$ to $r$ need not be streamed for snapshot $G_i$ because value for vertex $r$ is already precise and same in both snapshots.


In conclusion, while our algorithm is safe, it does not identify all vertices whose value remains the same across all snapshots. However, as the experimental results presented later in the paper will demonstrate, our algorithm is highly effective as it identifies nearly all the vertices whose property value remains unchanged across all snapshots.



\vspace{-0.1in}
\paragraph{Algorithm Summary} Algorithm~\ref{algo:QRG} shows how the QRS is found. The inputs to the algorithm are the Intersection Graph ($G_{\cap}$) and the Union Graph ($G_{\cup}$) of all the snapshots (line 1), the delta batches, which must be added to the Intersection Graph ($G_{\cap}$) to obtain the results for each snapshot of the graph. The number of delta batches equals the number of snapshots ($\Delta_{0}$, $\Delta_{1}$, ..., $\Delta_{n}$). Additionally, we should specify the query ($Q$) as an input to the algorithm. The output of the algorithm is the results of the query evaluation for each snapshot of the graph (line 3). Therefore, we have $n$ arrays to represent results of each snapshot, $R_{0}$, $R_{1}$, ..., $R_{n}$.

The first step of the algorithm is to compute the results of solving the query on the Intersection Graph ($G_{\cap}$) and the Union Graph ($G_{\cup}$). Therefore, we should define two result arrays, $R_{\cap}$ and $R_{\cup}$, to store the outcomes of the query evaluation on each graph. The size of these two arrays is proportional to the number of vertices in the graph, which is $m$ (lines 4-6). The \textsc{Compute} (Graph $G$, Query $Q$) function will evaluate the query $Q$ on the graph $G$ (lines 7-8). In the second step of the algorithm, we need to compare the results from the two value arrays $R_{\cap}$ and $R_{\cup}$. Therefore, we should use a for loop to iterate over each element of these arrays. If an element is the same in both $R_{\cap}$ and $R_{\cup}$, we should mark the vertex and add it to a set named $found$ (lines 9-15).

\begin{algorithm}[!t]
% \vspace{-0.25in}
\small
\begin{algorithmic}[1]

\State{\textbf{Inputs}: 
Intersection Graph ($G_{\cap}$); Union Graph ($G_{\cup}$);}
\State{Addition Edges Batches ($\Delta_{0}$, $\Delta_{1}$, ..., $\Delta_{n}$); Query ($Q$).}
\State{\textbf{Output}: Results for Solving Query ($Q$) on Each Snapshot of the Graphs ($R_{0}$, $R_{1}$, ..., $R_{n}$).}

\State \textcolor{teal}{$\rhd$ Compute $Q$ on $G_{\cap}$ and $G_{\cup}$ ($m$ is number of vertices)}
\State{$R_{\cap}$[$1$, $m$]: array results for $G_{\cap}$}
\State{$R_{\cup}$[$1$, $m$]: array results for $G_{\cup}$}
\State{$R_{\cap}$[$1$, $m$] $\leftarrow$ \textsc{Compute} ($G_{\cap}$, $Q$)}
\State{$R_{\cup}$[$1$, $m$] $\leftarrow$ \textsc{Compute} ($G_{\cup}$, $Q$)}

\State \textcolor{teal}{$\rhd$ Find Values That Are Same On Both $G_{\cap}$ and $G_{\cup}$}
\State{$found$: set for storing vertices with precise values}
\ForEach{$i$ $\in$ [$1$, $m$]}
\If{$R_{\cap}$[$i$] == $R_{\cup}$[$i$]}
\State{$found$ $\leftarrow$ $insert (i)$}
\EndIf
\EndFor

\State \textcolor{teal}{$\rhd$ Reduce the size of $G_{\cap}$ and delta batches and Find $G_{QRS}$}
\ForEach{$v$ $\in$ $found$}
\State{\textsc{RemoveIncomingEdges}($G_{\cap}$, $v$)}
\State{\textsc{RemoveDeltaAdditionBatches}($v$)}
\EndFor
\State{$G_{QRS}$ $\leftarrow$ $G_{\cap}$}

\State \textcolor{teal}{$\rhd$ Add the Batches to $G_{QRS}$ and Find the Results}
\ForEach{$i$ $\in$ [$0$, $n$]}
\State{$R_{i}$ $\leftarrow$ $\textsc{Increment}$ ($G_{QRS}$, $\Delta_{i}$)}
\EndFor

\State \textcolor{teal}{$\rhd$ Function for removing the incoming edges}
\Function{RemoveIncomingEdges}{Graph $G$, Vertex $v$}
% \ForEach{$vertex$ $u$ $\in$ $G$}
% \If{there is an $edge$ from $u$ to $v$}
% \State{remove $edge$ ($u$, $v$)}
% \EndIf
% \EndFor
\ForEach{$x$ $\in$ $G[v]$.inNeighbors}
\State{remove $edge$ ($x$, $v$)}
\EndFor
\EndFunction

\State \textcolor{teal}{$\rhd$ Function for removing edges from delta batches}
\Function{RemoveDeltaAdditionBatches}{Vertex $v$}
\ForEach{$i$ $\in$ [$0$, $n$]}
% \ForEach{$vertex$ $u$ $\in$ $\Delta_{i}$}
% \If{there is an $edge$ from $u$ to $v$}
% \State{remove $edge$ ($u$, $v$)}
% \EndIf
% \EndFor
\ForEach{$edge$ ($u$, $x$) $\in$ $\Delta_i$}
\If{$x$ == $v$}
\State{remove $edge$ ($u$, $v$)}
\EndIf
\EndFor
\EndFor
\EndFunction
%\Statex{}

\end{algorithmic}
\caption{Finding Q-Relevant Subgraph}
\label{algo:QRG}
\end{algorithm}


Next, we should reduce the size of the Intersection Graph ($G_{\cap}$) using the $found$ set. Currently, the $found$ set consists of all the vertices with the same value across all the snapshots. Therefore, we should remove the incoming edges of those vertices that are in the $found$ set using the \textsc{RemoveIncomingEdges} function. We should also remove the edges in the delta batches that have the same destination as those in the $found$ set using the \textsc{RemoveDeltaAddition} function (lines 16-21). Then, we can rename the reduced $G_{\cap}$ graph to $G_{QRS}$. Finally, we should incrementally add the reduced-size delta batches to the Q-Relevant Subgraph ($G_{QRS}$) to determine results of query $Q$ on each snapshot (lines 22-25).

The function \textsc{RemoveIncomingEdges}(Graph $G$, Vertex $v$) has two inputs: Graph $G$ and Vertex $v$. It iterates over each vertex in $G$, and if there is an edge leading to vertex $v$, it removes that edge (lines 26-31). The function \textsc{RemoveDeltaAddition}(Vertex $v$) takes a vertex as its input and iterates over all the edges in the delta batches. If it finds an edge to vertex $v$, it removes that edge (lines 32-41).

% \vspace{-0.25in}
\section{Concurrent Incremental Computations}
% \vspace{-0.025in}
Starting from the Q-Relevant Subgraph $G_{QRS}$, the query results for each snapshot can be found by incremental computation of the addition batch ($\Delta_i$).
One way to calculate query results for all snapshots is through a sequence of snapshot evaluations, i.e., $n$ rounds of incremental computation ($INCREMENT(G_{QRS},\Delta_i')$, where $i=0,...,n$ and $\Delta_i'$ is the reduced batch corresponding to $\Delta_i$). 
However, this approach has two issues: 1) resource under utilization and 2) data locality is not fully exploited.
First, incremental computation (especially for edge additions) is lightweight in comparison to full query evaluation, potentially leading to underutilization of machine resources. 
Second, same edges, either present in $G_{QRS}$ or $\Delta_i'$,  may be traversed multiple times across different snapshots, worsening cache locality.

Instead of evaluating snapshots one by one, we propose concurrent evaluation. An augmented graph with versioning and \textit{snapshot-oblivious} frontier are used for efficiency.
%Next, we describe these techniques in detail.

\begin{figure}[!t]
    \centering
    \includegraphics[width=0.95\linewidth]{diagrams/vqrs_example1.pdf}
    \vspace{-0.175in}
    \caption{(a) Versioned Graph; and (b) the augmented adjacency list (only lists for out-going edges are shown; bold edges are $G_{QRS}$ edges; and there are 4 snapshots)}
    \label{fig:vg_example}
    \vspace{-0.25in}
\end{figure}

% \begin{figure}[!t]
%  %\vspace{-0.15in}
%     \centering
%     % \includegraphics[width=1.0\columnwidth]{diagrams/version_graph.pdf}
%     % \vspace{-0.15in}
%     % \caption{Versioned Graph Example (only adjacency lists for out-going edges are shown; bold edges are $G_{QRS}$ edges; and there are four snapshots).}
%     % \label{fig:version_graph}
%     \begin{subfigure}[b]{0.30\textwidth}
%         \includegraphics[width=1\linewidth]{diagrams/vqrs_example.pdf}
%         \caption{}
%         \label{fig:vg_example} 
%     \end{subfigure}
%     \begin{subfigure}[b]{0.3\textwidth}
%         \includegraphics[width=1\linewidth]{diagrams/versioned_adj.pdf}
        
%         \caption{}
%         \label{fig:vg_adj} 
%     \end{subfigure}
%     \vspace{-0.1in}
%     \caption{(a) Versioned Graph Example and (b) the augmented adjacency list (only adjacency lists for out-going edges are shown; bold edges are $G_{QRS}$ edges; and there are four snapshots).}
%     \label{fig:version_graph}
% \vspace{-0.15in}
% \end{figure}

% \vspace{-0.075in}
\subsection{Versioned Graph Representation}
% \vspace{-0.025in}
% As shown earlier, $G_{\cap}$ contains edges that are common to all snapshots, and is a subset of $G_{\cup}$. 
We augment the Q-Relevant Subgraph $G_{QRS}$ with extra edge versioning information to show which snapshots an edge belongs to. A 64-bit variable is used for storing such version information of an edge (more bits can be added for supporting greater than 64 snapshots). The bit at each location indicates if the edge is present in the corresponding snapshot. For edges that are common to all snapshots, their version labels are all 1s (i.e., \texttt{1111....1111}). 

The augmented versioned graph has more edges than the Q-Relevant Subgraph -- edges in $G_{QRS}$ plus the reduced addition batches ($G_{QRS}\cup \Delta_0'\cup \Delta_1'\cup ... \Delta_{n}'$). 
Since $G_{QRS}$ is reduced from the Intersection Graph ($G_{\cap}$), it contains a subset of edges in $G_{\cap}$ that are common to all snapshots. Those 
common edges are stored at the beginning of the adjacency lists, followed by snapshot-specific edges.

An  augmented graph is shown in Figure~\ref{fig:vg_example}. 
$G_{QRS}$ has four edges (edges in bold) that have \texttt{1111}s as their version value in the adjacency lists.
Four graph snapshots are embedded into the augmented graph, which can be obtained by adding $\Delta_0'$ through $\Delta_3'$ to $G_{QRS}$, respectively.
Edge $\langle D\rightarrow C\rangle$ is present in both snapshots 0 and 3, so its version number is \texttt{1001}. Edges $\langle E\rightarrow C\rangle$ and $\langle E\rightarrow D\rangle$ are common to all snapshots. % and appear at the beginning of $E$'s adjacency list.  



% \vspace{-0.1in}
\subsection{Concurrent Snapshot Evaluation}
%\vspace{-0.025in}
Now we describe how multiple snapshots are evaluated concurrently.
The traditional graph query evaluation, the out-going edges of active vertices are evaluated by the edge function and vertices that have their values changed will be put into the frontier.
In the concurrent snapshot evaluation, there are two aspects to consider:
1) the ownership of edges that must be checked when traversing the versioned graph; and 2) the ownership of active vertices that help distinguish which vertex is active for which snapshot.
The edge ownership cannot be neglected as it affects the correctness of concurrent snapshot evaluation. We show that the ownership of active vertices can be further relaxed for better performance. In a basic concurrent evaluation design, it is intuitive to maintain a separate frontier for each snapshot since an active vertex may not be active for all snapshots; however, this introduces extra overhead due to the maintenance and access of multiple frontiers.
Instead, \texttt{UVVs} employs a design called \textit{snapshot-oblivious} frontier, inspired by recent works on concurrent graph query processing~\cite{yin2022glign,mazloumi2019multilyra}.
The \textit{snapshot-oblivious} frontier does not distinguish which vertex is active for which snapshot; given a batch of snapshots, it simply treats the vertex active for all snapshots by using a single frontier, which is the union of all separate frontiers. The correctness of \textit{snapshot-oblivious} frontier is guaranteed by the monotonic property of graph algorithms.

%Note that the \textit{snapshot-oblivious} frontier resembles \textit{query-oblivious} frontier~\cite{glign} in that they both compute results for multiple instances concurrently. \textit{query-oblivious} frontier is used for computing queries from different source vertices on the same static graph, while \textit{snapshot-oblivious} frontier is used for computing the same query on multiple graphs, referred to as snapshots.

\begin{algorithm}[t]
\caption{Concurrent Evaluation of Snapshots}
\label{algo:concurrent-snapshot}
\begin{algorithmic}[1]
\algnotext{EndFor}
\algnotext{EndParFor}
\algnotext{EndIf}
% \algnotext{EndWhile}
\fontsize{8}{8.5}
\selectfont
\Function{BatchEvaluation}{$G$, $n$, $\Delta[...]$, $f$}
\State ($F_{so}$,$F_{next}$) = ($\emptyset$, $\emptyset$) \textcolor{teal}{ $\rhd$  Snapshot-oblivious frontier }
% \State $F_{next} = \emptyset$
\State $R_{i\in [0:n-1]}$ = $R_{QRS}$ \textcolor{teal}{ $\rhd$  Initialize results for each snapshot  }
\ParFor{$\Delta_i$ \textbf{in} $\Delta[...]$} \textcolor{teal}{ $\rhd$  Processing addition batches } \label{line:inc_add_0}
\ParFor{$\langle u\rightarrow v, w\rangle$ \textbf{in} $\Delta_i$} 
\If{$f(\langle u\rightarrow v, w\rangle)$ improves $R_i[v]$}
    \State $R_i[v]$=$f(\langle u\rightarrow v, w\rangle)$
    \State $F_{so}$=$F_{so}\cup\{v\}$
\EndIf
\EndParFor
\EndParFor\label{line:inc_add_1}
\textcolor{teal}{$\rhd$ Concurrent snapshot evaluation }
\While {$F_{so} \neq \emptyset$}
\ParFor{$v$ \textbf{in} $F_{so}$} \label{line:snapshot_oblivious}
\For{$x$ \textbf{in} $v$'s out-going neighbors}
\For{$i$ \textbf{from} $0$ \textbf{to} $n-1$} \label{line:concurrent_eval_0}

\hspace{0.5in}\textcolor{teal}{ $\rhd$ Check if the edge belongs to snapshot $i$ }
\If{snapshotHasEdge($i$,$\langle v\rightarrow x\rangle$)}  
\If{$f(\langle v\rightarrow x, w\rangle)$ improves $R_i[x]$}
    \State $R_i[x]$=$f(\langle v\rightarrow x, w\rangle)$
    \State $F_{next}$=$F_{next}\cup\{x\}$
\EndIf
\EndIf
\EndFor
\EndFor
\EndParFor\label{line:concurrent_eval_1}
\State swap($F_{next}$, $F_{so}$)
\State $F_{next}=\emptyset$
\EndWhile
\State \Return{$R_{i\in [0:n-1]}$}
\EndFunction
\end{algorithmic}
\end{algorithm}

\begin{figure}[h]
 %\vspace{-0.05in}
    \centering
    \includegraphics[width=0.8\columnwidth]{diagrams/snap_oblivious_example1.pdf}
    \vspace{-0.1in}
    \caption{\emph{Snapshot-oblivious frontier} and concurrent snapshot traversal.}
    \label{fig:sof_example}
\vspace{-0.15in}
\end{figure}

Figure~\ref{fig:sof_example} shows an example of \textit{snapshot-oblivious} frontier and concurrent snapshot traversal. Two snapshots are considered in this case with their frontiers $F_0$ and $F_1$. 
The out-going edges of active vertices in $F_0$ and $F_1$ have their ownership checked before the edge function can be applied.
For example, active vertex $A$'s out-going edge $\langle A\rightarrow E\rangle$ that is owned by snapshot $S_0$ (the version label is \texttt{01}) will be evaluated for updating the results of $S_0$, while $\langle A\rightarrow D\rangle$ will not be evaluated for $S_0$ as it only belongs to $S_1$.
% So only $\langle A\rightarrow E\rangle$ is evaluated by the edge function for snapshot $S_0$. 
Regarding the ownership of active vertices, because of the use of \textit{snapshot-oblivious} frontier $F_{so}$, vertex $B$'s out-going edges are blindly evaluated for both snapshots even if it is only active for $S_1$. 
%Thanks to the monotonicity of graph algorithms that we studied, this redundant computation does not compromise the correctness. And we have found that 
The benefits of using \textit{snapshot-oblivious} frontier (without maintaining and checking separate frontiers) outweighs the extra computation overhead it introduces. Similar observation has been made in $\texttt{Glign}$~\cite{yin2022glign} about the \textit{query-oblivious} frontier.

The concurrent snapshot evaluation in Algorithm~\ref{algo:concurrent-snapshot} takes Q-Relevant Subgraph as input and to which addition batch $\Delta_i$ is applied to incrementally compute query results on snapshot $i$. The incremental computation for edge additions adds vertex $v$ to the frontier if a new edge $\langle u,v\rangle$ in addition batch improves the vertex value of $v$ (Line~\ref{line:inc_add_0} to ~\ref{line:inc_add_1}), e.g., a shorter shortest path is found after adding the edge.
Iterative computation continues till the frontier becomes empty. At Line~\ref{line:snapshot_oblivious}, the \textit{snapshot-oblivious} frontier does not distinguish between snapshots; it evaluates the vertex for all snapshots which may introduce redundant computation but still outweighs the overhead of tracking separate frontiers for each snapshot. Out-going edges of active vertices are evaluated for snapshots that contain the edges (Line~\ref{line:concurrent_eval_0} to ~\ref{line:concurrent_eval_1}). 

% \noindent Versioned graph representation vs. other evolving graph storage engines.
