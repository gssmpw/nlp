% \vspace{-0.1in}
\section{System}
Based on the proposed UVV finding algorithm and concurrent snapshots query evaluation engine, we developed a system for shared-memory. To our knowledge, this is the first system of its kind that supports concurrent evolving-graph query evaluation while using a very small portion of a large graph. \textbf{\emph{Our system is built upon Risgraph as it provides the most optimized implementation of KickStarter-based incremental approach that serves as the baseline in our evaluation.}} Our UVV system consists of several components described below:


\begin{itemize}
    \item An \emph{evolving graph engine} maintains data structures of the multiple snapshots. It is built on the Risgraph ~\cite{risgraph} streaming graph engine and adopts version management of Fig~\ref{fig:vg_example}(b). Risgraph leverages a fast addition query function and uses a pull-push hybrid mechanism and our support concurrent snapshots traversal.
    \item A graph reduction phase is added to the system that reduces the graph size to create a \emph{query-relevant} graph.
    \item We employ a snapshots scheduler, which takes user queries as input and maximizes the reuse of snapshots. A snapshot-oblivious frontier mechanism is added to Risgraph. Users can pick the snapshots of interest.
    \item Finally, a simple programming interface is provided where the user can program by following the vertex-centric paradigm and providing the system with the query source and snapshots to get query results.
\end{itemize}

\textbf{System Execution Engine} Two execution models are supported: concurrent and non-concurrent. Both perform graph reduction for a given user query. As Algorithm~\ref{alg1} depicted, we query intersection graph and union graph to find UVVs. The system then deletes all incoming edges of UVV vertices to create the query-relevant subgraph. The overhead of this step is included in query evaluation time.

In the non-concurrent execution mode, the system executes a query targeting multiple snapshots in two steps: the scheduling phase and the computation phase. During the scheduling phase, the query execution plan follows Fig~\ref{fig:commongraph}, and the engine follows the scheduling order to perform addition-only incremental computation for all the queried snapshots. %It's worth noting that there's no need to perform graph mutations when using our data structure from Figure~\ref{fig:vg_example}(b). In the concurrent execution mode, all snapshots are initialized with the same frontier array, and the execution follows the same pattern as in the non-concurrent mode.

% \vspace{-0.1in}
\section{Performance Evaluation}
\label{Evaluation}
%\vspace{-0.05in}
We have successfully implemented our idea on top of the RisGraph~\cite{risgraph} system which is the fastest streaming system that supports incremental algorithms for handling both edge additions and deletions. We made it suitable for analyzing an evolving graph by performing incremental computations incrementally over a versioned graph representation. We conducted our experiments on a shared memory machine on Google Cloud that has two Intel Xeon 2.60 GHz processors with 48 cores and 768GB memory. All codes are compiled by g++ version 9.4.0 and we run them on Ubuntu 20.04. 

% \vspace{-0.05in}
\paragraph*{Benchmarks} We evaluated five types of graph benchmarks: BFS (Breadth First Search), SSSP (Single Source Shortest Path), SSWP (Single Source Widest Path), SSNP (Single Source Narrowest Path), and Viterbi. Table~\ref{benchmarks} lists the benchmarks along with their edge functions.

% \vspace{-0.05in}
\paragraph*{Graph Data Sets} We used 5 real-world input graphs, as shown in Table~\ref{graphs}. The range for the number of vertices in our input graphs is from 68M to 2.6B, and the range for the number of edges is from 4.8M to 68.3M.


% \vspace{-0.1in}
\begin{table}[!t]
\caption{Benchmarks and their Edge Functions.}
% \vspace{-0.05in}
\label{benchmarks}
\vspace{-0.1in}
\small
\centering
\begin{tabular}{|l|l|} \hline

Alg.
%& \textsc{Needed} $( e(u,v) )$ \\ 
& EdgeFunction $( e(u,v) )$ \\ \hline \hline

BFS
%& $Val(v) < min( Val(u)+1, val(v))$ \\
& $CASMIN( Val(v), min( Val(u)+1, val(v) ) )$ \\ \hline

SSWP
%& $Val(v) < min( Val(u), wt(u,v) )$ \\
& $CASMAX( Val(v),  min( Val(u), wt(u,v) ) )$ \\ \hline

SSNP
%& $Val(v) > max( Val(u), wt(u,v) )$ \\
& $CASMIN( Val(v),  max( Val(u), wt(u,v) ) )$ \\ \hline

SSSP
%& $Val(v) > Val(u) + wt(u,v)$ \\
& $CASMIN( Val(v),  Val(u) + wt(u,v) )$ \\ \hline

Viterbi
%& $Val(v) < Val(u)/wt(u,v)$ \\
& $CASMAX( Val(v),  Val(u) / wt(u,v) )$ \\ \hline

\end{tabular}
\vspace{0.05in}
% \vspace{-0.05in}

\captionsetup{justification=centering}
\caption{Edges and Vertices of the Input Graphs.}
% \vspace{-0.05in}
\label{graphs}
\vspace{-0.1in}
\small
\centering
{\renewcommand{\arraystretch}{1.1}
\begin{tabular}{|l|c|c|c|} \hline
Input Graph & | Edges | &\!\!| Vertices |\!\! & \!\!|Avg degree|\!\!\\ \hline \hline
% Pokec (PL)~\cite{PL} & 30M & 1.6M & 37.51\\ \hline
LiveJournal (LJ)~\cite{LJ} & 68M & 4.8M & 28.26\\ \hline
Orkut (OR)~\cite{OR} & 117M & 3.1M & 76.28\\ \hline
WikipediaLinks (Wen)~\cite{Wen}\!\!& 437M & 13.5M & 64.32\\ \hline
Twitter (TW)~\cite{TW} & 1.46B & 41.6M & 70.51\\ \hline
Friendster (Fr)~\cite{friendster} & 2.58B & 68.3M & 55\\ \hline

\end{tabular}
}
\vspace{-0.15in}
\end{table}


% \begin{table*}[!t]
% \caption{Average Execution Time for KickStarter-based (KS) method, and the speedup of CommonGraph Work-Sharing (CG), Q-Relevant Subgraph Work-Sharing (QRS), Q-Relevant Subgraph Concurrent (CQRS) for 32 Snapshots and 150,000 batch size.}
% \label{tab:time}
% \vspace{-0.05in}
% \small
% \centering
% {\renewcommand{\arraystretch}{1.2}
% \begin{tabular}{|l|l||c|c|c|c|c|} \hline
% \multicolumn{1}{|c|}{\textbf{\textsf{G}}} & \textbf{\textsf{Query Evaluation Algorithm}} & \textsf{\textbf{BFS}} & \textsf{\textbf{SSSP}} & \textsf{\textbf{SSWP}} & \textsf{\textbf{SSNP}} 
% & \textsf{\textbf{Viterbi}} 
% \\ \hline \hline

% \multirow{4}{*}{\textsf{\textbf{LJ}}} &
%     \textsc{\textbf{KS}: KickStarter-based Time} & 768.68ms & 1127.91ms & 1106.95ms & 1074.22ms & 1317.5ms\\ \cline{2-7}
%     & \textsc{\textbf{CG}: CommonGraph Work-Sharing Speedup} & 1.21$\times$ & 1.30$\times$ & 1.43$\times$ & 1.46$\times$ & 1.11$\times$\\ \cline{2-7}
%     & \textsc{\textbf{QRS}: Q-Relevant Subgraph Work-Sharing Speedup} & 1.38$\times$ & 1.67$\times$ & 2.01$\times$ & 2.42$\times$ & 1.70$\times$\\ \cline{2-7}
%     & \textsc{\textbf{CQRS}: Q-Relevant Subgraph Concurrent Speedup} & 3.86$\times$ & 2.14$\times$ & 4.70$\times$ & 2.54$\times$ & 2.21$\times$
%     \\ \hline \hline
% \multirow{4}{*}{\textsf{\textbf{OR}}} &
%      \textsc{\textbf{KS}: KickStarter-based Time} & 670.06ms & 445.78ms & 737.68ms & 654.69ms & 644.71ms\\ \cline{2-7}
%     & \textsc{\textbf{CG}: CommonGraph Work-Sharing Speedup} & 1.12$\times$ & 1.18$\times$ & 1.53$\times$ & 1.27$\times$ & 1.17$\times$\\ \cline{2-7}
%     & \textsc{\textbf{QRS}: Q-Relevant Subgraph Work-Sharing Speedup} & 1.22$\times$ & 1.71$\times$ & 1.61$\times$ & 1.49$\times$ & 1.34$\times$\\ \cline{2-7}
%     & \textsc{\textbf{CQRS}: Q-Relevant Subgraph Concurrent Speedup} & 2.93$\times$ & 3.20$\times$ & 3.98$\times$ & 1.59$\times$ & 3.21$\times$
%     \\ \hline \hline
% \multirow{4}{*}{\textsf{\textbf{Wen}}} &
%      \textsc{\textbf{KS}: KickStarter-based Time} & 452.36ms & 910.38ms & 684.95ms & 489.25ms & 495.69ms\\ \cline{2-7}
%     & \textsc{\textbf{CG}: CommonGraph Work-Sharing Speedup} & 1.30$\times$ & 1.49$\times$ & 1.44$\times$ & 1.58$\times$ & 1.16$\times$\\ \cline{2-7}
%     & \textsc{\textbf{QRS}: Q-Relevant Subgraph Work-Sharing Speedup} & 1.70$\times$ & 1.79$\times$ & 1.85$\times$ & 2.42$\times$ & 1.55$\times$\\ \cline{2-7}
%     & \textsc{\textbf{CQRS}: Q-Relevant Subgraph Concurrent Speedup} & 3.18$\times$ & 4.50$\times$ & 4.33$\times$ & 3.06$\times$ & 2.16$\times$
%     \\ \hline \hline
% \multirow{4}{*}{\textsf{\textbf{TTW}}} &
%       \textsc{\textbf{KS}: KickStarter-based Time} & 478.21ms & 982.98ms & 980.81ms & 642.11ms & 873.31ms\\ \cline{2-7}
%     & \textsc{\textbf{CG}: CommonGraph Work-Sharing Speedup} & 1.10$\times$ & 1.95$\times$ & 1.68$\times$ & 1.30$\times$ & 1.06$\times$\\ \cline{2-7}
%     & \textsc{\textbf{QRS}: Q-Relevant Subgraph Work-Sharing Speedup} & 1.57$\times$ & 3.10$\times$ & 3.30$\times$ & 1.53$\times$ & 2.38$\times$\\ \cline{2-7}
%     & \textsc{\textbf{CQRS}: Q-Relevant Subgraph Concurrent Speedup} & 3.39$\times$ & 3.26$\times$ & 4.93$\times$ & 2.49$\times$ & 4.11$\times$
%     \\ \hline 
% \end{tabular}
% }
% %\vspace{-0.1in}
% \end{table*}



% \vspace{-0.05in}
\begin{table}[!t]
\caption{Average Execution Time for KickStarter-based (KS) method in milliseconds, and speedup for CommonGraph (CG), Q-Relevant Subgraph (QRS), Concurrent QRS (CQRS) given 64 Snapshots and 150,000 batch size.}

\label{tab:time}
\vspace{-0.1in}
\small
\centering
{\renewcommand{\arraystretch}{1.2}
\begin{tabular}{|l|l||c|c|c|c|c|} \hline
\multicolumn{1}{|c|}{\textbf{\textsf{G}}} & \textbf{\textsf{Alg.}} & \textsf{\textbf{BFS}} & \textsf{\textbf{SSSP}} & \textsf{\textbf{SSWP}} & \textsf{\textbf{SSNP}} 
& \textsf{\textbf{Viterbi}} 
\\ \hline \hline

\multirow{4}{*}{\textsf{\textbf{LJ}}} &
    \textsc{KS} & 1562.1 & 2292.2 & 2249.6 & 2183.0 & 2677.5\\ \cline{2-7}
    & \textsc{CG} & 1.17$\times$ & 1.24$\times$ & 1.32$\times$ & 1.39$\times$ & 1.07$\times$\\ \cline{2-7}
    & \textsc{QRS} & 1.30$\times$ & 1.53$\times$ & 2.08$\times$ & 2.43$\times$ & 1.67$\times$\\ \cline{2-7}
    & \textsc{CQRS} & 3.60$\times$ & 2.01$\times$ & 8.09$\times$ & 7.62$\times$ & 8.87$\times$
    \\ \hline \hline
\multirow{4}{*}{\textsf{\textbf{OR}}} &
     \textsc{KS} & 905.9 & 1361.7 & 1499.1 & 1330.5 & 1310.2\\ \cline{2-7}
    & \textsc{CG} & 1.11$\times$ & 1.06$\times$ & 1.44$\times$ & 1.17$\times$ & 1.08$\times$\\ \cline{2-7}
    & \textsc{QRS} & 1.56$\times$ & 1.14$\times$ & 1.60$\times$ & 1.53$\times$ & 1.25$\times$\\ \cline{2-7}
    & \textsc{CQRS} & 3.69$\times$ & 3.72$\times$ & 8.37$\times$ & 3.76$\times$ & 3.03$\times$
    \\ \hline \hline
\multirow{4}{*}{\textsf{\textbf{Wen}}} &
     \textsc{KS} & 919.3 & 1850.1 & 1391.9 & 994.2 & 1007.3\\ \cline{2-7}
    & \textsc{CG} & 1.23$\times$ & 1.42$\times$ & 1.37$\times$ & 1.46$\times$ & 1.10$\times$\\ \cline{2-7}
    & \textsc{QRS} & 1.55$\times$ & 1.77$\times$ & 1.85$\times$ & 2.25$\times$ & 1.52$\times$\\ \cline{2-7}
    & \textsc{CQRS} & 3.07$\times$ & 6.19$\times$ & 11.7$\times$ & 5.90$\times$ & 3.56$\times$
    \\ \hline \hline
\multirow{4}{*}{\textsf{\textbf{TW}}} &
      \textsc{KS} & 971.8 & 1997.6 & 1993.2 & 1304.9 & 1774.7\\ \cline{2-7}
    & \textsc{CG} & 1.06$\times$ & 1.86$\times$ & 1.61$\times$ & 1.20$\times$ & 1.03$\times$\\ \cline{2-7}
    & \textsc{QRS} & 1.44$\times$ & 2.94$\times$ & 3.44$\times$ & 1.51$\times$ & 2.61$\times$\\ \cline{2-7}
    & \textsc{CQRS} & 3.77$\times$ & 8.88$\times$ & 10.23$\times$ & 6.01$\times$ & 8.25$\times$
    \\ \hline \hline
\multirow{4}{*}{\textsf{\textbf{Fr}}} &
      \textsc{KS} & 1349.8 & 2680.9 & 1951.6 & 1824.5 & 2968.5\\ \cline{2-7}
    & \textsc{CG} & 1.32$\times$ & 1.13$\times$ & 1.25$\times$ & 1.2$\times$ & 1.16$\times$\\ \cline{2-7}
    & \textsc{QRS} & 2.08$\times$ & 1.4$\times$ & 3.57$\times$ & 4.3$\times$ & 1.82$\times$\\ \cline{2-7}
    & \textsc{CQRS} & 6.5$\times$ & 8.19$\times$ & 9.57$\times$ & 12.23$\times$ & 11.95$\times$
    \\ \hline
\end{tabular}
}
\vspace{-0.3in}
\end{table}


% \vspace{-0.05in}
% \begin{table*}[!t]
% \caption{Average Execution Time for KickStarter-based (KS) method, and the speedup of CommonGraph Work-Sharing (CG), Q-Relevant Subgraph Work-Sharing (QRS), Q-Relevant Subgraph Concurrent (CQRS) for 64 Snapshots and 150,000 batch size.}
% \label{tab:time}
% \vspace{-0.1in}
% \small
% \centering
% {\renewcommand{\arraystretch}{1.2}
% \begin{tabular}{|l|l||c|c|c|c|c|} \hline
% \multicolumn{1}{|c|}{\textbf{\textsf{G}}} & \textbf{\textsf{Query Evaluation Algorithm}} & \textsf{\textbf{BFS}} & \textsf{\textbf{SSSP}} & \textsf{\textbf{SSWP}} & \textsf{\textbf{SSNP}} 
% & \textsf{\textbf{Viterbi}} 
% \\ \hline \hline

% \multirow{4}{*}{\textsf{\textbf{LJ}}} &
%     \textsc{\textbf{KS}: KickStarter-based Time} & 1562.15ms & 2292.2ms & 2249.61ms & 2183.08ms & 2677.5ms\\ \cline{2-7}
%     & \textsc{\textbf{CG}: CommonGraph Work-Sharing Speedup} & 1.17$\times$ & 1.24$\times$ & 1.32$\times$ & 1.39$\times$ & 1.07$\times$\\ \cline{2-7}
%     & \textsc{\textbf{QRS}: Q-Relevant Subgraph Work-Sharing Speedup} & 1.30$\times$ & 1.53$\times$ & 2.08$\times$ & 2.43$\times$ & 1.67$\times$\\ \cline{2-7}
%     & \textsc{\textbf{CQRS}: Q-Relevant Subgraph Concurrent Speedup} & 3.60$\times$ & 2.01$\times$ & 8.09$\times$ & 7.62$\times$ & 8.87$\times$
%     \\ \hline \hline
% \multirow{4}{*}{\textsf{\textbf{OR}}} &
%      \textsc{\textbf{KS}: KickStarter-based Time} & 905.94ms & 1361.72ms & 1499.15ms & 1330.5ms & 1310.22ms\\ \cline{2-7}
%     & \textsc{\textbf{CG}: CommonGraph Work-Sharing Speedup} & 1.11$\times$ & 1.06$\times$ & 1.44$\times$ & 1.17$\times$ & 1.08$\times$\\ \cline{2-7}
%     & \textsc{\textbf{QRS}: Q-Relevant Subgraph Work-Sharing Speedup} & 1.56$\times$ & 1.14$\times$ & 1.60$\times$ & 1.53$\times$ & 1.25$\times$\\ \cline{2-7}
%     & \textsc{\textbf{CQRS}: Q-Relevant Subgraph Concurrent Speedup} & 3.69$\times$ & 3.72$\times$ & 8.37$\times$ & 3.76$\times$ & 3.03$\times$
%     \\ \hline \hline
% \multirow{4}{*}{\textsf{\textbf{Wen}}} &
%      \textsc{\textbf{KS}: KickStarter-based Time} & 919.3ms & 1850.13ms & 1391.99ms & 994.27ms & 1007.37ms\\ \cline{2-7}
%     & \textsc{\textbf{CG}: CommonGraph Work-Sharing Speedup} & 1.23$\times$ & 1.42$\times$ & 1.37$\times$ & 1.46$\times$ & 1.10$\times$\\ \cline{2-7}
%     & \textsc{\textbf{QRS}: Q-Relevant Subgraph Work-Sharing Speedup} & 1.55$\times$ & 1.77$\times$ & 1.85$\times$ & 2.25$\times$ & 1.52$\times$\\ \cline{2-7}
%     & \textsc{\textbf{CQRS}: Q-Relevant Subgraph Concurrent Speedup} & 3.07$\times$ & 6.19$\times$ & 11.7$\times$ & 5.90$\times$ & 3.56$\times$
%     \\ \hline \hline
% \multirow{4}{*}{\textsf{\textbf{TW}}} &
%       \textsc{\textbf{KS}: KickStarter-based Time} & 971.84ms & 1997.67ms & 1993.26ms & 1304.92ms & 1774.78ms\\ \cline{2-7}
%     & \textsc{\textbf{CG}: CommonGraph Work-Sharing Speedup} & 1.06$\times$ & 1.86$\times$ & 1.61$\times$ & 1.20$\times$ & 1.03$\times$\\ \cline{2-7}
%     & \textsc{\textbf{QRS}: Q-Relevant Subgraph Work-Sharing Speedup} & 1.44$\times$ & 2.94$\times$ & 3.44$\times$ & 1.51$\times$ & 2.61$\times$\\ \cline{2-7}
%     & \textsc{\textbf{CQRS}: Q-Relevant Subgraph Concurrent Speedup} & 3.77$\times$ & 8.88$\times$ & 10.23$\times$ & 6.01$\times$ & 8.25$\times$
%     \\ \hline \hline
% \multirow{4}{*}{\textsf{\textbf{Fr}}} &
%       \textsc{\textbf{KS}: KickStarter-based Time} & ms & ms & ms & ms & ms\\ \cline{2-7}
%     & \textsc{\textbf{CG}: CommonGraph Work-Sharing Speedup} & $\times$ & $\times$ & $\times$ & $\times$ & $\times$\\ \cline{2-7}
%     & \textsc{\textbf{QRS}: Q-Relevant Subgraph Work-Sharing Speedup} & $\times$ & $\times$ & $\times$ & $\times$ & $\times$\\ \cline{2-7}
%     & \textsc{\textbf{CQRS}: Q-Relevant Subgraph Concurrent Speedup} & $\times$ & $\times$ & $\times$ & $\times$ & $\times$
%     \\ \hline
% \end{tabular}
% }
% \vspace{-0.15in}
% \end{table*}

\vspace{0.05in}
\subsection{Speedups}
% \vspace{-0.05in}
    In our experiments we compare the execution times of query evaluation for five input graphs and 5 benchmarks across a sequence of 64 snapshots. Between two consecutive snapshots there are 150,000 edge updates, half of them are additions and half are deletions. 
%This corresponds to roughly 0.01\% of the edges in LiveJournal. 
We present our results in Table~\ref{tab:time}. 

\textbf{KickStarter-based (KS)} baseline incremental approach does full computation on the first snapshot followed by repeated incremental processing of addition and deletion batches to get results for subsequent snapshots (see Figure~\ref{fig:evolving_approaches}(b)). The first row of Table~\ref{tab:time} for each benchmark displays the KS execution times in milliseconds. Our implementation is based on Risgraph as it represents the most optimized implementation of KickStarter.

For \textbf{CommonGraph (CG)} we present results for the best performing implementation~\cite{CommonGraph} which is the work-sharing approach and present its speedups over KS in the second row of Table~\ref{tab:time} for each benchmark. As shown in Table~\ref{tab:time}, we did not observe a substantial speedups when implementing from CommonGraph because the RisGraph system is highly optimized, and the deletion operation is not as expensive compared to prior systems. The third row of Table~\ref{tab:time} gives the speedup achieved by the \textbf{Q-Relevant Subgraph work-sharing (QRS)} method over the KickStarter-based method. The QRS method significantly reduces the size of the graph over which incremental computations are performed and hence leads to a considerable speedup in obtaining results for individual snapshots. \textbf{\emph{However, there is overhead associated with generating the QRS graph (overhead cost) which we take into account by including it in the total query evaluation time. }}\textbf{The QRS approach yields speedups ranging from 1.25$\times$ to 4.3$\times$ over KS.}

The fourth row of Table~\ref{tab:time} gives the speed-up achieved by \textbf{Concurrent Q-Relevant Subgraph (CQRS)} over the KickStarter-based implementation. In this technique, we concurrently add the additional delta batches to the QRG. \textbf{\emph{We have included the time for QRS generation (overhead cost) in query evaluation time.}} \textbf{CQRS yields speedups ranging from 2.01$\times$ and 12.23$\times$ over KS.}
%\textbf{On average, the QRS and the CQRS approaches yield 1.88$\times$ and 3.28$\times$ speedups over the KickStarter-based implementation, respectively.}

\begin{figure*}[!t]
\vspace{0.05in}

    \centering
    \includegraphics[width=0.99\linewidth]{diagrams/size_reduction64_1.pdf}
    \vspace{-0.15in}
    \caption{Remaining fraction of edges in QRS (blue bars) and fraction of vertices whose values are computed via incremental computation (red bars), both normalized with respect to edges and vertices in $G_{\cap}$. This experiment uses 64 snapshots and a batch size of 150,000 edge updates (half additions and half deletions).}
    \label{reduction}

    \centering
    \includegraphics[width=0.99\linewidth]{diagrams/foundideal64_1.pdf}
    \vspace{-0.15in}
    \caption{Percentage of vertices that are UVVs (purple bars) vs. percentage of UVVs successfully detected by our algorithm (green bars). This experiment uses 64 snapshots and a batch size of 150,000 edge updates (half additions and half deletions).}
    \label{accuracy}

    \centering  
    \includegraphics[width=0.99\linewidth]{diagrams/breakdown_all645_1.pdf}
    \vspace{-0.1in}
    \caption{The normalized execution times for Kickstarter-based (KS), CommonGraph Work-Sharing (CG), Q-Relevant Subgraph (QRS), and Concurrent Q-Relevant Subgraph (CQRS) are presented with a breakdown of QRS generation time (overhead cost) and query evaluation times for 64 snapshots and 150,000 edge updates (half additions and half deletions).}
    \label{overhead}
    \vspace{-0.15in}  
\end{figure*}




% \vspace{-0.1in}
\subsection{UVV Detection and QRS Generation}
After detecting UVVs, we are able to determine the fraction of vertices who results need to be updated incrementally. After removing incoming edges of UVV vertices, the fraction of edges remaining in QRS is also known. The percentage of vertices ranges from 0.3\% to 42\% and edges involved incremental computation ranges from 0.16\% to 32\% (see Figure~\ref{reduction}). This large reduction is due to high accuracy with which UVVs are detected (see Figure~\ref{accuracy}).

Next we present the portion of QRS and CQRS execution times spent on QRS generation. Figure~\ref{overhead} shows the execution times for all methods normalized to the KickStarter-based approach (KS). The red segments in the QRS and CQRS bars represent the QRS generation times. 
%These times are included in query evaluation times because QRS generation must be performed for each query. 
The QRS generation consists of four steps. First, we solve the query on the intersection graph. Then, we incrementally add the missing edges to the intersection graph to obtain the results on the union graph. We compare the two value arrays, and if we find the same value for a node in both the $G_{\cup}$ and $G_{\cap}$ arrays, we should remove the incoming edges for that vertex. However, due to the much larger number of matches than mismatches, we implement this step in reverse. Instead of removing the incoming edges of the matching values, we add incoming edges for the mismatching values. On average, the QRS generation time accounts for 18.45\%/56.01\% of the total execution time for QRS/CQRS.

%\subsection{UVV Detection Effectiveness}
%because a batch size of 75,000 has a lower impact on larger graphs, resulting in more stable values across different snapshots. Consequently, the QRS graph contains fewer edges. Even for our smallest graph, LJ, as shown in Figure~\ref{reduction}, the reduction is substantial, which underscores the effectiveness of our approach.



% \vspace{-0.125in}
\subsection{Sensitivity to Number of Snapshots}
% \vspace{-0.025in}
We studied the performance of UVV for 32, 64, and 128 snapshots. Figure~\ref{sensitivity-snapshots}(a) shows high speedups across varying number of snapshots, benchmarks, and graphs. Speedups for 64 and 128 snapshots are close. While the speedups for 32 snapshots are lower. This is because the resources available on our server are not fully utilized by 32 snapshots but are by 64 and 128 snapshots. We studied the sensitivity to different batch sizes for the LiveJournal graph. We chose LiveJournal for this experiment because it is our smallest graph, allowing us to observe the effects of changes in batch sizes more clearly. Figure~\ref{sensitivity-snapshots}(b) shows that larger batch sizes give lower speedups because they lead to more updates and fewer UVVs. 


\begin{figure}[!ht]
    \centering
    \includegraphics[width=0.925\linewidth]{diagrams/sensitivity_snapshots_1.pdf}
    \vspace{-0.175in}
    \caption{(a) Sensitivity to the number of snapshots (32, 64, and 128) with batches of 150K edge updates for the LiveJournal graph; (b) Sensitivity to batch sizes of 50K, 100K, 150K, 200K, and 250K for the LiveJournal and the SSSP algorithm.}
    \label{sensitivity-snapshots}
    \vspace{-0.25in}
\end{figure}