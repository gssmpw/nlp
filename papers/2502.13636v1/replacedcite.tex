\section{Related Work.}
The matching problem is a well studied problem in computer science.
 In this section we give a brief overview of matchings in graphs and hypergraphs.
\subparagraph{Matching in Graphs and Streaming.}
The polynomial-time complexity of matchings in graphs is one of the classical results in theoretical computer science____.
While Preis____ presented the first linear time~$\frac{1}{2}$-approximation, Drake and Hougardy____ show a simpler algorithm with the same approximation ratio by path growing (PGA) in linear time. 
 Pettie and Sanders____ developed a~$\frac{2}{3}-\varepsilon$ approximation with expected running time of~$\mathcal{O}(m\log \frac{1}{\varepsilon})$.
 The GPA algorithm by Maue and Sanders____ bridges the the gap between greedy and path searching algorithms, showing that a combination of both works best in practice.
 Birn~et~al.____ developed a parallel algorithm in the CREW PRAM model with $\frac{1}{2}$-approximation guarantee and $O(\log^2n)$ work.
Feigenbaum~et~al.____ present a $\frac{1}{6}$-approximation for the weighted matching in  the semi-streaming setting using a blaming based analysis.
Paz and Schwartzman____ give a~$\frac{1}{2+\varepsilon}$-approximation algorithm based on  keeping track of the dual solution in a semi-streaming setting.
Their algorithm keeps track of a dual solution, admitting only edges into the primal solution that are heavier on to the stack and updating the dual solution.
 The edges are taken from the stack in reverse order and those that do not violate the matching property are added to the solution.
 The resulting matching is not necessarily maximal.
 Ghaffari and Wajc____ provide a simpler proof of the bound.
 Ferdous~et~al.____ show empirically that the algorithm by Paz and Schwartzman can compete quality-wise with offline $\frac{1}{2}$-approximation algorithms like GPA, while requiring less memory and time.
Recently, Ferdous~et~al.____ presented two semi-streaming algorithms for the related weighted~$k$-disjoint matching problem, building upon previous results of  Paz and Schwartzman and Huang and Sellier____ on streaming~$b$-matching.

\subparagraph{Hypergraph Matching.}
Hazan et~al.____ prove that the maximum~$k$-set packing problem and, therefore, the matching problem on ~$d$-uniform hypergraphs can be poorly approximated, and there is no approximation within a factor of~$\mathcal{O}(d/\log d)$.
Dufosse~et~al.____ engineered reduction rules for special~$d$-partite,~$d$-uniform hypergraphs.
 They use Karp-Sipser rules and a scaling argument to select more hyperedges. 
There are several approximation results and local search approaches, most notably by Hurkens and Schrijver____ and Cygan____  with an approximation guarantee of~$\mathcal{O}(1/\left(\frac{d+1+\varepsilon}{3}\right))$.
Hanguir and Stein____ developed three distributed algorithms to compute  matchings in hypergraphs, trading off between quality guarantee and number of rounds needed to compute a solution.


For the weighted~$k$-set packing problem Berman____ introduces a local search technique.
Improving on these results, Neuwohner____ presents a way to guarantee an approximation threshold of~$\frac{2}{k}$. We are not aware of any practical implementations of these techniques.
For the more general weighted hypergraph~$b$-matching problem, 
Großmann~et~al.____ presented effective data-reduction rules and local search methods.
In the online setting, when hyperedges arrive in adversarial order, and one must  immediately decide to include the incoming hyperedge or not in the matching, Trobst and Udwani____ show that no (randomized) algorithm can have a competitive ratio better than~$\frac{2+o(1)}{d}$.

We are unaware of any studies or implementations for streaming hypergraph matching.