% This must be in the first 5 lines to tell arXiv to use pdfLaTeX, which is strongly recommended.
\pdfoutput=1
% \PassOptionsToPackage{table,xcdraw,dvipsnames}{xcolor}
% In particular, the hyperref package requires pdfLaTeX in order to break URLs across lines.

\documentclass[11pt]{article}

% Change "review" to "final" to generate the final (sometimes called camera-ready) version.
% Change to "preprint" to generate a non-anonymous version with page numbers.
\usepackage[preprint]{acl}

% Standard package includes
\usepackage{times}
\usepackage{latexsym}
\usepackage{textcomp}
% \usepackage{minted}
% For proper rendering and hyphenation of words containing Latin characters (including in bib files)
\usepackage[T1]{fontenc}
% For Vietnamese characters
% \usepackage[T5]{fontenc}
% See https://www.latex-project.org/help/documentation/encguide.pdf for other character sets

% This assumes your files are encoded as UTF8
\usepackage[utf8]{inputenc}

% This is not strictly necessary, and may be commented out,
% but it will improve the layout of the manuscript,
% and will typically save some space.
\usepackage{microtype}

% This is also not strictly necessary, and may be commented out.
% However, it will improve the aesthetics of text in
% the typewriter font.
\usepackage{inconsolata}

%Including images in your LaTeX document requires adding
%additional package(s)
\usepackage{graphicx}

\usepackage{booktabs}
\usepackage{float}
\usepackage{xcolor}
\usepackage{colortbl}

% \usepackage{xcolor}
\usepackage{multicol}
\usepackage{multirow}
\usepackage{makecell}
\usepackage{tabularx}
% \usepackage[dvipsnames]{xcolor}
\usepackage{subcaption}

\usepackage{listings}
% \definecolor{our_red}{HTML}{FFEBF5}
\definecolor{our_gray}{HTML}{E5E5E5}
\definecolor{our_red}{HTML}{F8D5D8}
\definecolor{our_green}{HTML}{E4F7C1}
% \definecolor{def_color}{HTML}{DDF4FC}
\definecolor{our_blue}{HTML}{C2F3FF}



\lstset{
    language=Python,
    basicstyle=\ttfamily\footnotesize, % Monospaced font, small size
    commentstyle=\color{gray}\ttfamily, % Keep comments in monospace (not italic)
    keywordstyle=\color{blue}, % Keywords in blue
    stringstyle=\color{red}, % Strings in red
    captionpos=b, % Caption position (if needed)
    showstringspaces=false % Remove visible spaces in strings
}

\usepackage{tcolorbox}

\definecolor{commentgray}{RGB}{130, 130, 130}
\definecolor{stringblue}{RGB}{32, 74, 135}
\definecolor{keywordpurple}{RGB}{106, 13, 173}
\definecolor{classname}{RGB}{173, 23, 23}
\definecolor{entitycolor}{RGB}{150, 50, 50}
\definecolor{bordergray}{RGB}{200, 200, 200}

\lstdefinestyle{gollie}{
    language=Python,
    basicstyle=\ttfamily\footnotesize,
    keywordstyle=\color{keywordpurple}\bfseries,
    commentstyle=\color{commentgray},
    stringstyle=\color{stringblue},
    emph={BeBorn, LifeEvent, Entity, List}, emphstyle=\color{classname}, emphstyle=\color{classname},
    morekeywords={dataclass, List},
    frame=single,
    framesep=5pt,
    rulecolor=\color{bordergray},
    showstringspaces=false,
    breaklines=true,
    breakindent=0pt,
    breakatwhitespace=false
}
\newcommand{\highlight}[2]{%
    \begingroup
    \definecolor{hlcolor}{HTML}{#1}%
    \sethlcolor{hlcolor}%
    \hl{#2}%
    \endgroup
}


\definecolor{classgreen}{RGB}{34,139,34}
\definecolor{bluekeyword}{RGB}{0,0,255}
\definecolor{purpletype}{RGB}{106,13,173}
\definecolor{blackbold}{RGB}{0,0,0} 

% \lstdefinestyle{custompython}{
%     language=Python,
%     basicstyle=\ttfamily\footnotesize,
%     keywordstyle=[1]\color{classgreen}\bfseries,
%     keywordstyle=[2]\color{blackbold}\bfseries,
%     keywordstyle=[3]\color{purpletype},
%     emph={BeBorn, LifeEvent, Extradite, JusticeEvent}, emphstyle=\color{bluekeyword},
%     morekeywords={[1]class},
%     morekeywords={[2]@dataclass},
%     morekeywords={[3]str, List},
%     showstringspaces=false,
%     breaklines=true
% }

% \lstdefinestyle{custompython}{
%     language=Python,
%     basicstyle=\ttfamily\footnotesize,  % Increased font size and made text bold
%     keywordstyle=[1]\color{classgreen}\bfseries, % Class keyword style
%     keywordstyle=[2]\color{magenta}\bfseries, % Changed @dataclass to red for visibility
%     keywordstyle=[3]\color{purpletype}\bfseries, % Made List bold
%     emph={BeBorn, LifeEvent, Extradite, JusticeEvent}, emphstyle=\color{bluekeyword},
%     morekeywords={[1]class},
%     morekeywords={[2]@dataclass},
%     morekeywords={[3]str, List}, % List now follows the bolded type pattern
%     showstringspaces=false,
%     breaklines=true
% }

% \lstdefinestyle{custompython}{
%     language=Python,
%     basicstyle=\ttfamily\footnotesize,  
%     keywordstyle=[1]\color{blue}\bfseries, % Class keyword (e.g., class) in blue
%     keywordstyle=[2]\color{cyan}\bfseries, % Decorators (@dataclass) in dark green
%     keywordstyle=[3]\color{violet}\bfseries, % Type hints (e.g., List, str) in violet
%     emph={BeBorn, LifeEvent, Extradite, JusticeEvent}, emphstyle=\color{magenta}, % Custom class names in purple
%     morekeywords={[1]class, def, return, if, else, elif, for, while, try, except, import, from, as, with},
%     morekeywords={[2]@dataclass}, 
%     morekeywords={[3]str, List, Dict, Tuple, Set}, 
%     stringstyle=\color{violet}, % Strings in red
%     commentstyle=\color{darkgray}, % Comments in dark gray
%     showstringspaces=false,
%     breaklines=true
% }

\lstdefinestyle{custompython}{
    language=Python,
    basicstyle=\ttfamily\footnotesize,  
    backgroundcolor=\color{gray!5}, % Light gray background
    frame=single, % Add frame around code
    rulecolor=\color{gray}, % Frame color
    keywordstyle=[1]\color{blue}\bfseries, % Class keyword (e.g., class) in blue
    keywordstyle=[2]\color{cyan}\bfseries, % Decorators (@dataclass) in cyan
    keywordstyle=[3]\color{violet}\bfseries, % Type hints (e.g., List, str) in violet
    emph={BeBorn, LifeEvent, Extradite, JusticeEvent}, emphstyle=\color{magenta}, % Custom class names in magenta
    morekeywords={[1]class, def, return, if, else, elif, for, while, try, except, import, from, as, with},
    morekeywords={[2]@dataclass}, 
    morekeywords={[3]str, List, Dict, Tuple, Set}, 
    stringstyle=\color{violet}, % Strings in violet
    commentstyle=\color{darkgray}, % Comments in dark gray
    numberstyle=\tiny\color{gray}, % Line numbers in gray
    stepnumber=1, % Line number step size
    showstringspaces=false,
    breaklines=true, % Break long lines
    captionpos=b % Position caption at the bottom
}



\usepackage{listings} % For code formatting
% \usepackage{xcolor}  % For color formatting (optional)
\usepackage{graphicx} % For figure scaling and adjustments (optional)

% \lstdefinestyle{customjson}{
%   language=JSON,
%   basicstyle=\ttfamily\small,
%   backgroundcolor=\color{gray!5}, % Lighter gray background (change from black!10 to gray!5)
%   % keywordstyle=\color{blue}\bfseries, % Blue for keywords (e.g., "Event Definition")
%   stringstyle=\color{violet}, % Orange for strings
%   commentstyle=\color{cyan}, % Dark green for comments
%   morecomment=[l][\color{magenta}]{//}, % Purple for line comments
%   numberstyle=\tiny\color{gray}, % Gray numbers for line numbering
%   numbers=left, % Add line numbers
%   stepnumber=1, % Line number step size
%   breaklines=true, % Break long lines
%   frame=single, % Add frame around code
%   rulecolor=\color{gray}, % Frame color
%   captionpos=b % Position caption at the bottom
% }
\lstdefinelanguage{json}{
  basicstyle=\ttfamily\small,
  % numbers=left,
  % numberstyle=\tiny\color{gray},
  % stepnumber=1,
  % numbersep=5pt,
  showstringspaces=false,
  breaklines=true,
  literate=
    *{0}{{{\color{orange}0}}}{1}
     {1}{{{\color{orange}1}}}{1}
     {2}{{{\color{orange}2}}}{1}
     {3}{{{\color{orange}3}}}{1}
     {4}{{{\color{orange}4}}}{1}
     {5}{{{\color{orange}5}}}{1}
     {6}{{{\color{orange}6}}}{1}
     {7}{{{\color{orange}7}}}{1}
     {8}{{{\color{orange}8}}}{1}
     {9}{{{\color{orange}9}}}{1}
     {:}{{{\color{orange}{:}}}}{1}
     {,}{{{\color{orange}{,}}}}{1}
}

\lstdefinestyle{customjson}{
  language=JSON,
  basicstyle=\ttfamily\small,
  backgroundcolor=\color{gray!5}, % Lighter gray background
  keywordstyle=\color{black}, % Removed this line to prevent keyword highlighting
  stringstyle=\color{violet}, % Strings in violet
  commentstyle=\color{cyan}, % Dark cyan for comments
  morecomment=[l][\color{magenta}]{//}, % Purple for line comments
  numberstyle=\tiny\color{gray}, % Gray numbers for line numbering
  stepnumber=1, % Line number step size
  breaklines=true, % Break long lines
  frame=single, % Add frame around code
  rulecolor=\color{gray}, % Frame color
  captionpos=b % Position caption at the bottom
}

% \usepackage{xcolor}
\usepackage{soul}
\usepackage{textcomp}
% \usepackage{unicode-math}
\usepackage{pifont}
\usepackage{stfloats} 
\usepackage{float}
\usepackage{amssymb}
\newcommand{\zy}[1]{\textcolor{red}{#1}}
\newcommand{\zyc}[1]{\textcolor{red}{[ZY:] #1}}
\newcommand{\sriv}[1]{\textcolor{blue}{[SS:] #1}}
\newcommand{\srivc}[1]{\textcolor{blue}{#1}}
\newcommand{\sweta}[1]{\textcolor{ForestGreen}{[SP:] #1}}
\newcommand{\swetac}[1]{\textcolor{ForestGreen}{#1}}

% If the title and author information does not fit in the area allocated, uncomment the following
%
%\setlength\titlebox{<dim>}
%
% and set <dim> to something 5cm or larger.

\title{Instruction-Tuning LLMs for Event Extraction with Annotation Guidelines}

% Author information can be set in various styles:
% For several authors from the same institution:
% \author{Author 1 \and ... \and Author n \\
%         Address line \\ ... \\ Address line}
% if the names do not fit well on one line use
%         Author 1 \\ {\bf Author 2} \\ ... \\ {\bf Author n} \\
% For authors from different institutions:
% \author{Author 1 \\ Address line \\  ... \\ Address line
%         \And  ... \And
%         Author n \\ Address line \\ ... \\ Address line}
% To start a separate ``row'' of authors use \AND, as in
% \author{Author 1 \\ Address line \\  ... \\ Address line
%         \AND
%         Author 2 \\ Address line \\ ... \\ Address line \And
%         Author 3 \\ Address line \\ ... \\ Address line}



%%%%%% Equal Contrinution %%%%%%%%%
\author{
  Saurabh Srivastava$^*$, 
  Sweta Pati$^*$, 
  \textbf{Ziyu Yao}\\
  George Mason University, Fairfax, VA \\
  \{ssrivas6, spati, ziyuyao\}@gmu.edu
}
%%%%%% Equal Contrinution %%%%%%%%%


% \author{
%   Saurabh Srivastava, 
%   Sweta Pati, 
%   \textbf{Ziyu Yao}\\
%   George Mason University, Fairfax, VA \\
%   \{ssrivas6, spati, ziyuyao\}@gmu.edu
% }

%\author{
%  \textbf{First Author\textsuperscript{1}},
%  \textbf{Second Author\textsuperscript{1,2}},
%  \textbf{Third T. Author\textsuperscript{1}},
%  \textbf{Fourth Author\textsuperscript{1}},
%\\
%  \textbf{Fifth Author\textsuperscript{1,2}},
%  \textbf{Sixth Author\textsuperscript{1}},
%  \textbf{Seventh Author\textsuperscript{1}},
%  \textbf{Eighth Author \textsuperscript{1,2,3,4}},
%\\
%  \textbf{Ninth Author\textsuperscript{1}},
%  \textbf{Tenth Author\textsuperscript{1}},
%  \textbf{Eleventh E. Author\textsuperscript{1,2,3,4,5}},
%  \textbf{Twelfth Author\textsuperscript{1}},
%\\
%  \textbf{Thirteenth Author\textsuperscript{3}},
%  \textbf{Fourteenth F. Author\textsuperscript{2,4}},
%  \textbf{Fifteenth Author\textsuperscript{1}},
%  \textbf{Sixteenth Author\textsuperscript{1}},
%\\
%  \textbf{Seventeenth S. Author\textsuperscript{4,5}},
%  \textbf{Eighteenth Author\textsuperscript{3,4}},
%  \textbf{Nineteenth N. Author\textsuperscript{2,5}},
%  \textbf{Twentieth Author\textsuperscript{1}}
%\\
%\\
%  \textsuperscript{1}Affiliation 1,
%  \textsuperscript{2}Affiliation 2,
%  \textsuperscript{3}Affiliation 3,
%  \textsuperscript{4}Affiliation 4,
%  \textsuperscript{5}Affiliation 5
%\\
%  \small{
%    \textbf{Correspondence:} \href{mailto:email@domain}{email@domain}
%  }
%}

\begin{document}
\maketitle

% %%%%%% Equal Contrinution %%%%%%%%%
\def\thefootnote{*}\footnotetext{The first two authors contribute equally.}\def\thefootnote{\arabic{footnote}}

% %%%%%% Equal Contrinution %%%%%%%%%
\begin{abstract}
% some title suggestions
% 1. Schemas Are a Map, but Guidelines Are the Compass: Rethinking LLM Event Extraction

% 2. LLMs Are Not Mind Readers: Why Annotation Guidelines Are the Key to Event Extraction

% 3. LLMs Don’t Read Minds—They Need Guidelines: Unlocking Better Event Extraction

% 4. You Can’t Extract What You Don’t Understand: Why LLMs Need Guidelines, Not Just Schema

% \zyc{The abstract tells a quite different story compared to Intro. Change one of them to make the writing consistent.}
In this work, we study the effect of annotation guidelines---textual descriptions of event types and arguments, when instruction-tuning large language models for event extraction. We conducted a series of experiments with both human-provided and machine-generated guidelines in both full- and low-data settings. Our results demonstrate the promise of annotation guidelines when there is a decent amount of training data and highlight its effectiveness in improving cross-schema generalization and low-frequency event-type performance.\footnote{Our source code and datasets are available at \href{https://github.com/salokr/LLaMA-Events}{https://github.com/salokr/LLaMA-Events}.}


\end{abstract}

\section{Introduction}
Generative imitation-based policies are an increasingly powerful way to learn low-level robot behaviors from multimodal\footnote{Here, multimodality only refers to the training data's action distribution.} expert demonstrations~\citep{chi2024diffusionpolicy, fu2024mobile,zhao2023learningfinegrainedbimanualmanipulation}. 
Despite their impressive ability to learn diverse behaviors directly from high-dimensional observations, these policies still degrade in nuanced and unexpected ways at runtime.  
For example, consider the robot in the left of Figure~\ref{fig:front-fig} that must pick up a mug from the table. 
At training time, the generative policy learns a distribution over useful interaction modes such as grasping the cup by different parts (e.g., handle, lip and interior, etc.) shown in wrist camera photo in Figure~\ref{fig:front-fig}.

However, at runtime, the policy exhibits a range of degradations, from complete task failures (such as the robot knocking down the cup during grasping, shown in 
the center of Figure~\ref{fig:front-fig}), to inappropriate behaviors that are misaligned with the deployment context or preferences of an end-user (such as the robot placing its gripper inside of a cup of water when serving a guest shown in the middle of Figure~\ref{fig:front-fig}). 
While a common mitigation strategy involves re-training the policy via more demonstrations~\citep{shafiullah2024supervised} or interventions~\citep{ross2010efficient, liumulti},  runtime failures are not always an indication that the base policy is fundamentally incapable of producing the desired behavior. 
In fact, the base policy may already contain the ``right'' behavior mode within its distribution (e.g., grasping the cup by the handle), but due to putting too much probability mass on an undesired mode, the robot does not reliably choose the correct action plan at runtime. 

Runtime policy steering \citep{nakamoto2024steering,wang2024inference} offers an elegant solution to this mode-selection problem. 
By using an external \textit{verifier}
to select candidate plans proposed by an imperfect generative policy, the robot's behavior can be ``steered'' at runtime without any need for re-training.
Despite the initial successes demonstrated by the policy steering paradigm, the question still remains of how to fully unlock autonomous policy steering in the open world when the robot's environment, task context, and base policy's performance are constantly changing.

Policy steering can be approached via the stochastic model-predictive control framework of modern control theory, which decomposes the optimal action selection (i.e. \textit{generation}) problem of runtime policy steering into \textit{(a)} \textit{predicting} the outcomes of a given action plan and \textit{(b)} \textit{verifying} how well they align with user intent. However, this approach is only feasible when a physics-based, low-dimensional dynamics model is available for outcome prediction and a well-defined reward function can be specified for verification. In open-world environments, both of these requirements are challenging to fulfill due to the complexity of dynamics modeling and the difficulty of hand-crafting rewards to evaluate nuanced task requirements \citep{hadfield2017inverse}.

Our core idea is to leverage world models, which are well-suited for predicting action outcomes in open world settings, and VLMs, which have great potential as verifiers due to their commonsense reasoning abilities, to develop a divide-and-conquer approach to open-world policy steering. However, doing so naively is challenging as world models and VLMs operate on fundamentally different representations of reality.


To address this concern, we propose  \textbf{FOREWARN}: \textbf{F}iltering \textbf{O}ptions via \textbf{RE}presenting \textbf{W}orld-model \textbf{A}ction \textbf{R}ollouts via \textbf{N}arration.
To predict challenging action outcomes (e.g., interaction dynamics of a manipulator and a deformable bag), we use state-of-the-art world models
\citep{liumulti,wu2023daydreamer} to predict lower-dimensional latent state representations from high-dimensional observation-action data (shown in orange in the center of Figure~\ref{fig:front-fig}). 
To critique behavior outcomes under nuanced task specifications (e.g., ``Serve the cup of water to the guest''), we leverage vision-language models (VLMs)~\cite{dubey2024llama,openai2024gpt4technicalreport} as our open-world verifiers (shown in green in the center of Figure~\ref{fig:front-fig}).
Importantly, we demonstrate that \textit{aligning} the VLM to reason directly about the predicted latent states from the world model enables it to understand fine-grained outcomes that it cannot directly predict zero-shot nor understand from image reconstructions. 
Ultimately, this alignment step enables our ``VLM-in-the-loop'' policy steering approach to interpret action plans as behavior narrations and select high-quality plans by reasoning over those narrations even under novel task descriptions
(shown on the right of Figure~\ref{fig:front-fig}). 


We evaluate \textbf{FOREWARN} on a suite of robotic manipulation tasks, demonstrating how it can robustly filter proposed action plans to match user intent and task goals even when faced with variations not seen during training. In summary, our main contributions are:
\begin{itemize}
    \item Formalizing runtime policy steering a stochastic model-predictive control problem, revealing the \textit{generation-verification gap} \citep{Godel, cook2023complexity} and where state-of-the-art models have maximal potential to shine. 
    \item A latent space alignment strategy that enables a VLM to more reliably verify action plan outcomes predicted by a low-level, action-conditioned world model. 
    \item A novel, fully-autonomous policy steering framework that improves a base generative imitation-based policy by over $30\%$, even in novel task descriptions. 
    \item Extensive experiments in hardware showing that our latent-aligned VLM approach outperforms (by $\sim40\%$) altnerative VLM approaches that do \textit{not} decouple the prediction of outcomes from verification. 
\end{itemize}


\section{Approach}\label{sec:approach}
% \begin{tcolorbox}[colback=lightgray!10, colframe=black, title={Research Aim}]
% We investigate how structured annotation guidelines impact instruction-tuned LLMs for event extraction, assessing their effectiveness across training sizes, guideline sources, and cross-schema generalization.
% \end{tcolorbox}
% \vspace{-10pt}

% In this section, we formalize the event extraction task using LLMs (\textsection \ref{sec:formulation}) and detail the guideline generation procedure used to prompt LLMs (\textsection \ref{sec:guideline_generation}).

\subsection{Task Formulation}
\label{sec:formulation}
% \zyc{describe the instruction tuning formulation for EE. describe instructions with annotation guidelines. referring to Fig~\ref{fig:intro}.}
Given an input sentence \( X \), the goal of EE is to extract the structured event information $Y$ from the sentence, adhering to predefined schema constraints $\mathcal{E}$. The extraction task consists of (1) \textbf{Trigger Extraction}, which localizes an event trigger span and classifies its event type, and (2) \textbf{Argument Extraction}, where the task is to identify spans in \( X \) that serve as argument roles within the extracted event instance.  

When an autoregressive LLM is tasked with EE, the extraction of event instances is formulated in a generative way, with the LLM generating a sentence describing the extracted event instances. Specifically, the prompt to the LLM is defined as  $P = [I \oplus {E}_e \oplus X]$, where \(\oplus\) is the concatenation operation, \( I \) represents the {task instruction}, which specifies the structured output format and task definition, and \( {E}_e \in \mathcal{E}\) denotes the {event schema} of an interested type $e$ from a predefined set $\mathcal{E}$.
% which defines the event type \( e \) along with its argument roles. 


% Fine-tuning is performed via {causal language modeling}, leveraging a {next-token prediction objective} where the model generates structured event representations autoregressive. 
Let \( \mathcal{D} = \{(e_i, X_i, Y_i)\}_{i=1}^{N} \) denote a dataset of annotated event examples, where each \( X_i \) corresponds to a prompt instance \( P_i \) for the interested event type $e_i$. The objective function of instruction tuning for EE is as follows: $\mathcal{L}(\mathcal{D}; \theta) = - \sum_{i} \sum_{j} \log p_{\theta} (Y_{ij} \mid P_i, Y_{i,<j})$,
% Given a {pretrained model} parameterized by \( \theta \), we optimize the {conditional log-likelihood}:
% \begin{equation}
%     \mathcal{L}(\mathcal{D}; \theta) = - \sum_{i} \sum_{j} \log p_{\theta} (Y_{ij} \mid P_i, Y_{i,<j})
% \end{equation}
where \( Y_{i,<j} \) represents previously generated tokens in the structured output sequence, ensuring an autoregressive formulation. 

Existing work identified the structure of EE outputs to be critical~\cite{jiao-etal-2023-instruct, wang2023code4struct}. In particular, \citet{wang2023code4struct} found that formulating the EE output in a \emph{code format} can {take advantage of Programming Language features such as inheritance and type annotation to introduce external knowledge or add constraints.} In our work, we follow the same formatting strategy and represent the EE task as a code generation problem. Specifically, the event schema ${E}_e$ is represented as a Python class; accordingly, every extracted event instance is represented as a Python object of the corresponding event class. When there are multiple event instances implied in the input $X$, a list of Python objects will be generated; when there is no event specified in $X$, we expect an empty Python list to be the model output. An example is shown in Figure~\ref{fig:overview}.

% During training, we partition the sequence into a {source context} (\( P_i \)) and a {target output} (\( Y_i \)), applying the loss function {only to target tokens} (\emph{label loss}).
During training, we provide only the ground-truth event schema in the prompt; when the text input $X$ does not include any event, a random event schema will be chosen. 
At inference time, given a test instance $X$, we pair the input with every possible event type in the schema set $\mathcal{E}$, prompt the LLM to extract any implied event instances, and perform model evaluation based on the aggregated extraction outputs. As such, a well-performing LLM needs both extract the complete event instances and avoid events that are not indicated in $X$.
% the model generates output tokens autoregressively, until an end-of-sequence (EOS) token is generated or a predefined maximum length is reached.



% \subsection{Generating Annotation Guidelines with LLMs}
\subsection{Instruction-Tuning LLMs with Annotation Guidelines}
\label{sec:guideline_generation}
% \zyc{describe the guideline variants. need one or more figures of prompts + examples.}

Recent work by \citet{sainz2024gollie} demonstrated the effectiveness of integrating annotation guidelines in the code-format instructions of IE tasks. Specifically, when describing the event type schema $E_e$, a textual description is added to the event type and each of its argument roles (Figure~\ref{fig:overview}). As such, the LLM is expected to more easily understand the meaning of the event type while being instructed to extract any occurring events from the input $X$. While \citet{sainz2024gollie} evaluated annotation guidelines in the broad IE task, their main focus has been on Named Entity Recognition, rather than the complicated EE task. 
% However, since IE encompasses a broad range of sub-tasks, their analysis of guideline usability in EE remains underexplored. 
Furthermore, their approach assumed the availability of pre-existing human-curated guidelines, an assumption that may not always hold in real-world applications. To bridge this gap, we explore methods to automatically generate annotation guidelines and assess their effectiveness in comparison to human-authored ones.\footnote{Human-written guidelines for ACE05 are available \href{https://www.ldc.upenn.edu/sites/www.ldc.upenn.edu/files/english-events-guidelines-v5.4.3.pdf}{here}.}
\begin{table*}[t!]
\centering
\resizebox{.95\linewidth}{!}{%
\begin{tabular}{>{\centering\arraybackslash}p{3.8cm}p{17cm}}
    \toprule
    \multicolumn{2}{c}{\textbf{Examples of Annotation Guidelines for Event Type: Extradite (ACE05)}} \\
    % \midrule
    %     {\textbf{\textsc{Event Schema}}\newline(No Guideline)} &
    %     \lstinputlisting[style=custompython, aboveskip=-5pt, belowskip=-5pt]{tables/codes/beborn.py}\\
    \midrule
    \textbf{\textsc{Guideline-H}}\newline \small{Avg. Length - 107.67 tokens} & 
    {\textbf{Event Type:} {{An EXTRADITE Event occurs whenever a PERSON is sent by a state actor from one PLACE to another place for the purposes of legal proceedings there. }}\newline
    \textbf{Arguments:} \newline 
    - \textsc{Agent}: The extraditing agent.  \newline 
    - \textsc{Person}: The person being extradited.}\\
    \midrule
    \textbf{\textsc{Guideline-P}}  \newline \small{Avg. Length - 163.87 tokens}&
    {\textbf{Event Type:} The Extradition event refers to the formal process where one jurisdiction delivers a person accused \textbf{(...)} \highlight{D5E8D4}{The event can be triggered by terms such as `extradition' \textbf{(...)}} \highlight{BCD4E6}{Edge cases include situations where the term `extradition' is used metaphorically or in a non-legal context.} \newline
    \textbf{Arguments:} \newline 
    - \textsc{Agent}:\textbf{(...)} the agent is the organization or authority \textbf{(...)}. \highlight{D5E8D4}{Examples include `court', `government', \textbf{(...)}} \newline 
    - \textsc{Person}: \textbf{(...)} individual who is being transferred to another jurisdiction. \highlight{D5E8D4}{Examples are `she', \textbf{(...)}}
    }\\
    \midrule
    \textbf{\textsc{Guideline-PN}} \newline \small{Avg. Length - 285.24 tokens}&
    \textbf{Event Type:} The event is triggered by the formal request \textbf{(...)} for legal reasons. \highlight{D5E8D4}{Triggers such as `extradition' are indicative of this event type}, \highlight{FACEC6}{not `Transport' which involves general movement without legal context.} \\
    & \textbf{Arguments:} \\
    & - \textsc{Agent}: The agent is responsible for the legal and procedural aspects of the extradition,\textbf{(...)}. \highlight{D5E8D4}{An example is `the original court' \textbf{(...)}} \\
    & - \textsc{Person}: \textbf{(...)} one who is being moved from one place to another under legal authority. \highlight{D5E8D4}{For example, `he' \textbf{(...)}} \\
    \midrule
    \textbf{\textsc{Guideline-PS}} \newline \small{Avg. Length - 159.79 tokens}&
    \textbf{Event Type:} \textbf{(...)} person being moved to a new jurisdiction \textbf{(...)}. \highlight{FACEC6}{This differs from events like `TrialHearing' or `Convict', which focus on the legal proceedings and outcomes within a single jurisdiction.} \\
    & \textbf{Arguments:} \\
    & - \textsc{Agent}: \textbf{(...)} \highlight{BCD4E6}{Edge cases may include international organizations or coalitions} \textbf{(...)} \highlight{D5E8D4}{such as the U.N. \textbf{(...)}} \\
    & - \textsc{Person}: \highlight{FACEC6}{Unlike the `defendant' in events like `TrialHearing' or `Convict', the person in the `Extradite' event is specifically being transferred for legal proceedings or punishment.} \\
    \midrule
    \textbf{\textsc{Guideline-PN-Int}} \newline \small{Avg. Length - 439.94 tokens}&
    \textbf{(...)} \highlight{D5E8D4}{Key triggers include terms like `extradite', `extradition', and `extraditing'.} \highlight{FACEC6}{It is distinct from events like `ArrestJail' and `ReleaseParole', as it specifically involves \textbf{(...)}} \\
    & \textbf{Arguments:} \\
    & - \textsc{Agent}: The agent \textbf{(...)} typically a legal or governmental body. \highlight{D5E8D4}{Examples include `court', `government'\textbf{(...)}} \\
    & - \textsc{Person}: The person is the individual being extradited, the subject of the legal transfer. \highlight{D5E8D4}{Examples include `she', `him', and `her'.} \\
    \midrule
    \textbf{\textsc{Guideline-PS-Int}} \newline \small{Avg. Length - 434.64 tokens}&
    The 'Extradite' event involves the legal transfer of a person \textbf{(...)}. \highlight{FACEC6}{It is distinct from events like `ArrestJail', \textbf{(...)}, and `ReleaseParole' or `Pardon', \textbf{(...)}} \\
    & \textbf{Arguments:} \\
    & - \textsc{Agent}: The agent is the entity \textbf{(...)} \highlight{D5E8D4}{such as a court, government, or police department.} \highlight{BCD4E6}{This entity ensures the transfer is conducted according to legal protocols \textbf{(...)}} \\
    & - \textsc{Person}: \textbf{(...)} They are the central figure in the extradition process, \highlight{FACEC6}{distinct from a `defendant' in other legal events, \textbf{(...)}}\highlight{BCD4E6}{ This may include high-profile individuals or groups.} \\

    \bottomrule
    \end{tabular}

    }
    \caption{Examples of annotation guidelines for the event type \texttt{Extradite} from ACE05. Due to space limits, only \texttt{agent} and \texttt{person} were shown for arguments, and only 1 out of the 5 guideline samples were shown for \textbf{P}, \textbf{PN}, and \textbf{PS}.
    % For arguments, only \texttt{agent} and \texttt{person} were shown due to space limits.
    % \highlight{FACEC6}{Light red} highlights distinctions from other event types, \highlight{D5E8D4}{Light green} highlights example mentions, and \highlight{BCD4E6}{Light blue} highlights edge cases. 
    We highlight \highlight{FACEC6}{distinctions from other event types}, \highlight{D5E8D4}{example mentions}, and \highlight{BCD4E6}{edge cases} in guidelines.
    % On average, we note that the \textbf{-Adv} versions are the longest generations.
    % \zyc{add a fig showing the siblings of Extradite and "Transport" as one neg ET? From the truncated examples, the Adv is not obviously longer than others. Do we want to clarify its length for analysis purposes?} 
    }
    \label{tab:guideline-examples}
\end{table*}


To develop a scalable and cost-effective approach for guideline generation, we employ a reverse engineering strategy, leveraging both annotated event examples and the strong generative capabilities of LLMs. As illustrated in Figure X, we construct a guideline generation prompt for each event type $e$ by providing a few annotated examples $\{(X_i, Y_i)\}$ demonstrating the existence or non-existence of event instance of type $e$, and then prompt an LLM (GPT-4o in our experiment) to generate annotation guidelines for $e$. In total, we experimented with five variants of machine-generated guidelines:
% we construct guideline generation prompts by providing 10 annotated examples per event type $e$, using them to generate 5 distinct definitions for the event type and its corresponding argument roles. 
% Formally, each guideline generation prompt consists of a task instruction, an input text instance, and its corresponding event annotations $(t_i, a_i)$. Additionally, we investigate refining guidelines through the inclusion of annotations from different event types $\hat{e}$  and semantically similar event types, incorporating negative constraints (e.g., ``what not to do'') to enhance model robustness against corner cases.
% In our experiments, we prompt GPT-4 to elicit the guidelines \(E\) across five distinct settings, each designed to introduce different levels of specificity and contrast. 
\textbf{(1) Guideline-P:} We prompt the LLM with 10 positive annotated examples of type $e$ to generate the annotation guidelines. Inspired by \citet{sainz2024gollie}, we sample 5 distinct guidelines for each event type, which can be used during the model training to ensure that the model is exposed to multiple rephrasings of the guidelines rather than memorizing and overfitting to a specific one.
% The LLM generates guidelines based on 10 positive event instances without additional constraints.
\textbf{(2) Guideline-PN (Positive + Negative Examples):} In addition to 10 positive event annotations, we also provide 15 negative annotations where the input $X$ does not imply event instances of type $e$. Similarly, we prompt the LLM to generate 5 distinct guidelines for each event type.
% Extends Guideline-P by incorporating 15 additional samples from different event types, introducing edge cases and exceptions.
\textbf{(3) Guideline-PS (Positive + Sibling Events):} Similar to Guideline-PN, we prompt the LLM with both positive and negative event annotations. However, the negative annotations are selected from the sibling event types of the target type $e$ (e.g., Arrest vs. Jail), as defined by the event ontology. We hypothesize that the critical challenge for EE lies in distinguishing between sibling event types; hence, an instructed LLM can benefit from following annotation guidelines that particularly emphasize the difference between sibling event types. As in the earlier variants, we generate 5 guideline samples per event type.
\textbf{(4) Guideline-PN-Int} and \textbf{(5) Guideline-PS-Int:} Finally, we create two more variants that \underline{Int}egrate the 5 diverse guideline samples from Guideline-PN and Guideline-PS into a comprehensive one, respectively. 
% Intuitively, as these variants consolidate the diverse ways of describing each event schema, we expect them to be more effective guidelines for instructing LLMs in EE tasks. 
Examples of the 5 guideline variants are shown in Table~\ref{tab:guideline-examples}.
% \sriv{Should we highlight why 15? and 10? For negative samples: we picked 15 inspired by Degree}
% \zyc{Point people to Appendix for the prompt templates.}
The prompt templates used for generating guidelines and example generations
% across all settings including the 5 diverse variations and the consolidated versions 
are provided in Appendix\hyperref[sec:prompt-design]{~C} and~\ref{sec:app_dd}, respectively.
% Augments Guideline-P by adding 15 samples from semantically similar event types (e.g., Arrest vs. Jail), facilitating a more nuanced differentiation between event types.
% Finally, we create two new prompts by gathering 5 different distinct generations from Guideline-PN and Guideline-PS stages to generate \textbf{4) Guideline-PN-Adv}, and \textbf{(5) Guideline-PS-Adv}, respectively.


\section{Experiments}

\subsection{Experimental Setup} 
\paragraph{Datasets.}
We perform experiments on two standard EE datasets: \textbf{ACE05} \cite{doddington-etal-2004-automatic} and \textbf{RichERE} \cite{song-etal-2015-light}. Both of them exhibit fine-grained event distinctions, and RichERE includes sparser event annotations (i.e., fewer event-labeled sentences), which makes it more challenging. Moreover, RichERE does not come with human-written annotation guidelines. Datasets were split following the TextEE benchmark~\cite{huang2024textee} and then converted to code format automatically by our scripts. 
% We present additional details and data sampling strategies in Appendix~\ref{sec:appendix_preprocessing}.


\paragraph{Evaluation.} 
Following prior work~\cite{huang2024textee}, we evaluate the model on four F1 metrics:
\textbf{(1) Trigger Identification (TI)}, which measures correct trigger span extraction, \textbf{(2) Trigger Classification (TC)}, which additionally requires event-type correctness, \textbf{(3) Argument Identification (AI)}, which ensures correct argument role association with the predicted trigger, and \textbf{(4) Argument Classification (AC)}, which further requires role-type correctness and is thus the most comprehensive metric on a model's EE performance. When evaluating the model on the Guideline-P, PN, and PS variants, one guideline is randomly selected each time.

As a side benefit of representing events in a structured code format, we can easily evaluate an extracted event instance by directly instantiating its corresponding Python object based on the event schema's Python class definitions, checking if the object is valid (e.g., missing arguments or including hallucinated arguments) and comparing it with the ground truth. This code-based evaluation thus prevents the tedious string-matching process adopted in prior work~\cite{li-etal-2021-document}. 
% We include details of our code-based EE evaluation in Appendix~\ref{sec:app_additional_details} and will release our scripts as community resources in the future.

\paragraph{Model Training.} We experimented with the {LLaMA-3.1-8B-Instruct} model~\cite{grattafiori2024llama3herdmodels}, selected for its demonstrated proficiency in processing structured code-based inputs and generating coherent outputs. When instruction-tuning the model under the Guideline-P, PN, and PS variants, we randomly sample one of the generated guidelines, a strategy found to prevent the model from memorizing specific guidelines in ~\citet{sainz2024gollie}.
For parameter-efficient training, we implemented rsLoRA~\cite{kalajdzievski2023rankstabilizationscalingfactor} using the Unsloth framework~\cite{unsloth}. 
% Key hyperparameters—including LoRA rank (64), scaling factor $\alpha$ (128), and batch size (32)—were determined through preliminary experiments to balance computational efficiency with model performance. Models were trained for 10 epochs using a single NVIDIA A100 GPU (80GB VRAM) with early stopping based on their dev-set performance. 
% Further details on the experimental setup are included in Appendix~\ref{sec:app_additional_details}.

We include all details about datasets, evaluation, and model training in Appendix~\ref{sec:appendix_preprocessing}-\ref{sec:app_additional_details}. {We will open-source our scripts for automatically converting datasets in TextEE~\cite{huang2024textee} into Python code format (we dub the processed version as \textbf{PyCode-TextEE}) and for evaluating extracted events automatically via code execution at our GitHub repository \url{https://github.com/salokr/LLaMA-Events}.}

\begin{table*}[th!]
\centering
\setlength{\tabcolsep}{4pt}
\renewcommand{\arraystretch}{1.2}
\definecolor{rowgray}{gray}{0.97} 
\definecolor{rowgray2}{gray}{1.0} 
\definecolor{lightblue}{RGB}{173, 216, 250}
\resizebox{\textwidth}{!}{%
\begin{tabular}{lcccc|cccc|cccc|cccc}
\toprule

\multirow{2}{*}{\textbf{Experiments}} & \multicolumn{4}{c|}{\textbf{ACE w/o NS}} & \multicolumn{4}{c|}{\textbf{ACE w/ NS}} & \multicolumn{4}{c|}{\textbf{RichERE w/o NS}} & \multicolumn{4}{c}{\textbf{RichERE w/ NS}} \\

\cmidrule(lr){2-5} \cmidrule(lr){6-9} \cmidrule(lr){10-13} \cmidrule(lr){14-17}
& \textbf{TI} & \textbf{TC} & \textbf{AI} & \textbf{AC} & \textbf{TI} & \textbf{TC} & \textbf{AI} & \textbf{AC} & \textbf{TI} & \textbf{TC} & \textbf{AI} & \textbf{AC} & \textbf{TI} & \textbf{TC} & \textbf{AI} & \textbf{AC} \\
\midrule
\rowcolor{lightblue}\textbf{NoGuideline}      & 39.57  & 39.57  & 31.05  & 29.73  & \textbf{84.15}  & \textbf{84.15}  & \textbf{64.99}  & \textbf{61.96}  & \underline{35.11}  & \underline{35.11}  & 27.16  & 25.32  & 42.27  & 42.27  & 32.38  & 31.56  \\
\rowcolor{rowgray2}\textbf{Guideline-H}      & 40.71  & 40.71  & 30.76  & 28.64  & 56.30  & 56.30  & 44.82  & 43.13 & --  & --  & --  & --  & --  & --  & --  & -- \\
\rowcolor{rowgray}\textbf{Guideline-P}      & \textbf{51.46}  & \textbf{51.46}  & \textbf{37.82}  & \textbf{35.20}  & 72.86  & 72.86  & 55.01  & 53.73  & 34.38  & 34.38  & \underline{28.04}  & \underline{26.35}  & 67.92  & 67.92  & 52.29  & 44.93 \\
\rowcolor{rowgray2}\textbf{Guideline-PN}     & \underline{49.60}  & \underline{49.60}  & 35.80  & 32.81  & \underline{80.77}  & \underline{80.77}  & \underline{63.20}  & \underline{60.34}  & \textbf{40.89}  & \textbf{40.89}  & \textbf{30.04}  & \textbf{27.18}  & \underline{75.35}  & \underline{75.35}  & \textbf{60.85}  & \textbf{57.10}  \\
\rowcolor{rowgray}\textbf{Guideline-PS}     & 47.93  & 47.93  & \underline{37.19}  & \underline{34.88}  & 79.23  & 79.23  & 59.00  & 56.88  & 32.41  & 32.41  & 24.63  & 22.78  & \textbf{76.45}  & \textbf{76.45}  & \underline{60.42}  & \underline{56.26}  \\
\rowcolor{rowgray2}\textbf{Guideline-PN-Int} & 40.17  & 40.17  & 30.46  & 28.34  & 51.95  & 51.95  & 41.09  & 39.32  & 27.11  & 27.11  & 21.93  & 20.81  & 42.40  & 42.40  & 33.22  & 31.67 \\
\rowcolor{rowgray}\textbf{Guideline-PS-Int} & 39.51  & 39.51  & 31.27  & 30.26  & 53.70  & 53.70  & 42.62  & 41.10  & 31.61  & 31.61  & 26.70  & 24.96  & 52.60  & 52.60  & 41.06  & 39.46  \\
\bottomrule
\end{tabular}%
}

\caption{Evaluation results (\%) for end-to-end EE tasks trained on complete train data. Models trained \textbf{with Negative Samples (w/ NS)} include negative example augmentation. (\textbf{Best} and \underline{Second Best} performances)}
\label{tab:full_train_table}
\end{table*}
\subsection{RQ1: Do the annotation guidelines allow an LLM to more precisely extract occurring events?}
To assess the impact of incorporating annotation guidelines in the EE instructions, we compare instruction-tuning an LLM with and without guidelines. We hypothesize that including the annotation guidelines can help the LLM more easily distinguish between similar event types. To understand its impact, we also compare this approach with a ``negative sampling (NS)'' approach. Specifically, we instruction-tune the LLM on an augmented training set, where each training example is supplemented with 15 randomly selected negative samples, i.e., triplets of $(e_{neg}, X, \phi)$ with non-existing event type $e_{neg}$ yielding empty extraction output. We note that annotation guidelines and negative sampling are two complementary approaches for an LLM to learn to distinguish between event types. In our experiments, we thus evaluated the effect of annotation guidelines in two independent settings: (1) training on the original training set (\textbf{w/o NS}) and (2) training on the negative sample-augmented training set (\textbf{w/ NS}).


Table~\ref{tab:full_train_table} shows the results.
In the \textbf{w/o NS} setting, including annotation guidelines (\textbf{Guideline-P}, \textbf{PN}, and \textbf{PS}) consistently improves performance across both datasets. Our analysis in Section~\ref{sec:analyais} further validated that the guidelines indeed enable the LLM to understand the nuanced differences between event types. On {ACE w/o NS}, \textbf{Guideline-P} achieves the highest scores across all four metrics, leading to around 10\% TC and 5\% AC gains over \textbf{NoGuideline}.
Similarly, on {RichERE w/o NS}, \textbf{Guideline-PN} outperforms \textbf{NoGuideline} by about around 5\% TC and 2\% AC.  

Training the LLM with augmented negative samples, as we expected, helps the model better distinguish between event types; for example, \textbf{NoGuideline} in the \textbf{w/ NS} setting achieves 30\% higher AC on ACE and 6\% higher AC on RichERE, compared to its counterparts in the \textbf{w/o NS} setting. 
However, the effects of annotation guidelines in the \textbf{w/ NS} setting diverge between the two datasets. For {ACE}, adding the guidelines in the instruction does not offer a further advantage, where \textbf{NoGuideline} and \textbf{Guideline-PN} achieved a comparable, the best performance, while all other guideline variants do not show to help. On RichERE, however, the benefit of annotation guidelines complements the negative samples', where \textbf{Guideline-PN} and \textbf{Guideline-PS} achieve around 25\% gain on AC over \textbf{NoGuideline}. We notice that RichERE is annotated with a smaller training set but defines more fine-grained event schemas than ACE; for example, the courser-grained \texttt{Transport} event type in ACE is represented by two event types, i.e., \texttt{TransportPerson} and \texttt{TransportArtifact}. As the guideline provides not only a detailed description of an event type but also a comparison with similar ones (Table~\ref{tab:guideline-examples}), the LLM can leverage this information for better EE performance.

\begin{table*}[th!]
\centering
\setlength{\tabcolsep}{4pt}
\renewcommand{\arraystretch}{1.2}
\definecolor{rowgray}{gray}{0.97} 
\definecolor{rowgray2}{gray}{1.0} 
\definecolor{lightblue}{RGB}{173, 216, 250}
\resizebox{\textwidth}{!}{%
\begin{tabular}{lcccc|cccc|cccc|cccc}
\toprule
\multirow{2}{*}{\textbf{Experiments}} & \multicolumn{4}{c|}{\textbf{ACE w/o NS}} & \multicolumn{4}{c|}{\textbf{ACE w/ NS}} & \multicolumn{4}{c|}{\textbf{RichERE w/o NS}} & \multicolumn{4}{c}{\textbf{RichERE w/ NS}} \\
\cmidrule(lr){2-5} \cmidrule(lr){6-9} \cmidrule(lr){10-13} \cmidrule(lr){14-17}
& \textbf{TI} & \textbf{TC} & \textbf{AI} & \textbf{AC} & \textbf{TI} & \textbf{TC} & \textbf{AI} & \textbf{AC} & \textbf{TI} & \textbf{TC} & \textbf{AI} & \textbf{AC} & \textbf{TI} & \textbf{TC} & \textbf{AI} & \textbf{AC} \\
\midrule
\rowcolor{lightblue}\textbf{NoGuideline}      & 10.60  & 10.60  & 5.19  & 3.68  & 31.64  & 31.64  & 25.91  & 24.22  & 19.87  & 19.87  & 13.34  & 11.69  & 36.29  & 36.29  & 28.15  & 25.58  \\
\rowcolor{rowgray2}\textbf{Guideline-H}      & 29.01  & 29.01  & 16.37  & 14.78  & 32.62  & 32.62  & 25.35  & 22.87 & --  & --  & --  & --  & --  & --  & --  & -- \\
\rowcolor{rowgray}\textbf{Guideline-P}      & \underline{36.91}  & \underline{36.91}  & \underline{24.17}  & \underline{21.24}  & \underline{56.99}  & \underline{56.99}  & \textbf{43.44}  & \textbf{40.51}  & \textbf{40.28}  & \textbf{40.28}  & \textbf{21.97}  & \textbf{18.33}  & \underline{62.04}  & \underline{62.04}  & \underline{46.33}  & \underline{42.03}  \\
\rowcolor{rowgray2}\textbf{Guideline-PN}     & 30.94  & 30.94  & 19.27  & 17.64  & \textbf{60.29}  & \textbf{60.29}  & \underline{42.88}  & \underline{39.95}  & \underline{31.23}  & \underline{31.23}  & \underline{19.48}  & \underline{17.51}  & \textbf{67.16}  & \textbf{67.16}  & \textbf{47.85}  & \textbf{43.39}  \\
\rowcolor{rowgray}\textbf{Guideline-PS}     & \textbf{40.53}  & \textbf{40.53}  & \textbf{28.03}  & \textbf{26.12}  & 55.1  & 55.1  & 41.57  & 38.91  & 26.16  & 26.16  & 16.64  & 15.19  & 58.95  & 58.95  & 42.79  & 38.1  \\
\rowcolor{rowgray2} \textbf{Guideline-PN-Int} & 34.11  & 34.11  & 22.73  & 21.18  & 28.31  & 28.31  & 23.82  & 22.37  & 25.73  & 25.73  & 16.75  & 14.6  & 33.59  & 33.59  & 28.06  & 26.0  \\
\rowcolor{rowgray} \textbf{Guideline-PS-Int} & 30.04  & 30.04  & 19.69  & 16.9  & 27.96  & 27.96  & 21.55  & 20.37  & 23.33  & 23.33  & 15.35  & 13.38  & 34.92  & 34.92  & 27.31  & 25.04  \\
\bottomrule
\end{tabular}%
}
\caption{Evaluation results (\%) on full test data, for end-to-end EE tasks, trained on 2000 train data samples.}
\label{tab:train2000_table}
\end{table*}

\begin{table}[t!]

\setlength{\tabcolsep}{4pt}
\renewcommand{\arraystretch}{1.2}
\small
\definecolor{rowgray}{gray}{0.97} 
\definecolor{rowgray2}{gray}{1.0} 
\definecolor{lightblue}{RGB}{173, 216, 250}
\resizebox{\linewidth}{!}{%
\begin{tabular}{lcccc|cccc}
\toprule
 & \multicolumn{4}{c|}{\textbf{ACE w/ NS}} & \multicolumn{4}{c}{\textbf{RichERE w/ NS}} \\
\cmidrule(lr){2-5} \cmidrule(lr){6-9}
& \textbf{TI} & \textbf{TC} & \textbf{AI} & \textbf{AC} & \textbf{TI} & \textbf{TC} & \textbf{AI} & \textbf{AC} \\
\midrule
\rowcolor{lightblue}\textbf{NoGuide}      & \textbf{37.08}  & \textbf{37.08}  & \textbf{21.53}  & \textbf{19.18}  & 24.98  & 24.98  & 15.05  & 13.15  \\
\rowcolor{rowgray2}\textbf{H}      & 29.00  & 29.00  & 17.93  & 16.34  & --  & --  & --  & --  \\
\rowcolor{rowgray}\textbf{P}      & 27.95  & 27.95  & 15.94  & 14.21  & 23.93  & 23.93  & 13.56  & 12.71  \\
\rowcolor{rowgray2}\textbf{PN}     & 29.60  & 29.60  & 17.87  & 15.92  & \underline{27.43}  & \underline{27.43}  & \textbf{17.10}  & \textbf{15.28}  \\
\rowcolor{rowgray}\textbf{PS}     & \underline{29.85}  & \underline{29.85}  & \underline{19.49}  & \underline{17.04}  & 19.61  & 19.61  & 11.77  & 10.48  \\
\rowcolor{rowgray2}\textbf{PN-Int} & 24.34  & 24.34  & 14.08  & 12.56  & \textbf{27.59}  & \textbf{27.59}  & \underline{16.21}  & \underline{14.47}  \\
\rowcolor{rowgray}\textbf{PS-Int} & 22.51  & 22.51  & 13.59  & 12.48  & 18.99  & 18.99  & 10.67  & 9.56  \\
\bottomrule
\end{tabular}%
}
\caption{Evaluation results (\%) for end-to-end EE tasks on full test data, averaged over three runs using 100 training samples. {We did not experiment with the ``w/o NS'' setting because the model performance with 100 training samples is negligible for all variants.}}
\label{tab:train_100_results}
\vspace{-5pt}
\end{table}



% \subsection{RQ2: Do Contrastive and Diverse Guidelines Improve Event Extraction?}
\subsection{RQ2: Are machine-generated annotation guidelines effective?}
Interestingly, from Table~\ref{tab:full_train_table}, we noticed that the guidelines provided by the ACE annotators do not yield a performance gain and that the machine-generated guideline variants are not equally effective. Specifically, \textbf{Guideline-H} achieves a comparable performance in \textbf{w/o NS} and an inferior one in \textbf{w/ NS} on ACE; \textbf{Guideline-PN-Int} and \textbf{Guideline-PS-Int} provide either no or limited performance gain in both \textbf{w/o NS} and \textbf{w/ NS} settings, while \textbf{Guideline-P} and \textbf{Guideline-PS} are not consistently better than \textbf{NoGuideline}. \textbf{Guideline-PN} shows to be the most stable, outperforming \textbf{NoGuideline} on RichERE and performing comparably to the best model on ACE.

Qualitatively, as shown in Table~\ref{tab:guideline-examples}, the human-written guidelines (\textbf{Guideline-H}) lack explicit contrasts, making event boundaries ambiguous—for instance, \texttt{Transport} (a movement event) and \texttt{Extradite} (a justice event) both involve relocation, yet the fact that only the latter is legally enforced is not clarified in the guidelines.
% Still, only the latter is legally enforced. 
% Similarly, \textbf{Guideline-PS} focuses on sibling event types, such as \texttt{ChargeIndict} vs. \texttt{Convict}, yet fails to address distinctions across different event hierarchies, such as between movement-based versus legally driven ones. This suggests that sibling differentiation alone is insufficient. Meanwhile, \textbf{-Adv} variants underperform due to excessive consolidation, which collapses multiple distinctions into a single definition, diluting critical event contrasts. As shown in \citet{cai-etal-2024-improving-event}, exposure to diverse definitions improves event understanding, reinforcing why models trained with diverse definitions generalize better. By presenting varied definitions per training instance, these guidelines help models learn broader event boundaries and avoid overfitting to narrow schemas or lexical cues.
\textbf{Guideline-P} provides examples and edge cases of the target event, but these may not be sufficient for the model to distinguish between similar event types. While both \textbf{Guideline-PS} and \textbf{Guideline-PN} have supplied this comparison, \textbf{-PS} shows to be limited by focusing on only sibling differentiations (e.g., \texttt{Extradite} vs. \texttt{Convict}). Finally, surprisingly, the two \textbf{-Int} variants, despite being comprehensive, lead to mixed results. We observed that models tend to overfit to these comprehensive instructions. In contrast, training the models with 5 diverse guidelines per event type as in \textbf{-PN} and \textbf{-PS} avoids this issue, which shares a similar finding as \citet{cai-etal-2024-improving-event, sainz2024gollie}.

% Qualitatively, as shown in Table~\ref{tab:guideline-examples}, \textbf{Guideline-H}, while compact, lack explicit contrasts between event types. For instance, ACE defines \textbf{Extradite} as the process of moving a \textbf{Person} to a foreign state for legal proceedings and \textbf{Transport} as ``{\highlight{F0F0F0}{whenever an ARTIFACT or a PERSON is moved from one PLACE to another.}}'' Since both involve movement, the distinction is ambiguous. In contrast, \textbf{Guideline-PN} explicitly differentiates them by adding a. On the other hand, we hypothesize that while diverse and contrastive definitions help LLMs generalize better, the \textbf{-Adv} variants—generated by consolidating five machine-generated guidelines into one—failed to maintain contrast between event types. By sampling from multiple definitions during training, \textbf{Guideline-PN} and \textbf{-PS} expose the model to more varied perspectives, enhancing generalization. 
% Future work should ensure contrast is preserved in consolidated guidelines.    

\subsection{RQ3: Are the annotation guidelines helpful when there is only a small amount of training data?}
% , making it essential to determine how much data is necessary for effective training. 
With 2000 samples (Table~\ref{tab:train2000_table}), \textbf{Guideline-P}, \textbf{Guideline-PN} and \textbf{Guideline-P} improve \textbf{NoGuideline} on ACE and RichERE w/o NS by up to 30\% TC and 20\% AC. Unlike our observation on the full-training setting, this trend also holds in \textbf{ACE w/ NS}, where guidelines provide a similar advantage. 
Excitingly, the results also show that annotation guidelines can compensate for limited training data, enabling models trained with only 2000 samples to achieve performance comparable to full-data training. For example, on ACE, \textbf{Guideline-P w/ NS (2k)} outperforms \textbf{NoGuideline w/o NS (full)} by 10\% AC; on RichERE, \textbf{Guideline-PN w/ NS (2k)} outperforms \textbf{NoGuideline (full)} by 18\% AC in ``w/o NS '' and 12\% AC in ``w/ NS''.
% \swetac{For example, Guideline-P w/ NS (2000 samples) outperforms NoGuideline w/o NS (full training data) by up to 10.78\% AC on ACE and 16.71\% AC on RichERE.} This is particularly crucial since high-quality annotated data is often scarce.


%\srivc{\textbf{Guideline-P} and \textbf{Guideline-PN} in \textbf{w/ NS} with 2000 train samples outperform both \textbf{NoGuideline} and \textbf{Guideline-H} in \textbf{w/o NS} by about 5\% and 10\% AC on ACE and RichERE, respectively.} 

However, when training data is reduced to 100 samples (Table~\ref{tab:train_100_results}), the benefits become dataset-dependent. In \textbf{ACE w/ NS}, \textbf{NoGuideline} slightly outperforms guideline-based models, suggesting that with extremely limited data, the model resorts to memorization rather than learning schema constraints. In contrast, in \textbf{RichERE w/ NS}, which has more diverse and fine-grained event structures, guidelines remain beneficial—\textbf{Guideline-PN} surpasses \textbf{NoGuideline} by 2\% AC, indicating that guidelines help in settings where direct memorization is insufficient.

\begin{table*}[t]
\centering
\setlength{\tabcolsep}{3.5pt}
\renewcommand{\arraystretch}{1.2}
\definecolor{rowgray}{gray}{0.97} 
\definecolor{rowgray2}{gray}{1.0} 
\definecolor{lightblue2}{RGB}{173, 216, 240} % Light Blue
\definecolor{lightblue}{RGB}{173, 216, 250} % Light Blue
\resizebox{\textwidth}{!}{%
\begin{tabular}{lcccc|cccc|cccc|cccc}
\toprule
\multirow{2}{*}{\textbf{Experiments}} 
& \multicolumn{4}{c|}{\textbf{RichERE w/o NS → ACE}} 
& \multicolumn{4}{c|}{\textbf{RichERE w/ NS → ACE}} 
& \multicolumn{4}{c|}{\textbf{ACE w/o NS → RichERE}} 
& \multicolumn{4}{c}{\textbf{ACE w/ NS → RichERE}} \\
\cmidrule(lr){2-5} \cmidrule(lr){6-9} \cmidrule(lr){10-13} \cmidrule(lr){14-17}
& \textbf{TI} & \textbf{TC} & \textbf{AI} & \textbf{AC} 
& \textbf{TI} & \textbf{TC} & \textbf{AI} & \textbf{AC} 
& \textbf{TI} & \textbf{TC} & \textbf{AI} & \textbf{AC} 
& \textbf{TI} & \textbf{TC} & \textbf{AI} & \textbf{AC} \\
\midrule
\rowcolor{lightblue}\textbf{NoGuideline}      & 29.55  & 29.55  & 21.34  & 16.60  & 44.10  & 44.10  & 33.91  & 25.17  & 33.41  & 33.41  & 24.34  & 22.68  & 37.19  & 37.19  & 27.74  & 25.87 \\
\rowcolor{rowgray2}\textbf{Guideline-P}      & 31.78  & 31.78  & 22.51  & 15.90  & 61.69  & 61.69  & 39.83  & 27.93  & \textbf{42.95}  & \textbf{42.95}  & \textbf{31.61}  & \textbf{27.79}  & 54.72  & 54.72  & 38.63  & 35.00  \\
\rowcolor{rowgray}\textbf{Guideline-PN}     & \textbf{40.12}  & \textbf{40.12}  & \textbf{27.78}  & \textbf{19.77}  & \underline{63.97}  & \underline{63.97}  & \textbf{48.74}  & \textbf{36.24}  & 41.72  & 41.72  & 29.54  & 26.10  & \underline{64.87}  & \underline{64.87}  & \textbf{48.25}  & \textbf{44.51}  \\
\rowcolor{rowgray}\textbf{Guideline-PS}     & 29.28  & 29.28  & 20.13  & 15.38  & \textbf{64.23}  & \textbf{64.23}  & \underline{44.12}  & \underline{32.84}  & \underline{42.33}  & \underline{42.33}  & \underline{29.93}  & \underline{26.73}  & \textbf{65.54}  & \textbf{65.54}  & \underline{45.57}  & \underline{41.68}  \\
\rowcolor{rowgray2}\textbf{Guideline-PN-Int} & 27.00  & 27.00  & 18.91  & 14.66  & 35.35  & 35.35  & 28.07  & 21.82  & 28.65  & 28.65  & 22.13  & 19.87  & 38.60  & 38.60  & 27.46  & 26.02  \\
\rowcolor{rowgray}\textbf{Guideline-PS-Int} & \underline{31.96}  & \underline{31.96}  & \underline{23.60}  & \underline{19.00}  & 51.71  & 51.71  & 39.36  & 31.34  & 34.33  & 34.33  & 26.65  & 24.24  & 36.85  & 36.85  & 27.69  & 26.19  \\
\bottomrule
\rowcolor{lightblue2}\textbf{In-Distribution}      & 39.57  & 39.57  & 31.05  & 29.73  & 84.15  & 84.15  & 64.99  & 61.96  & 35.11  & 35.11  & 27.16  & 25.32  & 42.27  & 42.27  & 32.38  & 31.56  \\
\bottomrule
\end{tabular}%
}

\caption{{Evaluation of models (\%) in cross-schema generalization. \textbf{In-Distribution} represents the NoGuideline performance when trained and tested on the same dataset and the same setting (w/o or w/ NS). We did not experiment with Guideline-H as RichERE does not come with human-annotated guidelines.}
% Evaluating the generalization of annotation guidelines to unseen datasets. Models were trained on one dataset and tested on the opposite dataset:
% \textbf{RichERE → ACE:} Trained on RichERE, tested on ACE, while \textbf{ACE → RichERE:} shows the reverse. The last row, \textbf{In-Distribution} represents models trained and tested on the same dataset for the NoGuideline setting to provide a direct comparison of generalization performance. 
}
\label{tab:RQ4_f1_scores}
\end{table*}


%we can highlight that 2k w NS can be better than full wo NS
\subsection{RQ4: Do annotation guidelines improve cross-schema generalization?}
% \zyc{model generalizability? not sure yet. if we have time, pick 1 model and limited settings (based on what we want to highlight..but I'm still waiting for the results)}
% \zyc{Unclear what msgs you want to deliver here.}
In Table~\ref{tab:RQ4_f1_scores}, we evaluate different variants' generalizability to a new schema in EE. Notably, while ACE and RichERE share the same domain, RichERE has a finer schema design.
{In \textbf{RichERE w/o NS → ACE}, performance remains below the in-distribution baseline. 
% with \textbf{Guideline-PN} achieving 40\% TC and 19\% AC, indicating moderate transferability but still falling short. While TC reaches 40\%, 
While \textbf{Guideline-PN} achieves 40\% TC, nearly matching the in-distribution score, its AC drops by nearly 10\%, likely due to RichERE’s expanded argument roles that do not always align well with ACE’s simpler schema. This suggests that fine-to-coarse schema migration is partially feasible but still faces challenges in argument mapping. Contrastive learning helps mitigate some of this gap, as seen in \textbf{Guideline-PS (w/ NS)}, which improves TC to 64\% and AC to 32\%, highlighting the benefits of structured alignment. In contrast, \textbf{ACE → RichERE} generalizes even better, with \textbf{Guideline-PN (w/ NS)} achieving 64\% TC and 44\% AC, surpassing the in-distribution baseline by over 22\% TC and 12\% AC. This suggests that training on ACE, which has well-defined event boundaries, provides a stronger foundation for adapting to RichERE’s more detailed schema. Since RichERE introduces additional argument roles for certain events in ACE, structured guidelines play a key role in preventing role confusion and ensuring more consistent schema adaptation.}
%We present error analysis and successful transfer examples in Appendix~\ref{X}.


\subsection{Further Analysis}
\label{sec:analyais}
\begin{table}[t!]

\setlength{\tabcolsep}{4pt}
\renewcommand{\arraystretch}{1.2}
% \small
\definecolor{rowgray}{gray}{0.97} 
\definecolor{rowgray2}{gray}{1.0} 
\definecolor{lightblue}{RGB}{173, 216, 250}
\definecolor{lightblue2}{RGB}{173, 216, 240} % Light Blue
\resizebox{\linewidth}{!}{%
\begin{tabular}{lcccc|cccc}
\toprule
 & \multicolumn{4}{c|}{\textbf{ACE}} & \multicolumn{4}{c}{\textbf{RichERE}} \\
\cmidrule(lr){2-5} \cmidrule(lr){6-9}
& \textbf{TI} & \textbf{TC} & \textbf{AI} & \textbf{AC} & \textbf{TI} & \textbf{TC} & \textbf{AI} & \textbf{AC} \\
\midrule
\rowcolor{lightblue}\textbf{NoGuide w/o NS}  & 29.90 & 29.90 & 20.70 & 19.44 & 32.74 & 32.74 & 24.18 & 22.35  \\
\rowcolor{rowgray2}\textbf{PN w/o NS}  & 30.88 & 30.88 & 21.82 & 20.15 & 33.72 & 33.72 & 25.24 & 24.48  \\
\rowcolor{lightblue2}\textbf{NoGuide w/ NS}  & 79.81 & 79.81 & 56.41 & 53.85 & 45.70 & 45.70 & 35.68 & 32.69  \\
\rowcolor{rowgray2}\textbf{PN w/ NS}  & 77.95 & 77.95 & 57.30 & 54.21 & 69.10 & 69.10 & 49.26 & 44.10  \\
\bottomrule
\end{tabular}%
}
\caption{Evaluation results (\%) of LLaMA-3.2-1B-Instruct trained on full ACE and RichERE.}
\vspace{-15pt}
\label{tab:llama_3.2}
\end{table}

\paragraph{Generalization to a Smaller LLM}
We experimented with LLaMA-3.2-1B-Instruct for \textbf{NoGuideline} and the best-performing guideline variant \textbf{Guideline-PN}. 
%... \zyc{1-2 sentences to say if the advantages of guidelines generalize to smaller lLM or not.}
Results in Table~\ref{tab:llama_3.2} display a consistent observation compared to experiments with the larger LLaMA-3.1-8B model (Table~\ref{tab:full_train_table}). That is,
\textbf{Guideline-PN} achieves a comparable or better result than \textbf{NoGuideline} and shows the advantage of guidelines, particularly on \textbf{RichERE w/ NS}.
% show that guidelines improve performance across model scales. Although overall 1B F1 scores are slightly lower than those in Table~\ref{tab:full_train_table}, AC F1 in RichERE \textbf{NoGuideline w/ NS} surpasses the 8B model by 1.13 points. Since \textbf{Guideline-PN} consistently outperforms NoGuideline, guidelines remain a valuable strategy for enhancing extraction quality in smaller LLMs, especially when computational constraints limit model size.}

% Based on the main experiment results from Table~\ref{tab:llama_3.2}, we selected the guideline settings with the best performance across experiments for further exploration with the DeepSeek-R1-Distill-Qwen-1.5B model. Specifically, we chose NoGuideline, Guideline-PN, and Guideline-PS for additional testing with smaller LLMs, focusing on their effectiveness in specialized tasks. Guideline-PN w/ NS achieved the highest F1 score across all 4 metrics, with (61.14) in TC and (39.88) in AC (see Table~\ref{tab:deepseek_results}), demonstrating the value of guidelines and negative samples in improving performance for these tasks. Across all settings, adding negative samples improved performance, and the guideline-based methods consistently performed better as seen from the TC and AC F1 scores, when compared to the NoGuideline setting. While these results are strong, they also highlight the distinct capabilities of smaller models like DeepSeek-R1-Distill-Qwen-1.5B, which, although competitive, still lag behind larger models like LLaMA 3.1 8B in certain tasks.

\begin{figure}[t!]
    \centering
    \includegraphics[width=0.9\linewidth]{analysis_radar.pdf}
    \caption{Error categorization: CA (Context Ambiguity), PE (Parsing Errors), MAE (Missing Arguments/Events), AE (Argument Errors), TTE (Type/Trigger Errors), and LN (Label Noise).
    % Randomly picked 100 samples from ACE and RichERE and categorized the most common errors.
    %Error categorization on each dataset based on randomly picked 100 examples where NoGuideline w/o NS made mistakes.
    }
    \label{fig:error_cat}
    \vspace{-1.5em}
\end{figure}

\paragraph{Error Analysis}
We randomly selected 100 examples on each dataset where \textbf{NoGuideline w/o NS} made mistakes and compared them with errors made by other variants. The results in Figure~\ref{fig:error_cat} show that, on ACE w/o NS, including the annotation guidelines leads to increasing ungrammatical code outputs and parsing errors (PE), although it dramatically reduces the event type and trigger errors (TTE). In the case of w/ NS, guidelines help in almost all aspects, with the majority of remaining errors being caused by missing arguments or events (MAE) and label noise (LN). On RichERE, however, we observe that for both w/o and w/ NS cases, the annotation guidelines enhance the model performance in all dimensions.
% We analyzed 100 erroneous predictions across each guideline setting within the RichERE and ACE datasets, categorizing failures into six types: context ambiguity (CA), parsing errors (PE), missing multiple arguments/events (MMA), argument extraction errors (AE), trigger/type errors (WT/ET), and label noise (LN). CA errors occur when the model misinterprets the meaning of a phrase. For example, in "the officer is going to go in and swoop her away," the model misclassifies "go" and "swoop" as Transport events, misinterpreting the metaphor. LN arises when there is a mismatch between the ground truth annotations and the annotation guidelines. In "Secession by Iraqi Kurds could inspire Turkey's rebel Kurds, who for 15 years have been fighting for autonomy," the ground truth incorrectly labels "fighting" as a Demonstrate event. A similar issue was reported in the GOLLIE paper (cite GOLLIE). Figure~\ref{fig:error_cat} shows the distribution of these errors, with the following observations.

% In the No Guideline w/o NS setting, RichERE struggles with polysemous triggers (WT/ET: 39), contextual ambiguity (CA: 36), and missing arguments (MMA: 21). Negative sampling (No Guideline w/ NS) reduces CA (36 → 6) and WT/ET (39 → 12), improving event disambiguation, though AE (9 → 6) remains challenging due to complex argument structures. Guideline-P w/ NS further reduces MMA (21 → 9) and WT/ET (39 → 18) highlighting performance improvement through annotation guidelines, but CA remains high (24), indicating unresolved ambiguity. Error rates are the lowest in the Guideline-PN w/ NS setting, with CA dropping to 3 and WT/ET to 3, demonstrating the effectiveness of guideline generation through the inclusion of contrasting events, coupled with negative samples.

% In the No Guideline w/o NS setting, ACE shows notable errors with MMA (19), CA (13), and AE (19), indicating challenges in capturing multiple arguments, resolving ambiguity, and extracting arguments. Negative sampling (No Guideline w/ NS) reduces CA (13 → 5) and WT/ET (13 → 11), improving event disambiguation by exposing the model to both correct and incorrect event types, helping refine decision boundaries. Guideline-P w/ NS further lowers WT/ET (13 → 8) and AE (19 → 12), with minor improvements in other categories, suggesting that annotation guidelines reduce misclassifications in WT/ET, while clearer event role definitions enhance AE, minimizing confusion in argument assignment. In the Guideline-PN w/ NS setting, error rates are the lowest, with CA and AE dropping to 2, reinforcing that guidelines and negative sampling together improve event differentiation and argument extraction.



\paragraph{Effectiveness of Guidelines per Event Type Frequency}
% Recent event extraction (EE) studies highlight the challenge of low-frequency event types (ETs), where data sparsity limits model generalization and degrades extraction performance, particularly for argument classification (AC) and argument identification (AI) \cite{Exploring the Feasibility of ChatGPT for Event Extraction}. To evaluate whether annotation guidelines can mitigate this issue, we compare the best-performing guideline approach without negative sampling (Guideline-PN w/o NS) against the NoGuideline w/o NS on the ACE test set to assess the impact of guidelines.
As shown in Figure~\ref{fig:performance_by_type_frequency}, frequent event types show consistent improvements with annotation guidelines, as indicated by the green bars, suggesting that even well-represented events benefit from enhanced annotations. Only a few declines (red bars) occurred with mid- to low-frequency event types, whereas most event types still benefit from the guidelines. In fact, these event types, especially the very rare ones (top of the figure), generally present larger gains (i.e., longer green bars) than the more frequent types, which demonstrates a unique advantage of annotation guidelines in low-resource settings.
% The occurrence of red bars is minimal for medium-frequency events, indicating that guidelines are generally beneficial for these events as most of them are represented by green bars. In contrast, rare event types (top) display a higher occurrence of red bars, reinforcing prior findings that guidelines alone may not fully resolve data sparsity issues. However, certain low-resource events, such as \textit{Convict}, show substantial gains, indicating that guidelines can still be beneficial in some underrepresented cases.







\section{RELATED WORK}
\label{sec:relatedwork}
In this section, we describe the previous works related to our proposal, which are divided into two parts. In Section~\ref{sec:relatedwork_exoplanet}, we present a review of approaches based on machine learning techniques for the detection of planetary transit signals. Section~\ref{sec:relatedwork_attention} provides an account of the approaches based on attention mechanisms applied in Astronomy.\par

\subsection{Exoplanet detection}
\label{sec:relatedwork_exoplanet}
Machine learning methods have achieved great performance for the automatic selection of exoplanet transit signals. One of the earliest applications of machine learning is a model named Autovetter \citep{MCcauliff}, which is a random forest (RF) model based on characteristics derived from Kepler pipeline statistics to classify exoplanet and false positive signals. Then, other studies emerged that also used supervised learning. \cite{mislis2016sidra} also used a RF, but unlike the work by \citet{MCcauliff}, they used simulated light curves and a box least square \citep[BLS;][]{kovacs2002box}-based periodogram to search for transiting exoplanets. \citet{thompson2015machine} proposed a k-nearest neighbors model for Kepler data to determine if a given signal has similarity to known transits. Unsupervised learning techniques were also applied, such as self-organizing maps (SOM), proposed \citet{armstrong2016transit}; which implements an architecture to segment similar light curves. In the same way, \citet{armstrong2018automatic} developed a combination of supervised and unsupervised learning, including RF and SOM models. In general, these approaches require a previous phase of feature engineering for each light curve. \par

%DL is a modern data-driven technology that automatically extracts characteristics, and that has been successful in classification problems from a variety of application domains. The architecture relies on several layers of NNs of simple interconnected units and uses layers to build increasingly complex and useful features by means of linear and non-linear transformation. This family of models is capable of generating increasingly high-level representations \citep{lecun2015deep}.

The application of DL for exoplanetary signal detection has evolved rapidly in recent years and has become very popular in planetary science.  \citet{pearson2018} and \citet{zucker2018shallow} developed CNN-based algorithms that learn from synthetic data to search for exoplanets. Perhaps one of the most successful applications of the DL models in transit detection was that of \citet{Shallue_2018}; who, in collaboration with Google, proposed a CNN named AstroNet that recognizes exoplanet signals in real data from Kepler. AstroNet uses the training set of labelled TCEs from the Autovetter planet candidate catalog of Q1–Q17 data release 24 (DR24) of the Kepler mission \citep{catanzarite2015autovetter}. AstroNet analyses the data in two views: a ``global view'', and ``local view'' \citep{Shallue_2018}. \par


% The global view shows the characteristics of the light curve over an orbital period, and a local view shows the moment at occurring the transit in detail

%different = space-based

Based on AstroNet, researchers have modified the original AstroNet model to rank candidates from different surveys, specifically for Kepler and TESS missions. \citet{ansdell2018scientific} developed a CNN trained on Kepler data, and included for the first time the information on the centroids, showing that the model improves performance considerably. Then, \citet{osborn2020rapid} and \citet{yu2019identifying} also included the centroids information, but in addition, \citet{osborn2020rapid} included information of the stellar and transit parameters. Finally, \citet{rao2021nigraha} proposed a pipeline that includes a new ``half-phase'' view of the transit signal. This half-phase view represents a transit view with a different time and phase. The purpose of this view is to recover any possible secondary eclipse (the object hiding behind the disk of the primary star).


%last pipeline applies a procedure after the prediction of the model to obtain new candidates, this process is carried out through a series of steps that include the evaluation with Discovery and Validation of Exoplanets (DAVE) \citet{kostov2019discovery} that was adapted for the TESS telescope.\par
%



\subsection{Attention mechanisms in astronomy}
\label{sec:relatedwork_attention}
Despite the remarkable success of attention mechanisms in sequential data, few papers have exploited their advantages in astronomy. In particular, there are no models based on attention mechanisms for detecting planets. Below we present a summary of the main applications of this modeling approach to astronomy, based on two points of view; performance and interpretability of the model.\par
%Attention mechanisms have not yet been explored in all sub-areas of astronomy. However, recent works show a successful application of the mechanism.
%performance

The application of attention mechanisms has shown improvements in the performance of some regression and classification tasks compared to previous approaches. One of the first implementations of the attention mechanism was to find gravitational lenses proposed by \citet{thuruthipilly2021finding}. They designed 21 self-attention-based encoder models, where each model was trained separately with 18,000 simulated images, demonstrating that the model based on the Transformer has a better performance and uses fewer trainable parameters compared to CNN. A novel application was proposed by \citet{lin2021galaxy} for the morphological classification of galaxies, who used an architecture derived from the Transformer, named Vision Transformer (VIT) \citep{dosovitskiy2020image}. \citet{lin2021galaxy} demonstrated competitive results compared to CNNs. Another application with successful results was proposed by \citet{zerveas2021transformer}; which first proposed a transformer-based framework for learning unsupervised representations of multivariate time series. Their methodology takes advantage of unlabeled data to train an encoder and extract dense vector representations of time series. Subsequently, they evaluate the model for regression and classification tasks, demonstrating better performance than other state-of-the-art supervised methods, even with data sets with limited samples.

%interpretation
Regarding the interpretability of the model, a recent contribution that analyses the attention maps was presented by \citet{bowles20212}, which explored the use of group-equivariant self-attention for radio astronomy classification. Compared to other approaches, this model analysed the attention maps of the predictions and showed that the mechanism extracts the brightest spots and jets of the radio source more clearly. This indicates that attention maps for prediction interpretation could help experts see patterns that the human eye often misses. \par

In the field of variable stars, \citet{allam2021paying} employed the mechanism for classifying multivariate time series in variable stars. And additionally, \citet{allam2021paying} showed that the activation weights are accommodated according to the variation in brightness of the star, achieving a more interpretable model. And finally, related to the TESS telescope, \citet{morvan2022don} proposed a model that removes the noise from the light curves through the distribution of attention weights. \citet{morvan2022don} showed that the use of the attention mechanism is excellent for removing noise and outliers in time series datasets compared with other approaches. In addition, the use of attention maps allowed them to show the representations learned from the model. \par

Recent attention mechanism approaches in astronomy demonstrate comparable results with earlier approaches, such as CNNs. At the same time, they offer interpretability of their results, which allows a post-prediction analysis. \par


\section{Conclusion}
In this work, we propose a simple yet effective approach, called SMILE, for graph few-shot learning with fewer tasks. Specifically, we introduce a novel dual-level mixup strategy, including within-task and across-task mixup, for enriching the diversity of nodes within each task and the diversity of tasks. Also, we incorporate the degree-based prior information to learn expressive node embeddings. Theoretically, we prove that SMILE effectively enhances the model's generalization performance. Empirically, we conduct extensive experiments on multiple benchmarks and the results suggest that SMILE significantly outperforms other baselines, including both in-domain and cross-domain few-shot settings.

% Bibliography entries for the entire Anthology, followed by custom entries
%\bibliography{anthology,custom}
% Custom bibliography entries only
\section*{Acknowledgements}
This project was sponsored by the College of Computing and Engineering and the Department of Computer Science at George Mason University. This project was also supported by resources provided by the Office of Research Computing at George Mason University (URL: \url{https://orc.gmu.edu}) and funded in part by grants from the National Science Foundation (Award Number 2018631).
\bibliography{custom}

% \newpage % remove later on. Added by Ziyu

\appendix

% \section{Addtional Experimental Setup Details}
\section{Preprocessing and Data Sampling}
\label{sec:appendix_preprocessing}
For both datasets, ACE and RichERE, we follow the TextEE standardization \cite{huang2024textee} and formulate them as sentence-level EE tasks. We use the ``split 1'' data split of TextEE, but only sample a subset of 100 examples from its development (dev) set for better training efficiency. Specifically, we ensure that for each event type, two event instances will be included in our dev set, prioritizing those with larger coverages of arguments, with the remaining being examples with no event occurrences. The datasets are then converted to the code format shown in Figure~\ref{fig:overview}. Table~\ref{tab:data_stats} summarizes dataset statistics.
\begin{table}[th!]
\centering
\definecolor{rowgray}{gray}{0.97} 
\definecolor{rowgray2}{gray}{1.0} 
\resizebox{\columnwidth}{!}{%
\begin{tabular}{>{\centering\arraybackslash}p{3cm}>{\centering\arraybackslash}p{2cm}>{\centering\arraybackslash}p{1.5cm}>{\centering\arraybackslash}p{3cm}}
\toprule
\textbf{Dataset} & \textbf{\#Event Types} & \textbf{\#Role Types} & \textbf{\#Instances\newline (train/dev/test)} \\ 
\midrule
\textbf{ACE05~\cite{doddington-etal-2004-automatic}} & 33 & 22  & 16531/1870/2519  \\
\textbf{RichERE~\cite{song-etal-2015-light}} & 38 & 35  & 9105/973/1163 \\
\bottomrule
\end{tabular}
}
\caption{Dataset statistics. For efficiency purposes, in our experiments, we curated a subset of 100 examples as our development (dev) set.}
\label{tab:data_stats}
\end{table}

To perform the low-data experiments (RQ3), we additionally create the following subsets of the full training set for each dataset. \textbf{Train2k} includes uniformly sampled 2,000 examples from the full training set. \textbf{Train100-1/2/3} are three distinct subsets including 100 examples from the full training set, each of which was selected following the same procedure as how we prepare the dev set, ensuring all event types are included and prioritizing instances covering more arguments.

\section{Evaluation Methodology and Metrics}
\label{sec:app_additional_details}
\paragraph{Evaluation Methodology.} 
Our methodology contrasts with GoLLIE~\cite{sainz2024gollie}, which follows a pipeline-based structure and selectively includes only parent event types in its prompts, limiting granularity in event representation. For argument extraction, GoLLIE further restricts schema inclusion to sibling event types, introducing manual design choices that reduce automation and scalability. To ensure fair and comprehensive evaluation, we adopt a methodology that enumerates all possible event types for each test and development sample during prompt construction. Unlike setups where only the gold-standard event schema is included in the prompt, we avoid implicit event detection bias—if the correct event type were provided, the model would not need to identify the event type itself and could directly extract arguments, which would not reflect its real performance on real-world data. Due to these fundamental differences in methodology, we do not compare our results with GoLLIE.


\section{Prompt Design and Model Training}
\paragraph{Model.} 
We conducted experiments on an instruction-tuned LLaMA-3-8B model, selected for its demonstrated proficiency in processing structured code-based inputs and generating coherent outputs. For parameter-efficient adaptation, we implement RSLoRA \cite{kalajdzievski2023rankstabilizationscalingfactor}, applying LoRA transformations to all linear layers in the transformer blocks following the methodology of \citet{dettmers2024qlora}. Key hyperparameters—including LoRA rank (64), scaling factor $\alpha$ (128), and batch size (32)—were determined through preliminary experiments to balance computational efficiency with model performance. The models were trained for 10 epochs using a single NVIDIA A100 GPU (80GB VRAM), with early stopping triggered after three consecutive validation steps without improvement. We adopt a cosine learning rate scheduler with an initial rate of 1e-5 and a warmup period of 350 steps. Input sequences are padded to 3,000 tokens to maintain consistency while accommodating long-form code structures. To ensure reproducibility and minimize memory fragmentation, we implement deterministic padding and truncation strategies.

\paragraph{Prompt Design.}
\label{sec:prompt-design}
We adopt a structured prompt format consisting of four components: (i) task instruction,(ii) event schema, (iii) input text, and (iv) expected output, formatted as a structured event representation. Our approach follows a schema-first prompting strategy, where event definitions are explicitly encoded in a structured format to enhance model comprehension of event relations and argument constraints. For each input instance, a randomly sampled guideline definition is used to annotate the event schema, ensuring that the model is exposed to multiple rephrasings rather than memorizing and overfitting on a static definition. Formally, we prepare the input sequence as follows: ``\texttt{[BoS] \$-task\_instruction $(I)$ \$-annotated event schema $({E}_e)$~\$-input\_sample~$(X_i)$~[EoS]}'' where  the event schema ${E}_e$ for an event $e$ is annotated with one of the generated guideline definitions. 

% \begin{table*}[ht]
%     \centering
%     \begin{minipage}{\textwidth} % Ensure it spans the full page width
%         \lstinputlisting[style=customjson, 
%             caption={Prompt example for generating Guideline-P, Guideline-PN, and Guideline-PS.}, 
%             label={lst:guidelines}, 
%             aboveskip=10pt, 
%             belowskip=10pt]
%         {tables/codes/guideline_prompt.json} % Update the path to your JSON file
%     \end{minipage}
% \end{table*}



% \onecolumn

\begin{lstlisting}[style=customjson, caption={Prompt example for generating Guideline-P, Guideline-PN, and Guideline-PS.}, label={lst:guidelines1}, aboveskip=10pt, belowskip=10pt]
You are an expert in annotating NLP datasets for event extraction. Your task is to generate "detailed" annotation guidelines for the event type Acquit which is a child event type of super class JusticeEvent.

Input Format will be as following
```
Event Schema:
Event Name and its parent class
Arguments:
Arguments separated by new lines. If there are no arguments None will be given.

Examples
```
Instructions:
1) Identify and list all unique arguments related to the event type.
2) Define the event type and each argument. You can take help of examples below to understand the events and their arguments. 
3) Please remember that the examples may not cover all the arguments in the list. In some cases, you may not have arguments at all, in such cases, you can have an empty list for arguments. 
4) For each definition, provide 5 illustrative definitions in JSON format. For events you can add example triggers and the explanation of the events such as edge cases and other critical details starting with "The event can be triggered by ... ". Similarly for arguments also you can add examples, and detailed information for them including any edge case or domain knowledge starting with "Examples are ... ".
5) Remember to not generate any additional information such as examples, etc. and strictly follow the output format shown below.
6) Remember also to add detailed information for the events and arguments so that the annotators who are not familiar with machine learning and NLP can still solve the task. Remember to add required domain knowledge and please cover the edge cases when possible.
7) Remember that while generating examples for the event or attributes you should generate diverse set of triggers or argument values rather than picking them from the examples I have provided for each of the 5 generated guidelines.

Output Format:
{
  "Event Definition": [
    "Definition 1",
    "Definition 2",
    "Definition 3",
    "Definition 4",
    "Definition 5"
  ],
  "Arguments Definitions": {
    "Argument1": [
      "Definition 1",
      "Definition 2",
      "Definition 3",
      "Definition 4",
      "Definition 5"
    ],
    "Argument2": [
      "Definition 1",
      "Definition 2",
      "Definition 3",
      "Definition 4",
      "Definition 5"
    ]
    // Add additional arguments as necessary
  }
}

Event Schema:
Acquit which is a child event type of super class JusticeEvent
Arguments:
Argument 1 -> adjudicator
Argument 2 -> defendant

Example 1
### Input Text ###
Sentence 1.
### Event Trigger ###
[event trigger]
### Event Arguments ###
For argument "defendant" extracted spans ['x']
For argument "adjudicator" extracted spans ['y']

Example 2
### Input Text ###
Sentence 2.
### Event Trigger ###
[event trigger]
### Event Arguments ###
For argument "defendant" extracted spans ['a']

(...)
\end{lstlisting}




\paragraph{Prompt for Generating Consolidated Guidelines.}The exact prompts used for generating consolidated guidelines - Guideline-PN-Int, and Guideline-PS-Int is shared below


\begin{lstlisting}[style=customjson, caption={Prompt example for generating consolidated guidelines: Guideline-PN-Int, and Guideline-PS-Int.}, label={lst:guidelines2}, aboveskip=10pt, belowskip=10pt]
You are an expert in summarizing NLP event extraction guidelines. Your goal is to consolidate multiple detailed descriptions into a single concise, comprehensive "Intergrated" guideline.

### Input Format ###
Event Type: Event Type Name
```json
{
  "Event Definition": [
    "Definition 1",
    "Definition 2",
    "Definition 3",
    "Definition 4",
    "Definition 5"
  ],
  "Arguments Definitions": {
    "mention": [
      "Definition 1",
      "Definition 2",
      "Definition 3",
      "Definition 4",
      "Definition 5"
    ],
    "Argument1": [
      "Definition 1",
      "Definition 2",
      "Definition 3",
      "Definition 4",
      "Definition 5"
    ],
    // Add additional arguments as necessary
  }
}
```

### Task ###
1. Integrated the 5 definitions under "Event Definition" into a single definition:
   - Highlight all critical points and examples from the five definitions.
   - Ensure the description is concise, comprehensive, and clear, using formal language that non-experts can understand.

2. Do the same for each argument under "Arguments Definitions," producing a single intergrated definition for each. 

### Output Format ###
```json
{
  "Event Definition": "Consolidated intergrated guideline for the event type.",
  "Arguments Definitions": {
    "mention": "Consolidated intergrated guideline for the mention argument.",
    "Argument1": "Consolidated intergrated guideline for Argument1.",
    "Argument2": "Consolidated intergrated guideline for Argument2."
    // Add additional arguments as necessary
  }
}
```

### Guidelines to Summarize ###
Event Type: prompt_Acquit(JusticeEvent)
```json
{
    "Acquit(JusticeEvent)": {
        "description": [
            "Definition 1",
            "Definition 2",
            "Definition 3",
            "Definition 4",
            "Definition 5"
        ]
    },
    "attributes": {
        "mention": "The text span that triggers the event."
        "adjudicator": [
            "Definition 1",
            "Definition 2",
            "Definition 3",
            "Definition 4",
            "Definition 5"
        ],
        "defendant": [
            "Definition 1",
            "Definition 2",
            "Definition 3",
            "Definition 4",
            "Definition 5"
        ]
    }
}
```
\end{lstlisting}


\section{Dataset Examples Across Multiple Guideline Settings}
\label{sec:app_dd}
The below JSON example illustrates an event extraction task from the ACE dataset under the No Guideline setting. It defines how structured events are extracted from text, specifying event triggers, types, arguments, and roles. The instruction explains the task, the input provides a natural language sentence and its conversion into a structured Python-style format. The output presents the extracted event, including its trigger ("extradited") and associated arguments (e.g., "government" as the agent, "him" as the person).

\begin{lstlisting}[style=customjson, caption={Prompt example for generating consolidated guidelines: Guideline-PN-Int, and Guideline-PS-Int.}, label={lst:guidelines3}, aboveskip=10pt, belowskip=10pt]
{
  "doc_id": "APW_ENG_20030306.0191",
  "wnd_id": "APW_ENG_20030306.0191-6",
  "instance_id": "821",
  "dataset_name": "ace05-en",
  "task_type": "E2E",
  "is_auth": "0",
  "instruction": "# This is an event extraction task where the goal is to extract structured events from the text. A structured event contains an event trigger word, an event type, the arguments participating in the event, and their roles in the event. For each different event type, please output the extracted information from the text into python-style dictionaries where the first key will be 'mention' with the value of the event trigger. Next, please output the arguments and their roles following the same format. The event type definitions and their argument roles are defined next.",
  "input": "# The following lines describe the task definition\n\n@dataclass\nclass Extradite(JusticeEvent):\n    mention: str\n    agent: List\n    destination: List\n    origin: List\n    person: List\n\n# This is the text to analyze\ntext = \"The post-Milosevic government later extradited him to the U.N. war crimes tribunal in The Hague, the Netherlands.\"\n\n# The list called result should contain the instances for the following events according to the guidelines above:\nresult = \n",
  "output": "[Extradite(\n    mention=\"extradited\",\n    person=[\"him\"], \n    destination=[\"Hague\"], \n    agent=[\"government\"],\n    origin=[]\n)]"
}
\end{lstlisting}

\paragraph{NoGuideline}
Shown below is an example from the NoGuideline setting in python code format with no doc string and argument definitions.
\vspace{10pt} % Adds space before the listing
\lstinputlisting[style=custompython, aboveskip=-5pt, belowskip=-5pt]
{NoGuideline_example_instance.py}

\paragraph{Guideline-PN}
Shown below is an example from the Guideline-PN setting in python code format.
\vspace{10pt} % Adds space before the listing
\lstinputlisting[style=custompython, aboveskip=-5pt, belowskip=-5pt]
{Guideline_PN_example_instance.py}

\paragraph{Guideline-PN-Int}
Similarly, shown below is an example from the Guideline-PN-Int setting in python code format.
\vspace{10pt} % Adds space before the listing
\lstinputlisting[style=custompython, aboveskip=-5pt, belowskip=-5pt]
{Guideline_PN_Int_example.py}
\twocolumn



\end{document}
