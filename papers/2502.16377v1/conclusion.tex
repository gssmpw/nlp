\section{Conclusion}
We demonstrate that incorporating structured annotation guidelines improves the instruction-tuning of LLMs for EE, bridges the data gap when only a limited amount of training data is available, and enhances the model's cross-schema generalization. Our explorations of guideline generation also highlight the promise of automatically generating effective instructions.
% Our study demonstrates that incorporating structured annotation guidelines improves EE, particularly when they provide explicit contrast between similar event types rather than relying solely on hierarchical relationships. While guidelines emphasizing sibling distinctions improve performance in some cases, broader contrast across semantically related but conceptually distinct events (such as \texttt{Transport Vs. Expedite}) proves more beneficial. The addition of contrastive negative samples enhances event differentiation, but the effectiveness of guidelines in this setting varies by dataset—while structured guidance does not offer additional benefits in ACE, it complements negative samples in RichERE, likely due to its fine-grained schema. We also find that guidelines remain effective in data-scarce settings, allowing models trained on limited data with negative sampling to achieve performance comparable to full-data baselines without negative samples. In cross-schema generalization, guidelines facilitate schema migration, with well-defined event boundaries aiding transferability more than fine-grained distinctions alone. However, excessively detailed guidelines that consolidate multiple perspectives into a single definition can overwhelm the model, diluting key distinctions. Future work should explore optimizing guideline length and adaptively retrieving the most informative definitions to balance conciseness with contrastive richness.

\section{Limitations}
While our study demonstrates the benefits of structured annotation guidelines for event extraction, several limitations remain. First, our evaluation is limited to two datasets (ACE and RichERE), both within the news domain, which may not fully capture how guidelines generalize to other domains such as biomedical or legal text. Future work should assess whether schema differences in other domains exhibit similar trends. Second, while we analyze guideline length and diversity, we do not explicitly optimize guideline generation, leaving open the question of how to best balance conciseness and informativeness. Exploring adaptive methods that retrieve or refine guidelines dynamically during training and inference could further improve efficiency. Lastly, our study primarily focuses on instruction-tuning an LLM with predefined event schemas;     however, real-world applications often require handling previously unseen event types. Investigating how structured guidelines can aid zero-shot or few-shot event extraction remains an important avenue for future research.