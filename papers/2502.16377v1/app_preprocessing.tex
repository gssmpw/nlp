\section{Preprocessing and Data Sampling}
\label{sec:appendix_preprocessing}
For both datasets, ACE and RichERE, we follow the TextEE standardization \cite{huang2024textee} and formulate them as sentence-level EE tasks. We use the ``split 1'' data split of TextEE, but only sample a subset of 100 examples from its development (dev) set for better training efficiency. Specifically, we ensure that for each event type, two event instances will be included in our dev set, prioritizing those with larger coverages of arguments, with the remaining being examples with no event occurrences. The datasets are then converted to the code format shown in Figure~\ref{fig:overview}. Table~\ref{tab:data_stats} summarizes dataset statistics.
\begin{table}[th!]
\centering
\definecolor{rowgray}{gray}{0.97} 
\definecolor{rowgray2}{gray}{1.0} 
\resizebox{\columnwidth}{!}{%
\begin{tabular}{>{\centering\arraybackslash}p{3cm}>{\centering\arraybackslash}p{2cm}>{\centering\arraybackslash}p{1.5cm}>{\centering\arraybackslash}p{3cm}}
\toprule
\textbf{Dataset} & \textbf{\#Event Types} & \textbf{\#Role Types} & \textbf{\#Instances\newline (train/dev/test)} \\ 
\midrule
\textbf{ACE05~\cite{doddington-etal-2004-automatic}} & 33 & 22  & 16531/1870/2519  \\
\textbf{RichERE~\cite{song-etal-2015-light}} & 38 & 35  & 9105/973/1163 \\
\bottomrule
\end{tabular}
}
\caption{Dataset statistics. For efficiency purposes, in our experiments, we curated a subset of 100 examples as our development (dev) set.}
\label{tab:data_stats}
\end{table}

To perform the low-data experiments (RQ3), we additionally create the following subsets of the full training set for each dataset. \textbf{Train2k} includes uniformly sampled 2,000 examples from the full training set. \textbf{Train100-1/2/3} are three distinct subsets including 100 examples from the full training set, each of which was selected following the same procedure as how we prepare the dev set, ensuring all event types are included and prioritizing instances covering more arguments.
