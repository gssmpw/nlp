\begin{table*}[t!]
\centering
\resizebox{.95\linewidth}{!}{%
\begin{tabular}{>{\centering\arraybackslash}p{3.8cm}p{17cm}}
    \toprule
    \multicolumn{2}{c}{\textbf{Examples of Annotation Guidelines for Event Type: Extradite (ACE05)}} \\
    % \midrule
    %     {\textbf{\textsc{Event Schema}}\newline(No Guideline)} &
    %     \lstinputlisting[style=custompython, aboveskip=-5pt, belowskip=-5pt]{tables/codes/beborn.py}\\
    \midrule
    \textbf{\textsc{Guideline-H}}\newline \small{Avg. Length - 107.67 tokens} & 
    {\textbf{Event Type:} {{An EXTRADITE Event occurs whenever a PERSON is sent by a state actor from one PLACE to another place for the purposes of legal proceedings there. }}\newline
    \textbf{Arguments:} \newline 
    - \textsc{Agent}: The extraditing agent.  \newline 
    - \textsc{Person}: The person being extradited.}\\
    \midrule
    \textbf{\textsc{Guideline-P}}  \newline \small{Avg. Length - 163.87 tokens}&
    {\textbf{Event Type:} The Extradition event refers to the formal process where one jurisdiction delivers a person accused \textbf{(...)} \highlight{D5E8D4}{The event can be triggered by terms such as `extradition' \textbf{(...)}} \highlight{BCD4E6}{Edge cases include situations where the term `extradition' is used metaphorically or in a non-legal context.} \newline
    \textbf{Arguments:} \newline 
    - \textsc{Agent}:\textbf{(...)} the agent is the organization or authority \textbf{(...)}. \highlight{D5E8D4}{Examples include `court', `government', \textbf{(...)}} \newline 
    - \textsc{Person}: \textbf{(...)} individual who is being transferred to another jurisdiction. \highlight{D5E8D4}{Examples are `she', \textbf{(...)}}
    }\\
    \midrule
    \textbf{\textsc{Guideline-PN}} \newline \small{Avg. Length - 285.24 tokens}&
    \textbf{Event Type:} The event is triggered by the formal request \textbf{(...)} for legal reasons. \highlight{D5E8D4}{Triggers such as `extradition' are indicative of this event type}, \highlight{FACEC6}{not `Transport' which involves general movement without legal context.} \\
    & \textbf{Arguments:} \\
    & - \textsc{Agent}: The agent is responsible for the legal and procedural aspects of the extradition,\textbf{(...)}. \highlight{D5E8D4}{An example is `the original court' \textbf{(...)}} \\
    & - \textsc{Person}: \textbf{(...)} one who is being moved from one place to another under legal authority. \highlight{D5E8D4}{For example, `he' \textbf{(...)}} \\
    \midrule
    \textbf{\textsc{Guideline-PS}} \newline \small{Avg. Length - 159.79 tokens}&
    \textbf{Event Type:} \textbf{(...)} person being moved to a new jurisdiction \textbf{(...)}. \highlight{FACEC6}{This differs from events like `TrialHearing' or `Convict', which focus on the legal proceedings and outcomes within a single jurisdiction.} \\
    & \textbf{Arguments:} \\
    & - \textsc{Agent}: \textbf{(...)} \highlight{BCD4E6}{Edge cases may include international organizations or coalitions} \textbf{(...)} \highlight{D5E8D4}{such as the U.N. \textbf{(...)}} \\
    & - \textsc{Person}: \highlight{FACEC6}{Unlike the `defendant' in events like `TrialHearing' or `Convict', the person in the `Extradite' event is specifically being transferred for legal proceedings or punishment.} \\
    \midrule
    \textbf{\textsc{Guideline-PN-Int}} \newline \small{Avg. Length - 439.94 tokens}&
    \textbf{(...)} \highlight{D5E8D4}{Key triggers include terms like `extradite', `extradition', and `extraditing'.} \highlight{FACEC6}{It is distinct from events like `ArrestJail' and `ReleaseParole', as it specifically involves \textbf{(...)}} \\
    & \textbf{Arguments:} \\
    & - \textsc{Agent}: The agent \textbf{(...)} typically a legal or governmental body. \highlight{D5E8D4}{Examples include `court', `government'\textbf{(...)}} \\
    & - \textsc{Person}: The person is the individual being extradited, the subject of the legal transfer. \highlight{D5E8D4}{Examples include `she', `him', and `her'.} \\
    \midrule
    \textbf{\textsc{Guideline-PS-Int}} \newline \small{Avg. Length - 434.64 tokens}&
    The 'Extradite' event involves the legal transfer of a person \textbf{(...)}. \highlight{FACEC6}{It is distinct from events like `ArrestJail', \textbf{(...)}, and `ReleaseParole' or `Pardon', \textbf{(...)}} \\
    & \textbf{Arguments:} \\
    & - \textsc{Agent}: The agent is the entity \textbf{(...)} \highlight{D5E8D4}{such as a court, government, or police department.} \highlight{BCD4E6}{This entity ensures the transfer is conducted according to legal protocols \textbf{(...)}} \\
    & - \textsc{Person}: \textbf{(...)} They are the central figure in the extradition process, \highlight{FACEC6}{distinct from a `defendant' in other legal events, \textbf{(...)}}\highlight{BCD4E6}{ This may include high-profile individuals or groups.} \\

    \bottomrule
    \end{tabular}

    }
    \caption{Examples of annotation guidelines for the event type \texttt{Extradite} from ACE05. Due to space limits, only \texttt{agent} and \texttt{person} were shown for arguments, and only 1 out of the 5 guideline samples were shown for \textbf{P}, \textbf{PN}, and \textbf{PS}.
    % For arguments, only \texttt{agent} and \texttt{person} were shown due to space limits.
    % \highlight{FACEC6}{Light red} highlights distinctions from other event types, \highlight{D5E8D4}{Light green} highlights example mentions, and \highlight{BCD4E6}{Light blue} highlights edge cases. 
    We highlight \highlight{FACEC6}{distinctions from other event types}, \highlight{D5E8D4}{example mentions}, and \highlight{BCD4E6}{edge cases} in guidelines.
    % On average, we note that the \textbf{-Adv} versions are the longest generations.
    % \zyc{add a fig showing the siblings of Extradite and "Transport" as one neg ET? From the truncated examples, the Adv is not obviously longer than others. Do we want to clarify its length for analysis purposes?} 
    }
    \label{tab:guideline-examples}
\end{table*}
