\documentclass{article}


\usepackage{PRIMEarxiv}

\usepackage[utf8]{inputenc} % allow utf-8 input
\usepackage[T1]{fontenc}    % use 8-bit T1 fonts
\usepackage{hyperref}       % hyperlinks
\usepackage{url}            % simple URL typesetting
\usepackage{booktabs}       % professional-quality tables
\usepackage{amsfonts}       % blackboard math symbols
\usepackage{nicefrac}       % compact symbols for 1/2, etc.
\usepackage{microtype}      % microtypography
\usepackage{lipsum}
\usepackage{fancyhdr}       % header
\usepackage{graphicx}       % graphics
\usepackage{multirow}
\usepackage{colortbl}
\usepackage{hhline}
\usepackage{color, xcolor}
\usepackage{ulem}
\graphicspath{{media/}}     % organize your images and other figures under media/ folder

%Header
\pagestyle{fancy}
\thispagestyle{empty}
\rhead{ \textit{ }} 

% Update your Headers here
\fancyhead[LO]{Running Title for Header}
% \fancyhead[RE]{Firstauthor and Secondauthor} % Firstauthor et al. if more than 2 - must use \documentclass[twoside]{article}



  
%% Title
\title{PDM-SSD: Single-Stage Three-Dimensional Object Detector With Point Dilation
%%%% Cite as
%%%% Update your official citation here when published 
\thanks{\textit{\underline{Citation}}: 
\textbf{Authors. Title. Pages.... DOI:000000/11111.}} 
}

\author{
  Ao Liang, Haiyang Hua, Jian Fang, Wenyu Chen Huaici Zhao* \\
  Key Laboratory of Opto-Electronic Information Processing, Chinese Academy of Sciences, 110016, Shenyang\\
  Shenyang Institute of Automation, Chinese Academy of Sciences, 110016, Shenyang \\
  University of Chinese Academy of Sciences, 100049, Beijing \\
  \texttt{\{liangao, hczhao\}@sia.cn} \\
  %% examples of more authors
 % \\
  %% \AND
  %% Coauthor \\
  %% Affiliation \\
  %% Address \\
  %% \texttt{email} \\
  %% \And
  %% Coauthor \\
  %% Affiliation \\
  %% Address \\
  %% \texttt{email} \\
  %% \And
  %% Coauthor \\
  %% Affiliation \\
  %% Address \\
  %% \texttt{email} \\
}


\begin{document}
\maketitle


\begin{abstract}
One of the important reasons why grid/voxel-based three-dimensional (3D) object detectors can achieve robust results for sparse and incomplete targets in Light Detection And Ranging (LiDAR) scenes is that the repeated padding, convolution, and pooling layers in the feature learning process enlarge the model's receptive field, enabling features even in space not covered by point clouds. However, they require time- and memory-consuming 3D backbones. Point-based detectors are more suitable for practical application, but current detectors can only learn from the provided points, with limited receptive fields and insufficient global learning capabilities for such targets. In this paper, we present a novel Point Dilation Mechanism for single-stage 3D detection (PDM-SSD) that takes advantage of these two representations. Specifically, we first use a PointNet-style 3D backbone for efficient feature encoding. Then, a neck with Point Dilation Mechanism (PDM) is used to expand the feature space, which involves two key steps: point dilation and feature filling. The former expands points to a certain size grid centered around the sampled points in Euclidean space. The latter fills the unoccupied grid with feature for backpropagation using spherical harmonic coefficients and Gaussian density function in terms of direction and scale. Next, we associate multiple dilation centers and fuse coefficients to obtain sparse grid features through height compression. Finally, we design a hybrid detection head for joint learning, where on one hand, the scene heatmap is predicted to complement the voting point set for improved detection accuracy, and on the other hand, the target probability of detected boxes are calibrated through feature fusion. On the challenging Karlsruhe Institute of Technology and Toyota Technological Institute (KITTI) dataset, PDM-SSD achieves state-of-the-art results for multi-class detection among single-modal methods with an inference speed of 68 frames. We also demonstrate the advantages of PDM-SSD in detecting sparse and incomplete objects through numerous object-level instances. Additionally, PDM can serve as an auxiliary network to establish a connection between sampling points and object centers, thereby improving the accuracy of the model without sacrificing inference speed. Our code will be available at \url{https://github.com/AlanLiangC/PDM-SSD.git}.\end{abstract}


% keywords can be removed
\keywords{Autonomous driving \and 3D object detection \and Deep learning \and Point cloud proccessing}

\section{Introduction}
\section{Introduction}
\label{sec:intro}


\ps{Challenges of technology scaling}

The growing demand for computing performance has always been met by increasing the number of transistors per chip, which is only possible due to CMOS technology scaling.
However, as we keep pushing the boundaries of technology scaling, we encounter multiple challenges.
Firstly, whenever we transition to a more advanced technology node, the non-recurring cost due to physical design, verification, software, mask sets, and prototyping almost doubles \cite{cost-tech-node}.
As a result, designing a chip in an advanced technology node is only economically viable if the chip is manufactured in vast quantities.
Secondly, many chip components such as I/O drivers, analog circuits, or \gls{srams} have reached their scaling limit.
This means that we cannot shrink these components further, even if we use a more advanced technology with a smaller feature size.
Thirdly, advanced technology nodes suffer from high defect rates, diminishing the yield and inflating the recurring cost.
To tackle these challenges, new chip-design paradigms have been developed.

\ps{Why 2.5D integration?}

One of these new paradigms is 2.5D integration, where multiple silicon dies called chiplets are integrated into the same package.
Once designed, a single chiplet can be reused in multiple 2.5D stacked chips, which increases the ratio of production volume to non-recurring cost.
Another advantage is that multiple chiplets - fabricated in different technologies - can be integrated into the same package.
This means that only components that can take full advantage of technology scaling are built in bleeding-edge technologies.
Components that have reached their scaling limit are fabricated in more mature and hence less costly technology nodes.
Furthermore, chiplets are smaller than monolithic chips.
Therefore, manufacturing chiplets results in less silicon area loss due to fabrication defects and hence a higher yield.
Due to these economic advantages, chip vendors such as AMD \cite{amd-chiplet} and NVIDIA \cite{chiplet-book} have adopted the 2.5D integration paradigm.  

\ps{Challenges of 2.5D integration}

An important challenge when designing 2.5D stacked chips is the construction of a low-latency and high-throughput \gls{ici}. 
To build an \gls{ici}, we connect different chiplets using \gls{d2d} links.
These links are fabricated in an organic package substrate, silicon bridge, or silicon interposer, and they are connected to the chiplets using \gls{c4} bumps or microbumps.
The number of bumps per chiplet is limited, and so is the bandwidth of \gls{d2d} links.
In addition to having lower bandwidth than links in monolithic chips, \gls{d2d} links also have higher latency.
This latency is caused by wire delay and by \gls{phys} that are necessary in both the sending and the receiving chiplet.
\gls{phys} are needed to convert between protocols, voltage levels, and frequencies, which are usually different between on-chiplet links and \gls{d2d} links.
Due to these limitations, the \gls{ici} can quickly become a bottleneck.

\ps{How we solve these challenges differently than the related work does.}

Existing approaches to maximize the performance of the \gls{ici} either optimize the placement of chiplets (with potentially heterogeneous shapes) for a predetermined \gls{ici} topology 
\cite{ho,liu,seemuth,eris,osmolovskyi,tap25d,chiou}, select one topology out of a set of candidates \cite{coskun-1, coskun-2}, or they optimize the \gls{ici} topology for a 2D grid of homogeneously shaped chiplets on an active interposer \cite{butterdonut, cluscross, kite}.
To the best of our knowledge, there is no prior work on \gls{ici} topologies for chips with heterogeneously shaped chiplets or with passive silicon interposers or silicon bridges.
To fill this gap, we propose \name, a novel optimization methodology to jointly optimize the chiplet placement and \gls{ici} topology of such architectures.
\ifnb
\else
\newpage
\fi

\ps{Details on \name~and the key idea}

The key idea is as follows: 
We optimize the chiplet placement without a predetermined topology.
For each placement generated by an optimization algorithm, we infer a placement-based \gls{ici} topology by connecting chiplets that are in close proximity in that specific placement.
We then compute the latency and throughput of this combination of placement and topology for different traffic types.
These latencies and throughputs together with the total chip area are used to compute a user-defined quality-score of the placement, which is returned to the optimization algorithm.
Based on this quality score, the algorithm can further optimize the placement.
By following this iterative process, we jointly optimize the chiplet placement and the \gls{ici} topology.

\ps{Short evaluation-summary}

We provide our open-source framework implementing the proposed placement and topology co-optimization methodology, which we evaluate using both synthetic traffic and traffic traces.
A 2D grid of chiplets with a mesh topology is used as a baseline since many proposals for 2.5D stacked chips \cite{dataflow_accel_dnn, cifher, simba, hecaton, dojo} use such an architecture.
We reduce the latency of synthetic L1-to-L2 and L2-to-memory traffic, the two most important traffic types for cache coherency traffic, by up to 28\% and 62\% respectively.
For real traffic traces, we reduce the average packet latency for almost all traces and architectures considered (reduced by an 8\% or 18\% on average depending on the configuration of \gls{phys} within a chiplet).


\section{Related Work}
\label{sec:related}

\section{Related work}


Recent advances in single-image animatable head avatar generation can be categorized into mainly 2D-based and 3D-based approaches. 

\paragraph{\bf Image to 2D Animatable Avatar.}
2D-based methods, leveraging the power of convolutional neural networks (CNNs)~\cite{DBLP:conf/cvpr/KarrasLAHLA20,DBLP:conf/cvpr/IsolaZZE17,DBLP:conf/nips/GoodfellowPMXWOCB14}, often employ generative adversarial networks (GANs)~\cite{DBLP:conf/cvpr/StyleGAN} for direct image synthesis. Early approaches~\cite{DBLP:conf/cvpr/WangDYSW23,DBLP:conf/cvpr/BurkovPGL20,DBLP:conf/iccv/ZakharovSBL19} focus on injecting expression and pose features into the generator network, often utilizing architectures like U-Net or StyleGAN~\cite{DBLP:conf/cvpr/StyleGAN}.
Some other 2D methods~\cite{DBLP:journals/corr/abs-2407-03168,DBLP:conf/cvpr/ZhangQZZW0CW023,DBLP:conf/cvpr/HongZS022,DBLP:conf/mm/DrobyshevCKILZ22,DBLP:conf/cvpr/BurkovPGL20,DBLP:conf/nips/SiarohinLT0S19} represent expressions and poses as warping fields applied to the source image. 
Benefiting from advances in image and video diffusion networks, more recent 2D-based works~\cite{DBLP:journals/corr/abs-2410-07718,DBLP:journals/corr/abs-2406-08801,DBLP:conf/eccv/TianWZB24} get improved results with diffusion techniques. 
However, these methods still face challenges related to long generation times and significant computational resource demands. Audio-driven 2D control methods~\cite{DBLP:conf/cvpr/ZhangCWZSGSW23,DBLP:journals/corr/abs-2211-12368,DBLP:conf/iccv/GuoCLLBZ21} are easy to use but cannot explicitly control facial expressions and poses. 2D-based techniques often struggle with large pose or expression variations due to the lack of an explicit 3D structure, sometimes producing unrealistic distortions or identity changes. While some 2D methods~\cite{SadTalker,StyleHEAT,Pirenderer,DBLP:conf/cvpr/WangM021,MegaPortraits} incorporate 3D Morphable Models (3DMMs)~\cite{DBLP:conf/fgr/GerigMBELSV18,DBLP:journals/tog/LiBBL017,DBLP:conf/avss/PaysanKARV09,DBLP:conf/siggraph/BlanzV99} to mitigate these issues, they typically cannot achieve free-viewpoint rendering. 

\vspace{-0.1in}

\begin{figure*}[h]
    \centering
    \includegraphics[width=0.9\linewidth]{images/framework.pdf}
    \caption{\textbf{Overall Framework.} Our framework utilizes learnable query features attached to FLAME vertices to perform cross-attention with the extracted multi-level image features. The extracted features are then decoded to reconstruct the Gaussian avatar in the canonical space, which can be animated utilizing standard linear blend skinning (LBS) and corrective blendshapes as the FLAME model did and rendered in real-time on various platforms.}
    \label{fig:framework}
\end{figure*}

\paragraph{\bf Image to 3D Animatable Avatar.}
3D-aware methods offer improved geometric consistency and free-viewpoint rendering capabilities. Early 3D approaches~\cite{DBLP:conf/eccv/KhakhulinSLZ22,DBLP:conf/cvpr/XuYCWDJT20} utilize 3DMMs for head avatar reconstruction. With the advent of Neural Radiance Fields (NeRFs)~\cite{DBLP:conf/eccv/MildenhallSTBRN20}, many recent methods~\cite{DBLP:conf/siggraph/YuFZWYBCSWSW23,DBLP:conf/cvpr/MaZQLZ23,DBLP:conf/cvpr/LiZWZ0CZWB023,GPAvatar,ye2024real3d,deng2024portrait4d,deng2024portrait4d2,DBLP:conf/eccv/KiMC24,DBLP:conf/cvpr/BaiFWZSYS23,PointAvatar,Nerfies,INSTA} have adopted this representation for higher fidelity, particularly in modeling fine details like hair. However, NeRF-based~\cite{DBLP:conf/cvpr/ZhangZLHLWGCL024,HAvatar,DBLP:conf/cvpr/BaiTHSTQMDDOPTB23,AD-NeRF,DBLP:journals/tog/GaoZXHGZ22,DBLP:journals/tog/ParkSHBBGMS21,DBLP:conf/cvpr/AtharXSSS22,DBLP:journals/corr/abs-2112-05637,DBLP:conf/iccv/TretschkTGZLT21,DBLP:conf/cvpr/GafniTZN21,DBLP:conf/eccv/KiMC24,DBLP:conf/cvpr/BaiFWZSYS23,PointAvatar,Nerfies,DBLP:conf/siggraph/YuFZWYBCSWSW23,DBLP:conf/cvpr/MaZQLZ23,DBLP:conf/cvpr/LiZWZ0CZWB023} approaches often require extensive training data, including multi-view or single-view videos, raising privacy concerns and limiting generalization to unseen identities. Some methods~\cite{DBLP:conf/cvpr/SunWWLZZL23,DBLP:conf/3dim/ZhuangMKS22,DBLP:journals/pami/SunWZHWL24,DBLP:journals/tvcg/TangZYZCMW24,DBLP:conf/iclr/XuZLZBFS23} bypass this data requirement by training generators with random noise and then inverting them for identity-specific reconstruction, but inversion accuracy remains a challenge. Test-time optimization offers another alternative, but its computational cost limits practical applications. Several recent works~\cite{goha2023,hidenerf2023,gpavatar2024,ye2024real3d,ma2024cvthead,deng2024portrait4d,deng2024portrait4d2,GGHead} have explored one-shot 3D head reconstruction to address the limitations of data requirements and computational cost. These methods employ various techniques, such as tri-plane features, deformation fields, point-based expression fields, and vertex-feature transformers. Despite these advancements, NeRF-based methods often struggle with real-time rendering. 
Recently, 3D Gaussian Splatting~\cite{GaussianSplatting} has emerged as a promising alternative, offering both high-quality results and fast rendering speeds. However, existing Gaussian Splatting methods~\cite{GaussianAvatar,DBLP:conf/cvpr/XuCL00ZL24} typically rely on video data for training for each person, limiting their ability to generalize to new identities. Instead, the most recent work, GAGAvatar~\cite{GAGAvatar}, proposes a one-shot 3D Gaussian-based head avatar generation method. However, it still relies heavily on complex 2D neural post-processing to achieve optimal animation outcomes, thus it is not a pure 3D solution and the extra neural network hinders its application on various platforms. In contrast, our work generates Gaussian heads that are immediately animatable and renderable without additional networks or post-processing steps, enabling seamless integration into existing rendering pipelines for real-time animation and rendering across a wide range of platforms, including mobile phones. 

\section{Method}
\label{sec:method}

\subsection{Overview}
\begin{figure}[t]
	\begin{center}
		% \vspace{-0.2cm}
		\includegraphics[width=.8\textwidth]{AL_pics/pf.pdf}
	\end{center}
	\vspace{-0.3cm}
	\caption{Visualization on some very sparse and extremely incomplete targets on the KITTI dataset. For grid-based backbone networks, the grid continuously pads, convolves, and pools the operations, covering the space that the original point cloud does not occupy. The expansion of the receptive field is continuous and can better aggregate local features and combine features from different regions. Point-based methods can only extract features from existing points, and even if the number of surrounding points increases, the features remain unchanged. The receptive field is discontinuous and limited to local areas.}
	\label{fig3}
	\vspace{-0.3cm}
\end{figure}

As aforementioned, we aim to combine the advantages of grid-based and point-based detectors by addressing the issue of discontinuous receptive fields in current point-based models. To achieve this, we propose PDM-SSD, a novel and generic single-stage point-based 3D detector. The overall workflow of PDM-SSD is illustrated in Fig. \ref{fig2}, where the input LiDAR point clouds are first passed through the embedding network to expand the feature space of the points. Then, a PointNet-style 3D backbone network is employed to extract point-wise features. The 3D backbone network consists of several stages of downsampling modules, local feature aggregation modules, and multi-scale feature aggregation modules, ensuring that the sampled points learn rich geometric and semantic features. The neck network includes our proposed point dilation mechanism, where points are lifted to the grid level and feature filling is performed for the unoccupied space in the original point cloud using a special mechanism. The grid-wise features are used to regress the heatmap of the scene, providing global features of the objects through joint learning with the hybrid detection head.

In the following, we discuss in detail the implementation process of PDM-SSD. In Section \ref{sec:pf}, we examine the changes in the receptive fields of the grid-based and point-based models during the formulation learning process. In Sections \ref{sec:backbone}, \ref{sec:neck}, and \ref{sec:head}, we respectively introduce the specific structures of the PDM-SSD model's backbone, neck, and detection head. Finally, in Section \ref{sec:loss}, we explain the approach of joint learning in the model and the loss function.

\subsection{Problem formulation}
\label{sec:pf}
Let $P=\{p_n\}^N_{n=1}$ be a set of $N$ observed LiDAR points belonging to a scaene, where $p_n \in \mathbb{R}^{3}$ is a 3D point represented with spatial coordinates. Let $C$ be the centers location $C=\{c_m\}^M_{m=1}$ of the annotated ground-truth $M$ bounding boxes, $c_m=[c_{mx},~c_{my},~c_{mz},]\in \mathbb{R}^{3}$.
Due to the limitations of the resolution of the LiDAR, climate conditions, and occlusions, sparse and incomplete targets often appear in the scene, as shown in Fig. \ref{fig3}. Assuming the point cloud $P$ is voxelized into initial pillars $V=\{v_k\}^K_{k=1}$, where $K$ represents the number of pillars. Taking 2D sparse convolution as an example, after learning in the Grid-based detector with the convolutional layers stacked in the backbone network, the feature map can be simplified as:
\begin{equation}
	F^{g}_{i+1}=Pool(Conv(Padding(F^{g}_{i})))
\end{equation}
$i$ is the number of convolutional layers of the backbone. The padding layers in the network allocate the unoccupied space in the original point cloud, and the convolution layers and pooling layers fill these spaces with features. The receptive field of the current layer is:
\begin{equation}
	RF_{i+1}=RF_i+(k-1)\times S_i
\end{equation}
\begin{equation}
	S_i=\prod^{i}_{i=1}Sride_i
\end{equation}
Among them, $RF_{i+1}$ represents the receptive field of the current layer, $RF_i$ represents the receptive field of the previous layer, and $k$ represents the size of the convolution kernel. It can be seen that the receptive field of the model continuously increases, even learning features for the unoccupied space in the original point cloud. The feature maps of the Point-based backbone network can be represented as follows:
\begin{equation}
	\label{eq4}
	F^{p}_{i+1}=Pool(Group(Sampling(F^{p}_{i})))
\end{equation}
$i$ represents the stage of feature learning, $Sampling$ is the downsampling operation, $Group$ is the local feature aggregation operation, $Pool$ is the pooling layer, which is treated as a symmetric function to address the unordered nature of point clouds. Here, $F^{p}_{0}=E_{\theta}(P)$, where $E_{\theta}(\dots)$ is the feature augmentation network, usually a multi-layer perceptron or graph neural network, which expands the feature space of point clouds. As the number of learning stages increases, the query radius of the local feature extractor becomes larger to increase the receptive field of the model. However, the model only takes the original point cloud as input, and in the case where the target has points only in the local region, even if the query radius increases, the receptive field of the model remains the same, and the learning of local features still stays at the previous stage. Fig. \ref{fig3} illustrates this phenomenon intuitively. Under this problem, the model's ability to predict the semantics and geometry information of upsampled points on the target will decrease. The main purpose of PDM-SSD is to alleviate this problem.



\subsection{PointNet style 3D backbone}
\label{sec:backbone}
The currently popular point-based detectors, such as 3DSSD, IA-SSD, and DBQ-SSD, all use PointNet-style 3D backbones as the point semantic and geometric information extractors, and their advantages in detection accuracy and inference speed have been well demonstrated. In order to ensure lightweightness, our PDM-SSD also adopts this structure of 3D backbone. To highlight the performance gains after solving the problem of discontinuous receptive fields, we do not make too many changes to the backbone, following the design of IA-SSD in general. The specific structure is shown in Fig. \ref{fig2}.

The point cloud is first passed through a vanilla feature augmentation network stacked with MLP to increase the feature dimension. Then, it enters a feature extractor consisting of four repeated SA (Set Abstraction) modules to learn geometric and semantic information, as shown in Eq. \ref{eq4}. The point-wise features outputted at each stage are denoted as $\{F^p_{1}, F^p_{2}, F^p_{3}, F^p_{4}\}$. Specifically, before each stage, we downsample the point cloud to reduce the model's spatial complexity and feature redundancy. Then, we use PointNet for local feature extraction, which can be divided into three steps: 1) Point indexing: using the current stage sampled points as centers, perform vanilla ball query within the range of the sampled points in the previous stage to index the $k$ nearest points to each center within a certain radius $r$. 2) Feature learning: use MLP to learn the features of the points within the ball, further improving the learning depth. 3) Pooling: perform max-pooling operation on the features of the points within the ball in the feature dimension. Due to the unordered nature of point clouds, this operation ensures that even if the order of the point cloud is changed, the pooled features remain unchanged. In each stage, we simultaneously use two sets of combinations of radius and sampling number for multi-scale feature extraction, and aggregate the multi-scale features after the pooling layer. In summary, the point quantities of $F^p_{1}, F^p_{2}, F^p_{3}, F^p_{4}$ decrease while the feature dimensions gradually increase.

It should be noted that in the model, $F^p_{1}, F^p_{2}$ adopt Farthest Point Sampling (FPS) downsampling method, while $F^p_{3}, F^p_{4}$ adopt foreground point downsampling method with semantic embedding, following the approach of IA-SSD. The former ensures the global coverage of sampling points in the presence of a large number of redundant points, reducing global information loss. The latter ensures a high recall rate of foreground points, reducing target information loss. The specific implementation is shown in Fig. \ref{fig2}. We add a network branch $S(\cdot)$ to the SA module in the second and third stages to extract the semantic information of sampling points. Then, the sampling points in the next stage are selected from the points with the highest probability of being foreground points, as shown in the following equation:
\begin{equation}
	sample\_index_{i+1}=topK(S(F_i)) \quad i=2,3
\end{equation}
From this, we obtain a small number of foreground sampling points with their rich local geometric and semantic information in point-wise features.


\subsection{Neck with Point Dilation Mechanism}
\label{sec:neck}
\begin{figure}[t]
	\begin{center}
		% \vspace{-0.2cm}
		\includegraphics[width=.5\textwidth]{AL_pics/dilation.pdf}
	\end{center}
	\vspace{-0.3cm}
	\caption{Point dilation operation. The point cloud is first projected onto a 2D binary occupancy grid and then dilated with a structural element. The new feature map covers many areas that were not occupied by the original point cloud, especially the region where the target box is located (blue box). The feature at the center position is of great interest to the detector.}
	\label{fig4}
	\vspace{-0.3cm}
\end{figure}

The point-wise receptive field is still limited to the space occupied by the original point cloud until \ref{sec:backbone}. In order to expand the learning scope of the model, we propose the Point Dilation Mechanism, which consists of two steps: point dilation and feature filling.

\textbf{Point Dilation (PD)}. Dilation (usually represented by $\oplus$) is one of the basic operations in mathematical morphology. Image dilation is a commonly used morphological processing algorithm in the field of image processing, which utilizes a structuring element to probe and expand the shapes present in the input image, aiming to connect connected regions or eliminate noise \cite{wu2010morphological}. Our PD method borrows the idea of image dilation to achieve spatial expansion. Specifically, for $F_4^p$ obtained from \ref{sec:backbone}, we use Eq. \ref{eq6} to sparsely project it onto a grid, resulting in $F_4^g$.
\begin{equation}
	\label{eq6}
	F_i^g=G(F_i^p, W,H,\epsilon)
\end{equation}
Where $G(\cdot)$ is the projection function, $W$ and $H$ are the width and height of the feature map respectively, and $\epsilon$ is another hyperparameter representing the spatial range of the computed point cloud. It is worth noting that in order to maintain unique indices for the sparse grid, for the sampled points that are projected onto the same grid, we sum their features along the feature dimension to obtain the feature of the current cell. At this point, the feature map is very sparse. We set the occupied cells to 1 and binarize the feature map, and then perform dilation operation:
\begin{equation}
	F_i^g\oplus B=\cup_{b\in B} {F_i^g}_b
\end{equation}
Where $B$ is a structuring element, and we use a $5\times 5$ matrix consisting entirely of 1s. This process is illustrated in Fig. \ref{fig5}, and the dilated binary feature map contains a significant amount of unoccupied space in the original point cloud.

\begin{figure}[t]
	\begin{center}
		% \vspace{-0.2cm}
		\includegraphics[width=.5\textwidth]{AL_pics/ff.pdf}
	\end{center}
	\vspace{-0.3cm}
	\caption{Feature Filling operation. We propose a feature filling method based on spatial separation coefficient. We use point-wise feature learning for Angle Coefficient and Scale Coefficient. The former is achieved by the superposition of spherical harmonics, while the latter is achieved by Gaussian probability density function. The new feature is the weighted sum of the inflated center feature and these two coefficients.}
	\label{fig5}
	\vspace{-0.3cm}
\end{figure}

\textbf{Feature Filling (FF)}. Although PD expands the occupancy space of the feature map, there are no learnable features available on these new cells, which presents a significant challenge in filling the features. We believe that the filled features must adhere to several principles: 1) Learnability: In theory, unoccupied cells and occupied cells should have equal importance in providing information, especially for the position of the target area. We must ensure that these cells have sufficient learning depth and flexibility. 2) Spatial correlation: Previous works such as 3DSSD and IA-SSD have shown that point-wise features at this stage can already regress the rough center position of the target. Therefore, the newly filled features should have a certain correlation with the dilation center to prevent feature fragmentation and preserve the predictive ability of the original dilation center. 3) Cross-correlation: It is evident that multiple dilation centers may affect the features of the same cell. The current cell should have the ability to connect multiple dilation centers to achieve multi-center interaction and learn features with a larger receptive field. To address this, we propose a feature learning method based on spatial separation coefficients. Specifically, we consider the angle and scale aspects and fill the angle and scale influence coefficients for the structuring element $B$. The features of the newly filled cell are then obtained by weighting the dilation center with the separation coefficients. For cells affected by multiple centers, a simple height compression is used to aggregate the features.

\textbf{Angle Coefficient (AC) $\alpha$.} The perception of objects in 3d space by humans or machines is influenced by the observation angle and viewpoint. Taking ray tracing as an example, the color of a certain grid in space varies when observed from different angles. We extend this consensus to feature space, where the features in the surrounding space are related to the angle between them and the expansion center. The conditions of learnability and angle correlation naturally lead us to think of the commonly used spherical harmonics coefficients in simple lighting descriptions.
\begin{equation}
	\label{eq8}
	\nabla^2 f=\frac{1}{r^2}\frac{\partial}{\partial r}(r^2\frac{\partial f}{\partial r}) + \frac{1}{r^2 \sin{\theta}} \frac{\partial}{\partial \theta} (\sin{\theta} \frac{\partial f}{\partial \theta}) + \frac{1}{r^2 \sin{\theta}^2}\frac{\partial^2 f}{\partial \varphi
		^2}=0
\end{equation}
The spherical harmonics are the angular part of the solution to the Laplace equation in spherical coordinates. The Laplace equation in spherical coordinates can be written as Eq. \ref{eq8}. Using the method of separation of variables, we can assume that $f(r,\theta,\varphi)=R(r)Y(\theta,\varphi)=R(r)\Theta(\theta)\Phi(\varphi)$. Here, $Y(\theta,\varphi)$ represents the angular part of the solution, which is also known as the spherical harmonics. More intuitively, the spherical harmonics can also be expressed as:
\begin{equation}
	\label{eq9}
	Y^m_l(\theta,\varphi)=(-1)^m\sqrt{\frac{(2l+1)}{4\pi}\frac{(l-|m|)!}{(l+|m|)!}}P^m_l(\cos{\theta})\exp(im\varphi)
\end{equation}
\begin{equation}
	\label{eq10}
	P^m_l(x)=(1-x^2)^{\frac{|m|}{2}}\frac{d^{|m|}}{dx^{|m|}}P_l(x)
\end{equation}
\begin{equation}
	\label{eq11}
	P_l(x)=\frac{1}{2^l l!}\frac{d^l}{dx^l}(x^2-1)^l
\end{equation}
The spherical harmonic function is only dependent on angles, where $l$ and $m$ are the degree index and order of the associated Legendre polynomial $P^m_l$. In computer graphics, the spherical harmonic function is similar to the Fourier transform in representing lighting. The result is obtained by weighting multiple spherical harmonic coefficients with the basis of spherical harmonic functions. The more spherical harmonic coefficients are used, the stronger the expressive power and the closer it is to the original function. In this method, we treat the dilation center as a sphere and calculate the angular coefficients of the new cell on the structuring element $B$ based on the spherical harmonic coefficients. Specifically, we use a prediction network $SH_\theta(\cdot)$ to regress the 16 spherical harmonic coefficients of the fourth order in $F^g_4$ for each dilation center. Then, the weighted calculation result is used as the angular coefficients of the new cell, as shown in Eq. \ref{eq12}.
\begin{equation}
	\label{eq12}
	\alpha=\sum_{l=0} \sum_{m=-1}^{l+1} c_l^m Y^m_l(\theta,\varphi)
\end{equation}
The input of Eq. \ref{eq12} are the angle between the center point of the newly inflated cell and the center of the cell where the inflation occurs.

\textbf{Scale Coefficient (SC) $\beta$.} As mentioned above, the current point-wise feature $F^g_4$ already contains certain geometric information, so the new cell features around it are scale-dependent on the dilation center. Similar to AC, SC is also reflected by the values filled in the structuring element $B$. Specifically, we use a Gaussian density function to calculate SC, taking the relative position between the dilated new cell and the cell where the dilation center is located, as well as the learned variance, as inputs. The formula is as follows:
\begin{equation}
	\label{eq13}
	\beta= G(x,\mu,\Sigma)
\end{equation}
\begin{equation}
	\label{eq14}
	G(x,\mu,\Sigma)=G(x_1,x_2,\dots,x_D,\mu,\Sigma)=\frac{1}{(2\pi)^{D/2}{\vert\Sigma\vert}^{1/2}} exp(-\frac{1}{2}(x-\mu)^T{\Sigma}^{-1}(x-\mu))
\end{equation}

\begin{figure}[t]
	\begin{center}
		% \vspace{-0.2cm}
		\includegraphics[width=.5\textwidth]{AL_pics/coffiefusion.pdf}
	\end{center}
	\vspace{-0.3cm}
	\caption{(a) Coefficients fusion. In order to maintain the nonlinearity of different cell features, we first decompose the point-wise features and weight them separately using AC and SC. The final feature is the sum of these two parts. (b) Height compression. For cells that are influenced by multiple dilation centers, we directly add their multiple features together, retaining the effect of each dilation center.}
	\label{fig6}
	\vspace{-0.3cm}
\end{figure}

In this method, $G(x, \mu, \Sigma)$ represents independent and identically distributed bivariate Gaussians, where $\mu$ is the position of the inflated center cell and $x$ is the position of the new cell. $\Sigma=\sigma^2E$ is a scaled identity matrix. We also add a scale prediction branch network $S_\theta(\cdot)$ to regress the variance of the inflated center, and then multiply it by a 2D identity matrix $E$ to calculate the final scale coefficient using Eq. \ref{eq13}.

\textbf{Coefficients Fusion.} The new features filled during the dilation process with angle and scale coefficients have a direct linear relationship with the dilation center, which is detrimental to the robustness of deep models. To address this, we adopt a channel-splitting approach for coefficients fusion, as shown in Fig. \ref{fig6} (a). Specifically, we split the point-wise features of the dilation center into two parts in the depth dimension, and multiply them with the angle and scale coefficients respectively. Finally, we sum them up to obtain the new point-wise features. Through this operation, the relationship variables between the new features and the dilation center increase to binary, maintaining the non-linearity of the new features. For cells influenced by multiple dilation centers, we want to preserve the effects of all dilation centers to connect the surrounding features. We achieve this by using a height compression operation as shown in Fig. \ref{fig6} (b), where the effects of all active dilation centers at that location are stacked, reducing the spatial complexity of the training and inference processes. Thus, we obtain a relatively dense 2D feature map.


\subsection{Hybrid Head}
\label{sec:head}
\begin{figure}[t]
	\begin{center}
		% \vspace{-0.2cm}
		\includegraphics[width=.7\textwidth]{AL_pics/head.pdf}
	\end{center}
	\vspace{-0.3cm}
	\caption{Joint Learning. In the case of sparsity and extremely incomplete targets, on one hand, the vote points generated by the vote network may deviate from the target center, and on the other hand, the point-wise features have a prediction for the target category lower than the threshold. We use the auxiliary information generated by the neck to improve these issues. First, we supplement the vote point set with the learned heatmap. Then, we concatenate the point-wise features and grid features, and pass them through the channel attention network to form new features for predicting the parameters of the target box.}
	\label{fig7}
	\vspace{-0.3cm}
\end{figure}

We propose a hybrid detection head to simultaneously extract point-wise features and grid features, as shown in Fig. \ref{fig7}. For the point-wise features $F^p_4$, inspired by VoteNet, they are first input into a voting network to move the sampled points towards the center of the objects as much as possible. Then, the context features of the objects are extracted using the moved positions as centers. Finally, these features are used for semantic classification and regression of geometric parameters. This part follows the basic structure of a traditional point-based detection head. However, as described in Section \ref{sec:pf}, due to the limited receptive field, the information of objects in the scene is not fully explored, especially for sparse and locally scattered objects, where their voting centers may be inaccurate or the context features may have a probability of being classified as objects below the certain threshold.

The grid feature, with the support of angle and scale coefficients, not only expands the receptive field of point-wise features but also enhances the connectivity of features from multiple inflated centers, potentially enabling the extraction of holistic features from sparse and incomplete targets. To achieve this goal, we construct a sparse heatmap using annotation information to supervise the learning of grid features, following the design of the dense CenterHead in the training process. Additionally, we find that even treating this part of the network as an auxiliary learning task can significantly improve the performance of the point-based detection head. Therefore, we design the model from the perspectives of both auxiliary learning and joint learning.

\textbf{Auxiliary Learning (AL).} The $F^g_4$ used for heatmap regression is essentially obtained by dilating $F^p_4$ used for prediction. We believe that supervising $F^g_4$ with sparse heatmaps will also increase the receptive field of $F^p_4$, thereby obtaining more accurate detection results. In addition, another advantage of auxiliary learning is that the auxiliary network does not participate in prediction and inference stages of the model, thus preserving the inference speed of the point-based detector completely. We will demonstrate its superiority in the experimental section.

\textbf{Joint Learning (JL).} In joint learning, we will fully utilize grid features and learned scene heatmaps. As shown in Fig. \ref{fig7}, in addition to the votes generated by the vote network from point-wise features, we supplement the centers of the top $K$ maximum values in the predicted heatmap. These two sets of points are aggregated together to learn contextual features through the point-based head. Then, we contact the contextual point-wise features with the grid features of the cells where these points are located in the depth dimension. Finally, after a simple channel attention, the features are used for semantic classification and detection box parameter regression. The above operations can not only supplement the spatial position information of the scene targets provided by the heatmap, but also adaptively fuse point-wise features and grid features, thereby compensating for the inaccuracies caused by the limited receptive field of the point-based detector and the low semantic probabilities of the targets. We will explain this in detail in \ref{sec:pdm}.


\subsection{End-to-End Learning}
\label{sec:loss}
The proposed PDM-SSD can be trained in an end-to-end fashion. In the 3D backbone, the sampling loss is calculated.
\begin{equation}
	L_{sample}=-\sum_{c=1}^C{(Mask_i \cdot CELoss_i)}
\end{equation}
\begin{equation}
	CELoss_i = s_i \log(\hat{s_i}) + (1-s_i)\log{1-\hat{s_i}}
\end{equation}
\begin{equation}
	Mask_i=\sqrt[3]{\frac{\min{f^*,b^*}}{\max{f^*,b^*}}\times \frac{\min{l^*,r^*}}{\max{l^*,r^*}} \times \frac{\min{u^*,d^*}}{\max{u^*,d^*}}}
\end{equation}
$CELoss$ is the cross-entropy loss for predicting the category of sampled points, which first appeared in IA-SSD. It greatly improves the recall rate of foreground points in sampled points. $Mask$ is the weight for the centrality of sampled points. $f^*,b^*,l^*,r^*,u^*,d^*$ represent the distances between the sampled points and the six faces of the target box. We borrowed the design from 3DSSD, which believes that points closer to the center are more conducive to accurately regressing semantic categories and geometric parameters. By multiplying the centrality with the cross-entropy loss, the closer the sampled points are to the center, the greater the loss will be. The model will prioritize improving the foreground probability $s_i$ of these points, so that during inference, the model will prioritize selecting these points. Although SPSNet has shown that points closer to the center are not necessarily better, it is still better than methods that sample foreground points equally. Besides, SPSNet requires extra time to learn the stability of points.

The loss in the detection head is divided into two parts: the point-based loss $L_{p}$ and the grid-based heatmap prediction loss $L_{heatmap}$. The former consists of three components: the loss of the vote points $L_{vote}$, the loss of point semantics prediction $L_{cls}$, and the loss of target box geometric parameter regression $L_{reg}$. Additionally, we also include a regularization loss in the training process of PDM-SSD.
\begin{equation}
	L_{all}=L_{sample}+L_{p}+L_{heatmap}+L_2
\end{equation}
\begin{equation}
	L_{p}=L_{vote}+L_{cls}+L_{reg}
\end{equation}
In particular, the box generation loss can be further decomposed into location, size, angle-bin, angle-res, and corner parts:
\begin{equation}
	L_{reg}=L_{loc}+L_{size}+L_{angle-bin} + L_{angle-res} + L_{corner}
\end{equation}
All these losses will be jointly optimized using a multi-task learning approach.

\section{Experiments}
\label{sec:experiments}
In this section, we will provide detailed experimental results to demonstrate the efficiency and accuracy of PDM-SSD. Specifically, we introduced the specific settings and implementation details of the experiments in Section \ref{sec:setup}. Then, the comparison results between PDM-SSD and current state-of-the-art methods were reported in Section \ref{sec:comparison}. Following that, in Section \ref{sec:ablation}, a ablation study was conducted to demonstrate the rationality of the model design. Furthermore, the inference efficiency of PDM-SSD was analyzed in Section \ref{sec:runtime}. Finally, in Section \ref{sec:pdm}, we provided a large number of instances to demonstrate the superiority of PDM in sparse and incomplete object detection.

\subsection{Setup}
\label{sec:setup}
% \usepackage[normalem]{ulem}
% \usepackage{multirow}
% \usepackage{booktabs}


\begin{table}
	\centering
	\caption{Quantitative comparison with state-of-the-art methods on the KITTI \textit{test} set for \textit{Car} BEV and 3D detection, under the evaluation metric of 3D Average Precision ($AP$) of 40 sampling recall points. The best and our PDM-SSD results are highlighted in \textbf{bold} and \uline{underlined}, respectively}
	\resizebox{\textwidth}{!}{
	\begin{tabular}{c|c|c|ccc|ccc} 
		\toprule[0.75mm]
		\multirow{3}{*}{Method} & \multirow{3}{*}{Structure} & \multirow{3}{*}{Type} & \multicolumn{3}{c|}{$AP_{3D}@Car$(IoU=0.7)}                 & \multicolumn{3}{c}{$AP_{BEV}@Car$(IoU=0.7)}                  \\ 
		\cline{4-9}
		&                            &                       & \multicolumn{3}{c|}{R40}                         & \multicolumn{3}{c}{R40}                           \\
		&                            &                       & Easy           & Moderate       & Hard           & Easy           & Moderate       & Hard            \\ 
		\hline
		VoxelNet \cite{zhou_voxelnet_2017}                & Voxel-based                & 1-stage               & 77.47          & 65.11          & 57.73          & 87.95          & 78.39          & 71.29           \\
		PointPillars \cite{lang_pointpillars_2019}            & Voxel-based                & 1-stage               & 82.58          & 74.31          & 68.99          & 90.07          & 86.56          & 82.81           \\
		SECOND \cite{yan2018second}                  & Voxel-based                & 1-stage               & 84.65          & 75.96          & 68.71          & 89.39          & 83.77          & 78.59           \\
		3DIoULoss \cite{zhou2019iou}               & Voxel-based                & 2-stage               & 86.16          & 76.5           & 71.39          & 90.23          & 86.61          & 86.37           \\
		TANet \cite{liu2020tanet}                   & Voxel-based                & 1-stage               & 84.39          & 75.94          & 68.82          & 91.58          & 86.54          & 81.19           \\
		Part-A2 \cite{shi2020points}                 & Voxel-based                & 2-stage               & 87.81          & 78.49          & 73.51          & 91.7           & 87.79          & 84.61           \\
		CIASSD \cite{zheng2021cia}                  & Voxel-based                & 1-stage               & 89.59          & 80.28          & 72.87          & 93.74          & 89.84          & 82.39           \\
		SASSD \cite{he2020structure}                   & Voxel-based                & 1-stage               & 88.75          & 79.79          & 74.61          & 93.74          & 89.84          & 82.39           \\
		Associate-3Det \cite{du2020associate}          & Voxel-based                & 1-stage               & 85.99          & 77.4           & 70.53          & 91.4           & 88.09          & 82.96           \\
		SVGA-Net \cite{he2022svga}                & Voxel-based                & 1-stage               & 87.33          & 80.47          & 75.91          & -              & -              & -               \\ 
		\hline
		Fast Point R-CNN \cite{chen2019fast}        & Point-Voxel                & 2-stage               & 85.29          & 77.4           & 70.24          & 90.87          & 87.84          & 80.52           \\
		STD \cite{yang2019std}                     & Point-Voxel                & 2-stage               & 87.92          & 79.71          & 75.09          & -              & -              & -               \\
		PV-RCNN \cite{shi2020pv}                 & Point-Voxel                & 2-stage               & \uline{90.25}  & 81.43          & 76.82          & \uline{94.98}  & \uline{90.62}  & 86.14           \\
		EQ-PVRCNN \cite{yang2022unified}               & Point-Voxel                & 2-stage               & 90.13          & \uline{82.01}  & \uline{77.53}  & 94.55          & 89.09          & \uline{86.42}   \\
		VIC-Net \cite{jiang2021vic}                 & Point-Voxel                & 1-stage               & 88.25          & 80.61          & 75.83          & -              & -              & -               \\
		HVPR \cite{noh2021hvpr}                    & Point-Voxel                & 1-stage               & 86.38          & 77.92          & 73.04          & -              & -              & -               \\ 
		\hline
		PointRCNN \cite{shi2019pointrcnn}               & Point-based                & 2-stage               & 86.96          & 75.64          & 70.7           & 92.13          & 87.39          & 82.72           \\
		3D IoU-Net \cite{li20203d}              & Point-based                & 2-stage               & 87.96          & 79.03          & 72.78          & 94.76          & 88.38          & 81.93           \\
		3DSSD \cite{yang20203dssd}                   & Point-based                & 1-stage               & 88.36          & 79.57          & 74.55          & 92.66          & 89.02          & 85.86           \\
		IA-SSD \cite{zhang2022not}                  & Point-based                & 1-stage               & 88.34          & 80.13          & 75.04          & 92.79          & 89.333         & 84.35           \\
		DBQ-SSD \cite{yang2022dbq}                 & Point-based                & 1-stage               & 87.93          & 79.39          & 74.4           & -              & -              & -               \\
		\textbf{PDM-SSD}                 & Point-based                & 1-stage               & \textbf{88.74} & \textbf{80.87} & \textbf{75.63} & \textbf{93.07} & \textbf{89.92} & \textbf{85.12}  \\
		\bottomrule
	\end{tabular} }
	\label{tabel1}
\end{table}



\textbf{Benchmark datasets.} The KITTI dataset is a dataset sponsored by the Karlsruhe Institute of Technology and the Toyota Technological Institute at Chicago for research in the field of autonomous driving. The widely-used dataset contains 7481 training samples with annotations in the camera field of vision and 7518 testing samples. Following the common protocol, we further divide the training samples into a training set (3,712 samples) and a validation set (3,769 samples). Additionally, the samples are divided into three difficulty levels: simple, moderate, and hard based on the occlusion level, visibility, and bounding box size. The moderate average precision is the official ranking metric for both 3D and BEV detection on the KITTI website.

\textbf{Evaluation metrics.} To provide a comprehensive performance evaluation, we evaluated our PDM-SSD on both the KITTI 3D and BEV object detection benchmarks. Generally, average precision ($AP$) based on Intersection over Union (IoU) is commonly used for both the 3D and BEV tasks. The experiments primarily focused on the commonly-used \textit{Car} category and were evaluated using the average precision metric with an IoU threshold of 0.7. To ensure an objective comparison, we utilized both the $AP$ with 40 recall points ($AP_{40}$) and the {AP} with 11 recall points ($AP_{11}$). The 3D-NMS threshold for metric calculation was set at 0.1, and the object score threshold was set at 0.1.

\textbf{Training details.} To ensure fair comparison, the training parameters of PDM-SSD are kept consistent with IA-SSD. Specifically, we use the Adam optimizer with $\beta_1=0.9$ and $\beta_2=0.85$ to optimize PDM-SSD. The weight decay coefficient is set to 0.01, and the momentum coefficient is set to 0.9. The model is trained for 80 epochs with a batch size of 16 on a Nvidia A40 GPU. The initial learning rate is set to 0.01, which is decayed by 0.1 at 35 and 45 epochs and updated with the one cycle policy. We initialized the weights of the heatmap prediction network in the neck module with values generated from a normal distribution. The training range of point clouds in the KITTI dataset is $[0,-40,-3,70.4,40,1]$, corresponding to $[x_{min},y_{min},z_{min},x_{max},y_{max},z_{max}]$. The grid size in the neck module is $176\times 200$, the size of the structural element is $5\times 5$, and the scale of the spherical harmonic coefficients is 3. The hybrid head predicted heatmaps and added the top 256 points with the highest foreground probabilities to the vote point set. We applied common scene-level data augmentation strategies to enhance the robustness of the model, including: 1) randomly rotating the scene along the $z$ axis within the range of $[-4/\pi,4/\pi]$ with a probability of 50\%; 2) randomly flipping the scene along the $x-z$ plane; 3) randomly scaling the scene within the range of $[0.95,1.05]$. Moreover, a sufficient number of targets was necessary for PDM-SSD to learn a more complete feature distribution. To achieve this, we employed object-level data augmentation methods to transform objects from other scenes. Specifically, 15 cars were copied to the current scene.

\textbf{Base detector.} Currently, the well-performing point-based detectors are all PointNet-style detectors, and their structures are quite similar, with the main difference lying in the different downsampling methods. PDM-SSD takes IA-SSD, which performs the best in terms of accuracy and inference speed on KITTI, as the base detector. In the experiments, the backbone of PDM-SSD used for auxiliary training is exactly the same as IA-SSD, making it very intuitive to see the advantages of PDM-SSD. In Section \ref{sec:pdm}, we will also provide a detailed comparison between PDM-SSD and IA-SSD at the object level.

\begin{figure}
	\centering
	\includegraphics[width=1\linewidth]{AL_pics/detective_result.pdf}
	\caption{The detection results of PDM-SSD on a subset of KITTI \textit{val} set samples. From left to right, respectively: image and point clouds with predicted bounding boxes, space covered by grids after point dilatation, ground truth of space coverage heatmap in the scene, and predicted values of space coverage heatmap in the scene.}
	\label{fig:detectiveresult}
\end{figure}

\subsection{Comparison with State-of-the-Arts}
\label{sec:comparison}
% \usepackage[normalem]{ulem}
% \usepackage{multirow}
% \usepackage{booktabs}


\begin{table}
	\centering
	\caption{Quantitative comparison with state-of-the-art methods on the KITTI \textit{val} set for \textit{car} BEV and 3D detection, under the evaluation metric of 3D Average Precision (AP) of 11 and 40 sampling recall points. The best and our PDM-SSD results are highlighted in \textbf{bold} and \uline{underlined}, respectively}
	\resizebox{\textwidth}{!}{
	\begin{tabular}{c|c|c|ccc|ccc|ccc|ccc} 
		\toprule[0.75mm]
		\multirow{3}{*}{Method} & \multirow{3}{*}{Structure} & \multirow{3}{*}{Type} & \multicolumn{6}{c|}{$AP_{3D}@Car$(IoU=0.7)}                                                                & \multicolumn{6}{c}{$AP_{BEV}@Car$(IoU=0.7)}                                                                             \\ 
		\cline{4-15}
		&                            &                       & \multicolumn{3}{c|}{R11}                         & \multicolumn{3}{c|}{R40}                                 & \multicolumn{3}{c|}{R11}                                 & \multicolumn{3}{c}{R40}                           \\
		&                            &                       & Easy           & Moderate       & Hard           & Easy           & Moderate               & Hard           & Easy                   & Moderate       & Hard           & Easy           & Moderate       & Hard            \\
		\hline
		VoxelNet* \cite{zhou_voxelnet_2017}               & Voxel-based                & 1-stage               & 81.97          & 65.46          & 62.85          & -              & -                      & -              & 89.6                   & 84.81          & 78.57          & -              & -              & -               \\
		PointPillars* \cite{lang_pointpillars_2019}           & Voxel-based                & 1-stage               & 86.44          & 77.28          & 74.65          & 87.75          & 78.38                  & 75.18          & 89.66                  & 87.16          & 84.39          & 92.05          & 88.05          & 86.67           \\
		SECOND* \cite{yan2018second}                 & Voxel-based                & 1-stage               & 88.61          & 78.62          & 77.22          & 90.55          & 81.61                  & 78.61          & 90.01                  & 87.92          & 86.45          & 92.42          & 88.55          & 87.65           \\
		SECOND-iou* \cite{yan2018second}             & Voxel-based                & 1-stage               & 84.93          & 76.3           & 75.98          & 86.77          & 79.23                  & 77.17          & 87.9                   & 76.3           & 75.95          & 90.23          & 86.61          & 86.37           \\
		TANet \cite{liu2020tanet}                   & Voxel-based                & 1-stage               & 88.17          & 77.75          & 75.31          & -              & -                      & -              & -                      & -              & -              & -              & -              & -               \\
		Part-A2* \cite{shi2020points}                 & Voxel-based                & 2-stage               & 89.55          & 79.41          & 78.85          & \uline{92.15}  & 82.91                  & \uline{82.05}  & 90.2                   & 87.96          & \uline{87.56}  & 92.9           & 90.01          & \uline{88.35}   \\
		Part-A2-free* \cite{shi2020points}           & Voxel-based                & 1-stage               & 89.12          & 78.73          & 77.98          & 91.68          & 80.31                  & 78.1           & 90.1                   & 86.79          & 84.6           & 92.84          & 88.15          & 86.16           \\
		SASSD \cite{he2020structure}                   & Voxel-based                & 1-stage               & \uline{89.69}  & 79.41          & 78.33          & -              & -                      & -              & 90.59                  & 88.43          & 87.49          & -              & -              & -               \\
		Associate-3Det \cite{du2020associate}          & Voxel-based                & 1-stage               & 0              & 79.17          & -              & -              & -                      & -              & -                      & -              & -              & -              & -              & -               \\
		CIASSD \cite{zheng2021cia}                  & Voxel-based                & 1-stage               & 0              & \uline{79.81}  & -              & -              & -                      & -              & -                      & -              & -              & -              & -              & -               \\ 
		\hline
		PV-RCNN* \cite{shi2020pv}                & Point-Voxel                & 2-stage               & 89.26          & 79.16          & \uline{79.39}  & 91.37          & 82.78                  & 80.24          & 89.98                  & 87.7           & 86.59          & 92.72          & 88.59          & 88.04           \\
		Fast Point R-CNN \cite{chen2019fast}        & Point-Voxel                & 2-stage               & 0              & 79             & -              & -              & -                      & -              & -                      & -              & -              & -              & -              & -               \\
		STD \cite{yang2019std}                     & Point-Voxel                & 2-stage               & 0              & 79.8           & -              & -              & -                      & -              & -                      & -              & -              & -              & -              & -               \\
		VIC-Net \cite{jiang2021vic}                 & Point-Voxel                & 1-stage               & 0              & 79.25          & -              & -              & -                      & -              & -                      & -              & -              & -              & -              & -               \\ 
		\hline
		PointRCNN* \cite{shi2019pointrcnn}              & Point-based                & 2-stage               & 88.95          & 78.67          & 77.78          & 91.83          & 80.61                  & 78.18          & 89.92                  & 78.67          & 77.78          & 93.07          & 88.85          & 86.73           \\
		PointRCNN-iou* \cite{shi2019pointrcnn}          & Point-based                & 2-stage               & 89.09          & 78.78          & 78.26          & 91.89          & 80.68                  & 78.41          & 90.19                  & 87.49          & 85.91          & \uline{94.99}  & 88.82          & 86.71           \\
		3DSSD* \cite{yang20203dssd}                  & Point-based                & 1-stage               & 88.82          & 78.58          & 77.47          & -              & -                      & -              & 90.27                  & 87.87          & 86.35          & -              & -              & -               \\
		IA-SSD* \cite{zhang2022not}                 & Point-based                & 1-stage               & 88.78          & 79.12          & 78.12          & 89.52          & 82.86                  & 80.05          & 90.34                  & 88.19          & 86.78          & 93.17          & 89.54          & 88.64           \\
		DBQ-SSD \cite{yang2022dbq}                 & Point-based                & 1-stage               & 0              & 79.56          & -              & -              & -                      & -              & -                      & -              & -              & -              & -              & -               \\
		SPSNet* \cite{liang2023spsnet}                 & Point-based                & 1-stage               & 89.19          & 79.29          & 78.2           & 91.52          & 83.03                  & 80.15          & 90.31                  & 88.72          & 87.31          & 93.2           & 91.21          & 88.9            \\
	\textbf{	PDM-SSD(A)}              & Point-based                & 1-stage               & \textbf{89.12} & \textbf{79.37} & \textbf{78.33} & \textbf{91.57} & \textbf{83.23}         & \textbf{80.37} & \textbf{\uline{92.18}} & \textbf{88.54} & \textbf{86.86} & \textbf{93.17} & \textbf{91.1}  & \textbf{88.67}  \\
		\textbf{PDM-SSD}                 & Point-based                & 1-stage               & \textbf{89.3}  & \textbf{79.75} & \textbf{78.47} & \textbf{91.96} & \textbf{\uline{83.31}} & \textbf{80.59} & \textbf{90.25}         & \textbf{88.64} & \textbf{87.03} & \textbf{93.26} & \textbf{91.24} & \textbf{88.87}  \\
		\bottomrule
	\end{tabular}}
\label{tabel2}
\end{table}

\textbf{Note.} Our model is trained in a single-stage multi-class manner. Thanks to the contributions of mmlab \cite{openpcdet2020}, the PDM-SSD model was trained using their open-source OpenPCDet\footnote{\url{https://github.com/open-mmlab/OpenPCDet}} architecture. Additionally, to show respect and fair competition to the authors of the baseline models, we retested their models with our environment on the \textit{val} set, marked as $X^*$. In the following sections, unless otherwise specified, PDM-SSD refers to the results of joint training, PDM-SSD(J).

% \usepackage[normalem]{ulem}
% \usepackage{multirow}
% \usepackage{booktabs}


\begin{table}
	\centering
	\caption{Quantitative comparison with state-of-the-art methods on the KITTI \textit{val} set for \textit{cyclist} BEV and 3D detection, under the evaluation metric of 3D Average Precision (AP) of 11 and 40 sampling recall points. The best and our PDM-SSD results are highlighted in \textbf{bold} and \uline{underlined}, respectively}
	\resizebox{\textwidth}{!}{
	\begin{tabular}{c|c|c|ccc|ccc|ccc|ccc} 
		\toprule[0.75mm]
		\multirow{3}{*}{Method} & \multirow{3}{*}{Structure} & \multirow{3}{*}{Type} & \multicolumn{6}{c|}{$AP_{3D}@Cyclist$(IoU=0.5)}                                                                   & \multicolumn{6}{c}{$AP_{BEV}@Cyclist$(IoU=0.5)}                                                                            \\ 
		\cline{4-15}
		&                            &                       & \multicolumn{3}{c|}{R11}                                & \multicolumn{3}{c|}{R40}                                 & \multicolumn{3}{c|}{R11}                         & \multicolumn{3}{c}{R40}                                  \\
		&                            &                       & Easy                  & Moderate       & Hard           & Easy                   & Moderate       & Hard           & Easy           & Moderate       & Hard           & Easy           & Moderate               & Hard           \\ 
		\hline
		VoxelNet*\cite{zhou_voxelnet_2017}               & Voxel-based                & 1-stage               & 67.17                 & 47.65          & 45.11          & -                      & -              & -              & 74.41          & 52.18          & 52.49          & -              & -                      & -              \\
		PointPillars*\cite{lang_pointpillars_2019}           & Voxel-based                & 1-stage               & 86.47                 & 68.94          & 66.72          & 89.81                  & 69.71          & 66.91          & 86.66          & 67.21          & 66.72          & 90.01          & 67.45                  & 66.91          \\
		SECOND* \cite{yan2018second}                 & Voxel-based                & 1-stage               & 80.61                 & 67.14          & 63.11          & 82.97                  & 66.74          & 62.78          & 84.02          & 70.7           & 65.48          & 88.04          & 71.16                  & 66.89          \\
		SECOND-iou* \cite{yan2018second}             & Voxel-based                & 1-stage               & 80.44                 & 64.26          & 60.19          & 83.16                  & 63.75          & 60.24          & 84.64          & 66.83          & 63.43          & 86.45          & 67.73                  & 63.38          \\
		TANet \cite{liu2020tanet}                   & Voxel-based                & 1-stage               & 85.98                 & 64.95          & 60.4           & -                      & -              & -              & -              & -              & -              & -              & -                      & -              \\
		Part-A2* \cite{shi2020points}                 & Voxel-based                & 2-stage               & 85.5                  & 69.93          & 65.48          & 90.4                   & 70.1           & 66.9           & 86.88          & 73.32          & 70.84          & 91.93          & 74.6                   & 70.61          \\
		Part-A2-free* \cite{shi2020points}           & Voxel-based                & 2-stage               & 87.95                 & \uline{74.29}  & 69.91          & 91.92                  & 75.33          & 70.59          & 88.75          & \uline{76.31}  & \uline{73.68}  & 93.23          & 78.51                  & \uline{73.94}  \\
		SASSD \cite{he2020structure}                   & Voxel-based                & 1-stage               & 82.8                  & 63.37          & 61.6           & -                      & -              & -              & 86.78          & 71.54          & 65.85          & -              & -                      & -              \\ 
		\hline
		PV-RCNN* \cite{shi2020pv}                & Point-Voxel                & 2-stage               & 84.3                  & 69.29          & 63.59          & 87.05                  & 69.52          & 65.19          & 88.38          & 73.62          & 70.77          & 93.36          & 75.04                  & 70.44          \\ 
		\hline
		PointRCNN* \cite{shi2019pointrcnn}              & Point-based                & 2-stage               & 86.76                 & 71.68          & 65.77          & 91.94                  & 70.98          & 66.72          & 88.41          & 74.31          & 67.96          & 93.91          & 74.71                  & 70.26          \\
		PointRCNN-iou* \cite{shi2019pointrcnn}          & Point-based                & 2-stage               & 86.36                 & 70.99          & 66.23          & 91.48                  & 71.46          & 66.74          & 88.51          & 74.54          & 68.12          & 94.06          & 75.07                  & 70.22          \\
		IA-SSD* \cite{zhang2022not}                 & Point-based                & 1-stage               & 87.28                 & 72.28          & 66.67          & 91.98                  & 72.73          & 68.14          & 88.27          & 73.72          & 70.65          & 93.3           & 74.81                  & 70.68          \\
		SPSNet* \cite{liang2023spsnet}                 & Point-based                & 1-stage               & 88.16                 & 74.01          & \uline{71.2}   & 93.16                  & \uline{75.51}  & \uline{71.1}   & \uline{93.58}  & 75.03          & 72.72          & \uline{95.39}  & 77.57                  & 73.22          \\
		\textbf{PDM-SSD(A)}              & Point-based                & 1-stage               & \textbf{89.12} & \textbf{72.47} & \textbf{67.23} & \textbf{92.43} & \textbf{73.13}         & \textbf{68.47} & \textbf{92.18} & \textbf{73.74} & \textbf{71.16} & \textbf{94.07} & \textbf{81.14}  & \textbf{70.67}  \\
		\textbf{PDM-SSD}                 & Point-based                & 1-stage               & \textbf{\uline{92.04}} & \textbf{72.92} & \textbf{67.88} & \textbf{\uline{93.83}} & \textbf{73.44} & \textbf{68.67} & \textbf{92.99} & \textbf{73.96} & \textbf{71.59} & \textbf{95.24} & \textbf{\uline{85.24}} & \textbf{71.1}  \\
		\bottomrule
	\end{tabular}}
\label{tabel3}
\end{table}

\textbf{Evaluation on KITTI Dataset.} Tables \ref{tabel1} and \ref{tabel2} present the detection performance of PDM-SSD and some state-of-the-art models on \textit{Car} objects in the KITTI \textit{test} and \textit{val} benchmarks. We report their metrics in both 3D and BEV perspectives. In the KITTI benchmark, \textit{Car} objects are divided into three subsets ("Easy," "Moderate," and "Hard") based on difficulty levels. The results on the "Moderate" subset are commonly used as the primary indicator for final ranking. To provide a more intuitive comparison of PDM-SSD's superiority, we categorize the comparative models into three types: point-based, voxel-based, and point-voxel-based, with PDM-SSD belonging to the first category. Table \ref{tabel3} shows the detection results of PDM-SSD on \textit{Cyclist} objects. Table \ref{tabel4} displays the detection metrics of PDM-SSD, IA-SSD, and SPSNet at an IoU threshold of 0.5. Although this threshold is not commonly used for comparing model detection performance, it can reflect the differences in the models' object recall rates.

% \usepackage{multirow}
% \usepackage{booktabs}
% \usepackage{colortbl}


\begin{table}
	\centering
	\caption{Comparing IA-SSD, SPSNet-SIA and PDM-SSD on the KITTI \textit{val} set for \textit{Car} when IoU threshold is 0.5.}
	\setlength{\extrarowheight}{0pt}
	\addtolength{\extrarowheight}{\aboverulesep}
	\addtolength{\extrarowheight}{\belowrulesep}
	\setlength{\aboverulesep}{0pt}
	\setlength{\belowrulesep}{0pt}
	\resizebox{\textwidth}{!}{
	\begin{tabular}{c|ccc|ccc|ccc|ccc} 
		\toprule
		\multirow{3}{*}{Method}                     & \multicolumn{6}{c}{$AP_{3D}@Car$(IoU=0.5)}         & \multicolumn{6}{c}{$AP_{BEV}@Car$(IoU=0.5)}         \\ 
		\cline{2-13}
		& \multicolumn{3}{c|}{R11} & \multicolumn{3}{c|}{R40} & \multicolumn{3}{c|}{R11} & \multicolumn{3}{c}{R40}   \\
		& Easy  & Moderate & Hard  & Easy  & Moderate & Hard  & Easy  & Moderate & Hard  & Easy  & Moderate & Hard   \\ 
		\hline
		IA-SSD                                      & 90.74 & 90.14    & 89.65 & 96.24 & 95.44    & 94.86 & 90.74 & 90.16    & 89.72 & 96.25 & 95.53    & 95.03  \\
		SPSNet-SIA                                  & 96.19 & 90.17    & 89.69 & 98.04 & 95.5     & 94.99 & 96.24 & 95.81    & 89.75 & 98.07 & 97.28    & 95.11  \\ 
		\hline
		PDM-SSD                                     & 96.46 & 96.11    & 89.64 & 98.3  & 97.39    & 94.86 & 96.42 & 95.96    & 89.74 & 98.3  & 97.38    & 94.86  \\
		\rowcolor[rgb]{0.69,0.69,0.69} - IA-SSD     & +5.72  & +5.97     & -0.01 & +2.06  & +1.95     & 0     & +5.68  & +5.8      & +0.02  & +2.05  & +1.85     & -0.17  \\
		\rowcolor[rgb]{0.69,0.69,0.69} - SPSNet-SIA & +0.27  & +5.94     & -0.05 & +0.26  & +1.89     & -0.13 & +0.18  & +0.15     & -0.01 & +0.23  & +0.1      & -0.25  \\
		\bottomrule
	\end{tabular}}
\label{tabel4}
\end{table}

\textbf{Analysis.} It can be seen that: 
1) In the KITTI \textit{test} split, PDM-SSD performs the best among point-based detectors, with significant improvements of $0.58\% mAP$ and $1.17\% mAP$ over IA-SSD and DBQ-SSD, respectively. We believe that this is a result of PDM-SSD addressing the limited receptive field issue (details in \ref{sec:pdm}). 
2) PDM-SSD has advantages over some voxel-based and point-voxel-based detectors, outperforming Fast Point R-CNN by ($3.45\%, 3.47\%, 5.39\%$) and ($2.2\%, 2.08\%, 4.6\%$) in 3D and bev detection with 40 recall points, respectively. This is exciting because high-speed inference capability of 3D detectors is crucial for autonomous vehicles. 
3) As shown in Table \ref{tabel2}, with only the addition of PDM for auxiliary training, PDM-SSD(A) also achieves considerable gains over IA-SSD, with ($2.05\%, 0.37\%, 0.32\%$) improvement in 3D detection with 40 recall points. This indicates that PDM can help point-wise feature learning for more holistic features of objects, further enhancing the model's learning efficiency. 
4) In addition to the \textit{Car} category, PDM-SSD also provides significant assistance in detecting targets in the \textit{Cyclist} category, as shown in Table \ref{tabel3}. It outperforms IA-SSD by a large margin ($4.76\%, 0.64\%, 1.21\%$) with 11 recall points in terms of $AP_{3D}$, and achieves the best performance in the \textit{easy} level. 
5) As shown in Table \ref{tabel4}, when the IoU is 0.5, PDM-SSD outperforms IA-SSD in all metrics. It improves $AP_{3D}$ by $5.97\%$ and $1.95\%$ at 11 and 40 recall points, respectively, mainly due to the improvement in recall rate. This indicates that the complementarity of heatmap to the voting point set is useful, and recall rate is a prerequisite for a good detector performance. 
6) We conducted a Friedman test analysis \cite{jamaludin2022novel} to evaluate the effectiveness of our work. The final results are as follows: The Friedman test rank was conducted for all point-based detectors for all levels in the KITTI test set with $\alpha = 0.05$ and a degree of freedom of $d_f = 6$. The $p$ for $AP$ is $0.011 (\chi^2 = 16.57)$. As a result, PDM-SSD has an average rank of 1, which is the highest compared to other existing methods for $AP_{3D}$.

We visualized the detection results of three scenes in KITTI \textit{val} sets with PDM-SSD in Fig \ref{fig:detectiveresult}. The figures from left to right are respectively: the image and point clouds with predicted boxes, followed by the space covered by the dilated grid, the ground truth of the heatmap representing the covered space in the scene, and the predicted values of the heatmap. It can be observed that after dilation, the model's learnable space range has significantly increased, achieving almost complete coverage of the target area, especially the center position. The predicted heatmap also successfully identifies all the targets, proving the reliability of our learning method, which plays a positive role in complementing the vote points.




\subsection{Ablation Study}
\label{sec:ablation}
% \usepackage{multirow}
% \usepackage{booktabs}


\begin{table}[!h]
	\centering
	\caption{Performance of the model at different SH degrees.}
	\begin{tabular}{c|c|ccc} 
		\toprule
		\multirow{2}{*}{SH degree} & \multirow{2}{*}{numbers} & \multicolumn{3}{c}{R40}                           \\
		&                          & Easy           & Moderate       & Hard            \\ 
		\hline
		2                          & 9                        & 91.1           & 82.89          & 82.29           \\
		3                          & 16                       & \textbf{91.96} & \textbf{83.31} & \textbf{80.59}  \\
		4                          & 25                       & 91.82          & 83.24          & \textbf{80.59}  \\
		\bottomrule
	\end{tabular}
\label{tabel5}
\end{table}

Next, we will conduct ablation studies to evaluate the performance of our proposed modules in PDM-SSD on the KITTI validation split.

\textbf{SH degree.} In the illumination representation model, the higher the degree, the higher the fidelity to real scenes. However, higher degrees also require more coefficients to describe, leading to increased computational complexity. In practical applications, a suitable order is typically chosen to balance accuracy and computational efficiency based on the requirements. Table \ref{tabel5} shows the performance of PDM-SSD with orders 2, 3, and 4. It can be observed that the model performance is comparable between orders 3 and 4, and superior to order 2. Therefore, in order to reduce the number of model parameters and improve computational efficiency, we set the order to 3 with a total of 16 coefficients, which is also a common setting in many rendering models \cite{fridovich2022plenoxels,kerbl20233d}.

\textbf{Coefficients fusion.} In order to maintain the non-linear relationship between the padding feature and the dilation center feature, we adopted a coefficient fusion strategy as shown in Fig. \ref{fig6}. In contrast, directly summing the angle coefficient and the scale coefficient as the feature weight of the new cell, we compared the performance changes brought by these two coefficient fusion strategies in Table. \ref{tabel6}. It can be seen that truncating the features and separately weighting them with the two coefficients can bring higher benefits.



\subsection{Runtime Analysis}
\label{sec:runtime}
\begin{figure}
	\centering
	\includegraphics[width=1\linewidth]{AL_pics/issue1.pdf}
	\caption{Issue1. The error in the voting point position regressed from the sampling points increases the difficulty of predicting the target box. The first row represents the ground truth bounding box (yellow). The second row shows the inference results of IA-SSD, with green boxes indicating predicted bounding boxes with prediction probabilities, blue points representing selected foreground points, and purple points representing vote points learned from these foreground points. The third row represents the prediction results of our PDM-SSD, with the addition of pseudo foreground points obtained from the predicted heatmap. The fourth row shows the Gaussian coefficients of the foreground points used for final prediction, and the fifth row shows the angle coefficients.}
	\label{fig:issue1}
\end{figure}

% \usepackage{multirow}
% \usepackage{booktabs}


\begin{table}[h]
	\centering
	\caption{Model performance under two coefficient fusion methods.}
	\begin{tabular}{c|ccc} 
		\toprule
		\multirow{2}{*}{Fusion mode} & \multicolumn{3}{c}{R40}                           \\
		& Easy           & Moderate       & Hard            \\ 
		\hline
		straight                     & 90.92          & 82.84          & 80.29           \\
		our                          & \textbf{91.96} & \textbf{83.31} & \textbf{80.59}  \\
		\bottomrule
	\end{tabular}
\label{tabel6}
\end{table}

One of the advantages of PDM-SSD is that it uses a point-based 3D backbone, which allows the model to maintain fast inference speed. We analyze the parameter count and GFLOPs of each part of PDM-SSD. PDM-SSD has a very small parameter count, only 3.3M. The main difference between PDM-SSD and many other point-based detectors is the addition of the Neck module, which only contains 0.53M parameters and accounts for only 13\% of the overall GFLOPs, so it has a minimal impact on the model's inference speed. Table. \ref{tabel8} analyzes the inference speed of IA-SSD, PDM-SSD(A), and PDM-SSD. Our experiments were conducted on a single NVIDIA RTX 3090 with Intel i7-12700KF CPU@3.6GHz. The results show that the inference speed of PDM-SSD is 68FPS, slightly lower than the 84FPS of IA-SSD, but it fully meets the hardware limitations of current LiDAR devices and practical application requirements (0.46m/frame at a speed of 120km/h), and it also brings a 0.63\% performance gain. What's even more surprising is that PDM-SSD(A) can maintain the detection speed of 84FPS and bring a significant performance gain. Combining the advantages of PDM in sparse and incomplete object detection undoubtedly improves the safety in practical applications.

%% \usepackage{booktabs}


\begin{table}
	\centering
	\caption{The parameters and GFLOPs of PDM-SSD}
	\begin{tabular}{c|cccc} 
		\toprule
		& Backbone & Neck & Head  & All    \\ 
		\hline
		Parms (MB) & 0.51     & 0.53 & 2.31  & 3.3    \\
		GFLOPs     & 4.25     & 2.84 & 14.76 & 21.85  \\
		\bottomrule
	\end{tabular}
\label{tabel7}
\end{table}
\begin{table}[htb]
\centering
\caption{Average time consumption of LLMs across different processes.}
\label{tab:time_consumption}
\begin{tabular}{lcc}
\toprule[1.5pt]
\textbf{Model} & \textbf{Identify Sensitive Function} & \textbf{Thought \& Action} \\ 
\midrule[0.8pt]
GPT-4o & 3.1s & 8.3s \\ 
Qwen2.5:32b & 7.7s & 18.2s \\ 
Deepseek-v3 & 2.7s & 7.3s \\ 
\bottomrule[1.5pt]
\end{tabular}
\end{table}



\subsection{PDM Analysis}
\label{sec:pdm}
\begin{figure}
	\centering
	\includegraphics[width=1\linewidth]{AL_pics/issue2.pdf}
	\caption{Issue2. The detection box regressed from the features learned from context has a probability lower than the threshold. The first row represents the ground truth bounding box (yellow). The second row shows the inference results of IA-SSD, with green boxes indicating predicted bounding boxes with prediction probabilities, blue points representing selected foreground points, and purple points representing vote points learned from these foreground points. The third row represents the prediction results of our PDM-SSD, with the addition of pseudo foreground points obtained from the predicted heatmap. The fourth row shows the Gaussian coefficients of the foreground points used for final prediction, and the fifth row shows the angle coefficients.}
	\label{fig:issue2}
\end{figure}

In the previous sections, we have quantitatively analyzed the advantages of PDM-SSD in terms of overall performance (\ref{sec:comparison}) and runtime (\ref{sec:runtime}). In this section, we will analyze the role of PDM at the object level, particularly in detecting sparse and incomplete targets. As mentioned earlier, current point-based detectors suffer from two important issues due to discontinuous receptive fields. \textbf{\uppercase\expandafter{\romannumeral1}:} The error in the voting point position regressed from the sampling points increases the difficulty of predicting the target box. \textbf{\uppercase\expandafter{\romannumeral2}:} The detection box regressed from the features learned from context has a probability lower than the threshold. 
We will analyze the contributions made by PDM-SSD to address these two issues separately. 

\textbf{Issue \uppercase\expandafter{\romannumeral1}.} The features predicted by querying the vote points are crucial for determining the class and geometric parameters of the target bounding box. The position of the vote points has a significant impact on the prediction results. For sparse and particularly incomplete targets, the model's receptive field is limited when learning from only occupied points. It is difficult for the model to actively learn the overall features of the target and establish connections between points. This can easily lead to the deviation of vote points from the target center, resulting in a low IoU of the predicted bounding box compared to the threshold. In contrast, PDM-SSD learns from the dilated grid features to GT heatmap, which promotes the learning of global features by the dilated centers. Additionally, the predicted heatmap can complement the vote points, which we believe can alleviate the issue of large positional errors in vote points.

We illustrate this situation with some examples in Figure \ref{fig:issue1}. In each column of the figure, there is a sparse or incomplete target. The first row represents the ground truth bounding box (yellow). The second row shows the inference results of IA-SSD, with green boxes indicating predicted bounding boxes with prediction probabilities, blue points representing selected foreground points, and purple points representing vote points learned from these foreground points. The third row represents the prediction results of our PDM-SSD, with the addition of pseudo foreground points obtained from the predicted heatmap. The fourth row shows the Gaussian coefficients of the foreground points used for final prediction, and the fifth row shows the angle coefficients. From the figure, it can be observed that the vote points regressed by IA-SSD deviate from the center of the ground truth bounding box, while PDM-SSD regresses the vote points more accurately, leading to more accurate predictions. It is worth noting that in the seventh column, both IA-SSD and PDM-SSD regress similar vote point positions, which deviate from the ideal position. However, PDM-SSD still achieves good regression results with the addition of pseudo foreground points, while IA-SSD, due to only being able to utilize existing points, cannot make more accurate selections to significantly deviate the predicted bounding box from the ground truth. These examples provide a visual demonstration of the improvements made by PDM-SSD in addressing \textbf{Issue \uppercase\expandafter{\romannumeral1}}.

\textbf{Issue \uppercase\expandafter{\romannumeral2}.} The prediction results of the detector are determined by both IoU and prediction probability. This means that even if the vote points have small positioning errors and the predicted box is close to the ground truth in terms of position and size, the prediction result may still be excluded by the detector if the target probability is low. This is a common problem in the original point-based detector because object detection is essentially a combination of classification and regression tasks. In regression tasks, the spatial geometry information of points is more important, and with a large amount of training data, the model can easily learn accurate detection boxes from discrete and sparse points. However, for classification tasks, the semantic information of points is more important, and the limited receptive field at this time cannot make the point-wise features contain the overall information of the target, thus unable to obtain accurate semantic information of the target box. In the learning process, PDM-SSD connects the inflated centers through coefficients fusion and height compression, and obtains more global information from grid features through joint learning. We believe that it can improve the prediction probability for this type of target.

In Figure \ref{fig:issue2}, we demonstrate this situation where the IA-SSD accurately predicts the positions of vote points regressed from foreground points in the second row, but the target probabilities are all below the threshold (0.1). This undoubtedly affects the model's recall rate. However, PDM-SSD can achieve higher prediction probabilities, reducing the impact of this issue on the overall performance of the model.

\section{Dicussion}
\label{sec:dicussion}
PDM-SSD provides a new approach to address the issue of discontinuous receptive fields in point-based detectors. It eliminates the complex process of integrating point and grid representations in the backbone and instead utilizes PDM to directly lift and fill features for sampled points within the neck module. This not only maintains a lightweight model but also exploits the advantages of both representations.
We have extensively demonstrated the superiority of PDM-SSD through numerous experiments, as described in Section \ref{sec:experiments}. In terms of detection accuracy, PDM-SSD surpasses the state-of-the-art point-based detectors and can compete with some voxel-based models. In terms of inference speed, PDM-SSD can run efficiently at 68 FPS, which is much higher than the scanning frequency of current LiDARs, making it suitable for practical applications. The auxiliary trained PDM-SSD (A) achieves even higher detection accuracy without sacrificing inference efficiency. Additionally, the model parameters of PDM-SSD are only 3.3MB, greatly reducing the deployment difficulty. In the object-level experimental analysis, we found that PDM-SSD effectively mitigates the issues of large vote point errors and low object box prediction probabilities caused by limited receptive fields in current point-based detectors. Overall, PDM-SSD has indeed identified new problems and made further advancements in the current research, demonstrating its value.

\textbf{Limitations and outlook.} Objectively speaking, PDM-SSD still has the following issues: 1) We avoided the detection of pedestrians in the KITTI dataset because the small volume and limited impact of pedestrians make the $5\times 5$ structural element unsuitable for detecting such targets. In future work, we will select different sizes of $B$ for different classes of objects. 2) In the scale coefficient of feature padding, we set two variables as independently and identically distributed, following the setting of heatmap in CenterNet \cite{yin2021center}. However, we believe this is not optimal, and in future work, we will split the Gaussian into a mixture of scale and rotation to learn the covariance and embed more geometric information into the learning process. 3) The final context learning module we used is still PointNet, which may not fully utilize the mixed features provided by the mixed head. In future work, we will design more sophisticated learning modules.

\section{Conclusion}
\label{sec:conclusion}
In this article, we propose single stage point-based 3D object detector called PDM-SSD. Our goal is to alleviate the limited receptive field issue while maintaining the fast inference speed of point-based detectors. Currently, point-based detectors can only learn features from existing points, and their receptive field is discontinuous and limited when the query radius expands, especially for sparse or extremely incomplete objects. This not only leads to large position errors in regression vote points but also results in prediction probabilities for object boxes lower than the threshold. PDM-SSD expands the learning space of the model through Point Dilation, specifically covering the unoccupied space near the center of the object in the original point cloud. Then, it fills these spaces with features that can be backpropagated, and the information from multiple dilation centers is connected through height compression. Finally, these features are jointly learned through a hybrid head. The experimental results show that PDM-SSD achieves competitive performance in terms of detection accuracy, surpassing all current point-based models in multiple metrics. In terms of inference speed, it fully meets the current application requirements. More importantly, from the object-level experiments, we can intuitively see the contributions of PDM-SSD in addressing the issues of current point-based detectors.

\section*{Acknowledgments}
This document is the results of the research project funded by the CAS Innovation Fund (E01Z040101)
%Bibliography
\bibliographystyle{unsrt}  
\bibliography{references}  


\end{document}
