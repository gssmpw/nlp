% \usepackage{multirow}
% \usepackage{booktabs}


\begin{table}[!h]
	\centering
	\caption{Performance of the model at different SH degrees.}
	\begin{tabular}{c|c|ccc} 
		\toprule
		\multirow{2}{*}{SH degree} & \multirow{2}{*}{numbers} & \multicolumn{3}{c}{R40}                           \\
		&                          & Easy           & Moderate       & Hard            \\ 
		\hline
		2                          & 9                        & 91.1           & 82.89          & 82.29           \\
		3                          & 16                       & \textbf{91.96} & \textbf{83.31} & \textbf{80.59}  \\
		4                          & 25                       & 91.82          & 83.24          & \textbf{80.59}  \\
		\bottomrule
	\end{tabular}
\label{tabel5}
\end{table}

Next, we will conduct ablation studies to evaluate the performance of our proposed modules in PDM-SSD on the KITTI validation split.

\textbf{SH degree.} In the illumination representation model, the higher the degree, the higher the fidelity to real scenes. However, higher degrees also require more coefficients to describe, leading to increased computational complexity. In practical applications, a suitable order is typically chosen to balance accuracy and computational efficiency based on the requirements. Table \ref{tabel5} shows the performance of PDM-SSD with orders 2, 3, and 4. It can be observed that the model performance is comparable between orders 3 and 4, and superior to order 2. Therefore, in order to reduce the number of model parameters and improve computational efficiency, we set the order to 3 with a total of 16 coefficients, which is also a common setting in many rendering models \cite{fridovich2022plenoxels,kerbl20233d}.

\textbf{Coefficients fusion.} In order to maintain the non-linear relationship between the padding feature and the dilation center feature, we adopted a coefficient fusion strategy as shown in Fig. \ref{fig6}. In contrast, directly summing the angle coefficient and the scale coefficient as the feature weight of the new cell, we compared the performance changes brought by these two coefficient fusion strategies in Table. \ref{tabel6}. It can be seen that truncating the features and separately weighting them with the two coefficients can bring higher benefits.

