\begin{figure}
	\centering
	\includegraphics[width=1\linewidth]{AL_pics/issue2.pdf}
	\caption{Issue2. The detection box regressed from the features learned from context has a probability lower than the threshold. The first row represents the ground truth bounding box (yellow). The second row shows the inference results of IA-SSD, with green boxes indicating predicted bounding boxes with prediction probabilities, blue points representing selected foreground points, and purple points representing vote points learned from these foreground points. The third row represents the prediction results of our PDM-SSD, with the addition of pseudo foreground points obtained from the predicted heatmap. The fourth row shows the Gaussian coefficients of the foreground points used for final prediction, and the fifth row shows the angle coefficients.}
	\label{fig:issue2}
\end{figure}

In the previous sections, we have quantitatively analyzed the advantages of PDM-SSD in terms of overall performance (\ref{sec:comparison}) and runtime (\ref{sec:runtime}). In this section, we will analyze the role of PDM at the object level, particularly in detecting sparse and incomplete targets. As mentioned earlier, current point-based detectors suffer from two important issues due to discontinuous receptive fields. \textbf{\uppercase\expandafter{\romannumeral1}:} The error in the voting point position regressed from the sampling points increases the difficulty of predicting the target box. \textbf{\uppercase\expandafter{\romannumeral2}:} The detection box regressed from the features learned from context has a probability lower than the threshold. 
We will analyze the contributions made by PDM-SSD to address these two issues separately. 

\textbf{Issue \uppercase\expandafter{\romannumeral1}.} The features predicted by querying the vote points are crucial for determining the class and geometric parameters of the target bounding box. The position of the vote points has a significant impact on the prediction results. For sparse and particularly incomplete targets, the model's receptive field is limited when learning from only occupied points. It is difficult for the model to actively learn the overall features of the target and establish connections between points. This can easily lead to the deviation of vote points from the target center, resulting in a low IoU of the predicted bounding box compared to the threshold. In contrast, PDM-SSD learns from the dilated grid features to GT heatmap, which promotes the learning of global features by the dilated centers. Additionally, the predicted heatmap can complement the vote points, which we believe can alleviate the issue of large positional errors in vote points.

We illustrate this situation with some examples in Figure \ref{fig:issue1}. In each column of the figure, there is a sparse or incomplete target. The first row represents the ground truth bounding box (yellow). The second row shows the inference results of IA-SSD, with green boxes indicating predicted bounding boxes with prediction probabilities, blue points representing selected foreground points, and purple points representing vote points learned from these foreground points. The third row represents the prediction results of our PDM-SSD, with the addition of pseudo foreground points obtained from the predicted heatmap. The fourth row shows the Gaussian coefficients of the foreground points used for final prediction, and the fifth row shows the angle coefficients. From the figure, it can be observed that the vote points regressed by IA-SSD deviate from the center of the ground truth bounding box, while PDM-SSD regresses the vote points more accurately, leading to more accurate predictions. It is worth noting that in the seventh column, both IA-SSD and PDM-SSD regress similar vote point positions, which deviate from the ideal position. However, PDM-SSD still achieves good regression results with the addition of pseudo foreground points, while IA-SSD, due to only being able to utilize existing points, cannot make more accurate selections to significantly deviate the predicted bounding box from the ground truth. These examples provide a visual demonstration of the improvements made by PDM-SSD in addressing \textbf{Issue \uppercase\expandafter{\romannumeral1}}.

\textbf{Issue \uppercase\expandafter{\romannumeral2}.} The prediction results of the detector are determined by both IoU and prediction probability. This means that even if the vote points have small positioning errors and the predicted box is close to the ground truth in terms of position and size, the prediction result may still be excluded by the detector if the target probability is low. This is a common problem in the original point-based detector because object detection is essentially a combination of classification and regression tasks. In regression tasks, the spatial geometry information of points is more important, and with a large amount of training data, the model can easily learn accurate detection boxes from discrete and sparse points. However, for classification tasks, the semantic information of points is more important, and the limited receptive field at this time cannot make the point-wise features contain the overall information of the target, thus unable to obtain accurate semantic information of the target box. In the learning process, PDM-SSD connects the inflated centers through coefficients fusion and height compression, and obtains more global information from grid features through joint learning. We believe that it can improve the prediction probability for this type of target.

In Figure \ref{fig:issue2}, we demonstrate this situation where the IA-SSD accurately predicts the positions of vote points regressed from foreground points in the second row, but the target probabilities are all below the threshold (0.1). This undoubtedly affects the model's recall rate. However, PDM-SSD can achieve higher prediction probabilities, reducing the impact of this issue on the overall performance of the model.