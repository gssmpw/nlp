\begin{figure}[t]
	\begin{center}
		% \vspace{-0.2cm}
		\includegraphics[width=.8\textwidth]{AL_pics/pf.pdf}
	\end{center}
	\vspace{-0.3cm}
	\caption{Visualization on some very sparse and extremely incomplete targets on the KITTI dataset. For grid-based backbone networks, the grid continuously pads, convolves, and pools the operations, covering the space that the original point cloud does not occupy. The expansion of the receptive field is continuous and can better aggregate local features and combine features from different regions. Point-based methods can only extract features from existing points, and even if the number of surrounding points increases, the features remain unchanged. The receptive field is discontinuous and limited to local areas.}
	\label{fig3}
	\vspace{-0.3cm}
\end{figure}

As aforementioned, we aim to combine the advantages of grid-based and point-based detectors by addressing the issue of discontinuous receptive fields in current point-based models. To achieve this, we propose PDM-SSD, a novel and generic single-stage point-based 3D detector. The overall workflow of PDM-SSD is illustrated in Fig. \ref{fig2}, where the input LiDAR point clouds are first passed through the embedding network to expand the feature space of the points. Then, a PointNet-style 3D backbone network is employed to extract point-wise features. The 3D backbone network consists of several stages of downsampling modules, local feature aggregation modules, and multi-scale feature aggregation modules, ensuring that the sampled points learn rich geometric and semantic features. The neck network includes our proposed point dilation mechanism, where points are lifted to the grid level and feature filling is performed for the unoccupied space in the original point cloud using a special mechanism. The grid-wise features are used to regress the heatmap of the scene, providing global features of the objects through joint learning with the hybrid detection head.

In the following, we discuss in detail the implementation process of PDM-SSD. In Section \ref{sec:pf}, we examine the changes in the receptive fields of the grid-based and point-based models during the formulation learning process. In Sections \ref{sec:backbone}, \ref{sec:neck}, and \ref{sec:head}, we respectively introduce the specific structures of the PDM-SSD model's backbone, neck, and detection head. Finally, in Section \ref{sec:loss}, we explain the approach of joint learning in the model and the loss function.