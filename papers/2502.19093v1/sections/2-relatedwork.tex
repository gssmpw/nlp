\section{Related Work}
\label{sec:related-work}

\revised{
This section reviews the literature on historical visualization and datasets of visualization.
}

\subsection{Historical Visualization}

Historical visualizations reflect the societal context of their time, offering rich information for research.
Some prior research focuses on well-known historical visualizations and examines them from \revised{different perspectives}.
Koch~\cite{Koch2009Crediting} discusses the people's attitudes toward John Snow's cholera map~\cite{Snow1855Mode} in 19th-century London.
Shiode~\cite{Shiode2015mortality} \revised{utilizes historical records to quantitatively examine John Snow's waterborne transmission hypotheses.}
By retrospecting ``Napoleon's Grand Army''~\cite{Minard1869Carte}, Friendly reviews the contribution of Charles Joseph Minard and compares modern revisions of this classical visualization~\cite{Friendly2002Visions}.
There is also research on the provenance of different types of visualizations, such as scatter plot~\cite{Friendly2005Early} and heat map~\cite{Wilkinson2009History}. 
\revised{
This line of research focuses on close examination of individual visualizations with a microscopic view.
}

In addition to the microscopic view, some other works take a macroscopic view, reviewing the historical development of visualizations.
Friendly~\cite{Friendly2001Milestones} collects the chronological milestones of historical visualizations to show the development process of visualizations.
Correll and Garrison~\cite{Correll2024When} comprehensively examine the development of historical visualizations and illustrations related to the human body, highlighting the importance of culture in the understanding and reflection of visualizations.
There are also books~\cite{Friendly2008Brief, Friendly2021History, Rendgen2019History} illustrating the development of visualizations.

While these works elaborate on how, where, and why today's data visualization is developed and conceived, they inevitably partially overlook some \revised{``historical devices and trajectories of change that have not directly led to present-day forms''}~\cite{Ruokkeinen2023Developing}.
That is, what is promotive for modern visualization gets more attention and discussion, while earlier attempts that do not contribute much to later statistic diagrams are passed over~\cite{Ruokkeinen2023Developing}.
Our work focuses on historical Chinese visualizations, one of the many overlooked branches.
To this end, we collected a large number of historical Chinese visualizations.
Based on the dataset, we analyze these visualizations (\cref{sec:zuantuset}, \cref{sec:usage-scenarios}, and \cref{sec:discussion}) and sought to disseminate historical Chinese visualizations to a wider audience.
We aim to contribute to addressing the ``concern of eurocentric view''~\cite{Gunter2021Review} in previous work.





\subsection{\revised{Dataset of Visualizations}}

\revised{
With the popularization of data-driven research, datasets of visualizations are created for various purposes.
Specifically, they are useful as corpora for summarizing design patterns and benchmarks for empirical studies.
}
Segel et al. summarize different storytelling techniques from narrative visualization samples~\cite{Segel2010Narrative}.
Zhang et al. collect visualizations related to COVID-19 to discover ``who uses what kinds of data to communicate what messages''~\cite{Zhang2021Mapping}.
Borkin et al. collect \num{2070} visualizations to find out what visualizations are memorable~\cite{Borkin2013What}.
\revised{
Other works collect scholarly visualizations in IEEE VIS and IEEE TVCG papers to summarize common designs and research trends~\cite{Chen2021VIS30K, Deng2023VisImages}.
}
\revised{
Regarding historical visualizations, Friendly et al.'s Milestones Project~\cite{Friendly2001Milestones} gathers hundreds of significant inventions in the history of visualization.
Zhang et al.' OldVisOnline~\cite{Zhang2024OldVisOnline} curates a dataset of 13K historical visualizations.
}

\revised{
Existing historical visualization datasets fall short in that they typically overlook visualizations from non-European cultural frameworks, which may bring bias when using these datasets for analysis~\cite{Friendly2021History}.
This work constructs the first dataset dedicated to historical Chinese visualizations and illustrations as an initial effort to enhance existing historical visualization collections and draw attention to visualizations from underrepresented cultural frameworks.
}
\datasetName includes visualizations \revised{between 550 BCE and 1950 CE} in China.
Through this dataset, we aim to promote historical Chinese visualization and bring this knowledge to broader audiences.


