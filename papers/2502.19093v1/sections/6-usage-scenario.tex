\section{Usage Scenarios}
\label{sec:usage-scenarios}

In this section, we discuss usage scenarios of \datasetName.
The usage scenarios are adapted from OldVisOnline~\cite{Zhang2024OldVisOnline} and contextualized with historical Chinese visualizations from \datasetName.
We also examine unique opportunities for culture-focused design.
Note that the scenarios below are not mutually exclusive.

\subsection{Searching Visualization}

\datasetName Gallery (\cref{fig:gallery}) allows efficiently searching historical Chinese visualizations.
Consider a scenario where a researcher is reading a paper discussing tables in \ChineseTerm{\Shiji}{史記}{Records of the Grand Historian} while the images provided in the original paper are not sufficient for understanding.
These visualizations cannot easily be queried on general-purpose search engines (e.g., Google Image Search). 
\revised{Thus, researchers have to find digital versions of these books and manually look through many pages to find the visualization they want, which is time-consuming and frustrating.}
With \datasetName Gallery, the researcher can directly search for these visualizations with a book name, such as \term{Shiji}. 
The researcher may also find different versions of a book.
In addition, similarity-based recommendations can help the researcher find more similar visualizations appearing in books other than the ones being queried to provide additional context.
Searching supported by \datasetName Gallery is also vital for other scenarios discussed below.

\subsection{Textual Criticism}

Textual criticism is a scholarly discipline that studies texts, primarily ancient documents, to reconstruct the original version.
\revised{As a dataset of historical visualizations, \datasetName may be used for textual criticism, as described by Zhang et al.~\cite{Zhang2024OldVisOnline}.}

Take \ChineseTerm{\SiShiChuanShouTu}{四詩傳授圖}{The Apprenticeships of the Four Schools of \ShiJing} as an example.
\revised{
It records the master-apprentice relations of four schools of \term{\ShiJing}.
Here, we focus on one of the schools: \term{Lu}.
\Cref{fig:textual-criticism} shows the \term{Lu} apprenticeship relations in six editions of \term{\SiShiChuanShouTu} stored in \datasetName.
By examining the structure of these tree visualizations, we observe three variants of a branch in the tree, which are highlighted in \cref{fig:textual-criticism}.
Specifically, the master who taught \term{Son}, \term{Fu}, and \term{Ming} was presented differently as \term{Li} (\cref{fig:textual-criticism}(A)), \term{Qian} (\cref{fig:textual-criticism}(B)), and \term{Bao} (\cref{fig:textual-criticism}(C - F)).
A transcription error may cause the differences, which is common during the compilation of ancient books.
Such cases are worth further investigation by historians.
}



\begin{figure}[!t]
    \centering
    \includegraphics[width=\linewidth]{./assets/imgs/10-textual_criticism/textual_criticism.png}
    \caption{
        \revised{
        \textbf{Comparing editions of the \term{Lu} apprenticeship tree in \term{\SiShiChuanShouTu}:}
 Among the six editions, there are three variants of the apprenticeship relation of three apprentices: \term{Son}, \term{Fu}, and \term{Ming}.
        \href{https://commons.wikimedia.org/wiki/File:SBL003_毛詩擧要圖.pdf}{(A)},
        \href{https://tile.loc.gov/image-services/iiif/service:asian:lcnclscd:2014514341:2014514341v006:1E001-18b19a/full/pct:100/0/default.jpg}{(B)},
 and
        \href{https://iiif.lib.harvard.edu/manifests/view/drs:17209675$183i}{(C)}
        \href{https://digicoll.lib.berkeley.edu/record/73580?ln=en&v=pdf}{(D)}
        \href{https://www.digital.archives.go.jp/DAS/meta/listPhoto?LANG=default&BID=F1000000000000094161&ID=&NO=3&TYPE=JPEG&DL_TYPE=pdf}{(E)}
        \href{https://commons.wikimedia.org/wiki/File:ZHSY000019_監本纂圖重言重意互注點校毛詩二十卷圖譜一卷 (漢)毛萇 傳(漢)鄭玄 箋(唐)陸德明 釋文 宋刻本.pdf}{(F)}
 shows the master teaching the three apprentices as \term{Li}, \term{Qian}, and \term{Bao}, respectively.
 The published year (or estimated range) from the data sources is shown for each edition.
 }
 }
    \Description{
 Comparing six editions of the \term{Lu} apprenticeship diagram.
 }
    \label{fig:textual-criticism}
\end{figure}

\subsection{Investigating the History of Visualization}

\revised{
An understanding of the history of visualization is essential for its future development.
Due to the constraints on data retrieval and cultural barriers, as mentioned in \cref{sec:related-work}, the current study of visualization history is focused on the Eurocentric view.
Expanding our horizons to historical visualizations from different cultural contexts can help us understand visualization history more comprehensively and unbiasedly.
For example, when tracing the origin of tables, we observe Sima Qian's chronological tables in \term{\Shiji} created during the Western Han Dynasty, around 90 BCE (discussed in \cref{sec:taxonomy}).
Meanwhile, the origin of the formal chronology was previously attributed to Eusebius' \term{Chronicon}~\cite{Marchese2011Exploring}, which was created around 311 CE and similar to the vertical tables in \term{\Shiji}.
}

\subsection{Revitalizing Historical Graphic Designs}

\begin{figure}[!htb]
    \centering
    \includegraphics[width=\linewidth]{./assets/imgs/11-redesign/redesign.png}
    \caption{
 Two examples of contemporary designs inspired by historical designs created by Baur and Felsing~\cite{Baur2020Visual}:
 (A1) The \href{https://dl.ndl.go.jp/api/iiif/2574383/R0000010/full/full/0/default.jpg}{historical design} showing facial acupuncture points is from \term{San Cai Tu Hui}~\cite{Wang1609San}, 1609.
 (A2) The contemporary design incorporates smoother lines and combines visualization of acupuncture points with skeletal structures.
 (B1) A tree visualization of 64 hexagrams, stitched by the \href{https://dl.ndl.go.jp/api/iiif/2596442/R0000072/full/full/0/default.jpg}{left part}
 and the \href{https://dl.ndl.go.jp/api/iiif/2596442/R0000071/full/full/0/default.jpg}{right part}, from \term{Tu shu bian}~\cite{Huang1613Tu}, 1613.
 (B2) The contemporary design applies this visual encoding to map genetic codes to amino acids.
 }
    \Description{
 Two examples of contemporary designs inspired by historical designs created by Baur and Felsing.
 }
    \label{fig:redesign}
\end{figure}

Understanding and adapting visuals from history can facilitate design innovation and evoke audiences to rethink the role of contemporary designs.

\subsubsection{Inspire Design Innovation for Designers}
\label{sec:inspire-design-innovation}

By blending traditional visuals with modern design elements, designers may create unique expressions that respect cultural heritage while still appealing to contemporary tastes. 
Here, we introduce two interesting examples of such cross-cultural information designs created by Baur and Felsing~\cite{Baur2020Visual}.

\Cref{fig:redesign}(A) shows visualizations of facial acupuncture points.
\revised{In the historical design (\cref{fig:redesign}(A1)), two different styles of depiction are presented in the face: the left side of the face portrays pictorial facial features, while the right side is filled with extensive text annotations.}
In the contemporary design (\cref{fig:redesign}(A2)), the human face is also divided into two halves. 
The left side follows an anatomical style showing underlying skeletal structures, while the right side retains the historical design.
The contemporary design also uses color coding to distinguish different types of acupuncture points.

\Cref{fig:redesign}(B) demonstrates parallels between the structure of the 64 hexagrams derived from \term{Yin yang} and an amino acid codon table.
Both utilize a radial structure that expands outward from the center, with each layer progressively dividing until the outermost layer represents the hexagrams and the genetic codons, respectively~\cite{Baur2020Visual}.
Encoding codons based on the structure of the hexagram diagram facilitates the translation of a genetic code into an amino acid sequence.
\revised{The example indicates the potential of retargeting historical visualization designs to contemporary data.}

\subsubsection{Inspire Design Rethink for Audience}

Defamiliarization originates as a literary technique and serves as a method to question our habitual interpretations of everyday things~\cite{Bell2005Making}. 
In interface design, defamiliarization refers to altering familiar interfaces, interaction methods, or elements to encourage users to rethink and re-experience everyday interactions~\cite{Bell2005Making}.
\revised{Designers may leverage historical visual conventions to redesign contemporary visual elements and create a sense of visual defamiliarization for audiences}.
Visual defamiliarization, which involves presenting data through unfamiliar visual conventions, may encourage the audience to rethink the functional and cultural significance of contemporary design and help preserve historical visual conventions.
\revised{
Note that directly presenting unfamiliar visual conventions can lead to a cognitive burden on the audience, as discussed in~\cref{sec:discuss-convention}.
}


\subsection{Facilitating Culture-Focused Design}
\label{sec:culture-focused-design}

\revised{
Approaches to visual language can be placed on a continuum, according to Kostelnick~\cite{Kostelnick1995Cultural}, with the global approach at one end and the culture-focused approach at the other.
The global approach seeks to bridge the cultural gap by creating a rational, objective, and culturally neutral visual language.
The culture-focused approach emphasizes the strong correlation between visual communication effectiveness and cultural context, advocating for designs tailored to a specific cultural background.

While the global approach has become the guiding force in visual communication, culture-focused design deserves more attention.
Though they may be seen as niche or outdated, these designs emerging from specific historical and cultural contexts embody significant cultural value. 
We emphasize these design minorities to enrich the design diversity and promote design equity~\cite{Kostelnick2003Shaping}.
These designs can benefit certain situations, as users' cultural background can influence how they respond to information designs~\cite{Kostelnick1995Cultural}.

\datasetName can serve as a source of design materials to tailor designs for historical Chinese culture.}
The following lists two potential benefits of culture-focused designs suggested by Kostelnick and Hassett~\cite{Kostelnick2003Shaping}.

\begin{itemize}[leftmargin=3.5mm]
    \item \textbf{\revised{Streamline} comprehension:}
    \revised{Designs tailored for a specific culture can streamline the visual comprehension of those living in that culture~\cite{Kostelnick2003Shaping}.}
    Designers may leverage symbols and styles of historical Chinese graphics that encapsulate shared cultural knowledge and values to communicate more effectively with Chinese audiences.
    
    \item \textbf{Evoke emotional connotations:}
    \revised{Culture-focused designs can associate audiences with their cultural backgrounds to evoke emotional connotations.}
Using culturally specific visuals may lend credibility and authenticity to a design, positioning it within a particular tradition or discourse community, which can be beneficial on certain occasions~\cite{Kostelnick2003Shaping}.
    A commercial example discussed by~\cite{Kostelnick2003Shaping} is the cover design of ``The Old Farmer's Almanac'', the oldest continuously published periodical in North America. 
    Its cover has retained nearly two centuries of consistent visual features, including historical rural illustrations and vintage English typography~\cite{Kostelnick2003Shaping}. 
    These visual elements naturally connect readers to the country's profound natural and historical narrative, resonating deeply with its citizens.
\end{itemize}




\subsection{Games, Education, and Storytelling}

Historical Chinese \itemTypePlural and their redesigns can be combined with modern technologies to serve games, education, and storytelling.
Animation and interactivity can be applied to historical static visualizations to guide users' attention, enhancing their understanding of visualization encoding and historical contexts. 
These interactive ancient visualizations may even serve as serious games to support traditional cultural storytelling. 
For instance, the Xuanjitu project~\cite{RorySaur2022Xuanjitu} uses cyclic animations to display the positions of various poems within its palindrome structure, helping readers interpret historical literary works.
Moreover, historical Chinese visualizations contain rich visual expressions of cultural narratives, which may serve as resources for students to create their own historical data stories, thereby increasing classroom engagement and immersion~\cite{Lu2011ShadowStory}.
In recent years, more games such as Black Myth: Wukong~\cite{GameScience2024Black} have incorporated traditional cultural heritage into their scene and character designs. 
Graphics in \datasetName, such as for ancient clothing, architectural features, and perspective landscape paintings, may be used to train texture generation models for generating resources to be used in games.


