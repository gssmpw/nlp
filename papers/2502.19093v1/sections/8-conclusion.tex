\section{Conclusion}

We introduced \datasetName, a dataset of historical Chinese visualizations and illustrations.
To collect these graphics, we implemented a semi-automatic pipeline by which we extracted \numVisShort visualizations and \numIllusShort illustrations from historical Chinese books.
Based on \datasetName, we combine historical factors to analyze and explain design patterns in historical Chinese visualizations.
Our analysis highlighted the distinct features of historical Chinese visualizations, such as textualism and pictorialism.
We also envision the potential usage scenarios of \datasetName, including supporting textual criticism, and \revised{inspiring culture-focused designs.}
\revised{
Through our effort in curating the dataset, we aim to draw public attention to facilitate a basic understanding of historical Chinese visualizations.
Ancient Chinese culture is, however, only one of the underrepresented cultural communities in the common narrative of visualization history.
We call for more investigations into historical visualizations under other underrepresented cultures, which are critical for a comprehensive and unbiased understanding of the history of visualization.
}

