\begin{figure}[!ht]
    \centering
    \includegraphics[width=\linewidth]{./assets/imgs/3-framework/framework.pdf}
    \caption{
        \textbf{A framework of elements involved in historical visualizations as a visual communication channel:}
        The framework is adapted from frameworks in the literature~\cite{Zhang2021Mapping, Munzner2009Nested,Lasswell1948Structure}.
        We emphasize the impact of historical factors, such as politics, cultures, and religions, on the framework components.
        \revised{We also consider the influence on contemporary perspective when investigating the effect of historical visualizations.}
    }
    \Description{
        The framework for understanding historical visualizations builds on prior research, focusing on the influence of historical factors, such as politics, culture, and religion.
        It categorizes feedback into its impact on these historical factors and its effects on the current state of study.
    }
    \label{fig:framework}
\end{figure}

\section{\revised{Analyzing \datasetName}}
\label{sec:zuantuset}

\revised{
This section focuses on the content of \taxon{visualization} images in \datasetName.
}
We analyze historical Chinese visualizations based on the framework in \cref{fig:framework}, which is adapted from the literature~\cite{Zhang2021Mapping,Munzner2009Nested,Lasswell1948Structure}.
The framework is devised for \revised{understanding elements involved in historical visualizations as a visual communication channel}.
The primary function of the original framework~\cite{Zhang2021Mapping} is to organize and analyze surveyed visualizations, providing concepts that guide further inquiry and capture their commonalities.
\revised{
We adapt the framework to emphasize historical factors and the effect of historical visualizations on contemporary perspective.
}
\emph{Who} focuses on the creator of \itemTypePlural.
\emph{What data} refers to the domain of the data.
\emph{What message} refers to the desired purpose of \itemTypePlural.
\emph{What form} refers to the visualization taxonomy and visual patterns.

\revised{
\Cref{sec:dataset-sample} discusses the relations between form and domain of historical Chinese visualizations, corresponding to \emph{what data}.
\Cref{sec:taxonomy} focuses on \emph{what message} and \emph{what form} by analyzing visual patterns of historical Chinese visualizations together with historical factors.
}


\subsection{Visualization Forms and Domains}
\label{sec:dataset-sample}

\Cref{fig:relation-matrix} shows the distribution and correlation of visualization forms and domains in \datasetName.
We observe patterns, such as maps correlating with geography and tables and trees correlating with genealogy.
Through closer inspection, we observe that different types of books contribute visualizations differently in forms and domains.


\textbf{Books focusing on one form and domain:}
Some books focus on visualizations of a specific form and domain.
For example, visualizations in \ChineseTerm{\GuangYuJi}{廣輿記}{Enlarged Terrestrial Records}\footnote{
We use the template ``\ChineseTerm{Chinese Phonetic Alphabet}{Chinese term}{English translation}'' to introduce Chinese book and figure titles and terms.
}~\cite{Lu16621722Guang}, a Chinese geography book, are all maps.
\ChineseTerm{\DongTingQinShiZongPu}{洞庭秦氏宗譜}{The Genealogy of the Qin Family in Dongting}~\cite{Qin1873Dongting} recording family genealogy contribute thousands of family trees and tables, and \ChineseTerm{\BenCaoGangMu}{本草綱目}{The Compendium of Materia Medica}~\cite{Li1655Ben}, a famous medical book in ancient China, contributes thousands of illustrations of plants and animals but contains almost no visualizations.

\textbf{Books spanning multiple forms and domains:}
Other books, especially \ChineseTerm{\LeiShu}{類書}{a kind of reference book consisting of material quoted from many sources and arranged by category}, such as \ChineseTerm{\SiShuTuKao}{四書圖考}{Diagram Collection of the Four Books}~\cite{Du1827Si} and \ChineseTerm{\SiShuWuJingDaQuan}{四書五經大全}{Corpus of the Four Books and the Five Classics}~\cite{Hu1403Si} contribute visualizations across a wide range of domains.

 
\textbf{Where to find certain types of graphics:}
The correlation (\cref{fig:relation-matrix}) between form and domain suggests that to look for a specific form of visualization, and we may look into the book belonging to a highly correlated domain.
For example, genealogy books are the best source for looking for more tables and trees.



\begin{figure}[!htb]
    \centering
    \includegraphics[width=1.0\linewidth]{./assets/imgs/4-relation_vis_theme/relation_vis_theme.pdf}
    \caption{
        The correlation between form and domain of visualizations in \datasetName.
        \yzc{Give the exact number of each cell, each row, and each column.}
    }
    \Description{
        The image describes the connection between visualization forms and domains that can guide collection efforts.
    }
    \label{fig:relation-matrix}
\end{figure}


\subsection{Historical Chinese Visualizations}
\label{sec:taxonomy}

This section discusses the five different \revised{forms} of historical Chinese visualizations categorized through the classification process in \cref{sec:classification}: \taxon{map}, \taxon{node-link diagram}, \taxon{tree}, \taxon{table}, and \taxon{annotated diagram}.
For each form, we clarify its definition concerned in this paper.
We then give examples and summarize the characteristics of each form and its historical background through literature reviews.

\begin{figure*}[!htb]
    \includegraphics[width=\linewidth]{./assets/imgs/5-map/map.png}
    \caption{
        \textbf{Examples of map in \datasetName:}
        (Left)
        \ChineseTermWithUrl{\WangGongZhiTu}{王宮制圖}{Layout Plan of the Royal Palace}{https://www.digital.archives.go.jp/acv/auto_conversion/conv/jp2jpeg?ID=M2021050616304107841&p=6} from \term{Qi jing tu}~\cite{Wu1615Qi}, 1615.
        The layout plan shows a planimetric mode that rectangle 
        In the top-left corner is an ancestral temple dedicated to the spirits of deceased ancestors. 
        The annotated Chinese character may represent a tablet bearing the ancestors' names, which is rotated 90 degrees to indicate that the tablet may face the central point.
        (Middle)
        \ChineseTermWithUrl{\JiuBianZongTu}{九邊總圖}{Map of the Nine Garrisons}{https://ids.lib.harvard.edu/ids/iiif/23518593/full/full/0/default.jpg},
        from \term{Guang yu tu}~\cite{Zhu1566Guang}, 1566.
        The map shows the system built during the Ming dynasty (1368 - 1644) to protect the northern border and the Great Wall.
        We annotate pictorial elements such as mountains and rivers.
        Geographic location can be encoded by the scattered texts alone.
        The position of the annotated text indicates the location of the army and government office of Guizhou.
        The map is presented with grids whose length corresponds to 500 Li \revised{(A traditional Chinese unit of distance)}.
        (Right)
        \ChineseTermWithUrl{\DongXiFenShanTu}{東西分陜圖}{Map of Shaanxi}{\UrlOfDongXiFenShanTu}, from \term{Tian xia shan he liang jie kao}~\cite{Xu1723Tian}, 1723.
        Paragraphs are directly written on the map, which provides additional historical information about the specific location.
        We add annotations to the map to indicate some of these paragraphs and point them to the corresponding locations.
    }
    \Description{
        Examples of maps in \datasetName.
    }
    \label{fig:map}
\end{figure*}

\subsubsection{Map}


Map communicates geographical or location information.
In \datasetName, we found a variety of maps, such as geographical maps and layout plans.
We contrast map designs on a spectrum from quantitative to qualitative.

\textbf{Quantitative map design:}
Quantitative design is characterized by the use of  ``scientific'' measurements.
Such maps are typically used for utilitarian purposes, such as administration and military planning~\cite{Yee1994Reinterpreting}.
The quantitative design is common in modern maps that use cartographic approaches, such as the Mercator projection, and the adoption of more standardized cartographic norms~\cite{Yee1994Reinterpreting, Amelung2007New, Yee1994Traditional}.
Quantitative design was uncommon in Chinese maps until the exposure to Western maps in the late modern period when historical governors were aware of the power of the maps' practical function, and the need for more accurate maps increased~\cite{Amelung2007New}.

\textbf{Qualitative map design:}
Before modernization, most historical Chinese maps were characterized by qualitative design, which features textualism~\cite{Yee1994Taking, DorofeevaLichtmann2004Spatial} and pictorialism~\cite{Osawa2016Landscape, Yee1994Reinterpreting, Jiang2017heritage}.
Textualism refers to both a reliance on texts as sources of information in the compiling of maps and a reliance on text to complement the presentation of information in maps~\cite{Yee1994Taking}.
Pictorialism refers to the wide use of pictorial elements in maps~\cite{Osawa2016Landscape, Yee1994Reinterpreting, Jiang2017heritage}, from minor decorations to main information carriers.
The tendencies of textualism and pictorialism are also seen in other forms of historical Chinese visualizations.
In the following, we discuss four features of historical Chinese maps related to qualitative design.

\begin{itemize}[leftmargin=3.5mm]
    \item \textbf{Text accompaniment:}
          In historical Chinese cartography, text might not be merely an auxiliary element but a core component of the cartographic representation.
          According to the targeted task, maps may use texts to report corresponding local conditions~\cite{Yee1994Reinterpreting}.
          For instance, a map that is produced to report current local development may consist of texts indicating population, revenue, and ranking~\cite{Osawa2016Landscape}.
          Furthermore, text alone can form the main component of a map expressing geographic information (e.g., \cref{fig:map} (Right)).
          In this case, the Chinese characters may encode positional, directional, and semantic information.

          \item \textbf{Pictorial and planimetric \revised{design}:}
          Most historical Chinese maps were descriptive and used a lot of pictorial elements (styles of depicting the real world).
          Opposite to pictorial is planimetric, which uses a certain level of abstraction to characterize the geometry of real objects.
          In different maps, the ratio between these two styles varies.
          Typically, the planimetric approach was used to characterize roads and rivers (e.g., by single or double line \cref{fig:map}).
          The pictorial element is commonly used for buildings, landscapes, and mountains, serving as decorations.
          In local gazetteers, many landscape-style maps pictorially depict the local environments and affairs with a limited number of texts naming the places or objects. \yzc{Given reference or example to support this argument.}

    \item \textbf{\revised{Coordinate system}:}
    Before the introduction of Western cartography, most historical Chinese maps lacked scale and coordinate grids. 
    Although the ancient Chinese cartographer Pei Xiu introduced grids and \ChineseTerm{Fen Lu}{分率}{The Graduated Divisions}~\cite{needham1974science, Yee1994Reinterpreting}, the prototype of scale in China, and some maps used grids (\cref{fig:map} (Middle)), the scale of these grids were often inconsistent in a map~\cite{Yee1994Reinterpreting} and did not correspond to the meridian and latitude coordinate systems~\cite{Yee1994Reinterpreting}. 
    A possible explanation of this characteristic is that ancient Chinese maps did not primarily focus on geographical accuracy but rather on serving political, religious, artistic, or utilitarian purposes~\cite{Yee1994Reinterpreting}. 
    For example, a map can be used to demonstrate China's territorial integrity and the extent of its administrative authority~\cite{Yee1994Reinterpreting}, focusing on conveying political information rather than precise geographical measurements. 
    In this case, a scale or grid is not required.
    Furthermore, since ancient Chinese maps were often combined with textual annotations, the accompanying text provided quantitative information, which may mean that maps did not rely heavily on scales or coordinate grids to convey quantitative distance data~\cite{Yee1994Reinterpreting}.

    \item \textbf{Orientation:}
          Historical Chinese maps had a unique way of representing orientations.
 Unlike the current convention that the top indicates north, there was no standard for orientation in historical Chinese maps~\cite{Baur2019Cultural}.
 In these maps, the north may be pointed downward or even leftward.
 To indicate directions, historical Chinese maps rarely used arrows but a special way of rotating visual elements or texts within the map~\cite{Baur2019Cultural}.
 That is, the direction of a person or building is represented by the direction of the character or the rotated angle of the building.
 The premise of the practice is the multidirectionalism of Chinese characters that rotated characters are still recognizable and with distinct borders with each others~\cite{Baur2019Cultural}.
          \yzc{What is ``multidirectionalism''?}
          \yzc{characters in other languages are also recognizable after rotation. What is special for Chinese characters?}
 This special representation can be observed in many layout plans (\cref{fig:map}) that show the structure of a royal palace or ritual affairs in daily life.
\end{itemize}

The causes of the above characteristics vary. \yzc{This sentence does not add value.}
In ancient China, the primary producers of maps were the government and elites~\cite{Osawa2016Landscape}.
The lack of numerical measurement may rest with the government purpose, whose priority is neither the representation of precise nature and reality~\cite{Yee1994Taking} nor the curiosity to explore and describe the unknown lands~\cite{Osawa2016Landscape}, but the perpetuation of political power~\cite{Yee1994Taking}.
The primary function of historical Chinese maps was to report local conditions and customs, economic development, and records of resource exploration to the central government~\cite{Osawa2016Landscape}.
Also, as local officers had widely used the pictorial representation as a tradition, central governments collected and reprinted them further to disseminate the usage of pictorial elements~\cite{Osawa2016Landscape}.
The above features, while being criticized as ``immature and backward'' by some scholars~\cite{Yee1994Reinterpreting}, show that historical Chinese maps should be investigated under a more diverse opinion where the excellence of cartographic is measured by not only scientific purposes but also social, aesthetic, and even religious standards~\cite{Yee1994Reinterpreting, Jiang2017heritage}.

\begin{figure}[!htb]
    \centering
    \includegraphics[width=\linewidth]{./assets/imgs/6-node-link/node-link.png}
    \caption{
        \textbf{Examples of node-link diagram in \datasetName:}
        (Left)
        A
        \href{https://tile.loc.gov/image-services/iiif/service:asian:lcnclscd:2011457019:1A002:002b003a/full/pct:100/0/default.jpg}{star chart}
        from \term{Yuechi Xianzhi}~\cite{Xiong1850Yue}, 1850.
        Circles corresponding to stars are connected to form zodiac signs.
        (Middle)
        \ChineseTermWithUrl{Yiyou Taijitu}{易有太極圖}{\YiYouTaiJiTu}{https://ids.lib.harvard.edu/ids/iiif/17209716/full/full/0/default.jpg}, 
        from \term{Lui jing tu kao}~\cite{Yang1662Liu}. 
        It was devised by Zhou Dunyi, a Song dynasty philosopher, synthesizing aspects of Chinese Buddhism and Taoism with metaphysical discussions~\cite{Adler1999Zhou}.
        (Right)
        \ChineseTermWithUrl{\SiShiTu}{四始圖}{The Four Starts of \ShiJing}{https://tile.loc.gov/image-services/iiif/service:asian:lcnclscd:00510373:1D000:34b35a/full/pct:100/0/default.jpg}
        from \term{\SiShuWuJingDaQuan}~\cite{Hu1403Si}, 1403.
        \ChineseTerm{\ShiJing}{詩經}{Classic of Poetry} has four sections.
        \term{\SiShiTu} connects the name of each section and the first poem of each section, forming a knowledge network.
    }
    \Description{
        Examples of node-link diagrams in \datasetName.
    }
    \label{fig:node-link}
\end{figure}

\subsubsection{Node-link Diagram}

In this paper, we consider a node-link diagram as a visual representation that uses lines to connect objects and represent implicit relationships between them.
Among the node-link diagram in \datasetName, nodes are frequently presented as text.
The nodes may also be presented as circles, as in conventional design nowadays.

\textbf{Node as text:}
Linked texts can present textual knowledge.
According to Lackner~\cite{Lackner2007Diagrams}, in such designs, key terms are frequently placed in the center with lines guiding the reader in various directions to other textual elements. 
For example, \cref{fig:node-link} (Right) visualizes the relationship of the four sections of \ChineseTerm{\ShiJing}{詩經}{Classic of Poetry} and the first poem of each section.
The four sections from right to left in the figure are \chinese{風}, \chinese{小雅}, \chinese{大雅}, and \chinese{頌}.
The corresponding first poems are \chinese{關雎}, \chinese{鹿鳴}, \chinese{文王}, \chinese{清廟}.
In \cref{fig:node-link} (Right), the four initial poems are placed above the center character, ``\chinese{為}'', that functions as ``is'' in English.
The four corresponding first poems are placed below the center character.
The figure should be read following a top-to-bottom vertical order.
For example, the chain \chinese{關雎} $\rightarrow$ \chinese{為} (is) $\rightarrow$ \chinese{風} $\rightarrow$\chinese{始} (start) in \cref{fig:node-link} (Right) means ``\chinese{關雎} is the start of \chinese{風}''.
One of the patterns is that, in such a text-link diagram, terms that repeatedly occur in several parallel textual segments (e.g., \chinese{為}) are placed at the central of the graphic with other terms surrounding~\cite{Lackner2007Diagrams}.
The main objective of this pattern may be to reveal the parallelism among textual segments or to emphasize the crucial message of textual segments lies with the center through demonstrating the network-like structure~\cite{Lackner2007Diagrams}.
The text-link diagram may be used primarily for pedagogical purposes by showing the ``general meaning'' and characterizing a passage as a mnemonic element to help readers quickly remember paragraph~\cite{Lackner2007Diagrams}.

\textbf{Comparison with the contemporary visual convention:}
The design of such historical text-link diagrams looks similar to the SentenTree design~\cite{Hu2017Visualizing}, which is also a node-link diagram with text being nodes and links indicating word co-occurrence.
However, we note that the historical text-link diagrams exhibit a different visual convention from contemporary node-link diagrams.
In contemporary node-link diagrams, when there visually exists a path between two nodes, it indicates there exists a path in the underlying graph data structure.
In contrast, for historical text-link diagrams, such an indication is not valid.
For these diagram, only when there exist a visual path between two nodes, and that the two nodes are vertically aligned, can we infer the existence of a path in the underlying data structure.

\textbf{Node as circle:}
Linked circles can be found in historical Chinese star charts (\cref{fig:node-link} (Left)).
Although star charts carry positional information, their visual representation highly resembles a conventional node-link diagram, and the linking conveys the conceived connection among stars.
Others may utilize node-link diagrams to present the relationship of concepts in Taoist philosophy (\cref{fig:node-link} (Middle)).
With the incorporation of node-link diagrams and Taoist philosophical concepts, Taoist visualizations may historically serve as esoteric materials accessible to practitioners and provide visual aids to facilitate adepts' asceticism~\cite{Huang2015Picturing}.

\begin{figure}[!tb]
    \centering
    \includegraphics[width=\linewidth]{./assets/imgs/7-tree/tree.png}
    \caption{
        \textbf{Examples of tree in \datasetName:}
        The trees visualize the concept of \term{Changes} in \term{I Ching}.
        (Left)
        \href{https://dl.ndl.go.jp/pid/2596441/1/46}{A tree}
        from \term{Tu shu bian}~\cite{Huang1613Tu}, 1613.
        (Middle)
        \href{https://dl.ndl.go.jp/pid/2596441/1/58}{A radial tree}
        from \term{Tu shu bian}.
        (Right)
        \href{https://iiif.lib.harvard.edu/manifests/view/drs:53063658$144i}{An icicle tree}
        from \term{Yu zuan Xing li jing yi}~\cite{Li1717Yu}, 1717.
        \future{Reproduce these figures with their corresponding contemporary design to make their visual encoding clear.}
        \yzc{Instead of using ``Top'' and ``Bottom'' to refer to subfigures, I would suggest adding labels.}
 }
    \Description{
        Examples of trees in \datasetName.
    }
    \label{fig:tree}
\end{figure}

\subsubsection{Tree}

Tree visualizations show hierarchical data.
While hierarchical data may be visualized with a node-link diagram, making tree and node-link diagram potentially overlapping, in this work, we differentiate trees from node-link diagrams.
Our consideration is that \datasetName contains numerous tree visualizations that exhibit characteristics distinct from node-link diagrams.
Historical Chinese tree visualizations mainly consisted of tree visualizations for genealogy and abstract concepts.

\textbf{Visualizing genealogy:}
Genealogy in historical Chinese visualizations covers various domains, including not only family pedigrees but also apprenticeships and knowledge transmission, such as the visualization of apprenticeship relation in \cref{fig:textual-criticism}.

\textbf{Visualizing abstract concepts:}
A tree is also frequently used to visualize abstract concepts in historical Chinese visualizations, such as the concept of \term{Changes} in \ChineseTerm{I Ching}{易經}{Book of Changes}, a divination text in ancient China.
It corresponds to a binary tree structure with the first level representing \ChineseTerm{Taiji}{太極}{Supreme Ultimate}.
\emph{Taiji} generates \ChineseTerm{Liangyi}{兩儀}{Two Modes} on the second level.
The \term{two modes} generate \ChineseTerm{Sixiang}{四象}{Four Images} on the third level.
The \term{four images} generates \ChineseTerm{Bagua}{八卦}{Eight Trigrams} on the fourth level.
This binary tree may also be extended to the seventh level, resulting in 64 \ChineseTerm{Gua}{卦}{Hexagrams}.
The hierarchy can be visualized as trees with different designs, as \cref{fig:tree} shows.
This visualization can also be found with the accompaniment of other iconic concepts (e.g., the \term{Celestial Stems}, the \term{Sexagenary Cycle}, and the \term{Five Phases}) in China~\cite{Kalinowski2007Time}.
The functions of these divination diagrams vary from fortune-telling to the origin of binary system~\cite{Cammann1991Chinese}, and the actual use of them remains in debate.


\begin{figure*}[!htb]
    \centering
    \includegraphics[width=\linewidth]{./assets/imgs/8-tables-in-shiji/tables-in-shiji.png}
    \caption{
        \textbf{Examples of tables in \datasetName:}
        The chronological tables are from  \term{\Shiji}~\cite{Si1550Shi}
        (Top) \ChineseTermWithUrl{\SanDaiShiBiao}{三代世表}{Genealogical Table of the Three Ages}{https://iiif.lib.harvard.edu/manifests/view/drs:19140085408i} is an example of horizontal chronological table.
        The chronological order is from right to left.
        Note that this order is consistent with the right-to-left writing system of ancient Chinese.
        (Bottom) \ChineseTermWithUrl{\GaoZuGongChenHouZheNianBiao}{高祖功臣侯者年表}{Yearly Table of the Officials who became Marquises in the Time of Gaozu}{\UrlOfGaoZuGongChenHouZheNianBiao} is an example of a vertical chronological table.
        The chronological order is from top to down.
        (Due to space limit, the figure is incomplete.)
    }
    \Description{
       Examples of tables in \datasetName.
    }
    \label{fig:tables-in-shiji}
\end{figure*}

\subsubsection{Table}

In this work, we regard tables as visualization following the view~\cite{Riggsby2019Mosaics}, which suggests that the meaning of tables lies in their matrix-like structure rather than the data points within the cell.
In the following discussion, we show that historical Chinese tables are very much in line with this view in that their structures vary and serve different purposes.

Overall, we observe three types of historical Chinese tables in \datasetName: vertical, horizontal, and radial.
The vertical and horizontal tables were widely used in books related to history and social statistics, such as \ChineseTerm{\Shiji}{史記}{Records of the Grand Historian} and local gazetteers.
Due to the traditional Chinese book production methods, these long tables were usually divided into several pages.
Tables in \term{\Shiji} were usually employed to record administrative affairs such as enfeoffments and appointments~\cite{Vankeerberghen2007Tables}.
It has two forms of tables, which indicate different chronological order.
For the horizontal table (\cref{fig:tables-in-shiji} (Top)), readers should follow a temporal order from right to left (traditional Chinese reading direction) to read the chronology of a country or a dynasty, with the cells listed from the very right-hand side of the document rows showing the names of the country or emperors.
In this case, the horizontal table functions as a contemporary horizontal timeline~\cite{Rosenberg2010Cartographies}.
For vertical tables (\cref{fig:tables-in-shiji} (Bottom)), the very right side of the document shows emperors' reigns in chronological order from top to bottom, and each column represents a noble house from right to left~\cite{Vankeerberghen2007Tables}.

While these tables may seem simple to read and compile, their meaning goes beyond the data in the cells in historical contexts.
First, these tables can reveal historical insights. 
For example, in the vertical chronology, compared with a period where the entire column is filled with names of kingdoms, the period with few kingdoms may imply a dramatic loss of both territory and autonomy~\cite{Vankeerberghen2007Tables}.
This comparison is similar to grids with different color opacity in a modern heat map.
Second, political considerations may influence the choice of vertical and horizontal tables.
A horizontal table was used to present kingdoms because the long period available horizontally helps to present the continuity of the royal family\cite{Vankeerberghen2007Tables}.
The noble houses, however, were placed on vertical tables.
This may be because their fate was determined by the transition of central power (from top to bottom); these houses reflected the bureaucratic hierarchy, and their family continuity was not the main focus~\cite{Vankeerberghen2007Tables}.

Besides \term{\Shiji}, the vertical tables can also be found in family genealogies, where from top to bottom is the family pedigree.
There are also radial tables which can be viewed as the result of bending a horizontal table until the left and right sides are connected.
The original horizontal order (e.g., temporal) was then represented by evenly divided radians.
These tables usually appear in domains related to Chinese traditional solar terms and calendars.

\begin{figure}[!htb]
    \centering
    \includegraphics[width=.3\linewidth]{./assets/imgs/9-annotated_diagram/illus.jpeg}
    \caption{
        \textbf{\href{https://tile.loc.gov/image-services/iiif/service:asian:lcnclscd:2012402918:1A007:01b02a/full/pct:100/0/default.jpg}{An example of annotated diagrams in \datasetName}:}
        From \ChineseTerm{\ZhenJiuDaCheng}{鍼灸大成}{Compendium of Acupuncture and Moxibustion}~\cite{Yang1680Zhen}, 1680.
        In traditional Chinese medicine, different body elements are considered as a whole~\cite{Despeux2007Body}.
        In this visualization, circles representing acupuncture points are connected to form Meridian and Collateral (the passages transporting qi and blood).
        \yzc{This figure is too thin. I would recommend moving part of the caption into the figure as annotation.}
    }
    \Description{
        An example of an annotated diagram in \datasetName.
    }
    \label{fig:illus-diagram}
\end{figure}

\subsubsection{Annotated Diagram}

In this work, we regard annotated diagrams as a special type of illustrations accompanied by labels and lines connecting a concept to a part of the illustration. 
Most historical Chinese annotated diagrams visualize disease and the human body in relation to traditional philosophy.
The contributor to these visuals includes Taoism, divination through body examination, traditional medicine, and forensic medicine~\cite{Despeux2007Body}.
In general, the early Chinese annotated diagrams for the human body focus on depicting the body as a whole~\cite{Huang2015Picturing, Correll2024When} or as a microcosm in the image of the macrocosm~\cite{Despeux2007Body}, while these diagrams in the West pay more attention on musculoskeletal anatomy but lack of the focus on whole systems and features.
This tendency in ancient China can be partially explained by the influence of Taoist philosophy and traditional Chinese medicine.
For Taoism to depict a body, they preferred to emphasize the balance of \term{Yin} and \term{Yang} or the flow of \term{Qi}, and connect spiritual metaphor to the inner world of body~\cite{Despeux2007Body}.
In Chinese medicine, the different elements of the body were considered in relation to each other and within systems of correlation, where they focus on viscera and the circulation of the humours and energies along the meridians~\cite{Despeux2007Body}.
\Cref{fig:illus-diagram} shows the acupuncture points as well as Meridian and Collateral of a whole human body, which reflects concepts in both traditional Chinese medicine and Taoism.
Here, anatomy was not an important part, thus giving rise to the absence of musculoskeletal visuals.
Hence, traditional Taoism and traditional Chinese medicine are crucial factors that influenced the styles of historical Chinese annotated diagrams.

\subsection{Reflections}

\textbf{On overall patterns:}
Ancient Chinese can be regarded as a large discourse community that has its visual conventions~\cite{Kostelnick2003Shaping}.
The visual conventions can be reflected by \revised{the widespread understanding from ancient Chinese people on} visual coding methods and the preference for using pictorial representations in visualizations.
\Cref{sec:taxonomy} discusses that historical Chinese visualizations might serve as political reports, mnemonic elements, and pedagogical resources, but rarely as an analysis approach.
We observe that these visualizations prioritized visual communication over visual analysis.
Similarly, the data visualized in ancient China tended to be imaginary, conceptual, and relational rather than numerical.
Additionally, many historical Chinese visualizations adhere to textualism and pictorialism, which are characterized by the extensive use of pictorial elements in maps and annotated diagrams, non-linear texts in maps (where texts serve as glyphs), trees (where texts serve as leaves of a tree), node-link diagrams (where texts serve as nodes), and annotated diagrams (where texts serve as labels).

\textbf{\revised{On the framework}:} 
In the \cref{sec:taxonomy}, we primarily discuss historical factors, visualization producers, data, and communicated messages. 
This paper has not discussed the components to whom historical Chinese visualization was read and the effect on them. 
We leave these potential discussions to future researchers, which require a deeper historical background.

The following summarizes the discussions in \cref{sec:taxonomy} according to the framework components.
Regarding who produced these visualizations, many historical Chinese visualizations such as maps, node-link diagrams, and annotated diagrams might have been primarily produced by political groups or social elite classes in ancient China.
In terms of data domain and types, although we observe visualizations covering multiple topics—indicating that visualization had been widely used in ancient China—they were more focused on conceptual relationships rather than quantitative measurements.
Much of the data for these visualizations stems from traditional Chinese classics, such as the \term{I Ching} and the traditional Chinese cosmology constructed the forms and connotations of the \term{Taiji} diagram, and literary classics, such as the \term{\ShiJing}, constructed a Chinese early knowledge network.
These ancient Chinese philosophical ideas shaped, to some extent, the emphasis of visualizations on symbolic and qualitative relationships.
We aim to highlight the importance of incorporating historical and cultural contexts into the process of visualization creation and understanding.
We believe that historical factors are essential for understanding historical visualizations under specific ``data cultures'' regarding their formation, functionality, and unique patterns with underlying causes.


\textbf{On comparison with \revised{contemporary} visualizations:}
Here, we summarize the differences between visualizations in ancient China and contemporary according to Baur and Felsing~\cite{Baur2020Visual}.

\begin{itemize}[leftmargin=3.5mm]
    \item \textbf{Visual aesthetics:} Historical Chinese visualizations often prioritize harmonious layouts, flowing compositions, and the integration of textual and pictorial elements, reflecting cultural values of balance and unity~\cite{Baur2020Visual}. 
    In contrast, contemporary visual traditions sometimes emphasize geometric precision, symmetry, and separation of textual and visual elements.
    \yzc{Isn't ``symmetry'' part of ``harmonious layouts''?}
    
    \item \textbf{Purpose:} In ancient China, visualization was more inclined toward the simple display of data, such as in agriculture, handicrafts, and religion, and was influenced by the government~\cite{Yee1994Reinterpreting} and traditional philosophy. 
    Contemporary visualizations are more often used for data analysis, measurements, and exploration, driven by navigation and scientific research~\cite{Baur2020Visual}.

    \item \textbf{Flexibility of visual elements:} Historical Chinese visualizations often emphasized the interdependence within the visual system, with the semantics of visual elements being relatively fixed and less flexible for arbitrary combinations~\cite{Baur2020Visual}. 
    In contrast, contemporary visualizations' visual elements are semantically flexible and capable of conveying multiple meanings through legends and variables such as color and size, making them more suitable for recombination.
    Fixed semantics in ancient China, on the other hand, can evoke more direct cultural and emotional resonance, which is further discussed in \cref{sec:culture-focused-design}.
\end{itemize}

By comparing culture-specific historical visualizations with contemporary designs, we attempt to showcase the importance of constructing a more diverse visualization evaluation system . 
Traditional visualizations are rooted in a philosophical and cultural context and have significance beyond the data itself, so we should not only evaluate them in terms of functionality but should also consider the philosophical and aesthetic aspects of visualizations as cultural carriers.



