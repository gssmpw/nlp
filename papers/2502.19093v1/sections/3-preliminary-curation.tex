\section{\revised{Preliminary Data Curation for \datasetName}}
\label{sec:preliminary-curation}

\revised{
Before the large-scale data curation described in \cref{sec:large-scale-curation}, we went through a preliminary data curation process.
Through this process, we manually collected \numPreliminaryRelevantImages images of historical Chinese visual representations from the web and refined our scope for data collection.

\textbf{Manual collection from the web:}
We first manually collected \numChaofanYangRelevantImages images of historical Chinese visual representations from the web.
These visual representations include maps, genealogies, geometry, and paintings.
Their physical forms encompass a variety of materials, including stone, wooden boards, silk, and paper.
Their themes include geography, astronomy, medicine, and genealogy.
As we noticed a manually collected image was from Dunhuang, we looked for more relevant images from Pelliot chinois Dunhuang manuscripts\footnote{Last accessed on Feb 1, 2024 with the search keyword \href{https://gallica.bnf.fr/services/engine/search/sru?operation=searchRetrieve&version=1.2&query=(gallica all "Pelliot chinois")}{``Pelliot chinois''}.} in Gallica~\cite{NLFGallica}, and obtained another \numDunhuangRelevantImages images.
We also manually collected \numShugeBooks historical Chinese books from the Shuge digital library~\cite{Ceng2013Shuge} where we obtained \numShugeRelevantImages images of historical visual representations.

\textbf{Scope:}
Our initial objective of data collection was to gather images corresponding to \taxon{visualization}.
Specifically, we refer to \taxon{visualization} as a visual representation that uses graphical marks to encode abstract or spatial data\footnote{
    In this paper, we adopt a broad notion of \taxon{visualization} that includes visual representations that may not commonly be regarded as \taxon{visualization}.
    For example, despite the controversy on whether \taxon{map} should be categorized as \taxon{visualization}~\cite{Hograefer2020State,Friendly2010First,Chen2024Image,Zhang2025VisTaxa}, we include \taxon{map} as \taxon{visualization}.
    We also include \taxon{table} of structured data as \taxon{visualization}.
}.
Through our data collection practice, we observed that many figures presented in historical Chinese books did not fall into conventional notions of \taxon{visualization}.
These \taxon{non-visualization} figures generally follow the specification of \taxon{illustration} in Zhang et al.'s taxonomy~\cite{Zhang2025VisTaxa} that defines \taxon{illustration} as ``a visual representation that commonly uses drawings, sketches, or paintings to represent visual explanations of concepts, plans, processes, or scenes''.
We decided to include \taxon{illustration} as part of our dataset under two considerations.
Firstly, \taxon{illustration} may also serve some of the usage scenarios described in \cref{sec:usage-scenarios}.
Secondly, the border between \taxon{visualization} and \taxon{illustration} can be subjective, which is further discussed in~\cref{sec:lesson}.
By including \taxon{illustration} into our dataset, we aim to provide a more comprehensive dataset that users may redefine the boundary between \taxon{visualization} and \taxon{illustration} based on their needs.
Throughout the writing, we use \taxon{graphic} to refer to both \taxon{visualization} and \taxon{illustration}.

As we decided to expand the collection to include more data sources, the manual collection process was no longer scalable.
To improve the efficiency, we implement a semi-automatic pipeline for large-scale data curation, as described in \cref{sec:large-scale-curation}.
}



