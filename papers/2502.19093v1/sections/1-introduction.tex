\section{Introduction}
\label{sec:introduction}

Historical visualizations hold a pivotal role in the study of visualization history. 
By examining the visualization practices and design principles of different historical periods, researchers can gain a deeper understanding of the evolution and development of contemporary visualization methods.
Culture plays a significant role in shaping historical visualizations. 
Different civilizations exhibit distinct features in their visual communication and design aesthetics, which impose distinct requirements on how information is represented and conveyed within each cultural context.
Therefore, studying historical visualizations from diverse cultural perspectives is crucial for transcending singular or biased viewpoints and achieving a more objective and comprehensive understanding.

However, current studies on historical visualizations, whether focused on single case~\cite{Koch2009Crediting, Friendly2002Visions, Friendly2005Early, Wilkinson2009History, Marchese2011Exploring} or large corpora~\cite{Zhang2024OldVisOnline, Friendly2001Milestones}, are limited to eurocentric views~\cite{Gunter2021Review}.
The main reason could be the language barrier and cultural discrepancy~\cite{Friendly2021Discussion}, which hinders the inspection of visualizations across different cultures. 

To bridge this research gap, we investigate historical visualizations within the Chinese cultural context. 
By bringing Chinese historical visualizations into the broader discourse of visualization studies, we seek to stimulate further cross-cultural investigations of historical visualizations and enhance our understanding of diverse visualization practices.


Specifically, we propose \textit{\datasetName}, a collection of historical Chinese visualizations.
It is named after \textit{Zuan Tu} (``\chinese{纂圖}'' in Chinese), the classical Chinese term referring to the list of figures placed at the beginning of a book. 
Historical Chinese books contain diverse visual representations, some conveying data and concepts, while others do not encode data.
In this work, we collect all visual representations and termed them as ``graphics'' with those conveying data classified as visualizations and decorative elements referred to as illustrations.

Our work first collects a corpus of historical Chinese graphics from various sources, including a large number of historical Chinese books. 
In the corpus, there are over \numVisShort visualizations and \numIllusShort illustrations retrieved from data sources with more than \numBookShort books.
Hidden in ancient books and other mediums, these historical Chinese \itemTypePlural cannot be easily searched or accessed online.
The data curation process is introduced in~\cref{sec:preliminary-curation} and~\cref{sec:large-scale-curation}.

Based on the dataset, we study the visual patterns of historical Chinese visualizations and analyze the possible causes behind them in~\cref{sec:zuantuset}.
We envision several usage scenarios for our collection in~\cref{sec:usage-scenarios}. 
We also discuss our experiences working with historical documents under different cultures and share our experiences to guide future endeavors in~\cref{sec:discussion}.

In summary, the contributions of this work are:

\begin{itemize}[leftmargin=3.5mm]
    \item We contribute \datasetName, a large-scale historical Chinese visualization dataset.
        \datasetName is constructed through a semi-automatic pipeline to extract visualizations from historical Chinese books.
        \datasetName can be browsed with a gallery at: \url{https://zuantuset.github.io/gallery}.

    \item We introduce Chinese historical visualization into the field of historical visualization studies, examining its visual characteristics and formative factors. We also discuss the usage scenarios of \datasetName.
      
\end{itemize}




