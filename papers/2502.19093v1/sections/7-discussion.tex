\section{Discussion}
\label{sec:discussion}

\revised{
In \cref{sec:lesson}, we reflect on our data curation process.
In \cref{sec:discuss-convention}, we reflect on the cross-cultural visualizations in terms of their formation, value, and guidelines for practical usage.
}

\subsection{\revised{Reflections on Data Curation}}
\label{sec:lesson}

We discuss some of our considerations and challenges in the data curation process.
We also discuss future work to extend \datasetName.

\textbf{Number of classes in detection:}
We set the object detection model to detect two classes, \taxon{visualization} and \taxon{illustration}, based on the following considerations.
We differentiate the two classes as our primary focus is on visualizations.
Meanwhile, the boundary between visualizations and illustrations is not always clear.
We include illustrations in the dataset so that the border can be easily revised during iterations, compared with not including illustrations at the time of detection.
Similarly, we categorize visualization into multiple subcategories after obtaining the visualization images.
While an alternative approach is to set the goal to directly detect images belonging to these subcategories, this alternative loses the flexibility of iteratively revising the boundary of visualizations.



\textbf{A label management strategy in classification:}
A strategy that may improve efficiency and quality of developing taxonomy labels in large scale data curation is to cluster the images that are extremely similar in visual appearance and constrain them to the same taxon.
For example, the images corresponding to different editions of a same visualization should be clustered.
Our consideration is that developing a taxonomy is an iterative process that requires frequent modifications.
By clustering similar images, we can ensure that the label modifications are applied to all images in the same cluster.
For large datasets, this approach may save time and reduce the risk of missing some images with labels to be edited when revising the taxonomy labels.
While we did not implemented this strategy in our own data curation process, looking back, we believe it could have been beneficial.

\textbf{Challenge in defining taxa boundaries for the taxonomy:}
In this work, we first grouped graphics into \taxon{visualization} and \taxon{illustration} and then further classified them based on their visual appearance.
As is the case for other taxonomy development processes~\cite{Chen2024Image}, clearly defining the boundaries of taxa is challenging.
A particular challenge for developing a taxonomy of historical Chinese visualizations is that due to textualism and pictorialism, the boundaries between visualization, text, and illustration can be blurred.
Additionally, the level of understanding of the historical artifacts may also influence that judgment.
To improve the current taxonomy call for expertise from not only visualization experts but also historians.





\textbf{Future work:}
Building a dataset of visualizations from diverse cultural frameworks is a gradual process. 
We believe that in the future, the dataset's content and label quality can be further improved in the following aspects.
First, classifying visualizations within a specific culture requires domain knowledge.
We may involve more domain experts to refine the taxonomy.
We may also investigate using vision-language model to assist in the classification of historical Chinese visualizations~\cite{Springstein2024Visual}.
So far, our data source is limited to three digital libraries.
The timeline in \cref{fig:gallery} shows that most of the books collected currently are from the Ming dynasty and later.
To expand the data coverage for earlier periods, we plan to utilize additional data sources, such as the National Library of China.
We plan to continue to expand the coverage of the \datasetName, and explore more applications to revive historical Chinese visualizations.

\subsection{Reflections on Cross-Cultural Visualization}
\label{sec:discuss-convention}

\begin{figure}[!htb]
    \centering
    \includegraphics[width=\linewidth]{./assets/imgs/12-translate/translate.png}
    \caption{
        An example of visual translate \revised{created by Baur and Felsing}~\cite{Baur2020Visual}.
        The target visualization depicts different buildings occupied by ancient emperors during different time periods. 
        The redesign involves three visual translations.
        (A) translate rectangular building codes into ancient-style glyphs
        (B) a legend mapping building orientation to direction
        (C) arrows encode the emperor's movement trajectory over time
        The visualization is visually translated, making it more understandable to a wider audience.
    }
    \Description{
        An example of visual translation.
        The visualization represents ancient emperors' buildings and movements over time using glyphs for buildings, a legend for directions, arrows for movement, and colored circles for seasons, making it accessible to a broader audience.
    }
    \label{fig:translate}
\end{figure}

In this section, we use the concept of visual convention~\cite{Kostelnick2003Shaping} to reflect visualizations across different cultures from a higher-level perspective.
We discuss the formation and evolution of visual conventions across communities and examine practices for promoting them in contemporary applications.

\textbf{Definition of visual conventions:}
A discourse community is a group of people with shared goals, practices, and knowledge, especially in communicating.
These groups establish and use shared visual conventions. 
A visual convention is a system of shared symbols or practices that people recognize over time, enabling clear and efficient visual communication~\cite{Kostelnick2003Shaping}.
For example, musicians form a discourse community, and musical notation is one of their visual conventions~\cite{Kostelnick2003Shaping}.

\textbf{Impact factors for visual conventions:}
Visual conventions change due to technology advancements and shifts in sociocultural factors~\cite{Kostelnick2003Shaping}.
\revised{
As new communities form and old ones disappear, some conventions are forgotten over time, while others last and grow in influence.
On one hand, advancements in modern technology, such as web-based visualization, have shaped new contemporary visual conventions.
For example, in an interactive visualization, when a user hovers over an element, the de-highlighted elements are considered irrelevant to the currently hovered element. In contrast, the highlighted elements are considered relevant.
On the other hand, it also hindered the continuity of ancient conventions. 
Visual elements rendered in HTML are mostly restricted to fluent modernist lines, losing the historical hand-drawn style.
}
Cultural and organizational contexts also shape distinct visual conventions across nations and institutions~\cite{Kostelnick2003Shaping}. 
One potential explanation arises from Hall's theory about high-context and low-context culture~\cite{Hall1977Culture}. 
In high-context cultures (e.g., China, Korea, Japan~\cite{Kim1998High}), communication tends to be implicit, relying heavily on background context, social relationships, and environmental cues. 
This emphasis on context is reflected in design, where metaphor and symbolism are often employed to convey deeper meanings. 
In contrast, low-context cultures (e.g., America~\cite{Kim1998High}) rely on explicit information encoding with minimal dependence on context. 
This is reflected in designs that prioritize clarity and directness.


\textbf{Practice for utilizing visual conventions:}
Designers need to ensure that the audience can understand visual conventions from different cultures or histories when using them in design.
On one hand, as discussed in~\cref{sec:culture-focused-design}, it can streamline comprehension for those who live under that cultural framework and evoke a rethink of contemporary designs.
On the other hand, if designs are presented directly with unfamiliar visual conventions without any explanation, the audience may not understand the designs and become confused and frustrated.
Therefore, information designers should provide a ``visual translation''~\cite{Baur2019Cultural} to bridge that gap between different visual conventions, as well as maintain the characteristics for them.
Visual translation involves reinterpreting culturally specific visual designs and conventions using modern, universally recognized visual norms~\cite{Baur2019Cultural}. 
Critical cultural information embedded in the original design may not be visually apparent or understood by outsiders but can be conveyed through shared visual norms and appended to the original design.
\Cref{fig:translate} shows an example designed by Baur and Felsing~\cite{Baur2020Visual}.
It employs three visual translators (\cref{fig:translate}(A), \cref{fig:translate}(B), and \cref{fig:translate}(C)) to reveal the implicit knowledge embedded in historical Chinese layout plan visualizations.
\Cref{fig:translate} depicts the \term{Mingtang}, an ancient ceremonial building symbolizing cosmic order through its architecture.
Its layout guided the emperor's ceremonial activities, and each room corresponds to a specific month, determining where the emperor should be during different months of the year.
Three visual translations are employed to convey this information.
\Cref{fig:translate}(A) uses a glyph extracted from traditional Chinese architecture to replace the original rectangular encoding, which conveys ancient aesthetics and reduces cognitive load. 
\Cref{fig:translate}(B) acts as a legend indicating the relationship between the direction and buildings' orientation. 
\Cref{fig:translate}(C) uses a circular arrangement with sequential arrows to imply the function of guiding emperor activities.
Finally, these visual translations are integrated with the original visualizations, incorporating color-coded circles to symbolize the seasons, forming the final translated visualization. 
The new visualization preserves the historical Chinese visual conventions while aligning them with the visual conventions in modern society to improve understanding across different cultures.
