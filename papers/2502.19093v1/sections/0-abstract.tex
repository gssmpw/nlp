Historical visualizations are a valuable resource for studying the history of visualization and inspecting the cultural context where they were created.
When investigating historical visualizations, it is essential to consider contributions from different cultural frameworks to gain a comprehensive understanding.
While there is extensive research on historical visualizations within the European cultural framework, this work shifts the focus to ancient China, a cultural context that remains underexplored by visualization researchers. 
To this aim, we propose a semi-automatic pipeline to collect, extract, and label historical Chinese visualizations.
Through the pipeline, we curate \datasetName, a dataset with over \numVisShort visualizations and \numIllusShort illustrations.
We analyze distinctive design patterns of historical Chinese visualizations and their potential causes within the context of Chinese history and culture.
We illustrate potential usage scenarios for this dataset, summarize the unique challenges and solutions associated with collecting historical Chinese visualizations, and outline future research directions.
