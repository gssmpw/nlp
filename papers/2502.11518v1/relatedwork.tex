\section{Related Work}
% Other notable related reviews include the following:  

% \cite{hu2023toward}, \cite{firoozi2023foundation}, \cite{ma2024survey}, \cite{duan2022survey}, \cite{liu2024aligning} systematically present the current state of research in Embodied AI, but none of them specifically focus on the application of multi-agent systems.

% \cite{guo2024large},\cite{chen2024survey},\cite{lu2024merge} focuses on multi-agent systems based on large language models and the multi LLM collaboration, providing a comprehensive overview of their progress, applications, and challenges, but it does not delve deeply into the application of multi-agent systems in the embodied domain.

% \cite{ismail2018survey}, \cite{yan2013survey} firstly discussed Cooperative Multi-Agent Robot Systems, which is closely related to our topic. However, these papers were written before the generative AI era, and with the rapid development of technologies such as FMs in recent years, multi-agent systems have gained new dimensions, making it essential to revisit the latest developments in this area.

% \cite{hunt2024survey} explores the applications and development of language-based robots by examining communication between robots, between humans, and between human and robot. While language communication is a key component of multi-agent collaboration, it does not encompass all forms of collaboration.

% \cite{sun2024llm} introduces FM-based multi-agent reinforcement learning, focusing on its applications in embodied AI. However, it does not entirely focus on the embodied domain but rather emphasizes the perspective of reinforcement learning.

Although numerous surveys have examined embodied AI or MAS individually, few efforts have tackled the critical overlap between these fields, leaving significant knowledge gaps unaddressed.
Early studies on embodied AI~\cite{hu2023toward,firoozi2023foundation,ma2024survey} focus on single-agent perception-action loops. 
They discuss autonomy and sensorimotor learning in depth, yet devote limited attention to collaborative paradigms. 
Similarly, \cite{duan2022survey} and \cite{liu2024aligning} explore how agents interact with environments but still assume solitary agents with limited capacity for distributed teamwork.
% in complex tasks.

In contrast, recent surveys on FM-driven multi-agent systems~\cite{guo2024large,chen2024survey,lu2024merge} showcase promising results in semantic communication and emergent coordination, especially in virtual environments. 
However, these contributions remain detached from physical embodiment, where hardware constraints, sensor noise, and kinematic coordination pose significant challenges. 
Meanwhile, classic robotics surveys~\cite{ismail2018survey,yan2013survey} laid the groundwork for cooperative swarm behaviors but lack generative capabilities that facilitate role adaptation or zero-shot planning.

Several specialized reviews provide partial bridges. 
For instance, \cite{hunt2024survey} advances language-based human-robot interaction, yet overlooks non-linguistic coordination crucial in industrial or warehouse settings. 
Likewise, \cite{sun2024llm} integrates FMs with MARL but treats embodiment mostly as an implementation detail. 
Consequently, none of these views examine physical grounding, collaborative intelligence, and generative models under a unified lens.

Our survey addresses this gap by synthesizing insights from three converging axes: 
(i) embodied AI’s physical imperatives, 
(ii) multi-agent systems’ collaborative intelligence, and 
(iii) generative models’ adaptive reasoning. 
We propose a novel taxonomy that reconciles embodiment multiplicity (physical agents) with model multiplicity (virtual agents). 
Through case studies in cross-modal perception and emergent communication, we show how FMs overcome conventional multi-agent limitations in real-world embodied contexts. 
Finally, we identify underscored challenges, such as Sim2Real transfer for generative collectives, bridging the divide between robotics and FM-based coordination. 
Our work thus establishes conceptual foundations for a new generation of embodied systems, where physical constraints and generative collaboration progress in tandem.

%