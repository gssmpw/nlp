\section{Related Work}
\label{sec:related_work}
    Adaptive BCIs using ErrPs have previously been proposed.
    \cite{llera_use_2011} introduced an adaptive logistic regression based on interaction ErrPs. The weights of the classifier were modified based on the ErrP classification results. The approach was validated using both simulated and MEG data for a two-class MI paradigm, showing significant performance improvements compared to the baseline static classifier. \cite{schiatti_effect_2019} later applied the same approach to MI data recorded with EEG. 
    \cite{mousavi_improving_2017} introduced a new strategy by directly combining the ErrP frequency-domain information and the MI-related modulations to improve the classification of MI trials. They used common spatial patterns (CSP) for feature extraction and linear discriminant analysis (LDA) for classification of ErrPs and MI, combining the results with a logistic regression and observed significant improvements in performance with the proposed framework. This approach was further validated in an online follow-up study~\citep{mousavi_hybrid_2020}. 
    
    The ErrP information have often been used to validate the output of the BCI classifier. 
    An online BCI-speller based on code-modulated visual evoked potentials (c-VEP) and ErrP was validated in \cite{spuler_online_2012}. In this study, c-VEP trials were classified using a support vector machine (SVM) and a spatial filter (cannonical correlation analysis). The ErrP information was used to label trials for the training dataset.
    \cite{artusi_performance_2011} considered the classification of movement-related cortical potentials (MRCPs) into different motor tasks (e.g., slow vs. fast arm flexion). They also used the ErrP information to label trials before adding them to a the training dataset for an SVM classifier.
    This approach was also used recently by \cite{tao_enhancement_2023} in a two-class MI task, using regularized common spatial patterns (R-CSP) for feature extraction and the combination of Fisher's discriminant analysis (FDA) and SVM for classification. Lastly, for the classification of MI data using k-NN, \cite{haotian_online_2023} also used the trials labeled based on the ErrP to create a dataset and, after applying cross-validation to evaluate MI classification improvement, they expanded the training dataset with the new trials. 
    \cite{chiang_closed-loop_2021} used a similar approach to show the benefits of including the ErrP feedback information in the the adaptation of a convolutional neural network (CNN) for the classification of steady state visually evoked potentials (SSVEPs) and \cite{wang_toward_2024} for MI classification. However, as these studies used three- and four-class problems, respectively, the training dataset only included trials that did not elicit ErrPs. 
    All these studies reported improved performance when using the ErrP-based adaptation.