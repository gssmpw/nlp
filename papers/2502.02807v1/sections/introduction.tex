\section{Introduction}

{\bf Motivation.} Motivational Interviewing (MI) is a client-centered counseling technique aimed at addressing ambivalence and facilitating behavior change in clients, particularly for issues like binge drinking and substance use~\citep{miller2002motivational}. 
%By addressing ambivalence and enhancing self-efficacy, 
MI promotes intrinsic motivation and fosters a collaborative relationship between clients and counselors, ultimately improving clients' commitment to intervention programs~\citep{martins2009review}. 
%The effectiveness of MI techniques relies on psychotherapy skills that include not only dialogue strategies but also personalized counseling programs and change talk exploration throughout various stages of the counseling process~\citep{cole2023using}.

%The recent emergence of large language models (LLMs)~\citep{brown2020language,ouyang2022training,achiam2023gpt,dubey2024llama} presents new opportunities for generating diverse, flexible, and engaging dialogues. 
Although recent works building LLM-based counselor agents have shown some promising results, we observe at least three limitations with existing works. First, they do not follow the collaborative counseling style in MI. For example, \citet{steenstra2024virtual} describes MI in a prompt which may not provide LLM adequate guidance to perform a complex multi-turn conversational task such as MI counseling. Second, those works that use strategy planning overlook the client's state and the need to explore a diverse range of topics that can evoke change talk based on the client’s unique motivations~\citep{hsu2023helping,sun2024chain,xie2024few}. As a result, they fall short in providing client-centered counseling~\citep{steenstra2024virtual} and systematic change talk exploration, especially given the inherent preference bias of LLMs~\citep{kang2024can}. Finally, the evaluation of counselor behavior in most works focus on comparing the agent generated responses with the predefined ``ground truth'' responses at the turn level~\citep{sun2024chain}. Such an evaluation approach rules out other possible high quality responses.  Meanwhile, human evaluation of counselor agents remains costly~\citep{xie2024few,hsu2023helping,steenstra2024virtual}. 
%This research task thus still lacks a more comprehensive evaluation framework.

{\bf Objective.} To tackle these challenges, we propose counselor agents that organically integrate psychotherapy skills, incorporating three novel ideas. First, we design a client’s state inference module that models the client's state of mind to provide personalized counseling tailored to different clients. This module draws inspiration from the transtheoretical model of health behavior change~\citep{prochaska2005transtheoretical,prochaska2008initial}, which defines five stages of change and suggests that the counselor should adjust their goals and strategies according to the client’s current stage. Second, we introduce a topic tree based on a real counseling dataset and a knowledge base (Wikipedia) that systematically models the topics of change talk exploration in MI counseling. With the help of the topic tree, the counselor can engage with a greater diversity of topics and focus more effectively on capturing client feedback to evoke change talk from clients with various inherent motivations. To the best of our knowledge, this is the first attempt to build a topic exploration method for a counselor agent. Third, we design a ranking-based approach for MI strategies guided response generation. Unlike previous approaches, which generate responses based on all selected strategies, we use a ranking approach for generated candidate responses for each strategy, effectively reducing the overuse of preferred strategies and mitigating bias in strategy selection. Finally, we develop a comprehensive automated evaluation framework using simulated clients that are carefully designed to exhibit realistic client behaviors.  The above ideas have been incorporated into our proposed framework known as the STAR (State, Topic, Action, and Response) framework as shown in Figure~\ref{fig:framework}. 
%In STAR, the evaluation of counseling agents covers multiple aspects of psychotherapy, such as MI skill competence and client experience. 
%We also incorporate professional MI experts to assess the performance of our counselor agent and framework. The experimental results demonstrate that our counselor agent effectively aligns with psychotherapy skills and adheres to MI principles, highlighting its potential for use in real-world MI applications.

\begin{figure*}
    \centering
    \includegraphics[width=\textwidth]{figs/framework.pdf}
    \caption{The STAR Framework illustrates how an LLM-based counselor agent and a client simulator can be created in an MI-based counseling session. 
    % The client simulator can be substituted with a human client if necessary. 
    The counselor agent infers the client's state, explores topics that motivate change talk, and utilizes MI techniques to generate appropriate responses. Check marks represent the selected actions, strategies, or states. The client agent may be simulated by adapting the STAR framework, incorporating modules for state transition, dynamic engagement, action selection, and response generation, allowing the client agent to closely align with the client's profile and adapt its engagement level with the counselor accordingly. The ``Precon.'' and ``Cont.'' indicate Precontemplation and Contemplation, respectively.}
    % \caption{Proposed STAR Framework. It shows how a LLM-based counselor agent and client agent can be simulated in a MI-based counseling session. The client agent can be replaced by a human client if required. The counselor agent infers the client's state, explores topics that trigger the motivation to change, and employs MI techniques to generate responses. We use check mark button to represented the selected action, strategy or state. For client simulation, STAR includes state transition, dynamic engagement, action selection, and response generation modules making the client agent follow the client's profile closely and adjust its engagement level. }
    \label{fig:framework}
\end{figure*}

{\bf Contributions.}  Our work contributes to the field of counselor agent research as follows: (1) Based on the STAR framework, we develop CAMI, a counselor agent that integrates state inference, topic exploration, and strategy selection modules, all aligned with established psychotherapy skills. (2) Through ablation study, we show that the ability of CAMI to infer client's state significantly improves its performance in MI-based counseling. (3) Our experiment also shows that topic tree and topic exploration further enhances CAMI's effectiveness in evoking change talk. 
%To the best of our knowledge, this represents the first attempt to create a counselor agent that simulates the MI counseling process as a professional human counselor would. 
(4) Our evaluation, combining both automated methods and human expert assessments, demonstrates that CAMI achieves superior performance, outperforming several strong baseline models. 
%\jjcomment{Should this paragraph be better aligned with the paragraph on ``Objective''? In the Objective paragraph, there were three novel ideas presented. It seems the three novel ideas are re-organized into item (1) and item (2) above. Shall we present them in consistent ways?}