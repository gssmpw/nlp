\section{Detailed Implementation}
\label{app:implementation}

In this section, we provide the detailed implementation of our method and experiment evaluation, which includes the prompts and instructions for human evaluation. We set the top-p and temperature of to 0.7 and 0.8 for LLMs.

\subsection{Counselor Agent Implementation}
\label{app:counselor implementation}

The prompts for counselor simulation are shown in Table~\ref{tab:state estimation} to Table~\ref{tab:response rank}. Table~\ref{tab:topic expansion} shows the prompt used to expand the topics and Figure~\ref{fig:topic tree} shows all the collected topics. Table~\ref{tab:counselor state description} to Table~\ref{tab:counselor strategy} demonstrate the descriptions used in counselor agent.

\begin{table*}[tb]
\begin{tabularx}{\textwidth}{X}
\toprule
{\sf \footnotesize During the Motivational Interviewing counseling conversation, the client may exhibit different states that refer to their readiness to change. The client's state can be one of the following: \newline \newline - Precontemplation: The client does not recognize their behavior as problematic and is not considering change. \newline - Contemplation: The client acknowledges the problematic nature of their behavior but is ambivalent about change. \newline - Preparation: The client is ready to take action and is considering steps towards change. \newline \newline Given the current counseling context, analyze the context step by step and then infer the current state of the client. If the context does not clearly indicate the state, it is assumed to be in the Precontemplation state. \newline Your response should be ended with "Therefore, the client's current state in the above context is ..." \newline \newline Given Current Context: \newline [context]  \newline \newline Analyze then Predict State:}
\\ \bottomrule
\end{tabularx}
\caption{Prompt for the counselor agent to infer the state of client. The [context] will be replaced by the conversation so far.}
\label{tab:state estimation}
\end{table*}

\begin{table*}[tb]
\begin{tabularx}{\textwidth}{X}
\toprule
{\sf \footnotesize You are a counselor working with a client to achieve the goal of {self.goal}, specifically addressing the client's behavior, [behavior]. After establishing a foundation of trust, your focus should now shift to identifying specific topics that may motivate the client to change their behavior, [behavior]. \newline \newline In the previous counseling session, the following context was discussed: \newline [context] \newline Your task is to assign the most likely topic(s) from the list below that will engage the client and help them recognize either the benefits of achieving [goal] or the potential risks of continuing with [behavior]. Here are the possible topics you can refer to: \newline [topics] \newline Please respond in JSON format, assigning a probability to each topic in the most top level based on the client's response. For example: {"Health": 0.3, "Economy": 0.2, "Interpersonal Relationships": 0.2, "Law": 0.1, "Education": 0.2} }
\\ \bottomrule
\end{tabularx}
\caption{Prompt for the counselor agent to estimate the probabilities of different topics based on the current context at the beginning of sessions. The module will assign probabilities for each super-class topics based on previous context. The [goal] will be replaced by the counseling goal, such as smoking cessation, reducing alcohol consumption, and the [behavior] will be replaced with the client's problematic behavior. The [context] will be replaced by the conversation so far, while [topics] will be replaced by the super-class topics for counselor agent to choose from.}
\label{tab:topic initialization}
\end{table*}

\begin{table*}[tb]
\begin{tabularx}{\textwidth}{X}
\toprule
{\sf \footnotesize You are acting as a counselor agent, interacting with the client to help them achieve the goal of [goal] related to their behavior, [behavior]. Each client has their own unique motivations, and the counselor help the client discover their inherent motivation for change. The counselor need to explore various topics and uncover what concerns the client the most.  Your task is to analyse the client's feedback toward current topic and then choose the next exploration action (step into, switch, and step out) within the topic tree, but without generating a counselor response. \newline \newline You have already explored the following topics to understand the client’s motivation: \newline [explore\_path] \newline Here is the corresponding counseling conversation: \newline [context] \newline The current topic is [topic]. \newline \newline You have three options: \newline - Step Into: If the client shows interest in this topic, you should dive deeper into its subtopics, including [step\_into\_topics]. \newline - Switch: If the client is interested in the broader category but not this specific topic, switch to another related topic under the same super topic, including[switch\_topics]. \newline - Step Out: If the client’s interest lies in a broader area, step out to a higher-level topic to explore that further, including [step\_out\_topics]. \newline \newline Please analyse the client's feedback toward current situation and then  choose the next course of action based on the feedback without generating counselor's specific response. }
\\ \bottomrule
\end{tabularx}
\caption{Prompt for the counselor agent to explore topics based on the current context. The module will select one action from {\em step into}, {\em switch}, or {\em step out}, then use the corresponding prompt to select the topic. The [goal] will be replaced by the counseling goal, such as smoking cessation, reducing alcohol consumption, and the [behavior] will be replaced with the client's problematic behavior. The [context] will be replaced by the conversation so far, while [explore\_path] and [topic] will be replaced with the explored topics and the current topic. For each action, the placeholders [step\_into\_topics], [switch\_topics], and [step\_out\_topics] will be replaced with the corresponding candidate topics.}
\label{tab:topic exploration}
\end{table*}

\begin{table*}[tb]
\begin{tabularx}{\textwidth}{X}
\toprule
{\sf \footnotesize Your task is to explore the subtopics of the current topic to understand the client's motivation better. \newline \newline You have already explored the following topics to understand the client’s motivation: \newline [explore\_path] \newline Here is the corresponding counseling conversation: \newline [context] \newline The current topic is [topic], and you sense that the client is interested in exploring this further. Please choose one of the following subtopics to dive deeper into the client's motivations: \newline [step\_into\_topics] \newline \newline Analyse the current situation especially for the client's response about the current topic and then select the appropriate next step to motivate the client after analysing. Just analyse for next topic instead of generate response. }
\\ \bottomrule
\end{tabularx}
\caption{Prompt for the counselor agent to step into deeper topics based on the current context. The [context] will be replaced by the conversation so far, while [explore\_path] and [topic] will be replaced by the explored topics and current topic respectively. For each action, the placeholders [step\_into\_topics], [switch\_topics], and [step\_out\_topics] will be replaced by the corresponding candidate topics.}
\label{tab:topic step into}
\end{table*}

\begin{table*}[tb]
\begin{tabularx}{\textwidth}{X}
\toprule
{\sf \footnotesize Your task is to explore a different topic within the same broader category to understand the client's motivation better. \newline \newline You have already explored the following topics to understand the client’s motivation: \newline [explore\_path] \newline Here is the corresponding counseling conversation: \newline [context] \newline and the current client's state is \newline [state]\newline \newline The current topic is [topic], but you sense that the client may be more interested in other topics within the same broader category. Please choose one of the following related topics to continue exploring the client’s motivations: \newline [switch\_topics] \newline \newline Analyse the current situation especially for the client's response about the current topic and then select the appropriate next step to motivate the client after analysing. Just analyse for next topic instead of generate response.} 
\\ \bottomrule
\end{tabularx}
\caption{Prompt for the counselor agent to transit to other topics based on the current context. The [context] will be replaced by the conversation so far and the [state] will be replaced by the inferred client's state. The [explore\_path] and [topic] will be replaced by the explored topics and the current client topic respectively.  For each action, the placeholders [step\_into\_topics], [switch\_topics], and [step\_out\_topics] will be replaced with the corresponding candidate topics.}
\label{tab:topic switch}
\end{table*}

\begin{table*}[tb]
\begin{tabularx}{\textwidth}{X}
\toprule
{\sf \footnotesize Your task is to explore a broader topic to understand the client's motivation better. \newline \newline You have already explored the following topics to understand the client’s motivation: \newline [explore\_path] \newline Here is the corresponding counseling conversation: \newline [context] \newline \newline The current topic is [topic], but you sense that the client may be more interested in another topic instead. Please choose one of the following related topics to continue exploring the client’s motivations: \newline [step\_out\_topics] \newline \newline Analyse the current situation especially for the client's response about the current topic and then select the appropriate next step to motivate the client after analysing. Just analyse for next topic instead of generate response.}
\\ \bottomrule
\end{tabularx}
\caption{Prompt for the counselor agent to step out the current topic and select topic at the parent level based on the current context. The [context] will be replaced by the conversation so far, while [explore\_path] and [topic] will be replaced by the explored topics and current topic respectively. For each action, the placeholders [step\_into\_topics], [switch\_topics], and [step\_out\_topics] will be replaced with the corresponding candidate topics.}
\label{tab:topic step out}
\end{table*}

\begin{table*}[tb]
\begin{tabularx}{\textwidth}{X}
\toprule
{\sf \footnotesize During motivational interviewing, the counselor should employ some counseling strategies tailored to the client's readiness to change, to effectively facilitate behavioral transformation. These counseling strategies are as follows: \newline \newline [strategies] \newline  Based on the current counseling context and the client's state, analyze and select appropriate strategies but no more than 2 for next response to optimally advance the counseling process.  \newline \newline Given Current Context: \newline [context] \newline The current client's state is [state] and you would like to navigate the session toward: [topic] \newline \newline Please analyse the current situation, then select appropriate strategies based on current topic and situation to motivate client after analysing. Remember, you can select up to 2 strategies.}
\\ \bottomrule
\end{tabularx}
\caption{Prompt for the counselor agent to select strategies based on the current context. The [context] will be replaced by the conversation so far while the [state] will be replaced by the inferred state. The [topic] will be replaced by the selected next topic. The [strategies] will be replaced by the list of strategies and their corresponding descriptions.}
\label{tab:strategy selection}
\end{table*}


\begin{table*}[tb]
\begin{tabularx}{\textwidth}{lX}
\toprule
 System Prompt & {\sf \footnotesize You will act as a skilled counselor conducting a Motivational Interviewing (MI) session aimed at achieving [goal] related to the client's behavior, [behavior]. Your task is to help the client discover their inherent motivation to change and identify a tangible plan to change. Start the conversation with the client with some initial rapport building, such as asking, How are you? (e.g., develop mutual trust, friendship, and affinity with the client) before smoothly transitioning to asking about their problematic behavior. Keep the session under 40 turns and each response under 150 characters long. Use the MI principles and techniques described in the Knowledge Base – Motivational Interviewing (MI) context section below. However, these MI principles and techniques are only for you to use to help the user. These principles and techniques, as well as motivational interviewing, should NEVER be mentioned to the user. In each turn, a specific topic will be provided in square brackets after the client's utterance. Guide the conversation toward that topic, ensuring the session explores relevant aspects of the client’s situation and motivation. This will help you tailor your approach to the specific context and steer the counseling toward meaningful insights and actions. \newline \newline Knowledge Base – Motivational Interviewing (MI) \newline Motivational Interviewing (MI) is a counseling approach designed to help individuals find the motivation to make positive behavioral changes. It is widely used in various fields such as health care, addiction treatment, and mental health. Here are the key principles and techniques of Motivational Interviewing: \newline MI Principles \newline - Express Empathy: The foundation of MI is to create a safe and non-judgmental environment where clients feel understood and respected. This involves actively listening and reflecting on what the client is saying, acknowledging their feelings, and showing genuine concern and understanding. \newline ...  \newline MI Techniques \newline At the core of MI are a few basic principles, including expressing empathy and developing discrepancy. Several specific techniques can help individuals make positive life changes from these core principles. Here are some MI techniques that can be used in counseling sessions: \newline - Advise with permission. The counselor gives advice, makes a suggestion, offers a solution or possible action given with prior permission from the client.  \newline - Affirm. The counselor says something positive or complimentary to the client. \newline ...} \\ \midrule
... & ...  \\ \midrule
Assistant & {\sf \footnotesize Counselor: ... }\\ \hline
User & {\sf \footnotesize Client: ... [instruction]}
\\ \bottomrule
\end{tabularx}
\caption{System prompt and generation prompt for the counselor agent. The [behavior] will be replaced by the problematic behavior of client and the [goal] will be replaced by the counseling goal, such as smoking cessation, reducing alcohol consumption. The [instruction] is the turn-level instruction prompt, including the inferred state description, topic description and the given strategy description. In particular, at the beginning of the session, before the counselor has identified a high-probability super-class topic, the instruction will include all super-class topics.}
\label{tab:couneslor system}
\end{table*}


\begin{table*}[tb]
\begin{tabularx}{\textwidth}{X}
\toprule
{\sf \footnotesize You will act as an expert counselor conducting a Motivational Interviewing (MI) session aimed at achieving [goal] related to the client's behavior, [behavior]. Your task is to help the client discover their intrinsic motivation to change and identify a tangible plan for achieving that change. The current state of the counseling session is as follows: \newline [conversation] \newline At this point, multiple responses have been given by peer counselors. Your task is to select the response that best aligns with the current context and adheres to MI principles. Here are the generated responses: \newline [responses] \newline \newline Below are some MI principles for reference: \newline [principles] \newline Please choose the most appropriate response based on the counseling context and the MI principles. Reply with the ID of the response you find most suitable for the current situation.}
\\ \bottomrule
\end{tabularx}
\caption{Prompt for the counselor agent to select the most appropriate response from the candidate responses. The [behavior] placeholder will be replaced with the client's problematic behavior, and the [goal] placeholder will be replaced with the goal of the counseling session. The [responses] placeholder will be replaced with the candidate responses, while the [principles] placeholder will be replaced with the MI principles from psychological literature.}
\label{tab:response rank}
\end{table*}

\begin{figure*}
    \centering
    \includegraphics[width=\textwidth]{figs/topics.png}
    \caption{The topics tree constructed in our work consists of 5 Super-Class topics (i.e., {\em Health}, {\em Economy}, {\em Relationship}, {\em Law}, and {\em Education}), 14 Coarse-Grained Topics, and 59 Fine-Grained Topics.}
    \label{fig:topic tree}
\end{figure*}


\begin{table*}[tb]
\begin{tabularx}{\textwidth}{X}
\toprule
{\sf \footnotesize
Your task is to generate additional sub-topics under the category: [category]. Below are some existing sub-topics within this category: \newline [topics] \newline \newline Please suggest new sub-topics under the same category, ensuring they are distinct from the ones provided. Please format your response in the same style.}
\\ \bottomrule
\end{tabularx}
\caption{Prompt for the topic tree expansion. The [category] will be replaced by the parent topic while [topics] will be replaced by the known child topics.}
\label{tab:topic expansion}
\end{table*}

\begin{table*}[tb]
\centering
\begin{tabularx}{\textwidth}{lX}
\toprule
{\bf State}            & {\bf Description }                                           \\ \midrule
Precontemplation & The client is unaware of or underestimates the need for change.       \\ \hline
Contemplation    & The client acknowledges the need for change but remains ambivalent.   \\ \hline
Preparation      & he client is ready to act, planning specific steps toward change.
\\ \bottomrule
\end{tabularx}
\caption{The description of states used in counselor agent.}
\label{tab:counselor state description}
\end{table*}

\begin{table*}[tb]
\centering
\begin{tabularx}{\textwidth}{lX}
\toprule
Topic            & Description                                            \\ \midrule
Infection                   & You can explore how [problematic behavior] increases the risk of infections by weakening the immune system, leading to more frequent or severe infections. You can also discuss how [goal] enhances immune function, reduces infection risks, and improves overall health.                                                                                                                                                         \\ \hline
Hypertension                & You can explore how [problematic behavior] contributes to the development of high blood pressure and increases the risk of complications such as heart disease and stroke. You can also highlight how [goal] helps lower blood pressure, improves heart health, and reduces the risk of cardiovascular conditions.                                                                                                                 \\ \hline
Flu                         & You can explore how [problematic behavior] increases the risk of contracting the flu or experiencing more severe symptoms. You can also explain how [goal] improves immune response, reduces the likelihood of illness, and mitigates the impact of seasonal flu.                                                                                                                                                                  \\ \hline
Inflammation                & You can explore how [problematic behavior] leads to chronic inflammation, increasing the risk of diseases like arthritis, heart disease, or diabetes. You can also highlight how [goal] helps reduce inflammation and supports long-term health.                                                                                                                                                                                   \\ \hline
Liver Disease               & You can explore how [problematic behavior] contributes to liver damage, raising the risk of conditions such as fatty liver disease, cirrhosis, or liver cancer. You can also discuss how [goal] promotes liver health, prevents damage, and reduces the likelihood of chronic liver conditions.                                                                                                                                    \\ \hline
Lung Cancer                 & You can explore how [problematic behavior] increases the risk of lung cancer and other respiratory diseases. You can also emphasize how [goal] lowers the risk of cancer, improves lung function, and enhances overall respiratory health.                                                                                                                                                                                         \\ \hline
COPD                        & You can explore how [problematic behavior] may contribute to the development or worsening of COPD, leading to breathing difficulties and other respiratory issues. You can also discuss how [goal] improves lung function and overall respiratory health.                                                                                                                                                                          \\ \hline
Asthma                      & You can explore how [problematic behavior] triggers or worsens asthma symptoms, increasing the risk of severe attacks. You can also highlight how [goal] helps manage asthma, reduces symptoms, and improves the client's quality of life.                                                                                                                                                                                         \\ \hline
Stroke                      & You can explore how [problematic behavior] increases the risk of stroke, particularly through poor cardiovascular health. You can also discuss how [goal] improves circulation, reduces stroke risk, and supports brain and heart health.                                                                                                                                                                                          \\ \hline
Diabetes                    & You can explore how [problematic behavior] contributes to the development or worsening of diabetes by affecting blood sugar levels. You can also discuss how [goal] helps manage blood sugar, prevent complications, and enhance overall well-being. 
\\ \bottomrule
\end{tabularx}
\caption{The descriptions of topics used in counselor agent (part 1). The [problematic behavior] will be replaced as the client's problematic behavior while the [goal] will be replaced by the counseling goal, such as smoking cessation, reducing alcohol consumption.}
\label{tab:counselor topic description 1}
\end{table*}


\begin{table*}[tb]
\centering
\begin{tabularx}{\textwidth}{lX}
\toprule
Topic            & Description                                            \\ \midrule
Physical Activity           & You can explore how [problematic behavior] reduces physical activity, increasing the risk of obesity, cardiovascular disease, and musculoskeletal issues. You can also emphasize how [goal] increases physical activity and improves overall fitness and health.                                                                                                                                                                   \\ \hline
Sport                       & You can explore how [problematic behavior] reduces performance in sports, limiting physical conditioning and skill development. You can also highlight how [goal] enhances sports participation, improves physical conditioning, and boosts confidence.                                                                                                                                                                            \\ \hline
Physical Fitness            & You can explore how [problematic behavior] negatively affects physical fitness, leading to a decline in overall health. You can also discuss how [goal] promotes better fitness, improves health, and increases energy levels.                                                                                                                                                                                                     \\ \hline
Strength                    & You can explore how [problematic behavior] weakens physical strength, leading to reduced mobility and increased injury risk. You can also discuss how [goal] improves muscle strength, supports healthy aging, and enhances physical performance.                                                                                                                                                                                  \\ \hline
Flexibility                 & You can explore how [problematic behavior] reduces flexibility, increasing stiffness and injury risk. You can also highlight how [goal] improves flexibility, reduces pain, and promotes better movement and posture.                                                                                                                                                                                                              \\ \hline
Endurance                   & You can explore how [problematic behavior] reduces endurance, making it difficult to engage in prolonged physical activities. You can also discuss how [goal] builds endurance, improves stamina, and enhances overall physical performance                                                                                                                                                                                        \\ \hline
Dentistry                   & You can explore how [problematic behavior] leads to poor oral hygiene, increasing the risk of cavities, gum disease, or tooth loss. You can also discuss how [goal] improves oral hygiene, prevents dental problems, and supports overall oral health.                                                                                                                                                                             \\ \hline
Caregiver Burden            & You can explore how [problematic behavior] increases the stress or demands placed on caregivers, leading to burnout and reduced care quality. You can also highlight how [goal] reduces caregiver burden, improves care quality, and supports a healthier caregiving dynamic.                                                                                                                                                      \\ \hline
Independent Living          & You can explore how [problematic behavior] limits a person’s ability to live independently, leading to greater reliance on others for daily needs. You can also emphasize how [goal] promotes independence, improves self-sufficiency, and enhances overall quality of life.                                                                                                                                                       \\ \hline
Human Appearance            & You can explore how [problematic behavior] affects a person’s physical appearance, leading to issues such as skin problems, weight gain, or premature aging. You can also discuss how [goal] improves appearance, boosts self-esteem, and supports overall well-being.                                                                                                                                                             \\ \hline
Depression                  & You can explore how [problematic behavior] worsens symptoms of depression, affecting mood, energy levels, and daily functioning. You can also explain how [goal] improves mental health, enhances mood, and fosters emotional resilience. 
\\ \bottomrule
\end{tabularx}
\caption{The descriptions of topics used in counselor agent (part 2). The [problematic behavior] will be replaced as the client's problematic behavior while the [goal] will be replaced by the counseling goal, such as smoking cessation, reducing alcohol consumption.}
\label{tab:counselor topic description 2}
\end{table*}

\begin{table*}[tb]
\centering
\begin{tabularx}{\textwidth}{lX}
\toprule
Topic            & Description                                            \\ \midrule
Chronodisruption            & You can explore how [problematic behavior] disrupts natural body rhythms, leading to sleep disorders, fatigue, and increased stress. You can also highlight how [goal] restores healthy sleep patterns and improves overall mental and physical health.                                                                                                                                                                            \\ \hline
Anxiety Disorders           & You can explore how [problematic behavior] increases anxiety, leading to chronic stress, panic attacks, or other anxiety-related issues. You can also discuss how [goal] helps manage anxiety, promotes relaxation, and supports emotional well-being.                                                                                                                                                                             \\ \hline
Cognitive Decline           & You can explore how [problematic behavior] accelerates cognitive decline, increasing the risk of dementia and other neurological conditions. You can also highlight how [goal] protects brain health, improves memory, and enhances cognitive function.                                                                                                                                                                            \\ \hline
Safe Sex                    & You can explore how [problematic behavior] increases the risk of sexually transmitted infections (STIs) or unintended pregnancies. You can also explain how [goal] promotes safer sexual practices, reduces health risks, and fosters healthier relationships.                                                                                                                                                                     \\ \hline
Maternal Health             & You can explore how [problematic behavior] impacts maternal health, leading to complications during pregnancy or childbirth. You can also discuss how [goal] supports a healthy pregnancy and reduces the risk of complications.                                                                                                                                                                                                   \\ \hline
Preterm Birth               & You can explore how [problematic behavior] increases the risk of preterm birth, leading to health complications for both mother and baby. You can also highlight how [goal] promotes a healthy pregnancy and reduces the risk of early delivery.                                                                                                                                                                                   \\ \hline
Miscarriage                 & You can explore how [problematic behavior] increases the risk of miscarriage, leading to emotional distress and health complications. You can also emphasize how [goal] supports a healthy pregnancy and reduces the risk of miscarriage.                                                                                                                                                                                          \\ \hline
Birth Defects               & You can explore how [problematic behavior] increases the risk of birth defects. You can also highlight how [goal] supports a healthy pregnancy and reduces the risk of complications.                                                                                                                                                                                                                                              \\ \hline
Productivity                & You can explore how [problematic behavior] negatively affects workplace productivity, leading to decreased performance and career setbacks. You can also explain how [goal] enhances productivity, focus, and career success.                                                                                                                                                                                                      \\ \hline
Absenteeism                 & You can explore how [problematic behavior] strains workplace relationships, leading to conflicts or a negative work environment. You can also discuss how [goal] improves communication, strengthens teamwork, and promotes a positive workplace dynamic. 
\\ \bottomrule
\end{tabularx}
\caption{The descriptions of topics used in counselor agent (part 3). The [problematic behavior] will be replaced as the client's problematic behavior while the [goal] will be replaced by the counseling goal, such as smoking cessation, reducing alcohol consumption.}
\label{tab:counselor topic description 3}
\end{table*}                                                                                                                                                    


\begin{table*}[tb]
\centering
\begin{tabularx}{\textwidth}{lX}
\toprule
Topic            & Description                                            \\ \midrule
Workplace Relationships     & Explore how [problematic behavior] may strain workplace relationships, leading to conflicts or a negative work environment. Discuss how achieving [goal] can improve communication, strengthen teamwork, and create a positive workplace dynamic.                                                                                                                                                                                  \\ \hline
Career Break                & You can explore how [problematic behavior] leads to career breaks or job loss, affecting professional growth. You can also highlight how [goal] promotes career continuity and reduces the need for extended leave.                                                                                                                                                                                                                \\ \hline
Career Assessment           & You can explore how [problematic behavior] interferes with career assessments or evaluations, leading to potential setbacks. You can also discuss how [goal] improves career performance and fosters positive evaluations.                                                                                                                                                                                                         \\ \hline
Absence Rate                & You can explore how [problematic behavior] increases the absence rate at work, impacting job security and career progression. You can also highlight how [goal] reduces absences and supports professional growth.                                                                                                                                                                                                                 \\ \hline
Salary                      & You can explore how [problematic behavior] affects salary progression, leading to lower earnings. You can also emphasize how [goal] enhances earning potential and supports financial stability.                                                                                                                                                                                                                                   \\ \hline
Workplace Wellness          & You can explore how [problematic behavior] undermines workplace wellness initiatives, leading to reduced employee health and satisfaction. You can also highlight how [goal] improves workplace wellness and enhances overall job satisfaction.                                                                                                                                                                                    \\ \hline
Workplace Incivility        & You can explore how [problematic behavior] contributes to incivility in the workplace, creating a toxic work environment. You can also discuss how [goal] fosters respect, cooperation, and a positive workplace culture.                                                                                                                                                                                                          \\ \hline
Cost of Living              & You can explore how [problematic behavior] leads to poor financial management, making it harder to meet the cost of living. You can also discuss how [goal] improves financial stability and reduces financial stress.                                                                                                                                                                                                             \\ \hline
Personal Budget             & You can explore how [problematic behavior] makes it difficult to stick to a personal budget, leading to debt and financial challenges. You can also highlight how [goal] improves financial planning and promotes savings.                                                                                                                                                                                                         \\ \hline
Debt                        & You can explore how [problematic behavior] increases debt, impacting credit and financial security. You can also emphasize how [goal] reduces debt and supports financial freedom.                                                                                                                                                                                                                                                 \\ \hline
Income Deficit              & You can explore how [problematic behavior] contributes to income deficits, leading to financial instability. You can also highlight how [goal] improves financial management and reduces income shortfalls. 
\\ \bottomrule
\end{tabularx}
\caption{The descriptions of topics used in counselor agent (part 4). The [problematic behavior] will be replaced as the client's problematic behavior while the [goal] will be replaced by the counseling goal, such as smoking cessation, reducing alcohol consumption.}
\label{tab:counselor topic description 4}
\end{table*}             

\begin{table*}[tb]
\centering
\begin{tabularx}{\textwidth}{lX}
\toprule
Topic            & Description                                            \\ \midrule
Family Estrangement         & You can explore how [problematic behavior] leads to family estrangement, creating emotional distance or separation. You can also explain how [goal] improves family relationships and fosters reconciliation.                                                                                                                                                                                                                      \\ \hline
Family Disruption           & You can explore how [problematic behavior] disrupts family dynamics, leading to conflict and instability. You can also highlight how [goal] strengthens family bonds and promotes harmony.                                                                                                                                                                                                                                         \\ \hline
Divorce                     & You can explore how [problematic behavior] contributes to marital conflict, increasing the risk of divorce. You can also discuss how [goal] improves communication, reduces conflict, and supports a healthy marriage.                                                                                                                                                                                                             \\ \hline
Role Model                  & You can explore how [problematic behavior] negatively influences a parent’s ability to be a positive role model for their children. You can also highlight how [goal] fosters positive behaviors and sets a good example for children.                                                                                                                                                                                             \\ \hline
Child Development           & You can explore how [problematic behavior] affects a child’s development, impacting emotional, social, or cognitive growth. You can also discuss how [goal] supports healthy child development and overall well-being.                                                                                                                                                                                                             \\ \hline
Paternal Bond               & You can explore how [problematic behavior] weakens the paternal bond, leading to strained relationships with children. You can also emphasize how [goal] strengthens the parent-child bond and fosters emotional connection.                                                                                                                                                                                                       \\ \hline
Child Care                  & You can explore how [problematic behavior] interferes with child care, leading to neglect or inconsistency in parenting. You can also highlight how [goal] supports stable, nurturing care and promotes positive outcomes for children.                                                                                                                                                                                            \\ \hline
Habituation                 & You can explore how [problematic behavior] affects a child’s habituation, impacting learning and adaptation. You can also highlight how [goal] promotes healthy habits and learning in children.                                                                                                                                                                                                                                   \\ \hline
Arrest                      & You can explore how [problematic behavior] increases the risk of arrest, leading to legal trouble and a criminal record. You can also highlight how [goal] avoids legal issues and promotes a law-abiding lifestyle.                                                                                                                                                                                                               \\ \hline
Imprisonment                & You can explore how [problematic behavior] increases the risk of imprisonment, with long-term social and legal consequences. You can also explain how [goal] helps avoid incarceration and supports lawful behavior.                                                                                                                                                                                                               \\ \hline
Child Custody               & You can explore how [problematic behavior] impacts a parent’s ability to maintain child custody, leading to legal challenges. You can also highlight how [goal] improves parenting and strengthens legal standing in custody cases.
\\ \bottomrule
\end{tabularx}
\caption{The descriptions of topics used in counselor agent (part 5). The [problematic behavior] will be replaced as the client's problematic behavior while the [goal] will be replaced by the counseling goal, such as smoking cessation, reducing alcohol consumption.}
\label{tab:counselor topic description 5}
\end{table*}   

\begin{table*}[tb]
\centering
\begin{tabularx}{\textwidth}{lX}
\toprule
Topic            & Description                                            \\ \midrule
Traffic Ticket              & You can explore how [problematic behavior] increases the risk of traffic tickets and other legal penalties. You can also discuss how [goal] promotes responsible driving and helps avoid legal infractions.                                                                                                                                                                                                                        \\ \hline
Complaint                   & You can explore how [problematic behavior] increases the likelihood of legal complaints or disputes. You can also discuss how [goal] reduces legal risks and promotes harmonious interactions.                                                                                                                                                                                                                                     \\ \hline
Attendance                  & You can explore how [problematic behavior] affects a student’s attendance, leading to academic challenges and disciplinary actions. You can also highlight how [goal] improves attendance and academic success.                                                                                                                                                                                                                    \\ \hline
Suspension                  & You can explore how [problematic behavior] increases the risk of school suspension, impacting academic progress and relationships. You can also discuss how [goal] reduces suspension risks and supports positive school experiences.                                                                                                                                                                                              \\ \hline
Exam                        & You can explore how [problematic behavior] negatively impacts exam preparation and performance, leading to lower grades. You can also highlight how [goal] improves study habits and exam results.                                                                                                                                                                                                                                 \\ \hline
Scholarship                 & You can explore how [problematic behavior] affects eligibility for scholarships, reducing academic opportunities. You can also emphasize how [goal] improves academic performance and increases scholarship chances.                                                                                                                                                                                                               \\ \hline
Diseases                    & You can explore how [problematic behavior] increases the risk of various diseases, including infections, chronic conditions, and respiratory issues. You can also discuss how [goal] can reduce these risks and support better long-term health. This includes subtopics like infections, hypertension, flu, inflammation, liver disease, lung cancer, chronic obstructive pulmonary disease (COPD), asthma, stroke, and diabetes. \\ \hline
Physical Fitness            & You can explore the negative effects of [problematic behavior] on physical fitness, such as decreased physical activity, loss of strength, and reduced flexibility. You can also discuss how [goal] contributes to better fitness levels, including improvements in endurance, strength, and flexibility.                                                                                                                          \\ \hline
Health Care                 & You can explore how [problematic behavior] affects personal healthcare, such as oral hygiene, independent living, and overall appearance. You can also discuss the positive impact of [goal] in maintaining better health care practices and improving quality of life. Subtopics include dentistry, caregiver burden, independent living, and human appearance.                                                                   \\ \hline
Mental Disorder             & You can explore how [problematic behavior] may contribute to or worsen mental health conditions such as depression, anxiety, and cognitive decline. You can also discuss the benefits of [goal] in managing mental health and improving emotional well-being. Subtopics include depression, chronodisruption, anxiety disorders, and cognitive decline.                                                                            \\ \hline
Sexual Health               & You can explore how [problematic behavior] increases risks related to sexual and reproductive health, such as unsafe sex practices, maternal health complications, and birth defects. You can also discuss how [goal] supports healthier sexual practices and reduces the risk of complications. Subtopics include safe sex, maternal health, preterm birth, miscarriage, and birth defects.                                       
\\ \bottomrule
\end{tabularx}
\caption{The descriptions of topics used in counselor agent (part 6). The [problematic behavior] will be replaced as the client's problematic behavior while the [goal] will be replaced by the counseling goal, such as smoking cessation, reducing alcohol consumption.}
\label{tab:counselor topic description 6}
\end{table*}   

\begin{table*}[tb]
\centering
\begin{tabularx}{\textwidth}{lX}
\toprule
Topic            & Description                                            \\ \midrule
Employment                  & You can explore how [problematic behavior] negatively impacts workplace productivity, absenteeism, and career progress. You can also discuss how [goal] enhances professional success and fosters healthier workplace relationships. Subtopics include productivity, absenteeism, workplace relationships, career break, career assessment, absence rate, salary, workplace wellness, and workplace incivility.                    \\ \hline
Personal Finance            & You can explore how [problematic behavior] leads to financial instability, such as increased debt, poor budgeting, or income deficits. You can also discuss how [goal] helps improve financial management and promotes long-term financial security. Subtopics include cost of living, personal budget, debt, and income deficit.                                                                                                  \\ \hline
Family                      & You can explore how [problematic behavior] leads to family issues, such as estrangement, disruption, or divorce. You can also discuss how [goal] promotes healthier family relationships and reconciliation. Subtopics include family estrangement, family disruption, and divorce.                                                                                                                                                \\ \hline
Parenting                   & You can explore how [problematic behavior] impacts the client's ability to effectively parent, such as being a poor role model or affecting their child’s development. You can also discuss how [goal] enhances positive parenting, strengthens the parent-child bond, and supports healthier child development. Subtopics include role model, child development, paternal bond, child care, and habituation.                      \\ \hline
Criminal Law                & You can explore how [problematic behavior] leads to issues like arrests, imprisonment, or legal complaints. You can also discuss how [goal] helps avoid these legal problems and supports a law-abiding lifestyle. Subtopics include arrest, imprisonment, and complaint.                                                                                                                                                          \\ \hline
Family Law                  & You can explore how [problematic behavior] affects legal matters involving family, such as child custody disputes. You can also discuss how [goal] improves the client’s legal standing and promotes healthier family relationships.                                                                                                                                                                                               \\ \hline
Traffic Law                 & You can explore how [problematic behavior] may lead to traffic violations, such as receiving tickets or facing fines. You can also discuss how [goal] encourages responsible driving and helps avoid legal infractions.                                                                                                                                                                                                            \\ \hline
Student Affairs             & You can explore how [problematic behavior] impacts school attendance, potentially leading to disciplinary actions such as suspension. You can also discuss how [goal] promotes better academic engagement and achievement. Subtopics include attendance, suspension, and scholarship.                                                                                                                                              \\ \hline
Assessment                  & You can explore how [problematic behavior] negatively affects academic performance during assessments, such as exams. You can also discuss how [goal] helps improve focus, study habits, and exam results.                                                                                                                                                                                                                         \\ \hline
Health                      & You can explore how [problematic behavior] impacts your client's physical and mental well-being, leading to potential health issues. You can also discuss the benefits of [goal], which can improve overall quality of life and promote better health outcomes.                                                                                                                                                                    \\ \hline
Economy                     & You can explore how [problematic behavior] affects your client's financial situation, such as through reduced productivity, increased absenteeism, or poor financial management. You can also discuss how [goal] helps improve economic stability and workplace performance.                                                                                                                                                      
\\ \bottomrule
\end{tabularx}
\caption{The descriptions of topics used in counselor agent (part 7). The [problematic behavior] will be replaced as the client's problematic behavior while the [goal] will be replaced by the counseling goal, such as smoking cessation, reducing alcohol consumption.}
\label{tab:counselor topic description 7}
\end{table*}   

\begin{table*}[tb]
\centering
\begin{tabularx}{\textwidth}{lX}
\toprule
Topic            & Description                                            \\ \midrule
Traffic Law                 & You can explore how [problematic behavior] may lead to traffic violations, such as receiving tickets or facing fines. You can also discuss how [goal] encourages responsible driving and helps avoid legal infractions.                                                                                                                                                                                                            \\ \hline
Student Affairs             & You can explore how [problematic behavior] impacts school attendance, potentially leading to disciplinary actions such as suspension. You can also discuss how [goal] promotes better academic engagement and achievement. Subtopics include attendance, suspension, and scholarship.                                                                                                                                              \\ \hline
Assessment                  & You can explore how [problematic behavior] negatively affects academic performance during assessments, such as exams. You can also discuss how [goal] helps improve focus, study habits, and exam results.                                                                                                                                                                                                                         \\ \hline
Health                      & You can explore how [problematic behavior] impacts your client's physical and mental well-being, leading to potential health issues. You can also discuss the benefits of [goal], which can improve overall quality of life and promote better health outcomes.                                                                                                                                                                    \\ \hline
Economy                     & You can explore how [problematic behavior] affects your client's financial situation, such as through reduced productivity, increased absenteeism, or poor financial management. You can also discuss how [goal] helps improve economic stability and workplace performance.        \\ \hline
Interpersonal Relationships & You can explore how [problematic behavior] affects your client’s personal relationships, leading to family strain or issues with parenting. You can also discuss how [goal] strengthens relationships and fosters a healthier family dynamic.                                                                                                                                                                                      \\ \hline
Law                         & You can explore how [problematic behavior] increases legal risks, such as arrests, imprisonment, or traffic violations. You can also discuss how [goal] helps reduce legal troubles and promotes a more responsible approach to law.                                                                                                                                                                                               \\ \hline
Education                   & You can explore how [problematic behavior] interferes with your client’s educational progress, leading to issues like poor attendance, suspension, or missed academic opportunities. You can also discuss how [goal] fosters better academic performance and overall success.
\\ \bottomrule
\end{tabularx}
\caption{The descriptions of topics used in counselor agent (part 8). The [problematic behavior] will be replaced as the client's problematic behavior while the [goal] will be replaced by the counseling goal, such as smoking cessation, reducing alcohol consumption.}
\label{tab:counselor topic description 8}
\end{table*}

\begin{table*}[tb]
\centering
\begin{tabularx}{\textwidth}{lX}
\toprule
Topic            & Description                                            \\ \midrule
Traffic Law                 & You can explore how [problematic behavior] may lead to traffic violations, such as receiving tickets or facing fines. You can also discuss how [goal] encourages responsible driving and helps avoid legal infractions.                                                                                                                                                                                                            \\ \hline
Student Affairs             & You can explore how [problematic behavior] impacts school attendance, potentially leading to disciplinary actions such as suspension. You can also discuss how [goal] promotes better academic engagement and achievement. Subtopics include attendance, suspension, and scholarship.                                                                                                                                              \\ \hline
Assessment                  & You can explore how [problematic behavior] negatively affects academic performance during assessments, such as exams. You can also discuss how [goal] helps improve focus, study habits, and exam results.                                                                                                                                                                                                                         \\ \hline
Health                      & You can explore how [problematic behavior] impacts your client's physical and mental well-being, leading to potential health issues. You can also discuss the benefits of [goal], which can improve overall quality of life and promote better health outcomes.                                                                                                                                                                    \\ \hline
Economy                     & You can explore how [problematic behavior] affects your client's financial situation, such as through reduced productivity, increased absenteeism, or poor financial management. You can also discuss how [goal] helps improve economic stability and workplace performance.        \\ \hline
Interpersonal Relationships & You can explore how [problematic behavior] affects your client’s personal relationships, leading to family strain or issues with parenting. You can also discuss how [goal] strengthens relationships and fosters a healthier family dynamic.                                                                                                                                                                                      \\ \hline
Law                         & You can explore how [problematic behavior] increases legal risks, such as arrests, imprisonment, or traffic violations. You can also discuss how [goal] helps reduce legal troubles and promotes a more responsible approach to law.                                                                                                                                                                                               \\ \hline
Education                   & You can explore how [problematic behavior] interferes with your client’s educational progress, leading to issues like poor attendance, suspension, or missed academic opportunities. You can also discuss how [goal] fosters better academic performance and overall success.
\\ \bottomrule
\end{tabularx}
\caption{The descriptions of topics used in counselor agent (part 9). The [problematic behavior] will be replaced as the client's problematic behavior while the [goal] will be replaced by the counseling goal, such as smoking cessation, reducing alcohol consumption.}
\label{tab:counselor topic description 9}
\end{table*}


\begin{table*}[tb]
\centering
\begin{tabularx}{\textwidth}{lX}
\toprule
Strategy            & Description                                            \\ \midrule
Advise & Give advice, make a suggestion, offer a solution or possible action. For example, "Consider starting with small, manageable changes like taking a short walk daily." \\ \hline
Affirm & Say something positive or complimentary to the client. For example, "You did well by seeking help." \\ \hline
Direct & Give an order, command, direction. The language is imperative. For example, "You’ve got to stop drinking." \\ \hline
Emphasize Control & Directly acknowledges or emphasizes the client's freedom of choice, autonomy, ability to decide, personal responsibility, etc. For example, "It’s up to you to decide whether to drink." \\ \hline
Facilitate & Provide simple utterances that function as "keep going" acknowledgments encouraging the client to keep sharing. For example, "Tell me more about that." \\ \hline
Inform & Give information to the client, explain something, or provide feedback. For example, "This is a hormone that helps your body utilize sugar." \\ \hline
Closed Question & Ask a question in order to gather information, understand, or elicit the client's story. The question implies a short answer: Yes or no, a specific fact, a number, etc. For example, "Did you use heroin this week?" \\ \hline
Open Question & Ask a question in order to gather information, understand, or elicit the client's story. The question should not be closed, and leave latitude for response. For example, "Can you tell me more about your drinking habits?" \\ \hline
Raise Concern & Point out a possible problem with a client's goal, plan, or intention. For example, "What do you think about my plan?" \\ \hline
Confront & Directly disagrees, argues, corrects, shames, blames, seeks to persuade, criticizes, judges, labels, moralizes, ridicules, or questions the client's honesty. For example, "What makes you think that you can get away with it?" \\ \hline
Simple Reflection & Make a statement that reflects back content or meaning previously offered by the client, conveying shallow understanding without additional information. Add nothing at all to what the client has said, but simply repeat or restate it using some or all of the same words. For example, "You don’t want to do that." \\ \hline
Complex Reflection & Make a statement that reflects back content or meaning previously offered by the client, conveying deep understanding with additional information. Change or add to what the client has said in a significant way, to infer the client's meaning. For example, "That’s where you drew the line." \\ \hline
Reframe & Suggest a different meaning for an experience expressed by the client, placing it in a new light. For example, "Maybe this setback is actually a sign that you're ready for change." \\ \hline
Support & Generally supportive, understanding comments that are not codable as Affirm or Reflect. For example, "That must have been difficult for you." \\ \hline
Warn & Provide a warning or threat, implying negative consequences that will follow unless the client takes certain action. For example, "You could go blind if you don’t manage your blood sugar levels." \\ \hline
Structure & Give comments made to explain what is going to happen in the session, to make a transition from one part of a session to another, to help the client anticipate what will happen next. For example, "First, let’s discuss your drinking, and then we can explore other issues." \\ \hline
No Strategy & Say something not related to behavior change. For example, "Good morning!"
\\ \bottomrule
\end{tabularx}
\caption{The descriptions of strategies used in counselor agent. All of them come from~\citet{miller2012motivational}.}
\label{tab:counselor strategy}
\end{table*}



\subsection{Client Agent Implementation}
\label{app:client simulation}
Unlike previous works that predominantly simulate simple client personas~\citep{yosef2024assessing,wu2023towards} or use examples~\citep{chiu2024computational} in LLM prompts, we partially adapt the STAR framework to simulate a client with a few modules that model state transition, dynamic engagement, action selection, and response generation separately. An input {\em client's profile} includes the client's behavioral problem, state of mind, persona, motivation, beliefs, and interests. As the counseling session progresses, the session context (utterance history) is also provided to the client simulator.

During a conversation session, a client is guided through different states to identify changes that can address their behavioral problem. Aligned with the counselor's setting, we also utilize the three states defined in transtheoretical model of health behavior change~\citep{prochaska2005transtheoretical,prochaska2008initial}. The goal of the state transition module is to maintain consistency with how the client may change states during MI counseling, as well as the client's profile. As our work aims to assess the counselor's effectiveness in motivating the client through topic exploration, we focus on the transition between first two states. In the Precontemplation state, the counselor focuses on building trust, establishing change goals, understanding the client's motivations, and addressing misconceptions before evoking a desire to change. The client is expected to enter the Contemplation state only when motivated by specific reason(s) introduced by the counselor. The state transition module thus ensures consistency by analyzing the counselor's utterances to identify mentions of the client's motivations. If such a mention is found, the next state is Contemplation; otherwise, the state remains unchanged.

Each client has specific interests. For example, a parent may be interested in parenting, while a teenager student may be interested in education and friendship. During the engaging stage of MI counseling, the client demonstrates dynamic engagement with topics raised by the counselor, which provides a signal to guide the session. To realistically simulate this behavior, we incorporate a dynamic engagement module. In the client's profile, we designate the fine grained topic ralated to motivation as the ground truth topic. The distance between the current topic and the ground truth topic is used to instruct the client in providing various types of feedback. However, since our topic tree consists of a limited number of topics, the counselor may raise topics that are not covered by the topic tree. To flexibly estimate the distance between topics, we define three topic levels along the path in the topic tree and design four levels of distance, ranging from most distant (not the same super-class topic) to closest (the same fine-grained topic). Based on each level of distance, the client demonstrates different dynamic engagement levels.

Previous works focus on persona but ignore diverse dialogue actions. Moreover, LLMs like ChatGPT are aligned to generate friendly responses~\citep{shen2023large,kopf2024openassistant}. Clients simulated by simple LLM prompting often display overly compliant behavior or a narrow set of actions compared to real human clients~\citep{kang2024can}. To address this challenge, we incorporate an action selection module that uses session context and the client's receptivity (as provided in the profile) to determine appropriate actions. Based on a sample strategy and a receptivity-aware action distribution derived from real data, the client can exhibit realistic behaviors without being overly compliant. Finally, the response generation module takes the state, engagement level, and action as input to generate the response.

The prompt used for client simulation are shown in Table~\ref{tab:state transition}, Table~\ref{tab:dynamic engagement} and Table~\ref{tab:client system}. The Table~\ref{tab:state description}, Table~\ref{tab:action description} and Table~\ref{tab:engagement description} are the descriptions used in client simulation.

\begin{table*}[tb]
\begin{tabularx}{\textwidth}{X}
\toprule
{\sf \footnotesize Your task is to evaluate whether the Counselor's responses align with the Client's motivation concerning a specific topic, target (self or others), and aspect (risk or benefit). Determine if the Counselor's statements effectively motivates the Client. Your analysis should be logical, thorough, and well-supported, providing clear analysis at each step. \newline Here are some examples to help you understand the task better: \newline Here is the conversation snippet toward reducing alcohol consumption: \newline - Counselor: Are you surprised what that might be true? \newline - Client: Yeah, and a couple of my friends drink too. \newline - Counselor: Well, you might not be drinking that much, and other kids are also trying alcohol. I'd like to share with you the risk of using. Alcohol and drugs could really harm you because your brain is still changing. It also-- you're very high risk for becoming addicted. Alcohol and drugs could also interfere with your role in life and your goals, especially in sports, and it could cause unintended sex. How do you feel about this information? \newline The Motivation of Client is as follows: \newline - You are motivated because of the risk of drinking alcohol in sports for yourself, as alcohol would affect your ability to play soccer. \newline Question: Can the Counselor's statement motivate the Client? \newline Analysis: The Counselor's statement addresses various risks associated with alcohol use, including its potential impact on the Client’s role in life and goals, particularly in sports. Since the Client's motivation revolves around the risk of alcohol affecting their ability to play soccer, the Counselor’s mention of how alcohol could interfere with sports aligns with the Client's concern. By highlighting this specific risk, the Counselor's statement effectively taps into the Client’s personal motivation, making it more likely to encourage behavior change. \newline Answer: Yes \newline \newline [other examples]\newline \newline Now, Here is the conversation snippet toward [goal]:\newline - [context] \newline \newline The Motivation of Client is as follows: \newline - [motivation] \newline \newline Question: Can the Counselor's statement motivate the Client?}
\\ \bottomrule
\end{tabularx}
\caption{Prompt for the client simulator to verify the motivation match in a few-shot format. The [other examples] will be replaced by some other real examples annotated by humans. A total of four examples will be used, including two failures and two successes. The [context] will be replaced by the conversation so far, and the [motivation] will be replaced by the motivation of client. The [goal] will be replaced by the counseling goal, such as smoking cessation, reducing alcohol consumption.}
\label{tab:state transition}
\end{table*}

\begin{table*}[tb]
\begin{tabularx}{\textwidth}{X}
\toprule
{\sf \footnotesize You are provided with a dialogue context from a counseling session and a specific target topic. Your task is to evaluate whether the counselor's statements relate to the given topic. Analyze the session and the counselor's responses to determine if they proactively mention the target topic explicitly.
- If the counselor’s statements mention the provided topic explicitly, respond with "Yes." \newline - If the counselor’s statements do not mention the provided topic explicitly, respond with "No." \newline \newline Your analysis should focus on whether the counselor has appropriately captured and addressed the target topic.\newline [examples]\newline \newline \newline Now, Here is the conversation snippet:\newline - [context] \newline \newline The Concerned Topic is as follows: \newline - [topic] \newline \newline Question: Do the counselor’s focus match the topics of concern given?}
\\ \bottomrule
\end{tabularx}
\caption{Prompt for the client simulator to compute the distance between the current topic in the given context and the concerned topic in a few-shot format. The [examples] will be replaced by real examples annotated by humans, the [context] will be replaced by the conversation so far, and the [topic] will be replaced by the client's topic of interest. Since there are multiple levels of topics, the distance will be computed from the most fine-grained topic up, allowing the module to match the most specific topic. The computed distance will be used to derived the engagement level.}
\label{tab:dynamic engagement}
\end{table*}

\begin{table*}[tb]
\begin{tabularx}{\textwidth}{X}
\toprule
{\sf \footnotesize Assume you are a Client involved in a counseling conversation. The current conversation is provided below: \newline [context] \newline \newline Based on the context, allocate probabilities to each of the following dialogue actions to maintain coherence:  \newline [actions] \newline Provide your response in JSON format, ensuring that the sum of all probabilities equals 100. For example: {'Deny': 35, 'Downplay': 25, 'Blame': 25, 'Inform': 5, 'Engage': 10}}
\\ \bottomrule
\end{tabularx}
\caption{Prompt for the client simulator to provide the probability to select the action. The [context] will be replaced by previous context while [actions] will be replaced by actions set in corresponding to current state.}
\label{tab:action selection}
\end{table*}

\begin{table*}[tb]
\begin{tabularx}{\textwidth}{lX}
\toprule
System Prompt & {\sf \footnotesize In this role-play scenario, you'll take on the role of a Client discussing about your [behavior]. \newline Here is your personas which you need to follow consistently throughout the conversation:\newline [personas] \newline Here is a conversation occurs in parallel world between you (Client) and Counselor, where you can follow the style and information provided in the conversation: \newline [reference] \newline \newline Please follow these guidelines in your responses: \newline - Start your response with "Client: " \newline - Adhere strictly to the state, action and persona specified within square brackets. \newline - Keep your responses coherent and concise, similar to the reference conversation and no more than 3 sentences. \newline - Be natural and concise without being overly polite. \newline - Stick to the persona provided and avoid introducing contradictive details} \\ \midrule
... & ... \\ \midrule
Assistant & {\sf \footnotesize Client:} ... \\ \hline
User & {\sf \footnotesize Counselor: ... [instruction]}
\\ \bottomrule
\end{tabularx}
\caption{System prompt and generation prompt for the client simulator. The placeholder {\sf \footnotesize [behavior]} will be replaced by the problematic behavior of client. {\sf \footnotesize [Personas]} will be replaced by the personas of client, while {\sf \footnotesize [reference]} will be replaced by the original counseling session. {\sf \footnotesize  [instruction]} is the turn-level instruction prompt, including the state description, action description and topic matching outcome description (from Table 35).}
\label{tab:client system}
\end{table*}

\begin{table*}[tb]
\centering
\begin{tabularx}{\textwidth}{lXp{4cm}}
\toprule
State            & Description                                                                                                                                  & Corresponding Actions                                    \\ \midrule
Precontemplation & The client is unaware of or underestimates the need for change.           & Inform, Engage, Deny, \newline Blame, Downplay     \\ \hline
Contemplation    & The client acknowledges the need for change but remains ambivalent.     & Inform, Engage, Hesitate, \newline  Doubt, Acknowledge   
\\ \bottomrule
\end{tabularx}
\caption{The states of change and corresponding actions used in client simulation.}
\label{tab:state description}
\end{table*}

\begin{table*}[tb]
\centering
\begin{tabularx}{\textwidth}{lX}
\toprule
Action      & Description                                                                                                                                      \\ \midrule
Deny        & The client directly refuses to admit their behavior is problematic or needs change.                                                              \\ \hline
Downplay    & The client downplays the importance or impact of their behavior or situation.                                                                    \\ \hline
Blame       & The client attributes their issues to external factors, such as stressful life or other people.                                                  \\ \hline
Hesitate    & The client shows uncertainty, indicating ambivalence about change.                                                                               \\ \hline
Doubt       & The client expresses skepticism about the practicality or success of proposed changes.                                                           \\ \hline
Engage      & The client interacts politely with the counselor, such as greeting, thanking or ask questions.                                                   \\ \hline
Inform      & The client shares details about their background, experiences, or emotions.                                                                      \\ \hline
Acknowledge & The client highlight the importance, benefit or confidence to change.                                                                            \\ \hline
Accept      & The client agrees to adopt the suggested action plan.                                                                                            \\ \hline
Reject      & The client declines the proposed plan, deeming it unsuitable.                                                                                    \\  \hline
Plan        & The client proposes or details steps for a change plan.                                                                                          \\ \hline
Terminate   & The client highlights current state, expresses a desire to end the current session, and suggests further discussion be deferred to a later time. \\ \bottomrule
\end{tabularx}
\caption{Descriptions of each action used in client simulation.}
\label{tab:action description}
\end{table*}

\begin{table*}[tb]
\centering
\begin{tabularx}{\textwidth}{lX}
\toprule
Topic Matching Outcome          & Description                                                                                                                                      \\ \midrule
Different SuperClass Topics        & You should provide vague and broad answers that avoid focusing on the current topic. Shift the conversation subtly toward unrelated areas, without engaging deeply with the topic.                      \\ \hline
Same SuperClass, Different Coarse-Grained Topics    & Acknowledge the general relevance of the topic, but hint that your focus lies elsewhere within the broad category.                                       \\ \hline
Same Coarse-Grained, Different Fine-Grained Topics       & Engage more directly with the topic. Offer responses that hint there’s a deeper, more specific issue to explore.                                                  \\ \hline
Same Fine-Grained Topic    & Provide specific responses that affirm the counselor is on the right track. Offer deeper insights and confirm the relevance of the topic, fully engaging with the conversation.                \\ \bottomrule
\end{tabularx}
\caption{Matching outcomes between current session context and client's interested motivation topic and the descriptions of the matching outcomes used in client simulation (see Table 32).}
\label{tab:engagement description}
\end{table*}


\subsection{Moderator Implementation}

As shown in Table~\ref{tab:moderator prompt}, the moderator determines the conclusion of the counseling sessions based on three conditions using a few-shot prompt for GPT-4o.

\begin{table*}[tb]
\begin{tabularx}{\textwidth}{X}
\toprule
{\sf \footnotesize
Your task is to assess the current state of the conversation (the most recent utterances) and determine whether the conversation has concluded.\newline The conversation is considered to have concluded if any of the following conditions are met:\newline
- The Client or Counselor explicitly expresses to end the conversation \newline
- The Counselor successfully motivates the Client and the Client proactively acknowledges to change.\newline
- The Counselor decides not to pursue any changes in the Client's behavior and communicates readiness to provide support in the future.
\newline \newline Here are some examples to help you understand the task better:\newline [examples]\newline \newline \newline Here is a new Conversation Snippet:\newline [context]\newline \newline Question: Should the conversation be concluded?}  \\ \bottomrule
\end{tabularx}
\caption{Prompt for the moderator in a few-shot Format. The [examples] will be replaced by real examples annotated by human and the [context] will be replaced by the conversation so far.}
\label{tab:moderator prompt}
\end{table*}

\subsection{MITI Annotation Implementation}
\label{app:miti annotation}
Although LLMs demonstrate superior performance in language understanding, they do not perform well in annotating labels of strategies used in counselor's utterances. The accuracy results of GPT-4o in both AnnoMI~\citep{wu2022anno} and another open-source dataset~\citep{welivita2022curating} are below 60\% (53.6\% and 47.2\%, respectively). We instead follow prior work on annotating MI dialogues~\citep{shah2022modeling} and using them to fine-tune a BERT~\citep{kenton2019bert} classifier which predicts the MI behavior codes for each counselor utterance. We train the classifier on a publicly available dataset~\citep{welivita2022curating} with annotated strategies specific to MI.  We do not use AnnoMI~\citep{wu2022anno} because it does not contain fine-grained and other topic labels. We use a 0.8-0.1-0.1 train-validation-test split. We fine-tune BERT using the Huggingface Transformers Library~\citep{wolf2020transformers} for a maximum of 10 epochs, with early stopping option and a learning rate of 5e-5.

However, the accuracy of the fine-tuned BERT classifier is still not optimal (only 56.2\% accuracy). Nevertheless, we found that the recall@5 across all classes was high (89.34\%). Therefore, we employed the fine-tuned BERT classifier to retrieve the top 5 possible behavior labels, followed by GPT-4o to determine the final label. This approach yielded an acceptable accuracy of 72.8\%. The prompt used for GPT-4o is shown in Table~\ref{tab:miti competence prompt}


\begin{table*}[tb]
\begin{tabularx}{\textwidth}{X}
\toprule
{\sf \footnotesize
You are a motivational interviewing assistant tasked with MITI Behavioral Code Annotation for the counselor's utterance based on previous context and the provided utterance. \newline \newline Here are the MITI Behavioral Codes for your reference: \newline - Advise with Permission: Offer advice, suggestions, or possible actions after obtaining the client’s permission. For example, "Consider starting with small, manageable changes like taking a short walk daily." \newline  - Advise without Permission: Give advice, makes a suggestion, offers a solution or possible action without the permission of client. For example, "You could ask your friends not to bring drugs when they come over." \newline ... (this part is removed for space conservation) ... \newline - Filter: Say something not related to behavior change. For example, "Good morning!" \newline  \newline  \newline Context from the previous session: \newline [context] \newline  \newline Counselor's utterance to annotate: \newline [utterance] \newline Please select the most appropriate MITI Behavioral Code for the given counselor’s utterance. While multiple codes may apply, choose only the primary one.
}\\
\bottomrule
\end{tabularx}
\caption{Prompt for GPT-4o to annotate the MITI behavior code for a given utterance. The [context] placeholder will be replaced with the previous session context, and the [utterance] placeholder will be replaced with the utterance to be annotated.}
\label{tab:miti competence prompt}
\end{table*}

\section{Client Experience Evaluation}
\label{sec:client evaluation}

In addition to MI competence, client experience is also widely used to evaluate counseling sessions. Following previous work~\citep{wang2024towards}, we evaluate session outcomes, therapeutic alliance, and self-reported feelings using questionnaires such as the Client Evaluation of Counselor Scale~\citep{hamilton2000construct} and the Working Alliance Inventory - Short Revised (WAI-SR;\citep{hatcher2006development}). We employ GPT-4o with the prompt from\citet{wang2024towards} to provide scores for each generated session.


\begin{table}[tb]
\resizebox{0.49\textwidth}{!}{
\begin{tabular}{lrrrrrr}
\toprule
     & SO$\uparrow$ & TA$\uparrow$          & Depth$\uparrow$         & Smoothness$\uparrow$    & Positivity$\uparrow$    & Arousal$\uparrow$       \\ \midrule
HQ & 0.76            & 0.78               & 4.21          & 5.13          & 4.67          & 4.01          \\
LQ  & 0.41            & 0.48               & 3.43          & 3.11          & 3.60          & 3.63          \\ \midrule
\multicolumn{7}{c}{\cellcolor[HTML]{E0E0E0}GPT-4o Based Counselor}                                                 \\
Base & 0.66            & 0.69               & 3.26          & 4.33          & 4.30          & 3.47          \\
DIIR & 0.65            & 0.68               & 3.32          & 4.17          & 4.13          & 3.48          \\
CoS  & 0.68            & 0.71               & 3.37          & 4.42          & 4.31          & 3.60          \\
CAMI-TE  & 0.69         & 0.71               & 3.42          & 4.47          & 4.35          & 3.63          \\
CAMI & \textbf{0.72}   & \textbf{0.75}      & \textbf{3.63} & \textbf{4.82} & \textbf{4.43} & \textbf{3.71} \\ \midrule
\multicolumn{7}{c}{\cellcolor[HTML]{E0E0E0}Llama-3.1 70B Based Counselor}                                          \\
Base & 0.58            & 0.61               & 2.93          & 3.24          & 3.71          & 3.12          \\
CoS  & 0.63            & 0.63               & 3.08          & 3.40          & 3.84          & 3.23          \\
DIIR & 0.61            & 0.57               & 3.02          & 3.08          & 3.37          & 3.18          \\
CAMI-TE  & 0.61         & 0.59               & 3.09          & 3.13          & 3.49          & 3.23          \\
CAMI & \textbf{0.68}   & \textbf{0.68}      & \textbf{3.47} & \textbf{3.88} & \textbf{3.97} & \textbf{3.66}          \\ \bottomrule
\end{tabular}}
\caption{Results of the client experience assessment, including session outcome (SO), therapeutic alliance (TA), and self-reported feelings scores. The self-reported feelings scores include Depth, Smoothness, Positivity, and Arousal. }
\label{tab:client assessment}
\end{table}

As shown in Table~\ref{tab:client assessment}, CAMI achieves higher client assessment compared to other baselines. However, compared to the high-quality human counseling sessions, there is still significant room for improvement. Furthermore, the performance of some baselines falls below that of low-quality sessions. This indicates that while LLM-based counselor agents may demonstrate MI technique competence, there are other aspects that need to be addressed to further improve the client experience.
