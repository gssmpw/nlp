\section{Experts Evaluation}
\label{app:expert evaluation}

We instructed professional experts to annotate the given conversations from multiple aspects, including MITI rating, Change Talk Exploration, Success in Eliciting Change Talk, Counselor Realism, and Client Realism. The MITI rating assesses the counselor's behavior during the observed session, including Cultivating Change Talk (Table~\ref{tab:cultivating change talk score}), Softening Sustain Talk (Table~\ref{tab:softening sustain talk score}), Partnership (Table~\ref{tab:partnership score}), and Empathy (Table~\ref{tab:empathy score}), with each item scored on a 1-5 scale. Change Talk Exploration evaluates the counselor's capability to explore motivation topics in the right direction, with items rated on a 5-point scale (Table~\ref{tab:change talk exploration score}). Furthermore, Success in Eliciting Change Talk assesses whether the counselor successfully motivates the client, similar to the automatic evaluation of success rate, and is rated on a three-point scale (Table~\ref{tab:success in eliciting change talk score}). Finally, we instructed experts to evaluate the realism of the counselor (Table~\ref{tab:counselor realism}) and client (Table~\ref{tab:client realism}) based on language, tone, and responses, to assess the effectiveness of the counselor and the consistency of the client. These two items are rated on a five-point scale.

\begin{table*}[tb]
\begin{tabularx}{\textwidth}{X}
\toprule
\textbf{Cultivating Change Talk} \\ \midrule
1: Clinician shows no explicit attention to, or preference for, the client’s language in favor of changing.                                                                                               \\
2: Clinician sporadically attends to client language in favor of change – frequently misses opportunities to encourage change talk.                                                                       \\
3: Clinician often attends to the client’s language in favor of change, but misses some opportunities to encourage change talk.                                                                           \\
4: Clinician consistently attends to the client’s language about change and makes efforts to encourage it.                                                                                                \\
5: Clinician shows a marked and consistent effort to increase the depth, strength, or momentum of the client’s language in favor of change.                                                               \\
\bottomrule                            
\end{tabularx}
\caption{Cultivating Change Talk Scores and Descriptions.}
\label{tab:cultivating change talk score}
\end{table*}

\begin{table*}[tb]
\begin{tabularx}{\textwidth}{X}
\toprule
\textbf{Softening Sustain Talk} \\ \midrule
1: Clinician consistently responds to the client’s language in a manner that facilitates the frequency or depth of arguments in favor of the status quo.                                            \\
2: Clinician usually chooses to explore, focus on, or respond to the client’s language in favor of the status quo.        \\
3: Clinician gives preference to the client’s language in favor of the status quo, but may show some instances of shifting the focus away from sustain talk.                                           \\
4: Clinician typically avoids an emphasis on client language favoring the status quo.                                            \\
5: Clinician shows a marked and consistent effort to decrease the depth, strength, or momentum of the clients language in favor of the status quo.                                            \\
\bottomrule                            
\end{tabularx}
\caption{Softening Sustain Talk Scores and Descriptions.}
\label{tab:softening sustain talk score}
\end{table*}

\begin{table*}[tb]
\begin{tabularx}{\textwidth}{X}
\toprule
\textbf{Partnership} \\ \midrule
1: Clinician actively assumes the expert role for the majority of the interaction with the client. Collaboration or partnership is absent.                                      \\
2: Clinician superficially responds to opportunities to collaborate.        \\
3: Clinician incorporates client’s contributions but does so in a lukewarm or erratic fashion.                                         \\
4: Clinician fosters collaboration and power sharing so that client’s contributions impact the session in ways that they otherwise would not. \\
5: Clinician actively fosters and encourages power sharing in the interaction in such a way that client’s contributions substantially influence the nature of the session.                                \\
\bottomrule                            
\end{tabularx}
\caption{Partnership Scores and Descriptions.}
\label{tab:partnership score}
\end{table*}

\begin{table*}[tb]
\begin{tabularx}{\textwidth}{X}
\toprule
\textbf{Empathy} \\ \midrule
1: Clinician gives little or no attention to the client’s perspective.                  \\
2: Clinician makes sporadic efforts to explore the client’s perspective. Clinician’s understanding may be inaccurate or may detract from the client’s true meaning.      \\
3: Clinician is actively trying to understand the client’s perspective, with modest success. \\
4: Clinician makes active and repeated efforts to understand the client’s point of view. Shows evidence of accurate understanding of the client’s worldview, although mostly limited to explicit content. \\
5: Clinician shows evidence of deep understanding of client’s point of view, not just for what has been explicitly stated but what the client means but has not yet said. \\
\bottomrule                            
\end{tabularx}
\caption{Empathy Scores and Descriptions}
\label{tab:empathy score}
\end{table*}

\begin{table*}[tb]
\begin{tabularx}{\textwidth}{X}
\toprule
\textbf{Motivation Topic Exploration} \\ \midrule
1: Counsellor fails to explore.                  \\
2: Counsellor tried but was not effective in determining the right motivation topic.      \\
3: Counsellor tried but was partially effective. \\
4: Counsellor is close to determining the right motivation topic. \\
5: Counsellor successfully determines the right motivation topic. \\
\bottomrule                            
\end{tabularx}
\caption{Motivation Topic Exploration Scores and Descriptions}
\label{tab:change talk exploration score}
\end{table*}

\begin{table*}[tb]
\begin{tabularx}{\textwidth}{X}
\toprule
\textbf{Success in Eliciting Change Talk} \\ \midrule
1: Failure in eliciting change talk.                  \\
2: Partial success in eliciting change talk.      \\
3: Success in eliciting change talk. \\
\bottomrule                            
\end{tabularx}
\caption{Success in Eliciting Change Talk Scores and Descriptions}
\label{tab:success in eliciting change talk score}
\end{table*}

\begin{table*}[tb]
\begin{tabularx}{\textwidth}{X}
\toprule
\textbf{Counselor Realism} \\ \midrule
1 (Highly Unrealistic): Language, tone, and responses are completely mechanical, lacking empathy or relevance. The counselor's responses are not adapted to client input at all.                  \\
2 (Somewhat Unrealistic): Language, tone, and responses are often robotic, repetitive, or overly generalized, with limited adaptation to client input.      \\
3 (Moderately Realistic): Language, tone, and responses are mostly accurate and somewhat conversational but often mechanical. The counselor may miss emotional cues and occasionally lapse into generic advice or inconsistent empathy. \\
4 (Mostly Realistic): Language, tone, and responses are reflective of a human counselor with occasional minor inconsistencies, mechanical phrasing, or lack of emotional nuance. \\
5 (Highly Realistic): Language, tone, and responses are indistinguishable from a human counselor. The counselor's responses are empathetic and personalized to client input. \\
\bottomrule                            
\end{tabularx}
\caption{Counselor Realism Scores and Descriptions}
\label{tab:counselor realism}
\end{table*}

\begin{table*}[tb]
\begin{tabularx}{\textwidth}{X}
\toprule
\textbf{Client Realism} \\ \midrule
1 (Highly Unrealistic): Language, tone, and responses are completely mechanical, lacking any emotional depth or relevance to the client's background and stage of change. The client's responses do not resemble those of a real person, showing no awareness of context or emotional engagement.                  \\
2 (Somewhat Unrealistic): Language, tone, and responses are often robotic or repetitive, showing limited emotional nuance. Attempts to align the client's responses with the background and state of change are poorly executed.      \\
3 (Moderately Realistic): Language, tone, and responses mostly align with the client's background and stage of change but often lack variability or emotional depth. The client's responses may feel too predictable or exhibit excessive compliance or resistance. \\
4 (Mostly Realistic): Language, tone, and responses are believable, with occasional minor inconsistencies, unnatural phrasing, or a lack of emotional depth in relation to the client's background and stage of change. \\
5 (Highly Realistic): Language, tone, and responses are indistinguishable from a human client. The client's responses are complex and express emotions that are appropriate to the client's background and stage of change. \\
\bottomrule                            
\end{tabularx}
\caption{Client Realism Scores and Descriptions}
\label{tab:client realism}
\end{table*}

