% \section{Counselor Simulation}
\section{The STAR Framework}
\label{sec:counselor}
Our proposed STAR framework consists of four key modules that collaboratively mimic a counselor performing MI counseling, that is: (a) {\em \underline{S}tate Inference} (§\ref{sec: state estimation}), which infers the current client's state based on the transtheoretical model of health behavior change~\citep{prochaska2005transtheoretical,prochaska2008initial}. Based on the inferred client's state, the counselor may perform topic exploration before selecting a response strategy; (b) {\em \underline{T}opic Exploration} (§\ref{sec: topic exploration}), which explores a wide range of motivation topics in a topic tree to find the topic that may evoke change talk with the client; (c) {\em Strategy Selection}, also known as {\em \underline{A}ction Selection} (§\ref{sec: strategy selection}), which selects the appropriate strategy based on the current client's state, session context, and identified topic (if any); and (d) {\em \underline{R}esponse Generation \& Ranking} (see §\ref{sec: resopnse generation}), which generates a  few candidate responses based on the selected strategies and selects the most appropriate response that is coherent with the current context and aligned with MI principles.  In this study, we assume that the counselor is aware of the client's behavior problem before any counseling begins.

%{\color{blue} (EP: This para's content appears to duplicate the later sections)
%Right after a client's response, the counselor agent infers the client's state using the session context as input to a prompt submitted to the LLM. Then {\color{red} (EP: isn't topic exploration only done when the client is in pre-contemplation state?)} {\color{orange} (Yizhe: Yes, that is true. Thus we focus on the transition between precontemplaiton and contemplation state and end the sessions when the client is motivated.)}, the topic exploration module explores the topic tree and determines a topic that may be useful to evoke change talk. Simultaneously, the strategy selection module selects appropriate MI strategies to guide the counselor's response skillfully for. 
%To limit the response length, we restrict the number of selected strategies to fewer than two. The estimated state, along with the potential topic and strategies, are then used by the response generation module to generate the next response. Inspired by~\citet{hsu2023helping}, we generate one response for each selected strategy and for combined strategies, then rank these generated responses and choose the most coherent one to reply with. Details about the prompts can be found in the Appendix.
%} {\color{orange}(Yizhe: we can just remove it.)}

\subsection{Client's State Inference}
\label{sec: state estimation}

During MI counseling, a client may be in various mental states as they discuss their behavioral problems with the counselor. For counseling to be effective, the counselor must monitor the client's state and the current session context before responding ~\citep{prochaska2005transtheoretical,prochaska2008initial}. 

To define the client's states~\citep{prochaska1997transtheoretical, hashemzadeh2019transtheoretical}, we utilize the transtheoretical model of health behavior change~\citep{prochaska2005transtheoretical,prochaska2008initial}, consisting of five states: {\em Precontemplation}, {\em Contemplation}, {\em Preparation}, {\em Action}, and {\em Maintenance}.  In the {\em Precontemplation} stage, the client may not seriously consider behavior change nor be motivated to seek help. In the {\em Contemplation} stage, the client may weigh the pros and cons of changing their behavior. In the {\em Preparation} stage, the client has made a commitment to change. In the {\em Action} stage, the client actively takes steps to change their behaviors. In the {\em Maintenance}, the client seeks to avoid any temptation to return to the bad behavior. 
%we only consider the first three states, as MI counseling is particularly useful when a client is in the Precontemplation or Contemplation states. The 

Our primary goal is to assist clients in reaching the Preparation state, so our counselor agent focuses on the first three states. In the Precontemplation state, the counselor focuses on building trust, establishing change goals, understanding the client's motivations, and addressing misconceptions before encouraging the client to change. In the Contemplation state, the counselor listens to the client and addresses incorrect beliefs. In the Preparation state, the counselor provides the necessary information and tools to guide the discussion. 
We also aim to equip the counselor with the ability to %engage the client and evoke behavior change. We thus focus on the transition from Precontemplation to Contemplation where the counselor seeks to 
engage the client, to collaborate with the client to identify the motivating topic for change, and to soften the client's preference for the status quo.
%Given their strong capabilities in language understanding, classification, and generation~\citep{ouyang2022training,achiam2023gpt,dubey2024llama,brown2020language}, 

We thus prompt LLMs to infer the client's state based on the session context and state descriptions in a zero-shot manner. Our experiment results in §\ref{sec:topic exploration} show that the state-of-the-art LLMs perform this state inference task with high accuracy (see §~\ref{sec:state_inf_accuracy}).
\subsection{Topic Exploration}
\label{sec: topic exploration}

When the client is in the Precontemplation state, the counselor engages the client to develop rapport and cultivate change talk by supporting them in exploring topics that evoke motivation for behavior change. During the process, the counselor may explore different topics (e.g., smoking causing harm to family members, medication reducing cholesterol)~\citep{cole2023using} also known as the motivation topics. 
%While the current LLM-based agents are abundant in knowledge, they may have preference biases~\citep{kang2024can}{\color{red} (EP: topic bias?)}, which limit the diversity of change talk exploration. {\color{red} (EP: I thought this bias refers to bias in strategies, and not about topic exploration. Our work should be the first to introduce topic exploration.)} 
To our knowledge, topic exploration has not been studied in previous counseling agent research~\citep{hsu2023helping,xie2024few,sun2024chain} as most methods ignore the importance of change talk exploration. 

\noindent
\textbf{Topic Tree Construction.} We propose a three-level topic tree as an external knowledge base that facilitates the exploration of motivation topics by the counselor agent, aiming to evoke change talk for clients facing various behavioral challenges. As shown in Figure~\ref{fig:topic tree}, the topic tree is structured into superclass topics, coarse-grained topics, and fine-grained topics. To ensure standardization, we require all topic labels to match Wikipedia article titles, ensuring each topic is represented by a dedicated page. The construction of the topic tree draws on both real counseling sessions and the capabilities of GPT-4o, following this methodology:

%\begin{enumerate} 
%    \renewcommand{\labelenumi}{Step~\theenumi.} 
    %\item 
    \noindent
    \textit{Step 1: Derive Fine-Grained Topics.} Firstly, we derive a set of fine-grained topics from some counseling session corpus. Ideally, we would extract topics from a larger corpus to ensure comprehensive coverage. However, due to data limitations, we utilize the available AnnoMI dataset~\citep{wu2022anno,wu2023creation} and identify 28 fine-grained topics. 
    
    \noindent
    \textit{Step 2: Group Coarse-grained and Fine-grained Topics.} Next, we organize the derived fine-grained topics into coarse-grained topics based on Wikipedia's parent-child category relationships. Similarly, the coarse-grained topics are grouped into superclass topics.
    %, we group the fine-grained topics under the same category as an abstract topic.  To better model the hierarchical structure of topics, we create two levels of abstract topics. 
    Finally, we construct three levels of the topic tree: superclass topics (e.g., {\em Health} and {\em Economy}), coarse-grained topics (e.g., {\em Disease} and {\em Mental Disorders}), and fine-grained topics (e.g., {\em Depression} and {\em Hypertension}). 
    %All fine-grained topics can be linked to a Wikipedia page, and each abstract topic can be linked to a Wikipedia category. {\color{red} (EP: Do we have an article for each superclass and coarse-grained topic?)}
    
    \noindent 
    \textit{Step 3: Expand Topic Tree.} To broaden the range of topics, we expand the initial topic tree by prompting GPT-4o to generate additional coarse- and fine-grained topics, leveraging its extensive topical knowledge. We choose not to expand the superclass topics as they are already very general. Instead, we prompt GPT-4o to suggest topics under the superclass and coarse-grained topics in a few-shot manner (see Table~\ref{tab:topic expansion}). This step continues until there are no more new topics. Our final topic tree consists of 5 superclass topics, 14 coarse-grained topics, and 59 fine-grained topics, all verified as titles of Wikipedia pages, as shown in Figure~\ref{fig:topic tree}.
%\end{enumerate}

\noindent
\textbf{Topic Navigation.} We guide the counselor agent in navigating topics within the topic tree to initiate change talk in two stages. 
%Specifically, at each turn, the counselor takes in the client's feedback to the current topic. 
In the {\em initial engagement stage}, the counselor agent is provided with the superclass topics and prompted to %interact with the client based on these topics, aiming to 
explore these topics broadly with the client. The counselor agent should infer the client's interested topic (i.e., the one that evokes change talk) and estimate the probability of each superclass topic being the interested topic in each turn (see the prompt in Table~\ref{tab:topic initialization}). This phase ends after six turns which is similar to the number of turns for initial engagement of client proposed in~\citet{park2019designing}, or when a topic is assigned a probability higher than 40\%. The counselor will select the topic with the highest probability as the \textit{current client topic (or current topic)}.

In the subsequent {\em focused engagement stage}, when the inferred client topic does not match the client's actual motivation topic, the counselor agent performs one of the three navigation operations to identify other possible topics in the topic tree: (a) {\em Step Into}: This operation explores a sub-topic (either coarse-grained or fine-grained) of the current topic when the client provides some positive feedback, indicating a desire for a deeper discussion. The LLM is prompted to select the next topic from all sub-topics based on previous context and exploration path; (b) {\em Switch}: This operation allows the counselor to switch to another topic at the same level (i.e., sharing the same parent topic as the current topic) when the client shows limited engagement with the current topic but expresses interest in the parent topic. Similar to Step Into, the LLM is prompted to select the next topic from all candidate topics; and (c) {\em Step Out}: This operation involves revisiting the parent topic when the client does not wish to continue discussing the current topic or the parent topic. This occurs when the counselor has navigated to the wrong sub-tree. The prompt for selecting the navigation operations is shown in Table~\ref{tab:topic exploration}.
%In each operation, all the candidate topics are provided in instruction for selection by the agent.
%Based on the previous action, topic exploration module focuses on a specific topic according to the previous context, particularly considering client feedback, which guides the counselor's response generation. 
All the above navigation operations are facilitated by prompting the LLMs as shown in Appendix~\ref{app:counselor implementation}.  An example of topic exploration is also shown in Appendix~\ref{app:example}.

\subsection{Strategy Selection}
\label{sec: strategy selection}

Previous research has shown the importance of strategy selection in effective AI-based counseling~\citep{kang2024can,xie2024few,hsu2023helping,sun2024chain}. In our STAR framework, the Strategy Selection module select strategies from the Motivational Interviewing Skill Code (MISC)~\citep{miller2003manual} to guide LLMs in demonstrating MI skills, thereby aligning the LLMs with strategies that offer controllability and explainability. 
 Inspired by~\citet{sun2024chain}, we employ zero-shot prompts providing the definitions of different MI strategies~\citep{miller2003manual,miller2012motivational} and the current inferred client's state. A LLM prompt is then used to select MI strategies to guide response generation. 
 Both prior works~\citep{sun2024chain,xie2024few} and our own MI competency evaluation results (see Table~\ref{tab:miti}) highlight the effectiveness of this method.
 % Previous works~\citep{sun2024chain,xie2024few} and our experiment results in MI competency (Table~\ref{tab:miti}) demonstrate the effectiveness of this method through assessing the final generated response.

Following recent research~\citep{sun2024chain,xie2024few}, we allow the strategy selection module to select multiple strategies for generating a response (see Table~\ref{tab:strategy selection}). However, we limit the number of selected strategies to two so as to avoid an excessive number of strategies (e.g., more than five) in one generated counselor response. This limit is consistent with what we observe in the AnnoMI~\citep{wu2022anno,wu2023creation} dataset. 

\subsection{Response Generation \& Ranking}
\label{sec: resopnse generation}

Finally, the response generation module generates the counselor's response in the next turn using a {\em turn-by-turn} generation approach. 
Specifically, we construct an instruction prompt for the MI counselor, which includes the client's behavioral issue, the goal to achieve, the session context, and a turn-level instruction. The turn-level instruction further includes a description of the inferred client state, the current client topic, and the selected strategy(ies) (see Table~\ref{tab:couneslor system}).

When two strategies are selected, we follow the approach in \citet{hsu2023helping}, which generates an response for each single strategy as well as for the concatenated selected strategies. After generating the three candidate responses, we prompt the LLM to choose the most coherent  response based solely on the current session context (see Table~\ref{tab:response rank}).

