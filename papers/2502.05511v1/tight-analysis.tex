\section{Tightening the Analysis to 2-competitiveness}
\label{sec:tight-analysis}
Having identified the loose ends in the analysis of \cite{lund1999paging} above, we are able to tighten their analysis and prove the following lemma:
\begin{lemma}
    \label{lem:tight-analysis}
    Let $\mcA$ be a paging algorithm. Suppose that whenever $\mcA$ suffers a cache miss, the page $p$ that it chooses to evict is chosen from a distribution such that the following property is satisfied: for every page $q$ in the cache, the probability (over the choice of $p$ and the random sequence ahead, conditioned on the most recently requested page) that $q$ is next requested no later than the next request for $p$ is at least $1/c$. Then $\mcA$ is $c$-competitive against $\opt$.
\end{lemma}
\begin{proof}
    We make one small change (highlighted in green) to \Cref{item:clear-charges} in the charging scheme (\Cref{fig:charging-scheme}) from the analysis of \cite{lund1999paging}---whenever a new page is requested, any charges that this page might be giving are cleared, but also any charges that other pages might be having on the requested page are also cleared. This fixes the issue regarding uncleared charges discussed in \Cref{sec:looseness-uncleared-charges}.
    \begin{figure}[H]
        \begin{framed}
        Suppose that at time $t$, there is a request for page $s$.
            \vspace{-2mm}
                \begin{enumerate}
                    \item \label{item:updated-clear-charges} \begin{enumerate}
                        \item \label{item:updated-clear-charges-giving} Set $c(s)=\emptyset$.
                        \item \label{item:updated-clear-charges-bearing} \green{For any page $p$ that has $c(p)=s$, set $c(p)=\emptyset$.}
                    \end{enumerate}
                    \item \label{item:updated-assign-charge} If $\mcA$ evicts some page $p$, $c(p)$ is selected as follows:
                    \begin{enumerate}
                        \item \label{item:updated-p-not-in-opt+} If $p \notin \opt^+$, set $c(p)=p$.
                        \item \label{item:updated-p-in-opt+} If $p \in \opt^+$, find a page $q \in \mcA^- \setminus \opt^+$ that has no charges from $\opt^+ \setminus \mcA^-$. Set $c(p)=q$.
                    \end{enumerate}
                    \item \label{item:updated-opt-eviction-charge-reassign} If $\opt$ evicts some page $p$ and $c(p) \neq p$, set $c(p)=p$.
                \end{enumerate}
            \vspace{-2mm}
        \end{framed}
        \vspace{-4mm}
        \caption{Updated Charging Scheme}
        \label{fig:updated-charging-scheme}
    \end{figure}
    Let $\alpha(t)$ and $\beta(t)$ be the same random variables as defined in \eqref{eqn:alpha}, \eqref{eqn:beta} in the proof of \Cref{lem:lpr}. Further, let $\mcI$ and $\mcO$ be the same random variables as defined there as well. Note that even with the slightly-changed charging scheme, $\mcI$ remains quantitatively the same; the random variable $\mcO$ however changes---in particular, it potentially becomes a smaller number, because we are clearing charges more aggressively. The random variable
    \begin{align*}
        \sum_{t \le T}\beta(t) - \mcO
    \end{align*}
    nevertheless still counts the cache misses suffered by $\opt$ due to requests to savior pages, while potentially double-counting a few misses. However, to account for this double-counting, instead of dividing every miss by 2 (which was the issue discussed in \Cref{sec:looseness-double-counting}), we explicitly keep track if some $\beta(t)$ corresponds to a request to a doubly-charged page, and subtract it from our calculation. Formally, let
    \begin{align}
        \mcD = \sum_{t \le T}\Ind[\text{Page $\bp$ requested at time $t$ has two charges on it}]. \label{eqn:D}
    \end{align}
    Then, the quantity
    \begin{align*}
        \sum_{t \le T}\beta(t) - \mcD - \mcO
    \end{align*}
    counts the cache misses suffered by $\opt$ due to requests to savior pages more precisely, and without double-counting any miss. This crucially allows us to avoid an unnecessary factor of $2$.

    Lastly, in order for the analysis to go through, we have to also fix the final issue regarding requests to uncharged pages described in \Cref{sec:looseness-uncharged-non-first-time}. We simply do this by keeping track of an additional quantity that counts this, and add it to our calculation. Namely, let
    \begin{align}
        \mcU = \sum_{t \le T}\Ind[&\text{Page $\bp$ requested at time $t$ does not exist in $\opt$'s cache at this time because it was} \nonumber \\ 
        &\text{previously evicted by $\opt$, and $\bp$ has no charges on it}]. \label{eqn:U}
    \end{align}
    Then, combining all of the above, we have that
    \begin{align}
        \cost(\opt, T) &\ge \sum_{t \le T}\beta(t) - \mcD - \mcO + \mcU. \label{eqn:updated-final-accounting}
    \end{align}
    Comparing this to the accounting in \eqref{eqn:lpr-final-accounting}, it at least seems plausible that by not halving, we might be able to save a factor of 2. Namely, taking expectations, we obtain that
    \begin{align*}
        \E[\cost(\opt, T)] &\ge \sum_{t \le T} \E[\beta(t)] + \E[\mcI - \mcD - \mcO + \mcU] \\
        &\ge \frac{1}{c}\sum_{t \le T}\E[\alpha(t)] + \E[\mcI - \mcD - \mcO + \mcU] \\
        &= \frac{1}{c}\E[\cost(\mcA, T)] + \E[\mcI - \mcD - \mcO + \mcU],
    \end{align*}
    where the first inequality follows from the same analysis that we did in the proof of \Cref{lem:lpr}. It remains to argue that the quantity $\E[\mcI - \mcD - \mcO + \mcU]$ is nonnegative. Note that the random variable $\mcI - \mcD - \mcO + \mcU$ starts out being $0$ before the very first request at time $t=1$. We will argue that it always stays nonnegative thereafter via the following two claims.

    \begin{claim}
        \label{claim:potential-decrease-characterization}
         If the page $s$ requested at time $t$ satisfies the following condition: $s$ exists in $\opt$'s cache and $s$ does not exist in $\mcA$'s cache and $s$ is not giving a charge and $s$ is not bearing any charges, then the random variable $\mcI-\mcD-\mcO+\mcU$ decreases by 1 from $t$ to $t+1$. Furthermore, if the page $s$ requested at time $t$ does not satisfy this condition, then $\mcI-\mcD-\mcO+\mcU$ either increases or stays the same from $t$ to $t+1$.
    \end{claim}
    \begin{proof}
        We first show that if the page $s$ requested at time $t$ satisfies the condition, then $\mcI - \mcD - \mcO + \mcU$ decreases by 1. Since $s$ is in $\opt$'s cache, the request to $s$ does not change $\mcU$ (see the definition in \eqref{eqn:U}). Moreover, $s$ being in $\opt$'s cache implies it was either in the initial cache or previously requested, so $\mcI$ also remains unchanged. Similarly, because $s$ is not doubly charged, $\mcD$ remains unchanged. Thus, we only need to account for the change in $\mcO$.

        Because $s$ is neither giving nor bearing any charges, both \Cref{item:updated-clear-charges-giving} and \Cref{item:updated-clear-charges-bearing} cause no change in $\mcO$. Finally, because $s$ does not exist in $\mcA$'s cache and results in a cache miss, \Cref{item:updated-assign-charge} creates a new charge, causing $\mcO$ to increase by 1. In total, $\mcI-\mcD-\mcO+\mcU$ decreases by 1.

        \smallskip
        Next, we will show that if $s$ does not satisfy the condition, then $\mcI-\mcD-\mcO+\mcU$ either increases or stays the same. Towards this, consider each of the following cases: 
        
        \medskip\noindent\underline{Case 1: $s$ is a doubly-charged page.} \\
        In this case, $\mcD$ increases by 1. Note that one of the charges on $s$ is by $s$ itself, and the other charge is by some other page $q$. In particular, $s$ is not \textit{giving} a charge to a page other than itself. Thus, when $s$ is requested, \Cref{item:updated-clear-charges-giving} and \Cref{item:updated-clear-charges-bearing} ensure that the charge by $s$ on itself as well as the charge on $s$ by $q$ are \textit{both} dropped.\footnote{Note how \Cref{item:updated-clear-charges-bearing} was necessary to ensure this.} However, note also that $s$ is not in the cache of $\mcA$ (since it has a charge on itself, it was previously evicted by $\mcA$), and hence this request has caused a cache miss in $\mcA$, resulting in the creation of a new charge in \Cref{item:updated-assign-charge}. \Cref{item:updated-opt-eviction-charge-reassign} can only reassign a charge, and thus, the net change in $\mcO$ is $-1$. Finally, $\mcI$ and $\mcU$ remain the same. Therefore, the overall change in $\mcI - \mcD - \mcO + \mcU$ is 0. \medskip
        
        \noindent\underline{Case 2: $s$ is a singly-charged page.} \\
        This means that either $s$ is not in the cache of $\mcA$ and is charging itself, or $s$ is in the cache of $\mcA$ and is bearing a charge given by some other page $p$ at the time of its eviction. 
        
        In the former case, observe that none of $\mcI$, $\mcD$ or $\mcU$ are affected. The clearing of charges in \Cref{item:updated-clear-charges} causes $\mcO$ to decrease by 1, and the cache miss causes the creation of a new charge in \Cref{item:updated-assign-charge}. In total, $\mcI-\mcD-\mcO+\mcU$ stays unchanged.
        In the latter case, observe that the charge on $s$ is cleared in \Cref{item:updated-clear-charges-bearing}, causing $\mcO$ to decrease by 1. Furthermore, because $s$ is in the cache of $\mcA$, there is no creation of a new charge in \Cref{item:updated-assign-charge}. Thus, $\mcI-\mcD-\mcO+\mcU$ increases by 1.

        \medskip\noindent\underline{Case 3: $s$ is not in $\opt$'s cache.} \\
        If $s$ is either singly or doubly-charged, we fall under Case 1 or 2. If $s$ has no charges on it, then:
        
        \smallskip         \noindent\underline{Subcase 3a: $s$ has never been requested before.} \\
        In this case, $\mcI$ increases by 1. Also, $s$ cannot be giving or bearing any charges; thus \Cref{item:updated-clear-charges-giving} and \Cref{item:updated-clear-charges-bearing} cause no change to $\mcO$. However, the request to $s$ causes a cache miss in $\mcA$, resulting in an eviction, and the creation of a new charge. Thus, $\mcO$ increases by 1. Additionally, $\mcD$ and $\mcU$ remain the same. Thus, the net change in $\mcI - \mcD - \mcO + \mcU$ is 0.
        
        \smallskip         \noindent\underline{Subcase 3b: $s$ has been requested before.} \\
        This means that $s$ was previously in $\opt$'s cache at some time but was since evicted. Also, $s$ has no charges on it. Thus, $\mcU$ increases by 1, while $\mcI$ and $\mcD$ remain unchanged. It remains to reason about $\mcO$. Because $s$ has no charges on it, \Cref{item:updated-clear-charges-bearing} causes no change in $\mcO$.
        \Cref{item:updated-clear-charges-giving} either decreases $\mcO$ by 1 or causes no change to it.
        While the request to $s$ can cause a cache miss to $\mcA$, this can only result in the creation of a single new charge, and $\mcO$ can increase by at most 1 due to this. In total, $\mcI-\mcD-\mcO+\mcU$ either increases %
        or stays the same. 
        
        \medskip
        \noindent\underline{Case 4: $s$ is in $\mcA$'s cache.} \\
        This means that $s$ is giving no charges. Then, either $s$ is singly-charged or has no charges on it. It cannot be doubly-charged because it would have to be out of $\mcA$'s cache for that. If it is singly-charged, we fall under Case 2. If it has no charges on it, we reason as follows: both \Cref{item:updated-clear-charges} and \Cref{item:updated-assign-charge} leave $\mcO$ unaffected. Moreover, $\mcI$ and $\mcD$ remain unaffected. Finally, depending on whether $s$ is or isn't in $\opt$'s cache, $\mcU$ either stays the same or increases. Hence, $\mcI-\mcD-\mcO+\mcU$ either stays the same or increases. 
        
        \medskip
        \noindent\underline{Case 5: $s$ is giving a charge.} \\
        If $s$ has any charges on it, we fall under Case 1 or 2. So we assume that $s$ has no charges on it. This means that $s$ is not in $\mcA$'s cache, but is still in $\opt$'s cache. Note then that $\mcI, \mcD, \mcU$ remain unchanged. We reason about the change in $\mcO$ as follows: \Cref{item:updated-clear-charges-giving} clears the charge that $s$ is giving and decreases $\mcO$ by 1, whereas \Cref{item:updated-assign-charge} creates a new charge and increases $\mcO$ by 1. In total, $\mcI-\mcD-\mcO+\mcU$ remains unchanged.

        Thus, in any way that the requested page $s$ might not satisfy the condition, $\mcI-\mcD-\mcO+\mcU$ either increases or stays the same, concluding the proof.
    \end{proof}

    \begin{claim}
        If the page $s$ requested at time $t$ satisfies the condition: $s$ exists in $\opt$'s cache and $s$ does not exist in $\mcA$'s cache and $s$ is not giving a charge and $s$ is not bearing any charges, then $\mcI-\mcD-\mcO+\mcU$ is strictly positive just before this request.
    \end{claim}
    \begin{proof}
        Consider any time $t$ where the requested page $s$ satisfies the condition. We can associate to the page $s$ two past events occurring at times $t_1$ and $t_2$ (where $t_1 < t_2 < t$) such that: (1) $\mcA$ evicts $s$ at $t_1$, but $s$ continues to live in $\opt$'s cache, resulting in $s$ giving a charge to some other page $q$ in $\mcA$'s cache (\Cref{item:p-in-opt+}), and (2) $q$ gets requested at $t_2$, before the request to $s$ at $t$, thereby clearing the charge by $s$ on $q$ (as per \Cref{item:updated-clear-charges-bearing}). Observe that these two past events need to necessarily occur for $s$ to satisfy the condition. Now observe that when $q$ gets requested at $t_2$, $q$ is the bearer of a \textit{single} charge. Thus, over the course of this request to $q$, $\mcI, \mcD, \mcU$ remain unchanged. Because $s$'s charge on $q$ gets cleared, $\mcO$ decreases by $1$. Finally, because $q$ is in $\mcA$'s cache at this time, no new charges are added to $\mcO$. Thus, $\mcI - \mcD - \mcO + \mcU$ strictly \textit{increases} by 1 at step (2). Also, note that for each distinct time step where the requested page $s$ satisfies the condition, there is a distinct associated (past) step (2).
              


        Now consider the \textit{first} time $t_0$ where the requested page $s$ satisfies the condition. Then, by \Cref{claim:potential-decrease-characterization}, at every previous time step, $\mcI - \mcD - \mcO + \mcU$ either increased or remained the same. Given that $\mcI - \mcD - \mcO + \mcU$ started out being 0, and recalling that step (2) (which happened before $t_0$) caused a strict \textit{increase} in the quantity, we have that $\mcI - \mcD - \mcO + \mcU$ is strictly positive at $t_0$.

        Just after the request at $t_0$, $\mcI - \mcD - \mcO + \mcU$ decreases by 1 (by \Cref{claim:potential-decrease-characterization}), and is now only guaranteed to be nonnegative, instead of positive. Then, consider the next time $t$ when a page $s$ satisfying the condition is requested. We can again trace its associated (distinct) step (2) that happened in the past. If this happened before $t_0$, then $\mcI - \mcD - \mcO + \mcU$ stayed positive after the request at $t_0$. Alternatively, if this happened in between $t_0$ and $t$, $\mcI - \mcD - \mcO + \mcU$ still turns positive (if it ever became zero at all). In either case, $\mcI - \mcD - \mcO + \mcU$ is positive before the request at $t$. The claim follows by induction.
    \end{proof}

    The above two claims establish that if $\mcI - \mcD - \mcO + \mcU=0$ at any time $t$, then $\mcI - \mcD - \mcO + \mcU$ cannot decrease at time $t+1$. Furthermore, if at all $\mcI - \mcD - \mcO + \mcU$ does decrease (from being a positive number), it only decreases by 1. Together, this means that $\mcI - \mcD - \mcO + \mcU$ is always nonnegative, %
    and hence it is nonnegative in expectation. This concludes the proof of \Cref{lem:tight-analysis}
\end{proof}

\begin{corollary}
    \label{corollary:dom-is-2-competitive}
    The dominating distribution algorithm $\mcA_\dom$ is 2-competitive against $\opt$.
\end{corollary}
\begin{proof}
    This follows from \Cref{lem:tight-analysis} and \Cref{claim:dominating-distribution-probability}.
\end{proof}

\begin{remark}
    \label{remark:optimal-fifo}
    As in \Cref{remark:suboptimal-fifo}, if we set $c=1$ in \Cref{lem:tight-analysis}, our lemma says that a policy that has essentially seen the future and evicts the page that is next scheduled to be requested latest is 1-competitive against $\opt$, i.e., it is optimal. Thus, our analysis, while establishing the best-possible guarantee for algorithms satisfying the condition of the lemma, additionally recovers an alternate, charging-scheme-based proof of the optimality of the Farthest-in-Future eviction policy.
\end{remark}

\begin{remark}
    \label{remark:2-correct-answer-for-dom}
    The factor of 4 in the analysis of \cite{lund1994ip} constituted a factor of 2 arising due to doubly-charged pages, and a factor of 2 arising from the property of the dominating distribution algorithm. While we got rid of the first factor, the second factor appears to be necessary, stemming from an inherent property (\Cref{eqn:dominating-distribution-property}) of the algorithm itself. It would indeed be very interesting to see if this is not the case, and if the upper bound could further be improved from 2.
\end{remark}

\Cref{lem:tight-analysis} also allows us to improve the approximation guarantee for another intuitive \textit{deterministic} algorithm considered by \cite{lund1999paging}, namely the \textit{median} algorithm. At any cache miss, the median algorithm evicts the page in cache that has the largest median time of next request. For deterministic algorithms satisfying the condition of \Cref{lem:lpr}, \cite{lund1999paging} employ a slightly more specialized analysis of the charging scheme in \Cref{fig:charging-scheme} (Lemma 2.3 in \cite{lund1999paging}) to obtain a $(c+1)$-competitive guarantee against $\opt$ (instead of $2c$-competitiveness). Thereafter, by arguing that the median algorithm satisfies the condition with $c=4$, (Theorem 2.4 in \cite{lund1999paging}), they are able to show that the median algorithm is 5-competitive against $\opt$. Our tighter analysis in \Cref{lem:tight-analysis} applies to any algorithm (deterministic/randomized), and hence also improves the guarantee for deterministic algorithms obtained by \cite{lund1999paging}. More importantly, it improves the performance guarantee that we can state for the median algorithm.

\begin{corollary}
    \label{corollary:median-is-4-competitive}
    The median algorithm is 4-competitive against $\opt$.
\end{corollary}
