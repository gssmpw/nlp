\section{Conclusion}
\label{sec:conclusion}

% we propose reducing the refresh latency for RowHammer-preventive refresh  to reduce the performance overheads of existing RowHammer defenses. To this end, 
\om{8}{We} present the first rigorous experimental study on the \om{1}{effects of reduced refresh latency on} RowHammer vulnerability \om{5}{of modern DRAM chips}. \yct{8}{\om{8}{Our} analysis of \param{\nCHIPS{}} real DDR4 DRAM chips from three major manufacturers \yct{7}{demonstrate that charge restoration latency of \om{8}{RowHammer-}preventive refreshes can be significantly reduced}
% observe that potential victim rows can be preventively refreshed at significantly lower 
% latency 
at the expense of requiring slightly more preventive refreshes.}
% To leverage this observation, we \yct{20}{reduce the \yct{7}{preventive} refresh latency of existing RowHammer mitigation mechanisms \om{1}{by developing a new mechanism, \emph{\Xlong{} (\X{})}.}}
% , a memory controller-based low-cost mechanism that \agy{1}{reduces} the refresh latency of existing RowHammer solutions. 
% Our results show that by reducing the execution time spent on preventive refreshes, \X{} significantly improves system performance and energy efficiency of five state-off-the-art RowHammer mitigation mechanisms with small additional area overhead. Thus, \X{} enables new system design points in RowHammer mitigation: i)~performance and energy, and ii)~area overhead trade-off space.
\yct{7}{We propose a new mechanism, \emph{\Xlong{}} (\emph{\X{}}), which \om{8}{robustly} reduces the preventive refresh latency in existing RowHammer mitigation mechanisms. Our evaluation demonstrates that \X{} significantly enhances system performance and energy efficiency \om{8}{when used with} five state-of-the-art RowHammer mitigation mechanisms while introducing small additional area overhead. By doing so, \X{} enables new trade-offs in RowHammer mitigation.}
\yct{7}{We hope and expect that the understanding we develop via our rigorous experimental characterization and the resulting \X{} mechanism will inspire DRAM manufacturers and system designers to efficiently and scalably enable robust \om{8}{and efficient} operation as DRAM technology node scaling exacerbates read disturbance.}
% \ieycomment{7}{effectively? do other works use "scalably"? If so, then OK.}

\section*{Acknowledgments} {
We thank the anonymous reviewers of \om{8}{HPCA 2025 (both main submission and artifact evaluation), MICRO 2024, and ISCA 2024} for the encouraging feedback.
We thank the SAFARI Research Group members for valuable feedback and the stimulating scientific and intellectual environment.
We acknowledge the generous gift funding provided by our industrial partners (especially Google, Huawei, Intel, Microsoft, VMware), which has been instrumental in enabling the research we have been conducting on read disturbance in DRAM since 2011~\cite{kim2023flipping}.
This work was also in part supported by the Google Security and Privacy Research Award, the Microsoft Swiss Joint Research Center, \yct{3}{and the ETH Future Computing Laboratory (EFCL)}\yctcomment{5}{Oguz Hoca requested to include EFCL}.
}
\yctcomment{8}{Double-checked all references}