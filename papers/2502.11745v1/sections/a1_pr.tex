
\section{\yct{7}{\X{} \agy{7}{for Other \om{10}{Types of}\\Charge Restoration}}}
% \omcomment{7}{Did prior works not propose partial charge restoration. How is this different from those works?}
% \yctcomment{7}{I couldn't find any prior work that uses reduced latency for all periodic refreshes.}
\cql{GC1.3}{}\agy{1}{\X{} can be extended to \om{7}{also} reduce the \om{7}{charge} restoration latency for \om{11}{\emph{periodic refreshes and dynamic accesses}}\om{7}{, in addition to preventive refreshes (as we \om{10}{have done} so far).}}
To \om{10}{exemplify} the potential benefits of \agy{1}{such extensions}, 
\yct{7}{we analyze the impact of reducing charge restoration latency for periodic refreshes, which restore every cell's charge in the module every \gls{trefw} to prevent data retention failures, similarly to prior work~\cite{das2018vrldram}.
% \yct{10}{VRL-DRAM~\cite{das2018vrldram} profiles each DRAM row and calculates \gls{thpcrrow}, similar to \X's \gls{thpcr} profiling for a DRAM module. To guarantee that a row does \emph{not} receive more than \gls{thpcrrow}, VRL-DRAM implements a counter table that counts each row's partial periodic refresh count. When the memory controller performs a periodic refresh to a specific DRAM row, VRL-DRAM checks that row's counter and performs partial periodic refresh if the counter value is lower than the row's \gls{thpcrrow}. If the counter value is equal to the row's \gls{thpcrrow}, VRL-DRAM performs periodic refresh using nominal latency and resets the counter to zero. Unlike VRL-DRAM, \X{} uses a single \gls{thpcr} value for a DRAM module or chip, significantly reducing hardware cost.}
\yct{10}{To do so, we modify \X{} such the latency of \emph{all} periodic refreshes is reduced since our experimental characterization demonstrates that charge restoration latency can be reduced significantly without causing any data retention failures (\takeref{take:retention}).}
\yct{7}{We use a configuration that does \emph{not} employ any RowHammer mitigation mechanism (i.e., no preventive refreshes are issued) and sweep the \agy{7}{charge restoration latency for periodic refreshes.}}}
% \agycomment{7}{is this correct?}\yctcomment{7}{no}
% periodic refresh latency.} 
% We modify \X{} to perform periodic refreshes with reduced \yct{7}{charge restoration} latency \yct{7}{and compare to the baseline system that uses nominal charge restoration latency for periodic refreshes.} 


\srev{\figref{fig:periodic_ref} demonstrates \yct{7}{multi-core system} performance \sql{R3}(left subplot) and DRAM energy consumption (right subplot) of \X{} with reduced periodic refreshes \yct{7}{and the baseline system with nominal periodic refreshes} normalized to \yct{7}{a \agy{7}{hypothetical} system \agy{7}{which}
% \ieycomment{7}{nominal and reduced periodic refresh latencies? Onur said something here that I cannot fully parse (are you 'XXXXX' periodic refresh??}\yctcomment{7}{I changed here and the previous paragraph, I guess it is clear now, isn't it?} 
% where the system 
does \emph{not} perform any periodic refreshes} (y-axis) for different DRAM chip densities} (x-axis). \yct{7}{As DRAM chip density increases, the number of rows per bank increases. Hence, the number of rows refreshed with a single periodic refresh and the periodic refresh latency increase~\dramStandardCitations{}.} Six curves represent different \yct{7}{periodic} refresh latencies normalized to the nominal \yct{7}{periodic} refresh latency, the baseline system with nominal refresh latency is where \yct{7}{periodic} refresh latency is 1.00 \yct{7}{(marked with $\bm{\times}$)}. \yct{7}{The shades show the variation across tested workloads.}
% \agycomment{7}{I thought this also included reducing tRAS for dynamic accesses. If so, the legend should be Charge restoration latency}
% \agycomment{7}{not clear what happens when the density increases to cause performance and energy overheads. Do we increase the number of rows?}
% \agycomment{7}{instead of improving the text and saying that shades show the variation across workloads, you choose to drop it. Don't do that. Bring the shades back with 100\% confidence interval and say that they show the variation across tested workloads.}

\begin{figure}[ht]
\centering
\includegraphics[width=0.9\linewidth]{figures/fig19_periodic.pdf}
\caption{\yct{7}{System performance (left) and energy consumption (right) versus different DRAM chip capacities for different periodic refresh latencies}}
\label{fig:periodic_ref}
\end{figure}

% \agycomment{7}{I think all these best-observed ones should be best-observed (with a '-'). Check with grammarly and apply everywhere.} 
\srev{We make \param{three} observations from \figref{fig:periodic_ref}. 
\yct{7}{
First, \yct{7}{\X{} significantly improves system performance and energy efficiency for all DRAM chip densities and reduced periodic refresh latencies (i.e., $<1.00$) 
\agy{7}{compared to} \yct{7}{the baseline system that uses nominal charge restoration latency\yct{7}{, across all tested workloads.}} 
% \ieycomment{7}{in all tested multi-core workloads?}
For example, reducing periodic refresh latency by 64\% (i.e., \Xh's best-observed
charge restoration latency) improves multi-core system performance and energy efficiency by 23.31\%, and 36.49\% of a \param{512Gb} DRAM chip, respectively. 
Second, performance improvement and energy efficiency increase as periodic refresh latency decreases. For example, system performance increases by 8.83\% when periodic refresh latency decreases from 64\% to 36\%.}}
\yct{7}{Third, for all periodic refresh latency values, periodic refresh overheads increase as DRAM chip capacity increases.}
Based on these observations, we conclude that reducing the latency of periodic refreshes \agy{1}{improves \X{}'s performance benefits.}}

\yct{10}{To robustly employ this extension \om{11}{of \X{} to periodic refreshes}, \X{} needs to guarantee that a DRAM row does \emph{not} receive \gls{thpcr} consecutive periodic refreshes using reduced charge restoration latency. As each DRAM row is refreshed with periodic refresh once every refresh window (\gls{trefw}), \X{} can safely use reduced charge restoration latency for \gls{thpcr} refresh windows. After periodically refreshing each DRAM row using reduced charge restoration latency for \gls{thpcr} times, \X{} uses nominal charge restoration latency for the next refresh window to fully restore each cell's charge. To do so, \X{} implements a single counter that counts the number of refresh windows using reduced charge restoration latency and increments the counter every \gls{trefw}. When the counter reaches \gls{thpcr}, \X{} uses nominal charge restoration latency for that refresh window and resets the counter to zero.}
% \yctcomment{7}{We will add more on this in the extended version.}
% \ieycomment{7}{Instead of leaving this extension to future work, say something like: we provide the evaluation of this extension in the extended version of this paper[CITE] or "..metadata management, which we evaluate in the extended version of this paper[CITE]".}\yctcomment{7}{can we cite extended version before it is published?}\ieycomment{7}{Yes, just create another bib entry. Instead of HPCA put arXiv there. This is what I did (and also RowPress I think). RowPress did a lot of referring to the extended version, so you can even steal some sentences directly from there.}






