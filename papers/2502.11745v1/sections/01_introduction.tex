\section{Introduction}
\label{sec:introduction}

% \outline{DRAM is important, has a lot of problems, we need to protect DRAM.}
{To ensure system \om{1}{robustness (i.e., reliability, security, and safety)}, it is critical to {maintain} {memory isolation: accessing a memory address should \emph{not} cause unintended side-effects on data stored on other addresses. {Unfortunately}, with aggressive technology node scaling, dynamic random access memory (DRAM)~\cite{dennard1968dram}, the prevalent {main} memory technology}, suffers from increased read disturbance: accessing (reading) a DRAM cell degrades the data integrity of other physically close \yct{0}{and} \emph{unaccessed} DRAM cells\om{1}{~\cite{kim2014flipping, kim2023flipping, mutlu2017rowhammer, mutlu2023fundamentally, mutlu2019retrospective}}. \om{1}{This is because reading a DRAM cell causes physically nearby DRAM cells to lose their charge (i.e., charge leakage) and exhibit bitflips.}
\emph{RowHammer} \agy{0}{is a} prime example of such {DRAM read {disturbance} \om{2}{phenomena} {where} a row of DRAM cells (i.e., a DRAM row) can experience bitflips when another physically nearby DRAM row (i.e., aggressor row) is} repeatedly opened (i.e., hammered)~\rhmemisolationrefs{}.}
\om{1}{\emph{RowPress} is another example of DRAM read disturbance \om{3}{phenomenon} where a DRAM row can experience bitflips when another physically nearby DRAM row is kept open for a long time \om{2}{period} (i.e., pressed)~\cite{luo2023rowpress, luo2024rowpress}. \agy{1}{Prior work~\cite{luo2024experimental} shows that combining RowHammer and RowPress significantly exacerbates disturbance on a victim row and significantly reduces the number and duration of row activations needed to induce the first read disturbance bitflip.}}

{Many prior works demonstrate \om{1}{security} attacks on a wide range of systems \om{2}{that} exploit read disturbance {to escalate {privilege}, leak private data, and manipulate critical application outputs~\om{1}{\exploitingRowHammerAllCitations{}}. \yctcomment{2}{includes RH-RP work}
% \footnote{A survey of RowHammer studies and attacks can be found in~\cite{mutlu2019rowhammer}.}
To make matters worse, recent experimental studies\om{1}{~\understandingRowHammerAllCitations{}}\yctcomment{1}{HBM papers} have found that {newer DRAM chip generations are more susceptible to read disturbance}. {For example,} DRAM chips manufactured in {2018-2020} can experience RowHammer {bitflips} after an order of magnitude fewer row activations compared to the chips manufactured in {2012-2013}~\cite{kim2020revisiting}. \om{1}{Recent studies~\cite{olgun2023hbm, olgun2024read} also show that \agy{1}{modern} HBM2 DRAM chips are \agy{1}{as} vulnerable to both RowHammer and RowPress \agy{1}{as modern DDR4 and LPDDR4 DRAM chips}.}

\om{1}{To ensure robust DRAM operation, several prior works~\refreshBasedRowHammerDefenseCitations{} \agy{1}{propose to} perform an operation called \emph{preventive refresh}. Preventive refresh restores the \agy{1}{electrical} charge \agy{1}{stored in potential victim cells} (i.e., \om{2}{performs} charge restoration). This operation is performed before 
\agy{1}{the victim row gets disturbed significantly enough to experience bitflips due to RowHammer or RowPress. For precise timing of preventive refreshes, these works define a threshold called \emph{RowHammer threshold}, which represents the minimum hammer count needed to induce the first bitflip in a victim row.}
Existing RowHammer mitigation \yct{9}{mechanisms} can also prevent RowPress bitflips when they are configured to be more aggressive, which is practically equivalent to configuring them for smaller RowHammer thresholds~\cite{luo2023rowpress, luo2024rowpress, luo2024experimental}.}
% the aggressor row is activated more than a threshold value called \om{1}{\emph{RowHammer threshold}}, mitigating the risk of bitflips in nearby victim rows.
% \agy{2}{To ensure \om{1}{robust} operation, several prior works~\refreshBasedRowHammerDefenseCitations{} perform an operation called \om{1}{\emph{preventive refresh}} \yct{2}{before the aggressor row is activated more than a threshold value called \om{1}{\emph{RowHammer threshold}}}, where they restore the charge {(i.e., perform charge restoration)} in potential victim rows before a bitflip occurs.} 


{Charge restoration of a DRAM cell has a non-negligible latency\om{1}{, denoted by the (LP)DDRx timing parameter $t_{RAS}$}, which is typically $32-35 ns$~\dramStandardCitations{}. 
During charge restoration of cells in a DRAM row, the DRAM bank that contains the row becomes unavailable, and thus refreshing a DRAM row can \om{1}{delay} memory accesses and induce performance overheads.}
\om{1}{As read \om{2}{disturbance} in DRAM chips \om{3}{worsens}, RowHammer mitigation \om{1}{mechanisms} face a trade-off between system performance and area overhead. Some solutions rely on \emph{more aggressive} preventive refreshes, resulting in higher performance overheads, but lower area overheads (i.e., \emph{high-performance-overhead mitigations}), while other solutions introduce larger area overheads to detect a RowHammer attack more precisely, but incur lower performance overheads (i.e., \emph{high-area-overhead mitigations})~\rowHammerDefenseScalingProblemsCitations{}.}
% \yctcomment{1}{We added this to emph our benefit with low improvements.}
% \srev{As read disturbance in \sql{R2.1}DRAM chips worsens,  RowHammer mitigation mechanisms perform preventive refresh operations \om{1}{\emph{more aggressively}} and incur large performance overheads with low area overhead (i.e., high-performance-overhead mitigations) while other solutions introduce large area overheads to detect \om{1}{a} RowHammer attack more precisely \om{1}{but} incur low performance overheads (i.e., high-area-overhead mitigations)~\rowHammerDefenseScalingProblemsCitations{}.}}\yctcomment{1}{We added this to emph our benefit with low improvement.}
% {becomes} more expensive in terms of performance overhead, energy consumption, and hardware complexity
% \agy{1}{This is because, when configured for DRAM chips with larger RowHammer vulnerability, existing RowHammer mitigation mechanisms become more aggressive in performing their countermeasures. }


\srev{To mitigate read disturbance at \emph{both} low performance and area overheads, it is {critical} to understand
{the \om{1}{effects of the }refresh 
% (i.e., the main cause of the performance overheads induced by RowHammer mitigations)
on} {read disturbance in DRAM chips}. 
\om{2}{This understanding can lead to} more \om{2}{robust solutions for} current and future DRAM-based {memory} systems.}} 
\agy{3}{Many prior works~\understandingRowHammerAllCitations{} study the characteristics of DRAM read disturbance in various aspects.} \yct{1}{However, }\agy{0}{even though charge restoration and read disturbance \om{1}{both greatly affect} charge leakage in DRAM cells,
\emph{no} prior work rigorously studies the interaction between them and \om{2}{the} implications \om{2}{of this interaction} on RowHammer \yct{9}{mitigation mechanisms} \om{1}{and attacks}.}

\om{1}{\textbf{Our goal} is to enable RowHammer mitigation at \emph{both} low performance and area overheads by understanding RowHammer under reduced refresh latency \om{1}{in real DRAM chips}.}
% \agy{3}{2)~}safely reduce the time spent on \agy{3}{and thus the performance overhead of} preventive refreshes.
% {this is not newly added. I just highlighted to emph our goal and novelty.}
% \agycomment{3}{I moved it down. It was incorrectly placed.}
\yct{20}{To this end}, we present \yct{1}{i)} the first rigorous experimental study on the interactions between refresh latency and read disturbance characteristics of real modern DRAM chips~\yct{20}{and ii) demonstrate the potential benefits of our empirical observations by \om{1}{developing} a mechanism, \om{1}{\emph{\Xlong{} (\X{})}}. 
\om{2}{The key idea of \X{} is to reliably reduce} the latency of the preventive refreshes \om{1}{in} both high-performance-overhead and high-area-overhead RowHammer mitigation \om{1}{mechanisms}\om{2}{, thereby improving system performance and energy efficiency.} By doing so, \X{} introduces a novel perspective \om{1}{into} addressing the RowHammer vulnerability while incurring \yct{7}{small} additional area overhead over existing RowHammer mitigation \om{1}{mechanisms}.
% By doing so, \X{} enables new system design points in RowHammer mitigation performance and energy vs. area overhead tradeoff space.while introducing negligible additional area overhead
% \ieycomment{20}{Again, do we need to add this?} \yctcomment{20}{not necessarily}
% jeopardizing their security guarantees.
}
% , a new mechanism that dynamically tunes the latency of preventive refresh operations to reduce the performance overhead of the existing RowHammer \agy{1}{mitigation mechanisms} without jeopardizing \agy{1}{their} security guarantees.}}

\head{\om{1}{Experimental} characterization} We \agy{1}{present }the first rigorous experimental characterization \agy{1}{study} \om{1}{that examines the effects of refresh latency on RowHammer vulnerability, on \nCHIPS{} real DDR4 DRAM \agy{1}{chips}.} 
% \om{1}{We \om{2}{draw} two key takeaways from our experimental characterization. 
\om{2}{Our experimental characterization demonstrates that the latency of preventive refreshes can be safely reduced with either \emph{no} change or a relatively small reduction in the RowHammer threshold, and such reduced refresh latency does \emph{not} further exacerbate RowHammer vulnerability beyond the initial degradation, even after thousands of preventive refreshes.}
\om{2}{Therefore, the latency of a vast majority of preventive refreshes can be reduced reliably without jeopardizing the data integrity of a DRAM chip.}
% \om{2}{Second, reducing the refresh latency can be safely performed multiple times (e.g., 15K times for Mfr. S) without exacerbating RowHammer vulnerability, up to a safe point, while performing reduced refresh more times than this point may disrupt the cell's charge integrity and result in bitflips.}
% \om{2}{Therefore, the latency of a vast majority of preventive refreshes can be reduced reliably without jeopardizing the data integrity of a DRAM chip.}
}
% \agy{1}{\yct{1}{Second}, repeatedly performing partial charge restoration \om{1}{on a DRAM row} causes a reduction in the RowHammer threshold that saturates after a few repetitions of \om{1}{partial charge restorations}. Therefore, \om{1}{a DRAM row can be refreshed using reduced preventive refresh latency thousands of times without further worsening RowHammer vulnerability.}}
% We \agy{1}{perform this study, conducting \yct{1}{two} sets of experiments, which result in the following \param{two} key takeaways.}}
%\agy{2}{First, hammering a DRAM row reduces the row's data retention time and thus can eventually cause bitflips due to exacerbated charge leakage even if the read disturbance caused by hammering is \emph{not} strong enough to induce a bitflip by itself.\atbcomment{1}{Some possible reader questions: What is a data retention bitflip? What does it mean to reduce a row's retention time? This sounds all logical. Why is this a new finding?}\agycomment{2}{revised this part to address Ataberk's comments} Therefore, the RowHammer vulnerability of a DRAM row should be measured while accounting for the reduced data retention time due to hammering rows, as opposed to the methodology proposed by prior works~\cite{kim2014flipping, kim2020revisiting, orosa2021deeper, yaglikci2022understanding}, where the victim row is read immediately after the RowHammer test \emph{without} taking the effect on data retention time into account.}
% demonstrate the interactions between data retention time, refresh, and RowHammer vulnerability using two sets of studies.
% First, we study the effect of data retention time on RowHammer vulnerability. Our analysis shows that data \yctcomment{1}{What does our analysis show at the end?}\agycomment{just rephrase takeaway 1 in a shorter and slightly more abstract way.}

\head{\yct{20}{Potential benefits of reduced refresh latency}} 
\yct{20}{We showcase the benefits of our empirical observations \om{1}{by developing} a mechanism, \om{1}{\emph{\Xlong{} (\X{})}}}.
% \om{1}{\X{} reducing the latency of preventive refresh operations performed by RowHammer mitigation mechanisms. By doing so, \X{} partially restores the charge of the victim cells.} 
\X{} is implemented in the memory controller together with existing RowHammer \yct{9}{mitigation mechanisms}. 
\om{1}{\X{} carefully \om{2}{and reliably} reduces the latency of preventive refreshes, restoring the charge of the potential victim cells partially (i.e., partial charge restoration).
To ensure secure operation, \X{} adjusts the aggressiveness of \om{2}{a} RowHammer \yct{9}{mitigation mechanism} (i.e., configures the existing \yct{9}{mechanism} with the reduced RowHammer threshold) \om{2}{as some DRAM chips exhibit an increased vulnerability to RowHammer when they are refreshed using reduced refresh latency}.} 
\om{1}{To configure \X{}, we use our experimental data (i.e., reduced charge restoration latency, RowHammer threshold) from real DRAM characterization.}
We demonstrate that \X{} significantly increases the system performance~\yct{20}{(energy efficiency)} by reducing the overheads induced by \yct{20}{five} state-of-the-art RowHammer \yct{9}{mitigation mechanisms},
PARA~\cite{kim2014flipping}, 
RFM~\cite{jedec2020ddr5},
PRAC~\cite{jedec2024ddr5},
Hydra~\cite{qureshi2022hydra}, and 
Graphene~\cite{park2020graphene}
by \yct{20}{18.95\% (14.59\%), 12.28\% (11.56\%), 2.07\% (1.15\%), 2.56\% (2.18\%), and 5.37\% (4.50\%)} on average \yctcomment{7}{checked all numbers}across \param{62} workloads \om{2}{while introducing \om{3}{small} additional area overhead \om{3}{in the memory controller} (0.09\% of the area of a high-end Intel Xeon processor~\cite{wikichipcascade})}.

We make the following contributions:

\begin{itemize}

    \item We present the first rigorous characterization of the \om{1}{effects of reduced refresh latency on} RowHammer vulnerability \om{1}{in real DRAM chips}. Our experimental results on \param{\nCHIPS{}} real DRAM chips \om{1}{from three major manufacturers show that \om{2}{latency of a vast majority of preventive refreshes can be reduced reliably}.}
    
    \item \om{1}{To demonstrate the potential benefits of our experimental observations, we propose \om{1}{\emph{\Xlong{} (\X{})}}, a \om{1}{new} mechanism} that \om{2}{reliably} reduces the latency of preventive refreshes \om{3}{performed by} RowHammer mitigation mechanisms\om{3}{, and \om{6}{accordingly} adjusts the aggressiveness of RowHammer mitigation mechanisms}.
    % \yctcomment{3}{These are two separate things that PaCRAM does.}

    \item \ssrev{We showcase \X{}'s \om{1}{effectiveness by evaluating its} impact using five state-of-the-art RowHammer mitigation mechanisms. \om{1}{Our results} show that \X{} significantly reduces \agy{20}{their} performance \yct{20}{and energy} overheads while introducing \om{3}{small additional area overhead in the memory controller.}}

\end{itemize}