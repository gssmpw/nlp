\begin{abstract}

\agy{2}{Read disturbance in modern DRAM chips is a widespread weakness that is used for breaking memory isolation, one of the fundamental building blocks of \om{1}{system} security and privacy.}
RowHammer is a prime example of read disturbance in DRAM where repeatedly accessing (hammering) a row of DRAM cells (DRAM row) induces bitflips in physically nearby DRAM rows {(victim rows)}. 
% by worsening their volatility and reducing their data retention time. 
Unfortunately, shrinking technology node size exacerbates \agy{2}{RowHammer}
% in DRAM chips over generations 
{and} as such, significantly fewer accesses can induce bitflips \om{1}{in} newer DRAM chip generations. To ensure \om{1}{robust} DRAM operation, state-of-the-art mitigation mechanisms \agy{2}{restore the charge in potential victim rows (i.e., \om{2}{they perform} preventive refresh or charge restoration).} 
\srev{With newer DRAM chip \sql{R2.1}generations, these mechanisms perform preventive refresh more aggressively and cause larger performance, energy, or area overheads.}
\yct{20}{Therefore, it is essential to develop a better understanding and in-depth insights into the preventive refresh to secure real DRAM chips at low cost.}


\yct{20}{In this paper, our goal is to \om{2}{mitigate RowHammer at low cost by understanding} the preventive refresh latency and the impact of reduced refresh latency on RowHammer. To this end, we present the first rigorous experimental study on the interactions between refresh latency and RowHammer characteristics in real DRAM chips. Our experimental characterization \iey{20}{using} 388 real DDR4 DRAM chips from three major manufacturers demonstrates that a preventive refresh latency can be significantly reduced (by 64\%) at the expense of requiring slightly more (by 0.54\%) preventive refreshes.
% \agy{20}{effectively reducing overall time spent for refreshes by 62\%}.
% \agycomment{20}{So, time spent for refresh operations can be reduced down to $0.36\times1.054$ of the baseline?}
\srev{To \sql{R2.2}investigate the impact of reduced \yct{9}{preventive} refresh latency on system performance and energy efficiency, 
% To demonstrate the potential benefits of our empirical observations, 
we reduce the \yct{9}{preventive} refresh latency and adjust the aggressiveness of existing RowHammer solutions \om{1}{{by developing a new}} mechanism, \Xlong{} (\X{}).}
% \ieycomment{20}{Hmm. to even highlight the characterization part more we can say: By leveraging our X observations and Y takeaways from extensive experiments on real DRAM chips, we demonstrate the performance benefits of .....} 
\ssrev{Our results show that by reducing the \yct{9}{preventive} refresh latency, \X{} reduces the performance and energy overheads induced by \om{1}{five} state-of-the-art RowHammer \yct{9}{mitigation mechanisms} with \yct{7}{small} additional area overhead.
Thus, \X{} introduces a novel perspective \om{1}{into} addressing RowHammer vulnerability at low cost by leveraging our experimental observations.}} To aid future research, we open-source our \X{} implementation at \url{https://github.com/CMU-SAFARI/PaCRAM}.\yctcomment{2}{Repo is ready, can be made public.}
% \ieycomment{20}{This is dangerous. Now you are claiming that you do not have reliability issues. I believe you do not need this claim. You simply show the "benefits". This is your limitation, you cannot guarantee that there will not be any bitflips.}


% \ieycomment{20}{I believe you are not going to like it but maybe we can play with the wording here. Instead of "propose" we can maybe say "demonstrate the potential benefits of reduced refresh latency in X mitigation mechanisms." I will think about it more and come up with a paragraph.}
% \agy{2}{To reduce this overhead without sacrificing reliability, security, and safety, our goal is to reduce time spent for preventive refreshes and understand this reduced time's impact on RowHammer.}
% % Therefore, it is important to reduce \agy{2}{time spent for preventive} refreshes \agy{2}{and understand this reduction's impact on RowHammer's characteristics} to \yct{1}{achieve}
% %ensure reliable, secure, and safe operation at 
% % low performance overhead \yct{1}{without sacrificing \agy{2}{reliability, security, and safety.}}
% % \agy{2}{To this end,} we propose reducing the refresh latency for RowHammer-preventive refreshes to reduce their performance overheads. 
% To this end, we present the first rigorous experimental study on the interactions between refresh and RowHammer characteristics in real DRAM chips.
% We test \param{\nCHIPS} DDR4 DRAM chips from three major manufacturers and observe that \agy{2}{a preventive refresh operation's latency can be significantly reduced}  
% (by \param{64\%}) at the expense of requiring slightly more (by \param{0.54\%}) preventive refresh operations.\atbcomment{1}{It is not clear here (from the paragraph) if we reduce refresh latency per row, or for every row in the chip by the same amount. Not sure if this needs to be clear.}\agycomment{1}{We do not do it per-row basis. We find a safe latency value for all rows and apply it to all.} To leverage this observation, we propose \Xlong{} (\X{}), a memory controller-based low-cost mechanism that 
% dynamically tunes 
% the refresh latency and \yct{1}{adjusts} the aggressiveness of existing RowHammer \agy{2}{solutions}. \X{} significantly improves system performance by reducing the overhead induced by \yct{1}{four} \agy{2}{RowHammer solutions} by \param{8.74\%/13.99\%/1.05\%/2.62\%}, on average.
%and \agy{2}{significantly} improves system performance.
% by \param{5.3\%}, \param{6.2\%}, and \param{10.0\%} for three state-off-the-art solutions: Graphene, Hydra, and PARA, respectively.
%\yct{2}{on averaged across three state-off-the-art RowHammer solutions}.\agycomment{2}{Yahya please revise.}

\end{abstract}