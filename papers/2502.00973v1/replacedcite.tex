\section{Literature Review}
\label{sec.literature_review}

Professor E. Rafailov's research group at Aston University has developed LDF/FS wearable devices using VCSEs, showing comparable signal responses to conventional monitors in volunteer assessments ____. These devices employ LDF and FS for non-invasive early detection of vascular complications in diabetes and other conditions. LDF assesses tissue perfusion, oxygen saturation, and blood volume, while FS detects metabolic activity changes and AGEs accumulation, contributing to microvascular damage and inflammation in diabetes. 

LDF is a non-invasive method for estimating perfusion in the microcirculation ____. Introduced over 30 years ago, the technique uses laser radiation to probe tissue and analyze backscatter from moving red blood cells, primarily Hemoglobin (Hb). The main parameter recorded is the microcirculation or perfusion index, essential for organ nutrition, adaptation, and regulation. The method uses wavelet transformation, specifically adaptive wavelet analysis with complex-valued Morlet wavelets, to assess microvessel oscillatory processes over a wide frequency range. This has been the standard for over 15 years, replacing Fast Fourier Transform (FFT) and Butterworth filters ____. Continuous wavelet transformation is preferred for non-stationary LDF-gram (perfusion) due to its optimal ``time-frequency'' resolution, effectively tracking frequency and amplitude fluctuations in blood flow signals ____. The FS method uses laser probing to record fluorescence spectra of metabolic coenzymes, measuring NADH and FAD fluorescence intensity. This detects changes in metabolic activity in endothelial cells, indicating various physiological and pathological processes, and identifying cellular metabolic disorders related to diseases ____. 



Several studies have utilized wearable devices to assess blood microcirculation across diverse patient groups. Older adults typically exhibit higher perfusion levels in areas like the middle palm and dorsal forearm due to thinner skin, aiding in diagnostic precision ____. Conversely, younger individuals often show elevated wavelet parameters in blood perfusion oscillations, suggesting broad applicability in various pathologies. In endocrinology, wireless LDF devices have been used to evaluate microcirculatory function in type 2 diabetes patients and healthy individuals across different age brackets, revealing significant variations in perfusion levels ____. Notably, studies monitoring diabetes patients receiving intravenous alpha-lipoic acid therapy have shown improvements in microcirculatory and nutritional blood flow, particularly in limbs affected by diabetic complications ____. Additionally, wearable LDF devices have been instrumental in diagnosing vascular disorders during COVID-19 recovery, highlighting disruptions in microcirculatory function ____.

Further related works are described in Appendix Section \ref{sec.full_literature_review}.