\section{Full Literature Review}
\label{sec.full_literature_review}

Global health faces dual challenges from infectious diseases like COVID-19 and rising non-communicable diseases (NCDs). The World Health Organization (WHO) report highlights that the COVID-19 pandemic has caused significant disruptions in chronic disease services. Specifically, 53\% of countries reported disruptions in hypertension treatment, 49\% in diabetes care, 42\% in cancer treatment, and 31\% in cardiovascular emergency services. Additionally, over 50\% of countries postponed public screening programs for breast and cervical cancer due to the reassignment of healthcare staff to COVID-19 duties and the cancellation of planned treatments \cite{world2020impact, restrepo2008medication, yonel2018patients, mularczyk2022preventive}.
The use of remote monitoring devices without intervention is crucial to aid patients and healthcare professionals in timely classification and treatment. These devices can continuously monitor vital health indicators, detect abnormalities early, and alleviate the burden on the healthcare system.

\textbf{A promising new wearable technology: }

Fortunately, advancements in technology offer promising solutions. Professor E. Rafailov's research group at Aston University has made significant strides in developing LDF/FS wearable devices \cite{inproceedings19}. They developed the devices using VCSEs, which demonstrated signal responses comparable to conventional tabletop monitors through volunteer-based assessments. These devices use Laser Doppler Flowmetry (LDF) and fluorescence spectroscopy (FS) for non-invasive early detection of vascular complications in diabetes and other conditions. LDF assesses tissue perfusion, oxygen saturation, and blood volume by analyzing backscatter from red blood cells using near-infrared and infrared light. FS complements LDF by detecting metabolic activity changes and advanced glycation end-products (AGEs) accumulation in diabetes, which contributes to microvascular damage and inflammation. % . 

\textbf{Specific details of analyzing the microcirculation using LDF and wavelet transformation: }

Currently, several principal frequency bands are distinguished in microvascular oscillations, reflecting various regulatory mechanisms: endothelial 0.0095–0.02 Hz, neurogenic 0.02–0.06 Hz, myogenic 0.06–0.16 Hz, respiratory 0.16–0.4 Hz, and cardiac 0.4–1.6 Hz \cite{bagno2015wavelet}. 

\textbf{Additional capabilities of Fluorescence Spectroscopy (FS): }

Additionally, other structural proteins of capillary and skin membranes also exhibit fluorescence: pentosidine residues formed during collagen glycation. Pathogenic factors such as hyperglycemia and oxidative stress in diabetes lead to increased protein glycation and accumulation of advanced glycation end products (AGEs), affecting the properties of collagen with specific wavelengths of light. This can be used to study skin fluorescence related to AGE accumulation, which is associated with the accumulation of these substances. 

Currently, there is a growing interest in wearable electronic diagnostic devices because daily monitoring of parameters promises a new quality of diagnosis. Recently, multimodal approaches have been actively developed, allowing clinicians to obtain in vivo values of physiological and biochemical parameters, as well as to comprehensively assess the viability of the subcutaneous microcirculatory system. One of the first developments of wearable devices for estimating subcutaneous microcirculatory tissue system (MTS) parameters is the ``LAZMA PF'' analyzer, produced by Aston Medical Technology Ltd., UK, under the name ``FED-1B''. This device integrates a multimodal approach, specifically including 2 channels for laser Doppler flowmetry (LDF) and fluorescence spectroscopy (FS), designed into a new device named MDFED-2B\footnote{\url{https://amedtech.co.uk/product/mfed-2b/}}.

This technology is renowned for its non-invasive measurement capabilities in living tissues. Studies have been conducted at various sites such as the wrist, ankle, thigh, and fingertips. It has many applications, including research on metabolic and vascular complications of diabetes, automatic cerebral vascular analysis, and monitoring cerebral circulation in both healthy individuals and those with disorders. It provides continuous, non-invasive monitoring during diagnostic, treatment, and post-treatment phases. Spectral characteristic changes have been observed in conditions such as malignant tumors, surgical trauma, increased arterial pressure, and many others. It is also used to assess the functional status of the cerebral vascular system in patients with acute and chronic cerebrovascular disorders.

Monte Carlo modeling has shown a penetration depth of up to 2 mm for the LDF channel (deep vessels) and 1 mm for the FS channel . Sensors can monitor parameters such as perfusion, movement, skin temperature, and metabolic activity, providing crucial information for evaluating various physiological processes. 

\textbf{The technical details of the MDFED-2B wearable device and how it uses its two channels for LDF and FS measurements: }
The two channels combined in the design characteristics of the wearable device are used for multimodal optical diagnostics. A distinctive feature of the wearable devices under consideration is the absence of optical fibers in the design, which reduces common motion artifacts on the fibers. The wearable devices are placed on the skin for direct irradiation from a window on the underside of the device, recording the emitted (secondary) radiation from the biological tissue on the back of the device and transmitting measurement data to a PC via Bluetooth or Wi-Fi protocol. The “MDFED-2B” wearable device with 2 optical diagnostic channels uses an 850 nm VCSEL chip as the single-mode laser source with 0.8 mW power in the LDF channel, directly transmitting radiation to the skin. In the FS channel, a UV 365 nm LED is used, with a pulse power of 1.4 mW and an average power of 0.35 mW to excite endogenous NADH fluorescence at wavelengths between 460-470 nm. The amplitude of NADH fluorescence intensity (ANADH) is normalized to backscattered radiation to reduce the influence of varying blood filling in the biological tissue, which arises, among other reasons, from artifacts related to different pressures on the skin surface. The distance between the windows of the 2 channels is approximately 1 cm (the distance between the radiation source and detector). The placement of wearable devices for MTS diagnostics of the human body and wireless connection to a personal computer or smartphone is shown on symmetric regions of the limbs. Common locations for wearable devices on the body’s biological tissue depend on the diagnostic task: these are usually symmetrical points on the right and left of the upper and lower limbs, areas with direct arterio-venous connections (hand or fingertip) and with predominant nutritional blood flow (forearm or lower leg), and on the forehead at the supraorbital artery regions.

\textbf{The applications of the LDF/FS technology in the MDFED-2B wearable device for various medical conditions: }

Studies have applied this technology to measurements in various patients. Low perfusion parameters were observed in 19 patients with acute ischemic stroke (AIS) in both the affected and unaffected hemispheres, and were lower in patients with chronic cerebrovascular disease \cite{Alexey2017}. It evaluated the severity of microcirculatory and metabolic disorders in 41 rheumatic diseases and 76 diabetes patients. Research on perfusion in patients with joint microcirculatory disorders in the hands \cite{Zherebtsova2019MultimodalOD} . Cardiovascular risk in diabetic and elderly patients. A study on three groups, including 37 diabetic patients, 37 elderly individuals, and 58 young individuals, comparing average perfusion using the LDF (Laser Doppler Flowmetry) method, showed that blood microcirculation index values increase with age and the progression of diabetes \cite{Zherebtsov2023}. There was no statistically significant difference between patients and the older control group in average perfusion. However, the average energy of blood flow oscillations decreased in patients with diabetes in the endothelial, neurogenic, myogenic, and respiratory ranges . In terms of neurogenic and myogenic variability, statistically significant differences were found between the diabetic patient group and the older control group, reflecting the influence of sympathetic nervous distribution and vascular smooth muscle activity \cite{ralevic1995effects}.

In another study, diabetic patients had significantly lower endothelial, neurogenic, and myogenic low-frequency oscillation values compared to healthy controls when measured near the head of the first metatarsal bone.\cite{humeau2004spectral} When measuring on smooth skin, Jan and colleagues also showed reduced neurogenic and myogenic regulation in diabetes in response to heat. According to the authors, such changes in blood flow regulation are due to disruptions in the autonomic component of the peripheral nervous system, causing blood flow to be diverted to shunts.

The most common characteristic of microcirculatory disorders related to diabetes is the dysfunction of smooth muscle, endothelial cells, and perivascular nerves in the periphery, which explains the decrease in low-frequency perfusion oscillation values \cite{Zherebtsov2023}. A study on an animal model of diabetes also showed that neurogenic distribution disorders in peripheral vessels are the primary factor contributing to microcirculatory dysfunction, leading to endothelial dysfunction and impaired smooth muscle function in the vascular system \cite{jan2013skin, Zherebtsova2019MultimodalOD, Zherebtsov:19, inproceedingsage19}. Non-smokers had higher blood perfusion levels compared to smokers, while smokers exhibited greater variation in pulse frequency. These findings suggest that the LDF device is effective in detecting the cardiovascular impacts of smoking and could be useful for monitoring blood microcirculation and related pathologies in smokers. \cite{Saha2020wearable}. The device's advantages are its painlessness, quick results, no need for expensive consumables, and minimal impact on the patient. 

\textbf{The validation steps taken before using the LDF/FS wearable devices (MDFED-2B) for various diagnoses: }

Before deploying these new wearable devices for various diagnostic tasks, their potential for multipoint perfusion measurement was investigated. These devices were used to analyze the synchronization of blood flow on the skin in analogous regions of opposite limbs, both at rest and during various functional tests (occlusion or breath-holding). Studies have shown high synchronization of blood flow rhythms in the opposite limbs of healthy volunteers. The compact and highly sensitive devices can be used even outside clinical settings. Furthermore, these wearable devices show high repeatability of measurements at rest and during physiological tests, enhancing the diagnostic value of the measurements.

\textbf{Findings from an initial study using the LDF/FS wearable devices to investigate blood microcirculation: }

A study using new wearable devices examined blood microcirculation across age groups. Older adults showed higher perfusion levels in the middle palm and dorsal forearm, due to thinner skin reducing laser scattering and increasing diagnostic volume. Younger individuals had higher wavelet parameters in blood perfusion oscillations. These findings can aid in developing MTS study protocols for patients with various pathologies. \cite{Loktionova:19}.


\textbf{How the LDF/FS wearable device was used in a study on diabetes and microcirculation: }

Research using wearable diagnostic devices in endocrinology assessed microcirculatory function in  19 diabetes (DM) type 2 patients and 37 healthy individuals across two age groups. Results showed different average perfusion levels between healthy volunteers of different ages and between younger healthy volunteers and DM patients. Notably, wrist and fingertip perfusion levels in healthy groups showed no significant difference. This pilot study demonstrated that wireless LDF wearable devices are convenient for point-of-care testing, recording age-specific perfusion changes and changes related to diabetes development\cite{Zherebtsov:19}.

\textbf{A study using the LDF/FS wearable device to monitor the effects of alpha-lipoic acid treatment on microcirculation in diabetic patients: }

Another promising use of these wearable devices in endocrinology is monitoring 10 diabetes patients therapy involving intravenous alpha-lipoic acid. Studies showed a decrease in microcirculatory and nutritional blood flow and an increase in shunt blood flow during treatment. After treatment, patients' parameters approximated control group values, particularly in the lower limbs, which are more affected by diabetic complications due to higher stress factors. These changes suggest positive effects of the therapy \cite{article2022}.

\textbf{A study using the LDF/FS wearable device to investigate the impact of pregestational type 1 diabetes on microcirculation in pregnant women at different stages: }

The study examined multimodal wearable diagnostic devices' impact on pregestational type 1 diabetes in pregnant women, showing glucose variability monitoring's role in assessing vascular function and oxidative status. Ten pregnant women (ages 32, 7-22 weeks gestation) and seven healthy women (age 32) were monitored using the ``Libre Freestyle'' system. Results indicated reduced microvascular activity in pregnant patients' legs and increased NADH fluorescence, suggesting tissue respiration decline \cite{inproceedings2021}.

\textbf{A study using wearable LDF devices to analyze blood flow patterns in patients with COVID-19 during different stages of recovery: }

The study demonstrated the use of peripheral blood flow oscillation analysis with wearable LDF devices to diagnose vascular disorders in a COVID-19 patient during early and progressive recovery stages. Results showed a significant increase in neurogenic oscillation amplitude in the upper limbs, potentially leading to arteriolar and venular dilation and microcirculatory blood flow shunting, adversely affecting oxygen delivery and tissue metabolism. Wavelet analysis confirmed changes in average perfusion levels due to blood flow fluctuations, influenced by disease severity and specifics \cite{ diagnostics13050920}.

\textbf{The potential the wearable LDF/FS devices in medical diagnosis: }

Summarizing all the given data, the presented wearable diagnostic devices with 2 optical channels for LDF and FS are a promising approach for evaluating the functional state of MTS. The multimodal approach of using LDF and FS makes it possible to simultaneously obtain physiological and metabolic information, which helps to comprehensively assess the state of the microcirculatory system. The results presented in studies demonstrated the possibility of using wearable devices to obtain objective information on MTS status under normal and pathological conditions. However, more detailed studies with larger patient cohorts and extended analysis of physiological conditions should be conducted for further clinical implementation.

One of the crucial tasks is to investigate the effects of various treatment protocols and lifestyle changes on microcirculatory and metabolic parameters using these wearable devices. Another important direction is developing machine learning algorithms for automated data analysis and interpretation, which could significantly enhance the diagnostic capabilities of wearable devices.

The development and application of wearable diagnostic devices with LDF and FS channels represent a significant advancement in medical diagnostics, offering non-invasive, real-time monitoring of the microcirculatory system and metabolic state. These devices hold great potential for improving patient care, particularly in managing chronic diseases and monitoring treatment efficacy. Our research aims to leverage this technology to build a dataset for stress detection. The volunteers in our work have diverse medical histories, including migraine, diabetes, STEMI, and hypertension. By exploring cutaneous blood microcirculation parameters using a non-invasive wearable device equipped with LDF and FS channels, we can gain valuable insights. To the best of our knowledge, our work pioneers in publishing a large LDF/FS wearable device dataset for mental health assessment.


% Data is being collected from 132 volunteers, 69 males and 64 females aged 18-88 from 19 countries, predominantly Asian (86.4\%), with Caucasians (11.4\%) and Africans (2.3\%). They are participating in a study measuring transcutaneous blood flow using a wearable device with Laser Doppler Flowmetry (LDF) and Fluorescence Spectroscopy (FS) sensors on both hands. 
% Preliminary findings from a study reveal significant variations in vascular responses influenced by demographics and human characteristics. Measurements show that LDF and FS signals, as well as blood perfusion oscillations, are higher on the left hand. Men have higher M blood perfusion indices, although Kv100 values are lower compared to women. Racial analysis shows Caucasians have higher values than Asians. The LDF and FS readings increase with skin pigmentation and age, but Five rhythmic oscillations such as endothelial, neurogenic, myogenic, respiratory, and cardiac (AE, AN, AM, AR, and AC, respectively) tend to decrease post-40-year-old.
% The study highlights the impact of lifestyle on vascular responses: 21 smokers (15.9\%) show lower oscillation indices yet higher oscillation frequencies and LDF and FS indices compared to non-smokers. Additionally, 36 participants (27.3\%) experiencing non-wellbeing display increased LDF and generally higher oscillation amplitudes, indicating stress, while FS remains unchanged.
% The study notes that noncommunicable diseases (NCDs) affect vascular responses: Individuals with diabetes, obesity, hypertension, and hepatitis B have elevated LDF and FS signals, whereas those with a history of STEMI (10 participants, 7.5\%) show lower values. Additionally, all wavelet are reduced in those with chronic conditions, yet respiratory oscillation frequencies (F-M) are increased, significantly among migraine sufferers. During the study, 10 volunteers (7.5\%) who fell asleep showed significant decreases in Fluorescence Spectroscopy (FS) and myogenic oscillation frequency (F-M), while other parameters increased. 
% These findings highlight the device’s capability for early health status prediction and its utility in disease prevention and management, anticipating further benefits from expanded data collection and AI integration.



%The remainder of this paper is structured as follows. Section \ref{sec.study_design_and_data} provides a comprehensive overview of the clinical definitions and study design, and the data collection process, including participant demographics and device specifications. We also present a detailed analysis of the collected dataset, exploring its characteristics and potential challenges. Section \ref{sec.machinelearningandXAI} describes the machine learning models employed for DAS prediction and the XAI techniques utilized to understand their reasoning. Evaluation metrics used to assess the performance of these models are discussed in Section \ref{sec.evaluation}. Section \ref{sec.result} presents the results of our investigation, including the prediction accuracy and insights gleaned from XAI analysis. Finally, Section \ref{sec.conclusion} concludes the paper by summarizing our key findings, discussing limitations, and outlining potential future directions for research.


\newpage
\section{Detailed Study Design and Dataset Description}
\label{sec.appendix_study_design_and_data}

\subsection{Clinical Definition and Data collection}
\label{sec.clinical_definition}

The criteria of this study are described as follows:

\begin{itemize}
     \item This study focuses on volunteers aged 18 and above.
     \item Volunteers must not have any medical conditions related to dermatological diseases on both hands and middle fingers.
     \item  A total of 132 volunteers, aged 18 and above, of all genders, various occupations, and different ages, who are healthy and alert, were included.
\end{itemize}

Table \ref{tab:das21responses} shows the depression, anxiety and stress scale (DAS21) questionnaire responses which we gave to our volunteers.

\begin{table}[h!]
    \centering
    \caption{Depression, anxiety and stress scale (DAS21) questionnaire responses.}
    \setlength{\tabcolsep}{3pt}
    \renewcommand{\arraystretch}{1.6}
    \adjustbox{width=\columnwidth}{
    \begin{tabular}{|c|m{10cm}|c|c|c|c|}
        \hline
        No.& \centering \textbf{Question} & \textbf{0} & \textbf{1} & \textbf{2} & \textbf{3} \\ \hline
        
        1 (s) & I found it hard to wind down & 0 & 1 & 2 & 3 \\ \hline
      
        2 (a) & I was aware of dryness of my mouth & 0 & 1 & 2 & 3 \\ \hline
    
        3 (d) & I couldn’t seem to experience any positive feeling at all & 0 & 1 & 2 & 3 \\ \hline
     
        4 (a) & I experienced breathing difficulty (e.g. excessively rapid breathing, breathlessness in the absence of physical exertion) & 0 & 1 & 2 & 3 \\ \hline
     
        5 (d) & I found it difficult to work up the initiative to do things & 0 & 1 & 2 & 3 \\ \hline
        6 (s) & I tended to over-react to situations & 0 & 1 & 2 & 3 \\ \hline
        
        7 (a) & I experienced trembling (e.g. in the hands) & 0 & 1 & 2 & 3 \\ \hline

        8 (s) & I felt that I was using a lot of nervous energy & 0 & 1 & 2 & 3 \\ \hline
     
        9 (a) & I was worried about situations in which I might panic and make a fool of myself & 0 & 1 & 2 & 3 \\ \hline

        10 (d) & I felt that I had nothing to look forward to & 0 & 1 & 2 & 3 \\ \hline
 
        11 (s) & I found myself getting agitated & 0 & 1 & 2 & 3 \\ \hline
      
        12 (s) & I found it difficult to relax & 0 & 1 & 2 & 3 \\ \hline
    
        13 (d) & I felt down-hearted and blue & 0 & 1 & 2 & 3 \\ \hline
       
        14 (s) & I was intolerant of anything that kept me from getting on with what I was doing & 0 & 1 & 2 & 3 \\ \hline
    
        15 (a) & I felt I was close to panic & 0 & 1 & 2 & 3 \\ \hline
     
        16 (d) & I was unable to become enthusiastic about anything & 0 & 1 & 2 & 3 \\ \hline
    
        17 (d) & I felt I wasn’t worth much as a person & 0 & 1 & 2 & 3 \\ \hline
      
        18 (s) & I felt that I was rather touchy & 0 & 1 & 2 & 3 \\ \hline
    
        19 (a) & I was aware of the action of my heart in the absence of physical exertion (e.g. sense of heart rate increase, heart missing a beat) & 0 & 1 & 2 & 3 \\ \hline
     
        20 (a) & I felt scared without any good reason & 0 & 1 & 2 & 3 \\ \hline
  
        21 (d) & I felt that life was meaningless & 0 & 1 & 2 & 3 \\ \hline

    \end{tabular}
    }
    \label{tab:das21responses}
\end{table}


\subsection{Data Analysis}

Our dataset focuses on understanding the factors influencing depression, anxiety, and stress (DAS) levels. To achieve this, we have collected and integrated data from three key sources: personal information, wearable sensor readings, and the DAS-21 questionnaire. This diverse data representation allows us to create a comprehensive picture of each individual's background, psychological state, and physiological responses.

Similar to a medical record, the dataset includes essential demographic details for each participant. Our research involved a wide range of ages (18-94) with an average participant age of 40. We also ensured participant diversity by including a variety of races (Asian, White, African) and genders (55.2\% male, 44.5\% female), as shown in Fig. \ref{fig.race} and Fig. \ref{fig.gender}, respectively.


\begin{figure}[h]
    \centering
    % First row
    \begin{subfigure}[b]{0.32\textwidth}
        \centering
        \includegraphics[width=\textwidth]{images/race.png}
        \caption{Race}
        \label{fig.race}
    \end{subfigure}
    \hfill
    \begin{subfigure}[b]{0.32\textwidth}
        \centering
        \includegraphics[width=\textwidth]{images/gender.png}
        \caption{Gender}
        \label{fig.gender}
    \end{subfigure}
    \hfill
    \begin{subfigure}[b]{0.32\textwidth}
        \centering
        \includegraphics[width=\textwidth]{images/smoke.png}
        \caption{Smoke}
        \label{fig.smoke}
    \end{subfigure}
    \vspace{1em} % Add vertical space between rows
    % Second row
    \begin{subfigure}[b]{0.32\textwidth}
        \centering
        \includegraphics[width=\textwidth]{images/issue.png}
        \caption{Mental health status}
        \label{fig.issue}
    \end{subfigure}
    \hfill
    \begin{subfigure}[b]{0.32\textwidth}
        \centering
        \includegraphics[width=\textwidth]{images/sleep.png}
        \caption{Sleep during experiment}
        \label{fig.sleep}
    \end{subfigure}
    \hfill
    \begin{subfigure}[b]{0.32\textwidth}
        \centering
        \includegraphics[width=\textwidth]{images/bloodpreseure.png}
        \caption{Blood Pressure}
        \label{fig.bloodpresure}
    \end{subfigure}
    \caption{Data Analysis}
    \label{fig.DataRepresentation}
\end{figure}

In addition to the demographic information, we also investigate the effect of patients' routines and medical history, which is used to further understand potential contributors to DAS. As shown in Fig \ref{fig.smoke}, most of the anticipants do not smoke cigarettes, and the rate of people used to smoke is small compared to smoked people and non-smoke people. A similar observation is seen in the health issues attribute, in which the percentage of people who have problems with health issues is significantly smaller than normal people, as illustrated in Fig. \ref{fig.issue}.


The dataset incorporates data collected from wearable devices worn by participants. As shown in Fig. \ref{fig.sleep}, we even recorded details like sleep patterns (sleeping vs. awake) and the hand used for blood sample collection during the study. These devices continuously monitor various physiological and activity-related aspects, providing real-time health information. Examples of data collected include Body Mass Index (BMI), heart rate, and Blood Pressure. Each data point is linked to a specific time, which allows us to analyze trends and potential correlations between a participant's physiological responses and their mental state over time. The experiment duration is around 15 minutes, at the end of the experiment.


%\begin{figure}
%    \centering
%    \includegraphics[width=1\linewidth]{images/wellbeing-nonwellbeing1.png}
%    \caption{Blood perfusion (M*) with standard deviation and Maximum amplitude with standard deviation of the endothelial (A-E), neurogenic (A-N), myogenic (A-M), breath (A-R) and pulse (A-C) mechanism for Wellbeing vs Non-wellbeing, *, p$<$0.01, Mann-Whitney U test.}
%    \label{fig.wellbeing-nonwellbeing1}
%\end{figure}


\begin{table}[h]
\centering
\setlength{\tabcolsep}{2pt}
\renewcommand{\arraystretch}{3.5}
\caption{Blood perfusion (M*) with standard deviation and Maximum amplitude with standard
deviation of the endothelial (A-E), neurogenic (A-N), myogenic (A-M), breath (A-R) and pulse (A-C)
mechanism for Wellbeing vs Non-wellbeing, *, p<0.01, Mann-Whitney U test.}
\begin{adjustbox}{width=\textwidth}
\begin{tabular}{lcccccccccccccc}
\toprule
\textbf{Subgroup} & \textbf{M\_mean} & \textbf{p-value} & \textbf{A-E\_mean} & \textbf{p-value} & \textbf{A-N\_mean} & \textbf{p-value} & \textbf{A-M\_mean} & \textbf{p-value} & \textbf{A-R\_mean} & \textbf{p-value} & \textbf{A-C\_mean} & \textbf{p-value} \\ \midrule

All & \begin{tabular}{@{}c@{}}22.54 \\ (4.73 - 37.23)\end{tabular} & & \begin{tabular}{@{}c@{}}1.49 \\ (0.34 - 3.36)\end{tabular} & & \begin{tabular}{@{}c@{}}1.46 \\ (0.33 - 3)\end{tabular} & & \begin{tabular}{@{}c@{}}1.17 \\ (0.31 - 2.4)\end{tabular} & & \begin{tabular}{@{}c@{}}0.7 \\ (0.22 - 1.29)\end{tabular} & & \begin{tabular}{@{}c@{}}0.93 \\ (0.39 - 1.68)\end{tabular} & \\

Wellbeing  & \begin{tabular}{@{}c@{}}21.02 \\ (4.73 - 35.59)\end{tabular} & 0.016121 & \begin{tabular}{@{}c@{}}1.44 \\ (0.34 - 3.42)\end{tabular} & 0.171585 & \begin{tabular}{@{}c@{}}1.4 \\ (0.31 - 2.99)\end{tabular} & 0.27169 & \begin{tabular}{@{}c@{}}1.11 \\ (0.29 - 2.35)\end{tabular} & 0.123042 & \begin{tabular}{@{}c@{}}0.66 \\ (0.2 - 1.18)\end{tabular} & 0.0614 & \begin{tabular}{@{}c@{}}0.89 \\ (0.31 - 1.74)\end{tabular} & 0.06069 \\

Non-Wellbeing & \begin{tabular}{@{}c@{}}26.49 \\ (8.59 - 36.71)\end{tabular} & & \begin{tabular}{@{}c@{}}1.62 \\ (0.64 - 2.7)\end{tabular} & & \begin{tabular}{@{}c@{}}1.61 \\ (0.49 - 2.96)\end{tabular} & & \begin{tabular}{@{}c@{}}1.33 \\ (0.5 - 2.32)\end{tabular} & & \begin{tabular}{@{}c@{}}0.81 \\ (0.47 - 1.44)\end{tabular} & & \begin{tabular}{@{}c@{}}1.02 \\ (0.58 - 1.61)\end{tabular} & \\

\bottomrule
\end{tabular}
\end{adjustbox}
\label{tab.Wellbeing1}
\end{table}

\begin{figure}[h]
    \centering
    \includegraphics[width=1\linewidth]{images/wellbeing-nonwellbeing1.png}
    \caption{Blood perfusion (M*) with standard deviation and Maximum amplitude with a standard deviation of the endothelial (A-E), neurogenic (A-N), myogenic (A-M), breath (A-R) and pulse (A-C) mechanism for Wellbeing vs Non-wellbeing, *, p$<$0.01, Mann-Whitney U test.}
    \label{fig.wellbeing-nonwellbeing1}
\end{figure}

Table \ref{tab.Wellbeing1} and Fig. \ref{fig.wellbeing-nonwellbeing1} indicate that non-wellbeing individuals exhibit higher perfusion parameters and amplitude variations compared to wellbing individuals. For the perfusion parameter M, wellbeing individuals have a mean value of 21.02 (95\% CI, 4.73 - 35.59), whereas non-wellbeing individuals have a significantly higher M value of 26.49 (95\% CI, 8.59 - 36.71), (p=0.016, Mann-Whitney U test). The endothelial, neural, muscle, respiratory, and cardiovascular amplitude variations all tend to be higher by more than 0.2, but the differences are not statistically significant.


As shown in Table \ref{tab:Wellbeing_Kv100} and Fig. \ref{fig.wellbeing-nonwellbeing2}, non-wellbeing individuals exhibit a statistically significant higher amplitude of perfusion parameter fluctuations ($\delta$) compared to their wellbeing counterparts, with values of 4.79 (95\% CI, 2.25 - 7.65) versus 3.64 (95\% CI, 0.98 - 6.93), (p=0.007, Mann-Whitney U test). Additionally, they have a significantly higher temperature at the measurement site, 33.21 (30.14 - 35.82) compared to 30.65 (95\% CI, 22.68 - 35.6), (p=0.01, Mann-Whitney U test). Moreover, the metabolic index (POM) at the measurement site is also significantly elevated in non-wellbeing individuals, with values of 11.42 (95\% CI, 1.95 - 24.42) versus 7.7 (95\% CI, 0.85 - 20.43), (p=0.01, Mann-Whitney U test).









\begin{figure}
    \centering
    \includegraphics[width=1\linewidth]{images/wellbeing-nonwellbeing2.png}
    \caption{The parameters with standard deviation for Wellbeing vs Non-wellbeing: Kv100, $\delta$*, T*, A365, A460, Anadn, POM*, F-E; F-N; F-M; F-R; F-C, *, p$<$0.01, Mann-Whitney U test.}
    \label{fig.wellbeing-nonwellbeing2}
\end{figure}



\begin{table}[h]
\centering
\setlength{\tabcolsep}{2pt}
\renewcommand{\arraystretch}{3.5}
\caption{The parameters with standard deviation for Wellbeing vs Non-wellbeing: Kv100, $\delta^{*}$, $T^*$,
A365, A460, Anadn, ${POM}^{*}$, F-E; F-N; F-M; F-R; F-C, *, $p<0.01$, Mann-Whitney U test.}
\begin{adjustbox}{width=\textwidth}
\begin{tabular}{lccccccccccccc}
\toprule
\textbf{Subgroup}  & \textbf{Kv100\_mean} & \textbf{$\delta$\_mean} & \textbf{p-value} & \textbf{T\_mean} & \textbf{p-value} & \textbf{A365\_mean} & \textbf{A460\_mean} & \textbf{Anadn\_mean} & \textbf{POM\_mean} & \textbf{p-value} \\ \midrule

All & \begin{tabular}{@{}c@{}}21.09 \\ (6.86 - 49.55)\end{tabular} & \begin{tabular}{@{}c@{}}3.96 \\ (1.21 - 7.41)\end{tabular} & & \begin{tabular}{@{}c@{}}31.36 \\ (22.95 - 35.79)\end{tabular} & & \begin{tabular}{@{}c@{}}86.82 \\ (4.42 - 158.6)\end{tabular} & \begin{tabular}{@{}c@{}}59.3 \\ (12.92 - 106.52)\end{tabular} & \begin{tabular}{@{}c@{}}1.01 \\ (0.4 - 4.54)\end{tabular} & \begin{tabular}{@{}c@{}}8.74 \\ (0.99 - 22.15)\end{tabular} \\

Wellbeing & \begin{tabular}{@{}c@{}}21.12 \\ (6.71 - 48.75)\end{tabular} & \begin{tabular}{@{}c@{}}3.64 \\ (0.98 - 6.93)\end{tabular} & 0.007252 & \begin{tabular}{@{}c@{}}30.65 \\ (22.68 - 35.6)\end{tabular} & 0.018427 & \begin{tabular}{@{}c@{}}85.43 \\ (9.5 - 130.9)\end{tabular} & \begin{tabular}{@{}c@{}}60.64 \\ (17.2 - 106.8)\end{tabular} & \begin{tabular}{@{}c@{}}1.01 \\ (0.41 - 4.77)\end{tabular} & \begin{tabular}{@{}c@{}}7.7 \\ (0.85 - 20.43)\end{tabular} & 0.010656 \\

Non-Wellbeing & \begin{tabular}{@{}c@{}}21 \\ (7.56 - 48.44)\end{tabular} & \begin{tabular}{@{}c@{}}4.79 \\ (2.25 - 7.65)\end{tabular} & & \begin{tabular}{@{}c@{}}33.21 \\ (30.14 - 35.82)\end{tabular} & & \begin{tabular}{@{}c@{}}90.43 \\ (2.52 - 159.81)\end{tabular} & \begin{tabular}{@{}c@{}}55.81 \\ (12.05 - 88.06)\end{tabular} & \begin{tabular}{@{}c@{}}1 \\ (0.39 - 4.16)\end{tabular} & \begin{tabular}{@{}c@{}}11.42 \\ (1.95 - 24.42)\end{tabular} \\

\bottomrule
\end{tabular}
\end{adjustbox}
\label{tab:Wellbeing_Kv100}
\end{table}

\onecolumn
\section{Details of Experimental Setup}
\label{sec.details_experimental_setup}
This Appendix Section details the machine learning models employed for predicting depression, anxiety, and stress (DAS) levels, along with the Explainable Artificial Intelligence (XAI) technique used to interpret their decision-making processes.

\subsection{Experimental Setup: Machine Learning Models}

To identify the most effective approach for predicting Depression, Anxiety, and stress (DAS) levels, we explored various machine learning algorithms. These algorithms leverage the collected wearable device data and DAS-21 questionnaire scores to estimate a patient's mental issues.

\textbf{Support Vector Machine} (SVM) is a well-established method that is known for its ability to effectively handle high-dimensional datasets, even with a relatively small sample size. SVMs aim to find a hyperplane in the feature space that maximizes the margin between different classes. In SVM, new data points are then classified based on which side of the hyperplane they fall on. SVMs are powerful for classification tasks with high-dimensional data, such as ours with potentially many features extracted from wearable sensors. They are effective even with limited data and offer good generalization capabilities. However, SVMs can be computationally expensive for very large datasets and may be less interpretable than other algorithms on this list.

\textbf{Random Forest Classifier} is an ensemble learning algorithm that combines the predictions of multiple, independently trained decision trees. Each tree is built using a random subset of features and data points, promoting diversity within the ensemble. The final prediction is made by majority vote or averaging the individual tree predictions. The algorithms are robust to overfitting due to their inherent diversity. They can handle various data types and perform well even with missing values. This approach is particularly useful for datasets with potential noise or inconsistencies.

\textbf{Gradient Boosting Classifier} is an algorithm that works by iteratively building an ensemble of weak decision trees. Each tree learns from the errors of the previous one, ultimately leading to a more robust and accurate model. Gradient boosting is known for its flexibility and ability to handle various data types, making it a strong contender for our analysis.

Building upon gradient boosting, \textbf{CatBoost} specifically addresses challenges in healthcare data. It incorporates advanced techniques for handling categorical features, such as one-hot encoding or custom loss functions, which can be problematic for traditional gradient boosting. Additionally, CatBoost prioritizes model interpretability by providing feature importance scores and visualizations of decision boundaries. CatBoost excels in scenarios with a high volume of categorical features, common in healthcare data. It offers improved interpretability compared to standard gradient boosting, allowing us to understand the factors influencing model predictions

\textbf{LightGBM} (Light Gradient Boosting Machine) is a highly efficient implementation of the gradient boosting algorithm specifically designed for speed and performance. It utilizes techniques similar to gradient-based one-side sampling and feature importance sampling to focus on informative data points and reduce computational costs. LightGBM offers exceptional speed and accuracy, making it a compelling choice for large datasets. It is particularly efficient for memory usage, allowing for training in resource-constrained environments. LightGBM excels at handling large and complex datasets, making it suitable for our analysis where we have a high volume of data points from wearable devices. 

In addition to the machine learning algorithms above, we also implemented a \textbf{Multi-layer Perceptron} (MLP) for health issue prediction. Unlike simpler models, MLPs excel at identifying complex, non-linear relationships within the data. This capability could be particularly valuable for uncovering subtle patterns between physiological signals and mental health states. The MLP neural network has two hidden layers and a final output layer with a unit. The network employs layer normalization, ReLU activation for hidden layers, and dropout to prevent overfitting. Finally, a sigmoid or a softmax activation function is applied to the output layer to transform the final values to the probability of classes.

By evaluating the performance of these diverse algorithms, we aim to identify the one that best predicts DAS levels in the context of our specific dataset and research goals.

\subsection{Experimental Setup: Explainable AI}

In healthcare applications, ensuring trustworthy AI requires models to be not only accurate but also interpretable. Understanding the reasoning behind a model's predictions for DAS levels is crucial for building trust and confidence in its outputs. This empowers healthcare professionals and researchers to make informed decisions based on the predicted DAS levels and the underlying factors influencing those predictions. In this study, we leverage SHAP (Shapley Additive Explanations) to achieve interpretability and gain insights into the model's decision-making process for DAS prediction \cite{lundberg2017unified}.

SHAP, a powerful approach for achieving interpretability, assigns an attribution value (SHAP value) to each feature for a given DAS prediction. This value represents the contribution of that specific feature (e.g., a specific physiological sensor reading or a DAS-21 questionnaire response) to the model's final prediction. High positive SHAP values indicate that the feature has a strong positive influence on the predicted DAS level (potentially indicating a higher likelihood of depression, anxiety, or stress). Conversely, low negative SHAP values signify a negative influence (potentially indicating a lower likelihood). By analyzing the SHAP values for each feature, we can gain insights into the relative importance of various factors shaping the model's predictions about a user's mental state.

This interpretability allows us to answer several key questions:

\begin{itemize}
    \item Identification of key physiological and psychological indicators: What are the features from wearable sensor data and questionnaire scores of a patient that have the most significant influence on the model's predictions? Understanding these can pinpoint crucial physiological and psychological indicators associated with DAS levels. This knowledge can inform the development of targeted interventions and preventative measures for mental health.
    \item Validation of model fairness and mitigation of bias: Are the model's predictions fair across different demographics (age, gender, etc.)? Examining SHAP values across these groups helps ensure that the model is not unfairly biased towards certain populations. This is crucial in healthcare applications where fairness and unbiased decision-making are paramount.
    \item Enhanced model transparency and trust: How does the model arrive at its predictions? By explaining the rationale behind the model's predictions through SHAP values, we can foster trust and confidence in its use among healthcare professionals and researchers. This transparency is essential for the adoption and responsible use of AI in mental health assessments.
    
\end{itemize}

\newpage
\section{Training and Evaluation Metrics}

\subsection{Case study}
\label{sec.casestudy}
Our study employs two primary approaches to train machine learning models for predicting DAS levels: binary classification and multi-class classification, as shown in Table \ref{table:all_features} and Table \ref{table:sensor_important_features}. Both approaches leverage data from the DAS-21 questionnaire alongside potentially other features from the collected dataset. In addition, we consider three cases to investigate the models' performances: Using all collected features as shown in Table \ref{table:all_features}, using only features extracted from wearable devices, and using top-10 important features as illustrated in Table \ref{table:sensor_important_features}.

\begin{table}[h!]
\centering
\caption{Model investigation with all collected features, in which we trained models with different data split methods: 80:20, 5-fold, and leave one patient out (LOPO). In addition, "x" presents the cases that are being trained in our investigation.}
\begin{adjustbox}{max width=\textwidth}
\begin{tabular}{|l|c|c|c|c|c|c|}
\hline
\multirow{2}{*}{Model} & \multicolumn{6}{c|}{All Features} \\
\cline{2-7}
 & \multicolumn{2}{c|}{Split 80:20} & \multicolumn{2}{c|}{5-Fold Validation} & \multicolumn{2}{c|}{LOPO} \\
\cline{2-7}
 & Binary & Multi-class & Binary & Multi-class & Binary & Multi-class \\
\hline
Gradient Boosting & x & x & x & x & x & x \\ \hline
Catboost & x & x & x & x & x & x \\ \hline
LightGBM & x & x & x & x & x & x \\ \hline
SVM & x & x & x & x & x & x \\ \hline
Random Forest & x & x & x & x & x & x \\ \hline
MLP & x &  &  &  &  &  \\
\hline
\end{tabular}
\end{adjustbox}
\label{table:all_features}
\end{table}



\begin{table}[h!]
\centering
\renewcommand{\arraystretch}{1.2} % Increase the height of cells
\caption{Model investigation with sensor features and top-10 important features. We trained the models with various data split methods: 80/20 train-test split, 5-fold cross-validation, and LOPO cross-validation. The "x" marks in the table indicate the specific model-data combinations investigated in this study.}
\begin{adjustbox}{max width=\textwidth}
\begin{tabular}{|l|c|c|c|c|c|c|c|c|}
\hline
\multirow{2}{*}{Model} & \multicolumn{4}{c|}{Sensor Features} & \multicolumn{4}{c|}{Top-10 Important Features} \\
\cline{2-9}
 & \multicolumn{2}{c|}{5-Fold Validation} & \multicolumn{2}{c|}{LOPO} & \multicolumn{2}{c|}{5-Fold Validation} & \multicolumn{2}{c|}{LOPO} \\
\cline{2-9}
 & Binary & Multi-class & Binary & Multi-class & Binary & Multi-class & Binary & Multi-class \\
\hline
mGradient Boosting & x & x & x & x & x & x & x & x \\ \hline
Catboost & x & x & x & x & x & x & x & x \\ \hline
LightGBM & x & x & x & x & x & x & x & x \\ \hline
SVM & x & x & x & x & x & x & x & x \\ \hline
Random Forest & x & x & x & x & x & x & x & x \\ \hline
MLP & x & x & x & x &  &  &  &  \\
\hline
\end{tabular}
\end{adjustbox}
\label{table:sensor_important_features}
\end{table}

For binary classification, this approach simplifies the prediction task by transforming the DAS levels into a binary classification problem. In particular, we utilize the information from the DAS-21 questionnaire as labels, focusing on the mental health state of the participants. We categorize participants into two classes based on their DAS-21 scores:
\begin{itemize}
    \item Normal: This class comprises participants who score within the normal range for depression, anxiety, and stress according to established DAS-21 scoring guidelines.
    \item Abnormal: This class encompasses participants whose DAS-21 scores indicate potential symptoms of depression, anxiety, or stress.
\end{itemize}

By converting the problem into a binary classification task, we can train models to distinguish between individuals with normal mental health states and those with potential mental health concerns based on their DAS-21 responses and potentially other features from the dataset.

For multi-class classification, this approach aims for a more granular prediction by treating DAS levels as a multi-class classification problem. Instead of collapsing mental health states into two categories, we define multiple classes based on the established DAS-21 scoring ranges: Normal, stress, stress Anxiety, and stress anxiety depression.

\subsection{Data Split}
\subsubsection{Random 80:20 Split}
In machine learning, dividing the dataset into training and testing subsets is crucial for evaluating model performance. 
One common approach is the random 80:20 split, where 80\% of the randomly-shuffled data is used for training the model, and the remaining 20\% is reserved for testing. 
This method is favored for its balance between providing enough data for the model to learn effectively and retaining a significant portion for unbiased performance evaluation.

\begin{python}
from sklearn.model_selection import train_test_split

# Assuming `X` is the feature matrix and `y` is the target vector
X_train, X_test, y_train, y_test = train_test_split(X, y, test_size=0.2, random_state=42)

\end{python}

\subsubsection{Leave-one-patient-out (LOPO)}
Predictive models for diagnosis or treatment outcomes must generalize well across different patients. 
Leave-one-patient-out (LOPO) \cite{hastie2009elements} ensures that the model is evaluated on its ability to perform on new patients not seen during training.
For each iteration, all data points from one patient are used as the test set, while the data from all other patients are used to train the model. 
This process is repeated for each individual in the dataset, ensuring that every patient's data is used as a test set exactly once.

\begin{python}
def leave_one_person_out(patient_ids):
  loo = dict()
  for patient_id in set(patient_ids):
    loo[patient_id] = None

  for patient_id in loo:
    train_indices, test_indices = [], []

    for i in range(len(patient_ids)):
      if patient_ids[i] == patient_id:
        test_indices.append(i)
      else:
        train_indices.append(i)

    loo[patient_id] = [train_indices, test_indices]

  return loo


patient_ids = list(df_rawdata['Patient_ID'])
loo = leave_one_person_out(patient_ids)


for patient_id in loo:
  # Leave one out
  train_indices = np.array(loo[patient_id][0])
  test_indices = np.array(loo[patient_id][1])

  X_train = X[train_indices]
  X_test = X[test_indices]
  y_train = y[train_indices]
  y_test = y[test_indices]
  
\end{python}

\subsubsection{K-Folds}
K-fold cross-validation is a robust technique for training machine learning models in healthcare, ensuring reliable and unbiased performance evaluation \cite{hastie2009elements}. 
K-fold cross-validation mitigates the risk of overfitting, maximizes data utilization, and enhances the model's generalizability, which is crucial for accurately predicting patient outcomes and making informed medical decisions.
It involves randomly partitioning the dataset into k equally sized folds, where the model is trained on k-1 folds and validated on the remaining fold. 
This process is repeated k times, with each fold serving as the validation set once, and the performance metrics are averaged to provide a comprehensive assessment.
Note that for each fold there is no patient overlapping.
In the scope of this work, we used 5-fold cross-validation and 10 seeds for training, which means 5-fold was trained for 10 times to avoid varying results.

\begin{python}
def k_fold(patient_ids, number_of_folds = 5):
  patient_id_dict = dict()
  for patient_id in set(patient_ids):
    patient_id_dict[patient_id] = []

  for i in range(len(patient_ids)):
    patient_id = patient_ids[i]
    patient_id_dict[patient_id].append(i)

  kfold_dict = dict()
  i, k = 0, len(patient_id_dict.keys())//number_of_folds
  queue = list(patient_id_dict.keys())

  while queue:
    random.shuffle(queue)
    patient_id_test = [j for j in queue[0:k] ]
    patient_id_train = [j for j in set(patient_id_dict.keys())-set(patient_id_test) ]

    train_indices, test_indices = [], []
    for patient_id in patient_id_train:
      train_indices += patient_id_dict[patient_id]
    for patient_id in patient_id_test:
      test_indices += patient_id_dict[patient_id]
    kfold_dict[i] = [train_indices, test_indices]

    queue = queue[k:]
    i += 1

  return kfold_dict


patient_ids = list(df_rawdata['Patient_ID'])


for seed in range(10):
  kfold = k_fold(patient_ids, number_of_folds=5)
  for k in kfold:
    # K-fold
    train_indices = np.array(kfold[k][0])
    test_indices = np.array(kfold[k][1])

    X_train = X[train_indices]
    X_test = X[test_indices]
    y_train = y[train_indices]
    y_test = y[test_indices]

\end{python}
\subsection{Evaluation Metrics}
\label{sec.evaluation}
To assess the performance of the machine learning models for predicting Depression, Anxiety, and stress (DAS) levels, we employ two key evaluation metrics: Receiver Operating Characteristic (ROC) AUC (Area Under the Curve) and Precision-Recall (PR) AUC. These metrics provide a comprehensive assessment of the model's discriminative ability and its performance in handling class imbalances.

The ROC AUC metric measures the ability of the model to distinguish between classes. It plots the True Positive Rate (TPR) against the False Positive Rate (FPR) at various threshold settings. The AUC value ranges from 0 to 1, where a value of 1 indicates perfect classification and a value of 0.5 suggests performance no better than random guessing. A higher ROC AUC indicates that the model is better at differentiating between the positive and negative classes.

In our study, we calculate the ROC AUC to assess how well our model can predict DAS levels across the entire range of possible thresholds. This metric is particularly useful in healthcare applications, where the cost of false positives and false negatives can vary, and understanding the trade-offs between sensitivity and specificity is crucial.

While ROC AUC provides a good overall measure of performance, it can be insensitive to class imbalance.  In our case, the prevalence of depression, anxiety, and stress might be lower than the prevalence of healthy individuals.  The PR curve addresses this by plotting Precision against Recall.

\newpage
\section{ROC Plots}
This section shows the ROC AUC and ROC PR plots for all binary classification results above.

\subsection{All Features with 80:20 Split (Binary Classification)}
Figure \ref{fig:AUCplot-standard_classification-binary_gradientboosting}, \ref{fig:AUCplot-standard_classification-binary_catboost}, \ref{fig:AUCplot-standard_classification-binary_lightgbm}, \ref{fig:AUCplot-standard_classification-binary_svm}, \ref{fig:AUCplot-standard_classification-binary_randomforest}, \ref{fig:AUCplot-standard_classification-binary_mlp} shows the ROC AUC and ROC PR plots for all features binary classification models using Gradient Boosting, Catboost, LightGBM, SVM, Random Forest, and MLP respectively, with a random 80:20 split.
These figures are plots from Table \ref{tab:ml_metrics_All_features_binary_classification} in Section \ref{sec.result.8020}.

\begin{figure}[h]
    \centering
    \includegraphics[width=.7\textwidth]{AUC_plots/standard_classification/binary_gradientboosting_ROCAUC0.975_PRAUC0.9911.png}\hfill
    \caption{All features with 80:20 split, binary classification: Gradient Boosting}
    \label{fig:AUCplot-standard_classification-binary_gradientboosting}
\end{figure}

\begin{figure}[h]
    \centering
    \includegraphics[width=.7\textwidth]{AUC_plots/standard_classification/binary_catboost.png}\hfill
    \caption{All features with 80:20 split, binary classification: Catboost}
    \label{fig:AUCplot-standard_classification-binary_catboost}
\end{figure}

\begin{figure}[h]
    \centering
    \includegraphics[width=.7\textwidth]{AUC_plots/standard_classification/binary_lightgbm.png}\hfill
    \caption{All features with 80:20 split, binary classification: LightGBM}
    \label{fig:AUCplot-standard_classification-binary_lightgbm}
\end{figure}

\begin{figure}[h]
    \centering
    \includegraphics[width=.7\textwidth]{AUC_plots/standard_classification/binary_svm.png}\hfill
    \caption{All features with 80:20 split, binary classification: SVM}
    \label{fig:AUCplot-standard_classification-binary_svm}
\end{figure}

\begin{figure}[h]
    \centering
    \includegraphics[width=.7\textwidth]{AUC_plots/standard_classification/binary_randomforest.png}\hfill
    \caption{All features with 80:20 split, binary classification: Random forest}
    \label{fig:AUCplot-standard_classification-binary_randomforest}
\end{figure}

\begin{figure}[h]
    \centering
    \includegraphics[width=.7\textwidth]{AUC_plots/standard_classification/binary_mlp.png}\hfill
    \caption{All features with 80:20 split, binary classification: MLP}
    \label{fig:AUCplot-standard_classification-binary_mlp}
\end{figure}

\onecolumn
\subsection{All Features with Cross-Validation}

\subsubsection{Binary classification: LOPO}
Figure \ref{fig:AUCplot-allfeatures_binary_classification-gradientboosting_lopo}, \ref{fig:AUCplot-allfeatures_binary_classification-catboost_lopo}, \ref{fig:AUCplot-allfeatures_binary_classification-lightGBM_lopo}, \ref{fig:AUCplot-allfeatures_binary_classification-svm_lopo}, \ref{fig:AUCplot-allfeatures_binary_classification-randomforest_lopo}, \ref{fig:AUCplot-allfeatures_binary_classification-mlp_lopo} shows the ROC AUC and ROC PR plots for all features binary classification models using Gradient Boosting, Catboost, LightGBM, SVM, Random Forest, and MLP respectively, with a LOPO split.
These figures are plots from Table \ref{tab:BC_LOPO} in Section \ref{sec.result.allfeatures_crossval}.

\begin{figure}[h]
    \centering
    \includegraphics[width=.7\textwidth]{AUC_plots/allfeatures_binary_classification/gradientboosting_lopo.png}\hfill
    \caption{All features with LOPO, binary classification: Gradient Boosting}
    \label{fig:AUCplot-allfeatures_binary_classification-gradientboosting_lopo}
\end{figure}

\begin{figure}[h]
    \centering
    \includegraphics[width=.7\textwidth]{AUC_plots/allfeatures_binary_classification/catboost_lopo.png}\hfill
    \caption{All features with LOPO, binary classification: Catboost}
    \label{fig:AUCplot-allfeatures_binary_classification-catboost_lopo}
\end{figure}

\begin{figure}[h]
    \centering
    \includegraphics[width=.7\textwidth]{AUC_plots/allfeatures_binary_classification/lightGBM_lopo.png}\hfill
    \caption{All features with LOPO, binary classification: LightGBM}
    \label{fig:AUCplot-allfeatures_binary_classification-lightGBM_lopo}
\end{figure}

\begin{figure}[h]
    \centering
    \includegraphics[width=.7\textwidth]{AUC_plots/allfeatures_binary_classification/svm_lopo.png}\hfill
    \caption{All features with LOPO, binary classification: SVM}
    \label{fig:AUCplot-allfeatures_binary_classification-svm_lopo}
\end{figure}

\begin{figure}[h]
    \centering
    \includegraphics[width=.7\textwidth]{AUC_plots/allfeatures_binary_classification/randomforest_lopo.png}\hfill
    \caption{All features with LOPO, binary classification: Random Forest}
    \label{fig:AUCplot-allfeatures_binary_classification-randomforest_lopo}
\end{figure}

\begin{figure}[h]
    \centering
    \includegraphics[width=.7\textwidth]{AUC_plots/allfeatures_binary_classification/mlp_lopo.png}\hfill
    \caption{All features with LOPO, binary classification: MLP}
    \label{fig:AUCplot-allfeatures_binary_classification-mlp_lopo}
\end{figure}

\onecolumn
\subsubsection{Binary classification: K-folds}
Figure \ref{fig:AUCplot-allfeatures_binary_classification-gradientboosting_5fold}, \ref{fig:AUCplot-allfeatures_binary_classification-catboost_5fold}, \ref{fig:AUCplot-allfeatures_binary_classification-lightGBM_5fold}, \ref{fig:AUCplot-allfeatures_binary_classification-svm_5fold}, \ref{fig:AUCplot-allfeatures_binary_classification-randomforest_5fold}, \ref{fig:AUCplot-allfeatures_binary_classification-mlp_5fold} shows the ROC AUC and ROC PR plots for all features binary classification models using Gradient Boosting, Catboost, LightGBM, SVM, Random Forest, and MLP respectively, with a K-folds split.
These figures are plots from Table \ref{tab:BC_K_fold} in Section \ref{sec.result.allfeatures_crossval}.

\begin{figure}[h]
    \centering
    \includegraphics[width=.7\textwidth]{AUC_plots/allfeatures_binary_classification/gradientboosting_5fold.png}\hfill
    \caption{All features with K-folds, binary classification: Gradient Boosting}
    \label{fig:AUCplot-allfeatures_binary_classification-gradientboosting_5fold}
\end{figure}

\begin{figure}[h]
    \centering
    \includegraphics[width=.7\textwidth]{AUC_plots/allfeatures_binary_classification/catboost_5fold.png}\hfill
    \caption{All features with K-folds, binary classification: Catboost}
    \label{fig:AUCplot-allfeatures_binary_classification-catboost_5fold}
\end{figure}

\begin{figure}[h]
    \centering
    \includegraphics[width=.7\textwidth]{AUC_plots/allfeatures_binary_classification/lightGBM_5fold.png}\hfill
    \caption{All features with K-folds, binary classification: LightGBM}
    \label{fig:AUCplot-allfeatures_binary_classification-lightGBM_5fold}
\end{figure}

\begin{figure}[h]
    \centering
    \includegraphics[width=.7\textwidth]{AUC_plots/allfeatures_binary_classification/svm_5fold.png}\hfill
    \caption{All features with K-folds, binary classification: SVM}
    \label{fig:AUCplot-allfeatures_binary_classification-svm_5fold}
\end{figure}

\begin{figure}[h]
    \centering
    \includegraphics[width=.7\textwidth]{AUC_plots/allfeatures_binary_classification/randomforest_5fold.png}\hfill
    \caption{All features with K-folds, binary classification: Random Forest}
    \label{fig:AUCplot-allfeatures_binary_classification-randomforest_5fold}
\end{figure}

\begin{figure}[h]
    \centering
    \includegraphics[width=.7\textwidth]{AUC_plots/allfeatures_binary_classification/mlp_5fold.png}\hfill
    \caption{All features with K-folds, binary classification: MLP}
    \label{fig:AUCplot-allfeatures_binary_classification-mlp_5fold}
\end{figure}

\onecolumn
\subsection{Using multimodal sensor features classification}

\subsubsection{Binary classification: LOPO}
Figure \ref{fig:AUCplot-sensorfeatures_binary_classification-gradientboosting_lopo}, \ref{fig:AUCplot-sensorfeatures_binary_classification-catboost_lopo}, \ref{fig:AUCplot-sensorfeatures_binary_classification-lightgbm_lopo}, \ref{fig:AUCplot-sensorfeatures_binary_classification-svm_lopo}, \ref{fig:AUCplot-sensorfeatures_binary_classification-randomforest_lopo}, \ref{fig:AUCplot-sensorfeatures_binary_classification-mlp_lopo} shows the ROC AUC and ROC PR plots for multimodal sensor features binary classification models using Gradient Boosting, Catboost, LightGBM, SVM, Random Forest, and MLP respectively, with a LOPO split.
These figures are plots from Table \ref{tab:sensor_features_binary_LOPO} in Section \ref{sec.result.sensorfeatures}.

\begin{figure}[h]
    \centering
    \includegraphics[width=.7\textwidth]{AUC_plots/sensorfeatures_binary_classification/gradientboosting_lopo.png}\hfill
    \caption{Multimodal sensor features with LOPO, binary classification: Gradient Boosting}
    \label{fig:AUCplot-sensorfeatures_binary_classification-gradientboosting_lopo}
\end{figure}

\begin{figure}[h]
    \centering
    \includegraphics[width=.7\textwidth]{AUC_plots/sensorfeatures_binary_classification/catboost_lopo.png}\hfill
    \caption{Multimodal sensor features with LOPO, binary classification: Catboost}
    \label{fig:AUCplot-sensorfeatures_binary_classification-catboost_lopo}
\end{figure}

\begin{figure}[h]
    \centering
    \includegraphics[width=.7\textwidth]{AUC_plots/sensorfeatures_binary_classification/lightgbm_lopo.png}\hfill
    \caption{Multimodal sensor features with LOPO, binary classification: LightGBM}
    \label{fig:AUCplot-sensorfeatures_binary_classification-lightgbm_lopo}
\end{figure}

\begin{figure}[h]
    \centering
    \includegraphics[width=.7\textwidth]{AUC_plots/sensorfeatures_binary_classification/svm_lopo.png}\hfill
    \caption{Multimodal sensor features with LOPO, binary classification: SVM}
    \label{fig:AUCplot-sensorfeatures_binary_classification-svm_lopo}
\end{figure}

\begin{figure}[h]
    \centering
    \includegraphics[width=.7\textwidth]{AUC_plots/sensorfeatures_binary_classification/randomforest_lopo.png}\hfill
    \caption{Multimodal sensor features with LOPO, binary classification: Random Forest}
    \label{fig:AUCplot-sensorfeatures_binary_classification-randomforest_lopo}
\end{figure}

\begin{figure}[h]
    \centering
    \includegraphics[width=.7\textwidth]{AUC_plots/sensorfeatures_binary_classification/mlp_lopo.png}\hfill
    \caption{Multimodal sensor features with LOPO, binary classification: MLP}
    \label{fig:AUCplot-sensorfeatures_binary_classification-mlp_lopo}
\end{figure}

\onecolumn
\subsubsection{Binary classification: K-folds}
Figure \ref{fig:AUCplot-sensorfeatures_binary_classification-gradientboosting_kfold}, \ref{fig:AUCplot-sensorfeatures_binary_classification-catboost_kfold}, \ref{fig:AUCplot-sensorfeatures_binary_classification-lightgbm_kfold}, \ref{fig:AUCplot-sensorfeatures_binary_classification-svm_kfold}, \ref{fig:AUCplot-sensorfeatures_binary_classification-randomforest_kfold}, \ref{fig:AUCplot-sensorfeatures_binary_classification-mlp_kfold} shows the ROC AUC and ROC PR plots for multimodal sensor features binary classification models using Gradient Boosting, Catboost, LightGBM, SVM, Random Forest, and MLP respectively, with a K-folds split.
These figures are plots from Table \ref{tab:sensor_features_binary_kfold} in Section \ref{sec.result.sensorfeatures}.

\begin{figure}[h]
    \centering
    \includegraphics[width=.7\textwidth]{AUC_plots/sensorfeatures_binary_classification/gradientboosting_kfold.png}\hfill
    \caption{Multimodal sensor features with K-folds, binary classification: Gradient Boosting}
    \label{fig:AUCplot-sensorfeatures_binary_classification-gradientboosting_kfold}
\end{figure}

\begin{figure}[h]
    \centering
    \includegraphics[width=.7\textwidth]{AUC_plots/sensorfeatures_binary_classification/catboost_kfold.png}\hfill
    \caption{Multimodal sensor features with K-folds, binary classification: Catboost}
    \label{fig:AUCplot-sensorfeatures_binary_classification-catboost_kfold}
\end{figure}

\begin{figure}[h]
    \centering
    \includegraphics[width=.7\textwidth]{AUC_plots/sensorfeatures_binary_classification/lightgbm_kfold.png}\hfill
    \caption{Multimodal sensor features with K-folds, binary classification: LightGBM}
    \label{fig:AUCplot-sensorfeatures_binary_classification-lightgbm_kfold}
\end{figure}

\begin{figure}[h]
    \centering
    \includegraphics[width=.7\textwidth]{AUC_plots/sensorfeatures_binary_classification/svm_kfold.png}\hfill
    \caption{Multimodal sensor features with K-folds, binary classification: SVM}
    \label{fig:AUCplot-sensorfeatures_binary_classification-svm_kfold}
\end{figure}

\begin{figure}[h]
    \centering
    \includegraphics[width=.7\textwidth]{AUC_plots/sensorfeatures_binary_classification/randomforest_kfold.png}\hfill
    \caption{Multimodal sensor features with K-folds, binary classification: Random Forest}
    \label{fig:AUCplot-sensorfeatures_binary_classification-randomforest_kfold}
\end{figure}

\begin{figure}[h]
    \centering
    \includegraphics[width=.7\textwidth]{AUC_plots/sensorfeatures_binary_classification/mlp_kfold.png}\hfill
    \caption{Multimodal sensor features with K-folds, binary classification: MLP}
    \label{fig:AUCplot-sensorfeatures_binary_classification-mlp_kfold}
\end{figure}

\onecolumn
\subsection{Using Top-10 important features classification}
\subsubsection{Binary classification: LOPO}
Figure \ref{fig:AUCplot-top10features_binary_classification-gradientboosting_lopo}, \ref{fig:AUCplot-top10features_binary_classification-catboost_lopo}, \ref{fig:AUCplot-top10features_binary_classification-lightgbm_lopo}, \ref{fig:AUCplot-top10features_binary_classification-svm_lopo}, \ref{fig:AUCplot-top10features_binary_classification-randomforest_lopo}, \ref{fig:AUCplot-top10features_binary_classification-mlp_lopo} shows the ROC AUC and ROC PR plots for top 10 features binary classification models using Gradient Boosting, Catboost, LightGBM, SVM, Random Forest, and MLP respectively, with a LOPO split.
These figures are plots from Table \ref{tab:top10_binary_LOPO} in Section \ref{sec.result_top10}.

\begin{figure}[h]
    \centering
    \includegraphics[width=.7\textwidth]{AUC_plots/top10features_binary_classification/gradientboosting_lopo.png}\hfill
    \caption{Top 10 features with LOPO, binary classification: Gradient Boosting}
    \label{fig:AUCplot-top10features_binary_classification-gradientboosting_lopo}
\end{figure}

\begin{figure}[h]
    \centering
    \includegraphics[width=.7\textwidth]{AUC_plots/top10features_binary_classification/catboost_lopo.png}\hfill
    \caption{Top 10 features with LOPO, binary classification: Catboost}
    \label{fig:AUCplot-top10features_binary_classification-catboost_lopo}
\end{figure}

\begin{figure}[h]
    \centering
    \includegraphics[width=.7\textwidth]{AUC_plots/top10features_binary_classification/lightgbm_lopo.png}\hfill
    \caption{Top 10 features with LOPO, binary classification: LightGBM}
    \label{fig:AUCplot-top10features_binary_classification-lightgbm_lopo}
\end{figure}

\begin{figure}[h]
    \centering
    \includegraphics[width=.7\textwidth]{AUC_plots/top10features_binary_classification/svm_lopo.png}\hfill
    \caption{Top 10 features with LOPO, binary classification: SVM}
    \label{fig:AUCplot-top10features_binary_classification-svm_lopo}
\end{figure}

\begin{figure}[h]
    \centering
    \includegraphics[width=.7\textwidth]{AUC_plots/top10features_binary_classification/randomforest_lopo.png}\hfill
    \caption{Top 10 features with LOPO, binary classification: Random Forest}
    \label{fig:AUCplot-top10features_binary_classification-randomforest_lopo}
\end{figure}

\begin{figure}[h]
    \centering
    \includegraphics[width=.7\textwidth]{AUC_plots/top10features_binary_classification/mlp_lopo.png}\hfill
    \caption{Top 10 features with LOPO, binary classification: MLP}
    \label{fig:AUCplot-top10features_binary_classification-mlp_lopo}
\end{figure}

\onecolumn
\subsubsection{Binary classification: K-folds}
Figure \ref{fig:AUCplot-top10features_binary_classification-gradientboosting_5fold}, \ref{fig:AUCplot-top10features_binary_classification-catboost_5fold}, \ref{fig:AUCplot-top10features_binary_classification-lightgbm_5fold}, \ref{fig:AUCplot-top10features_binary_classification-svm_5fold}, \ref{fig:AUCplot-top10features_binary_classification-randomforest_5fold} shows the ROC AUC and ROC PR plots for top 10 features binary classification models using Gradient Boosting, Catboost, LightGBM, SVM, Random Forest, and MLP respectively, with a K-folds split.
These figures are plots from Table \ref{tab:top10_binary_Kfolds} in Section \ref{sec.result_top10}.

\begin{figure}[h]
    \centering
    \includegraphics[width=.7\textwidth]{AUC_plots/top10features_binary_classification/gradientboosting_5fold.png}\hfill
    \caption{Top 10 features with K-folds, binary classification: Gradient Boosting}
    \label{fig:AUCplot-top10features_binary_classification-gradientboosting_5fold}
\end{figure}

\begin{figure}[h]
    \centering
    \includegraphics[width=.7\textwidth]{AUC_plots/top10features_binary_classification/catboost_5fold.png}\hfill
    \caption{Top 10 features with K-folds, binary classification: Catboost}
    \label{fig:AUCplot-top10features_binary_classification-catboost_5fold}
\end{figure}

\begin{figure}[h]
    \centering
    \includegraphics[width=.7\textwidth]{AUC_plots/top10features_binary_classification/lightgbm_5fold.png}\hfill
    \caption{Top 10 features with K-folds, binary classification: LightGBM}
    \label{fig:AUCplot-top10features_binary_classification-lightgbm_5fold}
\end{figure}

\begin{figure}[h]
    \centering
    \includegraphics[width=.7\textwidth]{AUC_plots/top10features_binary_classification/svm_5fold.png}\hfill
    \caption{Top 10 features with K-folds, binary classification: SVM}
    \label{fig:AUCplot-top10features_binary_classification-svm_5fold}
\end{figure}

\begin{figure}[h]
    \centering
    \includegraphics[width=.7\textwidth]{AUC_plots/top10features_binary_classification/randomforest_5fold.png}\hfill
    \caption{Top 10 features with K-folds, binary classification: Random Forest}
    \label{fig:AUCplot-top10features_binary_classification-randomforest_5fold}
\end{figure}



%\end{document}
