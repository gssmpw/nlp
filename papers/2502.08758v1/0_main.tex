\documentclass[conference]{IEEEtran}
%****************this is required for ICMLCN**********
\usepackage[letterpaper, left=0.65in, right=0.65in, bottom=1in, top=0.8in]{geometry}
\IEEEoverridecommandlockouts
\special{papersize=8.5in,11in}
%****************************************************
%****************************************************%
%               LIST OF ACRONYM HERE                 %
%****************************************************%
\usepackage[acronym]{glossaries}
\newacronym{BS}{BS}{base station}
\newacronym{PS}{PS}{phase-shifter}
\newacronym{RL}{RL}{reinforcement learning}
\newacronym{AP}{AP}{analog precoder}
\newacronym{FC-HBF}{FC-HBF}{fully-connected HBF}
\newacronym{FSA-HBF}{FSA-HBF}{fixed subarray HBF}
\newacronym{DSA-HBF}{DSA-HBF}{dynamic subarray HBF}
\newacronym{BF}{BF}{beamforming}
\newacronym{UE}{UE}{user equipment}
\newacronym{AWGN}{AWGN}{additive white gaussian noise}
%\newacronym{BS}{BS}{base station}
\newacronym{MIMO}{MIMO}{multiple-input multiple-output}
\newacronym{MISO}{MISO}{multiple-input single-output}
\newacronym{RF}{RF}{radio frequency}
\newacronym{RIS}{RIS}{reconfigurable intelligent surfaces}
\newacronym{IOT}{IOT}{internet-of-things}
\newacronym{CL}{CL}{convolutional layer}
\newacronym{FDD}{FDD}{frequency division duplex}
\newacronym{TDD}{TDD}{time division duplex}
\newacronym{CSI}{CSI}{channel state information}
\newacronym{DNN}{DNN}{deep neural network}
\newacronym{DP}{DP}{digital precoder}
\newacronym{DL}{DL}{deep learning}
\newacronym{SVD}{SVD}{singular-value decomposition}
\newacronym{CNN}{CNN}{convolution neural network}
\newacronym{FDP}{FDP}{fully digital precoder}
\newacronym{SE}{SE}{spectral efficiency}
\newacronym{OFDM}{OFDM}{orthogonal frequency division multiplexing}
\newacronym{OMP}{OMP}{orthogonal matching pursuit}
\newacronym{FL}{FL}{fully-connected layer}
\newacronym{HSHO}{HSHO}{Hybrid Structured Heuristic Optimization}
\newacronym{HBF}{HBF}{hybrid beamforming}
\newacronym{IA}{IA}{initial access}
\newacronym{mm-Wave}{mm-Wave}{millimeter wave}
\newacronym{mMIMO}{mMIMO}{massive MIMO}
\newacronym{SINR}{SINR}{signal-to-interference-noise ratio}
\newacronym{SNR}{SNR}{signal-to-noise ratio}
%\newacronym{SS}{SS}{synchronization signal}
%\newacronym{SSB}{SSB}{synchronization signal burst}
\newacronym{RSSI}{RSSI}{received signal strength indicator}
\newacronym{PZF}{PZF}{phase zero forcing}
\newacronym{PSO}{PSO}{particle swarm optimization}
\newacronym{ZF}{ZF}{zero forcing}
\newacronym{O-FDP}{O-FDP}{optimal fully digital precoder}
\newacronym{JT}{JT}{joint transmission}
\newacronym{CU}{CU}{central unit}
\newacronym{MSE}{MSE}{mean square error}
\newacronym{CEL}{CEL}{cross entropy loss}
\newacronym{CB}{CB}{conjugate beamforming}
\newacronym{NC}{NC}{network controller}
\newacronym{CoMP}{CoMP}{coordinated multi point}
\newacronym{CF-mMIMO}{CF-mMIMO}{cell-free massive MIMO}
\newacronym{CF-HBF}{CF-HBF}{cell-free hybrid beamforming}
\newacronym{CF-BF}{CF-BF}{cell-free beamforming}
\newacronym{MLDG}{MLDG}{meta-learning domain generalization}
\newacronym{MAML}{MAML}{model agnostic meta-learning}
\newacronym{WSR}{WSR}{weighted sum rate}
\newacronym{WMMSE}{WMMSE}{weighted minimum mean square error}
\newacronym{NN}{NN}{neural network}
\newacronym{LOS}{LOS}{line of sight}
\newacronym{NLOS}{NLOS}{non line of sight}
\newacronym{ML}{ML}{machine learning}
\newacronym{FCL}{FCL}{fully connected layer}
\newacronym{SSL}{SSL}{semi-supervised learning}
 
\newif\ifDeepMIMOModel
\DeepMIMOModeltrue

\newif\ifSimpleNParamEq
\SimpleNParamEqtrue

\usepackage{lettrine}
% \usepackage[ruled,vlined,linesnumbered]{algorithm2e}
\usepackage{multirow}
\usepackage{longtable}
\usepackage[table,xcdraw]{xcolor}
\usepackage[linesnumbered,ruled,vlined]{algorithm2e}
\usepackage{float}
\floatplacement{figure}{H}
\makeatletter
\let\oldlt\longtable
\let\endoldlt\endlongtable
\def\longtable{\@ifnextchar[\longtable@i \longtable@ii}
\def\longtable@i[#1]{\begin{figure}[t]
\onecolumn
\begin{minipage}{0.5\textwidth}
\oldlt[#1]
}
\def\longtable@ii{\begin{figure}[t]
\onecolumn
\begin{minipage}{0.5\textwidth}
\oldlt
}
\def\endlongtable{\endoldlt
\end{minipage}
\twocolumn
\end{figure}}
\makeatother
\usepackage{ifpdf}
% *** GRAPHICS RELATED PACKAGES ***
\ifCLASSINFOpdf
  \usepackage[pdftex]{graphicx}
  \usepackage{graphicx,epstopdf}
  \graphicspath{{../pdf/}{../jpeg/}}
  \DeclareGraphicsExtensions{.pdf,.jpeg,.png,.eps}
\else
  \usepackage[dvips]{graphicx}
  \usepackage{graphicx,epstopdf}
  \graphicspath{{../eps/}}
  \DeclareGraphicsExtensions{}
\fi
\usepackage{tikz}
\usetikzlibrary{decorations.pathreplacing,calc}
\newcommand{\tikzmark}[1]{\tikz[overlay,remember picture] \node (#1) {};}

\newcommand*{\AddNote}[4]{%
    \begin{tikzpicture}[overlay, remember picture]
        \draw [decoration={brace,amplitude=0.5em},decorate,line width=.2mm,black]
            ($(#3)!(#1.north)!($(#3)-(0,1)$)$) --  
            ($(#3)!(#2.south)!($(#3)-(0,1)$)$)
                node [align=center, text width=2.5cm, pos=0.5, anchor=west] {#4};
    \end{tikzpicture}
}%

\usepackage{cite}
\usepackage{amsthm}
\usepackage{steinmetz}
\usepackage{amssymb}
\usepackage{balance}
\usepackage{eqparbox}
\usepackage{multirow}
\usepackage{bbm}
\usepackage{float}
\SetAlFnt{\small}


\usepackage{amsmath}

\usepackage{mathtools, nccmath}
\DeclarePairedDelimiter{\nint}\lfloor\rceil
\DeclarePairedDelimiter{\abs}\lvert\rvert

\usepackage{booktabs, siunitx}
% \usepackage[svgnames,table]{xcolor}
%\usepackage[tableposition=above]{caption}

% \usepackage{graphicx}
\usepackage[scaled]{DejaVuSansMono}
\usepackage[T1]{fontenc}

%\usepackage{stfloats}
%\usepackage[eulergreek]{sansmath}
% \usepackage[final]{graphicx}
\usepackage{color}
\usepackage{relsize}
\usepackage{mathtools}
\newcommand{\ex}{{\mathrm e}}
\newtheorem{definition}{Definition}
\newtheorem{lemma}{Lemma}
\newtheorem{theorem}{Theorem}
\newtheorem{remark}{Remark}

\newtheorem{proposition}{Proposition}
\newcommand{\bs}[1]{\boldsymbol{#1}}
\newcommand{\mc}[1]{\mathcal{#1}}
\newcommand{\ul}[1]{\underline{#1}}
\newcommand{\mb}[1]{\mathbf{#1}}
\newcommand{\mr}[1]{\mathrm{#1}}
\newcommand{\tr}{\mathrm{Tr}}
\newcommand{\fo}{\mathbf{F}_{\mathrm{opt}}}

% \newcommand{\ceil}[1]{\lceil {#1} \rceil}
\usepackage{mathtools}
\DeclarePairedDelimiter{\ceil}{\lceil}{\rceil}

\DeclareMathOperator*{\argmin}{arg\;min}
\DeclareMathOperator*{\argmax}{arg\;max}
\DeclareMathOperator*{\maximize}{maximize}
\DeclareMathOperator*{\minimize}{minimize}
\DeclareMathOperator*{\st}{subject\;to}
\renewcommand{\Pr}{\mathbb{P}} % (by default \Pr is rendered as "Pr")
\pagestyle{empty}
% \thispagestyle{plain}
% \pagestyle{plain}
\SetKwInput{KwInput}{Input}                % Set the Input
\SetKwInput{KwOutput}{Output}
\SetKwInput{KwOutputr}{Output Regression}              % set the Output
\SetKwInput{KwOutputc}{Output Classification}              % set the Output
\newcommand{\rank}{\operatornamewithlimits{rank}}
\newcommand{\trace}{\operatornamewithlimits{trace}}
%\newcommand{\argmin}{\operatornamewithlimits{argmin}}
%\newcommand{\argmax}{\operatornamewithlimits{argmax}}
\newcommand{\bseq}{\begin{subequations}}
\newcommand{\eseq}{\end{subequations}}
\newcommand{\baln}{\begin{align}}
\newcommand{\ealn}{\end{align}}
\newcommand{\balnd}{\begin{aligned}}
\newcommand{\ealnd}{\end{aligned}}
\newcommand{\beq}{\begin{equation}}
\newcommand{\eeq}{\end{equation}}
\newcommand{\beqn}{\begin{eqnarray}}
\newcommand{\eeqn}{\end{eqnarray}}
\newcommand{\beqno}{\begin{eqnarray*}}
\newcommand{\eeqno}{\end{eqnarray*}}
\newcommand{\bma}{\begin{displaymath}}
\newcommand{\ema}{\end{displaymath}}
\newcommand{\bnu}{\begin{enumerate}}
\newcommand{\enu}{\end{enumerate}}
\newcommand{\bce}{\begin{center}}
\newcommand{\ece}{\end{center}}
\newcommand{\btb}{\begin{tabular}}
\newcommand{\etb}{\end{tabular}}
\newcommand{\ba}{\begin{array}}
\newcommand{\ea}{\end{array}}
%\setlength\arraycolsep{2pt}
\usepackage{footnote}
\makesavenoteenv{tabular}
\makesavenoteenv{table}
\makeatletter 
\newcommand\semiHuge{\@setfontsize\semiHuge{21.1}{27.38}}
\makeatother

\usepackage{subfigure}
\usepackage{float}
% \setlength{\textfloatsep}{0pt}
% \setlength{\parskip}{0pt} 
%****************************************************
\begin{document}
\title{Compression of Site-Specific Deep Neural Networks for Massive MIMO Precoding}
\author{\IEEEauthorblockN{Ghazal~Kasalaee, Ali~Hasanzadeh~Karkan, Jean-François~Frigon, and François~Leduc-Primeau}
\IEEEauthorblockA{Department of Electrical Engineering, Polytechnique Montréal, Montréal, QC H3C 3A7, Canada\\
Emails: \{ghazal.kasalaee, ali.hasanzadeh-karkan, j-f.frigon, francois.leduc-primeau\}@polymtl.ca
    }
}

\maketitle
\IEEEpubidadjcol

\begin{abstract}
The deployment of deep learning (DL) models for precoding in massive multiple-input multiple-output (mMIMO) systems is often constrained by high computational demands and energy consumption. In this paper, we investigate the compute energy efficiency of mMIMO precoders using DL-based approaches, comparing them to conventional methods such as zero forcing and weighted minimum mean square error (WMMSE). Our energy consumption model accounts for both memory access and calculation energy within DL accelerators. We propose a framework that incorporates mixed-precision quantization-aware training and neural architecture search to reduce energy usage without compromising accuracy. Using a ray-tracing dataset covering various base station sites, we analyze how site-specific conditions affect the energy efficiency of compressed models. Our results show that deep neural network compression generates precoders with up to 35 times higher energy efficiency than WMMSE at equal performance, depending on the scenario and the desired rate. These results establish a foundation and a benchmark for the development of energy-efficient DL-based mMIMO precoders.
\end{abstract}




% \begin{IEEEkeywords}
% \end{IEEEkeywords}
% \IEEEpeerreviewmaketitle

\section{Introduction} \label{Sec:Intro}
\section{Introduction}
\label{sec:intro}
% Image editing methods in diffusion models depend on user-defined control directions - users can unlock their creativity using these methods by specifying the desired manipulation through prompts~\cite{gandikota2023concept}, reference images~\cite{ruiz2022dreambooth, kumari2022customdiffusion, gal2022image, chen2024trainingfreeregionalpromptingdiffusion}, or attribute vectors~\cite{parmar2023zero,hertz2022prompt}. In this work, we ask a fundamentally different question: \emph{Can we automatically discover the underlying visual structure of a concept within diffusion model's knowledge?} %Rather than requiring user-specified controls, we aim to decompose the model's internal knowledge into meaningful directions.

% This question touches on a fundamental limitation in how we interact with diffusion models. Current control methods ~\cite{zhang2023addingconditionalcontroltexttoimage, gandikota2023concept, ye2023ipadaptertextcompatibleimage,ye2023ipadaptertextcompatibleimage, hertz2024stylealignedimagegeneration, li2023photomaker, shi2024instantbooth, chen2024trainingfreeregionalpromptingdiffusion} require users to specify their desired manipulations in advance, limiting interactive creativity. This contrasts with natural human artistic workflows, where creators dynamically explore creative ideas while jointly refining them toward meaningful artistic outcomes~\cite{hoffmann2016modeling}. This synergy between specification and exploration is not new to generative models. Early GAN architectures naturally developed disentangled latent spaces that enabled continuous\cite{harkonen2020ganspace,radford2015unsupervised, wu2021stylespace, shen2020interfacegan}, compositional control over generated images. Users could explore these spaces to discover interesting variations that would be difficult to describe in words~\cite{wu2021stylespace}, then combine them to achieve their creative goals~\cite{grabe2022towards}. 


% While diffusion models have largely superseded GANs in conditional image synthesis~\cite{dhariwal2021diffusion},  their underlying structure remains less understood. Diffusion models achieve remarkable diversity through high-dimensional latents, unlike GANs' compact latent spaces.  With a single prompt, diffusion models can generate radically different variations through different random initializations of input noise. We ask - Is it possible to discover interpretable structure within this vast space of variations?

Text-to-image diffusion models are capable of generating remarkable visual variations from a single prompt through different random initializations. However, this vast creative potential remains largely opaque to users---while we can generate diverse images, we lack understanding of the underlying structure of these variations. This presents a fundamental challenge: how can we discover and expose the latent visual capabilities encoded within these models?

\let\thefootnote\relax \footnote{$^{*}$Correspondence to \texttt{gandikota.ro@northeastern.edu}}

The challenge touches on a key limitation in how we interact with diffusion models today. Current control methods require users to explicitly specify their desired edits in advance through prompts~\cite{gandikota2023concept}, reference images~\cite{zhang2023addingconditionalcontroltexttoimage, chen2024trainingfreeregionalpromptingdiffusion, ruiz2022dreambooth,kumari2022customdiffusion, Ryu_lora, hu2021lora}, or attribute vectors~\cite{ye2023ipadaptertextcompatibleimage, hertz2024stylealignedimagegeneration, li2023photomaker, shi2024instantbooth,parmar2023zero,hertz2022prompt}. That contrasts sharply with natural human creative workflows, where artists dynamically explore creative ideas and jointly refine them toward meaningful artistic outcomes~\cite{hoffmann2016modeling}. The need for pre-specified controls creates a barrier between users and the full creative potential of these models.

Interestingly, earlier generative models like GANs~\cite{gans,karras2019style,brock2018large} naturally developed more interpretable internal structures. Their compact latent spaces often exhibited emergent disentanglement~\cite{harkonen2020ganspace,radford2015unsupervised, wu2021stylespace, shen2020interfacegan}, enabling continuous and compositional control over generated images. Users could explore these spaces to discover interesting variations that would be difficult to describe in words~\cite{wu2021stylespace}, then combine them to achieve their creative goals~\cite{grabe2022towards}.

Diffusion models have largely superseded GANs in conditional image synthesis~\cite{dhariwal2021diffusion}, achieving greater diversity through much higher-dimensional latents. And yet an understanding of the underlying structure of these larger latent spaces has remained elusive. In this work, we ask a fundamental question: \emph{Can we automatically discover the visual structure within a diffusion model's knowledge of a concept?} Rather than requiring user-specified controls, we aim to decompose the model's internal representations into expressive directions that users can explore and combine.

To address these needs, we present \textbf{SliderSpace}, a framework that brings systematic explorability to diffusion models. Given just a text prompt, SliderSpace discovers a canonical set of meaningful, diverse, and controllable directions within the model's knowledge of that concept. Each direction is implemented as a low-rank adapter~\cite{hu2021lora} that can be scaled and composed with others, allowing users to explore and smoothly combine different aspects of variation, as shown in Figure~\ref{fig:intro}.

We ground SliderSpace discovery in three key requirements for meaningful decomposition of a diffusion model's visual manifold: 
\begin{enumerate}
    \item \textbf{Unsupervised Discovery:} The decomposition process should emerge from the intrinsic structure of the model's learned representation, rather than being guided by predefined attributes. This ensures we capture the true topology of the model's knowledge space rather than projecting our assumptions onto it.
    
    \item \textbf{Semantic Orthogonality:} Each discovered control must represent a distinct semantic direction. This is enforced in a semantic feature space, like CLIP, where every slider has an orthogonal effect in embeddings. This prevents discovering multiple controls that create similar semantic effects, making the system more efficient and easier.
    
    \item \textbf{Distribution Consistency:} Directions must induce consistent transformations across both random seeds and prompt variations. 
\end{enumerate}

These requirements naturally lead to our proposed framework, which we formalize in Section~\ref{sec:method}. As we show in our experiments, SliderSpace is architecture-agnostic, working with both conventional U-Net based models like Stable Diffusion~\cite{rombach2022high, rombach2022sd20, podell2023sdxl, turbo, dmd} and recent transformer-based architectures like Flux~\cite{flux}.

We demonstrate the expressiveness of SliderSpace through three applications: First, we show how SliderSpace can decompose high-level concepts into diverse and expressive components, revealing the natural axes of variation in the model's understanding. Second, we explore artistic style variation, where SliderSpace discovers directions that match or exceed the diversity of manually curated artist lists while being judged more useful by human evaluators. Finally, we show how SliderSpace can help reverse the mode collapse commonly observed in distilled diffusion models, restoring diversity while maintaining generation speed.

Beyond providing practical creative control, SliderSpace opens new avenues for understanding and utilizing the latent capabilities of diffusion models. By mapping these models' visual potential into intuitive, composable directions, we take a step toward making their creative possibilities more accessible and interpretable to users.

% Image editing methods in diffusion models unlock the creativity of users. In this work we ask an alternate question: \emph{Can we organize and expose what of the diffusion model is already capable of?}.
% Existing methods for controlling image generation typically require users to manually specify edit directions for desired changes. This process is time-consuming, requires technical expertise, and limits the spontaneity of the creative process. For instance, if a user wants to adjust the smile of a generated person, they must explicitly request this edit, often through imprecise prompt engineering or model fine-tuning. This approach of predefined controls or manual specifications restricts users from fully exploring the latent capabilities of the model. There may be interesting stylistic variations or attributes that the model can generate, but users have no easy way to discover or utilize these.

% Natural visual disentanglement was an emergent property in the latent space of Generative Adversarial Models (GANs) \cite{harkonen2020ganspace,radford2015unsupervised, wu2021stylespace, shen2020interfacegan}. In particular, it has been observed that StyleGAN~\cite{karras2019style} stylespace neurons offer detailed control over many meaningful aspects of images that would be difficult to describe in words~\cite{wu2021stylespace}. However, diffusion models do not share such a compact latent space~\cite{park2023unsupervised}; and efforts to uncover such a space in the semantic embeddings of the text conditioning have met with limited success \nik{Nick - is there a specific citation you were thinking about?}.

% In this work we introduce \textbf{SliderSpace}, which takes a step towards uncovering an analogous low dimensional representation of diffusion models' visual breadth; in essence treating the diffusion model as many generators sharing parameters, where a particular generator is defined by a specific prompt. For a given prompt we sample many random seeds (and optionally prompt expansions using an LLM), generate the corresponding images, and apply an off the shelf feature extractor (in this work CLIP, but our method can be applied to any differentiable feature extractor). We use PCA to analyze these features, and for each of the leading $k$ principal components we train a LoRA \cite{} which causes the diffusion model to produces images which increase the feature magnitude along that component when passed back through the same feature extractor. This leads to a 'Slider' for each principal component, because each LoRA can be scaled and applied to the original diffusion model, continuously varying those visual features in the generated results (as measured, in our case, by CLIP).

% There are many other works that enhance the controllability of diffusion models. One common approach is enabling users to add spatial constraints to a generation either manually, or via a reference image \cite{zhang2023addingconditionalcontroltexttoimage, chen2024trainingfreeregionalpromptingdiffusion}, a second is leveraging more abstract embeddings (e.g. identity, style) extracted from a reference image \cite{ye2023ipadaptertextcompatibleimage, hertz2024stylealignedimagegeneration, li2023photomaker, shi2024instantbooth}, a third is finetuning a foundation model to better generate a concept important to the user \cite{ruiz2022dreambooth, kumari2022customdiffusion, Ryu_lora, hu2021lora}, and a fourth (most relevant to this work) is finding low-rank adaptors of the model based on a prompt or small training set which can be scaled to provide continous control over one aspect of generated image (e.g. night vs day, basic vs luxury, etc.) \cite{gandikota2023concept}. SliderSpace is complementary to all of these methods and offers something distinct. All of the other methods we are aware require the user (and / or model designer) to know in advance what type of control they want. In contrast SliderSpace assists users in discovering and controlling hidden capabilities present in the diffusion model's distribution of possible generations.

%We propose that truly intuitive creative control in a text-to-image model should meet three key criteria: \emph{discoverability}, \emph{intuitiveness}, and \emph{specificity}. The model should reveal controllable attributes that may not be immediately obvious, offer controls that are easy to understand and manipulate, and ensure each control affects a distinct attribute of the generated image.

% We demonstrate the utility and power of SliderSpace using three applications built on top of SDXL-DMD \cite{dmd}, because its fast generation speed lends itself well to the continuous control offered by SliderSpace.

% First, we study concept decomposition (Section \ref{sec:concept_exp}), where we learn sliders for a specific concept (e.g. 'monster', 'waterfall', 'car'). Through quantitative metrics of diversity and text alignment we demonstrate that the learned sliders dramatically boost the diversity of generations when randomly applied without harming text alignment; we also ask humans to qualitatively judge these results in a user study where they find the SliderSpace results to be more 'Diverse', 'Useful', and 'Creative' than our baselines.

% Second, we attempt to compare the automatic discoveries of SliderSpace to a large scale manual study of artistic styles (Section \ref{sec:art_exp}), open-sourced by ParrotZone \cite{parrotzone}. In this study SDXL was prompted with over 4300 artist names,  and based on visual inspection the cases of successful stylistic mimicry recorded. Quantitatively SliderSpace more closely matches the distribution of artistic variation discovered by ParrotZone than other baselines, and in our user studies was judged to be significantly more 'Diverse' and 'Useful' than the baselines. To our surprise humans even judged SliderSpace results to be slightly more 'Diverse' than the results generated by the manually discovered artist names of \cite{parrotzone}.

% Third, we attempt to use SliderSpace to reverse the mode collapse commonly observed in distilled few-step diffusion models relative to the original teacher model (Section \ref{sec:diverse_exp}). We quantitatively demonstrate that applying SliderSpace to SDXL-DMD leads to more closely matching the distribution of images by the original teacher, SDXL.

%Through extensive experiments on various state-of-the-art text-to-image models, we demonstrate that SliderSpace significantly enhances user control and creative expression in AI-assisted image generation tasks. Our method enables a range of applications, including concept decomposition and control, diversity improvement in generated images, customization dissection and edits, and the exploration of artistic styles inherent in the model.

% SliderSpace goes beyond providing a practical tool for enhanced creative control. By mapping the visual potential of diffusion models it can open new avenues for generative creativity and deepens our understanding of each model's hidden potential.

\section{System model} \label{Sec:Baseline}
This work considers a multi-user \gls{mMIMO} system where a \gls{BS} with $N_{\sf{T}}$ transmit antennas serves $N_{\sf{U}}$ single-antenna users simultaneously. Let $x_u$ represent the transmitted symbol for each user. The received signal at the $u^{th}$ user can be represented as 
\begin{equation}\label{eq:signal_recived}
    \mathbf{y}_u =  \mathbf{h}_{u}^{\dagger} \sum_{\forall u}  \mathbf{w}_{u} x_u + \bs{\eta} \,,
\end{equation}
in which $\mathbf{h}_u \in \mathbb{C}^{N_{\sf{T}} \times 1}$ is the wireless channel vector between the \Gls{BS} and $u^{th}$ user; the term $\bs{\eta} \sim \mathcal{CN}(0, \sigma^2)$ denotes complex symmetric Gaussian noise with zero mean and a variance of $\sigma^2$; the downlink transmit \gls{FDP} vector is denoted as $\mathbf{W} = \left [ \mathbf{w}_1, \hdots, \mathbf{x}_u, \hdots, \mathbf{w}_{N_{\sf{U}}} \right ] \in \mathbb{C}^{N_{\sf{T}} \times N_{\sf{U}}}$.
The corresponding \gls{SINR} for user $u$ is formulated as
\begin{equation}
    \text{SINR}(\mb{w}_{u}) = \frac{ \big|\mb{h}^{\dagger}_{u} \mb{w}_{u} \big|^2}{\sum_{j \neq u} \big|\mb{h}^{\dagger}_{u} \mb{w}_{j} \big|^2 + \sigma^2} \,.
\end{equation}
The goal is to find such a precoding $\mathbf{W}$ that maximizes the throughput under the maximal transmit power $P_{\text{max}}$ constraint. Thus, the downlink sum rate maximization problem can be formulated as
\begin{align}\label{eq:sum rate-optimization}
    & \underset{\mb{W}}{\max}~ R(\mb{W}) \,, \\
    \text{s.t.}  &\sum_{\forall u} \mb{w}_{u}^{\dagger}  \mb{w}_{u} \leq P_{\sf{max}}\,,
\end{align}
where for a precoding matrix $\mathbf{W}$, the sum rate is
\begin{equation}\label{eq:sum rate}
    R(\mb{W}) = \sum_{\forall u} \text{log}_2 \Bigl(  1+ \text{SINR}(\mb{w}_{u}) \Bigr).
\end{equation}

\subsection{Problem Definition}

Beamforming is crucial in wireless communication, particularly at mmWave frequencies, where directional signal transmission is essential due to the limitations of omnidirectional antennas. \Glspl{DNN} have shown promise in maximizing the sum rate, but despite their enhanced performance, \glspl{DNN} come with high computational complexity and substantial energy requirements, posing challenges for deployment on resource-limited devices like edge platforms.

To tackle the complexities and energy demands associated with \gls{DNN}-based precoders in mMIMO systems, we focus on developing a method that preserves the high throughput of \glspl{DNN} while making them suitable for resource-constrained environments. This requires overcoming inefficiencies caused by uniformly applied quantization, which overlooks the varying sensitivity of different network layers to precision reduction.

In this work, we introduce a novel framework that combines mixed-precision quantization-aware training with \gls{NAS}. Our method is designed to significantly lower the energy consumption of \glspl{DNN} in precoder design while sustaining high throughput for site-specific \glspl{BS}. Extensive experiments validate that our approach achieves superior energy efficiency compared to existing \gls{DL}-based and conventional beamforming techniques.




\section{Methodology} \label{Sec:Proposed}
\section{Method}

\begin{figure*}[t]
    \centering
    \includegraphics[width=\linewidth]{figures/pipeline.png} \hfill

    \caption{An overview of our data synthesis pipeline. Starting from our seed data, we select a reference sample and collect \textsc{Reference-Level Feedback} on both the instruction and response. Instruction feedback is used to synthesize new instructions. We generate their corresponding responses, and then improve it using the response feedback.}
    \label{fig:pipeline}
\end{figure*}

In this section, we present our data synthesis pipeline that leverages \textsc{Reference-Level Feedback} to generate high-quality instruction-response pairs. An overview of the pipeline is presented in Figure \ref{fig:pipeline}, and the steps are detailed in the following subsections. Complete examples for each step can be found in Appendix \ref{sec:appendix_examples}, and the prompts used for each section can be found in Appendix \ref{sec:appendix_prompt_templates}.


\subsection{Feedback Collection}

Our pipeline begins with a seed dataset -- a small collection of carefully curated instruction-response pairs that serve as exemplars for synthesized data samples. It can be either manually crafted by human annotators or automatically selected using quality-based criteria. These reference samples are high-quality and exhibit desirable characteristics such as clarity and relevance, which we aim to replicate in our synthetic data. For \textsc{Reference-Level Feedback}, we systematically identify and capture such qualities through a framework that identifies the strength of each sample, as well as potential areas for improvement.

Unlike traditional approaches that collect feedback on generated responses at the sample-level, our method identifies the qualities that make reference samples high-quality and uses it for feedback. This feedback captures a richer signal than feedback collected at the sample-level, establishing higher quality standards for synthesis and providing more effective guidance for generating training data that exhibits similar properties to the reference samples.

For each reference sample in the seed dataset, we collect \textsc{Reference-Level Feedback} from both the instruction and the response:

\textbf{Instruction Feedback.} To collect feedback from a reference instruction and capture essential features that make it effective for training, we analyze key attributes (e.g., clarity and actionability). We also ensure comprehensive coverage along a wide breadth by collecting feedback along two dimensions: relevant subject areas (e.g. cellular biology, csv file manipulation, legislative processes) and relevant skills necessary to respond to the instruction (e.g. understanding of specific tools, knowledge of processes, analysis). This enables us to systematically identify desirable characteristics of instructions while maximizing the breadth of instruction types.

\textbf{Response Feedback.} When collecting feedback from a reference response, we identify key qualities that make it an effective response to the instruction. We evaluate along multiple critical dimensions, including factual accuracy, relevance to the instruction, and comprehensiveness. This feedback captures both the strengths of the reference response and specific areas where it can be improved upon.


\subsection{Data Synthesis}
Now, we use the collected \textsc{Reference-Level Feedback} from the previous stage to synthesize new data samples, while maintaining the quality standards established by our reference data. For each reference sample and its corresponding feedback, we employ a two-phase synthesis process, as illustrated in Figure \ref{fig:pipeline}:

\begin{enumerate}
    \item \textbf{Instruction Synthesis.} We provide an LLM the reference instruction as an example and the instruction feedback as guidelines to synthesize new instructions that maintain the qualities specified in the feedback. As depicted in Step 2 of Figure \ref{fig:pipeline}, we synthesize 10 new instructions for \textbf{subject-based} feedback, which produces instructions that align with the subject areas of the reference response. We also synthesize 10 new instructions for \textbf{skill-based} feedback, which produces instructions that align with the skills needed to respond to the reference instruction.
    
    \item \textbf{Response Synthesis and Refinement.} For each synthesized instruction, we first generate an initial response. We then enhance this response using the reference response feedback, instructing the language model to analyze the feedback and incorporate the relevant aspects. This process is shown in Step 3 of Figure \ref{fig:pipeline}.
    
    \paragraph{Note on relevance of response feedback.}
    Although the response feedback was originally collected for the reference response, many aspects of it can still remain applicable because of the shared characteristics between the reference and synthesized instructions. We acknowledge that not all feedback elements may transfer, and to account for this, we explicitly instruct the model to selectively apply only the relevant aspects of the feedback and ignore the irrelevant aspects. An example of this can be found in \ref{sec:appendix_examples}.
\end{enumerate}

This synthesis process enables us to synthesize new data, while systematically propagating the high-quality characteristics of reference samples.

\subsection{Theoretical Efficiency Analysis}
Our presented pipeline for data synthesis with \textsc{Reference-Level Feedback} is significantly more efficient than using traditional sample-level feedback methods, specifically in the frequency of feedback collection. While sample-level approaches require feedback for every synthesized sample, our method only requires feedback once for every reference sample. This translates to a reduction from $O(n)$ feedback collections, where $n$ represents the number of synthesized samples, to $O(1)$. However, it is also important to note that this efficiency gain comes with an initial fixed cost of collecting and curating seed data.

\section{Numerical Results} \label{Sec:Simulation}
\subsection{Dataset Definition}
A custom dataset was generated to accurately reflect the channel characteristics pertinent to beamforming in mMIMO systems. MATLAB’s Ray-Tracing toolbox simulated both \gls{LOS} and \gls{NLOS} conditions. The simulations positioned a \gls{BS} within the Montreal region, utilizing environmental data from OpenStreetMap \cite{OpenStreetMap} for realism. The base station employed a uniform planar array antenna with 8x8 elements, spaced at half-wavelength, operating at a frequency of 2\,GHz. The transmitter was set at a height of 20\,m and powered at 20\,W, assuming a system loss of 10\,dB. Users were placed in circular patterns around the base station at distances ranging from 50 to 350\,m and at 10-degree intervals. This configuration captured a wide variety of deployment scenarios and channel conditions. The ray-tracing simulations considered up to 10 reflections but excluded diffraction effects, focusing on signal reflections from buildings and terrain to emulate multipath propagation in urban environments accurately.

\subsection{Energy Consumption Examples}
We first present some examples of the energy consumed by the different approaches, as per the model presented in Sections~\ref{sec:energy_dnn} and \ref{sec:energy_baselines}.
We consider the ``UdeM-NLOS'' scenario.
In Table~\ref{table:energy_comparison}, we report the energy for the ``default'' variants of the method, that is, for \gls{WMMSE}, we set the stopping criterion to $10^{-5}$ to have near-optimal performance, and for the DNN, we use $C_{\text{out}}$\,$=$\,$64$ and $D_{\text{FCL}}$\,$=$\,$1024$ with the maximum weight resolution of 16 bits.
We see that the DNN consumes significantly less than WMMSE at the cost of a slight degradation in sum rate (on this scenario). ZF, on the other hand, is much less complex, but does not provide competitive performance in \gls{NLOS} conditions, or at low \gls{SNR}.
As a result, ZF is unlikely to be a favored solution in practice, since it results in significant under-utilization of the BS resources.

\begin{table}[t]
    \centering
    \caption{Energy Comparison of the Default Variants on ``U\lowercase{de}M-NLOS''  ($N_{\sf{T}}$\,$=$\,$64$,~$N_{\sf{U}}$\,$=$\,$4$,~SNR\,$=$\,15\,dB).}
    \begin{tabular}{ccc}
        \toprule
        \textbf{Method} & \textbf{Energy ($\mu J$)} & \textbf{Sum Rate} (bit/s/Hz) \\
        \midrule
        WMMSE ($I$\,$=$\,$92.8$) & 296 & 19.2\phantom{$~\pm~0.007$}\\
        \textbf{Default DNN} & \textbf{54.5} & \textbf{18.9}$~\pm~0.007$\\
        Zero-Forcing (ZF) & 0.008  & 15.3\phantom{$~\pm~0.007$}\\
        \bottomrule
    \end{tabular}
    \label{table:energy_comparison}
\end{table}

\begin{table}[t]
    \centering
    \caption{DNN Energy Consumption with Uniform Quantization 
    \\for $C_{\text{out}}$\,$=$\,$64$, $D_{\text{FCL}}$\,$=$\,$1024$ 
    on ``U\lowercase{de}M-NLOS''\\
    ($N_{\sf{T}} $\,$=$\,$ 64$, $N_{\sf{U}}$\,$=$\,$4$, SNR\,$=$\,15\,dB)}
    \resizebox{\columnwidth}{!}{
    \begin{tabular}{ccc}
        \toprule
        \textbf{Quantization Configs}
        & \textbf{Energy ($\mu J$)}& \textbf{Sum Rate} (bit/s/Hz)\\
        \midrule
        $[16,\,16,\,16,\,16]$& 54.5 & 18.9$~\pm~0.007$\\
        $[8,\,8,\,8,\,8]$ & 17.5 & 18.6$~\pm~0.012$ \\
        $[4,\,4,\,4,\,4]$ & 7.4 & 17.8$~\pm~0.014$\\
        $[2,\,2,\,2,\,2]$ & 4.6 & 17.1$~\pm~0.032$\\
        \bottomrule
    \end{tabular}
    }
    \label{tab:energy_consumption}
\end{table}

Next, to illustrate the impact of quantization, 
Table~\ref{tab:energy_consumption} lists the energy consumption of the DNN for various uniform bit-width configurations, for the same $C_{\text{out}}$ and $D_{\text{FCL}}$ as in Table~\ref{table:energy_comparison}. We see that lowering the weight resolution leads to substantial energy savings but at the cost of a moderate decrease in performance.

\begin{figure}[!t]
    \centering
    \includegraphics[width=\columnwidth]{fig/result_ICC_1_pareto_modified_final.pdf}
    \caption{Trade-off between energy efficiency and sum rate for \glspl{CNN} with varying $C_{\text{out}}$, $D_{\text{FCL}}$, and MPQ bit widths, on the ``UdeM-LOS'' scenario ($N_{\sf{T}}$\,$=$\,$64$, $N_{\sf{U}}$\,$=$\,$4$, average SNR\,$=$\,29\,dB). All the model configurations that were evaluated are shown, while the curves provide the Pareto front associated with each architecture configuration.}
    \label{fig:Neural Architecture Search (NAS)}
    \vspace{-10pt}
\end{figure}

\subsection{NAS Results}

Figure \ref{fig:Neural Architecture Search (NAS)} highlights the trade-offs between computational energy efficiency (bits/s/Hz/$\mu$J) and sum rate (bits/s/Hz) across diverse \gls{DNN} configurations generated as described in Section~\ref{Sec:Proposed}, on the ``UdeM-LOS'' scenario.
Each curve shows the Pareto front corresponding to a particular architecture configuration, while each point in this curve uses a different quantization configuration.

A few trends can be mentioned among the Pareto-optimal results for each architecture.
Firstly, the first layer, CONV, is often kept at high precision, particularly in smaller models, to maintain performance while reducing energy consumption. Interestingly, no Pareto-optimal model uses uniform quantization across all layers. Moreover, even models that achieve the highest sum rates do not employ more than two layers at the highest precision. These results emphasize the importance of efficiently distributing bit precision across layers to optimize energy consumption.

The configuration yielding the highest energy efficiency is ($C_{out}$\,$=$\,$8$, $D_{FCL}$\,$=$\,$512$) with quantization [2,\,8,\,8,\,8], achieving 26.2 \,bits/s/Hz/$\mu$J at a moderate sum rate of 28.5\,bits/s/Hz. On the other hand, the configuration achieving the highest sum rate is ($C_{out}$\,$=$\,$64$, $D_{FCL}$\,$=$\,$ 1024$) with quantization [16,\,16,\,4,\,8], which reaches 31.3\,bits/s/Hz but does not use full precision, making it more interesting from an energy efficiency perspective. An alternative worth mentioning is the model with the second-best sum rate of 31.1\,bits/s/Hz, achieved with an architecture of ($C_{out}$\,$=$\,$64$, $D_{FCL}$\,$=$\,$1024$) and quantization [16,\,2,\,2,\,2], which uses nearly $5\times$ less energy than the highest sum-rate model.
We do observe compute energy efficiency decreasing rapidly near the highest sum-rate, but of course this could simply mean that switching to a larger and/or different DNN architecture would be preferable at that point.

To further illustrate the importance of of NAS and model compression in finding efficient DNN precoders, Figure~\ref{fig:Neural Architecture Search (NAS)} includes horizontal and vertical arrows that quantify the impact on performance of the design choices. The horizontal arrow measures the difference in sum rate between the worst and best configurations at equal energy efficiency, revealing a 20\% improvement through optimal model selection. Similarly, the vertical arrow shows the difference in energy efficiency at an equal sum rate, demonstrating a $14.5\times$ gain.


\begin{figure}[!t]
\centering
\includegraphics[width=\columnwidth]{fig/result_ICC_5.pdf}
\caption{Comparison of energy efficiency (bits/s/Hz/$\mu$J) and sum rate (bits/s/Hz) across two environments: UdeM-NLOS (average SNR\,$=$\,15\,dB) and Okapark-LOS (average SNR\,$=$\,28\,dB). The proposed method achieves a superior balance of energy efficiency and sum rate performance compared to WMMSE. Results are derived for models with varying ($C_{out}$) and ($D_{FCL}$).}
\vspace{-9pt}
\label{fig:LSQ Quantization}
\end{figure}
\subsection{Impact of Deployment Environment and Energy Efficiency Comparison}

Figure~\ref{fig:LSQ Quantization} compares energy efficiency (bits/s/Hz/$\mu$J) and sum-rate (bits/s/Hz) across two contrasting deployment scenarios: UdeM-NLOS, characterized by challenging multipath conditions, and Okapark-LOS, offering clear \gls{LOS} signal propagation. These two scenarios highlight the adaptability and effectiveness of the proposed quantized models.
For the DNN precoders, each curve shows the Pareto-optimal configurations across the entire search space, whereas for WMMSE, the trade-off between sum-rate and energy efficiency is varied by adjusting the stopping criterion.

In the UdeM-NLOS environment, sum rates are constrained between 15 and 20 bits/s/Hz due to severe signal attenuation and multipath effects. Despite these limitations, the quantized models achieve significant energy savings, with improvements of up to $35\times$ in energy efficiency compared to the WMMSE baseline, all while maintaining competitive sum rates.

In contrast, the Okapark-LOS environment, which benefits from clear signal paths, supports higher sum rates ranging from about 30 to 40 bits/s/Hz. 
Depending on the desired sum-rate, the DNN precoders can provide improvements in energy efficiency ranging from $6.1\times$ to $1.2\times$. However, with the DNN architecture template and training method considered in this paper, the DNN precoder is unable to achieve the highest sum-rate that can be provided by WMMSE.
%Here, the quantized models continue outperforming the WMMSE baseline, delivering a 6.1× improvement in energy efficiency while maintaining similar or superior sum-rate performance. The highest sum rate, achieved by the DNN for Okapark-LOS ($C_{out} = 64$, $D_{FCL} = 1024$), is roughly 1.3× higher, highlighting an intriguing balance of performance and precision.

These results emphasize the adaptability of the quantized models across diverse deployment environments, and their ability to achieve the same performance with less compute energy. Interestingly, the energy gains provided by DNNs appear to be larger in more difficult (low SNR, non line-of-sight) environments.






\section{Conclusion} \label{Sec:Conclusion}
\section{Limitations and Future Work}
The proposed OpenFly platform incorporates various rendering engines/techniques to provide high-quality scenes. Specifically, this is the first attempt to use 3D GS reconstructed scenes to support real-to-sim training and testing, while in the reconstruction of large-scale areas, a few visual artifacts are inevitably present. Future work will focus on exploring more effective reconstruction methods to enhance realism in large-scale scenes. Besides, the proposed OpenFly-Agent is built upon the large VLN model architecture, which is not practical for real-time deployment on UAVs. To address this, future research should focus on developing more efficient architectures and effective quantization techniques. 


\section{Conclusion}
In this work, we present OpenFly, a platform designed for large-scale data collection in aerial Vision-and-Language Navigation (VLN). OpenFly integrates multiple rendering engines and advanced real-to-sim techniques for data generation, enabling efficient collection of diverse, high-quality aerial VLN data. The resulting large-scale dataset comprises 100k trajectories across 18 distinct scenes, spanning a wide range of altitudes and difficulty levels, which is significantly superior than existing ones. Furthermore, we propose OpenFly-Agent, a keyframe-aware aerial navigation model capable of directly predicting flight actions based on observations and language instructions. Extensive experiments validate the effectiveness of the proposed method, and establishing a comprehensive benchmark for future advancements in aerial navigation. 
%The toolchain, dataset, and code will be publicly released, providing a valuable resource for future research in this field.


\section*{Acknowledgement}
This work was supported by Ericsson - Global Artificial Intelligence Accelerator AI-Hub Canada in Montr\'{e}al and jointly funded by NSERC Alliance Grant 566589-21 (Ericsson, ECCC, Innov\'{E}\'{E}).

\bibliographystyle{IEEEtran}
% \bibliography{am_ger_eng,rubi_eng}
% \bibfont \footnotesize
\bibliography{0_main}


\end{document}