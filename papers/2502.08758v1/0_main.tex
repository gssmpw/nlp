\documentclass[conference]{IEEEtran}
%****************this is required for ICMLCN**********
\usepackage[letterpaper, left=0.65in, right=0.65in, bottom=1in, top=0.8in]{geometry}
\IEEEoverridecommandlockouts
\special{papersize=8.5in,11in}
%****************************************************
%****************************************************%
%               LIST OF ACRONYM HERE                 %
%****************************************************%
\usepackage[acronym]{glossaries}
\newacronym{WMMSE}{WMMSE}{weighted minimum mean square error}
\newacronym{FCL}{FCL}{fully connected layer}
\newacronym{NLOS}{NLOS}{non-line-of-sight}
% \newacronym{EE}{EE}{energy efficiency}
\newacronym{LOS}{LOS}{line-of-sight}
\newacronym{MAC}{MAC}{multiply-accumulate}
\newacronym{BS}{BS}{base station}
\newacronym{PS}{PS}{phase-shifter}
\newacronym{NAS}{NAS}{neural architecture search}
\newacronym{PTQ}{PTQ}{post-training quantization}
\newacronym{QAT}{QAT}{quantization-aware training}
\newacronym{LSQ}{LSQ}{Learned Step Size Quantization}
\newacronym{STE}{STE}{Straight Through Estimator}
\newacronym{MPQ}{MPQ}{Mixed-Precision Quantization}
\newacronym{RL}{RL}{reinforcement learning}
\newacronym{AP}{AP}{analog precoder}
\newacronym{FC-HBF}{FC-HBF}{fully-connected HBF}
\newacronym{FSA-HBF}{FSA-HBF}{fixed subarray HBF}
\newacronym{DSA-HBF}{DSA-HBF}{dynamic subarray HBF}
\newacronym{BF}{BF}{beamforming}
\newacronym{UE}{UE}{user equipment}
\newacronym{AWGN}{AWGN}{additive white gaussian noise}
%\newacronym{BS}{BS}{base station}
\newacronym{MIMO}{MIMO}{multiple-input multiple-output}
\newacronym{MISO}{MISO}{multiple-input single-output}
\newacronym{RF}{RF}{radio frequency}
\newacronym{RIS}{RIS}{reconfigurable intelligent surfaces}
\newacronym{IOT}{IOT}{internet-of-things}
\newacronym{QNN}{QNN}{Quantized Neural Network}
\newacronym{CL}{CL}{convolutional layer}
\newacronym{UdeM-NLOS}{UdeM-NLOS}{Universite De Montreal-Non-Line-Of-Sight}
\newacronym{Okapark-LOS}{Okapark-LOS}{Okapark-Line-Of-Sight}
\newacronym{FDD}{FDD}{frequency division duplex}
\newacronym{TDD}{TDD}{time division duplex}
\newacronym{CSI}{CSI}{channel state information}
\newacronym{DNN}{DNN}{Deep Neural Network}
\newacronym{DP}{DP}{digital precoder}
\newacronym{DL}{DL}{deep learning}
\newacronym{SVD}{SVD}{singular-value decomposition}
\newacronym{CNN}{CNN}{convolution neural network}
\newacronym{FDP}{FDP}{fully digital precoder}
\newacronym{SE}{SE}{spectral efficiency}
\newacronym{OFDM}{OFDM}{orthogonal frequency division multiplexing}
\newacronym{OMP}{OMP}{orthogonal matching pursuit}
\newacronym{FL}{FL}{fully-connected layer}
\newacronym{HSHO}{HSHO}{Hybrid Structured Heuristic Optimization}
\newacronym{HBF}{HBF}{hybrid beamforming}
\newacronym{IA}{IA}{initial access}
\newacronym{mm-Wave}{mm-Wave}{millimeter wave}
\newacronym{mMIMO}{mMIMO}{massive multiple-input multiple-output}
\newacronym{SINR}{SINR}{signal-to-interference-noise ratio}
\newacronym{SNR}{SNR}{signal-to-noise ratio}
%\newacronym{SS}{SS}{synchronization signal}
%\newacronym{SSB}{SSB}{synchronization signal burst}
\newacronym{RSSI}{RSSI}{received signal strength indicator}
\newacronym{PZF}{PZF}{phase zero forcing}
\newacronym{PSO}{PSO}{particle swarm optimization}
\newacronym{ZF}{ZF}{zero forcing}
\newacronym{O-FDP}{O-FDP}{optimal fully digital precoder}
\newacronym{JT}{JT}{joint transmission}
\newacronym{CU}{CU}{central unit}
\newacronym{MSE}{MSE}{mean square error}
\newacronym{CEL}{CEL}{cross entropy loss}
\newacronym{CB}{CB}{conjugate beamforming}
\newacronym{NC}{NC}{network controller}
\newacronym{CoMP}{CoMP}{coordinated multi point}
\newacronym{CF-mMIMO}{CF-mMIMO}{cell-free massive MIMO}
\newacronym{CF-HBF}{CF-HBF}{cell-free hybrid beamforming}
\newacronym{CF-BF}{CF-BF}{cell-free beamforming}
\newacronym{MLDG}{MLDG}{Meta-Learning Domain Generalization}
\newacronym{MAML}{MAML}{Model Agnostic Meta-Learning}
\newacronym{WSR}{WSR}{Weighted Sum Rate}

\newif\ifDeepMIMOModel
\DeepMIMOModeltrue

\newif\ifSimpleNParamEq
\SimpleNParamEqtrue

\usepackage{lettrine}
\usepackage[ruled,vlined,linesnumbered]{algorithm2e}
\usepackage{multirow}
\usepackage{longtable}
\usepackage[table,xcdraw]{xcolor}
\usepackage{array} 
\makeatletter
\let\oldlt\longtable
\let\endoldlt\endlongtable
\def\longtable{\@ifnextchar[\longtable@i \longtable@ii}
\def\longtable@i[#1]{\begin{figure}[t]
\onecolumn
\begin{minipage}{0.5\textwidth}
\oldlt[#1]
}
\def\longtable@ii{\begin{figure}[t]
\onecolumn
\begin{minipage}{0.5\textwidth}
\oldlt
}
\def\endlongtable{\endoldlt
\end{minipage}
\twocolumn
\end{figure}}
\makeatother
\usepackage{ifpdf}
% *** GRAPHICS RELATED PACKAGES ***
\ifCLASSINFOpdf
  \usepackage[pdftex]{graphicx}
  \usepackage{graphicx,epstopdf}
  \graphicspath{{../pdf/}{../jpeg/}}
  \DeclareGraphicsExtensions{.pdf,.jpeg,.png,.eps}
\else
  \usepackage[dvips]{graphicx}
  \usepackage{graphicx,epstopdf}
  \graphicspath{{../eps/}}
  \DeclareGraphicsExtensions{}
\fi
\usepackage{tikz}
\usetikzlibrary{decorations.pathreplacing,calc}
\newcommand{\tikzmark}[1]{\tikz[overlay,remember picture] \node (#1) {};}

\newcommand*{\AddNote}[4]{%
    \begin{tikzpicture}[overlay, remember picture]
        \draw [decoration={brace,amplitude=0.5em},decorate,line width=.2mm,black]
            ($(#3)!(#1.north)!($(#3)-(0,1)$)$) --  
            ($(#3)!(#2.south)!($(#3)-(0,1)$)$)
                node [align=center, text width=2.5cm, pos=0.5, anchor=west] {#4};
    \end{tikzpicture}
}%

\usepackage{cite}
\usepackage{amsthm}
\usepackage{steinmetz}
\usepackage{amssymb}
\usepackage{balance}
\usepackage{eqparbox}
\usepackage{multirow}
\usepackage{bbm}
\usepackage{float}
\SetAlFnt{\small}


\usepackage{amsmath}

\usepackage{mathtools, nccmath}
\DeclarePairedDelimiter{\nint}\lfloor\rceil
\DeclarePairedDelimiter{\abs}\lvert\rvert

\usepackage{booktabs, siunitx}
\usepackage{dcolumn}
\newcolumntype{d}[1]{D{.}{.}{#1}}
% \usepackage[svgnames,table]{xcolor}
%\usepackage[tableposition=above]{caption}

% \usepackage{graphicx}
\usepackage[scaled]{DejaVuSansMono}
\usepackage[T1]{fontenc}

%\usepackage{stfloats}
%\usepackage[eulergreek]{sansmath}
% \usepackage[final]{graphicx}
\usepackage{color}
\usepackage{relsize}
\usepackage{mathtools}
\newcommand{\ex}{{\mathrm e}}
\newtheorem{definition}{Definition}
\newtheorem{lemma}{Lemma}
\newtheorem{theorem}{Theorem}
\newtheorem{remark}{Remark}

\newtheorem{proposition}{Proposition}
\newcommand{\bs}[1]{\boldsymbol{#1}}
\newcommand{\mc}[1]{\mathcal{#1}}
\newcommand{\ul}[1]{\underline{#1}}
\newcommand{\mb}[1]{\mathbf{#1}}
\newcommand{\mr}[1]{\mathrm{#1}}
\newcommand{\tr}{\mathrm{Tr}}
\newcommand{\fo}{\mathbf{F}_{\mathrm{opt}}}

% \newcommand{\ceil}[1]{\lceil {#1} \rceil}
\usepackage{mathtools}
\DeclarePairedDelimiter{\ceil}{\lceil}{\rceil}

\DeclareMathOperator*{\argmin}{arg\;min}
\DeclareMathOperator*{\argmax}{arg\;max}
\DeclareMathOperator*{\maximize}{maximize}
\DeclareMathOperator*{\minimize}{minimize}
\DeclareMathOperator*{\st}{subject\;to}
\renewcommand{\Pr}{\mathbb{P}} % (by default \Pr is rendered as "Pr")
\pagestyle{empty}
% \thispagestyle{plain}
% \pagestyle{plain}
\SetKwInput{KwInput}{Input}                % Set the Input
\SetKwInput{KwOutput}{Output}
\SetKwInput{KwOutputr}{Output Regression}              % set the Output
\SetKwInput{KwOutputc}{Output Classification}              % set the Output
\newcommand{\rank}{\operatornamewithlimits{rank}}
\newcommand{\trace}{\operatornamewithlimits{trace}}
%\newcommand{\argmin}{\operatornamewithlimits{argmin}}
%\newcommand{\argmax}{\operatornamewithlimits{argmax}}
\newcommand{\bseq}{\begin{subequations}}
\newcommand{\eseq}{\end{subequations}}
\newcommand{\baln}{\begin{align}}
\newcommand{\ealn}{\end{align}}
\newcommand{\balnd}{\begin{aligned}}
\newcommand{\ealnd}{\end{aligned}}
\newcommand{\beq}{\begin{equation}}
\newcommand{\eeq}{\end{equation}}
\newcommand{\beqn}{\begin{eqnarray}}
\newcommand{\eeqn}{\end{eqnarray}}
\newcommand{\beqno}{\begin{eqnarray*}}
\newcommand{\eeqno}{\end{eqnarray*}}
\newcommand{\bma}{\begin{displaymath}}
\newcommand{\ema}{\end{displaymath}}
\newcommand{\bnu}{\begin{enumerate}}
\newcommand{\enu}{\end{enumerate}}
\newcommand{\bce}{\begin{center}}
\newcommand{\ece}{\end{center}}
\newcommand{\btb}{\begin{tabular}}
\newcommand{\etb}{\end{tabular}}
\newcommand{\ba}{\begin{array}}
\newcommand{\ea}{\end{array}}
%\setlength\arraycolsep{2pt}
\usepackage{footnote}
\makesavenoteenv{tabular}
\makesavenoteenv{table}
\makeatletter 
\newcommand\semiHuge{\@setfontsize\semiHuge{21.1}{27.38}}
\makeatother
% \setlength{\textfloatsep}{0pt}
% \setlength{\parskip}{0pt} 
%****************************************************
\begin{document}
\title{Compression of Site-Specific Deep Neural Networks for Massive MIMO Precoding}
\author{\IEEEauthorblockN{Ghazal~Kasalaee, Ali~Hasanzadeh~Karkan, Jean-François~Frigon, and François~Leduc-Primeau}
\IEEEauthorblockA{Department of Electrical Engineering, Polytechnique Montréal, Montréal, QC H3C 3A7, Canada\\
Emails: \{ghazal.kasalaee, ali.hasanzadeh-karkan, j-f.frigon, francois.leduc-primeau\}@polymtl.ca
    }
}

\maketitle
\IEEEpubidadjcol

\begin{abstract}
The deployment of deep learning (DL) models for precoding in massive multiple-input multiple-output (mMIMO) systems is often constrained by high computational demands and energy consumption. In this paper, we investigate the compute energy efficiency of mMIMO precoders using DL-based approaches, comparing them to conventional methods such as zero forcing and weighted minimum mean square error (WMMSE). Our energy consumption model accounts for both memory access and calculation energy within DL accelerators. We propose a framework that incorporates mixed-precision quantization-aware training and neural architecture search to reduce energy usage without compromising accuracy. Using a ray-tracing dataset covering various base station sites, we analyze how site-specific conditions affect the energy efficiency of compressed models. Our results show that deep neural network compression generates precoders with up to 35 times higher energy efficiency than WMMSE at equal performance, depending on the scenario and the desired rate. These results establish a foundation and a benchmark for the development of energy-efficient DL-based mMIMO precoders.
\end{abstract}




% \begin{IEEEkeywords}
% \end{IEEEkeywords}
% \IEEEpeerreviewmaketitle

\section{Introduction} \label{Sec:Intro}
\section{Introduction}
Backdoor attacks pose a concealed yet profound security risk to machine learning (ML) models, for which the adversaries can inject a stealth backdoor into the model during training, enabling them to illicitly control the model's output upon encountering predefined inputs. These attacks can even occur without the knowledge of developers or end-users, thereby undermining the trust in ML systems. As ML becomes more deeply embedded in critical sectors like finance, healthcare, and autonomous driving \citep{he2016deep, liu2020computing, tournier2019mrtrix3, adjabi2020past}, the potential damage from backdoor attacks grows, underscoring the emergency for developing robust defense mechanisms against backdoor attacks.

To address the threat of backdoor attacks, researchers have developed a variety of strategies \cite{liu2018fine,wu2021adversarial,wang2019neural,zeng2022adversarial,zhu2023neural,Zhu_2023_ICCV, wei2024shared,wei2024d3}, aimed at purifying backdoors within victim models. These methods are designed to integrate with current deployment workflows seamlessly and have demonstrated significant success in mitigating the effects of backdoor triggers \cite{wubackdoorbench, wu2023defenses, wu2024backdoorbench,dunnett2024countering}.  However, most state-of-the-art (SOTA) backdoor purification methods operate under the assumption that a small clean dataset, often referred to as \textbf{auxiliary dataset}, is available for purification. Such an assumption poses practical challenges, especially in scenarios where data is scarce. To tackle this challenge, efforts have been made to reduce the size of the required auxiliary dataset~\cite{chai2022oneshot,li2023reconstructive, Zhu_2023_ICCV} and even explore dataset-free purification techniques~\cite{zheng2022data,hong2023revisiting,lin2024fusing}. Although these approaches offer some improvements, recent evaluations \cite{dunnett2024countering, wu2024backdoorbench} continue to highlight the importance of sufficient auxiliary data for achieving robust defenses against backdoor attacks.

While significant progress has been made in reducing the size of auxiliary datasets, an equally critical yet underexplored question remains: \emph{how does the nature of the auxiliary dataset affect purification effectiveness?} In  real-world  applications, auxiliary datasets can vary widely, encompassing in-distribution data, synthetic data, or external data from different sources. Understanding how each type of auxiliary dataset influences the purification effectiveness is vital for selecting or constructing the most suitable auxiliary dataset and the corresponding technique. For instance, when multiple datasets are available, understanding how different datasets contribute to purification can guide defenders in selecting or crafting the most appropriate dataset. Conversely, when only limited auxiliary data is accessible, knowing which purification technique works best under those constraints is critical. Therefore, there is an urgent need for a thorough investigation into the impact of auxiliary datasets on purification effectiveness to guide defenders in  enhancing the security of ML systems. 

In this paper, we systematically investigate the critical role of auxiliary datasets in backdoor purification, aiming to bridge the gap between idealized and practical purification scenarios.  Specifically, we first construct a diverse set of auxiliary datasets to emulate real-world conditions, as summarized in Table~\ref{overall}. These datasets include in-distribution data, synthetic data, and external data from other sources. Through an evaluation of SOTA backdoor purification methods across these datasets, we uncover several critical insights: \textbf{1)} In-distribution datasets, particularly those carefully filtered from the original training data of the victim model, effectively preserve the model’s utility for its intended tasks but may fall short in eliminating backdoors. \textbf{2)} Incorporating OOD datasets can help the model forget backdoors but also bring the risk of forgetting critical learned knowledge, significantly degrading its overall performance. Building on these findings, we propose Guided Input Calibration (GIC), a novel technique that enhances backdoor purification by adaptively transforming auxiliary data to better align with the victim model’s learned representations. By leveraging the victim model itself to guide this transformation, GIC optimizes the purification process, striking a balance between preserving model utility and mitigating backdoor threats. Extensive experiments demonstrate that GIC significantly improves the effectiveness of backdoor purification across diverse auxiliary datasets, providing a practical and robust defense solution.

Our main contributions are threefold:
\textbf{1) Impact analysis of auxiliary datasets:} We take the \textbf{first step}  in systematically investigating how different types of auxiliary datasets influence backdoor purification effectiveness. Our findings provide novel insights and serve as a foundation for future research on optimizing dataset selection and construction for enhanced backdoor defense.
%
\textbf{2) Compilation and evaluation of diverse auxiliary datasets:}  We have compiled and rigorously evaluated a diverse set of auxiliary datasets using SOTA purification methods, making our datasets and code publicly available to facilitate and support future research on practical backdoor defense strategies.
%
\textbf{3) Introduction of GIC:} We introduce GIC, the \textbf{first} dedicated solution designed to align auxiliary datasets with the model’s learned representations, significantly enhancing backdoor mitigation across various dataset types. Our approach sets a new benchmark for practical and effective backdoor defense.




\section{System model} \label{Sec:Baseline}
This work considers a multi-user \gls{mMIMO} system where a \gls{BS} with $N_{\sf{T}}$ transmit antennas serves $N_{\sf{U}}$ single-antenna users simultaneously. Let $x_u$ represent the transmitted symbol for each user. The received signal at the $u^{th}$ user can be represented as 
\begin{equation}\label{eq:signal_recived}
    \mathbf{y}_u =  \mathbf{h}_{u}^{\dagger} \sum_{\forall u}  \mathbf{w}_{u} x_u + \bs{\eta} \,,
\end{equation}
in which $\mathbf{h}_u \in \mathbb{C}^{N_{\sf{T}} \times 1}$ is the wireless channel vector between the \Gls{BS} and $u^{th}$ user; the term $\bs{\eta} \sim \mathcal{CN}(0, \sigma^2)$ denotes complex symmetric Gaussian noise with zero mean and a variance of $\sigma^2$; the downlink transmit \gls{FDP} vector is denoted as $\mathbf{W} = \left [ \mathbf{w}_1, \hdots, \mathbf{x}_u, \hdots, \mathbf{w}_{N_{\sf{U}}} \right ] \in \mathbb{C}^{N_{\sf{T}} \times N_{\sf{U}}}$.
The corresponding \gls{SINR} for user $u$ is formulated as
\begin{equation}
    \text{SINR}(\mb{w}_{u}) = \frac{ \big|\mb{h}^{\dagger}_{u} \mb{w}_{u} \big|^2}{\sum_{j \neq u} \big|\mb{h}^{\dagger}_{u} \mb{w}_{j} \big|^2 + \sigma^2} \,.
\end{equation}
The goal is to find such a precoding $\mathbf{W}$ that maximizes the throughput under the maximal transmit power $P_{\text{max}}$ constraint. Thus, the downlink sum rate maximization problem can be formulated as
\begin{align}\label{eq:sum rate-optimization}
    & \underset{\mb{W}}{\max}~ R(\mb{W}) \,, \\
    \text{s.t.}  &\sum_{\forall u} \mb{w}_{u}^{\dagger}  \mb{w}_{u} \leq P_{\sf{max}}\,,
\end{align}
where for a precoding matrix $\mathbf{W}$, the sum rate is
\begin{equation}\label{eq:sum rate}
    R(\mb{W}) = \sum_{\forall u} \text{log}_2 \Bigl(  1+ \text{SINR}(\mb{w}_{u}) \Bigr).
\end{equation}

\subsection{Problem Definition}

Beamforming is crucial in wireless communication, particularly at mmWave frequencies, where directional signal transmission is essential due to the limitations of omnidirectional antennas. \Glspl{DNN} have shown promise in maximizing the sum rate, but despite their enhanced performance, \glspl{DNN} come with high computational complexity and substantial energy requirements, posing challenges for deployment on resource-limited devices like edge platforms.

To tackle the complexities and energy demands associated with \gls{DNN}-based precoders in mMIMO systems, we focus on developing a method that preserves the high throughput of \glspl{DNN} while making them suitable for resource-constrained environments. This requires overcoming inefficiencies caused by uniformly applied quantization, which overlooks the varying sensitivity of different network layers to precision reduction.

In this work, we introduce a novel framework that combines mixed-precision quantization-aware training with \gls{NAS}. Our method is designed to significantly lower the energy consumption of \glspl{DNN} in precoder design while sustaining high throughput for site-specific \glspl{BS}. Extensive experiments validate that our approach achieves superior energy efficiency compared to existing \gls{DL}-based and conventional beamforming techniques.




\section{Methodology} \label{Sec:Proposed}
\section{RoleMRC}
\label{sec:method}

In this section, we build RoleMRC. Figure\,\ref{fig:method} illustrates the overall pipeline of RoleMRC from top to bottom, which is divided into three steps.

\subsection{A Meta-pool of 10k Role Profiles}
\label{sec:meta_pool}
We first collect a meta-pool of 10k role profile using two open-source datasets, with Step 1 and 2.

\paragraph{Step 1: Persona Sampling.} We randomly sample 10.5k one-sentence demographic persona description from PersonaHub\,\cite{ge2024scaling}, such as ``\emph{A local business owner interested in economic trends}'', as shown at the top of Figure\,\ref{fig:method}. 

\paragraph{Step 2: Role Profile Standardization.} Next, we use a well-crafted prompt with gpt-4o\,\cite{gpt4o} to expand each sampled persona into a complete role profile, in reference to the 1-shot standardized example. Illustrated in the middle of Figure\,\ref{fig:method}, we require a standardized role profile consisting of seven components: \emph{Role Name and Brief Description}, \emph{Specific Abilities and Skills}, \emph{Speech Style}, \emph{Personality Characteristics}, \emph{Past Experience and Background}, \emph{Ability and Knowledge Boundaries} and \emph{Speech Examples}. %Setting standard specifications helps convert the generated role profiles into formatted records, which is beneficial for the post quality control. 
Standardizing these profiles ensures structured formatting, simplifying quality control. 
After manual checking and format filtering, we remove 333 invalid responses from gpt-4o, resulting in 10.2k final role profiles. We report complete persona-to-profile standardization prompt and structure tree of final role profiles in Appendix\,\ref{sec:app_prompt_1} and \,\ref{sec:app_tree}, respectively.

Machine Reading Comprehension (MRC) is one of the core tasks for LLMs to interact with human users. Consequently, we choose to synthesize fine-grained role-playing instruction-following data based on MRC. We first generate a retrieval pool containing 808.7k MRC data from the MSMARCO training set\,\cite{bajaj2016ms}. By leveraging SFR-Embedding\,\cite{SFR-embedding-2}, we perform an inner product search to identify the most relevant and least relevant MRC triplets (Passages, Question, Answer) for each role profile. For example, the middle part of Figure\,\ref{fig:method} shows that for the role \emph{Jessica Thompson, a resilient local business owner}, the most relevant question is about \emph{the skill of resiliency}, while the least relevant question is \emph{converting Fahrenheit to Celsius}. After review, we categorise the most relevant MRC triplet as within a role's knowledge boundary, and the least relevant MRC triplet as beyond their expertise.

\begin{figure}[t]
    \centering
    \includegraphics[width=1.0\linewidth]{figures/step3.png}
    \caption{The strategy of gradually synthesizing finer role-playing instructions in step 3 of Figure\,\ref{fig:method}.}
    \vspace{-1.0em}
    \label{fig:step3}
\end{figure}

\subsection{38k Role-playing Instructions}
Based on the role profiles, we then adopt \textbf{Step 3: Multi-stage Dialogue Synthesis} to generate 38k role-playing instructions, progressively increasing granularity across three categories %including three types with gradually finer granularity 
(Figure\,\ref{fig:step3}):
%\begin{itemize}
%[leftmargin=*,noitemsep,topsep=0pt]

\noindent \textbf{\underline{Free Chats.}} The simplest dialogues, free chats, are synthesized at first. Here, we ask gpt-4o to simulate and generate multi-turn open-domain conversations between the role and an imagined user based on the standardized role profile. When synthesizing the conversation, we additionally consider two factors: the \textbf{initial speaker} in the starting round of the conversation, and whether the role's speech has \textbf{a narration wrapped in brackets} at the beginning (e.g., \emph{(Aiden reviews the network logs, his eyes narrowing as he spots unusual activity) I found it!}). The narration refers to a short, vivid description of the role's speaking state from an omniscient perspective, which further strengthens the sense of role's depth and has been adopted in some role-playing datasets\,\cite{tu2024charactereval}. 

As shown on the left side of Figure\,\ref{fig:step3}, based on the aforementioned two factors, we synthesize four variations of Free Chats. In particular, when  narration is omitted, we deleted all the 
narration content in the speech examples from the role profile; %and for the case that 
when narration is allowed, we retain the narration content, and also add instructions to allow appropriate insertion of narration in the task prompt of gpt-4o. It worth to note that, in narration-allowed dialogues, not every response of the role has narration inserted to prevent overfitting. All categories of data in RoleMRC incorporate narration insertion and follow similar control mechanisms. The following sections will omit further details on narration.

\noindent \textbf{\underline{On-scene MRC Dialogues.}} The synthesis of on-scene MRC dialogues can be divided into two parts. The first part is similar to the free chats. As shown by the {\color{lightgreen}{green round rectangle}} in the upper part of Figure\,\ref{fig:step3}, we ask gpt-4o to synthesize a conversation (lower left corner of Figure\,\ref{fig:step3}) between the role and the user focusing on relevant passages. This part of the synthesis and the Free Chats share the entire meta-pool, so each consisting of 5k dialogues.

The remaining part forms eight types of single-turn role-playing Question Answering (QA). In the middle of Figure\,\ref{fig:step3}, we randomly select a group of roles and examined the most relevant MRCs they matched: if the question in the MRC is answerable, then the ground truth answer is stylized to match the role profile; otherwise, a seed script of ``unanswerable'' is randomly selected then stylized. The above process generates four groups of 1k data from type ``[1]'' to type``[4]''. According to the middle right side of Figure\,\ref{fig:step3}, we also select a group of roles and ensure that the least relevant MRCs they matched contain answerable QA pairs. Since the most irrelevant MRCs are outside the knowledge boundary of the roles, the role-playing responses to these questions are ``out-of-mind'' refusal or ``let-me-try'' attempt, thus synthesizing four groups of 1k data, from type ``[5]'' to type ``[8]''.

\noindent \textbf{\underline{Ruled Chats.}} We construct Ruled Chats by extending On-scene MRC Dialogues in categories ``[1]'' to ``[8]'' with incorporated three additional rules, as shown in the right bottom corner of Figure\,\ref{fig:step3}. For the \textbf{multi-turn rules}, we apply them to the four unanswerable scenarios ``[3]'', ``[4]'', ``[5]'', and ``[6]'', adding a user prompt that  forces the role to answer. Among them, data ``[3]'' and ``[4]'' maintain refusal since the questions in MRC are unanswerable; while ``[5]'' and ``[6]'' are transformed into attempts to answer despite knowledge limitations. For the \textbf{nested formatting rules}, we add new formatting instructions to the four categories of data ``[1]'', ``[2]'', ``[3]'', and ``[4]'', such as requiring emojis,  capitalization, specific punctuation marks, and controlling the total number of words, then modify the previous replies accordingly. For the last \textbf{prioritized rules}, we apply them to subsets ``[1]'' and ``[2]'' that contain normal stylized answers, inserting a  global refusal directive from the system, and thus creating a conflict between system instructions and the role's ability boundary.
%\end{itemize}

\begin{table}[t]
\resizebox{\columnwidth}{!}{%
  \begin{tabular}{c|c|c|c|c|c}
    \toprule
    & & \textbf{S*} & \textbf{P*} & \textbf{\#Turns} & \textbf{\#Words} \\ 
    \midrule
    \multirow{13.5}{*}{\textbf{RoleMRC}} 
    & \multicolumn{5}{c|}{\textbf{Free Chats}} \\ 
    \cmidrule(lr){2-6}
    & Chats & 5k & / & 9.47 & 38.62 \\ 
    \cmidrule(lr){2-6}
    & \multicolumn{5}{c|}{\textbf{On-scene MRC Dialogues}} \\ 
    \cmidrule(lr){2-6} 
    & On-scene Chats & 5k & / & 9.2 & 43.18 \\
    & Answer & 2k & 2k & 1 & 39.45 \\ 
    & No Answer & 2k & 2k & 1 & 47.09 \\ 
    & Refusal & 2k & 2k & 1 & 48.41 \\ 
    & Attempt & 2k & 2k & 1 & 47.92 \\ 
    \cmidrule(lr){2-6}
    & \multicolumn{5}{c|}{\textbf{Ruled Chats}} \\ 
    \cmidrule(lr){2-6}
    & Multi-turn & 2k & 2k & 2 & 42.47 \\ 
    & Nested & 1.6k & 1.6k & 1 & 46.17 \\ 
    & Prioritized & 2.4k & 2.4k & 1 & 42.65 \\ 
    \midrule
    & \textbf{Total} & 24k & 14k & 3.5 & 40.6 \\ 
    \midrule
    \multirow{3}{*}{\textbf{-mix}} 
    & RoleBench & 16k & / & 1 & 23.95 \\ 
    & RLHFlow & 40k & / & 1.39 & 111.79 \\ 
    & UltraFeedback & / & 14k & 1 & 199.28 \\ 
    \midrule
    & \textbf{Total} & 80k & 28k & 2 & 67.1 \\ 
    \bottomrule
  \end{tabular}}
  \vspace{-2mm}
  \caption{Statistics of RoleMRC. In particular, the column names S*, P*, \#Turns, and \#Words, stands for size of single-label data, size of pair-label data, average turns, and average number of words per reply, respectively. RoleMRC-mix expands RoleMRC by adding existing role-playing data.}
 \vspace{-3mm}
  \label{tab:roleMRC}
\end{table}

\subsection{Integration and Mix-up}
All the seed scripts and prioritized rules used for constructing On-scene Dialogues and Ruled Chats are reported in Appendix\,\ref{sec:app_scripts}. These raw responses are logically valid manual answers that remain unaffected by the roles' speaking styles, making them suitable as negative labels to contrast with the stylized answers. Thanks to these meticulous seed texts, we obtain high-quality synthetic data with stable output from gpt-4o. After integration, as shown in Table\,\ref{tab:roleMRC}, the final RoleMRC contains 24k single-label data for Supervised Fine-Tuning (SFT) and 14k pair-label data for Human Preference Optimization (HPO)\,\cite{ouyang2022training,rafailov2023direct,sampo,hong2024reference}. Considering that fine-tuning LLMs with relatively fixed data formats may lead to catastrophic forgetting\,\cite{kirkpatrick2017overcoming}, we create RoleMRC-mix as a robust version by incorporating external role-playing data (RoleBench\,\cite{wang2023rolellm}) and general instructions (RLHFlow\,\cite{dong2024rlhf}, UltraFeedback\,\cite{cui2023ultrafeedback}).


\section{Numerical Results} \label{Sec:Simulation}
\subsection{Dataset Definition}
A custom dataset was generated to accurately reflect the channel characteristics pertinent to beamforming in mMIMO systems. MATLAB’s Ray-Tracing toolbox simulated both \gls{LOS} and \gls{NLOS} conditions. The simulations positioned a \gls{BS} within the Montreal region, utilizing environmental data from OpenStreetMap \cite{OpenStreetMap} for realism. The base station employed a uniform planar array antenna with 8x8 elements, spaced at half-wavelength, operating at a frequency of 2\,GHz. The transmitter was set at a height of 20\,m and powered at 20\,W, assuming a system loss of 10\,dB. Users were placed in circular patterns around the base station at distances ranging from 50 to 350\,m and at 10-degree intervals. This configuration captured a wide variety of deployment scenarios and channel conditions. The ray-tracing simulations considered up to 10 reflections but excluded diffraction effects, focusing on signal reflections from buildings and terrain to emulate multipath propagation in urban environments accurately.

\subsection{Energy Consumption Examples}
We first present some examples of the energy consumed by the different approaches, as per the model presented in Sections~\ref{sec:energy_dnn} and \ref{sec:energy_baselines}.
We consider the ``UdeM-NLOS'' scenario.
In Table~\ref{table:energy_comparison}, we report the energy for the ``default'' variants of the method, that is, for \gls{WMMSE}, we set the stopping criterion to $10^{-5}$ to have near-optimal performance, and for the DNN, we use $C_{\text{out}}$\,$=$\,$64$ and $D_{\text{FCL}}$\,$=$\,$1024$ with the maximum weight resolution of 16 bits.
We see that the DNN consumes significantly less than WMMSE at the cost of a slight degradation in sum rate (on this scenario). ZF, on the other hand, is much less complex, but does not provide competitive performance in \gls{NLOS} conditions, or at low \gls{SNR}.
As a result, ZF is unlikely to be a favored solution in practice, since it results in significant under-utilization of the BS resources.

\begin{table}[t]
    \centering
    \caption{Energy Comparison of the Default Variants on ``U\lowercase{de}M-NLOS''  ($N_{\sf{T}}$\,$=$\,$64$,~$N_{\sf{U}}$\,$=$\,$4$,~SNR\,$=$\,15\,dB).}
    \begin{tabular}{ccc}
        \toprule
        \textbf{Method} & \textbf{Energy ($\mu J$)} & \textbf{Sum Rate} (bit/s/Hz) \\
        \midrule
        WMMSE ($I$\,$=$\,$92.8$) & 296 & 19.2\phantom{$~\pm~0.007$}\\
        \textbf{Default DNN} & \textbf{54.5} & \textbf{18.9}$~\pm~0.007$\\
        Zero-Forcing (ZF) & 0.008  & 15.3\phantom{$~\pm~0.007$}\\
        \bottomrule
    \end{tabular}
    \label{table:energy_comparison}
\end{table}

\begin{table}[t]
    \centering
    \caption{DNN Energy Consumption with Uniform Quantization 
    \\for $C_{\text{out}}$\,$=$\,$64$, $D_{\text{FCL}}$\,$=$\,$1024$ 
    on ``U\lowercase{de}M-NLOS''\\
    ($N_{\sf{T}} $\,$=$\,$ 64$, $N_{\sf{U}}$\,$=$\,$4$, SNR\,$=$\,15\,dB)}
    \resizebox{\columnwidth}{!}{
    \begin{tabular}{ccc}
        \toprule
        \textbf{Quantization Configs}
        & \textbf{Energy ($\mu J$)}& \textbf{Sum Rate} (bit/s/Hz)\\
        \midrule
        $[16,\,16,\,16,\,16]$& 54.5 & 18.9$~\pm~0.007$\\
        $[8,\,8,\,8,\,8]$ & 17.5 & 18.6$~\pm~0.012$ \\
        $[4,\,4,\,4,\,4]$ & 7.4 & 17.8$~\pm~0.014$\\
        $[2,\,2,\,2,\,2]$ & 4.6 & 17.1$~\pm~0.032$\\
        \bottomrule
    \end{tabular}
    }
    \label{tab:energy_consumption}
\end{table}

Next, to illustrate the impact of quantization, 
Table~\ref{tab:energy_consumption} lists the energy consumption of the DNN for various uniform bit-width configurations, for the same $C_{\text{out}}$ and $D_{\text{FCL}}$ as in Table~\ref{table:energy_comparison}. We see that lowering the weight resolution leads to substantial energy savings but at the cost of a moderate decrease in performance.

\begin{figure}[!t]
    \centering
    \includegraphics[width=\columnwidth]{fig/result_ICC_1_pareto_modified_final.pdf}
    \caption{Trade-off between energy efficiency and sum rate for \glspl{CNN} with varying $C_{\text{out}}$, $D_{\text{FCL}}$, and MPQ bit widths, on the ``UdeM-LOS'' scenario ($N_{\sf{T}}$\,$=$\,$64$, $N_{\sf{U}}$\,$=$\,$4$, average SNR\,$=$\,29\,dB). All the model configurations that were evaluated are shown, while the curves provide the Pareto front associated with each architecture configuration.}
    \label{fig:Neural Architecture Search (NAS)}
    \vspace{-10pt}
\end{figure}

\subsection{NAS Results}

Figure \ref{fig:Neural Architecture Search (NAS)} highlights the trade-offs between computational energy efficiency (bits/s/Hz/$\mu$J) and sum rate (bits/s/Hz) across diverse \gls{DNN} configurations generated as described in Section~\ref{Sec:Proposed}, on the ``UdeM-LOS'' scenario.
Each curve shows the Pareto front corresponding to a particular architecture configuration, while each point in this curve uses a different quantization configuration.

A few trends can be mentioned among the Pareto-optimal results for each architecture.
Firstly, the first layer, CONV, is often kept at high precision, particularly in smaller models, to maintain performance while reducing energy consumption. Interestingly, no Pareto-optimal model uses uniform quantization across all layers. Moreover, even models that achieve the highest sum rates do not employ more than two layers at the highest precision. These results emphasize the importance of efficiently distributing bit precision across layers to optimize energy consumption.

The configuration yielding the highest energy efficiency is ($C_{out}$\,$=$\,$8$, $D_{FCL}$\,$=$\,$512$) with quantization [2,\,8,\,8,\,8], achieving 26.2 \,bits/s/Hz/$\mu$J at a moderate sum rate of 28.5\,bits/s/Hz. On the other hand, the configuration achieving the highest sum rate is ($C_{out}$\,$=$\,$64$, $D_{FCL}$\,$=$\,$ 1024$) with quantization [16,\,16,\,4,\,8], which reaches 31.3\,bits/s/Hz but does not use full precision, making it more interesting from an energy efficiency perspective. An alternative worth mentioning is the model with the second-best sum rate of 31.1\,bits/s/Hz, achieved with an architecture of ($C_{out}$\,$=$\,$64$, $D_{FCL}$\,$=$\,$1024$) and quantization [16,\,2,\,2,\,2], which uses nearly $5\times$ less energy than the highest sum-rate model.
We do observe compute energy efficiency decreasing rapidly near the highest sum-rate, but of course this could simply mean that switching to a larger and/or different DNN architecture would be preferable at that point.

To further illustrate the importance of of NAS and model compression in finding efficient DNN precoders, Figure~\ref{fig:Neural Architecture Search (NAS)} includes horizontal and vertical arrows that quantify the impact on performance of the design choices. The horizontal arrow measures the difference in sum rate between the worst and best configurations at equal energy efficiency, revealing a 20\% improvement through optimal model selection. Similarly, the vertical arrow shows the difference in energy efficiency at an equal sum rate, demonstrating a $14.5\times$ gain.


\begin{figure}[!t]
\centering
\includegraphics[width=\columnwidth]{fig/result_ICC_5.pdf}
\caption{Comparison of energy efficiency (bits/s/Hz/$\mu$J) and sum rate (bits/s/Hz) across two environments: UdeM-NLOS (average SNR\,$=$\,15\,dB) and Okapark-LOS (average SNR\,$=$\,28\,dB). The proposed method achieves a superior balance of energy efficiency and sum rate performance compared to WMMSE. Results are derived for models with varying ($C_{out}$) and ($D_{FCL}$).}
\vspace{-9pt}
\label{fig:LSQ Quantization}
\end{figure}
\subsection{Impact of Deployment Environment and Energy Efficiency Comparison}

Figure~\ref{fig:LSQ Quantization} compares energy efficiency (bits/s/Hz/$\mu$J) and sum-rate (bits/s/Hz) across two contrasting deployment scenarios: UdeM-NLOS, characterized by challenging multipath conditions, and Okapark-LOS, offering clear \gls{LOS} signal propagation. These two scenarios highlight the adaptability and effectiveness of the proposed quantized models.
For the DNN precoders, each curve shows the Pareto-optimal configurations across the entire search space, whereas for WMMSE, the trade-off between sum-rate and energy efficiency is varied by adjusting the stopping criterion.

In the UdeM-NLOS environment, sum rates are constrained between 15 and 20 bits/s/Hz due to severe signal attenuation and multipath effects. Despite these limitations, the quantized models achieve significant energy savings, with improvements of up to $35\times$ in energy efficiency compared to the WMMSE baseline, all while maintaining competitive sum rates.

In contrast, the Okapark-LOS environment, which benefits from clear signal paths, supports higher sum rates ranging from about 30 to 40 bits/s/Hz. 
Depending on the desired sum-rate, the DNN precoders can provide improvements in energy efficiency ranging from $6.1\times$ to $1.2\times$. However, with the DNN architecture template and training method considered in this paper, the DNN precoder is unable to achieve the highest sum-rate that can be provided by WMMSE.
%Here, the quantized models continue outperforming the WMMSE baseline, delivering a 6.1× improvement in energy efficiency while maintaining similar or superior sum-rate performance. The highest sum rate, achieved by the DNN for Okapark-LOS ($C_{out} = 64$, $D_{FCL} = 1024$), is roughly 1.3× higher, highlighting an intriguing balance of performance and precision.

These results emphasize the adaptability of the quantized models across diverse deployment environments, and their ability to achieve the same performance with less compute energy. Interestingly, the energy gains provided by DNNs appear to be larger in more difficult (low SNR, non line-of-sight) environments.






\section{Conclusion} \label{Sec:Conclusion}
In this paper, we systematically investigate the position bias problem in the multi-constraint instruction following. To quantitatively measure the disparity of constraint order, we propose a novel Difficulty Distribution Index (CDDI). Based on the CDDI, we design a probing task. First, we construct a large number of instructions consisting of different constraint orders. Then, we conduct experiments in two distinct scenarios. Extensive results reveal a clear preference of LLMs for ``hard-to-easy'' constraint orders. To further explore this, we conduct an explanation study. We visualize the importance of different constraints located in different positions and demonstrate the strong correlation between the model's attention distribution and its performance.


\section*{Acknowledgement}
This work was supported by Ericsson - Global Artificial Intelligence Accelerator AI-Hub Canada in Montr\'{e}al and jointly funded by NSERC Alliance Grant 566589-21 (Ericsson, ECCC, Innov\'{E}\'{E}).

\bibliographystyle{IEEEtran}
% \bibliography{am_ger_eng,rubi_eng}
% \bibfont \footnotesize
\bibliography{0_main}


\end{document}