This work considers a multi-user \gls{mMIMO} system where a \gls{BS} with $N_{\sf{T}}$ transmit antennas serves $N_{\sf{U}}$ single-antenna users simultaneously. Let $x_u$ represent the transmitted symbol for each user. The received signal at the $u^{th}$ user can be represented as 
\begin{equation}\label{eq:signal_recived}
    \mathbf{y}_u =  \mathbf{h}_{u}^{\dagger} \sum_{\forall u}  \mathbf{w}_{u} x_u + \bs{\eta} \,,
\end{equation}
in which $\mathbf{h}_u \in \mathbb{C}^{N_{\sf{T}} \times 1}$ is the wireless channel vector between the \Gls{BS} and $u^{th}$ user; the term $\bs{\eta} \sim \mathcal{CN}(0, \sigma^2)$ denotes complex symmetric Gaussian noise with zero mean and a variance of $\sigma^2$; the downlink transmit \gls{FDP} vector is denoted as $\mathbf{W} = \left [ \mathbf{w}_1, \hdots, \mathbf{x}_u, \hdots, \mathbf{w}_{N_{\sf{U}}} \right ] \in \mathbb{C}^{N_{\sf{T}} \times N_{\sf{U}}}$.
The corresponding \gls{SINR} for user $u$ is formulated as
\begin{equation}
    \text{SINR}(\mb{w}_{u}) = \frac{ \big|\mb{h}^{\dagger}_{u} \mb{w}_{u} \big|^2}{\sum_{j \neq u} \big|\mb{h}^{\dagger}_{u} \mb{w}_{j} \big|^2 + \sigma^2} \,.
\end{equation}
The goal is to find such a precoding $\mathbf{W}$ that maximizes the throughput under the maximal transmit power $P_{\text{max}}$ constraint. Thus, the downlink sum rate maximization problem can be formulated as
\begin{align}\label{eq:sum rate-optimization}
    & \underset{\mb{W}}{\max}~ R(\mb{W}) \,, \\
    \text{s.t.}  &\sum_{\forall u} \mb{w}_{u}^{\dagger}  \mb{w}_{u} \leq P_{\sf{max}}\,,
\end{align}
where for a precoding matrix $\mathbf{W}$, the sum rate is
\begin{equation}\label{eq:sum rate}
    R(\mb{W}) = \sum_{\forall u} \text{log}_2 \Bigl(  1+ \text{SINR}(\mb{w}_{u}) \Bigr).
\end{equation}

\subsection{Problem Definition}

Beamforming is crucial in wireless communication, particularly at mmWave frequencies, where directional signal transmission is essential due to the limitations of omnidirectional antennas. \Glspl{DNN} have shown promise in maximizing the sum rate, but despite their enhanced performance, \glspl{DNN} come with high computational complexity and substantial energy requirements, posing challenges for deployment on resource-limited devices like edge platforms.

To tackle the complexities and energy demands associated with \gls{DNN}-based precoders in mMIMO systems, we focus on developing a method that preserves the high throughput of \glspl{DNN} while making them suitable for resource-constrained environments. This requires overcoming inefficiencies caused by uniformly applied quantization, which overlooks the varying sensitivity of different network layers to precision reduction.

In this work, we introduce a novel framework that combines mixed-precision quantization-aware training with \gls{NAS}. Our method is designed to significantly lower the energy consumption of \glspl{DNN} in precoder design while sustaining high throughput for site-specific \glspl{BS}. Extensive experiments validate that our approach achieves superior energy efficiency compared to existing \gls{DL}-based and conventional beamforming techniques.


