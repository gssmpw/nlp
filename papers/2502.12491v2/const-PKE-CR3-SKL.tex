% !TEX root = main.tex

\section{ABE-CR\textsuperscript{2}-SKL from Multi-Input ABE}\label{sec:ABR-CR2-SKL}
In this section, we show how to achieve ABE-CR-SKL with classical certificates (ABE-CR\textsuperscript{2}-SKL) from MI-ABE.
\subsection{Definitions of Multi-Input ABE}\label{sec:def-mi-abe}
First, we recall the definitions of MI-ABE.
The syntax of MI-ABE is as follows:

\begin{definition}\label{def:miabe}
A Multi-Input (Ciphertext-Policy) Attribute-Based Encryption scheme
$\MIABE$ is a tuple of four PPT algorithms $(\Setup, \KeyGen,
\allowbreak \Enc,
\Dec)$. Let $k=\poly(\secp)$ and let $\{\cX_\secp\}_\secp$,
$\{(\cY_\secp)^k\}_\secp$ and $R = \{R_\secp: \cX_\secp \times
(\cY_\secp)^k \ra \bit\}$ be the ciphertext attribute space, key
attribute space and the relation associated with $\MIABE$
respectively. Let $\bit^\ell$ be the message space. Let $s(\secp)$
denote the maximum bit string length required to describe PPT
circuits with input size $k\cdot n(\secp)$ and depth $d(\secp)$ for
polynomials $s, n$ and $d$. Let $\cX_\secp = \bit^{s(\secp)}$ and
$\cY = \bit^{n(\secp)}$. Let $R$ be such that $R(x, y_1, \ldots,
y_k) = 0 \iff x(y_1, \ldots, y_k) = \bot$ where $x \in \cX$ is
parsed as a $k$-input circuit and $(y_1, \ldots, y_k)\in \cY^k$. 

\begin{description}
\item[$\Setup(1^\secp)\ra(\pk,\msk)$:] The setup algorithm takes a security parameter $1^\secp$ and outputs a public key $\pk$ and master secret key $\msk$.

\item[$\KG(\msk, i, y_i)\ra \sk_i$:] The key generation algorithm
takes the master secret key $\msk$, an index $i \in [k]$, and a
key-attribute $y_i \in \cY$ as input. It outputs a secret-key
$\sk_i$.

\item[$\Enc(\pk, x, \msg)\ra \ct$:] The encryption algorithm takes
the public key $\pk$, a ciphertext attribute $x \in \cX$, and a
message $\msg \in \bit^\ell$ as input. It outputs a ciphertext
$\ct$.

\item[$\Dec(\ct, \sk_1, \ldots \sk_k)\ra \widetilde{\msg}$:]
The decryption algorithm takes as input a ciphertext $\ct$ and $k$
secret-keys $\sk_1, \ldots, \sk_k$. It outputs a value
$\widetilde{\msg} \in \bit^\ell \cup \bot$.

\item[Correctness:] We require that
\[
\Pr\left[
\Dec(\ct, \sk_1, \ldots, \sk_k) = \msg
 \ \Bigg\lvert
\begin{array}{rl}
 &(\pk,\msk) \la \Setup(1^\secp),\\
 & \forall i \in [k]: \sk_i \gets \KG(\msk,i,y_i), \\
 &\ct \gets \Enc(\pk,x,\msg)
\end{array}
\right] \ge 1 -\negl(\secp).
\]
holds for all $x\in \cX$ and $(y_1, \ldots, y_k) \in \cY^k$ such
that $R(x,y_1,\ldots,y_k)=0$ and $\msg\in \bin^\ell$.
\end{description}

\begin{remark}
We say that an MI-ABE scheme has \emph{polynomial-arity}, if it
allows $k$ to be an arbitrary polynomial in $\secp$.
\end{remark}

%\paragraph{Secret-Key Variant.}
%
%The aforementioned Secret-Key variant
%of MI-ABE is defined in a similar way, except that the algorithm
%$\Setup$ only outputs a master secret-key $\msk$ and the encryption
%algorithm $\Enc$ takes $\msk$ as input instead of the public-key.
%The correctness condition is defined analogously.
\end{definition}

\begin{definition}[Post-Quantum Selective Security for MI-ABE:]\label{def:pq_sel_MIABE}
We say that $\MIABE$ is a selective-secure MI-ABE scheme for relation
$R:\cX \times \cY^k \ra \bit$, if it satisfies the following
requirement, formalized from the experiment
$\expb{\MIABE, \qA}{sel}{ind}(1^\secp,\coin)$ between an adversary
$\qA$ and a challenger $\Ch$:

\begin{enumerate}
\item $\qA$ declares the challenge ciphertext attribute
$x$. $\Ch$ runs $(\pk,\msk)\gets\Setup(1^\secp)$ and sends $\pk$ to $\qA$.

\item $\qA$ can get access to the following key generation oracle.

\begin{description}
\item[$\Oracle{kg}(i, y_i)$:] It outputs $\sk_i \gets \KG(\msk, i,
y_i)$ to $\qA$.

\end{description}

\item At some point, $\qA$ sends $(\msg_0,\msg_1)$ to $\Ch$. Then, $\Ch$
generates $\ct^*\gets\Enc(\pk,x,\msg_\coin)$ and sends $\ct^*$ to
$\qA$.

\item Again, $\qA$ can get access to the oracle $\Oracle{kg}$.
\item $\qA$ outputs a guess $\coin^\prime$ for $\coin$ and the
experiment outputs $\coin'$.
\end{enumerate}

For an adversary to be admissible, we require that for all tuples
$(y_1, \ldots, y_k) \in \cY^k$ received by $\qA$, it must hold that
$R(x, y_1, \ldots, y_k) = 1$ (non-decrytable). Given this
constraint, we say $\MIABE$ satisfies selective security if for all
QPT $\qA$, the following holds:

\begin{align}
\advb{\MIABE,\qA}{sel}{ind}(1^\secp) \seteq
\abs{\Pr[\expb{\MIABE,\qA}{sel}{ind} (1^\secp,0) \ra 1] -
\Pr[\expb{\MIABE,\qA}{sel}{ind} (1^\secp,1) \ra 1] }\leq \negl(\secp).
\end{align}

%\paragraph{Post-Quantum Selective Security for Secret-Key MI-ABE.}
%
%This notion is also defined analogously to the Security for
%Public-Key MI-ABE, except that throughout the experiment, $\qA$
%has access to the encryption oracle $\Oracle{\Enc}$ that takes
%as input $(x', \msg)$ and outputs $\Enc(\msk, x', \msg)$.

\end{definition}
\begin{remark}
    We do not require quantum security in~\cref{def:pq_sel_MIABE} unlike~\cref{def:qsel_ind_ABE}.
\end{remark}

\paragraph{Comparison with the Definition of Agrawal et.
al \cite{C:ARYY23}.}

Their definition considers a key-policy variant of MI-ABE. It
consists of algorithms $(\Setup, \Enc, \KG_1, \cdots,\allowbreak
\KG_{k-1}, \KG_k, \Dec)$ where $\Enc$ works in a similar way
as our definition, i.e., $\Enc(\pk,\allowbreak x_0, \msg) \ra \ct$ for message
$\msg$ and attribute $x_0$. For each $i \in [k-1]$, $\KG_i$ works
as $\KG_i(\msk, x_i) \ra \sk_i$ for the $i$-th key-attribute
$x_i$. On the other hand, $\KG_k$ works as $\KG_k(\msk, f)
\ra \sk_f$ where $f$ is an arbitrary $k$-input function. The
guarantee is that decryption is feasible if and only if
$f(x_0, \ldots, x_{k-1}) = \bot$. The algorithm $\Dec$ has the
syntax $\Dec(\pk, \ct, \sk_1, \ldots, \sk_{k-1}, \sk_f) \ra \msg'$.


It is easy to see that this definition directly implies a
$(k-1)$-input ciphertext-policy MI-ABE with algorithms $(\Setup',
\Enc', \KG', \Dec')$. Specifically, $\Setup'(1^\secp) \seteq
\Setup(1^\secp)$, $\Enc'(\pk, x_0, \msg) \seteq \big(\pk, \Enc(\pk,
x_0, \msg)\big)$ and $\KG'$ can be defined as $\KG'(\msk, i, x_i)
\seteq \KG_i(\msk, x_i)$ for all $i \in [k-1]$. Finally, $\Dec'$ is
defined as $\Dec'(\ct = (\pk, \widetilde{\ct}), \sk_1, \ldots,
\sk_{k-1}) \seteq \Dec(\pk, \widetilde{\ct}, \sk_1, \ldots,
\sk_{k-1}, \sk_f)$ for the function $f(x_0, \ldots, x_{k-1}) =
C_{x_0}(x_1, \ldots, x_{k-1})$ where $C_{x_0}$ is the circuit
described by ciphertext-attribute $x_0$.

Unfortunately, \cite{C:ARYY23} only allows for a constant $k$, and
hence we do not currently know of a construction where
$k=\poly(\secp)$. In the recent work of \cite{myEPRINT:AgrKumYam24a}, a
variant of MI-ABE (for $k=\poly(\secp)$) was achieved from the LWE and evasive LWE assumptions, which is non-standard.
However, the scheme it implies would require the encryption algorithm $\Enc'$ to utilize the master secret-key
$\msk$.

\paragraph{Disucssion on MI-ABE and IO.}
We focus on MI-ABE satisfying~\cref{def:miabe} in this paragraph.
MI-ABE is not stronger than IO since IO is equivalent to multi-input functional encryption (MIFE)~\cite{EC:GGGJKL14} and MIFE trivially implies MI-ABE.
Although we do not know whether MIFE is separated from MI-ABE, it is likely since PKFE is separated from ABE~\cite{C:GarMahMoh17} and constructing FE schemes is significantly more challenging than ABE schemes.
Moreover, MI-ABE is not currently known from weaker assumptions than IO.
They are qualitatively different primitives, and MI-ABE could be constructed from weaker assumptions in the future.
We also know that MI-ABE implies witness encryption~\cite{SCN:BJKPW18},\ryo{but this does not imply anything about relationship between MI-ABE and PKFE.} and witness encryption is achieved from the evasive LWE assumption~\cite{C:Tsabary22,AC:VaiWeeWic22}.
Hence, MI-ABE is somewhere between witness encryption and IO.
Achieving MI-ABE from lattice assumptions is an interesting open problem.



% In the work of
% \cite{C:ARYY23}, such an MI-ABE was achieved, but only for
% constant-arity. Moreover, in the recent work of \cite{AKY24}, a
% variant of MI-ABE (for polynomial arity) was achieved from LWE and a
% variant of Evasive LWE, which is a better state of affairs than the
% state-of-the-art for post-quantum FE. Unfortunately, their MI-ABE
% only implies a secret-key variant of the MI-ABE definition in this
% work, i.e., the resulting encryption algorithm requires the master
% secret-key $\msk$ as input.\ryo{This part somewhat overlaps with the discussion in~\cref{sec:standard_crypto,sec:other-rel}.}

%\begin{}
%Secret-Key variant of MI-ABE to 
%consist of algorithms $\Setup', \KG', \Enc_0, \ldots, \Enc_{k},
%\Dec'$ where $\Enc_i(\msk, x_i, \msg_i) \ra \ct_i$ encrypts $\msg_i$
%under attribute $x_i$.  $\KG'(\msk, f) \ra \sk_f$ outputs a secret
%key $\sk_f$. The guarantee is that $\ct_0, \ldots, \ct_{k}$ can be
%decrypted only if $f(x_0, \ldots, x_{k})$ holds. Notice that we can
%define $f$ such that $f(x_0, \ldots, x_{k}) = x_0(x_1, \ldots,
%x_{k})$ where $x_0$ is interpreted as a circuit. Then, we can obtain
%a Secret-Key MI-ABE scheme $(\Setup, \KG, \Enc, \Dec)$ as per our
%definition as follows:
%
%\begin{itemize}
%\item  We set $\KG(\msk, i, y_i) \seteq \Enc_i(\msk, y_i, 0^\ell)
%\seteq \sk_i$ for all $i \in [k]$.
%\item $\Enc(\msk, x, \msg) \seteq (\pk, \Enc_0(\msk, x, \msg))
%\seteq \ct$.
%\item $\Dec(\ct=(\pk, \widetilde{\ct}), \sk_1, \cdots, \sk_k) \seteq
%\Dec'(\pk, \widetilde{\ct}, \sk_1, \cdots, \sk_k, \sk_f)$.
%\end{itemize}
%

%We remark that such a transformation was also described in prior
%work \cite{ARYY23}. Now, we can restate one of the results of their
%work as follows:
%
%\begin{theorem}\cite{AKY24}\label{thm:AKY}
%There exists a Secret-Key Multi-Input ABE scheme, assuming
%non-uniform $\kappa$-evasive LWE, non-uniform sub-exponential PRFs,
%and non-uniform sub-exponential LWE.
%\end{theorem}
%
%\nikhil{They show adaptive security relying on sub-exponential
%assumptions. Does selective security only need polynomial hardness?}




\subsection{Definitions of ABE-CR\textsuperscript{2}-SKL}
The syntax of ABE-CR\textsuperscript{2}-SKL is defined as follows. 

\begin{definition}[ABE-CR\textsuperscript{2}-SKL]\label{def:pke-cr2}
An ABE-CR\textsuperscript{2}-SKL scheme $\ABECRCRSKL$ is a tuple of
six algorithms $(\Setup,\qKG, \Enc, \allowbreak\qDec,\qDel, \Vrfy)$. 
Below, let $\cM$  be the message space of $\ABECRCRSKL$, $\cX$ be the
ciphertext-attribute space, and $\cY$ the key-attribute space.
\begin{description}
%\item[$\Setup(1^\secp,1^{\numkey})\ra\msk$:] The setup algorithm takes a security parameter $1^\lambda$ and a collusion bound $1^{\numkey}$, and outputs a master secret key $\msk$.
\item[$\Setup(1^\secp)\ra(\ek,\msk)$:] The setup algorithm takes a
    security parameter $1^\lambda$, and outputs an encryption key
    $\ek$ and a master secret-key $\msk$.

\item[$\qKG(\msk, y)\ra(\qsk,\vk)$:] The key generation algorithm takes
the master secret-key $\msk$ and a key-attribute $y\in \cY$ as inputs, and outputs a decryption key
$\qsk$ and a verification key $\vk$.

\item[$\Enc(\ek,\widetilde{x},\msg)\ra\ct$:] The encryption algorithm takes an
encryption key $\ek$, a ciphertext-attribute $\widetilde{x} \in \cX$, and a message $\msg
\in \cM$ as inputs, and outputs a ciphertext $\ct$.
%\ryo{Do we need $\tilde{x}$ instead of $x$?}
%\nikhil{I wanted something related to ciphertext-attribute space
%$\cX$, while not conflicting with $x$ corresponding to
%$\ket{x}_\theta$}.\ryo{Good.}

\item[$\qDec(\qsk,\ct)\ra\widetilde{\msg}$:] The decryption
    algorithm takes a decryption key $\qsk$ and a ciphertext $\ct$,
    and outputs a value $\widetilde{\msg}$ or $\bot$.

%\item[$\qCert(\qfsk)\ra\cert$:] The certification algorithm takes a function decryption key $\qfsk$, and outputs a classical string $\cert$.

\item[$\qDel(\qsk)\ra\cert$:] The deletion
algorithm takes a decryption key $\qsk$ and outputs a deletion
certificate $\cert$.

\item[$\Vrfy(\vk, \cert')\ra\top/\bot$:] The verification algorithm
takes a verification key $\vk$ and a certificate $\cert'$,
and outputs $\top$ or $\bot$.


\item[Decryption correctness:]For every $\msg \in \cM$, $\widetilde{x} \in \cX$
    and $y \in \cY$ such that $\widetilde{x}(y) = 0$ (decryptable), we have
\begin{align}
\Pr\left[
\qDec(\qsk, \ct) \allowbreak = \msg
\ :
\begin{array}{ll}
(\ek,\msk)\gets\Setup(1^\secp)\\
(\qsk,\vk)\gets\qKG(\msk, y)\\
\ct\gets\Enc(\ek,\widetilde{x},\msg)
\end{array}
\right] 
\ge 1-\negl(\secp).
\end{align}

\item[Verification correctness:] For every $y \in \cY$, we have
\begin{align}
\Pr\left[
\Vrfy(\vk,\cert)=\top
\ :
\begin{array}{ll}
(\ek,\msk)\gets\Setup(1^\secp)\\
(\qsk,\vk)\gets\qKG(\msk, y)\\
\cert\gets\qDel(\qsk)
\end{array}
\right] 
\ge 1-\negl(\secp).
\end{align}
\end{description}

\begin{remark}
We use the same experiment identifier
$\mathsf{sel}\textrm{-}\mathsf{ind}\textrm{-}\mathsf{kla}$ to refer to
the selective IND-KLA
security experiments of ABE-CR-SKL and
ABE-CR\textsuperscript{2}-SKL. This is for the sake of simplicity,
as the exact experiment will be clear from the context.
\end{remark}

\begin{description}
\item[Selective IND-KLA Security:] This security notion for an
ABE-CR\textsuperscript{2}-SKL scheme is defined in the same way as
for ABE-CR-SKL, except that instead of access to the oracle
$\Oracle{\qVrfy}$ in the experiment, $\qA$ receives access to the
following oracle:

\begin{description}

\item[$\Oracle{\Vrfy}(y, \cert)$:] Given $(y,\cert)$, it finds an
entry $(y,\vk,V)$ from $\List{\qKG}$. (If there is no such entry, it returns $\bot$.) 
It then runs $\decision \seteq \Vrfy(\vk, \cert)$ and returns
$\decision$ to $\qA$. If $V=\bot$, it updates the entry into
$(y,\vk,\decision)$.

%\takashi{changed the way of explaining update to better match that for ABE/FE.} %and updates $V\seteq \returned$ if $d=\top$. 
\end{description}

\end{description}
\end{definition}























\subsection{Construction of ABE-CR\textsuperscript{2}-SKL from MI-ABE}\label{sec-ABE-CR-SKL-classical-certificate}
\paragraph{Construction overview.}
%In addition, we present a construction of a Classically-Revocable
%ABE-CR-SKL (ABE-CR\textsuperscript{2}-SKL) scheme.  This offers two
%advantages over the aforementioned constructions of PKE-CR-SKL and ABE-CR-SKL. One is
%that the leased decryption keys have low entanglement, potentially making them
%easier to maintain. The other is that revocation, and consequently the
%verification, are both classical.  However, we require a stronger
%assumption of Multi-Input ABE (MI-ABE) with polynomial-arity for
%arbitrary polynomial-size circuits, a construction of which is currently not known from standard assumptions.
%
%Intuitively, MI-ABE allows to distribute attribute-keys corresponding
%to one of $k=\poly(\secp)$ ``slots''. The ciphertext-policy consists
%of a $k$-input circuit $C$. Consider now a set of $k$ keys
%$\{\abe.\sk_{i, u_i}\}_{i \in [k]}$ where $\abe.\sk_{i, u_i}$
%corresponds to the $i$-th slot and an attribute $u_i$. Decryption is
%feasible as long as $C(u_1, \ldots, u_n)$ satisfies the relation.
%Furthermore, it is guaranteed that an adversary can decrypt only if
%they have such a set of $k$-keys satisfying the multi-input circuit
%policy.
%Even though this primitive is not currently known without IO, we believe this is
%a meaningful connection because MI-ABE is a plausibly weaker primitive
%than IO or collusion-resistant FE. Evidently, a secret-key encryption variant of such an
%MI-ABE was recently constructed in the work of Agrawal, Kumari and Yamada
%\cite{AKY24} from $\kappa$-evasive LWE, PRFs and standard LWE.
%Although evasive LWE is a stronger assumption than standard LWE, this
%is a still a better state of affairs than what is known about
%post-quantum FE and iO.



We first provide the main idea
behind our ABE-CR\textsuperscript{2}-SKL based on MI-ABE, which follows along the lines of our
ABE-CR-SKL construction with some key-differences.
In this case, we directly rely on a BB84-based SKE-CD scheme as a
building block, instead of needing something similar to SKFE-CR-SKL.
Consider now an SKE-CD ciphertext $\skecd.\qct$ of the plaintext $0^\secp$.
Let its corresponding verification-key $\skecd.\vk$ be of the form
$\skecd.\vk = (x,
\theta)$, where $x$ and $\theta$ are $k$-bit strings. Then, $\skecd.\qct$ is
of the form $\ket{\psi_1} \otimes \cdots \otimes \ket{\psi_k}$, where
for $i \in [k]$, $\ket{\psi_i}$ is of the following form:

\begin{align}
\ket{\psi_i}=
\begin{cases}
    \ket{x[i]} & if~~ \theta[i]=0\\
    \ket{0}+(-1)^{x[i]}\ket{1} & if~~ \theta[i]=1
\end{cases}
\end{align}

Now, for each $i \in \{2, \dots, k\}$ and $u \in \bit$, consider the attribute-key
$\abe.\sk_{i, u}$ generated by the MI-ABE key-generation
algorithm for slot $i$ and attribute $t \| u$. Here, $t$ is a value chosen at
random and is common for all slots corresponding to a given decryption-key
$\qsk$. Additionally, for $u \in \bit$, consider the keys
$\abe.\sk_{1,u}$ corresponding to the attributes $t \| u \| y$ where
$y$ is the actual ABE key-attribute.

Consider now the following state $\rho_i$ for each $i \in [k]$:

\begin{align}
    \rho_i=
    \begin{cases}
        \ket{x[i]}\ket{\abe.\sk_{i,x[i]}} & if~~ \theta[i]=0\\
        \ket{0}\ket{\abe.\sk_{i,0}}+(-1)^{x[i]}\ket{1}\ket{\abe.\sk_{i,1}} & if~~ \theta[i]=1,
    \end{cases}
\end{align}

The quantum decryption-key of our scheme will be the tuple $\qsk =
(\rho_i)_{i \in [k]}$. The encryption algorithm is similar to the
ABE-CR-SKL scheme, and outputs an MI-ABE encryption of message $\msg$
under a policy $C_{\mathsf{ABE}} \| \tlC$ where $\tlC \gets
\CCSim(1^\secp)$ is a simulator of a compute and compare obfuscator and $C_{\mathsf{ABE}}$ is the actual ABE policy.
However, the MI-ABE relation is a little different. The relation is
such that for a set of $k$ attributes $y_1 = t_1 \| u_1 \| y$ and $y_2
= t_2 \| u_2, \ldots, y_k = t_k \| u_k$, it outputs $0$ (decryptable)
whenever $C_{\mathsf{ABE}}(y) = 0$ (the ABE relation is satisfied) AND $t_1 = \ldots = t_k$ AND
$\tlC(u_1 \| \ldots \| u_k) = \bot$.  Otherwise, it outputs $1$. As in
ABE-CR-SKL, the idea is that $\tlC$ is indistinguishable from $\tlC^*$
that is an obfuscation of the compute-and-compare circuit
$\cnc{D[\lock \xor r, \skecd.\sk]}{\lock, 0}$.  Here, $\lock$ and $r$ are
random values, and $\skecd.\sk$ is the secret-key of the SKE-CD scheme.  The
circuit $D$ is such that $D(u)$ outputs $\lock \xor r \xor
\SKECD.\CDec(\skecd.\sk, u)$, where $\SKECD.\CDec$ is the classical
decryption algorithm of the SKE-CD scheme.

Consequently, when every $\qsk = (\rho_i)_{i\in[k]}$ the adversary
receives is generated using an SKE-CD encryption of $r$ (instead of
$0^\secp$ as in the scheme), then any tuple of $k$ MI-ABE keys of the
adversary having attributes $\big(t_1 \| u_1 \| y, t_2 \| u_2 \ldots, t_k \| u_k\big)$
satisfies one of the three conditions:

\begin{itemize}
\item The values $t_1 \ldots t_k$ are not all the same.
\item $C_{\mathsf{ABE}}(y) \neq 0$.
\item For $u = u_1 \| \ldots \| u_k$, $D[\lock \xor r, \skecd.\sk](u)$
returns $\lock$.
\end{itemize}

Notice that the former condition ensures that the adversary cannot
interleave keys corresponding to different decryption-keys.
The last condition holds because the adversary never receives
$\abe.\sk_{i, 1 - x[i]}$ for any $i$ such that $\theta[i] = 0$.
Consequently, $u$ and $x$ are the same at all positions where
$\theta[i]=0$. This means that $\SKECD.\CDec(\skecd.\sk, u)$ outputs $r$ by the
classical decryption property of $\SKECD$ (Definition \ref{def:bb84}).
It is important to note that the positions $i$ where $\theta[i] = 1$
(the Hadamard positions) have no effect on the value output by
$\SKECD.\CDec$
as their purpose is just in the verification of deletion.
As a result, the security of
MI-ABE allows to simulate the adversary's view in this hybrid, as
no ``decryptable'' set of $k$ keys is given out.

Importantly, the switch from SKE-CD encryptions of $0^\secp$ to $r$ is
indistinguishable, given that the adversary deletes all the information
in the SKE-CD ciphertexts.  To enforce this, the deletion algorithm
requires the adversary to measure both the SKE-CD and MI-ABE registers
for each slot to obtain values $(c_i, d_i)_{i \in [k]}$. Then, given
the values $\{\abe.\sk_{i, 0} \xor \abe.\sk_{i, 1}\}_{i \in [k]}$ as
part of the verification key, the verification checks whether $x[i] =
c_i \xor d_i \cdot (\abe.\sk_{i, 0} \xor \abe.\sk_{i, 1})$ holds for
every $i \in [k]$ such that $\theta[i] = 1$. As a result, we are able
use a standard hybrid argument to turn any distinguisher (of the
$0^\secp$ and $r$ hybrids) into an attack on the certified deletion
security of the SKE-CD scheme.
%Consequently, we have that following theorem:
%\begin{theorem}
%There exists an ABE-CR\textsuperscript{2}-SKL scheme satisfying
%selective IND-KLA security, assuming the existence of a Multi-Input
%ABE scheme for polynomial-arity, compute-and-compare obfuscation, and
%a BB84-based SKE-CD scheme.
%\end{theorem}
%\fuyuki{I would suggest to move this subsection to the corresponding technical section. There are already many information including new concepts in the above parts. We might want to focus on providing the essence of our techniques. What do you think?}
%\nikhil{I was thinking of this as an optional sub-section for the
%reader but I am fine with either.}
%\fuyuki{Moved.}





\paragraph{Construction.}
We will construct an ABE-CR\textsuperscript{2}-SKL
scheme $\ABECRCRSKL = \ABECRCRSKL.(\Setup, \qKG, \Enc,$ $\qDec,
\qDel, \Vrfy)$ using the following building blocks:

\begin{itemize}
\item Multi-Input (Ciphertext-Policy) ABE Scheme $\MIABE =
\MIABE.(\Setup, \KG,\allowbreak \Enc, \Dec)$ for the following relation:

\begin{description}
    \item[$R(\widetilde{x} \| z,y_1, \cdots, y_k)$:] Let $\widetilde{x}$ be
interpreted as a circuit and $z$ as a $k$-input circuit.
\begin{itemize}
\item Parse $y_1 = t_1 \| u_1 \| y$.
\item Parse $y_i=t_i\|u_i$ for every $i\in 2, \ldots, k$.
\item If $t_i\ne t_j$ for some $i,j\in[k]$, output $1$. Otherwise, go to the next step.
\item If $z(u_1\|\cdots\|u_k) = \bot$ AND $\widetilde{x}(y) = 0$, output $0$
(decryptable). Else, output $1$.
\end{itemize}
\end{description}

\item Compute-and-Compare Obfuscation $\CCObf$ with the simulator
$\CCSim$.

\item BB84-based SKE-CD scheme $\SKECD =
\SKECD.(\KG,\qEnc,\qDec,\qDel,\Vrfy)$ with the classical
decryption algorithm $\SKECD.\CDec$.
\end{itemize}

The description of each algorithm of $\ABECRCRSKL$ is as follows.

\begin{description}
\item[$\ABECRCRSKL.\Setup(1^\secp)$:] $ $
\begin{itemize}
    \item Generate $(\abe.\pk, \abe.\msk)\gets\MIABE.\Setup(1^\secp)$.
        \item Generate $\skecd.\sk\gets\SKECD.\KG(1^\secp)$.
    \item Output $\ek\seteq\abe.\pk$ and
        $\msk\seteq(\abe.\msk,\skecd.\sk)$.
\end{itemize}



\item[$\ABECRCRSKL.\qKG(\msk, y)$:] $ $
\begin{itemize}
\item Parse $\msk=(\abe.\msk,\ske.\msk)$.
\item Sample a random value $t\gets\bit^\secp$.
\item Generate
$(\skecd.\qct,\skecd.\vk)\gets\SKECD.\qEnc(\skecd.\sk,0^\secp)$.
$\skecd.\vk$ is of the form
$(x,\theta)\in\bit^{k}\times\bit^{k}$, and $\skecd.\qct$ is of
the form
$\ket{\psi_1}_{\qreg{SKECD.CT_1}}\tensor\cdots\tensor\ket{\psi_{k}}_{\qreg{SKECD.CT_{k}}}$.

\item Generate $\abe.\sk_{1,b}\gets\MIABE.\KG(\abe.\msk, 1, t\|b\|y)$
  for each $b \in \bit$.

\item For every $i \in 2, \ldots, k$, do the following:
\begin{itemize}
\item Generate $\abe.\sk_{i,b}\gets\MIABE.\KG(\abe.\msk, i, t\|b)$
for each $b\in\bit$.
\end{itemize}

\item For every $i \in [k]$, do the following:
\begin{itemize}
\item Prepare a register $\qreg{ABE.SK_i}$ that is initialized
to $\ket{0\cdots0}_{\qreg{ABE.SK_i}}$.

\item Apply the map
$\ket{u}_{\qreg{SKECD.CT_i}}\ket{v}_{\qreg{ABE.SK_i}}\ra\ket{u}_{\qreg{SKECD.CT_i}}\ket{v\oplus\abe.\sk_{i,u}}_{\qreg{ABE.SK_i}}$
to the registers $\qreg{SKECD.CT_i}$ and $\qreg{ABE.SK_i}$, and
obtain the resulting state $\rho_i$. 
\end{itemize}

\item Output $\qsk:=(\rho_i)_{i \in [k]}$ and
$\vk\seteq\big(\skecd.\vk,
\{\abe.\sk_{i,0} \xor
\abe.\sk_{i,1}\}_{i\in[k]:\theta[i]=1}\big)$.

\end{itemize}

\item[$\ABECRCRSKL.\Enc(\ek,\widetilde{x},\msg)$:] $ $
\begin{itemize}
    \item Parse $\ek=\abe.\pk$.
\item Generate $\tlC\gets\CCSim(1^\secp,\pp_D,1)$, where $\pp_D$ consists of circuit parameters of $D$ defined in the security proof.
    \item Generate $\abe.\ct\gets\MIABE.\Enc(\abe.\pk,\widetilde{x} \| \tlC,\msg)$.
    \item Output $\ct\seteq\abe.\ct$.
\end{itemize}
 
\item[$\ABECRCRSKL.\qDec(\qsk,\ct)$:] $ $
\begin{itemize}
\item Parse $\qsk=(\rho_i)_{i\in[k]}$ and $\ct=\abe.\ct$. We
denote the register holding $\rho_i$ as $\qreg{SKECD.CT_i}\tensor\qreg{ABE.SK_i}$.

\item Prepare a register $\qreg{MSG}$ that is initialized to $\ket{0\cdots0}_{\qreg{MSG}}$

\item To the registers $\qreg{\bigotimes_{i\in[k]}ABE.SK_i}$ and
$\qreg{MSG}$, apply 
$\bigotimes_{i\in[k]}\ket{v_i}_{\qreg{ABE.SK_i}}\otimes\ket{w}_{\qreg{MSG}}\ra\bigotimes_{i\in[k]}\ket{v_i}_{\qreg{ABE.SK_i}}\otimes\ket{w\oplus\MIABE.\Dec(\abe.\ct,
v_1,\cdots,v_k)}_{\qreg{MSG}}$.

   \item Measure the register $\qreg{MSG}$ in the computational basis and output the result $\msg^\prime$.
\end{itemize}

\item[$\ABECRCRSKL.\qDel(\qsk)$:] $ $
\begin{itemize}
\item Parse $\qsk = (\rho_i)_{i \in [k]}$. Let the register 
holding $\rho_i$ be denoted as $\qreg{SKECD.CT_i}\tensor\qreg{ABE.SK_i}$.
\item For each $i \in [k]$, measure the registers $\qreg{SKECD_i}$ and $\qreg{ABE.SK_i}$ in the Hadamard basis to
obtain outcomes $c_i$ and $d_i$.
\item Output $\cert = (c_i,d_i)_{i \in [k]}$.
\end{itemize}



\item[$\ABECRCRSKL.\Vrfy(\vk,\cert)$:] $ $
\begin{itemize}
\item Parse $\vk = \big(\skecd.\vk=(x,\theta), \{\abe.\sk_{i,0}
\xor \abe.\sk_{i,1}\}_{i\in[k]:\theta[i]=1}\big)$ and $\cert =
(c_i,d_i)_{i \in [k]}$.

\item Output $\top$ if $x[i]=c_i\oplus
    d_i\cdot(\abe.\sk_{i,0}\oplus\abe.\sk_{i,1})$ holds for every
    $i\in[k]$ such that $\theta[i]=1$ and $\bot$ otherwise.

\end{itemize}
\end{description}

Let $\qsk \gets\ABECRCRSKL.\qKG(\msk, y)$. $\qsk$ is of the form
$(\rho_i)_{i \in [k]}$, where $\rho_i$ is of the following form, where
$(x, \theta)$ is the verification key of a BB84-based SKECD scheme:
\begin{align}
\rho_i=
\begin{cases}
    \ket{x[i]}\ket{\abe.\sk_{i,x[i]}} & if~~ \theta[i]=0\\
    \ket{0}\ket{\abe.\sk_{i,0}}+(-1)^{x[i]}\ket{1}\ket{\abe.\sk_{i,1}} & if~~ \theta[i]=1,
\end{cases}
\end{align}

Recall that $\abe.\sk_{i, b} = \MIABE.\KG(\abe.\msk, i, t\|b)$ for every
$i \in \{2, \ldots, k\}$ and $b \in \bit$. Moreover, 
$\abe.\sk_{1, b} = \MIABE.\KG(\abe.\msk, 1, t\|b\|y)$ for each $b \in
\bit$.

\paragraph{Decryption correctness.}
The MI-ABE relation defined in the construction is as follows:

\begin{description}
\item[$R(\widetilde{x} \| z,y_1, \cdots, y_k)$:] Let $\widetilde{x}$ be
interpreted as a circuit and $z$ as a $k$-input circuit.
\begin{itemize}
\item Parse $y_1 = t_1 \| u_1 \| y$.
\item Parse $y_i=t_i\|u_i$ for every $i\in 2, \ldots, k$.
\item If $t_i\ne t_j$ for some $i,j\in[k]$, output $1$. Otherwise, go to the next step.
\item If $z(u_1\|\cdots\|u_k) = \bot$ AND $\widetilde{x}(y) = 0$, output $0$
(decryptable). Else, output $1$.
\end{itemize}
\end{description}

Clearly, $z \seteq \tlC \gets \CCSim(1^\secp, \pp_D, 1)$ always outputs $\bot$
and $t_1 = \ldots = t_k$ holds by construction. Hence, the guarantee
follows from the decryption correctness of $\MIABE$, as long as
$\widetilde{x}(y) = 0$ holds, as desired.

\paragraph{Verification correctness.}

Recall that the verification only checks the Hadamard basis positions,
i.e, positions $i \in [k]$ such that $\theta[i] = 1$. Consider the
outcome $(c_i, d_i)$ obtained by measuring the state
$\ket{0}\ket{\abe.\sk_{i,0}}+(-1)^{x[i]}\ket{1}\ket{\abe.\sk_{i,1}}$
in the Hadamard basis, where $c_i$ denotes the first bit of the
outcome. It is easy to see that $x[i] = c_i \xor d_i \cdot
(\abe.\sk_{i, 0} \xor \abe.\sk_{i, 1})$ is satisfied. Hence, the
verification correctness follows.

\subsection{Proof of Selective IND-KLA Security}
Let $\qA$ be an adversary for the selective IND-KLA security of $\ABECRCRSKL$.
We consider the following sequence of experiments.
\begin{description}
\item[$\hybi{0}^\coin$:] This is
$\expc{\ABECRCRSKL,\qA}{sel}{ind}{kla}(1^\secp,\coin)$.

\begin{enumerate}
\item $\qA$ declares the challenge ciphertext attribute $x^* \in \cX$.
\item The challenger $\qCh$ generates
    $(\abe.\pk,\abe.\msk)\gets\MIABE.\Setup(1^\secp)$ and
    $\skecd.\sk \gets \SKECD.\KG(1^\secp)$, and sends $\ek\seteq\abe.\pk$ to $\qA$.

\item $\qA$ can get access to the following (stateful) oracles, where
the list $L_{\qKG}$ used by the oracles is initialized to an empty
list:

%\nikhil{As in ABE-CR-SKL proof, I've added the subscript y here.}

\begin{description}
\item[$\Oracle{\qKG}(y)$:] Given $y$, it finds an entry of the form
$(y,\vk_y,V_y)$ from $\List{\qKG}$. If there is such an entry, it returns $\bot$.
Otherwise, it generates $(\qsk_y,\vk_y)\la\qKG(\msk,y)$,
sends $\qsk_y$ to $\qA$, and adds $(y,\vk_y,\bot)$ to $\List{\qKG}$.
    
\item[$\Oracle{\Vrfy}(y, \cert)$:] Given $(y, \cert)$, it finds an
entry $(y,\vk_y,V_y)$ from $\List{\qKG}$. (If there is no such entry, it returns $\bot$.) 
It then runs $\decision \seteq \Vrfy(\vk_y, \cert)$ and returns
$\decision$ to $\qA$. If $V_y=\bot$, it updates the entry into
$(y,\vk_y,\decision)$.
\end{description}

\item $\qA$ sends $(\msg_0^*,\msg_1^*)\in \cM^2$ to the challenger. If
$V_j=\bot$ for some $j\in[q]$, $\qCh$ outputs $0$ as the final
output of this experiment. Otherwise, $\qCh$ generates
$\tlC^*\gets\CCSim(1^\secp, \pp_D, 1)$ and
$\abe.\ct^*\gets\MIABE.\Enc(\abe.\pk,x^* \| \tlC^*,\msg_\coin^*)$, and sends
$\ct^*\seteq\abe.\ct^*$ to $\qA$.

\item $\qA$ outputs $\coin^\prime$. $\qCh$ outputs $\coin'$ as the final output of the experiment.
\end{enumerate}

\item[$\hybi{1}^\coin$:]This is the same as $\hybi{0}^\coin$ except
that $\tlC^*$ is generated as $\tlC^*\gets\CCObf(1^\secp,D[\lock\oplus
r, \skecd.\sk], \lock,0)$, where $\lock\gets\bit^\secp$, $r\gets\bit^\secp$ and
$D[\lock\oplus r, \skecd.\sk](x)$ is a circuit that has $\lock\oplus
r$ and $\skecd.\sk$ hardwired and outputs $\lock\oplus
r\oplus\SKECD.\CDec(\skecd.\sk,x)$.

Since $\lock$ is chosen at random independently of all other
variables, from the security of compute-and-compare obfuscation, we
have that $\Hyb_0^\coin \approx \Hyb_1^\coin$.

\item[$\hybi{2}^\coin$:]This is the same as $\hybi{1}^\coin$ except
that $\skecd.\qct$ generated as part of $\ABECRCRSKL.\qKG(\msk, y)$
such that $x^*(y) = 0$, is generated as
$(\skecd.\qct,\allowbreak\skecd.\vk)\gets\SKECD.\qEnc(\skecd.\sk,r)$.

In the previous step, we changed the distribution of $\tlC^*$ used
to generate the challenge ciphertext so that $\tlC^*$ has
$\skecd.\sk$ hardwired. However, the ciphertext is given to $\qA$
after $\qA$ deletes all the leased secret keys satisfying the ABE
relation, and thus the
corresponding ciphertetexts of $\SKECD$.  Thus, we can use the security of
$\SKECD$ to argue that $\hybi{1}^\coin \approx \hybi{2}^\coin$. Let
$q$ be the number of key-queries made to $\Oracle{\qKG}$ that satisfy
the relation wrt $x^*$. Consider now $q+1$ hybrids $\Hyb_{1,0}^\coin, \cdots, 
\Hyb_{1,q}^\coin$ where for every $l \in [q]$, $\Hyb_{1,l}^\coin$
is such that the first $l$ of the $q$ keys are generated using
$\SKECD$ encryptions of $r$ (instead of $0^\secp$). Notice that
$\Hyb_{1,0}^\coin \equiv \Hyb_1^\coin$ and
$\Hyb_{1,q}^\coin \equiv \Hyb_2^\coin$. Suppose that
$\Hyb_{1,l}^\coin \not\approx \Hyb_{1,l+1}^\coin$ for some $l \in
[q]$. Let $\qD$ be a distinguisher with non-negligible advantage in
distinguishing the hybrids.
We will construct the following reduction to the IND-CVA-CD
security of $\SKECD$:

\begin{description}
\item Execution of $\qR^\qD$ in
$\expc{\SKECD,\qR}{ind}{cva}{cd}(1^\secp,b)$:
\begin{enumerate}
\item The challenger $\qCh$ computes $\sk \gets \KG(1^\secp)$.
\item $\qD$ declares a challenge ciphertext attribute $x^*$ to $\qR$. $\qR$ generates $(\abe.\pk, \abe.\msk) \gets
\MIABE.\Setup(\allowbreak1^\secp)$ and sends $\ek \seteq \abe.\pk$ to $\qD$.

\item $\qR$ sends $(r_0,r_1) \seteq (0^\secp, r)$ to $\qCh$.
\item $\qCh$ computes $(\qct^\star, \vk^\star) \gets \qEnc(\sk,
r_b)$ and sends $\qct^\star$ to $\qR$.

\item $\qR$ simulates the oracle $\Oracle{\qKG}(y)$ for $\qD$ as
follows. Given $y$, it finds an entry $(y, \vk_y, V_y)$ from
$L_{\qKG}$. If there is such an entry, it returns $\perp$. Otherwise,
it proceeds as follows:

\begin{itemize}
\item For the $j$-th query (of the $q$ satisfying key-queries), if $j \neq l+1$, $\qR$ computes 
$\qsk_y$ and $\vk_y$ as in $\Hyb_0^\coin$, except
that the values $(\skecd.\qct, \skecd.\vk)$ are computed using
$\Oracle{\Enc}(r)$ for $j \in [l]$ and using
$\Oracle{\Enc}(0^\secp)$ for $j \in \{l+2, \ldots, q\}$.

\item For the $j$-th query (of the $q$ satisfying key-queries), if $j = l+1$, $\qR$ computes 
$\qsk_{y}$ as in $\Hyb_0^\coin$, except that the value $\qct^\star$ is used instead
of $\skecd.\qct$. Consider the values $\{\abe.\sk_{i,0},
\abe.\sk_{i,1}\}_{i\in[k] : \theta[i]=1}$ computed during the
computation of $\qsk_{y}$. $\qR$ computes the value
$\widetilde{\vk}_{y} = \{\abe.\sk_{i,0}\xor
\abe.\sk_{i,1}\}_{i\in[k]}$.

\item If $x^*(y) \neq 0$ (unsatisfying key-queries), $\qR$ computes
$\qsk_y$ and $\vk_y$ as in $\Hyb_0^\coin$, except that $(\skecd.\qct,
\skecd.\vk)$ are computed using $\Oracle{\Enc}(0^\secp)$.
\end{itemize}

\item $\qR$ simulates the view of oracle $\Oracle{\Vrfy}(y, \cert)$ for
$\qD$ as follows. Given $(y, \cert)$, it finds an entry $(y, \vk_y,
V_y)$ from $L_{\qKG}$. If there is no such entry, it returns $\perp$.
Otherwise, it proceeds as follows:
\begin{itemize}
\item If $x^*(y) \neq 0$, $\qR$ simulates access to $\Oracle{\Vrfy}$ as in $\Hyb_0^\coin$.

\item If $y$ corresponds to the $j$-th satisfying key-query to
$\Oracle{\qKG}$ and $j \neq l+1$,
$\qR$ simulates access to $\Oracle{\Vrfy}$ as in $\Hyb_0^\coin$.

\item If $y$ corresponds to the $j$-th satisfying key-query to
$\Oracle{\qKG}$ and $j = l+1$, $\qR$ simulates access to
$\Oracle{\Vrfy}(y,\cert)$ as follows:

\begin{enumerate}
    \item Parse $\widetilde{\vk}_{y} = \{\abe.\sk_{i,0}\xor
\abe.\sk_{i,1}\}_{i\in[k]}$ and $\cert=(c_i,d_i)_{i \in [k]}$.
\item Compute $x[i] = c_i \xor d_i\cdot(\abe.\sk_{i,0} \xor
\abe.\sk_{i,1})$ for every $i \in [k]$.
\item If $\Oracle{\Vrfy}(x) = \sk$, set $d \seteq \top$. Else,
      set $d \seteq \bot$.
\end{enumerate}

It returns $d$ to $\qD$. If $V_{y} = \bot$, it updates the entry in
$L_{\qKG}$ into $(y, \vk_y, d)$.
\end{itemize}

\item $\qD$ sends $(m_0^*, m_1^*) \in \cM^2$ to $\qR$. $\qR$ 
computes $\ct^*$ as in Step $4.$ of $\Hyb_1^\coin$ using
$\skecd.\sk \seteq \sk$ obtained from $\qCh$. It sends $\ct^*$ to
$\qD$ if $V_y = \top$ for every entry of the form $(y, \vk_y, V_y)$ in
$L_{\qKG}$ such that $x^*(y) = 0$. Else, it outputs $\bot$.
\item $\qD$ guesses a bit $b'$. $\qR$ sends $b'$ to $\qCh$.
\end{enumerate}
\end{description}
Notice that the view of $\qD$ in the reduction is that of
$\Hyb_{1,l}^\coin$ if $b=0$ and $\Hyb_{1,l+1}^\coin$ if $b=1$.
Moreover, $\qD$ can only distinguish when $V_y = \top$ for all entries
$(y, \vk_y, V_y) \in L_{\qKG}$ such that $x^*(y) = 0$.
This means that $V_{y}$ corresponding to the $(l+1)$-th satisfying
query to $\Oracle{\qKG}$ also
satisfies $V_y = \top$. Consequently, the
corresponding verification check of $\qCh$ is also $\top$. Hence, $\qR$ succeeds
in breaking IND-CVA-CD security of $\SKECD$. Therefore, we have
$\Hyb_1^\coin \approx \Hyb_2^\coin$.
\end{description}

We will now bound the distinguishing gap between $\hybi{2}^0$ and
$\hybi{2}^1$ using the security of $\MIABE$. Consider a
decryption key $\qsk_y = (\rho_i)_{i\in[k]}$ such that $x^*(y) = 0$. Recall that a ciphertext of BB84-based $\SKECD$ is of the form
$\ket{\psi_1}\tensor\cdots \tensor\ket{\psi_k}$, where
    \begin{align}
    \ket{\psi_i}=
    \begin{cases}
        \ket{x[i]} & \textrm{if}~~ \theta[i]=0\\
        \ket{0}+(-1)^{x[i]}\ket{1} & \textrm{if}~~ \theta[i]=1
    \end{cases}
    \end{align}
As a result, we have
    \begin{align}
    \rho_i=
    \begin{cases}
        \ket{x[i]}\ket{\abe.\sk_{i,x[i]}} & \textrm{if}~~ \theta[i]=0\\
        \ket{0}\ket{\abe.\sk_{i,0}}+(-1)^{x[i]}\ket{1}\ket{\abe.\sk_{i,1}} & \textrm{if}~~ \theta[i]=1,
    \end{cases}
    \end{align}
where $\abe.\sk_{1,b}\gets\MIABE.\KG(\abe.\msk,1,t_y\|b\|y)$ for each
$b \in \bit$ and
$\abe.\sk_{i,b}\gets\MIABE.\KG(\abe.\msk,i,\allowbreak t_y\|b)$
for each $i \in \{2, \ldots, k\}$ and $b \in \bit$. Here, $t_y$ is a
random value that is common for each of the $k$ ``slots'' of the
decryption-key $\qsk_y$. Notice first that no
two values $t_y, t_w$ are equal for $t_w$ corresponding to some
$\qsk_w$ except with negligible probability.
Consequently, due to the defined relation $R$ and the selective
security of $\MIABE$, any set of $k$ secret keys is ``decryptable'' only if
they correspond to the same decryption key $\qsk_y$. Now, we notice
that $\abe.\sk_{i,1-x[i]}$ is not given to $\qA$ for any $i\in[k]$
such that $\theta[i]=0$. This means that for any set of $k$ $\MIABE$
keys corresponding to the decryption-key $\qsk_y$, the attributes
$(x'[1], \cdots, x'[k])$ satisfy $x'[i] = x[i]$ for all $i:\theta[i] =
0$. Consequently, the classical decryption property (See Definition
\ref{def:bb84}) of the BB84-based SKE-CD scheme guarantees that $\lock
\xor r \xor \SKECD.\CDec(\skecd.\sk, x') = \lock$ in these hybrids. This means that
$\tlC^*(x') = 0$ (instead of $\bot$).  As a result, any subset of
$\MIABE$ keys of size $k$ given to $\qA$ is a ``non-decryptable'' set
in the hybrids $\Hyb_2^0$ and $\Hyb_2^1$.
It is easy to see now that we can reduce to the selective security of
$\MIABE$. Specifically, the reduction specifies the target attribute
$x^* \| \tlC^*$ where
$\tlC^* \gets \CCObf(1^\secp, D[\lock\xor r, \skecd.\sk], \lock, 0)$
and simulates the view for $\qA$ by querying the key-generation oracle
of $\MIABE$ accordingly. Note that the reduction can handle
verification queries since it can obtain both $\abe.\sk_{i,0}$ and
$\abe.\sk_{i,1}$ for every $i\in[k]$ such that $\theta[i]=1$ that are
sufficient to perform the verification.  (Especially, it does not need
$\abe.\sk_{i,1-x[i]}$ for $i\in[k]$ such that $\theta[i]=0$ for
verification.) We now state the following theorem:

\begin{theorem}
There exists an ABE-CR\textsuperscript{2}-SKL scheme satisfying
selective IND-KLA Security, assuming the existence of a
selectively-secure Multi-Input ABE scheme for polynomial-arity,
Compute-and-Compare Obfuscation, and a BB84-based SKE-CD scheme.
\end{theorem}
