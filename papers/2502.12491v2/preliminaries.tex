% !TEX root = main.tex

\section{Preliminaries}\label{sec:prelim}
\ifnum\submission=1
\else
\paragraph{Notations and conventions.}
In this paper, standard math or sans serif font stands for classical algorithms (e.g., $C$ or $\algo{Gen}$) and classical variables (e.g., $x$ or $\keys{pk}$).
Calligraphic font stands for quantum algorithms (e.g., $\qalgo{Gen}$) and calligraphic font and/or the bracket notation for (mixed) quantum states (e.g., $\qstate{q}$ or $\ket{\psi}$).

Let $[\ell]$ denote the set of integers $\{1, \cdots, \ell \}$, $\secp$ denote a security parameter, and $y \seteq z$ denote that $y$ is set, defined, or substituted by $z$.
For a finite set $X$ and a distribution $D$, $x \chosen X$ denotes selecting an element from $X$ uniformly at random, and $x \chosen D$ denotes sampling an element $x$ according to $D$. Let $y \gets \algo{A}(x)$ and $y \gets \qalgo{A}(\qstate{x})$ denote assigning to $y$ the output of a probabilistic or deterministic algorithm $\algo{A}$ and a quantum algorithm $\qalgo{A}$ on an input $x$ and $\qstate{x}$, respectively.
PPT and QPT algorithms stand for probabilistic polynomial-time algorithms and polynomial-time quantum algorithms, respectively.
Let $\negl$ denote a negligible function.
For strings $x,y\in \bit^n$, $x\cdot y$ denotes $\bigoplus_{i\in[n]} x_i y_i$ where $x_i$ and $y_i$ denote the $i$th bit of $x$ and $y$, respectively.
%\fuyuki{Need to check if all the notations here are really used in this paper or not.}
%\nikhil{I think so, except we didn't use ; for randomness. Also need to check if everything is covered here.}
For random variables $X$ and $Y$, we use the notation $X \approx Y$ to denote that these are computationally indistinguishable. Likewise, $X \approx_s Y$ denotes that they are statistically indistinguishable.
\fi


\subsection{One Way to Hiding Lemmas}


\begin{lemma}[O2H Lemma~\cite{C:AmbHamUnr19}]\label{lem:O2Hprev}
Let $G,H:X\ra Y$ be any functions, $z$ be a random value, and $S\subseteq X$ be a random set such that $G(x)=H(x)$ holds for every $x\notin S$.
The tuple $(G,H,S,z)$ may have arbitrary joint distribution.
Furthermore, let $\qA$ be a quantum oracle algorithm that makes at most $q$ quantum queries.
Let $\qB$ be an algorithm such that $\qB^H$ on input $z$ chooses $i\gets[q]$, runs $\qA^H(z)$, measures $\qA$'s $i$-th query, and outputs the measurement outcome.
Then, we have:
\begin{align}
\abs{\Pr[\qA^{H}(z)=1]-\Pr[\qA^{G}(z)=1]} \leq 2q\cdot\sqrt{\Pr[\qB^{H}(z)\in S]}
\enspace.
\end{align}
\end{lemma}

We require a generalization of this lemma, where $\qA$ receives an
additional quantum oracle $\cQ$ in both worlds. Consequently, we
consider $\qB$ to be given oracle access to $\cQ$, which it uses to
simulate the oracle calls of $\qA$ to $\cQ$. Notice that if the
outputs of $\cQ$ were classical, we could have simply defined
augmented oracles $G'$ (likewise $H'$) based on $G$ (likewise $H$) and
$\cQ$.  However, the oracles $\cQ$ we consider will have classical
inputs and quantum state outputs. Consequently, the lemma we require
is stated as follows:

\begin{lemma}[O2H Lemma with Auxiliary Quantum Oracle]\label{lem:O2H}
Let $G, H: X \ra Y$ be any functions, $z$ be a random value, and $S
\subseteq X$ be a random set such that $G(x) = H(x)$ holds for every
$x \notin S$. The tuple $(G, H, S, z)$ may have arbitrary joint
distribution. Furthermore, let $\cQ$ be a quantum oracle that is
arbitrarily correlated with the tuple $(G, H, S, z)$, takes
classical input and produces a (possibly mixed) quantum state as
output. Let $\qA$ be a quantum oracle algorithm that makes at most $q$
quantum queries to the oracles $H$ or $G$, and arbitrarily many
queries to the oracle $\cQ$. Let $\qB$ be an algorithm such that
$\qB^{\cQ, H}$ on input $z$, chooses $i \gets [q]$, runs $\qA^{\cQ,
H}(z)$, measures $\qA$'s $i$-th query to $H$, and outputs the measurement
outcome. Then, we have:

\begin{align}
\abs{\Pr[\qA^{\cQ, H}(z)=1]-\Pr[\qA^{\cQ, G}(z)=1]} \leq
2q\cdot\sqrt{\Pr[\qB^{\cQ, H}(z)\in S]}
\enspace.
\end{align}

\end{lemma}

\ifnum\submission=1
We prove this lemma in~\cref{appsec:O2H_oracle}.
\else
% !TEX root = main.tex

\ifnum\submission=1
\section{Proof of O2H Lemma with Auxiliary Quantum Oracle}\label{appsec:O2H_oracle}
\else
\fi
\begin{proof}[Proof of~\cref{lem:O2H}]
Let us consider an adversary $\widetilde{\qA}$ that receives as input
the description $\langle \cQ\rangle$ of $\cQ$\footnote{The O2H Lemma (Lemma \ref{lem:O2Hprev}) holds even if $z$ is exponentially large, so the description of $\cQ$ need not be concise.}, along with the input
$z$ used by $\qA$. Given oracle access to $H$ (likewise $G$) and $(z,
\langle \cQ\rangle)$ as input, $\widetilde{\qA}$ simply runs
$\qA^{H, \cQ}(z)$ (likewise $\qA^{G, \cQ}(z)$) by simulating its queries to $\cQ$
using the description $\langle \cQ\rangle$. Then, the O2H Lemma (Lemma
\ref{lem:O2Hprev}) implies the existence of an algorithm
$\widetilde{\qB}^{H}(z, \langle \cQ\rangle)$ that chooses $i \gets
[q]$, runs $\widetilde{\qA}^H(z, \langle \cQ\rangle)$, measures its
$i$-th query to $H$ and outputs the measurement outcome. Observe that
the algorithms $\widetilde{A}$ and $\widetilde{B}$ do not make use of
the description $\langle \cQ\rangle$ except for simulating the queries
made by $\qA$.  Consequently, there exists an algorithm
$\qB^{\cQ, H}(z)$ equivalent to $\widetilde{\qB}^{H}(z, \langle \cQ\rangle)$ that directly runs $\qA$ (instead of $\widetilde{\qA}$)
and simulates its oracle queries to $\cQ$ using its own access to
$\cQ$.
\end{proof}

\fi
%\begin{proof}
%As in \cite{C:AmbHamUnr19}, we will first consider the tuple  
%$(G, H, S, z)$ to be fixed, and that $\qA$ is a \emph{unitary} oracle
%algorithm. Then, we will generalize this for the case of $(G, H, S,
%z)$ chosen from an arbitrary distribution and $\qA$ being a general
%quantum algorithm. Let us now begin by describing the different elements
%in the execution of $\qB$:
%
%\begin{itemize}
%\item $\ket{\psi_0}$ denotes the initial state of $\qA$, which
%depends on $z$ but not on $G, H$ or $S$.  This is
%a state on registers $\qreg{A, Q, R, \widetilde{A}, \widetilde{Q},
%\widetilde{R}}$ which are all explained below.
%
%\item $O_H$ is a unitary implementing the oracle $H$, i.e., if
%performs the map $\ket{a, q, r}_{\qreg{A, Q, R}} \mapsto \ket{a, q, r
%\xor H(q)}_{\qreg{A, Q, R}}$ where $\qreg{A, Q}$ and $\qreg{R}$ are
%the ancilla, query and response registers of $\qA$ respectively
%(corresponding to queries made to $H$).
%
%\item $O_{\cQ}$ is a unitary implementing the oracle $\cQ$ that
%acts on the $\qreg{\widetilde{A}}$, $\qreg{\widetilde{Q}}$ and
%$\qreg{\widetilde{R}}$ registers, which are the ancilla, query and
%response registers of $\qA$ meant for oracle queries to $\cQ$. The
%possibly mixed state obtained on register $\qreg{\widetilde{R}}$ (i.e. the state
%obtained by tracing out the $\qreg{\widetilde{A}}$ register) is the desired
%output of the oracle.
%
%\item $U, U_{\cQ}$ are state transition unitaries that act on
%the state after every query to the oracles $O_H, O_{\cQ}$ respectively.
%
%\item $\ket{\psi_i^H}$ is the state after the $i$-th query to $O_H$ is
%made followed by the subsequent queries to $O_{\cQ}$. Consequently,
%it is related to the previous state $\ket{\psi_{i-1}^H}$ as follows,
%assuming some $t$ queries to $\cQ$ are performed.
%
%$$\ket{\psi_i^H} = \Big((U_{\cQ}O_{\cQ})^t \cdot U O_H
%\Big)\ket{\psi_{i-1}^H}$$
%\end{itemize}
%
%As in Lemma 8 of \cite{C:AmbHamUnr19}, let $P_{\guess} \seteq \Pr[S
%\cap T \neq \emptyset : T \gets B^{\cQ, H}(z)]$ where $T$ is the set
%of measurement outcomes (on register $\qreg{Q}$) obtained by $B$. Let
%$P_\guess = \sum_{i=1}^{q}\frac{1}{q} B_i$\ryo{$P_\guess$ was defined above, so ``We can write $P_\guess =\sum_{i=1}^{q}\frac{1}{q} B_i$''?} where $B_i \seteq \norm{P_S
%\ket{\psi_{i-1}^H}}^2$ and $P_S$ is an orthogonal projector such that
%$P_S \seteq \sum_{T : S \cap T \neq \emptyset} \ketbra{T}$.
%Intuitively, $B_i$ is the probability of obtaining $T$ such that $S
%\cap T \neq \emptyset$.
%
%
%Consider now the value $D_i \seteq \norm{\ket{\psi^H_i} -
%\ket{\psi^G_i}}^2$.\ryo{Better to say $\ket{\psi_i^G}$ defined analogously (at the definition of $\ket{\psi_i^H}$).} Note that $D_0  = \norm{\ket{\psi_0} -
%\ket{\psi_0}}^2 = 0$. For $i \ge 1$, $D_i = \norm{\widetilde{U} O_H
%\ket{\psi_{i-1}^H} - \widetilde{U}O_G\ket{\psi_{i-1}^G}}^2$, where
%$\widetilde{U} = (U_{\cQ}O_{\cQ})^tU$. From the fact that
%$\widetilde{U}$ is a unitary, it follows that $D_i =
%\norm{(O_H\ket{\psi_{i-1}^H} - O_G\ket{\psi_{i-1}^H}) +
%(O_G\ket{\psi_{i-1}^H} - O_G\ket{\psi_{i-1}^G})}^2$. We observe that
%the rest of the analysis in \cite{C:AmbHamUnr19} does not depend on
%the unitary $\widetilde{U}$, and hence it follows that:
%
%$$\sqrt{D_i} \le \sqrt{D_{i-1}} + 2\sqrt{B_i}$$
%
%Consequently, Lemma 8 of their work gives us that:
%
%$$\norm{\ket{\psi_{q}^H} - \ket{\psi_q^G}} \le 2q\sqrt{P_\guess}$$
%
%Let us now denote the above states of the form $\ket{\psi^H_i}$ as
%$\ket{\psi^{H,S,z}_{i}}$ corresponding to the fixed $(G, H, S, z)$.
%Consider now the following mixed states:
%
%$$\rho_q^H \seteq \underset{G, H, S, z}{\mathbb{E}}[\ketbra{\psi_q^{H,S,z}}]$$
%$$\rho_q^G \seteq \underset{G, H, S, z}{\mathbb{E}}[\ketbra{\psi_q^{G,S,z}}]$$
%
%Our goal is to lower bound $P_\guess \seteq\mathbb{E}_{G,H,S,z}\Pr[P_\guess^{H,S,z}]$\ryo{Is this a definition of $P_\guess$? What is $P_\guess^{H,S,z}$?}. Note that in order to
%consider general quantum algorithms $\qA$, we will assume $\qA$
%performs a measurement to the state $\rho_q^H$ (or $\rho_q^G$)
%denoted by the POVM $\cE$. Let the resulting states be denoted as
%$\widetilde{\rho_q}^H$ and $\widetilde{\rho_q}^G$. We now observe that
%the analysis of Lemma 9 of \cite{C:AmbHamUnr19} generalizes
%directly, giving us the following inequality regarding the Fidelity
%between the final states:
%
%$$F(\widetilde{\rho_q}^H, \widetilde{\rho_q}^G) \ge 1 - 2q^2P_\guess$$
%
%Therefore, Theorem 3 of \cite{C:AmbHamUnr19} gives us:
%
%\begin{align}
%\abs{\Pr[\qA^{\cQ, H}(z)=1]-\Pr[\qA^{\cQ, G}(z)=1]} \leq
%2q\cdot\sqrt{P_\guess}
%\enspace.
%\end{align}
%
%\end{proof}

\begin{remark}
We assume that $\qB$ also outputs the measured index $i$. However,
this output is not taken into account for notation such as $\qB^{\cQ,
H}(z) \in S$ for the sake of simplicity.
\end{remark}


\subsection{Standard Cryptographic Tools}\label{sec:standard_crypto}


\subsubsection*{Attribute-Based Encryption.}

\begin{definition}[Attribute-Based Encryption]\label{def:ABE}
An ABE scheme $\ABE$ is a tuple of four PPT algorithms $(\Setup, \KG,
\Enc,\allowbreak \Dec)$. 
Below, let $\cX=\{\cX_\secp\}_\secp$, $\cY=\{\cY_\secp\}_\secp$, and $R=\{R_\secp:\cX_\secp \times \cY_\secp \ra \bin \}_\secp$ be the ciphertext attribute space, key attribute space, and the relation associated with $\ABE$, respectively.
We note that we will abuse the notation and occasionally drop the subscript for these spaces for notational simplicity.
We also note that the message space is set to be $\bin^\ell$ below. 
\begin{description}
\item[$\Setup(1^\secp)\ra(\pk,\msk)$:] The setup algorithm takes a security parameter $1^\secp$ and outputs a public key $\pk$ and master secret key $\msk$.

\item[$\KG(\msk,y, r)\ra\sk_y$:] The key generation algorithm $\KG$
takes a master secret key $\msk$, a key attribute $y \in
\cY$, and explicit randomness $r$. It outputs a decryption key
$\sk_y$.
Note that $\KG$ is deterministic.\footnote{In the standard syntax, $\KG$ does not take explicit randomness, and is probabilistic. This change is just for notational convention in our schemes.}

\item[$\Enc(\pk,x,\msg)\ra\ct$:] The encryption algorithm takes a
public key $\pk$, a ciphertext attribute $x \in \cX$, and a
message $\msg$, and outputs a ciphertext $\ct$.

\item[$\Dec(\sk_y,\ct)\ra z$:] The decryption algorithm takes a
secret key $\sk_y$ and a ciphertext $\ct$ and outputs
$z \in \{ \bot \} \cup \bin^\ell$.
%\nikhil{I removed the requirement of $\Dec$ needing $x$ separately.}

\item[Correctness:] We require that
\[
\Pr\left[
\Dec(\sk_y, \ct) = \msg
 \ :
\begin{array}{rl}
 &(\pk,\msk) \la \Setup(1^\secp),\\
 &r \leftarrow \bit^{\poly(\secp)},\\
 & \sk_y \gets \KG(\msk,y, r), \\
 &\ct \gets \Enc(\pk,x,\msg)
\end{array}
\right] \ge 1 -\negl(\secp).
\]
holds for all $x\in \cX$ and $y\in \cY$ such that $R(x,y)=0$ and $m\in \bin^\ell$.
\end{description}
\end{definition}

By setting $\cX$, $\cY$, and $R$ appropriately, we can recover
important classes of ABE. In particular, if we set
$\cX_\secp=\cY_\secp=\bin^*$ and define $R_\secp$ so that
$R_\secp(x,y)=0$ if $x=y$ and $R_\secp(x,y)=1$ otherwise, we recover
the definition of identity-based encryption (IBE). 
If we set $\cX_\secp=\bin^{n(\secp)}$ and $\cY_\secp$ to be the set of all circuits with input space $\bin^{n(\secp)}$ and size at most $s(\secp)$, where $n$ and $s$ are some polynomials, and define $R$ so that $R(x,y)=y(x)$, we recover the definition of (key policy) ABE for circuits. 

We introduce a new security notion for ABE that we call quantum selective-security for ABE where the adversary is allowed to get access to the key generation oracle in super-position.


\begin{definition}[Quantum Selective-Security for ABE]\label{def:qsel_ind_ABE}
We say that $\ABE$ is a \emph{selective-secure} ABE scheme for
relation $R:\cX\times \cY \to \bin$, if it satisfies the following
requirement, formalized by the experiment
$\expc{\ABE, \qA}{q}{sel}{ind}(1^\secp,\coin)$ between an adversary
$\qA$ and a challenger $\qCh$:
        \begin{enumerate}
            \item $\qA$ declares the challenge ciphertext attribute
                $x^*$. $\qCh$ runs $(\pk,\msk)\gets\Setup(1^\secp)$ and sends $\pk$ to $\qA$.
            \item $\qA$ can get access to the following quantum key generation oracle.
            \begin{description}

    \item[$\Oracle{qkg}(\qreg{Y},\qreg{Z})$:] Given two registers
        $\qreg{Y}$ and $\qreg{Z}$, it first applies the map
        $\ket{y}_{\qreg{Y}}\ket{b}_{\qreg{B}}\ra\ket{y}_{\qreg{Y}}\ket{b\oplus
        R(x^*,y)}_{\qreg{B}}$ and measures the register $\qreg{B}$, where $\qreg{B}$ is initialized to $\ket{0}_\qreg{B}$.
    If the result is $0$, it returns $\bot$. Otherwise, it
    chooses $r \leftarrow \bit^{\poly(\secp)}$, applies
    the map
    $\ket{y}_{\qreg{Y}}\ket{z}_{\qreg{Z}}\ra\ket{y}_{\qreg{Y}}\ket{z\oplus
    \KG(\msk,y,r)}_{\qreg{Z}}$ and returns the registers $\qreg{Y}$
    and $\qreg{Z}$.
    \end{description}

\item At some point, $\qA$ sends $(\msg_0,\msg_1)$ to $\qCh$. Then, $\qCh$
generates $\ct^*\gets\Enc(\pk,x^*,\allowbreak\msg_\coin)$ and sends $\ct^*$ to
$\qA$.

\item Again, $\qA$ can get access to the oracle $\Oracle{qkg}$.
\item $\qA$ outputs a guess $\coin^\prime$ for $\coin$ and the
experiment outputs $\coin'$.
\end{enumerate}
We say that $\ABE$ satisfies quantum selective security if, for all
QPT $\qA$, it holds that

\begin{align}
\advc{\ABE,\qA}{q}{sel}{ind}(1^\secp) \seteq \abs{\Pr[\expc{\ABE,\qA}{q}{sel}{ind} (1^\secp,0) \ra 1] - \Pr[\expc{\ABE,\qA}{q}{sel}{ind} (1^\secp,1) \ra 1] }\\\leq \negl(\secp).
\end{align}
\end{definition}

Boneh and Zhandry~\cite{C:BonZha13} introduced a similar quantum security notion for IBE and argued that it is straightforward to prove the quantum security of the IBE scheme by \cite{EC:AgrBonBoy10}, by leveraging the lattice trapdoor based proof technique.
It is easy to prove the quantum selective security of the ABE scheme for circuits by Boneh et al.~\cite{EC:BGGHNS14}, which relies on the lattice trapdoor based proof technique as well.
Formally, we have the following theorem.
\begin{theorem}
Assuming the hardness of the LWE problem, there exists a quantum selectively secure ABE scheme for all relations computable in polynomial time.
\end{theorem}
We elaborate on this in \cref{sec-quantum-secure-ABE}.





\ifnum\submission=1
\else
\paragraph{Compute-and-Compare Obfuscation.}
We define a class of circuits called compute-and-compare circuits as
follows:

\begin{definition}[Compute-and-Compare Circuits]\label{def:cc_circuits_searchability}
A compute-and-compare circuit $\cnc{P}{\lock,\msg}$ is of the form
\[
\cnc{P}{\lock,\msg}(x)\left\{
\begin{array}{ll}
    \msg&\textrm{if}\; P(x)=\lock\\
\bot&\text{otherwise}~
\end{array}
\right.
\]
where $P$ is a circuit, $\lock$ is a string called the lock value,
and $\msg$ is a message.
\end{definition}

We now introduce the definition of compute-and-compare obfuscation.
We assume that a program $P$ has an associated set of parameters $\pp_P$ (input size, output size, circuit size) which we do not need to hide.
\begin{definition}[Compute-and-Compare Obfuscation]\label{def:CCObf}
A PPT algorithm $\CCObf$ is an obfuscator for the family of distributions $D=\{D_\secp\}$ if the following holds:
\begin{description}
\item[Functionality Preserving:] There exists a negligible function
$\negl$ such that for all programs $P$, all lock values $\lock$, and
all messages $\msg$, it holds that

\begin{align}
\Pr[\forall x, \tlP(x)=\cnc{P}{\lock,\msg}(x) :
\tlP\la\CCObf(1^\secp,P,\lock,\msg)] \ge 1-\negl(\secp).
\end{align}
\item[Distributional Indistinguishability:] There exists an
efficient simulator $\Sim$ such that for all messages $\msg$, we have
\begin{align}
(\CCObf(1^\secp,P,\lock,\msg),\qaux)\approx(\CCSim(1^\secp,\pp_P,\abs{\msg}),\qaux),
\end{align}
where $(P,\lock,\qaux)\la D_\secp$.
\end{description}
\end{definition}

\begin{theorem}[\cite{FOCS:GoyKopWat17,FOCS:WicZir17}]
If the LWE assumption holds, there exists compute-and-compare obfuscation for all families of distributions $D=\{D_\secp\}$, where each $D_\secp$ outputs uniformly random lock value $\lock$ independent of $P$ and $\qaux$.
\end{theorem}
\fi
 
We present the definitions for SKE with certified deletion.
\begin{definition}[SKE-CD (Syntax)]\label{def:SKE-CD}
An SKE-CD scheme is a tuple of algorithms $(\KG,\qEnc,\qDec,\qDel,\Vrfy)$ with plaintext space $\Ms$ and key space $\Ks$.
\begin{description}
    \item[$\KG (1^\secp) \ra \sk$:] The key generation algorithm takes as input the security parameter $1^\secp$ and outputs a secret key $\sk \in \Ks$.
    \item[$\qEnc(\sk,\msg) \ra (\qct,\vk)$:] The encryption algorithm
    takes as input $\sk$ and a plaintext $\msg\in\Ms$ and outputs a
    ciphertext $\qct$ and a verification key $\vk$.

    \item[$\qDec(\sk,\qct) \ra \msg^\prime$:] The decryption algorithm
        takes as input $\sk$ and $\qct$ and outputs a plaintext $\msg^\prime \in \Ms$ or $\bot$.
    \item[$\qDel(\qct) \ra \cert$:] The deletion algorithm takes as
        input $\qct$ and outputs a certificate $\cert$.
    \item[$\Vrfy(\vk,\cert)\ra \top/\bot$:] The verification
        algorithm takes $\vk$ and $\cert$ as input and outputs
        $\top$ or $\bot$.

\item[Decryption correctness:] There exists a negligible function
    $\negl$ such that for any $\msg\in\Ms$, 
\begin{align}
\Pr\left[
\qDec(\sk,\qct)= \msg
\ :
\begin{array}{ll}
\sk\lrun \KG(1^\secp)\\
(\qct, \vk) \lrun \qEnc(\sk,\msg)
\end{array}
\right] 
\ge 1-\negl(\secp).
\end{align}

\item[Verification correctness:] There exists a negligible function
    $\negl$ such that for any $\msg\in\Ms$, 
\begin{align}
\Pr\left[
\Vrfy(\vk,\cert)=\top
\ :
\begin{array}{ll}
\sk\lrun \KG(1^\secp)\\
(\qct, \vk) \lrun \qEnc(\sk,\msg)\\
\cert \lrun \qDel(\qct)
\end{array}
\right] 
\ge 1-\negl(\secp).
\end{align}
\end{description}
\end{definition}


We introduce indistinguishability against Chosen Verification Attacks (CVA).
\begin{definition}[IND-CVA-CD Security]\label{def:reusable_sk-vo_certified_del}
We consider the following security experiment
$\expc{\CDSKE,\qA}{ind}{cva}{cd}(1^\secp,\coin)$.

\begin{enumerate}
    \item The challenger computes $\sk \la \KG(1^\secp)$.
    \item Thoughout the experiment, $\qA$ can get access to the following oracle.
    \begin{description}
    \item[$\Oracle{\qEnc}(\msg)$:] On input $\msg$, it generates
    $(\qct, \vk)\gets\qEnc(\sk,\msg)$ and returns $(\vk,\qct)$.  
    \end{description}
    \item $\qA$ sends $(\msg_0,\msg_1)\in\cM^2$ to the challenger. 
    %Also, $\qA$ may output $\bot$ instead of the challenge tuple and obtain $\sk$ at this stage. In that case, the experiment goes to Item $9$.
    \item The challenger computes $(\qct^*,\vk^*) \la
        \qEnc(\sk,\msg_\coin)$ and sends $\qct^*$ to $\qA$.
    \item Hereafter, $\qA$ can get access to the following oracle, where $V$ is initialized to $\bot$.
    \begin{description}
        \item[$\Oracle{\Vrfy}(\cert)$:] On input $\cert$, it returns $\sk$ and updates $V$ to $\top$ if $\Vrfy(\vk^*,\cert)=\top$. Otherwise, it returns $\bot$.
    \end{description}
    \item When $\qA$ outputs $\coin'\in \bit$, the experiment outputs $\coin^\prime$ if $V=\top$ and otherwise outputs $0$.
\end{enumerate}
We say that $\CDSKE$ is IND-CVA-CD secure if for any QPT $\qA$, it holds that
\begin{align}
\advc{\CDSKE,\qA}{ind}{cva}{cd}(1^\secp)\seteq \abs{\Pr[
\expc{\CDSKE,\qA}{ind}{cva}{cd}(1^\secp, 0)=1] - \Pr[
\expc{\CDSKE,\qA}{ind}{cva}{cd}(1^\secp, 1)=1] }\\\le \negl(\secp).
\end{align}
\end{definition}

\begin{definition}[BB84-Based SKE-CD]\label{def:bb84}
We say that an SKE-CD scheme $\CDSKE=(\KG,\qEnc,\qDec,\qDel,\Vrfy)$
is a BB84-based SKE-CD scheme if it satisfies the following
conditions.

\begin{itemize}
   \item Let $(\qct,\vk)\gets\qEnc(\sk,\msg)$. $\vk$ is of the form
    $(x,\theta)\in\bit^{\ctlen}\times\bit^{\ctlen}$,
    and $\qct$ is of the form $\ket{\psi_1}\tensor
    \cdots \tensor\ket{\psi_{\ctlen}}$, where
    \begin{align}
    \ket{\psi_i}=
    \begin{cases}
        \ket{x[i]} & if~~ \theta[i]=0\\
        \ket{0}+(-1)^{x[i]}\ket{1} & if~~ \theta[i]=1.
    \end{cases}
    \end{align}
    Moreover, there exists $n<\ctlen$ such that $\theta[i]=0$ for every $i\in[n+1,\ctlen]$. We call $x[n+1]\|\cdots\|x[\ctlen]$ a classical part of $\qct$. The parameter $n$ is specified by a construction. The classical part has information of $\theta$, and we can compute $\theta$ from it and $\sk$.
    %\fuyuki{The final condition is needed to prove that IND-CD security implies IND-CVA-CD security %for BB84 based schemes.}

    
        \item $\qDel(\qct)$ measures each qubit of $\qct$ in the
    Hadamard basis and outputs the measurement result
    $\cert\in\bit^{\ctlen}$.

    
    \item $\Vrfy(\vk,\cert)$ outputs $\top$ if $\cert[i]=x[i]$ holds
    for every $i\in[n]$ such that $\theta[i]=1$, and $0$
    otherwise.

\item \textbf{Classical Decryption Property:} There exists an
additional deterministic polynomial time algorithm $\CDec$ with the
following property. Let $(\qct,\vk)\gets\qEnc(\sk,\msg)$, where $\vk=(x,\theta)$. Let $u\in\bit^{\ctlen}$ be any string such that $u[i] =x[i]$ for all $i : \theta[i] = 0$. Then, the following holds:

$$\Pr\Big[\CDec\big(\sk, u\big)= \msg\Big] \ge 1 - \negl(\secp)$$ 

\end{itemize}
\end{definition}

\begin{theorem}\label{thm:SKECD-BB84}
There exists a BB84-based SKE-CD scheme satisfying IND-CVA-CD
security, assuming the existence of a CPA-secure Secret-Key
Encryption scheme.
\end{theorem}



Kitagawa and Nishimaki~\cite{AC:KitNis22} claimed the same statement as~\cref{thm:SKECD-BB84}.
However, their proof has a gap because known BB84-based SKE-CD schemes do not satisfy the unique certificate property, which they introduced. Hence, we prove \cref{thm:SKECD-BB84} in \cref{sec:SKECD-BB84}.
%\ryo{I added explanation about the issue in \cite{AC:KitNis22}.}


%We introduce non-malleability for SKE-CD.
%\begin{definition}[NM-CD Security]\label{def:NMCD-CDSKE}
%We consider the following security experiment $\expb{\CDSKE,\qA}{nm}{cd}(\secp)$.
%
%\begin{enumerate}
%    \item The challenger computes $\sk \la \KG(1^\secp)$.
%    \item $\qA$ can get access to the following oracle.
%    \begin{description}
%    \item[$\Oracle{\qEnc}(\msg)$] On input $\msg$, it generates $(\vk,\qct)\gets\qEnc(\sk,\msg)$ and returns $(\vk,\qct)$. Let $\cQ$ be the set of the inputs received from $\qA$. 
%    \end{description}
%    \item $\qA$ outputs $\qct^\prime$. The challenger computes $\msg^\prime\gets\qDec(\sk,\qct^\prime)$.
%    The challenger outputs $1$ if $\msg^\prime\notin\cQ\cup\{\bot\}$ and otherwise outputs $0$.
%\end{enumerate}
%We say that $\CDSKE$ satisfies NM-CD security if for any QPT $\qA$, it holds that
%\begin{align}
%\advb{\CDSKE,\qA}{nm}{cd}(\secp)\seteq\Pr[ \expb{\CDSKE,\qA}{nm}{cd}(\secp)=1]= \negl(\secp).
%\end{align}
%\end{definition}






