% !TEX root = main.tex


\section{Secret-Key Encryption with Collusion-Resistant SKL}

\subsection{Definitions of SKE-CR-SKL}
The syntax of SKE-CR-SKL is defined as follows. 
\begin{definition}[SKE-CR-SKL]
    An SKE-CR-SKL scheme $\SKECRSKL$ is a tuple of five algorithms
$(\Setup,\qKG, \Enc, \qDec, \qVrfy)$. 
Below, let $\cM$  be the message space of $\SKECRSKL$. 
\begin{description}
%\item[$\Setup(1^\secp,1^{\numkey})\ra\msk$:] The setup algorithm takes a security parameter $1^\lambda$ and a collusion bound $1^{\numkey}$, and outputs a master secret key $\msk$.
\item[$\Setup(1^\secp)\ra\msk$:] The setup algorithm takes a
security parameter $1^\lambda$ and outputs a master secret-key
$\msk$.

\item[$\qKG(\msk)\ra(\qdk,\vk,\tk)$:] The key generation algorithm
takes the master secret-key $\msk$ as input. It outputs a decryption
key $\qdk$, a verification key $\vk$, and a testing key
$\tk$.

\item[$\Enc(\msk,\msg)\ra\ct$:] The encryption algorithm takes the
master secret-key $\msk$ and a message $\msg \in \cM$, and outputs a
ciphertext $\ct$.

\item[$\qDec(\qdk,\ct)\ra\widetilde{\msg}$:] The decryption
algorithm takes a decryption key $\qdk$ and a ciphertext $\ct$,
and outputs a value $\widetilde{\msg}$.

%\item[$\qCert(\qfsk)\ra\cert$:] The certification algorithm takes a function decryption key $\qfsk$, and outputs a classical string $\cert$.

\item[$\qVrfy(\vk, \widetilde{\qdk})\ra\top/\bot$:] The verification
    algorithm takes a verification key $\vk$ and a (possibly
    malformed) decryption key $\widetilde{\qdk}$,
and outputs $\top$ or $\bot$.

%\item[$\KeyTest(\tk,\dk)\ra\top/\bot$:] The key testing algorithm takes a key-testing key $\tk$ and a classical decryption key $\dk$, and outputs $\top$ or $\bot$.

\item[Decryption Correctness:] For all $\msg \in \cM$, we have
\begin{align}
\Pr\left[
\qDec(\qdk, \ct) \allowbreak = \msg
\ :
\begin{array}{ll}
\msk\gets\Setup(1^\secp)\\
(\qdk,\vk,\tk)\gets\qKG(\msk)\\
\ct\gets\Enc(\msk,\msg)
\end{array}
\right] 
\ge 1-\negl(\secp).
\end{align}

\item[Verification Correctness:] We have 
\begin{align}
\Pr\left[
\Vrfy(\vk,\qdk)=\top
\ :
\begin{array}{ll}
\msk\gets\Setup(1^\secp)\\
(\qdk,\vk,\tk)\gets\qKG(\msk)\\
\end{array}
\right] 
\ge 1-\negl(\secp).
\end{align}

%\item[Decryption Robustness:]For any $\msg\in\cM$, we have 
%\begin{align}
%\Exp_{
%\begin{scriptsize}
%\begin{array}{lll}
%\msk\gets\Setup(1^\secp)\\
%(\qdk,\vk,\tk)\gets\qKG(\msk)\\
%\ct\gets\Enc(\msk,\msg)
%\end{array}
%\end{scriptsize}
%}
%\left[
%\Pr\left[
%\exists \dk \textrm{~~such that~~}
%\begin{array}{ll}
%\KeyTest(\tk,\dk)=\top\\
%\CDec(\dk,\ct)\ne \msg
%\end{array}
%\right]
%\right]
%=\negl(\secp).
%\end{align}

\end{description}
\end{definition}


\begin{definition}[Classical Decryption Property]
We say that $\SKECRSKL=(\Setup,\qKG,\Enc,\qDec,\qDel, \Vrfy)$ has
the classical decryption property if there exists a deterministic
polynomial time algorithm $\CDec$ such that given $\qdk$ in the
register $\qreg{DK}$ and ciphertext $\ct$, $\qDec$ applies the map
$\ket{u}_{\qreg{DK}}\ket{v}_{\qreg{MSG}}\ra\ket{u}_{\qreg{DK}}\ket{v\oplus\CDec(u,\ct)}_{\qreg{MSG}}$
and outputs the measurement result of the register $\qreg{MSG}$ in
the computational basis, where $\qreg{MSG}$ is initialized to
$\ket{0\cdots0}_{\qreg{MSG}}$.  \end{definition}

\begin{definition}[Key Testability]
We say that an SKE-CR-SKL scheme $\SKECRSKL$ with the classical
decryption property satisfies key testability, if there exists a
classical deterministic algorithm $\KeyTest$ that satisfies the
following conditions:

\begin{itemize}
\item \textbf{Syntax:} $\KeyTest$ takes as input a testing key $\tk$
and a classical string $\dk$ as input. It outputs $0$ or $1$.

\item \textbf{Correctness:} Let $\msk\gets\Setup(1^\secp)$ and
$(\qdk,\vk,\tk)\gets\qKG(\msk)$. We denote the register holding
$\qdk$ as $\qreg{DK}$. Let $\qreg{KT}$ be a register that is
initialized to $\ket{0}_{\qreg{KT}}$. If we apply the map
$\ket{u}_{\qreg{DK}}\ket{\beta}_{\qreg{KT}}\ra\ket{u}_{\qreg{DK}}\ket{\beta\oplus\KeyTest(\tk,u)}_{\qreg{KT}}$
to the registers $\qreg{DK}$ and $\qreg{KT}$ and then measure
$\qreg{KT}$ in the computational basis, we obtain $1$ with
overwhelming probability.

\item \textbf{Security:} Consider the following experiment
$\expb{\SKECRSKL,\qA}{key}{test}(1^\secp)$.

\begin{enumerate}
\item The challenger $\qCh$ runs $\msk\gets\Setup(1^\secp)$ and
initializes $\qA$ with input $\msk$. 

\item $\qA$ requests $q$ decryption keys for some polynomial $q$.
$\qCh$ generates $(\qdk_i,\vk_i,\tk_i)\gets\qKG(\msk)$
for every $i\in[q]$ and sends $(\qdk_i,\vk_i,\tk_i)_{i\in[q]}$ to
$\qA$.

\item $\qA$ sends $(k, \dk, \msg)$ to $\qCh$, where $k$ is
an index, $\dk$ is a classical string and $\msg$ is a message.
$\qCh$ generates $\ct\gets\Enc(\msk,\msg)$. $\qCh$ outputs $\top$ if
$\KeyTest(\tk_k,\dk)=1$ and $\CDec(\dk,\ct)\ne\msg$.
Otherwise, $\qCh$ outputs $\bot$.
\end{enumerate}

For all QPT $\qA$, the following must hold:

\begin{align}
\advb{\SKECRSKL,\qA}{key}{test}(1^\secp) \seteq
\Pr[\expb{\SKECRSKL,\qA}{key}{test}(1^\secp) \ra \top] \le
\negl(\secp).
\end{align} 
\end{itemize}
\end{definition}

\begin{definition}[OT-IND-KLA Security]\label{def:OT-IND-CPA_SKECRSKL}
%\takashi{Renamed from lessor security to IND-CPA security. (Alternatively, should we call it something like IND-KLA (Key Leasing Attack)?)}
We say that an SKE-CR-SKL scheme with key testability $\SKECRSKL$
is OT-IND-KLA secure, if it satisfies
the following requirement, formalized by the experiment
$\expc{\SKECRSKL,\qA}{ot}{ind}{kla}(1^\secp,\coin)$ between an
adversary $\qA$ and a challenger $\qCh$:

\begin{enumerate}
\item $\qCh$ runs $\msk\gets\Setup(1^\secp)$ and initializes
$\qA$ with the security parameter $1^\secp$.

\item $\qA$ requests $q$ decryption keys for some polynomial $q$.
The challenger generates $(\qdk_i,\vk_i,\tk_i)\gets\qKG(\msk)$
for every $i\in[q]$ and sends $(\qdk_i,\tk_i)_{i\in[q]}$ to $\qA$.

\item $\qA$ can get access to the following (stateful) verification
oracle $\Oracle{\qVrfy}$ where $V_i$ is initialized to be $\bot$:

\begin{description}
\item[ $\Oracle{\qVrfy}(i, \widetilde{\qdk})$:] It runs $d \gets
\Vrfy(\vk_i, \widetilde{\qdk})$ and returns $d$.  

If $V_i=\bot$ and $d=\top$, it updates $V_i\seteq \top$. 
               %\takashi{changed the way of explaining update to better match that for ABE/FE.}
               %and updates $V\seteq \returned$ if $d=\top$. 
\end{description}
\item $\qA$ sends $(\msg_0^*,\msg_1^*)\in \cM^2$ to the challenger.
If $V_i=\bot$ for some $i\in[q]$, $\qCh$ outputs
$0$ as the final output of this experiment. Otherwise, $\qCh$
generates $\ct^*\la\Enc(\msk,\msg_\coin^*)$ and sends
$\ct^*$ to $\qA$.

\item $\qA$ outputs a guess $\coin^\prime$ for $\coin$.
$\qCh$ outputs $\coin'$ as the final output of the
experiment.
\end{enumerate}

For all QPT $\qA$, it holds that:
\begin{align}
\advc{\SKECRSKL,\qA}{ot}{ind}{kla}(1^\secp) \seteq \abs{\Pr[\expc{\SKECRSKL,\qA}{ot}{ind}{kla} (1^\secp,0) \ra 1] - \Pr[\expc{\SKECRSKL,\qA}{ot}{ind}{kla} (1^\secp,1) \ra 1] }\leq \negl(\secp).
\end{align} 
\end{definition}

\subsection{Definitions of
SKE-CR\textsuperscript{2}-SKL}\label{def:ske-cr2}
The syntax of SKE-CR\textsuperscript{2}-SKL is defined as follows. 
\begin{definition}[SKE-CR\textsuperscript{2}-SKL]
An SKE-CR\textsuperscript{2}-SKL scheme $\SKECRCRSKL$ is a tuple of
six algorithms $(\Setup,\qKG, \Enc, \qDec, \allowbreak \qDel, \Vrfy)$. 
Below, let $\cM$  be the message space of $\SKECRCRSKL$. 
\begin{description}
\item[$\Setup(1^\secp)\ra\msk$:] The setup algorithm takes a
security parameter $1^\lambda$ and outputs a master secret-key
$\msk$.

\item[$\qKG(\msk)\ra(\qdk,\vk)$:] The key generation algorithm
takes the master secret-key $\msk$ as input. It outputs a decryption
key $\qdk$ and a certificate verification key $\vk$.

\item[$\Enc(\msk,\msg)\ra\ct$:] The encryption algorithm takes the
master secret-key $\msk$ and a message $\msg \in \cM$, and outputs a
ciphertext $\ct$.

\item[$\qDec(\qdk,\ct)\ra\widetilde{\msg}$:] The decryption
algorithm takes a decryption key $\qdk$ and a ciphertext $\ct$,
and outputs a value $\widetilde{\msg}$.

%\item[$\qCert(\qfsk)\ra\cert$:] The certification algorithm takes a function decryption key $\qfsk$, and outputs a classical string $\cert$.

\item[$\qDel(\qdk)\ra\cert$:] The decryption
algorithm takes a decryption key $\qdk$ and outputs a
certificate $\cert$.

\item[$\Vrfy(\vk, \cert')\ra\top/\bot$:] The verification
algorithm takes a verification key $\vk$ and a certificate $\cert'$
, and outputs $\top$ or $\bot$.

\item[Decryption Correctness:] For all $\msg \in \cM$, we have
\begin{align}
\Pr\left[
\qDec(\qdk, \ct) \allowbreak = \msg
\ :
\begin{array}{ll}
\msk\gets\Setup(1^\secp)\\
(\qdk,\vk)\gets\qKG(\msk)\\
\ct\gets\Enc(\msk,\msg)
\end{array}
\right] 
\ge 1-\negl(\secp).
\end{align}

\item[Verification Correctness:] We have 
\begin{align}
\Pr\left[
\Vrfy(\vk,\cert)=\top
\ :
\begin{array}{ll}
\msk\gets\Setup(1^\secp)\\
(\qdk,\vk)\gets\qKG(\msk)\\
\cert \gets\qDel(\qdk)
\end{array}
\right] 
\ge 1-\negl(\secp).
\end{align}

%\item[Decryption Robustness:]For any $\msg\in\cM$, we have 
%\begin{align}
%\Exp_{
%\begin{scriptsize}
%\begin{array}{lll}
%\msk\gets\Setup(1^\secp)\\
%(\qdk,\vk,\tk)\gets\qKG(\msk)\\
%\ct\gets\Enc(\msk,\msg)
%\end{array}
%\end{scriptsize}
%}
%\left[
%\Pr\left[
%\exists \dk \textrm{~~such that~~}
%\begin{array}{ll}
%\KeyTest(\tk,\dk)=\top\\
%\CDec(\dk,\ct)\ne \msg
%\end{array}
%\right]
%\right]
%=\negl(\secp).
%\end{align}

\end{description}
\end{definition}


\begin{description}
\item[IND-KLA Security:] The IND-KLA Security of an
SKE-CR\textsuperscript{2}-SKL scheme is defined in the same way as
the OT-IND-KLA Security (Definition \ref{def:OT-IND-CPA_SKECRSKL})
of an SKE-CR-SKL scheme, except that throughout the experiment,
$\qA$ receives access to the following oracle:

\begin{description}
\item[ $\Oracle{\Enc}(\msg)$:] It runs $\ct \gets \Enc(\msk,
\msg)$ and returns $\ct$.  

\end{description}

Furthermore, the access to the oracle $\Oracle{\qVrfy}$ is replaced
with access to the following oracle:
\begin{description}
\item[ $\Oracle{\Vrfy}(i, \textcolor{black}{\cert})$:] It runs $d
\gets \Vrfy(\vk_i,\textcolor{black}{\cert})$ and returns $d$.  

If $V_i=\bot$ and $d=\top$, it updates $V_i\seteq \top$. 
\end{description}
\end{description}