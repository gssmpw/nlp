% !TEX root = main.tex

\ifnum\submission=1
\section{More on Related Work} 
\label{sec:other-rel}
\else
\fi

\paragraph{Certified deletion.}
Broadbent and Islam~\cite{TCC:BroIsl20} introduced encryption with certified deletion, where we can generate classical certificates to guarantee that \emph{ciphertexts} were deleted. Subsequent works improved Broadbent and Islam's work and achieved advanced encryption with certified deletion~\cite{AC:HMNY21,ITCS:Poremba23,C:BarKhu23,EC:HKMNPY24,EC:BGKMRR24} and publicly verifiable deletion~\cite{AC:HMNY21,EC:BGKMRR24,TCC:KitNisYam23,TCC:BKMPW23}.
Compute-and-compare obfuscation with certified deletion introduced by Hiroka et al.~\cite{EC:HKMNPY24} is essentially the same as secure software leasing in the strong security model.

\paragraph{Single decryptor encryption.}
Georgiou and Zhandry~\cite{EPRINT:GeoZha20} introduced the notion of SDE. They constructed a public-key SDE scheme from one-shot signatures~\cite{STOC:AGKZ20} and extractable witness encryption with quantum auxiliary information~\cite{STOC:GGSW13,C:GKPVZ13}. Coladangelo, Liu, Liu, and Zhandry~\cite{C:CLLZ21} constructed a public-key SDE scheme from IO~\cite{JACM:BGIRSVY12} and extractable witness encryption or from subexponentially secure IO, subexponentially secure OWF, and LWE by combining the results by Culf and Vidick~\cite{Quantum:CulVid22}. Kitagawa and Nishimaki~\cite{AC:KitNis22} introduced the notion of single-decryptor functional encryption (SDFE), where each functional decryption key is copy-protected and constructed single decryptor PKFE for $\Ppoly$ from the subexponential hardness of IO and LWE.
These works consider the setting where the adversary receives only one copy-protected decryption key.
Liu, Liu, Qian, and Zhandry~\cite{TCC:LLQZ22} study SDE in the collusion-resistant setting, where the adversary receives multiple copy-protected decryption keys.
They constructed a public-key SDE scheme with bounded collusion-resistant copy-protected keys from subexponentially secure IO and subexponentially secure LWE.

\paragraph{Multi-copy revocable encryption.}
Ananth, Mutreja, and Poremba~\cite{myEPRINT:AnaMutPor24} introduced multi-copy revocable encryption.
This notion considers the setting where we can revoke \emph{ciphertexts} (not decryption keys), and the adversary receives multiple copies of the target \emph{ciphertext} (they are pure states). Hence, this notion is different from secure key leasing (or key-revocable cryptography).

