% !TEX root = main.tex



\section{Introduction}\label{sec:intro}

\paragraph{Secure key leasing.}
Encryption with secure key leasing (SKL) enables a secret key holder to generate a quantum decryption key and lease it securely to another party.
Once the lessee returns the quantum decryption key, they lose their ability to decrypt ciphertexts.
Since its introduction by Kitagawa and Nishimaki for secret key functional encryption (SKFE)~\cite{AC:KitNis22}, SKL has been extensively studied~\cite{EC:AKNYY23,TCC:AnaPorVai23,EPRINT:CGJL23,EPRINT:MorPorYam23,TCC:AnaHuHua24,EC:BGKMRR24,myEPRINT:KitMorYam24} due to its strong security guarantee and practical applications.

\paragraph{Collusion-resistant SKL.}
Most prior works~\cite{EC:AKNYY23,TCC:AnaPorVai23,EPRINT:CGJL23,EPRINT:MorPorYam23,TCC:AnaHuHua24} study how to achieve public-key encryption (PKE) with SKL schemes from standard cryptographic assumptions. All prior works on PKE-SKL have explored the setting where an adversary can obtain \emph{only one} quantum decryption key.
However, this single-key security model does not accurately reflect real-world scenarios. In practice, once a lessee returns their decryption key and it is verified (i.e., after revocation), the lessor may lease another decryption key, even to the same lessee. Moreover, in realistic settings, a single secret key holder may need to generate and lease \emph{multiple} quantum decryption keys to various entities. To accurately capture this setting, we consider adversaries capable of obtaining \emph{an unbounded number of quantum decryption keys}, even in the standard PKE setting.
We define this model as \emph{collusion-resistant} SKL.

Previous works~\cite{TCC:AnaPorVai23,TCC:AnaHuHua24} presented delegation tasks as a compelling application like the following:
Consider a scenario where a system administrator unexpectedly needs to take leave and must temporarily assign their responsibilities---including access to encrypted data---to a colleague by providing decryption keys.
In such cases, the single-key security model is inadequate, as the administrator may need to take leave multiple times or assign their responsibilities to different colleagues.
Another potential application of collusion-resistant PKE-SKL is in streaming services.
Encrypted videos are made accessible to subscribers through quantum decryption keys.
When a user unsubscribes, they return their leased keys, losing access to the content, and their subscription fees are canceled. By utilizing an attribute-based encryption~\cite{EC:SahWat05} variant of collusion-resistant PKE-SKL, it becomes possible to precisely control video access based on user attributes, such as location-restrictions or premium and basic subscription plans.

\paragraph{Our goal: collusion-resistant SKL from weaker assumptions.}
The notion of collusion-resistant SKL is both natural and compelling. However, it has yet to be achieved from well-established assumptions.  
All known PKE with SKL schemes based on standard assumptions~\cite{EC:AKNYY23,TCC:AnaPorVai23,EPRINT:CGJL23,TCC:AnaHuHua24,myEPRINT:KitMorYam24} do not remain secure in the collusion-resistant setting.  
While some existing constructions seem to imply collusion-resistant SKL, such as public-key functional encryption (PKFE) schemes with SKL~\cite{EC:AKNYY23,EC:BGKMRR24} and collusion-resistant single decryptor encryption (SDE)\footnote{In short, SDE is PKE where the decryption keys are copy-protected (i.e., unclonable). In general, SDE implies encryption with SKL (see the discussion by Agrawal et al.~\cite{EC:AKNYY23} for details).}~\cite{TCC:CakGoy24}, they rely on strong assumptions. These include post-quantum secure indistinguishability obfuscation (IO) or collusion-resistant PKFE, which in turn implies IO, albeit with a sub-exponential security loss~\cite{JACM:BitVai18,C:AnaJai15,EPRINT:AnaJaiSah15a}. Achieving them from well-established assumptions still remains elusive.
In this work, we aim to construct the first collusion-resistant SKL schemes based on weaker assumptions.

%Achieving post-quantum secure IO (or collusion-resistant PKFE) from well-founded assumptions is a big open problem.
% Agrawal et al.'s PKFE with SKL scheme can be instantiated with bounded collusion-resistant (standard) PKFE, which standard PKE implies. However, that instantiation is secure in the \emph{bounded} collusion-resistant setting.

%\paragraph{Single-decryptor encryption.}
%A closely related notion to SKL is single-decryptor encryption (SDE), where the decryption keys are copy-protected (i.e., unclonable). In general, SDE implies encryption with SKL (see the discussion by Agrawal et al.~\cite{EC:AKNYY23} for details).
%\c{C}akan and Goyal~\cite{TCC:CakGoy24} constructed public-key SDE and SDFE schemes with fully collusion-resistant copy-protected keys from subexponetially secure IO, OWFs, and LWE. That is, they consider security against adversaries that receive many quantum decryption keys.
%
%As we see above, all previous works that achieve PKE with collusion-resistant SKL require IO (or collusion-resistant PKFE), which we want to avoid. In this work, our goal is to achieve PKE and attribute-based encryption (ABE) with \emph{collusion-resistant SKL} from \emph{weaker} assumptions.


\subsection{Our Results}
Our main contributions are summarized as follows:

\begin{enumerate}
    \item \emph{Definition of collusion-resistant PKE-SKL (PKE-CR-SKL):} We formally define PKE-CR-SKL, ensuring that if all quantum decryption keys are returned and successfully verified, users lose decryption capabilities. We extend the indistinguishability against key leasing attacks (IND-KLA) security definition~\cite{EC:AKNYY23} to the collusion-resistant setting. One notable feature of this definition is that the adversary can send multiple queries to the verification oracle, which confirms the validity of returned decryption keys. Another important feature is that the public key and ciphertext size are independent of the number of leased decryption keys (up to logarithmic factors).
 % These features are important in the collusion-resistant setting since many lessees return quantum decryption keys.

    \item \emph{Construction of IND-KLA secure PKE with collusion-resistant SKL:} We propose an IND-KLA secure PKE-CR-SKL scheme based on the learning with errors (LWE) assumption.

     \item \emph{Attribute-based encryption with collusion-resistant SKL:} We construct an ABE-CR-SKL scheme, also based on the LWE assumption. Since PKE is a special case of ABE, ABE-CR-SKL trivially implies PKE-CR-SKL. We first present the PKE-CR-SKL scheme separately to provide a clearer foundation for understanding the ABE-CR-SKL construction.

     \item \emph{PKE and ABE with collusion-resistant SKL and
         classical certificates:} We also propose an IND-KLA secure
         ABE-CR-SKL scheme that utilizes \emph{classical} certificates, whereas the constructions above rely on quantum
         certificates.\footnote{Quantum decryption keys function as quantum certificates.} In this model, a classical certificate can be derived from a leased quantum decryption key, and successful verification guarantees security. Our scheme is based on multi-input ABE (MI-ABE), which is a potentially weaker assumption than collusion-resistant PKFE. We specify required properties for MI-ABE and discuss the relationship between MI-ABE and other primitives in \cref{sec:def-mi-abe}. ABE-CR-SKL with classical certificates trivially implies PKE-CR-SKL with classical certificates.
\end{enumerate}

We introduce fascinating techniques to achieve our results. These
include the classical decryption property and the notion of key-testability for secret key encryption with collusion-resistant SKL (SKE-CR-SKL) and secret key functional encryption with collusion-resistant SKL (SKFE-CR-SKL). Then, we transform SKE-CR-SKL (resp. SKFE-CR-SKL) into PKE-CR-SKL (resp. ABE-CR-SKL) by using classical ABE and compute-and-compare obfuscation~\cite{FOCS:WicZir17,FOCS:GoyKopWat17}. In these transformations, we use decryption keys of SKE-CR-SKL (resp. SKFE-CR-SKL) as key attributes of ABE in a superposition way.
See~\cref{sec:technical_overview} for the details.

\begin{table*}[!t]
\setlength\tabcolsep{0.5eM}
\begin{center}
\begin{minipage}[c]{\textwidth} \scriptsize
\begin{center}
 \begin{threeparttable}



\caption{{\scriptsize Comparison of SKL. VO means security in the presence of the verification oracle for certificates.
% \fuyuki{We might want to highlight the verification query resilience additionally. I think the property is crucial in the application of collusion-resistant ones.}
% \nikhil{We could also mention the number of superposition terms in the decryption keys. For example, \cite{TCC:AnaPorVai23} has classical revocation but exponentially many terms.}
% \ryo{I did not see the advantage of fewer number of superposition terms. Could you elaborate on it?}
% \nikhil{I am not certain but I think states with small superposition
% have low entanglement entropy like measures which are studied in
% subset state PRS works. I was under the impression that would make
% them easier to maintain. I guess its better to leave it out.}
}
%BRC: best range cover, minimum number of subtrees that cover [a; b].
}
\label{tbl:comparison_PKE-SKL}
\begin{tabular}{rccccc}\toprule
 & Primitive & Collusion-resistant SKL & VO & Certificate & Assumption\\
\midrule
\cite{EC:AKNYY23} & PKE & $-$ & $\checkmark$ & quantum  & PKE\\
\cite{EC:AKNYY23} & ABE & bounded & $\checkmark$ & quantum  & ABE\\
 \cite{EC:AKNYY23} & bCR PKFE\tnote{a} & bounded & $\checkmark$ & quantum  & PKE \\
  \cite{EC:AKNYY23} & PKFE & $\checkmark$ & $\checkmark$ & quantum & PKFE\tnote{b} \\
\cite{EC:BGKMRR24} & PKFE\tnote{c} & $\checkmark$ & $\checkmark$\tnote{d} & classical  & IO \\
 \cite{TCC:AnaPorVai23,TCC:AnaHuHua24} & PKE (FHE) & $-$ &  $-$ & classical & LWE\\
 \cite{EPRINT:CGJL23} & PKE (FHE) & $-$ & $-$ & classical  & LWE\\
 \cite{myEPRINT:KitMorYam24} & PKE & $-$ & $\checkmark$\tnote{e}  & classical  & PKE\\
 \midrule
  Ours1 & PKE & \colorbox[rgb]{1,0.9,0}{$\checkmark$}&  \colorbox[rgb]{1,0.9,0}{$\checkmark$} & quantum  &\colorbox[rgb]{1,0.9,0}{LWE}\\
 Ours2 & ABE\tnote{c} & \colorbox[rgb]{1,0.9,0}{$\checkmark$}  & \colorbox[rgb]{1,0.9,0}{$\checkmark$} & quantum & \colorbox[rgb]{1,0.9,0}{LWE}\\
  Ours3 & PKE & \colorbox[rgb]{1,0.9,0}{$\checkmark$} & \colorbox[rgb]{1,0.9,0}{$\checkmark$} &  \colorbox[rgb]{1,0.9,0}{classical}  & MI-ABE\\
  Ours4 & ABE\tnote{c} & \colorbox[rgb]{1,0.9,0}{$\checkmark$}  & \colorbox[rgb]{1,0.9,0}{$\checkmark$} & \colorbox[rgb]{1,0.9,0}{classical} & MI-ABE\\  
\bottomrule
\end{tabular}
 \begin{tablenotes}[flushleft,online,normal] %(default:normal)
 \item[a] bCR denotes bounded collusion-resistant. This scheme only satisfies the bounded collusion resistance of standard PKFE.
 \item[b] Collusion-resistant PKFE implies IO up to sub-exponential security loss.
 \item[c] These schemes are selectively secure, where adversaries must
     declare the target plaintexts (PKFE case) or attributes (ABE case) at the beginning.\nikhil{Same 'c' is used for ABE-CR-SKL and Bartusek et al. Our selective-security means ciphertext attribute is declared right?}\ryo{Yes. \cite{EC:BGKMRR24} considers FE, so plaintexts are declared. Ours considers ABE, so I added ``attributes''.}
 \item[d] This scheme has public verifiability (the verification key for certificates is public).
 \item[e] This scheme is secure even if the verification key is revealed to the adversary \emph{after} the adversary outputs a valid certificate.
 \end{tablenotes}
 \end{threeparttable}
\end{center}
\end{minipage}
\end{center}
\end{table*}


% \begin{figure*}[t]
\begin{center}
\includegraphics[width=.85\linewidth]{fig_overview_v3.pdf}
\end{center}
\caption{
FastAtlas Overview: In each frame, we compute charts spanning fully or partially visible triangles (a), determine texture space bounding boxes for the visible portions of the view-space projections of each chart, and tightly pack these boxes into atlases (b, here $2K \times 2K$). We simultaneously bijectively parameterize and shade the charts into their atlas boxes, obtaining high quality texture space shading (c), and use this shading to render the shaded frames (d).}
\label{fig:overview}
\label{fig:alg_overview}
\end{figure*}

\section{Overview}
\label{sec:overview}
Our work has two core contributions: a real-time, GPU-based algorithm for tight packing of general parameterized charts into compact atlases; and a real-time TSS method that
utilizes this packing.  

\paragraph*{FastAtlas Packing.}
FastAtlas runs entirely on the GPU as a series of compute shaders. It takes the bounding boxes of parameterized charts as input, and packs them into an atlas (Fig~\ref{fig:overview}b, Sec.~\ref{sec:pack}). As such, the only input it requires are the dimensions of the bounding boxes.
Its outputs are deterministic; identical input charts are packed into identical atlases. This is critical for TSS and similar applications, as it ensures that consecutive frames taken from the same camera view have the same shading. Even minute shading differences across such frames can cause sampling jitter, leading to undesirable flicker \cite{baker2012rock}. 
While prior methods such as \cite{mueller2018shading,hladky2019tessellated,hladky2021snakebinning,Neff2022MSA} cap the dimensions of the charts that can be packed as-is for a given atlas size, and scale down all charts that exceed these dimensions, we scale all charts by the same factor, and do so only when strictly necessary to achieve packing success (Figs~\ref{fig:atlas},~\ref{fig:sas_issues}). 

\paragraph*{TSS using FastAtlas.}
Our end-to-end TSS atlas generation method combines the packing method above with a novel approach for computing seamless per-frame charts. 
We define our charts as the connected components of the visible surfaces in each frame (Fig.~\ref{fig:overview}a), and efficiently compute them using a parallel union-find algorithm (Sec.~\ref{sec:visible}). Since the boundaries of these charts coincide with the contours of the rendered surface, they are {\em invisible} to the viewer. This approach 
eliminates the artifacts caused by shading discontinuities along visible seams (Fig.~\ref{fig:seams}). 

\begin{parWithWrapFigure}
\begin{wrapfigure}{l}{.27\columnwidth}%
\includegraphics[width=\linewidth]{fig_inset_view_plane.pdf}%
\end{wrapfigure}
We bijectively parametrize the {\em visible portions} of our charts by projecting them to view space (inset). This maps a constant number of texels to each pixel in the final rendered output, evenly distributing residual undersampling error across all image pixels. While conceptually straightforward, efficiently parameterizing charts containing partially visible triangles using viewspace projection is non-trivial, as the visible portions may no longer be triangular (e.g. green triangle in the inset); applying naive projection to triangles with vertices behind the camera may produce ill-posed results. Clipping triangles before projection is both computationally expensive and significantly complicates downstream operations. We avoid explicit clipping by observing that all that is required for atlas packing is the dimensions of, potentially conservative, bounding boxes of these projected visible portions. We compute such bounding boxes without explicit chart clipping by adapting a conservative screen coverage estimator \shortcite{Blinn:CalculatingScreenCoverage} (Sec.~\ref{sec:box}). We then pack the computed boxes using FastAtlas. 
\end{parWithWrapFigure}

Finally, we shade the visible portion of each chart into its corresponding atlas bounding box (Fig~\ref{fig:overview}c). 
The resulting texture is then used during rasterization as a standard texture map (Fig. ~\ref{fig:overview}d). 
Our framework is compatible with all existing approaches for texture space shading, including forward shading, raytraced illumination, or deferred shading in texture space \cite{baker:2016}. In the examples shown, we use the standard forward shading based rendering pipeline included in the G3D Innovation Engine \cite{G3D17}, a commercial grade renderer.


\subsection{Related Work}\label{sec:related_work}
\paragraph{Encryption with SKL.}
Agrawal et al.~\cite{EC:AKNYY23} presented PKE-SKL, ABE-SKL, and PKFE-SKL schemes.
Their PKE-SKL scheme is based on standard PKE, and its certificate is quantum. This scheme is not collusion-resistant.
Their ABE-SKL scheme is based on PKE-SKL and standard (collusion-resistant) ABE, and its certificates are ones of the underlying PKE-SKL.
This scheme is bounded collusion-resistant, that is, the adversary can
obtain an a-priori bounded number of quantum decryption keys that can decrypt target ciphertexts.
Their PKFE-SKL scheme is based on PKE-SKL and standard (collusion-resistant) PKFE, and its certificates are ones of the underlying PKE-SKL.
This scheme is collusion-resistant. (If we instantiate the PKFE-SKL scheme with bounded collusion-resistant PKFE, the scheme is bounded collusion-resistant with respect to both standard PKFE and SKL.)
All schemes mentioned above are secure in the presence of the verification oracle for certificates.

Bartusek et al.~\cite{EC:BGKMRR24} also presented a PKFE-SKL scheme.
Their scheme is collusion-resistant in the selective model, where the target plaintext is declared at the beginning of the game. In addition, their certificates are classical and publicly verifiable. However, their scheme relies on IO.

Ananth, Poremba, and Vaikuntanathan~\cite{TCC:AnaPorVai23} presented a PKE-SKL scheme with classical certificates based on the LWE assumption and an unproven conjecture.
Since the encryption algorithm of this scheme is essentially the same as the Dual-Regev PKE scheme~\cite{STOC:GenPeiVai08}\nikhil{Needs citation?}\ryo{done.}, their scheme achieves fully homomorphic encryption (FHE) with SKL. Later, Ananth, Hu, and Huang~\cite{TCC:AnaHuHua24} present a new security analysis for Ananth et al.'s scheme and removed the conjecture.
Chardouvelis, Goyal, Jain, and Liu~\cite{EPRINT:CGJL23} presented a PKE-SKL scheme with classical certificates based on the LWE assumption. Their scheme is based on the Regev PKE scheme~\cite{JACM:Regev09}\nikhil{Needs citation?}\ryo{done.}, so it also achieves FHE with SKL. In addition, their scheme also achieves classical communication, where all messages between senders and receivers are classical.
Kitagawa, Morimae, and Yamakawa~\cite{myEPRINT:KitMorYam24} presented
a PKE-SKL scheme with classical certificates based on PKE.
These three works do not achieve collusion resistance.
Ananth et al.~\cite{TCC:AnaPorVai23,TCC:AnaHuHua24} and Chardouvelis et al.~\cite{EPRINT:CGJL23} do not consider the verification oracle in their security definitions, while Kitagawa et al.~\cite{myEPRINT:KitMorYam24} does.\footnote{More precisely, the work considers adversaries that receive a verification key after they output a valid certificate. Kitagawa et al. show that their scheme can be converted to satisfy IND-KLA by Agrawal et al.~\cite{EC:AKNYY23}.}
We summarize the works on encryption with SKL in~\cref{tbl:comparison_PKE-SKL}.

\paragraph{Secure software leasing.}
Ananth and La Placa~\cite{EC:AnaLaP21} introduced secure software
leasing, which encodes classical programs into quantum programs that
we can securely lease. We can view SKL as secure software leasing for
decryption functions. However, previous works on secure software
leasing consider a sub-class of evasive
functions~\cite{EC:AnaLaP21,ARXIV:ColMajPor20,TCC:KitNisYam21,TCC:BJLPS21}
or PRFs~\cite{TCC:KitNisYam21}, which do not support decryption
functions. Moreover, they consider a weak security model in which
pirated programs use \emph{honest} evaluation
algorithms~\cite{EC:AnaLaP21,TCC:KitNis23,TCC:BJLPS21} or rely on the
quantum random oracle model~\cite{ARXIV:ColMajPor20}. Bartusek et
al.~\cite{EC:BGKMRR24} achieve secure software leasing supporting all
differing inputs circuits\footnote{Roughly speaking, a pair of
circuits $(C_0,C_1)$ is differing input if it is hard to find an input
$y$ such that $C_0(y)\ne C_1(y)$.} in a strong security model where
pirated programs can use arbitrary evaluation algorithms. However, their scheme relies on IO.



\paragraph{Multi-input ABE.}
Roughly speaking, MI-ABE is ABE that can support multiple ciphertext-attributes (or multiple key-attributes).
To achieve our classical certificates scheme, we need MI-ABE for polynomial-size circuits where the number of slots is polynomial, and we can generate ciphertexts for one slot using a public key. See~\cref{def:miabe} for the definition.
However, as we review previous works on MI-ABE for general circuits\footnote{Francati, Fior, Malavolta, and Venturi~\cite{EC:FFMV23} and Agrawal, Tomida, and Yadav~\cite{C:AgrTomYad23} proposed MI-ABE for restricted functionalities.} below, none of them achieves what we need without using IO.\footnote{IO implies multi-input functional encryption~\cite{EC:GGGJKL14}, which implies MI-ABE.}

Agrawal, Yadav, and Yamada~\cite{C:AgrYadYam22} proposed \emph{two-input} ABE for polynomial-size circuits based on lattices. However, the scheme is heuristic (no reduction-based security proof) and needs a master \emph{secret key to generate ciphertexts for all slots}.
Agrawal, Rossi, Yadav, and Yamada~\cite{C:ARYY23} proposed MI-ABE for
$\NCone$ from the evasive LWE assumption and MI-ABE for
polynomial-size circuits from the evasive LWE and tensor LWE
assumptions. Their scheme allows to generate ciphertexts for one slot using a public key. However, the number of slots is \emph{constant}.
Agrawal, Kumari, and Yamada~\cite{myEPRINT:AgrKumYam24a} proposed MI-ABE\footnote{Precisely speaking, their scheme is predicate encryption, which satisfies privacy for attributes.} for polynomial-size circuits based on the evasive LWE assumption.
In their scheme, the number of slots is polynomial. However, we need a master \emph{secret key to generate ciphertexts for all input-attributes}.

\paragraph{Broadcast encryption.}
Broadcast encryption~\cite{C:FiaNao93} enables a sender to generate ciphertexts intended for a specific subset of users.
Only the designated users can decrypt the ciphertexts, while even if all other users collude, they cannot recover the message. A key performance metric for broadcast encryption is the size of the public key and ciphertexts.
Several works have proposed optimal broadcast encryption schemes, where these sizes are $\poly(\log{N})$ and $N$ is the total number of users.
The constructions by Agrawal and Yamada~\cite{EC:AgrYam20} and Agrawal, Wichs, and Yamada~\cite{TCC:AgrWicYam20} rely on the LWE assumption and \emph{pairings}, which are not post-quantum secure. The construction by Wee~\cite{EC:Wee22} relies on the \emph{evasive LWE assumption}, which is a non-falsifiable assumption and has known counterexamples~\cite{AC:BrzUnaWoo24}. The construction by Brakerski and Vaikuntanathan~\cite{ITCS:BraVai22} relies on \emph{heuristics} and lacks a reduction-based proof.

While broadcast encryption is particularly well-suited for streaming services, PKE-CR-SKL can also be applied in this domain. Each approach has its own advantages and limitations. One advantage of broadcast encryption is that it allows the sender to specify recipients at the encryption phase, whereas PKE-CR-SKL does not.
However, PKE-CR-SKL offers three notable advantages over optimal broadcast encryption.
\begin{enumerate}
    \item \textbf{Efficient revocation:} Since decryption keys in broadcast encryption are classical, the system requires maintaining a revocation list, which senders use to revoke users, whereas PKE-CR-SKL eliminates the need for such lists.
    \item \textbf{Seamless user expansion:} In broadcast encryption, the user set is fixed during the setup phase, meaning that adding new users requires updating the encryption key. In contrast, PKE-CR-SKL allows new users to be added without requiring any key updates.
    \item \textbf{Weaker cryptographic assumptions:} Our PKE-CR-SKL scheme is based on the standard LWE assumption, whereas post-quantum secure optimal broadcast encryption relies on the evasive LWE assumption, which has counterexamples~\cite{AC:BrzUnaWoo24}.
\end{enumerate}
In terms of  asymptotic efficiency, both optimal broadcast encryption and PKE-CR-SKL achieve the same public key and ciphertext sizes.

\ifnum\submission=1
We discuss more related works in~\cref{sec:other-rel}.
\else
\section{Additional Related Work}\label{sec:add_related_work}

Mixture of Experts (MoE) and Memory Layers  are two well-established methods for scaling language models within a fixed FLOPS budget, as discussed below.

\paragraph{Mixture of Experts} MoE   replaces traditional feedforward layers with parallel `expert' layers, activating only one (or a few) per token via a lightweight router \citep{shazeer2017outrageously,lepikhin2021gshard,fedus2022switch,jiang2024mixtral,he2024mixture}. This allows scaling by increasing the number of experts without increasing active experts per token. However, all experts must reside on the accelerator, leading to higher memory usage.

\paragraph{Memory Layers} Memory layers store large sets of embeddings (continuous vectors) and retrieve nearest-neighbor embeddings during the forward pass via (approximate) similarity search \citep{weston2014memory,sukhbaatar2015end,lample2019large,berges2024memory}. These retrieved embeddings contribute to computations without adding to FLOPS. Despite advancements in similarity search \citep{lample2019large,johnson2019billion}, memory layers still need to reside on accelerators, which increases memory demands to impractical levels at larger scales \citep{berges2024memory}. Moreover, the embeddings in memory layers are typically updated through backpropagation, which also introduces sparse update challenges as memory scales.

Our approach focuses on input embedding layers, which we demonstrate can be efficiently offloaded from accelerators. This ensures constant memory usage and fixed FLOPS on the accelerator during inference.  Additionally, by modifying only the input embedding layer, our method integrates seamlessly with both MoE and memory layer techniques.






\fi



% \begin{table*}[!t]
% \setlength\tabcolsep{0.5eM}
% \begin{center}
% \begin{minipage}[c]{\textwidth} \scriptsize
% \begin{center}
%  \begin{threeparttable}


% \caption{\textcolor{red}{This table will be removed in the paper.}{\scriptsize Comparison of broadcast encryption and PKE-CR-SKL. Let $N$ be the number of users in broadcast encryption. In PKE-CR-SKL, we can think $N=2^\secp$, where $\secp$ is the security parameter.
% }
% %BRC: best range cover, minimum number of subtrees that cover [a; b].
% }
% \label{tbl:comparison_with_BE}
% \begin{tabular}{rcc}\toprule
%  & Optimal broadcast encryption & PKE-CR-SKL\\
% \midrule
% $\abs{\pk}$ & $\poly(\log{N})$ & $\poly(\log{N})$\\
% $\abs{\ct}$ & $\poly(\log{N})$ & $\poly(\log{N})$ \\
% Can specify recipients in $\Enc$ & $\checkmark$ & $\times$ \\
% No revocation list & $\times$ & $\checkmark$  \\
% No key update for dynamic join & $\times$ & $\checkmark$ \\
% From standard assumptions & $\times$ & $\checkmark$ \\
% \bottomrule
% \end{tabular}
%  % \begin{tablenotes}[flushleft,online,normal] %(default:normal)
%  % \item[a] bCR denotes bounded collusion-resistant. 
%  % \end{tablenotes}
%  \end{threeparttable}
% \end{center}
% \end{minipage}
% \end{center}
% \end{table*}




%\nikhil{I think we don't need this section if we have the index. However, the index takes up more space.}
%\ryo{We need this section in the submission version.}
\ifnum\submission=1
\subsection{Organization of the paper}
In~\cref{sec:technical_overview}, we present a technical overview.
In~\cref{sec:prelim}, we define the notation and preliminaries that we require in this work.  In~\cref{sec:cr-skl-defs}, we define the notion of secret key and public key encryption with collusion-resistant secure key leasing (SKE-CR-SKL and PKE-CR-SKL) and its security notions. In~\cref{sec:SKECRSKL-KT}, we construct an SKE-CR-SKL scheme with key testability. In~\cref{sec:PKE-CR-SKL}, we provide our PKE-CR-SKL scheme. Due to the page limitation, we present our ABE-CR-SKL and classical certificates ABE-CR-SKL schemes in~\cref{sec:ABE-SKL,sec:ABR-CR2-SKL}, respectively.
\else
\fi




