\section{Construction of \tPKECRCRSKL}

\begin{comment}
\nikhil{They seem to build (Adaptive-Secure) (Multi-Input) (Predicate) Encryption for polynomial arity for P. All three prefixes are stronger than what we need, which is (Selective-Secure) (Multi-Key) (Attribute-Based) Encryption.}
\end{comment}

The building blocks of the construction
$\PKECRCRSKL = \PKECRCRSKL.(\Setup, \qKG, \Enc,$ 
$\qDec, \qDel, \Vrfy)$ are as follows:

\begin{itemize}
\item Multi-Key (Ciphertext-Policy) ABE Scheme $\MKABE =
\MKABE.(\Setup, \KG, \Enc, \Dec)$ for the following relation:

\begin{description}
\item[$R(x, u_1, \cdots, u_\ell)$:] Let $x$ be
interpreted as an $\ell$-input circuit.
\begin{itemize}
\item If $x\big(u_1, \cdots, u_\ell) = \bot$, output $0$
(decryptable).
\item Else, output $1$.
\end{itemize}
\end{description}

\item Compute-and-Compare Obfuscation $\CCObf$ with the simulator
$\CCSim$.

\item Random Oracle $H$.

\item Secret-Key Encryption Scheme $\SKE = \SKE.(\Enc,
\Dec)$.
\end{itemize}

The construction is described as follows:

\begin{description}
\item[$\PKECRCRSKL.\Setup(1^\secp)$:] $ $
\begin{itemize}
\item Generate $(\abe.\pk,\abe.\msk_1, \cdots, \abe.\msk_\ell)\gets\MKABE.\Setup(1^\secp)$.
\item Generate $\ske.\sk\gets \bit^\secp$.
\item Output $\ek\seteq\abe.\pk$ and
    $\msk\seteq(\{\abe.\msk_i\}_{i\in[\ell]},\ske.\sk)$.
\end{itemize}

\item[$\PKECRCRSKL.\qKG(\msk)$:] $ $
\begin{itemize}
\item Parse $\msk=(\abe.\msk,\ske.\sk)$.
\item Sample random $r_1, \cdots, r_\ell$ such that $r_1 \xor \cdots
\xor r_\ell = 0^{\secp}$.
\item Sample a random value $y \gets \bit^\secp$.

\item For each $i \in [\ell]$, compute the following:
\begin{itemize}
\item Sample $(x_0^i, x_1^i) \leftarrow
\bit^\secp \times \bit^\secp$.

\item Compute $\ske.\ct_i = \SKE.\Enc(\ske.\sk, H(x_0^i) \xor r_i,
H(x_1^i) \xor r_i)$.

\item For each $b \in \bit$, compute $\abe.\dk_b^i =
\MKABE.\KG(\abe.\msk_i, y \| x_b^i \| \ske.\ct_i)_\qreg{ABE_i}$.

\item
Prepare the following state $\rho_i$:
$$\ket{0}_{\qreg{A_i}}\ket{x_0^i}_{\qreg{B_i}}\ket{\abe.\dk_0^i}_\qreg{ABE_i} +
\ket{1}_{\qreg{A_i}}\ket{x_1^i
}_{\qreg{B_i}}\ket{\abe.\dk_1^i}_\qreg{ABE_i}.$$


\end{itemize}
\item Output $\qdk = (\rho_i)_{i \in [\ell]}$ and $\vk =
\big((\abe.\dk_0^i \xor \abe.\dk_1^i)_{i \in [\ell]}, (x_0^i
\xor x_1^i)_{i \in [\ell]}\big)$.
\end{itemize}

\item[$\PKECRCRSKL.\Enc(\ek,\msg)$:] $ $
\begin{itemize}
\item Parse $\ek=\abe.\pk$.
\item Generate $\tlC\gets\CCSim(1^\secp)$.
\item Let $B[\tlC]$ be a circuit defined as follows:
\begin{description}
\item[$B{[}\tlC{]}(u_1, \cdots, u_\ell)$:] $ $
\begin{itemize}
\item For each $i \in [\ell]$, parse $u_i = y_i \| z_i$.
\item Output $\bot$ if $\big(\tlC(z_1 \| \cdots \| z_\ell) =
\bot\big) \land \big(y_1 = \cdots = y_\ell\big)$.
\item Otherwise, output $0$.
\end{itemize}
\end{description}

\item Generate $\abe.\ct\gets\MKABE.\Enc(\abe.\pk,B[\tlC],\msg)$.
\item Output $\ct\seteq\abe.\ct$.
\end{itemize}

\item[$\PKECRCRSKL.\qDec(\qdk,\ct)$:] $ $
\begin{itemize}
\item Parse $\ct=\abe.\ct$ and $\qdk = (\rho_i)_{i \in [\ell]}$. Let
the register holding $\rho_i$ be denoted as $\qreg{A_i} \otimes
\qreg{B_i} \otimes \qreg{ABE_i}$. 
\item Prepare a register $\qreg{MSG}$ that is initialized to
$\ket{0\cdots0}_{\qreg{MSG}}$.

\item Apply the map:
$\bigotimes_{i=1}^{\ell}\ket{v_i}_{\qreg{ABE_i}}\otimes
\ket{w}_{\qreg{MSG}}\ra
\bigotimes_{i=1}^{\ell}\ket{v_i}_{\qreg{ABE_i}}\otimes\ket{w\oplus\MKABE.\Dec(\abe.\ct, v_1, \cdots, v_l)}_{\qreg{MSG}}$ to the registers
$\bigotimes_{i=1}^\ell\qreg{ABE_i}$ and $\qreg{MSG}$.
\item Measure the register $\qreg{MSG}$ in the computational basis
and output the result $\msg^\prime$.
\end{itemize}

\item[$\PKECRCRSKL.\qDel(\qdk)$:] $ $
\begin{itemize}
\item Parse $\qdk = (\rho_i)_{i \in [\ell]}$. Let the register 
holding $\rho_i$ be denoted as $\qreg{A_i} \otimes \qreg{B_i}
\otimes \qreg{ABE_i}$.
\item For each $i \in [\ell]$, measure the registers $\qreg{A_i}$,
$\qreg{B_i}$, and $\qreg{ABE_i}$ in the Hadamard basis to
obtain outcomes $c_i, d_i$, and $e_i$.
\item Output $\cert = (c_i, d_i, e_i)_{i \in [\ell]}$.
\end{itemize}

\item[$\PKECRCRSKL.\Vrfy(\vk, \cert')$:] $ $
\begin{itemize}
\item Parse $\vk =
\big((\abe.\dk_0^i \xor \abe.\dk_1^i)_{i \in [\ell]}, (x_0^i \xor
x_1^i)_{i \in [\ell]}\big)$ and $\cert' = (c_i, d_i, e_i)_{i \in
[\ell]}$.

\item For each $i \in [\ell]$, check if $c_i = \big\langle d_i
\| e_i, (x_0^i \xor x_1^i) \| (\abe.\dk_0^i \xor
\abe.\dk_1^i)\big\rangle$. Output $\top$ if the check passes for all
$i \in [\ell]$. Else, output $\bot$.
\end{itemize}
\end{description}

\paragraph{Security of Building Block:}
Consider the following experiment $\mathsf{Exp}_\qA(1^\secp,
\coin)$:

\begin{enumerate}
\item $\qA$ sends $(m_0, m_1) \in \bit^\secp \times \bit^\secp$ to
$\qCh$ and $q = \poly(\secp)$.

\item $\qCh$ chooses $\sk \gets \bit^\secp$.

\item For each $j \in [q]$, $\qCh$ computes the following:
\begin{itemize}
\item Sample random $r_1^j, \cdots, r_\ell^j$ such that $r_1^j
\xor \cdots \xor r_\ell^j = m_\coin$.

\item For each $i \in [\ell]$, sample $x_0^{i, j}, x_1^{i, j} \gets
\bit^\secp \times \bit^\secp$ and prepare the following state:

$$\rho_i^j = \ket{0}\ket{x_0^{i,j}} + \ket{1}\ket{x_1^{i, j}}$$

$$\ct_i^j \gets \SKE.\Enc\Big(\sk, H(x_0^{i,j}) \xor r_i^j,
H(x_1^{i,j}) \xor r_i^j\Big)$$

\item Set $\rho^j = (\rho^j_i)_{i \in [\ell]}$ and $\ct^j =
(\ct_i^j)_{i \in [\ell]}$.
\end{itemize}

\item $\qCh$ sends $(\rho^j, \ct^j)_{j \in [q]}$ to $\qA$.

\item For each $j \in [q]$, $\qA$ sends
$(c_i^j, d_i^j)_{i \in [\ell]}$ to $\qCh$.

\item For each $(i, j) \in [\ell]\times[q]$, $\qCh$ checks if
$d_i^j\cdot(x^{i,j}_0 \xor x^{i,j}_1) = c_i^j$ holds. If all checks
pass, it sends $\sk$ to $\qA$. Else, it outputs $\perp$.

\item $\qA$ outputs a guess $\coin'$. $\qCh$ outputs $\coin'$ as the
final output of the experiment.
\end{enumerate}

We have the following security requirement:

$$\mathsf{Adv}_\qA(1^\secp) \seteq \Big\lvert
\Pr[\mathsf{Exp}_\qA(1^\secp, 0) \ra 1] -
\Pr[\mathsf{Exp}_\qA(1^\secp, 1) \ra 1]\Big\rvert \le \negl(\secp)$$

\paragraph{Security Proof of \tPKECRCRSKL:}

Let $\qA$ be an adversary for the IND-KLA security of $\PKECRSKL$.
We consider the following sequence of experiments.
\begin{description}
\item[$\hybi{0}^\coin$:] This is
$\expb{\PKECRCRSKL,\qA}{ind}{kla}(1^\secp,\coin)$.

\item[$\hybi{1}^\coin$:] This is the same as $\hybi{0}^\coin$ except
that instead of $\tlC$ being generated as $\tlC \gets
\CCSim(1^\secp)$, it is generated as follows:

$$\tlC \gets \CCObf(P[\sk], \lock, 0^\secp)$$ where $P[\sk]$ is a
circuit defined as follows:

\begin{description}
\item[$P{[}\sk{]}(z)$:] $ $
\begin{itemize}
\item Parse $z = z_1 \| \cdots \| z_\ell$.
\item For each $i \in [\ell]$, do the following:
\begin{itemize}
\item Parse $z_i = x^i_b \| \ske.\ct_i$.
\item Compute $r_i = \SKE.\Dec(\sk, \ske.\ct_i) \xor H(x_b^i)$.
\end{itemize}
\item Output $r_1 \xor \cdots \xor r_\ell$.
\end{itemize}
\end{description}

Notice that $\qA$'s view and $P[\sk]$ are independent of $\lock$.
Hence, from the security of compute-and-compare obfuscation, we have
that $\Hyb_1^\coin \approx \Hyb_0^\coin$.

\item[$\hybi{2}^\coin$:] This is the same as $\Hyb_1^\coin$ except
that all the decryption keys $\qdk_1, \cdots, \qdk_q$ are generated
with the following change: $r_1, \cdots, r_\ell$ are chosen at
random such that $r_1 \xor \cdots \xor r_\ell = \lock$. Conditioned
on every deletion check passing, this switch is undetectable by the
security of the building block. Hence, we have $\Hyb_2^\coin \approx
\Hyb_1^\coin$. 

\nikhil{Need to account for oracle queries using some kind of
key-testability style arguments.}

\item[$\hybi{3}^\coin$:] This is the same as $\Hyb_2^\coin$ except
that $\abe.\ct^\star$ is generated as $\abe.\ct^\star \gets
\MKABE.\Enc(\abe.\pk, B[\tlC], 0^\secp)$.
Since $P[\sk](z)$ outputs $r_1 \xor \cdots \xor r_\ell = \lock$,
$\tlC$ outputs $0^\secp$ instead of $\bot$. Now, by the
security of MK-ABE, $\qA$ can distinguish between $\Hyb_3^\coin$
and $\Hyb_2^\coin$ only if it has some set of $\ell$ keys satisfying
both conditions checked by the ciphertext attribute $B[\tlC]$.
However, it is easy to see that for any set of $\ell$ keys, at least
one of the conditions is false.

\end{description}