% !TEX root = main.tex

\section{Omitted Preliminaries}\label{sec:omitted_prelim}
\paragraph{Notations and conventions.}
In this paper, standard math or sans serif font stands for classical algorithms (e.g., $C$ or $\algo{Gen}$) and classical variables (e.g., $x$ or $\keys{pk}$).
Calligraphic font stands for quantum algorithms (e.g., $\qalgo{Gen}$) and calligraphic font and/or the bracket notation for (mixed) quantum states (e.g., $\qstate{q}$ or $\ket{\psi}$).

Let $[\ell]$ denote the set of integers $\{1, \cdots, \ell \}$, $\secp$ denote a security parameter, and $y \seteq z$ denote that $y$ is set, defined, or substituted by $z$.
For a finite set $X$ and a distribution $D$, $x \chosen X$ denotes selecting an element from $X$ uniformly at random, and $x \chosen D$ denotes sampling an element $x$ according to $D$. Let $y \gets \algo{A}(x)$ and $y \gets \qalgo{A}(\qstate{x})$ denote assigning to $y$ the output of a probabilistic or deterministic algorithm $\algo{A}$ and a quantum algorithm $\qalgo{A}$ on an input $x$ and $\qstate{x}$, respectively.
PPT and QPT algorithms stand for probabilistic polynomial-time algorithms and polynomial-time quantum algorithms, respectively.
Let $\negl$ denote a negligible function.
For strings $x,y\in \bit^n$, $x\cdot y$ denotes $\bigoplus_{i\in[n]} x_i y_i$ where $x_i$ and $y_i$ denote the $i$th bit of $x$ and $y$, respectively. \fuyuki{Need to check if all the notations here are really used in this paper or not.}
\nikhil{I think so, except we didn't use ; for randomness. Also need to check if everything is covered here.} For random variables $X$ and $Y$, we use the notation $X \approx Y$ to denote that these are computationally indistinguishable. On the other hand, $X \approx_s Y$ denotes statistically indistinguishability between the random variables.

\paragraph{Compute-and-Compare Obfuscation.}
We define a class of circuits called compute-and-compare circuits as
follows:

\begin{definition}[Compute-and-Compare Circuits]\label{def:cc_circuits_searchability}
A compute-and-compare circuit $\cnc{P}{\lock,\msg}$ is of the form
\[
\cnc{P}{\lock,\msg}(x)\left\{
\begin{array}{ll}
    \msg&\textrm{if}\; P(x)=\lock\\
\bot&\text{otherwise}~
\end{array}
\right.
\]
where $P$ is a circuit, $\lock$ is a string called the lock value,
and $\msg$ is a message.
\end{definition}

We now introduce the definition of compute-and-compare obfuscation.
We assume that a program $P$ has an associated set of parameters $\pp_P$ (input size, output size, circuit size) which we do not need to hide.
\begin{definition}[Compute-and-Compare Obfuscation]\label{def:CCObf}
A PPT algorithm $\CCObf$ is an obfuscator for the family of distributions $D=\{D_\secp\}$ if the following holds:
\begin{description}
\item[Functionality Preserving:] There exists a negligible function
$\negl$ such that for all programs $P$, all lock values $\lock$, and
all messages $\msg$, it holds that

\begin{align}
\Pr[\forall x, \tlP(x)=\cnc{P}{\lock,\msg}(x) :
\tlP\la\CCObf(1^\secp,P,\lock,\msg)] \ge 1-\negl(\secp).
\end{align}
\item[Distributional Indistinguishability:] There exists an
efficient simulator $\Sim$ such that for all messages $\msg$, we have
\begin{align}
(\CCObf(1^\secp,P,\lock,\msg),\qaux)\approx(\CCSim(1^\secp,\pp_P,\abs{\msg}),\qaux),
\end{align}
where $(P,\lock,\qaux)\la D_\secp$.
\end{description}
\end{definition}

\begin{theorem}[\cite{FOCS:GoyKopWat17,FOCS:WicZir17}]
If the LWE assumption holds, there exists compute-and-compare obfuscation for all families of distributions $D=\{D_\secp\}$, where each $D_\secp$ outputs uniformly random lock value $\lock$ independent of $P$ and $\qaux$.
\end{theorem}
