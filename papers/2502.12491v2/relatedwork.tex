\section{Related Work}
\label{sec:related_work}
\paragraph{Encryption with SKL.}
Agrawal et al.~\cite{EC:AKNYY23} presented PKE-SKL, ABE-SKL, and PKFE-SKL schemes.
Their PKE-SKL scheme is based on standard PKE, and its certificate is quantum. This scheme is not collusion-resistant.
Their ABE-SKL scheme is based on PKE-SKL and standard (collusion-resistant) ABE, and its certificates are ones of the underlying PKE-SKL.
This scheme is bounded collusion-resistant, that is, the adversary can
obtain an a-priori bounded number of quantum decryption keys that can decrypt target ciphertexts.
Their PKFE-SKL scheme is based on PKE-SKL and standard (collusion-resistant) PKFE, and its certificates are ones of the underlying PKE-SKL.
This scheme is collusion-resistant. (If we instantiate the PKFE-SKL scheme with bounded collusion-resistant PKFE, the scheme is bounded collusion-resistant with respect to both standard PKFE and SKL.)
All schemes mentioned above are secure in the presence of the verification oracle for certificates.

Bartusek et al.~\cite{EC:BGKMRR24} also presented a PKFE-SKL scheme.
Their scheme is collusion-resistant in the selective model, where the target plaintext is declared at the beginning of the game. In addition, their certificates are classical and publicly verifiable. However, their scheme relies on IO.

Ananth, Poremba, and Vaikuntanathan~\cite{TCC:AnaPorVai23} presented a PKE-SKL scheme with classical certificates based on the LWE assumption and an unproven conjecture.
Since the encryption algorithm of this scheme is essentially the same as the Dual-Regev PKE scheme~\cite{STOC:GenPeiVai08}\nikhil{Needs citation?}\ryo{done.}, their scheme achieves fully homomorphic encryption (FHE) with SKL. Later, Ananth, Hu, and Huang~\cite{TCC:AnaHuHua24} present a new security analysis for Ananth et al.'s scheme and removed the conjecture.
Chardouvelis, Goyal, Jain, and Liu~\cite{EPRINT:CGJL23} presented a PKE-SKL scheme with classical certificates based on the LWE assumption. Their scheme is based on the Regev PKE scheme~\cite{JACM:Regev09}\nikhil{Needs citation?}\ryo{done.}, so it also achieves FHE with SKL. In addition, their scheme also achieves classical communication, where all messages between senders and receivers are classical.
Kitagawa, Morimae, and Yamakawa~\cite{myEPRINT:KitMorYam24} presented
a PKE-SKL scheme with classical certificates based on PKE.
These three works do not achieve collusion resistance.
Ananth et al.~\cite{TCC:AnaPorVai23,TCC:AnaHuHua24} and Chardouvelis et al.~\cite{EPRINT:CGJL23} do not consider the verification oracle in their security definitions, while Kitagawa et al.~\cite{myEPRINT:KitMorYam24} does.\footnote{More precisely, the work considers adversaries that receive a verification key after they output a valid certificate. Kitagawa et al. show that their scheme can be converted to satisfy IND-KLA by Agrawal et al.~\cite{EC:AKNYY23}.}
We summarize the works on encryption with SKL in~\cref{tbl:comparison_PKE-SKL}.

\paragraph{Secure software leasing.}
Ananth and La Placa~\cite{EC:AnaLaP21} introduced secure software
leasing, which encodes classical programs into quantum programs that
we can securely lease. We can view SKL as secure software leasing for
decryption functions. However, previous works on secure software
leasing consider a sub-class of evasive
functions~\cite{EC:AnaLaP21,ARXIV:ColMajPor20,TCC:KitNisYam21,TCC:BJLPS21}
or PRFs~\cite{TCC:KitNisYam21}, which do not support decryption
functions. Moreover, they consider a weak security model in which
pirated programs use \emph{honest} evaluation
algorithms~\cite{EC:AnaLaP21,TCC:KitNis23,TCC:BJLPS21} or rely on the
quantum random oracle model~\cite{ARXIV:ColMajPor20}. Bartusek et
al.~\cite{EC:BGKMRR24} achieve secure software leasing supporting all
differing inputs circuits\footnote{Roughly speaking, a pair of
circuits $(C_0,C_1)$ is differing input if it is hard to find an input
$y$ such that $C_0(y)\ne C_1(y)$.} in a strong security model where
pirated programs can use arbitrary evaluation algorithms. However, their scheme relies on IO.



\paragraph{Multi-input ABE.}
Roughly speaking, MI-ABE is ABE that can support multiple ciphertext-attributes (or multiple key-attributes).
To achieve our classical certificates scheme, we need MI-ABE for polynomial-size circuits where the number of slots is polynomial, and we can generate ciphertexts for one slot using a public key. See~\cref{def:miabe} for the definition.
However, as we review previous works on MI-ABE for general circuits\footnote{Francati, Fior, Malavolta, and Venturi~\cite{EC:FFMV23} and Agrawal, Tomida, and Yadav~\cite{C:AgrTomYad23} proposed MI-ABE for restricted functionalities.} below, none of them achieves what we need without using IO.\footnote{IO implies multi-input functional encryption~\cite{EC:GGGJKL14}, which implies MI-ABE.}

Agrawal, Yadav, and Yamada~\cite{C:AgrYadYam22} proposed \emph{two-input} ABE for polynomial-size circuits based on lattices. However, the scheme is heuristic (no reduction-based security proof) and needs a master \emph{secret key to generate ciphertexts for all slots}.
Agrawal, Rossi, Yadav, and Yamada~\cite{C:ARYY23} proposed MI-ABE for
$\NCone$ from the evasive LWE assumption and MI-ABE for
polynomial-size circuits from the evasive LWE and tensor LWE
assumptions. Their scheme allows to generate ciphertexts for one slot using a public key. However, the number of slots is \emph{constant}.
Agrawal, Kumari, and Yamada~\cite{myEPRINT:AgrKumYam24a} proposed MI-ABE\footnote{Precisely speaking, their scheme is predicate encryption, which satisfies privacy for attributes.} for polynomial-size circuits based on the evasive LWE assumption.
In their scheme, the number of slots is polynomial. However, we need a master \emph{secret key to generate ciphertexts for all input-attributes}.

\paragraph{Broadcast encryption.}
Broadcast encryption~\cite{C:FiaNao93} enables a sender to generate ciphertexts intended for a specific subset of users.
Only the designated users can decrypt the ciphertexts, while even if all other users collude, they cannot recover the message. A key performance metric for broadcast encryption is the size of the public key and ciphertexts.
Several works have proposed optimal broadcast encryption schemes, where these sizes are $\poly(\log{N})$ and $N$ is the total number of users.
The constructions by Agrawal and Yamada~\cite{EC:AgrYam20} and Agrawal, Wichs, and Yamada~\cite{TCC:AgrWicYam20} rely on the LWE assumption and \emph{pairings}, which are not post-quantum secure. The construction by Wee~\cite{EC:Wee22} relies on the \emph{evasive LWE assumption}, which is a non-falsifiable assumption and has known counterexamples~\cite{AC:BrzUnaWoo24}. The construction by Brakerski and Vaikuntanathan~\cite{ITCS:BraVai22} relies on \emph{heuristics} and lacks a reduction-based proof.

While broadcast encryption is particularly well-suited for streaming services, PKE-CR-SKL can also be applied in this domain. Each approach has its own advantages and limitations. One advantage of broadcast encryption is that it allows the sender to specify recipients at the encryption phase, whereas PKE-CR-SKL does not.
However, PKE-CR-SKL offers three notable advantages over optimal broadcast encryption.
\begin{enumerate}
    \item \textbf{Efficient revocation:} Since decryption keys in broadcast encryption are classical, the system requires maintaining a revocation list, which senders use to revoke users, whereas PKE-CR-SKL eliminates the need for such lists.
    \item \textbf{Seamless user expansion:} In broadcast encryption, the user set is fixed during the setup phase, meaning that adding new users requires updating the encryption key. In contrast, PKE-CR-SKL allows new users to be added without requiring any key updates.
    \item \textbf{Weaker cryptographic assumptions:} Our PKE-CR-SKL scheme is based on the standard LWE assumption, whereas post-quantum secure optimal broadcast encryption relies on the evasive LWE assumption, which has counterexamples~\cite{AC:BrzUnaWoo24}.
\end{enumerate}
In terms of  asymptotic efficiency, both optimal broadcast encryption and PKE-CR-SKL achieve the same public key and ciphertext sizes.

\ifnum\submission=1
We discuss more related works in~\cref{sec:other-rel}.
\else
\section{Additional Related Work}\label{sec:add_related_work}

Mixture of Experts (MoE) and Memory Layers  are two well-established methods for scaling language models within a fixed FLOPS budget, as discussed below.

\paragraph{Mixture of Experts} MoE   replaces traditional feedforward layers with parallel `expert' layers, activating only one (or a few) per token via a lightweight router \citep{shazeer2017outrageously,lepikhin2021gshard,fedus2022switch,jiang2024mixtral,he2024mixture}. This allows scaling by increasing the number of experts without increasing active experts per token. However, all experts must reside on the accelerator, leading to higher memory usage.

\paragraph{Memory Layers} Memory layers store large sets of embeddings (continuous vectors) and retrieve nearest-neighbor embeddings during the forward pass via (approximate) similarity search \citep{weston2014memory,sukhbaatar2015end,lample2019large,berges2024memory}. These retrieved embeddings contribute to computations without adding to FLOPS. Despite advancements in similarity search \citep{lample2019large,johnson2019billion}, memory layers still need to reside on accelerators, which increases memory demands to impractical levels at larger scales \citep{berges2024memory}. Moreover, the embeddings in memory layers are typically updated through backpropagation, which also introduces sparse update challenges as memory scales.

Our approach focuses on input embedding layers, which we demonstrate can be efficiently offloaded from accelerators. This ensures constant memory usage and fixed FLOPS on the accelerator during inference.  Additionally, by modifying only the input embedding layer, our method integrates seamlessly with both MoE and memory layer techniques.






\fi



% \begin{table*}[!t]
% \setlength\tabcolsep{0.5eM}
% \begin{center}
% \begin{minipage}[c]{\textwidth} \scriptsize
% \begin{center}
%  \begin{threeparttable}


% \caption{\textcolor{red}{This table will be removed in the paper.}{\scriptsize Comparison of broadcast encryption and PKE-CR-SKL. Let $N$ be the number of users in broadcast encryption. In PKE-CR-SKL, we can think $N=2^\secp$, where $\secp$ is the security parameter.
% }
% %BRC: best range cover, minimum number of subtrees that cover [a; b].
% }
% \label{tbl:comparison_with_BE}
% \begin{tabular}{rcc}\toprule
%  & Optimal broadcast encryption & PKE-CR-SKL\\
% \midrule
% $\abs{\pk}$ & $\poly(\log{N})$ & $\poly(\log{N})$\\
% $\abs{\ct}$ & $\poly(\log{N})$ & $\poly(\log{N})$ \\
% Can specify recipients in $\Enc$ & $\checkmark$ & $\times$ \\
% No revocation list & $\times$ & $\checkmark$  \\
% No key update for dynamic join & $\times$ & $\checkmark$ \\
% From standard assumptions & $\times$ & $\checkmark$ \\
% \bottomrule
% \end{tabular}
%  % \begin{tablenotes}[flushleft,online,normal] %(default:normal)
%  % \item[a] bCR denotes bounded collusion-resistant. 
%  % \end{tablenotes}
%  \end{threeparttable}
% \end{center}
% \end{minipage}
% \end{center}
% \end{table*}




%\nikhil{I think we don't need this section if we have the index. However, the index takes up more space.}
%\ryo{We need this section in the submission version.}
\ifnum\submission=1