% !TEX root = main.tex

\section{Construction of SKFE-CR-SKL with Key Testability}\label{sec:SKFESKL-KT}

\subsection{Construction}\label{sec:SKFECRSKL-KT-construction}
We construct an SKFE-CR-SKL scheme for the functionality $F:\cX\times\cY\ra\cZ$ with key testability $\SKFESKL=
\SKFESKL.(\Setup,\qKG,\Enc,\qDec,\allowbreak\qVrfy)$ having the
additional algorithms $\SKFESKL.(\CDec, \KeyTest)$, using the
following building blocks.

\begin{itemize}
\item BB84-based SKE-CD scheme (Definition \ref{def:bb84}) $\SKECD =
\SKECD.(\KG,\qEnc,\qDec,\allowbreak \qDel,\Vrfy)$ having the classical
decryption algorithm $\SKECD.\CDec$.

\item Classical SKFE scheme $\SKFE=\SKFE.(\Setup,\KG,\Enc,\Dec)$ for the functionality $F:\cX\times\cY\ra\cZ$.

\item OWF $f:\bit^\secp\ra\bit^{p(\secp)}$ for some polynomial $p$.
\end{itemize}

The construction is as follows:

\begin{description}

\item[$\SKECRSKL.\Setup(1^\secp)$:] $ $
\begin{enumerate}
    \item Generate $\skecd.\sk\gets\SKECD.\KG(1^\secp)$.
    \item Generate $\skfe.\msk\gets\SKFE.\Setup(1^\secp)$.
%    \item Generate $(\sig.\vk,\sig.\sk)\gets\SIG.\KG(1^\secp)$.
    \item Output $\msk\seteq(\skecd.\sk, \skfe.\msk)$.
\end{enumerate}

\item[$\SKFESKL.\qKG(\msk,y)$:] $ $
\begin{enumerate}
    \item Parse $\msk=(\skecd.\sk, \skfe.\msk)$.
    \item Generate $\skfe.\sk_y \gets \SKFE.\KG(\skfe.\msk,y)$.
    \item Generate
        $(\skecd.\qct,\skecd.\vk)\gets\SKECD.\qEnc(\skecd.\sk,\skfe.\sk_y)$.
        Recall that $\skecd.\vk$ is of the form
        $(x,\theta)\in\bit^{\ctlen}\times\bit^{\ctlen}$, and
        $\skecd.\qct$ is of the form
        $\ket{\psi_1}_{\qreg{SKECD.CT_1}}\tensor\cdots\tensor\ket{\psi_{\ctlen}}_{\qreg{SKECD.CT_{\ctlen}}}$.

    \item Generate $s_{i,b}\la\bit^\secp$ and compute $t_{i,b}\la f(s_{i,b})$ for every $i\in[\ctlen]$ and $b\in\bit$. 
    Set $T\seteq
    t_{1,0}\|t_{1,1}\|\cdots\|t_{\ctlen,0}\|t_{\ctlen,1}$ and $S =
    \{s_{i,0} \xor s_{i, 1}\}_{i \in [\ctlen] \; : \;\theta[i] =
    1}$.
    \item Prepare a register $\qreg{S_i}$ that is initialized to
    $\ket{0^\secp}_{\qreg{S_i}}$ for every $i\in[\ctlen]$. 
    \item For every $i\in[\ctlen]$, apply the map
    \begin{align}
    \ket{u_i}_{\qreg{SKECD.CT_i}}\tensor\ket{v_i}_{\qreg{S_i}}
    \ra
    \ket{u_i}_{\qreg{SKECD.CT_i}}\tensor\ket{v_i\oplus s_{i,u_i}}_{\qreg{S_i}}
    \end{align}
    to the registers $\qreg{SKECD.CT_i}$ and $\qreg{S_i}$ and obtain the resulting state $\rho_i$.
    \item Output $\qsk_y = (\rho_i)_{i\in{[\ctlen]}}$,
    $\vk=(x,\theta,S)$, and $\tk=T$.
\end{enumerate}

\item[$\SKFESKL.\Enc(\msk, x)$:] $ $
\begin{enumerate}
    \item Parse $\msk = (\skecd.\sk, \skfe.\msk)$.
    \item Generate $\skfe.\ct\gets\SKFE.\Enc(\skfe.\msk,x)$.
    \item Output $\ct\seteq(\skecd.\sk, \skfe.\ct)$.
    
\end{enumerate}

\item[$\SKFESKL.\CDec(\sk, \ct)$:] $ $
\begin{enumerate}
\item Parse $\ct = (\skecd.\sk, \skfe.\ct)$. Parse $\sk$ as a string over
the registers $\qreg{SKECD.CT} = \qreg{SKECD.CT_1} \otimes \cdots
\otimes \qreg{SKECD.CT_{\ctlen}}$ and $\qreg{S} =
\qreg{S_1} \otimes \cdots \otimes \qreg{S_{\ctlen}}$. Let
$\widetilde{\sk}$ be the sub-string of $\sk$ on register
$\qreg{SKECD.CT}$.
\item Compute $\skfe.\sk \gets \SKECD.\CDec(\skecd.\sk, \widetilde{\sk})$.
\item Output $z\gets\SKFE.\Dec(\skfe.\sk,\skfe.\ct)$.
\end{enumerate}

\item[$\SKFESKL.\qDec(\qsk, \ct)$:] $ $
\begin{enumerate}
\item Parse $(\rho_i)_{i \in [\ctlen]}$. We denote the register
holding $\rho_i$ as $\qreg{SKECD.CT_i}\tensor\qreg{S_i}$ for
every $i\in[\ctlen]$.

\item Prepare a register $\qreg{MSG}$ of $\msglen$ qubits that is
initialized to $\ket{0\cdots0}_{\qreg{MSG}}$.

\item Apply the map
\begin{align}
\ket{u}_{\bigotimes_{i\in[\ctlen]}\qreg{SKECD.
CT_i}} \tensor\ket{w}_{\qreg{MSG}} \ra
\ket{u}_{\bigotimes_{i\in[\ctlen]}\qreg{SKECD.CT_i}}\tensor\ket{w\oplus
\SKFESKL.\CDec(u, \ct)}_{\qreg{MSG}}
\end{align}

to the registers
$\bigotimes_{i\in[\ctlen]}\qreg{SKECD.CT_i}$ and
$\qreg{MSG}$.
\item Measure $\qreg{MSG}$ in the computational basis and output the result $\msg^\prime$.
\end{enumerate}

\item[$\SKFESKL.\qVrfy(\vk,\qsk)$:] $ $
\begin{enumerate}
\item Parse $\vk = (x,\theta,S=\{s_{i,0} \xor
    s_{i,1}\}_{i\in[\ctlen]\; : \; \theta[i]=1})$ and
    $\qsk = (\rho_i)_{i\in[\ctlen]}$ where $\rho_i$ is a state on
    the registers $\qreg{SKECD.CT_i}$ and $\qreg{S_i}$.
\item For every $i \in [\ctlen]$, measure $\rho_i$ in the Hadamard
basis to get outcomes $c_i, d_i$ corresponding to 
the registers $\qreg{SKECD.CT_i}$ and $\qreg{S_i}$ respectively.

\item Output $\top$ if $x[i]=c_i \oplus d_i\cdot(s_{i,0}\oplus
        s_{i,1})$ holds for every $i\in[\ctlen]$ such that $\theta[i]=1$.
    Otherwise, output $\bot$.
\end{enumerate}

\item[$\SKFESKL.\KeyTest(\tk,\sk)$:] $ $
\begin{enumerate}
\item Parse $\sk$ as a string over the registers
$\qreg{SKECD.CT} = \qreg{SKECD.CT_1} \otimes \cdots \otimes
\qreg{SKECD.CT_{\ctlen}}$ and $\qreg{S} = \qreg{S_1} \otimes
\cdots \otimes \qreg{S_{\ctlen}}$.
Let $u_i$ denote the value on
$\qreg{SKECD.CT_i}$ and $v_i$ the value on $\qreg{S_i}$. Parse $\tk$
as $T=t_{1,0}\|t_{1,1}\|\cdots \|t_{\ctlen,0}\|t_{\ctlen,1}$.


\item Let $\Check[t_{i,0},t_{i_1}](u_i,v_i)$ be the deterministic
algorithm that outputs $1$ if $f(v_i)=t_{i,u_i}$ holds and $0$
otherwise.

\item Output $\Check[t_{1,0},t_{1,1}](u_1,v_1) \land
\Check[t_{2,0},t_{2,1}](u_2,v_2) \land \cdots \land
\Check[t_{\ctlen,0},t_{\ctlen,1}](u_{\ctlen},v_{\ctlen})$.
\end{enumerate}

\end{description}

\subsection{Proof of Selective
Single-Ciphertext KLA Security}\label{proof:sel-1ct-kla}
Let $\qA$ be an adversary for the selective
single-ciphertext KLA security of the
construction $\SKFESKL$ that makes use of a BB84-based
SKE-CD scheme $\SKECD$. Consider the hybrid $\Hyb_j^\coin$ defined
as  follows:

\begin{description}
\item[$\hybi{j}^\coin$:] $ $
\begin{enumerate}
\item Initialized with $1^\secp$, $\qA$ outputs $(x_0^*, x_1^*)$.
Sample $\msk \gets \SKFESKL.\Setup(1^\secp)$.

\item $\qA$ can get access to the following (stateful) oracles,
where the list $\List{\qKG}$ used by the oracles is initialized
to an empty list:

\begin{description}
\item[$\Oracle{\qKG}(y)$:] Given $y$, it finds an entry of the form
$(y,\vk,V)$ from $\List{\qKG}$. If there is such an entry, it
returns $\bot$. Otherwise it proceeds as follows:
\begin{enumerate}[(i)]
\item 
If this is the $k$-th query for $k \le j$
and $F(x_0^*, y) \neq F(x_1^*, y)$, then compute $(\qsk,
\vk, \tk) \gets \qKGt(\msk, y)$ where $\qKGt$ is defined below.
Otherwise, compute $(\qsk, \vk, \tk) \gets \qKG(\msk, y)$.

\item It sends $\qsk$ and $\tk$ to $\qA$ and adds $(y, \vk, \bot)$
to $L_{\qKG}$.
\end{enumerate}

\item[$\Oracle{\qVrfy}(y,\widetilde{\qsk})$:] Given
$(y,\widetilde{\qsk})$, it finds an entry $(y,\vk,V)$ from
$\List{\qKG}$. (If there is no such entry, it returns $\bot$.) It
then runs $\decision \gets \qVrfy(\vk,\widetilde{\qsk})$ and returns
$\decision$ to $\qA$. If $V=\top$, it updates the entry into
$(y,\vk,\decision)$. 
\end{description}

\item[$\qKGt(\msk)$:] Differences from $\qKG$ are colored in red:
\begin{enumerate}
    \item Parse $\msk=(\skecd.\sk, \skfe.\msk)$.
    \item \textcolor{red}{Generate $r \gets \bit^\secp$.}
    \item Generate
        $(\skecd.\qct,\skecd.\vk)\gets\SKECD.\qEnc(\skecd.\sk,\textcolor{red}{r})$.
        $\skecd.\vk$ is of the form
        $(x,\theta)\in\bit^{\ctlen}\times\bit^{\ctlen}$, and
        $\skecd.\qct$ is of the form
        $\ket{\psi_1}_{\qreg{SKECD.CT_1}}\tensor\cdots\tensor\ket{\psi_{\ctlen}}_{\qreg{SKECD.CT_{\ctlen}}}$.

    \item Generate $s_{i,b}\la\bit^\secp$ and compute $t_{i,b}\la f(s_{i,b})$ for every $i\in[\ctlen]$ and $b\in\bit$. 
    Set $T\seteq
    t_{1,0}\|t_{1,1}\|\cdots\|t_{\ctlen,0}\|t_{\ctlen,1}$ and $S =
    \{s_{i,0} \xor s_{i, 1}\}_{i \in [\ctlen] \; : \;\theta[i] =
    1}$.
    \item Prepare a register $\qreg{S_i}$ that is initialized to
    $\ket{0^\secp}_{\qreg{S_i}}$ for every $i\in[\ctlen]$. 
    \item For every $i\in[\ctlen]$, apply the map
    \begin{align}
    \ket{u_i}_{\qreg{SKECD.CT_i}}\tensor\ket{v_i}_{\qreg{S_i}}
    \ra
    \ket{u_i}_{\qreg{SKECD.CT_i}}\tensor\ket{v_i\oplus s_{i,u_i}}_{\qreg{S_i}}
    \end{align}
    to the registers $\qreg{SKECD.CT_i}$ and $\qreg{S_i}$ and obtain the resulting state $\rho_i$.
    \item Output $\qsk_y = (\rho_i)_{i\in{[\ctlen]}}$,
    $\vk=(x,\theta,S)$, and $\tk=T$.
\end{enumerate}

\item If there exists
an entry $(y,\vk,V)$ in $\List{\qKG}$ such that $F(x_0^*,y)\ne
F(x_1^*,y)$ and $V=\bot$, output $0$. Otherwise, generate
$\ct^*\la\Enc(\msk,x_\coin^*)$ and send $\ct^*$ to $\qA$.

\item $\qA$ continues to make queries to $\Oracle{\qKG}$. However, $\qA$ is not allowed to send $y$ such that $F(x_0^*,y)\ne F(x_1^*,y)$ to $\Oracle{\qKG}$.

\item $\qA$ outputs a guess $\coin^\prime$ for $\coin$. Output
$\coin'$.
\end{enumerate}
\end{description}

%Let $\Adv_\qA[\Hyb_j] \seteq \Big|\Pr[\Hyb_j^0 \ra 1] -
%\Pr[\Hyb_j^1 \ra 1]\Big|$.
Let $\qA$ make $q = \poly(\secp)$ many queries to $\Oracle{\qKG}$
before the challenge phase. We will now prove the following lemma:

\begin{lemma}
$\forall j \in \{0, \ldots, q-1\}$ and $\coin \in \bit: \Hyb_j^\coin
\approx_c \Hyb_{j+1}^\coin$.
\end{lemma}
\begin{proof}
Suppose $\Hyb_j^\coin \not \approx \Hyb_{j+1}^\coin$. Let $\qD$ be
a corresponding distinguisher. We will
construct a reduction $\qR$ that breaks the IND-CVA-CD security of
the BB84-based SKE-CD scheme $\SKECD$. The execution of $\qR^\qD$ in
the experiment $\expc{\SKECD,\qR}{ind}{cva}{cd}(1^\secp,b)$ proceeds
as follows:

\begin{description}
\item Execution of $\qR^\qD$ in
$\expc{\SKECD,\qR}{ind}{cva}{cd}(1^\secp,b)$:

\begin{enumerate}
\item The challenger $\qCh$ computes $\skecd.\sk \leftarrow
\SKECD.\KG(1^\lambda)$.
\item $\qR$ initializes $\qD$ with $1^\secp$ which outputs $(x_0^*,
x_1^*)$.
\item $\qR$ samples $\skfe.\msk \gets \SKFE.\Setup(1^\secp)$. It initializes a list
$L_{\qKG}$ to an empty list.
\item $\qR$ simulates the oracle $\Oracle{\qKG}$ for $\qD$ as follows:
\begin{description}
\item $\Oracle{\qKG}(y):$ Given $y$, it finds an entry of the form
$(y, \vk, V)$ from $L_{\qKG}$. If there is such an entry, it returns
$\bot$. Otherwise, it proceeds as follows for the $k$-th query:
\begin{enumerate}
\item If $k \le j$ and $F(x_0^*, y)
\neq F(x_1^*, y)$, then compute $(\qsk, \vk, \tk)$ as in
$\widetilde{\qKG}$ except that the pair $(\skecd.\qct, \skecd.\vk)$ is
obtained as the output of $\Oracle{\qEnc}(r)$ instead. Send $\qsk$ and
$\tk$ to $\qA$ and add $(y, \vk, \bot)$ to $L_\qKG$.

\item If $k=j+1$ and $F(x_0^*, y) \neq F(x_1^*, y)$,  compute
$\skfe.\sk_y \gets \SKFE.\KG(\skfe.\msk, y)$. Then, send $(r^*,
\skfe.\sk_y)$ to $\qCh$ where $r^*$ is a random value. On receiving
$\skecd.\ct^*$ from $\qCh$, compute $\qsk$ and $\tk$ as in $\qKG$,
except that $\skecd.\ct^*$ is used in place of $\skecd.\ct$.
Send $\qsk$ and $\tk$ to $\qA$ and add $(y, 0^\secp, \bot)$ to
$L_{\qKG}$.

\item If $k > j+1$ or $F(x_0^*, y) = F(x_1^*, y)$, then compute
$(\qsk, \vk, \tk)$ as in $\qKG$ except that the pair
$(\skecd.\qct, \skecd.\vk)$ is replaced with the output of
$\Oracle{\Enc}(\skfe.\sk_y)$ instead. Send $\qsk$ and $\tk$ to $\qA$
and add $(y, \vk, \bot)$ to $L_{\qKG}$.
\end{enumerate}
\end{description}

\item $\qR$ simulates the oracle $\Oracle{\qVrfy}$ for $\qD$ as
follows:

\begin{description}
\item $\Oracle{\qVrfy}(y, \widetilde{\qsk}):$ Given $(y,
\widetilde{\qsk})$, it finds an entry $(y, \vk, V)$ from $L_\qKG$
(If there is no such entry, it returns $\bot$.) It then proceeds as
follows, if $y$ corresponds to the $k$-th query made to
$\Oracle{\qKG}$ and $k \neq j+1$:

\begin{enumerate}
\item Parse $\vk = (x,\theta,S=\{s_{i,0} \xor
s_{i,1}\}_{i\in[\ctlen]\;:\;\theta[i]=1})$ and $\widetilde{\qsk} =
(\rho_i)_{i\in[\ctlen]}$.

\item For every $i \in [\ctlen]$, measure $\rho_i$ in the Hadamard
basis to get outcomes $c_i, d_i$ corresponding to the registers
$\qreg{SKECD.CT_i}$ and $\qreg{S_i}$ respectively.

\item Compute $\cert[i] = c_i \xor d_i \cdot (s_{i,0} \xor
s_{i,1})$ for every $i \in [\ctlen]$.

\item If $x[i] = \cert[i]$ holds
for every $i \in [\ctlen] : \theta[i] = 1$, then update the entry to
$(y, \vk, \top)$ and send $\top$ to $\qD$. Else, send $\bot$.
\end{enumerate}

If $k = j+1$, then it proceeds as follows:
\begin{enumerate}
\item Compute $\cert = \cert[1] \| \ldots \| \cert[\ctlen]$ where
each $\cert[i]$ is computed as in the previous case.
\item Send $\cert$ to $\qCh$. If $\qCh$ returns $\skecd.\sk$, send
$\top$ to $\qD$ and update the corresponding entry to $(y, \vk,
\top)$. Else, output $\bot$.
\end{enumerate}
\end{description}

\item $\qD$ requests the challenge ciphertext. If there exists
an entry $(y,\vk,V)$ in $\List{\qKG}$ such that $F(x_0^*,y)\ne
F(x_1^*,y)$ and $V=\bot$, output $0$. Otherwise, compute and send $\ct^\star =
(\skecd.\sk, \SKFE.\Enc(\skfe.\msk, x^*_\coin))$ to $\qD$.

\item $\qD$ continues to make queries to $\Oracle{\qKG}$. However, $\qA$ is not allowed to send $y$ such that $F(x_0^*,y)\ne F(x_1^*,y)$ to $\Oracle{\qKG}$.
Consequently, $\qR$ simulates these queries as per Step 4. (c) above.

\item $\qD$ outputs a guess $\coin^\prime$ for $\coin$ which $\qR$
forwards to $\qCh$. $\qCh$ outputs $\coin^\prime$ as the output of the
experiment.
\end{enumerate}
\end{description}

We will first argue that when $b=1$, the view of $\qD$ is exactly the
same as its view in the hybrid $\Hyb_j^\coin$. Notice that the for the
first $j$ queries, if $F(x_0^*, y) \neq F(x_1^*, y)$ for a key-query
corresponding to $y$, then the decryption key is computed by querying
the encryption oracle on a random plaintext. If this condition does
not hold, then the key is computed by querying the encryption oracle
on the corresponding SKFE key $\skfe.\sk_y$. The hybrid $\Hyb_j$ on
the other hand, directly computes these values, but there is no
difference in the distribution of the output ciphertexts. A similar
argument holds for the keys $\qdk_{j+2}, \ldots, \qdk_q$, which
contain encryptions of the corresponding keys $\skfe.\sk_y$. Note
that if $F(x_0^*, y^*) = F(x_1^*, y^*)$, where $y^*$ corresponds to
the $j+1$-th query, then the hybrids $\Hyb_j^\coin$ and
$\Hyb_{j+1}^\coin$ are identical. Hence, consider the case when
$F(x_0^*, y^*) \neq F(x_1^*, y^*)$. In this case, if $b=1$,
the value encrypted as part of the key $\qsk_{j+1}$ is
$\skfe.\sk_{y^*}$. This is the same as in $\Hyb_j^\coin$. As for the
verification oracle queries, notice that they are answered similarly
by the reduction and $\Hyb_j^\coin$ for all but the $j+1$-th key. For
the $j+1$-th key, the reduction works differently in that it forwards
the certificate $\cert$ to the verification oracle.  However, the
verification procedure of the BB84-based SKE-CD scheme checks the
validity of the value $\cert$ in the same way as the reduction, so
there is no difference.
Finally, notice that when $b=0$, the encrypted value is an independent
and random value, similar to the hybrid $\Hyb_{j+1}^\coin$.
Consequently, $\qR$ breaks the IND-CVA-CD security of $\SKECD$ with
non-negligible probability, a contradiction.
\end{proof}

Notice now that the hybrid $\Hyb_0^\coin$ is the same as the
experiment $\expc{\SKECRSKL,\qA}{sel}{1ct}{kla}\allowbreak(1^\secp,\coin)$. From
the previous lemma, we have that $\Hyb_0^\coin \approx_c
\Hyb_q^\coin$. However, we have that $\Hyb_q^0 \approx_c \Hyb_q^1$
holds from the selective single-ciphertext security of the underlying
SKFE scheme $\SKFE$. This is because any key-query corresponding to
$y$ after the $q$-th query is such that $F(x_0^*, y) = F(x_1^*, y)$.
For the first $q$ queries, wherever this condition doesn't hold, the
SKFE keys have been replaced with random values. Consequently, we have
that $\Hyb_0^0 \approx_c \Hyb_0^1$, which completes the proof. \qed

\subsection{Proof of Key-Testability}\label{proof:kt_SKFE}
First, we will argue the correctness requirement. Recall that
$\SKFESKL.\qKG$ applies the following map to a BB84 state
$\ket{x}_\theta$, where $(x, \theta) \in
\bit^{\ctlen}\times\bit^{\ctlen}$, for every $i \in [\ctlen]$:

\begin{align}
\ket{u_i}_{\qreg{SKECD.CT_i}}\tensor\ket{v_i}_{\qreg{S_i}}
\ra
\ket{u_i}_{\qreg{SKECD.CT_i}}\tensor\ket{v_i\oplus
s_{i,u_i}}_{\qreg{S_i}}
\end{align}
where $\qreg{SKECD.CT_i}$ denotes the register holding the $i$-th
qubit of $\ket{x}_\theta$ and $\qreg{S_i}$ is a register initialized
to $\ket{0\ldots0}_{\qreg{S_i}}$.

Consider applying the algorithm $\SKFESKL.\KeyTest$ in
superposition to the resulting state, i.e., performing the following
map, where $\qreg{\SKECD.CT} = \qreg{\SKECD.CT_1} \otimes \cdots
\otimes \qreg{\SKECD.CT_{\ctlen}}$ and $\qreg{S} =
\qreg{S_1} \otimes \cdots \otimes \qreg{S_{\ctlen}}$, and
$\qreg{KT}$ is initialized to $\ket{0}$:

\begin{align}
\ket{u}_{\qreg{SKECD.
CT}}\tensor\ket{v}_{\qreg{S}}\tensor\ket{\beta}_{\qreg{KT}} \ra
\ket{u}_{\qreg{SKECD.
CT}}\tensor\ket{v}_{\qreg{S}}\tensor\ket{\beta\oplus\SKFESKL.
\KeyTest(\tk, u\|v)}_{\qreg{KT}}
\end{align}

where $\tk = T = t_{1, 0}\|t_{1, 1} \| \cdots \| t_{\ctlen,
0}\|t_{\ctlen, 1}$. Recall that $\SKFESKL.\KeyTest$
outputs 1 if
and only if $\Check[t_{i,0}, t_{i, 1}](u_i,
v_i) = 1$ for every $i
\in [\ctlen]$, where $u_i, v_i$ denote the states of the registers
$\qreg{SKECD.CT_i}$ and $\qreg{S_i}$ respectively.
Recall that $\Check[t_{i,0},t_{i,1}](u_i, v_i)$ computes $f(v_i)$
and checks if it equals $t_{i, u_i}$. Since the construction chooses
$t_{i, u_i}$ such that $f(s_{i, u_i}) = t_{i, u_i}$, this check
always passes. Consequently, measuring register $\qreg{KT}$ always
produces outcome $1$.

We will now argue that the security requirement holds by showing the
following reduction to the security of the OWF $f$. Let $\qA$ be an
adversary that breaks the key-testability of $\SKFESKL$. Consider
a QPT reduction $\qR$ that works as follows in the OWF security
experiment:

\begin{description}
\item Execution of $\qR^\qA$ in
$\expa{f,\qR}{owf}(1^\secp)$:

\begin{enumerate}
\item The challenger chooses $s \leftarrow \bit^\lambda$ and sends
$y^* = f(s)$ to $\qR$.

\item $\qR$ runs $\SKFESKL.\Setup(1^\secp)$ and initializes $\qA$
with input $\msk$.

\item $\qR$ picks a random $k^* \in [q]$.

\item $\qR$ simulates the access of $\qA$ to the oracle
$\Oracle{\qKG}$ as follows, where the list $\List{\qKG}$ is
initialized to an empty list:

\begin{description}
\item[$\Oracle{\qKG}(y)$:] For the $k$-th query, do the following:
\begin{enumerate}
\item 
Given $y$, it finds an entry of the form
$(y,\tk)$ from $\List{\qKG}$. If there is such an entry, it
returns $\bot$.

\item
Otherwise, if $k \neq k^*$, it generates
$(\qsk_y,\vk,\tk)\la\qKG(\msk,y)$, sends $(\qsk_y,\vk,\tk)$ to $\qA$,
and adds $(y,\tk)$ to $\List{\qKG}$.

\item Otherwise, if $k = k^*$, it generates $(\qsk_y, \vk, \tk) \gets
\widetilde{\qKG}(\msk, y)$ (differences from $\qKG$ are colored in
\textcolor{red}{red}). It then sends $(\qsk_y, \vk, \tk)$ to $\qA$ and
adds $(y, \tk)$ to $L_{\qKG}$.
\end{enumerate}
\end{description}

\begin{description}
\item $\widetilde{\qKG}(\msk, y)$
\begin{enumerate}
    \item Parse $\msk=(\skecd.\sk, \skfe.\msk)$.
    \item Generate $\skfe.\sk_y \gets \SKFE.\KG(\skfe.\msk,y)$.
    \item Generate
        $(\skecd.\qct,\skecd.\vk)\gets\SKECD.\qEnc(\skecd.\sk,\skfe.\sk_y)$.
        Here, $\skecd.\vk$ is of the form
        $(x,\theta)\in\bit^{\ctlen}\times\bit^{\ctlen}$, and
        $\skecd.\qct$ is of the form
        $\ket{\psi_1}_{\qreg{SKECD.CT_1}}\tensor\cdots\tensor\ket{\psi_{\ctlen}}_{\qreg{SKECD.CT_{\ctlen}}}$.

\item \textcolor{red}{Choose an index $i^\star \in [\ctlen]$ such
that $\theta[i^\star] = 0$. For every $i \in [\ctlen]$ such
that $i \neq i^\star$, generate $s_{i,b}\la\bit^\secp$ and compute
$t_{i,b}\la f(s_{i,b})$ for every $b\in\bit$. For $i = i^\star$,
set $t_{i^\star, 1 - x[i^\star]} = y^*$. Then, generate $s_{i^\star,
x[i^\star]} \leftarrow \bit^\lambda$ and compute $t_{i^\star,
x[i^\star]} = f(s_{i^\star, x[i^\star]})$.}
   Set $T\seteq
    t_{1,0}\|t_{1,1}\|\cdots\|t_{\ctlen,0}\|t_{\ctlen,1}$ and $S =
    \{s_{i,0} \xor s_{i, 1}\}_{i \in [\ctlen] \; : \;\theta[i] =
    1}$.
    \item Prepare a register $\qreg{S_i}$ that is initialized to
    $\ket{0^\secp}_{\qreg{S_i}}$ for every $i\in[\ctlen]$. 
    \item For every $i\in[\ctlen]$, apply the map
    \begin{align}
    \ket{u_i}_{\qreg{SKECD.CT_i}}\tensor\ket{v_i}_{\qreg{S_i}}
    \ra
    \ket{u_i}_{\qreg{SKECD.CT_i}}\tensor\ket{v_i\oplus s_{i,u_i}}_{\qreg{S_i}}
    \end{align}
    to the registers $\qreg{SKECD.CT_i}$ and $\qreg{S_i}$ and obtain the resulting state $\rho_i$.
    \item Output $\qsk_y = (\rho_i)_{i\in{[\ctlen]}}$,
    $\vk=(x,\theta,S)$, and $\tk=T$.
\end{enumerate}
\end{description}

\item $\qA$ sends a tuple of classical strings $(y, \sk, x^*)$ to $\qR$.
$\qR$ outputs $\bot$ if there is no entry of the form $(y,\tk)$ in
$\List{\qKG}$ for some $\tk$. Also, if $k \neq k^*$, $\qR$ outputs $\bot$.
Otherwise, $\qR$ parses $\sk$ 
as a string over the registers $\qreg{SKECD.CT} = \qreg{SKECD.CT_1}
\otimes \cdots \otimes
\qreg{SKECD.CT_{\ctlen}}$ and $\qreg{S} = \qreg{S_1} \otimes
\cdots \otimes \qreg{S_{\ctlen}}$ and measures the register
$\qreg{S_{i^\star}}$ to obtain an outcome $s_{i^\star}$. $\qR$
then sends $s_{i^\star}$ to the challenger.
\end{enumerate}
\end{description}

Notice that the view of $\qA$ is the same as its view in the
key-testability experiment, as only the value $t_{i^\star,
1-x[i^\star]}$ is generated differently by forwarding the value $y$,
but this value is distributed identically to the original value.
Note that in both cases, $\qA$ receives no information about a
pre-image of $t_{i^\star, 1-x[i^\star]}$.
Now, $\qR$ guesses the index $k$ that $\qA$ targets with probability
$\frac1q$. By assumption, we have that $\CDec(\sk, \ct)
\neq F(x^*, y)$ where $\ct = \Enc(\msk, x^*)$. The value $\sk$ can be
parsed as a string over the registers $\qreg{SKECD.CT}$ and
$\qreg{S}$. Let $\widetilde{\sk}$ be the sub-string of $\sk$ on the
register $\qreg{SKECD.CT}$. Recall that $\CDec$ invokes the
algorithm $\SKECD.\CDec$ on input $\widetilde{\sk}$. We will
now recall a property of $\SKECD.\CDec$ that was specified in
Definition \ref{def:bb84}:

Let $(\qct, \vk = (x, \theta)) \gets \SKECD.\Enc(\skecd.\sk,
\skfe.\sk_y)$
where $\skecd.\sk \gets \SKECD.\KG(1^\secp)$. Now, let $u$ be any
arbitrary value such that $u[i] = x[i]$ for all $i : \theta[i] = 0$.
Then, the following holds:

$$\Pr\Big[\SKECD.\CDec(\skecd.\sk, u) = \skfe.\sk_y\Big] \ge 1 -
\negl(\secp)$$

Consequently, if $\widetilde{\sk}$ is such that $\widetilde{\sk}[i]
= x[i]$ for all $i: \theta[i] = 0$, where $(x, \theta)$ are
specified by $\vk_{k^\star}$, then $\SKECD.\CDec(\skecd.\sk,
\widetilde{\sk})$ outputs the value $\skfe.\sk_y$ with high
probability. Since $\CDec(\sk, \ct = (\skecd.\sk, \skfe.\ct =
\SKFE.\Enc\allowbreak(\skfe.\msk, x^*)))$ outputs
$\SKFE.\Dec(\SKECD.\CDec(\skecd.\sk, \widetilde{\sk}), \skfe.\ct)$, we
have that it outputs $x^*$ with high probability from the decryption
correctness of $\SKFE$. Therefore, it must be the case that there
exists some index $i$ for which $\widetilde{\sk} \neq x[i]$. With
probability $\frac{1}{\ctlen}$, this happens to be the guessed value
$i^\star$.  In this case, $\qA$ must output $s_{i^\star}$ on register
$\qreg{S_i}$ such that $f(s_{i^\star}) = t_{i^\star, 1 - x[i^\star]} =
y^*$. This concludes the proof. \qed

Since we have proved selective single-ciphertext security and
key-testability, we can now state the following theorem:

\begin{theorem}
Assuming the existence of a BB84-based SKE-CD scheme and the existence
of OWFs, there exists a selective single-ciphertext KLA secure
SKFE-CR-SKL scheme satisfying the key-testability property.
\end{theorem}

%\begin{align}
%\Pr\left[
%\SKECD.\CDec(\sk, u) = \SKECD.\CDec(\sk, u')
%\ \Big\vert
%\begin{array}{ll}
%\sk\lrun \SKECD.\KG(1^\secp)\\
%\end{array}
%\right] 
%\ge 1-\negl(\secp).
%\end{align}



\begin{comment}
We will now use the fact that for a BB84-based SKE-CD
scheme $\SKECD$, the deterministic algorithm $\SKECD.\CDec$
satisfies the following property. Let $\skecd.\sk$ be a secret key
of $\SKECD$ and $\aux_0$ be generated by $\SKECD.\qEnc(\skecd.\sk,
\cdot)$. Let $u, u' \in \bit^\ctlen$ be any
strings such that for all $i$ such that $\theta[i]=0$, $u[i] =
u[i']$. Then, it must be that $\SKECD.\CDec(\skecd.\sk, u,
\aux_0) = \SKECD.\CDec(\skecd.\sk, u', \aux_0)$.
As a
consequence, for all valid $\ct$, it must also hold that
$\SKECRSKL.\CDec(u \| \aux_0, \ct) = \SKECRSKL.\CDec(u' \|
\aux_0, \ct)$.

Let $\ct^\star = \SKECRSKL.\Enc(\msk, m)$. Now, let $\qA$ succeed in
breaking key-testability for the key with index $k$. Let $x'$ be the
sub-string of $\dk$ over the register $\qreg{SKECD.CT}$. By
assumption, we have that $\SKECRSKL.\CDec(x' \| \aux_0,
\ct^\star) \neq m$, where $\aux_0$ is generated as part of the
generation of $\qdk_k$. Let $(x, \theta, S)$ denote the verification
key for $\qdk_k$.
Since $\SKECRSKL.\CDec$ is applied in superposition based on the
state on register $\qreg{\SKECD.CT}$, from the
Classical Decryption Property of $\SKECRSKL$, there must be some
computational basis state $u^\star$ on register
$\qreg{SKECD.CT}$ corresponding to the decryption key $\qdk_k$ such
that $\SKECRSKL.\CDec(u^\star \| \aux_0, \ct^\star) = m$
holds. Using the aforementioned property, we have that
$\SKECRSKL.\CDec(x \| \aux_0, \ct^\star) = m$. Again from the
aforementioned property and the fact that $\KeyTest$ checks $\aux_0$
is consistent, there must be at least one index $i \in [\ctlen]$
such that $\theta[i] = 0$ and $x'[i] \neq x[i]$.  With probability
$\frac{1}{\ctlen}$, we have that the guessed index $i^\star$ equals
$i$. In this case, for $\qA$ to pass key-testability, it must be the
case that the value $s_{i^\star}$ is a valid pre-image of
$t_{i^\star, 1-x[i^\star]}$ under $f$.  Consequently, $\qR$ breaks
the security of the OWF $f$.

\nikhil{SKECD has a classical part apart from the BB84 state. I
changed this but the inconsistencies need to be fixed.}
\end{comment}
