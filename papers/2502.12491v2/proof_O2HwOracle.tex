% !TEX root = main.tex

\ifnum\submission=1
\section{Proof of O2H Lemma with Auxiliary Quantum Oracle}\label{appsec:O2H_oracle}
\else
\fi
\begin{proof}[Proof of~\cref{lem:O2H}]
Let us consider an adversary $\widetilde{\qA}$ that receives as input
the description $\langle \cQ\rangle$ of $\cQ$\footnote{The O2H Lemma (Lemma \ref{lem:O2Hprev}) holds even if $z$ is exponentially large, so the description of $\cQ$ need not be concise.}, along with the input
$z$ used by $\qA$. Given oracle access to $H$ (likewise $G$) and $(z,
\langle \cQ\rangle)$ as input, $\widetilde{\qA}$ simply runs
$\qA^{H, \cQ}(z)$ (likewise $\qA^{G, \cQ}(z)$) by simulating its queries to $\cQ$
using the description $\langle \cQ\rangle$. Then, the O2H Lemma (Lemma
\ref{lem:O2Hprev}) implies the existence of an algorithm
$\widetilde{\qB}^{H}(z, \langle \cQ\rangle)$ that chooses $i \gets
[q]$, runs $\widetilde{\qA}^H(z, \langle \cQ\rangle)$, measures its
$i$-th query to $H$ and outputs the measurement outcome. Observe that
the algorithms $\widetilde{A}$ and $\widetilde{B}$ do not make use of
the description $\langle \cQ\rangle$ except for simulating the queries
made by $\qA$.  Consequently, there exists an algorithm
$\qB^{\cQ, H}(z)$ equivalent to $\widetilde{\qB}^{H}(z, \langle \cQ\rangle)$ that directly runs $\qA$ (instead of $\widetilde{\qA}$)
and simulates its oracle queries to $\cQ$ using its own access to
$\cQ$.
\end{proof}
