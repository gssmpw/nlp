% !TEX root = main.tex

\section{SKE-CR-SKL with Key Testability}\label{sec:SKECRSKL-KT}
In this section, we show how to achieve SKE-CR-SKL introduced in~\cref{def:ske_cr_skl}.
\subsection{Construction}\label{sec:SKECRSKL-KT-construction}
We construct an SKE-CR-SKL scheme with key testability $\SKECRSKL=
\SKECRSKL.(\Setup,\qKG,\Enc,\qDec,\allowbreak\qVrfy)$ having the
additional algorithms $\CDec$ and $\KeyTest$, using the
following building blocks.

\begin{itemize}
\item BB84-based SKE-CD scheme (Definition \ref{def:bb84}) $\SKECD =
\SKECD.(\KG,\qEnc,\qDec,\allowbreak\qDel,\Vrfy)$ having the classical
decryption algorithm $\SKECD.\CDec$.

\item OWF $f:\bit^\secp\ra\bit^{p(\secp)}$ for some polynomial $p$.
\end{itemize}

Let $\cM \seteq \bit^{\msglen}$ be the plaintext space. The construction is as follows:

\begin{description}

\item[$\SKECRSKL.\Setup(1^\secp)$:] $ $
\begin{enumerate}
    \item Generate $r\gets\bit^{\msglen}$.
    \item Generate $\skecd.\sk\gets\SKECD.\KG(1^\secp)$.
%    \item Generate $(\sig.\vk,\sig.\sk)\gets\SIG.\KG(1^\secp)$.
    \item Output $\msk\seteq(\skecd.\sk, r)$.
\end{enumerate}

\item[$\SKECRSKL.\qKG(\msk)$:] $ $
\begin{enumerate}
    \item Parse $\msk=(\skecd.\sk, r)$.
    \item Generate
        $(\skecd.\qct,\skecd.\vk)\gets\SKECD.\qEnc(\skecd.\sk,r)$.
        $\skecd.\vk$ is of the form
        $(x,\theta)\in\bit^{\ctlen}\times\bit^{\ctlen}$, and
        $\skecd.\qct$ is of the form
        $\ket{\psi_1}_{\qreg{SKECD.CT_1}}\tensor\cdots\tensor\ket{\psi_{\ctlen}}_{\qreg{SKECD.CT_{\ctlen}}}$.

    \item Generate $s_{i,b}\la\bit^\secp$ and compute $t_{i,b}\la f(s_{i,b})$ for every $i\in[\ctlen]$ and $b\in\bit$. 
    Set $T\seteq
    t_{1,0}\|t_{1,1}\|\cdots\|t_{\ctlen,0}\|t_{\ctlen,1}$ and $S =
    \{s_{i,0} \xor s_{i, 1}\}_{i \in [\ctlen] \; : \;\theta[i] =
    1}$.
    \item Prepare a register $\qreg{S_i}$ that is initialized to
    $\ket{0^\secp}_{\qreg{S_i}}$ for every $i\in[\ctlen]$. 
    \item For every $i\in[\ctlen]$, apply the map
    \begin{align}
    \ket{u_i}_{\qreg{SKECD.CT_i}}\tensor\ket{v_i}_{\qreg{S_i}}
    \ra
    \ket{u_i}_{\qreg{SKECD.CT_i}}\tensor\ket{v_i\oplus s_{i,u_i}}_{\qreg{S_i}}
    \end{align}
    to the registers $\qreg{SKECD.CT_i}$ and $\qreg{S_i}$ and obtain the resulting state $\rho_i$.
    \item Output $\qdk = (\rho_i)_{i\in{[\ctlen]}}$,
    $\vk=(x,\theta,S)$, and $\tk=T$.
\end{enumerate}

\item[$\SKECRSKL.\Enc(\msk, \msg)$:] $ $
\begin{enumerate}
    \item Parse $\msk = (\skecd.\sk, r)$.
    \item Output $\ct\seteq(\skecd.\sk, r\oplus\msg)$.
    
\end{enumerate}

\item[$\SKECRSKL.\CDec(\dk, \ct)$:] $ $
\begin{enumerate}
\item Parse $\ct = (\skecd.\sk, z)$. Let
$\widetilde{\dk}$ be the sub-string of $\dk$ on register
$\qreg{SKECD.CT} = \qreg{SKECD.CT_1} \otimes \cdots \otimes \qreg{SKECD.CT_{\ctlen}}$.
\item Output $z \xor \SKECD.\CDec(\skecd.\sk, \widetilde{\dk})$.
\end{enumerate}

\item[$\SKECRSKL.\qDec(\qdk, \ct)$:] $ $
\begin{enumerate}
\item Parse $(\rho_i)_{i \in [\ctlen]}$. We denote the register
holding $\rho_i$ as $\qreg{SKECD.CT_i}\tensor\qreg{S_i}$ for
every $i\in[\ctlen]$.

\item Prepare a register $\qreg{MSG}$ of $\msglen$ qubits that is
initialized to $\ket{0\cdots0}_{\qreg{MSG}}$.

\item Apply the map
\begin{align}
&\ket{u}_{\bigotimes_{i\in[\ctlen]}\qreg{SKECD.
CT_i}} \tensor\ket{w}_{\qreg{MSG}} \ra\\
&\ket{u}_{\bigotimes_{i\in[\ctlen]}\qreg{SKECD.CT_i}}\tensor\ket{w\oplus
\SKECRSKL.\CDec(u, \ct)}_{\qreg{MSG}}
\end{align}

to the registers
$\bigotimes_{i\in[\ctlen]}\qreg{SKECD.CT_i}$ and
$\qreg{MSG}$.
\item Measure $\qreg{MSG}$ in the computational basis and output the result $\msg^\prime$.
\end{enumerate}

\item[$\SKECRSKL.\qVrfy(\vk,\widetilde{\qdk})$:] $ $
\begin{enumerate}
\item Parse $\vk = (x,\theta,S=\{s_{i,0} \xor
    s_{i,1}\}_{i\in[\ctlen]\; : \; \theta[i]=1})$ and
    $\qdk = (\rho_i)_{i\in[\ctlen]}$ where $\rho_i$ is a state on
    the registers $\qreg{SKECD.CT_i}$ and $\qreg{S_i}$.
\item For every $i \in [\ctlen]$, measure $\rho_i$ in the Hadamard
basis to get outcomes $c_i, d_i$ corresponding to 
the registers $\qreg{SKECD.CT_i}$ and $\qreg{S_i}$ respectively.

\item Output $\top$ if $x[i]=c_i \oplus d_i\cdot(s_{i,0}\oplus
        s_{i,1})$ holds for every $i\in[\ctlen]$ such that $\theta[i]=1$.
    Otherwise, output $\bot$.
\end{enumerate}

\item[$\SKECRSKL.\KeyTest(\tk,\dk)$:] $ $
\begin{enumerate}
\item Parse $\dk$ as a string over the registers
$\qreg{SKECD.CT} = \qreg{SKECD.CT_1} \otimes \cdots \otimes
\qreg{SKECD.CT_{\ctlen}}$ and $\qreg{S} = \qreg{S_1} \otimes
\cdots \otimes \qreg{S_{\ctlen}}$.
Let $u_i$ denote the value on
$\qreg{SKECD.CT_i}$ and $v_i$ the value on $\qreg{S_i}$. Parse $\tk$
as $T=t_{1,0}\|t_{1,1}\|\cdots \|t_{\ctlen,0}\|t_{\ctlen,1}$.


\item Let $\Check[t_{i,0},t_{i_1}](u_i,v_i)$ be the deterministic
algorithm that outputs $1$ if $f(v_i)=t_{i,u_i}$ holds and $0$
otherwise.

\item Output $\Check[t_{1,0},t_{1,1}](u_1,v_1) \land
\cdots \land
\Check[t_{\ctlen,0},t_{\ctlen,1}](u_{\ctlen},v_{\ctlen})$.
\end{enumerate}

\end{description}


\paragraph{Decryption correctness:} For a ciphertext $\ct =
(\skecd.\sk, r \xor \msg)$, the decryption algorithm $\qDec$ performs
the following computation:

\begin{align}
&\ket{u}_{\bigotimes_{i\in[\ctlen]}\qreg{SKECD.
CT_i}} \tensor\ket{w}_{\qreg{MSG}} \ra\\
&\ket{u}_{\bigotimes_{i\in[\ctlen]}\qreg{SKECD.CT_i}}\tensor\ket{w
\xor r \xor \msg \xor \SKECD.\CDec(\skecd.\sk, u)}_{\qreg{MSG}}
\end{align}

Recall that $\qreg{SKECD.CT} = \qreg{SKECD.CT_1} \otimes \cdots
\otimes \qreg{SKECD.CT_{\ctlen}}$ is a register corresponding to the
ciphertext of a BB84-based SKE-CD scheme $\SKECD$. Hence, from the Classical
Decryption property of $\SKECD$ (Definition \ref{def:bb84}), it must be that
$\SKECD.\CDec(\skecd.\sk, u) = r$ for every $u$ in the superposition
of $\skecd.\qct$. Consequently, $\msg$ is written onto the
$\qreg{MSG}$ register in each term of the superposition and decryption
correctness follows.


\paragraph{Verification correctness:}

Observe that for the Hadamard basis positions ($i \in [\ctlen]$ such that
$\theta[i] = 1$), $\rho_i$ is of the form:

$$\rho[i] = \ket{0}_{\qreg{SKECD.CT_i}}\ket{s_{i,0}}_{\qreg{S_i}}+
(-1)^{x[i]}\ket{1}_{\qreg{SKECD.CT_i}}\ket{s_{i,1}}_{\qreg{S_i}}$$

It is easy to see that measuring the state in the Hadamard basis gives
outcomes $c_i, d_i$ (on registers $\qreg{SKECD.CT_i}$ and
$\qreg{S_i}$ respectively) satisfying $x[i] = c_i \xor d_i \cdot
(s_{i,0} \xor s_{i,1})$. Hence, the verification correctness follows.

\begin{theorem} There exists an SKE-CR-SKL scheme satisfying
OT-IND-KLA security and Key-Testability, assuming the existence of a
BB84-based SKE-CD scheme and the existence of an OWF.
\end{theorem}

\ifnum\submission=1
We prove this theorem in~\cref{appsec:security_proofs_for_ske_cr_skl}.
\else
We prove this theorem in the subsequent sections.
% !TEX root = main.tex
\ifnum\submission=1
\section{Security Proofs for SKE-CR-SKL}\label{appsec:security_proofs_for_ske_cr_skl}
\fi
\subsection{Proof of OT-IND-KLA Security}\label{proof:ot-ind}\nikhil{Move to Appendix? Commented out to check Pg Limit.}
\ryo{I set the submission flag. If we switch the flag to 1, these proofs are moved to Appendix. See the main.tex.}


Let $\qA$ be an adversary for the OT-IND-KLA security of the
construction $\SKECRSKL$ that makes use of a BB84-based
SKE-CD scheme $\SKECD$. Consider the hybrid $\Hyb_j^\coin$ defined
as  follows:

\begin{description}
\item[$\hybi{j}^\coin$:] $ $
\begin{enumerate}
\item The challenger $\qCh$ runs $\msk \gets
\SKECRSKL.\Setup(1^\secp)$ 
and initializes $\qA$ with input $1^\secp$.

\item $\qA$ requests $q$ decryption keys for some polynomial 
$q$. For each $k \in [j]$, $\qCh$ generates
$(\qdk_i,\vk_i,\tk_i) \gets \qKGt(\msk)$ where $\qKGt$ is defined 
as follows (the difference from $\SKECRSKL.\qKG$ is colored 
in red):

\item[$\qKGt(\msk)$:] $ $
\begin{enumerate}
    \item Parse $\msk=(\skecd.\sk, r)$.
    \item \textcolor{red}{Sample $\widetilde{r} \gets
        \bit^{\msglen}$.}
    \item Generate
$(\skecd.\qct,\skecd.\vk)\gets\SKECD.\qEnc(\skecd.\sk,
\textcolor{red}{\widetilde{r}})$.
$\skecd.\vk$ is of the form
$(x,\theta)\in\bit^{\ctlen}\times\bit^{\ctlen}$, and $\skecd.\qct$
can be described as
$\ket{\psi_1}_{\qreg{SKECD.CT_1}}\tensor\cdots\tensor\ket{\psi_{\ctlen}}_{\qreg{SKECD.CT_{\ctlen}}}$.

    \item Generate $s_{i,b}\la\bit^\secp$ and compute $t_{i,b}\la
        f(s_{i,b})$ for every $i\in[\ctlen]$ and $b\in\bit$. 
    Set $T\seteq t_{1,0}\|t_{1,1}\|\cdots\|t_{\ctlen,0}\|t_{\ctlen,1}$ and $S =
    \{s_{i,0} \xor s_{i, 1}\}_{i \in [\ctlen] \; : \;\theta[i] = 1}$.
    \item Prepare a register $\qreg{S_i}$ that is initialized to $\ket{0\cdots0}_{\qreg{S_i}}$ for every $i\in[\ctlen]$. 
    \item For every $i\in[\ctlen]$, apply the map
    \begin{align}
    \ket{u_i}_{\qreg{SKECD.CT_i}}\tensor\ket{v_i}_{\qreg{S_i}}
    \ra
    \ket{u_i}_{\qreg{SKECD.CT_i}}\tensor\ket{v_i\oplus s_{i,u_i}}_{\qreg{S_1}}
    \end{align}
    to the registers $\qreg{SKECD.CT_i}$ and $\qreg{S_i}$ and obtain the resulting state $\rho_i$.
\item Output $\qdk=(\rho_i)_{i\in[\ctlen]}$, $\vk=(x,\theta,S)$, and $\tk=T$.
\end{enumerate}

\item On the other hand, for $k = j+1, \ldots, q$, $\qCh$ generates
$(\qdk_k, \vk_k,\tk_k) \gets \SKECRSKL.\qKG(\msk)$. Then,
$\qCh$ sends $(\qdk_k,\tk_k)_{k\in[q]}$ to $\qA$.

\item $\qA$ can
get access to the following (stateful) verification oracle
$\Oracle{\qVrfy}$ where $V_i$ is initialized to $\bot$:

\begin{description} \item[
        $\Oracle{\qVrfy}(i, \widetilde{\qdk})$:] It runs $d
        \gets \SKECRSKL.\qVrfy(\vk_i, \widetilde{\qdk})$ and returns $d$.
        If $V_i=\bot$ and $d=\top$, it updates $V_i\seteq
        \top$. 
        %\takashi{changed the way of explaining update to
        %better match that for ABE/FE.} and updates $V\seteq
        %\returned$ if $d=\top$. 
\end{description}

\item $\qA$ sends $(\msg_0^*,\msg_1^*)\in 
\bit^{\msglen} \times \bit^{\msglen}$ to $\qCh$. If $V_i=\bot$ for
some $i\in[q]$, $\qCh$ outputs $0$ as the final output of this
experiment. Otherwise, $\qCh$ generates
$\ct^*\la\SKECRSKL.\Enc(\msk,\msg_\coin^*)$ and sends $\ct^*$ to
$\qA$.

\item $\qA$ outputs a guess $\coin^\prime$ for $\coin$. The 
challenger outputs $\coin'$ as the final output of the 
experiment.
\end{enumerate}
\end{description}

%Let $\Adv_\qA[\Hyb_j] \seteq \Big|\Pr[\Hyb_j^0 \ra 1] -
%\Pr[\Hyb_j^1 \ra 1]\Big|$.
We will now prove the following lemma:

\begin{lemma}
$\forall j \in \{0, \ldots, q-1\}$ and $\coin \in \bit: \Hyb_j^\coin \approx \Hyb_{j+1}^\coin$.
\end{lemma}
\begin{proof}
Suppose $\Hyb_j^\coin \not \approx \Hyb_{j+1}^\coin$. Let $\qD$ be
a corresponding distinguisher. We will
construct a reduction $\qR$ that breaks the IND-CVA-CD security of
the BB84-based SKE-CD scheme $\SKECD$. The execution of $\qR^\qD$ in
the experiment $\expc{\SKECD,\qR}{ind}{cva}{cd}(1^\secp,b)$ proceeds
as follows:

\begin{description}
\item Execution of $\qR^\qD$ in
$\expc{\SKECD,\qR}{ind}{cva}{cd}(1^\secp,b)$:

\begin{enumerate}
\item The challenger $\qCh$ computes $\skecd.\sk \leftarrow
\SKECD.\KG(1^\lambda)$.
\item $\qR$ samples $(r_0, r_1) \leftarrow \bit^{\msglen} \times
    \bit^{\msglen}$ and
sends it to $\qCh$.
\item $\qCh$ computes $(\skecd.\qct^\star, \skecd.\vk^\star)
\leftarrow
\SKECD.\qEnc(\skecd.\sk, r_b)$ and sends $\skecd.\qct^\star$ to $\qR$.
\item $\qR$ initializes $\qD$ with $1^\secp$. $\qD$ requests $q$ keys for some polynomial $q$.

\item For each $k \in [j]$, $\qR$ computes $\qdk_k$ as follows:
\begin{itemize}
\item Sample a random value $\widetilde{r} \gets
    \bit^{\msglen}$.
\item Compute $(\skecd.\widetilde{\qct}, \skecd.\widetilde{\vk}) \leftarrow
    \Oracle{\qEnc}(\widetilde{r})$.
\item Compute $\qdk_k$ by executing Steps 2.(c)-2.(g) as in
$\Hyb_j^0$, but using $\skecd.\widetilde{\qct}$ in place of
$\skecd.\qct$.
\end{itemize}

\item $\qR$ computes $\qdk_{j+1}$ by executing Steps 2.(c)-2.(g) as
in $\Hyb_j^0$, but using $\skecd.\qct^\star$ in place of
$\skecd.\qct$.

\item For each $k \in [j+2,q]$, $\qR$ computes $\qdk_k$ as follows:
\begin{itemize}
\item Compute $(\skecd.\widetilde{\qct}, \skecd.\widetilde{\vk}) \leftarrow
\Oracle{\qEnc}(r_1)$.
\item Compute $\qdk_k$ by executing Steps 2.(c)-2.(g) as in
$\Hyb_j^0$, but using $\skecd.\widetilde{\qct}$ in place of
$\skecd.\qct$.
\end{itemize}
\item $\qR$ sends $\qdk_1, \ldots, \qdk_q$ to $\qD$ and initializes
$V_k = \bot$ for every $k \in [q]$.
\item If $k \neq j+1$, $\qR$ simulates
the response  of oracle $\Oracle{\qVrfy}(k, \widetilde{\qdk})$ as follows:

\begin{itemize}
\item Parse $\widetilde{\vk} =
(x,\theta,S=\{s_{i,0} \xor
s_{i,1}\}_{i\in[\ctlen]\;:\;\theta[i]=1})$ and $\widetilde{\qdk} =
(\rho_i)_{i\in[\ctlen]}$.

\item For every $i \in [\ctlen]$, measure $\rho_i$ in the Hadamard
basis to get outcomes $c_i, d_i$ corresponding to the registers
$\qreg{SKECD.CT_i}$ and $\qreg{S_i}$ respectively.

\item Compute $\cert[i] = c_i \xor d_i \cdot (s_{i,0} \xor
s_{i,1})$ for every $i \in [\ctlen]$.

\item If $x[i] = \cert[i]$ holds
for every $i \in [\ctlen] : \theta[i] = 1$, then update $V_k = \top$
and send $\top$ to $\qD$. Else, send $\bot$.
\end{itemize}

\item If $k = j+1$, $\qR$ simulates
the response  of oracle $\Oracle{\qVrfy}(k, \widetilde{\qdk})$ as
follows:

\begin{itemize}
\item Compute $\cert = \cert[1] \| \ldots \| \cert[\ctlen]$,
where each $\cert[i]$ is computed as in Step 9.

\item Send $\cert$ to $\qCh$. If $\qCh$ returns $\skecd.\sk$, send $\top$ to $\qD$ and update $V_{j+1}= \top$. Else if $\qCh$ returns $\bot$, send $\bot$ to $\qD$.
\end{itemize}

\item $\qD$ sends $(\msg_0^\star, \msg_1^\star) \in \bit^{\msglen} \times
\bit^{\msglen}$ to $\qR$. If $V_i = \bot$ for any $i \in [q]$, $\qR$
sends $0$ to $\qCh$.
$\qR$ computes $\ct^\star = (\skecd.\sk, r_1 \xor \msg^\star_\coin)$,
where $\skecd.\sk$ is obtained from $\qCh$ in Step 10. $\qR$ sends $\ct^*$ to $\qD$.

\item $\qD$ outputs a guess $b'$ which $\qR$ forwards to $\qCh$.
$\qCh$ outputs $b'$ as the final output of the experiment.
\end{enumerate}
\end{description}

We will first argue that when $b=1$, the view of $\qD$ is exactly
the same as its view in the hybrid $\Hyb_j^\coin$. Notice that the
reduction computes the first $j$ decryption keys by querying the
encryption oracle on random plaintexts. $\Hyb_j$ on the other hand,
directly computes them but there is no difference in the
output ciphertexts. A similar argument holds for the keys
$\qdk_{j+2}, \ldots, \qdk_q$, which contain encryptions of the same
random value $r_1$. Moreover, if $b=1$, the value encrypted as part
of the key $\qdk_{j+1}$ is also $r_1$. This is the same as in
$\Hyb_j^\coin$. As for the verification oracle queries, notice that
they are answered similarly by the reduction and $\Hyb_j^\coin$ for
all but the $j+1$-th key. For the $j+1$-th key, the reduction works
differently in that it forwards the certificate $\cert$ to the
verification oracle.  However, the verification procedure of the
BB84-based SKE-CD scheme checks the validity of the value $\cert$ in
the same way as the reduction, so there is no difference.

Finally, notice that when $b=0$, the encrypted value is random and
independent of $r_1$, similar to the hybrid $\Hyb_{j+1}^\coin$.
Consequently, $\qR$ breaks the IND-CVA-CD security of $\SKECD$ with
non-negligible probability, a contradiction.
\end{proof}

Notice now that the hybrid $\Hyb_0^\coin$ is the same as the
experiment $\expc{\SKECRSKL,\qA}{ot}{ind}{kla}\allowbreak(1^\secp,\coin)$. From
the previous lemma, we have that $\Hyb_0^\coin \approx
\Hyb_q^\coin$. However, we have that $\Hyb_q^0 \approx \Hyb_q^1$
because $\Hyb_q^0$ and $\Hyb_q^1$ do not encrypt $r$ at all as part
of the decryption keys, but they mask the plaintext with $r$.
Consequently, we have that $\Hyb_0^0 \approx \Hyb_0^1$, which
completes the proof. \qed



\subsection{Proof of Key-Testability}\label{proof:kt}\nikhil{Move to Appendix? Commented out to check Pg Limit.}

First, we will argue the correctness requirement. Recall that
$\SKECRSKL.\qKG$ applies the following map to a BB84 state
$\ket{x}_\theta$, where $(x, \theta) \in
\bit^{\ctlen}\times\bit^{\ctlen}$, for every $i \in [\ctlen]$:

\begin{align}
\ket{u_i}_{\qreg{SKECD.CT_i}}\tensor\ket{v_i}_{\qreg{S_i}}
\ra
\ket{u_i}_{\qreg{SKECD.CT_i}}\tensor\ket{v_i\oplus
s_{i,u_i}}_{\qreg{S_i}}
\end{align}
where $\qreg{SKECD.CT_i}$ denotes the register holding the $i$-th
qubit of $\ket{x}_\theta$ and $\qreg{S_i}$ is a register initialized
to $\ket{0\ldots0}_{\qreg{S_i}}$.

Consider applying the algorithm $\SKECRSKL.\KeyTest$ in
superposition to the resulting state, i.e., performing the following
map, where $\qreg{\SKECD.CT} = \qreg{\SKECD.CT_1} \otimes \cdots
\otimes \qreg{\SKECD.CT_{\ctlen}}$ and $\qreg{S} =
\qreg{S_1} \otimes \cdots \otimes \qreg{S_{\ctlen}}$, and
$\qreg{KT}$ is initialized to $\ket{0}$:

\begin{align}
\ket{u}_{\qreg{SKECD.
CT}}\tensor\ket{v}_{\qreg{S}}\tensor\ket{\beta}_{\qreg{KT}} \ra
\ket{u}_{\qreg{SKECD.
CT}}\tensor\ket{v}_{\qreg{S}}\tensor\ket{\beta\oplus\SKECRSKL.
\KeyTest(\tk, u\|v)}_{\qreg{KT}}
\end{align}

where $\tk = T = t_{1, 0}\|t_{1, 1} \| \cdots \| t_{\ctlen,
0}\|t_{\ctlen, 1}$. Recall that $\SKECRSKL.\KeyTest$
outputs 1 if
and only if $\Check[t_{i,0}, t_{i, 1}](u_i,
v_i) = 1$ for every $i
\in [\ctlen]$, where $u_i, v_i$ denote the states of the registers
$\qreg{SKECD.CT_i}$ and $\qreg{S_i}$ respectively.
Recall that $\Check[t_{i,0},t_{i,1}](u_i, v_i)$ computes $f(v_i)$
and checks if it equals $t_{i, u_i}$. Since the construction chooses
$t_{i, u_i}$ such that $f(s_{i, u_i}) = t_{i, u_i}$, this check
always passes. Consequently, measuring register $\qreg{KT}$ always
produces outcome $1$.

We will now argue that the security requirement holds by showing the
following reduction to the security of the OWF $f$. Let $\qA$ be an
adversary that breaks the key-testability of $\SKECRSKL$. Consider
a QPT reduction $\qR$ that works as follows in the OWF security
experiment:

\begin{description}
\item Execution of $\qR^\qA$ in
$\expa{f,\qR}{owf}(1^\secp)$:

\begin{enumerate}
\item The challenger chooses $s \leftarrow \bit^\lambda$ and sends
$y \seteq f(s)$ to $\qR$.

\item $\qR$ runs $\SKECRSKL.\Setup(1^\secp)$ and initializes $\qA$
with input $\msk$.

\item $\qA$ requests $q$ decryption keys for some polynomial $q$.
$\qR$ picks a random index $k^\star \in [q]$. For every $k \neq
k^\star$, $\qR$ generates $(\qdk_k, \vk_k, \tk_k)$ by computing the
function $f$ as needed. For the index $k^\star$, $\qR$ computes
$(\qdk_{k^\star}, \vk_{k^\star}, \tk_{k^\star})$ as follows (the
difference is colored in \textcolor{red}{red}):

\begin{enumerate}
\item Parse $\msk=(\skecd.\sk, r)$.
\item Generate
$(\skecd.\qct,\skecd.\vk)\gets\SKECD.\qEnc(\skecd.\sk,r)$.
$\skecd.\vk$ is of the form
$(x,\theta)\in\bit^{\ctlen}\times\bit^{\ctlen}$, 
$\skecd.\qct$ is of the form
$\ket{\psi_1}_{\qreg{SKECD.
CT_1}}\tensor\cdots\tensor\ket{\psi_{\ctlen}}_{\qreg{SKECD.
CT_{\ctlen}}}$.
\item \textcolor{red}{Choose an index $i^\star \in [\ctlen]$ such
that $\theta[i^\star] = 0$. For every $i \in [\ctlen]$ such
that $i \neq i^\star$, generate $s_{i,b}\la\bit^\secp$ and compute
$t_{i,b}\la f(s_{i,b})$ for every $b\in\bit$. For $i = i^\star$,
set $t_{i^\star, 1 - x[i^\star]} \seteq y$. Then, generate $s_{i^\star,
x[i^\star]} \leftarrow \bit^\lambda$ and compute $t_{i^\star,
x[i^\star]} = f(s_{i^\star, x[i^\star]})$.} Set $T\seteq
t_{1,0}\|t_{1,1}\|\cdots\|t_{\ctlen,0}\|t_{\ctlen,1}$ and $S =
\{s_{i,0} \xor s_{i, 1}\}_{i \in [\ctlen] \; : \;\theta[i] = 1}$.

\item Prepare a register $\qreg{S_i}$ that is initialized to
$\ket{0\cdots0}_{\qreg{S_i}}$ for every $i\in[\ctlen]$. 

\item For every $i\in[\ctlen]$, apply the map
\begin{align}
    \ket{u_i}_{\qreg{SKECD.CT_i}}\tensor\ket{v_i}_{\qreg{S_i}}
    \ra
    \ket{u_i}_{\qreg{SKECD.CT_i}}\tensor\ket{v_i\oplus
    s_{i,u_i}}_{\qreg{S_i}}
\end{align}

    to the registers $\qreg{SKECD.CT_i}$ and $\qreg{S_i}$ and obtain the resulting state $\rho_i$.
\item Compute $\qdk_{k^\star}=(\rho_i)_{i\in{[\ctlen]}}$,
$\vk_{k^\star}=(x,\theta,S)$, and $\tk_{k^\star}=T$.
\end{enumerate}

\item $\qR$ sends $(\qdk_i, \vk_i, \tk_i)$ to $\qA$ for every $i \in
[q]$.

\item $\qA$ sends $(k, \dk, \msg)$ to $\qR$. If $k \neq k^\star$, $\qR$
aborts. Otherwise, $\qR$ parses $\dk$ 
as a string over the registers $\qreg{SKECD.CT} = \qreg{SKECD.CT_1}
\otimes \cdots \otimes
\qreg{SKECD.CT_{\ctlen}}$ and $\qreg{S} = \qreg{S_1} \otimes
\cdots \otimes \qreg{S_{\ctlen}}$ and measures the register
$\qreg{S_{i^\star}}$ to obtain an outcome $s_{i^\star}$. $\qR$
then sends $s_{i^\star}$ to the challenger.
\end{enumerate}
\end{description}

Notice that the view of $\qA$ is the same as its view in the
key-testability experiment, as only the value $t_{i^\star,
1-x[i^\star]}$ is generated differently by forwarding the value $y$,
but this value is distributed identically to the original value.
Note that in both cases, $\qA$ receives no information about a
pre-image of $t_{i^\star, 1-x[i^\star]}$.
Now, $\qR$ guesses the index $k$ that $\qA$ targets with probability
$\frac1q$. By assumption, we have that $\CDec(\dk, \ct)
\neq \msg$ where $\ct = \Enc(\msk, \msg)$. The value $\dk$ can be
parsed as a string over the registers $\qreg{SKECD.CT}$ and
$\qreg{S}$. Let $\widetilde{\dk}$ be the sub-string of $\dk$ on the
register $\qreg{SKECD.CT}$. Recall that $\CDec$ invokes the
algorithm $\SKECD.\CDec$ on input $\widetilde{\dk}$. We will
now recall a property of $\SKECD.\CDec$ that was specified in
Definition \ref{def:bb84}:

Let $(\qct, \vk = (x, \theta)) \gets \SKECD.\Enc(\skecd.\sk, r)$
where $\skecd.\sk \gets \SKECD.\KG(1^\secp)$. Now, let $u$ be any
arbitrary value such that $u[i] = x[i]$ for all $i : \theta[i] = 0$.
Then, the following holds:

$$\Pr\Big[\SKECD.\CDec(\skecd.\sk, u) = r\Big] \ge 1 -
\negl(\secp)$$

Consequently, if $\widetilde{\dk}$ is such that $\widetilde{\dk}[i]
= x[i]$ for all $i: \theta[i] = 0$, where $(x, \theta)$ are
specified by $\vk_{k^\star}$, then $\SKECD.\CDec(\skecd.\sk,
\widetilde{\dk})$ outputs the value $r$ with high probability. Since
$\CDec(\dk, \ct = (\skecd.\sk, r \xor \msg))$ outputs $r \xor \msg \xor
\SKECD.\CDec(\skecd.\sk, \widetilde{\dk})$, we have that
$\CDec(\dk, \ct = (\skecd.\sk, r \xor \msg)) = \msg$. Therefore, it must
be the case that there exists some index $i$ for which
$\widetilde{\dk}[i] \neq x[i]$. With probability $\frac{1}{\ctlen}$,
this happens to be the guessed value $i^\star$.  In this case, $\qA$
must output $s_{i^\star}$ on register $\qreg{S_i}$ such that
$f(s_{i^\star}) = t_{i^\star, 1 - x[i^\star]} = y$. This concludes
the proof. \qed

Since we have proved OT-IND-KLA security (Section
\ref{proof:ot-ind}) and Key-Testability (Section
\ref{proof:kt}), we can now state the following theorem:



%\begin{align}
%\Pr\left[
%\SKECD.\CDec(\sk, u) = \SKECD.\CDec(\sk, u')
%\ \Big\vert
%\begin{array}{ll}
%\sk\lrun \SKECD.\KG(1^\secp)\\
%\end{array}
%\right] 
%\ge 1-\negl(\secp).
%\end{align}



\begin{comment}
We will now use the fact that for a BB84-based SKE-CD
scheme $\SKECD$, the deterministic algorithm $\SKECD.\CDec$
satisfies the following property. Let $\skecd.\sk$ be a secret key
of $\SKECD$ and $\aux_0$ be generated by $\SKECD.\qEnc(\skecd.\sk,
\cdot)$. Let $u, u' \in \bit^\ctlen$ be any
strings such that for all $i$ such that $\theta[i]=0$, $u[i] =
u[i']$. Then, it must be that $\SKECD.\CDec(\skecd.\sk, u,
\aux_0) = \SKECD.\CDec(\skecd.\sk, u', \aux_0)$.
As a
consequence, for all valid $\ct$, it must also hold that
$\SKECRSKL.\CDec(u \| \aux_0, \ct) = \SKECRSKL.\CDec(u' \|
\aux_0, \ct)$.

Let $\ct^\star = \SKECRSKL.\Enc(\msk, m)$. Now, let $\qA$ succeed in
breaking key-testability for the key with index $k$. Let $x'$ be the
sub-string of $\dk$ over the register $\qreg{SKECD.CT}$. By
assumption, we have that $\SKECRSKL.\CDec(x' \| \aux_0,
\ct^\star) \neq m$, where $\aux_0$ is generated as part of the
generation of $\qdk_k$. Let $(x, \theta, S)$ denote the verification
key for $\qdk_k$.
Since $\SKECRSKL.\CDec$ is applied in superposition based on the
state on register $\qreg{\SKECD.CT}$, from the
Classical Decryption Property of $\SKECRSKL$, there must be some
computational basis state $u^\star$ on register
$\qreg{SKECD.CT}$ corresponding to the decryption key $\qdk_k$ such
that $\SKECRSKL.\CDec(u^\star \| \aux_0, \ct^\star) = m$
holds. Using the aforementioned property, we have that
$\SKECRSKL.\CDec(x \| \aux_0, \ct^\star) = m$. Again from the
aforementioned property and the fact that $\KeyTest$ checks $\aux_0$
is consistent, there must be at least one index $i \in [\ctlen]$
such that $\theta[i] = 0$ and $x'[i] \neq x[i]$.  With probability
$\frac{1}{\ctlen}$, we have that the guessed index $i^\star$ equals
$i$. In this case, for $\qA$ to pass key-testability, it must be the
case that the value $s_{i^\star}$ is a valid pre-image of
$t_{i^\star, 1-x[i^\star]}$ under $f$.  Consequently, $\qR$ breaks
the security of the OWF $f$.

\nikhil{SKECD has a classical part apart from the BB84 state. I
changed this but the inconsistencies need to be fixed.}
\end{comment}

\fi

