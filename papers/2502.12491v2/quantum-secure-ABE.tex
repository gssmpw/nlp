% !TEX root = main.tex

\section{Quantum Secure ABE for All Relations from LWE}\label{sec-quantum-secure-ABE}

We show there exists quantum selective-secure ABE for all relations computable in polynomial time based on the LWE assumption.

\subsection{Preparation}

\begin{definition}[Quantum-Accessible Pseudo-Random Function]\label{def:prf}
Let $\{\PRF_{k}: \bin^{\ell_1} \ra \allowbreak \bin^{\ell_2} \mid k \in \bin^\secp\}$ be a family of polynomially computable functions, where $\ell_1$ and $\ell_2$ are some polynomials of $\secp$.
We say that $\PRF$ is a quantum-accessible pseudo-random function (QPRF) family if for any QPT adversary $\qA$, it holds that
\begin{align}
\advt{\qA}{prf}(\secp)
= \abs{\Pr[\qA^{\ket{\PRF_{k}(\cdot)}}(1^\secp) \gets 1 \mid k \gets \bit^{\secp}]
-\Pr[\qA^{\ket{\sfR(\cdot)}}(1^\secp) \gets 1 \mid \sfR \gets \cU]
}\leq\negl(\secp),
\end{align}
where $\cU$ is the set of all functions from $\bit^{\ell_1}$ to $\bit^{\ell_2}$. 
\end{definition}

\begin{theorem}[\cite{FOCS:Zhandry12}]\label{thm:qprf}
If there exists a OWF, there exists a QPRF.
\end{theorem}

\subsection{Proofs}

We first define Key-Policy ABE for polynomial size circuits and briefly see that it can be used to instantiate ABE for any relations computable in polynomial time, even under quantum selective-security.  
Then, we prove that the Key-Policy ABE scheme for polynomial size circuits by Boneh et al.~\cite{EC:BGGHNS14} with a light modification using QPRF satisfies quantum selective-security under the LWE assumption.  

\textbf{Key-Policy ABE for Circuits:}
Let $\cX_\secp=\bin^{n(\secp)}$ and $\cY_\secp$ be the set of all circuits with input space $\bin^{n(\secp)}$ and size at most $s(\secp)$, where $n$ and $s$ are some polynomials.
Let $R_\secp$ be the following relation:
$$R_\secp(x,y)=0 \iff y(x) = 0$$
An ABE scheme for such $\{\cX_\secp\}_\secp, \{\cY_\secp\}_\secp$,
and $\{R_\secp\}_\secp$ is referred to as a Key-Policy ABE scheme
for circuits. 

\begin{lemma}\label{lma:kp_to_cp}
If there exists a quantum selective-secure Key-Policy ABE scheme
for circuits, then there exists a quantum selective-secure ABE scheme for all relations computable in polynomial time.
\end{lemma}
\begin{proof}
Let $\cX_\secp\subseteq\bit^n$, $\cY_\secp\subseteq \bit^\ell$, and $R_\secp:\cX_\secp\times\cY_\secp\ra\bit$, where $n$ and $\ell$ are polynomials and $R_\secp$ is computable in polynomial time.
We construct ABE scheme $\ABE = (\Setup,\KG,\Enc,\Dec)$ with attribute spaces $\{\cX_\secp\}$ and $\{\cY_\secp\}$ and relation $\{R_\secp\}$, using Key-Policy ABE scheme for circuits $\ABE' =
(\Setup,\KG',\Enc,\Dec)$ with the following attribute spaces $\{\cX^\prime_\secp\}$ and $\{\cY^\prime_\secp\}$.
\begin{itemize}
\item $\cX^\prime_\secp=\bit^n$.
\item Let $C_{\secp,y}$ be the circuit such that $C_{\secp,y}(x)=R_\secp(x,y)$ for every $x\in\bit^n$ and $y\in\bit^\ell$, and let $s$ be the maximum size of $C_y$. Note that $s$ is a polynomial in $\secp$. Then, $\cY^\prime_\secp$ is the set of all circuits with input length $n$ and size at most $s^\prime$.
\end{itemize}
The scheme is as follows:
$\KG'(\msk, y, r) = \KG(\msk, C_{\secp,y}, r)$.
It is easy to see that the correctness of $\ABE$ follows from that of $\ABE^\prime$. For the
quantum selective security of $\ABE'$, consider a reduction $\qR$
to the quantum selective security of $\ABE$.

\begin{description}
\item Execution of $\qR^{\qA}$ in Experiment $\expc{\ABE,
\qA}{q}{sel}{ind}(1^\secp,\coin)$:
\begin{enumerate}
\item $\qR$ receives challenge attribute $x^* \in \cX_\secp$ from $\qA$ and
forwards it to the challenger $\Ch$.
\item For each oracle query $(\qreg{Y}, \qreg{Z})$ made by $\qA$,
$\qR$ performs the following map on
register $\qreg{Z}$ initialized to $\ket{0^{s(\secp)}}$:
$$\ket{y}_{\qreg{Y}}\ket{z}_{\qreg{Z}} \mapsto
\ket{y}_{\qreg{Y}}\ket{z \xor C_{\secp,y}}_{\qreg{Z}}$$
\item Then, $\qR$ queries the registers $\qreg{Y}, \qreg{Z}$ to
$\Oracle{qkg}$ followed by returning the registers $\qreg{Y},
\qreg{Z}$ to $\qA$.
\item When $\qA$ sends $(\msg_0, \msg_1)$ to $\qR$, $\qR$ forwards it to
$\Ch$.
\item On receiving $\ct^\star \leftarrow \Enc(\pk, x^*, \msg_\coin)$ from
$\Ch$, $\qR$ forwards it to $\qA$.
\item Finally, when $\qA$ outputs a guess $\coin'$, $\qR$ sends
    $\coin'$ to $\Ch$.
\end{enumerate}
\end{description}

Since the view of $\qA$ is identical to that in the
quantum selective security experiment for scheme $\ABE'$, $\qR$ ends
up breaking the quantum selective security of $\ABE$.
\end{proof}


\begin{theorem}
Assuming the polynomial hardness of the
LWE problem, there exists a quantum selective-secure
Key-Policy ABE scheme for circuits.
\end{theorem}
\begin{proof}
We claim that the Key-Policy ABE scheme for circuits by Boneh et al.
\cite{EC:BGGHNS14} based on LWE is quantum selective-secure.
Actually, we will alter the key-generation algorithm of their scheme
as follows: $\KG(\msk, y, k) = \KG'(\msk, y, \PRF_k(y))$ where
$\KG'$ is the key-generation algorithm of their construction with
explicit random coins $\PRF_k(y)$, and $\{\PRF_k\}_k$ is QPRF.
This is a common technique utilized in quantum security proofs (See
for Eg. \cite{EC:BonZha13,C:BonZha13}) that allows one to use a
common random value for every term of a superposition. In the
following discussion, whenever we discuss a hybrid titled
"$\mathsf{Game}\;i$" for some value $i$, it refers to the
corresponding hybrid in Theorem 4.2 of \cite{EC:BGGHNS14}. Also, any
indistinguishability claims between $\mathsf{Game}$ hybrids that are
mentioned to be previously established, will refer to their work.
Consider the following sequence of hybrids for the aforementioned
ABE scheme $\ABE$ and a QPT adversary $\qA$:

\begin{description}
\item[$\hybi{0}^\coin$:] This is the same as the experiment
    $\expc{\ABE,\qA}{q}{sel}{ind}(1^\secp, \coin)$.

%\item[$\hybi{1}^\coin$:] This is the same as $\Hyb_0^\coin$, except
%that the PRF family $\{\PRF_k\}_k$ is replaced with a $2q$-wise
%independent function family $\{f_k\}_k$, where $q$ is an upper bound
%on the number of queries made by $\qA$.

%This switch is
%computationally indistinguishable as shown in the work of Zhandry
%(\cite{C:Zhandry12}).

\item[$\hybi{1}^\coin$:] This is similar to $\Hyb_0^\coin$, except
that the public-key $\pk$ is generated based on the challenge
attribute $x^*$, as in the hybrid $\mathsf{Game}\;1$. It was shown
that the hybrids $\mathsf{Game}\;0$ (the original experiment) and
$\mathsf{Game}\;1$ are statistically indistinguishable. By the same
argument, it follows that $\Hyb_0^\coin \approx_s \Hyb_1^\coin$.

\item[$\hybi{2}^\coin$:] This is similar to $\Hyb_1^\coin$, except
that the public-key is further modified as in $\mathsf{Game}\;2$.
Unlike the change made in $\Hyb_1^\coin$ though, this change
requires the key-queries to be answered differently as in
$\mathsf{Game}\;2$. More specifically, the hybrid now has a
punctured master secret-key that still allows it to simulate
every key-query $y$ such that $C_y(x^*) \neq \bot$. It was shown
previously that $\mathsf{Game}\;1 \approx_s \mathsf{Game}\;2$.
Consider now the intermediate hybrids $\widetilde{\Hyb}_1^\coin,
\widetilde{\Hyb}_2^\coin$ that are similar to $\Hyb_1^\coin,
\Hyb_2^\coin$ respectively, except that all the superposition
key-queries in these hybrids are responded using independent (true)
randomness in every term of the superposition. We will now restate
the following lemma by \cite{EC:BonZha13}, which we will use to
argue that $\widetilde{\Hyb}_1^\coin \approx_s
\widetilde{\Hyb}_2^\coin$.

\begin{lemma}\cite{EC:BonZha13}
Let $\cY$ and $\cZ$ be sets and for each $y \in \cY$, let $D_y$ and
$D'_y$ be distributions on $\cZ$ such that $\SD(D_y, D'_y) \le
\epsilon$.  Let $O:\cY \ra \cZ$ and $O':\cY \ra \cZ$ be functions
such that $O(y)$ outputs $z \gets D_y$ and $O'(y)$ outputs $z' \gets
D'_y$. Then, $O(y)$ and $O'(y)$ are $\epsilon'$-statistically
indistinguishable by quantum algorithms making $q$ superposition
oracle queries, such that $\epsilon' = \sqrt{8C_0q^3\epsilon}$ where
$C_0$ is a constant.
\end{lemma}

Recall that $\cY$ denotes the set of key-attributes. Let us fix a
challenge ciphertext attribute $x^*$ for the following discussion.
For each $y \in \cY$ let the distributions $D^1_y[\pk], D^2_y[\pk]$
correspond to how a key for attribute $y$ is sampled in the hybrids
$\widetilde{\Hyb}^\coin_1, \widetilde{\Hyb}^\coin_2$ respectively,
conditioned on the public key being $\pk$. Note that for each $i \in
[2]$, we consider $D_y^i[\pk]$ to also output the public-key $\pk$
along with the secret-key for $y$. We know that on average over
$\pk$, $D_y^1[\pk] \approx_s D_y^2[\pk]$ holds for all $y \in \cY$.
It now follows from the above lemma that on average over $\pk$, the
analogously defined oracles $O_1[\pk], O_2[\pk]$ are statistically
indistinguishable by algorithms making polynomially many
superposition queries. Observe that with access to oracle $O_i[\pk]$
for each $i \in [2]$, the view of $\qA$ in
$\widetilde{\Hyb}^\coin_i$ (conditioned on the public-key being
$\pk$) can easily
be recreated as the oracle outputs $\pk$, which can then be used to
compute the challenge ciphertext corresponding to $\coin$.
Consequently, it follows that the hybrids $\widetilde{\Hyb}^\coin_1$
and $\widetilde{\Hyb}^\coin_2$ are statistically indistinguishable.
Since $\widetilde{\Hyb}^\coin_i \approx \Hyb^\coin_i$ holds for each
$i\in[2]$ (by the quantum-security of $\PRF$), we have that
$\Hyb^\coin_1 \approx \Hyb^\coin_2$.

\item[$\hybi{3}^\coin$:] This is similar to $\Hyb_2^\coin$, except
that the challenge ciphertext is chosen uniformly at random, as in
$\mathsf{Game}\;3$.

Observe that $\Hyb_3^0 \equiv \Hyb_3^1$. It was shown previously
that $\mathsf{Game}\;3 \approx \mathsf{Game}\;2$ by a reduction to
LWE. Specifically, the reduction prepares the setup based on the LWE
sample and the challenge ciphertext $x^*$, and plants the LWE
challenge in the challenge ciphertext. It is easy to see that a
similar reduction works in our case, thereby showing that
$\Hyb_2^\coin \approx \Hyb_3^\coin$. Consequently, it follows that
$\Hyb_0^0 \approx \Hyb_0^1$.
\end{description}
This completes the proof.
\end{proof}






