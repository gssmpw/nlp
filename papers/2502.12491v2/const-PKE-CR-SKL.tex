% !TEX root = main.tex

\section{PKE-CR-SKL from LWE}\label{sec:PKE-CR-SKL}

In this section, we show how to achieve PKE-CR-SKL from SKE-CR-SKL, standard ABE, and compute-and-compare obfuscation.

\subsection{Construction}
We construct a PKE-CR-SKL scheme
$\PKECRSKL=\PKECRSKL.(\Setup,\qKG,\allowbreak\Enc,\qDec,\qVrfy)$ with
message space $\cM = \bit^{\ell}$ using the following building blocks.

\begin{itemize}
\item ABE scheme $\ABE=\ABE.(\Setup,\KG,\Enc,\Dec)$ for the following relation $R$.
\begin{description}
\item[$R(x,y)$:] Interpret $x$ as a circuit. Then, output $0$ (decryptable) if $\bot=x(y)$ and otherwise $1$.
\end{description}
\item Compute-and-Compare Obfuscation $\CCObf$ with the simulator $\CCSim$.

%\nikhil{Can we rely directly on Predicate Encryption instead of ABE
%+ CC-Obf?}\fuyuki{I do not think so. The way those two are used is different from the way they are used when %constructing PE.}

\item SKE-CR-SKL scheme with Key Testability
$\SKECRSKL=\SKECRSKL.(\Setup,\allowbreak\qKG,\Enc,\qDec,\qVrfy,\KeyTest)$. It also has the classical decryption algorithm $\SKECRSKL.\CDec$.
%\item Quantum-Secure PRF family $\{\PRF_k: \bit^{\poly(\secp)}
%\rightarrow \bit^{\poly(\secp)}\}_{k \in \bit^\secp}$.
\end{itemize} 

The construction is as follows.

\begin{description}
\item[$\PKECRSKL.\Setup(1^\secp)$:] $ $
\begin{itemize}
    \item Generate $(\abe.\pk,\abe.\msk)\gets\ABE.\Setup(1^\secp)$.
    \item Generate $\ske.\msk\gets\SKECRSKL.\Setup(1^\secp)$.
    \item Output $\ek\seteq\abe.\pk$ and $\msk\seteq(\abe.\msk,\ske.\msk)$.
\end{itemize}



\item[$\PKECRSKL.\qKG(\msk)$:] $ $
\begin{itemize}
    \item Parse $\msk=(\abe.\msk,\ske.\msk)$.
    \item Generate $(\ske.\qdk,\ske.\vk,\ske.\tk)\gets\SKECRSKL.\qKG(\ske.\msk)$. We denote the register holding $\ske.\qdk$ as $\qreg{SKE.DK}$.
    \item Prepare a register $\qreg{ABE.SK}$ that is initialized to $\ket{0\cdots0}_{\qreg{ABE.SK}}$.
    \item Choose explicit randomness $\key \leftarrow \bit^{\secp}$.
    \item Apply the map
        $\ket{u}_{\qreg{SKE.DK}}\ket{v}_{\qreg{ABE.SK}}\ra\ket{u}_{\qreg{SKE.DK}}\ket{v\oplus\ABE.\KG(\abe.\msk,u,
        \key)}_{\qreg{ABE.SK}}$ to the registers $\qreg{SKE.DK}$ and
        $\qreg{ABE.SK}$, and obtain $\qdk$ over the registers
        $\qreg{SKE.DK}$ and $\qreg{ABE.SK}$.

    \item Output $\qdk$ and
        $\vk\seteq(\abe.\msk,\ske.\vk,\ske.\tk, \key)$.
\end{itemize}



\item[$\PKECRSKL.\Enc(\ek,\msg)$:] $ $
\begin{itemize}
    \item Parse $\ek=\abe.\pk$.
    \item Generate $\tlC\gets\CCSim(1^\secp,\pp_D,1)$, where $\pp_D$ consists of circuit parameters of $D$ defined in the security proof.
    \item Generate $\abe.\ct\gets\ABE.\Enc(\abe.\pk,\tlC,\msg)$.
    \item Output $\ct\seteq\abe.\ct$.
\end{itemize}
 
\item[$\PKECRSKL.\qDec(\qdk,\ct)$:] $ $
\begin{itemize}
   \item Parse $\ct=\abe.\ct$. We denote the register holding $\qdk$ as $\qreg{SKE.DK}\tensor\qreg{ABE.SK}$.
   \item Prepare a register $\qreg{MSG}$ that is initialized to $\ket{0\cdots0}_{\qreg{MSG}}$
   \item Apply the map $\ket{v}_{\qreg{ABE.SK}}\ket{w}_{\qreg{MSG}}\ra\ket{v}_{\qreg{ABE.SK}}\ket{w\oplus\ABE.\Dec(v,\abe.\ct)}_{\qreg{MSG}}$ to the registers $\qreg{ABE.SK}$ and $\qreg{MSG}$.
   \item Measure the register $\qreg{MSG}$ in the computational basis and output the result $\msg^\prime$.
\end{itemize}


\item[$\PKECRSKL.\qVrfy(\vk,\qdk^\prime)$:] $ $
\begin{itemize}
    \item Parse $\vk=(\abe.\msk,\ske.\vk,\ske.\tk, \key)$. We denote the register holding $\qdk^\prime$ as $\qreg{SKE.DK}\tensor\qreg{ABE.SK}$.
    \item Prepare a register $\qreg{SKE.KT}$ that is initialized to $\ket{0}_{\qreg{SKE.KT}}$.
    \item Apply the map $\ket{u}_{\qreg{SKE.DK}}\ket{\beta}_{\qreg{SKE.KT}}\ra\ket{u}_{\qreg{SKE.DK}}\ket{\beta\oplus\SKECRSKL.\KeyTest(\ske.\tk,u)}_{\qreg{SKE.KT}}$ to the registers $\qreg{SKE.DK}$ and $\qreg{SKE.KT}$.
    \item Measure $\qreg{SKE.KT}$ in the computational basis and output $\bot$ if the result is $0$. Otherwise, go to the next step.
    \item Apply the map
        $\ket{u}_{\qreg{SKE.DK}}\ket{v}_{\qreg{ABE.SK}}\ra\ket{u}_{\qreg{SKE.DK}}\ket{v\oplus\ABE.\KG(\abe.\msk,u,
        \key)}_{\qreg{ABE.SK}}$ to the registers $\qreg{SKE.DK}$ and $\qreg{ABE.SK}$.
    \item Trace out the register $\qreg{ABE.SK}$ and obtain $\ske.\qdk^\prime$ over $\qreg{SKE.DK}$.
    \item Output $\top$ if $\top=\SKECRSKL.\qVrfy(\ske.\vk,\ske.\qdk')$
        and $\bot$ otherwise.

\end{itemize}
\end{description}

\paragraph{Decryption correctness.}
The key
$\qdk$ output by $\PKECRSKL.\qKG$ is of the form $\sum_u \alpha_u
\ket{u}_{\qreg{SKE.DK}}\ket{\abe.\sk_u}_{\qreg{ABE.SK}}$, where
$\abe.\sk_u\gets\ABE.\KG(\abe.\msk,u, \key)$.
Let $\ct\gets\PKECRSKL.\Enc(\ek,\msg)$.
On applying $\ket{v}_{\qreg{ABE.SK}}\ket{w}_{\qreg{MSG}}\ra\ket{v}_{\qreg{ABE.SK}}\ket{w\oplus\ABE.\Dec(v,\abe.\ct)}_{\qreg{MSG}}$ to $\sum_u \alpha_u \ket{u}_{\qreg{SKE.DK}}\ket{\abe.\sk_u}_{\qreg{ABE.SK}} \tensor \ket{0\cdots0}_{\qreg{MSG}}$, with overwhelming probability, the result is negligibly close to
\begin{align}
\sum_u \alpha_u \ket{u}_{\qreg{SKE.DK}}\ket{\abe.\sk_u}_{\qreg{ABE.SK}} \tensor \ket{\msg}_{\qreg{MSG}}
\end{align}
since $\tlC(u)=\bot$ and thus $R(\tlC,u)=0$ for $\tlC\gets\CCSim(1^\secp,\pp_D,1)$ and almost every string $u$.
Therefore, we see that $\PKECRSKL$ satisfies decryption correctness.

\paragraph{Verification correctness.}
Let $\qdk\gets\PKECRSKL.\qKG(\msk)$.
It is clear that the state $\ske.\qdk^\prime$ obtained when computing $\PKECRSKL.\qVrfy(\vk,\qdk)$ is the same as $\ske.\qdk$ generated when generating $\qdk$.
Therefore, the verification correctness of $\PKECRSKL$ follows from that of $\SKECRSKL$.

\subsection{Proof of IND-KLA Security}
Let $\qA$ be an adversary for the IND-KLA security of $\PKECRSKL$.
We consider the following sequence of experiments.
\begin{description}
\item[$\hybi{0}^\coin$:]This is $\expb{\PKECRSKL,\qA}{ind}{kla}(1^\secp,\coin)$.
\begin{enumerate}
\item The challenger $\qCh$ generates $(\abe.\pk,\abe.\msk)\gets\ABE.\Setup(1^\secp)$ and $\ske.\msk\gets\SKECRSKL.\Setup(1^\secp)$, and sends $\ek\seteq\abe.\pk$ to $\qA$.

\item $\qA$ requests $q$ decryption keys for some polynomial $q$.
$\qCh$ generates $\qdk_i$ as follows for every $i\in[q]$:

\begin{itemize}
    \item Generate $(\ske.\qdk_i,\ske.\vk_i,\ske.\tk_i)\gets\SKECRSKL.\qKG(\ske.\msk)$. We denote the register holding $\ske.\qdk_i$ as $\qreg{SKE.DK_i}$.
    \item Prepare a register $\qreg{ABE.SK_i}$ that is initialized to $\ket{0\cdots0}_{\qreg{ABE.SK_i}}$.
    \item Choose explicit randomness $\key_i \leftarrow \bit^\secp$.
    \item Apply the map
        $\ket{u}_{\qreg{SKE.DK_i}}\ket{v}_{\qreg{ABE.SK_i}}\ra\ket{u}_{\qreg{SKE.DK_i}}\ket{v\oplus\ABE.\KG(\abe.\msk,u,
        \key_i)}_{\qreg{ABE.SK_i}}$ to the registers $\qreg{SKE.DK_i}$ and $\qreg{ABE.SK_i}$, and obtain $\qdk_i$ over the registers $\qreg{SKE.DK_i}$ and $\qreg{ABE.SK_i}$.
\end{itemize}
$\qCh$ sends $\qdk_1,\ldots,\qdk_q$ to $\qA$.
\item $\qA$ can get access to the following (stateful) verification
oracle $\Oracle{\qVrfy}$ where $V_i$ is initialized to be $\bot$:

\begin{description}
\item[ $\Oracle{\qVrfy}(i,\widetilde{\qdk})$:] It computes $d$ as follows.  

\begin{enumerate}
    \item Let the register holding $\widetilde{\qdk}$ be $\qreg{SKE.DK_i}\tensor\qreg{ABE.SK_i}$.
    \item Prepare a register $\qreg{SKE.KT_i}$ that is initialized to $\ket{0}_{\qreg{SKE.KT_i}}$.
    \item Apply the map $\ket{u}_{\qreg{SKE.DK_i}}\ket{\beta}_{\qreg{SKE.KT_i}}\ra\ket{u}_{\qreg{SKE.DK_i}}\ket{\beta\oplus\SKECRSKL.\KeyTest(\ske.\tk_i,u)}_{\qreg{SKE.KT_i}}$ to the registers $\qreg{SKE.DK_i}$ and $\qreg{SKE.KT_i}$.
    \item Measure $\qreg{SKE.KT_i}$ in the computational basis and
        set $d\seteq\bot$ if the result is $0$. Otherwise, go to the next step.
    \item Apply the map
        $\ket{u}_{\qreg{SKE.DK_i}}\ket{v}_{\qreg{ABE.SK_i}}\ra\ket{u}_{\qreg{SKE.DK_i}}\ket{v\oplus\ABE.\KG(\abe.\msk,u,\key_i)}_{\qreg{ABE.SK_i}}$ to the registers $\qreg{SKE.DK_i}$ and $\qreg{ABE.SK_i}$.
    \item Trace out the register $\qreg{ABE.SK_i}$ and obtain $\ske.\qdk^\prime$ over $\qreg{SKE.DK_i}$.
    \item Set $d\seteq\top$ if
        $\top=\SKECRSKL.\qVrfy(\ske.\vk_i,\ske.\qdk')$ and set $d\seteq\bot$ otherwise.
It returns $d$ to $\qA$. Finally, if $V_i=\bot$ and $d=\top$, it updates $V_i\seteq \top$. 
\end{enumerate}
            \end{description}
            \item $\qA$ sends $(\msg_0^*,\msg_1^*)\in \cM^2$ to
                $\qCh$. If $V_i=\bot$ for some $i\in[q]$,
                $\qCh$ outputs $0$ as the final output of this
                experiment. Otherwise, $\qCh$ generates
                $\tlC^*\gets\CCSim(1^\secp, \pp_D, 1)$ and $\abe.\ct^*\gets\ABE.\Enc(\abe.\pk,\tlC^*,\msg_\coin^*)$, and sends $\ct^*\seteq\abe.\ct^*$ to $\qA$.
            \item $\qA$ outputs $\coin^\prime$. $\qCh$ outputs $\coin'$ as the final output of the experiment.
        \end{enumerate}
        
\item[$\hybi{1}^\coin$:]This is the same as $\hybi{0}^\coin$ except
that $\tlC^*$ is generated as
$\tlC^*\gets\CCObf(1^\secp,D[\ske.\ct^*], \lock, 0)$, where
$\ske.\ct^*\gets\SKECRSKL.\Enc(\ske.\msk,0^\secp)$,
$\lock\gets\bit^\secp$, and $D[\ske.\ct^*](x)$ is a circuit that has
$\ske.\ct^*$ hardwired and outputs $\SKECRSKL.\CDec(x,\ske.\ct^*)$.

\end{description}

We pick $\lock$ as a uniformly random string that is completely independent of other variables such as $\ske.\ct^*$.
Thus, from the security of $\CCObf$, we have $\Hyb_0^\coin \approx
\Hyb_1^\coin$.

\begin{description}
\item[$\hybi{2}^\coin$:]This is the same as $\hybi{1}^\coin$ except
that $\ske.\ct^*$ hardwired into the obfuscated circuit $\tlC^*$ is
generated as $\ske.\ct^*\gets\SKECRSKL.\Enc(\ske.\msk,\lock)$.
\end{description}

From the OT-IND-KLA security of $\SKECRSKL$, we can show that
$\hybi{1}^\coin \approx \hybi{2}^\coin$. Suppose that $\Hyb_1^\coin
\not \approx \Hyb_2^\coin$ and $\qD$ is a corresponding
distinguisher. We consider the following reduction
$\qR$:

\begin{description}
\item Execution of $\qR^{\qD}$ in
$\expc{\SKECRSKL,\qR}{ot}{ind}{kla}(1^\secp, b)$:

\begin{enumerate}
\item $\qCh$ runs $\ske.\msk \gets \SKECRSKL.\Setup(1^\secp)$ and initializes $\qR$
with the security parameter $1^\secp$.
\item $\qR$ generates $(\abe.\pk, \abe.\msk) \gets
\ABE.\Setup(1^\secp)$ and sends $\ek \seteq \abe.\pk$ to $\qD$.
\item $\qD$ requests $q$ decryption keys for some polynomial $q$.
$\qR$ requests $q$ decryption keys. $\qCh$ generates $(\ske.\qdk_i,
\ske.\vk_i, \ske.\tk_i) \gets \SKECRSKL.\qKG(\msk)$ for every $i \in [q]$
and sends $(\ske.\qdk_i, \ske.\tk_i)_{i\in[q]}$ to $\qR$.
\item $\qR$ computes $\qdk_1, \ldots, \qdk_q$ as in Step $2.$ of
$\Hyb_0^\coin$, except that the received values
$(\ske.\qdk_i)_{i \in [q]}$ are used instead of the original ones.
\item $\qR$ simulates the access to $\Oracle{\qVrfy}(i,
\widetilde{\qdk})$ for $\qD$ as follows:
\begin{description}
\item $\Oracle{\qVrfy}(i, \widetilde{\qdk}):$
\begin{enumerate}
\item Perform Step $3.(\textrm{a})$-$3.(\textrm{f})$ of $\Hyb_0^\coin$ to obtain
$\ske.\qdk'$, but using the received value $\ske.\tk_i$
instead of the original one.
\item Set $d \seteq \top$ if $\top =
\SKECRSKL.\Oracle{\qVrfy}(i, \ske.\qdk')$ and set $d \seteq \bot$
otherwise.
It returns $d$ to $\qD$. Finally, if $V_i = \bot$ and $d = \top$, it
updates $V_i = \top$.
\end{enumerate}
\end{description}
\item $\qR$ sends $(\ske.\msg_0^*, \ske.\msg_1^*) \seteq (0^\secp, \lock)$
to $\qCh$ and receives $\ske.\ct^* \gets \SKECRSKL.\Enc(\ske.\msk,
\ske.\msg^*_b)$.
\item $\qD$ sends $(\msg_0^*, \msg_1^*) \in \cM^2$ to $\qR$. If $V_i =
\bot$ for some $i\in[q]$, $\qR$ outputs $0$. Otherwise, $\qR$
generates $\tlC^* \gets \CCObf(1^\secp, D[\ske.\ct^*], \lock, 0)$
and $\abe.\ct^* \gets \ABE.\Enc(\abe.\pk, \tlC^*, \msg^*_\coin)$ and
sends $\ct^* \seteq \abe.\ct^*$ to $\qD$.
\item $\qD$ outputs a bit $b'$. $\qR$ outputs $b'$ and $\qCh$
outputs $b'$ as the final output of the experiment.
\end{enumerate}
\end{description}

It is easy to see that the view of $\qD$ is the same as that in
$\Hyb^\coin_2$ when $\lock$ is encrypted in $\ske.\ct^*$ and that of
$\Hyb^\coin_1$ when $0^\secp$ is encrypted. Moreover, for $\qD$ to
distinguish between the two hybrids, it must be the case that $V_i =
\top$ for all $i \in [q]$, which directly implies that the $q$
analogous values checked by $\SKECRSKL.\Oracle{\qVrfy}$ must also be
$\top$.  Consequently, $\qR$ breaks the OT-IND-KLA security of
$\SKECRSKL$. Therefore, it must be that $\Hyb_1^\coin \approx
\Hyb_2^\coin$.

\begin{description}
\item[$\hybi{3}^\coin$:]This is the same as $\hybi{2}^\coin$ except
that $\qdk_i$ is generated as follows for every $i\in[q]$. (The
difference is colored in red.)
\begin{itemize}
    \item Generate $(\ske.\qdk_i,\ske.\vk_i,\ske.\tk_i)\gets\SKECRSKL.\qKG(\ske.\msk)$. We denote the register holding $\ske.\qdk_i$ as $\qreg{SKE.DK_i}$.
    \textcolor{red}{
        \item Prepare a register $\qreg{ABE.R_i}$ that is initialized to $\ket{0}_{\qreg{ABE.R_i}}$.
    \item Apply the map $\ket{u}_{\qreg{SKE.DK_i}}\ket{\beta}_{\qreg{ABE.R_i}}\ra\ket{u}_{\qreg{SKE.DK_i}}\ket{\beta\oplus R(\tlC^*,u)}_{\qreg{ABE.R_i}}$
to the registers $\qreg{SKE.DK_i}$ and $\qreg{ABE.R_i}$. (Note that we
can generate $\tlC^*$ at the beginning of the game.)
    \item Measure $\qreg{ABE.R_i}$ in the computational basis and set $\qdk_i\seteq\bot$ if the result is $0$. Otherwise, go to the next step.
    }
    \item Prepare a register $\qreg{ABE.SK_i}$ that is initialized to $\ket{0\cdots0}_{\qreg{ABE.SK_i}}$.

    \item Sample explicit randomness $\key_i \gets \bit^\secp$.
    \item Apply the map
        $\ket{u}_{\qreg{SKE.DK_i}}\ket{v}_{\qreg{ABE.SK_i}}\ra\ket{u}_{\qreg{SKE.DK_i}}\ket{v\oplus\ABE.\KG(\abe.\msk,u,\key_i)}_{\qreg{ABE.SK_i}}$ to the registers $\qreg{SKE.DK_i}$ and $\qreg{ABE.SK_i}$, and obtain $\qdk_i$ over the registers $\qreg{SKE.DK_i}$ and $\qreg{ABE.SK_i}$.
\end{itemize}
\end{description}

From the decryption correctness of $\SKECRSKL$, the added procedure that checks $R(\tlC^*,u)$ in superposition does not affect the final state $\qdk_i$ with overwhelming probability since $R(\tlC^*,u)=1$ in this hybrid for any $u$ that appears in $\ske.\qdk_i$ when describing it in the computational basis.
Therefore, we have $\Hyb_2^\coin \approx \Hyb_3^\coin$.

\begin{description}
\item[$\hybi{4}^\coin$:]This is the same as $\hybi{3}^\coin$ except
that the oracle $\Oracle{\qVrfy}$ behaves as follows. (The
difference is colored in red.)
\begin{description}
\item[ $\Oracle{\qVrfy}(i,\widetilde{\qdk})$:] It computes $d$ as follows.  
\begin{enumerate}[(a)]
    \item Let the register holding $\widetilde{\qdk}$ be $\qreg{SKE.DK_i}\tensor\qreg{ABE.SK_i}$.
    \item Prepare a register $\qreg{SKE.KT_i}$ that is initialized to $\ket{0}_{\qreg{SKE.KT_i}}$.
    \item Apply the map $\ket{u}_{\qreg{SKE.DK_i}}\ket{\beta}_{\qreg{SKE.KT_i}}\ra\ket{u}_{\qreg{SKE.DK_i}}\ket{\beta\oplus\SKECRSKL.\KeyTest(\ske.\tk_i,u)}_{\qreg{SKE.KT_i}}$ to the registers $\qreg{SKE.DK_i}$ and $\qreg{SKE.KT_i}$.
    \item Measure $\qreg{SKE.KT_i}$ in the computational basis and set $d\seteq\bot$ if the result is $0$. Otherwise, go to the next step.
    \textcolor{red}{
    \item Prepare a register $\qreg{ABE.R_i}$ that is initialized to $\ket{0}_{\qreg{ABE.R_i}}$.
    \item Apply the map $\ket{u}_{\qreg{SKE.DK_i}}\ket{\beta}_{\qreg{ABE.R_i}}\ra\ket{u}_{\qreg{SKE.DK_i}}\ket{\beta\oplus R(\tlC^*,u)}_{\qreg{ABE.R_i}}$ to the registers $\qreg{SKE.DK_i}$ and $\qreg{ABE.R_i}$.
    \item Measure $\qreg{ABE.R_i}$ in the computational basis and set $d\seteq\bot$ if the result is $0$. Otherwise, go to the next step.
    }
    \item Apply the map
        $\ket{u}_{\qreg{SKE.DK_i}}\ket{v}_{\qreg{ABE.SK_i}}\ra\ket{u}_{\qreg{SKE.DK_i}}\ket{v\oplus\ABE.\KG(\abe.\msk,u,\key_i)}_{\qreg{ABE.SK_i}}$ to the registers $\qreg{SKE.DK_i}$ and $\qreg{ABE.SK_i}$.
    \item Trace out the register $\qreg{ABE.SK_i}$ and obtain $\ske.\qdk^\prime$ over $\qreg{SKE.DK_i}$.
    \item Set $d\seteq\top$ if $\top=\SKECRSKL.\qVrfy(\ske.\vk_i,
        \ske.\qdk')$ and set $d\seteq\bot$ otherwise.
    Return $d$ to $\qA$. Finally, if $V_i=\bot$ and $d=\top$, update $V_i\seteq \top$.
\end{enumerate}
            \end{description}
\end{description}

%Suppose $\qA$ makes a query $(i,\widetilde{\qdk})$ to $\Oracle{\qVrfy}$ and the measurement result of $\qreg{SKE.KT_i}$ is $\top$.

Suppose there exists a QPT distinguisher $\qD$ that has
non-negligible advantage in distinguishing $\Hyb_3^\coin$ and
$\Hyb_4^\coin$. Let $\qD$ make $q = \poly(\lambda)$ many queries to
the oracle $\Oracle{\qVrfy}(\cdot, \cdot)$. We will now consider the
following QPT algorithm $\qA_\oh$ with access to an oracle
$\Oracle{\qKG}$ and an oracle $\cO$ that runs $\qD$ as follows:

\begin{description}
\item $\underline{\qA_\oh^{\Oracle{\qKG}, \cO}(\abe.\pk, \ske.\msk)}:$
\begin{enumerate}
\item $\qA_\oh$ initializes $\qD$ with input $\ek = \abe.\pk$.
\item When $\qD$ requests $q$ decryption keys, $\qA_\oh$ queries
$\Oracle{\qKG}$ on input $q$ and receives $(\{\qdk_i\}_{i \in [q]}, \{\vk_i\}_{i \in [q]})$. It forwards $\{\qdk_i\}_{i\in [q]}$ to $\qD$.

\item $\qA_\oh$ simulates the access of $\Oracle{\qVrfy}$ for $\qD$ as
follows:
\begin{description}
\item $\underline{\Oracle{\qVrfy}(y, \widetilde{\qdk})}:$
\begin{enumerate}
\item Execute Steps (a)-(e) of $\Oracle{\qVrfy}$ as in
    $\Hyb_4^\coin$.
    \item Apply the map
        $\ket{u}_{\qreg{SKE.DK_y}}\ket{\beta}_{\qreg{ABE.R_y}}\ra\ket{u}_{\qreg{SKE.DK_y}}\ket{\beta\oplus
        \calO(u)}_{\qreg{ABE.R_y}}$ to the registers
        $\qreg{SKE.DK_y}$ and $\qreg{ABE.R_y}$. 
\item Execute Steps (g)-(j) of $\Oracle{\qVrfy}$ as in
$\Hyb_4^\coin$.
\end{enumerate}
\end{description}

\item $\qD$ sends $(\msg_0^*, \msg_1^*) \in \cM^2$ to $\qA_\oh$. If $V_i =
\bot$ for some $i \in [q]$, $\qA_\oh$ outputs $0$. Otherwise,
$\qA_\oh$ generates $\tlC^* \gets \CCObf(1^\secp, D[\ske.\ct^*], \lock,
0)$, where $\ske.\ct^* \gets \SKECRSKL.\Enc(\ske.\msk, \lock)$.
It then generates $\abe.\ct^* \gets\ABE.\Enc(\abe.\pk, \tlC^*, \msg^*_\coin)$ and sends $\ct^* \seteq\abe.\ct^*$ to $\qD$.
\item $\qD$ outputs a guess $b'$. $\qA_\oh$ outputs $b'$.
\end{enumerate}
\end{description}

Let $H$ be an oracle that for every input $u$, outputs $1$.
Consider now the extractor $\qB_\oh^{\Oracle{\qKG}, H}$
as specified by the O2H Lemma (Lemma \ref{lem:O2H}). We will now
construct a reduction $\qR$ that runs $\qB_\oh$ by simulating the oracles
$\Oracle{\qKG}$ and $H$ for $\qB_\oh$, and breaks the
key-testability of the $\SKECRSKL$ scheme.

\begin{description}
\item Execution of $\qR$ in
$\expb{\SKECRSKL,\qR}{key}{test}(1^\secp)$:

\begin{enumerate}
\item The challenger $\qCh$ runs $\ske.\msk \leftarrow
\SKECRSKL.\Setup(1^\secp)$ and
initializes $\qR$ with input $\ske.\msk$.
\item $\qR$ samples $(\abe.\pk, \abe.\msk) \gets
\ABE.\Setup(1^\secp)$ and initializes $\qB_\oh$ with the input
$(\abe.\pk, \ske.\msk)$.
\item When $\qB_\oh$ queries input $q$ to $\Oracle{\qKG}$, $\qR$
generates the decryption-keys $\qdk_1, \ldots, \qdk_q$ in the same way
as in $\Hyb_3^\coin$ and sends them to $\qB_\oh$.

\item When $\qB_\oh$ queries an input $u$ to $H$, $\qR$ responds with $1$.

\item $\qB_\oh$ outputs measured index $y$ and measurement outcome $\dk$. $\qR$ sends $(y, \dk, \lock)$
to $\qCh$.

\end{enumerate}
\end{description}

We will now claim that with non-negligible probability, $\qR$ obtains
values $\dk$ and $y$ such that $R(\tlC^*, \dk) = 0$. By the definition of
$R$ and $\tlC^*$ and the decryption correctness of $\SKECRSKL$, this
will imply that $\SKECRSKL.\CDec(\dk, \allowbreak\ske.\ct^\star) \neq
\lock$. Moreover, $\KeyTest(\ske.\tk_y, \dk)$ also holds.
Consequently, this will imply $\qR$ breaks the key-testability of
$\SKECRSKL$. To prove this, we will rely on the One-Way to Hiding
(O2H) Lemma (Lemma \ref{lem:O2H}).  Consider an oracle $G$ which takes
as input $u$ and outputs $R(\tlC^*, u)$ and an oracle $H$ which takes
as input $u$ and outputs $1$. Notice that if the oracle $\calO = G$,
then the view of $\qD$ as run by $\qA_\oh$ is the same as in
$\Hyb_4^\coin$, while if $\calO = H$, the view of $\qD$ is the same as
in $\Hyb_3^\coin$. By the O2H Lemma, we have the following, where $z =
(\abe.\pk, \ske.\msk)$.

\begin{align}
\abs{\Pr[\qA_\oh^{\Oracle{\qKG},
H}(z)=1]-\Pr[\qA_\oh^{\Oracle{\qKG}, G}(z)=1]} \leq
2q\cdot\sqrt{\Pr[\qB_\oh^{\Oracle{\qKG}, H}(z)\in S]}
\enspace.
\end{align}
where $S$ is a set where
the oracles $H$ and $G$ differ, which happens only for inputs $u$ s.t.
$R(\tlC^*, u) = 0$. Since $\qR$ obtains $\dk$ and $y$ as the
output of $\qB_\oh^{\Oracle{\qKG},H}(z)$, the argument
follows that $\Hyb_3^\coin \approx \Hyb_4^\coin$.

\begin{description}
\item[$\hybi{5}^\coin$:]This is the same as $\hybi{4}^\coin$ except that $\ct^*\seteq\abe.\ct^*$ is generated as $\abe.\ct^*\gets\ABE.\Enc(\abe.\pk,\tlC^*,0^{\msglen})$.
\end{description}

The view of $\qA$ in $\hybi{4}^\coin$ and $\hybi{5}^\coin$ can be
simulated with $\abe.\pk$ and the access to the quantum key
generation oracle $\Oracle{qkg}$. This is because before $\ABE.\KG$
is required to be applied in both the generation of $\{\qdk_i\}_{i
\in [q]}$ and to compute the responses of $\Oracle{\qVrfy}$, the
relation check $R(\tlC^*, u)$ is already applied in
superposition. Thus, we have $\Hyb_4^\coin \approx \Hyb_5^\coin$.

Lastly, $\hybi{5}^{0}$ and $\hybi{5}^{1}$ are exactly the same experiment and thus we have $\abs{\Pr[\hybi{5}^0=1]-\Pr[\hybi{5}^1=1]}=\negl(\secp)$.
Then, from the above arguments, we obtain
\begin{align}
   & \abs{\Pr[\expb{\PKECRSKL,\qA}{ind}{kla}(1^\secp,0)=1]-\Pr[\expb{\PKECRSKL,\qA}{ind}{kla}(1^\secp,1)=1]}\\
=&\abs{\Pr[\hybi{0}^0=1]-\Pr[\hybi{0}^1=1]}\le\negl(\secp).
\end{align}
This completes the proof. \qed

Given the fact that SKE-CR-SKL with Key-Testability (implied by
BB84-based SKE-CD and OWFs), Compute-and-Compare Obfuscation, and
Ciphertext-Policy ABE for General Circuits are all implied by the
LWE assumption, we state the following theorem:

\begin{theorem}
There exists a PKE-CR-SKL scheme satisfying IND-KLA security,
assuming the polynomial hardness of the LWE assumption.
\end{theorem}

\nikhil{At this point, we can probably just state the ABE-CR-SKL and ABE-CR\textsuperscript{2}-SKL theorems and move them to the Appendix.}
