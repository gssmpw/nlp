% !TEX root = main.tex

\section{Technical Overview}\label{sec:technical_overview}

\newcommand{\qDel}{\qalgo{Del}}
\newcommand{\cnc}[2]{\mathbf{CC}[#1,#2]}
\newcommand{\lock}{\mathsf{lock}}
\newcommand{\CCObf}{\mathsf{CC}.\mathsf{Obf}}
\newcommand{\CCSim}{\mathsf{CC}.\mathsf{Sim}}
\newcommand{\tlP}{\widetilde{P}}
\newcommand{\CDSKE}{\mathsf{SKECD}}
\newcommand{\SKECD}{\mathsf{SKECD}}
\newcommand{\CDec}{\mathsf{CDec}}
\newcommand{\ctlen}{\ell_{\ct}}
\newcommand{\NMSKECD}{\mathsf{NMSKECD}}
\newcommand{\sig}{\mathsf{sig}}
\newcommand{\sgn}{\mathsf{sgn}}
\newcommand{\skecd}{\mathsf{skecd}}
\renewcommand{\Check}{\mathsf{Check}}
\newcommand{\msglen}{\ell_{\msg}}
\newcommand{\PKECRSKL}{\mathsf{PKE}\textrm{-}\mathsf{CR}\textrm{-}\mathsf{SKL}}
\newcommand{\SKECRSKL}{\mathsf{SKE}\textrm{-}\mathsf{CR}\textrm{-}\mathsf{SKL}}
\newcommand{\KeyTest}{\mathsf{KeyTest}}


We now provide a technical overview of this work.  
Our primary focus here is on constructing PKE-CR-SKL and ABE-CR-SKL based on the LWE assumption.  
For an overview of our variants with classical certificates based on MI-ABE, please refer to \cref{sec-ABE-CR-SKL-classical-certificate}. 
%\fuyuki{I moved the overview for classical certificate constructions, and added these sentences. %Please check \cref{sec-ABE-CR-SKL-classical-certificate}.}

\subsection{Defining PKE with Collusion-Resistant SKL}

We will begin by describing the definition of PKE with Secure
Key-Leasing (PKE-SKL) \cite{EC:AKNYY23,TCC:AnaPorVai23}, and then get into our collusion-resistant generalization
of it. PKE-SKL is a cryptographic primitive consisting of five
algorithms: $\Setup, \qKG, \Enc, \qDec$ and $\qVrfy$. The algorithm
$\Setup$ samples a public encryption-key $\ek$ and a ``master''
secret-key $\msk$. The encryption algorithm $\Enc$ takes a message
$\msg$ and $\ek$ as inputs and produces a corresponding ciphertext
$\ct$. Both these algorithms are classical and similar to their
counterparts in standard PKE. On the other hand, the key-generation
algorithm $\qKG$ is quantum, and produces a pair of keys $(\qdk, \vk)$
given $\msk$ as input. Here, $\qdk$ is a quantum decryption-key using
which $\qDec$ can decrypt arbitrarily many ciphertexts encrypted under
$\ek$. The other key $\vk$ is a classical verification-key, the
purpose of which will be clear in a moment.

The setting to consider is one where an adversary is given a
decryption-key $\qdk$, and is later asked to return it back.
Intuitively, we wish to guarantee that if $\qdk$ is returned, the
adversary can no longer decrypt ciphertexts
encrypted under $\ek$. Since a malicious adversary may
even send a malformed key $\widetilde{\qdk}$, we need a way to tell
whether the decryption-key has been correctly returned.  This
is the purpose of the algorithm $\qVrfy$, which performs such a check
using the corresponding (private) verification-key $\vk$.  It is
required that if the state $\qdk$ is sent back undisturbed, then
$\qVrfy$ must accept. On the other hand, if $\qVrfy$ accepts (even for
a possibly malformed state), then the adversary must lose the ability
to decrypt. This loss in the ability to decrypt is captured formally
by a cryptographic game, where an adversary is asked to distinguish
between ciphertexts of different messages, after passing the
verification check.

So far, we have described the notion of PKE-SKL. In this work, we
study the notion of PKE with Collusion-Resistant SKL (PKE-CR-SKL).
This primitive has the same syntax as PKE-SKL, but the aforementioned
security requirement is now stronger. Specifically, an adversary that
obtains polynomially-many decryption keys and (verifiably) returns
them all, should no longer be able to decrypt. This is defined formally
in our notion of IND-KLA (Indistinguishability under Key-Leasing
Attacks) security (Definition \ref{def:IND-CPA_PKECRSKL}). As standard
in cryptography, it is characterized by a game between a challenger
$\qCh$ and a QPT adversary $\qA$. An informal description of the game
follows:

\begin{description}
\item \textbf{IND-KLA Game in the Collusion-Resistant Setting}
\begin{enumerate}
\item $\qCh$ samples $(\ek, \msk) \gets \Setup(1^\secp)$ and sends
$\ek$ to $\qA$.
\item Then, $\qA$ requests $q$ decryption keys corresponding to $\ek$,
where $q$ is some arbitrary polynomial in $\secp$.\footnote{Without loss of generality, we can assume that $\qA$ asks for sufficiently many keys at the beginning itself. Hence, even adversaries that can request additional keys after accessing the verification oracle are covered by this definition.}
\item $\qCh$ generates $(\qdk_i, \vk_i)_{i \in [q]}$ using $q$
independent
invocations of $\qKG(\msk)$. It sends $\{\qdk_i\}_{i\in[q]}$ to $\qA$.

\item Corresponding to each index $i \in [q]$, $\qA$ is allowed
oracle access to the algorithm $\qVrfy(\vk_i, \cdot)$ in the following
sense: For a quantum state $\widetilde{\qdk}$ queried by $\qA$, the
oracle evaluates $\qVrfy(\vk_i, \widetilde{\qdk})$ and measures the
verification result. It then returns the obtained classical outcome
(which indicates accept/reject). We emphasize that $\qA$ is allowed to
interleave queries corresponding to indices $i \in [q]$, and can also
make its queries adaptively.

\item If at-least one query of $\qA$ to $\qVrfy(\vk_i, \cdot)$
produces an accept output for every $i \in [q]$, the game proceeds to
the challenge phase. Otherwise, the game aborts.
\item In the challenge phase, $\qA$ specifies a pair of messages
$(\msg_0, \msg_1)$. $\qCh$ sends $\msg_\coin$ to $\qA$ for a random
bit $\coin$.
\item $\qA$ outputs $\coin'$.
\end{enumerate}
\end{description}

$\qA$ wins the game whenever $\coin = \coin'$. The security
requirement is that for every QPT adversary, the winning probability
conditioned on the game not aborting is negligibly close to $1/2$.
Observe that the
adversary is allowed to make polynomially-many attempts in order to
verifiably return a decryption-key $\qdk_i$, and unsuccessful attempts
are not penalized.  We also emphasize that in the definition, $q$ is
an unbounded polynomial, i.e., the construction is not allowed to
depend on $q$ in any way.


\subsection{Insecurity of Direct Extensions of Prior Work}

Next, we will provide some intuition regarding the insecurity of
direct extensions of prior works to this stronger setting.

Let us consider the PKE-SKL scheme due to Agrawal et al.
\cite{EC:AKNYY23} for demonstration. The decryption-keys in their
scheme are of the following form\footnote{In actuality, they use a
parallel repetition to amplify security.}:

$$\frac{1}{\sqrt{2}}\big(\ket{0}\ket{\sk_0} + \ket{1}\ket{\sk_1}\big)$$

Here, $\sk_0, \sk_1$ are secret-keys corresponding to public-keys
$\pk_0, \pk_1$ respectively of a standard PKE scheme. Specifically,
the pairs $(\pk_0, \sk_0)$ and $(\pk_1, \sk_1)$ are generated using
independent invocations of the setup algorithm of PKE. The PKE-SKL
public-key consists of the pair $(\pk_0, \pk_1)$. The encryption
algorithm outputs ciphertexts of the form $(\ct_0, \ct_1)$, where for
each $i \in \bit$, $\ct_i$ encrypts the plaintext under $\pk_i$.  The
verification procedure requires the adversary to return the
decryption-key, and checks if it is the same as the above state. The intuition is that if the adversary retains the ability to decrypt, then it cannot pass the verification check with probability close to 1. For instance, measuring the state provides the adversary with $\sk_0$ or
$\sk_1$ which is sufficient for decryption, but it clearly destroys
the above quantum state.

Consider now the scenario where the public-key $(\pk_0, \pk_1)$ is
fixed, and $n = \poly(\secp)$ copies of the above decryption-key are
given to an adversary. In this case, it is easy to see that the
adversary can simply measure the states to obtain both $\sk_0$ and
$\sk_1$. Even though this process is destructive, since $\sk_0$ and
$\sk_1$ completely describe the state, the adversary can just recreate
many copies of it and pass the deletion checks. Consequently, one
might be tempted to encode other secret information in the Hadamard
basis by introducing random phases. However, we observe that such
approaches also fail due to simple gentle-measurement attacks.  For
example, consider the combined state of the $n$ decryption-keys in
such a case:

$$\frac{1}{2^{n/2}}\Big(\ket{0\ldots0}\ket{\sk_0 \ldots \sk_0} -
\ket{00\ldots1}\ket{\sk_0\sk_0\ldots\sk_1} + \cdots -
\ket{1\ldots1}\ket{\sk_1\ldots\sk_1}\Big)$$

Clearly, only the last term of the superposition does not contain
$\sk_0$, and the term only has negligible amplitude.
Hence, one can compute $\sk_0$ on another register and measure
it, without disturbing the state more than a negligible amount. As a result, the verification checks can be passed while retaining the ability to decrypt. We note that all existing constructions of encryption with SKL can be broken with similar collusion attacks. 

The reason our scheme does not run into such an attack is because we
rely on the notion of Attribute-Based Encryption (ABE), which
enables exponentially-many secret-keys for every public-key.
Consequently, different decryption-keys can be generated as
superpositions of different ABE secret-keys. However, it is not
clear how this intuition alone can be used to establish security in
a provable manner, and we require additional ideas. We now describe
these ideas at a high-level.

\subsection{Idea Behind the PKE-CR-SKL Scheme}

In order to explain the basic idea behind our PKE-CR-SKL scheme, we
will first introduce a key-building block, a primitive called
SKE-CR-SKL. This is basically a secret-key variant of PKE-CR-SKL,
i.e., the setup algorithm only samples a master secret-key $\ske.\msk$, and
the encryption algorithm encrypts plaintexts under $\ske.\msk$. The
security requirement is similar. An adversary that receives polynomially
many decryption-keys and returns them all, should not be able to
distinguish between ciphertexts of different messages. We refer to
this security notion as one-time IND-KLA (OT-IND-KLA) security
(Definition \ref{def:OT-IND-CPA_SKECRSKL}).
The ``one-time'' prefix refers to the fact that unlike in PKE-CR-SKL,
the adversary does not have the ability to perform chosen plaintext
attacks. In other words, the adversary does not see any ciphertexts
before it is required to return its decryption-keys. Although this is
a weak security guarantee, it suffices for our PKE-CR-SKL scheme. The
description of the PKE-CR-SKL scheme now follows.

The key-generation procedure involves first generating an SKE-CR-SKL
decryption-key, which is represented in the computational basis as
$\ske.\qdk = \sum_u \alpha_u \ket{u}$.
Actually, our SKE-CR-SKL scheme needs to
satisfy another crucial property, which we call the classical
decryption property. This property requires the existence of a
classical deterministic algorithm $\CDec$ with the following
guarantee.  For any SKE-CR-SKL ciphertext $\ske.\ct$, $\CDec(u, \ske.\ct)$
correctly decrypts $\ske.\ct$ for every string $u$ in the superposition of
$\ske.\qdk$. Our construction exploits this fact with the help of an
ABE scheme as follows.

The actual decryption key is generated as $\qdk \seteq \sum_u
\alpha_u \ket{u} \otimes \ket{\abe.\sk_u}$ where $\abe.\sk_u$ is an ABE secret-key
corresponding to the key-attribute $u$. The idea is to now have the
encryption algorithm encrypt the plaintext using the ABE scheme,
under a carefully chosen ciphertext-policy. Specifically, we wish to
embed an SKE-CR-SKL ciphertext $\ske.\ct^*$ as part of the
policy-circuit, such that the outcome of $\CDec(u, \ske.\ct^*)$
determines whether the key $\abe.\sk_u$ can decrypt the ABE
ciphertext or not. This allows us to consider two ciphertexts
$\ske.\ct_0^*, \ske.\ct_1^*$ of different plaintexts, such that when
$\ske.\ct_0^*$
is embedded in the policy, then every ABE key $\abe.\sk_u$ satisfies
the ABE relation. On the other hand, no ABE key satisfies the relation
when $\ske.\ct_1^*$ is
embedded. Observe that in the former case, $\qdk$ allows for
decryption without disturbing the state (by gentle measurement) while
in the latter case, the security of ABE ensures that no adversary can
distinguish ciphertexts of different messages\footnote{This requires
that the ABE scheme is secure even given superposition access to the
key-generation oracle.}. Hence, if we were to undetectably switch the
policy-circuit from one corresponding to $\ske.\ct_0^*$ to one with
$\ske.\ct_1^*$, we are done.

While we cannot argue this directly, our main idea is that this switch
is undetectable using OT-IND-KLA security, given that all the
SKE-CR-SKL keys $\ske.\qdk$ are returned. This can be enforced because
all the leased decryption keys $\qdk$ are required to be returned.
Consequently, the verification procedure uncomputes the ABE secret-key
register of the returned keys, followed by verifying all the obtained
SKE-CR-SKL keys $\ske.\qdk'$. However, there is one problem with the
template as we have described it so far. This is that both
$\ske.\ct_0^*, \ske.\ct_1^*$ are SKE-CR-SKL ciphertexts, which
inherently depend on the master secret-key. In order to remove this
dependence and achieve public-key encryption, the actual encryption
algorithm uses a dummy ciphertext-policy $\tlC \gets \CCSim(1^\secp)$
where $\CCSim$ is the simulator of a compute-and-compare obfuscation
scheme, a notion we will explain shortly.  We will then rely on the
security of this obfuscation scheme to switch $\tlC$ in the security proof, to
an appropriate obfuscated circuit with $\ske.\ct_0^*$ embedded in
it.
% This dummy ciphertext-policy is reminiscent of a technique by Goyal et al.~\cite{SIAMCOMP:GoyKopWat20}.\ryo{We might want to refer to \cite{STOC:GoyKopWat18} since this
% technique is reminiscent of their PLBE scheme.} \nikhil{I am not
% familiar with their work. Could you reference it?}\ryo{done.}

In more detail, the ABE scheme allows a key with attribute $u$ to
decrypt if and only if the ciphertext policy-circuit $\tlC$ satisfies
$\tlC(u) = \bot$. Observe that in the construction, $\tlC$ is
generated as $\tlC \gets
\CCSim(1^\secp)$, which outputs $\bot$ on every input $u$. Consider
now a circuit $\tlC^*$ that is an obfuscation of the circuit
$\cnc{D[\ske.\ct^*_0]}{\lock, 0}$ which is described as follows:

\begin{description}
\item {\bf Description of $\cnc{D[\ske.\ct^*_0]}{\lock, 0}:$}
\begin{itemize}
\item $\ske.\ct_0^*$ is an SKE-CR-SKL encryption of the (dummy) message $0^\secp$.
\item $D[\ske.\ct_0^*]$ is a circuit with $\ske.\ct^*_0$ hardwired. It
is defined as $D[\ske.\ct_0^*](x) = \CDec(x, \ske.\ct_0^*)$.
\item $\lock$ is a value chosen uniformly at random, independently of all
other values.
\item On input $x$, the circuit outputs $0$ if $D[\ske.\ct_0^*](x) = \lock$. Otherwise, it outputs $\bot$.
\end{itemize}
\end{description}

The above circuit belongs to a sub-class of circuits known as
compute-and-compare circuits. Recall that our goal was to avoid the use
of IO. We can get away with using IO for this sub-class of
circuits, as these so-called compute-and-compare obfuscation schemes are known
from LWE \cite{FOCS:GoyKopWat17,FOCS:WicZir17}.

The security of the obfuscation can now be used to argue that the
switch from $\tlC$ to $\tlC^*$ is indistinguishable. Note that to
invoke this security guarantee, $\lock$ must be a
uniform value that is independent of all other values. Next, we
can rely on the OT-IND-KLA security of SKE-CR-SKL to switch the ciphertext
$\ske.\ct_0^*$ embedded in the circuit $D$ to some other ciphertext
$\ske.\ct_1^*$. The switch would be indistinguishable given that the
SKE-CR-SKL decryption-keys are revoked, which we can enforce as
mentioned previously. Crucially, we will generate $\ske.\ct_1^*$ as an
encryption of the value $\lock$ corresponding to the above
compute-and-compare circuit.\footnote{Although the compute-and-compare
circuit depends on $\lock$ in this hybrid, the switch is still
justified by OT-IND-KLA security.} This ensures that for every attribute $u$
in the superposition of an SKE-CR-SKL decryption-key $\ske.\qdk =
\sum_u \alpha_u \ket{u}$, the algorithm $\CDec(u, \ske.\ct_1^*)$
outputs the value $\lock$. As a consequence, the circuit $\tlC^*$
will output $0$ instead of $\bot$, meaning the key $\abe.\sk_u$ does
not satisfy the relation in this hybrid, as desired.

It will be clear in the next subsection that our SKE-CR-SKL scheme is implied
by OWFs. Since compute-and-compare obfuscation is known from LWE
\cite{FOCS:GoyKopWat17,FOCS:WicZir17}, and so is Attribute-Based Encryption for
polynomial-size circuits
\cite{EC:BGGHNS14}\footnote{We show that their ABE scheme is secure
even with superposition access to the key-generation oracle.}, we have
the following theorem:

\begin{theorem}
There exists a PKE-CR-SKL scheme satisfying IND-KLA security, assuming
the polynomial hardness of the LWE assumption.
\end{theorem}

We also observe that using similar ideas as in PKE-CR-SKL, one can
obtain an analogous notion of selectively-secure ABE with
collusion-resistant secure-key
leasing (ABE-CR-SKL) for arbitrary polynomial-time computable circuits. Intuitively, this
primitive allows the adversary to declare a target ciphertext-attribute and then make arbitrarily many key-queries
adaptively, even ones which satisfy the ABE relation. Then, as long as
the adversary verifiably returns all the keys that satisfy the
relation, it must lose the ability to decrypt. This is captured
formally in the notion of selective IND-KLA security (Definition
\ref{def:sel_ind_ABE_SKL}). To realize this primitive, we introduce a
notion called secret-key functional-encryption with
collusion-resistant secure key-leasing
(SKFE-CR-SKL), which is a functional encryption analogue of SKE-CR-SKL.
From this primitive, we require a notion called selective
single-ciphertext security (Definition \ref{def:sel-1ct-SKFE}), that
is similar to the OT-IND-KLA security of SKE-CR-SKL. Like SKE-CR-SKL,
we observe that SKFE-CR-SKL is also implied by OWFs.
The other elements of the ABE-CR-SKL construction are the
same as in PKE-CR-SKL, namely compute-and-compare obfuscation and an
ABE scheme. Consequently, we have the following theorem:

\begin{theorem}
There exists an ABE-CR-SKL scheme satisfying selective IND-KLA security, assuming
the polynomial hardness of the LWE assumption.
\end{theorem}

\subsection{Constructing SKE-CR-SKL}

Next, we will describe how we realize the aforementioned
building-block of SKE-CR-SKL satisfying the classical decryption
property (Definition \ref{def:CDEC_SKE-CR-SKL}). Our construction will
make use of a BB84-based secret-key encryption with certified-deletion
(SKE-CD) scheme, a brief description of which is as follows. This is
an encryption scheme where the ciphertexts are quantum BB84 states
\cite{wiesner1983conjugate,BB84}.  Given such a ciphertext, an
adversary is later asked to provide a certificate of deletion. If a
certificate is provided and verified to be correct, it is guaranteed
that the adversary learns nothing about the plaintext even if the
secret-key is later revealed.  Crucially, we require that
the SKE-CD scheme also satisfies a classical decryption property
(Definition \ref{def:bb84}). Intuitively, this requires that if the ciphertext
is of the form $\skecd.\qct = \sum_u \alpha_u \ket{u}$, then every string $u$ in the
superposition can be used to decrypt correctly. Specifically, there
exists a classical deterministic algorithm $\SKECD.\CDec$ such that
$\SKECD.\CDec(\skecd.\sk, u)$ correctly decrypts $\skecd.\qct$, where
$\skecd.\sk$ is the secret-key.

Let us now recall the functionality offered by SKE-CR-SKL.
The encryption algorithm $\Enc$ takes as input a master secret-key
$\ske.\msk$
and a plaintext $\msg$ and outputs a classical ciphertext
$\ske.\ct$. The
key-generation algorithm $\qKG$ takes as input $\ske.\msk$ and produces a
quantum decryption key $\ske.\qdk$ along with a corresponding
verification-key $\ske.\vk$. Decryption of $\ske.\ct$ can be performed by $\qDec$ using
$\ske.\qdk$ without disturbing the state by more than a negligible amount. Furthermore, an
adversary that receives $q$ (unbounded polynomially many)
decryption-keys can be asked to return all of them, before which it
does not get to see any ciphertext encrypted under $\ske.\msk$. Each
returned key can be verified using the verification algorithm and
the corresponding verification key. If all $q$ keys are
verifiably returned, then it is guaranteed that the adversary cannot
distinguish a pair of ciphertexts (of different messages) encrypted
under $\ske.\msk$. This requirement, termed as OT-IND-KLA security, is
captured formally in Definition \ref{def:OT-IND-CPA_SKECRSKL}. The
intuition behind the construction is now described as follows:

Let us begin with the simple encryption algorithm. $\Enc(\ske.\msk, \msg)$
produces a classical output $\ske.\ct= (\skecd.\sk, z \seteq \msg \xor r)$, where
$\skecd.\sk$ and $r$ are values specified by $\ske.\msk$. The value
$\skecd.\sk$ is a secret key of an SKE-CD scheme $\SKECD$, while $r$
is a string chosen uniformly at random. Clearly, one can retrieve
$\msg$ from $\ske.\ct$ given $r$. Consequently, each decryption key
$\ske.\qdk$
is essentially an $\SKECD$ encryption of $r$.  The idea is that the
secret-key $\skecd.\sk$ can be obtained from $\ske.\ct$, which can then be
used to retrieve $r$ from $\ske.\qdk$. As a consequence,
collusion-resistance (OT-IND-KLA security) can be argued based on the
security of $\SKECD$ by utilizing the following observations:

\begin{itemize}
\item Each decryption-key contains an $\SKECD$ encryption of $r$ using
independent randomness.
\item The adversary must return all the decryption keys (containing
$\SKECD$ ciphertexts) before it receives the challenge ciphertext
(containing the $\SKECD$ secret-key).
\end{itemize}
Furthermore, since $\ske.\qdk$ is essentially an
$\SKECD$ ciphertext, the classical decryption property of SKE-CR-SKL
follows easily from the analogous classical decryption property of
BB84-based SKE-CD (Definition \ref{def:bb84}).

\subsection{Handling Verification Queries}

In our previous discussion about the idea behind the PKE-CR-SKL
scheme, we left out some details regarding the following. We did not discuss how the key-generation oracle of the ABE scheme can be used to
simulate the adversary's view, in the hybrid where $\ske.\ct_1^*$ is
embedded in the circuit $\tlC$. Firstly, we note that the ABE scheme
can handle superposition key-queries, which we establish by a
straightforward argument about the LWE-based ABE scheme of Boneh et
al.
\cite{EC:BGGHNS14}.
Recall now that in this hybrid, for every leased
decryption-key $\qdk = \sum_u \alpha_u \ket{u} \otimes
\ket{\abe.\sk_u}$, each
$\abe.\sk_u$ can be obtained (in superposition) by querying $\ske.\qdk
= \sum_u
\alpha_u \ket{u}$ to the oracle of ABE. However, we observe that responses to
verification queries made by the adversary are not so straightforward
to simulate. This is because for each verification query, the ABE
secret-key register needs to be uncomputed, for which we will again
rely on the ABE key-generation oracle. Specifically, the problem is
that for a malformed key $\widetilde{\qdk}$, there may exist some
$\widetilde{u}$ in the superposition which actually satisfies the ABE
relation. Recall that by definition of this relation, it follows that
$\CDec(\widetilde{u}, \ske.\ct_1^*)$ incorrectly decrypts the 
ciphertext $\ske.\ct_1^*$.
To fix this issue, we upgrade our SKE-CR-SKL
scheme to satisfy another property called key-testability. This
involves the existence of a classical algorithm $\KeyTest$ which
accepts or rejects. The property requires that when $\KeyTest$ is
applied in superposition followed by post-selecting on it accepting,
every $\widetilde{u}$ in the superposition decrypts correctly. As a
result, we are able to apply this key-testing procedure to the
register of the SKE-CR-SKL key, followed by simulating the adversary's
view using the ABE oracle. Note that we will now consider
$\qKG(\ske.\msk)$ to output an additional testing-key $\ske.\tk$ along
with $\ske.\qdk$ and $\ske.\vk$. The key-testability requirements are
specified in more detail as follows:

%Next, let us take a closer look at the key-testability property and
%how the SKE-CR-SKL scheme can be upgraded to achieve it. Note that we
%now consider $\qKG(\ske.\msk)$ to output an additional testing-key
%$\ske.\tk$ along with $\ske.\qdk$ and $\ske.\vk$. Key-testability
%requires the following properties to hold:

\begin{itemize}

\item \textbf{Security:} There exists an algorithm $\KeyTest$ such
that no QPT adversary can produce a classical value $\dk$ and a message
$\msg$ such that:
\begin{itemize}
\item $\CDec(\dk, \ske.\ct) \neq \msg$, where $\ske.\ct \gets
\Enc(\ske.\msk, \msg)$.
\item $\KeyTest(\ske.\tk, \dk) = 1$.
\end{itemize}

\item \textbf{Correctness:} 
If $\KeyTest$ is applied in superposition to $\ske.\qdk$ and the output is
measured to obtain outcome $1$, $\ske.\qdk$ should be almost
undisturbed.
\end{itemize}


\subsection{Upgrading SKE-CR-SKL with Key-Testability}

We will now explain how the SKE-CR-SKL construction is modified to
satisfy the aforementioned key-testability property. First, we mention
a classical decryption property of $\SKECD$ (Definition
\ref{def:bb84}) that we crucially rely on. A ciphertext $\skecd.\qct$
(corresponding to some message $\msg$) of $\SKECD$ is essentially a
BB84 state $\ket{x}_\theta$. The property guarantees the existence of
an algorithm $\SKECD.\CDec$ such that for any string $u$ that matches
$x$ at all computational basis positions specified by $\theta$,
$\SKECD.\CDec(\skecd.\sk, u) = \msg$ with overwhelming probability.
Recall now that in our SKE-CR-SKL construction, ciphertexts are of the
form $\ske.\ct = (\skecd.\sk, z = r \xor \msg)$ and decryption-keys
are essentially $\SKECD$ encryptions of the value $r$. Consequently,
the algorithm $\CDec$ of $\SKECRSKL$ works as follows:

\begin{description}
\item $\underline{\CDec(u, \ske.\ct)}:$
\begin{itemize}
\item Parse $\ske.\ct = (\skecd.\sk, z)$.
\item Output $z \xor \SKECD.\CDec(\skecd.\sk, u)$.
\end{itemize}
\end{description}

As a result, it is sufficient for us to ensure that no QPT adversary
can output a value $\dk$ such that
$\SKECD.\CDec(\skecd.\sk,\allowbreak \dk)$
produces a value different from $r$. By the aforementioned
classical-decryption property of $\SKECD$, it suffices to bind the
adversary to the computational basis values of a ciphertext
$\skecd.\qct = \ket{x}_\theta$. For this, we employ a technique
reminiscent of the Lamport-signature scheme \cite{Lamport79}. Thereby, an additional ``signature'' register that is entangled with
the $\SKECD$ ciphertext $\skecd.\qct = \ket{x}_\theta$ is utilized. We
note that similar
techniques for signing BB84 states were employed in previous works on
certified deletion \cite{EC:HKMNPY24,TCC:KitNisYam23}.
Specifically, let $\qreg{SKECD.CT_i}$ denote the
register holding the $i$-th qubit of $\skecd.\qct$ and $s_{i,0}, s_{i,1}$ be
randomly chosen pre-images from the domain of an OWF $f$.  Then, the
following map is performed on registers $\qreg{SKECD.CT_i}$ and
$\qreg{S_i}$ where the latter is initialized to $\ket{0 \ldots 0}$:

$$\ket{u_i}_{\qreg{SKECD.CT_i}} \otimes \ket{v_i}_{\qreg{S_i}}
\ra \ket{u_i}_{\qreg{SKECD.CT_i}} \otimes \ket{v_i \xor
s_{i,u_i}}_\qreg{S_i}$$

Let $\rho_i$ be the resulting state. The SKE-CR-SKL decryption-key
$\ske.\qdk$ is
set to be the state $\rho_1 \otimes \ldots \otimes \rho_{\ctlen}$
where $\ctlen$ is the length of $\skecd.\qct$. Thereby, the testing-key
$\ske.\tk$ will consist of the values $f(s_{i, 0}), f(s_{i, 1})$ for each
$i \in [\ctlen]$. Observe now that for a returned (possibly altered)
decryption-key $\widetilde{\qdk}$ (or $\ske.\widetilde{\qdk}$), it is possible to check for each
qubit whether the superposition term $u_i$ is associated with the
correct pre-image $s_{i,u_i}$. This can be done by forward
evaluating the pre-image register and comparing it with the value
$f(s_{i, u_i})$ that is specified in $\ske.\tk$. This is essentially the
$\KeyTest$ algorithm. It is easy to see that this procedure does
not disturb the state when applied to the unaltered
decryption key $\qdk$ (or $\ske.\qdk$). Moreover, observe that the adversary does not
receive the pre-image $s_{i, 1-x[i]}$ for any computational basis
position $i$. Consequently, we show that the adversary cannot
produce a value $\dk$ whose computational basis values are
inconsistent with those of $x$, unless it can invert outputs of $f$.
From the previous discussion, it follows that if the computational
basis values of $\dk$ are consistent with $x$, then $\dk$ cannot
result in the incorrect decryption of $\ske.\ct$.

