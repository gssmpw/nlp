\section{Previous Works}
%% Identifying
The Federal Aviation Administration (FAA) recently introduced a rule known as remote identification, aimed at increasing accountability in UAV operations by requiring UAVs to broadcast sensitive data, such as their identity \cite{Tedeschi2024, FAA2023}. However, this measure can be circumvented by rogue or unauthorized UAVs.

%% Detection
Detecting unauthorized UAVs is crucial for anti-UAV systems \cite{Sadovskis2022}. Primary detection techniques \cite{Zhang2023, You2023, Lee2023} include:
\begin{itemize}
  \item Radar systems that analyze radio waves reflected by UAVs.
  \item Acoustic sensors that identify UAVs based on rotor noise signatures.
  \item Visual or image detection methods using cameras and infrared sensors, often enhanced by deep neural networks.
\end{itemize}

%% Tracking and Pursuing
Tracking involves continuously monitoring the trajectory of detected UAVs using various technologies such as radar, acoustic sensors, visual analysis, GPS, and RF signal tracking, to maintain precise control over airspace.

%% Tracking: Radar
He et al. proposed a method combining coherent and non-coherent processing to enhance micro-UAV tracking under poor signal-to-noise ratios \cite{He2023}. An et al. developed a novel direction-of-arrival (DOA) estimator that improves tracking of unresolved scattering centers \cite{An2024}. Jin et al. examined the effects of location errors on radar-based UAV detection and introduced an adaptive beam control scheme to counteract jamming inaccuracies \cite{Jin2024}.

%% Tracking: Visual
Significant advancements in UAV tracking include:
\begin{itemize}
  \item Background-aware correlation filters by Kiani Galoogahi et al. that adjust tracking models according to environmental changes \cite{KianiGaloogahi2017}.
  \item Continuous convolution operators by Danelljan et al. for fine-grained UAV tracking \cite{Danelljan2016}.
  \item A learning response interference suppression (RIS) correlation filter by Li et al. to enhance consistency and minimize distractions during UAV tracking \cite{Li2023a}.
\end{itemize}
Li et al. also proposed a computationally efficient tracking algorithm validated using fixed-wing UAV video data \cite{Li2022}, and a general architecture for UAV-to-UAV detection and tracking within a space-air-ground integrated network (SAGIN) was introduced to provide extensive, cost-effective tracking \cite{Li2023}.

%% Neutralization
Neutralization strategies include net guns, RF jammers, and laser systems to incapacitate or disable rogue UAVs, ensuring the security of airspace \cite{Park2021, Jurn2021}.