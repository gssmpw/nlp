\begin{figure*}[t!]
    \centering
    \newcommand{\textbox}[2]{
        \begin{minipage}[t][3.8cm][b]{\linewidth}
            \raggedright\scriptsize\baselineskip=10pt
            \sffamily
            \textbf{Q:} #1 \\[5pt]
            \textbf{Choices:} \\[0pt] #2
        \end{minipage}
    }
    % 두 번째 subfigure
    \begin{subfigure}[t]{.19\linewidth}
        \centering
        \begin{tikzpicture}
            \node[anchor=south west] (img) at (0,2) {\includegraphics[width=.85\linewidth]{latex/figures/images/fig2.png}};
            \node[anchor=south west, align=left] at (0, 0) {\textbox{
                How many points are \\[0pt] on the circle?
            }{
                (A) 0 \quad (B) 1 \\[0pt]
                (C) 2 \quad (D) 3 \quad (E) 4
            }};
        \end{tikzpicture}
        \caption{Concyclic}
    \end{subfigure}
    % 세 번째 subfigure
    \begin{subfigure}[t]{.19\linewidth}
        \centering
        \begin{tikzpicture}
            \node[anchor=south west] (img) at (0,2) {\includegraphics[width=.85\linewidth]{latex/figures/images/fig3.png}};
            \node[anchor=south west, align=left] at (0, 0) {\textbox{
                How are $\overline{\rm {AB}}$ and $\overline{\rm {BC}}$ related?
            }{
                (A) Perpendicular \\[0pt]
                (B) Collinear (C) Otherwise
            }};
        \end{tikzpicture}
        \caption{TwoLines}
    \end{subfigure}
    % 첫 번째 subfigure
    \begin{subfigure}[t]{.19\linewidth}
        \centering
        \begin{tikzpicture}
            \node[anchor=south west] (img) at (0,2) {\includegraphics[width=.85\linewidth]{latex/figures/images/fig1.png}};
           \node[anchor=south west, align=left] at (0, 0){\textbox{
                What kind of object is \\[0pt] in the diagram?
            }{
                (A) Segment (B) Triangle \\[0pt]
                (C) Square (D) Pentagon
            }};
        \end{tikzpicture}
        \caption{ObjectShape}
    \end{subfigure}
    % 네 번째 subfigure
    \begin{subfigure}[t]{.19\linewidth}
        \centering
        \begin{tikzpicture}
            \node[anchor=south west] (img) at (0,2) {\includegraphics[width=.85\linewidth]{latex/figures/images/fig5.png}};
            \node[anchor=south west, align=left] at (0, 0) {\textbox{
                What is the shape of $\square \rm{ABCD}$?
            }{
                (A) Parallelogram \\[0pt]
                (B) Trapezoid (C) Rectangle
            }};
        \end{tikzpicture}
        \caption{SquareShape}
    \end{subfigure}
    % 다섯 번째 subfigure
    \begin{subfigure}[t]{.19\linewidth}
        \centering
        \begin{tikzpicture}
            \node[anchor=south west] (img) at (0,2) {\includegraphics[width=.85\linewidth]{latex/figures/images/fig4.png}};
            \node[anchor=south west, align=left] at (0, 0) {\textbox{
                What is the degree of $\angle \rm{ABC}$?
            }{
                (A) $15^\circ$ \quad (B) $20^\circ$ \\[0pt]
                (C) $25^\circ$ \quad $\cdots$ \quad (N) $75^\circ$
                % [15$^\circ$, 20$^\circ$, $\dots$, 70$^\circ$, 75$^\circ$] \\[0pt]
                % Test
            }};
        \end{tikzpicture}
        \caption{AngleDetection}
    \end{subfigure}
    \caption{Illustration of the proposed visual feature perception benchmark. We introduce five different diagram classification tasks that require visual feature perception to answer geometry-related questions.}
    \label{fig:benchmark}
\end{figure*}

% \begin{figure*}[t!]
%     \centering
%     \begin{subfigure}[t]{.19\linewidth}
%         \centering
%         \includegraphics[width=\linewidth]{latex/figures/images/objectshape.pdf}
%         \caption{ObjectShape}
%     \end{subfigure}
%     \begin{subfigure}[t]{.19\linewidth}
%         \centering
%         \includegraphics[width=\linewidth]{latex/figures/images/concyclic.pdf}
%         \caption{Concyclic}
%     \end{subfigure}
%     \begin{subfigure}[t]{.19\linewidth}
%         \centering
%         \includegraphics[width=\linewidth]{latex/figures/images/twolines.pdf}
%         \caption{TwoLines}
%     \end{subfigure}
%     \begin{subfigure}[t]{.19\linewidth}
%         \centering
%         \includegraphics[width=\linewidth]{latex/figures/images/squareshape.pdf}
%         \caption{SquareShape}
%     \end{subfigure}
%     \begin{subfigure}[t]{.19\linewidth}
%         \centering
%         \includegraphics[width=\linewidth]{latex/figures/images/angledetection.pdf}
%         \caption{AngleDetection}
%     \end{subfigure}
%     \caption{Illustration of the proposed visual feature perception benchmark. We introduce five different diagram classification tasks that require visual feature perception to answer geometry-related questions.
%     % (a) ObjectShape prompts the vision encoder to identify the number of lines present in the diagram. (b) Concyclic requires determining whether given points lie on a circle. (c) TwoLines involves detecting lines AB and BC among various geometric objects and classifying their relationship at point B. (d) SquareShape, similar to TwoLines, asks the vision encoder to recognize quadrilateral ABCD and assess the relationships among its sides. (e) AngleDetection necessitates identifying angle ABC and interpreting textual indication of its measure at point B.
%     }
%     \label{fig:benchmark}
% \end{figure*}