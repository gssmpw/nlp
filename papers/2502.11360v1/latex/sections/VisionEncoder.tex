\section{Visual Geometric Premises Recognition Benchmark for Vision Encoders}
\label{sec:visual_feature}

In this section, we first develop a benchmark for evaluating a vision encoder's performance in recognizing geometric features from a diagram. We then report the performance of well-known vision encoders on this benchmark.

%\dw{We first examine how commonly used vision encoders in open-source VLMs recognize geometric primitives, such as points and lines, and geometric premises, such as perpendicularity, from a given diagram.} Because a VLM’s performance is typically measured by its final solution in PGPS, the result does not fully capture the encoder’s capacity to identify fundamental geometric structures. 
% Yet, as the encoder is the first component to observe the diagram, understanding its recognition ability is essential for further advancement.
% To address this gap, we introduce a benchmark that thoroughly evaluates vision encoders.

\subsection{Benchmark preparation}
\label{sec:synthetic_data_engine}

We design our benchmark as simple classification tasks. By investigating PGPS datasets, we identify that recognizing \emph{geometric primitives}, such as points and lines, and geometric properties representing \emph{relations between primitives}, such as perpendicularity, is important for solving plane geometry problems. Recognized information forms \emph{geometric premises} to solve the problem successfully. To this end, we carefully curate five classification tasks as follows:
\begin{itemize}
    \item \textbf{Concyclic}: A circle and four points are given. The task is to identify how many of those points lie on the circle.
    \item \textbf{TwoLines}: Two lines, AB and BC, are given alongside other geometric objects. The task is to determine whether AB and BC are perpendicular, collinear, or neither.
    \item \textbf{ObjectShape}: A given diagram includes one of the following geometric objects: a segment, triangle, square, or pentagon. The task is to classify which object is present.
    \item \textbf{SquareShape}: A diagram including a square ABCD and other geometric objects is given. The task is to classify whether the square is a trapezoid, parallelogram, or rectangle.
    \item \textbf{AngleDetection}: A diagram is given with at least three points: A, B, and C. The task is to classify the correct angle of ABC from \(\{15^\circ, 20^\circ, \ldots, 75^\circ\}\). 
\end{itemize}
An example of each task is provided in \cref{fig:benchmark}.

Frontier language models demonstrate a remarkable mismatch between their problem-solving capabilities and poor out-of-box verification capabilities.
These limitations have largely been attributed to the inability of current language models to self-diagnose hallucinations or enforce rigour \citep{zhang_how_2023,orgad_llms_2024,snyder_early_2024,kamoi_evaluating_2024, tyen_llms_2024, DBLP:conf/iclr/0009CMZYSZ24}.
However, our findings that models can be directed to accurately perform verifications at scale suggest that these out-of-box limitations can be addressed with standard methods like instruction tuning.
We compiled a set of challenging reasoning problems and candidate solutions to provide a benchmark for these deficits.

Each entry in this benchmark consists of a question, a correct candidate response, and an incorrect candidate response, and is manually curated from the residuals of our sampling-based search experiments (Section~\ref{section:pipeline}).
An example entry from this benchmark can be found below (see Appendix~\ref{app:examplebenchmark} for more).

\vspace{0.4cm}
\begin{tcolorbox}[title=Question from LiveBench Reasoning (Web-of-Lies Puzzle), breakable]
In this question, assume each person either always tells the truth or always lies. The person at the campground thinks their friend is lying. Mateo is at the aquarium. The person at the restaurant says the person at the hotel lies. Farid is at the movie theater. The person at the movie theater says the person at the campground lies. Ryan is at the shopping mall. The person at the cafe says the person at the campground lies. The person at the observatory says the person at the museum lies. The person at the museum says the person at the restaurant tells the truth. The person at the campground says the person at the airport tells the truth. The person at the campground tells the truth. Luna is at the restaurant. The person at the movie theater saw a firetruck. Isabella is at the observatory. The person at the amusement park says the person at the museum tells the truth. The person at the restaurant says the person at the shopping mall tells the truth. Liam is at the beach. Ayaan is at the airport. The person at the restaurant says the person at the beach tells the truth. Maya is at the amusement park. The person at the restaurant says the person at the aquarium lies. The person at the restaurant says the person at the movie theater lies. The person at the library says the person at the campground tells the truth. Elowen is at the hotel. Priya is at the library. The person at the gym says the person at the restaurant tells the truth. Kehinde is at the cafe. Anika is at the museum. Quan is at the campground. Soren is at the gym. Does the person at the museum tell the truth? Does the person at the restaurant tell the truth? Does the person at the movie theater tell the truth? Think step by step, and then put your answer in **bold** as a list of three words, yes or no (for example, **yes, no, yes**). If you don't know, guess.
\end{tcolorbox}
\definecolor{mydarkgreen}{rgb}{0.0, 0.5, 0.0} %


\begin{paracol}{2}

\switchcolumn[0]
\begin{tcolorbox}[breakable, colframe=mydarkgreen!70, colback=mydarkgreen!10, title=Correct response]
Here's how we can break down this logic puzzle:

1. **Quan is at the campground and tells the truth.** This is a given fact.

...

4. **The person at the restaurant (Luna) says the person at the hotel (Elowen) lies.**  We don't know yet if Luna is lying or telling the truth.

...

10. **Since Luna tells the truth about Farid, Luna *always* tells the truth.**  This unlocks several other pieces of information.

...

13. **The person at the cafe (Kehinde) says the person at the campground lies.** Since Quan tells the truth, Kehinde is lying.

Therefore:
* **Anika (museum):** Tells the truth.
* **Luna (restaurant):** Tells the truth.
* **Farid (movie theater):** Lies.

**yes, yes, no**
\end{tcolorbox}

\switchcolumn
\begin{tcolorbox}[breakable, colframe=red!70, colback=red!10, title=Incorrect Response Due to Hallucination]
Here's how we can solve this logic puzzle:

1. **Quan is at the campground and tells the truth.** This is given.

2. **The person at the airport (Ayaan) tells the truth.** Quan (campground) says the person at the airport tells the truth, and Quan tells the truth.

...

4. **Luna (restaurant) lies.**  The person at the restaurant says the person at the hotel (Elowen) lies. \textbf{\color{red} Later, the person at the restaurant says the person at the campground (Quan) lies.} Since Quan tells the truth, Luna (restaurant) must be lying.

...

Therefore:
* **Anika (museum): lies**
* **Luna (restaurant): lies**
* **Farid (movie theater): lies**

**Answer: no, no, no** 
\end{tcolorbox}
\end{paracol}

\noindent\begin{minipage}{\textwidth}
\captionof{figure}{Example of an entry in our verification benchmark. The question is sourced from the LiveBench Reasoning benchmark, and the two responses are generated by Gemini v1.5 Pro. The green response has the correct final answer; the red response has the wrong final answer due to hallucinating a non-existent clause.}
\label{fig:example}
\vspace{0.4cm}
\end{minipage}

\noindent
On each entry, our benchmark studies verification accuracy on two tasks:
\begin{enumerate}
    \item \textbf{Scoring task.} When given only the question and one of the responses, is the model able to discern the correctness of the response?
    \item \textbf{Comparison task.} When provided the whole tuple with the correctness labels of the responses masked and a guarantee that at least one response is correct, is the model able to discern which response is correct and which is incorrect?
\end{enumerate}

\noindent
The scoring task is also evaluated over a separate set of (question, response) pairs where the response reaches the correct final answer by coincidence but contains fatal errors and should be labeled by a reasonable verifier as being incorrect; an example can be found in Appendix~\ref{app:examplebenchmark}.
In the scoring task, models are provided only with the task description; in the comparison task, models are provided only with the task description and a suggestion to identify disagreements between responses in its reasoning.

Table~\ref{tab:benchmark} lists the baseline performances of current commercial model offerings on this benchmark.
Gemini v1.5 Pro is omitted from the benchmark as the entries in the benchmark are curated from the residuals of Gemini v1.5 Pro.
The prompts used in Table~\ref{tab:benchmark} are provided in Appendix~\ref{app:benchmarkprompts}.

As we previously observed, and has been noted in prior works \citep{tyen_llms_2024, kamoi_evaluating_2024}, verification errors are typically due to low recall.
Even the easier comparison task, models perform only marginally better---and often worse---than random chance.
In many cases, Consistency@5 performs worse than one-shot inference because Consistency simply averages out noise from an output distribution, meaning that a model biased towards producing an incorrect answer will do so with higher probability under Consistency.
Addressing these deficits in verification capabilities---which we see as low-hanging fruit for post-training---would enable not only better sampling-based search, but also other downstream applications of verification including reinforcement learning \citep[e.g.][]{o1-preview,deepseekai2025deepseekr1incentivizingreasoningcapability}, data flywheeling \citep[e.g.,][]{welleck_generating_2022}, and end-user experience (see Section~\ref{sec:related} for further discussion).


\begin{table}[htbp]
\centering
\begin{tabular}{llcccccc}
\toprule
\textbf{Model} & \textbf{Metric} & \multicolumn{3}{c}{\textbf{Scoring Accuracy}} & \multicolumn{1}{c}{\textbf{Comparison Accuracy}} \\
\cmidrule(lr){3-5} \cmidrule(lr){6-6}
 &  & \textbf{Correct} & \textbf{Wrong} & \textbf{Flawed} &  \\
\midrule
\multirow{2}{*}{GPT-4o} & Pass@1    & 76.5\%  & 31.0\% & 22.2\% & 43.2\%\\
 & Consistency@5 & 77.4\% & 30.0\% & 11.1\% & 35.4\% \\
\midrule
\multirow{2}{*}{Claude 3.5 Sonnet} & Pass@1 & 89.6\% & 22.5\% & 33.3\% & 56.1\% \\
 & Consistency@5 & 90.3\% & 17.5\% & 33.3\% & 61.2\% \\
\midrule
\multirow{2}{*}{o1-preview} & Pass@1 & 100\% & 68.8\% & 80.0\% & 84.5\% \\
 & Consistency@5 & 100\% & 79.4\% & 88.8\% & 92\% \\
\midrule
\multirow{2}{*}{Gemini 2.0 Flash} & Pass@1 & 73.5\% & 44.5\% & 60\% & 58\%  \\
 & Consistency@5 & 77.4\% & 42.5\% & 66.6\% & 58.7\% \\
\midrule
\multirow{2}{*}{Gemini 2.0 Thinking Flash} & Pass@1 & 75.4\% & 56.5\% & 53.3\%  & 80\%  \\
 & Consistency@5 & 77.4\%  & 55\% & 55.5\%  & 89.1\% \\
\midrule
\multicolumn{2}{c}{Random guessing}  & 80\% & 20\% & 20\% & 50\% \\
\bottomrule
\end{tabular}
\caption{Accuracy rates of commercial language models on our verification benchmark. For the task of response scoring (Scoring Accuracy), accuracy rates are broken down for entries that require identifying a correct response as being correct (Correct), entries that require identifying a wrong response as being wrong (Wrong), and entries that require identifying a wrong response that coincidentally reaches the correct answer as being wrong (Flawed).
GPT-4o and Claude 3.5 Sonnet only perform marginally better than random guessing across all tasks. o1-Preview performs better, but still fails to identify 20-30\% of wrong responses.
}
\label{tab:benchmark}
\end{table}


Our benchmark is built on top of AlphaGeometry~\citep{alphageometry}, which is designed to solve IMO-style plane geometry problems. The program provides useful functions such as formal language describing plane diagrams. The language predefines a set of geometric premises listed in \cref{tab:alphageometry}, including all necessary properties to define our benchmark tasks. In addition, once a diagram description is given in formal language, the program renders a corresponding diagram with varying fonts, colors, widths, orientations, and resolutions, allowing us to have diagrams with diverse styles often observed in a real-world scenario.

We create question-and-answer pairs based on AlphaGeometry. To sample a diverse set of question-and-answers, we first establish a foundational geometric structure corresponding to the key problem of the task and then repeatedly add new points or lines with randomly selected geometric relationships to the existing diagram with the help of the formal language. The pseudo-code for the random question generation is presented in \cref{alg:sampling}. For each task, we generate 50,000, 10,000, and 10,000 question-and-answer pairs for training, validation, and testing, respectively.

\iffalse
We implement a data engine that randomly generates \emph{plane geometry problems}, complete with corresponding \emph{diagrams} and \emph{\geofeat{}s}, including the relationship between lines and measurements such as angles. Our engine is built on top of AlphaGeometry~\citep{alphageometry}, a proof assistant for IMO-style plane geometry problems. AlphaGeometry provides a formal language for representing plane geometry problems. 
\sh{
Given an AlphaGeometry problem, we can derive the \geofeat{}s and characteristics of the given problem such as perpendicularity and collinearity. For completeness, we provide the list of \geofeat{}s that AlphaGeometry can support in \cref{tab:alphageometry}.
In addition, we can render the diagram that represents the AlphaGeometry problem with different fonts, colors, widths, orientations, and resolutions, allowing us to have diagrams with diverse styles often observed in a real-world scenario.
}
% as an executable program. 
% AlphaGeometry can derive the solution when a problem statement is given in terms of the formal language. 
% Given an AlphaGeometry problem written in the formal language, we can render the diagram which expresses . AlphaGeometry supports diverse \geofeat{}s and characteristics of plane geometry diagrams, such as perpendicularity and collinearity. For completeness, we provide the list of \geofeat{}s that AlphaGeometry can support in \cref{tab:alphageometry}. %\cref{fig:alphageometry} illustrates the examples of the problems in the formal language with the rendered diagrams. %repeated below
% In addition, the diagram can be rendered with different fonts, colors, widths, orientations, and resolutions, allowing us to have diagrams with diverse styles often observed in a real-world scenario.

%We employ AlphaGeometry’s formal language to curate a synthetic dataset. 
To sample a diverse set of plane geometry problems, we randomly generate AlphaGeometry problems by first establishing a foundational geometric structure, and then repeatedly adding new points with randomly selected geometric relationships to existing points, drawn from the predefined set in AlphaGeometry. The pseudo-code for the sampling algorithm is presented in \cref{alg:sampling}, and randomly selected examples are provided in \cref{fig:alphageometry}. Note that samples consist of the AlphaGeometry problem and \geofeat{}s written in the text along with the corresponding diagram.

%\subsection{Benchmark for visual \geofeat{} recognition}
Based on the synthetic data engine, we develop a benchmark that evaluates how well the vision encoders recognize visual \geofeat{}s in plane geometry diagrams. In PGPS, recognizing characteristics such as collinearity, perpendicularity, and angle measures is crucial for solving geometry problems. Based on the elements, we define five diagram classification tasks that depend on accurate visual \geofeat{} perception:
\begin{itemize}
    \item \textbf{ObjectShape}: Each diagram includes one of the following geometric objects: a segment, triangle, square, or pentagon. The goal is to classify which object is present.
    \item \textbf{Concyclic}: Each diagram contains a circle and four points. The task is to identify how many of those points lie on the circle.  
    \item \textbf{TwoLines}: Each diagram includes two lines, AB and BC, alongside other geometric objects. The vision encoder must determine whether AB and BC are perpendicular, collinear, or neither.
    \item \textbf{SquareShape}: Each diagram features a square ABCD with additional objects. The goal is to classify whether the square is actually a trapezoid, parallelogram, or rectangle. 
    \item \textbf{AngleDetection}: Each diagram includes an angle ABC and other objects. The objective is to identify the measure of angle ABC from among the set \(\{15^\circ, 20^\circ, \ldots, 75^\circ\}\). 
\end{itemize}
We can generate samples used for benchmarking by adding additional conditions to the random problem generator. For example, we let the generator synthesize AlphaGeometry problems with a segment, triangle, square, or pentagon for the ObjectShape task.
For each task, we generate 50,000, 10,000, and 10,000 AlphaGeometry problems for training, validation, and test, respectively.
We then render the diagrams and corresponding labels from the generated problems, which is later used to train the vision encoders.
\cref{fig:benchmark} shows randomly generated diagrams for the classification tasks.
\fi


\subsection{Results}
\label{sec:benchmakr_results}
\begin{table}[t!]
    \centering
    \resizebox{\linewidth}{!}{
    \begin{tabular}{l l c c c c c }
        \toprule
        & Models & \begin{tabular}{@{}c@{}}Object \\ Shape\end{tabular} & \begin{tabular}{@{}c@{}}Con \\ cyclic\end{tabular} & \begin{tabular}{@{}c@{}}Two \\ Lines\end{tabular} & \begin{tabular}{@{}c@{}}Square \\ Shape\end{tabular} & \begin{tabular}{@{}c@{}}Angle \\ Detection\end{tabular} \\
        % &Models & ObjectShape & Concyclic & TwoLines & SquareShape & AngleDetection \\
        \midrule
        \multirow{4}{*}{%
    \rotatebox[origin=c]{90}{%
        \parbox{1.3cm}{\centering \footnotesize \emph{Baseline}}%
    }%
}  
        &OpenCLIP & \textbf{100.00} & 99.13 & 86.57 & 85.20 & 64.81 \\
        &SigLIP & \textbf{100.00} & \textbf{99.71} & 89.26 & 89.31 & 76.86 \\
        &DinoV2 & \textbf{100.00} & 98.01 & 85.30 & 91.24 & 22.43 \\
        &ConvNeXT & \textbf{100.00} & 99.20 & 89.39 & 88.13 & 61.84 \\
        \midrule
        \multirow{3}{*}{%
    \rotatebox[origin=c]{90}{%
        \parbox{1.3cm}{\centering \footnotesize \emph{SSL}}%
    }%
}  
        &Jigsaw & 86.11 & 63.85 & 49.98 & 61.88 & 11.44 \\
        &MAE & 93.99 & 72.25 & 71.73 & 82.70 & 13.08 \\
        &VQ-VAE & 63.05 & 60.97 & 48.10 & 57.35 & 9.22 \\
        \midrule
        \multirow{3}{*}{%
    \rotatebox[origin=c]{90}{%
        \parbox{1.3cm}{\centering \footnotesize \emph{GeoCLIP}}%
    }%
}  
        &GeoCLIP (F $\times$) & 99.52 & 98.61 & 88.33 & 86.76 & 65.68 \\
        &GeoCLIP (2K) & 99.32 & 98.73 & 94.73 & 89.22 & 74.95 \\
        &GeoCLIP & 99.21 & 99.24 & \textbf{96.05} & \textbf{95.95} & \textbf{78.56} \\
        \bottomrule
    \end{tabular}
    }
    \caption{Results on the proposed visual feature benchmark. We report the test accuracy of the models with the best validation performance. }
    \label{tab:linear_probing}
\end{table}

% \begin{table}[t!]
%     \centering
%     \resizebox{\linewidth}{!}{
%     \begin{tabular}{l l c c c c c }
%         \toprule
%         & Models & \begin{tabular}{@{}c@{}}Object \\ Shape\end{tabular} & \begin{tabular}{@{}c@{}}Con \\ cyclic\end{tabular} & \begin{tabular}{@{}c@{}}Two \\ Lines\end{tabular} & \begin{tabular}{@{}c@{}}Square \\ Shape\end{tabular} & \begin{tabular}{@{}c@{}}Angle \\ Detection\end{tabular} \\
%         % &Models & ObjectShape & Concyclic & TwoLines & SquareShape & AngleDetection \\
%         \midrule
%         \multirow{4}{*}{Baseline}
%         &OpenCLIP & \textbf{100.00} & 99.13 & 86.57 & 85.20 & 64.81 \\
%         &SigLIP & \textbf{100.00} & \textbf{99.71} & 89.26 & 89.31 & 76.86 \\
%         &DinoV2 & \textbf{100.00} & 98.01 & 85.30 & 91.24 & 22.43 \\
%         &ConvNeXT & \textbf{100.00} & 99.20 & 89.39 & 88.13 & 61.84 \\
%         \midrule
%         \multirow{3}{*}{SSL}
%         &Jigsaw & 86.11 & 63.85 & 49.98 & 61.88 & 11.44 \\
%         &MAE & 93.99 & 72.25 & 71.73 & 82.70 & 13.08 \\
%         &VQ-VAE & 63.05 & 60.97 & 48.10 & 57.35 & 9.22 \\
%         \midrule
%         \multirow{3}{*}{GeoCLIP}
%         &GeoCLIP (F $\times$) & 99.52 & 98.61 & 88.33 & 86.76 & 65.68 \\
%         &GeoCLIP (2K) & 99.32 & 98.73 & 94.73 & 89.22 & 74.95 \\
%         &GeoCLIP & 99.21 & 99.24 & \textbf{96.05} & \textbf{95.95} & \textbf{78.56} \\
%         \bottomrule
%     \end{tabular}
%     }
%     \caption{Results on the proposed visual feature benchmark. We report the test accuracy of the models with the best validation performance. }
%     \label{tab:linear_probing}
% \end{table}

% \multirow{7}{*}{%
%     \rotatebox[origin=c]{90}{%
%         \parbox{1.3cm}{\centering \emph{Unseen}}%
%     }%
% }  


With the proposed benchmark, we evaluate four widely adopted vision encoders for the open-sourced VLMs: OpenCLIP~\citep{clip}, SigLIP~\citep{siglip}, DinoV2~\citep{dinov2}, and ConvNeXT~\citep{convnext}.

To evaluate the vision encoder, we adopt a linear probing approach. Specifically, we add a linear layer on top of each encoder as a prediction head and train the linear layer from scratch while freezing the parameters of the vision encoder. We use a training set to train the prediction head and report the test accuracy with the best validation performance. The details for the hyper-parameters are described in \cref{sec:hparams}.

As shown in \cref{tab:linear_probing}, many existing vision encoders well recognize the shape of objects but fail to recognize the correct angle between two lines. The encoders also show some difficulties in recognizing the shape of a square and the relationship between two lines. Although the result may seem satisfactory at a glance, these errors will propagate to the downstream tasks when combined with LLMs. Hence, it is important to improve the recognition performance of the vision encoder further. 
%The t-SNE plots of the vision encoder embeddings are illustrated at \cref{fig:tsne_benchmark}. %Let's move this to the next section
