\section{Details of Benchmark}

\subsection{Training details}

\label{sec:hparams}

% \paragraph{Linear probing.}
To evaluate the visual feature perception of the vision encoder, we utilize a linear probing approach, which involves freezing the vision
encoder parameters and training a simple linear classifier on top of its features.

We train the linear classifier on the training set of each task for 50 epochs with batch size 128 and learning rate $1\text{e-}4$.
We use Adam optimizer for optimization.

\subsection{Visualization of the vision encoders}

We visualize the embeddings of the vision encoders used in \cref{sec:benchmakr_results} at \cref{fig:tsne_benchmark}.

\begin{figure}[t!]
    \centering
    \begin{subfigure}[t]{.32\linewidth}
        \centering
        \includegraphics[width=\linewidth]{latex/figures/images/visual_tsne_openclip.png}
        \caption{OpenCLIP}
    \end{subfigure}
    \begin{subfigure}[t]{.32\linewidth}
        \centering
        \includegraphics[width=\linewidth]{latex/figures/images/visual_tsne_siglip.png}
        \caption{SigLIP}
    \end{subfigure}
    \begin{subfigure}[t]{.32\linewidth}
        \centering
        \includegraphics[width=\linewidth]{latex/figures/images/visual_tsne_convnext.png}
        \caption{ConvNeXT}
    \end{subfigure}
    \begin{subfigure}[t]{.32\linewidth}
        \centering
        \includegraphics[width=\linewidth]{latex/figures/images/visual_tsne_dinov2.png}
        \caption{DinoV2}
    \end{subfigure}
    \begin{subfigure}[t]{.32\linewidth}
        \centering
        \includegraphics[width=\linewidth]{latex/figures/images/visual_tsne_ours.png}
        \caption{GeoCLIP}
    \end{subfigure}
    \caption{
    The embeddings of the vision encoders on the diagrams of TwoLines task. We visualize the embeddings of the vision encoders on the diagrams of TwoLines task. The blue, orange, and green dots are the diagrams where the two lines AB and BC are collinear, perpendicular, and otherwise, respectively.
    \label{fig:tsne_benchmark}
    }
\end{figure}


% \paragraph{GeoCLIP.}
% We start from OpenCLIP~\citep{clip}, a pre-trained model where the architecture is ViT-L/14 with image resolution $336\times 336$. To train OpenCLIP with GeoCLIP, we use total of 200,000 diagram-caption pairs generated with our synthetic data engine.
% We set the batch size and weight decay to 256 and 0.2, respectively.
% We optimize for 50 epochs using Adam optimizer~\citep{adam} and a cosine annealing scheduler with 2,000 warmup steps and the maximum learning rate is set to be $1\text{e-}4$.
% For the domain adaptation parts, i.e., applying CLIP on the diagram-caption pairs and the diagram pairs of target domains, we vary the batch size to 32.
