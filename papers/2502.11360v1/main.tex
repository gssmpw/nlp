% This must be in the first 5 lines to tell arXiv to use pdfLaTeX, which is strongly recommended.
\pdfoutput=1
% In particular, the hyperref package requires pdfLaTeX in order to break URLs across lines.

\documentclass[11pt]{article}

% Change "review" to "final" to generate the final (sometimes called camera-ready) version.
% Change to "preprint" to generate a non-anonymous version with page numbers.
\usepackage[preprint]{acl}

% Comments
\newcommand{\zq}[1]{\textcolor{red}{\textbf{ZQ}: #1}} 

% Standard package includes
\usepackage{times}
\usepackage{latexsym}
\usepackage{amsfonts}
\usepackage{amsmath}

% For proper rendering and hyphenation of words containing Latin characters (including in bib files)
\usepackage[T1]{fontenc}
% For Vietnamese characters
% \usepackage[T5]{fontenc}
% See https://www.latex-project.org/help/documentation/encguide.pdf for other character sets

% This assumes your files are encoded as UTF8
\usepackage[utf8]{inputenc}

% This is not strictly necessary, and may be commented out,
% but it will improve the layout of the manuscript,
% and will typically save some space.
\usepackage{microtype}

% This is also not strictly necessary, and may be commented out.
% However, it will improve the aesthetics of text in
% the typewriter font.
\usepackage{inconsolata}

%Including images in your LaTeX document requires adding
%additional package(s)
\usepackage{graphicx}

\usepackage{booktabs}       % professional-quality tables

\usepackage{multirow}
% \usepackage{autoref}
\usepackage[capitalise]{cleveref}
\crefformat{section}{\S#2#1#3} % see manual of cleveref, section 8.2.1
\crefformat{subsection}{\S#2#1#3}
\crefformat{subsubsection}{\S#2#1#3}

\usepackage{todonotes}
\usepackage{subcaption}

% \usepackage{soul}
% \usepackage[normalem]{ulem}
\usepackage{algorithm}
\usepackage[noend]{algpseudocode}

\newcommand{\tabitem}{~~\llap{\textbullet}~~}

\newcommand{\dw}[1]{\textcolor{red}{#1}}
\newcommand{\sh}[1]{\textcolor{blue}{#1}}
\newcommand{\geoclip}{GeoCLIP}
\newcommand{\geovlm}{GeoDANO}
\newcommand{\geofeat}{geometric premises}
\newcommand{\captionstyle}{GeoCLIP-style}


\usepackage{pict2e}

\newcommand{\measuredrightanglewithdot}{%
  \mathord{%
    \mspace{1mu}%
    \text{\mrawd}%
    \mspace{1mu}%
  }%
}

\newcommand{\mrawd}{%
  \setlength{\unitlength}{1ex}%
  \begin{picture}(1,1)
  \roundcap
  \polyline(0,1)(0,0)(1,0)
  \put(0,0){\arc[0,90]{0.5}}
  \put(0.2,0.2){\circle*{0.1}}
  \end{picture}%
}

\newcommand{\measuredrightanglewithsquare}{%
  \mathord{%
    \mspace{1mu}%
    \text{\msquare}%
    \mspace{1mu}%
  }%
}

\newcommand{\msquare}{%
  \setlength{\unitlength}{1ex}%
  \begin{picture}(1,1)
  \roundcap
  \polyline(0,1)(0,0)(1,0)
  \polyline(0, 0.5)(0.5, 0.5)(0.5,0)
  \end{picture}%
}

% If the title and author information does not fit in the area allocated, uncomment the following
%
%\setlength\titlebox{<dim>}
%
% and set <dim> to something 5cm or larger.

\title{GeoDANO: Geometric VLM with Domain Agnostic Vision Encoder}

% Author information can be set in various styles:
% For several authors from the same institution:
% \author{Author 1 \and ... \and Author n \\
%         Address line \\ ... \\ Address line}
% if the names do not fit well on one line use
%         Author 1 \\ {\bf Author 2} \\ ... \\ {\bf Author n} \\
% For authors from different institutions:
% \author{Author 1 \\ Address line \\  ... \\ Address line
%         \And  ... \And
%         Author n \\ Address line \\ ... \\ Address line}
% To start a separate ``row'' of authors use \AND, as in
% \author{Author 1 \\ Address line \\  ... \\ Address line
%         \AND
%         Author 2 \\ Address line \\ ... \\ Address line \And
%         Author 3 \\ Address line \\ ... \\ Address line}

% \author{
%   Seunghyuk Cho \\
%   POSTECH GSAI \\
%   \texttt{shhj1998@postech.ac.kr} \\\And
%   Zhenyue Qin \\
%   Independent researcher \\
%   \texttt{zhenyue.qin@yale.edu} \\\And
%   Yang Liu \\
%   Independent researcher \\
%   \texttt{lyf1082@gmail.com} \\\And
%   Youngbin Choi \\
%   POSTECH GSAI \\
%   \texttt{shhj1998@postech.ac.kr} \\\And
%   Seungbeom Lee \\
%   POSTECH GSAI \\
%   \texttt{shhj1998@postech.ac.kr} \\\And
%   Dongwoo Kim \\
%   POSTECH CSE \& GSAI \\
%   \texttt{shhj1998@postech.ac.kr} \\\And}

\author{
 \textbf{Seunghyuk Cho\textsuperscript{1}},
 \textbf{Zhenyue Qin\textsuperscript{3}},
 \textbf{Yang Liu\textsuperscript{3}},
 \textbf{Youngbin Choi\textsuperscript{1}},
\\
 \textbf{Seungbeom Lee\textsuperscript{1}},
 \textbf{Dongwoo Kim\textsuperscript{1,2}}
\\
 \textsuperscript{1}Graduate School of Artificial Intelligence, POSTECH,
\\
 \textsuperscript{2}Department of Computer Science and Engineering, POSTECH, 
\\
 \textsuperscript{3}Independent Researcher
\\
 \small{
   \textbf{Correspondence to:} Dongwoo Kim \href{mailto:dongwoo.kim@postech.ac.kr}{<dongwoo.kim@postech.ac.kr>}
 }
}

\begin{document}
\maketitle

\begin{abstract}
    \begin{abstract}

To develop generalizable models in multi-agent reinforcement learning, recent approaches have been devoted to discovering task-independent skills for each agent, which generalize across tasks and facilitate agents' cooperation. However, particularly in partially observed settings, such approaches struggle with sample efficiency and generalization capabilities due to two primary challenges: (a) How to incorporate global states into coordinating the skills of different agents? (b) How to learn generalizable and consistent skill semantics when each agent only receives partial observations? To address these challenges, we propose a framework called \textbf{M}asked \textbf{A}utoencoders for \textbf{M}ulti-\textbf{A}gent \textbf{R}einforcement \textbf{L}earning (MA2RL), which encourages agents to infer unobserved entities by reconstructing entity-states from the entity perspective. The entity perspective helps MA2RL generalize to diverse tasks with varying agent numbers and action spaces. Specifically, we treat local entity-observations as masked contexts of the global entity-states, and MA2RL can infer the latent representation of dynamically masked entities, facilitating the assignment of task-independent skills and the learning of skill semantics. Extensive experiments demonstrate that MA2RL achieves significant improvements relative to state-of-the-art approaches, demonstrating extraordinary performance, remarkable zero-shot generalization capabilities and advantageous transferability.

 % Additional rewards transform the original MTRL problem into a multi-objective MTRL problem, and the coupling relationship between the outputs of SP and ACP further complicates the optimization process. To solve this challenge, TSAC assigns a virtual expected budget to convert the multi-objective MTRL into a constrained single-objective formulation and then employs the Lagrangian method to transform a constrained single-objective optimization into an unconstrained one. The multiplier in the Lagrangian method automatically adjusts the weights during the training process, promoting cooperation between SP and ACP.
\end{abstract}
\begin{IEEEImpStatement}
The Current policies trained by Multi-Agent Reinforcement Learning (MARL) predominantly rely on meticulously designed structured environments, which considerably constrain the agents' generalization capabilities across multitasking and cross-task skill reuse. In this paper, we design a novel masked autoencoders for MARL to coordinate the skills of different agents and learn generalizable and consistent skill semantics when each agent only receives partial observations. Experimental results demonstrate that our proposed MA2RL framework significantly enhances both the asymptotic performance and generalization capabilities of the generalizable models. Specifically, MA2RL introduces masked autoencoders tailored for MARL, aimed at enhancing generalizable models. The framework holds promise for inspiring further explorations into the generalization of multi-agent reinforcement learning.
\end{IEEEImpStatement}


% Note that keywords are not normally used for peerreview papers.
\begin{IEEEkeywords}
Multi-Agent reinforcement learning, generalization, self-supervised learning.
\end{IEEEkeywords}


\IEEEpeerreviewmaketitle
\end{abstract}

% 
% 
The widespread integration of communication networks and smart devices in modern control systems has increased the vulnerability of industrial systems to online cyber-attacks, e.g., Industroyer, Blackenergy, etc \citep{osti_1505628}.
% Modern control systems have seen a large push to include communication networks and smart devices to increase performance, made possible by improvements in communication device cost and energy consumption. This trend has been coupled with the usage of open-standard communication protocols among industrial control systems, making them vulnerable to online cyber-attacks such as Industroyer, Blackenergy, etc \citep{osti_1505628}. 
To counter this, methods have been developed to improve security by achieving attack detection, mitigation, and monitoring, among others \citep{sandberg2022secure}. This paper focuses on active attack diagnosis to mitigate stealthy attacks. 
%
%\subsection{Literature review}

Active diagnosis techniques rely on the inclusion of additional moduli to control systems
% inclusion within the control system of additional moduli 
to alter the behavior of the system compared to information known by the attacker. 
For instance, the concept of additive watermarking was introduced in \cite{mo2015physical}, where noise signals of known mean and variance are added at the plant and compensated for it at the controller. 
This compensation, however, is not exact, causing some performance degradation. Thus, trade-offs between performance and detectability  are necessary \citep{zhu2023detection}.
% A later work \citep{zhu2023detection} designs the watermark signal by trading performance for detection. Thus, although additive watermarking serves as a good detection scheme, they endure performance losses even in the nominal case. 

In encrypted control \citep{darup2021encrypted}, the sensor data is encrypted, sent to the controller, and then operated on directly. Encrypted input signals are sent back to the plant for decryption. Although encryption is widespread in IT security, in control systems it presents some concerns, such as the introduction of time delays \citep{stabile2024verifiable}, while it may present inherent weaknesses \citep{alisic2023model}.
% they are not preferred as they introduce time delays \citep{stabile2024verifiable} which can cause instability, and some encryption schemes can be very weak  \citep{alisic2023model}. 

In moving target defense \citep{griffioen2020moving}, the plant is augmented with fictitious dynamics, known to the controller. The plant output is transmitted to the controller along with the fictitious states over a network under attack. 
The additional measurements then aide in the detection of attacks. 
This comes at the cost of higher communication bandwidth needs, which increases rapidly with the dimension of the augmented systems.
% Since the dynamics of the fictitious dynamics are exactly known to the controller, the attack is detected easily. However, when the scale of the system increases, the communication bandwidth used by moving the target defense approach increases rapidly. 

Other recently proposed works include two-way coding \citep{fang2019two}, a weak encryuption technique, and dynamic masking \citep{abdalmoaty2023privacy}, which enhances privacy as well as security, have been shown to be effective against zero-dynamics attacks.
% Two-way coding \citep{fang2019two} and dynamic masking \citep{abdalmoaty2023privacy} are other recently proposed approaches. Two-way coding is another form of weak encryption technique whilst dynamic masking proposes an architecture that enhances both privacy and security. These schemes are shown to be effective against zero dynamics attacks but remain to be studied for other classes of attacks. 
% Recent extensions include \citep{mukherjee2021secure,ramos2024privacy}.
% Some other works which are related are \citep{mukherjee2021secure}, an extension of \cite{fang2019two}. The work \citep{ramos2024privacy} is an extension of moving target defense for multi-agent systems. 
Furthermore, filtering techniques for attack detection are proposed by \cite{murguia2020security,hashemi2022codesign,escudero2023safety}, while not focusing on stealthy attacks.
% The works \citep{murguia2020security,hashemi2022codesign,escudero2023safety} develop filtering techniques to guarantee safety, without being focused on stealthy covert attacks.

Multiplicative watermarking (mWM) has been proposed by the authors as a diagnosis technique \citep{ferrari2020switching}. mWM consists of a pair of filters on each communication channel between the plant and its controller; the scheme is affine to weak encryption, whereby ``encoding'' and ``decoding'' are done by changing signals' dynamic characteristics through inverse pairs of filters. This enables original signals to be recovered exactly, and thus does not lead to performance degradation.
% A multiplicative watermark is an affine to a weak encryption technique, through which the signal is ``encoded'' by a filter, changing its dynamic behavior. The use of inverse pairs means that the original signal can be recovered, through ``decoding'' via an inverse filter. As such, differently to techniques based on additive watermarking, no performance is lost due to the injection of noise, and there are no bandwidth limitations.

%\subsection{Contributions}
One of the critical features of multiplicative watermarking is that to detect stealthy attacks, the mWM filter parameters must be switched over time. In this paper, an algorithm to optimally design the mWM parameters after a switching event is presented, enhancing detection performance, without changing the switching time.
% This is done without changing the switching time, which is taken as given.

\textcolor{black}{
To formalize the filter design problem, we suppose the defender is interested in optimal performance against adversaries injecting covert attacks with matched system parameters \citep{smith2015covert}, including the mWM parameters prior to the switch. This scenario represents a worst case where malicious agents can take full control of the system while remaining undetected.
Thus, the attack strategy is explicitly included within the formulation of the closed-loop system, and the mWM filters are chosen by solving an optimization problem minimizing the attack-energy-constrained output-to-output gain (AEC-OOG) \citep{anand2023risk}, a variation of the output-to-output gain proposed in  \cite{teixeira2015strategic}.
}
The main contributions of this paper are:
% We consider an adversary injecting a covert attack with matched system parameters \citep{smith2015covert}, i.e., an attacker with full knowledge of the control system parameters, including those of the mWM filters before the switch. This scenario is taken as a worst case, as it has been shown that this class of attacks can be made stealthy. To quantitatively define a cost, the output-to-output gain (OOG) \citep{teixeira2015strategic} is leveraged,
% a metric introduced to evaluate the impact of an additive attack in a control system. %Specifically, OOG evaluates the worst-case performance loss that an attacker injecting an undetectable attack can obtain. 
% Here, the maximum performance loss caused by a stealthy adversary with limited energy is taken, the attack-energy-constrained OOG (AEC-OOG) \citep{anand2023risk}. The main contributions of this paper are:
\begin{enumerate}
%[label=\alph*.]
\item The problem of optimally designing the switching mWM filters is formulated as an optimization problem, with the AEC-OOG is taken as the objective;%where the AEC-OOG is taken as the impact metric; 
\item The worst-case scenario of a covert attack with exact knowledge of plant and mWM filter parameters is embedded within the design problem;
% The optimization problem is defined to incorporate the worst-case scenario of a covert attack with exact knowledge of plant and mWM filter parameters;
\item The feasibility of the optimization problem is shown to be dependent only on stability conditions; 
\item A solution scheme is proposed to promote randomization of the mWM filter parameters such that an eavesdropping adversary cannot remain stealthy.
\end{enumerate} 

This builds on the results of \cite{ferrari2020switching}, where the focus was on the design of the switching protocols, rather than the parameters themselves.
Compared to previous work \citep{gallo2021design}, this paper introduces an optimization problem which is always feasible (thanks to the use of AEC-OOG in the objective), while also considering a more sophisticated class of covert attacks, where the presence of watermark is known to the adversary. 
Moreover, this paper poses a different objective than \citep{zhang2023hybrid}; indeed, while \citep{zhang2023hybrid} provided a design strategy to ensure certain privacy properties, in this paper we address the problem of optimal parameter design following a switching event.


%\subsection{Organization}
The rest of the paper is organized as follows. 
After formulating the problem in Section~\ref{sec:PF}, we propose our design algorithm in Section~\ref{sec:main}, and analyze its properties. It is then evaluated through a numerical example in Section~\ref{sec:NE}, and concluding remarks are given Section~\ref{sec:Con}.
% We provide the problem background in Section~\ref{sec:PF}. We formulate the design problem in Section~\ref{sec:main}, together with an analysis of its properties. The proposed algorithm is evaluated through a numerical example in Section \ref{sec:NE}. Concluding remarks are offered in Section \ref{sec:Con}.
\section{Related Work}

\subsection{View-Dependent Control}
View-dependent representations have a long history in computer graphics.
In his pioneering work, Rademacher proposed interpolating between \textit{key viewpoints} and associated \textit{key deformations} to manipulate 3D models~\cite{rademacher1999view}.
Other researchers have extended the idea to create 3D animation systems~\cite{10.1111:j.1467-8659.2004.00772.x}, streamline the modeling process~\cite{DBLP:journals/corr/abs-2103-15472}, and integrate physical simulation\cite{koyama2013view}.
Of particular note, Rivers et al.~\cite{rivers25Dcartoonmodels} introduced \textit{2.5D Cartoon Models}, a combination of planar meshes transformed, based upon view angle, so as to appears three dimensional.
Our work draws upon these works but is, to our knowledge, the first work to attempt to use view-dependent techniques to retarget 3D motion onto 2D characters.   

\subsection{Animation from 2D Images}

% output is still 2D
Many researchers have proposed different methods for creating animations from 2D images. Hornung et al.~\cite{Hornung2007anim2Dpicmotion} presented a method to deform a character from a photograph given user-provided joint annotations.
\textit{Toonsynth}~\cite{Dvoroznak18-SIG} and \textit{Neural Puppet}~\cite{poursaeed2020neural} both present methods to create new images of hand-drawn characters from examples.
% output is 3D model
Other researchers have proposed methods of obtaining 3D geometry from 2D sketches~\cite{igarashi2006teddy, Dvoroznak20-SA} and images~\cite{ArtiSketch,weng2019photo}.
% focus on sketches specifically
A number of works have specifically focused on childlike drawings.
Lingens et al.~\cite{lingens2020towards} proposed an evolutionary algorithm for animating children's drawings. 
\textit{MagicToon}~\cite{feng2017magictoon} creates a 3D model from childlike drawings for AR applications.
Similar to our work, Smith et al.~\cite{SmithHodgins} proposed a method for animating childlike drawings using 3D skeletal motion. 
However, the resulting animations are only suitable for use in 2D applications and the type of motions it supports are limited.

Unlike these previous works, our solution can be used in 3D contexts but does not create a 3D model. We instead relying upon a view-dependent formulation of the animated character.
% \section{Mobile Networks Powered by \glspl{LLM}}
\label{sec:LLM_enabled_MNs}
\begin{figure*}[t!]
\centering
\includegraphics[width=.99\textwidth]{Fig1.eps}
    \caption{Possible architectural designs for integrated \gls{LLM} and \gls{MNO} ecosystem.}
    \label{fig:LLM_possible_architectures}
\end{figure*}
The historical data of the \gls{MNO}, archived over years of expertise, constitutes a solid foundation for training the \gls{LLM} using structured and unstructured multi-modal inputs (as illustrated in Fig.~\ref{fig:LLM_possible_architectures}a) such as user intents, network logs, alarm descriptions, trouble tickets, \gls{PCAP} files (e.g. from Wireshark or tcpdump), dashboard screenshots, audio recordings (e.g. from \gls{IVR} systems), video feeds (e.g. from infrastructure surveillance), and \gls{NWDAF} analytics. To this end, a separate collection framework aggregates data from various sources into a centralized repository, and extracts most informative features such as warnings, error codes, timestamps, and user/gNB/session/bearer/\gls{QoS} flow/slice IDs. The extracted features are then converted into unified embeddings that are combined into a common vector space with suitable metadata (e.g. to differentiate data formats). The resulting vector store is used to fine-tune the \gls{LLM} to deeply internalize \gls{MNO}-specific knowledge \cite{Bariah2023understanding}. This allows the \gls{LLM} to learn patterns, sequences, and deviations that correlate with normal or faulty network operations. This is made possible using a timestamp-based cross-referencing to link different entries from several data sources, allowing detailed description and context for each flagged event as well as the resolution workflow for the spotted anomalies.

In live mobile networks, fresh multi-modal data is continuously fed into the \gls{LLM}, either uploaded in batches or streamed in real-time. The \gls{LLM} analyzes this data and identifies potential anomalous behaviors in light of its accumulated learning. In case of new anomalies not covered during the fine-tuning stage, the \gls{LLM} can rely on clustering techniques to group similar patterns and flag outliers as suspected behaviors. The \gls{LLM} is also capable of using \gls{RAG}-enabled external knowledge databases such as \gls{3GPP} documents \cite{Said2024instruct}, \gls{IEEE} standards, \gls{IETF} RFCs and vendors documentation \cite{soman2023observations} to compare the actual network behavior with the expected one to identify misconfigurations and spot unusual trends in protocols and communication flows. Well-crafted prompts, on the other hand, can guide the \gls{LLM} responses to provide focused solutions. Paradigms such as the \gls{CoT} reasoning can be used to break down the \gls{LLM} insights into a series of simplified and actionable sub-tasks. It can be extended by the \gls{ToT} technique to explore different reasoning paths and identify the most optimal solution. The \gls{LLM} can naturally produce stepwise reasoning if datasets used for fine-tuning contain \gls{CoT} and \gls{ToT} examples, or through creative prompting \cite{Zhou2024survey}. In parallel, \gls{NOC} engineers can intervene to confirm, guide or reject the \gls{LLM} findings, if needed, e.g. using its intuitive conversational interface. Through continuous self-learning, the \gls{LLM} will dynamically adapt to evolving network conditions, optimizing its performance over time \cite{Chaparadza2023optimization}.

%For instance, when a network experiences latency issues, the \gls{LLM} seamlessly analyze multi-modal information from diverse origins to identify the root cause, e.g. overloaded \gls{UPF} due to insufficient capacity, and then suggest a solution, e.g. step-by-step instructions including suitable code scripts for the involved \glspl{NF} to autonomously reroute traffic or modify policies. Conventional 5G networks can only alert about anomalies using suitable \gls{NWDAF} analytics that track the violated thresholds and notify the \gls{OAM} center to display the details on complex dashboards.

By incorporating \glspl{LLM} (e.g. as \glspl{NF}) into upcoming 6G networks, expected to be designed with \gls{SbD} principles \cite{Khaloopour2024Resilience}, \glspl{LLM} will naturally inherit the same built-in security safeguards rather than adding them as an afterthought. This design-driven approach focuses on proactive threat management, end-to-end encryption, authentication, network slicing isolation, \gls{AI}-driven threat detection with automated reactions, and stateless designs, fostering a resilient \gls{LLM}.
%The design-driven security in 5G and upcoming 6G networks ensures that security is natively integrated at every layer of the architecture rather than added as an afterthought. This approach focuses on proactive threat management, end-to-end encryption, authentication, network slicing, and \gls{AI}-driven threat detection and automated reactions to counter evolving cyber threats.



\section{GeoCLIP-DA}

\subsection{Domain adaptation data}

We adopt GeoCLIP to the two PGPS benchmarks: GeoQA~\citep{geoqa} and PGPS9K~\citep{pgps}.
For PGPS9K, we use the Geometry3K split.
\cref{fig:domain_adaptation_samples} shows the pairs used to adapt the domain of GeoCLIP.

\begin{figure}[t!]
    \centering
    \includegraphics[width=\linewidth]{latex/figures/images/domain_adaptation_samples.pdf}
    \caption{Examples of diagram pairs curated for domain adaptation. For each row, the first diagram is from the target domain, and the remaining diagrams are from the source domain. To generate source domain diagrams, we translate the target diagram by our diagram generator with the textual description of the target image.}
    \label{fig:domain_adaptation_samples}
\end{figure}

\subsection{Training details}

We start from OpenCLIP~\citep{clip}, a pre-trained model where the architecture is ViT-L/14 with image resolution $336\times 336$. To train OpenCLIP, we use total of 200,000 diagram-caption pairs generated with our synthetic data engine.
For the domain adaptation to GeoQA and Geometry3K datasets, we randomly sample 50 diagrams and translate the diagram and caption styles following the procedure described in \cref{sec:domain_adaptation}. Finally, \geoclip{} is fine-tuned via \cref{eqn:da}.
We name the GeoQA and Geometry3K adopted \geoclip{} as \geoclip{}-DA.

We set the batch size for the source domain diagram-caption pairs to 256. 
For the domain adaptation parts, i.e., applying CLIP on the diagram-caption pairs and the diagram pairs of target domains, we vary the batch size to 32.
We set weight decay to 0.2.
We optimize for 50 epochs using Adam optimizer~\citep{adam} and a cosine annealing scheduler with 2,000 warmup steps and the maximum learning rate is set to be $1\text{e-}4$.
We train the model with eight RTX3090 GPUs for approximately 24 hours.

\section{Improving the Vision Encoder Geometric Premises Recognition}

In this section, we first propose \geoclip{}, a new vision encoder designed to recognize geometric premises from diverse styles of diagrams.
To transfer the recognition to real-world PGPS benchmarks, we then propose a domain adaptation technique for \geoclip{} that leverages a small set of diagram–caption pairs from the target domains. 

% \subsection{Overall architecture}

% We begin by summarizing the architecture of our VLM, a combination of a vision encoder and a language model. For the vision encoder, we use \geoclip{}-DA, introduced in \cref{sec:domain_adaptation} with a two-layer MLP of GeLU activation as the projection layers following LLaVA-OneVision~\citep{llava-next}. For the language model, we employ LLama-3-8B-Instruct~\citep{llama}.
% For a given diagram and question pair in PGPS, we feed the vision encoder with the given diagram, and then the output of the encoder is used as an input token of LLM through the projection layer. The question text is then fed into the LLM, followed by the diagram embedding.
%The input to the language model is a concatenation of the diagram embedding, the textual input, and previously generated tokens, following a scheme similar to LLaVA~\citep{llava}.


\subsection{\geoclip{}}
% \subsection{Improving visual \geofeat{} recognition}
\label{sec:geoclip}

To make a vision encoder recognize geometric diagrams better, we propose a \geoclip{}, a vision encoder trained with CLIP objective with a newly developed 200,000 diagram-caption examples.
From the random diagram generator developed in \cref{sec:synthetic_data_engine}, we additionally sample 200,000 diagrams written in the formal language. Directly rendering these samples can result in a diagram that may not preserve the geometric properties. For example, the perpendicularity between two lines cannot be observed from the diagram without having the right angle sign, i.e., $\measuredrightanglewithsquare$. Therefore, we ensure to render the images containing all necessary geometric premises from its visual illustration.

For the caption of a diagram, we filter out some geometric properties from the original description of a diagram used to render the image. Specifically, we only keep the following four properties, concyclic, perpendicularity, angle measures, and length measures, from the visual premises shown in \cref{tab:alphageometry}. After that, we convert the remaining descriptions written in the formal language into natural language. We filter out some properties for two reasons.
First, some properties are not recognizable from the rendered diagram without additional information, e.g., congruency. These properties are listed as non-visual premises in \cref{tab:alphageometry}. Second, collinearity and parallelity occur so frequently that they can marginalize others.
Some examples of generated captions after filtering and translating are provided in the right-most column of \cref{fig:alphageometry}. We call the filtered caption as \emph{\captionstyle{} caption}.

With this dataset, we fine-tune OpenCLIP~\citep{clip} according to the CLIP objective which is formulated as:
% To be specific, given a collection of image–caption pairs \(\mathcal{D} := \{(D_i, X_i)\}_{i=1}^N\), we train the vision encoder \(g\) and the text encoder \(h\) using the following contrastive learning objective:
\begin{align}
    &\mathcal{L}_{\textrm{CLIP}}(\mathcal{D}, g, h) := \nonumber \\
    &\,\,\,\,\,\mathbb{E}_{\mathcal{D}} \!\biggl[ -\log \frac{\exp \bigl( g(D_i)^T \, h(X_i) / \tau \bigr)}{\sum_{X \in \{X_i\}_i} \exp \bigl( g(D_i)^T \, h(X) / \tau \bigr)} \biggr],
    \label{eq:clip}
\end{align}
where \(\mathcal{D} := \{(D_i, X_i)\}_{i=1}^N\) is the diagram-caption pairs, $g$ is the vision encoder, $h$ is the text encoder, and \(\tau\) is a temperature parameter.
% in \cref{eq:clip}, 
We named the resulting vision encoder as \geoclip{}. \cref{sec:hparams} provides the details, including hyper-parameters.

We compare the performance of \geoclip{} to other self-supervised approaches trained with the same dataset. We test three self-supervised approaches: Jigsaw~\citep{geoqa, geoqa-plus}, MAE~\citep{scagps, geox}, and VQ-VAE~\citep{unimath} used in previous work to improve the recognition performance of plane diagrams. We use the same architecture used for \geoclip{} for Jigsaw and MAE with the hyper-parameters used in the previous works. For VQ-VAE, we follow the architecture of \citet{unimath}. 
All model performances are measured through the linear probing used in \cref{sec:benchmakr_results}. 
% We measure the performances of the models through the linear probing used in \cref{sec:benchmakr_results}.

As shown in \cref{tab:linear_probing}, \geoclip{} recognizes geometric features better than existing baselines and self-supervised methods. The self-supervised approaches generally perform poorly for the benchmark, justifying the choice of the objective. We also compare the performance of \geoclip{} against other encoders such as OpenCLIP. Note that although we outperform the other encoders in difficult tasks such as SquareShape and AngleDetection, these results might be \emph{unfair} since the training set of \geoclip{} is similar to the diagrams in the benchmark.
The t-SNE plots of the embeddings from the vision encoders are illustrated at \cref{fig:tsne_benchmark}.

We further ablate the filtering process in \geoclip{}. To this end, we compare \geoclip{} with its two variants: \emph{GeoCLIP (F $\times$)}, which uses the captions generated without filtering. We also test \emph{GeoCLIP (2K)}, which is trained on only 2,000 pairs, to see the effectiveness of the large-scale dataset.
The results in \cref{tab:linear_probing} imply both the filtering and the training set size matter in enhancing geometric properties recognition.

% \paragraph{Solution program and numerical measurements prediction.}

% We then change the output of PGPS to predict not only the solution programs but also the numerical measurements in the diagram and text.
% While the previous PGPS models extract the numerical measurements from the text or assume as given features, our revised target enables problem solving even if the numerical measurements are only provided in the diagram. 

% Furthermore, the utilization of the vision encoder in the VLM would be enhanced due to the numerical measurements prediction from the diagram.

%To achieve this, we employ the diagram and geometric properties pairs obtained from the programs generated by the synthetic data engine.
%In forming the textual captions, we retain only the following geometric properties: concyclic, perpendicularity, and angle measure, which are basically visual features. Other properties are excluded for two reasons. First, certain properties cannot be inferred from the diagram without additional information, e.g., congruency, so those are considered non-visual features. Second, some visual features occur so frequently that they overshadow others, such as collinearity and parallelity. Consequently, those are removed as well.
% \sh{Examples of generated diagram and caption pairs are visualized at \cref{fig:geoclip}.}
% \todo{We need an example of this process in Appendix (which information is retained and removed from the original description.)}

%\cref{fig:alphageometry} shows the examples of sampled diagram-caption pairs.
%We produce a total of 200,000 diagram–caption pairs. 

%Using this dataset, we train OpenCLIP~\citep{clip}, whose architecture is ViT-L/14~\citep{vit} with an image resolution of $336 \times 336$, according to the CLIP objective at \cref{eq:clip}. Additional details regarding hyper-parameters are provided in \cref{sec:geoclip_hparams}.


\subsection{Domain adaptation of \geoclip{}}
% \subsection{Enhancing domain generalization capability}
\label{sec:domain_adaptation}

Although \geoclip{} enhances the geometric premises recognition on the benchmark set, the diagram styles in existing PGPS benchmarks differ, necessitating further adaptation. To overcome this challenge, we propose a domain adaptation method for \geoclip{}. To this end, we propose a few-shot domain adaptation method utilizing a few labeled diagrams.

% to transfer the knowledge from a source to a target domain. %To overcome this challenge, we propose a simple yet effective domain adaptation method that makes the vision encoder focus on important geometric information instead of irrelevant attributes, such as color and font family.
A domain-agnostic vision encoder must match the same diagrams drawn in different styles. To do so, we need a target domain diagram translated into the source domain style or the source diagrams translated into the target domain style. With these translated images, we can guide the model to focus on key geometric information instead of irrelevant attributes, such as color and font family. However, in practice, it is difficult to obtain the same diagrams with different styles. 

We develop a way to translate the target diagrams into source style.
Thankfully, since well-known PGPS datasets come with diagram captions written in formal languages~\citep{intergps}, we can easily convert them to the AlphaGeometry-style descriptions. 
Given the translated descriptions, we utilize the rendering engine of AlphaGeometry to translate the target domain images into the source domain. 
With the translation, we can generate the same diagram in the source domain style. 
\cref{fig:domain_adaptation_samples} provides examples of the diagram pairs with different styles. However, in some cases, the original description contains geometric premises that are unrecognizable from the diagram, such as \(\angle ACB = 35.0\) in \cref{fig:geoqa_example}. Therefore, we apply the same filtering process used in \geoclip{} to translate the AlphaGeometry-style descriptions into natural languages.

%In practice, however, not only do the diagrams have different styles, but also the captions. For example, the textual description of the GeoQA diagram in \cref{fig:geoqa_example} includes non-visual details, such as \(\angle ACB = 35.0\), which we removed from the dataset used to train \geoclip{}. Therefore, we apply the same filtering process used to curate captions for \geoclip{}. To this end, 

Formally, let $\mathcal{D}_{S} := \{(D_S^{(i)}, X_S^{(i)}) \}_{i=1}^{N_S}$ be the diagram-caption pairs from source domain $S$, e.g., the synthetic diagrams, and let $\mathcal{D}_{T_j} := \{(D_{T_j}^{(i)}, X_{T_j}^{(i)}) \}_{i=1}^{N_{T_j}}$ be the set of diagram-caption pairs of target domain $T_j$, e.g., the PGPS benchmarks. With the translation process described above, we can synthesize a style-transferred diagram-caption pair $(\hat{D}_{T_j}^{(i)}, \hat{X}_{T_j}^{(i)})$ for each diagram $D_{T_j}^{(i)}$ and caption $X_{T_j}^{(i)}$ in target domain $T_j$.

We perform domain adaptation by fine-tuning the vision encoder through the style-transferred diagram-caption pairs.
Let $\hat{\mathcal{D}}_{T_j}$ be a collection of the original diagram and style-transferred captions, i.e., $\hat{\mathcal{D}}_{T_j} = \{(D_{T_j}^{(i)}, \hat{X}_{T_j}^{(i)})\}_{i=1}^{N_{T_j}}$, and let $\hat{\mathcal{D}}_{T_jS}$ be a collection of the original and style transferred diagram pairs, i.e., $\hat{\mathcal{D}}_{T_jS}= \{(D_{T_j}^{(i)}, \hat{D}_{T_j}^{(i)}) \}_{i=1}^{N_{T_j}}$. The cross-domain adaptation objective is written as
%jointly training with diagram–caption pairs from both the source and target domains, as well as with the diagram pairs produced by the synthetic engine. Formally, we update the vision and text encoders $g$ and $h$ using the CLIP objective in \cref{eq:clip}, defined as:
\begin{align}
    &\mathcal{L}_{\textrm{CLIP-DA}}(\mathcal{D}_S, \{\mathcal{D}_{T_j}\}_j, g, h) := 
    \mathcal{L}_{\textrm{CLIP}}(\mathcal{D}_S, g, h) + \nonumber\\
    &\,\,\,\,\,\,\,\,\Sigma_j \mathcal{L}_{\textrm{CLIP}}(\hat{\mathcal{D}}_{T_j}, g, h) + \mathcal{L}_{\textrm{CLIP}}(\hat{\mathcal{D}}_{T_jS}, g, g),
    \label{eqn:da}
\end{align}
where $g$ and $h$ are the vision and text encoders of \geoclip{}, respectively. Note that we do not use the original captions from the target domain, since our goal is to adapt the vision encoder to the target domain, not the text encoder.
% We refer to the domain-adapted vision encoder as \geoclip{}-DA.
% Figure x illustrates the vision encoder pre-training phase.

% \paragraph{Unified solution program language.}

% Although GeoCLIP-DA is capable of capturing relevant visual \geofeat{}s across diverse diagram styles, the difference in solution program languages often restrict the VLM to fully leverage the enhanced visual embeddings. To address the gap, we unify the programming languages of the PGPS datasets, i.e., GeoQA and PGPS9K, by implementing a converter that transforms the solution programs from GeoQA into the PGPS9K format.
% We also convert the prediction target to predict both solution program and the numerical measurements as mentioned in \cref{sec:geoclip}.
% Examples of the training data are at \cref{fig:training_data}.

% \subsection{Training details - \dw{can we move this to experiments?}}

% For the remaining experiments, we designate GeoQA and PGPS9K as the target domains. In GeoQA, $N_{T_j}=50$ diagrams are sampled and manually annotated with \geoclip{}–style captions, followed by the generation of $N_{S_j}=100$ synthetic diagrams for each sampled diagram.
% In PGPS9K, where visual features are explicitly provided, the given features are converted into natural language to produce \geoclip{}–style captions. From this set, 50 diagram–caption pairs are chosen, and corresponding AlphaGeometry programs are generated. Subsequently, each sampled target diagram is supplemented with $N_{S_j}=100$ synthetic diagrams for domain adaptation.

% Next, we build the VLM using GeoCLIP-DA. We freeze the vision encoder and train the projection and language model on the revised training data of GeoQA and PGPS9K. Further details on hyper-parameters are provided in \cref{sec:vlm_hparams}.
In this section, we empirically compare the proposed algorithm on both sequence windows and time windows with existing methods.
\paragraph{Datasets} For the sequence-based model, we used two synthetic datasets and two cross-language datasets. The statistics of the datasets are provided in Table \ref{table:statistics}:

\begin{table}[t]
    \centering
    \caption{The statistics of the datasets. The datasets satisfy $1 \leq \|\vx\|\|\vy\| \leq R $.}
    \label{table:statistics}
    \begin{tabular}{|c|c|c|c|c|c|}
    \hline
        Dataset & $n$ & $m_x$ & $m_y$ & $N$ & $R$ \\ \hline
        SYNTHETIC(1) & 100,000 & 1,000 & 2,000 & 50,000 & 65 \\ \hline
        SYNTHETIC(2) & 100,000 & 1,000 & 2,000 & 50,000 & 724 \\ \hline
        APR & 23,235 & 28,017 & 42,833 & 10,000 & 773 \\ \hline
        PAN11 & 88,977 & 5,121 & 9,959 & 10,000 & 5,548 \\ \hline
        EURO & 475,834 & 7,247 & 8,768 & 100,000 & 107,840 \\ \hline
    \end{tabular}
\end{table}

\begin{itemize}
    \item Synthetic: The elements of the two synthetic datasets are initially uniformly sampled from the range (0,1), then multiplied by a coefficient to adjust the maximum column squared norm $R$. The X matrix has 1,000 rows, and the Y matrix has 2,000 rows, each with 100,000 columns. The window size is set to 50,000.
    \item APR: The Amazon Product Reviews (APR) dataset is a publicly available collection containing product reviews and related information from the Amazon website. This dataset consists of millions of sentences in both English and French. We structured it into a review matrix where the X matrix has 28,017 rows, and the Y matrix has 42,833 rows, with both matrices sharing 23,235 columns. The window size is 10,000.
    \item PAN11: PANPC-11 (PAN11) is a dataset designed for text analysis, particularly for tasks such as plagiarism detection, author identification, and near-duplicate detection. The dataset includes texts in English and French. The X and Y matrices contain 5,121 and 9,959 rows, respectively, with both matrices having 88,977 columns. The window size is 10,000.
\end{itemize}
We evaluate the time-based model on another real-world dataset:
\begin{itemize}
    \item EURO: The Europarl (EURO) dataset is a widely used multilingual parallel corpus, comprising the proceedings of the European Parliament. We selected a subset of its English and French text portions. The X and Y matrices contain 7,247 and 8,768 rows, respectively, and both matrices share 475,834 columns. Timestamps are generated using the $Poisson$ $Arrival$ $Process$ with a rate parameter of $\lambda=2$. The window size is set to 100,000, with approximately 30,000 columns of data on average in each window.
\end{itemize}

\paragraph{Setup} For the sequence-based model, we compare the proposed hDS-COD and  aDS-COD with EH-COD~\cite{yao2024approximate} and DI-COD~\cite{yao2024approximate}. We do not consider the Sampling algorithm as a baseline, as its performance is inferior to that of EH-COD and DI-CID, as demonstrated in \cite{yao2024approximate}. %The hDS-COD is adjusted by the parameter $\ell$ and the maximum number of levels $L = \log{R}$, where $R$ is the prior estimate of the maximum squared column norm of the dataset. DI-COD similarly requires a prior estimate of $R$ to limit the maximum number of levels $L = \log{(R/\varepsilon})$. In contrast, aDS-COD and EH-COD do not require an estimate of $R$; their error-space balance is controlled by the parameter $\ell = \frac{1}{\varepsilon}$. 
For the time-based model, we compare the proposed hDS-COD and  aDS-COD with EH-COD and the Sampling algorithm since DI-COD cannot be applied to time-based sliding window model. To achieve the same error bound, the maximum number of levels for hDS-COD is set to $L = \log{(\varepsilon NR)}$, and the initial threshold for aDS-COD is set to $1$.

Our experiments aim to illustrate the trade-offs between space and approximation errors. The x-axis represents two metrics for space: final sketch size and total space cost. The final sketch size refers to the number of columns in the result sketches $\mA$ and $\mB$ generated by the algorithm, representing a compression ratio. The total space cost refers to the maximum space required during the algorithm's execution, measured by the number of columns.We evaluate the approximation performance of all algorithms based on correlation errors $\operatorname{corr-err}(\mathbf{X}_W \mathbf{Y}_W^\top, \mathbf{A} \mathbf{B}^\top)$, which is reflected on the y-axis. Every 1,000 iterations, all algorithms query the window and record the average and maximum errors across all sampled windows.

The experiments for all algorithms were conducted using MATLAB (R2023a), with all algorithms running on a Windows server equipped with 32GB of memory and a single processor of Intel i9-13900K.

\paragraph{Performance} Figure \ref{fig:error vs l} and Figure \ref{fig:error vs space} illustrate the space efficiency comparison of the algorithms on sequence-based datasets. Panels (a-d) show the average errors across all sampled windows, while panels (e-h) display the maximum errors.

Figure \ref{fig:error vs l} evaluates the compression effect of the final sketch. The hDS-COD, aDS-COD, and EH-COD show similar compression performances. But the DS series is more stable, particularly on the synthetic datasets, where they significantly outperform EH-COD and DI-COD. The performance of hDS-COD and aDS-COD is nearly the same, indicating that the adaptive threshold trick in aDS-COD does not have a noticeable negative impact on it, maintaining the same error as hDS-COD.

Figure \ref{fig:error vs space} measures the total space cost of the algorithms. hDS-COD and aDS-COD show a significant advantage over existing methods, as they can achieve the  $\varepsilon$-approximation error with much less space. For the same space cost, the correlation errors of hDS-COD and aDS-COD are much smaller than those of EH-COD and DI-COD. Also, aDS-COD has better space efficiency than hDS-COD because aDS only uses a single-level structure while hDS requires $\log R+1$ levels. We find that hDS-COD requires more space on  SYNTHETIC(2) dataset compared to SYNTHETIC(1) dataset. This phenomenon occurs because SYNTHETIC(2) dataset has a larger $R$, which confirms the dependence on $R$ as stated in Theorem~\ref{thm:hds}. 

Figure \ref{fig:time-based} compares the performance of algorithms on time-based windows. Panels (a) and (b) present the error against the final sketch size, which show that our aDS-COD and hDS-COD algorithms enjoy similar performance as EH-COD and significantly outperform the sampling algorithm. On the other hand, as shown in panels (c) and (d), our methods outperform baselines in terms of total space cost.

Software development is increasingly conceived as a collaboration activity between developers and AIs. Indeed, IDEs already implement features to enable interactive development, with AI suggesting implementations that are reused by developers.

Although multiple studies show this interaction can be successful, there is still limited understanding of how the models must be configured and used in the context of code generation tasks. This study addresses this gap, systematically investigating the impact of several key parameters, including the repeated submission of a prompt to accommodate for the non-deterministic nature of the models.

Our study reveals several key findings about the usage of ChatGPT. In particular, we discovered how creativity, although up to a limited extent, is useful to increase the range of methods whose code can be generated correctly. A major role is played by parameter top-p, which is commonly underrated, and instead has a major impact on the correctness of the results, with lower values producing better results. Finally, prompts should be submitted multiple times, with $5$ repetitions combined with a temperature of $1.2$ resulting in an effective configuration in our experiments.  

Future work concerns two main research directions. One is about replicating this experiment with other AI assistants, to validate our findings in multiple contexts. The second research direction concerns finding strategies to deal with the need to submit the same prompt multiple times to obtain a useful result, and thus developing approaches able to select or merge multiple responses automatically. 

\newpage
\section{Limitations} \label{sec:limitations}

While the above results demonstrate an important step toward flexible and robust humanoid locomotion, our proposed technique is not a panacea. 
%
Both HLIP and CI-MPC require parameter tuning, and their combination only increases the complexity of this process. While we used only one set of parameters for all the experiments, we did find that some parameters induced sharp tradeoffs. For example, a lower weight on base orientation tracking gave more natural-looking gaits, but reduced push recovery performance.
%


Our CI-MPC implementation uses significantly simplified collision geometries. This enables fast solve times, but precludes behaviors that involve contact away from the hands and the feet. As a result, the robot is not able to automatically recover from a fall. Furthermore, our CI-MPC solver's performance is reliant on smooth collision geometries, as sharp corners introduce problematic discontinuous gradients. 
%
Similarly, self-collisions present a major failure mode in the current implementation. Adding self-collision constraints either in the optimization problem \cite{grandia2021multi} or with a high order control barrier function \cite{khazoom2024tailoring, ames2019control, singletary2021safety} presents an obvious next step for improving reliability.

Finally, there are instances in which HLIP's suggested contact sequence guides the robot in an unhelpful direction. For example, if the robot is standing and pushed to the left, HLIP might suggest lifting the right leg, depending on the timing of the gait cycle. This could be mitigated with a richer reduced-order model, but illustrates a trade-off inherent to guiding whole-body behaviors with a reduced-order model.




% \section*{Acknowledgments}



% Bibliography entries for the entire Anthology, followed by custom entries
%\bibliography{anthology,custom}
% Custom bibliography entries only
\bibliography{custom}

\clearpage
\appendix
\section*{Appendix}
\section{Synthetic Data Engine}
\begin{table}[t!]
    \centering
    \begin{tabular}{l l}
    \toprule
    Visual premises & Non-visual premises \\
    \midrule
    \tabitem Perpendicularity & \tabitem Middle point \\
    \tabitem Collinearity & \tabitem Congruency in degree \\
    \tabitem Concyclicity & \tabitem Congruency in length \\
    \tabitem Parallelity & \tabitem Congruency in ratio \\
    \tabitem Angle measure & \tabitem Triangle similarity \\
    \tabitem Length measure & \tabitem Triangle congruency \\
    & \tabitem Circumcenter \\
    & \tabitem Foot \\
    \bottomrule
    \end{tabular}
    \caption{Geometric premises used in AlphaGeometry. \emph{Visual premises} denotes the geometric premises which can be directly perceived from the diagram. \emph{Non-visual premises} requires reasoning to be recognized.}
    \label{tab:alphageometry}
\end{table}

\begin{table}[t!]
    \centering
    \begin{tabular}{l l}
    \toprule
    Visual premises & Non-visual premises \\
    \midrule
    \tabitem Perpendicularity & \tabitem Middle point \\
    \tabitem Collinearity & \tabitem Congruency in degree \\
    \tabitem Concyclicity & \tabitem Congruency in length \\
    \tabitem Parallelity & \tabitem Congruency in ratio \\
    \tabitem Angle measure & \tabitem Triangle similarity \\
    \tabitem Length measure & \tabitem Triangle congruency \\
    & \tabitem Circumcenter \\
    & \tabitem Foot \\
    \bottomrule
    \end{tabular}
    \caption{Geometric premises used in AlphaGeometry. \emph{Visual premises} denotes the geometric premises which can be directly perceived from the diagram. \emph{Non-visual premises} requires reasoning to be recognized.}
    \label{tab:alphageometry}
\end{table}

In this section, we provide the details of our synthetic data engine. Based on AlphaGeometry~\citep{alphageometry}, we generate synthetic diagram and caption pairs by randomly sampling a AlphaGeometry program with \cref{alg:sampling}.

\begin{algorithm}[t!]
\caption{Sampling process of the synthetic data engine}
 \textbf{Input} Geometric relations $R$, geometric objects $O$, number of clauses $n_c$ \\
 \textbf{Output} AlphaGeometry program $c$
\begin{algorithmic}[1]
\State Initialize points and clauses with the sampled object: $P, C \sim O$ 
\For{$i \gets 1$ to $n_c$}
    \State Generate points: $P_{\text{new}}$
    \State Sample relation and points: $r, P_{\text{old}} \sim R, P$
    \State Construct clause: $C_{\text{new}} = r(P_\text{new}, P_\text{old})$
    \State Update points and clauses: $P, C \gets P \cup P_{\text{new}}, C \cup C_{\text{new}}$
\EndFor
\State Generate program with points and clauses: $c \gets \text{Clauses2Program}(P, C)$
\State \textbf{return} $c$
\end{algorithmic}
\label{alg:sampling}
\end{algorithm}

Examples for randomly sampled AlphaGeometry problems and their corresponding diagrams and lists of geometric premises are described in \cref{fig:alphageometry}.
The types of geometric premises that appear in our synthetic data engine are listed in \cref{tab:alphageometry}.


\section{Details of Benchmark}

\subsection{Training details}

\label{sec:hparams}

% \paragraph{Linear probing.}
To evaluate the visual feature perception of the vision encoder, we utilize a linear probing approach, which involves freezing the vision
encoder parameters and training a simple linear classifier on top of its features.

We train the linear classifier on the training set of each task for 50 epochs with batch size 128 and learning rate $1\text{e-}4$.
We use Adam optimizer for optimization.

\subsection{Visualization of the vision encoders}

We visualize the embeddings of the vision encoders used in \cref{sec:benchmakr_results} at \cref{fig:tsne_benchmark}.

\begin{figure}[t!]
    \centering
    \begin{subfigure}[t]{.32\linewidth}
        \centering
        \includegraphics[width=\linewidth]{latex/figures/images/visual_tsne_openclip.png}
        \caption{OpenCLIP}
    \end{subfigure}
    \begin{subfigure}[t]{.32\linewidth}
        \centering
        \includegraphics[width=\linewidth]{latex/figures/images/visual_tsne_siglip.png}
        \caption{SigLIP}
    \end{subfigure}
    \begin{subfigure}[t]{.32\linewidth}
        \centering
        \includegraphics[width=\linewidth]{latex/figures/images/visual_tsne_convnext.png}
        \caption{ConvNeXT}
    \end{subfigure}
    \begin{subfigure}[t]{.32\linewidth}
        \centering
        \includegraphics[width=\linewidth]{latex/figures/images/visual_tsne_dinov2.png}
        \caption{DinoV2}
    \end{subfigure}
    \begin{subfigure}[t]{.32\linewidth}
        \centering
        \includegraphics[width=\linewidth]{latex/figures/images/visual_tsne_ours.png}
        \caption{GeoCLIP}
    \end{subfigure}
    \caption{
    The embeddings of the vision encoders on the diagrams of TwoLines task. We visualize the embeddings of the vision encoders on the diagrams of TwoLines task. The blue, orange, and green dots are the diagrams where the two lines AB and BC are collinear, perpendicular, and otherwise, respectively.
    \label{fig:tsne_benchmark}
    }
\end{figure}


% \paragraph{GeoCLIP.}
% We start from OpenCLIP~\citep{clip}, a pre-trained model where the architecture is ViT-L/14 with image resolution $336\times 336$. To train OpenCLIP with GeoCLIP, we use total of 200,000 diagram-caption pairs generated with our synthetic data engine.
% We set the batch size and weight decay to 256 and 0.2, respectively.
% We optimize for 50 epochs using Adam optimizer~\citep{adam} and a cosine annealing scheduler with 2,000 warmup steps and the maximum learning rate is set to be $1\text{e-}4$.
% For the domain adaptation parts, i.e., applying CLIP on the diagram-caption pairs and the diagram pairs of target domains, we vary the batch size to 32.

\section{GeoCLIP-DA}

\subsection{Domain adaptation data}

We adopt GeoCLIP to the two PGPS benchmarks: GeoQA~\citep{geoqa} and PGPS9K~\citep{pgps}.
For PGPS9K, we use the Geometry3K split.
\cref{fig:domain_adaptation_samples} shows the pairs used to adapt the domain of GeoCLIP.

\begin{figure}[t!]
    \centering
    \includegraphics[width=\linewidth]{latex/figures/images/domain_adaptation_samples.pdf}
    \caption{Examples of diagram pairs curated for domain adaptation. For each row, the first diagram is from the target domain, and the remaining diagrams are from the source domain. To generate source domain diagrams, we translate the target diagram by our diagram generator with the textual description of the target image.}
    \label{fig:domain_adaptation_samples}
\end{figure}

\subsection{Training details}

We start from OpenCLIP~\citep{clip}, a pre-trained model where the architecture is ViT-L/14 with image resolution $336\times 336$. To train OpenCLIP, we use total of 200,000 diagram-caption pairs generated with our synthetic data engine.
For the domain adaptation to GeoQA and Geometry3K datasets, we randomly sample 50 diagrams and translate the diagram and caption styles following the procedure described in \cref{sec:domain_adaptation}. Finally, \geoclip{} is fine-tuned via \cref{eqn:da}.
We name the GeoQA and Geometry3K adopted \geoclip{} as \geoclip{}-DA.

We set the batch size for the source domain diagram-caption pairs to 256. 
For the domain adaptation parts, i.e., applying CLIP on the diagram-caption pairs and the diagram pairs of target domains, we vary the batch size to 32.
We set weight decay to 0.2.
We optimize for 50 epochs using Adam optimizer~\citep{adam} and a cosine annealing scheduler with 2,000 warmup steps and the maximum learning rate is set to be $1\text{e-}4$.
We train the model with eight RTX3090 GPUs for approximately 24 hours.

\section{\geovlm{}}
\label{sec:vlm_details}

\subsection{Modification of training data}

Our fine-tuning strategy differs slightly from previous works~\citep{unigeo,pgps,geox}. 
Here, we clarify the difference between our approach and previous approaches. In previous works, the VLM is trained to produce the solution program given diagram and problem description as shown in \cref{fig:pgps_examples}. An interesting observation from GeoQA and PGPS9K datasets is that the numerical measurements, such as angles, lengths, and volumes, are not written in the problem description but given as additional conditions, and the numerals are substituted as a variable in the problem description as shown in \cref{fig:geoqa_example}. Therefore, the VLM only needs to produce the solution program without having optical character recognition (OCR) from the diagram. The variables are automatically substituted by the actual numbers when the program is executed. Therefore, the vision encoders do not need to learn OCR from the image.

However, this approach cannot be generalized to a wider class of problems where the numerals are embedded in the diagram instead of written in the problem description. Some variants of MathVerse, such as the vision-dominant problems, fall into this category as well. To incorporate OCR into the solution of the problem, we modify some problem statements in the training set, such that the numerical measurements are only shown in the diagram and not in the statements. We further modify the solution problem so that the solution contains OCR results as a part of the final output. Finally, we unify the language of the solution programs used in GeoQA and PGPS9K by converting GeoQA programs into PGPS9K format. The unification makes the output of VLM consistent since both datasets use different types of formal languages.

\cref{fig:training_data} shows examples of the modified input pairs and solutions, where the first problem statement does not have numerical measurements and the OCR results are in the part of the output solution program.

% \paragraph{Solution program and numerical values prediction.}
% \cref{fig:training_data} reveals the training data for GeoDANO.

\begin{figure}[t!]
    \centering
    \includegraphics[width=\linewidth]{latex/figures/images/training_data.pdf}
    \caption{Examples of the training data for GeoDANO. While previous PGPS models require the only to predict the solution steps and assume the numerical values are explicitly given, GeoDANO is trained to predict both the solution steps and the numerical values in the diagram and text.}
    \label{fig:training_data}
\end{figure}

% \subsection{Architecture}
% \label{sec:vlm_arch}

% We begin by summarizing the architecture of our VLM, a combination of a vision encoder and a language model. For the vision encoder, we use \geoclip{}-DA, with a two-layer MLP of GeLU activation as the projection layers following LLaVA-OneVision~\citep{llava-next}. For the language model, we employ LLama-3-8B-Instruct~\citep{llama}.
% For a given diagram and question pair in PGPS, we feed the vision encoder with the given diagram, and then the output of the encoder is used as an input token of LLM through the projection layer. The question text is then fed into the LLM, followed by the diagram embedding.



% \subsection{Fine-tuning strategy.}
% \label{sec:vlm_details}

% \paragraph{Modification process.}
% Our fine-tuning strategy differs slightly from previous works~\citep{unigeo,pgps,geox}. 
% Here, we clarify the difference between our approach and previous approaches. In previous works, the VLM is trained to produce the solution program given diagram and problem description as shown in \cref{fig:pgps_examples}. An interesting observation from GeoQA and PGPS9K datasets is that the numerical measurements, such as angles, lengths, and volumes, are not written in the problem description but given as additional conditions, and the numerals are substituted as a variable in the problem description as shown in \cref{fig:geoqa_example}. Therefore, the VLM only needs to produce the solution program without having optical character recognition (OCR) from the diagram. The variables are automatically substituted by the actual numbers when the program is executed. Therefore, the vision encoders do not need to learn OCR from the image.

% However, this approach cannot be generalized to a wider class of problems where the numerals are embedded in the diagram instead of written in the problem description. Some variants of MathVerse, such as the vision-dominant problems, fall into this category as well. To incorporate OCR into the solution of the problem, we modify some problem statements in the training set, such that the numerical measurements are only shown in the diagram and not in the statements. We further modify the solution problem so that the solution contains OCR results as a part of the final output. Finally, we unify the language of the solution programs used in GeoQA and PGPS9K by converting GeoQA programs into PGPS9K format. The unification makes the output of VLM consistent since both datasets use different types of formal languages.

% \cref{fig:training_data} shows examples of the modified input pairs and solutions, where the first problem statement does not have numerical measurements and the OCR results are in the part of the output solution program.

\subsection{Training details}
We begin by summarizing the architecture of our VLM, a combination of a vision encoder and a language model. For the vision encoder, we use \geoclip{}-DA, with a two-layer MLP of GeLU activation as the projection layers following LLaVA-OneVision~\citep{llava-next}. For the language model, we employ LLama-3-8B-Instruct~\citep{llama}.
For a given diagram and question pair in PGPS, we feed the vision encoder with the given diagram, and then the output of the encoder is used as an input token of LLM through the projection layer. The question text is then fed into the LLM, followed by the diagram embedding.

With the modified training data, we apply supervised fine-tuning on the VLM, i.e., the gradient only flows through the prediction of numerical values and solution steps, not the diagram and text.

We train the VLM with AdamW optimizer~\citep{adamw} and cosine annealing scheduler with warmp up ratio 0.03 and maximum learning rate $1\text{e-}5$.
We use LoRA~\citep{lora} with rank 128.
We set the batch size to 16 and train with 5 epochs.
We train the VLM with four A100-80GB GPUs for approximately 24 hours.
% \paragraph{Hyper-parameters}

% We train the VLM with AdamW optimizer~\citep{adamw} and cosine annealing scheduler with warmp up ratio 0.03 and maximum learning rate $1\text{e-}5$.
% We use LoRA~\citep{lora} with rank 128.
% We set the batch size to 16 and train with 5 epochs.

% \subsection{Prediction results}

% \begin{figure}[t!]
    \centering
    \includegraphics[width=.92\linewidth]{latex/figures/images/tsne.pdf}
    \caption{Visualization of OpenCLIP and GeoCLIP-DA embeddings. The orange, green, and blue dots represent PGPS9K, GeoQA, and synthetic diagrams, respectively. In the top row, the three diagrams on the left and right are those with the highest cosine similarities to the center under OpenCLIP and GeoCLIP-DA, respectively.}
    \label{fig:tsne}
    \vskip -0.19in
\end{figure}

% We provide quantitative analysis of GeoDANO on MathVerse.

\end{document}
