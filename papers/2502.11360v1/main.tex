% This must be in the first 5 lines to tell arXiv to use pdfLaTeX, which is strongly recommended.
\pdfoutput=1
% In particular, the hyperref package requires pdfLaTeX in order to break URLs across lines.

\documentclass[11pt]{article}

% Change "review" to "final" to generate the final (sometimes called camera-ready) version.
% Change to "preprint" to generate a non-anonymous version with page numbers.
\usepackage[preprint]{acl}

% Comments
\newcommand{\zq}[1]{\textcolor{red}{\textbf{ZQ}: #1}} 

% Standard package includes
\usepackage{times}
\usepackage{latexsym}
\usepackage{amsfonts}
\usepackage{amsmath}

% For proper rendering and hyphenation of words containing Latin characters (including in bib files)
\usepackage[T1]{fontenc}
% For Vietnamese characters
% \usepackage[T5]{fontenc}
% See https://www.latex-project.org/help/documentation/encguide.pdf for other character sets

% This assumes your files are encoded as UTF8
\usepackage[utf8]{inputenc}

% This is not strictly necessary, and may be commented out,
% but it will improve the layout of the manuscript,
% and will typically save some space.
\usepackage{microtype}

% This is also not strictly necessary, and may be commented out.
% However, it will improve the aesthetics of text in
% the typewriter font.
\usepackage{inconsolata}

%Including images in your LaTeX document requires adding
%additional package(s)
\usepackage{graphicx}

\usepackage{booktabs}       % professional-quality tables

\usepackage{multirow}
% \usepackage{autoref}
\usepackage[capitalise]{cleveref}
\crefformat{section}{\S#2#1#3} % see manual of cleveref, section 8.2.1
\crefformat{subsection}{\S#2#1#3}
\crefformat{subsubsection}{\S#2#1#3}

\usepackage{todonotes}
\usepackage{subcaption}

% \usepackage{soul}
% \usepackage[normalem]{ulem}
\usepackage{algorithm}
\usepackage[noend]{algpseudocode}

\newcommand{\tabitem}{~~\llap{\textbullet}~~}

\newcommand{\dw}[1]{\textcolor{red}{#1}}
\newcommand{\sh}[1]{\textcolor{blue}{#1}}
\newcommand{\geoclip}{GeoCLIP}
\newcommand{\geovlm}{GeoDANO}
\newcommand{\geofeat}{geometric premises}
\newcommand{\captionstyle}{GeoCLIP-style}


\usepackage{pict2e}

\newcommand{\measuredrightanglewithdot}{%
  \mathord{%
    \mspace{1mu}%
    \text{\mrawd}%
    \mspace{1mu}%
  }%
}

\newcommand{\mrawd}{%
  \setlength{\unitlength}{1ex}%
  \begin{picture}(1,1)
  \roundcap
  \polyline(0,1)(0,0)(1,0)
  \put(0,0){\arc[0,90]{0.5}}
  \put(0.2,0.2){\circle*{0.1}}
  \end{picture}%
}

\newcommand{\measuredrightanglewithsquare}{%
  \mathord{%
    \mspace{1mu}%
    \text{\msquare}%
    \mspace{1mu}%
  }%
}

\newcommand{\msquare}{%
  \setlength{\unitlength}{1ex}%
  \begin{picture}(1,1)
  \roundcap
  \polyline(0,1)(0,0)(1,0)
  \polyline(0, 0.5)(0.5, 0.5)(0.5,0)
  \end{picture}%
}

% If the title and author information does not fit in the area allocated, uncomment the following
%
%\setlength\titlebox{<dim>}
%
% and set <dim> to something 5cm or larger.

\title{GeoDANO: Geometric VLM with Domain Agnostic Vision Encoder}

% Author information can be set in various styles:
% For several authors from the same institution:
% \author{Author 1 \and ... \and Author n \\
%         Address line \\ ... \\ Address line}
% if the names do not fit well on one line use
%         Author 1 \\ {\bf Author 2} \\ ... \\ {\bf Author n} \\
% For authors from different institutions:
% \author{Author 1 \\ Address line \\  ... \\ Address line
%         \And  ... \And
%         Author n \\ Address line \\ ... \\ Address line}
% To start a separate ``row'' of authors use \AND, as in
% \author{Author 1 \\ Address line \\  ... \\ Address line
%         \AND
%         Author 2 \\ Address line \\ ... \\ Address line \And
%         Author 3 \\ Address line \\ ... \\ Address line}

% \author{
%   Seunghyuk Cho \\
%   POSTECH GSAI \\
%   \texttt{shhj1998@postech.ac.kr} \\\And
%   Zhenyue Qin \\
%   Independent researcher \\
%   \texttt{zhenyue.qin@yale.edu} \\\And
%   Yang Liu \\
%   Independent researcher \\
%   \texttt{lyf1082@gmail.com} \\\And
%   Youngbin Choi \\
%   POSTECH GSAI \\
%   \texttt{shhj1998@postech.ac.kr} \\\And
%   Seungbeom Lee \\
%   POSTECH GSAI \\
%   \texttt{shhj1998@postech.ac.kr} \\\And
%   Dongwoo Kim \\
%   POSTECH CSE \& GSAI \\
%   \texttt{shhj1998@postech.ac.kr} \\\And}

\author{
 \textbf{Seunghyuk Cho\textsuperscript{1}},
 \textbf{Zhenyue Qin\textsuperscript{3}},
 \textbf{Yang Liu\textsuperscript{3}},
 \textbf{Youngbin Choi\textsuperscript{1}},
\\
 \textbf{Seungbeom Lee\textsuperscript{1}},
 \textbf{Dongwoo Kim\textsuperscript{1,2}}
\\
 \textsuperscript{1}Graduate School of Artificial Intelligence, POSTECH,
\\
 \textsuperscript{2}Department of Computer Science and Engineering, POSTECH, 
\\
 \textsuperscript{3}Independent Researcher
\\
 \small{
   \textbf{Correspondence to:} Dongwoo Kim \href{mailto:dongwoo.kim@postech.ac.kr}{<dongwoo.kim@postech.ac.kr>}
 }
}

\begin{document}
\maketitle

\begin{abstract}
    Since 2020, GitGuardian has been detecting checked-in hard-coded secrets in GitHub repositories. During 2020-2023, GitGuardian has observed an upward annual trend and a four-fold increase in hard-coded secrets, with 12.8 million exposed in 2023. However, removing all the secrets from software artifacts is not feasible due to time constraints and technical challenges. Additionally, the security risks of the secrets are not equal, protecting assets ranging from obsolete databases to sensitive medical data. Thus, secret removal should be prioritized by security risk reduction, which existing secret detection tools do not support. \textit{The goal of this research is to aid software practitioners in prioritizing secrets removal efforts through our security risk-based tool}. We present RiskHarvester, a risk-based tool to compute a security risk score based on the value of the asset and ease of attack on a database. We calculated the value of asset by identifying the sensitive data categories present in a database from the database keywords in the source code. We utilized data flow analysis, SQL, and Object Relational Mapper (ORM) parsing to identify the database keywords. To calculate the ease of attack, we utilized passive network analysis to retrieve the database host information. To evaluate RiskHarvester, we curated RiskBench, a benchmark of 1,791 database secret-asset pairs with sensitive data categories and host information manually retrieved from 188 GitHub repositories. RiskHarvester demonstrates precision of (95\%) and recall (90\%) in detecting database keywords for the value of asset and precision of (96\%) and recall (94\%) in detecting valid hosts for ease of attack. Finally, we conducted a survey (52 respondents) to understand whether developers prioritize secret removal based on security risk score. We found that 86\% of the developers prioritized the secrets for removal with descending security risk scores.
\end{abstract}

\section{Introduction}

Large language models (LLMs) have achieved remarkable success in automated math problem solving, particularly through code-generation capabilities integrated with proof assistants~\citep{lean,isabelle,POT,autoformalization,MATH}. Although LLMs excel at generating solution steps and correct answers in algebra and calculus~\citep{math_solving}, their unimodal nature limits performance in plane geometry, where solution depends on both diagram and text~\citep{math_solving}. 

Specialized vision-language models (VLMs) have accordingly been developed for plane geometry problem solving (PGPS)~\citep{geoqa,unigeo,intergps,pgps,GOLD,LANS,geox}. Yet, it remains unclear whether these models genuinely leverage diagrams or rely almost exclusively on textual features. This ambiguity arises because existing PGPS datasets typically embed sufficient geometric details within problem statements, potentially making the vision encoder unnecessary~\citep{GOLD}. \cref{fig:pgps_examples} illustrates example questions from GeoQA and PGPS9K, where solutions can be derived without referencing the diagrams.

\begin{figure}
    \centering
    \begin{subfigure}[t]{.49\linewidth}
        \centering
        \includegraphics[width=\linewidth]{latex/figures/images/geoqa_example.pdf}
        \caption{GeoQA}
        \label{fig:geoqa_example}
    \end{subfigure}
    \begin{subfigure}[t]{.48\linewidth}
        \centering
        \includegraphics[width=\linewidth]{latex/figures/images/pgps_example.pdf}
        \caption{PGPS9K}
        \label{fig:pgps9k_example}
    \end{subfigure}
    \caption{
    Examples of diagram-caption pairs and their solution steps written in formal languages from GeoQA and PGPS9k datasets. In the problem description, the visual geometric premises and numerical variables are highlighted in green and red, respectively. A significant difference in the style of the diagram and formal language can be observable. %, along with the differences in formal languages supported by the corresponding datasets.
    \label{fig:pgps_examples}
    }
\end{figure}



We propose a new benchmark created via a synthetic data engine, which systematically evaluates the ability of VLM vision encoders to recognize geometric premises. Our empirical findings reveal that previously suggested self-supervised learning (SSL) approaches, e.g., vector quantized variataional auto-encoder (VQ-VAE)~\citep{unimath} and masked auto-encoder (MAE)~\citep{scagps,geox}, and widely adopted encoders, e.g., OpenCLIP~\citep{clip} and DinoV2~\citep{dinov2}, struggle to detect geometric features such as perpendicularity and degrees. 

To this end, we propose \geoclip{}, a model pre-trained on a large corpus of synthetic diagram–caption pairs. By varying diagram styles (e.g., color, font size, resolution, line width), \geoclip{} learns robust geometric representations and outperforms prior SSL-based methods on our benchmark. Building on \geoclip{}, we introduce a few-shot domain adaptation technique that efficiently transfers the recognition ability to real-world diagrams. We further combine this domain-adapted GeoCLIP with an LLM, forming a domain-agnostic VLM for solving PGPS tasks in MathVerse~\citep{mathverse}. 
%To accommodate diverse diagram styles and solution formats, we unify the solution program languages across multiple PGPS datasets, ensuring comprehensive evaluation. 

In our experiments on MathVerse~\citep{mathverse}, which encompasses diverse plane geometry tasks and diagram styles, our VLM with a domain-adapted \geoclip{} consistently outperforms both task-specific PGPS models and generalist VLMs. 
% In particular, it achieves higher accuracy on tasks requiring geometric-feature recognition, even when critical numerical measurements are moved from text to diagrams. 
Ablation studies confirm the effectiveness of our domain adaptation strategy, showing improvements in optical character recognition (OCR)-based tasks and robust diagram embeddings across different styles. 
% By unifying the solution program languages of existing datasets and incorporating OCR capability, we enable a single VLM, named \geovlm{}, to handle a broad class of plane geometry problems.

% Contributions
We summarize the contributions as follows:
We propose a novel benchmark for systematically assessing how well vision encoders recognize geometric premises in plane geometry diagrams~(\cref{sec:visual_feature}); We introduce \geoclip{}, a vision encoder capable of accurately detecting visual geometric premises~(\cref{sec:geoclip}), and a few-shot domain adaptation technique that efficiently transfers this capability across different diagram styles (\cref{sec:domain_adaptation});
We show that our VLM, incorporating domain-adapted GeoCLIP, surpasses existing specialized PGPS VLMs and generalist VLMs on the MathVerse benchmark~(\cref{sec:experiments}) and effectively interprets diverse diagram styles~(\cref{sec:abl}).

\iffalse
\begin{itemize}
    \item We propose a novel benchmark for systematically assessing how well vision encoders recognize geometric premises, e.g., perpendicularity and angle measures, in plane geometry diagrams.
	\item We introduce \geoclip{}, a vision encoder capable of accurately detecting visual geometric premises, and a few-shot domain adaptation technique that efficiently transfers this capability across different diagram styles.
	\item We show that our final VLM, incorporating GeoCLIP-DA, effectively interprets diverse diagram styles and achieves state-of-the-art performance on the MathVerse benchmark, surpassing existing specialized PGPS models and generalist VLM models.
\end{itemize}
\fi

\iffalse

Large language models (LLMs) have made significant strides in automated math word problem solving. In particular, their code-generation capabilities combined with proof assistants~\citep{lean,isabelle} help minimize computational errors~\citep{POT}, improve solution precision~\citep{autoformalization}, and offer rigorous feedback and evaluation~\citep{MATH}. Although LLMs excel in generating solution steps and correct answers for algebra and calculus~\citep{math_solving}, their uni-modal nature limits performance in domains like plane geometry, where both diagrams and text are vital.

Plane geometry problem solving (PGPS) tasks typically include diagrams and textual descriptions, requiring solvers to interpret premises from both sources. To facilitate automated solutions for these problems, several studies have introduced formal languages tailored for plane geometry to represent solution steps as a program with training datasets composed of diagrams, textual descriptions, and solution programs~\citep{geoqa,unigeo,intergps,pgps}. Building on these datasets, a number of PGPS specialized vision-language models (VLMs) have been developed so far~\citep{GOLD, LANS, geox}.

Most existing VLMs, however, fail to use diagrams when solving geometry problems. Well-known PGPS datasets such as GeoQA~\citep{geoqa}, UniGeo~\citep{unigeo}, and PGPS9K~\citep{pgps}, can be solved without accessing diagrams, as their problem descriptions often contain all geometric information. \cref{fig:pgps_examples} shows an example from GeoQA and PGPS9K datasets, where one can deduce the solution steps without knowing the diagrams. 
As a result, models trained on these datasets rely almost exclusively on textual information, leaving the vision encoder under-utilized~\citep{GOLD}. 
Consequently, the VLMs trained on these datasets cannot solve the plane geometry problem when necessary geometric properties or relations are excluded from the problem statement.

Some studies seek to enhance the recognition of geometric premises from a diagram by directly predicting the premises from the diagram~\citep{GOLD, intergps} or as an auxiliary task for vision encoders~\citep{geoqa,geoqa-plus}. However, these approaches remain highly domain-specific because the labels for training are difficult to obtain, thus limiting generalization across different domains. While self-supervised learning (SSL) methods that depend exclusively on geometric diagrams, e.g., vector quantized variational auto-encoder (VQ-VAE)~\citep{unimath} and masked auto-encoder (MAE)~\citep{scagps,geox}, have also been explored, the effectiveness of the SSL approaches on recognizing geometric features has not been thoroughly investigated.

We introduce a benchmark constructed with a synthetic data engine to evaluate the effectiveness of SSL approaches in recognizing geometric premises from diagrams. Our empirical results with the proposed benchmark show that the vision encoders trained with SSL methods fail to capture visual \geofeat{}s such as perpendicularity between two lines and angle measure.
Furthermore, we find that the pre-trained vision encoders often used in general-purpose VLMs, e.g., OpenCLIP~\citep{clip} and DinoV2~\citep{dinov2}, fail to recognize geometric premises from diagrams.

To improve the vision encoder for PGPS, we propose \geoclip{}, a model trained with a massive amount of diagram-caption pairs.
Since the amount of diagram-caption pairs in existing benchmarks is often limited, we develop a plane diagram generator that can randomly sample plane geometry problems with the help of existing proof assistant~\citep{alphageometry}.
To make \geoclip{} robust against different styles, we vary the visual properties of diagrams, such as color, font size, resolution, and line width.
We show that \geoclip{} performs better than the other SSL approaches and commonly used vision encoders on the newly proposed benchmark.

Another major challenge in PGPS is developing a domain-agnostic VLM capable of handling multiple PGPS benchmarks. As shown in \cref{fig:pgps_examples}, the main difficulties arise from variations in diagram styles. 
To address the issue, we propose a few-shot domain adaptation technique for \geoclip{} which transfers its visual \geofeat{} perception from the synthetic diagrams to the real-world diagrams efficiently. 

We study the efficacy of the domain adapted \geoclip{} on PGPS when equipped with the language model. To be specific, we compare the VLM with the previous PGPS models on MathVerse~\citep{mathverse}, which is designed to evaluate both the PGPS and visual \geofeat{} perception performance on various domains.
While previous PGPS models are inapplicable to certain types of MathVerse problems, we modify the prediction target and unify the solution program languages of the existing PGPS training data to make our VLM applicable to all types of MathVerse problems.
Results on MathVerse demonstrate that our VLM more effectively integrates diagrammatic information and remains robust under conditions of various diagram styles.

\begin{itemize}
    \item We propose a benchmark to measure the visual \geofeat{} recognition performance of different vision encoders.
    % \item \sh{We introduce geometric CLIP (\geoclip{} and train the VLM equipped with \geoclip{} to predict both solution steps and the numerical measurements of the problem.}
    \item We introduce \geoclip{}, a vision encoder which can accurately recognize visual \geofeat{}s and a few-shot domain adaptation technique which can transfer such ability to different domains efficiently. 
    % \item \sh{We develop our final PGPS model, \geovlm{}, by adapting \geoclip{} to different domains and training with unified languages of solution program data.}
    % We develop a domain-agnostic VLM, namely \geovlm{}, by applying a simple yet effective domain adaptation method to \geoclip{} and training on the refined training data.
    \item We demonstrate our VLM equipped with GeoCLIP-DA effectively interprets diverse diagram styles, achieving superior performance on MathVerse compared to the existing PGPS models.
\end{itemize}

\fi 

\section{Related Work}
\label{sec:RelatedWork}

Within the realm of geophysical sciences, super-resolution/downscaling is a challenge that scientists continue to tackle. There have been several works involved in downscaling applications such as river mapping \cite{Yin2022}, coastal risk assessment \cite{Rucker2021}, estimating soil moisture from remotely sensed images \cite{Peng2017SoilMoisture} and downscaling of satellite based precipitation estimates \cite{Medrano2023PrecipitationDownscaling} to name a few. We direct the reader to \cite{Karwowska2022SuperResolutionSurvey} for a comprehensive review of satellite based downscaling applications and methods. Pertaining to our objective of downscaling \acp{WFM}, we can draw comparisons with several existing works. 
In what follows, we provide a brief review of functionally adjacent works to contrast the novelty of our proposed model and its role in addressing gaps in literature. 

When it comes to downscaling \ac{WFM}, several works use statistical downscaling techniques. These works downscale images by using statistical techniques that utilize relationships between neighboring water fraction pixels. For instance, \cite{Li2015SRFIM} treat the super-resolution task as a sub-pixel mapping problem, wherein the input fraction of inundated pixels must be exactly mapped to the output patch of inundated pixels. 
% In doing so, they are able to apply a discrete particle swarm optimization method to maximize the Flood Inundation Spatial Dependence Index (FISDI). 
\cite{Wang2019} improved upon these approaches by including a spectral term to fully utilize spectral information from multi spectral remote sensing image band. \cite{Wang2021} on the other hand also include a spectral correlation term to reduce the influence of linear and non-linear imaging conditions. All of these approaches are applied to water fraction obtained via spectral unmixing \cite{wang2013SpectralUnmixing} and are designed to work with multi spectral information from MODIS. However, we develop our model with the intention to be used with water fractions directly derived from the output of satellites. One such example is NOAA/VIIRS whose water fraction extraction method is described in \cite{Li2013VIIRSWFM}. \cite{Li2022VIIRSDownscaling} presented a work wherein \ac{WFM} at 375-m flood products from VIIRS were downscaled 30-m flood event and depth products by expressing the inundation mechanism as a function of the \ac{DEM}-based water area and the VIIRS water area.

On the other hand, the non-linear nature of the mapping task lends itself to the use of neural networks. Several models have been adapted from traditional single image digital super-resolution in computer vision literature \cite{sdraka2022DL4downscalingRemoteSensing}. Existing deep learning models in single image super-resolution are primarily dominated by \ac{CNN} based models. Specifically, there has been an upward trend in residual learning models. \acp{RDN} \cite{Zhang2018ResidualDenseSuperResolution} introduced residual dense blocks that employed a contiguous memory mechanism that aimed to overcome the inability of very deep \acp{CNN} to make full use of hierarchical features. 
\acp{RCAN} \cite{Zhang2018RCANSuperResolution} introduced an attention mechanism to exploit the inter-channel dependencies in the intermediate feature transformations. There have also been some works that aim to produce more lightweight \ac{CNN}-based architectures \cite{Zheng2019IMDN,Xiaotong2020LatticeNET}. Since the introduction of the vision transformer \cite{Vaswani2017Attention} that utilized the self-attention mechanism -- originally used for modeling text sequences -- by feeding a sequence 2D sub-image extracted from the original image. Using this approach \cite{LuESRT2022} developed a light-weight and efficient transformer based approach for single image super-resolution. 


For the task of super-resolution of \acp{WFM}, we discuss some works whose methodology is similar to ours even though they differ in their problem setting. \cite{Yin2022} presented a cascaded spectral spatial model for super-resolution of MODIS imagery with a scaling factor 10. Their architecture consists of two stages; first multi-spectral MODIS imagery is converted into a low-resolution \ac{WFM} via spectral unmixing by passing it through a deep stacked residual \ac{CNN}. The second stage involved the super-resolution mapping of these \acp{WFM} using a nested multi-level \ac{CNN} model. Similar to our work, the input fraction images are obtained with zero errors which may not be reflective of reality since there tends to be sensor noise, the spatial distribution of whom cannot be easily estimated. We also note that none of these works directly tackle flood inundation since they've been trained with river map data during non-flood circumstance and \textit{ergo} do not face a data scarcity problem as we do. 
% In this work, apart from the final product of \acp{WFM}, we are not presented with any additional spectral information about the low resolution image. This was intended to work directly with products that can generate \ac{WFM} either directly (VIIRS) or indirectly (Landsat).
\cite{Jia2019} used a deep \ac{CNN} for land mapping that consists of several classes such as building, low vegetation, background and trees. 
\cite{Kumar2021} similarly employ a \ac{CNN} based model for downscaling of summer monsoon rainfall data over the Indian subcontinent. Their proposed Super-Resolution Convolutional Neural Network (SRCNN) has a downscaling factor of 4. 
\cite{Shang2022} on the other hand, proposed a super-resolution mapping technique using Generative Adversarial Networks (GANs). They first generate high resolution fractional images, somewhat analogous to our \ac{WFM}, and are then mapped to categorical land cover maps involving forest, urban, agriculture and water classes. 
\cite{Qin2020} interestingly approach lake area super-resolution for Landsat and MODIS data as an unsupervised problem using a \ac{CNN} and are able to extend to other scaling factors. \cite{AristizabalInundationMapping2020} performed flood inundation mapping using \ac{SAR} data obtained from Sentinel-1. They showed that \ac{DEM}-based features helped to improve \ac{SAR}-based predictions for quadratic discriminant analysis, support vector machines and k-nearest neighbor classifiers. While almost all of the aforementioned works can be adapted to our task. We stand out in the following ways (i) We focus on downscaling of \acp{WFM} directly, \textit{i.e.,} we do not focus on the algorithm to compute the \ac{WFM} from multi-channel satellite data and (ii) We focus on producing high resolution maps only for instances of flood inundation. The latter point produces a data scarcity issue which we seek to remedy with synthetic data. 


%%%%%%%%%%%%%%%%% Additional unused information %%%%%%%%%%%%%%%%


%     \item[\cite{Wang2021}] Super-Resolution Mapping Based on Spatial–Spectral Correlation for Spectral Imagery
%     \begin{itemize}
%         \item Not a deep neural network approach. SRM based on spatial–spectral correlation (SSC) is proposed in order to overcome the influence of linear and nonlinear imaging conditions and utilize more accurate spectral properties.
%         \item (fig 1) there are two main SRM types: (1) the initialization-then-optimization SRM, where the class labels are allocated randomly to subpixels, and the location of each subpixel is optimized to obtain the final SRM result. and (2)soft-then-hard SRM, which involves two steps: the subpixel sharpening and the class allocation.  
%         \item SSC procedures: (1) spatial correlation is performed by the MSAM to reduce the influences of linear imaging conditions on image quality. (2) A spectral correlation that utilizes spectral properties based on the nonlinear KLD is proposed to reduce the influences of nonlinear imaging conditions. (3) spatial and spectral correlations are then combined to obtain an optimization function with improved linear and nonlinear performances. And finally (4) by maximizing the optimization function, a class allocation method based on the SA is used to assign LC labels to each subpixel, obtaining the final SRM result.
%         \item (Comparable) 
%     \end{itemize}
%     %--------------------------------------------------------------------
% \cite{Wang2021} account for the influence of linear and non-linear imaging conditions by involving more accurate spectral properties. 
%     %--------------------------------------------------------------------
%     \item[\cite{Yin2022}] A Cascaded Spectral–Spatial CNN Model for Super-Resolution River Mapping With MODIS Imagery
%     \begin{itemize}
%         \item produce  Landsat-like  fine-resolution (scale of 10)  river  maps  from  MODIS images. Notice the original coarse-resolution remotely sensed images, not the river fraction images.
%         \item combined  CNN  model that  contains  a spectral  unmixing  module  and  an  SRM  module, and the SRM module is made up of an encoder and a decoder that are connected through a series of convolutional blocks. 
%         \item With an adaptive cross-entropy loss function to address class imbalance.	
%         \item The overall accuracy, the omission error, the  commission  error,  and  the  mean  intersection  over  union (MIOU)  calculated  to  assess  the results.
%         \item partially comparable with ours, only the SRM module part
%     %--------------------------------------------------------------------

% To decouple the description of the objective and the \ac{ML} model architecture, the motivation for the model architecture is described in \secref{sec:Methodology}. 


%     \item[\cite{Wang2019}] Improving Super-Resolution Flood Inundation Mapping for Multi spectral Remote Sensing Image by Supplying More Spectral 
%     \begin{itemize}
%         \item proposed the SRFIM-MSI,where a new spectral term is added to the traditional SRFIM to fully utilize the spectral information from multi spectral remote sensing image band. 
%         \item The original SRFIM \cite{Huang2014, Li2015} obtains the sub pixel spatial distribution of flood inundation within mixed pixels by maximizing their spatial correlation while maintaining the original proportions of flood inundation within the mixed pixels. The SRFIM is formulated as a maximum combined optimization issue according to the principle of spatial correlation.
%         \item follow the terminology in \cite{Wang2021}, this is an initialization-then-optimization SRM. 
%         \item (Comparable) 
%     \end{itemize}
%     %--------------------------------------------------------------------


%--------------------------------------------------------------------
%     \item[\cite{Jia2019}] Super-Resolution Land Cover Mapping Based on the Convolutional Neural Network
%     \begin{itemize}
%         \item SRMCNN (Super-resolution mapping CNN) is proposed to obtain fine-scale land cover maps from coarse remote sensing images. Specifically, an encoder-decoder CNN is used to determined the labels (i.e., land cover classes) of the subpixels within mixed pixels.
%         \item There were three main parts in SRMCNN. The first part was a three-sequential convolutional layer with ReLU and pooling. The second part is up-sampling, for which a multi transposed-convolutional layer was adopted. To keep the feature learned in the previous layer, a skip connection was used to concatenate the output of the corresponding convolution layer. The last part was the softmax classifier, in which the feature in the antepenultimate layer was classified and class probabilities are obtained.
%         \item The loss: the optimal allocation of classes to the subpixels of mixed pixel is achieved by maximizing the spatial dependence between neighbor pixels under constraint that the class proportions within the mixed pixels are preserved.
%         \item (Preferred), this paper is designed to classify background, Building, Low Vegetation, or Tree in the land. But we can easily adapt to our problem and should compare with this paper.
%     \end{itemize}
%     %--------------------------------------------------------------------

%     \item[\cite{Kumar2021}] Deep learning–based downscaling of summer monsoon rainfall data over Indian region
%     \begin{itemize}
%         \item down-scaling (scale of 4) rainfall data. The output image is not binary image.
%         \item three algorithms: SRCNN, stacked SRCNN, and DeepSD are employed, based on \cite{Vandal2019}
%         \item mean square error and pattern correlation coefficient are used as evaluation metrics.
%         \item SRCNN: super-resolution-based convolutional neural networks (SRCNN) first upgrades the low-resolution image to the higher resolution size by using bicubic interpolation. Suppose the interpolated image is referred to as Y; SRCNNs’task is to retrieve from Y an image F(Y) which is close to the high-resolution ground truth image X.
%         \item stacked SRCNN: stack 2 or more SRCNN blocks to increasing the scaling factor.
%         \item DeepSD: uses topographies as an additional input to stacked SRCNN.
%         \item These algorithms are not designed for binary output images, but if prefer, the ``modified'' stacked SRCNN or DeepSD can be used as baseline algorithms.
%     
%     \item[\cite{Shang2022}] Super resolution Land Cover Mapping Using a Generative Adversarial Network
%     \begin{itemize}
%         \item propose an end-to-end SRM model based on a generative adversarial network (GAN), that is, GAN-SRM, to improve the two-step learning-based SRM methods. 
%         \item Two-step SRM method: The first step is fraction-image super-resolution (SR), which reconstructs a high-spatial-resolution fraction image from the low input, methods like SVR, or CNN has been widely adopted. The second step is converting the high-resolution fraction images to a categorical land cover map, such as with a soft-max function to assign each high-resolution pixel to a unique category value.
%         \item The proposed GAN-SRM model includes a generative network and a discriminative network, so that both the fraction-image SR and the conversion of the fraction images to categorical map steps are fully integrated to reduce the resultant uncertainty. 
%         \item applied to the National Land Cover Database (NLCD), which categorized land into four typical classes:forest, urban, agriculture,and water. scale factor of 8. 
%         \item (Preferred), we should compare with this work.
%     \end{itemize}
%     %--------------------------------------------------------------------

%   \item[\cite{Qin2020}] Achieving Higher Resolution Lake Area from Remote Sensing Images Through an Unsupervised Deep Learning Super-Resolution Method
%   \begin{itemize}
%       \item propose an unsupervised deep gradient network (UDGN) to generate a higher resolution lake area from remote sensing images.
%       \item UDGN models the internal recurrence of information inside the single image and its corresponding gradient map to generate images with higher spatial resolution. 
%       \item A single image super-resolution approach, not comparable
%   \end{itemize}
%     %--------------------------------------------------------------------




%     \item[\cite{Demiray2021}] D-SRGAN: DEM Super-Resolution with Generative Adversarial Networks
%     \begin{itemize}
%         \item A GAN based model is proposed to increase the spatial resolution of a given DEM dataset up to 4 times without additional information related to data.
%         \item Rather than processing each image in a sequence independently, our generator architecture uses a recurrent layer to update the state of the high-resolution reconstruction in a manner that is consistent with both the previous state and the newly received data. The recurrent layer can thus be understood as performing a Bayesian update on the ensemble member, resembling an ensemble Kalman filter. 
%         \item A single image super-resolution approach, not comparable
%     \end{itemize}
%     %--------------------------------------------------------------------
%     \item[\cite{Leinonen2021}] Stochastic Super-Resolution for Downscaling Time-Evolving Atmospheric Fields With a Generative Adversarial Network
%     \begin{itemize}
%         \item propose a super-resolution GAN that operates on sequences of two-dimensional images and creates an ensemble of predictions for each input. The spread between the ensemble members represents the uncertainty of the super-resolution reconstruction.
%         \item for sequence of input images, not comparable with ours.
%     \end{itemize} 
%     %--------------------------------------------------------------------

% \end{itemize}




% \section{Related Work}\label{sec:related_works}
\gls{bp} estimation from \gls{ecg} and \gls{ppg} waveforms has received significant attention due to its potential for continuous, unobtrusive monitoring. Earlier work relied on classical machine learning with handcrafted features, but deep learning methods have since emerged as more robust alternatives. Convolutional or recurrent architectures designed for \gls{ecg}/\gls{ppg} have shown strong performance, including ResUNet with self-attention~\cite{Jamil}, U-Net variants~\cite{Mahmud_2022}, and hybrid \gls{cnn}--\gls{rnn} models~\cite{Paviglianiti2021ACO}. These architectures often outperform traditional feature-engineering approaches, particularly when both \gls{ecg} and \gls{ppg} signals are used~\cite{Paviglianiti2021ACO}.

Nevertheless, many existing methods train solely on \gls{ecg}/\gls{ppg} data, which, while plentiful~\cite{mimiciii,vitaldb,ptb-xl}, often exhibit significant variability in signal quality and patient-specific characteristics. This variability poses challenges for achieving robust generalization across populations. Recent work has explored transfer learning to overcome these issues; for example, Yang \emph{et~al.}~\cite{yang2023cross} studied the transfer of \gls{eeg} knowledge to \gls{ecg} classification tasks, achieving improved performance and reduced training costs. Joshi \emph{et~al.}~\cite{joshi2021deep} also explored the transfer of \gls{eeg} knowledge using a deep knowledge distillation framework to enhance single-lead \gls{ecg}-based sleep staging. However, these studies have largely focused on within-modality or narrow domain adaptations, leaving open the broader question of whether an \gls{eeg}-based foundation model can serve as a versatile starting point for generalized biosignal analysis.

\gls{eeg} has become an attractive candidate for pre-training large models not only because of the availability of large-scale \gls{eeg} repositories~\cite{TUEG} but also due to its rich multi-channel, temporal, and spectral dynamics~\cite{jiang2024large}. While many time-series modalities (for example, voice) also exhibit rich temporal structure, \gls{eeg}, \gls{ecg}, and \gls{ppg} share common physiological origins and similar noise characteristics, which facilitate the transfer of temporal pattern recognition capabilities. In other words, our hypothesis is that the underlying statistical properties and multi-dimensional dynamics in \gls{eeg} make it particularly well-suited for learning robust representations that can be effectively adapted to \gls{ecg}/\gls{ppg} tasks. Our work is the first to validate the feasibility of fine-tuning a transformer-based model initially trained on EEG (CEReBrO~\cite{CEReBrO}) for arterial \gls{bp} estimation using \gls{ecg} and \gls{ppg} data.

Beyond accuracy, real-world deployment of \gls{bp} estimation models calls for efficient inference. Traditional deep networks can be computationally expensive, motivating recent interest in quantization and other compression techniques~\cite{nagel2021whitepaperneuralnetwork}. Few studies have combined large-scale pre-training with post-training quantization for \gls{bp} monitoring. Hence, our method integrates these two aspects: leveraging a potent \gls{eeg}-based foundation model and applying quantization for a compact, high-accuracy cuffless \gls{bp} solution.
\section{Visual Geometric Premises Recognition Benchmark for Vision Encoders}
\label{sec:visual_feature}

In this section, we first develop a benchmark for evaluating a vision encoder's performance in recognizing geometric features from a diagram. We then report the performance of well-known vision encoders on this benchmark.

%\dw{We first examine how commonly used vision encoders in open-source VLMs recognize geometric primitives, such as points and lines, and geometric premises, such as perpendicularity, from a given diagram.} Because a VLM’s performance is typically measured by its final solution in PGPS, the result does not fully capture the encoder’s capacity to identify fundamental geometric structures. 
% Yet, as the encoder is the first component to observe the diagram, understanding its recognition ability is essential for further advancement.
% To address this gap, we introduce a benchmark that thoroughly evaluates vision encoders.

\subsection{Benchmark preparation}
\label{sec:synthetic_data_engine}

We design our benchmark as simple classification tasks. By investigating PGPS datasets, we identify that recognizing \emph{geometric primitives}, such as points and lines, and geometric properties representing \emph{relations between primitives}, such as perpendicularity, is important for solving plane geometry problems. Recognized information forms \emph{geometric premises} to solve the problem successfully. To this end, we carefully curate five classification tasks as follows:
\begin{itemize}
    \item \textbf{Concyclic}: A circle and four points are given. The task is to identify how many of those points lie on the circle.
    \item \textbf{TwoLines}: Two lines, AB and BC, are given alongside other geometric objects. The task is to determine whether AB and BC are perpendicular, collinear, or neither.
    \item \textbf{ObjectShape}: A given diagram includes one of the following geometric objects: a segment, triangle, square, or pentagon. The task is to classify which object is present.
    \item \textbf{SquareShape}: A diagram including a square ABCD and other geometric objects is given. The task is to classify whether the square is a trapezoid, parallelogram, or rectangle.
    \item \textbf{AngleDetection}: A diagram is given with at least three points: A, B, and C. The task is to classify the correct angle of ABC from \(\{15^\circ, 20^\circ, \ldots, 75^\circ\}\). 
\end{itemize}
An example of each task is provided in \cref{fig:benchmark}.

\begin{table}[t]
    \centering
    \caption{The performance of different pre-trained models on ImageNet and infrared semantic segmentation datasets. The \textit{Scratch} means the performance of randomly initialized models. The \textit{PT Epochs} denotes the pre-training epochs while the \textit{IN1K FT epochs} represents the fine-tuning epochs on ImageNet \citep{imagenet}. $^\dag$ denotes models reproduced using official codes. $^\star$ refers to the effective epochs used in \citet{iBOT}. The top two results are marked in \textbf{bold} and \underline{underlined} format. Supervised and CL methods, MIM methods, and UNIP models are colored in \colorbox{orange!15}{\rule[-0.2ex]{0pt}{1.5ex}orange}, \colorbox{gray!15}{\rule[-0.2ex]{0pt}{1.5ex}gray}, and \colorbox{cyan!15}{\rule[-0.2ex]{0pt}{1.5ex}cyan}, respectively.}
    \label{tab:benchmark}
    \centering
    \scriptsize
    \setlength{\tabcolsep}{1.0mm}{
    \scalebox{1.0}{
    \begin{tabular}{l c c c c  c c c c c c c c}
        \toprule
         \multirow{2}{*}{Methods} & \multirow{2}{*}{\makecell[c]{PT \\ Epochs}} & \multicolumn{2}{c}{IN1K FT} & \multicolumn{4}{c}{Fine-tuning (FT)} & \multicolumn{4}{c}{Linear Probing (LP)} \\
         \cmidrule{3-4} \cmidrule(lr){5-8} \cmidrule(lr){9-12} 
         & & Epochs & Acc & SODA & MFNet-T & SCUT-Seg & Mean & SODA & MFNet-T & SCUT-Seg & Mean \\
         \midrule
         \textcolor{gray}{ViT-Tiny/16} & & &  & & & & & & & & \\
         Scratch & - & - & - & 31.34 & 19.50 & 41.09 & 30.64 & - & - & - & - \\
         \rowcolor{gray!15} MAE$^\dag$ \citep{mae} & 800 & 200 & \underline{71.8} & 52.85 & 35.93 & 51.31 & 46.70 & 23.75 & 15.79 & 27.18 & 22.24 \\
         \rowcolor{orange!15} DeiT \citep{deit} & 300 & - & \textbf{72.2} & 63.14 & 44.60 & 61.36 & 56.37 & 42.29 & 21.78 & 31.96 & 32.01 \\
         \rowcolor{cyan!15} UNIP (MAE-L) & 100 & - & - & \underline{64.83} & \textbf{48.77} & \underline{67.22} & \underline{60.27} & \underline{44.12} & \underline{28.26} & \underline{35.09} & \underline{35.82} \\
         \rowcolor{cyan!15} UNIP (iBOT-L) & 100 & - & - & \textbf{65.54} & \underline{48.45} & \textbf{67.73} & \textbf{60.57} & \textbf{52.95} & \textbf{30.10} & \textbf{40.12} & \textbf{41.06}  \\
         \midrule
         \textcolor{gray}{ViT-Small/16} & & & & & & & & & & & \\
         Scratch & - & - & - & 41.70 & 22.49 & 46.28 & 36.82 & - & - & - & - \\
         \rowcolor{gray!15} MAE$^\dag$ \citep{mae} & 800 & 200 & 80.0 & 63.36 & 42.44 & 60.38 & 55.39 & 38.17 & 21.14 & 34.15 & 31.15 \\
         \rowcolor{gray!15} CrossMAE \citep{crossmae} & 800 & 200 & 80.5 & 63.95 & 43.99 & 63.53 & 57.16 & 39.40 & 23.87 & 34.01 & 32.43 \\
         \rowcolor{orange!15} DeiT \citep{deit} & 300 & - & 79.9 & 68.08 & 45.91 & 66.17 & 60.05 & 44.88 & 28.53 & 38.92 & 37.44 \\
         \rowcolor{orange!15} DeiT III \citep{deit3} & 800 & - & 81.4 & 69.35 & 47.73 & 67.32 & 61.47 & 54.17 & 32.01 & 43.54 & 43.24 \\
         \rowcolor{orange!15} DINO \citep{dino} & 3200$^\star$ & 200 & \underline{82.0} & 68.56 & 47.98 & 68.74 & 61.76 & 56.02 & 32.94 & 45.94 & 44.97 \\
         \rowcolor{orange!15} iBOT \citep{iBOT} & 3200$^\star$ & 200 & \textbf{82.3} & 69.33 & 47.15 & 69.80 & 62.09 & 57.10 & 33.87 & 45.82 & 45.60 \\
         \rowcolor{cyan!15} UNIP (DINO-B) & 100 & - & - & 69.35 & 49.95 & 69.70 & 63.00 & \underline{57.76} & \underline{34.15} & \underline{46.37} & \underline{46.09} \\
         \rowcolor{cyan!15} UNIP (MAE-L) & 100 & - & - & \textbf{70.99} & \underline{51.32} & \underline{70.79} & \underline{64.37} & 55.25 & 33.49 & 43.37 & 44.04 \\
         \rowcolor{cyan!15} UNIP (iBOT-L) & 100 & - & - & \underline{70.75} & \textbf{51.81} & \textbf{71.55} & \textbf{64.70} & \textbf{60.28} & \textbf{37.16} & \textbf{47.68} & \textbf{48.37} \\ 
        \midrule
        \textcolor{gray}{ViT-Base/16} & & & & & & & & & & & \\
        Scratch & - & - & - & 44.25 & 23.72 & 49.44 & 39.14 & - & - & - & - \\
        \rowcolor{gray!15} MAE \citep{mae} & 1600 & 100 & 83.6 & 68.18 & 46.78 & 67.86 & 60.94 & 43.01 & 23.42 & 37.48 & 34.64 \\
        \rowcolor{gray!15} CrossMAE \citep{crossmae} & 800 & 100 & 83.7 & 68.29 & 47.85 & 68.39 & 61.51 & 43.35 & 26.03 & 38.36 & 35.91 \\
        \rowcolor{orange!15} DeiT \citep{deit} & 300 & - & 81.8 & 69.73 & 48.59 & 69.35 & 62.56 & 57.40 & 34.82 & 46.44 & 46.22 \\
        \rowcolor{orange!15} DeiT III \citep{deit3} & 800 & 20 & \underline{83.8} & 71.09 & 49.62 & 70.19 & 63.63 & 59.01 & \underline{35.34} & 48.01 & 47.45 \\
        \rowcolor{orange!15} DINO \citep{dino} & 1600$^\star$ & 100 & 83.6 & 69.79 & 48.54 & 69.82 & 62.72 & 59.33 & 34.86 & 47.23 & 47.14 \\
        \rowcolor{orange!15} iBOT \citep{iBOT} & 1600$^\star$ & 100 & \textbf{84.0} & 71.15 & 48.98 & 71.26 & 63.80 & \underline{60.05} & 34.34 & \underline{49.12} & \underline{47.84} \\
        \rowcolor{cyan!15} UNIP (MAE-L) & 100 & - & - & \underline{71.47} & \textbf{52.55} & \underline{71.82} & \textbf{65.28} & 58.82 & 34.75 & 48.74 & 47.43 \\
        \rowcolor{cyan!15} UNIP (iBOT-L) & 100 & - & - & \textbf{71.75} & \underline{51.46} & \textbf{72.00} & \underline{65.07} & \textbf{63.14} & \textbf{39.08} & \textbf{52.53} & \textbf{51.58} \\
        \midrule
        \textcolor{gray}{ViT-Large/16} & & & & & & & & & & & \\
        Scratch & - & - & - & 44.70 & 23.68 & 49.55 & 39.31 & - & - & - & - \\
        \rowcolor{gray!15} MAE \citep{mae} & 1600 & 50 & \textbf{85.9} & 71.04 & \underline{51.17} & 70.83 & 64.35 & 52.20 & 31.21 & 43.71 & 42.37 \\
        \rowcolor{gray!15} CrossMAE \citep{crossmae} & 800 & 50 & 85.4 & 70.48 & 50.97 & 70.24 & 63.90 & 53.29 & 33.09 & 45.01 & 43.80 \\
        \rowcolor{orange!15} DeiT3 \citep{deit3} & 800 & 20 & \underline{84.9} & \underline{71.67} & 50.78 & \textbf{71.54} & \underline{64.66} & \underline{59.42} & \textbf{37.57} & \textbf{50.27} & \underline{49.09} \\
        \rowcolor{orange!15} iBOT \citep{iBOT} & 1000$^\star$ & 50 & 84.8 & \textbf{71.75} & \textbf{51.66} & \underline{71.49} & \textbf{64.97} & \textbf{61.73} & \underline{36.68} & \underline{50.12} & \textbf{49.51} \\
        \bottomrule
    \end{tabular}}}
    \vspace{-2mm}
\end{table}

Our benchmark is built on top of AlphaGeometry~\citep{alphageometry}, which is designed to solve IMO-style plane geometry problems. The program provides useful functions such as formal language describing plane diagrams. The language predefines a set of geometric premises listed in \cref{tab:alphageometry}, including all necessary properties to define our benchmark tasks. In addition, once a diagram description is given in formal language, the program renders a corresponding diagram with varying fonts, colors, widths, orientations, and resolutions, allowing us to have diagrams with diverse styles often observed in a real-world scenario.

We create question-and-answer pairs based on AlphaGeometry. To sample a diverse set of question-and-answers, we first establish a foundational geometric structure corresponding to the key problem of the task and then repeatedly add new points or lines with randomly selected geometric relationships to the existing diagram with the help of the formal language. The pseudo-code for the random question generation is presented in \cref{alg:sampling}. For each task, we generate 50,000, 10,000, and 10,000 question-and-answer pairs for training, validation, and testing, respectively.

\iffalse
We implement a data engine that randomly generates \emph{plane geometry problems}, complete with corresponding \emph{diagrams} and \emph{\geofeat{}s}, including the relationship between lines and measurements such as angles. Our engine is built on top of AlphaGeometry~\citep{alphageometry}, a proof assistant for IMO-style plane geometry problems. AlphaGeometry provides a formal language for representing plane geometry problems. 
\sh{
Given an AlphaGeometry problem, we can derive the \geofeat{}s and characteristics of the given problem such as perpendicularity and collinearity. For completeness, we provide the list of \geofeat{}s that AlphaGeometry can support in \cref{tab:alphageometry}.
In addition, we can render the diagram that represents the AlphaGeometry problem with different fonts, colors, widths, orientations, and resolutions, allowing us to have diagrams with diverse styles often observed in a real-world scenario.
}
% as an executable program. 
% AlphaGeometry can derive the solution when a problem statement is given in terms of the formal language. 
% Given an AlphaGeometry problem written in the formal language, we can render the diagram which expresses . AlphaGeometry supports diverse \geofeat{}s and characteristics of plane geometry diagrams, such as perpendicularity and collinearity. For completeness, we provide the list of \geofeat{}s that AlphaGeometry can support in \cref{tab:alphageometry}. %\cref{fig:alphageometry} illustrates the examples of the problems in the formal language with the rendered diagrams. %repeated below
% In addition, the diagram can be rendered with different fonts, colors, widths, orientations, and resolutions, allowing us to have diagrams with diverse styles often observed in a real-world scenario.

%We employ AlphaGeometry’s formal language to curate a synthetic dataset. 
To sample a diverse set of plane geometry problems, we randomly generate AlphaGeometry problems by first establishing a foundational geometric structure, and then repeatedly adding new points with randomly selected geometric relationships to existing points, drawn from the predefined set in AlphaGeometry. The pseudo-code for the sampling algorithm is presented in \cref{alg:sampling}, and randomly selected examples are provided in \cref{fig:alphageometry}. Note that samples consist of the AlphaGeometry problem and \geofeat{}s written in the text along with the corresponding diagram.

%\subsection{Benchmark for visual \geofeat{} recognition}
Based on the synthetic data engine, we develop a benchmark that evaluates how well the vision encoders recognize visual \geofeat{}s in plane geometry diagrams. In PGPS, recognizing characteristics such as collinearity, perpendicularity, and angle measures is crucial for solving geometry problems. Based on the elements, we define five diagram classification tasks that depend on accurate visual \geofeat{} perception:
\begin{itemize}
    \item \textbf{ObjectShape}: Each diagram includes one of the following geometric objects: a segment, triangle, square, or pentagon. The goal is to classify which object is present.
    \item \textbf{Concyclic}: Each diagram contains a circle and four points. The task is to identify how many of those points lie on the circle.  
    \item \textbf{TwoLines}: Each diagram includes two lines, AB and BC, alongside other geometric objects. The vision encoder must determine whether AB and BC are perpendicular, collinear, or neither.
    \item \textbf{SquareShape}: Each diagram features a square ABCD with additional objects. The goal is to classify whether the square is actually a trapezoid, parallelogram, or rectangle. 
    \item \textbf{AngleDetection}: Each diagram includes an angle ABC and other objects. The objective is to identify the measure of angle ABC from among the set \(\{15^\circ, 20^\circ, \ldots, 75^\circ\}\). 
\end{itemize}
We can generate samples used for benchmarking by adding additional conditions to the random problem generator. For example, we let the generator synthesize AlphaGeometry problems with a segment, triangle, square, or pentagon for the ObjectShape task.
For each task, we generate 50,000, 10,000, and 10,000 AlphaGeometry problems for training, validation, and test, respectively.
We then render the diagrams and corresponding labels from the generated problems, which is later used to train the vision encoders.
\cref{fig:benchmark} shows randomly generated diagrams for the classification tasks.
\fi


\subsection{Results}
\label{sec:benchmakr_results}
\begin{table}[t!]
    \centering
    \resizebox{\linewidth}{!}{
    \begin{tabular}{l l c c c c c }
        \toprule
        & Models & \begin{tabular}{@{}c@{}}Object \\ Shape\end{tabular} & \begin{tabular}{@{}c@{}}Con \\ cyclic\end{tabular} & \begin{tabular}{@{}c@{}}Two \\ Lines\end{tabular} & \begin{tabular}{@{}c@{}}Square \\ Shape\end{tabular} & \begin{tabular}{@{}c@{}}Angle \\ Detection\end{tabular} \\
        % &Models & ObjectShape & Concyclic & TwoLines & SquareShape & AngleDetection \\
        \midrule
        \multirow{4}{*}{%
    \rotatebox[origin=c]{90}{%
        \parbox{1.3cm}{\centering \footnotesize \emph{Baseline}}%
    }%
}  
        &OpenCLIP & \textbf{100.00} & 99.13 & 86.57 & 85.20 & 64.81 \\
        &SigLIP & \textbf{100.00} & \textbf{99.71} & 89.26 & 89.31 & 76.86 \\
        &DinoV2 & \textbf{100.00} & 98.01 & 85.30 & 91.24 & 22.43 \\
        &ConvNeXT & \textbf{100.00} & 99.20 & 89.39 & 88.13 & 61.84 \\
        \midrule
        \multirow{3}{*}{%
    \rotatebox[origin=c]{90}{%
        \parbox{1.3cm}{\centering \footnotesize \emph{SSL}}%
    }%
}  
        &Jigsaw & 86.11 & 63.85 & 49.98 & 61.88 & 11.44 \\
        &MAE & 93.99 & 72.25 & 71.73 & 82.70 & 13.08 \\
        &VQ-VAE & 63.05 & 60.97 & 48.10 & 57.35 & 9.22 \\
        \midrule
        \multirow{3}{*}{%
    \rotatebox[origin=c]{90}{%
        \parbox{1.3cm}{\centering \footnotesize \emph{GeoCLIP}}%
    }%
}  
        &GeoCLIP (F $\times$) & 99.52 & 98.61 & 88.33 & 86.76 & 65.68 \\
        &GeoCLIP (2K) & 99.32 & 98.73 & 94.73 & 89.22 & 74.95 \\
        &GeoCLIP & 99.21 & 99.24 & \textbf{96.05} & \textbf{95.95} & \textbf{78.56} \\
        \bottomrule
    \end{tabular}
    }
    \caption{Results on the proposed visual feature benchmark. We report the test accuracy of the models with the best validation performance. }
    \label{tab:linear_probing}
\end{table}

% \begin{table}[t!]
%     \centering
%     \resizebox{\linewidth}{!}{
%     \begin{tabular}{l l c c c c c }
%         \toprule
%         & Models & \begin{tabular}{@{}c@{}}Object \\ Shape\end{tabular} & \begin{tabular}{@{}c@{}}Con \\ cyclic\end{tabular} & \begin{tabular}{@{}c@{}}Two \\ Lines\end{tabular} & \begin{tabular}{@{}c@{}}Square \\ Shape\end{tabular} & \begin{tabular}{@{}c@{}}Angle \\ Detection\end{tabular} \\
%         % &Models & ObjectShape & Concyclic & TwoLines & SquareShape & AngleDetection \\
%         \midrule
%         \multirow{4}{*}{Baseline}
%         &OpenCLIP & \textbf{100.00} & 99.13 & 86.57 & 85.20 & 64.81 \\
%         &SigLIP & \textbf{100.00} & \textbf{99.71} & 89.26 & 89.31 & 76.86 \\
%         &DinoV2 & \textbf{100.00} & 98.01 & 85.30 & 91.24 & 22.43 \\
%         &ConvNeXT & \textbf{100.00} & 99.20 & 89.39 & 88.13 & 61.84 \\
%         \midrule
%         \multirow{3}{*}{SSL}
%         &Jigsaw & 86.11 & 63.85 & 49.98 & 61.88 & 11.44 \\
%         &MAE & 93.99 & 72.25 & 71.73 & 82.70 & 13.08 \\
%         &VQ-VAE & 63.05 & 60.97 & 48.10 & 57.35 & 9.22 \\
%         \midrule
%         \multirow{3}{*}{GeoCLIP}
%         &GeoCLIP (F $\times$) & 99.52 & 98.61 & 88.33 & 86.76 & 65.68 \\
%         &GeoCLIP (2K) & 99.32 & 98.73 & 94.73 & 89.22 & 74.95 \\
%         &GeoCLIP & 99.21 & 99.24 & \textbf{96.05} & \textbf{95.95} & \textbf{78.56} \\
%         \bottomrule
%     \end{tabular}
%     }
%     \caption{Results on the proposed visual feature benchmark. We report the test accuracy of the models with the best validation performance. }
%     \label{tab:linear_probing}
% \end{table}

% \multirow{7}{*}{%
%     \rotatebox[origin=c]{90}{%
%         \parbox{1.3cm}{\centering \emph{Unseen}}%
%     }%
% }  


With the proposed benchmark, we evaluate four widely adopted vision encoders for the open-sourced VLMs: OpenCLIP~\citep{clip}, SigLIP~\citep{siglip}, DinoV2~\citep{dinov2}, and ConvNeXT~\citep{convnext}.

To evaluate the vision encoder, we adopt a linear probing approach. Specifically, we add a linear layer on top of each encoder as a prediction head and train the linear layer from scratch while freezing the parameters of the vision encoder. We use a training set to train the prediction head and report the test accuracy with the best validation performance. The details for the hyper-parameters are described in \cref{sec:hparams}.

As shown in \cref{tab:linear_probing}, many existing vision encoders well recognize the shape of objects but fail to recognize the correct angle between two lines. The encoders also show some difficulties in recognizing the shape of a square and the relationship between two lines. Although the result may seem satisfactory at a glance, these errors will propagate to the downstream tasks when combined with LLMs. Hence, it is important to improve the recognition performance of the vision encoder further. 
%The t-SNE plots of the vision encoder embeddings are illustrated at \cref{fig:tsne_benchmark}. %Let's move this to the next section

\section{Improving the Vision Encoder Geometric Premises Recognition}

In this section, we first propose \geoclip{}, a new vision encoder designed to recognize geometric premises from diverse styles of diagrams.
To transfer the recognition to real-world PGPS benchmarks, we then propose a domain adaptation technique for \geoclip{} that leverages a small set of diagram–caption pairs from the target domains. 

% \subsection{Overall architecture}

% We begin by summarizing the architecture of our VLM, a combination of a vision encoder and a language model. For the vision encoder, we use \geoclip{}-DA, introduced in \cref{sec:domain_adaptation} with a two-layer MLP of GeLU activation as the projection layers following LLaVA-OneVision~\citep{llava-next}. For the language model, we employ LLama-3-8B-Instruct~\citep{llama}.
% For a given diagram and question pair in PGPS, we feed the vision encoder with the given diagram, and then the output of the encoder is used as an input token of LLM through the projection layer. The question text is then fed into the LLM, followed by the diagram embedding.
%The input to the language model is a concatenation of the diagram embedding, the textual input, and previously generated tokens, following a scheme similar to LLaVA~\citep{llava}.


\subsection{\geoclip{}}
% \subsection{Improving visual \geofeat{} recognition}
\label{sec:geoclip}

To make a vision encoder recognize geometric diagrams better, we propose a \geoclip{}, a vision encoder trained with CLIP objective with a newly developed 200,000 diagram-caption examples.
From the random diagram generator developed in \cref{sec:synthetic_data_engine}, we additionally sample 200,000 diagrams written in the formal language. Directly rendering these samples can result in a diagram that may not preserve the geometric properties. For example, the perpendicularity between two lines cannot be observed from the diagram without having the right angle sign, i.e., $\measuredrightanglewithsquare$. Therefore, we ensure to render the images containing all necessary geometric premises from its visual illustration.

For the caption of a diagram, we filter out some geometric properties from the original description of a diagram used to render the image. Specifically, we only keep the following four properties, concyclic, perpendicularity, angle measures, and length measures, from the visual premises shown in \cref{tab:alphageometry}. After that, we convert the remaining descriptions written in the formal language into natural language. We filter out some properties for two reasons.
First, some properties are not recognizable from the rendered diagram without additional information, e.g., congruency. These properties are listed as non-visual premises in \cref{tab:alphageometry}. Second, collinearity and parallelity occur so frequently that they can marginalize others.
Some examples of generated captions after filtering and translating are provided in the right-most column of \cref{fig:alphageometry}. We call the filtered caption as \emph{\captionstyle{} caption}.

With this dataset, we fine-tune OpenCLIP~\citep{clip} according to the CLIP objective which is formulated as:
% To be specific, given a collection of image–caption pairs \(\mathcal{D} := \{(D_i, X_i)\}_{i=1}^N\), we train the vision encoder \(g\) and the text encoder \(h\) using the following contrastive learning objective:
\begin{align}
    &\mathcal{L}_{\textrm{CLIP}}(\mathcal{D}, g, h) := \nonumber \\
    &\,\,\,\,\,\mathbb{E}_{\mathcal{D}} \!\biggl[ -\log \frac{\exp \bigl( g(D_i)^T \, h(X_i) / \tau \bigr)}{\sum_{X \in \{X_i\}_i} \exp \bigl( g(D_i)^T \, h(X) / \tau \bigr)} \biggr],
    \label{eq:clip}
\end{align}
where \(\mathcal{D} := \{(D_i, X_i)\}_{i=1}^N\) is the diagram-caption pairs, $g$ is the vision encoder, $h$ is the text encoder, and \(\tau\) is a temperature parameter.
% in \cref{eq:clip}, 
We named the resulting vision encoder as \geoclip{}. \cref{sec:hparams} provides the details, including hyper-parameters.

We compare the performance of \geoclip{} to other self-supervised approaches trained with the same dataset. We test three self-supervised approaches: Jigsaw~\citep{geoqa, geoqa-plus}, MAE~\citep{scagps, geox}, and VQ-VAE~\citep{unimath} used in previous work to improve the recognition performance of plane diagrams. We use the same architecture used for \geoclip{} for Jigsaw and MAE with the hyper-parameters used in the previous works. For VQ-VAE, we follow the architecture of \citet{unimath}. 
All model performances are measured through the linear probing used in \cref{sec:benchmakr_results}. 
% We measure the performances of the models through the linear probing used in \cref{sec:benchmakr_results}.

As shown in \cref{tab:linear_probing}, \geoclip{} recognizes geometric features better than existing baselines and self-supervised methods. The self-supervised approaches generally perform poorly for the benchmark, justifying the choice of the objective. We also compare the performance of \geoclip{} against other encoders such as OpenCLIP. Note that although we outperform the other encoders in difficult tasks such as SquareShape and AngleDetection, these results might be \emph{unfair} since the training set of \geoclip{} is similar to the diagrams in the benchmark.
The t-SNE plots of the embeddings from the vision encoders are illustrated at \cref{fig:tsne_benchmark}.

We further ablate the filtering process in \geoclip{}. To this end, we compare \geoclip{} with its two variants: \emph{GeoCLIP (F $\times$)}, which uses the captions generated without filtering. We also test \emph{GeoCLIP (2K)}, which is trained on only 2,000 pairs, to see the effectiveness of the large-scale dataset.
The results in \cref{tab:linear_probing} imply both the filtering and the training set size matter in enhancing geometric properties recognition.

% \paragraph{Solution program and numerical measurements prediction.}

% We then change the output of PGPS to predict not only the solution programs but also the numerical measurements in the diagram and text.
% While the previous PGPS models extract the numerical measurements from the text or assume as given features, our revised target enables problem solving even if the numerical measurements are only provided in the diagram. 

% Furthermore, the utilization of the vision encoder in the VLM would be enhanced due to the numerical measurements prediction from the diagram.

%To achieve this, we employ the diagram and geometric properties pairs obtained from the programs generated by the synthetic data engine.
%In forming the textual captions, we retain only the following geometric properties: concyclic, perpendicularity, and angle measure, which are basically visual features. Other properties are excluded for two reasons. First, certain properties cannot be inferred from the diagram without additional information, e.g., congruency, so those are considered non-visual features. Second, some visual features occur so frequently that they overshadow others, such as collinearity and parallelity. Consequently, those are removed as well.
% \sh{Examples of generated diagram and caption pairs are visualized at \cref{fig:geoclip}.}
% \todo{We need an example of this process in Appendix (which information is retained and removed from the original description.)}

%\cref{fig:alphageometry} shows the examples of sampled diagram-caption pairs.
%We produce a total of 200,000 diagram–caption pairs. 

%Using this dataset, we train OpenCLIP~\citep{clip}, whose architecture is ViT-L/14~\citep{vit} with an image resolution of $336 \times 336$, according to the CLIP objective at \cref{eq:clip}. Additional details regarding hyper-parameters are provided in \cref{sec:geoclip_hparams}.


\subsection{Domain adaptation of \geoclip{}}
% \subsection{Enhancing domain generalization capability}
\label{sec:domain_adaptation}

Although \geoclip{} enhances the geometric premises recognition on the benchmark set, the diagram styles in existing PGPS benchmarks differ, necessitating further adaptation. To overcome this challenge, we propose a domain adaptation method for \geoclip{}. To this end, we propose a few-shot domain adaptation method utilizing a few labeled diagrams.

% to transfer the knowledge from a source to a target domain. %To overcome this challenge, we propose a simple yet effective domain adaptation method that makes the vision encoder focus on important geometric information instead of irrelevant attributes, such as color and font family.
A domain-agnostic vision encoder must match the same diagrams drawn in different styles. To do so, we need a target domain diagram translated into the source domain style or the source diagrams translated into the target domain style. With these translated images, we can guide the model to focus on key geometric information instead of irrelevant attributes, such as color and font family. However, in practice, it is difficult to obtain the same diagrams with different styles. 

We develop a way to translate the target diagrams into source style.
Thankfully, since well-known PGPS datasets come with diagram captions written in formal languages~\citep{intergps}, we can easily convert them to the AlphaGeometry-style descriptions. 
Given the translated descriptions, we utilize the rendering engine of AlphaGeometry to translate the target domain images into the source domain. 
With the translation, we can generate the same diagram in the source domain style. 
\cref{fig:domain_adaptation_samples} provides examples of the diagram pairs with different styles. However, in some cases, the original description contains geometric premises that are unrecognizable from the diagram, such as \(\angle ACB = 35.0\) in \cref{fig:geoqa_example}. Therefore, we apply the same filtering process used in \geoclip{} to translate the AlphaGeometry-style descriptions into natural languages.

%In practice, however, not only do the diagrams have different styles, but also the captions. For example, the textual description of the GeoQA diagram in \cref{fig:geoqa_example} includes non-visual details, such as \(\angle ACB = 35.0\), which we removed from the dataset used to train \geoclip{}. Therefore, we apply the same filtering process used to curate captions for \geoclip{}. To this end, 

Formally, let $\mathcal{D}_{S} := \{(D_S^{(i)}, X_S^{(i)}) \}_{i=1}^{N_S}$ be the diagram-caption pairs from source domain $S$, e.g., the synthetic diagrams, and let $\mathcal{D}_{T_j} := \{(D_{T_j}^{(i)}, X_{T_j}^{(i)}) \}_{i=1}^{N_{T_j}}$ be the set of diagram-caption pairs of target domain $T_j$, e.g., the PGPS benchmarks. With the translation process described above, we can synthesize a style-transferred diagram-caption pair $(\hat{D}_{T_j}^{(i)}, \hat{X}_{T_j}^{(i)})$ for each diagram $D_{T_j}^{(i)}$ and caption $X_{T_j}^{(i)}$ in target domain $T_j$.

We perform domain adaptation by fine-tuning the vision encoder through the style-transferred diagram-caption pairs.
Let $\hat{\mathcal{D}}_{T_j}$ be a collection of the original diagram and style-transferred captions, i.e., $\hat{\mathcal{D}}_{T_j} = \{(D_{T_j}^{(i)}, \hat{X}_{T_j}^{(i)})\}_{i=1}^{N_{T_j}}$, and let $\hat{\mathcal{D}}_{T_jS}$ be a collection of the original and style transferred diagram pairs, i.e., $\hat{\mathcal{D}}_{T_jS}= \{(D_{T_j}^{(i)}, \hat{D}_{T_j}^{(i)}) \}_{i=1}^{N_{T_j}}$. The cross-domain adaptation objective is written as
%jointly training with diagram–caption pairs from both the source and target domains, as well as with the diagram pairs produced by the synthetic engine. Formally, we update the vision and text encoders $g$ and $h$ using the CLIP objective in \cref{eq:clip}, defined as:
\begin{align}
    &\mathcal{L}_{\textrm{CLIP-DA}}(\mathcal{D}_S, \{\mathcal{D}_{T_j}\}_j, g, h) := 
    \mathcal{L}_{\textrm{CLIP}}(\mathcal{D}_S, g, h) + \nonumber\\
    &\,\,\,\,\,\,\,\,\Sigma_j \mathcal{L}_{\textrm{CLIP}}(\hat{\mathcal{D}}_{T_j}, g, h) + \mathcal{L}_{\textrm{CLIP}}(\hat{\mathcal{D}}_{T_jS}, g, g),
    \label{eqn:da}
\end{align}
where $g$ and $h$ are the vision and text encoders of \geoclip{}, respectively. Note that we do not use the original captions from the target domain, since our goal is to adapt the vision encoder to the target domain, not the text encoder.
% We refer to the domain-adapted vision encoder as \geoclip{}-DA.
% Figure x illustrates the vision encoder pre-training phase.

% \paragraph{Unified solution program language.}

% Although GeoCLIP-DA is capable of capturing relevant visual \geofeat{}s across diverse diagram styles, the difference in solution program languages often restrict the VLM to fully leverage the enhanced visual embeddings. To address the gap, we unify the programming languages of the PGPS datasets, i.e., GeoQA and PGPS9K, by implementing a converter that transforms the solution programs from GeoQA into the PGPS9K format.
% We also convert the prediction target to predict both solution program and the numerical measurements as mentioned in \cref{sec:geoclip}.
% Examples of the training data are at \cref{fig:training_data}.

% \subsection{Training details - \dw{can we move this to experiments?}}

% For the remaining experiments, we designate GeoQA and PGPS9K as the target domains. In GeoQA, $N_{T_j}=50$ diagrams are sampled and manually annotated with \geoclip{}–style captions, followed by the generation of $N_{S_j}=100$ synthetic diagrams for each sampled diagram.
% In PGPS9K, where visual features are explicitly provided, the given features are converted into natural language to produce \geoclip{}–style captions. From this set, 50 diagram–caption pairs are chosen, and corresponding AlphaGeometry programs are generated. Subsequently, each sampled target diagram is supplemented with $N_{S_j}=100$ synthetic diagrams for domain adaptation.

% Next, we build the VLM using GeoCLIP-DA. We freeze the vision encoder and train the projection and language model on the revised training data of GeoQA and PGPS9K. Further details on hyper-parameters are provided in \cref{sec:vlm_hparams}.
\section{Experimental Analysis}
\label{sec:exp}
We now describe in detail our experimental analysis. The experimental section is organized as follows:
%\begin{enumerate}[noitemsep,topsep=0pt,parsep=0pt,partopsep=0pt,leftmargin=0.5cm]
%\item 

\noindent In {\bf 
Section~\ref{exp:setup}}, we introduce the datasets and methods to evaluate the previously defined accuracy measures.

%\item
\noindent In {\bf 
Section~\ref{exp:qual}}, we illustrate the limitations of existing measures with some selected qualitative examples.

%\item 
\noindent In {\bf 
Section~\ref{exp:quant}}, we continue by measuring quantitatively the benefits of our proposed measures in terms of {\it robustness} to lag, noise, and normal/abnormal ratio.

%\item 
\noindent In {\bf 
Section~\ref{exp:separability}}, we evaluate the {\it separability} degree of accurate and inaccurate methods, using the existing and our proposed approaches.

%\item
\noindent In {\bf 
Section~\ref{sec:entropy}}, we conduct a {\it consistency} evaluation, in which we analyze the variation of ranks that an AD method can have with an accuracy measures used.

%\item 
\noindent In {\bf 
Section~\ref{sec:exectime}}, we conduct an {\it execution time} evaluation, in which we analyze the impact of different parameters related to the accuracy measures and the time series characteristics. 
We focus especially on the comparison of the different VUS implementations.
%\end{enumerate}

\begin{table}[tb]
\caption{Summary characteristics (averaged per dataset) of the public datasets of TSB-UAD (S.: Size, Ano.: Anomalies, Ab.: Abnormal, Den.: Density)}
\label{table:charac}
%\vspace{-0.2cm}
\footnotesize
\begin{center}
\scalebox{0.82}{
\begin{tabular}{ |r|r|r|r|r|r|} 
 \hline
\textbf{\begin{tabular}[c]{@{}c@{}}Dataset \end{tabular}} & 
\textbf{\begin{tabular}[c]{@{}c@{}}S. \end{tabular}} & 
\textbf{\begin{tabular}[c]{c@{}} Len.\end{tabular}} & 
\textbf{\begin{tabular}[c]{c@{}} \# \\ Ano. \end{tabular}} &
\textbf{\begin{tabular}[c]{c@{}c@{}} \# \\ Ab. \\ Points\end{tabular}} &
\textbf{\begin{tabular}[c]{c@{}c@{}} Ab. \\ Den. \\ (\%)\end{tabular}} \\ \hline
Dodgers \cite{10.1145/1150402.1150428} & 1 & 50400   & 133.0     & 5612.0  &11.14 \\ \hline
SED \cite{doi:10.1177/1475921710395811}& 1 & 100000   & 75.0     & 3750.0  & 3.7\\ \hline
ECG \cite{goldberger_physiobank_2000}   & 52 & 230351  & 195.6     & 15634.0  &6.8 \\ \hline
IOPS \cite{IOPS}   & 58 & 102119  & 46.5     & 2312.3   &2.1 \\ \hline
KDD21 \cite{kdd} & 250 &77415   & 1      & 196.5   &0.56 \\ \hline
MGAB \cite{markus_thill_2020_3762385}   & 10 & 100000  & 10.0     & 200.0   &0.20 \\ \hline
NAB \cite{ahmad_unsupervised_2017}   & 58 & 6301   & 2.0      & 575.5   &8.8 \\ \hline
NASA-M. \cite{10.1145/3449726.3459411}   & 27 & 2730   & 1.33      & 286.3   &11.97 \\ \hline
NASA-S. \cite{10.1145/3449726.3459411}   & 54 & 8066   & 1.26      & 1032.4   &12.39 \\ \hline
SensorS. \cite{YAO20101059}   & 23 & 27038   & 11.2     & 6110.4   &22.5 \\ \hline
YAHOO \cite{yahoo}  & 367 & 1561   & 5.9      & 10.7   &0.70 \\ \hline 
\end{tabular}}
\end{center}
\end{table}











\subsection{Experimental Setup and Settings}
\label{exp:setup}
%\vspace{-0.1cm}

\begin{figure*}[tb]
  \centering
  \includegraphics[width=1\linewidth]{figures/quality.pdf}
  %\vspace{-0.7cm}
  \caption{Comparison of evaluation measures (proposed measures illustrated in subplots (b,c,d,e); all others summarized in subplots (f)) on two examples ((A)AE and OCSM applied on MBA(805) and (B) LOF and OCSVM applied on MBA(806)), illustrating the limitations of existing measures for scores with noise or containing a lag. }
  \label{fig:quality}
  %\vspace{-0.1cm}
\end{figure*}

We implemented the experimental scripts in Python 3.8 with the following main dependencies: sklearn 0.23.0, tensorflow 2.3.0, pandas 1.2.5, and networkx 2.6.3. In addition, we used implementations from our TSB-UAD benchmark suite.\footnote{\scriptsize \url{https://www.timeseries.org/TSB-UAD}} For reproducibility purposes, we make our datasets and code available.\footnote{\scriptsize \url{https://www.timeseries.org/VUS}}
\newline \textbf{Datasets: } For our evaluation purposes, we use the public datasets identified in our TSB-UAD benchmark. The latter corresponds to $10$ datasets proposed in the past decades in the literature containing $900$ time series with labeled anomalies. Specifically, each point in every time series is labeled as normal or abnormal. Table~\ref{table:charac} summarizes relevant characteristics of the datasets, including their size, length, and statistics about the anomalies. In more detail:

\begin{itemize}
    \item {\bf SED}~\cite{doi:10.1177/1475921710395811}, from the NASA Rotary Dynamics Laboratory, records disk revolutions measured over several runs (3K rpm speed).
	\item {\bf ECG}~\cite{goldberger_physiobank_2000} is a standard electrocardiogram dataset and the anomalies represent ventricular premature contractions. MBA(14046) is split to $47$ series.
	\item {\bf IOPS}~\cite{IOPS} is a dataset with performance indicators that reflect the scale, quality of web services, and health status of a machine.
	\item {\bf KDD21}~\cite{kdd} is a composite dataset released in a SIGKDD 2021 competition with 250 time series.
	\item {\bf MGAB}~\cite{markus_thill_2020_3762385} is composed of Mackey-Glass time series with non-trivial anomalies. Mackey-Glass data series exhibit chaotic behavior that is difficult for the human eye to distinguish.
	\item {\bf NAB}~\cite{ahmad_unsupervised_2017} is composed of labeled real-world and artificial time series including AWS server metrics, online advertisement clicking rates, real time traffic data, and a collection of Twitter mentions of large publicly-traded companies.
	\item {\bf NASA-SMAP} and {\bf NASA-MSL}~\cite{10.1145/3449726.3459411} are two real spacecraft telemetry data with anomalies from Soil Moisture Active Passive (SMAP) satellite and Curiosity Rover on Mars (MSL).
	\item {\bf SensorScope}~\cite{YAO20101059} is a collection of environmental data, such as temperature, humidity, and solar radiation, collected from a sensor measurement system.
	\item {\bf Yahoo}~\cite{yahoo} is a dataset consisting of real and synthetic time series based on the real production traffic to some of the Yahoo production systems.
\end{itemize}


\textbf{Anomaly Detection Methods: }  For the experimental evaluation, we consider the following baselines. 

\begin{itemize}
\item {\bf Isolation Forest (IForest)}~\cite{liu_isolation_2008} constructs binary trees based on random space splitting. The nodes (subsequences in our specific case) with shorter path lengths to the root (averaged over every random tree) are more likely to be anomalies. 
\item {\bf The Local Outlier Factor (LOF)}~\cite{breunig_lof_2000} computes the ratio of the neighbor density to the local density. 
\item {\bf Matrix Profile (MP)}~\cite{yeh_time_2018} detects as anomaly the subsequence with the most significant 1-NN distance. 
\item {\bf NormA}~\cite{boniol_unsupervised_2021} identifies the normal patterns based on clustering and calculates each point's distance to normal patterns weighted using statistical criteria. 
\item {\bf Principal Component Analysis (PCA)}~\cite{aggarwal_outlier_2017} projects data to a lower-dimensional hyperplane. Outliers are points with a large distance from this plane. 
\item {\bf Autoencoder (AE)} \cite{10.1145/2689746.2689747} projects data to a lower-dimensional space and reconstructs it. Outliers are expected to have larger reconstruction errors. 
\item {\bf LSTM-AD}~\cite{malhotra_long_2015} use an LSTM network that predicts future values from the current subsequence. The prediction error is used to identify anomalies.
\item {\bf Polynomial Approximation (POLY)} \cite{li_unifying_2007} fits a polynomial model that tries to predict the values of the data series from the previous subsequences. Outliers are detected with the prediction error. 
\item {\bf CNN} \cite{8581424} built, using a convolutional deep neural network, a correlation between current and previous subsequences, and outliers are detected by the deviation between the prediction and the actual value. 
\item {\bf One-class Support Vector Machines (OCSVM)} \cite{scholkopf_support_1999} is a support vector method that fits a training dataset and finds the normal data's boundary.
\end{itemize}

\subsection{Qualitative Analysis}
\label{exp:qual}



We first use two examples to demonstrate qualitatively the limitations of existing accuracy evaluation measures in the presence of lag and noise, and to motivate the need for a new approach. 
These two examples are depicted in Figure~\ref{fig:quality}. 
The first example, in Figure~\ref{fig:quality}(A), corresponds to OCSVM and AE on the MBA(805) dataset (named MBA\_ECG805\_data.out in the ECG dataset). 

We observe in Figure~\ref{fig:quality}(A)(a.1) and (a.2) that both scores identify most of the anomalies (highlighted in red). However, the OCSVM score points to more false positives (at the end of the time series) and only captures small sections of the anomalies. On the contrary, the AE score points to fewer false positives and captures all abnormal subsequences. Thus we can conclude that, visually, AE should obtain a better accuracy score than OCSVM. Nevertheless, we also observe that the AE score is lagged with the labels and contains more noise. The latter has a significant impact on the accuracy of evaluation measures. First, Figure~\ref{fig:quality}(A)(c) is showing that AUC-PR is better for OCSM (0.73) than for AE (0.57). This is contradictory with what is visually observed from Figure~\ref{fig:quality}(A)(a.1) and (a.2). However, when using our proposed measure R-AUC-PR, OCSVM obtains a lower score (0.83) than AE (0.89). This confirms that, in this example, a buffer region before the labels helps to capture the true value of an anomaly score. Overall, Figure~\ref{fig:quality}(A)(f) is showing in green and red the evolution of accuracy score for the 13 accuracy measures for AE and OCSVM, respectively. The latter shows that, in addition to Precision@k and Precision, our proposed approach captures the quality order between the two methods well.

We now present a second example, on a different time series, illustrated in Figure~\ref{fig:quality}(B). 
In this case, we demonstrate the anomaly score of OCSVM and LOF (depicted in Figure~\ref{fig:quality}(B)(a.1) and (a.2)) applied on the MBA(806) dataset (named MBA\_ECG806\_data.out in the ECG dataset). 
We observe that both methods produce the same level of noise. However, LOF points to fewer false positives and captures more sections of the abnormal subsequences than OCSVM. 
Nevertheless, the LOF score is slightly lagged with the labels such that the maximum values in the LOF score are slightly outside of the labeled sections. 
Thus, as illustrated in Figure~\ref{fig:quality}(B)(f), even though we can visually consider that LOF is performing better than OCSM, all usual measures (Precision, Recall, F, precision@k, and AUC-PR) are judging OCSM better than AE. On the contrary, measures that consider lag (Rprecision, Rrecall, RF) rank the methods correctly. 
However, due to threshold issues, these measures are very close for the two methods. Overall, only AUC-ROC and our proposed measures give a higher score for LOF than for OCSVM.

\subsection{Quantitative Analysis}
\label{exp:case}

\begin{figure}[t]
  \centering
  \includegraphics[width=1\linewidth]{figures/eval_case_study.pdf}
  %\vspace*{-0.7cm}
  \caption{\commentRed{
  Comparison of evaluation measures for synthetic data examples across various scenarios. S8 represents the oracle case, where predictions perfectly align with labeled anomalies. Problematic cases are highlighted in the red region.}}
  %\vspace*{-0.5cm}
  \label{fig:eval_case_study}
\end{figure}
\commentRed{
We present the evaluation results for different synthetic data scenarios, as shown in Figure~\ref{fig:eval_case_study}. These scenarios range from S1, where predictions occur before the ground truth anomaly, to S12, where predictions fall within the ground truth region. The red-shaded regions highlight problematic cases caused by a lack of adaptability to lags. For instance, in scenarios S1 and S2, a slight shift in the prediction leads to measures (e.g., AUC-PR, F score) that fail to account for lags, resulting in a zero score for S1 and a significant discrepancy between the results of S1 and S2. Thus, we observe that our proposed VUS effectively addresses these issues and provides robust evaluations results.}

%\subsection{Quantitative Analysis}
%\subsection{Sensitivity and Separability Analysis}
\subsection{Robustness Analysis}
\label{exp:quant}


\begin{figure}[tb]
  \centering
  \includegraphics[width=1\linewidth]{figures/lag_sensitivity_analysis.pdf}
  %\vspace*{-0.7cm}
  \caption{For each method, we compute the accuracy measures 10 times with random lag $\ell \in [-0.25*\ell,0.25*\ell]$ injected in the anomaly score. We center the accuracy average to 0.}
  %\vspace*{-0.5cm}
  \label{fig:lagsensitivity}
\end{figure}

We have illustrated with specific examples several of the limitations of current measures. 
We now evaluate quantitatively the robustness of the proposed measures when compared to the currently used measures. 
We first evaluate the robustness to noise, lag, and normal versus abnormal points ratio. We then measure their ability to separate accurate and inaccurate methods.
%\newline \textbf{Sensitivity Analysis: } 
We first analyze the robustness of different approaches quantitatively to different factors: (i) lag, (ii) noise, and (iii) normal/abnormal ratio. As already mentioned, these factors are realistic. For instance, lag can be either introduced by the anomaly detection methods (such as methods that produce a score per subsequences are only high at the beginning of abnormal subsequences) or by human labeling approximation. Furthermore, even though lag and noises are injected, an optimal evaluation metric should not vary significantly. Therefore, we aim to measure the variance of the evaluation measures when we vary the lag, noise, and normal/abnormal ratio. We proceed as follows:

\begin{enumerate}[noitemsep,topsep=0pt,parsep=0pt,partopsep=0pt,leftmargin=0.5cm]
\item For each anomaly detection method, we first compute the anomaly score on a given time series.
\item We then inject either lag $l$, noise $n$ or change the normal/abnormal ratio $r$. For 10 different values of $l \in [-0.25*\ell,0.25*\ell]$, $n \in [-0.05*(max(S_T)-min(S_T)),0.05*(max(S_T)-min(S_T))]$ and $r \in [0.01,0.2]$, we compute the 13 different measures.
\item For each evaluation measure, we compute the standard deviation of the ten different values. Figure~\ref{fig:lagsensitivity}(b) depicts the different lag values for six AD methods applied on a data series in the ECG dataset.
\item We compute the average standard deviation for the 13 different AD quality measures. For example, figure~\ref{fig:lagsensitivity}(a) depicts the average standard deviation for ten different lag values over the AD methods applied on the MBA(805) time series.
\item We compute the average standard deviation for the every time series in each dataset (as illustrated in Figure~\ref{fig:sensitivity_per_data}(b to j) for nine datasets of the benchmark.
\item We compute the average standard deviation for the every dataset (as illustrated in Figure~\ref{fig:sensitivity_per_data}(a.1) for lag, Figure~\ref{fig:sensitivity_per_data}(a.2) for noise and Figure~\ref{fig:sensitivity_per_data}(a.3) for normal/abnormal ratio).
\item We finally compute the Wilcoxon test~\cite{10.2307/3001968} and display the critical diagram over the average standard deviation for every time series (as illustrated in Figure~\ref{fig:sensitivity}(a.1) for lag, Figure~\ref{fig:sensitivity}(a.2) for noise and Figure~\ref{fig:sensitivity}(a.3) for normal/abnormal ratio).
\end{enumerate}

%height=8.5cm,

\begin{figure}[tb]
  \centering
  \includegraphics[width=\linewidth]{figures/sensitivity_per_data_long.pdf}
%  %\vspace*{-0.3cm}
  \caption{Robustness Analysis for nine datasets: we report, over the entire benchmark, the average standard deviation of the accuracy values of the measures, under varying (a.1) lag, (a.2) noise, and (a.3) normal/abnormal ratio. }
  \label{fig:sensitivity_per_data}
\end{figure}

\begin{figure*}[tb]
  \centering
  \includegraphics[width=\linewidth]{figures/sensitivity_analysis.pdf}
  %\vspace*{-0.7cm}
  \caption{Critical difference diagram computed using the signed-rank Wilkoxon test (with $\alpha=0.1$) for the robustness to (a.1) lag, (a.2) noise and (a.3) normal/abnormal ratio.}
  \label{fig:sensitivity}
\end{figure*}

The methods with the smallest standard deviation can be considered more robust to lag, noise, or normal/abnormal ratio from the above framework. 
First, as stated in the introduction, we observe that non-threshold-based measures (such as AUC-ROC and AUC-PR) are indeed robust to noise (see Figure~\ref{fig:sensitivity_per_data}(a.2)), but not to lag. Figure~\ref{fig:sensitivity}(a.1) demonstrates that our proposed measures VUS-ROC, VUS-PR, R-AUC-ROC, and R-AUC-PR are significantly more robust to lag. Similarly, Figure~\ref{fig:sensitivity}(a.2) confirms that our proposed measures are significantly more robust to noise. However, we observe that, among our proposed measures, only VUS-ROC and R-AUC-ROC are robust to the normal/abnormal ratio and not VUS-PR and R-AUC-PR. This is explained by the fact that Precision-based measures vary significantly when this ratio changes. This is confirmed by Figure~\ref{fig:sensitivity_per_data}(a.3), in which we observe that Precision and Rprecision have a high standard deviation. Overall, we observe that VUS-ROC is significantly more robust to lag, noise, and normal/abnormal ratio than other measures.




\subsection{Separability Analysis}
\label{exp:separability}

%\newline \textbf{Separability Analysis: } 
We now evaluate the separability capacities of the different evaluation metrics. 
\commentRed{The main objective is to measure the ability of accuracy measures to separate accurate methods from inaccurate ones. More precisely, an appropriate measure should return accuracy scores that are significantly higher for accurate anomaly scores than for inaccurate ones.}
We thus manually select accurate and inaccurate anomaly detection methods and verify if the accuracy evaluation scores are indeed higher for the accurate than for the inaccurate methods. Figure~\ref{fig:separability} depicts the latter separability analysis applied to the MBA(805) and the SED series. 
The accurate and inaccurate anomaly scores are plotted in green and red, respectively. 
We then consider 12 different pairs of accurate/inaccurate methods among the eight previously mentioned anomaly scores. 
We slightly modify each score 50 different times in which we inject lag and noises and compute the accuracy measures. 
Figure~\ref{fig:separability}(a.4) and Figure~\ref{fig:separability}(b.4) are divided into four different subplots corresponding to 4 pairs (selected among the twelve different pairs due to lack of space). 
Each subplot corresponds to two box plots per accuracy measure. 
The green and red box plots correspond to the 50 accuracy measures on the accurate and inaccurate methods. 
If the red and green box plots are well separated, we can conclude that the corresponding accuracy measures are separating the accurate and inaccurate methods well. 
We observe that some accuracy measures (such as VUS-ROC) are more separable than others (such as RF). We thus measure the separability of the two box-plots by computing the Z-test. 

\begin{figure*}[tb]
  \centering
  \includegraphics[width=1\linewidth]{figures/pairwise_comp_example_long.pdf}
  %\vspace*{-0.5cm}
  \caption{Separability analysis applied on 4 pairs of accurate (green) and inaccurate (red) methods on (a) the MBA(805) data series, and (b) the SED data series.}
  %\vspace*{-0.3cm}
  \label{fig:separability}
\end{figure*}

We now aggregate all the results and compute the average Z-test for all pairs of accurate/inaccurate datasets (examples are shown in Figures~\ref{fig:separability}(a.2) and (b.2) for accurate anomaly scores, and in Figures~\ref{fig:separability}(a.3) and (b.3) for inaccurate anomaly scores, for the MBA(805) and SED series, respectively). 
Next, we perform the same operation over three different data series: MBA (805), MBA(820), and SED. 
Then, we depict the average Z-test for these three datasets in Figure~\ref{fig:separability_agg}(a). 
Finally, we show the average Z-test for all datasets in Figure~\ref{fig:separability_agg}(b). 


We observe that our proposed VUS-based and Range-based measures are significantly more separable than other current accuracy measures (up to two times for AUC-ROC, the best measures of all current ones). Furthermore, when analyzed in detail in Figure~\ref{fig:separability} and Figure~\ref{fig:separability_agg}, we confirm that VUS-based and Range-based are more separable over all three datasets. 

\begin{figure}[tb]
  \centering
  \includegraphics[width=\linewidth]{figures/agregated_sep_analysis.pdf}
  %\vspace*{-0.5cm}
  \caption{Overall separability analysis (averaged z-test between the accuracy values distributions of accurate and inaccurate methods) applied on 36 pairs on 3 datasets.}
  \label{fig:separability_agg}
\end{figure}


\noindent \textbf{Global Analysis: } Overall, we observe that VUS-ROC is the most robust (cf. Figure~\ref{fig:sensitivity}) and separable (cf. Figure~\ref{fig:separability_agg}) measure. 
On the contrary, Precision and Rprecision are non-robust and non-separable. 
Among all previous accuracy measures, only AUC-ROC is robust and separable. 
Popular measures, such as, F, RF, AUC-ROC, and AUC-PR are robust but non-separable.

In order to visualize the global statistical analysis, we merge the robustness and the separability analysis into a single plot. Figure~\ref{fig:global} depicts one scatter point per accuracy measure. 
The x-axis represents the averaged standard deviation of lag and noise (averaged values from Figure~\ref{fig:sensitivity_per_data}(a.1) and (a.2)). The y-axis corresponds to the averaged Z-test (averaged value from Figure~\ref{fig:separability_agg}). 
Finally, the size of the points corresponds to the sensitivity to the normal/abnormal ratio (values from Figure~\ref{fig:sensitivity_per_data}(a.3)). 
Figure~\ref{fig:global} demonstrates that our proposed measures (located at the top left section of the plot) are both the most robust and the most separable. 
Among all previous accuracy measures, only AUC-ROC is on the top left section of the plot. 
Popular measures, such as, F, RF, AUC-ROC, AUC-PR are on the bottom left section of the plot. 
The latter underlines the fact that these measures are robust but non-separable.
Overall, Figure~\ref{fig:global} confirms the effectiveness and superiority of our proposed measures, especially of VUS-ROC and VUS-PR.


\begin{figure}[tb]
  \centering
  \includegraphics[width=\linewidth]{figures/final_result.pdf}
  \caption{Evaluation of all measures based on: (y-axis) their separability (avg. z-test), (x-axis) avg. standard deviation of the accuracy values when varying lag and noise, (circle size) avg. standard deviation of the accuracy values when varying the normal/abnormal ratio.}
  \label{fig:global}
\end{figure}




\subsection{Consistency Analysis}
\label{sec:entropy}

In this section, we analyze the accuracy of the anomaly detection methods provided by the 13 accuracy measures. The objective is to observe the changes in the global ranking of anomaly detection methods. For that purpose, we formulate the following assumptions. First, we assume that the data series in each benchmark dataset are similar (i.e., from the same domain and sharing some common characteristics). As a matter of fact, we can assume that an anomaly detection method should perform similarly on these data series of a given dataset. This is confirmed when observing that the best anomaly detection methods are not the same based on which dataset was analyzed. Thus the ranking of the anomaly detection methods should be different for different datasets, but similar for every data series in each dataset. 
Therefore, for a given method $A$ and a given dataset $D$ containing data series of the same type and domain, we assume that a good accuracy measure results in a consistent rank for the method $A$ across the dataset $D$. 
The consistency of a method's ranks over a dataset can be measured by computing the entropy of these ranks. 
For instance, a measure that returns a random score (and thus, a random rank for a method $A$) will result in a high entropy. 
On the contrary, a measure that always returns (approximately) the same ranks for a given method $A$ will result in a low entropy. 
Thus, for a given method $A$ and a given dataset $D$ containing data series of the same type and domain, we assume that a good accuracy measure results in a low entropy for the different ranks for method $A$ on dataset $D$.

\begin{figure*}[tb]
  \centering
  \includegraphics[width=\linewidth]{figures/entropy_long.pdf}
  %\vspace*{-0.5cm}
  \caption{Accuracy evaluation of the anomaly detection methods. (a) Overall average entropy per category of measures. Analysis of the (b) averaged rank and (c) averaged rank entropy for each method and each accuracy measure over the entire benchmark. Example of (b.1) average rank and (c.1) entropy on the YAHOO dataset, KDD21 dataset (b.2, c.2). }
  \label{fig:entropy}
\end{figure*}

We now compute the accuracy measures for the nine different methods (we compute the anomaly scores ten different times, and we use the average accuracy). 
Figures~\ref{fig:entropy}(b.1) and (b.2) report the average ranking of the anomaly detection methods obtained on the YAHOO and KDD21 datasets, respectively. 
The x-axis corresponds to the different accuracy measures. We first observe that the rankings are more separated using Range-AUC and VUS measures for these two datasets. Figure~\ref{fig:entropy}(b) depicts the average ranking over the entire benchmark. The latter confirms the previous observation that VUS measures provide more separated rankings than threshold-based and AUC-based measures. We also observe an interesting ranking evolution for the YAHOO dataset illustrated in Figure~\ref{fig:entropy}(b.1). We notice that both LOF and MatrixProfile (brown and pink curve) have a low rank (between 4 and 5) using threshold and AUC-based measures. However, we observe that their ranks increase significantly for range-based and VUS-based measures (between 2.5 and 3). As we noticed by looking at specific examples (see Figure~\ref{exp:qual}), LOF and MatrixProfile can suffer from a lag issue even though the anomalies are well-identified. Therefore, the range-based and VUS-based measures better evaluate these two methods' detection capability.


Overall, the ranking curves show that the ranks appear more chaotic for threshold-based than AUC-, Range-AUC-, and VUS-based measures. 
In order to quantify this observation, we compute the Shannon Entropy of the ranks of each anomaly detection method. 
In practice, we extract the ranks of methods across one dataset and compute Shannon's Entropy of the different ranks. 
Figures~\ref{fig:entropy}(c.1) and (c.2) depict the entropy of each of the nine methods for the YAHOO and KDD21 datasets, respectively. 
Figure~\ref{fig:entropy}(c) illustrates the averaged entropy for all datasets in the benchmark for each measure and method, while Figure~\ref{fig:entropy}(a) shows the averaged entropy for each category of measures.
We observe that both for the general case (Figure~\ref{fig:entropy}(a) and Figure~\ref{fig:entropy}(c)) and some specific cases (Figures~\ref{fig:entropy}(c.1) and (c.2)), the entropy is reducing when using AUC-, Range-AUC-, and VUS-based measures. 
We report the lowest entropy for VUS-based measures. 
Moreover, we notice a significant drop between threshold-based and AUC-based. 
This confirms that the ranks provided by AUC- and VUS-based measures are consistent for data series belonging to one specific dataset. 


Therefore, based on the assumption formulated at the beginning of the section, we can thus conclude that AUC, range-AUC, and VUS-based measures are providing more consistent rankings. Finally, as illustrated in Figure~\ref{fig:entropy}, we also observe that VUS-based measures result in the most ordered and similar rankings for data series from the same type and domain.










\subsection{Execution Time Analysis}
\label{sec:exectime}

In this section, we evaluate the execution time required to compute different evaluation measures. 
In Section~\ref{sec:synthetic_eval_time}, we first measure the influence of different time series characteristics and VUS parameters on the execution time. In Section~\ref{sec:TSB_eval_time}, we  measure the execution time of VUS (VUS-ROC and VUS-PR simultaneously), R-AUC (R-AUC-ROC and R-AUC-PR simultaneously), and AUC-based measures (AUC-ROC and AUC-PR simultaneously) on the TSB-UAD benchmark. \commentRed{As demonstrated in the previous section, threshold-based measures are not robust, have a low separability power, and are inconsistent. 
Such measures are not suitable for evaluating anomaly detection methods. Thus, in this section, we do not consider threshold-based measures.}


\subsubsection{Evaluation on Synthetic Time Series}\hfill\\
\label{sec:synthetic_eval_time}

We first analyze the impact that time series characteristics and parameters have on the computation time of VUS-based measures. 
to that effect, we generate synthetic time series and labels, where we vary the following parameters: (i) the number of anomalies {\bf$\alpha$} in the time series, (ii) the average \textbf{$\mu(\ell_a)$} and standard deviation $\sigma(\ell_a)$ of the anomalies lengths in the time series (all the anomalies can have different lengths), (iii) the length of the time series \textbf{$|T|$}, (iv) the maximum buffer length \textbf{$L$}, and (v) the number of thresholds \textbf{$N$}.


We also measure the influence on the execution time of the R-AUC- and AUC- related parameter, that is, the number of thresholds ($N$).
The default values and the range of variation of these parameters are listed in Table~\ref{tab:parameter_range_time}. 
For VUS-based measures, we evaluate the execution time of the initial VUS implementation, as well as the two optimized versions, VUS$_{opt}$ and VUS$_{opt}^{mem}$.

\begin{table}[tb]
    \centering
    \caption{Value ranges for the parameters: number of anomalies ($\alpha$), average and standard deviation anomaly length ($\mu(\ell_a)$,$\sigma(\ell_a)$), time series length ($|T|$), maximum buffer length ($L$), and number of thresholds ($N$).}
    \begin{tabular}{|c|c|c|c|c|c|c|} 
 \hline
 Param. & $\alpha$ & $\mu(\ell_a)$ & $\sigma(\ell_{a})$ & $|T|$ & $L$ & $N$ \\ [0.5ex] 
 \hline\hline
 \textbf{Default} & 10 & 10 & 0 & $10^5$ & 5 & 250\\ 
 \hline
 Min. & 0 & 0 & 0 & $10^3$ & 0 & 2 \\
 \hline
 Max. & $2*10^3$ & $10^3$ & $10$ & $10^5$ & $10^3$ & $10^3$ \\ [1ex] 
 \hline
\end{tabular}
    \label{tab:parameter_range_time}
\end{table}


Figure~\ref{fig:sythetic_exp_time} depicts the execution time (averaged over ten runs) for each parameter listed in Table~\ref{tab:parameter_range_time}. 
Overall, we observe that the execution time of AUC-based and R-AUC-based measures is significantly smaller than VUS-based measures.
In the following paragraph, we analyze the influence of each parameter and compare the experimental execution time evaluation to the theoretical complexity reported in Table~\ref{tab:complexity_summary}.

\vspace{0.2cm}
\noindent {\bf [Influence of $\alpha$]}:
In Figure~\ref{fig:sythetic_exp_time}(a), we observe that the VUS, VUS$_{opt}$, and VUS$_{opt}^{mem}$ execution times are linearly increasing with $\alpha$. 
The increase in execution time for VUS, VUS$_{opt}$, and VUS$_{opt}^{mem}$ is more pronounced when we vary $\alpha$, in contrast to $l_a$ (which nevertheless, has a similar effect on the overall complexity). 
We also observe that the VUS$_{opt}^{mem}$ execution time grows slower than $VUS_{opt}$ when $\alpha$ increases. 
This is explained by the use of 2-dimensional arrays for the storage of predictions, which use contiguous memory locations that allow for faster access, decreasing the dependency on $\alpha$.

\vspace{0.2cm}
\noindent {\bf [Influence of $\mu(\ell_a)$]}:
As shown in Figure~\ref{fig:sythetic_exp_time}(b), the execution time variation of VUS, VUS$_{opt}$, and VUS$_{opt}^{mem}$ caused by $\ell_a$ is rather insignificant. 
We also observe that the VUS$_{opt}$ and VUS$_{opt}^{mem}$ execution times are significantly lower when compared to VUS. 
This is explained by the smaller dependency of the complexity of these algorithms on the time series length $|T|$. 
Overall, the execution time for both VUS$_{opt}$ and VUS$_{opt}^{mem}$ is significantly lower than VUS, and follows a similar trend. 

\vspace{0.2cm}
\noindent {\bf [Influence of $\sigma(\ell_a)$]}: 
As depicted in Figure~\ref{fig:sythetic_exp_time}(d) and inferred from the theoretical complexities in Table~\ref{tab:complexity_summary}, none of the measures are affected by the standard deviation of the anomaly lengths.

\vspace{0.2cm}
\noindent {\bf [Influence of $|T|$]}:
For short time series (small values of $|T|$), we note that O($T_1$) becomes comparable to O($T_2$). 
Thus, the theoretical complexities approximate to $O(NL(T_1+T_2))$, $O(N*(T_1+T_2))+O(NLT_2)$ and $O(N(T_1+T_2))$ for VUS, VUS$_{opt}$, and VUS$_{opt}^{mem}$, respectively. 
Indeed, we observe in Figure~\ref{fig:sythetic_exp_time}(c) that the execution times of VUS, VUS$_{opt}$, and VUS$_{opt}^{mem}$ are similar for small values of $|T|$. However, for larger values of $|T|$, $O(T_1)$ is much higher compared to $O(T_2)$, thus resulting in an effective complexity of $O(NLT_1)$ for VUS, and $O(NT_1)$ for VUS$_{opt}$, and VUS$_{opt}^{mem}$. 
This translates to a significant improvement in execution time complexity for VUS$_{opt}$ and VUS$_{opt}^{mem}$ compared to VUS, which is confirmed by the results in Figure~\ref{fig:sythetic_exp_time}(c).

\vspace{0.2cm}
\noindent {\bf [Influence of $N$]}: 
Given the theoretical complexity depicted in Table~\ref{tab:complexity_summary}, it is evident that the number of thresholds affects all measures in a linear fashion.
Figure~\ref{fig:sythetic_exp_time}(e) demonstrates this point: the results of varying $N$ show a linear dependency for VUS, VUS$_{opt}$, and VUS$_{opt}^{mem}$ (i.e., a logarithmic trend with a log scale on the y axis). \commentRed{Moreover, we observe that the AUC and range-AUC execution time is almost constant regardless of the number of thresholds used. The latter is explained by the very efficient implementation of AUC measures. Therefore, the linear dependency on the number of thresholds is not visible in Figure~\ref{fig:sythetic_exp_time}(e).}

\vspace{0.2cm}
\noindent {\bf [Influence of $L$]}: Figure~\ref{fig:sythetic_exp_time}(f) depicts the influence of the maximum buffer length $L$ on the execution time of all measures. 
We observe that, as $L$ grows, the execution time of VUS$_{opt}$ and VUS$_{opt}^{mem}$ increases slower than VUS. 
We also observe that VUS$_{opt}^{mem}$ is more scalable with $L$ when compared to VUS$_{opt}$. 
This is consistent with the theoretical complexity (cf. Table~\ref{tab:complexity_summary}), which indicates that the dependence on $L$ decreases from $O(NL(T_1+T_2+\ell_a \alpha))$ for VUS to $O(NL(T_2+\ell_a \alpha)$ and $O(NL(\ell_a \alpha))$ for $VUS_{opt}$, and $VUS_{opt}^{mem}$.





\begin{figure*}[tb]
  \centering
  \includegraphics[width=\linewidth]{figures/synthetic_res.pdf}
  %\vspace*{-0.5cm}
  \caption{Execution time of VUS, R-AUC, AUC-based measures when we vary the parameters listed in Table~\ref{tab:parameter_range_time}. The solid lines correspond to the average execution time over 10 runs. The colored envelopes are to the standard deviation.}
  \label{fig:sythetic_exp_time}
\end{figure*}


\vspace{0.2cm}
In order to obtain a more accurate picture of the influence of each of the above parameters, we fit the execution time (as affected by the parameter values) using linear regression; we can then use the regression slope coefficient of each parameter to evaluate the influence of that parameter. 
In practice, we fit each parameter individually, and report the regression slope coefficient, as well as the coefficient of determination $R^2$.
Table~\ref{tab:parameter_linear_coeff} reports the coefficients mentioned above for each parameter associated with VUS, VUS$_{opt}$, and VUS$_{opt}^{mem}$.



\begin{table}[tb]
    \centering
    \caption{Linear regression slope coefficients ($C.$) for VUS execution times, for each parameter independently. }
    \begin{tabular}{|c|c|c|c|c|c|c|} 
 \hline
 Measure & Param. & $\alpha$ & $l_a$ & $|T|$ & $L$ & $N$\\ [0.5ex] 
 \hline\hline
 \multirow{2}{*}{$VUS$} & $C.$ & 21.9 & 0.02 & 2.13 & 212 & 6.24\\\cline{2-7}
 & {$R^2$} & 0.99 & 0.15 & 0.99 & 0.99 & 0.99 \\   
 \hline
  \multirow{2}{*}{$VUS_{opt}$} & $C.$ & 24.2  & 0.06 & 0.19 & 27.8 & 1.23\\\cline{2-7}
  & $R^2$& 0.99 & 0.86 & 0.99 & 0.99 & 0.99\\ 
 \hline
 \multirow{2}{*}{$VUS_{opt}^{mem}$} & $C.$ & 21.5 & 0.05 & 0.21 & 15.7 & 1.16\\\cline{2-7}
  & $R^2$ & 0.99 & 0.89 & 0.99 & 0.99 & 0.99\\[1ex] 
 \hline
\end{tabular}
    \label{tab:parameter_linear_coeff}
\end{table}

Table~\ref{tab:parameter_linear_coeff} shows that the linear regression between $\alpha$ and the execution time has a $R^2=0.99$. Thus, the dependence of execution time on $\alpha$ is linear. We also observe that VUS$_{opt}$ execution time is more dependent on $\alpha$ than VUS and VUS$_{opt}^{mem}$ execution time.
Moreover, the dependence of the execution time on the time series length ($|T|$) is higher for VUS than for VUS$_{opt}$ and VUS$_{opt}^{mem}$. 
More importantly, VUS$_{opt}$ and VUS$_{opt}^{mem}$ are significantly less dependent than VUS on the number of thresholds and the maximal buffer length. 







\subsubsection{Evaluation on TSB-UAD Time Series}\hfill\\
\label{sec:TSB_eval_time}

In this section, we verify the conclusions outlined in the previous section with real-world time series from the TSB-UAD benchmark. 
In this setting, the parameters $\alpha$, $\ell_a$, and $|T|$ are calculated from the series in the benchmark and cannot be changed. Moreover, $L$ and $N$ are parameters for the computation of VUS, regardless of the time series (synthetic or real). Thus, we do not consider these two parameters in this section.

\begin{figure*}[tb]
  \centering
  \includegraphics[width=\linewidth]{figures/TSB2.pdf}
  \caption{Execution time of VUS, R-AUC, AUC-based measures on the TSB-UAD benchmark, versus $\alpha$, $\ell_a$, and $|T|$.}
  \label{fig:TSB}
\end{figure*}

Figure~\ref{fig:TSB} depicts the execution time of AUC, R-AUC, and VUS-based measures versus $\alpha$, $\mu(\ell_a)$, and $|T|$.
We first confirm with Figure~\ref{fig:TSB}(a) the linear relationship between $\alpha$ and the execution time for VUS, VUS$_{opt}$ and VUS$_{opt}^{mem}$.
On further inspection, it is possible to see two separate lines for almost all the measures. 
These lines can be attributed to the time series length $|T|$. 
The convergence of VUS and $VUS_{opt}$ when $\alpha$ grows shows the stronger dependence that $VUS_{opt}$ execution time has on $\alpha$, as already observed with the synthetic data (cf. Section~\ref{sec:synthetic_eval_time}). 

In Figure~\ref{fig:TSB}(b), we observe that the variation of the execution time with $\ell_a$ is limited when compared to the two other parameters. We conclude that the variation of $\ell_a$ is not a key factor in determining the execution time of the measures.
Furthermore, as depicted in Figure~\ref{fig:TSB}(c), $VUS_{opt}$ and $VUS_{opt}^{mem}$ are more scalable than VUS when $|T|$ increases. 
We also confirm the linear dependence of execution time on the time series length for all the accuracy measures, which is consistent with the experiments on the synthetic data. 
The two abrupt jumps visible in Figure~\ref{fig:TSB}(c) are explained by significant increases of $\alpha$ in time series of the same length. 

\begin{table}[tb]
\centering
\caption{Linear regression slope coefficients ($C.$) for VUS execution time, for all time series parameters all-together.}
\begin{tabular}{|c|ccc|c|} 
 \hline
Measure & $\alpha$ & $|T|$ & $l_a$ & $R^2$ \\ [0.5ex] 
 \hline\hline
 \multirow{1}{*}{${VUS}$} & 7.87 & 13.5 & -0.08 & 0.99  \\ 
 %\cline{2-5} & $R^2$ & \multicolumn{3}{c|}{ 0.99}\\
 \hline
 \multirow{1}{*}{$VUS_{opt}$} & 10.2 & 1.70 & 0.09 & 0.96 \\
 %\cline{2-5} & $R^2$ & \multicolumn{3}{c|}{0.96}\\
\hline
 \multirow{1}{*}{$VUS_{opt}^{mem}$} & 9.27 & 1.60 & 0.11 & 0.96 \\
 %\cline{2-5} & $R^2$ & \multicolumn{3}{c|}{0.96} \\
 \hline
\end{tabular}
\label{tab:parameter_linear_coeff_TSB}
\end{table}



We now perform a linear regression between the execution time of VUS, VUS$_{opt}$ and VUS$_{opt}^{mem}$, and $\alpha$, $\ell_a$ and $|T|$.
We report in Table~\ref{tab:parameter_linear_coeff_TSB} the slope coefficient for each parameter, as well as the $R^2$.  
The latter shows that the VUS$_{opt}$ and VUS$_{opt}^{mem}$ execution times are impacted by $\alpha$ at a larger degree than $\alpha$ affects VUS. 
On the other hand, the VUS$_{opt}$ and VUS$_{opt}^{mem}$ execution times are impacted to a significantly smaller degree by the time series length when compared to VUS. 
We also confirm that the anomaly length does not impact the execution time of VUS, VUS$_{opt}$, or VUS$_{opt}^{mem}$.
Finally, our experiments show that our optimized implementations VUS$_{opt}$ and VUS$_{opt}^{mem}$ significantly speedup the execution of the VUS measures (i.e., they can be computed within the same order of magnitude as R-AUC), rendering them practical in the real world.











\subsection{Summary of Results}


Figure~\ref{fig:overalltable} depicts the ranking of the accuracy measures for the different tests performed in this paper. The robustness test is divided into three sub-categories (i.e., lag, noise, and Normal vs. abnormal ratio). We also show the overall average ranking of all accuracy measures (most right column of Figure~\ref{fig:overalltable}).
Overall, we see that VUS-ROC is always the best, and VUS-PR and Range-AUC-based measures are, on average, second, third, and fourth. We thus conclude that VUS-ROC is the overall winner of our experimental analysis.

\commentRed{In addition, our experimental evaluation shows that the optimized version of VUS accelerates the computation by a factor of two. Nevertheless, VUS execution time is still significantly slower than AUC-based approaches. However, it is important to mention that the efficiency of accuracy measures is an orthogonal problem with anomaly detection. In real-time applications, we do not have ground truth labels, and we do not use any of those measures to evaluate accuracy. Measuring accuracy is an offline step to help the community assess methods and improve wrong practices. Thus, execution time should not be the main criterion for selecting an evaluation measure.}

We present RiskHarvester, a risk-based tool to compute a security risk score based on the value of the asset and ease of attack on a database. We calculated the value of asset by identifying the sensitive data categories present in a database from the database keywords. We utilized data flow analysis, SQL, and Object Relational Mapper (ORM) parsing to identify the database keywords. To calculate the ease of attack, we utilized passive network analysis to retrieve the database host information. To evaluate RiskHarvester, we curated RiskBench, a benchmark of 1,791 database secret-asset pairs with sensitive data categories and host information manually retrieved from 188 GitHub repositories. RiskHarvester demonstrates precision of (95\%) and recall (90\%) in detecting database keywords for the value of asset and precision of (96\%) and recall (94\%) in detecting valid hosts for ease of attack. Finally, we conducted an online survey to understand whether developers prioritize secret removal based on security risk score. We found that 86\% of the developers prioritized the secrets for removal with descending security risk scores.

\newpage
\section{Limitations}
Our experiments and analyses have been primarily conducted on LLaVA, LLaVA-Next, and Qwen2-VL. While these multimodal large language models are highly representative, our exploration should be extended to a broader range of model architectures. Such an expansion would enable us to uncover more intriguing findings and gain more robust and comprehensive insights. Additionally, we should apply our analytical framework and experimental evaluations to models of varying sizes, ensuring that our conclusions are not only diverse but also applicable across different scales of architecture.



% \section*{Acknowledgments}



% Bibliography entries for the entire Anthology, followed by custom entries
%\bibliography{anthology,custom}
% Custom bibliography entries only
\bibliography{custom}

\clearpage
\appendix
\section*{Appendix}
\section{Synthetic Data Engine}
\begin{table}[t!]
    \centering
    \begin{tabular}{l l}
    \toprule
    Visual premises & Non-visual premises \\
    \midrule
    \tabitem Perpendicularity & \tabitem Middle point \\
    \tabitem Collinearity & \tabitem Congruency in degree \\
    \tabitem Concyclicity & \tabitem Congruency in length \\
    \tabitem Parallelity & \tabitem Congruency in ratio \\
    \tabitem Angle measure & \tabitem Triangle similarity \\
    \tabitem Length measure & \tabitem Triangle congruency \\
    & \tabitem Circumcenter \\
    & \tabitem Foot \\
    \bottomrule
    \end{tabular}
    \caption{Geometric premises used in AlphaGeometry. \emph{Visual premises} denotes the geometric premises which can be directly perceived from the diagram. \emph{Non-visual premises} requires reasoning to be recognized.}
    \label{tab:alphageometry}
\end{table}

\begin{table}[t!]
    \centering
    \begin{tabular}{l l}
    \toprule
    Visual premises & Non-visual premises \\
    \midrule
    \tabitem Perpendicularity & \tabitem Middle point \\
    \tabitem Collinearity & \tabitem Congruency in degree \\
    \tabitem Concyclicity & \tabitem Congruency in length \\
    \tabitem Parallelity & \tabitem Congruency in ratio \\
    \tabitem Angle measure & \tabitem Triangle similarity \\
    \tabitem Length measure & \tabitem Triangle congruency \\
    & \tabitem Circumcenter \\
    & \tabitem Foot \\
    \bottomrule
    \end{tabular}
    \caption{Geometric premises used in AlphaGeometry. \emph{Visual premises} denotes the geometric premises which can be directly perceived from the diagram. \emph{Non-visual premises} requires reasoning to be recognized.}
    \label{tab:alphageometry}
\end{table}

In this section, we provide the details of our synthetic data engine. Based on AlphaGeometry~\citep{alphageometry}, we generate synthetic diagram and caption pairs by randomly sampling a AlphaGeometry program with \cref{alg:sampling}.

\begin{algorithm}[t!]
\caption{Sampling process of the synthetic data engine}
 \textbf{Input} Geometric relations $R$, geometric objects $O$, number of clauses $n_c$ \\
 \textbf{Output} AlphaGeometry program $c$
\begin{algorithmic}[1]
\State Initialize points and clauses with the sampled object: $P, C \sim O$ 
\For{$i \gets 1$ to $n_c$}
    \State Generate points: $P_{\text{new}}$
    \State Sample relation and points: $r, P_{\text{old}} \sim R, P$
    \State Construct clause: $C_{\text{new}} = r(P_\text{new}, P_\text{old})$
    \State Update points and clauses: $P, C \gets P \cup P_{\text{new}}, C \cup C_{\text{new}}$
\EndFor
\State Generate program with points and clauses: $c \gets \text{Clauses2Program}(P, C)$
\State \textbf{return} $c$
\end{algorithmic}
\label{alg:sampling}
\end{algorithm}

Examples for randomly sampled AlphaGeometry problems and their corresponding diagrams and lists of geometric premises are described in \cref{fig:alphageometry}.
The types of geometric premises that appear in our synthetic data engine are listed in \cref{tab:alphageometry}.


\section{Details of Benchmark}

\subsection{Training details}

\label{sec:hparams}

% \paragraph{Linear probing.}
To evaluate the visual feature perception of the vision encoder, we utilize a linear probing approach, which involves freezing the vision
encoder parameters and training a simple linear classifier on top of its features.

We train the linear classifier on the training set of each task for 50 epochs with batch size 128 and learning rate $1\text{e-}4$.
We use Adam optimizer for optimization.

\subsection{Visualization of the vision encoders}

We visualize the embeddings of the vision encoders used in \cref{sec:benchmakr_results} at \cref{fig:tsne_benchmark}.

\begin{figure}[t!]
    \centering
    \begin{subfigure}[t]{.32\linewidth}
        \centering
        \includegraphics[width=\linewidth]{latex/figures/images/visual_tsne_openclip.png}
        \caption{OpenCLIP}
    \end{subfigure}
    \begin{subfigure}[t]{.32\linewidth}
        \centering
        \includegraphics[width=\linewidth]{latex/figures/images/visual_tsne_siglip.png}
        \caption{SigLIP}
    \end{subfigure}
    \begin{subfigure}[t]{.32\linewidth}
        \centering
        \includegraphics[width=\linewidth]{latex/figures/images/visual_tsne_convnext.png}
        \caption{ConvNeXT}
    \end{subfigure}
    \begin{subfigure}[t]{.32\linewidth}
        \centering
        \includegraphics[width=\linewidth]{latex/figures/images/visual_tsne_dinov2.png}
        \caption{DinoV2}
    \end{subfigure}
    \begin{subfigure}[t]{.32\linewidth}
        \centering
        \includegraphics[width=\linewidth]{latex/figures/images/visual_tsne_ours.png}
        \caption{GeoCLIP}
    \end{subfigure}
    \caption{
    The embeddings of the vision encoders on the diagrams of TwoLines task. We visualize the embeddings of the vision encoders on the diagrams of TwoLines task. The blue, orange, and green dots are the diagrams where the two lines AB and BC are collinear, perpendicular, and otherwise, respectively.
    \label{fig:tsne_benchmark}
    }
\end{figure}


% \paragraph{GeoCLIP.}
% We start from OpenCLIP~\citep{clip}, a pre-trained model where the architecture is ViT-L/14 with image resolution $336\times 336$. To train OpenCLIP with GeoCLIP, we use total of 200,000 diagram-caption pairs generated with our synthetic data engine.
% We set the batch size and weight decay to 256 and 0.2, respectively.
% We optimize for 50 epochs using Adam optimizer~\citep{adam} and a cosine annealing scheduler with 2,000 warmup steps and the maximum learning rate is set to be $1\text{e-}4$.
% For the domain adaptation parts, i.e., applying CLIP on the diagram-caption pairs and the diagram pairs of target domains, we vary the batch size to 32.

\section{Visual Geometric Premises Recognition Benchmark for Vision Encoders}
\label{sec:visual_feature}

In this section, we first develop a benchmark for evaluating a vision encoder's performance in recognizing geometric features from a diagram. We then report the performance of well-known vision encoders on this benchmark.

%\dw{We first examine how commonly used vision encoders in open-source VLMs recognize geometric primitives, such as points and lines, and geometric premises, such as perpendicularity, from a given diagram.} Because a VLM’s performance is typically measured by its final solution in PGPS, the result does not fully capture the encoder’s capacity to identify fundamental geometric structures. 
% Yet, as the encoder is the first component to observe the diagram, understanding its recognition ability is essential for further advancement.
% To address this gap, we introduce a benchmark that thoroughly evaluates vision encoders.

\subsection{Benchmark preparation}
\label{sec:synthetic_data_engine}

We design our benchmark as simple classification tasks. By investigating PGPS datasets, we identify that recognizing \emph{geometric primitives}, such as points and lines, and geometric properties representing \emph{relations between primitives}, such as perpendicularity, is important for solving plane geometry problems. Recognized information forms \emph{geometric premises} to solve the problem successfully. To this end, we carefully curate five classification tasks as follows:
\begin{itemize}
    \item \textbf{Concyclic}: A circle and four points are given. The task is to identify how many of those points lie on the circle.
    \item \textbf{TwoLines}: Two lines, AB and BC, are given alongside other geometric objects. The task is to determine whether AB and BC are perpendicular, collinear, or neither.
    \item \textbf{ObjectShape}: A given diagram includes one of the following geometric objects: a segment, triangle, square, or pentagon. The task is to classify which object is present.
    \item \textbf{SquareShape}: A diagram including a square ABCD and other geometric objects is given. The task is to classify whether the square is a trapezoid, parallelogram, or rectangle.
    \item \textbf{AngleDetection}: A diagram is given with at least three points: A, B, and C. The task is to classify the correct angle of ABC from \(\{15^\circ, 20^\circ, \ldots, 75^\circ\}\). 
\end{itemize}
An example of each task is provided in \cref{fig:benchmark}.

\begin{table}[t]
    \centering
    \caption{The performance of different pre-trained models on ImageNet and infrared semantic segmentation datasets. The \textit{Scratch} means the performance of randomly initialized models. The \textit{PT Epochs} denotes the pre-training epochs while the \textit{IN1K FT epochs} represents the fine-tuning epochs on ImageNet \citep{imagenet}. $^\dag$ denotes models reproduced using official codes. $^\star$ refers to the effective epochs used in \citet{iBOT}. The top two results are marked in \textbf{bold} and \underline{underlined} format. Supervised and CL methods, MIM methods, and UNIP models are colored in \colorbox{orange!15}{\rule[-0.2ex]{0pt}{1.5ex}orange}, \colorbox{gray!15}{\rule[-0.2ex]{0pt}{1.5ex}gray}, and \colorbox{cyan!15}{\rule[-0.2ex]{0pt}{1.5ex}cyan}, respectively.}
    \label{tab:benchmark}
    \centering
    \scriptsize
    \setlength{\tabcolsep}{1.0mm}{
    \scalebox{1.0}{
    \begin{tabular}{l c c c c  c c c c c c c c}
        \toprule
         \multirow{2}{*}{Methods} & \multirow{2}{*}{\makecell[c]{PT \\ Epochs}} & \multicolumn{2}{c}{IN1K FT} & \multicolumn{4}{c}{Fine-tuning (FT)} & \multicolumn{4}{c}{Linear Probing (LP)} \\
         \cmidrule{3-4} \cmidrule(lr){5-8} \cmidrule(lr){9-12} 
         & & Epochs & Acc & SODA & MFNet-T & SCUT-Seg & Mean & SODA & MFNet-T & SCUT-Seg & Mean \\
         \midrule
         \textcolor{gray}{ViT-Tiny/16} & & &  & & & & & & & & \\
         Scratch & - & - & - & 31.34 & 19.50 & 41.09 & 30.64 & - & - & - & - \\
         \rowcolor{gray!15} MAE$^\dag$ \citep{mae} & 800 & 200 & \underline{71.8} & 52.85 & 35.93 & 51.31 & 46.70 & 23.75 & 15.79 & 27.18 & 22.24 \\
         \rowcolor{orange!15} DeiT \citep{deit} & 300 & - & \textbf{72.2} & 63.14 & 44.60 & 61.36 & 56.37 & 42.29 & 21.78 & 31.96 & 32.01 \\
         \rowcolor{cyan!15} UNIP (MAE-L) & 100 & - & - & \underline{64.83} & \textbf{48.77} & \underline{67.22} & \underline{60.27} & \underline{44.12} & \underline{28.26} & \underline{35.09} & \underline{35.82} \\
         \rowcolor{cyan!15} UNIP (iBOT-L) & 100 & - & - & \textbf{65.54} & \underline{48.45} & \textbf{67.73} & \textbf{60.57} & \textbf{52.95} & \textbf{30.10} & \textbf{40.12} & \textbf{41.06}  \\
         \midrule
         \textcolor{gray}{ViT-Small/16} & & & & & & & & & & & \\
         Scratch & - & - & - & 41.70 & 22.49 & 46.28 & 36.82 & - & - & - & - \\
         \rowcolor{gray!15} MAE$^\dag$ \citep{mae} & 800 & 200 & 80.0 & 63.36 & 42.44 & 60.38 & 55.39 & 38.17 & 21.14 & 34.15 & 31.15 \\
         \rowcolor{gray!15} CrossMAE \citep{crossmae} & 800 & 200 & 80.5 & 63.95 & 43.99 & 63.53 & 57.16 & 39.40 & 23.87 & 34.01 & 32.43 \\
         \rowcolor{orange!15} DeiT \citep{deit} & 300 & - & 79.9 & 68.08 & 45.91 & 66.17 & 60.05 & 44.88 & 28.53 & 38.92 & 37.44 \\
         \rowcolor{orange!15} DeiT III \citep{deit3} & 800 & - & 81.4 & 69.35 & 47.73 & 67.32 & 61.47 & 54.17 & 32.01 & 43.54 & 43.24 \\
         \rowcolor{orange!15} DINO \citep{dino} & 3200$^\star$ & 200 & \underline{82.0} & 68.56 & 47.98 & 68.74 & 61.76 & 56.02 & 32.94 & 45.94 & 44.97 \\
         \rowcolor{orange!15} iBOT \citep{iBOT} & 3200$^\star$ & 200 & \textbf{82.3} & 69.33 & 47.15 & 69.80 & 62.09 & 57.10 & 33.87 & 45.82 & 45.60 \\
         \rowcolor{cyan!15} UNIP (DINO-B) & 100 & - & - & 69.35 & 49.95 & 69.70 & 63.00 & \underline{57.76} & \underline{34.15} & \underline{46.37} & \underline{46.09} \\
         \rowcolor{cyan!15} UNIP (MAE-L) & 100 & - & - & \textbf{70.99} & \underline{51.32} & \underline{70.79} & \underline{64.37} & 55.25 & 33.49 & 43.37 & 44.04 \\
         \rowcolor{cyan!15} UNIP (iBOT-L) & 100 & - & - & \underline{70.75} & \textbf{51.81} & \textbf{71.55} & \textbf{64.70} & \textbf{60.28} & \textbf{37.16} & \textbf{47.68} & \textbf{48.37} \\ 
        \midrule
        \textcolor{gray}{ViT-Base/16} & & & & & & & & & & & \\
        Scratch & - & - & - & 44.25 & 23.72 & 49.44 & 39.14 & - & - & - & - \\
        \rowcolor{gray!15} MAE \citep{mae} & 1600 & 100 & 83.6 & 68.18 & 46.78 & 67.86 & 60.94 & 43.01 & 23.42 & 37.48 & 34.64 \\
        \rowcolor{gray!15} CrossMAE \citep{crossmae} & 800 & 100 & 83.7 & 68.29 & 47.85 & 68.39 & 61.51 & 43.35 & 26.03 & 38.36 & 35.91 \\
        \rowcolor{orange!15} DeiT \citep{deit} & 300 & - & 81.8 & 69.73 & 48.59 & 69.35 & 62.56 & 57.40 & 34.82 & 46.44 & 46.22 \\
        \rowcolor{orange!15} DeiT III \citep{deit3} & 800 & 20 & \underline{83.8} & 71.09 & 49.62 & 70.19 & 63.63 & 59.01 & \underline{35.34} & 48.01 & 47.45 \\
        \rowcolor{orange!15} DINO \citep{dino} & 1600$^\star$ & 100 & 83.6 & 69.79 & 48.54 & 69.82 & 62.72 & 59.33 & 34.86 & 47.23 & 47.14 \\
        \rowcolor{orange!15} iBOT \citep{iBOT} & 1600$^\star$ & 100 & \textbf{84.0} & 71.15 & 48.98 & 71.26 & 63.80 & \underline{60.05} & 34.34 & \underline{49.12} & \underline{47.84} \\
        \rowcolor{cyan!15} UNIP (MAE-L) & 100 & - & - & \underline{71.47} & \textbf{52.55} & \underline{71.82} & \textbf{65.28} & 58.82 & 34.75 & 48.74 & 47.43 \\
        \rowcolor{cyan!15} UNIP (iBOT-L) & 100 & - & - & \textbf{71.75} & \underline{51.46} & \textbf{72.00} & \underline{65.07} & \textbf{63.14} & \textbf{39.08} & \textbf{52.53} & \textbf{51.58} \\
        \midrule
        \textcolor{gray}{ViT-Large/16} & & & & & & & & & & & \\
        Scratch & - & - & - & 44.70 & 23.68 & 49.55 & 39.31 & - & - & - & - \\
        \rowcolor{gray!15} MAE \citep{mae} & 1600 & 50 & \textbf{85.9} & 71.04 & \underline{51.17} & 70.83 & 64.35 & 52.20 & 31.21 & 43.71 & 42.37 \\
        \rowcolor{gray!15} CrossMAE \citep{crossmae} & 800 & 50 & 85.4 & 70.48 & 50.97 & 70.24 & 63.90 & 53.29 & 33.09 & 45.01 & 43.80 \\
        \rowcolor{orange!15} DeiT3 \citep{deit3} & 800 & 20 & \underline{84.9} & \underline{71.67} & 50.78 & \textbf{71.54} & \underline{64.66} & \underline{59.42} & \textbf{37.57} & \textbf{50.27} & \underline{49.09} \\
        \rowcolor{orange!15} iBOT \citep{iBOT} & 1000$^\star$ & 50 & 84.8 & \textbf{71.75} & \textbf{51.66} & \underline{71.49} & \textbf{64.97} & \textbf{61.73} & \underline{36.68} & \underline{50.12} & \textbf{49.51} \\
        \bottomrule
    \end{tabular}}}
    \vspace{-2mm}
\end{table}

Our benchmark is built on top of AlphaGeometry~\citep{alphageometry}, which is designed to solve IMO-style plane geometry problems. The program provides useful functions such as formal language describing plane diagrams. The language predefines a set of geometric premises listed in \cref{tab:alphageometry}, including all necessary properties to define our benchmark tasks. In addition, once a diagram description is given in formal language, the program renders a corresponding diagram with varying fonts, colors, widths, orientations, and resolutions, allowing us to have diagrams with diverse styles often observed in a real-world scenario.

We create question-and-answer pairs based on AlphaGeometry. To sample a diverse set of question-and-answers, we first establish a foundational geometric structure corresponding to the key problem of the task and then repeatedly add new points or lines with randomly selected geometric relationships to the existing diagram with the help of the formal language. The pseudo-code for the random question generation is presented in \cref{alg:sampling}. For each task, we generate 50,000, 10,000, and 10,000 question-and-answer pairs for training, validation, and testing, respectively.

\iffalse
We implement a data engine that randomly generates \emph{plane geometry problems}, complete with corresponding \emph{diagrams} and \emph{\geofeat{}s}, including the relationship between lines and measurements such as angles. Our engine is built on top of AlphaGeometry~\citep{alphageometry}, a proof assistant for IMO-style plane geometry problems. AlphaGeometry provides a formal language for representing plane geometry problems. 
\sh{
Given an AlphaGeometry problem, we can derive the \geofeat{}s and characteristics of the given problem such as perpendicularity and collinearity. For completeness, we provide the list of \geofeat{}s that AlphaGeometry can support in \cref{tab:alphageometry}.
In addition, we can render the diagram that represents the AlphaGeometry problem with different fonts, colors, widths, orientations, and resolutions, allowing us to have diagrams with diverse styles often observed in a real-world scenario.
}
% as an executable program. 
% AlphaGeometry can derive the solution when a problem statement is given in terms of the formal language. 
% Given an AlphaGeometry problem written in the formal language, we can render the diagram which expresses . AlphaGeometry supports diverse \geofeat{}s and characteristics of plane geometry diagrams, such as perpendicularity and collinearity. For completeness, we provide the list of \geofeat{}s that AlphaGeometry can support in \cref{tab:alphageometry}. %\cref{fig:alphageometry} illustrates the examples of the problems in the formal language with the rendered diagrams. %repeated below
% In addition, the diagram can be rendered with different fonts, colors, widths, orientations, and resolutions, allowing us to have diagrams with diverse styles often observed in a real-world scenario.

%We employ AlphaGeometry’s formal language to curate a synthetic dataset. 
To sample a diverse set of plane geometry problems, we randomly generate AlphaGeometry problems by first establishing a foundational geometric structure, and then repeatedly adding new points with randomly selected geometric relationships to existing points, drawn from the predefined set in AlphaGeometry. The pseudo-code for the sampling algorithm is presented in \cref{alg:sampling}, and randomly selected examples are provided in \cref{fig:alphageometry}. Note that samples consist of the AlphaGeometry problem and \geofeat{}s written in the text along with the corresponding diagram.

%\subsection{Benchmark for visual \geofeat{} recognition}
Based on the synthetic data engine, we develop a benchmark that evaluates how well the vision encoders recognize visual \geofeat{}s in plane geometry diagrams. In PGPS, recognizing characteristics such as collinearity, perpendicularity, and angle measures is crucial for solving geometry problems. Based on the elements, we define five diagram classification tasks that depend on accurate visual \geofeat{} perception:
\begin{itemize}
    \item \textbf{ObjectShape}: Each diagram includes one of the following geometric objects: a segment, triangle, square, or pentagon. The goal is to classify which object is present.
    \item \textbf{Concyclic}: Each diagram contains a circle and four points. The task is to identify how many of those points lie on the circle.  
    \item \textbf{TwoLines}: Each diagram includes two lines, AB and BC, alongside other geometric objects. The vision encoder must determine whether AB and BC are perpendicular, collinear, or neither.
    \item \textbf{SquareShape}: Each diagram features a square ABCD with additional objects. The goal is to classify whether the square is actually a trapezoid, parallelogram, or rectangle. 
    \item \textbf{AngleDetection}: Each diagram includes an angle ABC and other objects. The objective is to identify the measure of angle ABC from among the set \(\{15^\circ, 20^\circ, \ldots, 75^\circ\}\). 
\end{itemize}
We can generate samples used for benchmarking by adding additional conditions to the random problem generator. For example, we let the generator synthesize AlphaGeometry problems with a segment, triangle, square, or pentagon for the ObjectShape task.
For each task, we generate 50,000, 10,000, and 10,000 AlphaGeometry problems for training, validation, and test, respectively.
We then render the diagrams and corresponding labels from the generated problems, which is later used to train the vision encoders.
\cref{fig:benchmark} shows randomly generated diagrams for the classification tasks.
\fi


\subsection{Results}
\label{sec:benchmakr_results}
\begin{table}[t!]
    \centering
    \resizebox{\linewidth}{!}{
    \begin{tabular}{l l c c c c c }
        \toprule
        & Models & \begin{tabular}{@{}c@{}}Object \\ Shape\end{tabular} & \begin{tabular}{@{}c@{}}Con \\ cyclic\end{tabular} & \begin{tabular}{@{}c@{}}Two \\ Lines\end{tabular} & \begin{tabular}{@{}c@{}}Square \\ Shape\end{tabular} & \begin{tabular}{@{}c@{}}Angle \\ Detection\end{tabular} \\
        % &Models & ObjectShape & Concyclic & TwoLines & SquareShape & AngleDetection \\
        \midrule
        \multirow{4}{*}{%
    \rotatebox[origin=c]{90}{%
        \parbox{1.3cm}{\centering \footnotesize \emph{Baseline}}%
    }%
}  
        &OpenCLIP & \textbf{100.00} & 99.13 & 86.57 & 85.20 & 64.81 \\
        &SigLIP & \textbf{100.00} & \textbf{99.71} & 89.26 & 89.31 & 76.86 \\
        &DinoV2 & \textbf{100.00} & 98.01 & 85.30 & 91.24 & 22.43 \\
        &ConvNeXT & \textbf{100.00} & 99.20 & 89.39 & 88.13 & 61.84 \\
        \midrule
        \multirow{3}{*}{%
    \rotatebox[origin=c]{90}{%
        \parbox{1.3cm}{\centering \footnotesize \emph{SSL}}%
    }%
}  
        &Jigsaw & 86.11 & 63.85 & 49.98 & 61.88 & 11.44 \\
        &MAE & 93.99 & 72.25 & 71.73 & 82.70 & 13.08 \\
        &VQ-VAE & 63.05 & 60.97 & 48.10 & 57.35 & 9.22 \\
        \midrule
        \multirow{3}{*}{%
    \rotatebox[origin=c]{90}{%
        \parbox{1.3cm}{\centering \footnotesize \emph{GeoCLIP}}%
    }%
}  
        &GeoCLIP (F $\times$) & 99.52 & 98.61 & 88.33 & 86.76 & 65.68 \\
        &GeoCLIP (2K) & 99.32 & 98.73 & 94.73 & 89.22 & 74.95 \\
        &GeoCLIP & 99.21 & 99.24 & \textbf{96.05} & \textbf{95.95} & \textbf{78.56} \\
        \bottomrule
    \end{tabular}
    }
    \caption{Results on the proposed visual feature benchmark. We report the test accuracy of the models with the best validation performance. }
    \label{tab:linear_probing}
\end{table}

% \begin{table}[t!]
%     \centering
%     \resizebox{\linewidth}{!}{
%     \begin{tabular}{l l c c c c c }
%         \toprule
%         & Models & \begin{tabular}{@{}c@{}}Object \\ Shape\end{tabular} & \begin{tabular}{@{}c@{}}Con \\ cyclic\end{tabular} & \begin{tabular}{@{}c@{}}Two \\ Lines\end{tabular} & \begin{tabular}{@{}c@{}}Square \\ Shape\end{tabular} & \begin{tabular}{@{}c@{}}Angle \\ Detection\end{tabular} \\
%         % &Models & ObjectShape & Concyclic & TwoLines & SquareShape & AngleDetection \\
%         \midrule
%         \multirow{4}{*}{Baseline}
%         &OpenCLIP & \textbf{100.00} & 99.13 & 86.57 & 85.20 & 64.81 \\
%         &SigLIP & \textbf{100.00} & \textbf{99.71} & 89.26 & 89.31 & 76.86 \\
%         &DinoV2 & \textbf{100.00} & 98.01 & 85.30 & 91.24 & 22.43 \\
%         &ConvNeXT & \textbf{100.00} & 99.20 & 89.39 & 88.13 & 61.84 \\
%         \midrule
%         \multirow{3}{*}{SSL}
%         &Jigsaw & 86.11 & 63.85 & 49.98 & 61.88 & 11.44 \\
%         &MAE & 93.99 & 72.25 & 71.73 & 82.70 & 13.08 \\
%         &VQ-VAE & 63.05 & 60.97 & 48.10 & 57.35 & 9.22 \\
%         \midrule
%         \multirow{3}{*}{GeoCLIP}
%         &GeoCLIP (F $\times$) & 99.52 & 98.61 & 88.33 & 86.76 & 65.68 \\
%         &GeoCLIP (2K) & 99.32 & 98.73 & 94.73 & 89.22 & 74.95 \\
%         &GeoCLIP & 99.21 & 99.24 & \textbf{96.05} & \textbf{95.95} & \textbf{78.56} \\
%         \bottomrule
%     \end{tabular}
%     }
%     \caption{Results on the proposed visual feature benchmark. We report the test accuracy of the models with the best validation performance. }
%     \label{tab:linear_probing}
% \end{table}

% \multirow{7}{*}{%
%     \rotatebox[origin=c]{90}{%
%         \parbox{1.3cm}{\centering \emph{Unseen}}%
%     }%
% }  


With the proposed benchmark, we evaluate four widely adopted vision encoders for the open-sourced VLMs: OpenCLIP~\citep{clip}, SigLIP~\citep{siglip}, DinoV2~\citep{dinov2}, and ConvNeXT~\citep{convnext}.

To evaluate the vision encoder, we adopt a linear probing approach. Specifically, we add a linear layer on top of each encoder as a prediction head and train the linear layer from scratch while freezing the parameters of the vision encoder. We use a training set to train the prediction head and report the test accuracy with the best validation performance. The details for the hyper-parameters are described in \cref{sec:hparams}.

As shown in \cref{tab:linear_probing}, many existing vision encoders well recognize the shape of objects but fail to recognize the correct angle between two lines. The encoders also show some difficulties in recognizing the shape of a square and the relationship between two lines. Although the result may seem satisfactory at a glance, these errors will propagate to the downstream tasks when combined with LLMs. Hence, it is important to improve the recognition performance of the vision encoder further. 
%The t-SNE plots of the vision encoder embeddings are illustrated at \cref{fig:tsne_benchmark}. %Let's move this to the next section

\section{\geovlm{}}
\label{sec:vlm_details}

\subsection{Modification of training data}

Our fine-tuning strategy differs slightly from previous works~\citep{unigeo,pgps,geox}. 
Here, we clarify the difference between our approach and previous approaches. In previous works, the VLM is trained to produce the solution program given diagram and problem description as shown in \cref{fig:pgps_examples}. An interesting observation from GeoQA and PGPS9K datasets is that the numerical measurements, such as angles, lengths, and volumes, are not written in the problem description but given as additional conditions, and the numerals are substituted as a variable in the problem description as shown in \cref{fig:geoqa_example}. Therefore, the VLM only needs to produce the solution program without having optical character recognition (OCR) from the diagram. The variables are automatically substituted by the actual numbers when the program is executed. Therefore, the vision encoders do not need to learn OCR from the image.

However, this approach cannot be generalized to a wider class of problems where the numerals are embedded in the diagram instead of written in the problem description. Some variants of MathVerse, such as the vision-dominant problems, fall into this category as well. To incorporate OCR into the solution of the problem, we modify some problem statements in the training set, such that the numerical measurements are only shown in the diagram and not in the statements. We further modify the solution problem so that the solution contains OCR results as a part of the final output. Finally, we unify the language of the solution programs used in GeoQA and PGPS9K by converting GeoQA programs into PGPS9K format. The unification makes the output of VLM consistent since both datasets use different types of formal languages.

\cref{fig:training_data} shows examples of the modified input pairs and solutions, where the first problem statement does not have numerical measurements and the OCR results are in the part of the output solution program.

% \paragraph{Solution program and numerical values prediction.}
% \cref{fig:training_data} reveals the training data for GeoDANO.







\begin{table*}[t]
\centering
\begin{tabular}{lcc}
\toprule
Model & Number of training images & Number of training videos \\
\midrule
V-JEPA & - & ~2M \\
VideoMAE & - & 240K \\
V-WALT & ~970M & ~36M \\
\bottomrule
\end{tabular}
\caption{
\textbf{Training data comparison}}
\label{tab:training data}
\end{table*}






% \subsection{Architecture}
% \label{sec:vlm_arch}

% We begin by summarizing the architecture of our VLM, a combination of a vision encoder and a language model. For the vision encoder, we use \geoclip{}-DA, with a two-layer MLP of GeLU activation as the projection layers following LLaVA-OneVision~\citep{llava-next}. For the language model, we employ LLama-3-8B-Instruct~\citep{llama}.
% For a given diagram and question pair in PGPS, we feed the vision encoder with the given diagram, and then the output of the encoder is used as an input token of LLM through the projection layer. The question text is then fed into the LLM, followed by the diagram embedding.



% \subsection{Fine-tuning strategy.}
% \label{sec:vlm_details}

% \paragraph{Modification process.}
% Our fine-tuning strategy differs slightly from previous works~\citep{unigeo,pgps,geox}. 
% Here, we clarify the difference between our approach and previous approaches. In previous works, the VLM is trained to produce the solution program given diagram and problem description as shown in \cref{fig:pgps_examples}. An interesting observation from GeoQA and PGPS9K datasets is that the numerical measurements, such as angles, lengths, and volumes, are not written in the problem description but given as additional conditions, and the numerals are substituted as a variable in the problem description as shown in \cref{fig:geoqa_example}. Therefore, the VLM only needs to produce the solution program without having optical character recognition (OCR) from the diagram. The variables are automatically substituted by the actual numbers when the program is executed. Therefore, the vision encoders do not need to learn OCR from the image.

% However, this approach cannot be generalized to a wider class of problems where the numerals are embedded in the diagram instead of written in the problem description. Some variants of MathVerse, such as the vision-dominant problems, fall into this category as well. To incorporate OCR into the solution of the problem, we modify some problem statements in the training set, such that the numerical measurements are only shown in the diagram and not in the statements. We further modify the solution problem so that the solution contains OCR results as a part of the final output. Finally, we unify the language of the solution programs used in GeoQA and PGPS9K by converting GeoQA programs into PGPS9K format. The unification makes the output of VLM consistent since both datasets use different types of formal languages.

% \cref{fig:training_data} shows examples of the modified input pairs and solutions, where the first problem statement does not have numerical measurements and the OCR results are in the part of the output solution program.

\subsection{Training details}
We begin by summarizing the architecture of our VLM, a combination of a vision encoder and a language model. For the vision encoder, we use \geoclip{}-DA, with a two-layer MLP of GeLU activation as the projection layers following LLaVA-OneVision~\citep{llava-next}. For the language model, we employ LLama-3-8B-Instruct~\citep{llama}.
For a given diagram and question pair in PGPS, we feed the vision encoder with the given diagram, and then the output of the encoder is used as an input token of LLM through the projection layer. The question text is then fed into the LLM, followed by the diagram embedding.

With the modified training data, we apply supervised fine-tuning on the VLM, i.e., the gradient only flows through the prediction of numerical values and solution steps, not the diagram and text.

We train the VLM with AdamW optimizer~\citep{adamw} and cosine annealing scheduler with warmp up ratio 0.03 and maximum learning rate $1\text{e-}5$.
We use LoRA~\citep{lora} with rank 128.
We set the batch size to 16 and train with 5 epochs.
We train the VLM with four A100-80GB GPUs for approximately 24 hours.
% \paragraph{Hyper-parameters}

% We train the VLM with AdamW optimizer~\citep{adamw} and cosine annealing scheduler with warmp up ratio 0.03 and maximum learning rate $1\text{e-}5$.
% We use LoRA~\citep{lora} with rank 128.
% We set the batch size to 16 and train with 5 epochs.

% \subsection{Prediction results}

% \begin{figure*}[t]
\centering
\includegraphics[width=0.49\linewidth]{figures/tsne-small.png}
\includegraphics[width=0.49\linewidth]{figures/tsne-large.png}
\caption{LW/non-LW pairs visualized across all models of the Llama family. Larger models tend to use their representation space more efficiently, leading to greater subword token representation similarity.}
\label{fig:tsne}
\end{figure*}


% We provide quantitative analysis of GeoDANO on MathVerse.

\end{document}
