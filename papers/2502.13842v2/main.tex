% This must be in the first 5 lines to tell arXiv to use pdfLaTeX, which is strongly recommended.
\pdfoutput=1
% In particular, the hyperref package requires pdfLaTeX in order to break URLs across lines.

\documentclass[11pt]{article}
\usepackage{blindtext} % For dummy text
\makeatletter
\newcommand\blfootnote[1]{%
  \begingroup
  \renewcommand\thefootnote{\fnsymbol{footnote}}%
  \footnotetext{#1}%
  \endgroup
}
\makeatother

% Change "review" to "final" to generate the final (sometimes called camera-ready) version.
% Change to "preprint" to generate a non-anonymous version with page numbers.
% \usepackage[review]{acl}
\usepackage[preprint]{acl}
\usepackage{algorithm}
% Standard package includes
\usepackage{times}
\usepackage{latexsym}

% For proper rendering and hyphenation of words containing Latin characters (including in bib files)
\usepackage[T1]{fontenc}
% For Vietnamese characters
% \usepackage[T5]{fontenc}
% See https://www.latex-project.org/help/documentation/encguide.pdf for other character sets

% This assumes your files are encoded as UTF8
\usepackage[utf8]{inputenc}

% This is not strictly necessary, and may be commented out,
% but it will improve the layout of the manuscript,
% and will typically save some space.
\usepackage{microtype}

% This is also not strictly necessary, and may be commented out.
% However, it will improve the aesthetics of text in
% the typewriter font.
\usepackage{inconsolata}
\usepackage{amsmath}

%Including images in your LaTeX document requires adding
%additional package(s)
\usepackage{graphicx}

% If the title and author information does not fit in the area allocated, uncomment the following
%
%\setlength\titlebox{<dim>}
%
% and set <dim> to something 5cm or larger.

\usepackage{microtype}
% \usepackage{graphicx}
\usepackage{caption}
\usepackage{subcaption}
% \usepackage{subfigure}
\usepackage{booktabs} % for professional tables
\usepackage[utf8]{inputenc} % allow utf-8 input
\usepackage[T1]{fontenc}    % use 8-bit T1 fonts
\usepackage{url}            % simple URL typesetting
\usepackage{booktabs}       % professional-quality tables
\usepackage{amsfonts}       % blackboard math symbols
\usepackage{nicefrac}       % compact symbols for 1/2, etc.
\usepackage{microtype}      % microtypography
\usepackage{xcolor}         % colors
% \newcommand{\theHalgorithm}{\arabic{algorithm}}
\usepackage{wrapfig}
\usepackage{booktabs}
\usepackage{enumitem}
\usepackage{makecell}
\usepackage{diagbox}
\usepackage{bm}
\usepackage{dsfont}
% \usepackage{subfig}
\usepackage{multirow}
\usepackage{varwidth}
\usepackage{pifont}
\usepackage{makecell}
\usepackage{wrapfig}
\usepackage{graphicx} 
\usepackage{comment}
\usepackage{color}
\usepackage[colaction]{multicol}
\usepackage{colortbl}
%\usepackage{algorithmic} %%Zhao: Can we don't this package.
%\usepackage[ruled, linesnumbered]{algorithm2e} %%%Zhao: Can we don't use this package. I have a lot of proofs to write.
\usepackage{algpseudocode} %%%Zhao: I added this, don't delete it.
% \usepackage{algorithm}%%%Zhao: I added this, don't delete it.
\usepackage{xspace}
\usepackage{rotating}
\usepackage{longtable}
\usepackage{adjustbox}
\usepackage{threeparttable}
% \usepackage{algorithmic}
\usepackage{listings}

\usepackage{hyperref}
% \usepackage[table]{xcolor}
\definecolor{myblue}{HTML}{bfcdf0}
\definecolor{myred}{HTML}{e07f7f}

\newcommand{\aname}{\textsc{ITT}\xspace}

\title{Inner Thinking Transformer: \\ Leveraging Dynamic Depth Scaling to Foster Adaptive Internal Thinking}

% Author information can be set in various styles:
% For several authors from the same institution:
% \author{Author 1 \and ... \and Author n \\
%         Address line \\ ... \\ Address line}
% if the names do not fit well on one line use
%         Author 1 \\ {\bf Author 2} \\ ... \\ {\bf Author n} \\
% For authors from different institutions:
% \author{Author 1 \\ Address line \\  ... \\ Address line
%         \And  ... \And
%         Author n \\ Address line \\ ... \\ Address line}
% To start a separate ``row'' of authors use \AND, as in
% \author{Author 1 \\ Address line \\  ... \\ Address line
%         \AND
%         Author 2 \\ Address line \\ ... \\ Address line \And
%         Author 3 \\ Address line \\ ... \\ Address line}

\author{
    Yilong Chen$^{1,2}$,
    ~Junyuan Shang$^{3\ddagger}$,
    ~Zhenyu Zhang$^{3}$,
    ~Yanxi Xie$^{4}$,
    ~\textbf{Jiawei Sheng}$^1$,
    ~\textbf{Tingwen Liu}$^{1,2\dagger}$, 
    \\~\textbf{Shuohuan Wang}$^3$\textbf{,}~\textbf{Yu Sun}$^3$\textbf{,}~\textbf{Hua Wu}$^3$\textbf{,}~\textbf{Haifeng Wang}$^3$  \\ 
    \normalsize $^1$ Institute of Information Engineering, Chinese Academy of Sciences\\
    \normalsize $^2$ School of Cyber Security, University of Chinese Academy of Sciences\\
    \normalsize $^3$ Baidu Inc.\\
    \normalsize $^4$ School of Artificial Intelligence, Beijing Normal University\\
    \small \{\texttt{chenyilong, shengjiawei, liutingwen\}@iie.ac.cn} \\
    \small \{\texttt{shangjunyuan, zhangzhenyu07, wangshuohuan, sunyu02\}@baidu.com} \\
}



%\author{
%  \textbf{First Author\textsuperscript{1}},
%  \textbf{Second Author\textsuperscript{1,2}},
%  \textbf{Third T. Author\textsuperscript{1}},
%  \textbf{Fourth Author\textsuperscript{1}},
%\\
%  \textbf{Fifth Author\textsuperscript{1,2}},
%  \textbf{Sixth Author\textsuperscript{1}},
%  \textbf{Seventh Author\textsuperscript{1}},
%  \textbf{Eighth Author \textsuperscript{1,2,3,4}},
%\\
%  \textbf{Ninth Author\textsuperscript{1}},
%  \textbf{Tenth Author\textsuperscript{1}},
%  \textbf{Eleventh E. Author\textsuperscript{1,2,3,4,5}},
%  \textbf{Twelfth Author\textsuperscript{1}},
%\\
%  \textbf{Thirteenth Author\textsuperscript{3}},
%  \textbf{Fourteenth F. Author\textsuperscript{2,4}},
%  \textbf{Fifteenth Author\textsuperscript{1}},
%  \textbf{Sixteenth Author\textsuperscript{1}},
%\\
%  \textbf{Seventeenth S. Author\textsuperscript{4,5}},
%  \textbf{Eighteenth Author\textsuperscript{3,4}},
%  \textbf{Nineteenth N. Author\textsuperscript{2,5}},
%  \textbf{Twentieth Author\textsuperscript{1}}
%\\
%\\
%  \textsuperscript{1}Affiliation 1,
%  \textsuperscript{2}Affiliation 2,
%  \textsuperscript{3}Affiliation 3,
%  \textsuperscript{4}Affiliation 4,
%  \textsuperscript{5}Affiliation 5
%\\
%  \small{
%    \textbf{Correspondence:} \href{mailto:email@domain}{email@domain}
%  }
%}

\begin{document}
\maketitle
\begin{abstract}

Large language models (LLMs) face inherent performance bottlenecks under parameter constraints, particularly in processing critical tokens that demand complex reasoning. Empirical analysis reveals challenging tokens induce abrupt gradient spikes across layers, exposing architectural stress points in standard Transformers. Building on this insight, we propose Inner Thinking Transformer (ITT), which reimagines layer computations as implicit thinking steps. ITT dynamically allocates computation through Adaptive Token Routing, iteratively refines representations via Residual Thinking Connections, and distinguishes reasoning phases using Thinking Step Encoding. ITT enables deeper processing of critical tokens without parameter expansion. Evaluations across 162M-466M parameter models show ITT achieves 96.5\% performance of a 466M Transformer using only 162M parameters, reduces training data by 43.2\%, and outperforms Transformer/Loop variants in 11 benchmarks. By enabling elastic computation allocation during inference, ITT balances performance and efficiency through architecture-aware optimization of implicit thinking pathways.

\end{abstract}

\blfootnote{$^\dagger$Corresponding author. $^\ddagger$ Project lead. Preliminary work.}


\section{Introduction}

Large language models (LLMs)~\cite{claude,Gpt-4,touvronLlamaOpenFoundation2023}  have demonstrated remarkable performance across numerous natural language tasks. Recent studies~\cite{fernandez2024hardware, hoffmann2022training, chen2024scaling} indicate that scaling laws for LLM parameters exhibit diminishing returns under constrained data availability and computational resource budgets. Scaling model parameters increases computational and deployment costs, making high-performance models impractical for resource-constrained environments. Meanwhile, smaller models encounter performance bottlenecks primarily attributable to limited parameter space.

Recent approaches, such as Test-Time Scaling ("Slow-Thinking")~\cite{muennighoff2025simple,snell2024scaling,ma2024inference}, aim to enhance performance by allocating more computation during the inference search process. While effective, these methods are limited by the reliance on accurately generating key tokens, which can lead to catastrophic reasoning failures~\cite{chen2023token,singh2024exposing,jiang2024peek}, especially in smaller models. Some works enhance model performance through layer sharing~\cite{li2024lisa,li2024crosslayer}, recursion~\cite{ng2024loopneuralnetworksparameter,dehghani2019universaltransformers,geiping2025scalingtesttimecomputelatent}, or implicit reasoning~\cite{deng2023implicit,shalev2024distributional}, but they fail to flexibly improve the model's reasoning ability on key tokens, which either suffer from insufficient performance or  redundant overhead.

% these methods suffer from a critical limitation: LLM reasoning accuracy is highly dependent on correctly generating key tokens that demand complex planning. Errors in such tokens often lead to catastrophic reasoning failures, particularly in smaller models, where excessive reflection further constrains performance ceilings. Prior attempts to mitigate this—such as prompting models to produce concise reasoning chains or pre-reasoning before generating critical tokens, but fail to address the root cause: insufficient intrinsic model capabilities.

\begin{figure}[t]  
\centering  
\includegraphics[width=7.6cm]{figs/intro_v2.pdf}
\caption{
The Transformer, constrained by a limited number of parameters, tends to make errors on difficult samples. We treat each single computation in the model's layers as one step of inner thinking. By training the model to allocate more inner thinking steps at specific layers and organize thinking results, the model can achieve better results without scaling parameters.
}
\vspace{-2mm}
% \cyl{using FFN as exmaple}
\label{fig:intro_motivation}
\end{figure}

% In contrast to existing work, 

In this work, we aim to explore\textit{ how the model can allocate more computation to individual tokens, enhancing testing performance without increasing parameters.}
Through analysis in Section~\ref{sec.obs}, we explore how models learn and reason about critical tokens. Our findings reveal that \textit{simple tokens are resolved efficiently in early layers with stable low-gradient flows, while complex tokens cause difficulties across layers, with sudden gradient spikes indicating architectural or parametric issues.} The differentiated properties of layers inspire us to propose a novel perspective on the model’s internal reasoning process: \textit{Inner Thinking}.  Inner thinking conceptualizes the evolution of hidden states layer by layer, with each layer representing a distinct implicit reasoning step for deriving a single token.
% This work provides foundational insights into LLM reasoning limitations and paves the way for architecture-aware optimization strategies.

Intuitively, we can extend and combine multiple inner thinking steps to break the model's performance bottleneck.
Therefore, we propose a novel approach called Inner Thinking Transformer (\aname). \aname enhances token-level reasoning by dynamically allocating additional thinking steps to key tokens and iteratively accumulating residual thinking results to refine tokens' representations. As shown in Figure~\ref{fig:intro_motivation}, the model learns to “think” more deeply on important information during training. Specifically, we design a dynamic token-wise depth architecture based on\textit{ Adaptive Token Routing} networks and adopt a\textit{ Residual Thinking Connection} mechanism (RTC) that gradually converges toward better outcomes at each step. In addition, we introduce a\textit{ Thinking Step Encoding} scheme to better differentiate between successive thinking steps.

Notably, while trained under specific thinking settings, our architecture can \textit{flexibly allocate more computational resources during testing time} to improve performance or achieve a balanced trade-off between resources and performance (see Figure~\ref{fig:res_3figs}). \textit{The routing network autonomously develops a thinking pattern that strategically balances depth and breadth}: specific thinking steps are allocated for intensive processing of complex tokens, while more efficient pathways handle simpler tokens (see Figure~\ref{fig:router_visual}). In general, \aname mitigates the performance bottleneck in reasoning for individual tokens and can be combined with COT methods to resolving reasoning challenges for critical tokens.

Experimentally, we construct both vanilla Transformer,  Loop variants and \aname variants across three scales (162M, 230M, and 466M parameters) following the LLaMA architecture. Evaluated on an 11-task benchmark, \aname consistently outperforms Transformer and Loop variants with an equivalent parameters. \aname achieves higher performance with the same FLOPs and saves 43.2\% of the training data budget compared to Transformer. Notably, the ITT ×4 -162M model significantly surpasses the 230M Transformer and even achieves 96.5\% performance of 466M Transformer. Overall, \aname introduces an inherent test-time scaling in the model, achieving both performance and efficiency balance through its elastic deep computation paradigm.



\section{Observation} \label{sec.obs}
\begin{figure}[t]  
\centering  
\includegraphics[width=7.6cm]{figs/curve_gard.pdf}
\caption{
Layer's Gradient Nuclear Norm of the Attention matrices of GPT-2 on hard or simple samples.
}
\vspace{-2mm}
% \cyl{using FFN as exmaple}
\label{fig:obs}
\end{figure}

To investigate how models learn about critical tokens, our empirical analysis of GPT-2's attention matrices through gradient nuclear norm (GNN) measurements~\cite{li2024happenedllmslayerstrained} reveals systematic patterns in layer-wise dynamics. Using the AQuA corpus~\cite{ling2017program}, we firstly train GPT-2 in 100 samples then categorize samples in evaluation into easy (model answers correctly) and hard (model answers incorrectly). In Figure~\ref{fig:obs}, for easy samples, GNN values decay exponentially across early layers (L0-L2) and final layers (L11), stabilizing below 3 in layers (L3-L10). In contrast, hard samples exhibit persistent GNN oscillations throughout all 12 layers, punctuated by abrupt spikes at strategic layer positions (L3, L5, L7, L9).

These observations reveal one of the underlying reasons for the presence of hard-to-learn samples in models: as shown in Figure~\ref{fig:intro_motivation}, certain parameters face significant optimization difficulties due to architectural limitations (e.g., insufficient depth) or parameter constraints. Many studies suggest that Transformer layers exhibit unique functional characteristics and training variances~\cite{alizadeh2024duollmframeworkstudyingadaptive,sun2025transformerlayerspainters,takase2023lessonsparametersharinglayers}.. This inspires us to propose a framework where \textit{each layer transformation in the model is viewed as a single thinking step on latent information.} By studying the inner thinking process, we aim to
% aim to explore the functionality of each step and 
design corresponding architectures to optimize model's learning difficulty and inference performance.



\section{Method} \label{sec.method}
In this section, we introduce our Inner Thinking (\aname) framework (Figure~\ref{fig:main}) to enhance transformer models by dynamically deepening token-level reasoning. We begin in Section~\ref{method:step} by formalizing inner thinking steps within the transformer. Section~\ref{method:residual} then details the Residual Thinking Connection, where inner steps are extended via residual accumulation. In Section~\ref{method:routing}, we present the Adaptive Token Routing, which employs a weight predictor to select the most critical tokens for further thinking. Finally, Section~\ref{method:bp} demonstrate how \aname enhances learning efficiency in backporpogation.

\begin{figure*}[t]
\begin{minipage}[c]{\textwidth}
\centering
% \subfloat[MHA to DHA Transformation]{
  \includegraphics[width=\textwidth]{figs/main_3.pdf}
  \caption{
  An illustration of \aname: ITT uses Adaptive Token Routing to select and weight important tokens for each inner thinking step. Based on Thinking Step Encoding and Residual Thinking Connection, ITT layer iterates thinking multiple times, accumulating each step's results for improved layer output.
  }\label{fig:main}
% }
\end{minipage}
\vspace{-2mm}
\end{figure*}



\subsection{Inner Thinking Step in Transformer}
\label{method:step}

Traditional reasoning in Transformer models typically relies on token-by-token generation. Given an input $x$, the output sequence $y = (y_1, y_2, \ldots, y_N)$ is generated as
\begin{equation} \small
P(y \mid x) = \prod_{n=1}^{N} P(y_n \mid y_{<n}, x),
\label{eq:sequentially} \end{equation}
However, errors in key tokens can propagate, potentially leading to an incorrect result. To investigate the intrinsic mechanisms in single-token generating, we propose a novel concept of \emph{Inner Thinking} in model's depth that decomposes the generation of each token into a series of internal thinking steps. Specifically, given an initial state $x^{(0)}$, we define Inner Thinking as
\begin{equation} \small
X^{(t)} = f^{(t)}\big(x^{(t-1)}\big), \quad t = 1, 2, \ldots, T,
\label{eq:inner_thinking} \end{equation}
where $f^{(t)}(\cdot)$ represents the transformation corresponding to the $t$-th thinking step (consist of one or more Transformer layers) and $T$ is the maximum number of steps. The final token is then generated based on the output of the last thinking step:
\begin{equation} \small
P(y \mid x) = \operatorname{softmax}\big(W\, x^{(T)} + b\big),
\label{eq:lmout} \end{equation}
with $W$ and $b$ denoting the weights and bias for the output projection. Define $\mathcal{L}(\cdot, y)$ measures the discrepancy between final state $X^{(T)}$ and the target token $y$, we have two scenarios:

% \paragraph{Early Exit:} If at an intermediate step $t_0 < T$ the state $X^{(t_0)}$ is sufficiently close to the target (i.e., if $\mathcal{L}\big(X^{(t_0)}, y\big) < \epsilon$, where $\epsilon$ is a predefined threshold), the model can terminate further processing and output the token via $y = \psi\big(X^{(t_0)}\big)$, where $\psi(\cdot)$ is the decoding function. This indicates that the model performs well, allowing us to obtain correct results using only a subset of Inner Thinking Steps, thereby improving efficiency.
\paragraph{Early Exit:} If at an intermediate step $t_0 < T$, the state $x^{(t_0)}$ is close enough to the target (i.e., $\mathcal{L}\big(x^{(t_0)}, y\big) < \epsilon$, where $\epsilon$ is a threshold), the model can stop and output the token as $y = \psi\big(x^{(t_0)}\big)$, where $\psi(\cdot)$ is the decoding function. This allows the model to achieve correct results with fewer Inner Thinking Steps, improving efficiency.
    
\paragraph{Performance Deficiency:} Conversely, if even after all $T$ internal steps the discrepancy remains high (i.e., $\mathcal{L}\big(x^{(T)}, y\big) > \epsilon$), it indicates that the Inner Thinking was insufficient to correctly approximate the target. This scenario highlights potential areas for improvement in the model's reasoning capacity or its internal step design.


\subsection{Residual Thinking Connection}
\label{method:residual}

% \begin{algorithm}[ht!]
\caption{\textit{NovelSelect}}
\label{alg:novelselect}
\begin{algorithmic}[1]
\State \textbf{Input:} Data pool $\mathcal{X}^{all}$, data budget $n$
\State Initialize an empty dataset, $\mathcal{X} \gets \emptyset$
\While{$|\mathcal{X}| < n$}
    \State $x^{new} \gets \arg\max_{x \in \mathcal{X}^{all}} v(x)$
    \State $\mathcal{X} \gets \mathcal{X} \cup \{x^{new}\}$
    \State $\mathcal{X}^{all} \gets \mathcal{X}^{all} \setminus \{x^{new}\}$
\EndWhile
\State \textbf{return} $\mathcal{X}$
\end{algorithmic}
\end{algorithm}

Under the framework defined in Section~\ref{method:step}, we aim to enhance the model's performance to reduce \textbf{Performance Deficiencies}. For challenging examples, high gradient values are observed in Section~\ref{sec.obs}, indicating that the model faces optimization difficulties. 
To address these issues, a natural approach is to increase the number of inner thinking steps in one layer's computation. Therefore, we propose a \emph{Residual Thinking Connection} (RTC) mechanism that train model's layer parameters to\textbf{ learn iterative thinking capabilities}, reducing the difficulty of single-step thinking and enabling multiple uses of parameters to break performance bottlenecks.


Let $x^{(0)} \in \mathbb{R}^{d}$ denote RTC Layer input of a token representation, where $d$ is the hidden dimension. We denote $f:\mathbb{R}^{d}\rightarrow\mathbb{R}^{d}$ as the layer  transformation,  $T$ is the maximum number of thinking steps. In RTC, the final output after $t$ iterative steps is computed by cumulatively accumulating each step's outputs:
\begin{equation}
\small
\begin{split}
x^{(t)} &= \sum_{i=1}^{t} \left( f\big(x^{(i-1)}\big) \odot \phi^{(i)}  \right), 
t = 1, \ldots, T \\
\end{split}
\label{eq:rtc}
\end{equation}
where $\phi^{(t)} \in \mathbb{R}^{d}$ the learnable thinking position encoding associated with the $t$-th inner thinking step, which measuring the differences and importance of each step. Rather than processing the input representation only once, RTC Layer iteratively refine it by adding the residual contributions of each step's layer-output together with a learnable encoding. 
Compared to direct looping~\cite{ng2024loopneuralnetworksparameter,dehghani2019universaltransformers},\textit{ RTC not only enables deeper thinking but also effectively measures and combines each thinking step, allowing them to complement each other.} RTC provides the foundation for scaling Inner Thinking during testing.

\subsection{Adaptive Token Routing}
\label{method:routing}

 RTC in Section~\ref{method:residual} provides a method to enhance inner thinking. However,\textit{ different tokens require a varying number of thinking steps in the model}, as show in Section~\ref{sec.obs}. Moreover, we aim for the model to learn detailed, task-specific information at each step. To avoid unnecessary computation and information interference from processing all tokens at once, we introduce Adaptive Token Routing (ATR). Inspired by deep conditional computation~\cite{raposo2024mixtureofdepthsdynamicallyallocatingcompute,zhang2024pmodbuildingmixtureofdepthsmllms}, ATR, based on a routing network, selects the most important tokens for thinking at each step.
 
 Let the input sequence be denoted by \( X \in \mathbb{R}^{n \times d} \), where \( n \) is the sequence length. We perform a forward pass to obtain the output \( Y^{(0)} \in \mathbb{R}^{n \times d} \), and then linear weight predictor $\mathcal{R}^{(0)} \in \mathbb{R}^{d \times 1} $ is applied to \( Y^{(0)} \) to generate an importance score:
 % for each token's output: 
 \begin{equation} \small
Y^{(0)} = f(X), \quad
w^{(1)} = \mathcal{R}^{(1)}(Y^{(0)}) \in \mathbb{R}^{n},
\end{equation}
 and we denote by \(P_\rho(w^{(1)})\) the \(\rho\)-th percentile of these scores, with \(\rho\) being a predefined selection ratio. For a given thinking step \(t\), the calculation process in \aname layer can be formulated as:
\begin{equation} \small
Y_i^{(t)\prime}=\left\{\begin{array}{lll}
\alpha^{(t)} w^{(t)}_i f\left(Y^{(t-1)}_i\right), & \text { if } \quad w^{(t)}_i>P_\rho(w^{(t)}), \\
Y^{(t-1)}_i, & \text { if } \quad w^{(t)}_i \leq P_\rho(w^{(t)}),
\end{array}\right.
\end{equation}
where $\alpha^{(t)}$ is a hyperparam in $t$ step, \( w^{(t)}_i>P_\rho(w^{(t)})\) is the indicator function selecting only the tokens with predicted weights exceeding the threshold. The router \(\mathcal{R}^{(t)}\)\textit{modulates the decision to execute an additional thinking iteration based on the current token representation and the step-specific encoding.} For tokens deemed important, the model applies an extra weighted transformation. Conversely, tokens that do not meet the selection criteria bypass the extra processing, preserving their previous representation. The router's weights are part of the gradient path, allowing the routing parameters to be updated through backpropagation. 

Finally, \aname (in Figure~\ref{fig:main}) combine the results of each step using RTC, following Equation~\ref{eq:rtc}:
\begin{equation}
\small
\begin{split}
Y^{(t)} &= Y^{(0)} \odot \phi^{(0)}  + \sum_{i=1}^{t} \left( Y_i^{(i)\prime} \odot \phi^{(i)}  \right) , \\
t &= 1, \ldots, T.
\end{split}
\end{equation}
This unified update thus integrates RTC with dynamic, token-level routing, enabling the model to \textit{adaptively allocate computational resources only where deeper thinking is required}. By iteratively selecting a subset of tokens for deeper processing, the model can efficiently reinforce key tokens without increasing the model parameter.
In practice, the ITT layer can be flexibly improved based on the model layers. We insert the ITT layer at regular intervals alongside the model's original layers to construct a flexible inner thinking model, and optimize all model parameters using the language modeling cross-entropy loss: $\mathbb{L} = \mathbb{L}_{\text{CE}}$.



\subsection{Optimization}
\label{method:bp}

In this section, we prove Residual Thinking Learning \textit{extends single-step optimization into multi-step optimization,  making it easier to converge} during backpropagation compared to a direct one-step mapping. Let \( y^* \in \mathbb{R}^d \) the corresponding ground-truth, \( \Theta' \) represents the origin Layer parameters, and \( \theta \) represents th \aname layer parameters. The optimization objective is to minimize the loss:
\begin{equation} \small
\mathcal{L}\bigl(F(x; \Theta', \theta), y^*\bigr) = \mathcal{L}\bigl(G(f_T(x; \theta); \Theta'), y^*\bigr).
\end{equation}
For each step's parameter \( \theta \), the gradient is computed using the chain rule:
\begin{equation} \small
\frac{\partial \mathcal{L}}{\partial \theta} = \frac{\partial \mathcal{L}}{\partial Y^{(t)}} \cdot \prod_{j=t+1}^{T} \left[I + \frac{\partial \Delta_j(Y^{(j)}; \theta)}{\partial Y^{(j)}}\right] \cdot \frac{\partial \Delta_k(Y^{(0)}; \theta)}{\partial \theta}.
\end{equation}
Assuming that the corrections \( \Delta_j \) are small, we can approximate the product term by the identity matrix \( I \), yielding:
\begin{equation} \small
\frac{\partial \mathcal{L}}{\partial \theta} \approx \frac{\partial \mathcal{L}}{\partial Y^{(t)}} \cdot \frac{\partial \Delta_k(Y^{(0)}; \theta)}{\partial \theta}.
\end{equation}
This shows that the gradient update at each small step is nearly equal to the global gradient multiplied by the derivative of the local mapping, aligning with global loss reduction. Assuming each iteration reduces the error by a factor of \( c \), this leads to exponential decay \( c^t \), proving that iterative corrections ensure stable, efficient convergence. In summary, our method avoids excessive scaling or distortion from deep chain propagation. It extends single-step optimization to multi-step, easing convergence and preventing gradient vanishing or explosion.


\section{Experiments}

% This must be in the first 5 lines to tell arXiv to use pdfLaTeX, which is strongly recommended.
\pdfoutput=1
% In particular, the hyperref package requires pdfLaTeX in order to break URLs across lines.

\documentclass[11pt]{article}

% Change "review" to "final" to generate the final (sometimes called camera-ready) version.
% Change to "preprint" to generate a non-anonymous version with page numbers.
\usepackage{acl}

% Standard package includes
\usepackage{times}
\usepackage{latexsym}

% Draw tables
\usepackage{booktabs}
\usepackage{multirow}
\usepackage{xcolor}
\usepackage{colortbl}
\usepackage{array} 
\usepackage{amsmath}

\newcolumntype{C}{>{\centering\arraybackslash}p{0.07\textwidth}}
% For proper rendering and hyphenation of words containing Latin characters (including in bib files)
\usepackage[T1]{fontenc}
% For Vietnamese characters
% \usepackage[T5]{fontenc}
% See https://www.latex-project.org/help/documentation/encguide.pdf for other character sets
% This assumes your files are encoded as UTF8
\usepackage[utf8]{inputenc}

% This is not strictly necessary, and may be commented out,
% but it will improve the layout of the manuscript,
% and will typically save some space.
\usepackage{microtype}
\DeclareMathOperator*{\argmax}{arg\,max}
% This is also not strictly necessary, and may be commented out.
% However, it will improve the aesthetics of text in
% the typewriter font.
\usepackage{inconsolata}

%Including images in your LaTeX document requires adding
%additional package(s)
\usepackage{graphicx}
% If the title and author information does not fit in the area allocated, uncomment the following
%
%\setlength\titlebox{<dim>}
%
% and set <dim> to something 5cm or larger.

\title{Wi-Chat: Large Language Model Powered Wi-Fi Sensing}

% Author information can be set in various styles:
% For several authors from the same institution:
% \author{Author 1 \and ... \and Author n \\
%         Address line \\ ... \\ Address line}
% if the names do not fit well on one line use
%         Author 1 \\ {\bf Author 2} \\ ... \\ {\bf Author n} \\
% For authors from different institutions:
% \author{Author 1 \\ Address line \\  ... \\ Address line
%         \And  ... \And
%         Author n \\ Address line \\ ... \\ Address line}
% To start a separate ``row'' of authors use \AND, as in
% \author{Author 1 \\ Address line \\  ... \\ Address line
%         \AND
%         Author 2 \\ Address line \\ ... \\ Address line \And
%         Author 3 \\ Address line \\ ... \\ Address line}

% \author{First Author \\
%   Affiliation / Address line 1 \\
%   Affiliation / Address line 2 \\
%   Affiliation / Address line 3 \\
%   \texttt{email@domain} \\\And
%   Second Author \\
%   Affiliation / Address line 1 \\
%   Affiliation / Address line 2 \\
%   Affiliation / Address line 3 \\
%   \texttt{email@domain} \\}
% \author{Haohan Yuan \qquad Haopeng Zhang\thanks{corresponding author} \\ 
%   ALOHA Lab, University of Hawaii at Manoa \\
%   % Affiliation / Address line 2 \\
%   % Affiliation / Address line 3 \\
%   \texttt{\{haohany,haopengz\}@hawaii.edu}}
  
\author{
{Haopeng Zhang$\dag$\thanks{These authors contributed equally to this work.}, Yili Ren$\ddagger$\footnotemark[1], Haohan Yuan$\dag$, Jingzhe Zhang$\ddagger$, Yitong Shen$\ddagger$} \\
ALOHA Lab, University of Hawaii at Manoa$\dag$, University of South Florida$\ddagger$ \\
\{haopengz, haohany\}@hawaii.edu\\
\{yiliren, jingzhe, shen202\}@usf.edu\\}



  
%\author{
%  \textbf{First Author\textsuperscript{1}},
%  \textbf{Second Author\textsuperscript{1,2}},
%  \textbf{Third T. Author\textsuperscript{1}},
%  \textbf{Fourth Author\textsuperscript{1}},
%\\
%  \textbf{Fifth Author\textsuperscript{1,2}},
%  \textbf{Sixth Author\textsuperscript{1}},
%  \textbf{Seventh Author\textsuperscript{1}},
%  \textbf{Eighth Author \textsuperscript{1,2,3,4}},
%\\
%  \textbf{Ninth Author\textsuperscript{1}},
%  \textbf{Tenth Author\textsuperscript{1}},
%  \textbf{Eleventh E. Author\textsuperscript{1,2,3,4,5}},
%  \textbf{Twelfth Author\textsuperscript{1}},
%\\
%  \textbf{Thirteenth Author\textsuperscript{3}},
%  \textbf{Fourteenth F. Author\textsuperscript{2,4}},
%  \textbf{Fifteenth Author\textsuperscript{1}},
%  \textbf{Sixteenth Author\textsuperscript{1}},
%\\
%  \textbf{Seventeenth S. Author\textsuperscript{4,5}},
%  \textbf{Eighteenth Author\textsuperscript{3,4}},
%  \textbf{Nineteenth N. Author\textsuperscript{2,5}},
%  \textbf{Twentieth Author\textsuperscript{1}}
%\\
%\\
%  \textsuperscript{1}Affiliation 1,
%  \textsuperscript{2}Affiliation 2,
%  \textsuperscript{3}Affiliation 3,
%  \textsuperscript{4}Affiliation 4,
%  \textsuperscript{5}Affiliation 5
%\\
%  \small{
%    \textbf{Correspondence:} \href{mailto:email@domain}{email@domain}
%  }
%}

\begin{document}
\maketitle
\begin{abstract}
Recent advancements in Large Language Models (LLMs) have demonstrated remarkable capabilities across diverse tasks. However, their potential to integrate physical model knowledge for real-world signal interpretation remains largely unexplored. In this work, we introduce Wi-Chat, the first LLM-powered Wi-Fi-based human activity recognition system. We demonstrate that LLMs can process raw Wi-Fi signals and infer human activities by incorporating Wi-Fi sensing principles into prompts. Our approach leverages physical model insights to guide LLMs in interpreting Channel State Information (CSI) data without traditional signal processing techniques. Through experiments on real-world Wi-Fi datasets, we show that LLMs exhibit strong reasoning capabilities, achieving zero-shot activity recognition. These findings highlight a new paradigm for Wi-Fi sensing, expanding LLM applications beyond conventional language tasks and enhancing the accessibility of wireless sensing for real-world deployments.
\end{abstract}

\section{Introduction}

In today’s rapidly evolving digital landscape, the transformative power of web technologies has redefined not only how services are delivered but also how complex tasks are approached. Web-based systems have become increasingly prevalent in risk control across various domains. This widespread adoption is due their accessibility, scalability, and ability to remotely connect various types of users. For example, these systems are used for process safety management in industry~\cite{kannan2016web}, safety risk early warning in urban construction~\cite{ding2013development}, and safe monitoring of infrastructural systems~\cite{repetto2018web}. Within these web-based risk management systems, the source search problem presents a huge challenge. Source search refers to the task of identifying the origin of a risky event, such as a gas leak and the emission point of toxic substances. This source search capability is crucial for effective risk management and decision-making.

Traditional approaches to implementing source search capabilities into the web systems often rely on solely algorithmic solutions~\cite{ristic2016study}. These methods, while relatively straightforward to implement, often struggle to achieve acceptable performances due to algorithmic local optima and complex unknown environments~\cite{zhao2020searching}. More recently, web crowdsourcing has emerged as a promising alternative for tackling the source search problem by incorporating human efforts in these web systems on-the-fly~\cite{zhao2024user}. This approach outsources the task of addressing issues encountered during the source search process to human workers, leveraging their capabilities to enhance system performance.

These solutions often employ a human-AI collaborative way~\cite{zhao2023leveraging} where algorithms handle exploration-exploitation and report the encountered problems while human workers resolve complex decision-making bottlenecks to help the algorithms getting rid of local deadlocks~\cite{zhao2022crowd}. Although effective, this paradigm suffers from two inherent limitations: increased operational costs from continuous human intervention, and slow response times of human workers due to sequential decision-making. These challenges motivate our investigation into developing autonomous systems that preserve human-like reasoning capabilities while reducing dependency on massive crowdsourced labor.

Furthermore, recent advancements in large language models (LLMs)~\cite{chang2024survey} and multi-modal LLMs (MLLMs)~\cite{huang2023chatgpt} have unveiled promising avenues for addressing these challenges. One clear opportunity involves the seamless integration of visual understanding and linguistic reasoning for robust decision-making in search tasks. However, whether large models-assisted source search is really effective and efficient for improving the current source search algorithms~\cite{ji2022source} remains unknown. \textit{To address the research gap, we are particularly interested in answering the following two research questions in this work:}

\textbf{\textit{RQ1: }}How can source search capabilities be integrated into web-based systems to support decision-making in time-sensitive risk management scenarios? 
% \sq{I mention ``time-sensitive'' here because I feel like we shall say something about the response time -- LLM has to be faster than humans}

\textbf{\textit{RQ2: }}How can MLLMs and LLMs enhance the effectiveness and efficiency of existing source search algorithms? 

% \textit{\textbf{RQ2:}} To what extent does the performance of large models-assisted search align with or approach the effectiveness of human-AI collaborative search? 

To answer the research questions, we propose a novel framework called Auto-\
S$^2$earch (\textbf{Auto}nomous \textbf{S}ource \textbf{Search}) and implement a prototype system that leverages advanced web technologies to simulate real-world conditions for zero-shot source search. Unlike traditional methods that rely on pre-defined heuristics or extensive human intervention, AutoS$^2$earch employs a carefully designed prompt that encapsulates human rationales, thereby guiding the MLLM to generate coherent and accurate scene descriptions from visual inputs about four directional choices. Based on these language-based descriptions, the LLM is enabled to determine the optimal directional choice through chain-of-thought (CoT) reasoning. Comprehensive empirical validation demonstrates that AutoS$^2$-\ 
earch achieves a success rate of 95–98\%, closely approaching the performance of human-AI collaborative search across 20 benchmark scenarios~\cite{zhao2023leveraging}. 

Our work indicates that the role of humans in future web crowdsourcing tasks may evolve from executors to validators or supervisors. Furthermore, incorporating explanations of LLM decisions into web-based system interfaces has the potential to help humans enhance task performance in risk control.






\section{Related Work}
\label{sec:relatedworks}

% \begin{table*}[t]
% \centering 
% \renewcommand\arraystretch{0.98}
% \fontsize{8}{10}\selectfont \setlength{\tabcolsep}{0.4em}
% \begin{tabular}{@{}lc|cc|cc|cc@{}}
% \toprule
% \textbf{Methods}           & \begin{tabular}[c]{@{}c@{}}\textbf{Training}\\ \textbf{Paradigm}\end{tabular} & \begin{tabular}[c]{@{}c@{}}\textbf{$\#$ PT Data}\\ \textbf{(Tokens)}\end{tabular} & \begin{tabular}[c]{@{}c@{}}\textbf{$\#$ IFT Data}\\ \textbf{(Samples)}\end{tabular} & \textbf{Code}  & \begin{tabular}[c]{@{}c@{}}\textbf{Natural}\\ \textbf{Language}\end{tabular} & \begin{tabular}[c]{@{}c@{}}\textbf{Action}\\ \textbf{Trajectories}\end{tabular} & \begin{tabular}[c]{@{}c@{}}\textbf{API}\\ \textbf{Documentation}\end{tabular}\\ \midrule 
% NexusRaven~\citep{srinivasan2023nexusraven} & IFT & - & - & \textcolor{green}{\CheckmarkBold} & \textcolor{green}{\CheckmarkBold} &\textcolor{red}{\XSolidBrush}&\textcolor{red}{\XSolidBrush}\\
% AgentInstruct~\citep{zeng2023agenttuning} & IFT & - & 2k & \textcolor{green}{\CheckmarkBold} & \textcolor{green}{\CheckmarkBold} &\textcolor{red}{\XSolidBrush}&\textcolor{red}{\XSolidBrush} \\
% AgentEvol~\citep{xi2024agentgym} & IFT & - & 14.5k & \textcolor{green}{\CheckmarkBold} & \textcolor{green}{\CheckmarkBold} &\textcolor{green}{\CheckmarkBold}&\textcolor{red}{\XSolidBrush} \\
% Gorilla~\citep{patil2023gorilla}& IFT & - & 16k & \textcolor{green}{\CheckmarkBold} & \textcolor{green}{\CheckmarkBold} &\textcolor{red}{\XSolidBrush}&\textcolor{green}{\CheckmarkBold}\\
% OpenFunctions-v2~\citep{patil2023gorilla} & IFT & - & 65k & \textcolor{green}{\CheckmarkBold} & \textcolor{green}{\CheckmarkBold} &\textcolor{red}{\XSolidBrush}&\textcolor{green}{\CheckmarkBold}\\
% LAM~\citep{zhang2024agentohana} & IFT & - & 42.6k & \textcolor{green}{\CheckmarkBold} & \textcolor{green}{\CheckmarkBold} &\textcolor{green}{\CheckmarkBold}&\textcolor{red}{\XSolidBrush} \\
% xLAM~\citep{liu2024apigen} & IFT & - & 60k & \textcolor{green}{\CheckmarkBold} & \textcolor{green}{\CheckmarkBold} &\textcolor{green}{\CheckmarkBold}&\textcolor{red}{\XSolidBrush} \\\midrule
% LEMUR~\citep{xu2024lemur} & PT & 90B & 300k & \textcolor{green}{\CheckmarkBold} & \textcolor{green}{\CheckmarkBold} &\textcolor{green}{\CheckmarkBold}&\textcolor{red}{\XSolidBrush}\\
% \rowcolor{teal!12} \method & PT & 103B & 95k & \textcolor{green}{\CheckmarkBold} & \textcolor{green}{\CheckmarkBold} & \textcolor{green}{\CheckmarkBold} & \textcolor{green}{\CheckmarkBold} \\
% \bottomrule
% \end{tabular}
% \caption{Summary of existing tuning- and pretraining-based LLM agents with their training sample sizes. "PT" and "IFT" denote "Pre-Training" and "Instruction Fine-Tuning", respectively. }
% \label{tab:related}
% \end{table*}

\begin{table*}[ht]
\begin{threeparttable}
\centering 
\renewcommand\arraystretch{0.98}
\fontsize{7}{9}\selectfont \setlength{\tabcolsep}{0.2em}
\begin{tabular}{@{}l|c|c|ccc|cc|cc|cccc@{}}
\toprule
\textbf{Methods} & \textbf{Datasets}           & \begin{tabular}[c]{@{}c@{}}\textbf{Training}\\ \textbf{Paradigm}\end{tabular} & \begin{tabular}[c]{@{}c@{}}\textbf{\# PT Data}\\ \textbf{(Tokens)}\end{tabular} & \begin{tabular}[c]{@{}c@{}}\textbf{\# IFT Data}\\ \textbf{(Samples)}\end{tabular} & \textbf{\# APIs} & \textbf{Code}  & \begin{tabular}[c]{@{}c@{}}\textbf{Nat.}\\ \textbf{Lang.}\end{tabular} & \begin{tabular}[c]{@{}c@{}}\textbf{Action}\\ \textbf{Traj.}\end{tabular} & \begin{tabular}[c]{@{}c@{}}\textbf{API}\\ \textbf{Doc.}\end{tabular} & \begin{tabular}[c]{@{}c@{}}\textbf{Func.}\\ \textbf{Call}\end{tabular} & \begin{tabular}[c]{@{}c@{}}\textbf{Multi.}\\ \textbf{Step}\end{tabular}  & \begin{tabular}[c]{@{}c@{}}\textbf{Plan}\\ \textbf{Refine}\end{tabular}  & \begin{tabular}[c]{@{}c@{}}\textbf{Multi.}\\ \textbf{Turn}\end{tabular}\\ \midrule 
\multicolumn{13}{l}{\emph{Instruction Finetuning-based LLM Agents for Intrinsic Reasoning}}  \\ \midrule
FireAct~\cite{chen2023fireact} & FireAct & IFT & - & 2.1K & 10 & \textcolor{red}{\XSolidBrush} &\textcolor{green}{\CheckmarkBold} &\textcolor{green}{\CheckmarkBold}  & \textcolor{red}{\XSolidBrush} &\textcolor{green}{\CheckmarkBold} & \textcolor{red}{\XSolidBrush} &\textcolor{green}{\CheckmarkBold} & \textcolor{red}{\XSolidBrush} \\
ToolAlpaca~\cite{tang2023toolalpaca} & ToolAlpaca & IFT & - & 4.0K & 400 & \textcolor{red}{\XSolidBrush} &\textcolor{green}{\CheckmarkBold} &\textcolor{green}{\CheckmarkBold} & \textcolor{red}{\XSolidBrush} &\textcolor{green}{\CheckmarkBold} & \textcolor{red}{\XSolidBrush}  &\textcolor{green}{\CheckmarkBold} & \textcolor{red}{\XSolidBrush}  \\
ToolLLaMA~\cite{qin2023toolllm} & ToolBench & IFT & - & 12.7K & 16,464 & \textcolor{red}{\XSolidBrush} &\textcolor{green}{\CheckmarkBold} &\textcolor{green}{\CheckmarkBold} &\textcolor{red}{\XSolidBrush} &\textcolor{green}{\CheckmarkBold}&\textcolor{green}{\CheckmarkBold}&\textcolor{green}{\CheckmarkBold} &\textcolor{green}{\CheckmarkBold}\\
AgentEvol~\citep{xi2024agentgym} & AgentTraj-L & IFT & - & 14.5K & 24 &\textcolor{red}{\XSolidBrush} & \textcolor{green}{\CheckmarkBold} &\textcolor{green}{\CheckmarkBold}&\textcolor{red}{\XSolidBrush} &\textcolor{green}{\CheckmarkBold}&\textcolor{red}{\XSolidBrush} &\textcolor{red}{\XSolidBrush} &\textcolor{green}{\CheckmarkBold}\\
Lumos~\cite{yin2024agent} & Lumos & IFT  & - & 20.0K & 16 &\textcolor{red}{\XSolidBrush} & \textcolor{green}{\CheckmarkBold} & \textcolor{green}{\CheckmarkBold} &\textcolor{red}{\XSolidBrush} & \textcolor{green}{\CheckmarkBold} & \textcolor{green}{\CheckmarkBold} &\textcolor{red}{\XSolidBrush} & \textcolor{green}{\CheckmarkBold}\\
Agent-FLAN~\cite{chen2024agent} & Agent-FLAN & IFT & - & 24.7K & 20 &\textcolor{red}{\XSolidBrush} & \textcolor{green}{\CheckmarkBold} & \textcolor{green}{\CheckmarkBold} &\textcolor{red}{\XSolidBrush} & \textcolor{green}{\CheckmarkBold}& \textcolor{green}{\CheckmarkBold}&\textcolor{red}{\XSolidBrush} & \textcolor{green}{\CheckmarkBold}\\
AgentTuning~\citep{zeng2023agenttuning} & AgentInstruct & IFT & - & 35.0K & - &\textcolor{red}{\XSolidBrush} & \textcolor{green}{\CheckmarkBold} & \textcolor{green}{\CheckmarkBold} &\textcolor{red}{\XSolidBrush} & \textcolor{green}{\CheckmarkBold} &\textcolor{red}{\XSolidBrush} &\textcolor{red}{\XSolidBrush} & \textcolor{green}{\CheckmarkBold}\\\midrule
\multicolumn{13}{l}{\emph{Instruction Finetuning-based LLM Agents for Function Calling}} \\\midrule
NexusRaven~\citep{srinivasan2023nexusraven} & NexusRaven & IFT & - & - & 116 & \textcolor{green}{\CheckmarkBold} & \textcolor{green}{\CheckmarkBold}  & \textcolor{green}{\CheckmarkBold} &\textcolor{red}{\XSolidBrush} & \textcolor{green}{\CheckmarkBold} &\textcolor{red}{\XSolidBrush} &\textcolor{red}{\XSolidBrush}&\textcolor{red}{\XSolidBrush}\\
Gorilla~\citep{patil2023gorilla} & Gorilla & IFT & - & 16.0K & 1,645 & \textcolor{green}{\CheckmarkBold} &\textcolor{red}{\XSolidBrush} &\textcolor{red}{\XSolidBrush}&\textcolor{green}{\CheckmarkBold} &\textcolor{green}{\CheckmarkBold} &\textcolor{red}{\XSolidBrush} &\textcolor{red}{\XSolidBrush} &\textcolor{red}{\XSolidBrush}\\
OpenFunctions-v2~\citep{patil2023gorilla} & OpenFunctions-v2 & IFT & - & 65.0K & - & \textcolor{green}{\CheckmarkBold} & \textcolor{green}{\CheckmarkBold} &\textcolor{red}{\XSolidBrush} &\textcolor{green}{\CheckmarkBold} &\textcolor{green}{\CheckmarkBold} &\textcolor{red}{\XSolidBrush} &\textcolor{red}{\XSolidBrush} &\textcolor{red}{\XSolidBrush}\\
API Pack~\cite{guo2024api} & API Pack & IFT & - & 1.1M & 11,213 &\textcolor{green}{\CheckmarkBold} &\textcolor{red}{\XSolidBrush} &\textcolor{green}{\CheckmarkBold} &\textcolor{red}{\XSolidBrush} &\textcolor{green}{\CheckmarkBold} &\textcolor{red}{\XSolidBrush}&\textcolor{red}{\XSolidBrush}&\textcolor{red}{\XSolidBrush}\\ 
LAM~\citep{zhang2024agentohana} & AgentOhana & IFT & - & 42.6K & - & \textcolor{green}{\CheckmarkBold} & \textcolor{green}{\CheckmarkBold} &\textcolor{green}{\CheckmarkBold}&\textcolor{red}{\XSolidBrush} &\textcolor{green}{\CheckmarkBold}&\textcolor{red}{\XSolidBrush}&\textcolor{green}{\CheckmarkBold}&\textcolor{green}{\CheckmarkBold}\\
xLAM~\citep{liu2024apigen} & APIGen & IFT & - & 60.0K & 3,673 & \textcolor{green}{\CheckmarkBold} & \textcolor{green}{\CheckmarkBold} &\textcolor{green}{\CheckmarkBold}&\textcolor{red}{\XSolidBrush} &\textcolor{green}{\CheckmarkBold}&\textcolor{red}{\XSolidBrush}&\textcolor{green}{\CheckmarkBold}&\textcolor{green}{\CheckmarkBold}\\\midrule
\multicolumn{13}{l}{\emph{Pretraining-based LLM Agents}}  \\\midrule
% LEMUR~\citep{xu2024lemur} & PT & 90B & 300.0K & - & \textcolor{green}{\CheckmarkBold} & \textcolor{green}{\CheckmarkBold} &\textcolor{green}{\CheckmarkBold}&\textcolor{red}{\XSolidBrush} & \textcolor{red}{\XSolidBrush} &\textcolor{green}{\CheckmarkBold} &\textcolor{red}{\XSolidBrush}&\textcolor{red}{\XSolidBrush}\\
\rowcolor{teal!12} \method & \dataset & PT & 103B & 95.0K  & 76,537  & \textcolor{green}{\CheckmarkBold} & \textcolor{green}{\CheckmarkBold} & \textcolor{green}{\CheckmarkBold} & \textcolor{green}{\CheckmarkBold} & \textcolor{green}{\CheckmarkBold} & \textcolor{green}{\CheckmarkBold} & \textcolor{green}{\CheckmarkBold} & \textcolor{green}{\CheckmarkBold}\\
\bottomrule
\end{tabular}
% \begin{tablenotes}
%     \item $^*$ In addition, the StarCoder-API can offer 4.77M more APIs.
% \end{tablenotes}
\caption{Summary of existing instruction finetuning-based LLM agents for intrinsic reasoning and function calling, along with their training resources and sample sizes. "PT" and "IFT" denote "Pre-Training" and "Instruction Fine-Tuning", respectively.}
\vspace{-2ex}
\label{tab:related}
\end{threeparttable}
\end{table*}

\noindent \textbf{Prompting-based LLM Agents.} Due to the lack of agent-specific pre-training corpus, existing LLM agents rely on either prompt engineering~\cite{hsieh2023tool,lu2024chameleon,yao2022react,wang2023voyager} or instruction fine-tuning~\cite{chen2023fireact,zeng2023agenttuning} to understand human instructions, decompose high-level tasks, generate grounded plans, and execute multi-step actions. 
However, prompting-based methods mainly depend on the capabilities of backbone LLMs (usually commercial LLMs), failing to introduce new knowledge and struggling to generalize to unseen tasks~\cite{sun2024adaplanner,zhuang2023toolchain}. 

\noindent \textbf{Instruction Finetuning-based LLM Agents.} Considering the extensive diversity of APIs and the complexity of multi-tool instructions, tool learning inherently presents greater challenges than natural language tasks, such as text generation~\cite{qin2023toolllm}.
Post-training techniques focus more on instruction following and aligning output with specific formats~\cite{patil2023gorilla,hao2024toolkengpt,qin2023toolllm,schick2024toolformer}, rather than fundamentally improving model knowledge or capabilities. 
Moreover, heavy fine-tuning can hinder generalization or even degrade performance in non-agent use cases, potentially suppressing the original base model capabilities~\cite{ghosh2024a}.

\noindent \textbf{Pretraining-based LLM Agents.} While pre-training serves as an essential alternative, prior works~\cite{nijkamp2023codegen,roziere2023code,xu2024lemur,patil2023gorilla} have primarily focused on improving task-specific capabilities (\eg, code generation) instead of general-domain LLM agents, due to single-source, uni-type, small-scale, and poor-quality pre-training data. 
Existing tool documentation data for agent training either lacks diverse real-world APIs~\cite{patil2023gorilla, tang2023toolalpaca} or is constrained to single-tool or single-round tool execution. 
Furthermore, trajectory data mostly imitate expert behavior or follow function-calling rules with inferior planning and reasoning, failing to fully elicit LLMs' capabilities and handle complex instructions~\cite{qin2023toolllm}. 
Given a wide range of candidate API functions, each comprising various function names and parameters available at every planning step, identifying globally optimal solutions and generalizing across tasks remains highly challenging.



\section{Preliminaries}
\label{Preliminaries}
\begin{figure*}[t]
    \centering
    \includegraphics[width=0.95\linewidth]{fig/HealthGPT_Framework.png}
    \caption{The \ourmethod{} architecture integrates hierarchical visual perception and H-LoRA, employing a task-specific hard router to select visual features and H-LoRA plugins, ultimately generating outputs with an autoregressive manner.}
    \label{fig:architecture}
\end{figure*}
\noindent\textbf{Large Vision-Language Models.} 
The input to a LVLM typically consists of an image $x^{\text{img}}$ and a discrete text sequence $x^{\text{txt}}$. The visual encoder $\mathcal{E}^{\text{img}}$ converts the input image $x^{\text{img}}$ into a sequence of visual tokens $\mathcal{V} = [v_i]_{i=1}^{N_v}$, while the text sequence $x^{\text{txt}}$ is mapped into a sequence of text tokens $\mathcal{T} = [t_i]_{i=1}^{N_t}$ using an embedding function $\mathcal{E}^{\text{txt}}$. The LLM $\mathcal{M_\text{LLM}}(\cdot|\theta)$ models the joint probability of the token sequence $\mathcal{U} = \{\mathcal{V},\mathcal{T}\}$, which is expressed as:
\begin{equation}
    P_\theta(R | \mathcal{U}) = \prod_{i=1}^{N_r} P_\theta(r_i | \{\mathcal{U}, r_{<i}\}),
\end{equation}
where $R = [r_i]_{i=1}^{N_r}$ is the text response sequence. The LVLM iteratively generates the next token $r_i$ based on $r_{<i}$. The optimization objective is to minimize the cross-entropy loss of the response $\mathcal{R}$.
% \begin{equation}
%     \mathcal{L}_{\text{VLM}} = \mathbb{E}_{R|\mathcal{U}}\left[-\log P_\theta(R | \mathcal{U})\right]
% \end{equation}
It is worth noting that most LVLMs adopt a design paradigm based on ViT, alignment adapters, and pre-trained LLMs\cite{liu2023llava,liu2024improved}, enabling quick adaptation to downstream tasks.


\noindent\textbf{VQGAN.}
VQGAN~\cite{esser2021taming} employs latent space compression and indexing mechanisms to effectively learn a complete discrete representation of images. VQGAN first maps the input image $x^{\text{img}}$ to a latent representation $z = \mathcal{E}(x)$ through a encoder $\mathcal{E}$. Then, the latent representation is quantized using a codebook $\mathcal{Z} = \{z_k\}_{k=1}^K$, generating a discrete index sequence $\mathcal{I} = [i_m]_{m=1}^N$, where $i_m \in \mathcal{Z}$ represents the quantized code index:
\begin{equation}
    \mathcal{I} = \text{Quantize}(z|\mathcal{Z}) = \arg\min_{z_k \in \mathcal{Z}} \| z - z_k \|_2.
\end{equation}
In our approach, the discrete index sequence $\mathcal{I}$ serves as a supervisory signal for the generation task, enabling the model to predict the index sequence $\hat{\mathcal{I}}$ from input conditions such as text or other modality signals.  
Finally, the predicted index sequence $\hat{\mathcal{I}}$ is upsampled by the VQGAN decoder $G$, generating the high-quality image $\hat{x}^\text{img} = G(\hat{\mathcal{I}})$.



\noindent\textbf{Low Rank Adaptation.} 
LoRA\cite{hu2021lora} effectively captures the characteristics of downstream tasks by introducing low-rank adapters. The core idea is to decompose the bypass weight matrix $\Delta W\in\mathbb{R}^{d^{\text{in}} \times d^{\text{out}}}$ into two low-rank matrices $ \{A \in \mathbb{R}^{d^{\text{in}} \times r}, B \in \mathbb{R}^{r \times d^{\text{out}}} \}$, where $ r \ll \min\{d^{\text{in}}, d^{\text{out}}\} $, significantly reducing learnable parameters. The output with the LoRA adapter for the input $x$ is then given by:
\begin{equation}
    h = x W_0 + \alpha x \Delta W/r = x W_0 + \alpha xAB/r,
\end{equation}
where matrix $ A $ is initialized with a Gaussian distribution, while the matrix $ B $ is initialized as a zero matrix. The scaling factor $ \alpha/r $ controls the impact of $ \Delta W $ on the model.

\section{HealthGPT}
\label{Method}


\subsection{Unified Autoregressive Generation.}  
% As shown in Figure~\ref{fig:architecture}, 
\ourmethod{} (Figure~\ref{fig:architecture}) utilizes a discrete token representation that covers both text and visual outputs, unifying visual comprehension and generation as an autoregressive task. 
For comprehension, $\mathcal{M}_\text{llm}$ receives the input joint sequence $\mathcal{U}$ and outputs a series of text token $\mathcal{R} = [r_1, r_2, \dots, r_{N_r}]$, where $r_i \in \mathcal{V}_{\text{txt}}$, and $\mathcal{V}_{\text{txt}}$ represents the LLM's vocabulary:
\begin{equation}
    P_\theta(\mathcal{R} \mid \mathcal{U}) = \prod_{i=1}^{N_r} P_\theta(r_i \mid \mathcal{U}, r_{<i}).
\end{equation}
For generation, $\mathcal{M}_\text{llm}$ first receives a special start token $\langle \text{START\_IMG} \rangle$, then generates a series of tokens corresponding to the VQGAN indices $\mathcal{I} = [i_1, i_2, \dots, i_{N_i}]$, where $i_j \in \mathcal{V}_{\text{vq}}$, and $\mathcal{V}_{\text{vq}}$ represents the index range of VQGAN. Upon completion of generation, the LLM outputs an end token $\langle \text{END\_IMG} \rangle$:
\begin{equation}
    P_\theta(\mathcal{I} \mid \mathcal{U}) = \prod_{j=1}^{N_i} P_\theta(i_j \mid \mathcal{U}, i_{<j}).
\end{equation}
Finally, the generated index sequence $\mathcal{I}$ is fed into the decoder $G$, which reconstructs the target image $\hat{x}^{\text{img}} = G(\mathcal{I})$.

\subsection{Hierarchical Visual Perception}  
Given the differences in visual perception between comprehension and generation tasks—where the former focuses on abstract semantics and the latter emphasizes complete semantics—we employ ViT to compress the image into discrete visual tokens at multiple hierarchical levels.
Specifically, the image is converted into a series of features $\{f_1, f_2, \dots, f_L\}$ as it passes through $L$ ViT blocks.

To address the needs of various tasks, the hidden states are divided into two types: (i) \textit{Concrete-grained features} $\mathcal{F}^{\text{Con}} = \{f_1, f_2, \dots, f_k\}, k < L$, derived from the shallower layers of ViT, containing sufficient global features, suitable for generation tasks; 
(ii) \textit{Abstract-grained features} $\mathcal{F}^{\text{Abs}} = \{f_{k+1}, f_{k+2}, \dots, f_L\}$, derived from the deeper layers of ViT, which contain abstract semantic information closer to the text space, suitable for comprehension tasks.

The task type $T$ (comprehension or generation) determines which set of features is selected as the input for the downstream large language model:
\begin{equation}
    \mathcal{F}^{\text{img}}_T =
    \begin{cases}
        \mathcal{F}^{\text{Con}}, & \text{if } T = \text{generation task} \\
        \mathcal{F}^{\text{Abs}}, & \text{if } T = \text{comprehension task}
    \end{cases}
\end{equation}
We integrate the image features $\mathcal{F}^{\text{img}}_T$ and text features $\mathcal{T}$ into a joint sequence through simple concatenation, which is then fed into the LLM $\mathcal{M}_{\text{llm}}$ for autoregressive generation.
% :
% \begin{equation}
%     \mathcal{R} = \mathcal{M}_{\text{llm}}(\mathcal{U}|\theta), \quad \mathcal{U} = [\mathcal{F}^{\text{img}}_T; \mathcal{T}]
% \end{equation}
\subsection{Heterogeneous Knowledge Adaptation}
We devise H-LoRA, which stores heterogeneous knowledge from comprehension and generation tasks in separate modules and dynamically routes to extract task-relevant knowledge from these modules. 
At the task level, for each task type $ T $, we dynamically assign a dedicated H-LoRA submodule $ \theta^T $, which is expressed as:
\begin{equation}
    \mathcal{R} = \mathcal{M}_\text{LLM}(\mathcal{U}|\theta, \theta^T), \quad \theta^T = \{A^T, B^T, \mathcal{R}^T_\text{outer}\}.
\end{equation}
At the feature level for a single task, H-LoRA integrates the idea of Mixture of Experts (MoE)~\cite{masoudnia2014mixture} and designs an efficient matrix merging and routing weight allocation mechanism, thus avoiding the significant computational delay introduced by matrix splitting in existing MoELoRA~\cite{luo2024moelora}. Specifically, we first merge the low-rank matrices (rank = r) of $ k $ LoRA experts into a unified matrix:
\begin{equation}
    \mathbf{A}^{\text{merged}}, \mathbf{B}^{\text{merged}} = \text{Concat}(\{A_i\}_1^k), \text{Concat}(\{B_i\}_1^k),
\end{equation}
where $ \mathbf{A}^{\text{merged}} \in \mathbb{R}^{d^\text{in} \times rk} $ and $ \mathbf{B}^{\text{merged}} \in \mathbb{R}^{rk \times d^\text{out}} $. The $k$-dimension routing layer generates expert weights $ \mathcal{W} \in \mathbb{R}^{\text{token\_num} \times k} $ based on the input hidden state $ x $, and these are expanded to $ \mathbb{R}^{\text{token\_num} \times rk} $ as follows:
\begin{equation}
    \mathcal{W}^\text{expanded} = \alpha k \mathcal{W} / r \otimes \mathbf{1}_r,
\end{equation}
where $ \otimes $ denotes the replication operation.
The overall output of H-LoRA is computed as:
\begin{equation}
    \mathcal{O}^\text{H-LoRA} = (x \mathbf{A}^{\text{merged}} \odot \mathcal{W}^\text{expanded}) \mathbf{B}^{\text{merged}},
\end{equation}
where $ \odot $ represents element-wise multiplication. Finally, the output of H-LoRA is added to the frozen pre-trained weights to produce the final output:
\begin{equation}
    \mathcal{O} = x W_0 + \mathcal{O}^\text{H-LoRA}.
\end{equation}
% In summary, H-LoRA is a task-based dynamic PEFT method that achieves high efficiency in single-task fine-tuning.

\subsection{Training Pipeline}

\begin{figure}[t]
    \centering
    \hspace{-4mm}
    \includegraphics[width=0.94\linewidth]{fig/data.pdf}
    \caption{Data statistics of \texttt{VL-Health}. }
    \label{fig:data}
\end{figure}
\noindent \textbf{1st Stage: Multi-modal Alignment.} 
In the first stage, we design separate visual adapters and H-LoRA submodules for medical unified tasks. For the medical comprehension task, we train abstract-grained visual adapters using high-quality image-text pairs to align visual embeddings with textual embeddings, thereby enabling the model to accurately describe medical visual content. During this process, the pre-trained LLM and its corresponding H-LoRA submodules remain frozen. In contrast, the medical generation task requires training concrete-grained adapters and H-LoRA submodules while keeping the LLM frozen. Meanwhile, we extend the textual vocabulary to include multimodal tokens, enabling the support of additional VQGAN vector quantization indices. The model trains on image-VQ pairs, endowing the pre-trained LLM with the capability for image reconstruction. This design ensures pixel-level consistency of pre- and post-LVLM. The processes establish the initial alignment between the LLM’s outputs and the visual inputs.

\noindent \textbf{2nd Stage: Heterogeneous H-LoRA Plugin Adaptation.}  
The submodules of H-LoRA share the word embedding layer and output head but may encounter issues such as bias and scale inconsistencies during training across different tasks. To ensure that the multiple H-LoRA plugins seamlessly interface with the LLMs and form a unified base, we fine-tune the word embedding layer and output head using a small amount of mixed data to maintain consistency in the model weights. Specifically, during this stage, all H-LoRA submodules for different tasks are kept frozen, with only the word embedding layer and output head being optimized. Through this stage, the model accumulates foundational knowledge for unified tasks by adapting H-LoRA plugins.

\begin{table*}[!t]
\centering
\caption{Comparison of \ourmethod{} with other LVLMs and unified multi-modal models on medical visual comprehension tasks. \textbf{Bold} and \underline{underlined} text indicates the best performance and second-best performance, respectively.}
\resizebox{\textwidth}{!}{
\begin{tabular}{c|lcc|cccccccc|c}
\toprule
\rowcolor[HTML]{E9F3FE} &  &  &  & \multicolumn{2}{c}{\textbf{VQA-RAD \textuparrow}} & \multicolumn{2}{c}{\textbf{SLAKE \textuparrow}} & \multicolumn{2}{c}{\textbf{PathVQA \textuparrow}} &  &  &  \\ 
\cline{5-10}
\rowcolor[HTML]{E9F3FE}\multirow{-2}{*}{\textbf{Type}} & \multirow{-2}{*}{\textbf{Model}} & \multirow{-2}{*}{\textbf{\# Params}} & \multirow{-2}{*}{\makecell{\textbf{Medical} \\ \textbf{LVLM}}} & \textbf{close} & \textbf{all} & \textbf{close} & \textbf{all} & \textbf{close} & \textbf{all} & \multirow{-2}{*}{\makecell{\textbf{MMMU} \\ \textbf{-Med}}\textuparrow} & \multirow{-2}{*}{\textbf{OMVQA}\textuparrow} & \multirow{-2}{*}{\textbf{Avg. \textuparrow}} \\ 
\midrule \midrule
\multirow{9}{*}{\textbf{Comp. Only}} 
& Med-Flamingo & 8.3B & \Large \ding{51} & 58.6 & 43.0 & 47.0 & 25.5 & 61.9 & 31.3 & 28.7 & 34.9 & 41.4 \\
& LLaVA-Med & 7B & \Large \ding{51} & 60.2 & 48.1 & 58.4 & 44.8 & 62.3 & 35.7 & 30.0 & 41.3 & 47.6 \\
& HuatuoGPT-Vision & 7B & \Large \ding{51} & 66.9 & 53.0 & 59.8 & 49.1 & 52.9 & 32.0 & 42.0 & 50.0 & 50.7 \\
& BLIP-2 & 6.7B & \Large \ding{55} & 43.4 & 36.8 & 41.6 & 35.3 & 48.5 & 28.8 & 27.3 & 26.9 & 36.1 \\
& LLaVA-v1.5 & 7B & \Large \ding{55} & 51.8 & 42.8 & 37.1 & 37.7 & 53.5 & 31.4 & 32.7 & 44.7 & 41.5 \\
& InstructBLIP & 7B & \Large \ding{55} & 61.0 & 44.8 & 66.8 & 43.3 & 56.0 & 32.3 & 25.3 & 29.0 & 44.8 \\
& Yi-VL & 6B & \Large \ding{55} & 52.6 & 42.1 & 52.4 & 38.4 & 54.9 & 30.9 & 38.0 & 50.2 & 44.9 \\
& InternVL2 & 8B & \Large \ding{55} & 64.9 & 49.0 & 66.6 & 50.1 & 60.0 & 31.9 & \underline{43.3} & 54.5 & 52.5\\
& Llama-3.2 & 11B & \Large \ding{55} & 68.9 & 45.5 & 72.4 & 52.1 & 62.8 & 33.6 & 39.3 & 63.2 & 54.7 \\
\midrule
\multirow{5}{*}{\textbf{Comp. \& Gen.}} 
& Show-o & 1.3B & \Large \ding{55} & 50.6 & 33.9 & 31.5 & 17.9 & 52.9 & 28.2 & 22.7 & 45.7 & 42.6 \\
& Unified-IO 2 & 7B & \Large \ding{55} & 46.2 & 32.6 & 35.9 & 21.9 & 52.5 & 27.0 & 25.3 & 33.0 & 33.8 \\
& Janus & 1.3B & \Large \ding{55} & 70.9 & 52.8 & 34.7 & 26.9 & 51.9 & 27.9 & 30.0 & 26.8 & 33.5 \\
& \cellcolor[HTML]{DAE0FB}HealthGPT-M3 & \cellcolor[HTML]{DAE0FB}3.8B & \cellcolor[HTML]{DAE0FB}\Large \ding{51} & \cellcolor[HTML]{DAE0FB}\underline{73.7} & \cellcolor[HTML]{DAE0FB}\underline{55.9} & \cellcolor[HTML]{DAE0FB}\underline{74.6} & \cellcolor[HTML]{DAE0FB}\underline{56.4} & \cellcolor[HTML]{DAE0FB}\underline{78.7} & \cellcolor[HTML]{DAE0FB}\underline{39.7} & \cellcolor[HTML]{DAE0FB}\underline{43.3} & \cellcolor[HTML]{DAE0FB}\underline{68.5} & \cellcolor[HTML]{DAE0FB}\underline{61.3} \\
& \cellcolor[HTML]{DAE0FB}HealthGPT-L14 & \cellcolor[HTML]{DAE0FB}14B & \cellcolor[HTML]{DAE0FB}\Large \ding{51} & \cellcolor[HTML]{DAE0FB}\textbf{77.7} & \cellcolor[HTML]{DAE0FB}\textbf{58.3} & \cellcolor[HTML]{DAE0FB}\textbf{76.4} & \cellcolor[HTML]{DAE0FB}\textbf{64.5} & \cellcolor[HTML]{DAE0FB}\textbf{85.9} & \cellcolor[HTML]{DAE0FB}\textbf{44.4} & \cellcolor[HTML]{DAE0FB}\textbf{49.2} & \cellcolor[HTML]{DAE0FB}\textbf{74.4} & \cellcolor[HTML]{DAE0FB}\textbf{66.4} \\
\bottomrule
\end{tabular}
}
\label{tab:results}
\end{table*}
\begin{table*}[ht]
    \centering
    \caption{The experimental results for the four modality conversion tasks.}
    \resizebox{\textwidth}{!}{
    \begin{tabular}{l|ccc|ccc|ccc|ccc}
        \toprule
        \rowcolor[HTML]{E9F3FE} & \multicolumn{3}{c}{\textbf{CT to MRI (Brain)}} & \multicolumn{3}{c}{\textbf{CT to MRI (Pelvis)}} & \multicolumn{3}{c}{\textbf{MRI to CT (Brain)}} & \multicolumn{3}{c}{\textbf{MRI to CT (Pelvis)}} \\
        \cline{2-13}
        \rowcolor[HTML]{E9F3FE}\multirow{-2}{*}{\textbf{Model}}& \textbf{SSIM $\uparrow$} & \textbf{PSNR $\uparrow$} & \textbf{MSE $\downarrow$} & \textbf{SSIM $\uparrow$} & \textbf{PSNR $\uparrow$} & \textbf{MSE $\downarrow$} & \textbf{SSIM $\uparrow$} & \textbf{PSNR $\uparrow$} & \textbf{MSE $\downarrow$} & \textbf{SSIM $\uparrow$} & \textbf{PSNR $\uparrow$} & \textbf{MSE $\downarrow$} \\
        \midrule \midrule
        pix2pix & 71.09 & 32.65 & 36.85 & 59.17 & 31.02 & 51.91 & 78.79 & 33.85 & 28.33 & 72.31 & 32.98 & 36.19 \\
        CycleGAN & 54.76 & 32.23 & 40.56 & 54.54 & 30.77 & 55.00 & 63.75 & 31.02 & 52.78 & 50.54 & 29.89 & 67.78 \\
        BBDM & {71.69} & {32.91} & {34.44} & 57.37 & 31.37 & 48.06 & \textbf{86.40} & 34.12 & 26.61 & {79.26} & 33.15 & 33.60 \\
        Vmanba & 69.54 & 32.67 & 36.42 & {63.01} & {31.47} & {46.99} & 79.63 & 34.12 & 26.49 & 77.45 & 33.53 & 31.85 \\
        DiffMa & 71.47 & 32.74 & 35.77 & 62.56 & 31.43 & 47.38 & 79.00 & {34.13} & {26.45} & 78.53 & {33.68} & {30.51} \\
        \rowcolor[HTML]{DAE0FB}HealthGPT-M3 & \underline{79.38} & \underline{33.03} & \underline{33.48} & \underline{71.81} & \underline{31.83} & \underline{43.45} & {85.06} & \textbf{34.40} & \textbf{25.49} & \underline{84.23} & \textbf{34.29} & \textbf{27.99} \\
        \rowcolor[HTML]{DAE0FB}HealthGPT-L14 & \textbf{79.73} & \textbf{33.10} & \textbf{32.96} & \textbf{71.92} & \textbf{31.87} & \textbf{43.09} & \underline{85.31} & \underline{34.29} & \underline{26.20} & \textbf{84.96} & \underline{34.14} & \underline{28.13} \\
        \bottomrule
    \end{tabular}
    }
    \label{tab:conversion}
\end{table*}

\noindent \textbf{3rd Stage: Visual Instruction Fine-Tuning.}  
In the third stage, we introduce additional task-specific data to further optimize the model and enhance its adaptability to downstream tasks such as medical visual comprehension (e.g., medical QA, medical dialogues, and report generation) or generation tasks (e.g., super-resolution, denoising, and modality conversion). Notably, by this stage, the word embedding layer and output head have been fine-tuned, only the H-LoRA modules and adapter modules need to be trained. This strategy significantly improves the model's adaptability and flexibility across different tasks.


\section{Experiment}
\label{s:experiment}

\subsection{Data Description}
We evaluate our method on FI~\cite{you2016building}, Twitter\_LDL~\cite{yang2017learning} and Artphoto~\cite{machajdik2010affective}.
FI is a public dataset built from Flickr and Instagram, with 23,308 images and eight emotion categories, namely \textit{amusement}, \textit{anger}, \textit{awe},  \textit{contentment}, \textit{disgust}, \textit{excitement},  \textit{fear}, and \textit{sadness}. 
% Since images in FI are all copyrighted by law, some images are corrupted now, so we remove these samples and retain 21,828 images.
% T4SA contains images from Twitter, which are classified into three categories: \textit{positive}, \textit{neutral}, and \textit{negative}. In this paper, we adopt the base version of B-T4SA, which contains 470,586 images and provides text descriptions of the corresponding tweets.
Twitter\_LDL contains 10,045 images from Twitter, with the same eight categories as the FI dataset.
% 。
For these two datasets, they are randomly split into 80\%
training and 20\% testing set.
Artphoto contains 806 artistic photos from the DeviantArt website, which we use to further evaluate the zero-shot capability of our model.
% on the small-scale dataset.
% We construct and publicly release the first image sentiment analysis dataset containing metadata.
% 。

% Based on these datasets, we are the first to construct and publicly release metadata-enhanced image sentiment analysis datasets. These datasets include scenes, tags, descriptions, and corresponding confidence scores, and are available at this link for future research purposes.


% 
\begin{table}[t]
\centering
% \begin{center}
\caption{Overall performance of different models on FI and Twitter\_LDL datasets.}
\label{tab:cap1}
% \resizebox{\linewidth}{!}
{
\begin{tabular}{l|c|c|c|c}
\hline
\multirow{2}{*}{\textbf{Model}} & \multicolumn{2}{c|}{\textbf{FI}}  & \multicolumn{2}{c}{\textbf{Twitter\_LDL}} \\ \cline{2-5} 
  & \textbf{Accuracy} & \textbf{F1} & \textbf{Accuracy} & \textbf{F1}  \\ \hline
% (\rownumber)~AlexNet~\cite{krizhevsky2017imagenet}  & 58.13\% & 56.35\%  & 56.24\%& 55.02\%  \\ 
% (\rownumber)~VGG16~\cite{simonyan2014very}  & 63.75\%& 63.08\%  & 59.34\%& 59.02\%  \\ 
(\rownumber)~ResNet101~\cite{he2016deep} & 66.16\%& 65.56\%  & 62.02\% & 61.34\%  \\ 
(\rownumber)~CDA~\cite{han2023boosting} & 66.71\%& 65.37\%  & 64.14\% & 62.85\%  \\ 
(\rownumber)~CECCN~\cite{ruan2024color} & 67.96\%& 66.74\%  & 64.59\%& 64.72\% \\ 
(\rownumber)~EmoVIT~\cite{xie2024emovit} & 68.09\%& 67.45\%  & 63.12\% & 61.97\%  \\ 
(\rownumber)~ComLDL~\cite{zhang2022compound} & 68.83\%& 67.28\%  & 65.29\% & 63.12\%  \\ 
(\rownumber)~WSDEN~\cite{li2023weakly} & 69.78\%& 69.61\%  & 67.04\% & 65.49\% \\ 
(\rownumber)~ECWA~\cite{deng2021emotion} & 70.87\%& 69.08\%  & 67.81\% & 66.87\%  \\ 
(\rownumber)~EECon~\cite{yang2023exploiting} & 71.13\%& 68.34\%  & 64.27\%& 63.16\%  \\ 
(\rownumber)~MAM~\cite{zhang2024affective} & 71.44\%  & 70.83\% & 67.18\%  & 65.01\%\\ 
(\rownumber)~TGCA-PVT~\cite{chen2024tgca}   & 73.05\%  & 71.46\% & 69.87\%  & 68.32\% \\ 
(\rownumber)~OEAN~\cite{zhang2024object}   & 73.40\%  & 72.63\% & 70.52\%  & 69.47\% \\ \hline
(\rownumber)~\shortname  & \textbf{79.48\%} & \textbf{79.22\%} & \textbf{74.12\%} & \textbf{73.09\%} \\ \hline
\end{tabular}
}
\vspace{-6mm}
% \end{center}
\end{table}
% 

\subsection{Experiment Setting}
% \subsubsection{Model Setting.}
% 
\textbf{Model Setting:}
For feature representation, we set $k=10$ to select object tags, and adopt clip-vit-base-patch32 as the pre-trained model for unified feature representation.
Moreover, we empirically set $(d_e, d_h, d_k, d_s) = (512, 128, 16, 64)$, and set the classification class $L$ to 8.

% 

\textbf{Training Setting:}
To initialize the model, we set all weights such as $\boldsymbol{W}$ following the truncated normal distribution, and use AdamW optimizer with the learning rate of $1 \times 10^{-4}$.
% warmup scheduler of cosine, warmup steps of 2000.
Furthermore, we set the batch size to 32 and the epoch of the training process to 200.
During the implementation, we utilize \textit{PyTorch} to build our entire model.
% , and our project codes are publicly available at https://github.com/zzmyrep/MESN.
% Our project codes as well as data are all publicly available on GitHub\footnote{https://github.com/zzmyrep/KBCEN}.
% Code is available at \href{https://github.com/zzmyrep/KBCEN}{https://github.com/zzmyrep/KBCEN}.

\textbf{Evaluation Metrics:}
Following~\cite{zhang2024affective, chen2024tgca, zhang2024object}, we adopt \textit{accuracy} and \textit{F1} as our evaluation metrics to measure the performance of different methods for image sentiment analysis. 



\subsection{Experiment Result}
% We compare our model against the following baselines: AlexNet~\cite{krizhevsky2017imagenet}, VGG16~\cite{simonyan2014very}, ResNet101~\cite{he2016deep}, CECCN~\cite{ruan2024color}, EmoVIT~\cite{xie2024emovit}, WSCNet~\cite{yang2018weakly}, ECWA~\cite{deng2021emotion}, EECon~\cite{yang2023exploiting}, MAM~\cite{zhang2024affective} and TGCA-PVT~\cite{chen2024tgca}, and the overall results are summarized in Table~\ref{tab:cap1}.
We compare our model against several baselines, and the overall results are summarized in Table~\ref{tab:cap1}.
We observe that our model achieves the best performance in both accuracy and F1 metrics, significantly outperforming the previous models. 
This superior performance is mainly attributed to our effective utilization of metadata to enhance image sentiment analysis, as well as the exceptional capability of the unified sentiment transformer framework we developed. These results strongly demonstrate that our proposed method can bring encouraging performance for image sentiment analysis.

\setcounter{magicrownumbers}{0} 
\begin{table}[t]
\begin{center}
\caption{Ablation study of~\shortname~on FI dataset.} 
% \vspace{1mm}
\label{tab:cap2}
\resizebox{.9\linewidth}{!}
{
\begin{tabular}{lcc}
  \hline
  \textbf{Model} & \textbf{Accuracy} & \textbf{F1} \\
  \hline
  (\rownumber)~Ours (w/o vision) & 65.72\% & 64.54\% \\
  (\rownumber)~Ours (w/o text description) & 74.05\% & 72.58\% \\
  (\rownumber)~Ours (w/o object tag) & 77.45\% & 76.84\% \\
  (\rownumber)~Ours (w/o scene tag) & 78.47\% & 78.21\% \\
  \hline
  (\rownumber)~Ours (w/o unified embedding) & 76.41\% & 76.23\% \\
  (\rownumber)~Ours (w/o adaptive learning) & 76.83\% & 76.56\% \\
  (\rownumber)~Ours (w/o cross-modal fusion) & 76.85\% & 76.49\% \\
  \hline
  (\rownumber)~Ours  & \textbf{79.48\%} & \textbf{79.22\%} \\
  \hline
\end{tabular}
}
\end{center}
\vspace{-5mm}
\end{table}


\begin{figure}[t]
\centering
% \vspace{-2mm}
\includegraphics[width=0.42\textwidth]{fig/2dvisual-linux4-paper2.pdf}
\caption{Visualization of feature distribution on eight categories before (left) and after (right) model processing.}
% 
\label{fig:visualization}
\vspace{-5mm}
\end{figure}

\subsection{Ablation Performance}
In this subsection, we conduct an ablation study to examine which component is really important for performance improvement. The results are reported in Table~\ref{tab:cap2}.

For information utilization, we observe a significant decline in model performance when visual features are removed. Additionally, the performance of \shortname~decreases when different metadata are removed separately, which means that text description, object tag, and scene tag are all critical for image sentiment analysis.
Recalling the model architecture, we separately remove transformer layers of the unified representation module, the adaptive learning module, and the cross-modal fusion module, replacing them with MLPs of the same parameter scale.
In this way, we can observe varying degrees of decline in model performance, indicating that these modules are indispensable for our model to achieve better performance.

\subsection{Visualization}
% 


% % 开始使用minipage进行左右排列
% \begin{minipage}[t]{0.45\textwidth}  % 子图1宽度为45%
%     \centering
%     \includegraphics[width=\textwidth]{2dvisual.pdf}  % 插入图片
%     \captionof{figure}{Visualization of feature distribution.}  % 使用captionof添加图片标题
%     \label{fig:visualization}
% \end{minipage}


% \begin{figure}[t]
% \centering
% \vspace{-2mm}
% \includegraphics[width=0.45\textwidth]{fig/2dvisual.pdf}
% \caption{Visualization of feature distribution.}
% \label{fig:visualization}
% % \vspace{-4mm}
% \end{figure}

% \begin{figure}[t]
% \centering
% \vspace{-2mm}
% \includegraphics[width=0.45\textwidth]{fig/2dvisual-linux3-paper.pdf}
% \caption{Visualization of feature distribution.}
% \label{fig:visualization}
% % \vspace{-4mm}
% \end{figure}



\begin{figure}[tbp]   
\vspace{-4mm}
  \centering            
  \subfloat[Depth of adaptive learning layers]   
  {
    \label{fig:subfig1}\includegraphics[width=0.22\textwidth]{fig/fig_sensitivity-a5}
  }
  \subfloat[Depth of fusion layers]
  {
    % \label{fig:subfig2}\includegraphics[width=0.22\textwidth]{fig/fig_sensitivity-b2}
    \label{fig:subfig2}\includegraphics[width=0.22\textwidth]{fig/fig_sensitivity-b2-num.pdf}
  }
  \caption{Sensitivity study of \shortname~on different depth. }   
  \label{fig:fig_sensitivity}  
\vspace{-2mm}
\end{figure}

% \begin{figure}[htbp]
% \centerline{\includegraphics{2dvisual.pdf}}
% \caption{Visualization of feature distribution.}
% \label{fig:visualization}
% \end{figure}

% In Fig.~\ref{fig:visualization}, we use t-SNE~\cite{van2008visualizing} to reduce the dimension of data features for visualization, Figure in left represents the metadata features before model processing, the features are obtained by embedding through the CLIP model, and figure in right shows the features of the data after model processing, it can be observed that after the model processing, the data with different label categories fall in different regions in the space, therefore, we can conclude that the Therefore, we can conclude that the model can effectively utilize the information contained in the metadata and use it to guide the model for classification.

In Fig.~\ref{fig:visualization}, we use t-SNE~\cite{van2008visualizing} to reduce the dimension of data features for visualization.
The left figure shows metadata features before being processed by our model (\textit{i.e.}, embedded by CLIP), while the right shows the distribution of features after being processed by our model.
We can observe that after the model processing, data with the same label are closer to each other, while others are farther away.
Therefore, it shows that the model can effectively utilize the information contained in the metadata and use it to guide the classification process.

\subsection{Sensitivity Analysis}
% 
In this subsection, we conduct a sensitivity analysis to figure out the effect of different depth settings of adaptive learning layers and fusion layers. 
% In this subsection, we conduct a sensitivity analysis to figure out the effect of different depth settings on the model. 
% Fig.~\ref{fig:fig_sensitivity} presents the effect of different depth settings of adaptive learning layers and fusion layers. 
Taking Fig.~\ref{fig:fig_sensitivity} (a) as an example, the model performance improves with increasing depth, reaching the best performance at a depth of 4.
% Taking Fig.~\ref{fig:fig_sensitivity} (a) as an example, the performance of \shortname~improves with the increase of depth at first, reaching the best performance at a depth of 4.
When the depth continues to increase, the accuracy decreases to varying degrees.
Similar results can be observed in Fig.~\ref{fig:fig_sensitivity} (b).
Therefore, we set their depths to 4 and 6 respectively to achieve the best results.

% Through our experiments, we can observe that the effect of modifying these hyperparameters on the results of the experiments is very weak, and the surface model is not sensitive to the hyperparameters.


\subsection{Zero-shot Capability}
% 

% (1)~GCH~\cite{2010Analyzing} & 21.78\% & (5)~RA-DLNet~\cite{2020A} & 34.01\% \\ \hline
% (2)~WSCNet~\cite{2019WSCNet}  & 30.25\% & (6)~CECCN~\cite{ruan2024color} & 43.83\% \\ \hline
% (3)~PCNN~\cite{2015Robust} & 31.68\%  & (7)~EmoVIT~\cite{xie2024emovit} & 44.90\% \\ \hline
% (4)~AR~\cite{2018Visual} & 32.67\% & (8)~Ours (Zero-shot) & 47.83\% \\ \hline


\begin{table}[t]
\centering
\caption{Zero-shot capability of \shortname.}
\label{tab:cap3}
\resizebox{1\linewidth}{!}
{
\begin{tabular}{lc|lc}
\hline
\textbf{Model} & \textbf{Accuracy} & \textbf{Model} & \textbf{Accuracy} \\ \hline
(1)~WSCNet~\cite{2019WSCNet}  & 30.25\% & (5)~MAM~\cite{zhang2024affective} & 39.56\%  \\ \hline
(2)~AR~\cite{2018Visual} & 32.67\% & (6)~CECCN~\cite{ruan2024color} & 43.83\% \\ \hline
(3)~RA-DLNet~\cite{2020A} & 34.01\%  & (7)~EmoVIT~\cite{xie2024emovit} & 44.90\% \\ \hline
(4)~CDA~\cite{han2023boosting} & 38.64\% & (8)~Ours (Zero-shot) & 47.83\% \\ \hline
\end{tabular}
}
\vspace{-5mm}
\end{table}

% We use the model trained on the FI dataset to test on the artphoto dataset to verify the model's generalization ability as well as robustness to other distributed datasets.
% We can observe that the MESN model shows strong competitiveness in terms of accuracy when compared to other trained models, which suggests that the model has a good generalization ability in the OOD task.

To validate the model's generalization ability and robustness to other distributed datasets, we directly test the model trained on the FI dataset, without training on Artphoto. 
% As observed in Table 3, compared to other models trained on Artphoto, we achieve highly competitive zero-shot performance, indicating that the model has good generalization ability in out-of-distribution tasks.
From Table~\ref{tab:cap3}, we can observe that compared with other models trained on Artphoto, we achieve competitive zero-shot performance, which shows that the model has good generalization ability in out-of-distribution tasks.


%%%%%%%%%%%%
%  E2E     %
%%%%%%%%%%%%


\section{Conclusion}
In this paper, we introduced Wi-Chat, the first LLM-powered Wi-Fi-based human activity recognition system that integrates the reasoning capabilities of large language models with the sensing potential of wireless signals. Our experimental results on a self-collected Wi-Fi CSI dataset demonstrate the promising potential of LLMs in enabling zero-shot Wi-Fi sensing. These findings suggest a new paradigm for human activity recognition that does not rely on extensive labeled data. We hope future research will build upon this direction, further exploring the applications of LLMs in signal processing domains such as IoT, mobile sensing, and radar-based systems.

\section*{Limitations}
While our work represents the first attempt to leverage LLMs for processing Wi-Fi signals, it is a preliminary study focused on a relatively simple task: Wi-Fi-based human activity recognition. This choice allows us to explore the feasibility of LLMs in wireless sensing but also comes with certain limitations.

Our approach primarily evaluates zero-shot performance, which, while promising, may still lag behind traditional supervised learning methods in highly complex or fine-grained recognition tasks. Besides, our study is limited to a controlled environment with a self-collected dataset, and the generalizability of LLMs to diverse real-world scenarios with varying Wi-Fi conditions, environmental interference, and device heterogeneity remains an open question.

Additionally, we have yet to explore the full potential of LLMs in more advanced Wi-Fi sensing applications, such as fine-grained gesture recognition, occupancy detection, and passive health monitoring. Future work should investigate the scalability of LLM-based approaches, their robustness to domain shifts, and their integration with multimodal sensing techniques in broader IoT applications.


% Bibliography entries for the entire Anthology, followed by custom entries
%\bibliography{anthology,custom}
% Custom bibliography entries only
\bibliography{main}
\newpage
\appendix

\section{Experiment prompts}
\label{sec:prompt}
The prompts used in the LLM experiments are shown in the following Table~\ref{tab:prompts}.

\definecolor{titlecolor}{rgb}{0.9, 0.5, 0.1}
\definecolor{anscolor}{rgb}{0.2, 0.5, 0.8}
\definecolor{labelcolor}{HTML}{48a07e}
\begin{table*}[h]
	\centering
	
 % \vspace{-0.2cm}
	
	\begin{center}
		\begin{tikzpicture}[
				chatbox_inner/.style={rectangle, rounded corners, opacity=0, text opacity=1, font=\sffamily\scriptsize, text width=5in, text height=9pt, inner xsep=6pt, inner ysep=6pt},
				chatbox_prompt_inner/.style={chatbox_inner, align=flush left, xshift=0pt, text height=11pt},
				chatbox_user_inner/.style={chatbox_inner, align=flush left, xshift=0pt},
				chatbox_gpt_inner/.style={chatbox_inner, align=flush left, xshift=0pt},
				chatbox/.style={chatbox_inner, draw=black!25, fill=gray!7, opacity=1, text opacity=0},
				chatbox_prompt/.style={chatbox, align=flush left, fill=gray!1.5, draw=black!30, text height=10pt},
				chatbox_user/.style={chatbox, align=flush left},
				chatbox_gpt/.style={chatbox, align=flush left},
				chatbox2/.style={chatbox_gpt, fill=green!25},
				chatbox3/.style={chatbox_gpt, fill=red!20, draw=black!20},
				chatbox4/.style={chatbox_gpt, fill=yellow!30},
				labelbox/.style={rectangle, rounded corners, draw=black!50, font=\sffamily\scriptsize\bfseries, fill=gray!5, inner sep=3pt},
			]
											
			\node[chatbox_user] (q1) {
				\textbf{System prompt}
				\newline
				\newline
				You are a helpful and precise assistant for segmenting and labeling sentences. We would like to request your help on curating a dataset for entity-level hallucination detection.
				\newline \newline
                We will give you a machine generated biography and a list of checked facts about the biography. Each fact consists of a sentence and a label (True/False). Please do the following process. First, breaking down the biography into words. Second, by referring to the provided list of facts, merging some broken down words in the previous step to form meaningful entities. For example, ``strategic thinking'' should be one entity instead of two. Third, according to the labels in the list of facts, labeling each entity as True or False. Specifically, for facts that share a similar sentence structure (\eg, \textit{``He was born on Mach 9, 1941.''} (\texttt{True}) and \textit{``He was born in Ramos Mejia.''} (\texttt{False})), please first assign labels to entities that differ across atomic facts. For example, first labeling ``Mach 9, 1941'' (\texttt{True}) and ``Ramos Mejia'' (\texttt{False}) in the above case. For those entities that are the same across atomic facts (\eg, ``was born'') or are neutral (\eg, ``he,'' ``in,'' and ``on''), please label them as \texttt{True}. For the cases that there is no atomic fact that shares the same sentence structure, please identify the most informative entities in the sentence and label them with the same label as the atomic fact while treating the rest of the entities as \texttt{True}. In the end, output the entities and labels in the following format:
                \begin{itemize}[nosep]
                    \item Entity 1 (Label 1)
                    \item Entity 2 (Label 2)
                    \item ...
                    \item Entity N (Label N)
                \end{itemize}
                % \newline \newline
                Here are two examples:
                \newline\newline
                \textbf{[Example 1]}
                \newline
                [The start of the biography]
                \newline
                \textcolor{titlecolor}{Marianne McAndrew is an American actress and singer, born on November 21, 1942, in Cleveland, Ohio. She began her acting career in the late 1960s, appearing in various television shows and films.}
                \newline
                [The end of the biography]
                \newline \newline
                [The start of the list of checked facts]
                \newline
                \textcolor{anscolor}{[Marianne McAndrew is an American. (False); Marianne McAndrew is an actress. (True); Marianne McAndrew is a singer. (False); Marianne McAndrew was born on November 21, 1942. (False); Marianne McAndrew was born in Cleveland, Ohio. (False); She began her acting career in the late 1960s. (True); She has appeared in various television shows. (True); She has appeared in various films. (True)]}
                \newline
                [The end of the list of checked facts]
                \newline \newline
                [The start of the ideal output]
                \newline
                \textcolor{labelcolor}{[Marianne McAndrew (True); is (True); an (True); American (False); actress (True); and (True); singer (False); , (True); born (True); on (True); November 21, 1942 (False); , (True); in (True); Cleveland, Ohio (False); . (True); She (True); began (True); her (True); acting career (True); in (True); the late 1960s (True); , (True); appearing (True); in (True); various (True); television shows (True); and (True); films (True); . (True)]}
                \newline
                [The end of the ideal output]
				\newline \newline
                \textbf{[Example 2]}
                \newline
                [The start of the biography]
                \newline
                \textcolor{titlecolor}{Doug Sheehan is an American actor who was born on April 27, 1949, in Santa Monica, California. He is best known for his roles in soap operas, including his portrayal of Joe Kelly on ``General Hospital'' and Ben Gibson on ``Knots Landing.''}
                \newline
                [The end of the biography]
                \newline \newline
                [The start of the list of checked facts]
                \newline
                \textcolor{anscolor}{[Doug Sheehan is an American. (True); Doug Sheehan is an actor. (True); Doug Sheehan was born on April 27, 1949. (True); Doug Sheehan was born in Santa Monica, California. (False); He is best known for his roles in soap operas. (True); He portrayed Joe Kelly. (True); Joe Kelly was in General Hospital. (True); General Hospital is a soap opera. (True); He portrayed Ben Gibson. (True); Ben Gibson was in Knots Landing. (True); Knots Landing is a soap opera. (True)]}
                \newline
                [The end of the list of checked facts]
                \newline \newline
                [The start of the ideal output]
                \newline
                \textcolor{labelcolor}{[Doug Sheehan (True); is (True); an (True); American (True); actor (True); who (True); was born (True); on (True); April 27, 1949 (True); in (True); Santa Monica, California (False); . (True); He (True); is (True); best known (True); for (True); his roles in soap operas (True); , (True); including (True); in (True); his portrayal (True); of (True); Joe Kelly (True); on (True); ``General Hospital'' (True); and (True); Ben Gibson (True); on (True); ``Knots Landing.'' (True)]}
                \newline
                [The end of the ideal output]
				\newline \newline
				\textbf{User prompt}
				\newline
				\newline
				[The start of the biography]
				\newline
				\textcolor{magenta}{\texttt{\{BIOGRAPHY\}}}
				\newline
				[The ebd of the biography]
				\newline \newline
				[The start of the list of checked facts]
				\newline
				\textcolor{magenta}{\texttt{\{LIST OF CHECKED FACTS\}}}
				\newline
				[The end of the list of checked facts]
			};
			\node[chatbox_user_inner] (q1_text) at (q1) {
				\textbf{System prompt}
				\newline
				\newline
				You are a helpful and precise assistant for segmenting and labeling sentences. We would like to request your help on curating a dataset for entity-level hallucination detection.
				\newline \newline
                We will give you a machine generated biography and a list of checked facts about the biography. Each fact consists of a sentence and a label (True/False). Please do the following process. First, breaking down the biography into words. Second, by referring to the provided list of facts, merging some broken down words in the previous step to form meaningful entities. For example, ``strategic thinking'' should be one entity instead of two. Third, according to the labels in the list of facts, labeling each entity as True or False. Specifically, for facts that share a similar sentence structure (\eg, \textit{``He was born on Mach 9, 1941.''} (\texttt{True}) and \textit{``He was born in Ramos Mejia.''} (\texttt{False})), please first assign labels to entities that differ across atomic facts. For example, first labeling ``Mach 9, 1941'' (\texttt{True}) and ``Ramos Mejia'' (\texttt{False}) in the above case. For those entities that are the same across atomic facts (\eg, ``was born'') or are neutral (\eg, ``he,'' ``in,'' and ``on''), please label them as \texttt{True}. For the cases that there is no atomic fact that shares the same sentence structure, please identify the most informative entities in the sentence and label them with the same label as the atomic fact while treating the rest of the entities as \texttt{True}. In the end, output the entities and labels in the following format:
                \begin{itemize}[nosep]
                    \item Entity 1 (Label 1)
                    \item Entity 2 (Label 2)
                    \item ...
                    \item Entity N (Label N)
                \end{itemize}
                % \newline \newline
                Here are two examples:
                \newline\newline
                \textbf{[Example 1]}
                \newline
                [The start of the biography]
                \newline
                \textcolor{titlecolor}{Marianne McAndrew is an American actress and singer, born on November 21, 1942, in Cleveland, Ohio. She began her acting career in the late 1960s, appearing in various television shows and films.}
                \newline
                [The end of the biography]
                \newline \newline
                [The start of the list of checked facts]
                \newline
                \textcolor{anscolor}{[Marianne McAndrew is an American. (False); Marianne McAndrew is an actress. (True); Marianne McAndrew is a singer. (False); Marianne McAndrew was born on November 21, 1942. (False); Marianne McAndrew was born in Cleveland, Ohio. (False); She began her acting career in the late 1960s. (True); She has appeared in various television shows. (True); She has appeared in various films. (True)]}
                \newline
                [The end of the list of checked facts]
                \newline \newline
                [The start of the ideal output]
                \newline
                \textcolor{labelcolor}{[Marianne McAndrew (True); is (True); an (True); American (False); actress (True); and (True); singer (False); , (True); born (True); on (True); November 21, 1942 (False); , (True); in (True); Cleveland, Ohio (False); . (True); She (True); began (True); her (True); acting career (True); in (True); the late 1960s (True); , (True); appearing (True); in (True); various (True); television shows (True); and (True); films (True); . (True)]}
                \newline
                [The end of the ideal output]
				\newline \newline
                \textbf{[Example 2]}
                \newline
                [The start of the biography]
                \newline
                \textcolor{titlecolor}{Doug Sheehan is an American actor who was born on April 27, 1949, in Santa Monica, California. He is best known for his roles in soap operas, including his portrayal of Joe Kelly on ``General Hospital'' and Ben Gibson on ``Knots Landing.''}
                \newline
                [The end of the biography]
                \newline \newline
                [The start of the list of checked facts]
                \newline
                \textcolor{anscolor}{[Doug Sheehan is an American. (True); Doug Sheehan is an actor. (True); Doug Sheehan was born on April 27, 1949. (True); Doug Sheehan was born in Santa Monica, California. (False); He is best known for his roles in soap operas. (True); He portrayed Joe Kelly. (True); Joe Kelly was in General Hospital. (True); General Hospital is a soap opera. (True); He portrayed Ben Gibson. (True); Ben Gibson was in Knots Landing. (True); Knots Landing is a soap opera. (True)]}
                \newline
                [The end of the list of checked facts]
                \newline \newline
                [The start of the ideal output]
                \newline
                \textcolor{labelcolor}{[Doug Sheehan (True); is (True); an (True); American (True); actor (True); who (True); was born (True); on (True); April 27, 1949 (True); in (True); Santa Monica, California (False); . (True); He (True); is (True); best known (True); for (True); his roles in soap operas (True); , (True); including (True); in (True); his portrayal (True); of (True); Joe Kelly (True); on (True); ``General Hospital'' (True); and (True); Ben Gibson (True); on (True); ``Knots Landing.'' (True)]}
                \newline
                [The end of the ideal output]
				\newline \newline
				\textbf{User prompt}
				\newline
				\newline
				[The start of the biography]
				\newline
				\textcolor{magenta}{\texttt{\{BIOGRAPHY\}}}
				\newline
				[The ebd of the biography]
				\newline \newline
				[The start of the list of checked facts]
				\newline
				\textcolor{magenta}{\texttt{\{LIST OF CHECKED FACTS\}}}
				\newline
				[The end of the list of checked facts]
			};
		\end{tikzpicture}
        \caption{GPT-4o prompt for labeling hallucinated entities.}\label{tb:gpt-4-prompt}
	\end{center}
\vspace{-0cm}
\end{table*}
% \section{Full Experiment Results}
% \begin{table*}[th]
    \centering
    \small
    \caption{Classification Results}
    \begin{tabular}{lcccc}
        \toprule
        \textbf{Method} & \textbf{Accuracy} & \textbf{Precision} & \textbf{Recall} & \textbf{F1-score} \\
        \midrule
        \multicolumn{5}{c}{\textbf{Zero Shot}} \\
                Zero-shot E-eyes & 0.26 & 0.26 & 0.27 & 0.26 \\
        Zero-shot CARM & 0.24 & 0.24 & 0.24 & 0.24 \\
                Zero-shot SVM & 0.27 & 0.28 & 0.28 & 0.27 \\
        Zero-shot CNN & 0.23 & 0.24 & 0.23 & 0.23 \\
        Zero-shot RNN & 0.26 & 0.26 & 0.26 & 0.26 \\
DeepSeek-0shot & 0.54 & 0.61 & 0.54 & 0.52 \\
DeepSeek-0shot-COT & 0.33 & 0.24 & 0.33 & 0.23 \\
DeepSeek-0shot-Knowledge & 0.45 & 0.46 & 0.45 & 0.44 \\
Gemma2-0shot & 0.35 & 0.22 & 0.38 & 0.27 \\
Gemma2-0shot-COT & 0.36 & 0.22 & 0.36 & 0.27 \\
Gemma2-0shot-Knowledge & 0.32 & 0.18 & 0.34 & 0.20 \\
GPT-4o-mini-0shot & 0.48 & 0.53 & 0.48 & 0.41 \\
GPT-4o-mini-0shot-COT & 0.33 & 0.50 & 0.33 & 0.38 \\
GPT-4o-mini-0shot-Knowledge & 0.49 & 0.31 & 0.49 & 0.36 \\
GPT-4o-0shot & 0.62 & 0.62 & 0.47 & 0.42 \\
GPT-4o-0shot-COT & 0.29 & 0.45 & 0.29 & 0.21 \\
GPT-4o-0shot-Knowledge & 0.44 & 0.52 & 0.44 & 0.39 \\
LLaMA-0shot & 0.32 & 0.25 & 0.32 & 0.24 \\
LLaMA-0shot-COT & 0.12 & 0.25 & 0.12 & 0.09 \\
LLaMA-0shot-Knowledge & 0.32 & 0.25 & 0.32 & 0.28 \\
Mistral-0shot & 0.19 & 0.23 & 0.19 & 0.10 \\
Mistral-0shot-Knowledge & 0.21 & 0.40 & 0.21 & 0.11 \\
        \midrule
        \multicolumn{5}{c}{\textbf{4 Shot}} \\
GPT-4o-mini-4shot & 0.58 & 0.59 & 0.58 & 0.53 \\
GPT-4o-mini-4shot-COT & 0.57 & 0.53 & 0.57 & 0.50 \\
GPT-4o-mini-4shot-Knowledge & 0.56 & 0.51 & 0.56 & 0.47 \\
GPT-4o-4shot & 0.77 & 0.84 & 0.77 & 0.73 \\
GPT-4o-4shot-COT & 0.63 & 0.76 & 0.63 & 0.53 \\
GPT-4o-4shot-Knowledge & 0.72 & 0.82 & 0.71 & 0.66 \\
LLaMA-4shot & 0.29 & 0.24 & 0.29 & 0.21 \\
LLaMA-4shot-COT & 0.20 & 0.30 & 0.20 & 0.13 \\
LLaMA-4shot-Knowledge & 0.15 & 0.23 & 0.13 & 0.13 \\
Mistral-4shot & 0.02 & 0.02 & 0.02 & 0.02 \\
Mistral-4shot-Knowledge & 0.21 & 0.27 & 0.21 & 0.20 \\
        \midrule
        
        \multicolumn{5}{c}{\textbf{Suprevised}} \\
        SVM & 0.94 & 0.92 & 0.91 & 0.91 \\
        CNN & 0.98 & 0.98 & 0.97 & 0.97 \\
        RNN & 0.99 & 0.99 & 0.99 & 0.99 \\
        % \midrule
        % \multicolumn{5}{c}{\textbf{Conventional Wi-Fi-based Human Activity Recognition Systems}} \\
        E-eyes & 1.00 & 1.00 & 1.00 & 1.00 \\
        CARM & 0.98 & 0.98 & 0.98 & 0.98 \\
\midrule
 \multicolumn{5}{c}{\textbf{Vision Models}} \\
           Zero-shot SVM & 0.26 & 0.25 & 0.25 & 0.25 \\
        Zero-shot CNN & 0.26 & 0.25 & 0.26 & 0.26 \\
        Zero-shot RNN & 0.28 & 0.28 & 0.29 & 0.28 \\
        SVM & 0.99 & 0.99 & 0.99 & 0.99 \\
        CNN & 0.98 & 0.99 & 0.98 & 0.98 \\
        RNN & 0.98 & 0.99 & 0.98 & 0.98 \\
GPT-4o-mini-Vision & 0.84 & 0.85 & 0.84 & 0.84 \\
GPT-4o-mini-Vision-COT & 0.90 & 0.91 & 0.90 & 0.90 \\
GPT-4o-Vision & 0.74 & 0.82 & 0.74 & 0.73 \\
GPT-4o-Vision-COT & 0.70 & 0.83 & 0.70 & 0.68 \\
LLaMA-Vision & 0.20 & 0.23 & 0.20 & 0.09 \\
LLaMA-Vision-Knowledge & 0.22 & 0.05 & 0.22 & 0.08 \\

        \bottomrule
    \end{tabular}
    \label{full}
\end{table*}




\end{document}


\subsection{Setup}

\paragraph{Data.} To pretrain \aname models and baseline models, we employ the RedPajama~\cite{Redpajama}, which parallels the LLaMA training data across seven domains: CommonCrawl, C4, GitHub, Wikipedia, Books, ArXiv, and Stack-Exchange. This dataset comprises a 2 million tokens validation set and a 50 billion tokens training set.

\paragraph{Training.} Our experimental framework utilizes the Sheared-LLaMA codebase \cite{xiaShearedLLaMAAccelerating2023} implemented on the Composer package \cite{mosaicml2022composer}, and is executed on 8 NVIDIA A100 GPUs (80GB). The models are trained with a sequence length of 4096, employing a global batch size of 256.
% Baseline models underwent training  during the continued pre-training phases. 
\aname models are trained for 50000 steps (50B token budget). The learning rates were set at 3e-4 for all parameters. The baselines and all \aname models follow the same training setup, starting from random initialization and training on the same dataset.

\paragraph{Evaluation.} We employed the lm-evaluation-harness \cite{eval-harness} to evaluate our models. For common sense and reading comprehension tasks, we report 0-shot accuracy for SciQ \cite{sciqa}, PIQA \cite{piqa}, WinoGrande (WG) \cite{WinoGrande:conf/aaai/SakaguchiBBC20}, ARC Easy(ARC-E) \cite{clark2018think}, and 10-shot HellaSwag (Hella.) \cite{HellaSwag:conf/acl/ZellersHBFC19}, alongside 25-shot accuracy for ARC Challenge (ARC-C) \cite{arcChallenge:journals/corr/abs-1803-05457}. For continued QA and text understanding, we report 0-shot accuracy for LogiQA \cite{liu2020logiqa}, 32-shot BoolQ \cite{clark2019boolq}, and 0-shot LAMBADA (Lam.) \cite{paperno2016lambada}. All reported results are calculated with the mean and stderr of multiple experiments.

\paragraph{Baseline.} Following the architecture of LLaMA2, we constructed models at three parameter scales: 162M, 230M, and 466M, with hidden dimensions of 1024, 1536, and 2048, as shown in Table~\ref{tab:model_setting}. 
For each parameter scale, we develop three variants: 
\begin{itemize} 
\item Vanilla Transformers in LLaMA architecture~\cite{touvronLlamaOpenFoundation2023}. \vspace{-2mm}
\item The Loop Neural Network design~\cite{ng2024loopneuralnetworksparameter,dehghani2019universaltransformers} implements recurrence for iterative refinement. \vspace{-2mm}
% We replace every other layer of the original model with a Loop Neural Network layer.
\item Our \aname architecture, adaptively selecting a subset of tokens for deeper thinking.
% inserts multiple \aname layers into the Transformer, which Improved depth expansion is achieved through residual thinking connections and positional encoding.
\end{itemize}
We experiment with three thinking step scaling factors—$2\times$, $3\times$ and $4\times$.
% —to showcase \aname's ability to enhance model capacity and maintaining model efficiency.
% All models are the same initialized and pre-trained on 50 billion tokens.
We replace every other layer of original model with a Loop or \aname layer.




\subsection{Result}

\begin{figure*}[t]
\begin{minipage}[c]{\textwidth}
\centering
% \subfloat[MHA to DHA Transformation]{
  \includegraphics[width=\textwidth]{figs/3_figs0.pdf}
  \caption{ \textbf{Left:} Loss curves for 162M-models pre-trained on 50B tokens. \textbf{Middle:} Eval Perplexity curves for 162M-models pre-trained on 50B tokens. \textbf{Right:} Eval Perplexity for 230M-models with Training FLOPs.}
  \label{fig:exp_3figs}
% }
\end{minipage}
\vspace{-2mm}
\end{figure*}
\begin{figure}[t]  
\vspace{-2mm}
\centering  
\includegraphics[width=6cm]{figs/step.pdf}
\vspace{-3mm}
\caption{
Average accuracy after training 50B tokens for the ITT and Loop models (162M, 230M, 460M) under different thinking step configurations.
% Perplexity vs. FLOPs for different selection strategies. Lower left region indicates better performance-efficiency balance. 
% Dashed line shows overall tradeoff trend.
% Optimal \configurations achieve 10.26 perplexity at 3.66B FLOPs.
}
\vspace{-3mm}
% \cyl{using FFN as exmaple}
\label{fig:elastic_scaling}
\end{figure}

\begin{figure*}[t]
\begin{minipage}[c]{\textwidth}
\centering
% \subfloat[MHA to DHA Transformation]{
  \includegraphics[width=\textwidth]{figs/3-figs-2.pdf}
  \caption{ \textbf{Left:} 
  Perplexity vs. FLOPs for different selection strategies. Lower left region indicates better performance-efficiency balance.
  % Average score with activated parameters, with point size representing total parameters. 
  \textbf{Middle:} The average weights by the learned Thinking Step Encoding in the ITT x4 model (230M, 460M) across different thinking steps.
  \textbf{Right:} 3-2 step's Router Weight Distribution in ITT $\times$4.}\label{fig:res_3figs}
% }
\end{minipage}
\vspace{-2mm}
\end{figure*}

\paragraph{Foundational Capabilities.}
Table~\ref{table:main-results} shows the performance improvements of \aname (pink) and Loop (blue) on LLaMA 2's 162M, 230M, and 466M versions. Both methods enhance model performance by increasing computational allocation during training and inference without expanding parameters. \textbf{Thanks to its unique RTC design, \aname achieves better test-time scaling performance} than Loop, as shown in Figure~\ref{fig:elastic_scaling}. For example, the 162M \aname $\times$4 configuration improves the baseline by 1.7\% with 4-step deep thinking in 50\% of layers, while Loop improves only by 0.3\% after 4 iterations. \textbf{The advantages of \aname become clearer as model scale increases}, with improvements of 1.7\%, 2.1\%, and 1.7\% for the 162M, 230M, and 466M models. \aname shows overall enhancement across nearly all metrics, with notable improvements in ARC-E, BoolQ, and LAMBADA, reflecting gains in generative and reasoning abilities.
\paragraph{Convergence.} Figure~\ref{fig:exp_3figs} Left and Middle visualize the training loss and eval perplexity during 50B-token pre-training for LLaMA 2-2162M, Loop$\times$4, and \aname$\times$4. \textbf{\aname demonstrates superior training stability and efficiency}, with smoother, lower perplexity trajectories compared to LLaMA 2-230M and Loop. Notably, \aname$\times$4 shows a 0.09 loss reduction compared to baseline and 0.4 to Loop at 50B tokens. \textbf{\aname also reveals remarkable data efficiency}: it matches LLaMA 2-162M's performance using only 56.8\% of the training data, showcasing its capability in parameter-efficient scaling and data-efficient learning.

\paragraph{Computational Efficiency.} As shown in Figure~\ref{fig:exp_3figs} (Right), Figure~\ref{fig:res_3figs} (Left), and Table~\ref{table:main-results}, \textbf{\aname maintains high computational efficiency during test-time scaling}. With 3-step deep thinking, \aname incurs only 84\% of Loop's computational cost, dropping to 70\% at 4 steps. Remarkably, \textbf{\aname outperforms Loop with fewer computational FLOPs}, achieving performance similar to models with more parameters. Our experiments show that \aname$\times$2 outperforms Loop$\times$3 while using only 72\% of the computation and exceeds the 230M Dense model with just 70.4\% of the parameters. These results highlight the substantial computational efficiency gains from token-wise selective inner thinking in the \aname framework.

\paragraph{Elastic Thinking.} Our experiments show that \textbf{\aname models can elastically allocate computations for inner thinking.} As seen in Table~\ref{tab:elastic_main}, with 4-step thinking and 70\% token participation during training, we can flexibly adjust token selections to enhance performance (e.g., 10.21 PPL in the 70\%, 70\%, 90\% setting, 0.31 PPL lower than the training config), or reduce token selections to lower costs with no performance loss (e.g., 10.47 PPL in the 50\%, 50\%, 50\% setting). We can even \textit{remove a thinking step while maintaining near-identical results} to the training configuration. Figure~\ref{fig:res_3figs} Left shows the FLOPs and Eval PPL of ITT's elastic inference. Compared to the baselines, ITT achieves a performance-efficiency balance, with the dashed line illustrating \textbf{the near-linear tradeoff trend of ITT during testing}. ITT’s elastic thinking enables flexible deployment in diverse scenarios.


\begin{table}[t]
\centering
\small
\setlength{\tabcolsep}{2.5mm}{
\begin{tabular}{@{}lcc@{}}
\toprule
\textbf{Method - Select Ratio in Steps} & \textbf{FLOPs} & \textbf{Perplexity} $\downarrow$\\
\midrule
LLaMA2-162M & 1.88 & 11.13 \\
\midrule
\aname×4 - 90\%,~90\%,~90\% & 4.42 & 10.27 (\textcolor{green!70!black}{-0.86}) \\
\aname×4 - 90\%,~90\%,~0\% & 3.57 & 10.40 (\textcolor{green!70!black}{-0.73}) \\
\aname×4 - 90\%,~0\%,~90\% & 3.57 & 10.36 (\textcolor{green!70!black}{-0.77}) \\
\aname×4 - 0\%,~90\%,~90\% & 3.57 & 10.56 (\textcolor{green!70!black}{-0.57}) \\
\aname×4 - 90\%,~70\%,~90\% & 4.23 & 10.23 (\textcolor{green!70!black}{-0.90}) \\
\aname×4 - 70\%,~70\%,~90\% & 4.04 & 10.21 (\textcolor{green!70!black}{-0.92}) \\
\midrule
\aname×4 - 70\%,~70\%,~70\%$^\dagger$  & 3.85 & 10.52 (\textcolor{green!70!black}{-0.61}) \\
\midrule
\aname×4 - 70\%,~70\%,~50\% & 3.66 & 10.26 (\textcolor{green!70!black}{-0.87}) \\
\aname×4 - 70\%,~50\%,~50\% & 3.47 & 10.34 (\textcolor{green!70!black}{-0.79}) \\
% \midrule
\aname×4 - 50\%,~50\%,~50\% & 3.29 & 10.47 (\textcolor{green!70!black}{-0.66}) \\
\midrule
Loop×4 - 100\%,~100\%,~100\%$^\dagger$  & 4.70 & 10.78 (\textcolor{green!70!black}{-0.35}) \\
\bottomrule
\end{tabular}
}

\caption{Eval Perplexity with different token selection ratios for extended 3-steps thinking. $^\dagger$ refers to the model's training configuration.}\label{tab:elastic_main}
\vspace{-2mm}
\end{table}

% \begin{table}[!t]
% \centering
% \scalebox{0.68}{
%     \begin{tabular}{ll cccc}
%       \toprule
%       & \multicolumn{4}{c}{\textbf{Intellipro Dataset}}\\
%       & \multicolumn{2}{c}{Rank Resume} & \multicolumn{2}{c}{Rank Job} \\
%       \cmidrule(lr){2-3} \cmidrule(lr){4-5} 
%       \textbf{Method}
%       &  Recall@100 & nDCG@100 & Recall@10 & nDCG@10 \\
%       \midrule
%       \confitold{}
%       & 71.28 &34.79 &76.50 &52.57 
%       \\
%       \cmidrule{2-5}
%       \confitsimple{}
%     & 82.53 &48.17
%        & 85.58 &64.91
     
%        \\
%        +\RunnerUpMiningShort{}
%     &85.43 &50.99 &91.38 &71.34 
%       \\
%       +\HyReShort
%         &- & -
%        &-&-\\
       
%       \bottomrule

%     \end{tabular}
%   }
% \caption{Ablation studies using Jina-v2-base as the encoder. ``\confitsimple{}'' refers using a simplified encoder architecture. \framework{} trains \confitsimple{} with \RunnerUpMiningShort{} and \HyReShort{}.}
% \label{tbl:ablation}
% \end{table}
\begin{table*}[!t]
\centering
\scalebox{0.75}{
    \begin{tabular}{l cccc cccc}
      \toprule
      & \multicolumn{4}{c}{\textbf{Recruiting Dataset}}
      & \multicolumn{4}{c}{\textbf{AliYun Dataset}}\\
      & \multicolumn{2}{c}{Rank Resume} & \multicolumn{2}{c}{Rank Job} 
      & \multicolumn{2}{c}{Rank Resume} & \multicolumn{2}{c}{Rank Job}\\
      \cmidrule(lr){2-3} \cmidrule(lr){4-5} 
      \cmidrule(lr){6-7} \cmidrule(lr){8-9} 
      \textbf{Method}
      & Recall@100 & nDCG@100 & Recall@10 & nDCG@10
      & Recall@100 & nDCG@100 & Recall@10 & nDCG@10\\
      \midrule
      \confitold{}
      & 71.28 & 34.79 & 76.50 & 52.57 
      & 87.81 & 65.06 & 72.39 & 56.12
      \\
      \cmidrule{2-9}
      \confitsimple{}
      & 82.53 & 48.17 & 85.58 & 64.91
      & 94.90&78.40 & 78.70& 65.45
       \\
      +\HyReShort{}
       &85.28 & 49.50
       &90.25 & 70.22
       & 96.62&81.99 & \textbf{81.16}& 67.63
       \\
      +\RunnerUpMiningShort{}
       % & 85.14& 49.82
       % &90.75&72.51
       & \textbf{86.13}&\textbf{51.90} & \textbf{94.25}&\textbf{73.32}
       & \textbf{97.07}&\textbf{83.11} & 80.49& \textbf{68.02}
       \\
   %     +\RunnerUpMiningShort{}
   %    & 85.43 & 50.99 & 91.38 & 71.34 
   %    & 96.24 & 82.95 & 80.12 & 66.96
   %    \\
   %    +\HyReShort{} old
   %     &85.28 & 49.50
   %     &90.25 & 70.22
   %     & 96.62&81.99 & 81.16& 67.63
   %     \\
   % +\HyReShort{} 
   %     % & 85.14& 49.82
   %     % &90.75&72.51
   %     & 86.83&51.77 &92.00 &72.04
   %     & 97.07&83.11 & 80.49& 68.02
   %     \\
      \bottomrule

    \end{tabular}
  }
\caption{\framework{} ablation studies. ``\confitsimple{}'' refers using a simplified encoder architecture. \framework{} trains \confitsimple{} with \RunnerUpMiningShort{} and \HyReShort{}. We use Jina-v2-base as the encoder due to its better performance.
}
\label{tbl:ablation}
\end{table*}
\subsection{Ablation Studies}

In Table~\ref{tab:abla}, we compare the ablation results of \aname×4 with 162M parameters to the baseline under zero-shot pretraining on 50B tokens, based on Eval PPL. The specific analysis is as follows:


\paragraph{Residual Thinking Connection.} Removing this core mechanism causes the largest performance drop (+0.77 PPL), validating our hypothesis about multi-step reasoning. The residual accumulation enables iterative refinement of token representations, particularly crucial for processing linguistically complex patterns. Without RTC, the model may also lose the ability for elastic computation.
% loses its capacity to reinforce key tokens through progressive transformations.


\paragraph{Thinking Position Encoding.} Thinking Position Encoding provides the model with key information for each thinking step. As shown in Table~\ref{tab:abla}, removing it results in +0.31 PPL., as  model loses  information about importance of each thinking step. 

\begin{figure*}[t]
% \vspace{-2mm}
\begin{minipage}[c]{\textwidth}
\centering
% \subfloat[MHA to DHA Transformation]{
  \includegraphics[width=\textwidth]{figs/token_select_2.pdf }
  % \vspace{-2mm}
  \caption{ \textbf{Left:} Visualization of inner thinking routers' choices in \aname x4 -162M. "3-2" refers to the second thinking step in the 3rd layer (\aname layer). \aname allocates slow thinking to \textcolor{red!60!black}{difficult tokens} and fast thinking to \textcolor{blue!60!black}{easy tokens}. \textbf{Right:} The prediction probabilities for the tokens 'three' and 'stand' from \textcolor{blue!60!black}{LLaMA} and \textcolor{red!60!black}{ITT}.}\label{fig:router_visual}
% }
\end{minipage}
\vspace{-2mm}
\end{figure*}

\paragraph{Adaptive Token Routing.} Disabling the dynamic routing mechanism results in a moderate PPL. increase (+0.19), but significantly impacts computational efficiency. This demonstrates the router's dual role: while maintaining prediction quality through selective processing, it achieves more than 50\% FLOPs reduction by focusing computation on 50\% most critical tokens in each step.

\paragraph{Router Setting.} Our experiments validate three critical design choices: The RTC design of ITT relies on explicit token selection signals (e.g., a 0.5 threshold in Sigmoid) for error correction and progressive disambiguation. The cumulative probability characteristic of Top-P conflicts with this deterministic routing mechanism, leading to a disruption in the iterative processing chain of key tokens. Sigmoid Normalization outperforms Tanh by 0.13 PPL., as it provides unambiguous activation signals for token selection whereas Tanh's negative values may disrupt RTC. Only Select Reweighting surpasses symmetric approaches by 0.15 PPL. through focused computation – selectively enhancing critical tokens while preserving original features for others. This targeted refinement minimizes interference between primary and augmented features.



\subsection{Analysis} \label{sec.exp.ana}

\paragraph{More Thinking for Better Performance.} As shown in Figure~\ref{fig:res_3figs} Left, \textbf{the performance gains from \aname's deep thinking mechanism do not diminish with more iterations}, unlike Loop's diminishing returns. The 162M \aname $\times$4 configuration improves 0.6\% over $\times$3, while Loop $\times$4 only shows a 0.3\% gain over $\times$3. This suggests that with sufficient computational resources, increasing \aname's thinking steps can unlock additional capabilities. The architectural advantage of \textbf{\aname becomes more apparent with larger model widths}, implying that smaller ITT models can adopt wider hidden dimensions to boost representational capacity.


\paragraph{Deeper Thinking with Fewer Tokens.} In Table~\ref{tab:elastic}, \aname x4 reduces the selection rate of the 4th step to 50\%, achieving a -0.26 PPL reduction compared to the training config, showing that \textbf{fewer tokens are needed for deeper thinking steps.} Additionally, \textbf{different thinking steps compensate for each other}, maintaining a PPL advantage of over 0.7 even when a step is removed. Figure~\ref{fig:res_3figs} Middle shows the average Position Encoding values, indicating that the model prioritizes earlier steps while assigning high weights to deeper ones. This demonstrates the \textbf{model's ability to optimize deep thinking with fewer, more impactful tokens}, with potential for even deeper thinking steps.

\paragraph{Routing Analysis.} Visualization of token selection paths (Figure~\ref{fig:router_visual}) demonstrates that approximately 30\%-50\% of tokens receive iterative thinking, with task-critical tokens (e.g., verbs, semantic pivots in red) being more likely to undergo multi-step thinking than low-information tokens. Moreover,the dynamic routing exhibits complementary thinking across steps:  In consecutive steps, \textbf{important tokens are prioritized for deeper thinking}. However, the 3-3 and 7-3 steps demonstrate compensatory choices for broader thinking. These two steps focus on simple tokens that were not given attention in previous steps, compensating for any missed details. Finally, interpretability analysis in Figure~\ref{fig:router_visual} Right demonstrate that\textbf{ ITT extend inner thinking steps, thereby preventing the failures observed in the baseline model.} This routing strategy developed during training, allows \textbf{ITT to achieve both depth and comprehensiveness}.



\section{Related Work}

\paragraph{Recurrent Computation} The concept of recurrence in machine learning traces back to foundational works on neural computation \citep{braitenberg1986vehicles} and LSTM networks \citep{gers2000lstm}. Modern extensions integrate recurrence into transformers through depth recurrence \citep{dehghani2019universal,lan2019albert,ng2024loopneuralnetworksparameter}. Recent works have re-discovered this idea for implicit reasoning~\cite{deng2023implicit,hao2024traininglargelanguagemodels} and test-time scaling~\cite{geiping2025scalingtesttimecomputelatent}. In contrast, \aname establishes a general-purpose recursive reasoning framework within individual layers and designs the Residual Thinking Cnnection (RTC) for enhanced capability.


\paragraph{Dynamic Computation Allocation} Dynamic Computation Allocation, like Mixture-of-Expert (MoE), reduce computational overhead by activating only a subset of networks~\cite{fedus2022switch, riquelme2021scaling, zhou2022mixture, jiang2024mixtral, xue2024openmoe}. Some works focus on elastic computation in depth, such as early exit~\cite{elhoushi2024layerskip,chen2023ee_llm}, parameter sharing~\cite{li2024lisa, li2024crosslayer} or using token-routing for dynamic layer skipping~\cite{zhang2024pmodbuildingmixtureofdepthsmllms}. Inspired by these works, ITT designs an elastic deep thinking architecture with Adaptive Token Routing (ATR) for efficient and adaptive computational resources allocation.



\section{Conclusion}
We propose \aname, a dynamic architecture enabling LLMs to allocate additional computation to critical tokens through adaptive inner thinking steps. By integrating token-wise depth routing, residual thinking connections, and step encoding, \aname enhance inner thinking without parameters expansion. Experiments demonstrate its potential for balancing efficiency with enhanced capabilities. 

% superior performance over Transformers and Loop variants across scales, highlighting iThis work advances elastic computation allocation in LLMs, offering a pathway to overcome inherent parameter limitations.


\section*{Limitations}

While \aname demonstrates promising results, several limitations warrant discussion: First, our current implementation employs fixed routing patterns during training, potentially limiting dynamic adaptation to diverse token complexities. Second, our experiments focus on models up to 466M parameters - validation at larger scales could reveal new architectural interactions. Third, the residual thinking connections introduce additional memory overhead during backward passes, requiring optimization for industrial deployment. Finally, while our step encoding effectively differentiates thinking stages, more sophisticated temporal modeling might further enhance reasoning depth. These limitations present valuable directions for future research.

\section*{Ethical Considerations}
Our work adheres to ethical AI principles through three key aspects: 1) All experiments use publicly available datasets with proper anonymization, 2) The enhanced parameter efficiency reduces environmental impact from model training/inference, and 3) Our architecture-agnostic approach promotes accessible performance improvements without proprietary dependencies. We acknowledge potential risks of enhanced reasoning capabilities being misapplied, and recommend implementing output verification mechanisms when deploying \aname-based systems. Our work is committed to advancing accessible and efficient NLP technologies, fostering a more inclusive and automated future for AI.


\section*{Acknowledgments}
% We thank the anonymous reviewers for their insightful comments and constructive suggestions.
We would like to thank members of the IIE KDsec group for their valuable feedback and discussions. We sincerely thank Sean McLeish for his diligent review and critical feedback on this work. We are very grateful to Mengzhou Xia for providing the concise and effective ShearingLLaMA experimental code and for her assistance during the reproduction process. Work done during Yilong Chen's internship in Baidu Inc. This research is supported by the Youth Innovation Promotion Association of CAS (Grant No.2021153).
% \section*{Acknowledgments}


\bibliography{custom}

\newpage
\appendix

\section{Appendix}

\subsection{Algorithm}
As described in Section~\ref{sec.method}, the core algorithm of our proposed Inner Thinking Transformer implements fine-grained token-level reasoning optimization through dynamic depth computation. The detailed procedure is presented in Algorithm~\ref{alg:l4resrep-posmodblockv1}, which features three key innovations: 

\begin{itemize}
    \item \textbf{Adaptive Capacity Scheduling} with temperature annealing: The $\text{getCapacity}$ function gradually increases processed token count during initial training stages, enabling coarse-to-fine learning dynamics.
    
    \item \textbf{Hierarchical Residual Architecture}: Each thinking step $t$ scales and fuses current results ($\alpha^{(t)}\cdot\phi^{(t)}$) with positional encoding before integrating with previous hidden states.
    
    \item \textbf{Multi-grained Routing Network} utilizes hierarchical routing modules $\{\mathcal{R}^{(0)},...,\mathcal{R}^{(T)}\}$ to automatically identify critical tokens at different depth levels.
\end{itemize}

Notably, when training step $P$ stabilizes, the processing capacity $C$ progressively expands to cover all tokens, equipping the network with self-adaptive depth allocation capabilities. Theoretically, this algorithm extends the model's effective depth to $T+1$ times the baseline while maintaining FLOPs overhead of merely $O(kT/S)$. This establishes a parameter-efficient approach for enhancing reasoning capacity through explicit computation budgeting.

\subsection{Extend Related Work}

\paragraph{Recurrent Computation}
The concept of recurrence in machine learning traces back to foundational works on neural computation \citep{braitenberg1986vehicles} and LSTM networks \citep{gers2000lstm}. Modern extensions integrate recurrence into transformers through depth recurrence \citep{dehghani2019universal,lan2019albert,ng2024loopneuralnetworksparameter}, with recent improvements demonstrating algorithmic generalization via randomized unrolling \citep{schwarzschild2021randomized,mcleish2024}. From an optimization perspective, these models relate to energy-based gradient dynamics \citep{lecun2006contrastive} and test-time adaptation \citep{boudiaf2022testtime}. Recent works have introduced it for implicit reasoning~\cite{deng2023implicit,hao2024traininglargelanguagemodels} and test-time scaling~\cite{geiping2025scalingtesttimecomputelatent}. Inspired by these, ITT focuses on recursive reasoning within individual layers and designs the RTC architecture with theoretical support to enhance this capability.

\paragraph{Dynamic Computation Allocation}Dynamic Computation Allocation in architectures, like Sparse Mixture-of-Expert (MoE), utilize input adaptivity to reduce computational overhead by activating only a subset of subnetworks, or "experts," for each input token \cite{fedus2022switch, riquelme2021scaling, zhou2022mixture, jiang2024mixtral, xue2024openmoe}. Recent developments have introduced heterogeneous experts, integrating experts with varying capacities and specializations \cite{wu2024multihead, he2024millionexperts, dean2021pathways, zhou2022mixture}. Some works focus on elastic computation in depth, such as early exit~\cite{elhoushi2024layerskip,chen2023ee_llm}, parameter sharing~\cite{li2024lisa, li2024crosslayer} or using token-routing for dynamic layer skipping (Mixture of Depth)~\cite{zhang2024pmodbuildingmixtureofdepthsmllms}. Inspired by these works, ITT designs an elastic deep thinking architecture and uses Residual Thinking Connections to address the issue of non-continuous layer skipping.

\begin{algorithm}[ht!]
\caption{\textit{NovelSelect}}
\label{alg:novelselect}
\begin{algorithmic}[1]
\State \textbf{Input:} Data pool $\mathcal{X}^{all}$, data budget $n$
\State Initialize an empty dataset, $\mathcal{X} \gets \emptyset$
\While{$|\mathcal{X}| < n$}
    \State $x^{new} \gets \arg\max_{x \in \mathcal{X}^{all}} v(x)$
    \State $\mathcal{X} \gets \mathcal{X} \cup \{x^{new}\}$
    \State $\mathcal{X}^{all} \gets \mathcal{X}^{all} \setminus \{x^{new}\}$
\EndWhile
\State \textbf{return} $\mathcal{X}$
\end{algorithmic}
\end{algorithm}

\subsection{Theoretical Proof of Multi-Step Residual Thinking Connection's Convergence}

In this Section, we provide a theoretical derivation showing that multi-step residual learning, used in Transformer architectures, is more effective than direct one-step learning in terms of gradient flow and convergence. We show that the multi-step process allows for more stable gradient propagation and faster convergence through geometric decay of the error, in contrast to the difficulties caused by gradient vanishing or explosion in direct one-step learning.

In deep learning models, especially in transformer-based architectures, the issue of gradient propagation across multiple layers has been a key challenge. Residual learning, where each layer updates the model with small corrections rather than directly mapping inputs to outputs, has shown promise in improving the stability of training and facilitating deeper networks. In this section, we will theoretically compare multi-step residual learning with direct one-step mapping to highlight why the former leads to better convergence and stability.

Let us consider the overall goal of a Transformer model. The final output \( F(x; \Theta) \) is a function of the input \( x \), parameterized by the model's parameters \( \Theta \), and is trained to minimize the loss function
\[
\mathcal{L}(F(x; \Theta), y^*)\,,
\]
where \( y^* \) is the target output.

For a single block \( B \) within the Transformer, we define an iterative process where the output at step \( k \), denoted by \( y_k \), is updated by adding a small residual term:
\[
y_{k+1} = y_k + \Delta_k(y_k; \theta)\,,
\]
where \( \theta \) is the shared parameter used for the residual function \( \Delta_k \). The goal is to iteratively refine the output by accumulating these residuals. After \( K \) iterations, the final output becomes:
\[
y_K = y_0 + \sum_{k=0}^{K-1} \Delta_k(y_k; \theta)\,,
\]
where \( y_0 \) is the initial input to the block.

\paragraph{Gradient Propagation in Direct One-Step Mapping}
In the direct one-step mapping, we try to learn the function \( F(x; \theta) \) directly from the input to the output. The loss function is defined as:
\[
\mathcal{L} = \mathcal{L}(F(x; \theta), y^*)\,.
\]
The gradient of the loss function with respect to the parameters \( \theta \) is:
\[
\frac{\partial \mathcal{L}}{\partial \theta} = \frac{\partial \mathcal{L}}{\partial F(x; \theta)} \cdot \frac{\partial F(x; \theta)}{\partial \theta}\,.
\]
In deep networks, the term \( \frac{\partial F(x; \theta)}{\partial \theta} \) involves multiple layers of non-linear transformations. This can cause the gradients to either vanish or explode as they propagate back through the layers, leading to unstable training. Specifically, when \( \theta \) is deep within the network, the gradient may be subject to shrinking (vanishing) or growing (exploding) due to the repeated chain rule applications, which impedes effective training.

\paragraph{Gradient Propagation in Multi-Step Residual Learning}
Now, we consider the multi-step residual learning process. After \( K \) iterations, the output of the block is:
\[
y_K = y_0 + \sum_{k=0}^{K-1} \Delta_k(y_k; \theta)\,.
\]
We want to compute the gradient of the loss function \( \mathcal{L} \) with respect to the shared parameters \( \theta \). Using the chain rule, the gradient of \( y_K \) with respect to \( \theta \) is:
\[
\frac{\partial y_K}{\partial \theta} = \frac{\partial y_K}{\partial y_{K-1}} \cdot \frac{\partial y_{K-1}}{\partial y_{K-2}} \cdots \frac{\partial y_1}{\partial \theta}\,.
\]
For each residual update, we have:
\[
\frac{\partial y_{k+1}}{\partial y_k} = I + \frac{\partial \Delta_k(y_k; \theta)}{\partial y_k}\,,
\]
where \( I \) is the identity matrix, and \( \frac{\partial \Delta_k(y_k; \theta)}{\partial y_k} \) represents the gradient of the residual function. Therefore, the total gradient is:
\[
\frac{\partial y_K}{\partial \theta} = \prod_{k=0}^{K-1} \left(I + \frac{\partial \Delta_k(y_k; \theta)}{\partial y_k}\right) \cdot \frac{\partial \Delta_0(y_0; \theta)}{\partial \theta}\,.
\]
If each residual update \( \Delta_k(y_k; \theta) \) is small, we can approximate:
\[
I + \frac{\partial \Delta_k(y_k; \theta)}{\partial y_k} \approx I\,.
\]
This leads to:
\[
\frac{\partial y_K}{\partial \theta} \approx \frac{\partial \Delta_0(y_0; \theta)}{\partial \theta}\,.
\]
Thus, the gradient flow in each step is relatively stable and doesn't suffer from drastic shrinking or explosion, allowing for efficient and stable training.

\paragraph{Convergence in Direct One-Step Learning}
For direct one-step learning, the model learns the entire transformation from \( x \) to \( y \) in one step, which can be represented as:
\[
y = F(x; \theta)\,.
\]
The training objective is to minimize the loss function:
\[
\mathcal{L} = \mathcal{L}(F(x; \theta), y^*)\,.
\]
However, due to the complexity of the non-linear function \( F(x; \theta) \), the gradients can either vanish or explode as they propagate through the layers. In the worst case, the gradients may become extremely small (vanishing gradients) or extremely large (exploding gradients), causing the optimization process to stall or fail to converge to an optimal solution.

\paragraph{Convergence in Multi-Step Residual Learning}
In multi-step residual learning, each step updates the output with a small correction, and the final output is the sum of all the incremental corrections. The error at step \( k \) is given by:
\[
e_k = T(x) - y_k\,,
\]
where \( T(x) \) is the target. The error at step \( k+1 \) is:
\[
e_{k+1} = T(x) - y_{k+1} = e_k - \Delta_k(y_k; \theta)\,.
\]
If the residual updates \( \Delta_k(y_k; \theta) \) are small, the error at each step decreases geometrically:
\[
\| e_{k+1} \| \leq c \| e_k \| \quad \text{for some constant} \quad 0 < c < 1\,.
\]
After \( K \) iterations, the error will decrease exponentially:
\[
\| e_K \| \leq c^K \| e_0 \|\,.
\]
This shows that the error decays exponentially with the number of steps, leading to fast convergence as the number of iterations increases.

\subsection{Extend Analysis} 

\paragraph{Router Weights Visulization} The observed normal distribution of routing weights in the ITT framework, with its distinctive concentration within the 0.6-0.8 range, emerges as a self-regulating mechanism that fundamentally reconciles computational efficiency with model effectiveness. This central tendency facilitates dynamic resource allocation through probabilistic token selection, where moderately high weights enable smooth computational load balancing while preserving residual information pathways. The distribution's avoidance of extreme values inherently supports flexible top-k adjustments, allowing the system to scale computation across contexts without abrupt performance degradation - a critical feature for processing variable-length inputs and maintaining throughput consistency.

The weight concentration further ensures training stability through continuous differentiability across routing decisions. By preventing abrupt 0/1 selection thresholds, the architecture maintains stable gradient flows during backpropagation, effectively distributing learning signals between activated and bypassed tokens. 

\label{sec:appendix}

\begin{table}[t]
\centering
\small
\setlength{\tabcolsep}{2.5mm}{
\begin{tabular}{@{}lcc@{}}
\toprule
\textbf{Method - Select Ratio in Steps} & \textbf{FLOPs} & \textbf{Perplexity} $\downarrow$\\
\midrule
LLaMA2-162M & 1.88 & 11.13 \\
\midrule
\aname×4 - 90\%,~90\%,~90\% & 4.42 & 10.27 (\textcolor{green!70!black}{-0.86}) \\
\midrule
\aname×4 - 90\%,~90\%,~0\% & 3.57 & 10.40 (\textcolor{green!70!black}{-0.73}) \\
\aname×4 - 90\%,~0\%,~90\% & 3.57 & 10.36 (\textcolor{green!70!black}{-0.77}) \\
\aname×4 - 0\%,~90\%,~90\% & 3.57 & 10.56 (\textcolor{green!70!black}{-0.57}) \\
\midrule
\aname×4 - 90\%,~90\%,~70\% & 4.23 & 10.25 (\textcolor{green!70!black}{-0.88}) \\
\aname×4 - 90\%,~70\%,~90\% & 4.23 & 10.23 (\textcolor{green!70!black}{-0.90}) \\
\aname×4 - 70\%,~70\%,~90\% & 4.04 & 10.21 (\textcolor{green!70!black}{-0.92}) \\
\aname×4 - 90\%,~70\%,~70\% & 4.04 & 10.22 (\textcolor{green!70!black}{-0.91}) \\
\midrule
\aname×4 - 70\%,~70\%,~70\%$^\dagger$  & 3.85 & 10.52 (\textcolor{green!70!black}{-0.61}) \\
\midrule
\aname×4 - 70\%,~70\%,~50\% & 3.66 & 10.26 (\textcolor{green!70!black}{-0.87}) \\
\aname×4 - 70\%,~50\%,~70\% & 3.66 & 10.26 (\textcolor{green!70!black}{-0.87}) \\
\aname×4 - 50\%,~70\%,~70\% & 3.66 & 10.29 (\textcolor{green!70!black}{-0.84}) \\
\midrule
\aname×4 - 70\%,~50\%,~50\% & 3.47 & 10.34 (\textcolor{green!70!black}{-0.79}) \\
\aname×4 - 50\%,~50\%,~70\% & 3.47 & 10.36 (\textcolor{green!70!black}{-0.77}) \\
\aname×4 - 50\%,~70\%,~50\% & 3.47 & 10.34 (\textcolor{green!70!black}{-0.79}) \\
% \midrule
\aname×4 - 50\%,~50\%,~50\% & 3.29 & 10.47 (\textcolor{green!70!black}{-0.66}) \\
\midrule
Loop×4 - 100\%,~100\%,~100\%$^\dagger$  & 4.70 & 10.78 (\textcolor{green!70!black}{-0.35}) \\
\bottomrule
\end{tabular}
}

\caption{Eval Perplexity in the \aname setting is performed for extend 3 steps' thinking. $^\dagger$ refers to the model's training configuration.}\label{tab:elastic}
\vspace{-2mm}
\end{table}

\begin{table}[t]
\centering

% \vskip 0.15in
\small{\resizebox{0.9\columnwidth}{!}{%
\begin{tabular}{@{}lcccc@{}}
\toprule
\textbf{Model Setting} & \textbf{L.2-162M} &  \textbf{L.2-230M} &   \textbf{L.2-466M} \\ \midrule
\textit{hidden size }        & 1024  & 1536 & 2048  \\
\textit{intermediate size }        &  2560  & 2560 & 4096  \\
\textit{attention heads}         &  32  & 32 & 32  \\
\textit{num kv heads}      &  32  & 16 & 32 \\
\textit{layers }         & 8  & 8 & 8\\
\midrule
% \textbf{\# Activate}  & 162M   & 230M &  \\
\textbf{\# Params}  & 162M   & 230M & 466M \\
\bottomrule
\end{tabular}
}}
\caption{Detailed configuration, activation parameters, and total parameters of the models included in our study. L.2-162M represents the LLaMA-2 architecture model with 162M total parameters.}
\label{tab:model_setting}
\end{table}

\end{document}
