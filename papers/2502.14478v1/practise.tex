% \subsection{RQ4: Practical Use of the Synthesised Predicates}
% \label{sec:practical}

% Are the predicates synthesised by \tool useful?
% %
% To answer this question, we used \tool's results to enable \emph{new}
% applications of an existing SL-based state-of-the-art deductive
% synthesiser \suslik~\cite{polikarpova2019structuring}.
% %
% Given definitions of SL predicates and a Hoare-style program
% \emph{specification} (\ie, its parameters, pre- and post-condition
% written in Separation Logic), \suslik produces a C program that
% \emph{provably} satisfies that specification, as well as a formal
% proof of that fact in one of several SL implementations on top of Coq
% proof assistant~\cite{Appel:ESOP11, KrebbersTB17,Nanevski-al:POPL10},
% providing the ultimate correctness guarantees for synthesis
% results~\cite{WatanabeGPPS21}.
% %
% The main downside of \suslik is that it requires the user to provide
% SL predicates---a non-negligeable price to pay, which is, however,
% amortised across multiple synthesis tasks involving the same data
% types.
% %
% With \tool, we can, therefore, enhance the expressivity of \suslik by
% allowing the user to provide memory graphs
% instead of the formal SL predicate definitions.

% Following the informally described translation procedure in
% \autoref{sec:bst}, we implement an automated conversion of the
% synthesised \prolog-style SL predicates  into \suslik's specification
% language, and take the predicates as the specifications
% to synthesise a number of
% easy-to-specify programs as shown in \autoref{tab:results}.
% % \footnote{We found that negative number is not well-supported. In our balanced tree predicate, we use -1 to represent the empty tree's height. However, \suslik synthesis only work when we manually modify it to 0.}
% For example, a program fully deallocating a sorted doubly-linked
% list (DLL) can be specified by the following Hoare triple, which makes
% use of the newly synthesised predicate for a sorted DLL
% (\code{srt_dll}) in the precondition and the empty heap assertion
% \code{emp} in the postcondition:
% %
% \begin{lstlisting}
%   { srt_dll(x, a, s) } void free(loc x) { emp }
% \end{lstlisting}
% %

% Non-trivial synthesised programs performing data structure
% transformations such as {sorting} and {flattening} are listed in in
% \autoref{tab:results}'s {Transform} category.
% %
% As a specific example, a specification describing a function that
% flattens a BST into a sorted DLL is defined as follows:
% %
% \begin{lstlisting}
%   { r :-> 0 * bst(x, s) } void bst_to_srt_dll(loc x, loc r) { r :-> y * srt_dll(y, a, s) }
% \end{lstlisting}
% %

% The columns in the table in \autoref{tab:results} also show the ratio
% of AST nodes between the the synthesised programs and \suslik
% specifications, demonstrating the gain in expressivity.
% %
% Finally, the table shows the time it took to synthesise the programs.
% %
% Synthesis efficiency for some of those case studies in \suslik can be
% improved by explicitly providing search parameters such as number of
% considered predicate unfoldings to the synthesiser.
% %
% All programs in the table were automatically synthesised by \suslik,
% along with their SL proofs embedded into Coq via VST
% framework~\cite{Appel:ESOP11}, from the predicates obtained via \tool
% in the scope of the experiments in \autoref{sec:done}. An example of
% the synthesised C program that transforms a sorted DLL into a singly
% linked list is given in \autoref{fig:transform}.

% \begin{figure}[t]
%   \centering  
%   \begin{minipage}[b]{0.56\textwidth}
%       \centering
%       {\footnotesize{
%       
    \begin{tabular}{c| c|c|c| c}
    \toprule
    No. & Category & Program & Code/Spec & Time    \\	  
    \midrule
    1 & \multirow{4}{*}{{{Deallocate}}} & sll & 5.5x & 0.2s\\
    2 & & bst & 8.0x & 0.2s\\
    3 & & dll\_seg & 2.4x & 0.2s\\
    4 & & {multilist} & 16.0x & 0.3s\\
    \midrule
    5 &  \multirow{3}{*}{{{Copy}}} & lseg & 2.0x & 0.8s\\
    6 & & bst & 3.5x & 3.3s\\
    7 & & balanced tree & 3.0x & 1.9s\\
    \midrule 
    8 & \multirow{2}{*}{{{Size}}} & sll\_len & 2.1x & 0.4s  \\
    9 & & balanced tree & 3.5x & 0.6s  \\
    \midrule 
    10 & \multirow{8}{*}{{{Transform}}} & sll $\rightarrow$ dlseg & 2.5x & 0.4s  \\
    11 & & {srt\_dll $\rightarrow$ sll} & 3.1x & 7.4s  \\
    12 & & {dll $\rightarrow$ bst} & 15.0x & 42.8s  \\
    13 & & {btree $\rightarrow$ bktree} & 13.6x & 11.8s  \\
    14 & & {multilist $\xrightarrow{}$ sll} & 5.0x & 8.8s  \\
    15 & & {btree $\xrightarrow{}$ dll} & 9.6x & 7.1s  \\
    16 & & {bst $\xrightarrow{}$ srtl} & 11.6x & 10.3s  \\
    17 & & {dll $\xrightarrow{}$ srt\_dll} & 7.3x & 9.3s  \\
      \bottomrule
      %
    \end{tabular}
    
    % & btree $\rightarrow$ dll ? & 1.4x & 0.4  \\

%       }}
%     \caption{Example programs synthesised by \suslik from SL
%       specifications stated using predicates produced by \tool.}
    
%     %
%     \label{tab:results}
%   \end{minipage}
%   \hfill
%   \begin{minipage}[b]{0.4\textwidth}
%     \centering
    
% \begin{minted}[fontsize=\scriptsize]{c}
% // pre:  {f :-> x ** sorted_dll(x, z, s)}
% // post: {f :-> y ** sll(y, s)}

% void srt_dll_to_sll (loc f) {
% loc x1 = READ_LOC(f, 0);
% if (x1 == 0)  {
%   WRITE_INT(f, 0, 0);
%   return;
% } else {
%   int vx11 = READ_INT(x1, 0);
%   loc nxtx11 = READ_LOC(x1, 1);
%   loc z1 = READ_LOC(x1, 2);
%   WRITE_LOC(f, 0, nxtx11);
%   srt_dll_to_sll(f);
%   loc y11 = READ_LOC(f, 0);
%   loc y2 = (loc)malloc(2 * sizeof(loc));
%   free(x1);
%   WRITE_LOC(f, 0, y2);
%   WRITE_LOC(y2, 1, y11);
%   WRITE_INT(y2, 0, (int)vx11);
%   return;
% }}
% \end{minted}
% \caption{An example \suslik output (\#{11}): a C program for
%   converting a sorted DLL to SLL. \texttt{loc}, \texttt{READ\_LOC},
%   \etc are macro-definitions around ordinary C types and operations. }
%         \label{fig:transform}
% \end{minipage}

% % \setlength{\abovecaptionskip}{5pt}
% \setlength{\belowcaptionskip}{-10pt}

% \end{figure}

% The outcome of this experiment hints a new way to synthesise
% \emph{provably-correct} heap-manipulating programs via memory graphs
% describing the properties of the input and output, instead of
% providing the formal SL predicates as in deductive synthesis
% \cite{polikarpova2019structuring}, or providing the concrete input
% memory graphs before the program execution and the
% \emph{corresponding} output memory graphs, as in inductive synthesis
% \cite{roy2013concrete, SinghS11}.
