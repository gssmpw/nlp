%! TEX root = paper.tex

%%% Arrays in mathpartir
\makeatletter % allow us to mention @-commands
\def\arcr{\@arraycr}
\makeatother

\definecolor{shadecolor}{gray}{1.00}
\definecolor{ddarkgray}{gray}{0.4}
\definecolor{darkgray}{gray}{0.7}
\definecolor{light-gray-in-algo}{gray}{0.8}
\definecolor{light-gray}{gray}{0.9}
\definecolor[named]{ACMBlue}{cmyk}{1,0.1,0,0.1}
\definecolor[named]{ACMYellow}{cmyk}{0,0.16,1,0}
\definecolor[named]{ACMOrange}{cmyk}{0,0.42,1,0.01}
\definecolor[named]{ACMRed}{cmyk}{0,0.90,0.86,0}
\definecolor[named]{ACMLightBlue}{cmyk}{0.49,0.01,0,0}
\definecolor[named]{ACMGreen}{cmyk}{0.20,0,1,0.19}
\definecolor[named]{ACMPurple}{cmyk}{0.55,1,0,0.15}
\definecolor[named]{ACMDarkBlue}{cmyk}{1,0.58,0,0.21}
\definecolor[named]{PaleGreen}{RGB}{196, 255, 231}
\definecolor[named]{PaleOrange}{RGB}{255, 213, 169}
\definecolor{intnull}{RGB}{213,229,255}

% These are already defined in dvipsnames
% \definecolor{teal}{rgb}{0.0, 0.5, 0.5}
% \definecolor{magenta}{rgb}{1.0, 0.0, 1.0}

\newcommand{\whitebox}[1]{\colorbox{white}{#1}}
\newcommand{\graybox}[1]{\colorbox{light-gray}{#1}}
\newcommand{\darkgraybox}[1]{\colorbox{darkgray}{#1}}
\newcommand{\gbm}[1]{\graybox{${#1}$}}
\newcommand{\wbm}[1]{\whitebox{${#1}$}}
\newcommand{\optim}[1]{{\textcolor{ddarkgray}{{#1}}}}

\newcommand{\naive}{na\"{i}ve\xspace}

\newcommand{\angled}[1]{\left\langle{#1}\right\rangle}
\newcommand{\paren}[1]{\left({#1}\right)}
\newcommand{\parenbig}[1]{\big({#1}\big)}

\newcommand{\etc}{\emph{etc}\xspace}
\newcommand{\ie}{\emph{i.e.}\xspace}
\newcommand{\Ie}{\emph{I.e.}\xspace}
\newcommand{\eg}{\emph{e.g.}\xspace}
\newcommand{\Eg}{\emph{E.g.}\xspace}
\newcommand{\etal}{\emph{et~al.}\xspace}
\newcommand{\adhoc}{\emph{ad hoc}\xspace}
\newcommand{\viz}{\emph{viz.}\xspace}
\newcommand{\dom}[1]{\mathsf{dom}\paren{#1}\xspace}
\newcommand{\im}[1]{\mathsf{im}\paren{#1}}
\newcommand{\aka}{\textit{a.k.a.}\xspace}
\newcommand{\cf}{\textit{cf.}\xspace}
\newcommand{\wrt}{\emph{w.r.t.}\xspace}
\newcommand{\Iff}{\emph{iff}\xspace}
\newcommand{\loef}{L\"{o}f}
\newcommand{\eqdef}{\triangleq}

% \newcommand{\cached}[1]{\boxed{#1}}
\newcommand{\cached}[1]{#1}

\newcommand{\denot}[1]{\llbracket{#1}\rrbracket}
\newcommand{\Eqframe}{\approx}
\newcommand{\eqframe}[2]{{#1} \approx {#2}}

\newcommand{\ifext}[2]{\ifdefined\extflag{#1}\else{#2}\fi}
%\newcommand{\ifext}[2]{{#1}}
\newcommand{\ifcomm}[1]{\ifdefined\extcomm{#1}\else{}\fi}

% \theoremstyle{definition}
% \newtheorem{definition}{Definition}[section]

%\newcommand{\mute}[1]{\ifcomm{#1}}
\newcommand{\mute}[1]{{#1}}

% remarks
\newcommand{\todo}[1]{\mute{\textcolor{ACMRed}{(TODO: {#1})}}}
\newcommand{\is}[1]{{\textcolor{ACMRed}{(Ilya: {#1})}}}
\newcommand{\zy}[1]{{\textcolor{blue}{(Ziyi: {#1})}}}

\newcommand{\mypara}[1]{\smallskip\textbf{\emph{#1.}}}
%\newcommand{\mypara}[1]{\smallskip\noindent\textbf{\emph{#1.}}}
% \newcommand{\mypara}[1]{\paragraph{#1}}

\newcommand{\tname}[1]{\textsf{#1}\xspace}

\newcommand{\popper}{\tname{Popper}}
\newcommand{\aspal}{\tname{ASPAL}}
\newcommand{\aspsyn}{\tname{ASPSynth}}
\newcommand{\shape}{\tname{ShaPE}}
\newcommand{\Shape}{\tname{\textbf{S}haPE}}
\newcommand{\vcdryad}{\tname{VCDryad}}
\newcommand{\grass}{\tname{GRASShopper}}
\newcommand{\veri}{\tname{VeriFast}}
\newcommand{\precis}{\tname{Precis}}
\newcommand{\prolog}{\tname{Prolog}}
\newcommand{\clingo}{\tname{Clingo}}
\newcommand{\suslik}{\tname{SuSLik}}
\newcommand{\sling}{\tname{SLING}}
\newcommand{\cypress}{\tname{Cypress}}
\newcommand{\synquid}{\tname{Synquid}}
\newcommand{\cyclist}{\tname{Cyclist}}
\newcommand{\robo}{\tname{ROBoSuSLik}}
\newcommand{\spt}{\tname{SPT}}
\newcommand{\synbad}{\tname{Synbad}}
\newcommand{\logicfull}{Synthetic Separation Logic\xspace}
\newcommand{\Synfull}{Cyclic Program Synthesis\xspace}
\newcommand{\synfull}{cyclic program synthesis\xspace}
\newcommand{\tool}{\tname{Sippy}}
\newcommand{\ggen}{\tname{Grippy}}
\newcommand{\impsynt}{\tname{ImpSynt}}
\newcommand{\logic}{SSL\xspace}
% \newcommand{\newlogic}{SSL$_{\circlearrowleft}$\xspace}
\newcommand{\newlogic}{SSL\xspace}
\newcommand{\exCount}{54\xspace}
\newcommand{\newbench}{seven\xspace}
\newcommand{\timeout}{30\xspace}

\definecolor{pblue}{rgb}{0.13,0.13,1}
\definecolor{pgreen}{rgb}{0,0.5,0}
\definecolor{pred}{rgb}{0.9,0,0}
\definecolor{pgrey}{rgb}{0.46,0.45,0.48}

\definecolor{ckeyword}{HTML}{7F0055}
\definecolor{ccomment}{HTML}{3F7F5F}
\definecolor{cnumber}{HTML}{2A0099}

\lstdefinelanguage{SynLang}{
  keywords={new, let, if, else, null, return, while},
  ndkeywords={bool, int, void, loc, set, pred, where},
  mathescape=true,
  showspaces=false,
  escapechar=$,
  %$
  showtabs=false,
  breaklines=true,
  showstringspaces=false,
  breakatwhitespace=true,
  lineskip=-0.9pt,
  morecomment=[l]{//}, % l is for line comment
  morecomment=[s]{/*}{*/}, % s is for start and end delimiter
  basewidth={0.54em, 0.4em},%
  basicstyle=\footnotesize\ttfamily,
  keywordstyle={\color{ACMPurple}\ttfamily\bfseries},
  ndkeywordstyle={\color{pblue}\ttfamily\bfseries},
  commentstyle={\color{ccomment}\itshape},
  numbers=none,
  moredelim=**[is][\color{red}]{@}{@},
}

\lstset{
  language=SynLang, 
  escapeinside={(*@}{@*)},
}

\lstdefinestyle{numbers}
{
  numbers=left,
  numberstyle=\scriptsize\sf,
  xleftmargin=15pt
}

% \newcommand{\code}[1]{\lstinline[language=SynLang,basicstyle=\small\ttfamily,mathescape=true]{#1}}
\newcommand{\tinycode}[1]{\lstinline[language=SynLang,basicstyle=\footnotesize\ttfamily,mathescape=true]{#1}}


% \newcommand{\x}{\times}
% \newcommand{\tth}{^{\text{th}}\xspace}
% \newcommand{\pow}[1]{\wp({#1})}
\newcommand{\set}[1]{\left\{{#1}\right\}}
% \newcommand{\many}[1]{\overline{#1}}


% \newcommand{\mcode}[1]{{\ensuremath{\pcode{#1}}}}
\newcommand{\mcode}[1]{\textmintinline[fontsize=\small]{prolog}{#1}}

%% Predicates
\newcommand{\rtree}{\pred{rtree}}
\newcommand{\children}{\pred{children}}
\newcommand{\andor}{\textsc{And/Or}\xspace}
\newcommand{\AND}{\textsc{And}\xspace}
\newcommand{\OR}{\textsc{Or}\xspace}

% https://tex.stackexchange.com/questions/34094/suppress-line-numbering-for-specific-lines-in-listings-package
\let\origthelstnumber\thelstnumber
\makeatletter
\newcommand*\Suppressnumber{%
  \lst@AddToHook{OnNewLine}{%
    \let\thelstnumber\relax%
     \advance\c@lstnumber-\@ne\relax%
    }%
}
\newcommand*\Reactivatenumber{%
  \lst@AddToHook{OnNewLine}{%
   \let\thelstnumber\origthelstnumber%
   \advance\c@lstnumber\@ne\relax}%
}
\makeatother

%%%%%%%%%%%%%%%%%%%%%%%%%%%%%%%%%%%%%%%%%%%
%% Commands for drawing traces
%%%%%%%%%%%%%%%%%%%%%%%%%%%%%%%%%%%%%%%%%%%

\newtcbox{\tracebox}[1][]{on line,size=fbox,#1}
\newcommand{\traceval}[2][black]{
  \tracebox[colframe=#1!80,colback=#1!10]{\ensuremath{#2}}
}
\newcommand{\tighttraceval}[2][black]{
  \tracebox[
    boxsep=0.1em,
    left=0mm,right=0mm,top=0mm,bottom=0mm,
    left skip=0.2em, right skip=0em,
    colframe=#1!80,colback=#1!10
  ]
  {\ensuremath{#2}}
}
\newcommand{\snugtraceval}[2][black]{
  \tracebox[
    boxsep=0.2em,
    left=0mm,right=0mm,top=0mm,bottom=0mm,
    % left skip=0.2em, right skip=0em,
    colframe=#1!80,colback=#1!10
  ]
  {\ensuremath{#2}}
}


%%%%%%%%%%%%%%%%%%%%%%%%%%
% New logic macros
%%%%%%%%%%%%%%%%%%%%%%%%%%


% Label for rule group
\newcommand\flab[3]{
  \node[frame label] at (#1.north west)
    {\small \textbf{#2}\hspace{0.5em}#3}
}

% Highlight substitutions in green
\newcommand\substhilt[1]{{\color{green!50!black}#1}}


\newcommand{\generate}{\ensuremath{\mathsf{generate}}}
\newcommand{\test}{\ensuremath{\mathsf{test}}}
\newcommand{\constrain}{\ensuremath{\mathsf{constrain}}}

\newcommand{\mynext}{\ensuremath{\mathsf{next}}}
\newcommand{\myvalue}{\ensuremath{\mathsf{value}}}

% \setmintedinline{fontsize=\small}

\newcommand{\code}[1]{\lstinline[basicstyle=\small\ttfamily,mathescape=true,escapechar=$,]{#1}}
\newcommand{\codeinmath}[1]{\text{\small\ensuremath{\mathtt{#1}}}}
\newcommand{\textmintinline}[2]{\text{\mintinline{#1}{#2}}}
\newcommand{\pcode}[1]{\mintinline[fontsize=\small]{prolog}{#1}}
\newcommand{\ccode}[1]{\mintinline[fontsize=\small]{c}{#1}}

\newcommand{\pia}{(Proof in Appendix \ref{app:proofs})\xspace}
\newcommand{\dia}{(Detailed in Appendix)\xspace}
\newcommand{\di}[1]{(Detailed in Appendix \ref{#1})\xspace}
% \newcommand{\pia}[1]{(proof in Appendix \ref{#1})\xspace}

\newcommand{\sym}[1]{\langle \text{#1} \rangle}
\newcommand{\pre}[1]{\lceil \text{#1} \rfloor}
