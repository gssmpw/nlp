\documentclass[acmsmall,dvipsnames,screen]{acmart}
\settopmatter{printfolios=true,printccs=true,printacmref=true}


\setcopyright{rightsretained}
\acmJournal{PACMPL}
\acmYear{2025} 
\acmVolume{9} 
\acmNumber{OOPSLA2} 
\acmArticle{} 
\acmMonth{04}
\acmDOI{10.1145/XXXXXXX}

%% Copyright information
%% Supplied to authors (based on authors' rights management selection;
%% see authors.acm.org) by publisher for camera-ready submission;
%% use 'none' for review submission.
% \setcopyright{none}
%\setcopyright{acmcopyright}
%\setcopyright{acmlicensed}
%\setcopyright{rightsretained}
%\copyrightyear{2018}           %% If different from \acmYear

%% Bibliography style
\bibliographystyle{ACM-Reference-Format}
%% Citation style
%% Note: author/year citations are required for papers published as an
%% issue of PACMPL.
% \citestyle{acmauthoryear}   %% For author/year citations
% \setcitestyle{round}

\renewcommand{\figurename}{Fig.}
%\renewcommand{\rmdefault}{ptm}



\usepackage{xspace}
\usepackage[frozencache,cachedir=.]{minted}
\usepackage{tikz}
\usepackage{hhline}
\usepackage{tcolorbox}
\usepackage{multirow}
\usepackage{wrapfig}
\usepackage{mathpartir}
\usepackage{cancel}
\usepackage{pgfplots}
\usepackage{adjustbox}
\usepackage[flushleft]{threeparttable}
\usepackage{bm}
\usetikzlibrary{arrows,arrows.meta,shapes,shapes.arrows,calc,positioning,math,fit,tikzmark,decorations.markings}
\makeatletter
\tikzset{
    block filldraw1/.style={% only the fill and draw styles
        draw, fill=light-gray-in-algo},
    block filldraw2/.style={% only the fill and draw styles
        draw, fill=ddarkgray},
    % block rect/.style={% fill, draw + rectangle (without measurements)
    %     block filldraw, rectangle},
    % block/.style={% fill, draw, rectangle + minimum measurements
    %     block rect, minimum height=0.8cm, minimum width=6em},
    % from/.style args={#1 to #2}{% without transformations
    %     above right={0cm of #1},% needs positioning library
    %     /utils/exec=\pgfpointdiff
    %         {\tikz@scan@one@point\pgfutil@firstofone(#1)\relax}
    %         {\tikz@scan@one@point\pgfutil@firstofone(#2)\relax},
    %     minimum width/.expanded=\the\pgf@x,
    %     minimum height/.expanded=\the\pgf@y}
        }
\makeatother
\usepackage{listings}
\usepackage{enumitem}
\usepackage{algorithm}
\usepackage[noend]{algpseudocode}
\usepackage{subcaption}
% \usepackage{hyperref}
% \usepackage{booktabs}
\usepackage{amsmath}
\usepackage{fontawesome}
\usepackage[utf8]{inputenc}
\usepackage{algorithm}
% \usepackage{cite}

% \hypersetup{colorlinks,
%   linkcolor=ACMDarkBlue,
%   citecolor=ACMPurple,
%   urlcolor=ACMDarkBlue,
%   filecolor=ACMDarkBlue}


%%
%% Submission ID.
%% Use this when submitting an article to a sponsored event. You'll
%% receive a unique submission ID from the organizers
%% of the event, and this ID should be used as the parameter to this command.
%%\acmSubmissionID{123-A56-BU3}

%%
%% For managing citations, it is recommended to use bibliography
%% files in BibTeX format.
%%
%% You can then either use BibTeX with the ACM-Reference-Format style,
%% or BibLaTeX with the acmnumeric or acmauthoryear sytles, that include
%% support for advanced citation of software artefact from the
%% biblatex-software package, also separately available on CTAN.
%%
%% Look at the sample-*-biblatex.tex files for templates showcasing
%% the biblatex styles.
%%

%%
%% The majority of ACM publications use numbered citations and
%% references.  The command \citestyle{authoryear} switches to the
%% "author year" style.
%%
%% If you are preparing content for an event
%% sponsored by ACM SIGGRAPH, you must use the "author year" style of
%% citations and references.
%% Uncommenting
%% the next command will enable that style.
%%\citestyle{acmauthoryear}

\usepackage{bbm}
\usepackage{graphicx}
\usepackage{amsmath,amssymb,amsthm,amsfonts}

\usepackage{paralist}
\usepackage{bm}
\usepackage{xspace}
\usepackage{url}
\usepackage{prettyref}
\usepackage{boxedminipage}
\usepackage{wrapfig}
\usepackage{ifthen}
\usepackage{color}
\usepackage{xspace}

\newcommand{\ii}{{\sc Indicator-Instance}\xspace}
\newcommand{\midd}{{\sf mid}}


\usepackage{amsmath,amsthm,amsfonts,amssymb}
\usepackage{mathtools}
\usepackage{graphicx}


% \usepackage{fullpage}

\usepackage{nicefrac}

\newtheorem{inftheorem}{Informal Theorem}
\newtheorem{claim}{Claim}
\newtheorem*{definition*}{Definition}
\newtheorem{example}{Example}

\DeclareMathOperator*{\argmax}{arg\,max}
\DeclareMathOperator*{\argmin}{arg\,min}
\usepackage{subcaption}

\newtheorem{problem}{Problem}
\usepackage[utf8]{inputenc}
\newcommand{\rank}{\mathsf{rank}}
\newcommand{\tr}{\mathsf{Tr}}
\newcommand{\tv}{\mathsf{TV}}
\newcommand{\opt}{\mathsf{OPT}}
\newcommand{\rr}{\textsc{R}\space}
\newcommand{\alg}{\textsf{Alg}\space}
\newcommand{\sd}{\textsf{sd}_\lambda}
\newcommand{\lblq}{\mathfrak{lq} (X_1)}
\newcommand{\diag}{\textsf{diag}}
\newcommand{\sign}{\textsf{sgn}}
\newcommand{\BC}{\texttt{BC} }
\newcommand{\MM}{\texttt{MM} }
\newcommand{\Nexp}{N_{\mathrm{exp}}}
\newcommand{\Nrep}{N_{\mathrm{replay}}}
\newcommand{\Drep}{D_{\mathrm{replay}}}
\newcommand{\Nsim}{N_{\mathrm{sim}}}
\newcommand{\piBC}{\pi^{\texttt{BC}}}
\newcommand{\piRE}{\pi^{\texttt{RE}}}
\newcommand{\piEMM}{\pi^{\texttt{MM}}}
\newcommand{\mmd}{\texttt{Mimic-MD} }
\newcommand{\RE}{\texttt{RE} }
\newcommand{\dem}{\pi^E}
\newcommand{\Rlint}{\mathcal{R}_{\mathrm{lin,t}}}
\newcommand{\Rlipt}{\mathcal{R}_{\mathrm{lip,t}}}
\newcommand{\Rlin}{\mathcal{R}_{\mathrm{lin}}}
\newcommand{\Rlip}{\mathcal{R}_{\mathrm{lip}}}
\newcommand{\Rmax}{R_{\mathrm{max}}}
\newcommand{\Rall}{\mathcal{R}_{\mathrm{all}}}
\newcommand{\Rdet}{\mathcal{R}_{\mathrm{det}}}
\newcommand{\Fmax}{F_{\mathrm{max}}}
\newcommand{\Nmax}{\mathcal{N}_{\mathrm{max}}}
\newcommand{\piref}{\pi^{\mathrm{ref}}}
\newcommand{\green}{\text{\color{green!75!black} green}\;}
\newcommand{\thetaBC}{\widehat{\theta}^{\textsf{BC}}}
\newcommand{\ent}{\mathcal{E}_{\Theta,n,\delta}}
\newcommand{\eNt}{\mathcal{E}_{\Theta_t,\Nexp,\delta}}
\newcommand{\eNtH}{\mathcal{E}_{\Theta_t,\Nexp,\delta/H}}

\newcommand{\eref}[1]{(\ref{#1})}
\newcommand{\sref}[1]{Sec. \ref{#1}}
\newcommand{\dr}{\widehat{d}_{\mathrm{replay}}}
\newcommand{\figref}[1]{Fig. \ref{#1}}

\usepackage{xcolor}
\definecolor{expert}{HTML}{008000}
\definecolor{error}{HTML}{f96565}
\newcommand{\GKS}[1]{{\textcolor{violet}{\textbf{GKS: #1}}}}
\newcommand{\Q}[1]{{\textcolor{red}{\textbf{Question #1}}}}
\newcommand{\ZSW}[1]{{\textcolor{orange}{\textbf{ZSW: #1}}}}
\newcommand{\JAB}[1]{{\textcolor{teal}{\textbf{JAB: #1}}}}
\newcommand{\jab}[1]{{\textcolor{teal}{\textbf{JAB: #1}}}}
\newcommand{\SAN}[1]{{\textcolor{blue}{\textbf{SC: #1}}}}
\newcommand{\scnote}[1]{\SAN{#1}}
\newcommand{\norm}[1]{\left\lVert #1 \right\rVert}

\usepackage{color-edits}
\addauthor{sw}{blue}

\usepackage{thmtools}
\usepackage{thm-restate}

\usepackage{tikz}
\usetikzlibrary{arrows,calc} 
\newcommand{\tikzAngleOfLine}{\tikz@AngleOfLine}
\def\tikz@AngleOfLine(#1)(#2)#3{%
\pgfmathanglebetweenpoints{%
\pgfpointanchor{#1}{center}}{%
\pgfpointanchor{#2}{center}}
\pgfmathsetmacro{#3}{\pgfmathresult}%
}

\declaretheoremstyle[
    headfont=\normalfont\bfseries, 
    bodyfont = \normalfont\itshape]{mystyle} 
\declaretheorem[name=Theorem,style=mystyle,numberwithin=section]{thm}

% \usepackage{algorithm}
% \usepackage{algorithmic}
\usepackage[linesnumbered,algoruled,boxed,lined,noend]{algorithm2e}

\usepackage{listings}
\usepackage{amsmath}
\usepackage{amsthm}
\usepackage{tikz}
\usepackage{caption}
\usepackage{mdwmath}
\usepackage{multirow}
\usepackage{mdwtab}
\usepackage{eqparbox}
\usepackage{multicol}
\usepackage{amsfonts}
\usepackage{tikz}
\usepackage{multirow,bigstrut,threeparttable}
\usepackage{amsthm}
\usepackage{bbm}
\usepackage{epstopdf}
\usepackage{mdwmath}
\usepackage{mdwtab}
\usepackage{eqparbox}
\usetikzlibrary{topaths,calc}
\usepackage{latexsym}
\usepackage{cite}
\usepackage{amssymb}
\usepackage{bm}
\usepackage{amssymb}
\usepackage{graphicx}
\usepackage{mathrsfs}
\usepackage{epsfig}
\usepackage{psfrag}
\usepackage{setspace}
\usepackage[%dvips,
            CJKbookmarks=true,
            bookmarksnumbered=true,
            bookmarksopen=true,
%						bookmarks=false,
            colorlinks=true,
            citecolor=red,
            linkcolor=blue,
            anchorcolor=red,
            urlcolor=blue
            ]{hyperref}
%\usepackage{algorithm}
\usepackage[linesnumbered,algoruled,boxed,lined]{algorithm2e}
\usepackage{algpseudocode}
\usepackage{stfloats}
\RequirePackage[numbers]{natbib}

\usepackage{comment}
\usepackage{mathtools}
\usepackage{blkarray}
\usepackage{multirow,bigdelim,dcolumn,booktabs}

\usepackage{xparse}
\usepackage{tikz}
\usetikzlibrary{calc}
\usetikzlibrary{decorations.pathreplacing,matrix,positioning}

\usepackage[T1]{fontenc}
\usepackage[utf8]{inputenc}
\usepackage{mathtools}
\usepackage{blkarray, bigstrut}
\usepackage{gauss}

\newenvironment{mygmatrix}{\def\mathstrut{\vphantom{\big(}}\gmatrix}{\endgmatrix}

\newcommand{\tikzmark}[1]{\tikz[overlay,remember picture] \node (#1) {};}

%% Adapted form https://tex.stackexchange.com/questions/206898/braces-for-cases-in-tabular-environment/207704#207704
\newcommand*{\BraceAmplitude}{0.4em}%
\newcommand*{\VerticalOffset}{0.5ex}%  
\newcommand*{\HorizontalOffset}{0.0em}% 
\newcommand*{\blocktextwid}{3.0cm}%
\NewDocumentCommand{\InsertLeftBrace}{%
	O{} % #1 = draw options
	O{\HorizontalOffset,\VerticalOffset} % #2 = optional brace shift options
	O{\blocktextwid} % #3 = optional text width
	m   % #4 = top tikzmark
	m   % #5 = bottom tikzmark
	m   % #6 = node text
}{%
	\begin{tikzpicture}[overlay,remember picture]
	\coordinate (Brace Top)    at ($(#4.north) + (#2)$);
	\coordinate (Brace Bottom) at ($(#5.south) + (#2)$);
	\draw [decoration={brace, amplitude=\BraceAmplitude}, decorate, thick, draw=black, #1]
	(Brace Bottom) -- (Brace Top) 
	node [pos=0.5, anchor=east, align=left, text width=#3, color=black, xshift=\BraceAmplitude] {#6};
	\end{tikzpicture}%
}%
\NewDocumentCommand{\InsertRightBrace}{%
	O{} % #1 = draw options
	O{\HorizontalOffset,\VerticalOffset} % #2 = optional brace shift options
	O{\blocktextwid} % #3 = optional text width
	m   % #4 = top tikzmark
	m   % #5 = bottom tikzmark
	m   % #6 = node text
}{%
	\begin{tikzpicture}[overlay,remember picture]
	\coordinate (Brace Top)    at ($(#4.north) + (#2)$);
	\coordinate (Brace Bottom) at ($(#5.south) + (#2)$);
	\draw [decoration={brace, amplitude=\BraceAmplitude}, decorate, thick, draw=black, #1]
	(Brace Top) -- (Brace Bottom) 
	node [pos=0.5, anchor=west, align=left, text width=#3, color=black, xshift=\BraceAmplitude] {#6};
	\end{tikzpicture}%
}%
\NewDocumentCommand{\InsertTopBrace}{%
	O{} % #1 = draw options
	O{\HorizontalOffset,\VerticalOffset} % #2 = optional brace shift options
	O{\blocktextwid} % #3 = optional text width
	m   % #4 = top tikzmark
	m   % #5 = bottom tikzmark
	m   % #6 = node text
}{%
	\begin{tikzpicture}[overlay,remember picture]
	\coordinate (Brace Top)    at ($(#4.west) + (#2)$);
	\coordinate (Brace Bottom) at ($(#5.east) + (#2)$);
	\draw [decoration={brace, amplitude=\BraceAmplitude}, decorate, thick, draw=black, #1]
	(Brace Top) -- (Brace Bottom) 
	node [pos=0.5, anchor=south, align=left, text width=#3, color=black, xshift=\BraceAmplitude] {#6};
	\end{tikzpicture}%
}%

\usetikzlibrary{patterns}

\definecolor{cof}{RGB}{219,144,71}
\definecolor{pur}{RGB}{186,146,162}
\definecolor{greeo}{RGB}{91,173,69}
\definecolor{greet}{RGB}{52,111,72}

% provide arXiv number if available:
% \arxiv{cs.IT/1502.00326}

% put your definitions there:

%\newtheorem{remark}{Remark} \def\remref#1{Remark~\ref{#1}}
%\newtheorem{conjecture}{Conjecture} \def\remref#1{Remark~\ref{#1}}
%\newtheorem{example}{Example}

%\theorembodyfont{\itshape}
%\newtheorem{theorem}{Theorem}
%\newtheorem{proposition}{Proposition}
%\newtheorem{lemma}{Lemma} \def\lemref#1{Lemma~\ref{#1}}
%\newtheorem{corollary}{Corollary}


%\theorembodyfont{\rmfamily}
%\newtheorem{definition}{Definition}
%\numberwithin{equation}{section}
% \theoremstyle{plain}
% \newtheorem{theorem}{Theorem}
% \newtheorem{Example}{Example}
% \newtheorem{lemma}{Lemma}
% \newtheorem{remark}{Remark}
% \newtheorem{corollary}{Corollary}
% \newtheorem{definition}{Definition}
% \newtheorem{conjecture}{Conjecture}
% \newtheorem{question}{Question}
% \newtheorem*{induction}{Induction Hypothesis}
% \newtheorem*{folklore}{Folklore}
% \newtheorem{assumption}{Assumption}

\def \by {\bar{y}}
\def \bx {\bar{x}}
\def \bh {\bar{h}}
\def \bz {\bar{z}}
\def \cF {\mathcal{F}}
\def \bP {\mathbb{P}}
\def \bE {\mathbb{E}}
\def \bR {\mathbb{R}}
\def \bF {\mathbb{F}}
\def \cG {\mathcal{G}}
\def \cM {\mathcal{M}}
\def \cB {\mathcal{B}}
\def \cN {\mathcal{N}}
\def \var {\mathsf{Var}}
\def\1{\mathbbm{1}}
\def \FF {\mathbb{F}}


\newenvironment{keywords}
{\bgroup\leftskip 20pt\rightskip 20pt \small\noindent{\bfseries
Keywords:} \ignorespaces}%
{\par\egroup\vskip 0.25ex}
\newlength\aftertitskip     \newlength\beforetitskip
\newlength\interauthorskip  \newlength\aftermaketitskip















%%%%%%%%%%%%%%%%%%%%%%%%%%%% by Wu %%%%%%%%%%%%%%%%%%%%%%%%%%%%
\usepackage{xspace}

\newcommand{\Lip}{\mathrm{Lip}}
\newcommand{\stepa}[1]{\overset{\rm (a)}{#1}}
\newcommand{\stepb}[1]{\overset{\rm (b)}{#1}}
\newcommand{\stepc}[1]{\overset{\rm (c)}{#1}}
\newcommand{\stepd}[1]{\overset{\rm (d)}{#1}}
\newcommand{\stepe}[1]{\overset{\rm (e)}{#1}}
\newcommand{\stepf}[1]{\overset{\rm (f)}{#1}}


\newcommand{\floor}[1]{{\left\lfloor {#1} \right \rfloor}}
\newcommand{\ceil}[1]{{\left\lceil {#1} \right \rceil}}

\newcommand{\blambda}{\bar{\lambda}}
\newcommand{\reals}{\mathbb{R}}
\newcommand{\naturals}{\mathbb{N}}
\newcommand{\integers}{\mathbb{Z}}
\newcommand{\Expect}{\mathbb{E}}
\newcommand{\expect}[1]{\mathbb{E}\left[#1\right]}
\newcommand{\Prob}{\mathbb{P}}
\newcommand{\prob}[1]{\mathbb{P}\left[#1\right]}
\newcommand{\pprob}[1]{\mathbb{P}[#1]}
\newcommand{\intd}{{\rm d}}
\newcommand{\TV}{{\sf TV}}
\newcommand{\LC}{{\sf LC}}
\newcommand{\PW}{{\sf PW}}
\newcommand{\htheta}{\hat{\theta}}
\newcommand{\eexp}{{\rm e}}
\newcommand{\expects}[2]{\mathbb{E}_{#2}\left[ #1 \right]}
\newcommand{\diff}{{\rm d}}
\newcommand{\eg}{e.g.\xspace}
\newcommand{\ie}{i.e.\xspace}
\newcommand{\iid}{i.i.d.\xspace}
\newcommand{\fracp}[2]{\frac{\partial #1}{\partial #2}}
\newcommand{\fracpk}[3]{\frac{\partial^{#3} #1}{\partial #2^{#3}}}
\newcommand{\fracd}[2]{\frac{\diff #1}{\diff #2}}
\newcommand{\fracdk}[3]{\frac{\diff^{#3} #1}{\diff #2^{#3}}}
\newcommand{\renyi}{R\'enyi\xspace}
\newcommand{\lpnorm}[1]{\left\|{#1} \right\|_{p}}
\newcommand{\linf}[1]{\left\|{#1} \right\|_{\infty}}
\newcommand{\lnorm}[2]{\left\|{#1} \right\|_{{#2}}}
\newcommand{\Lploc}[1]{L^{#1}_{\rm loc}}
\newcommand{\hellinger}{d_{\rm H}}
\newcommand{\Fnorm}[1]{\lnorm{#1}{\rm F}}
%% parenthesis
\newcommand{\pth}[1]{\left( #1 \right)}
\newcommand{\qth}[1]{\left[ #1 \right]}
\newcommand{\sth}[1]{\left\{ #1 \right\}}
\newcommand{\bpth}[1]{\Bigg( #1 \Bigg)}
\newcommand{\bqth}[1]{\Bigg[ #1 \Bigg]}
\newcommand{\bsth}[1]{\Bigg\{ #1 \Bigg\}}
\newcommand{\xxx}{\textbf{xxx}\xspace}
\newcommand{\toprob}{{\xrightarrow{\Prob}}}
\newcommand{\tolp}[1]{{\xrightarrow{L^{#1}}}}
\newcommand{\toas}{{\xrightarrow{{\rm a.s.}}}}
\newcommand{\toae}{{\xrightarrow{{\rm a.e.}}}}
\newcommand{\todistr}{{\xrightarrow{{\rm D}}}}
\newcommand{\eqdistr}{{\stackrel{\rm D}{=}}}
\newcommand{\iiddistr}{{\stackrel{\text{\iid}}{\sim}}}
%\newcommand{\var}{\mathsf{var}}
\newcommand\indep{\protect\mathpalette{\protect\independenT}{\perp}}
\def\independenT#1#2{\mathrel{\rlap{$#1#2$}\mkern2mu{#1#2}}}
\newcommand{\Bern}{\text{Bern}}
\newcommand{\Poi}{\mathsf{Poi}}
\newcommand{\iprod}[2]{\left \langle #1, #2 \right\rangle}
\newcommand{\Iprod}[2]{\langle #1, #2 \rangle}
\newcommand{\indc}[1]{{\mathbf{1}_{\left\{{#1}\right\}}}}
\newcommand{\Indc}{\mathbf{1}}
\newcommand{\regoff}[1]{\textsf{Reg}_{\mathcal{F}}^{\text{off}} (#1)}
\newcommand{\regon}[1]{\textsf{Reg}_{\mathcal{F}}^{\text{on}} (#1)}

\definecolor{myblue}{rgb}{.8, .8, 1}
\definecolor{mathblue}{rgb}{0.2472, 0.24, 0.6} % mathematica's Color[1, 1--3]
\definecolor{mathred}{rgb}{0.6, 0.24, 0.442893}
\definecolor{mathyellow}{rgb}{0.6, 0.547014, 0.24}


\newcommand{\red}{\color{red}}
\newcommand{\blue}{\color{blue}}
\newcommand{\nb}[1]{{\sf\blue[#1]}}
\newcommand{\nbr}[1]{{\sf\red[#1]}}

\newcommand{\tmu}{{\tilde{\mu}}}
\newcommand{\tf}{{\tilde{f}}}
\newcommand{\tp}{\tilde{p}}
\newcommand{\tilh}{{\tilde{h}}}
\newcommand{\tu}{{\tilde{u}}}
\newcommand{\tx}{{\tilde{x}}}
\newcommand{\ty}{{\tilde{y}}}
\newcommand{\tz}{{\tilde{z}}}
\newcommand{\tA}{{\tilde{A}}}
\newcommand{\tB}{{\tilde{B}}}
\newcommand{\tC}{{\tilde{C}}}
\newcommand{\tD}{{\tilde{D}}}
\newcommand{\tE}{{\tilde{E}}}
\newcommand{\tF}{{\tilde{F}}}
\newcommand{\tG}{{\tilde{G}}}
\newcommand{\tH}{{\tilde{H}}}
\newcommand{\tI}{{\tilde{I}}}
\newcommand{\tJ}{{\tilde{J}}}
\newcommand{\tK}{{\tilde{K}}}
\newcommand{\tL}{{\tilde{L}}}
\newcommand{\tM}{{\tilde{M}}}
\newcommand{\tN}{{\tilde{N}}}
\newcommand{\tO}{{\tilde{O}}}
\newcommand{\tP}{{\tilde{P}}}
\newcommand{\tQ}{{\tilde{Q}}}
\newcommand{\tR}{{\tilde{R}}}
\newcommand{\tS}{{\tilde{S}}}
\newcommand{\tT}{{\tilde{T}}}
\newcommand{\tU}{{\tilde{U}}}
\newcommand{\tV}{{\tilde{V}}}
\newcommand{\tW}{{\tilde{W}}}
\newcommand{\tX}{{\tilde{X}}}
\newcommand{\tY}{{\tilde{Y}}}
\newcommand{\tZ}{{\tilde{Z}}}

\newcommand{\sfa}{{\mathsf{a}}}
\newcommand{\sfb}{{\mathsf{b}}}
\newcommand{\sfc}{{\mathsf{c}}}
\newcommand{\sfd}{{\mathsf{d}}}
\newcommand{\sfe}{{\mathsf{e}}}
\newcommand{\sff}{{\mathsf{f}}}
\newcommand{\sfg}{{\mathsf{g}}}
\newcommand{\sfh}{{\mathsf{h}}}
\newcommand{\sfi}{{\mathsf{i}}}
\newcommand{\sfj}{{\mathsf{j}}}
\newcommand{\sfk}{{\mathsf{k}}}
\newcommand{\sfl}{{\mathsf{l}}}
\newcommand{\sfm}{{\mathsf{m}}}
\newcommand{\sfn}{{\mathsf{n}}}
\newcommand{\sfo}{{\mathsf{o}}}
\newcommand{\sfp}{{\mathsf{p}}}
\newcommand{\sfq}{{\mathsf{q}}}
\newcommand{\sfr}{{\mathsf{r}}}
\newcommand{\sfs}{{\mathsf{s}}}
\newcommand{\sft}{{\mathsf{t}}}
\newcommand{\sfu}{{\mathsf{u}}}
\newcommand{\sfv}{{\mathsf{v}}}
\newcommand{\sfw}{{\mathsf{w}}}
\newcommand{\sfx}{{\mathsf{x}}}
\newcommand{\sfy}{{\mathsf{y}}}
\newcommand{\sfz}{{\mathsf{z}}}
\newcommand{\sfA}{{\mathsf{A}}}
\newcommand{\sfB}{{\mathsf{B}}}
\newcommand{\sfC}{{\mathsf{C}}}
\newcommand{\sfD}{{\mathsf{D}}}
\newcommand{\sfE}{{\mathsf{E}}}
\newcommand{\sfF}{{\mathsf{F}}}
\newcommand{\sfG}{{\mathsf{G}}}
\newcommand{\sfH}{{\mathsf{H}}}
\newcommand{\sfI}{{\mathsf{I}}}
\newcommand{\sfJ}{{\mathsf{J}}}
\newcommand{\sfK}{{\mathsf{K}}}
\newcommand{\sfL}{{\mathsf{L}}}
\newcommand{\sfM}{{\mathsf{M}}}
\newcommand{\sfN}{{\mathsf{N}}}
\newcommand{\sfO}{{\mathsf{O}}}
\newcommand{\sfP}{{\mathsf{P}}}
\newcommand{\sfQ}{{\mathsf{Q}}}
\newcommand{\sfR}{{\mathsf{R}}}
\newcommand{\sfS}{{\mathsf{S}}}
\newcommand{\sfT}{{\mathsf{T}}}
\newcommand{\sfU}{{\mathsf{U}}}
\newcommand{\sfV}{{\mathsf{V}}}
\newcommand{\sfW}{{\mathsf{W}}}
\newcommand{\sfX}{{\mathsf{X}}}
\newcommand{\sfY}{{\mathsf{Y}}}
\newcommand{\sfZ}{{\mathsf{Z}}}


\newcommand{\calA}{{\mathcal{A}}}
\newcommand{\calB}{{\mathcal{B}}}
\newcommand{\calC}{{\mathcal{C}}}
\newcommand{\calD}{{\mathcal{D}}}
\newcommand{\calE}{{\mathcal{E}}}
\newcommand{\calF}{{\mathcal{F}}}
\newcommand{\calG}{{\mathcal{G}}}
\newcommand{\calH}{{\mathcal{H}}}
\newcommand{\calI}{{\mathcal{I}}}
\newcommand{\calJ}{{\mathcal{J}}}
\newcommand{\calK}{{\mathcal{K}}}
\newcommand{\calL}{{\mathcal{L}}}
\newcommand{\calM}{{\mathcal{M}}}
\newcommand{\calN}{{\mathcal{N}}}
\newcommand{\calO}{{\mathcal{O}}}
\newcommand{\calP}{{\mathcal{P}}}
\newcommand{\calQ}{{\mathcal{Q}}}
\newcommand{\calR}{{\mathcal{R}}}
\newcommand{\calS}{{\mathcal{S}}}
\newcommand{\calT}{{\mathcal{T}}}
\newcommand{\calU}{{\mathcal{U}}}
\newcommand{\calV}{{\mathcal{V}}}
\newcommand{\calW}{{\mathcal{W}}}
\newcommand{\calX}{{\mathcal{X}}}
\newcommand{\calY}{{\mathcal{Y}}}
\newcommand{\calZ}{{\mathcal{Z}}}

\newcommand{\bara}{{\bar{a}}}
\newcommand{\barb}{{\bar{b}}}
\newcommand{\barc}{{\bar{c}}}
\newcommand{\bard}{{\bar{d}}}
\newcommand{\bare}{{\bar{e}}}
\newcommand{\barf}{{\bar{f}}}
\newcommand{\barg}{{\bar{g}}}
\newcommand{\barh}{{\bar{h}}}
\newcommand{\bari}{{\bar{i}}}
\newcommand{\barj}{{\bar{j}}}
\newcommand{\bark}{{\bar{k}}}
\newcommand{\barl}{{\bar{l}}}
\newcommand{\barm}{{\bar{m}}}
\newcommand{\barn}{{\bar{n}}}
\newcommand{\baro}{{\bar{o}}}
\newcommand{\barp}{{\bar{p}}}
\newcommand{\barq}{{\bar{q}}}
\newcommand{\barr}{{\bar{r}}}
\newcommand{\bars}{{\bar{s}}}
\newcommand{\bart}{{\bar{t}}}
\newcommand{\baru}{{\bar{u}}}
\newcommand{\barv}{{\bar{v}}}
\newcommand{\barw}{{\bar{w}}}
\newcommand{\barx}{{\bar{x}}}
\newcommand{\bary}{{\bar{y}}}
\newcommand{\barz}{{\bar{z}}}
\newcommand{\barA}{{\bar{A}}}
\newcommand{\barB}{{\bar{B}}}
\newcommand{\barC}{{\bar{C}}}
\newcommand{\barD}{{\bar{D}}}
\newcommand{\barE}{{\bar{E}}}
\newcommand{\barF}{{\bar{F}}}
\newcommand{\barG}{{\bar{G}}}
\newcommand{\barH}{{\bar{H}}}
\newcommand{\barI}{{\bar{I}}}
\newcommand{\barJ}{{\bar{J}}}
\newcommand{\barK}{{\bar{K}}}
\newcommand{\barL}{{\bar{L}}}
\newcommand{\barM}{{\bar{M}}}
\newcommand{\barN}{{\bar{N}}}
\newcommand{\barO}{{\bar{O}}}
\newcommand{\barP}{{\bar{P}}}
\newcommand{\barQ}{{\bar{Q}}}
\newcommand{\barR}{{\bar{R}}}
\newcommand{\barS}{{\bar{S}}}
\newcommand{\barT}{{\bar{T}}}
\newcommand{\barU}{{\bar{U}}}
\newcommand{\barV}{{\bar{V}}}
\newcommand{\barW}{{\bar{W}}}
\newcommand{\barX}{{\bar{X}}}
\newcommand{\barY}{{\bar{Y}}}
\newcommand{\barZ}{{\bar{Z}}}

\newcommand{\hX}{\hat{X}}
\newcommand{\Ent}{\mathsf{Ent}}
\newcommand{\awarm}{{A_{\text{warm}}}}
\newcommand{\thetaLS}{{\widehat{\theta}^{\text{\rm LS}}}}

\newcommand{\jiao}[1]{\langle{#1}\rangle}
\newcommand{\gaht}{\textsc{GoodActionHypTest}\;}
\newcommand{\iaht}{\textsc{InitialActionHypTest}\;}
\newcommand{\true}{\textsf{True}\;}
\newcommand{\false}{\textsf{False}\;}

% \usepackage[capitalize,noabbrev]{cleveref}
% \crefname{lemma}{Lemma}{Lemmas}
% \Crefname{lemma}{Lemma}{Lemmas}
% \crefname{thm}{Theorem}{Theorems}
% \Crefname{thm}{Theorem}{Theorems}
% \Crefname{assumption}{Assumption}{Assumptions}
% \Crefname{inftheorem}{Informal Theorem}{Informal Theorems}
% \crefformat{equation}{(#2#1#3)}

% % if you use cleveref..
% \usepackage[capitalize,noabbrev]{cleveref}
% \crefname{lemma}{Lemma}{Lemmas}
% \crefname{proposition}{Proposition}{Propositions}
% \crefname{remark}{Remark}{Remarks}
% \crefname{corollary}{Corollary}{Corollaries}
% \crefname{definition}{Definition}{Definitions}
% \crefname{conjecture}{Conjecture}{Conjectures}
% \crefname{figure}{Fig.}{Figures}

\def\sectionautorefname{Sec.}
\def\subsectionautorefname{Sec.}
\def\subsectionautorefname{Sec.}
\def\subsubsectionautorefname{Sec.}
\def\figureautorefname{Fig.}
\def\tableautorefname{Tab.}
\def\equationautorefname{Eq.}
\def\algorithmautorefname{Algorithm}

\newtheorem{theorem}{Theorem}[section]
% \newtheorem{proof}{Proof}[section]
\newtheorem{definition}{Definition}[section]

% \setlength{\parindent}{0.15in}
% \titlespacing*{\section}{0pt}{*1.5}{*1}
% \titlespacing*{\subsection}{0pt}{*1.5}{*1}
% \titlespacing*{\paragraph}{0pt}{*0.5}{*0.5}
% \setlength{\topsep}{0cm}
% \setlength{\parskip}{0pt}

\setlist[itemize]{leftmargin=*}
\setlist[enumerate]{leftmargin=*}

%\renewcommand{\shortauthors}{Anonymised}
%%
%% end of the preamble, start of the body of the document source.
\begin{document}

%%
%% The "title" command has an optional parameter,
%% allowing the author to define a "short title" to be used in page headers.
\title{Inductive Synthesis of Inductive Heap Predicates}

\author{Ziyi Yang}
\affiliation{%
  \institution{National University of Singapore}
    \country{Singapore}
}
\email{yangziyi@u.nus.edu}
\orcid{0000-0002-8015-7846}

\author{Ilya Sergey}
\affiliation{%
  \institution{National University of Singapore}
    \country{Singapore}
}
\email{ilya@nus.edu.sg}
\orcid{0000-0003-4250-5392}

% \vspace{-30pt}
\begin{abstract}
  \begin{abstract}  
Test time scaling is currently one of the most active research areas that shows promise after training time scaling has reached its limits.
Deep-thinking (DT) models are a class of recurrent models that can perform easy-to-hard generalization by assigning more compute to harder test samples.
However, due to their inability to determine the complexity of a test sample, DT models have to use a large amount of computation for both easy and hard test samples.
Excessive test time computation is wasteful and can cause the ``overthinking'' problem where more test time computation leads to worse results.
In this paper, we introduce a test time training method for determining the optimal amount of computation needed for each sample during test time.
We also propose Conv-LiGRU, a novel recurrent architecture for efficient and robust visual reasoning. 
Extensive experiments demonstrate that Conv-LiGRU is more stable than DT, effectively mitigates the ``overthinking'' phenomenon, and achieves superior accuracy.
\end{abstract}  

\end{abstract}

\begin{CCSXML}
<ccs2012>
   <concept>
       <concept_id>10003752.10010124.10010138.10010140</concept_id>
       <concept_desc>Theory of computation~Program specifications</concept_desc>
       <concept_significance>500</concept_significance>
       </concept>
   <concept>
       <concept_id>10010147.10010178.10010187.10010196</concept_id>
       <concept_desc>Computing methodologies~Logic programming and answer set programming</concept_desc>
       <concept_significance>500</concept_significance>
       </concept>
 </ccs2012>
\end{CCSXML}

\ccsdesc[500]{Theory of computation~Program specifications}
\ccsdesc[500]{Computing methodologies~Logic programming and answer set programming}

\keywords{inductive program synthesis, answer set programming,
  Separation Logic}

\maketitle
%%
%% This command processes the author and affiliation and title
%% information and builds the first part of the formatted document.


\section{Introduction}


\begin{figure}[t]
\centering
\includegraphics[width=0.6\columnwidth]{figures/evaluation_desiderata_V5.pdf}
\vspace{-0.5cm}
\caption{\systemName is a platform for conducting realistic evaluations of code LLMs, collecting human preferences of coding models with real users, real tasks, and in realistic environments, aimed at addressing the limitations of existing evaluations.
}
\label{fig:motivation}
\end{figure}

\begin{figure*}[t]
\centering
\includegraphics[width=\textwidth]{figures/system_design_v2.png}
\caption{We introduce \systemName, a VSCode extension to collect human preferences of code directly in a developer's IDE. \systemName enables developers to use code completions from various models. The system comprises a) the interface in the user's IDE which presents paired completions to users (left), b) a sampling strategy that picks model pairs to reduce latency (right, top), and c) a prompting scheme that allows diverse LLMs to perform code completions with high fidelity.
Users can select between the top completion (green box) using \texttt{tab} or the bottom completion (blue box) using \texttt{shift+tab}.}
\label{fig:overview}
\end{figure*}

As model capabilities improve, large language models (LLMs) are increasingly integrated into user environments and workflows.
For example, software developers code with AI in integrated developer environments (IDEs)~\citep{peng2023impact}, doctors rely on notes generated through ambient listening~\citep{oberst2024science}, and lawyers consider case evidence identified by electronic discovery systems~\citep{yang2024beyond}.
Increasing deployment of models in productivity tools demands evaluation that more closely reflects real-world circumstances~\citep{hutchinson2022evaluation, saxon2024benchmarks, kapoor2024ai}.
While newer benchmarks and live platforms incorporate human feedback to capture real-world usage, they almost exclusively focus on evaluating LLMs in chat conversations~\citep{zheng2023judging,dubois2023alpacafarm,chiang2024chatbot, kirk2024the}.
Model evaluation must move beyond chat-based interactions and into specialized user environments.



 

In this work, we focus on evaluating LLM-based coding assistants. 
Despite the popularity of these tools---millions of developers use Github Copilot~\citep{Copilot}---existing
evaluations of the coding capabilities of new models exhibit multiple limitations (Figure~\ref{fig:motivation}, bottom).
Traditional ML benchmarks evaluate LLM capabilities by measuring how well a model can complete static, interview-style coding tasks~\citep{chen2021evaluating,austin2021program,jain2024livecodebench, white2024livebench} and lack \emph{real users}. 
User studies recruit real users to evaluate the effectiveness of LLMs as coding assistants, but are often limited to simple programming tasks as opposed to \emph{real tasks}~\citep{vaithilingam2022expectation,ross2023programmer, mozannar2024realhumaneval}.
Recent efforts to collect human feedback such as Chatbot Arena~\citep{chiang2024chatbot} are still removed from a \emph{realistic environment}, resulting in users and data that deviate from typical software development processes.
We introduce \systemName to address these limitations (Figure~\ref{fig:motivation}, top), and we describe our three main contributions below.


\textbf{We deploy \systemName in-the-wild to collect human preferences on code.} 
\systemName is a Visual Studio Code extension, collecting preferences directly in a developer's IDE within their actual workflow (Figure~\ref{fig:overview}).
\systemName provides developers with code completions, akin to the type of support provided by Github Copilot~\citep{Copilot}. 
Over the past 3 months, \systemName has served over~\completions suggestions from 10 state-of-the-art LLMs, 
gathering \sampleCount~votes from \userCount~users.
To collect user preferences,
\systemName presents a novel interface that shows users paired code completions from two different LLMs, which are determined based on a sampling strategy that aims to 
mitigate latency while preserving coverage across model comparisons.
Additionally, we devise a prompting scheme that allows a diverse set of models to perform code completions with high fidelity.
See Section~\ref{sec:system} and Section~\ref{sec:deployment} for details about system design and deployment respectively.



\textbf{We construct a leaderboard of user preferences and find notable differences from existing static benchmarks and human preference leaderboards.}
In general, we observe that smaller models seem to overperform in static benchmarks compared to our leaderboard, while performance among larger models is mixed (Section~\ref{sec:leaderboard_calculation}).
We attribute these differences to the fact that \systemName is exposed to users and tasks that differ drastically from code evaluations in the past. 
Our data spans 103 programming languages and 24 natural languages as well as a variety of real-world applications and code structures, while static benchmarks tend to focus on a specific programming and natural language and task (e.g. coding competition problems).
Additionally, while all of \systemName interactions contain code contexts and the majority involve infilling tasks, a much smaller fraction of Chatbot Arena's coding tasks contain code context, with infilling tasks appearing even more rarely. 
We analyze our data in depth in Section~\ref{subsec:comparison}.



\textbf{We derive new insights into user preferences of code by analyzing \systemName's diverse and distinct data distribution.}
We compare user preferences across different stratifications of input data (e.g., common versus rare languages) and observe which affect observed preferences most (Section~\ref{sec:analysis}).
For example, while user preferences stay relatively consistent across various programming languages, they differ drastically between different task categories (e.g. frontend/backend versus algorithm design).
We also observe variations in user preference due to different features related to code structure 
(e.g., context length and completion patterns).
We open-source \systemName and release a curated subset of code contexts.
Altogether, our results highlight the necessity of model evaluation in realistic and domain-specific settings.







\section{Overview and Key Ideas}
\label{sec:overview}


In this section, we provide a brief outline of the basics of
Separation Logic (SL) and its inductive predicates. Next, we explain
how to use the existing ILP system \popper~\cite{cropper2021learning}
to synthesise such predicates from \emph{both positive and negative}
examples, and how we handle with learning from positive examples only.
%
We conclude with the high-level workflow of our SL predicate
synthesiser \tool.

\subsection{Inductive Predicates in Separation Logic}
\label{sec:sl}

\begin{figure}
  \centering
%  \belowcaptionskip=-10pt
%  \abovecaptionskip=5pt
  \includegraphics[width=0.8\textwidth]{figure/sll.pdf}
% \setlength{\abovecaptionskip}{0pt}

  \caption{Memory layout of a singly linked list.}
  \label{fig:sll}
\end{figure}
% 
Consider a schematic memory layout depicted in \autoref{fig:sll}
corresponding to a singly linked list (SLL).
%
The list has a recurring structure with each of its elements
represented by a consecutive pair of memory locations (the ``head''
one referred to by a pointer variable~\code{x}), the first one storing
its data value (or \emph{payload})~\code{v} and the second containing
the address \code{y} of the tail of the list. 
%
Knowing these shape constraints, the entire list can be traversed
recursively by starting from the head and following the tail pointers.
% thus, getting access to its payload.


The idea of defining the repetitive shape of a heap-based linked
structure, such as SLL, is precisely captured by Separation Logic and
its inductive (\ie, well-founded recursive) predicates. One encoding
of an SLL heap shape via the SL predicate \code{sll} is given below:

% The idea of defining the repetitive shape of heap-based linked
% structures
% %, such as SLL,  %for page break reason
% is captured by Separation Logic and
% its inductive (\ie, well-founded recursive) predicates. One
% encoding
% of an SLL heap shape   is given via the SL predicate:

\begin{lstlisting}
  pred sll(loc x, set s) where
    | x = 0 $\Rightarrow$ { s = {}; emp }
    | x $\neq$ 0 $\Rightarrow$ { s = {v} $\cup$ s1; x $\mapsto$ v * (x+1) $\mapsto$ y * sll(y, s1) }
\end{lstlisting}

\noindent
%
The predicate \code{sll} is parameterised by a location (\ie, pointer variable)
\code{x} and a payload set of the data structure
\code{s};
%
% \is{We could use algebraic lists to encode payloads instead of sets.
%   We should either address it or explain our choice.}
%
it holds true for any
\emph{heap fragment} that follows the shape of a linked list (and
contains no extra heap space).
%
What exactly that shape is, is defined by the two \emph{clauses} (\aka
constructors) of the predicate. 
%
The first one handles the case when \code{x} is a null-pointer,
constraining the payload set \code{s} of the list to be empty
(\code{\{\}}); the same holds for the list-carrying heap---which is
denoted by a standard SL assertion \code{emp}.\footnote{For
  simplicity, our examples use mathematical sets to encode the data
  payload, assuming uniqueness of the elements, instead of, \eg, an
  algebraic list. This is not a conceptual or practical limitation of
  our approach, as we will show in \autoref{sec:done}.}
%
The second clause describes a more interesting case, in which \code{x}
is not null, and so the payload can be split to an element \code{v}
and the residual payload \code{s1} (for simplicity of this example, we
assume that all elements of the list are unique).
%
Furthermore, the heap carrying the list is specified to have two
consecutive locations, \code{x} and \code{x + 1}, storing \code{v}
and some (existentially quantified) pointer value \code{y}, as denoted
by the SL \emph{points-to} notation $\mapsto$.
%
Finally, the rest of the SLL-carrying heap is 
the recursive occurrence of the same predicate \code{sll(y, s1)} (with
different arguments), thus replicating the recursive structure of the
layout from \autoref{fig:sll}.
%
% The heap-related constraints appearing after the semicolon in the
% clauses are traditionally referred to as \emph{spatial},
% while the remaining constraints (such as, \eg, the statement \code{s
%   == \{\}} constraining the empty list payload) are usually called
% \emph{pure}.



The logical connective \code{*} appearing in the second clause of the
\code{sll} predicate is known as the \emph{separating conjunction}
(sometimes pronounced ``and separately'') and is the main enabling
feature of Separation Logic~\cite{OHearn-al:CSL01}.
%
It implicitly constrains the symbolic heaps it connects in a spatial
assertion to have \emph{disjoint} domains.
%
Specifically, in this example it implies that the heap fragment
captured by \code{sll(y, s1)} does not contain memory locations
referred to by either \code{x} or \code{x+1}.
%
Such disjointness constraint  is what makes it
possible to avoid extensive reasoning about aliasing when using SL
specifications, making them \emph{modular}, \ie, holding true in the
context of any heap that is larger than what is affected by the
specified program.
%


% \subsection{\popper an ASP-based ILP system}

% \pagebreak

% \vspace{-5pt}
  
\subsection{From  Memory Graphs to Heap Predicates}
% \subsection{Synthesising Heap Predicates via ILP} maybe?
\label{sec:popper}

Our goal is to synthesise inductive SL predicates from examples of
concrete memory graphs.
%
To do so, we phrase both SL predicates and the memory graphs that
satisfy them in terms of Logic Programming. For example, the \prolog
predicate below defines a sorted singly linked list:
%
\begin{minted}[fontsize=\small]{prolog}
  srtl(X, S) :- empty(S), nullptr(X).
  srtl(X, S) :- next(X,Y), value(X,V), srtl(Y,SY), min_set(V,S), insert(SY,V,S).
\end{minted}
%
The predicate above defines a sorted singly linked list by enhancing
the ordinary singly linked list predicate with the constraint
\pcode{min_set(V, S)} that states that the value \pcode{V} is equal to
the smallest value in the set \pcode{S}. The \pcode{insert(S1, V, S)}
and \pcode{empty(S)} (\ie, \pcode{s == {v} ++ s1} and \pcode{s == {}}
in the SLL example) are defined using \prolog built-in predicates that
correspond to ordinary functions in set theory.
%
Other \prolog-style predicates used in the synthesised SL solutions
are data-structure specific and are extracted from the user-provided
memory graphs.
%
We leave till later (\autoref{sec:sldomain}) the issue of ensuring
that a \prolog predicate is also a \emph{valid} SL predicate in the
sense that it does not use SL connectives in a contradictory way,
allowing one to derive falsehood from its definition.


%
% Most of the predicates contributing to the definition of \code{srtl}
% come from predefined vocabularies, which the synthesiser will be aware
% of. During the search, it will be looking for definitions that can fix
% a number of clauses and combine available constraints, using the
% user-provided examples to validate its guesses.
%

\begin{figure}[t]
%\vspace{-15pt}
\centering
% \scalebox{0.75}{{\small{
\begin{tabular}{c}
    \begin{tikzpicture}
        \node[circle, draw] (n1) at (0,1.5) {p11};
        \node[circle, draw] (n2) at (2,1.5) {p12};
        \node[circle, draw] (n3) at (4,1.5) {p13};
        \node[draw] (null) at (6,1.5) {null};
        \node[draw] (v1) at (0,0) {1};
        \node[draw] (v2) at (2,0) {2};
        \node[draw] (v3) at (4,0) {3};
        \draw[->] (n1) --  node [above,midway] {\mynext} (n2);
        \draw[->] (n2) --  node [above,midway] {\mynext} (n3);
        \draw[->] (n3) --  node [above,midway] {\mynext} (null);
        \draw[->] (n1) --  node [midway] [above,midway,sloped] {\myvalue} (v1);
        \draw[->] (n2) --  node [midway] [above,midway,sloped] {\myvalue} (v2);
        \draw[->] (n3) --  node [midway] [above,midway,sloped] {\myvalue} (v3);
    \end{tikzpicture}
\\
\\
  \begin{tikzpicture}
        \node[circle, draw] (n1) at (0,1.5) {p21};
        \node[circle, draw] (n2) at (1.8,1.5) {p22};
        \node[circle, draw] (n3) at (3.6,1.5) {p23};
        \node[circle, draw] (n4) at (5.4,1.5) {p24};
        \node[draw] (null) at (7.2,1.5) {null};
        \node[draw] (v1) at (0,0) {2};
        \node[draw] (v2) at (1.8,0) {4};
        \node[draw] (v3) at (3.6,0) {6};
        \node[draw] (v4) at (5.4,0) {9};
        \draw[->] (n1) --  node [above,midway] {\mynext} (n2);
        \draw[->] (n2) --  node [above,midway] {\mynext} (n3);
        \draw[->] (n3) --  node [above,midway] {\mynext} (n4);
        \draw[->] (n4) --  node [above,midway] {\mynext} (null);
        \draw[->] (n1) --  node [above,midway,sloped] {\myvalue} (v1);
        \draw[->] (n2) --  node [above,midway,sloped] {\myvalue} (v2);
        \draw[->] (n3) --  node [above,midway,sloped] {\myvalue} (v3);
        \draw[->] (n4) --  node [above,midway,sloped] {\myvalue} (v4);
    \end{tikzpicture}
\end{tabular}
}}
}
{\small{
\begin{tabular}{c}
    \begin{tikzpicture}
        \node[circle, draw] (n1) at (0,1.5) {p11};
        \node[circle, draw] (n2) at (2,1.5) {p12};
        \node[circle, draw] (n3) at (4,1.5) {p13};
        \node[draw] (null) at (6,1.5) {null};
        \node[draw] (v1) at (0,0) {1};
        \node[draw] (v2) at (2,0) {2};
        \node[draw] (v3) at (4,0) {3};
        \draw[->] (n1) --  node [above,midway] {\mynext} (n2);
        \draw[->] (n2) --  node [above,midway] {\mynext} (n3);
        \draw[->] (n3) --  node [above,midway] {\mynext} (null);
        \draw[->] (n1) --  node [midway] [above,midway,sloped] {\myvalue} (v1);
        \draw[->] (n2) --  node [midway] [above,midway,sloped] {\myvalue} (v2);
        \draw[->] (n3) --  node [midway] [above,midway,sloped] {\myvalue} (v3);
    \end{tikzpicture}
\\
\\
  \begin{tikzpicture}
        \node[circle, draw] (n1) at (0,1.5) {p21};
        \node[circle, draw] (n2) at (1.8,1.5) {p22};
        \node[circle, draw] (n3) at (3.6,1.5) {p23};
        \node[circle, draw] (n4) at (5.4,1.5) {p24};
        \node[draw] (null) at (7.2,1.5) {null};
        \node[draw] (v1) at (0,0) {2};
        \node[draw] (v2) at (1.8,0) {4};
        \node[draw] (v3) at (3.6,0) {6};
        \node[draw] (v4) at (5.4,0) {9};
        \draw[->] (n1) --  node [above,midway] {\mynext} (n2);
        \draw[->] (n2) --  node [above,midway] {\mynext} (n3);
        \draw[->] (n3) --  node [above,midway] {\mynext} (n4);
        \draw[->] (n4) --  node [above,midway] {\mynext} (null);
        \draw[->] (n1) --  node [above,midway,sloped] {\myvalue} (v1);
        \draw[->] (n2) --  node [above,midway,sloped] {\myvalue} (v2);
        \draw[->] (n3) --  node [above,midway,sloped] {\myvalue} (v3);
        \draw[->] (n4) --  node [above,midway,sloped] {\myvalue} (v4);
    \end{tikzpicture}
\end{tabular}
}}

\\[5pt]
\begin{minted}[fontsize=\small]{prolog}
  pos(srtl(p11,[1,2,3])). pos(srtl(p21,[2,4,6,9])).

  % Encoding of the first memory graph from Fig.2
  next(p11, p12). next(p12, p13). next(p13, null). 
  value(p11, 1).  value(p12, 2).  value(p13, 3).

  % Encoding of the second memory graph from Fig.2
  next(p21, p22). next(p22, p23). next(p23, p24). next(p24, null).
  value(p21, 2).  value(p22, 4).  value(p23, 6).  value(p24, 9).
\end{minted}

%\setlength{\abovecaptionskip}{5pt}
% \setlength{\belowcaptionskip}{-10pt}
    \caption{Positive examples of sorted list heap graphs, with the corresponding logic encoding.}
    \label{fig:sorted-list}
\end{figure}

The top of \autoref{fig:sorted-list} shows two memory graphs of sorted
lists that can be used to synthesise \pcode{srtl()}.
%
For a more natural representation in terms of Logic Programming, we
use Java-style naming of structure components, \ie, fields such as
{\small{\textsf{value}}} and {\small{\textsf{next}}} instead of
C-style integer pointer offsets;
%
these fields provide the data-specific predicates (\ie, \pcode{next()}
and \pcode{value()}) to the synthesiser. In the bottom of
\autoref{fig:sorted-list}, we provide the corresponding logic
representations of the inputs to the synthesiser, consisting of
positive examples (\ie, instances of the sought predicate that are
expected to be true), and the background knowledge (\ie, encoding of
the corresponding memory graphs) that should be used to derive those
examples using predicate candidates.
%
Given all this information, a synthesiser should be able to generate
the predicate \pcode{srtl()} that satisfies the positive examples.
%
That is, using the traditional program synthesis from
input-output pairs as an analogy~\cite{GulwaniPS17}, the facts in
background knowledge (\eg, \pcode{next(p11, p12)}) are inputs, the
examples (\eg, \pcode{pos(srtl(p11, [1,2,3]))}) are the outputs, and
the solution is the program (\ie, the predicate) to be synthesised.

% \begin{figure}[t]
% % \setlength{\belowcaptionskip}{-10pt}
% % \setlength{\abovecaptionskip}{5pt}
% \begin{minted}[fontsize=\small]{prolog}
%   pos(srtl(p11,[1,2,3])). pos(srtl(p21,[2,4,6,9])).

%   % Encoding of the first memory graph from Fig.2
%   next(p11, p12). next(p12, p13). next(p13, null). 
%   value(p11, 1).  value(p12, 2).  value(p13, 3).

%   % Encoding of the second memory graph from Fig.2
%   next(p21, p22). next(p22, p23). next(p23, p24). next(p24, null).
%   value(p21, 2).  value(p22, 4).  value(p23, 6).  value(p24, 9).
% \end{minted}
% \caption{Positive examples for \code{srtl} in \popper. \todo{merge to one figure?}}
% \label{fig:examples}
% \end{figure}


\subsection{Predicate Synthesis via Answer Set Programming}
\label{sec:asp}

We observe that the synthesis of SL predicates can be regarded as a
Logic Programming synthesis task, studied extensively in the field of
Inductive Logic Programming
(ILP)~\cite{Muggleton91,cropper2020turning}. 
%
In ILP, the synthesis of definition is done by generating hypotheses
(\ie, predicates) and testing them against the provided examples.
%
Efficient generation of hypotheses in ILP is typically implemented
using Answer Set Programming (ASP)~\cite{gebser2022answer,aspguide}, a
constraint solving-based search-and-optimisation methodology that
allows for effectively pruning the search space of candidate
definitions and is used in many state-of-the-art ILP systems:
\aspal~\cite{corapi2011inductive}, \popper\cite{cropper2021learning},
and \aspsyn~\cite{bembenek2023smt}.

To see how ASP can be used for synthesising logic predicates, we first
provide a brief introduction to its principles using very basic
examples. 
%
Considered a declarative logic programming paradigm, ASP can be
regarded as a syntactic extension of \tname{Datalog}, but with a
different semantics called \emph{stable model
  semantics}~\cite{gelfond1988stable}. The output of an ASP program is
a set of \emph{models} (\ie, so-called answer set) that satisfy the
program constraints. A (normal) ASP program consists of a set of
\emph{clauses} that are composed of a head (on the left of
$\leftarrow$) and a body (on the right of $\leftarrow$) as:
%
\[
   a\ \leftarrow\ b_1,\ldots\ b_m,\ \neg\ c_1,\ \ldots,\ \neg\ c_n.
\]
%
which can be read as "if $b_1,\ldots\ b_m$ are true and
$c_1,\ \ldots,\ c_n$ are false, then $a$ is true". 
The statements $a,~b_i,~c_i$ are called \emph{literals} and are
declared in the format of \pcode{pred_name(X1, ..., Xn)} (\ie, a
predicate with arity $n$). 
%
A clause is called an \emph{integrity constraint} when its head (\ie,
the statement on the left-hand side of $\leftarrow$) is empty, which
means it is inconsistent if the body is true; a clause is called a
\emph{fact} when its body is empty, which means the head is always
true.

Instead of describing the formal definition of a stable
model, we show simple examples of ASP program and the corresponding
answer sets below:
%
\[
  % \begin{footnotesize}
    \begin{tabular}{c|l|l}
      No. &ASP Program&Answer Sets\\\hline
      1&\pcode{a :- b. b.}&\pcode{{a,b}}\\
      2&\pcode{a :- not b. b.}&\pcode{{b}}\\
      3&\pcode{a :- not b. b :- not a.}&\pcode{{a},{b}}\\
      4&\pcode{a :- not b. b :- not a. :- a.}&\pcode{{b}}\\
    \end{tabular}
  % \end{footnotesize}
\]
%
The arity of the literals \pcode{a} and \pcode{b} is 0, and
\pcode{:-}, \pcode{not} in the programs mean $\leftarrow$ and $\neg$
correspondingly.
%
The programs in the table above and their answer sets should be
interpreted as follows.

\begin{itemize}
\item Program 1 is a simple program with a rule (general clause)
  \pcode{a :- b} and the fact \pcode{b} postulating that \pcode{b} is
  true. The answer set is \pcode{{a,b}} meaning that \pcode{a} and
  \pcode{b} can be true together, given the constraints.
%
\item Program 2 is similar to Program 1, but with the rule \pcode{a :-
    not b} instead, which means \pcode{a} is true when \pcode{b} is
  false. The answer set is \pcode{{b}}, no clause is making \pcode{a}
  true.
%
\item Program 3 is a program with two rules. The answer set is
  \pcode{{a},{b}} because \pcode{a} is true when \pcode{b} is false,
  and \pcode{b} is true when \pcode{a} is false, so the answer set is
  the combination of the two cases.
%
\item Program 4 is extended from Program 3 with another clause, that
  is an integrity constraint \pcode{:- a} forcing \pcode{a} to be false. The answer set is \pcode{{b}}
  because \pcode{b} is true when \pcode{a} is false, and the program is
  consistent only in this case (in contrast with Program~3).
\end{itemize}

\noindent
%
As Program 4 demonstrates, the integrity constraint can be used to
prune the answer sets---a very useful feature for synthesis tasks
(more discussion on ASP versus SMT is in \autoref{sec:related}).

Each program above can be regarded as an \emph{enumeration in the
  powerset} of the set with two elements, \pcode{a} and \pcode{b},
returning \emph{all sets} that satisfy the relations (\ie, the ASP
clauses) between the elements.
%
An ILP system, such as \popper~\cite{cropper2021learning}, relies on
an ASP solver to encode the enumerative search among all possible
combinations of literals to synthesise logic predicates.

As a concrete example of ILP via ASP, consider synthesising the
definition of a predicate \pcode{plus_two(A, B)} using six literals:
\pcode{succ(A, A)}, \pcode{succ(A, C)}, \pcode{succ(B, A)},
\pcode{succ(B, B)}, \pcode{succ(B, C)}, and \pcode{succ(C, B)} to
build the body of the predicate, with examples \pcode{plus_two(1,3)}
and \pcode{plus_two(2,4)}.
%
An ASP-based synthesiser would try to find a definition of
\pcode{plus_two(A, B)} as a suitable subset of all their $2^6=64$
possible combinations.
%
While doing so, it would also make use of the natural restrictions
that can be encoded as integrity constraints, such as
%
(1)~no free variable is allowed in the body (hence \pcode{{succ(A, B), succ(C, B)}} is
not a valid answer set because \pcode{C} is free), and 
%
(2)~all input variables \pcode{A} and \pcode{B} should appear in the
body (hence \pcode{{succ(B, C)}} is not a valid synthesis candidate).
%
As we will show, such constraints are also useful for encoding the
domain-specific knowledge about validity of SL predicates, and can be
efficiently solved by ASP solvers.
%
% Finally, 
% answer set \pcode{{succ(A,C), succ(C,B)}} pass
% the input examples' test such as \pcode{plus_two(1,3)} and
% \pcode{plus_two(2,4)}.

Moreover, the incrementality of ASP solvers make it possible to
constrain the search space continuously~\cite{gebser2019multi}. For
instance, assume the following hypothesis is obtained during the
search:
%
\begin{minted}[fontsize=\small]{prolog}
  plus_two(A, B) :- succ(A, A), succ(B, B).
\end{minted}
%
% which corresponds to the answer set
% %
% \begin{minted}[fontsize=\scriptsize]{prolog}
%   head(0,srtl,(X,S)). body(0,value,(X,V)). body(0,min_set,(V,S)).
% \end{minted}
% %
% That is, the clause 0 has one head predicate \pcode{srtl(X, S)} and
% two body predicates \pcode{value(X, V)} and \pcode{min_set(V, S)}.

% MARK: should be 5 pages above
%
%
After testing it by \prolog against the examples, \popper finds that
none of the provided positive examples can be derived using this
solution.
%
As the result, other hypotheses that are more \emph{specialised} (\ie,
more constrained in the bodies) than it, such as the definition of
\pcode{plus_two()} below.
%
\begin{minted}[fontsize=\small]{prolog}
  plus_two(A, B) :- succ(A, A), succ(B, A), succ(B, B).
\end{minted}
%
will also entail no positive examples.
%
To this end, with the help of ASP, a classic ILP performs search for
a candidate hypothesis that passed all tests and has the smallest size
(\ie, number of literals in the predicate). Such \emph{optimal}
hypothesis for our example is the synthesised solution:
%
\begin{minted}[fontsize=\small]{prolog}
  plus_two(A, B) :- succ(A, C), succ(C, B).
\end{minted}


\subsection{Synthesis without Negative Examples}
\label{sec:approach}

The classic ILP comes with an important limitation: in general, it
requires both \emph{positive} and \emph{negative} examples to learn a
predicate. As we explain below, the need for the latter kind of
examples makes it challenging to employ ILP \emph{as-is} as a
pragmatic approach for synthesising SL predicates.


\paragraph{Why Negative Examples are Necessary in ILP}

Let us get back to our examples with synthesising a sorted singly
linked list predicate from positive examples of memory graphs in
\autoref{fig:sorted-list}. 
%
With the conventional ILP, the learned hypothesis by \popper is as
follows, and it is not what we need:
%
\begin{minted}[fontsize=\small]{prolog}
  srtl(X, S) :- empty(S), nullptr(X).
  srtl(X, S) :- next(X, Y), value(X, V), insert(SY, V, S), srtl(Y, SY).
\end{minted}
%
The learned hypothesis does not define a sorted list,
but an ordinary (unsorted) singly linked list.
%
The reason is: in the absence of the negative example, this is a
consistent hypothesis that is smaller in size than the correct
definition of \code{srtl}.
So if we want to learn the correct predicate, we need to provide negative examples that are inconsistent with the incorrect hypothesis, such as
%
\begin{minted}[fontsize=\small]{prolog}
  neg(srtl(p11, [1,2,3])).
  % Encoding of a negative example
  next(n1, n2). next(n2, n3). next(n3, null). 
  value(n1, 2). value(n2, 1). value(n3, 3).
\end{minted}
%
which is a singly linked but not sorted list. To summarise, when
performing synthesis via classic ILP, negative examples are necessary
to avoid the predicate being \emph{too general}.
%



\paragraph{Challenges in Obtaining Negative Examples.}
What makes things worse is that ILP systems rely on
\emph{representative} negative examples to correctly prune the
generality, which are hard to obtain automatically.
%
% The difference between positive and negative examples is that, a
% good set of positive examples only needs to guarantee that all
% instances follow the predicate, while a good set of negative
% examples needs to cover all possible ways that the predicate can be
% wrong (\ie, being too general).
%
The difference between positive and negative examples is that, a good
set of positive examples (\(\mathbf{Pos}\)) only needs to guarantee
that all instances follow the predicate (\(\mathcal{P}\)), while a
good set of negative examples (\(\mathbf{Neg}\)) need to be much more
elaborated, so it could cover any possible way in which the predicate
can be wrong. This difference is expressed by the following
quantifications:
\[
  \forall \mathbf{e^+} \in \mathbf{Pos}, \mathcal{P}(\mathbf{e^+})
  \quad \text{vs.} \quad
  \forall \mathcal{P'} \subset \mathcal{P}, \exists \mathbf{e^-} \in \mathbf{Neg},  \mathcal{P'}(\mathbf{e^-}) \land \neg\mathcal{P}(\mathbf{e^-})
\]
That is, unlike a good positive example set, which is only quantified
over the examples, a good negative example set is quantified over the
\emph{predicates} and the examples, which makes it much harder to
achieve. As a concrete example of this phenomenon, imagine learning a
predicate for balanced binary trees.
%
A good set of negative examples would contain instances where
(a)~the height of the left subtree is too large, (b)~the height of the
right subtree is too large, (c)~the imbalance manifests recursively in
both left and right subtrees.
%
Without all these rather specific negative instances, it is possible
to learn a predicate with a constraint on the subtrees
\pcode{height(Left) <= height(Right) + 1}, which is not wrong but is
imprecise. This is not just an issue for SL predicates domain, but
also for general logical learning, witnesses by the fact that in
existing ILP benchmarks~\cite{cropper2021learning,thakkar2021example}
and specification mining framework \cite{10.1145/3622876} high-quality
negative examples are often crafted manually.

%\vspace{-3pt}
\begin{figure}[t]
%  \abovecaptionskip=4pt
%  \belowcaptionskip=-10pt
  \centering
  \includegraphics[width=0.5\textwidth]{figure/examples.pdf}
  \caption{The effect of positive and negative examples on search.}
      \label{fig:illustration}
\end{figure}


This state of affairs brings us to the two key novel ideas of this
work that enable efficient synthesis of SL predicates only from
positive examples.

\subsubsection{Key Idea 1: Learning with Specificity.}
\label{sec:most-specific}


%
As discussed above, without negative examples a solution delivered by
\popper, while valid, may not be specific enough.
%
To provide more intuition on the space of possible design choices in
finding the best solution, together with the reason why positive-only
learning is possible, let us consider the illustration in
\autoref{fig:illustration}. The ``up''/``down'' in this figure
(informally) means ``more general/specific'', where the top/bottom
are constant True/False (\ie, the lattice is defined by
subsumption~\cite{muggleton1995inverse}).
%
Providing two positive examples, \code{p1} and \code{p2}, restricts
the search space for the solution hypothesis to the intersection of
their own spaces, with the most general one chosen as the solution.
%
Adding a negative example \code{n1} provides more restrictions, thus
allowing for more specific most-general solution.
%
From this diagram, one can see that, even without a negative example,
we can have a
\emph{tighter}~\cite{DBLP:journals/pacmpl/AstorgaSDWMX21} solution
(\ie, with stronger restrictions given the same number of clauses) if
we consider not the most general, but the most specific candidate in
the intersection of the search spaces defined by \code{p1} and
\code{p2} (generally, positive examples only).

Therefore, the basic idea of our positive-only learning is: to learn
\emph{the most specific} predicate admitting all provided examples.
The only problem is: what is the definition of ``specificity''? As the
opposite of ``generality'' (the program with the smallest number of
constraints), it is not practical to take the \emph{largest}
hypothesis as the most specific, as it would lead to \emph{redundant}
constraints.
%
As an example, consider the following valid formulation of the sorted
linked list predicate that requires, in its second clause, that
\code{T} = \code{SY}~$\cup \set{\codeinmath{V}}$ and \code{S} =
\code{T}~$\cup \set{\codeinmath{V}}$:
%
\begin{minted}[fontsize=\small]{prolog}
srtl(X, S) :- empty(S), nullptr(X).
srtl(X, S) :- next(X, Y), value(X, V), srtl(Y, SY), insert(SY, V, T), insert(T, V, S).
\end{minted}
%
Clearly, the last conjunction \pcode{insert(T, V, S)} is redundant and
can be removed because of the properties of the \pcode{insert(...)}
predicate.
%


To eliminate such candidates with redundancies, our approach for
positive-only learning encodes intrinsic logical properties of customised
predicates to \emph{minimise} the generated SL predicates.
%
Our tool comes with a pre-defined collection of properties of
predicates for common arithmetic (\eg, calculation, comparison) and
set operations (\eg, insertion, union) that are included into the
synthesis automatically.
%
More customised predicates can be added
by the user (with additional clause minimisation rules). 
%
After performing the minimisation hinted above (detailed in
\autoref{sec:normalise}), we define the \emph{specificity} of a predicate
candidate based on its size \wrt other available candidates
(\cf~\autoref{sec:tool}). The solution that is locally-optimal (\ie, the
strongest in the search space) will be adopted as the most specific predicate that is implied by
all the positive examples.



\subsubsection{Key Idea 2: Separation Logic-Based Pruning.}
\label{sec:pruning}



The domain of our synthesis, \ie, Separation Logic, provides effective
ways to prune the search space and accelerate the synthesis process. 
%
Postponing the detailed explanation of the optimisations until
\autoref{sec:SLsynthesis}, as an example, consider an important
property the separating conjunction stating that the fact
\code{x}~$\mapsto$~\code{a * y} $\mapsto$ \code{b} implies
\code{x}~$\neq$~\code{y} because of the disjointness assumption
enforced by \code{*}.
%
This property can be encoded as a pruning strategy via ASP integrity
constraints (\autoref{sec:asp}) that are generated by our tool for
each synthesis task.
%
% \begin{minted}[fontsize=\scriptsize]{prolog}
%  :- clause(C), pointer(X), #count{V: body_lit(C, next, (X, V))} > 1.
% \end{minted}
%
Even for a \emph{doubly linked list}, one of the simplest predicates
in our benchmark (\cf~\autoref{sec:done}), without such optimisations,
the synthesis time is increased from 3 to 339 seconds; the synthesis
of more complex predicates does not terminate in 20 minutes without
SL-specific pruning.

\subsection{Automatically Generating Positive Examples}

So far, we assumed that the positive examples are provided by the
user, in the format of memory graphs. 
%
In practice, one may expect that such examples can be obtained in a
more automated way, \eg, from the available programs that manipulate
with the respective data structures.
%
For instance, an existing work on shape analysis~\cite{le2019sling} uses
a debugger for extracting memory graphs from a program's execution,
with the assumption that (1)~the user indicates the line of code to
extract the memory graph, and (2)~an input for the program is provided.
%
Unfortunately, this rules out a large set of programs that
\emph{expect} a data structure rather than generate one: without a
suitable input we simply cannot run them to obtain a graph, and constructing
such input is a task not much easier than encoding a memory graph manually.

% However, considering a function "removing a certain element from a
% binary search tree", we cannot obtain the trees without a constructor
% function, thus the memory graphs cannot be extracted because of
% non-executability of the function itself.

To address this issue, we implemented \ggen---a tool that can
automatically generates positive example in the form of arbitrary valid
memory graphs of a data structure from the program that manipulates
with the structure, without requiring the user to provide any input.
%
The only assumption we make is that the program in question is
instrumented with test assertions, which can be used to filter out the
invalid memory graphs from the randomly generated ones. We further
show in \autoref{sec:verification} that \ggen can effectively generate
input graphs for synthesising SL predicates from real
heap-manipulating programs, reducing the specification burden for
proving their correctness.

% \vspace{-5pt}


\begin{figure}[!t]
  \centering
  % \abovecaptionskip=5pt
  % \belowcaptionskip=-15pt
      
      \begin{adjustbox}{width=0.96\textwidth}
        \section{Problem definition}
\label{sec:problem-def}

The first step in building and using a \CSE{} or SciML model is defining the problem scope: the model's intended purpose, application domain and operating environment, required quantities of interest (QoI) and their scales, and how prior knowledge informs model conceptualization.

\subsection{Model purpose}

\begin{essrec}[Specify prior knowledge and model purpose]
Define the model's intended use and document the essential model properties that must be satisfied. Document any known limitations and constraints of the chosen approach. This ensures appropriate data selection and physics-informed objectives while preventing model misuse outside its intended scope.
\end{essrec}

A SciML model's purpose, as discussed in Section~\ref{sec:scope}, dictates all subsequent modeling choices.
This purpose determines required outer-loop processes and essential properties.

To highlight the importance of specifying the target outer-loop process, consider a model used for explanatory modeling. An explanatory model must simulate all system processes, like ice-sheet thickness and velocity evolution. In contrast, a risk assessment model needs only decision-relevant quantities, like ice-sheet mass loss under varying emissions scenarios. Design and control models, meanwhile, have different requirements than those for risk assessment.
The model purpose dictates the data types and formulations needed to train a SciML model. The impact of this purpose on data requirements and physics-informed objectives varies by application domain. Thus, the exact model formulation should be chosen in light of these problem-specific considerations.


\subsection{Verification, calibration, validation and application domain}

\begin{essrec}[Specify verification, calibration, validation, and application domains]
Define the specific conditions under which the model will operate across the verification, calibration, validation, and prediction phases. These domains are specified by relevant boundary conditions, forcing functions, geometry, and timescales. Account for potential differences between these domains and address any data distribution shifts that could affect model performance. This ensures the selected model architecture and training data align with the intended use while preventing unreliable predictions when operating outside validated conditions.
\end{essrec}

The trustworthy development and deployment of a model (see Figure~\ref{fig:model-development}) requires using the model in regards to verification, calibration, validation, and application domains. These domains are defined by the conditions under which the model operates during these respective phases and must be defined before model construction because they determine viable model classes. Key features include boundary conditions, forcing functions, geometry, and timescales. For ice sheets, examples include surface mass balance, land mass topography, and ocean temperatures.

Each domain will often require the prediction of different quantities of interest under different conditions. Moreover the complexity of the processes being modeled typically increase when transitioning from verification to calibration, to validation to prediction. Additionally the amount of data to complement or inform the model decreases as we move through these domains. For example, the verification domain for our ice-sheet examplar predicts the entire state of the ice-sheet for simple manufactured or analytical solutions. The calibration domain predicts Humboldt Glacier surface velocity under steady-state preindustrial conditions. The validation domain predicts grounding-line change rates from the first decade of this century. The application domain predicts glacier mass change in 2100. Figure~\ref{fig:computational-domains} illustrates these distinct domains. When transitioning between domains, data shifts across domains must be considered. For example, a model trained only on calibration data from recent ice-sheet forcings may fail to predict ice-sheet properties under different future conditions.

\begin{figure}[htb]
    \centering
    \includegraphics[width=0.65\linewidth]{application-domain.pdf}
    \caption{Verification, validation, calibration and application domains.}
    \label{fig:computational-domains}
\end{figure}


\subsection{Quantities of interest}

\begin{essrec}[Carefully select and specify the quantities of interest]
Select and specify the model outputs (quantities of interest, QoI) required for the intended use, considering their form and scale. For risk assessment and design applications, identify the minimal set of QoIs needed for decision-making or optimization. For explanatory modeling, specify the broader range of QoIs needed to capture system behavior. This choice fundamentally determines the required model complexity, training data requirements, and computational approach needed to achieve reliable predictions.
\end{essrec}

Quantities of interest (QoI) are the model outputs required by users. Their form and scale depend on modeling purpose and application domain. We now discuss key considerations for QoI selection.

Risk assessment requires reproducing only decision-critical QoI. For ice-sheets, these include sea-level rise from mass loss and infrastructure damage costs. Design applications similarly need few QoI to evaluate objectives and constraints, like thermal and structural stresses in aerospace vehicles. Design models need accurate QoI predictions only along optimizer trajectories\footnote{For each design iteration the model may still need to be accurate across all uncertain model inputs}, while risk assessment models must predict across many conditions. Explanatory modeling demands more extensive QoI sets, such as complete ice-sheet depth and velocity fields for studying calving. Therefore, simple surrogates often suffice for risk assessment and design, but explanatory modeling may require operators or reduced order models.


\subsection{Model conceptualization}


\begin{essrec}[Select and document model structure]
Select a model structure that fits the model's purpose, domain, and quantities of interest based on relevant prior knowledge such as conservation laws or system properties. Document the alternative model structures considered and the reasoning behind the final selection, including how available resources and computational constraints influenced the choice. This systematic approach ensures the model balances usability, reliability, and feasibility while maintaining transparency about structural assumptions and limitations.
\end{essrec}

Model conceptualization, which follows problem definition, involves selecting model structure based on prior knowledge. While essential to \CSE{} model development~\cite{Jakeman_LN_EMS_2006}, this step requires clear identification of the application domain and relevant QoI.

Model structure selection draws on key prior knowledge: conservation laws, system invariances like rotational and translational symmetries. These guide method selection---for example, symplectic time integrators~\cite{ruth1983canonical} preserve system dynamics properties. Moreover, this knowledge informs and justifies the selection of candidate model structures.
A \CSE{} modeler chooses between model types like lumped versus distributed PDE models, and linear versus nonlinear PDEs. The optimal choice depends on application domain, QoI, and available resources. For example, linear PDEs may introduce more error but their lower computational cost enables better error and uncertainty characterization for tasks like optimal design.
Similar considerations guide SciML model selection. For example, Gaussian processes excel at predicting scalar QoI with few inputs and limited data, but become intractable for larger datasets without variational inference approximations~\cite{Liu_CO_KBS_2018}. In contrast, deep neural networks handle high-dimensional data but require large datasets. The intended use also shapes model structure and training, e.g., optimization applications require controlling derivative errors~\cite{bouhlel2020scalable,o2024derivative} to ensure convergence~\cite{cao2024lazy,luo2023efficient}. These approximation errors must be understood and quantified where possible.

\CSE{} has a strong history of using prior knowledge to formulate governing equations for complex phenomena and deriving numerical methods that respect important physical properties. However, all models are approximate and the best model must balance usability, reliability, and feasibility~\cite{Hamilton_PSFJEMS_2022}. While SciML methods can be usable and feasible, more attention is needed to establish their trustworthiness. In the following two sections, we discuss how \CSE{} V\&V can improve the trustworthiness of SciML research.

\section{Verification}
\label{sec:verification}

Verification increases the trustworthiness of numerical models by demonstrating that the numerical method can adequately solve the equations of the desired mathematical model and the code correctly implements the algorithm. Verification consists of code verification and solution verification, which enhance credibility and trust in the model's predictions. Code and solution verification are well-established in \CSE{} to reduce algorithmic errors. However, verification for SciML models has received less attention due to the field's young age and unique challenges. Moreoever, because SciML models heavily rely on data, unlike \CSE{} models, existing \CSE{} verification notions need to be adapted for SciML.

\subsection{Code verification}
\label{sec:code-verification}

\begin{essrec}[Verify code implementation with test problems]
Evaluate the SciML model's accuracy on simple manufactured test problems using verification data that is independent from training data. Assess how the model error responds to variations in training data samples and optimization parameters while increasing both model complexity and training data size. This systematic testing approach reveals implementation issues, quantifies the impact of sampling and optimization choices, and establishes confidence in the model's numerical implementation.
\end{essrec}

Code verification ensures that a computer code correctly implements the intended mathematical model. For \CSE{} models, this involves confirming that numerical methods and algorithms are free from programming errors (``bugs"). PDE-based \CSE{} models commonly use the method of manufactured solutions (MMS) to verify code on arbitrarily complex solutions. MMS substitutes a user-provided solution into the governing equations, then differentiates it to obtain the exact forcing function and boundary conditions. These solutions check if the code produces the known theoretical convergence rate as the numerical discretization is refined. If the observed order of convergence is less than theoretical, causes such as software bugs, insufficient mesh refinement, or singularities and discontinuities affecting convergence must be identified.

Code verification for SciML models is important but challenging due to the large role of data and nonconvex numerical optimization. Three main challenges limit code verification for many SciML models.
First, while theoretical analysis of SciML models is increasing~\cite{schwab2019deep,schwab2023deep,opschoor2022exponential,leshno1993multilayer,lanthaler2023curse,kovachki2021universal,kovachki2023neural}, many models like neural networks do not generally admit known convergence rates outside specific map classes~\cite{schwab2019deep,schwab2023deep,opschoor2022exponential,herrmann2024neural}, despite their universal approximation properties~\cite{hornik1989multilayer,cybenko1989approximation,leshno1993multilayer}.
Second, generalizable procedures to refine models, such as consistently refining neural-network width and depth as data increases, do not exist.
Finally, regardless of data amount and model unknowns, modeling error often plateaus at a much higher level than machine precision due to nonconvex optimization issues like local minima and saddle points~\cite{Dauphin_PGCGB_NIPS_2014,Bottouleon_CN_SIAMR_2018}.

Developing theoretical and algorithmic advances to address the three main challenges limiting code verification can substantially improve the trustworthiness of SciML models. Convergence-based code verification is currently possible only for certain SciML models with theory that bounds approximation errors in terms of model complexity and training data amount, such as operator methods~\cite{Turnage_et_al_arxiv_2024}, polynomial chaos expansions~\cite{Cohen_M_SMAIJCM_2017,xiu2002wiener}, and Gaussian processes~\cite{Burt_RV_PMLR_2019}.

For SciML models without supporting theory, convergence tests should still be conducted and reported. Studies providing evidence of model convergence engender greater trustworthiness than those that do not, even when the empirically estimated convergence rate cannot be compared to theoretical rates. For example, observing Monte Carlo-type sampling rates in a regime of interest for a fixed overparametrized model can provide intuition into whether the model should be enhanced.

To account for the heavy reliance of SciML models on training data optimization, code verification should be adapted in two ways.
First, report errors in the ML model for a given complexity and data amount for different realizations of the training data to quantify the impact of sampling error, which is not present in \CSE{} models.
Second, because most SciML algorithms introduce optimization error, conduct verification studies that artificially generate data from a random realization of an ML model, then compare the recovered parameter values with the true parameter values or compare the predictions of the learned and true approximations, or at the very least compare the predictions of the two models. Additionally, quantify the sensitivity of the SciML model error to randomness in the optimization by varying the random seed and initial guess passed to the optimizer (see Section~\ref{sec:loss-and-opt}).
All verification tests must employ test data or \emph{verification data}, independent of the training data, to measure the accuracy of the SciML model.


\subsection{Solution verification}

\begin{essrec}[Verify solution accuracy with realistic benchmarks]
Test the model's performance on well-designed, realistic benchmark problems that reflect the intended application domain. Quantify how the model error varies with different training data samples and optimization parameters. When feasible, examine error patterns across different model complexities and data amounts; otherwise, focus on verifying the specific configuration intended for deployment. This ensures the model meets accuracy requirements under realistic conditions while accounting for uncertainties in training and optimization.
\end{essrec}

Code verification establishes a code's ability to reproduce known idealized solutions, while solution verification, performed after code verification, assesses the code's accuracy on more complex yet tractable problems defined by more realistic boundary conditions, forcing, and data. For example, code verification of ice sheets may use manufactured solutions, whereas solution verification may use more realistic MISMIP benchmarks~\cite{Cornford_et_al_TC_2020}. In solution verification, the numerical solution cannot be compared to a known exact solution, and the convergence rate to a known solution cannot be established. Instead, solution verification must use other procedures to estimate the error introduced by the numerical discretization.

Solution verification establishes whether the exact conditions of a model result in the expected theoretical convergence rate or if unexpected features like shocks or singularities prevent it. The most common approach for \CSE{} models compares the difference between consecutive solutions as the numerical discretization is refined and uses Richardson extrapolation to estimate errors. A posteriori error estimation techniques that require solving an adjoint equation can also be used.

While thorough solution verification of CSE models is challenging, these difficulties are further amplified for SciML models. Currently, solution verification of SciML models simply consists of evaluating a trained model's performance using test data separate from the training data. However, this is insufficient as solution verification requires quantifying the impact of increasing data and model complexity on model error. Yet, unfortunately, performing a posteriori error estimation for many SciML models using techniques like Richardson extrapolation is difficult due to the confounding of model expressivity, statistical sampling errors, and variability introduced by converging to local solutions or saddle points of nonconvex optimization, making it challenging to monotonically decrease the error of SciML models such as neural networks. 

Until convergence theory for SciML models improves and automated procedures are developed to change SciML model hyperparameters as data increases, solution verification of SciML models should repeat the sensitivity tests proposed for code verification (Section~\ref{sec:code-verification}) with two key differences:
First, verification experiments used to generate verification data must be specifically designed for solution verification, as not all verification data equally informs solution verification efforts, similar to observations made when creating validation datasets for \CSE{} models~\cite{Oberkampf_T_PAS_2002}. See Section~\ref{sec:data-sources} for more information on important properties of verification benchmarks.
Second, while ideally the convergence of SciML errors on realistic benchmarks should be investigated, it may be computationally impractical. Thus, solution verification should prioritize quantifying errors using the model complexity and data amount that will be used when deploying the SciML model to its application domain.

\section{Validation}
\label{sec:validation}

Verification establishes if a model can accurately produce the behavior of a system described by governing equations. In contrast, validation assesses whether a \CSE{} model's governing equations---or data for SciML models---and the model's implementation can reproduce the physical system's important properties, as determined by the model's purpose.

Validation requires three main steps: (1) solve an inverse problem to calibrate the model to observational data; (2) compare the model's output with observational data collected explicitly for validation; and (3) quantify the uncertainty in model predictions when interpolating or extrapolating from the validation domain to the application domain. We will expand on these steps below.
But first note that the issues affecting the verification of SciML models also affect calibration and validation. Consequently, we will not revisit them here but rather will highlight the unique challenges in validating SciML models.

\subsection{Calibration}

\begin{essrec}[Perform probabilistic calibration]
Calibrate the trained SciML model using observational data to optimize its predictive accuracy for the application domain. Implement Bayesian inference when possible to generate probabilistic parameter estimates and quantify model uncertainty. Choose calibration metrics that account for both model and experimental uncertainties, and select calibration data strategically to maximize information content within experimental constraints. This approach enables reliable uncertainty estimation and optimal use of available observational data.
\end{essrec}

Once a \CSE{} model has been verified, it must be calibrated to match experimental data that contains observational noise. This calibration requires solving an inverse problem~\cite{Stuart_AN_2010}, which can be either deterministic or statistical (e.g., Bayesian). The deterministic approach formulates the inverse problem as a (nonlinear) optimization problem that minimizes the mismatch between model and experimental data. This formulation requires regularization to ensure well-posedness, typically chosen using the L-curve~\cite{hansen1999curve} or the Morozov discrepancy principle~\cite{anzengruber2009morozov}. The Bayesian approach replaces the misfit with a likelihood function based on the noise model, while using a prior distribution for regularization. This prior distribution ensures well-posedness while encoding typical parameter ranges and correlation lengths. We recommend Bayesian methods for calibration because they provide insight into the uncertainty of the reconstructed model parameters. 

The calibration of SciML operator, reduced-order, and hybrid CSE-SciML models is distinct from SciML training and follows similar principles to \CSE{} model calibration. These models are first trained using simulation data for solution verification. Next, observational data (called \emph{calibration data}) determines the optimal model input values that match experimental outputs. For instance, calibrating a SciML ice-sheet model such as that of Ref.~\cite{He_PKS_JCP_2023} requires finding optimal friction field parameters of a trained SciML model, which best predict observed glacier surface velocities, given the noise in the observational data.

Calibration typically improves a model's predictive accuracy on its application domain, but the informative value of calibration data varies significantly. Therefore, researchers should select calibration data strategically to maximize information content within their experimental budget. See Section~\ref{sec:data-sources} for further discussion on collecting informative data.

\subsection{Model validation}

\begin{essrec}[Validate model against purpose-specific requirements]
Define validation metrics that align with the model's intended purpose. Then validate the model using independent data that was not used for training or calibration, ensuring it captures essential physics and boundary conditions of interest. If validation reveals inadequate performance, iterate by collecting additional training data, refining the model structure, or gathering more calibration data until the model achieves satisfactory accuracy for its intended application. This systematic approach will help ensure the model meets stakeholder requirements while maintaining scientific rigor.
\end{essrec}

Model validation is the ``substantiation that a model within its domain of applicability possesses a satisfactory range of accuracy consistent with the intended application of the model''~\cite{Refsgaard_H_AWR_2004}. Validation involves comparing computational results with observational data, then determining if the agreement meets the model's intended purpose~\cite{Lee_et_all_AIAA_2016}. For \CSE{} models with unacceptable validation agreement, modelers must either collect additional calibration data or refine the model structure until reaching acceptable accuracy. SciML models follow a similar iterative process but offer an additional option: to collect more training data.

Model validation must occur after calibration and requires independent data not used for calibration or training. For our conceptual ice-sheet model, calibration matches surface velocities assumed to represent pre-industrial conditions, while validation assesses the calibrated model's ability to predict grounding line change rates at the start of this decade. Performance metrics must target the specific modeling purpose. For optimization tasks, metrics should measure the distance from true optima obtained via the SciML model or bound the associated error~\cite{cao2024lazy}. For uncertainty estimation, metrics should quantify errors in uncertainty statistics through moment discrepancies or density-based measures like (shifted) reverse and forward Kullback--Leibler divergences.
For explanatory SciML modeling, validation metrics must also assess physical fidelity: adherence to physical laws, conservation properties (such as mass and energy), and other constraints. As with verification, the validation concept should encompass \emph{data validation}, particularly whether training data adequately represents the application space.

Validation determines whether a model is acceptable for its specific purpose rather than universally correct. The definition of acceptable is subjective, depending on validation metrics and accuracy requirements established by model stakeholders in alignment with the problem definition and model purpose (see Section~\ref{sec:problem-def}). Moreoever, validation itself does not constitute final model acceptance, which must be based on model accuracy in the application domain, as discussed in Section~\ref{sec:prediction}.

Two additional considerations complete our discussion of model validation. First, this validation differs from the concept of \emph{cross validation}, which estimates ML model accuracy on data representative of the training domain during development. The validation described here assesses accuracy in a separate validation domain. Second, validation data varies in informative value. Validation experiments should ``capture the essential physics of interest, including all relevant physical modeling data and initial and boundary conditions required by the code''~\cite{Oberkampf_T_NED_2008}. Most critically, validation data must remain independent from training and calibration data. 

\subsection{Prediction}
\label{sec:prediction}

\begin{essrec}[Quantify prediction uncertainties]
Assess and quantify all sources of uncertainty affecting model predictions in the application domain, including numerical approximation errors, input and parameter uncertainties, sampling errors from finite training data, and optimization errors. Propagate these uncertainties through the model using appropriate techniques to estimate relevant statistics that match validation criteria. Define acceptance thresholds for prediction uncertainty to ensure the model's reliability for its intended use while acknowledging inherent limitations in uncertainty quantification.
\end{essrec}

Although extensive data may be available for model calibration, validation data is typically scarcer and may not represent the model's intended application domain. According to Schwer~\cite{Schwer_EWC_2007}, ``The original reason for developing a model was to make predictions for applications of the model where no experimental data could, or would, be obtained.'' Therefore, minimizing validation metrics at nominal conditions cannot sufficiently validate a model. Modelers must also quantify accuracy and uncertainty when predictions are extrapolated to the application domain.

SciML models, like \CSE{} models, are subject to numerous sources of uncertainty. The impact of these uncertainties on model predictions must be quantified. Several sources of uncertainty affect \CSE{} models. These include: numerical errors, from approximating the solution to governing equations; input uncertainty arises, which is caused by inexact knowledge of model inputs; parameter uncertainty, which stems from inexact knowledge of model coefficients; and model structure error representing the difference between the model and reality. SciML models contain all these uncertainties. They also incorporate additional uncertainties from sampling and optimization errors, as discussed previously.

Sampling error arises from training a model with a finite amount of possibly noisy data. For a fixed ML model structure and zero optimization error, this error decreases as the amount of data increases. Optimization error represents the difference between the optimized solution, which is often a local optimum, and the global solution for fixed training data. Optimization error can enter \CSE{} models during calibration. Optimization error affects SciML models more significantly because it occurs both during calibration and training. Linear approximations, for example, based on polynomials, achieve zero optimization error during training to machine precision. However, nonlinear approximations such as neural networks often produce non-trivial optimization errors. Stochastic gradient descent demonstrates this by producing different parameter estimates due to stochastic optimization randomness and initial guesses.

The identified sources of modeling uncertainty require parameterization for sampling. Expert knowledge typically guides the construction of prior distributions that represent parametric uncertainty. This parameterization should occur during problem definition. Bayesian calibration updates these priors into posterior distributions using calibration data. The model must then propagate all uncertainties onto predictions in the application domain. Monte Carlo quadrature accomplishes this propagation by drawing random samples from the uncertainty distributions. The method collects model predictions at these samples and computes empirical estimates of important statistics defined by validation criteria, such as mean and variance.

We emphasize the impact of all sources of error and uncertainty must be quantified. Simply estimating the impact of error caused by using finite sample sets, for example estimated by generative models such as variational autoencoders of Gaussian processes is insufficient. Moreover, complete elimination of uncertainty is impossible. Consequently, model acceptance, like validation, must rely on subjective accuracy criteria established through stakeholder communication. For instance, acceptance criteria for predicted sea-level change from melting ice-sheets at year 2100 may specify that prediction precision reaches 1\% of the mean value. Yet, engineering applications, such as those focused on aerospace design, may have much higher accuracy requirements.

The aforementioned Monte-Carlo based UQ procedure effectively quantifies the impact of parameterized uncertainties on model predictions. However, model structure error remains difficult to parameterize in both SciML and \CSE{} modeling. Validation can partially assess model structure error. However, experiments rarely cover all conditions of use. Specifically, validation tests only the model's interpolation ability within the convex hull of available data and assumptions. This limitation creates challenges when applying the model outside its validation domain. Some progress exists in quantifying extrapolation error for ``models based upon highly-reliable theory that is augmented with less-reliable embedded models''~\cite{Oliver_TSM_CMAME_2015}. However, such hybrid CSE-SciML models rely on well-established physics-based governing equations to support extrapolation confidence. Pure SciML models still require substantial research to develop reliable methods for estimating model structure uncertainty.


      \end{adjustbox}
      \caption{The Workflow of \tool}
      
    \label{fig:extended}
\end{figure}

%\vspace{-5pt}


\subsection{Putting It All Together}

We implemented our algorithm for inductive synthesis of SL predicates
as the core of our tool called \tool. \autoref{fig:extended} shows the
high-level workflow of \tool. Starting from the left, the user can
provide positive examples (\ie, structure-specific heap graphs) for
the synthesiser by either using our graph generator \ggen on programs
that expect the data structure instance (\cf \autoref{sec:generator}),
or by manually writing them (\eg, in the style of Story-Board Programming
\cite{singh2012spt}).
%
Given the graph examples, \tool synthesises an inductive predicate
definition, which can be further used for program verification in SL-based
verifiers (\autoref{sec:verification}), for program synthesis
(\autoref{sec:synthesis}), or by any other SL-based tool.


\section{MDMs Can Generate Low-Perplexity Sentence More Efficiently}
\label{sec:positive}

Although parallel sampling (\cref{eq:parallel_sample}) in MDM inference introduces a known distributional mismatch with respect to ground-truth data, the practical implications of this divergence in sampling quality remain insufficiently explored. The inherent sparsity and ambiguity of natural language suggest that such mismatches may not necessarily manifest as perceptible inconsistencies in the generated text. In this section, we adopt the $n$-gram language model as an analytical framework to rigorously investigate both the efficiency and quality of sampling in MDMs. Despite their simplicity, $n$-gram models, which capture local dependencies in language, have long served as foundational tools for a broad range of NLP tasks, including text generation, machine translation, and speech recognition. Notably, recent studies have demonstrated that $n$-gram models can achieve comparable performance with modern large language models across diverse datasets, highlighting their continued relevance in NLP research. Within this framework, we show that MDMs are capable of efficiently generating $n$-gram languages while preserving high output quality.

\textbf{$n$-Gram Language.} The $n$-gram language model provides a statistical framework for modeling natural language by estimating the likelihood of a token conditioned on its preceding \(n-1\) tokens. Formally, let $\gV$ be the vocabulary and $q$ be the $n$-gram model, given a sequence $\vx = (x_1, x_2, \dots, x_L)\in\gV^L$, the $n$-gram model approximates the probability of the sequence as
\begin{equation}
    q(\vx) = \prod_{i=1}^L q(x_i \mid x_{\max\{1,i-n+1\}}, \dots, x_{i-1}),
\end{equation}
where $q(x_i \mid x_{\max\{1,i-n+1\}}, \dots, x_{i-1})$ denotes the conditional probability of token $x_i$ given its preceding \(n-1\) tokens. 
%

To establish the main theoretical results, we begin with the following assumption:

\begin{assumption}[Learning with Small Error]
\label{ass:perfect_learning}
    Let $q$ denote the $n$-gram language model with vocabulary $\gV$, and let $p_\mathbf{\theta}$ represent the reverse model trained to approximate the reverse process of the $n$-gram language under a masking schedule $\alpha_t$. Assume there exists $\delta > 0$ such that the KL divergence between $p_\mathbf{\theta}$ and the reverse process distribution of the $n$-gram language is bounded by $\delta$, i.e.,
    \begin{equation*}
        \DKL{q_{0|t}(x_0^i \mid \vx_t)}{p_\mathbf{\theta}(x_0^i \mid \vx_t)} < \delta, \quad \forall\ t \text{ and } \vx_t.
    \end{equation*}
\end{assumption}

It is worth noting that $p_\mathbf{\theta}(x_0^i \mid \vx_t) = q_{0|t}(x_0^i \mid \vx_t)$ represents the optimal solution to the ELBO loss during training. \cref{ass:perfect_learning} implies that the MDM model is well-trained and approximates the ground-truth distribution with only a small error.

During MDM inference, the time interval $[0, 1]$ is discretized into $T$ steps, where $t_i = \frac{i}{T},\ i \in [T]$, and the reverse process is solved sequentially. The following theorem establishes that the distribution generated by the reverse process, even with a small number of sampling steps, achieves near-optimal perplexity. Consequently, MDMs exhibit high efficiency in generating $n$-gram languages.

\begin{theorem}[Perplexity Bounds for $n$-Gram Language Generation]
\label{thm:acceleration_ngram}
    For any $n$-gram language $q$ and any $\epsilon > 0$, let $p_\mathsf{\theta}$ denote the reverse model. Under \cref{ass:perfect_learning}, there exists a masking schedule $\alpha_t$ such that, with $T = O\big( (\frac{2\log|\gV|}{\epsilon})^{(n-1)} \big)$ sampling steps, the perplexity of the MDM is upper-bounded by:
    \begin{equation}
        \begin{gathered}
            \operatorname{PPL}(p_\mathsf{\theta}) \leq \operatorname{PPL}(q) (1 + \delta + \epsilon), \\
            \text{where} \quad \operatorname{PPL}(p) = 2^{\mathbb{E}_{\vx \sim q} \frac{\log (p(\vx))}{|\vx|}}.
        \end{gathered}
    \end{equation}
\end{theorem}

\begin{remark}
For a given data distribution $q$, the perplexity of a language model $p$ achieves its global minimum when $p = q$.
\end{remark}

\cref{thm:acceleration_ngram} demonstrates that MDMs can efficiently generate $n$-gram languages with high fidelity. To ensure a perplexity gap of at most $\epsilon$ during sampling, the number of required sampling steps is bounded by $O\big( \big(\frac{2 \log |\gV|}{\epsilon}\big)^{(n-1)} \big)$. This bound is independent of the sequence length $L$ and scales polynomially with respect to the logarithm of the vocabulary size $|\gV|$ and the inverse of $\epsilon$, with a polynomial degree of \(n-1\).

\textbf{Practical Insights.} Perplexity is a widely adopted metric for evaluating the quality of text generation models. Models with lower perplexity are typically better at producing coherent, fluent, and contextually appropriate text. In \cref{thm:acceleration_ngram}, we establish that MDMs can achieve near-optimal perplexity, underscoring their ability to generate high-quality text while maintaining sampling efficiency. Recent work has shown that $n$-gram models trained on one trillion tokens with a dynamic $n$ can achieve perplexity levels comparable to large language models across diverse datasets. For instance, when the median $n$ is approximately 7, the required number of sampling steps is significantly smaller than the sequence length $L$, even as $L$ increases. MDMs capitalize on this property by employing parallel sampling, which generates multiple tokens simultaneously at each step. Consequently, the number of sampling steps required by MDMs remains independent of $L$, offering substantial efficiency gains over autoregressive models, which scale linearly with $L$. This efficiency enables MDMs to handle long-sequence generation tasks with efficiency while maintaining high-quality outputs. 
\section{Computational lower bound for learning stochastic block model}\label{sec:lb-learning}

\subsection{Computational lower bound for learning the edge connection probability matrix}

In this section, we prove \cref{thm:lb-edge-probability} by showing that there exists an efficient algorithm that reduces testing to learning in SBM. 
The reduction of algorithm \cref{alg:reduction-test-learning} is similar to that of \cref{alg:reduction-test-recovery}. The proof of \cref{thm:lb-edge-probability} is also a similar proof by contradiction to the proof of \cref{thm:main-theorem-weak-recovery}.

Before describing the algorithm, we restate \cref{thm:lb-edge-probability} here for completeness.
\begin{theorem}[Restatement of \cref{thm:lb-edge-probability}]
\label{thm:lb-edge-probability-restatement}
    Let $k,d\in \N^+$ be such that $k\leq n^{o(1)}, d\leq o(n)$.
    Assume that for any $d'\in \N^+$ such that $0.999 d\leq d'\leq d$, Conjecture \ref{conj:eldlr} holds with distribution $P$ given by $\SSBM(n,\frac{d'}{n},\e,k)$ and distribution $Q$ given by \Erdos-\Renyi graph model $\bbG(n, \frac{d'}{n})$. 
    Then given graph $G\sim \SSBM(n,\frac{d}{n},\e,k)$, no $\exp\Paren{n^{0.99}}$ time algorithm can output $\theta\in [0,1]^{n\times n}$ achieving error rate $\normf{\theta-\thetanull}^2\leq 0.99kd/4$ with constant probability, where $\thetanull$ is the ground truth sampled edge connection probability matrix.
\end{theorem}

The reduction that we consider is the following.

\begin{algorithmbox}[Reduction from testing to learning]
    \label{alg:reduction-test-learning}
    \mbox{}\\
    \textbf{Input:} A random graph $G$ with equal probability sampled from \Erdos-\Renyi model or stochastic block model. \\
    \textbf{Output:} Testing statistics $g(Y)\in \R$, where $Y$ is the centered adjacency matrix\\
    \textbf{Algorithm:} 
    \begin{enumerate}[1.]
        \item Obtain subgraph $G_1$ by subsampling each edge with probability $1-\eta=0.999$, and let $G_2= G\setminus G_1$. 
        \item Run learning algorithm on $G_1$, and obtain estimator $\hat{\theta}\in \R^{n\times n}$
        \item Obtain $\hat{M}$ by running correlation preserving projection on $\hat{\theta}-\frac{d}{n}\Ind \Ind^{\top}$ to the set $\cK=\Set{M\in [-1,1]^{n\times n}: M+\frac{1}{k} \Ind \Ind^{\top} \succeq 0 \,, \Tr(M + \frac{1}{k} \Ind \Ind^{\top}) \leq n}$. 
        \item Construct the testing statistics $g(Y)=\iprod{\hat{M},Y_2-\frac{\eta d}{n}\Ind \Ind^{\top}}$, where $Y_2$ is the adjacency matrix for the graph $G_2$.
    \end{enumerate}
\end{algorithmbox}

Before proving \cref{thm:lb-edge-probability}, we first show the relationship between learning edge connection probability and weak recovery.
 \begin{lemma}\label[lemma]{lem:reduction-learning-recovery}
     Consider the distribution of $\SSBM(n,\frac{d}{n},\e,k)$ with $d\le n^{o(1)}$. 
     Suppose give graph $Y\sim \SSBM(n,\frac{d}{n},\e,k)$, the estimator $\hat{\theta}\in \R^{n\times n}$ achieves error rate $\normf{\hat{\theta}- \thetanull}\leq \frac{1}{2}\sqrt{0.99kd}$ with constant probability, then $\hat{\theta}-d/n$ achieves weak recovery when $\e^2 d\geq 0.99k^2$.
 \end{lemma}
\begin{proof}
By the relation between edge connection probability matrix $\thetanull$ and the community matrix $M^\circ$, We have
    \begin{equation*}
        \iprod{\hat{\theta}-\frac{d}{n}\Ind \Ind^\top,M^\circ}=\iprod{\hat{\theta}-\theta^\circ,M^\circ}+\iprod{\theta^\circ-\frac{d}{n}\Ind \Ind^\top,M^\circ}=\iprod{\hat{\theta}-\theta^\circ,M^\circ}+\iprod{\frac{\e d}{n}M^\circ,M^\circ}\,.
    \end{equation*}
    For the first term, since with constant probability, $\normf{\hat{\theta}-\theta^\circ}\leq \sqrt{0.99kd}$, we have
    \begin{equation*}
      \Abs{\iprod{\hat{\theta}-\theta^\circ,M^\circ}}\leq \normf{M^\circ}\normf{\hat{\theta}-\theta^\circ}\leq 
        \normf{M^\circ} \sqrt{0.99kd}\,.
    \end{equation*}
    For the second term, since with overwhelming high probability, $\normf{M^\circ}\geq \frac{n}{\sqrt{k}}(1-\frac{1}{k})$, we have
    \begin{equation*}
        \iprod{\frac{\e d}{n}M^\circ,M^\circ}=\frac{\e d}{n}\normf{M^\circ}^2\geq \frac{\e d }{2\sqrt{k}} \normf{M^\circ}\,.
    \end{equation*}
    Therefore, when $\e^2 d> 0.999 k^2$, we have 
    \begin{equation*}
        \iprod{\hat{\theta}-\frac{d}{n}\Ind \Ind^{\top},M^\circ}\geq \frac{\e d }{2\sqrt{k}} \normf{M^\circ}-\normf{M^\circ} \frac{\sqrt{0.99kd}}{2}\geq \Omega\Paren{\frac{\e d \normf{M^\circ}}{\sqrt{k}}} \,.
    \end{equation*}
    On the other hand, by triangle inequality
    \begin{equation*}
        \Normf{\hat{\theta}-\frac{d}{n}\Ind \Ind^{\top}}\leq  \Normf{\hat{\theta}-\theta^\circ}+ \Normf{\theta^\circ-\frac{d}{n}\Ind \Ind^{\top}}\leq O(\sqrt{kd}+\frac{\e d}{\sqrt{k}}) \leq O\Paren{\e d/\sqrt{k}}\,,
    \end{equation*}
Therefore we have 
\begin{equation*}
    \iprod{\hat{\theta}-\frac{d}{n}\Ind \Ind^{\top},M^\circ}\geq \Omega(\normf{M^\circ}\cdot \normf{\hat{\theta}-\frac{d}{n}\Ind \Ind^{\top}})\,.
\end{equation*}
    We thus conclude that with constant probability, $\hat{\theta}-\frac{d}{n}\Ind \Ind^\top$ achieves weak recovery when $\e^2 d\geq 0.99k^2$.
\end{proof}
With \cref{lem:reduction-learning-recovery}, the proof of lower bound for learning the edge connection probability matrix of stochastic block model follows as a corollary.
\begin{proof}[Proof of \cref{thm:lb-edge-probability}]
    By \cref{lem:reduction-learning-recovery}, suppose an $\exp\Paren{n^{0.99}}$ time algorithm achieves error rate less than $0.99\sqrt{kd}$ in estimating the edge connection probability matrix, then in \cref{alg:reduction-test-learning}, $\hat{\theta}-\frac{d}{n}$ achieves weak recovery when $\e^2 d=0.99k^2$.
    We let $f(Y)=\mathbf{1}_{g(Y)\geq 0.001 \e^2 d^2/k}$. 

    We show that with constant probability under $P$, we have $f(Y)=1$.    
    We essentially follow the proof of \cref{lem:lb_sbm} with $\delta$ taken as a constant, except that we take a different strategy for bounding
    $\iprod{W_2-\tilde{W}_2, \hat{M}}$.
    By \cref{lem:spectral-concentration-sbm}, we have, with probability at least $1-o(1)$, the following spectral radius bounds on the symmetric random matrices
\begin{equation*}
    \normop{W_2-\tilde{W}_2}\leq O\Paren{\sqrt{d\log(n)}\cdot \sqrt{\frac{d}{n}}}\,.
\end{equation*}
Therefore, by Trace inequality, we have
\begin{equation*}
\begin{split}
|\iprod{W_2-\tilde{W}_2, \hat{M}}|
& = |\iprod{W_2-\tilde{W}_2, \hat{M}+\frac{1}{k\delta}\Ind \Ind^{\top}} - \iprod{W_2-\tilde{W}_2, \frac{1}{k\delta}\Ind \Ind^{\top}}| \\
& \leq |\iprod{W_2-\tilde{W}_2, \hat{M}+\frac{1}{k\delta}\Ind \Ind^{\top}}| + |\iprod{W_2-\tilde{W}_2, \frac{1}{k\delta}\Ind \Ind^{\top}}| \\
& \leq \normop{W_2-\tilde{W}_2} \Tr(\hat{M}+\frac{1}{k\delta}\Ind \Ind^{\top}) + \normop{W_2-\tilde{W}_2} \Tr(\frac{1}{k\delta}\Ind \Ind^{\top}) \\
& \leq O\Paren{\sqrt{d\log(n)}\cdot \sqrt{\frac{d}{n}} (1+\frac{1}{k})\frac{n}{\delta}}\\
& = O\Paren{(d+\frac{d}{k})\frac{\sqrt{n\log(n)}}{\delta}} \,.
\end{split}
\end{equation*}

    With the same reasoning, by \cref{lem:ub_ER}, with probability at least $1-\exp(-n^{0.001})$ under distribution $Q$, we have $f(Y)=0$. 
    Therefore, we have $\RPQ(f)\geq \exp(n^{0.001})$. 
    Since $f(A)$ can be evaluated in $O\Paren{\exp\Paren{n^{0.99}}}$ time, assuming conjecture \ref{conj:low-degree} we have
   \begin{equation*}
       R_{P,Q}(f)\coloneqq \frac{\E f(A)}{\sqrt{\text{Var}_Q(f(A))}} \lesssim \max_{\text{deg}(f)\leq n^{0.99}}\frac{\E f(A)}{\sqrt{\text{Var}_Q(f(A))}}\,.
   \end{equation*}
    On the other hand, by low-degree lower bound stated in \cref{thm:ldlr-sbm}, we have 
    \begin{equation*}
       \max_{\text{deg}(f)\leq n^{0.99}}\frac{\E f(A)}{\sqrt{\text{Var}_Q(f(A))}}\leq \exp(k^2)\,. 
    \end{equation*}
Since we have $\exp(n^{0.001})\gg\exp(k^2)$ when $k\leq n^{o(1)}$, this leads to a contradiction. 
\end{proof}

\subsection{Computational lower bound for learning graphon}
In this part, we give formal proof of \cref{thm:lb-learning-graphon}. 

\begin{theorem}[Restatement of \cref{thm:lb-learning-graphon}]
    Let $k,d\in \N^+$ be such that $k\leq O(1), d\leq o(n)$.
    Assume that Conjecture \ref{conj:low-degree} holds with distribution $P$ given by $\SSBM(n,\frac{d}{n},\e,k)$ and distribution $Q$ given by \Erdos-\Renyi graph model $\bbG(n, \frac{d}{n})$. 
    Then no $\exp\Paren{n^{0.99}}$ time algorithm can output a $\poly(n)$-block graphon function $\hat{W}:[0,1]\times [0,1]\to [0,1]$ such that $\GW(\hat{W},\Wnull) \leq \frac{d}{3n}\sqrt{\frac{k}{d}}$  with $1-o(1)$ probability under distribution $P$ and distribution $Q$(where $\Wnull$ is the underlying graphon of the corresponding distribution).
\end{theorem}
\begin{proof}
Let $W_0$ be the graphon function underlying the distribution $\bbG(n,\frac{d}{n})$ and $W_1$ be the graphon function underlying the distribution $\SSBM(n,\frac{d}{n},\e,k)$, we have $\GW(W_0,W_1)\geq \frac{d}{n}\sqrt{\frac{0.99k}{d}}$ when $\e^2 d\geq 0.99k^2$. 

Now suppose there is a polynomial time algorithm, which given random graph $G$ sampled from an arbitrary symmetric $k$-stochastic block model, outputs an $n$-block graphon function $\hat{W}:[0,1]\times [0,1]\to [0,1]$ achieving error $\frac{d}{3n}\sqrt{\frac{k}{d}}$ with probability $1-o(1)$.
Then one can construct the testing statistics by taking
\begin{equation*}
f(Y) =
\begin{cases}
    1, & \text{if } \GW(\hat{W}, W_0) \leq \frac{d}{3n} \sqrt{\frac{k}{d}} \\
    0, & \text{otherwise}
\end{cases}
\end{equation*}
We have $f(Y)=1$ with probability $1-o(1)$ under the distribution of symmetric stochastic block model $\SSBM(n,\frac{d}{n},\e,k)$.
By triangle inequality, we have $f(Y)=0$ with probability $1-o(1)$ under the distribution $\bbG(n,\frac{d}{n})$. 
Therefore we have $\RPQ(f)\geq \omega(1)$.

Now since the function $\hat{W}$ can be represented as a symmetric matrix with $\poly(n)$ number of rows and columns, and moreove since $W_0$ is a constant function,
\begin{equation*}
    \GW(\hat{W},W_0)= \int_0^1 \int_0^1 (\hat{W}(x,y)-W_0(x,y))^2 dx dy\,.
\end{equation*}
Therefore, the function $f(\cdot )$ can be evaluated in polynomial time. 
This contradicts the low-degree lower bound (\cref{thm:ldlr-sbm}) assuming \cref{conj:low-degree}.
\end{proof}

















\begin{table}[ht!]
\centering
\caption{\textbf{Generator Model Architecture}}
\label{tab:gen}
\resizebox{0.95\linewidth}{!}{%
\begin{tabular}{@{}cccc@{}}
\toprule
\textbf{Layer Type} & \textbf{Kernels} & \textbf{Dimensions}            & \textbf{Activation} \\
\toprule
\rowcolor{Gray}
Convolution         & 128              & (\# input channels +  1, 1)    & Linear              \\

Convolution         & 128              & (\# input channels / 2 + 1, 1) & ELU                 \\

\rowcolor{Gray}
Upsampling          & -                & (scale factor – 1, 1)          & -                   \\

Convolution         & 128              & (\# input channels + 1, 1)     & ELU                 \\

\rowcolor{Gray}
Convolution         & 128              & (\# input channels / 2 + 1, 1) & ELU                 \\

Concatenate         & -                & -                              & -                   \\

\rowcolor{Gray}
Convolution         & 256              & (\# input channels / 2 + 1, 1) & ELU                 \\

Concatenate         & -                & -                              & -                   \\

\rowcolor{Gray}
Convolution         & 512              & (\# input channels / 2 + 1, 3) & ELU                 \\

Concatenate         & -                & -                              & -                   \\

\rowcolor{Gray}
Convolution         & 1                & (\# input channels + 1, 1)     & None                \\
\bottomrule

\end{tabular}
}
\end{table}
\definecolor{darkgreen}{rgb}{0.0, 0.5, 0.0}
\definecolor{violet}{rgb}{0.56, 0.0, 1.0}
\section{Evaluation}
We apply our methodology to derive counterfactual policies for various MDPs, addressing three main research questions: (1) how does our policy's performance compare to the Gumbel-max SCM approach; (2) how do the counterfactual stability and monotonicity assumptions impact the probability bounds; and (3) how fast is our approach compared with the Gumbel-max SCM method?

\begin{figure*}
    \centering
    %
    \resizebox{0.6\textwidth}{!}{
        \begin{tikzpicture}[scale=1.0, every node/.style={scale=1.0}]
            \draw[thick, black] (-3, -0.25) rectangle (10, 0.25);
            %
            \draw[black, line width=1pt] (-2.5, 0.0) -- (-2,0.0);
            \fill[black] (-2.25,0.0) circle (2pt); %
            \node[right] at (-2,0.0) {\small Observed Path};
            
            %
            \draw[blue, line width=1pt] (1.0,0.0) -- (1.5,0.0);
            \node[draw=blue, circle, minimum size=4pt, inner sep=0pt] at (1.25,0.0) {}; %
            \node[right] at (1.5,0.0) {\small Interval CFMDP Policy};
            
            %
            \draw[red, line width=1pt] (5.5,0) -- (6,0);
            \node[red] at (5.75,0) {$\boldsymbol{\times}$}; %
            \node[right] at (6,0) {\small Gumbel-max SCM Policy};
        \end{tikzpicture}
    }\\
    %
    \subfigure[\footnotesize Lowest cumulative reward: Interval CFMDP ($312$), Gumbel-max SCM ($312$)]{%
        \resizebox{0.76\columnwidth}{!}{
             \begin{tikzpicture}
                \begin{axis}[
                    xlabel={$t$},
                    ylabel={Mean reward at time step $t$},
                    title={Optimal Path},
                    grid=both,
                    width=20cm, height=8.5cm,
                    every axis/.style={font=\Huge},
                    %
                ]
                \addplot[
                    color=black, %
                    mark=*, %
                    line width=2pt,
                    mark size=3pt,
                    error bars/.cd,
                    y dir=both, %
                    y explicit, %
                    error bar style={line width=1pt,solid},
                    error mark options={line width=1pt,mark size=4pt,rotate=90}
                ]
                coordinates {
                    (0, 0.0)  +- (0, 0.0)
                    (1, 0.0)  +- (0, 0.0) 
                    (2, 1.0)  +- (0, 0.0) 
                    (3, 1.0)  +- (0, 0.0)
                    (4, 2.0)  +- (0, 0.0)
                    (5, 3.0) +- (0, 0.0)
                    (6, 5.0) +- (0, 0.0)
                    (7, 100.0) +- (0, 0.0)
                    (8, 100.0) +- (0, 0.0)
                    (9, 100.0) +- (0, 0.0)
                };
                %
                \addplot[
                    color=blue, %
                    mark=o, %
                    line width=2pt,
                    mark size=3pt,
                    error bars/.cd,
                    y dir=both, %
                    y explicit, %
                    error bar style={line width=1pt,solid},
                    error mark options={line width=1pt,mark size=4pt,rotate=90}
                ]
                 coordinates {
                    (0, 0.0)  +- (0, 0.0)
                    (1, 0.0)  +- (0, 0.0) 
                    (2, 1.0)  +- (0, 0.0) 
                    (3, 1.0)  +- (0, 0.0)
                    (4, 2.0)  +- (0, 0.0)
                    (5, 3.0) +- (0, 0.0)
                    (6, 5.0) +- (0, 0.0)
                    (7, 100.0) +- (0, 0.0)
                    (8, 100.0) +- (0, 0.0)
                    (9, 100.0) +- (0, 0.0)
                };
                %
                \addplot[
                    color=red, %
                    mark=x, %
                    line width=2pt,
                    mark size=6pt,
                    error bars/.cd,
                    y dir=both, %
                    y explicit, %
                    error bar style={line width=1pt,solid},
                    error mark options={line width=1pt,mark size=4pt,rotate=90}
                ]
                coordinates {
                    (0, 0.0)  +- (0, 0.0)
                    (1, 0.0)  +- (0, 0.0) 
                    (2, 1.0)  +- (0, 0.0) 
                    (3, 1.0)  +- (0, 0.0)
                    (4, 2.0)  +- (0, 0.0)
                    (5, 3.0) +- (0, 0.0)
                    (6, 5.0) +- (0, 0.0)
                    (7, 100.0) +- (0, 0.0)
                    (8, 100.0) +- (0, 0.0)
                    (9, 100.0) +- (0, 0.0)
                };
                \end{axis}
            \end{tikzpicture}
         }
    }
    \hspace{1cm}
    \subfigure[\footnotesize Lowest cumulative reward: Interval CFMDP ($19$), Gumbel-max SCM ($-88$)]{%
         \resizebox{0.76\columnwidth}{!}{
            \begin{tikzpicture}
                \begin{axis}[
                    xlabel={$t$},
                    ylabel={Mean reward at time step $t$},
                    title={Slightly Suboptimal Path},
                    grid=both,
                    width=20cm, height=8.5cm,
                    every axis/.style={font=\Huge},
                    %
                ]
                \addplot[
                    color=black, %
                    mark=*, %
                    line width=2pt,
                    mark size=3pt,
                    error bars/.cd,
                    y dir=both, %
                    y explicit, %
                    error bar style={line width=1pt,solid},
                    error mark options={line width=1pt,mark size=4pt,rotate=90}
                ]
              coordinates {
                    (0, 0.0)  +- (0, 0.0)
                    (1, 1.0)  +- (0, 0.0) 
                    (2, 1.0)  +- (0, 0.0) 
                    (3, 1.0)  +- (0, 0.0)
                    (4, 2.0)  +- (0, 0.0)
                    (5, 3.0) +- (0, 0.0)
                    (6, 3.0) +- (0, 0.0)
                    (7, 2.0) +- (0, 0.0)
                    (8, 2.0) +- (0, 0.0)
                    (9, 4.0) +- (0, 0.0)
                };
                %
                \addplot[
                    color=blue, %
                    mark=o, %
                    line width=2pt,
                    mark size=3pt,
                    error bars/.cd,
                    y dir=both, %
                    y explicit, %
                    error bar style={line width=1pt,solid},
                    error mark options={line width=1pt,mark size=4pt,rotate=90}
                ]
              coordinates {
                    (0, 0.0)  +- (0, 0.0)
                    (1, 1.0)  +- (0, 0.0) 
                    (2, 1.0)  +- (0, 0.0) 
                    (3, 1.0)  +- (0, 0.0)
                    (4, 2.0)  +- (0, 0.0)
                    (5, 3.0) +- (0, 0.0)
                    (6, 3.0) +- (0, 0.0)
                    (7, 2.0) +- (0, 0.0)
                    (8, 2.0) +- (0, 0.0)
                    (9, 4.0) +- (0, 0.0)
                };
                %
                \addplot[
                    color=red, %
                    mark=x, %
                    line width=2pt,
                    mark size=6pt,
                    error bars/.cd,
                    y dir=both, %
                    y explicit, %
                    error bar style={line width=1pt,solid},
                    error mark options={line width=1pt,mark size=4pt,rotate=90}
                ]
                coordinates {
                    (0, 0.0)  +- (0, 0.0)
                    (1, 1.0)  +- (0, 0.0) 
                    (2, 1.0)  +- (0, 0.0) 
                    (3, 1.0)  +- (0, 0.0)
                    (4, 2.0)  += (0, 0.0)
                    (5, 3.0)  += (0, 0.0)
                    (6, 3.17847) += (0, 0.62606746) -= (0, 0.62606746)
                    (7, 2.5832885) += (0, 1.04598233) -= (0, 1.04598233)
                    (8, 5.978909) += (0, 17.60137623) -= (0, 17.60137623)
                    (9, 5.297059) += (0, 27.09227512) -= (0, 27.09227512)
                };
                \end{axis}
            \end{tikzpicture}
         }
    }\\[-1.5pt]
    \subfigure[\footnotesize Lowest cumulative reward: Interval CFMDP ($14$), Gumbel-max SCM ($-598$)]{%
         \resizebox{0.76\columnwidth}{!}{
             \begin{tikzpicture}
                \begin{axis}[
                    xlabel={$t$},
                    ylabel={Mean reward at time step $t$},
                    title={Almost Catastrophic Path},
                    grid=both,
                    width=20cm, height=8.5cm,
                    every axis/.style={font=\Huge},
                    %
                ]
                \addplot[
                    color=black, %
                    mark=*, %
                    line width=2pt,
                    mark size=3pt,
                    error bars/.cd,
                    y dir=both, %
                    y explicit, %
                    error bar style={line width=1pt,solid},
                    error mark options={line width=1pt,mark size=4pt,rotate=90}
                ]
                coordinates {
                    (0, 0.0)  +- (0, 0.0)
                    (1, 1.0)  +- (0, 0.0) 
                    (2, 2.0)  +- (0, 0.0) 
                    (3, 1.0)  +- (0, 0.0)
                    (4, 0.0)  +- (0, 0.0)
                    (5, 1.0) +- (0, 0.0)
                    (6, 2.0) +- (0, 0.0)
                    (7, 2.0) +- (0, 0.0)
                    (8, 3.0) +- (0, 0.0)
                    (9, 2.0) +- (0, 0.0)
                };
                %
                \addplot[
                    color=blue, %
                    mark=o, %
                    line width=2pt,
                    mark size=3pt,
                    error bars/.cd,
                    y dir=both, %
                    y explicit, %
                    error bar style={line width=1pt,solid},
                    error mark options={line width=1pt,mark size=4pt,rotate=90}
                ]
                coordinates {
                    (0, 0.0)  +- (0, 0.0)
                    (1, 1.0)  +- (0, 0.0) 
                    (2, 2.0)  +- (0, 0.0) 
                    (3, 1.0)  +- (0, 0.0)
                    (4, 0.0)  +- (0, 0.0)
                    (5, 1.0) +- (0, 0.0)
                    (6, 2.0) +- (0, 0.0)
                    (7, 2.0) +- (0, 0.0)
                    (8, 3.0) +- (0, 0.0)
                    (9, 2.0) +- (0, 0.0)
                };
                %
                \addplot[
                    color=red, %
                    mark=x, %
                    line width=2pt,
                    mark size=6pt,
                    error bars/.cd,
                    y dir=both, %
                    y explicit, %
                    error bar style={line width=1pt,solid},
                    error mark options={line width=1pt,mark size=4pt,rotate=90}
                ]
                coordinates {
                    (0, 0.0)  +- (0, 0.0)
                    (1, 0.7065655)  +- (0, 0.4553358) 
                    (2, 1.341673)  +- (0, 0.67091621) 
                    (3, 1.122926)  +- (0, 0.61281824)
                    (4, -1.1821935)  +- (0, 13.82444042)
                    (5, -0.952399)  +- (0, 15.35195457)
                    (6, -0.72672) +- (0, 20.33508414)
                    (7, -0.268983) +- (0, 22.77861454)
                    (8, -0.1310835) +- (0, 26.31013314)
                    (9, 0.65806) +- (0, 28.50670214)
                };
                %
            %
            %
            %
            %
            %
            %
            %
            %
            %
            %
            %
            %
            %
            %
            %
            %
            %
            %
                \end{axis}
            \end{tikzpicture}
         }
    }
    \hspace{1cm}
    \subfigure[\footnotesize Lowest cumulative reward: Interval CFMDP ($-698$), Gumbel-max SCM ($-698$)]{%
         \resizebox{0.76\columnwidth}{!}{
            \begin{tikzpicture}
                \begin{axis}[
                    xlabel={$t$},
                    ylabel={Mean reward at time step $t$},
                    title={Catastrophic Path},
                    grid=both,
                    width=20cm, height=8.5cm,
                    every axis/.style={font=\Huge},
                    %
                ]
                \addplot[
                    color=black, %
                    mark=*, %
                    line width=2pt,
                    mark size=3pt,
                    error bars/.cd,
                    y dir=both, %
                    y explicit, %
                    error bar style={line width=1pt,solid},
                    error mark options={line width=1pt,mark size=4pt,rotate=90}
                ]
                coordinates {
                    (0, 1.0)  +- (0, 0.0)
                    (1, 2.0)  +- (0, 0.0) 
                    (2, -100.0)  +- (0, 0.0) 
                    (3, -100.0)  +- (0, 0.0)
                    (4, -100.0)  +- (0, 0.0)
                    (5, -100.0) +- (0, 0.0)
                    (6, -100.0) +- (0, 0.0)
                    (7, -100.0) +- (0, 0.0)
                    (8, -100.0) +- (0, 0.0)
                    (9, -100.0) +- (0, 0.0)
                };
                %
                \addplot[
                    color=blue, %
                    mark=o, %
                    line width=2pt,
                    mark size=3pt,
                    error bars/.cd,
                    y dir=both, %
                    y explicit, %
                    error bar style={line width=1pt,solid},
                    error mark options={line width=1pt,mark size=4pt,rotate=90}
                ]
                coordinates {
                    (0, 0.0)  +- (0, 0.0)
                    (1, 0.504814)  +- (0, 0.49997682) 
                    (2, 0.8439835)  +- (0, 0.76831917) 
                    (3, -8.2709165)  +- (0, 28.93656754)
                    (4, -9.981082)  +- (0, 31.66825363)
                    (5, -12.1776325) +- (0, 34.53463233)
                    (6, -13.556076) +- (0, 38.62845372)
                    (7, -14.574418) +- (0, 42.49603359)
                    (8, -15.1757075) +- (0, 46.41913968)
                    (9, -15.3900395) +- (0, 50.33563368)
                };
                %
                \addplot[
                    color=red, %
                    mark=x, %
                    line width=2pt,
                    mark size=6pt,
                    error bars/.cd,
                    y dir=both, %
                    y explicit, %
                    error bar style={line width=1pt,solid},
                    error mark options={line width=1pt,mark size=4pt,rotate=90}
                ]
                coordinates {
                    (0, 0.0)  +- (0, 0.0)
                    (1, 0.701873)  +- (0, 0.45743556) 
                    (2, 1.1227805)  +- (0, 0.73433129) 
                    (3, -8.7503255)  +- (0, 30.30257976)
                    (4, -10.722092)  +- (0, 33.17618589)
                    (5, -13.10721)  +- (0, 36.0648089)
                    (6, -13.7631645) +- (0, 40.56553451)
                    (7, -13.909043) +- (0, 45.23829402)
                    (8, -13.472517) +- (0, 49.96270296)
                    (9, -12.8278835) +- (0, 54.38618735)
                };
                %
            %
            %
            %
            %
            %
            %
            %
            %
            %
            %
            %
            %
            %
            %
            %
            %
            %
            %
                \end{axis}
            \end{tikzpicture}
         }
    }
    \caption{Average instant reward of CF paths induced by policies on GridWorld $p=0.4$.}
    \label{fig: reward p=0.4}
\end{figure*}

\subsection{Experimental Setup}
To compare policy performance, we measure the average rewards of counterfactual paths induced by our policy and the Gumbel-max policy by uniformly sampling $200$ counterfactual MDPs from the ICFMDP and generating $10,000$ counterfactual paths over each sampled CFMDP. \jl{Since the interval CFMDP depends on the observed path, we select $4$  paths of varying optimality to evaluate how the observed path impacts the performance of both policies: an optimal path, a slightly suboptimal path that could reach the optimal reward with a few changes, a catastrophic path that enters a catastrophic, terminal state with low reward, and an almost catastrophic path that was close to entering a catastrophic state.} When measuring the average probability bound widths and execution time needed to generate the ICFMDPs, we averaged over $20$ randomly generated observed paths
\footnote{Further training details are provided in Appendix \ref{app: training details}, and the code is provided at \href{https://github.com/ddv-lab/robust-cf-inference-in-MDPs}{https://github.com/ddv-lab/robust-cf-inference-in-MDPs}
%
%
.}.

\subsection{GridWorld}
\jl{The GridWorld MDP is a $4 \times 4$ grid where an agent must navigate from the top-left corner to the goal state in the bottom-right corner, avoiding a dangerous terminal state in the centre. At each time step, the agent can move up, down, left, or right, but there is a small probability (controlled by hyper-parameter $p$) of moving in an unintended direction. As the agent nears the goal, the reward for each state increases, culminating in a reward of $+100$ for reaching the goal. Entering the dangerous state results in a penalty of $-100$. We use two versions of GridWorld: a less stochastic version with $p=0.9$ (i.e., $90$\% chance of moving in the chosen direction) and a more stochastic version with $p=0.4$.}

\paragraph{GridWorld ($p=0.9$)}
When $p=0.9$, the counterfactual probability bounds are typically narrow (see Table \ref{tab:nonzero_probs} for average measurements). Consequently, as shown in Figure \ref{fig: reward p=0.9}, both policies are nearly identical and perform similarly well across the optimal, slightly suboptimal, and catastrophic paths.
%
However, for the almost catastrophic path, the interval CFMDP path is more conservative and follows the observed path more closely (as this is where the probability bounds are narrowest), which typically requires one additional step to reach the goal state than the Gumbel-max SCM policy.
%

\paragraph{GridWorld ($p=0.4$)}
\jl{When $p=0.4$, the GridWorld environment becomes more uncertain, increasing the risk of entering the dangerous state even if correct actions are chosen. Thus, as shown in Figure \ref{fig: reward p=0.4}, the interval CFMDP policy adopts a more conservative approach, avoiding deviation from the observed policy if it cannot guarantee higher counterfactual rewards (see the slightly suboptimal and almost catastrophic paths), whereas the Gumbel-max SCM is inconsistent: it can yield higher rewards, but also much lower rewards, reflected in the wide error bars.} For the catastrophic path, both policies must deviate from the observed path to achieve a higher reward and, in this case, perform similarly.
%
%
%
%
\subsection{Sepsis}
The Sepsis MDP \citep{oberst2019counterfactual} simulates trajectories of Sepsis patients. Each state consists of four vital signs (heart rate, blood pressure, oxygen concentration, and glucose levels), categorised as low, normal, or high.
and three treatments that can be toggled on/off at each time step (8 actions in total). Unlike \citet{oberst2019counterfactual}, we scale rewards based on the number of out-of-range vital signs, between $-1000$ (patient dies) and $1000$ (patient discharged). \jl{Like the GridWorld $p=0.4$ experiment, the Sepsis MDP is highly uncertain, as many states are equally likely to lead to optimal and poor outcomes. Thus, as shown in Figure \ref{fig: reward sepsis}, both policies follow the observed optimal and almost catastrophic paths to guarantee rewards are no worse than the observation.} However, improving the catastrophic path requires deviating from the observation. Here, the Gumbel-max SCM policy, on average, performs better than the interval CFMDP policy. But, since both policies have lower bounds clipped at $-1000$, neither policy reliably improves over the observation. In contrast, for the slightly suboptimal path, the interval CFMDP policy performs significantly better, shown by its higher lower bounds. 
Moreover, in these two cases, the worst-case counterfactual path generated by the interval CFMDP policy is better than that of the Gumbel-max SCM policy,
indicating its greater robustness.
%
\begin{figure*}
    \centering
     \resizebox{0.6\textwidth}{!}{
        \begin{tikzpicture}[scale=1.0, every node/.style={scale=1.0}]
            \draw[thick, black] (-3, -0.25) rectangle (10, 0.25);
            %
            \draw[black, line width=1pt] (-2.5, 0.0) -- (-2,0.0);
            \fill[black] (-2.25,0.0) circle (2pt); %
            \node[right] at (-2,0.0) {\small Observed Path};
            
            %
            \draw[blue, line width=1pt] (1.0,0.0) -- (1.5,0.0);
            \node[draw=blue, circle, minimum size=4pt, inner sep=0pt] at (1.25,0.0) {}; %
            \node[right] at (1.5,0.0) {\small Interval CFMDP Policy};
            
            %
            \draw[red, line width=1pt] (5.5,0) -- (6,0);
            \node[red] at (5.75,0) {$\boldsymbol{\times}$}; %
            \node[right] at (6,0) {\small Gumbel-max SCM Policy};
        \end{tikzpicture}
    }\\
    \subfigure[\footnotesize Lowest cumulative reward: Interval CFMDP ($8000$), Gumbel-max SCM ($8000$)]{%
         \resizebox{0.76\columnwidth}{!}{
             \begin{tikzpicture}
                \begin{axis}[
                    xlabel={$t$},
                    ylabel={Mean reward at time step $t$},
                    title={Optimal Path},
                    grid=both,
                    width=20cm, height=8.5cm,
                    every axis/.style={font=\Huge},
                    %
                ]
                \addplot[
                    color=black, %
                    mark=*, %
                    line width=2pt,
                    mark size=3pt,
                ]
                coordinates {
                    (0, -50.0)
                    (1, 50.0)
                    (2, 1000.0)
                    (3, 1000.0)
                    (4, 1000.0)
                    (5, 1000.0)
                    (6, 1000.0)
                    (7, 1000.0)
                    (8, 1000.0)
                    (9, 1000.0)
                };
                %
                \addplot[
                    color=blue, %
                    mark=o, %
                    line width=2pt,
                    mark size=3pt,
                    error bars/.cd,
                    y dir=both, %
                    y explicit, %
                    error bar style={line width=1pt,solid},
                    error mark options={line width=1pt,mark size=4pt,rotate=90}
                ]
                coordinates {
                    (0, -50.0)  +- (0, 0.0)
                    (1, 50.0)  +- (0, 0.0) 
                    (2, 1000.0)  +- (0, 0.0) 
                    (3, 1000.0)  +- (0, 0.0)
                    (4, 1000.0)  +- (0, 0.0)
                    (5, 1000.0) +- (0, 0.0)
                    (6, 1000.0) +- (0, 0.0)
                    (7, 1000.0) +- (0, 0.0)
                    (8, 1000.0) +- (0, 0.0)
                    (9, 1000.0) +- (0, 0.0)
                };
                %
                \addplot[
                    color=red, %
                    mark=x, %
                    line width=2pt,
                    mark size=6pt,
                    error bars/.cd,
                    y dir=both, %
                    y explicit, %
                    error bar style={line width=1pt,solid},
                    error mark options={line width=1pt,mark size=4pt,rotate=90}
                ]
                coordinates {
                    (0, -50.0)  +- (0, 0.0)
                    (1, 50.0)  +- (0, 0.0) 
                    (2, 1000.0)  +- (0, 0.0) 
                    (3, 1000.0)  +- (0, 0.0)
                    (4, 1000.0)  +- (0, 0.0)
                    (5, 1000.0) +- (0, 0.0)
                    (6, 1000.0) +- (0, 0.0)
                    (7, 1000.0) +- (0, 0.0)
                    (8, 1000.0) +- (0, 0.0)
                    (9, 1000.0) +- (0, 0.0)
                };
                %
                \end{axis}
            \end{tikzpicture}
         }
    }
    \hspace{1cm}
    \subfigure[\footnotesize Lowest cumulative reward: Interval CFMDP ($-5980$), Gumbel-max SCM ($-8000$)]{%
         \resizebox{0.76\columnwidth}{!}{
            \begin{tikzpicture}
                \begin{axis}[
                    xlabel={$t$},
                    ylabel={Mean reward at time step $t$},
                    title={Slightly Suboptimal Path},
                    grid=both,
                    width=20cm, height=8.5cm,
                    every axis/.style={font=\Huge},
                    %
                ]
               \addplot[
                    color=black, %
                    mark=*, %
                    line width=2pt,
                    mark size=3pt,
                ]
                coordinates {
                    (0, -50.0)
                    (1, 50.0)
                    (2, -50.0)
                    (3, -50.0)
                    (4, -1000.0)
                    (5, -1000.0)
                    (6, -1000.0)
                    (7, -1000.0)
                    (8, -1000.0)
                    (9, -1000.0)
                };
                %
                \addplot[
                    color=blue, %
                    mark=o, %
                    line width=2pt,
                    mark size=3pt,
                    error bars/.cd,
                    y dir=both, %
                    y explicit, %
                    error bar style={line width=1pt,solid},
                    error mark options={line width=1pt,mark size=4pt,rotate=90}
                ]
                coordinates {
                    (0, -50.0)  +- (0, 0.0)
                    (1, 50.0)  +- (0, 0.0) 
                    (2, -50.0)  +- (0, 0.0) 
                    (3, 20.0631)  +- (0, 49.97539413)
                    (4, 71.206585)  +- (0, 226.02033693)
                    (5, 151.60797) +- (0, 359.23292559)
                    (6, 200.40593) +- (0, 408.86185176)
                    (7, 257.77948) +- (0, 466.10372804)
                    (8, 299.237465) +- (0, 501.82579506)
                    (9, 338.9129) +- (0, 532.06124996)
                };
                %
                \addplot[
                    color=red, %
                    mark=x, %
                    line width=2pt,
                    mark size=6pt,
                    error bars/.cd,
                    y dir=both, %
                    y explicit, %
                    error bar style={line width=1pt,solid},
                    error mark options={line width=1pt,mark size=4pt,rotate=90}
                ]
                coordinates {
                    (0, -50.0)  +- (0, 0.0)
                    (1, 20.00736)  +- (0, 49.99786741) 
                    (2, -12.282865)  +- (0, 267.598755) 
                    (3, -47.125995)  +- (0, 378.41755832)
                    (4, -15.381965)  +- (0, 461.77616558)
                    (5, 41.15459) +- (0, 521.53189262)
                    (6, 87.01595) +- (0, 564.22243126 )
                    (7, 132.62376) +- (0, 607.31338037)
                    (8, 170.168145) +- (0, 641.48013693)
                    (9, 201.813135) +- (0, 667.29441777)
                };
                %
                %
                %
                %
                %
                %
                %
                %
                %
                %
                %
                %
                %
                %
                %
                %
                %
                %
                %
                \end{axis}
            \end{tikzpicture}
         }
    }\\[-1.5pt]
    \subfigure[\footnotesize Lowest cumulative reward: Interval CFMDP ($100$), Gumbel-max SCM ($100$)]{%
         \resizebox{0.76\columnwidth}{!}{
             \begin{tikzpicture}
                \begin{axis}[
                    xlabel={$t$},
                    ylabel={Mean reward at time step $t$},
                    title={Almost Catastrophic Path},
                    grid=both,
                    every axis/.style={font=\Huge},
                    width=20cm, height=8.5cm,
                    %
                ]
               \addplot[
                    color=black, %
                    mark=*, %
                    line width=2pt,
                    mark size=3pt,
                ]
                coordinates {
                    (0, -50.0)
                    (1, 50.0)
                    (2, 50.0)
                    (3, 50.0)
                    (4, -50.0)
                    (5, 50.0)
                    (6, -50.0)
                    (7, 50.0)
                    (8, -50.0)
                    (9, 50.0)
                };
                %
                %
                \addplot[
                    color=blue, %
                    mark=o, %
                    line width=2pt,
                    mark size=3pt,
                    error bars/.cd,
                    y dir=both, %
                    y explicit, %
                    error bar style={line width=1pt,solid},
                    error mark options={line width=1pt,mark size=4pt,rotate=90}
                ]
                coordinates {
                    (0, -50.0)  +- (0, 0.0)
                    (1, 50.0)  +- (0, 0.0) 
                    (2, 50.0)  +- (0, 0.0) 
                    (3, 50.0)  +- (0, 0.0)
                    (4, -50.0)  +- (0, 0.0)
                    (5, 50.0) +- (0, 0.0)
                    (6, -50.0) +- (0, 0.0)
                    (7, 50.0) +- (0, 0.0)
                    (8, -50.0) +- (0, 0.0)
                    (9, 50.0) +- (0, 0.0)
                };
                %
                \addplot[
                    color=red, %
                    mark=x, %
                    line width=2pt,
                    mark size=6pt,
                    error bars/.cd,
                    y dir=both, %
                    y explicit, %
                    error bar style={line width=1pt,solid},
                    error mark options={line width=1pt,mark size=4pt,rotate=90}
                ]
                coordinates {
                    (0, -50.0)  +- (0, 0.0)
                    (1, 50.0)  +- (0, 0.0) 
                    (2, 50.0)  +- (0, 0.0) 
                    (3, 50.0)  +- (0, 0.0)
                    (4, -50.0)  +- (0, 0.0)
                    (5, 50.0) +- (0, 0.0)
                    (6, -50.0) +- (0, 0.0)
                    (7, 50.0) +- (0, 0.0)
                    (8, -50.0) +- (0, 0.0)
                    (9, 50.0) +- (0, 0.0)
                };
                %
                %
                %
                %
                %
                %
                %
                %
                %
                %
                %
                %
                %
                %
                %
                %
                %
                %
                %
                \end{axis}
            \end{tikzpicture}
         }
    }
    \hspace{1cm}
    \subfigure[\footnotesize Lowest cumulative reward: Interval CFMDP ($-7150$), Gumbel-max SCM ($-9050$)]{%
         \resizebox{0.76\columnwidth}{!}{
            \begin{tikzpicture}
                \begin{axis}[
                    xlabel={$t$},
                    ylabel={Mean reward at time step $t$},
                    title={Catastrophic Path},
                    grid=both,
                    width=20cm, height=8.5cm,
                    every axis/.style={font=\Huge},
                    %
                ]
               \addplot[
                    color=black, %
                    mark=*, %
                    line width=2pt,
                    mark size=3pt,
                ]
                coordinates {
                    (0, -50.0)
                    (1, -50.0)
                    (2, -1000.0)
                    (3, -1000.0)
                    (4, -1000.0)
                    (5, -1000.0)
                    (6, -1000.0)
                    (7, -1000.0)
                    (8, -1000.0)
                    (9, -1000.0)
                };
                %
                %
                \addplot[
                    color=blue, %
                    mark=o, %
                    line width=2pt,
                    mark size=3pt,
                    error bars/.cd,
                    y dir=both, %
                    y explicit, %
                    error bar style={line width=1pt,solid},
                    error mark options={line width=1pt,mark size=4pt,rotate=90}
                ]
                coordinates {
                    (0, -50.0)  +- (0, 0.0)
                    (1, -50.0)  +- (0, 0.0) 
                    (2, -50.0)  +- (0, 0.0) 
                    (3, -841.440725)  += (0, 354.24605512) -= (0, 158.559275)
                    (4, -884.98225)  += (0, 315.37519669) -= (0, 115.01775)
                    (5, -894.330425) += (0, 304.88572805) -= (0, 105.669575)
                    (6, -896.696175) += (0, 301.19954514) -= (0, 103.303825)
                    (7, -897.4635) += (0, 299.61791279) -= (0, 102.5365)
                    (8, -897.77595) += (0, 298.80392585) -= (0, 102.22405)
                    (9, -897.942975) += (0, 298.32920557) -= (0, 102.057025)
                };
                %
                \addplot[
                    color=red, %
                    mark=x, %
                    line width=2pt,
                    mark size=6pt,
                    error bars/.cd,
                    y dir=both, %
                    y explicit, %
                    error bar style={line width=1pt,solid},
                    error mark options={line width=1pt,mark size=4pt,rotate=90}
                ]
            coordinates {
                    (0, -50.0)  +- (0, 0.0)
                    (1, -360.675265)  +- (0, 479.39812699) 
                    (2, -432.27629)  +- (0, 510.38620897) 
                    (3, -467.029545)  += (0, 526.36009628) -= (0, 526.36009628)
                    (4, -439.17429)  += (0, 583.96638919) -= (0, 560.82571)
                    (5, -418.82704) += (0, 618.43027478) -= (0, 581.17296)
                    (6, -397.464895) += (0, 652.67322574) -= (0, 602.535105)
                    (7, -378.49052) += (0, 682.85407033) -= (0, 621.50948)
                    (8, -362.654195) += (0, 707.01412023) -= (0, 637.345805)
                    (9, -347.737935) += (0, 729.29076479) -= (0, 652.262065)
                };
                %
                %
                %
                %
                %
                %
                %
                %
                %
                %
                %
                %
                %
                %
                %
                %
                %
                %
                %
                \end{axis}
            \end{tikzpicture}
         }
    }
    \caption{Average instant reward of CF paths induced by policies on Sepsis.}
    \label{fig: reward sepsis}
\end{figure*}

%
%
%
\subsection{Interval CFMDP Bounds}
%
%
Table \ref{tab:nonzero_probs} presents the mean counterfactual probability bound widths (excluding transitions where the upper bound is $0$) for each MDP, averaged over 20 observed paths. We compare the bounds under counterfactual stability (CS) and monotonicity (M) assumptions, CS alone, and no assumptions. This shows that the assumptions marginally reduce the bound widths, indicating the assumptions tighten the bounds without excluding too many causal models, as intended.
\renewcommand{\arraystretch}{1}

\begin{table}
\centering
\caption{Mean width of counterfactual probability bounds}
\resizebox{0.8\columnwidth}{!}{%
\begin{tabular}{|c|c|c|c|}
\hline
\multirow{2}{*}{\textbf{Environment}} & \multicolumn{3}{c|}{\textbf{Assumptions}} \\ \cline{2-4}
 & \textbf{CS + M} & \textbf{CS} & \textbf{None\tablefootnote{\jl{Equivalent to \citet{li2024probabilities}'s bounds (see Section \ref{sec: equivalence with Li}).}}} \\ \hline
\textbf{GridWorld} ($p=0.9$) & 0.0817 & 0.0977 & 0.100 \\ \hline
\textbf{GridWorld} ($p=0.4$) & 0.552  & 0.638  & 0.646 \\ \hline
\textbf{Sepsis} & 0.138 & 0.140 & 0.140 \\ \hline
\end{tabular}
}
\label{tab:nonzero_probs}
\end{table}


\subsection{Execution Times}
Table \ref{tab: times} compares the average time needed to generate the interval CFMDP vs.\ the Gumbel-max SCM CFMDP for 20 observations.
The GridWorld algorithms were run single-threaded, while the Sepsis experiments were run in parallel.
Generating the interval CFMDP is significantly faster as it uses exact analytical bounds, whereas the Gumbel-max CFMDP requires sampling from the Gumbel distribution to estimate counterfactual transition probabilities. \jl{Since constructing the counterfactual MDP models is the main bottleneck in both approaches, ours is more efficient overall and suitable for larger MDPs.}
\begin{table}
\centering
\caption{Mean execution time to generate CFMDPs}
\resizebox{0.99\columnwidth}{!}{%
\begin{tabular}{|c|c|c|}
\hline
\multirow{2}{*}{\textbf{Environment}} & \multicolumn{2}{c|}{\textbf{Mean Execution Time (s)}} \\ \cline{2-3} 
                                      & \textbf{Interval CFMDP} & \textbf{Gumbel-max CFMDP} \\ \hline
\textbf{GridWorld ($p=0.9$) }                  & 0.261                   & 56.1                      \\ \hline
\textbf{GridWorld ($p=0.4$)  }                 & 0.336                   & 54.5                      \\ \hline
\textbf{Sepsis}                                 & 688                     & 2940                      \\ \hline
\end{tabular}%
}
\label{tab: times}
\end{table}

% \subsection{RQ4: Practical Use of the Synthesised Predicates}
% \label{sec:practical}

% Are the predicates synthesised by \tool useful?
% %
% To answer this question, we used \tool's results to enable \emph{new}
% applications of an existing SL-based state-of-the-art deductive
% synthesiser \suslik~\cite{polikarpova2019structuring}.
% %
% Given definitions of SL predicates and a Hoare-style program
% \emph{specification} (\ie, its parameters, pre- and post-condition
% written in Separation Logic), \suslik produces a C program that
% \emph{provably} satisfies that specification, as well as a formal
% proof of that fact in one of several SL implementations on top of Coq
% proof assistant~\cite{Appel:ESOP11, KrebbersTB17,Nanevski-al:POPL10},
% providing the ultimate correctness guarantees for synthesis
% results~\cite{WatanabeGPPS21}.
% %
% The main downside of \suslik is that it requires the user to provide
% SL predicates---a non-negligeable price to pay, which is, however,
% amortised across multiple synthesis tasks involving the same data
% types.
% %
% With \tool, we can, therefore, enhance the expressivity of \suslik by
% allowing the user to provide memory graphs
% instead of the formal SL predicate definitions.

% Following the informally described translation procedure in
% \autoref{sec:bst}, we implement an automated conversion of the
% synthesised \prolog-style SL predicates  into \suslik's specification
% language, and take the predicates as the specifications
% to synthesise a number of
% easy-to-specify programs as shown in \autoref{tab:results}.
% % \footnote{We found that negative number is not well-supported. In our balanced tree predicate, we use -1 to represent the empty tree's height. However, \suslik synthesis only work when we manually modify it to 0.}
% For example, a program fully deallocating a sorted doubly-linked
% list (DLL) can be specified by the following Hoare triple, which makes
% use of the newly synthesised predicate for a sorted DLL
% (\code{srt_dll}) in the precondition and the empty heap assertion
% \code{emp} in the postcondition:
% %
% \begin{lstlisting}
%   { srt_dll(x, a, s) } void free(loc x) { emp }
% \end{lstlisting}
% %

% Non-trivial synthesised programs performing data structure
% transformations such as {sorting} and {flattening} are listed in in
% \autoref{tab:results}'s {Transform} category.
% %
% As a specific example, a specification describing a function that
% flattens a BST into a sorted DLL is defined as follows:
% %
% \begin{lstlisting}
%   { r :-> 0 * bst(x, s) } void bst_to_srt_dll(loc x, loc r) { r :-> y * srt_dll(y, a, s) }
% \end{lstlisting}
% %

% The columns in the table in \autoref{tab:results} also show the ratio
% of AST nodes between the the synthesised programs and \suslik
% specifications, demonstrating the gain in expressivity.
% %
% Finally, the table shows the time it took to synthesise the programs.
% %
% Synthesis efficiency for some of those case studies in \suslik can be
% improved by explicitly providing search parameters such as number of
% considered predicate unfoldings to the synthesiser.
% %
% All programs in the table were automatically synthesised by \suslik,
% along with their SL proofs embedded into Coq via VST
% framework~\cite{Appel:ESOP11}, from the predicates obtained via \tool
% in the scope of the experiments in \autoref{sec:done}. An example of
% the synthesised C program that transforms a sorted DLL into a singly
% linked list is given in \autoref{fig:transform}.

% \begin{figure}[t]
%   \centering  
%   \begin{minipage}[b]{0.56\textwidth}
%       \centering
%       {\footnotesize{
%       
    \begin{tabular}{c| c|c|c| c}
    \toprule
    No. & Category & Program & Code/Spec & Time    \\	  
    \midrule
    1 & \multirow{4}{*}{{{Deallocate}}} & sll & 5.5x & 0.2s\\
    2 & & bst & 8.0x & 0.2s\\
    3 & & dll\_seg & 2.4x & 0.2s\\
    4 & & {multilist} & 16.0x & 0.3s\\
    \midrule
    5 &  \multirow{3}{*}{{{Copy}}} & lseg & 2.0x & 0.8s\\
    6 & & bst & 3.5x & 3.3s\\
    7 & & balanced tree & 3.0x & 1.9s\\
    \midrule 
    8 & \multirow{2}{*}{{{Size}}} & sll\_len & 2.1x & 0.4s  \\
    9 & & balanced tree & 3.5x & 0.6s  \\
    \midrule 
    10 & \multirow{8}{*}{{{Transform}}} & sll $\rightarrow$ dlseg & 2.5x & 0.4s  \\
    11 & & {srt\_dll $\rightarrow$ sll} & 3.1x & 7.4s  \\
    12 & & {dll $\rightarrow$ bst} & 15.0x & 42.8s  \\
    13 & & {btree $\rightarrow$ bktree} & 13.6x & 11.8s  \\
    14 & & {multilist $\xrightarrow{}$ sll} & 5.0x & 8.8s  \\
    15 & & {btree $\xrightarrow{}$ dll} & 9.6x & 7.1s  \\
    16 & & {bst $\xrightarrow{}$ srtl} & 11.6x & 10.3s  \\
    17 & & {dll $\xrightarrow{}$ srt\_dll} & 7.3x & 9.3s  \\
      \bottomrule
      %
    \end{tabular}
    
    % & btree $\rightarrow$ dll ? & 1.4x & 0.4  \\

%       }}
%     \caption{Example programs synthesised by \suslik from SL
%       specifications stated using predicates produced by \tool.}
    
%     %
%     \label{tab:results}
%   \end{minipage}
%   \hfill
%   \begin{minipage}[b]{0.4\textwidth}
%     \centering
    
% \begin{minted}[fontsize=\scriptsize]{c}
% // pre:  {f :-> x ** sorted_dll(x, z, s)}
% // post: {f :-> y ** sll(y, s)}

% void srt_dll_to_sll (loc f) {
% loc x1 = READ_LOC(f, 0);
% if (x1 == 0)  {
%   WRITE_INT(f, 0, 0);
%   return;
% } else {
%   int vx11 = READ_INT(x1, 0);
%   loc nxtx11 = READ_LOC(x1, 1);
%   loc z1 = READ_LOC(x1, 2);
%   WRITE_LOC(f, 0, nxtx11);
%   srt_dll_to_sll(f);
%   loc y11 = READ_LOC(f, 0);
%   loc y2 = (loc)malloc(2 * sizeof(loc));
%   free(x1);
%   WRITE_LOC(f, 0, y2);
%   WRITE_LOC(y2, 1, y11);
%   WRITE_INT(y2, 0, (int)vx11);
%   return;
% }}
% \end{minted}
% \caption{An example \suslik output (\#{11}): a C program for
%   converting a sorted DLL to SLL. \texttt{loc}, \texttt{READ\_LOC},
%   \etc are macro-definitions around ordinary C types and operations. }
%         \label{fig:transform}
% \end{minipage}

% % \setlength{\abovecaptionskip}{5pt}
% \setlength{\belowcaptionskip}{-10pt}

% \end{figure}

% The outcome of this experiment hints a new way to synthesise
% \emph{provably-correct} heap-manipulating programs via memory graphs
% describing the properties of the input and output, instead of
% providing the formal SL predicates as in deductive synthesis
% \cite{polikarpova2019structuring}, or providing the concrete input
% memory graphs before the program execution and the
% \emph{corresponding} output memory graphs, as in inductive synthesis
% \cite{roy2013concrete, SinghS11}.

\putsec{related}{Related Work}

\noindent \textbf{Efficient Radiance Field Rendering.}
%
The introduction of Neural Radiance Fields (NeRF)~\cite{mil:sri20} has
generated significant interest in efficient 3D scene representation and
rendering for radiance fields.
%
Over the past years, there has been a large amount of research aimed at
accelerating NeRFs through algorithmic or software
optimizations~\cite{mul:eva22,fri:yu22,che:fun23,sun:sun22}, and the
development of hardware
accelerators~\cite{lee:cho23,li:li23,son:wen23,mub:kan23,fen:liu24}.
%
The state-of-the-art method, 3D Gaussian splatting~\cite{ker:kop23}, has
further fueled interest in accelerating radiance field
rendering~\cite{rad:ste24,lee:lee24,nie:stu24,lee:rho24,ham:mel24} as it
employs rasterization primitives that can be rendered much faster than NeRFs.
%
However, previous research focused on software graphics rendering on
programmable cores or building dedicated hardware accelerators. In contrast,
\name{} investigates the potential of efficient radiance field rendering while
utilizing fixed-function units in graphics hardware.
%
To our knowledge, this is the first work that assesses the performance
implications of rendering Gaussian-based radiance fields on the hardware
graphics pipeline with software and hardware optimizations.

%%%%%%%%%%%%%%%%%%%%%%%%%%%%%%%%%%%%%%%%%%%%%%%%%%%%%%%%%%%%%%%%%%%%%%%%%%
\myparagraph{Enhancing Graphics Rendering Hardware.}
%
The performance advantage of executing graphics rendering on either
programmable shader cores or fixed-function units varies depending on the
rendering methods and hardware designs.
%
Previous studies have explored the performance implication of graphics hardware
design by developing simulation infrastructures for graphics
workloads~\cite{bar:gon06,gub:aam19,tin:sax23,arn:par13}.
%
Additionally, several studies have aimed to improve the performance of
special-purpose hardware such as ray tracing units in graphics
hardware~\cite{cho:now23,liu:cha21} and proposed hardware accelerators for
graphics applications~\cite{lu:hua17,ram:gri09}.
%
In contrast to these works, which primarily evaluate traditional graphics
workloads, our work focuses on improving the performance of volume rendering
workloads, such as Gaussian splatting, which require blending a huge number of
fragments per pixel.

%%%%%%%%%%%%%%%%%%%%%%%%%%%%%%%%%%%%%%%%%%%%%%%%%%%%%%%%%%%%%%%%%%%%%%%%%%
%
In the context of multi-sample anti-aliasing, prior work proposed reducing the
amount of redundant shading by merging fragments from adjacent triangles in a
mesh at the quad granularity~\cite{fat:bou10}.
%
While both our work and quad-fragment merging (QFM)~\cite{fat:bou10} aim to
reduce operations by merging quads, our proposed technique differs from QFM in
many aspects.
%
Our method aims to blend \emph{overlapping primitives} along the depth
direction and applies to quads from any primitive. In contrast, QFM merges quad
fragments from small (e.g., pixel-sized) triangles that \emph{share} an edge
(i.e., \emph{connected}, \emph{non-overlapping} triangles).
%
As such, QFM is not applicable to the scenes consisting of a number of
unconnected transparent triangles, such as those in 3D Gaussian splatting.
%
In addition, our method computes the \emph{exact} color for each pixel by
offloading blending operations from ROPs to shader units, whereas QFM
\emph{approximates} pixel colors by using the color from one triangle when
multiple triangles are merged into a single quad.


\section{Conclusion}
In this work, we propose a simple yet effective approach, called SMILE, for graph few-shot learning with fewer tasks. Specifically, we introduce a novel dual-level mixup strategy, including within-task and across-task mixup, for enriching the diversity of nodes within each task and the diversity of tasks. Also, we incorporate the degree-based prior information to learn expressive node embeddings. Theoretically, we prove that SMILE effectively enhances the model's generalization performance. Empirically, we conduct extensive experiments on multiple benchmarks and the results suggest that SMILE significantly outperforms other baselines, including both in-domain and cross-domain few-shot settings.

% \section{Limitation and Future Work}
\label{sec:limit}

We have discussed some limitations of our approach in
\autoref{sec:fail} in the inductive predicate synthesis. Here we
discuss some other high-level limitations on the positive-only
learning and future works.

\paragraph{More on positive-only learning}


% Though we have mentioned that the positive-only learning is also applicable to other domains, we only discuss the issue of negative examples in  Separation Logic.
% Here we show different scenarios where (considerate) negative examples are not easy to obtain out of the context of Separation Logic.
% \begin{itemize}
%   \item \emph{Scenario 1:} To know what are the features on friendship among a group of people, we can ask them to fill a questionnaire about who are their friends. However, a questionnaire about who are \emph{not} their friends will not be an easy one to answer.
%   \item \emph{Scenario 2:} To learn the property of isosceles triangles, negative examples are not hard to provide; but if all negative examples provided are scalene, the learned predicates can be the property of equilateral triangles, which is not wrong but not specific enough.
% \end{itemize}

% It is both intuitive and supported by data that negation is not as
% welcomed as positive one. In daily knowledge representation,
% statements like "a tiger is an animal" are more common than "water is
% not an animal"; \citet{hossain2022analysis} conclude that negation is
% even less common in natural language understanding corpora than
% general-purpose English (22.6\%-29.9\%). Therefore, we believe that
% the positive-only learning can be a more practical setting for many
% applications.

Some basic assumptions of our positive-only learning restrict the
extension to more general settings. For example, noisy tolerance is
something not applicable to our current approach, because we need to
guarantee that the learned predicates are valid for all provided
examples. It is to be explored on whether the specificity can be
defined in other settings like probabilistic
ILP~\cite{de2008probabilistic} or differentiable
ILP~\cite{evans2018learning}.


\paragraph{Domain refinement and theory-based learning}

Since the learning domain of \popper is the definite clause, though we
have already encoded large number of SL-specific constraints in the ASP
encoding, either the hypothesis generation or the hypothesis testing
can be further refined. For example, the pure and spatial part of SL
predicates can be learned incrementally as the multilist example in
\autoref{sec:evaluation}. However, the encoding makes it hard to do
the incremental efficient.

Another direction is to use the theory-based learning approach. See in
most existing synthesisers, SMT solver is used to check the validity
of the candidate predicates. In our current synthesis, the search
space is traversed without awareness of the memory graphs' concrete
data, which is inherited from \popper. The good point of data
unawareness is to make less assumption on the input memory graphs: in
a real memory heap, some pointers can point to certain location which
is part of a data structure, but the pointer itself are not
necessarily to be part of the data structure though with reachability.
If we can have the assumption that the pointers with reachability is
of one data structure, we can use data-aware search to accelerate the
search.


% \bibliographystyle{plain}
\bibliography{references}

% \newpage

\subsection{Lloyd-Max Algorithm}
\label{subsec:Lloyd-Max}
For a given quantization bitwidth $B$ and an operand $\bm{X}$, the Lloyd-Max algorithm finds $2^B$ quantization levels $\{\hat{x}_i\}_{i=1}^{2^B}$ such that quantizing $\bm{X}$ by rounding each scalar in $\bm{X}$ to the nearest quantization level minimizes the quantization MSE. 

The algorithm starts with an initial guess of quantization levels and then iteratively computes quantization thresholds $\{\tau_i\}_{i=1}^{2^B-1}$ and updates quantization levels $\{\hat{x}_i\}_{i=1}^{2^B}$. Specifically, at iteration $n$, thresholds are set to the midpoints of the previous iteration's levels:
\begin{align*}
    \tau_i^{(n)}=\frac{\hat{x}_i^{(n-1)}+\hat{x}_{i+1}^{(n-1)}}2 \text{ for } i=1\ldots 2^B-1
\end{align*}
Subsequently, the quantization levels are re-computed as conditional means of the data regions defined by the new thresholds:
\begin{align*}
    \hat{x}_i^{(n)}=\mathbb{E}\left[ \bm{X} \big| \bm{X}\in [\tau_{i-1}^{(n)},\tau_i^{(n)}] \right] \text{ for } i=1\ldots 2^B
\end{align*}
where to satisfy boundary conditions we have $\tau_0=-\infty$ and $\tau_{2^B}=\infty$. The algorithm iterates the above steps until convergence.

Figure \ref{fig:lm_quant} compares the quantization levels of a $7$-bit floating point (E3M3) quantizer (left) to a $7$-bit Lloyd-Max quantizer (right) when quantizing a layer of weights from the GPT3-126M model at a per-tensor granularity. As shown, the Lloyd-Max quantizer achieves substantially lower quantization MSE. Further, Table \ref{tab:FP7_vs_LM7} shows the superior perplexity achieved by Lloyd-Max quantizers for bitwidths of $7$, $6$ and $5$. The difference between the quantizers is clear at 5 bits, where per-tensor FP quantization incurs a drastic and unacceptable increase in perplexity, while Lloyd-Max quantization incurs a much smaller increase. Nevertheless, we note that even the optimal Lloyd-Max quantizer incurs a notable ($\sim 1.5$) increase in perplexity due to the coarse granularity of quantization. 

\begin{figure}[h]
  \centering
  \includegraphics[width=0.7\linewidth]{sections/figures/LM7_FP7.pdf}
  \caption{\small Quantization levels and the corresponding quantization MSE of Floating Point (left) vs Lloyd-Max (right) Quantizers for a layer of weights in the GPT3-126M model.}
  \label{fig:lm_quant}
\end{figure}

\begin{table}[h]\scriptsize
\begin{center}
\caption{\label{tab:FP7_vs_LM7} \small Comparing perplexity (lower is better) achieved by floating point quantizers and Lloyd-Max quantizers on a GPT3-126M model for the Wikitext-103 dataset.}
\begin{tabular}{c|cc|c}
\hline
 \multirow{2}{*}{\textbf{Bitwidth}} & \multicolumn{2}{|c|}{\textbf{Floating-Point Quantizer}} & \textbf{Lloyd-Max Quantizer} \\
 & Best Format & Wikitext-103 Perplexity & Wikitext-103 Perplexity \\
\hline
7 & E3M3 & 18.32 & 18.27 \\
6 & E3M2 & 19.07 & 18.51 \\
5 & E4M0 & 43.89 & 19.71 \\
\hline
\end{tabular}
\end{center}
\end{table}

\subsection{Proof of Local Optimality of LO-BCQ}
\label{subsec:lobcq_opt_proof}
For a given block $\bm{b}_j$, the quantization MSE during LO-BCQ can be empirically evaluated as $\frac{1}{L_b}\lVert \bm{b}_j- \bm{\hat{b}}_j\rVert^2_2$ where $\bm{\hat{b}}_j$ is computed from equation (\ref{eq:clustered_quantization_definition}) as $C_{f(\bm{b}_j)}(\bm{b}_j)$. Further, for a given block cluster $\mathcal{B}_i$, we compute the quantization MSE as $\frac{1}{|\mathcal{B}_{i}|}\sum_{\bm{b} \in \mathcal{B}_{i}} \frac{1}{L_b}\lVert \bm{b}- C_i^{(n)}(\bm{b})\rVert^2_2$. Therefore, at the end of iteration $n$, we evaluate the overall quantization MSE $J^{(n)}$ for a given operand $\bm{X}$ composed of $N_c$ block clusters as:
\begin{align*}
    \label{eq:mse_iter_n}
    J^{(n)} = \frac{1}{N_c} \sum_{i=1}^{N_c} \frac{1}{|\mathcal{B}_{i}^{(n)}|}\sum_{\bm{v} \in \mathcal{B}_{i}^{(n)}} \frac{1}{L_b}\lVert \bm{b}- B_i^{(n)}(\bm{b})\rVert^2_2
\end{align*}

At the end of iteration $n$, the codebooks are updated from $\mathcal{C}^{(n-1)}$ to $\mathcal{C}^{(n)}$. However, the mapping of a given vector $\bm{b}_j$ to quantizers $\mathcal{C}^{(n)}$ remains as  $f^{(n)}(\bm{b}_j)$. At the next iteration, during the vector clustering step, $f^{(n+1)}(\bm{b}_j)$ finds new mapping of $\bm{b}_j$ to updated codebooks $\mathcal{C}^{(n)}$ such that the quantization MSE over the candidate codebooks is minimized. Therefore, we obtain the following result for $\bm{b}_j$:
\begin{align*}
\frac{1}{L_b}\lVert \bm{b}_j - C_{f^{(n+1)}(\bm{b}_j)}^{(n)}(\bm{b}_j)\rVert^2_2 \le \frac{1}{L_b}\lVert \bm{b}_j - C_{f^{(n)}(\bm{b}_j)}^{(n)}(\bm{b}_j)\rVert^2_2
\end{align*}

That is, quantizing $\bm{b}_j$ at the end of the block clustering step of iteration $n+1$ results in lower quantization MSE compared to quantizing at the end of iteration $n$. Since this is true for all $\bm{b} \in \bm{X}$, we assert the following:
\begin{equation}
\begin{split}
\label{eq:mse_ineq_1}
    \tilde{J}^{(n+1)} &= \frac{1}{N_c} \sum_{i=1}^{N_c} \frac{1}{|\mathcal{B}_{i}^{(n+1)}|}\sum_{\bm{b} \in \mathcal{B}_{i}^{(n+1)}} \frac{1}{L_b}\lVert \bm{b} - C_i^{(n)}(b)\rVert^2_2 \le J^{(n)}
\end{split}
\end{equation}
where $\tilde{J}^{(n+1)}$ is the the quantization MSE after the vector clustering step at iteration $n+1$.

Next, during the codebook update step (\ref{eq:quantizers_update}) at iteration $n+1$, the per-cluster codebooks $\mathcal{C}^{(n)}$ are updated to $\mathcal{C}^{(n+1)}$ by invoking the Lloyd-Max algorithm \citep{Lloyd}. We know that for any given value distribution, the Lloyd-Max algorithm minimizes the quantization MSE. Therefore, for a given vector cluster $\mathcal{B}_i$ we obtain the following result:

\begin{equation}
    \frac{1}{|\mathcal{B}_{i}^{(n+1)}|}\sum_{\bm{b} \in \mathcal{B}_{i}^{(n+1)}} \frac{1}{L_b}\lVert \bm{b}- C_i^{(n+1)}(\bm{b})\rVert^2_2 \le \frac{1}{|\mathcal{B}_{i}^{(n+1)}|}\sum_{\bm{b} \in \mathcal{B}_{i}^{(n+1)}} \frac{1}{L_b}\lVert \bm{b}- C_i^{(n)}(\bm{b})\rVert^2_2
\end{equation}

The above equation states that quantizing the given block cluster $\mathcal{B}_i$ after updating the associated codebook from $C_i^{(n)}$ to $C_i^{(n+1)}$ results in lower quantization MSE. Since this is true for all the block clusters, we derive the following result: 
\begin{equation}
\begin{split}
\label{eq:mse_ineq_2}
     J^{(n+1)} &= \frac{1}{N_c} \sum_{i=1}^{N_c} \frac{1}{|\mathcal{B}_{i}^{(n+1)}|}\sum_{\bm{b} \in \mathcal{B}_{i}^{(n+1)}} \frac{1}{L_b}\lVert \bm{b}- C_i^{(n+1)}(\bm{b})\rVert^2_2  \le \tilde{J}^{(n+1)}   
\end{split}
\end{equation}

Following (\ref{eq:mse_ineq_1}) and (\ref{eq:mse_ineq_2}), we find that the quantization MSE is non-increasing for each iteration, that is, $J^{(1)} \ge J^{(2)} \ge J^{(3)} \ge \ldots \ge J^{(M)}$ where $M$ is the maximum number of iterations. 
%Therefore, we can say that if the algorithm converges, then it must be that it has converged to a local minimum. 
\hfill $\blacksquare$


\begin{figure}
    \begin{center}
    \includegraphics[width=0.5\textwidth]{sections//figures/mse_vs_iter.pdf}
    \end{center}
    \caption{\small NMSE vs iterations during LO-BCQ compared to other block quantization proposals}
    \label{fig:nmse_vs_iter}
\end{figure}

Figure \ref{fig:nmse_vs_iter} shows the empirical convergence of LO-BCQ across several block lengths and number of codebooks. Also, the MSE achieved by LO-BCQ is compared to baselines such as MXFP and VSQ. As shown, LO-BCQ converges to a lower MSE than the baselines. Further, we achieve better convergence for larger number of codebooks ($N_c$) and for a smaller block length ($L_b$), both of which increase the bitwidth of BCQ (see Eq \ref{eq:bitwidth_bcq}).


\subsection{Additional Accuracy Results}
%Table \ref{tab:lobcq_config} lists the various LOBCQ configurations and their corresponding bitwidths.
\begin{table}
\setlength{\tabcolsep}{4.75pt}
\begin{center}
\caption{\label{tab:lobcq_config} Various LO-BCQ configurations and their bitwidths.}
\begin{tabular}{|c||c|c|c|c||c|c||c|} 
\hline
 & \multicolumn{4}{|c||}{$L_b=8$} & \multicolumn{2}{|c||}{$L_b=4$} & $L_b=2$ \\
 \hline
 \backslashbox{$L_A$\kern-1em}{\kern-1em$N_c$} & 2 & 4 & 8 & 16 & 2 & 4 & 2 \\
 \hline
 64 & 4.25 & 4.375 & 4.5 & 4.625 & 4.375 & 4.625 & 4.625\\
 \hline
 32 & 4.375 & 4.5 & 4.625& 4.75 & 4.5 & 4.75 & 4.75 \\
 \hline
 16 & 4.625 & 4.75& 4.875 & 5 & 4.75 & 5 & 5 \\
 \hline
\end{tabular}
\end{center}
\end{table}

%\subsection{Perplexity achieved by various LO-BCQ configurations on Wikitext-103 dataset}

\begin{table} \centering
\begin{tabular}{|c||c|c|c|c||c|c||c|} 
\hline
 $L_b \rightarrow$& \multicolumn{4}{c||}{8} & \multicolumn{2}{c||}{4} & 2\\
 \hline
 \backslashbox{$L_A$\kern-1em}{\kern-1em$N_c$} & 2 & 4 & 8 & 16 & 2 & 4 & 2  \\
 %$N_c \rightarrow$ & 2 & 4 & 8 & 16 & 2 & 4 & 2 \\
 \hline
 \hline
 \multicolumn{8}{c}{GPT3-1.3B (FP32 PPL = 9.98)} \\ 
 \hline
 \hline
 64 & 10.40 & 10.23 & 10.17 & 10.15 &  10.28 & 10.18 & 10.19 \\
 \hline
 32 & 10.25 & 10.20 & 10.15 & 10.12 &  10.23 & 10.17 & 10.17 \\
 \hline
 16 & 10.22 & 10.16 & 10.10 & 10.09 &  10.21 & 10.14 & 10.16 \\
 \hline
  \hline
 \multicolumn{8}{c}{GPT3-8B (FP32 PPL = 7.38)} \\ 
 \hline
 \hline
 64 & 7.61 & 7.52 & 7.48 &  7.47 &  7.55 &  7.49 & 7.50 \\
 \hline
 32 & 7.52 & 7.50 & 7.46 &  7.45 &  7.52 &  7.48 & 7.48  \\
 \hline
 16 & 7.51 & 7.48 & 7.44 &  7.44 &  7.51 &  7.49 & 7.47  \\
 \hline
\end{tabular}
\caption{\label{tab:ppl_gpt3_abalation} Wikitext-103 perplexity across GPT3-1.3B and 8B models.}
\end{table}

\begin{table} \centering
\begin{tabular}{|c||c|c|c|c||} 
\hline
 $L_b \rightarrow$& \multicolumn{4}{c||}{8}\\
 \hline
 \backslashbox{$L_A$\kern-1em}{\kern-1em$N_c$} & 2 & 4 & 8 & 16 \\
 %$N_c \rightarrow$ & 2 & 4 & 8 & 16 & 2 & 4 & 2 \\
 \hline
 \hline
 \multicolumn{5}{|c|}{Llama2-7B (FP32 PPL = 5.06)} \\ 
 \hline
 \hline
 64 & 5.31 & 5.26 & 5.19 & 5.18  \\
 \hline
 32 & 5.23 & 5.25 & 5.18 & 5.15  \\
 \hline
 16 & 5.23 & 5.19 & 5.16 & 5.14  \\
 \hline
 \multicolumn{5}{|c|}{Nemotron4-15B (FP32 PPL = 5.87)} \\ 
 \hline
 \hline
 64  & 6.3 & 6.20 & 6.13 & 6.08  \\
 \hline
 32  & 6.24 & 6.12 & 6.07 & 6.03  \\
 \hline
 16  & 6.12 & 6.14 & 6.04 & 6.02  \\
 \hline
 \multicolumn{5}{|c|}{Nemotron4-340B (FP32 PPL = 3.48)} \\ 
 \hline
 \hline
 64 & 3.67 & 3.62 & 3.60 & 3.59 \\
 \hline
 32 & 3.63 & 3.61 & 3.59 & 3.56 \\
 \hline
 16 & 3.61 & 3.58 & 3.57 & 3.55 \\
 \hline
\end{tabular}
\caption{\label{tab:ppl_llama7B_nemo15B} Wikitext-103 perplexity compared to FP32 baseline in Llama2-7B and Nemotron4-15B, 340B models}
\end{table}

%\subsection{Perplexity achieved by various LO-BCQ configurations on MMLU dataset}


\begin{table} \centering
\begin{tabular}{|c||c|c|c|c||c|c|c|c|} 
\hline
 $L_b \rightarrow$& \multicolumn{4}{c||}{8} & \multicolumn{4}{c||}{8}\\
 \hline
 \backslashbox{$L_A$\kern-1em}{\kern-1em$N_c$} & 2 & 4 & 8 & 16 & 2 & 4 & 8 & 16  \\
 %$N_c \rightarrow$ & 2 & 4 & 8 & 16 & 2 & 4 & 2 \\
 \hline
 \hline
 \multicolumn{5}{|c|}{Llama2-7B (FP32 Accuracy = 45.8\%)} & \multicolumn{4}{|c|}{Llama2-70B (FP32 Accuracy = 69.12\%)} \\ 
 \hline
 \hline
 64 & 43.9 & 43.4 & 43.9 & 44.9 & 68.07 & 68.27 & 68.17 & 68.75 \\
 \hline
 32 & 44.5 & 43.8 & 44.9 & 44.5 & 68.37 & 68.51 & 68.35 & 68.27  \\
 \hline
 16 & 43.9 & 42.7 & 44.9 & 45 & 68.12 & 68.77 & 68.31 & 68.59  \\
 \hline
 \hline
 \multicolumn{5}{|c|}{GPT3-22B (FP32 Accuracy = 38.75\%)} & \multicolumn{4}{|c|}{Nemotron4-15B (FP32 Accuracy = 64.3\%)} \\ 
 \hline
 \hline
 64 & 36.71 & 38.85 & 38.13 & 38.92 & 63.17 & 62.36 & 63.72 & 64.09 \\
 \hline
 32 & 37.95 & 38.69 & 39.45 & 38.34 & 64.05 & 62.30 & 63.8 & 64.33  \\
 \hline
 16 & 38.88 & 38.80 & 38.31 & 38.92 & 63.22 & 63.51 & 63.93 & 64.43  \\
 \hline
\end{tabular}
\caption{\label{tab:mmlu_abalation} Accuracy on MMLU dataset across GPT3-22B, Llama2-7B, 70B and Nemotron4-15B models.}
\end{table}


%\subsection{Perplexity achieved by various LO-BCQ configurations on LM evaluation harness}

\begin{table} \centering
\begin{tabular}{|c||c|c|c|c||c|c|c|c|} 
\hline
 $L_b \rightarrow$& \multicolumn{4}{c||}{8} & \multicolumn{4}{c||}{8}\\
 \hline
 \backslashbox{$L_A$\kern-1em}{\kern-1em$N_c$} & 2 & 4 & 8 & 16 & 2 & 4 & 8 & 16  \\
 %$N_c \rightarrow$ & 2 & 4 & 8 & 16 & 2 & 4 & 2 \\
 \hline
 \hline
 \multicolumn{5}{|c|}{Race (FP32 Accuracy = 37.51\%)} & \multicolumn{4}{|c|}{Boolq (FP32 Accuracy = 64.62\%)} \\ 
 \hline
 \hline
 64 & 36.94 & 37.13 & 36.27 & 37.13 & 63.73 & 62.26 & 63.49 & 63.36 \\
 \hline
 32 & 37.03 & 36.36 & 36.08 & 37.03 & 62.54 & 63.51 & 63.49 & 63.55  \\
 \hline
 16 & 37.03 & 37.03 & 36.46 & 37.03 & 61.1 & 63.79 & 63.58 & 63.33  \\
 \hline
 \hline
 \multicolumn{5}{|c|}{Winogrande (FP32 Accuracy = 58.01\%)} & \multicolumn{4}{|c|}{Piqa (FP32 Accuracy = 74.21\%)} \\ 
 \hline
 \hline
 64 & 58.17 & 57.22 & 57.85 & 58.33 & 73.01 & 73.07 & 73.07 & 72.80 \\
 \hline
 32 & 59.12 & 58.09 & 57.85 & 58.41 & 73.01 & 73.94 & 72.74 & 73.18  \\
 \hline
 16 & 57.93 & 58.88 & 57.93 & 58.56 & 73.94 & 72.80 & 73.01 & 73.94  \\
 \hline
\end{tabular}
\caption{\label{tab:mmlu_abalation} Accuracy on LM evaluation harness tasks on GPT3-1.3B model.}
\end{table}

\begin{table} \centering
\begin{tabular}{|c||c|c|c|c||c|c|c|c|} 
\hline
 $L_b \rightarrow$& \multicolumn{4}{c||}{8} & \multicolumn{4}{c||}{8}\\
 \hline
 \backslashbox{$L_A$\kern-1em}{\kern-1em$N_c$} & 2 & 4 & 8 & 16 & 2 & 4 & 8 & 16  \\
 %$N_c \rightarrow$ & 2 & 4 & 8 & 16 & 2 & 4 & 2 \\
 \hline
 \hline
 \multicolumn{5}{|c|}{Race (FP32 Accuracy = 41.34\%)} & \multicolumn{4}{|c|}{Boolq (FP32 Accuracy = 68.32\%)} \\ 
 \hline
 \hline
 64 & 40.48 & 40.10 & 39.43 & 39.90 & 69.20 & 68.41 & 69.45 & 68.56 \\
 \hline
 32 & 39.52 & 39.52 & 40.77 & 39.62 & 68.32 & 67.43 & 68.17 & 69.30  \\
 \hline
 16 & 39.81 & 39.71 & 39.90 & 40.38 & 68.10 & 66.33 & 69.51 & 69.42  \\
 \hline
 \hline
 \multicolumn{5}{|c|}{Winogrande (FP32 Accuracy = 67.88\%)} & \multicolumn{4}{|c|}{Piqa (FP32 Accuracy = 78.78\%)} \\ 
 \hline
 \hline
 64 & 66.85 & 66.61 & 67.72 & 67.88 & 77.31 & 77.42 & 77.75 & 77.64 \\
 \hline
 32 & 67.25 & 67.72 & 67.72 & 67.00 & 77.31 & 77.04 & 77.80 & 77.37  \\
 \hline
 16 & 68.11 & 68.90 & 67.88 & 67.48 & 77.37 & 78.13 & 78.13 & 77.69  \\
 \hline
\end{tabular}
\caption{\label{tab:mmlu_abalation} Accuracy on LM evaluation harness tasks on GPT3-8B model.}
\end{table}

\begin{table} \centering
\begin{tabular}{|c||c|c|c|c||c|c|c|c|} 
\hline
 $L_b \rightarrow$& \multicolumn{4}{c||}{8} & \multicolumn{4}{c||}{8}\\
 \hline
 \backslashbox{$L_A$\kern-1em}{\kern-1em$N_c$} & 2 & 4 & 8 & 16 & 2 & 4 & 8 & 16  \\
 %$N_c \rightarrow$ & 2 & 4 & 8 & 16 & 2 & 4 & 2 \\
 \hline
 \hline
 \multicolumn{5}{|c|}{Race (FP32 Accuracy = 40.67\%)} & \multicolumn{4}{|c|}{Boolq (FP32 Accuracy = 76.54\%)} \\ 
 \hline
 \hline
 64 & 40.48 & 40.10 & 39.43 & 39.90 & 75.41 & 75.11 & 77.09 & 75.66 \\
 \hline
 32 & 39.52 & 39.52 & 40.77 & 39.62 & 76.02 & 76.02 & 75.96 & 75.35  \\
 \hline
 16 & 39.81 & 39.71 & 39.90 & 40.38 & 75.05 & 73.82 & 75.72 & 76.09  \\
 \hline
 \hline
 \multicolumn{5}{|c|}{Winogrande (FP32 Accuracy = 70.64\%)} & \multicolumn{4}{|c|}{Piqa (FP32 Accuracy = 79.16\%)} \\ 
 \hline
 \hline
 64 & 69.14 & 70.17 & 70.17 & 70.56 & 78.24 & 79.00 & 78.62 & 78.73 \\
 \hline
 32 & 70.96 & 69.69 & 71.27 & 69.30 & 78.56 & 79.49 & 79.16 & 78.89  \\
 \hline
 16 & 71.03 & 69.53 & 69.69 & 70.40 & 78.13 & 79.16 & 79.00 & 79.00  \\
 \hline
\end{tabular}
\caption{\label{tab:mmlu_abalation} Accuracy on LM evaluation harness tasks on GPT3-22B model.}
\end{table}

\begin{table} \centering
\begin{tabular}{|c||c|c|c|c||c|c|c|c|} 
\hline
 $L_b \rightarrow$& \multicolumn{4}{c||}{8} & \multicolumn{4}{c||}{8}\\
 \hline
 \backslashbox{$L_A$\kern-1em}{\kern-1em$N_c$} & 2 & 4 & 8 & 16 & 2 & 4 & 8 & 16  \\
 %$N_c \rightarrow$ & 2 & 4 & 8 & 16 & 2 & 4 & 2 \\
 \hline
 \hline
 \multicolumn{5}{|c|}{Race (FP32 Accuracy = 44.4\%)} & \multicolumn{4}{|c|}{Boolq (FP32 Accuracy = 79.29\%)} \\ 
 \hline
 \hline
 64 & 42.49 & 42.51 & 42.58 & 43.45 & 77.58 & 77.37 & 77.43 & 78.1 \\
 \hline
 32 & 43.35 & 42.49 & 43.64 & 43.73 & 77.86 & 75.32 & 77.28 & 77.86  \\
 \hline
 16 & 44.21 & 44.21 & 43.64 & 42.97 & 78.65 & 77 & 76.94 & 77.98  \\
 \hline
 \hline
 \multicolumn{5}{|c|}{Winogrande (FP32 Accuracy = 69.38\%)} & \multicolumn{4}{|c|}{Piqa (FP32 Accuracy = 78.07\%)} \\ 
 \hline
 \hline
 64 & 68.9 & 68.43 & 69.77 & 68.19 & 77.09 & 76.82 & 77.09 & 77.86 \\
 \hline
 32 & 69.38 & 68.51 & 68.82 & 68.90 & 78.07 & 76.71 & 78.07 & 77.86  \\
 \hline
 16 & 69.53 & 67.09 & 69.38 & 68.90 & 77.37 & 77.8 & 77.91 & 77.69  \\
 \hline
\end{tabular}
\caption{\label{tab:mmlu_abalation} Accuracy on LM evaluation harness tasks on Llama2-7B model.}
\end{table}

\begin{table} \centering
\begin{tabular}{|c||c|c|c|c||c|c|c|c|} 
\hline
 $L_b \rightarrow$& \multicolumn{4}{c||}{8} & \multicolumn{4}{c||}{8}\\
 \hline
 \backslashbox{$L_A$\kern-1em}{\kern-1em$N_c$} & 2 & 4 & 8 & 16 & 2 & 4 & 8 & 16  \\
 %$N_c \rightarrow$ & 2 & 4 & 8 & 16 & 2 & 4 & 2 \\
 \hline
 \hline
 \multicolumn{5}{|c|}{Race (FP32 Accuracy = 48.8\%)} & \multicolumn{4}{|c|}{Boolq (FP32 Accuracy = 85.23\%)} \\ 
 \hline
 \hline
 64 & 49.00 & 49.00 & 49.28 & 48.71 & 82.82 & 84.28 & 84.03 & 84.25 \\
 \hline
 32 & 49.57 & 48.52 & 48.33 & 49.28 & 83.85 & 84.46 & 84.31 & 84.93  \\
 \hline
 16 & 49.85 & 49.09 & 49.28 & 48.99 & 85.11 & 84.46 & 84.61 & 83.94  \\
 \hline
 \hline
 \multicolumn{5}{|c|}{Winogrande (FP32 Accuracy = 79.95\%)} & \multicolumn{4}{|c|}{Piqa (FP32 Accuracy = 81.56\%)} \\ 
 \hline
 \hline
 64 & 78.77 & 78.45 & 78.37 & 79.16 & 81.45 & 80.69 & 81.45 & 81.5 \\
 \hline
 32 & 78.45 & 79.01 & 78.69 & 80.66 & 81.56 & 80.58 & 81.18 & 81.34  \\
 \hline
 16 & 79.95 & 79.56 & 79.79 & 79.72 & 81.28 & 81.66 & 81.28 & 80.96  \\
 \hline
\end{tabular}
\caption{\label{tab:mmlu_abalation} Accuracy on LM evaluation harness tasks on Llama2-70B model.}
\end{table}

%\section{MSE Studies}
%\textcolor{red}{TODO}


\subsection{Number Formats and Quantization Method}
\label{subsec:numFormats_quantMethod}
\subsubsection{Integer Format}
An $n$-bit signed integer (INT) is typically represented with a 2s-complement format \citep{yao2022zeroquant,xiao2023smoothquant,dai2021vsq}, where the most significant bit denotes the sign.

\subsubsection{Floating Point Format}
An $n$-bit signed floating point (FP) number $x$ comprises of a 1-bit sign ($x_{\mathrm{sign}}$), $B_m$-bit mantissa ($x_{\mathrm{mant}}$) and $B_e$-bit exponent ($x_{\mathrm{exp}}$) such that $B_m+B_e=n-1$. The associated constant exponent bias ($E_{\mathrm{bias}}$) is computed as $(2^{{B_e}-1}-1)$. We denote this format as $E_{B_e}M_{B_m}$.  

\subsubsection{Quantization Scheme}
\label{subsec:quant_method}
A quantization scheme dictates how a given unquantized tensor is converted to its quantized representation. We consider FP formats for the purpose of illustration. Given an unquantized tensor $\bm{X}$ and an FP format $E_{B_e}M_{B_m}$, we first, we compute the quantization scale factor $s_X$ that maps the maximum absolute value of $\bm{X}$ to the maximum quantization level of the $E_{B_e}M_{B_m}$ format as follows:
\begin{align}
\label{eq:sf}
    s_X = \frac{\mathrm{max}(|\bm{X}|)}{\mathrm{max}(E_{B_e}M_{B_m})}
\end{align}
In the above equation, $|\cdot|$ denotes the absolute value function.

Next, we scale $\bm{X}$ by $s_X$ and quantize it to $\hat{\bm{X}}$ by rounding it to the nearest quantization level of $E_{B_e}M_{B_m}$ as:

\begin{align}
\label{eq:tensor_quant}
    \hat{\bm{X}} = \text{round-to-nearest}\left(\frac{\bm{X}}{s_X}, E_{B_e}M_{B_m}\right)
\end{align}

We perform dynamic max-scaled quantization \citep{wu2020integer}, where the scale factor $s$ for activations is dynamically computed during runtime.

\subsection{Vector Scaled Quantization}
\begin{wrapfigure}{r}{0.35\linewidth}
  \centering
  \includegraphics[width=\linewidth]{sections/figures/vsquant.jpg}
  \caption{\small Vectorwise decomposition for per-vector scaled quantization (VSQ \citep{dai2021vsq}).}
  \label{fig:vsquant}
\end{wrapfigure}
During VSQ \citep{dai2021vsq}, the operand tensors are decomposed into 1D vectors in a hardware friendly manner as shown in Figure \ref{fig:vsquant}. Since the decomposed tensors are used as operands in matrix multiplications during inference, it is beneficial to perform this decomposition along the reduction dimension of the multiplication. The vectorwise quantization is performed similar to tensorwise quantization described in Equations \ref{eq:sf} and \ref{eq:tensor_quant}, where a scale factor $s_v$ is required for each vector $\bm{v}$ that maps the maximum absolute value of that vector to the maximum quantization level. While smaller vector lengths can lead to larger accuracy gains, the associated memory and computational overheads due to the per-vector scale factors increases. To alleviate these overheads, VSQ \citep{dai2021vsq} proposed a second level quantization of the per-vector scale factors to unsigned integers, while MX \citep{rouhani2023shared} quantizes them to integer powers of 2 (denoted as $2^{INT}$).

\subsubsection{MX Format}
The MX format proposed in \citep{rouhani2023microscaling} introduces the concept of sub-block shifting. For every two scalar elements of $b$-bits each, there is a shared exponent bit. The value of this exponent bit is determined through an empirical analysis that targets minimizing quantization MSE. We note that the FP format $E_{1}M_{b}$ is strictly better than MX from an accuracy perspective since it allocates a dedicated exponent bit to each scalar as opposed to sharing it across two scalars. Therefore, we conservatively bound the accuracy of a $b+2$-bit signed MX format with that of a $E_{1}M_{b}$ format in our comparisons. For instance, we use E1M2 format as a proxy for MX4.

\begin{figure}
    \centering
    \includegraphics[width=1\linewidth]{sections//figures/BlockFormats.pdf}
    \caption{\small Comparing LO-BCQ to MX format.}
    \label{fig:block_formats}
\end{figure}

Figure \ref{fig:block_formats} compares our $4$-bit LO-BCQ block format to MX \citep{rouhani2023microscaling}. As shown, both LO-BCQ and MX decompose a given operand tensor into block arrays and each block array into blocks. Similar to MX, we find that per-block quantization ($L_b < L_A$) leads to better accuracy due to increased flexibility. While MX achieves this through per-block $1$-bit micro-scales, we associate a dedicated codebook to each block through a per-block codebook selector. Further, MX quantizes the per-block array scale-factor to E8M0 format without per-tensor scaling. In contrast during LO-BCQ, we find that per-tensor scaling combined with quantization of per-block array scale-factor to E4M3 format results in superior inference accuracy across models. 


\end{document}
\endinput
%%
%% End of file `sample-acmsmall-submission.tex'.
