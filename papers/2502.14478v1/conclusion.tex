
\section{Conclusion}
\label{sec:conclusion}

We presented the first approach for synthesising property-rich
inductive predicates for data structures in Separation Logic (SL) from
concrete heap graph examples, by positive-only learning via Answer Set
Programming, with SL-based pruning.
%
% The essence of our approach is to extend the classical inductive logic
% programming by (1) supporting positive-only learning and (2) tune the search
% for SL predicates, both with the help of Answer Set Programming.
%
Our framework \tool is capable of automatically learning predicates
for complex structures with payload constraints and mutual recursion,
facilitating applications of SL-based tools for deductive verification
and program synthesis. 
%
In the future, we are planning to explore other possible applications
of our predicate synthesiser for program
repair~\cite{Tonder-LeGoues:ICSE18}, program
comprehension~\cite{DBLP:conf/iwpc/BoockmannL22}, and Computer Science
education~\cite{marron2012abstracting}.

\begin{acks}
  We thank Vladimir Gladshtein, Yunjeong Lee, Peter
  Müller, Hila Peleg, George Pîrlea, and Qiyuan Zhao for their feedback on
  earlier drafts of this paper.
%
  We also thank the reviewers of OOPSLA'25 for their constructive and
  insightful comments.
%
  This work was partially supported by a Singapore Ministry of
  Education (MoE) Tier 3 grant ``Automated Program Repair''
  MOE-MOET32021-0001.
\end{acks}



\section*{Data Availability}

The implementations of \tool, \ggen, and the benchmark harness
necessary for reproducing our experimental results in
\autoref{sec:evaluation} are available online~\cite{sippy-artefact}. 

% at
% \url{https://zenodo.org/records/13918252}. An easy-to-setup version
% will be submitted for the artefact evaluation.

% Acknowledgements

% Yunjeong Lee
% George Pîrlea
% Qiyuan Zhao
% Vladimir Gladshtein
% Peter Müller
% Hila Peleg
% Vikram Goyal
