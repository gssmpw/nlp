\vspace{-5pt}

\section{Automated Heap Graph Generation via \ggen}
\label{sec:generator}


As presented, \tool requires input graphs to be provided explicitly.
%
The final technical contribution of this work is \ggen---an auxiliary
tool that allows one to \emph{automatically generate valid structure
  graphs} via programs that take them as inputs and are most likely
already available to the user.

Valid memory graphs can be obtained by extracting concrete
heap states from the program execution, provided concrete inputs, as,
\eg, done by the \sling tool~\cite{le2019sling}.
%
This approach is, however, only applicable if the program in question
\emph{generates} a data structure instance, and is problematic if it
\emph{expects} it as an input.
%
Our idea for automated graph generating is inspired by works on
fuzz-testing and uses a structure-expecting program as a
\emph{validator} for candidate graphs.
%
Given a program with assertions, \ggen generates \emph{arbitrary}
input memory graphs, subject to basic validity constraints (\eg, any
node should be reachable from its root), and keeps the graphs that
pass the assertions together with example facts summarised from those
graphs by traversing them and accumulating their payload in a set, so
they can be used as inputs for \tool.
%
The details are given in \autoref{alg:generateGraphs}.

One can argue that the generator defined this way can also effectively
produce \emph{negative} examples, thus removing the need for our
positive-only learning.
%
In practice, however, the number of the negative examples can be large
due to the randomness of the generator that has no knowledge of the
data structure (\cf~\autoref{sec:verification}). This makes it
impractical to use them for ILP-based synthesis that cannot
effectively discriminate good (\ie, informative) negative examples
from arbitrary junk.


\begin{algorithm}[t]
  \caption{Generating random memory graphs and examples}
    \label{alg:generateGraphs}
    \begin{algorithmic}[1]
    \Require $p$: Program with assertions as tests
    \Require $n$: Number of graphs to be generated
    % \Ensure $n$ memory graphs that passed the test
    \State Initialize $\mathit{GraphsAndExamples} \gets \emptyset$
    \For{$\mathit{sz} \gets 1$ to $max\_size$}
        \State $\mathit{cnt}$ = 0
        \While{$\mathit{cnt} < \lceil n/\mathit{max\_size} \rceil$}
        \State $\mathit{node}_1, \dots, \mathit{node}_{\mathit{sz}} \gets \mathit{init\_node}()$
        \State $\mathit{node}_i.key \gets \text{random(int)}$ for $i \in [1, \mathit{sz}]$
        \State $\mathit{node}_i.pointer \gets \text{random(node)}$ for $i \in [1, \mathit{sz}]$
        \If {$p(\mathit{node}_1, \dots, \mathit{node}_{\mathit{sz}})$ passes the tests}
        \State $\text{cnt} \gets \text{cnt} + 1$
        \State $\mathit{example} \gets \text{summarise\_graph}(node_1, \dots, node_{\mathit{sz}})$
        \State $\mathit{GraphsAndExamples}.\text{add}(\text{graph}(\mathit{node}_1, \dots, \mathit{node}_{\mathit{sz}}), \mathit{example})$
        \EndIf
        \EndWhile
    \EndFor
    
    \State \Return $\mathit{GraphsAndExamples}$
    \end{algorithmic}
\end{algorithm}

