\appendix
\label{sec:appendix}

\section{Answer Set Programming in \tool and Graph \ggen.}
In this appendix, we provide a brief overview of the ASP encoding in \tool and \ggen. The notations are standard in ASP, and we will explain them in our context without describing the semantics. An interested reader can refer to the ASP textbook~\cite{lifschitz2019answer} or a tutorial~\cite{aspguide}.

\subsection{Minimisation of Redundant Predicates}
\label{app:minimisation}

The minimisation procedure in \autoref{sec:normalise} of the main text is easily encoded by
the following constraint:
%
\begin{minted}[fontsize=\small]{prolog}
  :- entail(H, H0), entail(H0, H), lit(H0)<lit(H), output(H).
\end{minted}
%
The ASP definition of the \pcode{entail(H, H0)} is provided as part of
our framework for several common first-order theories (\eg, integers,
finite sets, and lists).



\subsection{ASP-based Pruning by SL Semantics}
\label{app:slsemantics}


The listed constraints present SL semantics based pruning shown in \autoref{sec:semantics} of the main paper.
%
\begin{enumerate}
\item \emph{Basic Reachability}: for a pointer domain $P$ and a node $A$, if another node is reached from $A$ by a pointer in $P$, then $A$ must be the "this" node.
\begin{minted}[fontsize=\small]{prolog}
  :- pointer(P), body_lit(_, P, (A, _)), not this(A).
\end{minted}
%
\item \emph{Basic Assumptions}: "this" node in the basic case is either the null node or equal to another node.
\begin{minted}[fontsize=\small]{prolog}
:- this(X), not null_base(X), not eq_base(X, _).
\end{minted}
%
\item \emph{Restricted use of} \code{null}: if a node $A$ is a null pointer in clause $T$, then it can only appear twice in the literals of the clause.
\begin{minted}[fontsize=\small]{prolog}
:- body_lit(T, nullptr, (A,)), #count{P,Vars : var_in_literal(T,P,Vars,A)} != 2.
\end{minted}
%
\item \emph{Quasi-well-founded recursion of payload}: for the pure variable $B1$ in the head of the clause $T$, it should not be less than the pure variable $B2$ in the body of the clause $T$ in partial order.
\begin{minted}[fontsize=\small]{prolog}
:- pure_in_head(T, B1), pure_in_body(T, B2), not partial_le(T, 1, B2, B1).
\end{minted}
Note that by default in \tool, the variables of set and list theories follow the partial order, but not for integer theory: if the integer relation is to describe height or length, the \pcode{po_type(int)} predicate can be manually added as input to accelerate the search (458s to 130s for the balanced tree predicate). Actually we have different parameters in \tool for more practical uses.
%
\item \emph{Heap functionality}: there should be at most one pointer reached from a single domain $P$ in a clause $T$.
\begin{minted}[fontsize=\small]{prolog}
:- clause(T), pointer(P),
   #count{Var : var_in_body_pos(T, P, _, Var)} > 1.
\end{minted}
%
\end{enumerate}
%


\subsection{Auxiliary Placeholders}
\label{app:auxiliary}

On the implementation level for the auxiliary placeholder introduced in \autoref{sec:auxiliary} of the main text, this requires adding an ASP constraint
that forces the parameter of the placeholder predicate (\pcode{Y}
here) to appear \emph{twice} in the whole clause, so it could be later
translated into a single occurrence of a free variable as follows:

\begin{minted}[fontsize=\small]{prolog}
  :- body_lit(T, anynumber, (A,)),
     #count{P,Vars : var_in_literal(T,P,Vars,A)} != 2.
\end{minted}


\vspace{1em}
To sum up, the declarative encoding of ASP gives us a compositional way to encode the different domain knowledge, then achieves the efficient search by the advanced ASP solving. 
