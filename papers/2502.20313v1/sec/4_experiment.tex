
\section{Experiments}
\label{sec:exp}

\subsection{Implementation details}
\noindent \textbf{FlexVAR tokenizer.}
Our scalable VQVAE tokenizer is configured with a downsampling factor of 16 and is initialized with the pre-trained weights from LlamaGen \cite{llamagen}, the codebook size is set to 8912, and the latent space dimension is set to 32. The quantization of each scale shares the same codebook. We follow the VQVAE training recipe of LlamaGen. The training is on OpenImages \cite{openimages} with a constant learning rate of $10^{-4}$, AdamW optimizer with $\beta_1 = 0.9$, $\beta_2 = 0.95$, weight decay = 0.05,  a batch size of 128, and for 20 epochs. $K$ is set to 5 by default, indicating that each latent space is randomly sampled into 5 different resolutions.

\noindent \textbf{FlexVAR transformer.}
We provide FlexVAR in three scales, with detailed configurations for each scale provided in Tab \ref{tab:scale}.
FlexVAR is trained on the ImageNet-1K 256$\times$256 using 80GB A100 GPUs. The training process employs the AdamW optimizer with $\beta_1 = 0.9$, $\beta_2 = 0.95$, and a weight decay rate of 0.05. The learning rate is set to 1e-4, with the training epochs varying between 180 and 350 depending on the model scale.

\begin{table}[ht]
  \centering
  \footnotesize
  % \vspace{-10pt}
% \vspace{-3mm}
  % \begin{threeparttable}  
  % \resizebox{1.0\textwidth}{
    \renewcommand\arraystretch{1.2} % 1.95
    % \begin{tabular}{l|ccccc}
  \resizebox{0.48\textwidth}{!}{
    \begin{tabular}{c|ccccc}
      % \toprule
      \Xhline{0.7pt}

       \textbf{Model name} & \textbf{Layers} & \textbf{Params.} & \textbf{Heads} & \textbf{Dims.} & \textbf{Epoch}\\ 
      \hline
      FlexVAR-$d16$ & 16 & 310M & 16 & 1024 & 180 \\     
      FlexVAR-$d20$ & 20 & 600M & 20 & 1280 & 250  \\     
      FlexVAR-$d24$ & 24 & 1.0B & 24 & 1536 & 350 \\     
      \Xhline{0.7pt}
      \end{tabular}
      }
  \vspace{-2mm}
  \caption{Architectural design and training epochs of FlexVAR.}
      \label{tab:scale}
  \end{table}  
  \vspace{-2mm}



% This must be in the first 5 lines to tell arXiv to use pdfLaTeX, which is strongly recommended.
\pdfoutput=1
% In particular, the hyperref package requires pdfLaTeX in order to break URLs across lines.

\documentclass[11pt]{article}

% Change "review" to "final" to generate the final (sometimes called camera-ready) version.
% Change to "preprint" to generate a non-anonymous version with page numbers.
\usepackage{acl}

% Standard package includes
\usepackage{times}
\usepackage{latexsym}

% Draw tables
\usepackage{booktabs}
\usepackage{multirow}
\usepackage{xcolor}
\usepackage{colortbl}
\usepackage{array} 
\usepackage{amsmath}

\newcolumntype{C}{>{\centering\arraybackslash}p{0.07\textwidth}}
% For proper rendering and hyphenation of words containing Latin characters (including in bib files)
\usepackage[T1]{fontenc}
% For Vietnamese characters
% \usepackage[T5]{fontenc}
% See https://www.latex-project.org/help/documentation/encguide.pdf for other character sets
% This assumes your files are encoded as UTF8
\usepackage[utf8]{inputenc}

% This is not strictly necessary, and may be commented out,
% but it will improve the layout of the manuscript,
% and will typically save some space.
\usepackage{microtype}
\DeclareMathOperator*{\argmax}{arg\,max}
% This is also not strictly necessary, and may be commented out.
% However, it will improve the aesthetics of text in
% the typewriter font.
\usepackage{inconsolata}

%Including images in your LaTeX document requires adding
%additional package(s)
\usepackage{graphicx}
% If the title and author information does not fit in the area allocated, uncomment the following
%
%\setlength\titlebox{<dim>}
%
% and set <dim> to something 5cm or larger.

\title{Wi-Chat: Large Language Model Powered Wi-Fi Sensing}

% Author information can be set in various styles:
% For several authors from the same institution:
% \author{Author 1 \and ... \and Author n \\
%         Address line \\ ... \\ Address line}
% if the names do not fit well on one line use
%         Author 1 \\ {\bf Author 2} \\ ... \\ {\bf Author n} \\
% For authors from different institutions:
% \author{Author 1 \\ Address line \\  ... \\ Address line
%         \And  ... \And
%         Author n \\ Address line \\ ... \\ Address line}
% To start a separate ``row'' of authors use \AND, as in
% \author{Author 1 \\ Address line \\  ... \\ Address line
%         \AND
%         Author 2 \\ Address line \\ ... \\ Address line \And
%         Author 3 \\ Address line \\ ... \\ Address line}

% \author{First Author \\
%   Affiliation / Address line 1 \\
%   Affiliation / Address line 2 \\
%   Affiliation / Address line 3 \\
%   \texttt{email@domain} \\\And
%   Second Author \\
%   Affiliation / Address line 1 \\
%   Affiliation / Address line 2 \\
%   Affiliation / Address line 3 \\
%   \texttt{email@domain} \\}
% \author{Haohan Yuan \qquad Haopeng Zhang\thanks{corresponding author} \\ 
%   ALOHA Lab, University of Hawaii at Manoa \\
%   % Affiliation / Address line 2 \\
%   % Affiliation / Address line 3 \\
%   \texttt{\{haohany,haopengz\}@hawaii.edu}}
  
\author{
{Haopeng Zhang$\dag$\thanks{These authors contributed equally to this work.}, Yili Ren$\ddagger$\footnotemark[1], Haohan Yuan$\dag$, Jingzhe Zhang$\ddagger$, Yitong Shen$\ddagger$} \\
ALOHA Lab, University of Hawaii at Manoa$\dag$, University of South Florida$\ddagger$ \\
\{haopengz, haohany\}@hawaii.edu\\
\{yiliren, jingzhe, shen202\}@usf.edu\\}



  
%\author{
%  \textbf{First Author\textsuperscript{1}},
%  \textbf{Second Author\textsuperscript{1,2}},
%  \textbf{Third T. Author\textsuperscript{1}},
%  \textbf{Fourth Author\textsuperscript{1}},
%\\
%  \textbf{Fifth Author\textsuperscript{1,2}},
%  \textbf{Sixth Author\textsuperscript{1}},
%  \textbf{Seventh Author\textsuperscript{1}},
%  \textbf{Eighth Author \textsuperscript{1,2,3,4}},
%\\
%  \textbf{Ninth Author\textsuperscript{1}},
%  \textbf{Tenth Author\textsuperscript{1}},
%  \textbf{Eleventh E. Author\textsuperscript{1,2,3,4,5}},
%  \textbf{Twelfth Author\textsuperscript{1}},
%\\
%  \textbf{Thirteenth Author\textsuperscript{3}},
%  \textbf{Fourteenth F. Author\textsuperscript{2,4}},
%  \textbf{Fifteenth Author\textsuperscript{1}},
%  \textbf{Sixteenth Author\textsuperscript{1}},
%\\
%  \textbf{Seventeenth S. Author\textsuperscript{4,5}},
%  \textbf{Eighteenth Author\textsuperscript{3,4}},
%  \textbf{Nineteenth N. Author\textsuperscript{2,5}},
%  \textbf{Twentieth Author\textsuperscript{1}}
%\\
%\\
%  \textsuperscript{1}Affiliation 1,
%  \textsuperscript{2}Affiliation 2,
%  \textsuperscript{3}Affiliation 3,
%  \textsuperscript{4}Affiliation 4,
%  \textsuperscript{5}Affiliation 5
%\\
%  \small{
%    \textbf{Correspondence:} \href{mailto:email@domain}{email@domain}
%  }
%}

\begin{document}
\maketitle
\begin{abstract}
Recent advancements in Large Language Models (LLMs) have demonstrated remarkable capabilities across diverse tasks. However, their potential to integrate physical model knowledge for real-world signal interpretation remains largely unexplored. In this work, we introduce Wi-Chat, the first LLM-powered Wi-Fi-based human activity recognition system. We demonstrate that LLMs can process raw Wi-Fi signals and infer human activities by incorporating Wi-Fi sensing principles into prompts. Our approach leverages physical model insights to guide LLMs in interpreting Channel State Information (CSI) data without traditional signal processing techniques. Through experiments on real-world Wi-Fi datasets, we show that LLMs exhibit strong reasoning capabilities, achieving zero-shot activity recognition. These findings highlight a new paradigm for Wi-Fi sensing, expanding LLM applications beyond conventional language tasks and enhancing the accessibility of wireless sensing for real-world deployments.
\end{abstract}

\section{Introduction}

In today’s rapidly evolving digital landscape, the transformative power of web technologies has redefined not only how services are delivered but also how complex tasks are approached. Web-based systems have become increasingly prevalent in risk control across various domains. This widespread adoption is due their accessibility, scalability, and ability to remotely connect various types of users. For example, these systems are used for process safety management in industry~\cite{kannan2016web}, safety risk early warning in urban construction~\cite{ding2013development}, and safe monitoring of infrastructural systems~\cite{repetto2018web}. Within these web-based risk management systems, the source search problem presents a huge challenge. Source search refers to the task of identifying the origin of a risky event, such as a gas leak and the emission point of toxic substances. This source search capability is crucial for effective risk management and decision-making.

Traditional approaches to implementing source search capabilities into the web systems often rely on solely algorithmic solutions~\cite{ristic2016study}. These methods, while relatively straightforward to implement, often struggle to achieve acceptable performances due to algorithmic local optima and complex unknown environments~\cite{zhao2020searching}. More recently, web crowdsourcing has emerged as a promising alternative for tackling the source search problem by incorporating human efforts in these web systems on-the-fly~\cite{zhao2024user}. This approach outsources the task of addressing issues encountered during the source search process to human workers, leveraging their capabilities to enhance system performance.

These solutions often employ a human-AI collaborative way~\cite{zhao2023leveraging} where algorithms handle exploration-exploitation and report the encountered problems while human workers resolve complex decision-making bottlenecks to help the algorithms getting rid of local deadlocks~\cite{zhao2022crowd}. Although effective, this paradigm suffers from two inherent limitations: increased operational costs from continuous human intervention, and slow response times of human workers due to sequential decision-making. These challenges motivate our investigation into developing autonomous systems that preserve human-like reasoning capabilities while reducing dependency on massive crowdsourced labor.

Furthermore, recent advancements in large language models (LLMs)~\cite{chang2024survey} and multi-modal LLMs (MLLMs)~\cite{huang2023chatgpt} have unveiled promising avenues for addressing these challenges. One clear opportunity involves the seamless integration of visual understanding and linguistic reasoning for robust decision-making in search tasks. However, whether large models-assisted source search is really effective and efficient for improving the current source search algorithms~\cite{ji2022source} remains unknown. \textit{To address the research gap, we are particularly interested in answering the following two research questions in this work:}

\textbf{\textit{RQ1: }}How can source search capabilities be integrated into web-based systems to support decision-making in time-sensitive risk management scenarios? 
% \sq{I mention ``time-sensitive'' here because I feel like we shall say something about the response time -- LLM has to be faster than humans}

\textbf{\textit{RQ2: }}How can MLLMs and LLMs enhance the effectiveness and efficiency of existing source search algorithms? 

% \textit{\textbf{RQ2:}} To what extent does the performance of large models-assisted search align with or approach the effectiveness of human-AI collaborative search? 

To answer the research questions, we propose a novel framework called Auto-\
S$^2$earch (\textbf{Auto}nomous \textbf{S}ource \textbf{Search}) and implement a prototype system that leverages advanced web technologies to simulate real-world conditions for zero-shot source search. Unlike traditional methods that rely on pre-defined heuristics or extensive human intervention, AutoS$^2$earch employs a carefully designed prompt that encapsulates human rationales, thereby guiding the MLLM to generate coherent and accurate scene descriptions from visual inputs about four directional choices. Based on these language-based descriptions, the LLM is enabled to determine the optimal directional choice through chain-of-thought (CoT) reasoning. Comprehensive empirical validation demonstrates that AutoS$^2$-\ 
earch achieves a success rate of 95–98\%, closely approaching the performance of human-AI collaborative search across 20 benchmark scenarios~\cite{zhao2023leveraging}. 

Our work indicates that the role of humans in future web crowdsourcing tasks may evolve from executors to validators or supervisors. Furthermore, incorporating explanations of LLM decisions into web-based system interfaces has the potential to help humans enhance task performance in risk control.






\section{Related Work}
\label{sec:relatedworks}

% \begin{table*}[t]
% \centering 
% \renewcommand\arraystretch{0.98}
% \fontsize{8}{10}\selectfont \setlength{\tabcolsep}{0.4em}
% \begin{tabular}{@{}lc|cc|cc|cc@{}}
% \toprule
% \textbf{Methods}           & \begin{tabular}[c]{@{}c@{}}\textbf{Training}\\ \textbf{Paradigm}\end{tabular} & \begin{tabular}[c]{@{}c@{}}\textbf{$\#$ PT Data}\\ \textbf{(Tokens)}\end{tabular} & \begin{tabular}[c]{@{}c@{}}\textbf{$\#$ IFT Data}\\ \textbf{(Samples)}\end{tabular} & \textbf{Code}  & \begin{tabular}[c]{@{}c@{}}\textbf{Natural}\\ \textbf{Language}\end{tabular} & \begin{tabular}[c]{@{}c@{}}\textbf{Action}\\ \textbf{Trajectories}\end{tabular} & \begin{tabular}[c]{@{}c@{}}\textbf{API}\\ \textbf{Documentation}\end{tabular}\\ \midrule 
% NexusRaven~\citep{srinivasan2023nexusraven} & IFT & - & - & \textcolor{green}{\CheckmarkBold} & \textcolor{green}{\CheckmarkBold} &\textcolor{red}{\XSolidBrush}&\textcolor{red}{\XSolidBrush}\\
% AgentInstruct~\citep{zeng2023agenttuning} & IFT & - & 2k & \textcolor{green}{\CheckmarkBold} & \textcolor{green}{\CheckmarkBold} &\textcolor{red}{\XSolidBrush}&\textcolor{red}{\XSolidBrush} \\
% AgentEvol~\citep{xi2024agentgym} & IFT & - & 14.5k & \textcolor{green}{\CheckmarkBold} & \textcolor{green}{\CheckmarkBold} &\textcolor{green}{\CheckmarkBold}&\textcolor{red}{\XSolidBrush} \\
% Gorilla~\citep{patil2023gorilla}& IFT & - & 16k & \textcolor{green}{\CheckmarkBold} & \textcolor{green}{\CheckmarkBold} &\textcolor{red}{\XSolidBrush}&\textcolor{green}{\CheckmarkBold}\\
% OpenFunctions-v2~\citep{patil2023gorilla} & IFT & - & 65k & \textcolor{green}{\CheckmarkBold} & \textcolor{green}{\CheckmarkBold} &\textcolor{red}{\XSolidBrush}&\textcolor{green}{\CheckmarkBold}\\
% LAM~\citep{zhang2024agentohana} & IFT & - & 42.6k & \textcolor{green}{\CheckmarkBold} & \textcolor{green}{\CheckmarkBold} &\textcolor{green}{\CheckmarkBold}&\textcolor{red}{\XSolidBrush} \\
% xLAM~\citep{liu2024apigen} & IFT & - & 60k & \textcolor{green}{\CheckmarkBold} & \textcolor{green}{\CheckmarkBold} &\textcolor{green}{\CheckmarkBold}&\textcolor{red}{\XSolidBrush} \\\midrule
% LEMUR~\citep{xu2024lemur} & PT & 90B & 300k & \textcolor{green}{\CheckmarkBold} & \textcolor{green}{\CheckmarkBold} &\textcolor{green}{\CheckmarkBold}&\textcolor{red}{\XSolidBrush}\\
% \rowcolor{teal!12} \method & PT & 103B & 95k & \textcolor{green}{\CheckmarkBold} & \textcolor{green}{\CheckmarkBold} & \textcolor{green}{\CheckmarkBold} & \textcolor{green}{\CheckmarkBold} \\
% \bottomrule
% \end{tabular}
% \caption{Summary of existing tuning- and pretraining-based LLM agents with their training sample sizes. "PT" and "IFT" denote "Pre-Training" and "Instruction Fine-Tuning", respectively. }
% \label{tab:related}
% \end{table*}

\begin{table*}[ht]
\begin{threeparttable}
\centering 
\renewcommand\arraystretch{0.98}
\fontsize{7}{9}\selectfont \setlength{\tabcolsep}{0.2em}
\begin{tabular}{@{}l|c|c|ccc|cc|cc|cccc@{}}
\toprule
\textbf{Methods} & \textbf{Datasets}           & \begin{tabular}[c]{@{}c@{}}\textbf{Training}\\ \textbf{Paradigm}\end{tabular} & \begin{tabular}[c]{@{}c@{}}\textbf{\# PT Data}\\ \textbf{(Tokens)}\end{tabular} & \begin{tabular}[c]{@{}c@{}}\textbf{\# IFT Data}\\ \textbf{(Samples)}\end{tabular} & \textbf{\# APIs} & \textbf{Code}  & \begin{tabular}[c]{@{}c@{}}\textbf{Nat.}\\ \textbf{Lang.}\end{tabular} & \begin{tabular}[c]{@{}c@{}}\textbf{Action}\\ \textbf{Traj.}\end{tabular} & \begin{tabular}[c]{@{}c@{}}\textbf{API}\\ \textbf{Doc.}\end{tabular} & \begin{tabular}[c]{@{}c@{}}\textbf{Func.}\\ \textbf{Call}\end{tabular} & \begin{tabular}[c]{@{}c@{}}\textbf{Multi.}\\ \textbf{Step}\end{tabular}  & \begin{tabular}[c]{@{}c@{}}\textbf{Plan}\\ \textbf{Refine}\end{tabular}  & \begin{tabular}[c]{@{}c@{}}\textbf{Multi.}\\ \textbf{Turn}\end{tabular}\\ \midrule 
\multicolumn{13}{l}{\emph{Instruction Finetuning-based LLM Agents for Intrinsic Reasoning}}  \\ \midrule
FireAct~\cite{chen2023fireact} & FireAct & IFT & - & 2.1K & 10 & \textcolor{red}{\XSolidBrush} &\textcolor{green}{\CheckmarkBold} &\textcolor{green}{\CheckmarkBold}  & \textcolor{red}{\XSolidBrush} &\textcolor{green}{\CheckmarkBold} & \textcolor{red}{\XSolidBrush} &\textcolor{green}{\CheckmarkBold} & \textcolor{red}{\XSolidBrush} \\
ToolAlpaca~\cite{tang2023toolalpaca} & ToolAlpaca & IFT & - & 4.0K & 400 & \textcolor{red}{\XSolidBrush} &\textcolor{green}{\CheckmarkBold} &\textcolor{green}{\CheckmarkBold} & \textcolor{red}{\XSolidBrush} &\textcolor{green}{\CheckmarkBold} & \textcolor{red}{\XSolidBrush}  &\textcolor{green}{\CheckmarkBold} & \textcolor{red}{\XSolidBrush}  \\
ToolLLaMA~\cite{qin2023toolllm} & ToolBench & IFT & - & 12.7K & 16,464 & \textcolor{red}{\XSolidBrush} &\textcolor{green}{\CheckmarkBold} &\textcolor{green}{\CheckmarkBold} &\textcolor{red}{\XSolidBrush} &\textcolor{green}{\CheckmarkBold}&\textcolor{green}{\CheckmarkBold}&\textcolor{green}{\CheckmarkBold} &\textcolor{green}{\CheckmarkBold}\\
AgentEvol~\citep{xi2024agentgym} & AgentTraj-L & IFT & - & 14.5K & 24 &\textcolor{red}{\XSolidBrush} & \textcolor{green}{\CheckmarkBold} &\textcolor{green}{\CheckmarkBold}&\textcolor{red}{\XSolidBrush} &\textcolor{green}{\CheckmarkBold}&\textcolor{red}{\XSolidBrush} &\textcolor{red}{\XSolidBrush} &\textcolor{green}{\CheckmarkBold}\\
Lumos~\cite{yin2024agent} & Lumos & IFT  & - & 20.0K & 16 &\textcolor{red}{\XSolidBrush} & \textcolor{green}{\CheckmarkBold} & \textcolor{green}{\CheckmarkBold} &\textcolor{red}{\XSolidBrush} & \textcolor{green}{\CheckmarkBold} & \textcolor{green}{\CheckmarkBold} &\textcolor{red}{\XSolidBrush} & \textcolor{green}{\CheckmarkBold}\\
Agent-FLAN~\cite{chen2024agent} & Agent-FLAN & IFT & - & 24.7K & 20 &\textcolor{red}{\XSolidBrush} & \textcolor{green}{\CheckmarkBold} & \textcolor{green}{\CheckmarkBold} &\textcolor{red}{\XSolidBrush} & \textcolor{green}{\CheckmarkBold}& \textcolor{green}{\CheckmarkBold}&\textcolor{red}{\XSolidBrush} & \textcolor{green}{\CheckmarkBold}\\
AgentTuning~\citep{zeng2023agenttuning} & AgentInstruct & IFT & - & 35.0K & - &\textcolor{red}{\XSolidBrush} & \textcolor{green}{\CheckmarkBold} & \textcolor{green}{\CheckmarkBold} &\textcolor{red}{\XSolidBrush} & \textcolor{green}{\CheckmarkBold} &\textcolor{red}{\XSolidBrush} &\textcolor{red}{\XSolidBrush} & \textcolor{green}{\CheckmarkBold}\\\midrule
\multicolumn{13}{l}{\emph{Instruction Finetuning-based LLM Agents for Function Calling}} \\\midrule
NexusRaven~\citep{srinivasan2023nexusraven} & NexusRaven & IFT & - & - & 116 & \textcolor{green}{\CheckmarkBold} & \textcolor{green}{\CheckmarkBold}  & \textcolor{green}{\CheckmarkBold} &\textcolor{red}{\XSolidBrush} & \textcolor{green}{\CheckmarkBold} &\textcolor{red}{\XSolidBrush} &\textcolor{red}{\XSolidBrush}&\textcolor{red}{\XSolidBrush}\\
Gorilla~\citep{patil2023gorilla} & Gorilla & IFT & - & 16.0K & 1,645 & \textcolor{green}{\CheckmarkBold} &\textcolor{red}{\XSolidBrush} &\textcolor{red}{\XSolidBrush}&\textcolor{green}{\CheckmarkBold} &\textcolor{green}{\CheckmarkBold} &\textcolor{red}{\XSolidBrush} &\textcolor{red}{\XSolidBrush} &\textcolor{red}{\XSolidBrush}\\
OpenFunctions-v2~\citep{patil2023gorilla} & OpenFunctions-v2 & IFT & - & 65.0K & - & \textcolor{green}{\CheckmarkBold} & \textcolor{green}{\CheckmarkBold} &\textcolor{red}{\XSolidBrush} &\textcolor{green}{\CheckmarkBold} &\textcolor{green}{\CheckmarkBold} &\textcolor{red}{\XSolidBrush} &\textcolor{red}{\XSolidBrush} &\textcolor{red}{\XSolidBrush}\\
API Pack~\cite{guo2024api} & API Pack & IFT & - & 1.1M & 11,213 &\textcolor{green}{\CheckmarkBold} &\textcolor{red}{\XSolidBrush} &\textcolor{green}{\CheckmarkBold} &\textcolor{red}{\XSolidBrush} &\textcolor{green}{\CheckmarkBold} &\textcolor{red}{\XSolidBrush}&\textcolor{red}{\XSolidBrush}&\textcolor{red}{\XSolidBrush}\\ 
LAM~\citep{zhang2024agentohana} & AgentOhana & IFT & - & 42.6K & - & \textcolor{green}{\CheckmarkBold} & \textcolor{green}{\CheckmarkBold} &\textcolor{green}{\CheckmarkBold}&\textcolor{red}{\XSolidBrush} &\textcolor{green}{\CheckmarkBold}&\textcolor{red}{\XSolidBrush}&\textcolor{green}{\CheckmarkBold}&\textcolor{green}{\CheckmarkBold}\\
xLAM~\citep{liu2024apigen} & APIGen & IFT & - & 60.0K & 3,673 & \textcolor{green}{\CheckmarkBold} & \textcolor{green}{\CheckmarkBold} &\textcolor{green}{\CheckmarkBold}&\textcolor{red}{\XSolidBrush} &\textcolor{green}{\CheckmarkBold}&\textcolor{red}{\XSolidBrush}&\textcolor{green}{\CheckmarkBold}&\textcolor{green}{\CheckmarkBold}\\\midrule
\multicolumn{13}{l}{\emph{Pretraining-based LLM Agents}}  \\\midrule
% LEMUR~\citep{xu2024lemur} & PT & 90B & 300.0K & - & \textcolor{green}{\CheckmarkBold} & \textcolor{green}{\CheckmarkBold} &\textcolor{green}{\CheckmarkBold}&\textcolor{red}{\XSolidBrush} & \textcolor{red}{\XSolidBrush} &\textcolor{green}{\CheckmarkBold} &\textcolor{red}{\XSolidBrush}&\textcolor{red}{\XSolidBrush}\\
\rowcolor{teal!12} \method & \dataset & PT & 103B & 95.0K  & 76,537  & \textcolor{green}{\CheckmarkBold} & \textcolor{green}{\CheckmarkBold} & \textcolor{green}{\CheckmarkBold} & \textcolor{green}{\CheckmarkBold} & \textcolor{green}{\CheckmarkBold} & \textcolor{green}{\CheckmarkBold} & \textcolor{green}{\CheckmarkBold} & \textcolor{green}{\CheckmarkBold}\\
\bottomrule
\end{tabular}
% \begin{tablenotes}
%     \item $^*$ In addition, the StarCoder-API can offer 4.77M more APIs.
% \end{tablenotes}
\caption{Summary of existing instruction finetuning-based LLM agents for intrinsic reasoning and function calling, along with their training resources and sample sizes. "PT" and "IFT" denote "Pre-Training" and "Instruction Fine-Tuning", respectively.}
\vspace{-2ex}
\label{tab:related}
\end{threeparttable}
\end{table*}

\noindent \textbf{Prompting-based LLM Agents.} Due to the lack of agent-specific pre-training corpus, existing LLM agents rely on either prompt engineering~\cite{hsieh2023tool,lu2024chameleon,yao2022react,wang2023voyager} or instruction fine-tuning~\cite{chen2023fireact,zeng2023agenttuning} to understand human instructions, decompose high-level tasks, generate grounded plans, and execute multi-step actions. 
However, prompting-based methods mainly depend on the capabilities of backbone LLMs (usually commercial LLMs), failing to introduce new knowledge and struggling to generalize to unseen tasks~\cite{sun2024adaplanner,zhuang2023toolchain}. 

\noindent \textbf{Instruction Finetuning-based LLM Agents.} Considering the extensive diversity of APIs and the complexity of multi-tool instructions, tool learning inherently presents greater challenges than natural language tasks, such as text generation~\cite{qin2023toolllm}.
Post-training techniques focus more on instruction following and aligning output with specific formats~\cite{patil2023gorilla,hao2024toolkengpt,qin2023toolllm,schick2024toolformer}, rather than fundamentally improving model knowledge or capabilities. 
Moreover, heavy fine-tuning can hinder generalization or even degrade performance in non-agent use cases, potentially suppressing the original base model capabilities~\cite{ghosh2024a}.

\noindent \textbf{Pretraining-based LLM Agents.} While pre-training serves as an essential alternative, prior works~\cite{nijkamp2023codegen,roziere2023code,xu2024lemur,patil2023gorilla} have primarily focused on improving task-specific capabilities (\eg, code generation) instead of general-domain LLM agents, due to single-source, uni-type, small-scale, and poor-quality pre-training data. 
Existing tool documentation data for agent training either lacks diverse real-world APIs~\cite{patil2023gorilla, tang2023toolalpaca} or is constrained to single-tool or single-round tool execution. 
Furthermore, trajectory data mostly imitate expert behavior or follow function-calling rules with inferior planning and reasoning, failing to fully elicit LLMs' capabilities and handle complex instructions~\cite{qin2023toolllm}. 
Given a wide range of candidate API functions, each comprising various function names and parameters available at every planning step, identifying globally optimal solutions and generalizing across tasks remains highly challenging.



\section{Preliminaries}
\label{Preliminaries}
\begin{figure*}[t]
    \centering
    \includegraphics[width=0.95\linewidth]{fig/HealthGPT_Framework.png}
    \caption{The \ourmethod{} architecture integrates hierarchical visual perception and H-LoRA, employing a task-specific hard router to select visual features and H-LoRA plugins, ultimately generating outputs with an autoregressive manner.}
    \label{fig:architecture}
\end{figure*}
\noindent\textbf{Large Vision-Language Models.} 
The input to a LVLM typically consists of an image $x^{\text{img}}$ and a discrete text sequence $x^{\text{txt}}$. The visual encoder $\mathcal{E}^{\text{img}}$ converts the input image $x^{\text{img}}$ into a sequence of visual tokens $\mathcal{V} = [v_i]_{i=1}^{N_v}$, while the text sequence $x^{\text{txt}}$ is mapped into a sequence of text tokens $\mathcal{T} = [t_i]_{i=1}^{N_t}$ using an embedding function $\mathcal{E}^{\text{txt}}$. The LLM $\mathcal{M_\text{LLM}}(\cdot|\theta)$ models the joint probability of the token sequence $\mathcal{U} = \{\mathcal{V},\mathcal{T}\}$, which is expressed as:
\begin{equation}
    P_\theta(R | \mathcal{U}) = \prod_{i=1}^{N_r} P_\theta(r_i | \{\mathcal{U}, r_{<i}\}),
\end{equation}
where $R = [r_i]_{i=1}^{N_r}$ is the text response sequence. The LVLM iteratively generates the next token $r_i$ based on $r_{<i}$. The optimization objective is to minimize the cross-entropy loss of the response $\mathcal{R}$.
% \begin{equation}
%     \mathcal{L}_{\text{VLM}} = \mathbb{E}_{R|\mathcal{U}}\left[-\log P_\theta(R | \mathcal{U})\right]
% \end{equation}
It is worth noting that most LVLMs adopt a design paradigm based on ViT, alignment adapters, and pre-trained LLMs\cite{liu2023llava,liu2024improved}, enabling quick adaptation to downstream tasks.


\noindent\textbf{VQGAN.}
VQGAN~\cite{esser2021taming} employs latent space compression and indexing mechanisms to effectively learn a complete discrete representation of images. VQGAN first maps the input image $x^{\text{img}}$ to a latent representation $z = \mathcal{E}(x)$ through a encoder $\mathcal{E}$. Then, the latent representation is quantized using a codebook $\mathcal{Z} = \{z_k\}_{k=1}^K$, generating a discrete index sequence $\mathcal{I} = [i_m]_{m=1}^N$, where $i_m \in \mathcal{Z}$ represents the quantized code index:
\begin{equation}
    \mathcal{I} = \text{Quantize}(z|\mathcal{Z}) = \arg\min_{z_k \in \mathcal{Z}} \| z - z_k \|_2.
\end{equation}
In our approach, the discrete index sequence $\mathcal{I}$ serves as a supervisory signal for the generation task, enabling the model to predict the index sequence $\hat{\mathcal{I}}$ from input conditions such as text or other modality signals.  
Finally, the predicted index sequence $\hat{\mathcal{I}}$ is upsampled by the VQGAN decoder $G$, generating the high-quality image $\hat{x}^\text{img} = G(\hat{\mathcal{I}})$.



\noindent\textbf{Low Rank Adaptation.} 
LoRA\cite{hu2021lora} effectively captures the characteristics of downstream tasks by introducing low-rank adapters. The core idea is to decompose the bypass weight matrix $\Delta W\in\mathbb{R}^{d^{\text{in}} \times d^{\text{out}}}$ into two low-rank matrices $ \{A \in \mathbb{R}^{d^{\text{in}} \times r}, B \in \mathbb{R}^{r \times d^{\text{out}}} \}$, where $ r \ll \min\{d^{\text{in}}, d^{\text{out}}\} $, significantly reducing learnable parameters. The output with the LoRA adapter for the input $x$ is then given by:
\begin{equation}
    h = x W_0 + \alpha x \Delta W/r = x W_0 + \alpha xAB/r,
\end{equation}
where matrix $ A $ is initialized with a Gaussian distribution, while the matrix $ B $ is initialized as a zero matrix. The scaling factor $ \alpha/r $ controls the impact of $ \Delta W $ on the model.

\section{HealthGPT}
\label{Method}


\subsection{Unified Autoregressive Generation.}  
% As shown in Figure~\ref{fig:architecture}, 
\ourmethod{} (Figure~\ref{fig:architecture}) utilizes a discrete token representation that covers both text and visual outputs, unifying visual comprehension and generation as an autoregressive task. 
For comprehension, $\mathcal{M}_\text{llm}$ receives the input joint sequence $\mathcal{U}$ and outputs a series of text token $\mathcal{R} = [r_1, r_2, \dots, r_{N_r}]$, where $r_i \in \mathcal{V}_{\text{txt}}$, and $\mathcal{V}_{\text{txt}}$ represents the LLM's vocabulary:
\begin{equation}
    P_\theta(\mathcal{R} \mid \mathcal{U}) = \prod_{i=1}^{N_r} P_\theta(r_i \mid \mathcal{U}, r_{<i}).
\end{equation}
For generation, $\mathcal{M}_\text{llm}$ first receives a special start token $\langle \text{START\_IMG} \rangle$, then generates a series of tokens corresponding to the VQGAN indices $\mathcal{I} = [i_1, i_2, \dots, i_{N_i}]$, where $i_j \in \mathcal{V}_{\text{vq}}$, and $\mathcal{V}_{\text{vq}}$ represents the index range of VQGAN. Upon completion of generation, the LLM outputs an end token $\langle \text{END\_IMG} \rangle$:
\begin{equation}
    P_\theta(\mathcal{I} \mid \mathcal{U}) = \prod_{j=1}^{N_i} P_\theta(i_j \mid \mathcal{U}, i_{<j}).
\end{equation}
Finally, the generated index sequence $\mathcal{I}$ is fed into the decoder $G$, which reconstructs the target image $\hat{x}^{\text{img}} = G(\mathcal{I})$.

\subsection{Hierarchical Visual Perception}  
Given the differences in visual perception between comprehension and generation tasks—where the former focuses on abstract semantics and the latter emphasizes complete semantics—we employ ViT to compress the image into discrete visual tokens at multiple hierarchical levels.
Specifically, the image is converted into a series of features $\{f_1, f_2, \dots, f_L\}$ as it passes through $L$ ViT blocks.

To address the needs of various tasks, the hidden states are divided into two types: (i) \textit{Concrete-grained features} $\mathcal{F}^{\text{Con}} = \{f_1, f_2, \dots, f_k\}, k < L$, derived from the shallower layers of ViT, containing sufficient global features, suitable for generation tasks; 
(ii) \textit{Abstract-grained features} $\mathcal{F}^{\text{Abs}} = \{f_{k+1}, f_{k+2}, \dots, f_L\}$, derived from the deeper layers of ViT, which contain abstract semantic information closer to the text space, suitable for comprehension tasks.

The task type $T$ (comprehension or generation) determines which set of features is selected as the input for the downstream large language model:
\begin{equation}
    \mathcal{F}^{\text{img}}_T =
    \begin{cases}
        \mathcal{F}^{\text{Con}}, & \text{if } T = \text{generation task} \\
        \mathcal{F}^{\text{Abs}}, & \text{if } T = \text{comprehension task}
    \end{cases}
\end{equation}
We integrate the image features $\mathcal{F}^{\text{img}}_T$ and text features $\mathcal{T}$ into a joint sequence through simple concatenation, which is then fed into the LLM $\mathcal{M}_{\text{llm}}$ for autoregressive generation.
% :
% \begin{equation}
%     \mathcal{R} = \mathcal{M}_{\text{llm}}(\mathcal{U}|\theta), \quad \mathcal{U} = [\mathcal{F}^{\text{img}}_T; \mathcal{T}]
% \end{equation}
\subsection{Heterogeneous Knowledge Adaptation}
We devise H-LoRA, which stores heterogeneous knowledge from comprehension and generation tasks in separate modules and dynamically routes to extract task-relevant knowledge from these modules. 
At the task level, for each task type $ T $, we dynamically assign a dedicated H-LoRA submodule $ \theta^T $, which is expressed as:
\begin{equation}
    \mathcal{R} = \mathcal{M}_\text{LLM}(\mathcal{U}|\theta, \theta^T), \quad \theta^T = \{A^T, B^T, \mathcal{R}^T_\text{outer}\}.
\end{equation}
At the feature level for a single task, H-LoRA integrates the idea of Mixture of Experts (MoE)~\cite{masoudnia2014mixture} and designs an efficient matrix merging and routing weight allocation mechanism, thus avoiding the significant computational delay introduced by matrix splitting in existing MoELoRA~\cite{luo2024moelora}. Specifically, we first merge the low-rank matrices (rank = r) of $ k $ LoRA experts into a unified matrix:
\begin{equation}
    \mathbf{A}^{\text{merged}}, \mathbf{B}^{\text{merged}} = \text{Concat}(\{A_i\}_1^k), \text{Concat}(\{B_i\}_1^k),
\end{equation}
where $ \mathbf{A}^{\text{merged}} \in \mathbb{R}^{d^\text{in} \times rk} $ and $ \mathbf{B}^{\text{merged}} \in \mathbb{R}^{rk \times d^\text{out}} $. The $k$-dimension routing layer generates expert weights $ \mathcal{W} \in \mathbb{R}^{\text{token\_num} \times k} $ based on the input hidden state $ x $, and these are expanded to $ \mathbb{R}^{\text{token\_num} \times rk} $ as follows:
\begin{equation}
    \mathcal{W}^\text{expanded} = \alpha k \mathcal{W} / r \otimes \mathbf{1}_r,
\end{equation}
where $ \otimes $ denotes the replication operation.
The overall output of H-LoRA is computed as:
\begin{equation}
    \mathcal{O}^\text{H-LoRA} = (x \mathbf{A}^{\text{merged}} \odot \mathcal{W}^\text{expanded}) \mathbf{B}^{\text{merged}},
\end{equation}
where $ \odot $ represents element-wise multiplication. Finally, the output of H-LoRA is added to the frozen pre-trained weights to produce the final output:
\begin{equation}
    \mathcal{O} = x W_0 + \mathcal{O}^\text{H-LoRA}.
\end{equation}
% In summary, H-LoRA is a task-based dynamic PEFT method that achieves high efficiency in single-task fine-tuning.

\subsection{Training Pipeline}

\begin{figure}[t]
    \centering
    \hspace{-4mm}
    \includegraphics[width=0.94\linewidth]{fig/data.pdf}
    \caption{Data statistics of \texttt{VL-Health}. }
    \label{fig:data}
\end{figure}
\noindent \textbf{1st Stage: Multi-modal Alignment.} 
In the first stage, we design separate visual adapters and H-LoRA submodules for medical unified tasks. For the medical comprehension task, we train abstract-grained visual adapters using high-quality image-text pairs to align visual embeddings with textual embeddings, thereby enabling the model to accurately describe medical visual content. During this process, the pre-trained LLM and its corresponding H-LoRA submodules remain frozen. In contrast, the medical generation task requires training concrete-grained adapters and H-LoRA submodules while keeping the LLM frozen. Meanwhile, we extend the textual vocabulary to include multimodal tokens, enabling the support of additional VQGAN vector quantization indices. The model trains on image-VQ pairs, endowing the pre-trained LLM with the capability for image reconstruction. This design ensures pixel-level consistency of pre- and post-LVLM. The processes establish the initial alignment between the LLM’s outputs and the visual inputs.

\noindent \textbf{2nd Stage: Heterogeneous H-LoRA Plugin Adaptation.}  
The submodules of H-LoRA share the word embedding layer and output head but may encounter issues such as bias and scale inconsistencies during training across different tasks. To ensure that the multiple H-LoRA plugins seamlessly interface with the LLMs and form a unified base, we fine-tune the word embedding layer and output head using a small amount of mixed data to maintain consistency in the model weights. Specifically, during this stage, all H-LoRA submodules for different tasks are kept frozen, with only the word embedding layer and output head being optimized. Through this stage, the model accumulates foundational knowledge for unified tasks by adapting H-LoRA plugins.

\begin{table*}[!t]
\centering
\caption{Comparison of \ourmethod{} with other LVLMs and unified multi-modal models on medical visual comprehension tasks. \textbf{Bold} and \underline{underlined} text indicates the best performance and second-best performance, respectively.}
\resizebox{\textwidth}{!}{
\begin{tabular}{c|lcc|cccccccc|c}
\toprule
\rowcolor[HTML]{E9F3FE} &  &  &  & \multicolumn{2}{c}{\textbf{VQA-RAD \textuparrow}} & \multicolumn{2}{c}{\textbf{SLAKE \textuparrow}} & \multicolumn{2}{c}{\textbf{PathVQA \textuparrow}} &  &  &  \\ 
\cline{5-10}
\rowcolor[HTML]{E9F3FE}\multirow{-2}{*}{\textbf{Type}} & \multirow{-2}{*}{\textbf{Model}} & \multirow{-2}{*}{\textbf{\# Params}} & \multirow{-2}{*}{\makecell{\textbf{Medical} \\ \textbf{LVLM}}} & \textbf{close} & \textbf{all} & \textbf{close} & \textbf{all} & \textbf{close} & \textbf{all} & \multirow{-2}{*}{\makecell{\textbf{MMMU} \\ \textbf{-Med}}\textuparrow} & \multirow{-2}{*}{\textbf{OMVQA}\textuparrow} & \multirow{-2}{*}{\textbf{Avg. \textuparrow}} \\ 
\midrule \midrule
\multirow{9}{*}{\textbf{Comp. Only}} 
& Med-Flamingo & 8.3B & \Large \ding{51} & 58.6 & 43.0 & 47.0 & 25.5 & 61.9 & 31.3 & 28.7 & 34.9 & 41.4 \\
& LLaVA-Med & 7B & \Large \ding{51} & 60.2 & 48.1 & 58.4 & 44.8 & 62.3 & 35.7 & 30.0 & 41.3 & 47.6 \\
& HuatuoGPT-Vision & 7B & \Large \ding{51} & 66.9 & 53.0 & 59.8 & 49.1 & 52.9 & 32.0 & 42.0 & 50.0 & 50.7 \\
& BLIP-2 & 6.7B & \Large \ding{55} & 43.4 & 36.8 & 41.6 & 35.3 & 48.5 & 28.8 & 27.3 & 26.9 & 36.1 \\
& LLaVA-v1.5 & 7B & \Large \ding{55} & 51.8 & 42.8 & 37.1 & 37.7 & 53.5 & 31.4 & 32.7 & 44.7 & 41.5 \\
& InstructBLIP & 7B & \Large \ding{55} & 61.0 & 44.8 & 66.8 & 43.3 & 56.0 & 32.3 & 25.3 & 29.0 & 44.8 \\
& Yi-VL & 6B & \Large \ding{55} & 52.6 & 42.1 & 52.4 & 38.4 & 54.9 & 30.9 & 38.0 & 50.2 & 44.9 \\
& InternVL2 & 8B & \Large \ding{55} & 64.9 & 49.0 & 66.6 & 50.1 & 60.0 & 31.9 & \underline{43.3} & 54.5 & 52.5\\
& Llama-3.2 & 11B & \Large \ding{55} & 68.9 & 45.5 & 72.4 & 52.1 & 62.8 & 33.6 & 39.3 & 63.2 & 54.7 \\
\midrule
\multirow{5}{*}{\textbf{Comp. \& Gen.}} 
& Show-o & 1.3B & \Large \ding{55} & 50.6 & 33.9 & 31.5 & 17.9 & 52.9 & 28.2 & 22.7 & 45.7 & 42.6 \\
& Unified-IO 2 & 7B & \Large \ding{55} & 46.2 & 32.6 & 35.9 & 21.9 & 52.5 & 27.0 & 25.3 & 33.0 & 33.8 \\
& Janus & 1.3B & \Large \ding{55} & 70.9 & 52.8 & 34.7 & 26.9 & 51.9 & 27.9 & 30.0 & 26.8 & 33.5 \\
& \cellcolor[HTML]{DAE0FB}HealthGPT-M3 & \cellcolor[HTML]{DAE0FB}3.8B & \cellcolor[HTML]{DAE0FB}\Large \ding{51} & \cellcolor[HTML]{DAE0FB}\underline{73.7} & \cellcolor[HTML]{DAE0FB}\underline{55.9} & \cellcolor[HTML]{DAE0FB}\underline{74.6} & \cellcolor[HTML]{DAE0FB}\underline{56.4} & \cellcolor[HTML]{DAE0FB}\underline{78.7} & \cellcolor[HTML]{DAE0FB}\underline{39.7} & \cellcolor[HTML]{DAE0FB}\underline{43.3} & \cellcolor[HTML]{DAE0FB}\underline{68.5} & \cellcolor[HTML]{DAE0FB}\underline{61.3} \\
& \cellcolor[HTML]{DAE0FB}HealthGPT-L14 & \cellcolor[HTML]{DAE0FB}14B & \cellcolor[HTML]{DAE0FB}\Large \ding{51} & \cellcolor[HTML]{DAE0FB}\textbf{77.7} & \cellcolor[HTML]{DAE0FB}\textbf{58.3} & \cellcolor[HTML]{DAE0FB}\textbf{76.4} & \cellcolor[HTML]{DAE0FB}\textbf{64.5} & \cellcolor[HTML]{DAE0FB}\textbf{85.9} & \cellcolor[HTML]{DAE0FB}\textbf{44.4} & \cellcolor[HTML]{DAE0FB}\textbf{49.2} & \cellcolor[HTML]{DAE0FB}\textbf{74.4} & \cellcolor[HTML]{DAE0FB}\textbf{66.4} \\
\bottomrule
\end{tabular}
}
\label{tab:results}
\end{table*}
\begin{table*}[ht]
    \centering
    \caption{The experimental results for the four modality conversion tasks.}
    \resizebox{\textwidth}{!}{
    \begin{tabular}{l|ccc|ccc|ccc|ccc}
        \toprule
        \rowcolor[HTML]{E9F3FE} & \multicolumn{3}{c}{\textbf{CT to MRI (Brain)}} & \multicolumn{3}{c}{\textbf{CT to MRI (Pelvis)}} & \multicolumn{3}{c}{\textbf{MRI to CT (Brain)}} & \multicolumn{3}{c}{\textbf{MRI to CT (Pelvis)}} \\
        \cline{2-13}
        \rowcolor[HTML]{E9F3FE}\multirow{-2}{*}{\textbf{Model}}& \textbf{SSIM $\uparrow$} & \textbf{PSNR $\uparrow$} & \textbf{MSE $\downarrow$} & \textbf{SSIM $\uparrow$} & \textbf{PSNR $\uparrow$} & \textbf{MSE $\downarrow$} & \textbf{SSIM $\uparrow$} & \textbf{PSNR $\uparrow$} & \textbf{MSE $\downarrow$} & \textbf{SSIM $\uparrow$} & \textbf{PSNR $\uparrow$} & \textbf{MSE $\downarrow$} \\
        \midrule \midrule
        pix2pix & 71.09 & 32.65 & 36.85 & 59.17 & 31.02 & 51.91 & 78.79 & 33.85 & 28.33 & 72.31 & 32.98 & 36.19 \\
        CycleGAN & 54.76 & 32.23 & 40.56 & 54.54 & 30.77 & 55.00 & 63.75 & 31.02 & 52.78 & 50.54 & 29.89 & 67.78 \\
        BBDM & {71.69} & {32.91} & {34.44} & 57.37 & 31.37 & 48.06 & \textbf{86.40} & 34.12 & 26.61 & {79.26} & 33.15 & 33.60 \\
        Vmanba & 69.54 & 32.67 & 36.42 & {63.01} & {31.47} & {46.99} & 79.63 & 34.12 & 26.49 & 77.45 & 33.53 & 31.85 \\
        DiffMa & 71.47 & 32.74 & 35.77 & 62.56 & 31.43 & 47.38 & 79.00 & {34.13} & {26.45} & 78.53 & {33.68} & {30.51} \\
        \rowcolor[HTML]{DAE0FB}HealthGPT-M3 & \underline{79.38} & \underline{33.03} & \underline{33.48} & \underline{71.81} & \underline{31.83} & \underline{43.45} & {85.06} & \textbf{34.40} & \textbf{25.49} & \underline{84.23} & \textbf{34.29} & \textbf{27.99} \\
        \rowcolor[HTML]{DAE0FB}HealthGPT-L14 & \textbf{79.73} & \textbf{33.10} & \textbf{32.96} & \textbf{71.92} & \textbf{31.87} & \textbf{43.09} & \underline{85.31} & \underline{34.29} & \underline{26.20} & \textbf{84.96} & \underline{34.14} & \underline{28.13} \\
        \bottomrule
    \end{tabular}
    }
    \label{tab:conversion}
\end{table*}

\noindent \textbf{3rd Stage: Visual Instruction Fine-Tuning.}  
In the third stage, we introduce additional task-specific data to further optimize the model and enhance its adaptability to downstream tasks such as medical visual comprehension (e.g., medical QA, medical dialogues, and report generation) or generation tasks (e.g., super-resolution, denoising, and modality conversion). Notably, by this stage, the word embedding layer and output head have been fine-tuned, only the H-LoRA modules and adapter modules need to be trained. This strategy significantly improves the model's adaptability and flexibility across different tasks.


\section{Experiment}
\label{s:experiment}

\subsection{Data Description}
We evaluate our method on FI~\cite{you2016building}, Twitter\_LDL~\cite{yang2017learning} and Artphoto~\cite{machajdik2010affective}.
FI is a public dataset built from Flickr and Instagram, with 23,308 images and eight emotion categories, namely \textit{amusement}, \textit{anger}, \textit{awe},  \textit{contentment}, \textit{disgust}, \textit{excitement},  \textit{fear}, and \textit{sadness}. 
% Since images in FI are all copyrighted by law, some images are corrupted now, so we remove these samples and retain 21,828 images.
% T4SA contains images from Twitter, which are classified into three categories: \textit{positive}, \textit{neutral}, and \textit{negative}. In this paper, we adopt the base version of B-T4SA, which contains 470,586 images and provides text descriptions of the corresponding tweets.
Twitter\_LDL contains 10,045 images from Twitter, with the same eight categories as the FI dataset.
% 。
For these two datasets, they are randomly split into 80\%
training and 20\% testing set.
Artphoto contains 806 artistic photos from the DeviantArt website, which we use to further evaluate the zero-shot capability of our model.
% on the small-scale dataset.
% We construct and publicly release the first image sentiment analysis dataset containing metadata.
% 。

% Based on these datasets, we are the first to construct and publicly release metadata-enhanced image sentiment analysis datasets. These datasets include scenes, tags, descriptions, and corresponding confidence scores, and are available at this link for future research purposes.


% 
\begin{table}[t]
\centering
% \begin{center}
\caption{Overall performance of different models on FI and Twitter\_LDL datasets.}
\label{tab:cap1}
% \resizebox{\linewidth}{!}
{
\begin{tabular}{l|c|c|c|c}
\hline
\multirow{2}{*}{\textbf{Model}} & \multicolumn{2}{c|}{\textbf{FI}}  & \multicolumn{2}{c}{\textbf{Twitter\_LDL}} \\ \cline{2-5} 
  & \textbf{Accuracy} & \textbf{F1} & \textbf{Accuracy} & \textbf{F1}  \\ \hline
% (\rownumber)~AlexNet~\cite{krizhevsky2017imagenet}  & 58.13\% & 56.35\%  & 56.24\%& 55.02\%  \\ 
% (\rownumber)~VGG16~\cite{simonyan2014very}  & 63.75\%& 63.08\%  & 59.34\%& 59.02\%  \\ 
(\rownumber)~ResNet101~\cite{he2016deep} & 66.16\%& 65.56\%  & 62.02\% & 61.34\%  \\ 
(\rownumber)~CDA~\cite{han2023boosting} & 66.71\%& 65.37\%  & 64.14\% & 62.85\%  \\ 
(\rownumber)~CECCN~\cite{ruan2024color} & 67.96\%& 66.74\%  & 64.59\%& 64.72\% \\ 
(\rownumber)~EmoVIT~\cite{xie2024emovit} & 68.09\%& 67.45\%  & 63.12\% & 61.97\%  \\ 
(\rownumber)~ComLDL~\cite{zhang2022compound} & 68.83\%& 67.28\%  & 65.29\% & 63.12\%  \\ 
(\rownumber)~WSDEN~\cite{li2023weakly} & 69.78\%& 69.61\%  & 67.04\% & 65.49\% \\ 
(\rownumber)~ECWA~\cite{deng2021emotion} & 70.87\%& 69.08\%  & 67.81\% & 66.87\%  \\ 
(\rownumber)~EECon~\cite{yang2023exploiting} & 71.13\%& 68.34\%  & 64.27\%& 63.16\%  \\ 
(\rownumber)~MAM~\cite{zhang2024affective} & 71.44\%  & 70.83\% & 67.18\%  & 65.01\%\\ 
(\rownumber)~TGCA-PVT~\cite{chen2024tgca}   & 73.05\%  & 71.46\% & 69.87\%  & 68.32\% \\ 
(\rownumber)~OEAN~\cite{zhang2024object}   & 73.40\%  & 72.63\% & 70.52\%  & 69.47\% \\ \hline
(\rownumber)~\shortname  & \textbf{79.48\%} & \textbf{79.22\%} & \textbf{74.12\%} & \textbf{73.09\%} \\ \hline
\end{tabular}
}
\vspace{-6mm}
% \end{center}
\end{table}
% 

\subsection{Experiment Setting}
% \subsubsection{Model Setting.}
% 
\textbf{Model Setting:}
For feature representation, we set $k=10$ to select object tags, and adopt clip-vit-base-patch32 as the pre-trained model for unified feature representation.
Moreover, we empirically set $(d_e, d_h, d_k, d_s) = (512, 128, 16, 64)$, and set the classification class $L$ to 8.

% 

\textbf{Training Setting:}
To initialize the model, we set all weights such as $\boldsymbol{W}$ following the truncated normal distribution, and use AdamW optimizer with the learning rate of $1 \times 10^{-4}$.
% warmup scheduler of cosine, warmup steps of 2000.
Furthermore, we set the batch size to 32 and the epoch of the training process to 200.
During the implementation, we utilize \textit{PyTorch} to build our entire model.
% , and our project codes are publicly available at https://github.com/zzmyrep/MESN.
% Our project codes as well as data are all publicly available on GitHub\footnote{https://github.com/zzmyrep/KBCEN}.
% Code is available at \href{https://github.com/zzmyrep/KBCEN}{https://github.com/zzmyrep/KBCEN}.

\textbf{Evaluation Metrics:}
Following~\cite{zhang2024affective, chen2024tgca, zhang2024object}, we adopt \textit{accuracy} and \textit{F1} as our evaluation metrics to measure the performance of different methods for image sentiment analysis. 



\subsection{Experiment Result}
% We compare our model against the following baselines: AlexNet~\cite{krizhevsky2017imagenet}, VGG16~\cite{simonyan2014very}, ResNet101~\cite{he2016deep}, CECCN~\cite{ruan2024color}, EmoVIT~\cite{xie2024emovit}, WSCNet~\cite{yang2018weakly}, ECWA~\cite{deng2021emotion}, EECon~\cite{yang2023exploiting}, MAM~\cite{zhang2024affective} and TGCA-PVT~\cite{chen2024tgca}, and the overall results are summarized in Table~\ref{tab:cap1}.
We compare our model against several baselines, and the overall results are summarized in Table~\ref{tab:cap1}.
We observe that our model achieves the best performance in both accuracy and F1 metrics, significantly outperforming the previous models. 
This superior performance is mainly attributed to our effective utilization of metadata to enhance image sentiment analysis, as well as the exceptional capability of the unified sentiment transformer framework we developed. These results strongly demonstrate that our proposed method can bring encouraging performance for image sentiment analysis.

\setcounter{magicrownumbers}{0} 
\begin{table}[t]
\begin{center}
\caption{Ablation study of~\shortname~on FI dataset.} 
% \vspace{1mm}
\label{tab:cap2}
\resizebox{.9\linewidth}{!}
{
\begin{tabular}{lcc}
  \hline
  \textbf{Model} & \textbf{Accuracy} & \textbf{F1} \\
  \hline
  (\rownumber)~Ours (w/o vision) & 65.72\% & 64.54\% \\
  (\rownumber)~Ours (w/o text description) & 74.05\% & 72.58\% \\
  (\rownumber)~Ours (w/o object tag) & 77.45\% & 76.84\% \\
  (\rownumber)~Ours (w/o scene tag) & 78.47\% & 78.21\% \\
  \hline
  (\rownumber)~Ours (w/o unified embedding) & 76.41\% & 76.23\% \\
  (\rownumber)~Ours (w/o adaptive learning) & 76.83\% & 76.56\% \\
  (\rownumber)~Ours (w/o cross-modal fusion) & 76.85\% & 76.49\% \\
  \hline
  (\rownumber)~Ours  & \textbf{79.48\%} & \textbf{79.22\%} \\
  \hline
\end{tabular}
}
\end{center}
\vspace{-5mm}
\end{table}


\begin{figure}[t]
\centering
% \vspace{-2mm}
\includegraphics[width=0.42\textwidth]{fig/2dvisual-linux4-paper2.pdf}
\caption{Visualization of feature distribution on eight categories before (left) and after (right) model processing.}
% 
\label{fig:visualization}
\vspace{-5mm}
\end{figure}

\subsection{Ablation Performance}
In this subsection, we conduct an ablation study to examine which component is really important for performance improvement. The results are reported in Table~\ref{tab:cap2}.

For information utilization, we observe a significant decline in model performance when visual features are removed. Additionally, the performance of \shortname~decreases when different metadata are removed separately, which means that text description, object tag, and scene tag are all critical for image sentiment analysis.
Recalling the model architecture, we separately remove transformer layers of the unified representation module, the adaptive learning module, and the cross-modal fusion module, replacing them with MLPs of the same parameter scale.
In this way, we can observe varying degrees of decline in model performance, indicating that these modules are indispensable for our model to achieve better performance.

\subsection{Visualization}
% 


% % 开始使用minipage进行左右排列
% \begin{minipage}[t]{0.45\textwidth}  % 子图1宽度为45%
%     \centering
%     \includegraphics[width=\textwidth]{2dvisual.pdf}  % 插入图片
%     \captionof{figure}{Visualization of feature distribution.}  % 使用captionof添加图片标题
%     \label{fig:visualization}
% \end{minipage}


% \begin{figure}[t]
% \centering
% \vspace{-2mm}
% \includegraphics[width=0.45\textwidth]{fig/2dvisual.pdf}
% \caption{Visualization of feature distribution.}
% \label{fig:visualization}
% % \vspace{-4mm}
% \end{figure}

% \begin{figure}[t]
% \centering
% \vspace{-2mm}
% \includegraphics[width=0.45\textwidth]{fig/2dvisual-linux3-paper.pdf}
% \caption{Visualization of feature distribution.}
% \label{fig:visualization}
% % \vspace{-4mm}
% \end{figure}



\begin{figure}[tbp]   
\vspace{-4mm}
  \centering            
  \subfloat[Depth of adaptive learning layers]   
  {
    \label{fig:subfig1}\includegraphics[width=0.22\textwidth]{fig/fig_sensitivity-a5}
  }
  \subfloat[Depth of fusion layers]
  {
    % \label{fig:subfig2}\includegraphics[width=0.22\textwidth]{fig/fig_sensitivity-b2}
    \label{fig:subfig2}\includegraphics[width=0.22\textwidth]{fig/fig_sensitivity-b2-num.pdf}
  }
  \caption{Sensitivity study of \shortname~on different depth. }   
  \label{fig:fig_sensitivity}  
\vspace{-2mm}
\end{figure}

% \begin{figure}[htbp]
% \centerline{\includegraphics{2dvisual.pdf}}
% \caption{Visualization of feature distribution.}
% \label{fig:visualization}
% \end{figure}

% In Fig.~\ref{fig:visualization}, we use t-SNE~\cite{van2008visualizing} to reduce the dimension of data features for visualization, Figure in left represents the metadata features before model processing, the features are obtained by embedding through the CLIP model, and figure in right shows the features of the data after model processing, it can be observed that after the model processing, the data with different label categories fall in different regions in the space, therefore, we can conclude that the Therefore, we can conclude that the model can effectively utilize the information contained in the metadata and use it to guide the model for classification.

In Fig.~\ref{fig:visualization}, we use t-SNE~\cite{van2008visualizing} to reduce the dimension of data features for visualization.
The left figure shows metadata features before being processed by our model (\textit{i.e.}, embedded by CLIP), while the right shows the distribution of features after being processed by our model.
We can observe that after the model processing, data with the same label are closer to each other, while others are farther away.
Therefore, it shows that the model can effectively utilize the information contained in the metadata and use it to guide the classification process.

\subsection{Sensitivity Analysis}
% 
In this subsection, we conduct a sensitivity analysis to figure out the effect of different depth settings of adaptive learning layers and fusion layers. 
% In this subsection, we conduct a sensitivity analysis to figure out the effect of different depth settings on the model. 
% Fig.~\ref{fig:fig_sensitivity} presents the effect of different depth settings of adaptive learning layers and fusion layers. 
Taking Fig.~\ref{fig:fig_sensitivity} (a) as an example, the model performance improves with increasing depth, reaching the best performance at a depth of 4.
% Taking Fig.~\ref{fig:fig_sensitivity} (a) as an example, the performance of \shortname~improves with the increase of depth at first, reaching the best performance at a depth of 4.
When the depth continues to increase, the accuracy decreases to varying degrees.
Similar results can be observed in Fig.~\ref{fig:fig_sensitivity} (b).
Therefore, we set their depths to 4 and 6 respectively to achieve the best results.

% Through our experiments, we can observe that the effect of modifying these hyperparameters on the results of the experiments is very weak, and the surface model is not sensitive to the hyperparameters.


\subsection{Zero-shot Capability}
% 

% (1)~GCH~\cite{2010Analyzing} & 21.78\% & (5)~RA-DLNet~\cite{2020A} & 34.01\% \\ \hline
% (2)~WSCNet~\cite{2019WSCNet}  & 30.25\% & (6)~CECCN~\cite{ruan2024color} & 43.83\% \\ \hline
% (3)~PCNN~\cite{2015Robust} & 31.68\%  & (7)~EmoVIT~\cite{xie2024emovit} & 44.90\% \\ \hline
% (4)~AR~\cite{2018Visual} & 32.67\% & (8)~Ours (Zero-shot) & 47.83\% \\ \hline


\begin{table}[t]
\centering
\caption{Zero-shot capability of \shortname.}
\label{tab:cap3}
\resizebox{1\linewidth}{!}
{
\begin{tabular}{lc|lc}
\hline
\textbf{Model} & \textbf{Accuracy} & \textbf{Model} & \textbf{Accuracy} \\ \hline
(1)~WSCNet~\cite{2019WSCNet}  & 30.25\% & (5)~MAM~\cite{zhang2024affective} & 39.56\%  \\ \hline
(2)~AR~\cite{2018Visual} & 32.67\% & (6)~CECCN~\cite{ruan2024color} & 43.83\% \\ \hline
(3)~RA-DLNet~\cite{2020A} & 34.01\%  & (7)~EmoVIT~\cite{xie2024emovit} & 44.90\% \\ \hline
(4)~CDA~\cite{han2023boosting} & 38.64\% & (8)~Ours (Zero-shot) & 47.83\% \\ \hline
\end{tabular}
}
\vspace{-5mm}
\end{table}

% We use the model trained on the FI dataset to test on the artphoto dataset to verify the model's generalization ability as well as robustness to other distributed datasets.
% We can observe that the MESN model shows strong competitiveness in terms of accuracy when compared to other trained models, which suggests that the model has a good generalization ability in the OOD task.

To validate the model's generalization ability and robustness to other distributed datasets, we directly test the model trained on the FI dataset, without training on Artphoto. 
% As observed in Table 3, compared to other models trained on Artphoto, we achieve highly competitive zero-shot performance, indicating that the model has good generalization ability in out-of-distribution tasks.
From Table~\ref{tab:cap3}, we can observe that compared with other models trained on Artphoto, we achieve competitive zero-shot performance, which shows that the model has good generalization ability in out-of-distribution tasks.


%%%%%%%%%%%%
%  E2E     %
%%%%%%%%%%%%


\section{Conclusion}
In this paper, we introduced Wi-Chat, the first LLM-powered Wi-Fi-based human activity recognition system that integrates the reasoning capabilities of large language models with the sensing potential of wireless signals. Our experimental results on a self-collected Wi-Fi CSI dataset demonstrate the promising potential of LLMs in enabling zero-shot Wi-Fi sensing. These findings suggest a new paradigm for human activity recognition that does not rely on extensive labeled data. We hope future research will build upon this direction, further exploring the applications of LLMs in signal processing domains such as IoT, mobile sensing, and radar-based systems.

\section*{Limitations}
While our work represents the first attempt to leverage LLMs for processing Wi-Fi signals, it is a preliminary study focused on a relatively simple task: Wi-Fi-based human activity recognition. This choice allows us to explore the feasibility of LLMs in wireless sensing but also comes with certain limitations.

Our approach primarily evaluates zero-shot performance, which, while promising, may still lag behind traditional supervised learning methods in highly complex or fine-grained recognition tasks. Besides, our study is limited to a controlled environment with a self-collected dataset, and the generalizability of LLMs to diverse real-world scenarios with varying Wi-Fi conditions, environmental interference, and device heterogeneity remains an open question.

Additionally, we have yet to explore the full potential of LLMs in more advanced Wi-Fi sensing applications, such as fine-grained gesture recognition, occupancy detection, and passive health monitoring. Future work should investigate the scalability of LLM-based approaches, their robustness to domain shifts, and their integration with multimodal sensing techniques in broader IoT applications.


% Bibliography entries for the entire Anthology, followed by custom entries
%\bibliography{anthology,custom}
% Custom bibliography entries only
\bibliography{main}
\newpage
\appendix

\section{Experiment prompts}
\label{sec:prompt}
The prompts used in the LLM experiments are shown in the following Table~\ref{tab:prompts}.

\definecolor{titlecolor}{rgb}{0.9, 0.5, 0.1}
\definecolor{anscolor}{rgb}{0.2, 0.5, 0.8}
\definecolor{labelcolor}{HTML}{48a07e}
\begin{table*}[h]
	\centering
	
 % \vspace{-0.2cm}
	
	\begin{center}
		\begin{tikzpicture}[
				chatbox_inner/.style={rectangle, rounded corners, opacity=0, text opacity=1, font=\sffamily\scriptsize, text width=5in, text height=9pt, inner xsep=6pt, inner ysep=6pt},
				chatbox_prompt_inner/.style={chatbox_inner, align=flush left, xshift=0pt, text height=11pt},
				chatbox_user_inner/.style={chatbox_inner, align=flush left, xshift=0pt},
				chatbox_gpt_inner/.style={chatbox_inner, align=flush left, xshift=0pt},
				chatbox/.style={chatbox_inner, draw=black!25, fill=gray!7, opacity=1, text opacity=0},
				chatbox_prompt/.style={chatbox, align=flush left, fill=gray!1.5, draw=black!30, text height=10pt},
				chatbox_user/.style={chatbox, align=flush left},
				chatbox_gpt/.style={chatbox, align=flush left},
				chatbox2/.style={chatbox_gpt, fill=green!25},
				chatbox3/.style={chatbox_gpt, fill=red!20, draw=black!20},
				chatbox4/.style={chatbox_gpt, fill=yellow!30},
				labelbox/.style={rectangle, rounded corners, draw=black!50, font=\sffamily\scriptsize\bfseries, fill=gray!5, inner sep=3pt},
			]
											
			\node[chatbox_user] (q1) {
				\textbf{System prompt}
				\newline
				\newline
				You are a helpful and precise assistant for segmenting and labeling sentences. We would like to request your help on curating a dataset for entity-level hallucination detection.
				\newline \newline
                We will give you a machine generated biography and a list of checked facts about the biography. Each fact consists of a sentence and a label (True/False). Please do the following process. First, breaking down the biography into words. Second, by referring to the provided list of facts, merging some broken down words in the previous step to form meaningful entities. For example, ``strategic thinking'' should be one entity instead of two. Third, according to the labels in the list of facts, labeling each entity as True or False. Specifically, for facts that share a similar sentence structure (\eg, \textit{``He was born on Mach 9, 1941.''} (\texttt{True}) and \textit{``He was born in Ramos Mejia.''} (\texttt{False})), please first assign labels to entities that differ across atomic facts. For example, first labeling ``Mach 9, 1941'' (\texttt{True}) and ``Ramos Mejia'' (\texttt{False}) in the above case. For those entities that are the same across atomic facts (\eg, ``was born'') or are neutral (\eg, ``he,'' ``in,'' and ``on''), please label them as \texttt{True}. For the cases that there is no atomic fact that shares the same sentence structure, please identify the most informative entities in the sentence and label them with the same label as the atomic fact while treating the rest of the entities as \texttt{True}. In the end, output the entities and labels in the following format:
                \begin{itemize}[nosep]
                    \item Entity 1 (Label 1)
                    \item Entity 2 (Label 2)
                    \item ...
                    \item Entity N (Label N)
                \end{itemize}
                % \newline \newline
                Here are two examples:
                \newline\newline
                \textbf{[Example 1]}
                \newline
                [The start of the biography]
                \newline
                \textcolor{titlecolor}{Marianne McAndrew is an American actress and singer, born on November 21, 1942, in Cleveland, Ohio. She began her acting career in the late 1960s, appearing in various television shows and films.}
                \newline
                [The end of the biography]
                \newline \newline
                [The start of the list of checked facts]
                \newline
                \textcolor{anscolor}{[Marianne McAndrew is an American. (False); Marianne McAndrew is an actress. (True); Marianne McAndrew is a singer. (False); Marianne McAndrew was born on November 21, 1942. (False); Marianne McAndrew was born in Cleveland, Ohio. (False); She began her acting career in the late 1960s. (True); She has appeared in various television shows. (True); She has appeared in various films. (True)]}
                \newline
                [The end of the list of checked facts]
                \newline \newline
                [The start of the ideal output]
                \newline
                \textcolor{labelcolor}{[Marianne McAndrew (True); is (True); an (True); American (False); actress (True); and (True); singer (False); , (True); born (True); on (True); November 21, 1942 (False); , (True); in (True); Cleveland, Ohio (False); . (True); She (True); began (True); her (True); acting career (True); in (True); the late 1960s (True); , (True); appearing (True); in (True); various (True); television shows (True); and (True); films (True); . (True)]}
                \newline
                [The end of the ideal output]
				\newline \newline
                \textbf{[Example 2]}
                \newline
                [The start of the biography]
                \newline
                \textcolor{titlecolor}{Doug Sheehan is an American actor who was born on April 27, 1949, in Santa Monica, California. He is best known for his roles in soap operas, including his portrayal of Joe Kelly on ``General Hospital'' and Ben Gibson on ``Knots Landing.''}
                \newline
                [The end of the biography]
                \newline \newline
                [The start of the list of checked facts]
                \newline
                \textcolor{anscolor}{[Doug Sheehan is an American. (True); Doug Sheehan is an actor. (True); Doug Sheehan was born on April 27, 1949. (True); Doug Sheehan was born in Santa Monica, California. (False); He is best known for his roles in soap operas. (True); He portrayed Joe Kelly. (True); Joe Kelly was in General Hospital. (True); General Hospital is a soap opera. (True); He portrayed Ben Gibson. (True); Ben Gibson was in Knots Landing. (True); Knots Landing is a soap opera. (True)]}
                \newline
                [The end of the list of checked facts]
                \newline \newline
                [The start of the ideal output]
                \newline
                \textcolor{labelcolor}{[Doug Sheehan (True); is (True); an (True); American (True); actor (True); who (True); was born (True); on (True); April 27, 1949 (True); in (True); Santa Monica, California (False); . (True); He (True); is (True); best known (True); for (True); his roles in soap operas (True); , (True); including (True); in (True); his portrayal (True); of (True); Joe Kelly (True); on (True); ``General Hospital'' (True); and (True); Ben Gibson (True); on (True); ``Knots Landing.'' (True)]}
                \newline
                [The end of the ideal output]
				\newline \newline
				\textbf{User prompt}
				\newline
				\newline
				[The start of the biography]
				\newline
				\textcolor{magenta}{\texttt{\{BIOGRAPHY\}}}
				\newline
				[The ebd of the biography]
				\newline \newline
				[The start of the list of checked facts]
				\newline
				\textcolor{magenta}{\texttt{\{LIST OF CHECKED FACTS\}}}
				\newline
				[The end of the list of checked facts]
			};
			\node[chatbox_user_inner] (q1_text) at (q1) {
				\textbf{System prompt}
				\newline
				\newline
				You are a helpful and precise assistant for segmenting and labeling sentences. We would like to request your help on curating a dataset for entity-level hallucination detection.
				\newline \newline
                We will give you a machine generated biography and a list of checked facts about the biography. Each fact consists of a sentence and a label (True/False). Please do the following process. First, breaking down the biography into words. Second, by referring to the provided list of facts, merging some broken down words in the previous step to form meaningful entities. For example, ``strategic thinking'' should be one entity instead of two. Third, according to the labels in the list of facts, labeling each entity as True or False. Specifically, for facts that share a similar sentence structure (\eg, \textit{``He was born on Mach 9, 1941.''} (\texttt{True}) and \textit{``He was born in Ramos Mejia.''} (\texttt{False})), please first assign labels to entities that differ across atomic facts. For example, first labeling ``Mach 9, 1941'' (\texttt{True}) and ``Ramos Mejia'' (\texttt{False}) in the above case. For those entities that are the same across atomic facts (\eg, ``was born'') or are neutral (\eg, ``he,'' ``in,'' and ``on''), please label them as \texttt{True}. For the cases that there is no atomic fact that shares the same sentence structure, please identify the most informative entities in the sentence and label them with the same label as the atomic fact while treating the rest of the entities as \texttt{True}. In the end, output the entities and labels in the following format:
                \begin{itemize}[nosep]
                    \item Entity 1 (Label 1)
                    \item Entity 2 (Label 2)
                    \item ...
                    \item Entity N (Label N)
                \end{itemize}
                % \newline \newline
                Here are two examples:
                \newline\newline
                \textbf{[Example 1]}
                \newline
                [The start of the biography]
                \newline
                \textcolor{titlecolor}{Marianne McAndrew is an American actress and singer, born on November 21, 1942, in Cleveland, Ohio. She began her acting career in the late 1960s, appearing in various television shows and films.}
                \newline
                [The end of the biography]
                \newline \newline
                [The start of the list of checked facts]
                \newline
                \textcolor{anscolor}{[Marianne McAndrew is an American. (False); Marianne McAndrew is an actress. (True); Marianne McAndrew is a singer. (False); Marianne McAndrew was born on November 21, 1942. (False); Marianne McAndrew was born in Cleveland, Ohio. (False); She began her acting career in the late 1960s. (True); She has appeared in various television shows. (True); She has appeared in various films. (True)]}
                \newline
                [The end of the list of checked facts]
                \newline \newline
                [The start of the ideal output]
                \newline
                \textcolor{labelcolor}{[Marianne McAndrew (True); is (True); an (True); American (False); actress (True); and (True); singer (False); , (True); born (True); on (True); November 21, 1942 (False); , (True); in (True); Cleveland, Ohio (False); . (True); She (True); began (True); her (True); acting career (True); in (True); the late 1960s (True); , (True); appearing (True); in (True); various (True); television shows (True); and (True); films (True); . (True)]}
                \newline
                [The end of the ideal output]
				\newline \newline
                \textbf{[Example 2]}
                \newline
                [The start of the biography]
                \newline
                \textcolor{titlecolor}{Doug Sheehan is an American actor who was born on April 27, 1949, in Santa Monica, California. He is best known for his roles in soap operas, including his portrayal of Joe Kelly on ``General Hospital'' and Ben Gibson on ``Knots Landing.''}
                \newline
                [The end of the biography]
                \newline \newline
                [The start of the list of checked facts]
                \newline
                \textcolor{anscolor}{[Doug Sheehan is an American. (True); Doug Sheehan is an actor. (True); Doug Sheehan was born on April 27, 1949. (True); Doug Sheehan was born in Santa Monica, California. (False); He is best known for his roles in soap operas. (True); He portrayed Joe Kelly. (True); Joe Kelly was in General Hospital. (True); General Hospital is a soap opera. (True); He portrayed Ben Gibson. (True); Ben Gibson was in Knots Landing. (True); Knots Landing is a soap opera. (True)]}
                \newline
                [The end of the list of checked facts]
                \newline \newline
                [The start of the ideal output]
                \newline
                \textcolor{labelcolor}{[Doug Sheehan (True); is (True); an (True); American (True); actor (True); who (True); was born (True); on (True); April 27, 1949 (True); in (True); Santa Monica, California (False); . (True); He (True); is (True); best known (True); for (True); his roles in soap operas (True); , (True); including (True); in (True); his portrayal (True); of (True); Joe Kelly (True); on (True); ``General Hospital'' (True); and (True); Ben Gibson (True); on (True); ``Knots Landing.'' (True)]}
                \newline
                [The end of the ideal output]
				\newline \newline
				\textbf{User prompt}
				\newline
				\newline
				[The start of the biography]
				\newline
				\textcolor{magenta}{\texttt{\{BIOGRAPHY\}}}
				\newline
				[The ebd of the biography]
				\newline \newline
				[The start of the list of checked facts]
				\newline
				\textcolor{magenta}{\texttt{\{LIST OF CHECKED FACTS\}}}
				\newline
				[The end of the list of checked facts]
			};
		\end{tikzpicture}
        \caption{GPT-4o prompt for labeling hallucinated entities.}\label{tb:gpt-4-prompt}
	\end{center}
\vspace{-0cm}
\end{table*}
% \section{Full Experiment Results}
% \begin{table*}[th]
    \centering
    \small
    \caption{Classification Results}
    \begin{tabular}{lcccc}
        \toprule
        \textbf{Method} & \textbf{Accuracy} & \textbf{Precision} & \textbf{Recall} & \textbf{F1-score} \\
        \midrule
        \multicolumn{5}{c}{\textbf{Zero Shot}} \\
                Zero-shot E-eyes & 0.26 & 0.26 & 0.27 & 0.26 \\
        Zero-shot CARM & 0.24 & 0.24 & 0.24 & 0.24 \\
                Zero-shot SVM & 0.27 & 0.28 & 0.28 & 0.27 \\
        Zero-shot CNN & 0.23 & 0.24 & 0.23 & 0.23 \\
        Zero-shot RNN & 0.26 & 0.26 & 0.26 & 0.26 \\
DeepSeek-0shot & 0.54 & 0.61 & 0.54 & 0.52 \\
DeepSeek-0shot-COT & 0.33 & 0.24 & 0.33 & 0.23 \\
DeepSeek-0shot-Knowledge & 0.45 & 0.46 & 0.45 & 0.44 \\
Gemma2-0shot & 0.35 & 0.22 & 0.38 & 0.27 \\
Gemma2-0shot-COT & 0.36 & 0.22 & 0.36 & 0.27 \\
Gemma2-0shot-Knowledge & 0.32 & 0.18 & 0.34 & 0.20 \\
GPT-4o-mini-0shot & 0.48 & 0.53 & 0.48 & 0.41 \\
GPT-4o-mini-0shot-COT & 0.33 & 0.50 & 0.33 & 0.38 \\
GPT-4o-mini-0shot-Knowledge & 0.49 & 0.31 & 0.49 & 0.36 \\
GPT-4o-0shot & 0.62 & 0.62 & 0.47 & 0.42 \\
GPT-4o-0shot-COT & 0.29 & 0.45 & 0.29 & 0.21 \\
GPT-4o-0shot-Knowledge & 0.44 & 0.52 & 0.44 & 0.39 \\
LLaMA-0shot & 0.32 & 0.25 & 0.32 & 0.24 \\
LLaMA-0shot-COT & 0.12 & 0.25 & 0.12 & 0.09 \\
LLaMA-0shot-Knowledge & 0.32 & 0.25 & 0.32 & 0.28 \\
Mistral-0shot & 0.19 & 0.23 & 0.19 & 0.10 \\
Mistral-0shot-Knowledge & 0.21 & 0.40 & 0.21 & 0.11 \\
        \midrule
        \multicolumn{5}{c}{\textbf{4 Shot}} \\
GPT-4o-mini-4shot & 0.58 & 0.59 & 0.58 & 0.53 \\
GPT-4o-mini-4shot-COT & 0.57 & 0.53 & 0.57 & 0.50 \\
GPT-4o-mini-4shot-Knowledge & 0.56 & 0.51 & 0.56 & 0.47 \\
GPT-4o-4shot & 0.77 & 0.84 & 0.77 & 0.73 \\
GPT-4o-4shot-COT & 0.63 & 0.76 & 0.63 & 0.53 \\
GPT-4o-4shot-Knowledge & 0.72 & 0.82 & 0.71 & 0.66 \\
LLaMA-4shot & 0.29 & 0.24 & 0.29 & 0.21 \\
LLaMA-4shot-COT & 0.20 & 0.30 & 0.20 & 0.13 \\
LLaMA-4shot-Knowledge & 0.15 & 0.23 & 0.13 & 0.13 \\
Mistral-4shot & 0.02 & 0.02 & 0.02 & 0.02 \\
Mistral-4shot-Knowledge & 0.21 & 0.27 & 0.21 & 0.20 \\
        \midrule
        
        \multicolumn{5}{c}{\textbf{Suprevised}} \\
        SVM & 0.94 & 0.92 & 0.91 & 0.91 \\
        CNN & 0.98 & 0.98 & 0.97 & 0.97 \\
        RNN & 0.99 & 0.99 & 0.99 & 0.99 \\
        % \midrule
        % \multicolumn{5}{c}{\textbf{Conventional Wi-Fi-based Human Activity Recognition Systems}} \\
        E-eyes & 1.00 & 1.00 & 1.00 & 1.00 \\
        CARM & 0.98 & 0.98 & 0.98 & 0.98 \\
\midrule
 \multicolumn{5}{c}{\textbf{Vision Models}} \\
           Zero-shot SVM & 0.26 & 0.25 & 0.25 & 0.25 \\
        Zero-shot CNN & 0.26 & 0.25 & 0.26 & 0.26 \\
        Zero-shot RNN & 0.28 & 0.28 & 0.29 & 0.28 \\
        SVM & 0.99 & 0.99 & 0.99 & 0.99 \\
        CNN & 0.98 & 0.99 & 0.98 & 0.98 \\
        RNN & 0.98 & 0.99 & 0.98 & 0.98 \\
GPT-4o-mini-Vision & 0.84 & 0.85 & 0.84 & 0.84 \\
GPT-4o-mini-Vision-COT & 0.90 & 0.91 & 0.90 & 0.90 \\
GPT-4o-Vision & 0.74 & 0.82 & 0.74 & 0.73 \\
GPT-4o-Vision-COT & 0.70 & 0.83 & 0.70 & 0.68 \\
LLaMA-Vision & 0.20 & 0.23 & 0.20 & 0.09 \\
LLaMA-Vision-Knowledge & 0.22 & 0.05 & 0.22 & 0.08 \\

        \bottomrule
    \end{tabular}
    \label{full}
\end{table*}




\end{document}


\subsection{State-of-the-art image generation}
We compare FlexVAR with existing generative methods on the ImageNet-1K benchmark, including GAN, diffusion models, random-scan, raster-scan, and scaling-scan autoregressive models.  As shown in Tab. \ref{tab:main}, \ref{tab:main-512}.

\noindent \textbf{Overall comparison on ImageNet 256$\times$256.} To ensure a fair comparison, we present models with a size smaller than 1B in Tab. \ref{tab:main}. Our FlexVAR achieves state-of-the-art performance in all generative methods, and performs remarkably well compared to the VAR counterparts.  Specifically, we achieve -0.45, -0.56, and -0.12 FID improvement compared with VAR at different model sizes.

\noindent \textbf{Zero-shot inference with more steps.} 
We use 13 steps for image generation without training, as shown in the last row of Tab. \ref{tab:main}. FlexVAR can flexibly adopt more steps to improve image quality. By using 13 inference steps, FlexVAR further enhances the performance to 2.08 FID and 315 IS, manifesting strong flexibility and generalization capabilities.
The specific steps design is detailed in the Appendix.


% 表的部分
\begin{table}[h]
  \centering
  \resizebox{0.48\textwidth}{!}{
    \begin{tabular}{lccccc}
      \toprule
     \textbf{Model} & \textbf{Training Free} &  \textbf{FID} & \textbf{IS}  & \textbf{Params.} \\ 
      \hline
      BigGAN~\citep{biggan} & \multirow{1}{*}{$\times$} &  8.43  &  177.9 & 112M \\     
      \cline{2-5}
      ADM~\citep{adm} & $\times$ & 23.24  & 101.0  & 554M  \\     
      DiT-XL/2~\citep{dit} & $\times$ & 3.04  & 240.8  & 675M  \\     
      \cline{2-5}
      MaskGIT~\citep{chang2022maskgit} & $\times$& 7.32  & 156.0  & 227M  \\     
      \cline{2-5}
       VQGAN~\citep{esser2021taming}  & $\times$ & 26.52  & 66.8  & 1.4B  \\     
       VAR-$d36$~\citep{esser2021taming} &$\times$ & 2.63  &  303.2  & 2.3B  \\     
       \rowcolor{gray!10}
       FlexVAR-$d24$ (ours) & $\checkmark$ & 4.43  &  314.4  & 1.0B  \\     
      \bottomrule
    \end{tabular}
  }
  \caption{Zero-shot inference on ImageNet 512$\times$512 conditional generation. \textbf{Training Free} indicates whether the model is trained at the 512$\times$512 resolution.}
  \label{tab:main-512}
  % \vspace{-5mm}
\end{table}

\noindent \textbf{Zero-shot inference on ImageNet 512$\times$512 benchmark.} 
We use FlexVAR-$d24$ to generate 512$\times$512 images and evaluate on ImageNet-512 benchmark without training, as shown in Tab. \ref{tab:main-512}. Surprisingly, our FlexVAR-$d24$ exhibits competitive performance when compared to VAR, despite FlexVAR being trained only on resolutions $\leq$ 256$\times$256 and having only 1.0B parameters.


\subsection{Ablation study}
% \begin{table}[h]
%   \centering
%   % \footnotesize
%   % \vspace{-3mm}
%     % \renewcommand\arraystretch{1.2} % 1.95
%   \resizebox{0.48\textwidth}{!}{
%     \begin{tabular}{c|ccccc}
%       \toprule
%       % \Xhline{0.7pt}

%       note & \textbf{Pred. type} & \textbf{Tokenizer} &  \textbf{Pos. embed.} & \textbf{FID} & \textbf{IS} \\ 
%       \hline
%      ori. VAR & residual &  VAR & fixed-length & 4.00  & 226.04 \\     
%      -& GT &  VAR & fix-length & 7.14 & 166.20 \\       
%      -& GT & Llamagen  & fix-length & 6.18 & 180.36 \\       
%      -& GT & ours  & fix-length & 3.82 & 229.35 \\   
%       \rowcolor{gray!10}
%      final& GT &  ours  & scalable & 3.71 & 230.22 \\       
%       \bottomrule
%       % \Xhline{0.7pt}
%       \end{tabular}
%       }
%   \caption{Ablation of diverse designs of FlexVAR. We report the results with model scale -$d20$ trained 40 epochs ($\sim$ 160K iterations) on ImageNet for analysis.}
%       \label{tab:ab-comp}
%   % \vspace{-2mm}
%   \end{table}  



\begin{table}[t]
  \centering
  % \footnotesize
  % \vspace{-3mm}
    % \renewcommand\arraystretch{1.2} % 1.95
  \resizebox{0.48\textwidth}{!}{
    \begin{tabular}{ccc|cc}
      \toprule
      % \Xhline{0.7pt}

      \textbf{Pred. type} & \textbf{VQVAE} &  \textbf{PE} & \textbf{FID} & \textbf{IS} \\ 
      \hline
     Residual &  VAR & fixed-length & 4.00  & 226.04 \\     
     GT &  VAR & fix-length & $\mathrm{N/A}$  &  $\mathrm{N/A}$ \\ 
     GT & Llamagen  & fix-length & 17.75 & 234.12 \\       
     GT & ours  & fix-length & 3.82 & 229.35 \\   
      \rowcolor{gray!10}
     GT &  ours  & scalable & 3.71 & 230.22 \\       
      \bottomrule
      % \Xhline{0.7pt}
      \end{tabular}
      }
  \caption{Ablation of diverse designs. We use the \textit{next-scale-prediction} paradigm, explore the effects of different prediction types (residual/GT), VQVAE tokenizers (Llamagen/VAR/ours), and positional embedding (fix-length/scalable). $\mathrm{N/A}$ denotes the model does not converge during training. We report the results with model scale -$d20$ trained 40 epochs on ImageNet-1K.  }
      \label{tab:ab-comp}
  % \vspace{-2mm}
  \end{table}  
	
We conduct ablation studies on various  design choices  in FlexVAR
% to show their contribution to the final results
.
Due to the limited computational resources, we report the results trained with a short training scheme in Tab. \ref{tab:ab-comp}, \ref{tab:ab-mamba}, \ref{tab:ab-posi}, \textit{i.e.}, 40 epochs  ($\sim$ 70K iterations).

\noindent \textbf{Component-wise ablations.} 
To understand the effect of each component, we start with standard VAR and progressively add each design. (Tab. \ref{tab:ab-comp}):
\begin{itemize}[itemsep=2pt,topsep=0pt,parsep=0pt]
\item \textbf{Baseline:} VAR uses a residual prediction paradigm and exhibits decent performance (1$^{st}$ result), but its flexibility in image generation does not meet expectations (as described in Sec. \ref{sec:intro}).
\item \textbf{Prediction type:} It is infeasible to directly convert the prediction type to GT, as seen in the 2$^{nd}$  and 3$^{rd}$ results. We employ the VQVAE tokenizers from VAR and Llamagen, both of which yield inferior performance. This is not surprising, as the current tokenizers lack robustness to images with varying latent space, while we force these tokenizers to obtain multi-scale latent features during training  (we provide a detailed analysis in Fig. \ref{fig:abla-vae}).
% which motivates us to train a scalable tokenizer to adapt to the GT prediction paradigm.
\item \textbf{Tokenizer:}  Our scalable tokenizer obtains reasonable multi-scale latent features during training, resulting in an improvement of -13.87 FID (the 4$^{th}$ results). However, flexible image generation is not accomplished yet.
\item \textbf{Position embedding:} The introduction of our scalable PE (last result) provides high flexibility for image generation, and further enhances the performance to 3.71 FID.
\end{itemize}


\begin{figure}[h]
\begin{center}
   \includegraphics[width=0.99\linewidth]{figs/pdf/abla-vae.pdf}
\end{center}
   \caption{
    Compared with VQVAE tokenizers \cite{var, llamagen} for multi-scale reconstructing images, we downsample the latent features in VQVAE to multiple scales and then use the VQVAE Decoder to reconstruct images. We upsample images $<$ 100 pixels using bilinear interpolation for a better view.   }
\label{fig:abla-vae}
\end{figure}

\noindent \textbf{Reconstruct images with different VQVAEs.} 
In Fig. \ref{fig:abla-vae}, we reconstruct multi-scale images by scaling the latent features in VQVAE tokenizers. Existing VQVAE tokenizers typically do not support scaling the latent features across a range of small to large scales. 
VAR's VQVAE \cite{var} uses a residual-based training recipe, directly applying it to non-residual image reconstruction does not yield the anticipated results (the 1$^{st}$ row).
The VQVAE tokenizer from Llamagen \cite{llamagen} shows excellent reconstruction performance only at the original latent space, indicating that it is not feasible for scale-wise autoregressive modeling (the 2$^{nd}$ row).





\noindent \textbf{Transfer FlexVAR to Mamba.} 
Recent work, AiM \cite{aim}, uses the Mamba architecture for token-wise autoregressive modeling. Inspired by this, we modify FlexVAR with Mamba to evaluate the performance (Tab. \ref{tab:ab-mamba}).
With similar model parameters, Mamba demonstrates competitive results compared to transformer models, indicating the GT prediction paradigm can effectively adapt to linear attention mechanisms like Mamba. However, considering that this Mamba architecture does not reflect the speed advantage, we do not integrate Mamba into our final version.

Mamba's inherent unidirectional attention mechanism prevents image tokens from achieving global attention within the same scale. To address this issue, we employ 8 scanning paths in different Mamba layers to capture global information. The specific Mamba architecture is detailed in the Appendix.



% \begin{figure}[h]
%   \centering
%   \includegraphics[width=0.99\linewidth]{figs/pdf/mamba.pdf}
%   \vspace{-5mm}
%   \caption{Sptial scan paths for Mamba.}
%   \label{fig:mamba}
% \end{figure}

% \vspace{-5mm}

% 表的部分
\begin{table}[h]
  \centering
  \resizebox{0.48\textwidth}{!}{
    \begin{tabular}{cccccc}
      \toprule
     \textbf{Depth} & \textbf{Atten. type} &  \textbf{FID} & \textbf{IS}  & \textbf{Params.} & \textbf{Time} \\ 
      \hline
     \multirow{2}{*}{-$d16$} & Transformer  & 4.32  & 209.87 & 310M & 0.2 \\     
      & Mamba  & 4.22  & 200.04 & 370M & 0.2 \\     
      \hline
     \multirow{2}{*}{-$d20$} & Transformer  & 3.71  & 230.22 & 600M & 0.3 \\     
      & Mamba  & 3.80  & 216.45 & 700M & 0.3 \\   
      \bottomrule
    \end{tabular}
  }
  \caption{Ablation of the Mamba architectural. }
  \label{tab:ab-mamba}
  % \vspace{-5mm}
\end{table}

\begin{table}[h]
  \centering
  \resizebox{0.48\textwidth}{!}{
    \begin{tabular}{ccc|cc}
      \toprule
      % \Xhline{0.7pt}
      \textbf{Step} & \textbf{h-w coordinates} & \textbf{learnable} & \textbf{FID} & \textbf{IS} \\ 
      \hline
      fix-length & fixed-length & True & 3.82 & 229.35 \\     
      $\times$ & fixed-length & True & 3.87 & 224.25 \\     
      $\times$ & scaleable & False & 3.74 & 224.04 \\     
      \rowcolor{gray!10}
      $\times$ & scaleable & True & 3.71 & 230.22 \\     
      \bottomrule
      % \Xhline{0.7pt}
      \end{tabular}
      }
  \caption{Ablation of Position Embedding. We report the results with model scale -$d20$ trained 40 epochs ($\sim$ 70K iterations). $\times$ denotes that the corresponding Position Embedding is removed.}
      \label{tab:ab-posi}
  % \vspace{-2mm}
  \end{table}  

\noindent \textbf{Position Embedding.} 
In Tab. \ref{tab:ab-posi}, we experiment with several types of step PE and x-y coordinate PE. To make the model robust to inference steps and enable it to generate images at any resolution, we remove the fixed-length step embedding (results in the second row), and the performance showed only slight changes. We adopt a non-parametric variant, similar to ViT \cite{vit}, which shows a 0.03 FID difference compared to the learnable variant.




\subsection{Analysis of the GT prediction paradigm}
\noindent \textbf{Convergence rate.}
We compared the training loss of VAR and our FlexVAR, as shown in Fig. \ref{fig:loss}. FlexVAR demonstrates significantly lower loss values and faster convergence, indicating that predicting ground-truth rather than residuals is more friendly for training.
This may be attributed to the residuals at different scales lacking semantic continuity, and this implicit prediction approach might limit the training convergence rate.

\begin{figure}[h]
\begin{center}
   \includegraphics[width=0.99\linewidth]{figs/pdf/loss.pdf}
\end{center}
   \caption{
    Training loss of VAR \textit{vs.} FlexVAR. FlexVAR demonstrates a faster convergence rate. We report the results with trained 40 epochs ($\sim$ 70K iterations). 
   }
\label{fig:loss}
\end{figure}


\noindent \textbf{Generate images at any resolution.}
We show generated images at different resolutions using FlexVAR-$d24$ in Fig \ref{fig:abs} and Fig. \ref{fig:any-reso}. By controlling the inference steps, our FlexVAR can generate images at any resolution, despite being trained only on images with resolutions $\leq$ 256px. The generated images demonstrate strong semantic consistency across multiple scales, and the higher resolutions exhibit more detailed clarity. See the Appendix for more zero-shot high-resolution generation samples and step designs. 

\begin{figure}[h]
\begin{center}
   \includegraphics[width=0.99\linewidth]{figs/pdf/any_reso.pdf}
\end{center}
   \caption{
    Generated samples from 80px to 512px. FlexVAR demonstrates strong consistency across various scales and can generate 512px images, despite the model being trained only on images with resolutions $\leq$ 256. Zoom in for a better view.
   }
\label{fig:any-reso}
\end{figure}


\noindent \textbf{Generate images at any ratio.}
We show generated samples with various aspect ratios in Fig. \ref{fig:abs} and Fig. \ref{fig:any-ratio}. By controlling the aspect ratio at each step of the inference process, our FlexVAR allows for generating images with various aspect ratios, demonstrating the flexibility and controllability of our FlexVAR.

\begin{figure}[h]
\begin{center}
   \includegraphics[width=0.99\linewidth]{figs/pdf/any_ratio.pdf}
\end{center}
   \caption{
    Generated samples with various aspect ratios.  FlexVAR-$d24$ is used. FlexVAR demonstrates good visual quality across images with various aspect ratios.
   }
\label{fig:any-ratio}
\end{figure}


\noindent \textbf{Generate images at any step.}
In Fig. \ref{fig:any-step}, we investigate the FID and IS for generating 256$\times$256 images from 6 to 16 steps with 3 different sizes (depth 16, 20, 24). As the number of steps increases, the quality of the generated images improves. The improvement is more significant in larger models (\textit{e.g.}, FlexVAR-$d24$), as larger transformers are thought able to learn more complex and fine-grained image distributions. 
During training, we use up to 10 steps to avoid OOM (out-of-memory) problem. Surprisingly, in the inference stage, using 13 steps results in a performance gain of -0.13 FID. This observation indicates that our FlexVAR is flexible with respect to inference steps, allowing for fewer steps to speed up image generation or more steps to achieve higher-quality images. The details of various step designs are provided in the Appendix.
\begin{figure}[h]
\begin{center}
   \includegraphics[width=0.99\linewidth]{figs/pdf/any_step.pdf}
\end{center}
   \caption{
    Zero-shot image generation at different steps (from 6 to 13 steps). FID and IS are used for evaluation. We use $\leq$ 10 steps for training, and FlexVAR can zero-shot transfer to 13 steps during inference and achieve better results.
   }
\label{fig:any-step}
\end{figure}


\noindent \textbf{Refine image at high resolution.}
In Fig. \ref{fig:super_reso}, we input low-resolution images (e.g., 256$\times$256) and enable FlexVAR to output high-resolution refined images. Despite being trained only on $\leq$ 256px images, FlexVAR effectively refines image details by increasing the input image resolution, such as the eyes of the dogs in the example. This demonstrates the high flexibility of FlexVAR in image-to-image generation.
\begin{figure}[h]
\begin{center}
   \includegraphics[width=0.99\linewidth]{figs/pdf/super_reso.pdf}
\end{center}
   \caption{
    Zero-shot image refinement at high resolution. Zoom in for a better view.
   }
\label{fig:super_reso}
\end{figure}


\noindent \textbf{Image in-painting and out-painting.}
For in-painting and out-painting, we teacher-force ground-truth tokens outside the mask and let the model only generate tokens within the mask. Class label information is also injected. The results are visualized in Fig. \ref{fig:edit-paint}. Without modifications to the architecture design or training, FlexVAR achieves decent results on these image-to-image tasks.
\begin{figure}[h]
\begin{center}
   \includegraphics[width=0.99\linewidth]{figs/pdf/edit-paint.pdf}
\end{center}
   \caption{
    Zero-shot evaluation in/out-painting. The results show that  FlexVAR can generalize to novel downstream tasks without special design and finetuning.
   }
\label{fig:edit-paint}
\end{figure}


\noindent \textbf{Image extension.}
For image extension, we generate images with an aspect ratio of 1:2 for the target class, with the ground-truth tokens forced to be in the center. FlexVAR shows decent results in image extension, indicating the strong generalization ability of our FlexVAR.
\begin{figure}[h]
\begin{center}
   \includegraphics[width=0.99\linewidth]{figs/pdf/edit-extent.pdf}
\end{center}
   \caption{
    Zero-shot evaluation image expansion. The results show that  FlexVAR can generalize to novel downstream tasks without special design and fine-tuning.
   }
\label{fig:edit-extent}
\end{figure}



\noindent \textbf{Failure case.}
FlexVAR fails to generate images with a resolution 3$\times$ or more than the training resolution, as illustrated in Fig. \ref{fig:failure}. These cases typically feature noticeable wavy textures and blurry areas in the details. This failure is likely due to the overly homogeneous structure of the current training dataset. \textit{i.e.}, ImageNet-1K generally lacks multi-scale objects ranging from coarse to fine, leading to errors in generating details of high-resolution objects.

We hypothesize that training the model with a more complex dataset that includes images with fine-grained details, the model might become robust for higher resolutions.
%File: formatting-instructions-latex-2025.tex
%release 2025.0
\documentclass[letterpaper]{article} % DO NOT CHANGE THIS
\usepackage{aaai25}  % DO NOT CHANGE THIS
\usepackage{times}  % DO NOT CHANGE THIS
\usepackage{helvet}  % DO NOT CHANGE THIS
\usepackage{courier}  % DO NOT CHANGE THIS
\usepackage[hyphens]{url}  % DO NOT CHANGE THIS
\usepackage{graphicx} % DO NOT CHANGE THIS
\urlstyle{rm} % DO NOT CHANGE THIS
\def\UrlFont{\rm}  % DO NOT CHANGE THIS
\usepackage{natbib}  % DO NOT CHANGE THIS AND DO NOT ADD ANY OPTIONS TO IT
\usepackage{caption} % DO NOT CHANGE THIS AND DO NOT ADD ANY OPTIONS TO IT
\frenchspacing  % DO NOT CHANGE THIS
\setlength{\pdfpagewidth}{8.5in}  % DO NOT CHANGE THIS
\setlength{\pdfpageheight}{11in}  % DO NOT CHANGE THIS
%
% Keep the \pdfinfo as shown here. There’s no need
% for you to add the /Title and /Author tags.
\pdfinfo{
/TemplateVersion (2025.1)
}
%
% These are recommended to typeset algorithms but not required. See the subsubsection on algorithms. Remove them if you don't have algorithms in your paper.
\usepackage{algorithm}
\usepackage{algorithmic}
\usepackage{booktabs}
\usepackage{amsmath}
\usepackage{multirow}
\usepackage{amssymb}
% \usepackage{algpseudocode}

\usepackage[utf8]{inputenc} % allow utf-8 input
\usepackage[T1]{fontenc}    % use 8-bit T1 fonts
% \usepackage[hidelinks]{hyperref}      % hyperlinks
\usepackage{url}            % simple URL typesetting
\usepackage{amsfonts}       % blackboard math symbols
\usepackage{nicefrac}       % compact symbols for 1/2, etc.
\usepackage{microtype}      % microtypography
\usepackage{xcolor}         % colors
\usepackage{graphicx}         % 
\usepackage{xurl}
% \usepackage{algorithm,algorithmic}
\urlstyle{same}
\usepackage{float}
\usepackage{lipsum}
\usepackage{colortbl}
\usepackage{makecell}
\usepackage{xcolor}
\usepackage{wrapfig}
% \newcommand{\answerYes}[1]{\textcolor{blue}{#1}} 
% \newcommand{\answerNo}[1]{\textcolor{teal}{#1}} 
% \newcommand{\answerNA}[1]{\textcolor{gray}{#1}} 
% \newcommand{\answerTODO}[1]{\textcolor{red}{#1}} 
% \renewcommand\thesection{\Alph{section}}
%
% These are are recommended to typeset listings but not required. See the subsubsection on listing. Remove this block if you don't have listings in your paper.
\usepackage{newfloat}
\usepackage{listings}
\DeclareCaptionStyle{ruled}{labelfont=normalfont,labelsep=colon,strut=off} % DO NOT CHANGE THIS
\lstset{%
	basicstyle={\footnotesize\ttfamily},% footnotesize acceptable for monospace
	numbers=left,numberstyle=\footnotesize,xleftmargin=2em,% show line numbers, remove this entire line if you don't want the numbers.
	aboveskip=0pt,belowskip=0pt,%
	showstringspaces=false,tabsize=2,breaklines=true}
\floatstyle{ruled}
\newfloat{listing}{tb}{lst}{}
\floatname{listing}{Listing}
%
% Keep the \pdfinfo as shown here. There's no need
% for you to add the /Title and /Author tags.
\pdfinfo{
/TemplateVersion (2025.1)
}

% DISALLOWED PACKAGES
% \usepackage{authblk} -- This package is specifically forbidden
% \usepackage{balance} -- This package is specifically forbidden
% \usepackage{color (if used in text)
% \usepackage{CJK} -- This package is specifically forbidden
% \usepackage{float} -- This package is specifically forbidden
% \usepackage{flushend} -- This package is specifically forbidden
% \usepackage{fontenc} -- This package is specifically forbidden
% \usepackage{fullpage} -- This package is specifically forbidden
% \usepackage{geometry} -- This package is specifically forbidden
% \usepackage{grffile} -- This package is specifically forbidden
% \usepackage{hyperref} -- This package is specifically forbidden
% \usepackage{navigator} -- This package is specifically forbidden
% (or any other package that embeds links such as navigator or hyperref)
% \indentfirst} -- This package is specifically forbidden
% \layout} -- This package is specifically forbidden
% \multicol} -- This package is specifically forbidden
% \nameref} -- This package is specifically forbidden
% \usepackage{savetrees} -- This package is specifically forbidden
% \usepackage{setspace} -- This package is specifically forbidden
% \usepackage{stfloats} -- This package is specifically forbidden
% \usepackage{tabu} -- This package is specifically forbidden
% \usepackage{titlesec} -- This package is specifically forbidden
% \usepackage{tocbibind} -- This package is specifically forbidden
% \usepackage{ulem} -- This package is specifically forbidden
% \usepackage{wrapfig} -- This package is specifically forbidden
% DISALLOWED COMMANDS
% \nocopyright -- Your paper will not be published if you use this command
% \addtolength -- This command may not be used
% \balance -- This command may not be used
% \baselinestretch -- Your paper will not be published if you use this command
% \clearpage -- No page breaks of any kind may be used for the final version of your paper
% \columnsep -- This command may not be used
% \newpage -- No page breaks of any kind may be used for the final version of your paper
% \pagebreak -- No page breaks of any kind may be used for the final version of your paperr
% \pagestyle -- This command may not be used
% \tiny -- This is not an acceptable font size.
% \vspace{- -- No negative value may be used in proximity of a caption, figure, table, section, subsection, subsubsection, or reference
% \vskip{- -- No negative value may be used to alter spacing above or below a caption, figure, table, section, subsection, subsubsection, or reference

\setcounter{secnumdepth}{0} %May be changed to 1 or 2 if section numbers are desired.

% The file aaai25.sty is the style file for AAAI Press
% proceedings, working notes, and technical reports.
%

% Title

% Your title must be in mixed case, not sentence case.
% That means all verbs (including short verbs like be, is, using,and go),
% nouns, adverbs, adjectives should be capitalized, including both words in hyphenated terms, while
% articles, conjunctions, and prepositions are lower case unless they
% directly follow a colon or long dash
\title{Efficient Reinforcement Learning Through Adaptively Pretrained Visual Encoder}
\author{
    %Authors
    % All authors must be in the same font size and format.
    Yuhan Zhang\textsuperscript{\rm 1}\textsuperscript{\rm 2}\equalcontrib
    Guoqing Ma\textsuperscript{\rm 1}\textsuperscript{\rm 3}\equalcontrib,
    Guangfu Hao\textsuperscript{\rm 1}\textsuperscript{\rm 2},
    Liangxuan Guo\textsuperscript{\rm 1}\textsuperscript{\rm 3},
    Yang Chen\textsuperscript{\rm 1}\textsuperscript{\rm 4},
    Shan Yu\textsuperscript{\rm 1}\textsuperscript{\rm 3}\textsuperscript{\rm 4}\thanks{Corresponding author.}
}
\affiliations{
    %Afiliations
    \textsuperscript{\rm 1}Laboratory of Brain Atlas and Brain-inspired Intelligence, Institute of Automation, Chinese Academy of Sciences\\
    \textsuperscript{\rm 2}School of Artificial Intelligence, University of Chinese Academy of Sciences\\
    \textsuperscript{\rm 3}School of Future Technology, University of Chinese Academy of Sciences\\
    \textsuperscript{\rm 4}Key Laboratory of Brain Cognition and Brain-inspired Intelligence Technology, Chinese Academy of Sciences\\

    \{zhangyuhan2022, maguoqing2022, haoguangfu2021, guoliangxuan2021, yang.chen, shan.yu\}@ia.ac.cn
}

%Example, Single Author, ->> remove \iffalse,\fi and place them surrounding AAAI title to use it
\iffalse
\title{My Publication Title --- Single Author}
\author {
    Author Name
}
\affiliations{
    Affiliation\\
    Affiliation Line 2\\
    name@example.com
}
\fi

\iffalse
%Example, Multiple Authors, ->> remove \iffalse,\fi and place them surrounding AAAI title to use it
\title{My Publication Title --- Multiple Authors}
\author {
    % Authors
    First Author Name\textsuperscript{\rm 1,\rm 2},
    Second Author Name\textsuperscript{\rm 2},
    Third Author Name\textsuperscript{\rm 1}
}
\affiliations {
    % Affiliations
    \textsuperscript{\rm 1}Affiliation 1\\
    \textsuperscript{\rm 2}Affiliation 2\\
    firstAuthor@affiliation1.com, secondAuthor@affilation2.com, thirdAuthor@affiliation1.com
}
\fi


% REMOVE THIS: bibentry
% This is only needed to show inline citations in the guidelines document. You should not need it and can safely delete it.
\usepackage{bibentry}
% END REMOVE bibentry

\begin{document}

\maketitle

\begin{abstract}
While Reinforcement Learning (RL) agents can successfully learn to handle complex tasks, effectively generalizing acquired skills to unfamiliar settings remains a challenge. One of the reasons behind this is the visual encoders used are task-dependent, preventing effective feature extraction in different settings. To address this issue, recent studies have tried to pretrain encoders with diverse visual inputs in order to improve their performance. However, they rely on existing pretrained encoders without further exploring the impact of pretraining period. In this work, we propose APE: efficient reinforcement learning through \textbf{A}daptively \textbf{P}retrained visual \textbf{E}ncoder—a framework that utilizes adaptive augmentation strategy during the pretraining phase and extracts generalizable features with only a few interactions within the task environments in the policy learning period. Experiments are conducted across various domains, including DeepMind Control Suite, Atari Games and Memory Maze benchmarks, to verify the effectiveness of our method. Results show that mainstream RL methods, such as DreamerV3 and DrQ-v2, achieve state-of-the-art performance when equipped with APE. In addition, APE significantly improves the sampling efficiency using only visual inputs during learning, approaching the efficiency of state-based method in several control tasks. These findings demonstrate the potential of adaptive pretraining of encoder in enhancing the generalization ability and efficiency of visual RL algorithms.
\end{abstract}

% Uncomment the following to link to your code, datasets, an extended version or similar.
%
% \begin{links}
%     \link{Code}{https://aaai.org/example/code}
%     \link{Datasets}{https://aaai.org/example/datasets}
%     \link{Extended version}{https://aaai.org/example/extended-version}
% \end{links}

\section{Introduction}

Deep Reinforcement Learning (Deep RL) has made great advances in recent years. Notable algorithms such as MuZero \cite{MuZero}, Player of Games \cite{PlayerofGame} and ReBeL \cite{ReBel} have been proposed to solve many challenging decision making problems.
While these advances have primarily focused on state-based inputs, significant progress has also been made in visual RL, i.e., leveraging image inputs for policy learning \cite{CURL, Dreamer, DreamerV2, dreamerv3, Kostrikov2020ImageAI}. 

However, visual RL agents learning from these high-demensional observations suffer from problems of low efficiency and often overfitting to specific environments \cite{Song2019ObservationalOI, CURL}.
Since the performance of these agents depends heavily on the quality of extracted features, the critical role of enhancing visual encoders has been highlighted in both model-free and model-based algorithms \cite{Yarats2019ImprovingSE,Dreamer, Poudel2023ReCoReRC}. 
\begin{figure}[t]
\centering
\includegraphics[width=0.9\columnwidth]{figs/vis_walker1.pdf} % Reduce the figure size so that it is slightly narrower than the column. Don't use precise values for figure width.This setup will avoid overfull boxes.
\caption{Visualization of ResNet-18 model with different pretraining strategy using LayerCAM \cite{9462463}, which indicates that APE is able to extract more precise outline of the Walker than other initialization settings. The first row displays the pure feature maps, which are also presented together with the image in the second row.}
\label{fig:vis init}
\end{figure}

In visual RL, various approaches have been explored to improve representation learning, among which data augmentations are often used to increase data diversity \cite{Wang2020ImprovingGI, Raileanu2021AutomaticDA,  10.1145/3510414, Liu2023RobustRL}. 
% Both weak data augmentations \cite{Kostrikov2020ImageAI, Yarats2021MasteringVC} and strong augmentations \cite{Hansen2021StabilizingDQ, Yuan2022PreTrainedIE} are used to enhance the performance of RL agents. 
The challenge lies in extracting generalizable features rather than focusing on task-specific details, leading to difficulties in transferring learned skills to unseen scenarios \cite{Lee2019NetworkRA, Laskin2020ReinforcementLW}.

One promising direction is to exploit cross-domain knowledge learned by pretrained models \cite{RRL, Yuan2022PreTrainedIE}, which has shown great success in improving data efficiency and generalization ability in recent deep learning \cite{Devlin2019BERTPO, Baevski2020wav2vec2A}. In computer vision, since these models have typically been trained on extensive sets of natural images, their features inherently possess general knowledge about the world \cite{Hu2023ForPV}. This approach has the potential to enable RL agents to extract useful features more effectively, enhancing their ability to learn and generalize across different domains. Unsupervised learning, e.g., contrastive learning, is particularly advantageous in this regard as it enables pretrained models to extract meaningful features from unlabeled visual data, effectively addressing the issue of data scarcity and high labeling costs \cite{MoCo, SimCLR}.

Nevertheless, current RL methods simply implement existing pretrained models as visual encoders and augment observations in the downstream policy learning period \cite{RRL, Hu2023ForPV}.
As illustrated in Fig. \ref{fig:vis init}, the features learned by image classification models with the prevailing pretraining strategies (shown in the left three columns) exhibit limited generalization capabilities.
This also results in a lack of exploration of pretraining augmentations, which prove to be an important factor when applying pretrained encoders under great distribution shifts \cite{Geirhos2021PartialSI, Burns2023WhatMP}.

Given this, here we propose APE, a framework where the RL agent learns efficiently through \textbf{A}daptively \textbf{P}retrained visual \textbf{E}ncoder. This novel framework uses an adaptive closed-loop augmentation strategy in contrastive pretraining to learn transferable representations from a wide range of real-world images. Comparison in Fig. \ref{fig:vis init} indicates that APE helps to extract more generalizable features than other pretraining strategies. In addition, it works efficiently, requiring minimal interactions with the targeted environment during policy learning period. We evaluate our method on various challenging visual RL domains, including DeepMind Control (DMC) Suite \cite{dmc}, the Atari 100K benchmark \cite{Bellemare2012TheAL}, and Memory Maze \cite{MemoryMaze}. Experiments demonstrate that APE significantly improves the sampling efficiency and performance of the base RL method. Intersetingly, we found that the real RL enviorments are not necessary to test the pretrained encoder. Linear probes, a common protocol for evaluating the quality of learned representations \cite{SimCLR}, can serve as a useful metric to assess the quality of pretrained encoders quite effectively. 
The main contribution of this paper can be summarized as follows:


\begin{itemize}
    \item We propose a cross-domain RL framework with a fixed encoder pretrained on a wide variety of natural images using adaptive augmentation adjustment. This helps to produces more generalizable representations for the downstream RL tasks.
    
    \item We demonstrate the generality of APE to both model-based and model-free methods, underscoring its adaptability and effectiveness in enhancing learning performance across diverse RL approaches.
    % Our analysis focuses on MBRL-based APE, where the policy is primarily learned from reconstructed latents, thus attach greater importance to the quality of extracted features.
    
    \item APE is developed without any auxiliary tasks or other sensory informantion during policy learning period, effectively decoupling the pretraining phase from subsequent behavior learning tasks. This simple yet powerful design contributes to APE's superior performance on various visual RL benchmarks, approaching the performance of state-based Soft-Actor-Critic (SAC) \cite{SAC} in several control tasks.
\end{itemize}

\section{Related Works}

\subsection{Contrastive Learning}
In computer vision (CV), contrastive learning has gained popularity for its ability to learn generalizable representations leveraging unlabeled images and videos \cite{Oord2018RepresentationLW, SimCLR, MoCo}. Prior studies have emphasized the pivotal role of data augmentation in facilitating unsupervised training \cite{res7, res8, res9}.
Experiments conducted in SimCLR approach \cite{SimCLR} highlight the significant impact of data augmentations, which is re-confirmed by MoCo \cite{MoCo} and its modification MoCo v2 \cite{mocov2}.
% Being an incremental study of MoCo v1/2, MoCo v3\cite{Chen2021AnES}
AdDA \cite{Zhang2023AdaptiveDA} focuses on exploring the effect of dynamic adjustment on augmentation compositions, which enables the network to acquire more generalizable features. 
We adopt the feedback structure \cite{Zhang2023AdaptiveDA}  in the pretraining period and implement it on a different network architecture, which proves to be more suitable for RL tasks \cite{Yuan2022PreTrainedIE}.
% Since the transferability of visual features has been well-explored in CV, a growing body of research is focused on rendering this approach as representation learning method in RL tasks\cite{Zhan2020LearningVR}.




\subsection{Representation Learning in RL}
There are extensive works in RL studying the impact of representation learning \cite{ Lin2020LearningTS, Liu2023RobustRL}, among which contrastive learning is often applied to acquire useful features \cite{Zhan2020LearningVR, Du2021CuriousRL, Schwarzer2021PretrainingRF}. CURL \cite{CURL} trains a visual representation encoder using contrastive loss, significantly improving sampling efficiency over prior pixel-based methods. Proto-RL \cite{Yarats2021ReinforcementLW} learns contrastive visual representations in dynamic RL environments without access to task-specific rewards. To make full use of context information, MLR \cite{Yu2022MaskbasedLR} introduces mask-based reconstruction to promote contrastive representation learning in RL. 
However, prior methods rely completely on data collected in target environments, which limits their generalization to unseen scenarios and hinders their adaptability to new tasks or environments. It also leads to additional sampling costs. APE, on the other hand, is pretrained on a distribution of real-world samples that wider than what policy can provide. 

Besides, the interpretability of extracted features is a key focus \cite{Lin2020SPACEUO, Delfosse2022BoostingOR, Delfosse2024InterpretableCB}, leading to improved performance and robustness of the agent. The efficiency gains of our method also result from a more interpretable encoder, aiding the agent in capturing key factors of observations in policy-making period.

\begin{figure*}[t]
\centering
\includegraphics[width=0.8\textwidth]{figs/pipeline1.pdf} % Reduce the figure size so that it is slightly narrower than the column.
\caption{APE pipeline for MBRL. The training phase is divided into two parts, namely the Adaptive Pretraining period (within the 
% \textcolor{cyan!80!black}{blue} 
blue area) and the Downstream Policy Learning period (within the 
% \textcolor{yellow!80!black}{yellow}
yellow area). A wide variety of real-world images are augmented using an adaptive data augmentation strategy in the first period, which dynamically updates the sampling probability of each augmentation composition in the next pretraining epoch. In the second stage, the pretrained vision encoder is implemented in a generic RL framework as a perception module for the policy.}
\label{fig1}
\end{figure*}

\subsection{Generalization for Image-Based RL}
Since image augmentation has been successfully applied in CV for improving performance on object classification tasks, different approaches of transformation were investigated and incorporated in RL pipelines \cite{Laskin2020ReinforcementLW, Kostrikov2020ImageAI, Stooke2020DecouplingRL}. 
DrAC \cite{Raileanu2021AutomaticDA} contributes to the proper use of data augmentation for actor-critic algorithms and proposes an automatically selecting approach. 
% ATC \cite{Stooke2020DecouplingRL} associates pairs of observations separated by a short time difference and introduces a new kind of augmentation to enable replay of latent images. 
SVEA \cite{Hansen2021StabilizingDQ} investigates the factors contributing to instability when employing augmentation within off-policy RL methods. DrQ \cite{Kostrikov2020ImageAI} together with DrQ-v2 \cite{Yarats2021MasteringVC} introduces a simple augmentation method for model-free RL algorithms utilizing input perturbations and regularization techniques, which we use to evaluate the generality of APE.
However, most previous methods attach more importance to the policy training period and straightforwardly augment the observations of the target environments \cite{Zhao2024AnEG}. Thus, they fall short in providing the requisite data diversity, which is essential for generalization over large domain gaps \cite{Yuan2022PreTrainedIE}. 
On the contrary, APE leverages an adaptively pretrained encoder without neglecting the potential benefits of pretraining augmentation strategy in RL, which has been confirmed in recent studies for its effectiveness in enhancing RL performance \cite{Burns2023WhatMP}.

\subsection{Pretrained Visual Encoders for RL}

Instead of training with expensive collected data, researches have also been made to bridge the domain gap between cross-domain datasets and the inputs of the target environments \cite{Ma2022VIPTU, Hu2023ForPV}. Using a pretrained ResNet encoder, RRL \cite{RRL} brings a straightforward approach to fuse extracted features into a standard RL pipeline. PIE-G \cite{Yuan2022PreTrainedIE} further demonstrates the effectiveness of supervised pretrained encoders by using early layer features from frozen models, with strongly augmented representations. By combining pretrained visual encoder and proprioceptive information, MVP outperforms supervised encoders in motor control tasks \cite{Xiao2022MaskedVP}. While pretrained models in aid of model-free RL have been studied, there lacks exploration on Model-Based Reinforcement Learning (MBRL) algorithms. These methods rely compeletely on reconstructed latents, thus further highlights the significance of representation learning \cite{Poudel2023ReCoReRC}.
Besides, extra tasks or sensory data are often needed during policy learning period while APE works without such intensive task-specific data.

\section{Preliminaries}
The proposed APE expands on both model-based and model-free RL methods. Detailed analyses are conducted on a mainstream MBRL framework, DreamerV3 \cite{dreamerv3}, which only learns from the representations extracted from original image observations. This integration allows APE to inherit DreamerV3's generality, operating with fixed hyperparameters across various domains. This section provides an overall description of our MBRL Backbone.
\subsubsection{Latent dynamics.}
The latent dynamics of DreamerV3 are modeled as a recurrent state space model (RSSM) which consists of the following five components:

\begin{figure*}[t]
\centering
\includegraphics[width=0.9\textwidth]{figs/all1.pdf} % Reduce the figure size so that it is slightly narrower than the column.
\caption{Training curves for DMC vision benchmarks.}
\label{figdmc}
\end{figure*}

% \begin{center}
% Encoder: \quad $z_t \sim q_{\theta}(z_t | z_{t-1},a_{t-1},o_t)$ \\
% Dynamics model: \quad $\hat{z}_t \sim p_{\theta}(z_t | z_{t-1},a_{t-1})$
% \end{center}

\begin{equation}
\begin{aligned}
    \label{qua4}
    &\text{Encoder:} &z_t \sim f_{\theta}(z_t \mid z_{t-1},a_{t-1},o_t)\\
    &\text{Dynamics model:} &\hat{z}_t \sim p_{\theta}^{D}(\hat{z}_t \mid z_{t-1},a_{t-1})\\
    &\text{Reward predictor:} & \hat{r}_t \sim p_{\theta}^{R}(\hat{r}_t \mid z_t,z_{t-1},a_{t-1})\\
    &\text{Continue predictor:}& \hat{c}_t \sim p_{\theta}^{C}(\hat{c}_t \mid z_t,z_{t-1},a_{t-1})\\
    &\text{Decoder:}& \hat{x}_t\sim g_{\theta}(\hat{x}_t \mid z_t,z_{t-1},a_{t-1})
\end{aligned}
\tag{1}
\end{equation}

\noindent Here the dynamics model is designed to predict the next latent representation $\hat{z}_t$, while the feature ${z}_t$ generated by the encoder is used in the reward and continue predictor. The decoder using a convolutional neural network (CNN) helps in reconstructing visual inputs. 

\subsubsection{Agent learning.}
The actor-critic algorithm is employed to learn behaviors from the feature sequences predicted by the world model \cite{2018arXiv180310122H}. The actor aims to maximize the expected return $R_t$ for each state $s_t$ while the critic is trained to predict the return of each state $s_t$ with the current action $a_t$. Given $\gamma$ as the discount factor for the future rewards, the agent model are defined as follows:



\begin{equation}
\begin{aligned}
    \label{qua4}
    \text{Actor:} \quad & a_t \sim \pi_{\phi}(a_t \mid s_t)\\
    \text{Critic:} \quad &  V_{\psi} \approx  \mathbb{E}_{\pi_{\phi}, p_{\theta}}[\sum_{k=0}^{\infty}\gamma^kr_{t+k}]
\end{aligned}
\tag{2}
\end{equation}

The overall loss of the agent can be found in Appendix C.
% We adopt the agent learning settings in DreamerV3\cite{dreamerv3}, with the overall loss of the actor-critic algorithm remains unchanged.

% \begin{equation}
% \begin{aligned}
%     \label{qua4}
%     \mathcal{L}(\phi) \\
%     \mathcal{L}(\psi) 
% \end{aligned}
% \tag{8}
% \end{equation}


\section{Methodology}

We consider the visual task as a Partially Observable Markov Decision Process (POMDP) \cite{Bellman1957AMD} due to the partial state observability from images. We denote the state space, the observation space, the action space and the reward function as $\mathcal{S, O, A }$ and $r$ respectively. The goal for an agent is to find a policy $\pi^*$ to maximize the expected cumulative return $E_p(\sum_{t=1}^Tr_t)$.
As shown in Fig. \ref{fig1}, our method decouples the pretraining period from the downstream control task and thus consists of two main parts: Adaptive Pretraining and Policy Learning, which are described as follows.



\subsection{Adaptive Pretraining}
Dynamic adjustment on data augmentation compositions is applied on MoCo v2 to explore the importance of visual encoder in RL methods. Instead of providing a complete search space for pretext task, APE provides the network with alternative compositions to learn robust and generalized representations. Specifically, two image features $q$ and $k^+$ extracted from two augmented views of a same image serve as a query \cite{MoCo} and a key. The set $\{k^-\}$ is made up of the outputs from other images as negative samples. For each augmentation composition, InfoNCE \cite{Oord2018RepresentationLW} is applied to maximize the agreement between $q$ and $k^+$:
\begin{equation}
     % \resizebox{0.9\hsize}{!}
     \ell_{q}=-\log{\frac{{\rm exp}(q\cdot k^+/\tau)}{{\rm exp}(q\cdot k^+/\tau)+\sum_{k^-}{\rm exp}(q\cdot k^-/\tau)} } 
\tag{3} 
\end{equation}

\noindent where $\tau$ is a temperature parameter and all the embeddings are $\ell_2$ normalized.
% For MoCo v3 implementation, symmetrized loss is adopted as follows\cite{Chen2021AnES, Chen2020ExploringSS}:
% \begin{equation}
%      \resizebox{0.9\hsize}{!}{$\ell_{z_i,z'_i,Q}=-{\rm log}\frac{{\rm exp}(z_i\cdot z'_i/\tau)}{{\rm exp}(z_i\cdot z'_i/\tau)+\sum_{k\in Q}{\rm exp}(z_i\cdot k/\tau)} $ }
% \tag{2} 
% \end{equation}
In our augmentation strategy, each batch is divided into $N$ sub-batches with the sampling probability $p_i$, i.e., $\sum_{i=1}^{N} p_i=1$, which is initialized as $1/N$ for a fair assignment.
The overall loss $\mathcal{L}_z$ of all the augmentation compositions is formulated as follows:

\begin{figure*}[t]
\centering
\includegraphics[width=0.9\textwidth]{figs/loss_1.pdf} % Reduce the figure size so that it is slightly narrower than the column.
\caption{Loss comparison between DreamerV3, encoder with frozen random initialized parameters, encoder with trainable random initialized parameters and APE. The last layer of the frozen random initialized encoder is finetuned during training. The absolute value of actor loss is used.}
\label{figloss}
\end{figure*}

\begin{equation}
    \mathcal{L}_z=
    \begin{matrix} 
    \sum_{i=1}^{n} \ell_{q}p_i
    \end{matrix} 
    \tag{4} 
\end{equation}
\noindent Here $\mathcal{L}_z$ enables the encoder networks to maintain consistency across all sub-batches by utilizing the same key and query encoder.
The closed-loop feedback structure works by utilizing the sampling probability, which is dynamically updated at the end of every epoch by:
\begin{equation}
    \label{qua3}
    p^{t+1}=Softmax(\alpha(1-Acc^t))\tag{5} 
\end{equation}

\noindent where $\alpha$ is set to 0.8 for 7 compositions, and 1 for 3 compositions, thus speeds up the process of exploration when given more augmentation choices. This updating strategy decreases the size of those well-explored compositions and attaches more importance to the ones with lower pretext task accuracy in the next epoch.
\subsection{Policy Learning}




The pretrained encoder projects the high-dimensional image observations $o_t$ into low-dimensional latent features $z_t$, which are then transferred to the downstream agents that learn a control policy. The first three layers of the encoder are frozen to maintain generalization ability while parameters in the last layer are optimized together with the world model to adapt to environments with distribution shifts.

% \subsubsection{Latent dynamics.}
\label{wm}

\begin{figure}[t]
\centering
\includegraphics[width=0.9\columnwidth]{figs/state1.pdf} % Reduce the figure size so that it is slightly narrower than the column. Don't use precise values for figure width.This setup will avoid overfull boxes.
\caption{Exploration of states space in different phases during policy learning period. Data for 100 environment steps are sampled and visualized by Principal Component Analysis (PCA) in each stage. To compare fairly, axes are set to have identical ranges within the same stage. Thus the larger the state area, the higher the efficiency in exploration.}
\label{figstate}
\end{figure}



All model parameters $\theta$ in the latent dynamics except for the frozen ones in visual encoder's first three layers are optimized end-to-end to minimize the following objectives:

\begin{equation}
\setlength{\abovedisplayskip}{3pt}
\begin{aligned}
    \label{qualoss}
    \mathcal{L}_{rew}(\theta) = &-\log(p_{\theta}^{R}(\hat{r}_t \mid z_t,z_{t-1},a_{t-1}))\\
     \mathcal{L}_{con}(\theta)= &-\log(p_{\theta}^{C}(\hat{c}_t \mid z_t,z_{t-1},a_{t-1}))\\
     \mathcal{L}_{rec}(\theta)= &-\log(g_{\theta}(\hat{x}_t \mid z_t,z_{t-1},a_{t-1}))\\
     \mathcal{L}_{obs}(\theta) = & \beta_1 \max (1, \text{KL}[\text{sg}(f_{\theta}(z_t \mid z_{t-1},a_{t-1},o_t))\\
      & \parallel p_{\theta}^{D}(\hat{z}_t \mid z_{t-1},a_{t-1})])\\
     & + \beta_2 \max (1, \text{KL}[(f_{\theta}(z_t \mid z_{t-1},a_{t-1},o_t))\\
     & \parallel \text{sg}(p_{\theta}^{D}(\hat{z}_t \mid z_{t-1},a_{t-1}))])\\
\end{aligned}
\tag{6}
\end{equation}

\noindent where $\text{sg}(\cdot)$ denotes the stop-gradient operator. The fixed hyperparameters are set to $\beta_1 = 0.5$ and $\beta_2 = 0.1$.  
The overall loss of world model can be formulated as follows:

\begin{equation}
\setlength{\abovedisplayskip}{3pt}
    \label{qua4}
    \mathcal{L}(\theta)=\mathcal{L}_{rew}(\theta) + \mathcal{L}_{con}(\theta) + \mathcal{L}_{rec}(\theta) + \mathcal{L}_{obs}(\theta)
\tag{7}
\end{equation}

% \noindent where the first three losses aims to train the reward predictor, the continue predictor and the decoder respectively, while the observation loss is designed to model the visual inputs. 

\begin{table}[h!]

  \begin{center}
    
    
    \label{table:pre} %table的标号必须在caption后面

    \begin{tabular}{c|c|c} % <-- Alignments: 1st column left, 2nd middle and 3rd right, with vertical lines in between
      \toprule  %添加表格头部粗线
      {Method}& $f_{\rm main}$ & {Acc. $(\%) $}\\
      \midrule
      \midrule
      MoCo v2 & — & 90.84\\
      APE &Jitter  & 91.08\\
       APE & Blur  & \textbf{91.7}\\
        % \midrule
        % MoCo & Res50 & 1  & 92.6 \\
        % MoCo v2 & Res50 & 1 & 93.9 \\
        % APE(Blur) &  Res50 & 7 & \textbf{94.5}\\
        % \midrule
        %  MoCo v3 & Vit-S & 1 & 94.68\\
        %  APE(Blur) & Vit-S & 7& \\

      \bottomrule %添加表格底部粗线
    \end{tabular}
  \end{center}

  \caption{Comparison of different augmentation settings using linear probes on ImageNet-100 validation set. We report top-5 classification accuracy and bold the highest result. 
    % \textit{Augs} indicates the number of augmentation compositions used in the pretraining period, while \textit{Arc} means the architecture of each model.
    }
    \label{table:linear}
\end{table}

Taking a multi-task view, the optimization of latent dynamics can be mainly divided into two parts, namely observation modeling and reward modeling \cite{HarmonyDreamTH}. APE works by contributing to the first modeling task, which is attached more importance in MBRL frameworks.




\begin{figure*}[t]
\centering
\includegraphics[width=0.9\textwidth]{figs/atari1.pdf} % Reduce the figure size so that it is slightly narrower than the column.
\caption{Training curves for Atari 100k benchmarks.}
\label{fig:Atari}
\end{figure*}

\begin{figure}[t]
\centering
\includegraphics[width=0.9\columnwidth]{figs/mem_1.pdf} % Reduce the figure size so that it is slightly narrower than the column.
\caption{Training curves for Memory Maze benchmarks. }
\label{fig:mem}
\end{figure}

\begin{figure}[t]
\centering
\includegraphics[width=0.9\columnwidth]{figs/pieg1.pdf} % Reduce the figure size so that it is slightly narrower than the column. Don't use precise values for figure width.This setup will avoid overfull boxes.
\caption{Comparison of DreamerV3-based and DrQ-v2-based APE against other ResNet pretrained algorithms (RRL and PIE-G), together with SAC:state, which learns on proprioceptive observations. 
The bars sharing the same color family (green, orange, and blue) denote algorithm groups following the same downstream strategy.
The performance gains are calculated based on the RL backbone of each group (SAC:vision, DrQ-v2 and DreamerV3), with APE showing the most significant improvement.
}
% Bars of the same color family, i.e., green, orange and blue, 
% % {\color{green!50!black}{green}}, \textcolor{orange}{orange} and \textcolor{cyan}{blue}, 
% represent a group of algorithms with the same downstream strategy.}
\label{figpieg}
\end{figure}












\section{Experiments}
Several experiments are conducted to evaluate the performance of APE using fixed hyperparameters, with details provided in Appendix A and C. We investigate the following questions: (a) Can APE improve the agent’s generalization ability and sampling efficiency on various visual RL benchmarks? (b) Can APE generalize to both model-
based and model-free methods? (c) Why APE works and how do choice of different settings affects the performance? By default, the encoder uses the ResNet18 architecture \cite{He2015DeepRL}. Results reported are averaged over at least 3 runs.






\subsection{Pretraining Encoders}
% \subsubsection{Technical details}
We pretrain APE on ImageNet-100, a randomly selected subset of the common ImageNet-1k \cite{5206848}, which has also been utilized in pervious works \cite{res21, Zhang2023AdaptiveDA} for pretext tasks.
Dynamic adjustment is made on the applied frequency of five data augmentations, including random color jittering, random grayscale conversion, random gaussian blur, random resized crop and random horizontal flip. Results under linear classification protocol are reported in Table \ref{table:linear}. The augmentation with varying applied frequency during pretraining is denoted as the main augmentation strategy ($f_{\rm main}$). In our method, the default $f_{\rm main}$ is random gaussian blur, which proved to be the most promising setting in AdDA.









% \section{Experimental Results}

% Detailed results for each environments are reported in this section. Thanks to the efficiency of DreamerV3, APE achieves competitive performance with state-of-the-art methods on various RL benchmarks.



\subsection{DMC Results}
Being a widely studied benchmark in control tasks, DMC provides a reasonably challenging and diverse set of environments. 
We evaluate the sample efficiency of APE on DMC vision tasks for 1M environment steps. As shown in Fig.\ref{figdmc}, experiments conducted on those tasks demonstrates that APE benefits from the strong feature extraction capabilities learned from ImageNet, leading to enhanced training efficiency and asymptotic performance when applied to control tasks.
Detailed comparison results of DMC scores are reported in Appendix B. 




We illustrate the corresponding loss curves of \texttt{DMC walker walk} task learned with different encoders in Fig. \ref{figloss}. Encoders with random initialization have the same
network architecture as APE, with frozen or trainable random initialized parameters. Intuitively, a pretrained encoder helps accelerate the convergence of observation loss (shown in Fig. \ref{figloss}(a)), since it provides prior knowledge for extracting visual features. Moreover, model loss demonstrated in Fig. \ref{figloss}(a) indicates that APE also helps in the muti-task optimization of latent dynamics, as the overall model loss with pretrained encoder converges more easily than others. Besides, actor loss in Fig. \ref{figloss}(b) suggests that world model equipped with improved encoder is able to predict better future outcomes of potential actions, and thus speed up the actor's learning process.
Furthermore, by visualizing the states space in Fig. \ref{figstate}, we demonstrate that APE enables more sufficient exploration in states with larger visualization area, thereby enhancing the downstream performance. Visualization of reconstructions is provided in  Appendix B.

\subsection{Results on Other Benchmarks}

Fig. \ref{fig:Atari} indicates that APE achieves better or comparable performance using same hyperparameters on 4 Atari tasks. This environment is often used as a benchmark for investigating data-efficiency in RL algorithms. Following the common setup of Atari 100k, we set the environment steps to 40k in tasks considered. The performance on Atari benchmarks highlights the robustness and generalization capability of APE in various RL settings.

Additional experiments have also been made on Memory Maze, which is a 3D domain of randomized mazes generated from a first-person perspective, which measures the long-term memory of the agent and requires it to localize itself by integrating information over time. In this paper, tasks on Memory Maze are trained for 2M steps due to limited computational resources.
As shown in Fig. \ref{fig:mem}, APE is superior over the DreamerV3 baseline on these tasks that require semantic understanding of the environment, making it a promising candidate for complex real-world applications requiring sophisticated decision-making processes.


\subsection{Comparison with Other Pretrained Algorithms}
As shown in Fig. \ref{figpieg}, we compare the performances of two ResNet pretrained algorithms (RRL and PIE-G) and their base algorithms (SAC:vision and DrQ-v2) on three DMC benchmarks. APE outperforms all those methods on the 100K and 500K environment step benchmarks and achieves comparable performance with SAC:state (an agent that learns directly from states) at 100K environment step. To compare fairly, we reimplement DrQ-v2-based APE (denoted as APE (DrQ-v2)) to show that our findings and approach are not limited to MBRL framework.
The results of SAC:state, RRL and DrQ-v2 are from the paper of RRL \cite{RRL} and DrQ-v2 \cite{Yarats2021MasteringVC}, while the others are reproduced and averaged over at least 3 runs.
Detailed results are reported in the Appendix B.
\subsection{Ablation Studies}
 

\subsubsection{Pretraining does work.}
\begin{figure}[t]
\centering
\includegraphics[width=0.9\columnwidth]{figs/random1.pdf} % Reduce the figure size so that it is slightly narrower than the column. Don't use precise values for figure width.This setup will avoid overfull boxes.
\caption{Choosing suitable pretraining strategy weighs more than increasing the depth of encoder network. We compare APE with random initialized encoder with frozen parameters, random initialized encoder with trainable parameters and DreamerV3. The last layer of the frozen random initialized encoder is finetuned during training. ‘-18’ and ‘-4’ denote the number of layers used in the encoder.}
\label{fig:init random}
\end{figure}

Experiments are conducted to figure out whether deeper encoder helps to extract more discriminative features (shown in Fig. \ref{fig:init random}). \textit{Random Enc}s with frozen or trainable initialized parameters have the same network architecture as APE and are included as baselines to eliminate the effect of varied network size. By comparing the performance of \textit{Random Enc} and DreamerV3, it is important to note that deeper networks do not always guarantee the extraction of better features, which leads to improved performance of APE in our tasks. This underscores the significant role of the pretraining period for RL algorithms.





\subsubsection{Augmentations matter.}

In Fig. \ref{fig:aug}, we focus on the applied frequency of random gaussian blur and random color jittering to investigate the effect of data agumentations on visual representations in RL tasks. We observe that the sampling efficiency varies when changing the augmentation strategy.
Results also indicates that linear probes may serve as a useful metric of pretrained model quality under the same network architecture, with relative findings made on imitation learning \cite{Hu2023ForPV}.

\begin{figure}[t]
\centering
\includegraphics[width=0.9\columnwidth]{figs/aug1.pdf} % Reduce the figure size so that it is slightly narrower than the column. Don't use precise values for figure width.This setup will avoid overfull boxes.
\caption{Different choices of augmentation strategy. APE with random gaussian blur as its main augmentation strategy outperforms other settings.}
\label{fig:aug}

\end{figure}
\begin{figure}[t]
\centering
\includegraphics[width=0.9\columnwidth]{figs/arc1.pdf} % Reduce the figure size so that it is slightly narrower than the column. Don't use precise values for figure width.This setup will avoid overfull boxes.
\caption{Different choices of network architectures. This figure indicates that APE with ResNet18 achieves better results compared with a deeper APE (ResNet50).}
\label{fig:arc}
\end{figure}


\subsubsection{Different choices of architectures.}

As shown in Fig. \ref{fig:arc}, we further explore the impact of architectures in DMC tasks. Following the settings of ResNet18 architecture, we freeze the first three layers of ResNet50 and update the last layer during training.
For a fair comparison, the latent dimension of the three architectures are kept same (4096) and both the ResNet18 and the ResNet50 architecture are pretrained on ImageNet-100 with the same $f_{\rm main}$.
Results demonstrate that the increase of depth and complexity of the network, which lead to more abstract representations, may compromises the performance of the fine-grained control tasks. 
Comparisons with ViT-based pretrained encoder \cite{He2021MaskedAA} are reported in the Appendix B.


\section{Conclusion}

In this paper, we propose APE, a simple yet effective method that implements adaptively pretrained encoder in RL frameworks. Unlike previous methods, APE is pretrained on a wide range of existing real-world images using a dynamic augmentation strategy, which helps the network to acquire more generalizable features in the downstream policy learning period.  
Experimental results show that our method surpasses state-of-the-art visual RL algorithms in learning efficiency and performance across various challenging domains. Besides, APE approaches the performance of state-based SAC in several control tasks, underscoring the effectiveness of augmentation strategy in the pretraining period.

\section{Acknowledgments}
This work was supported in part by the Strategic Priority Research Program of the Chinese Academy of Sciences (CAS)(XDB1010302), CAS Project for Young Scientists in Basic Research, Grant No. YSBR-041 and the International Partnership Program of the Chinese Academy of Sciences (CAS) (173211KYSB20200021).

\bigskip

\bibliography{aaai25}
\clearpage
\onecolumn
\appendix
\setcounter{figure}{0}
\setcounter{table}{0}
{\Large\textbf{Appendix of APE: Efficient Reinforcement Learning through Adaptively Pretrained Visual Encoder}}

\section{A\quad Environment}
% As shown in Fig. \ref{fig:env}, we evaluate our method on the following three reinforcement learning(RL) benchmarks. 
\subsection{DeepMind Control (DMC) Suite \cite{dmc}}
Being a widely used RL benchmark, DMC contains a variety of continuous control tasks with a standardised structure and interpretable rewards. In this paper, we test the effectiveness of our method using DMC vision tasks, where the agent is required to learn low-level locomotion and manipulation skills operating purely from pixels. Visualized observations are in the first line of Fig. \ref{fig:env}.

\subsection{Memory Maze \cite{MemoryMaze}}
Agents in this benchmark is repeatedly tasked to navigate through randomized 3D mazes with various objects to reach. To succeed efficiently, agents must remember object locations, maze layouts, and their own positions. An ideal agent with long-term memory can explore each maze once and quickly find the shortest path to requested targets. 
The visualizations of the environment are shown in the second line of Fig. \ref{fig:env}, with \texttt{Agent Inputs} refers to the first-person perspective inputs for the agent.

\subsection{Atari 100k \cite{Bellemare2012TheAL}}
The Atari 100k task contains 26 video games with up to 18 discrete actions, which are often serve as benchmarks for sample efficiency. Considering frame skipping (4 frames skipped) and repeated actions within those frames, the 100k sample constraint equates to 400k actual game frames. Given the wide domain gap between real-world images and Atari observations (reported in Appendix B), we consider five tasks in our evaluation. Visualized observations are illustrated in the third line of Fig. \ref{fig:env}.

\begin{figure*}[h]
\centering
\includegraphics[width=0.7\textwidth]{figs/env1.pdf} % Reduce the figure size so that it is slightly narrower than the column.
\caption{Tasks across three different domains are included in our paper to evaluate the effectiveness of APE.}
\label{fig:env}
\end{figure*}

% \clearpage

\section{B\quad Additional Results}

\subsection{Comparison with PIE-G \cite{Yuan2022PreTrainedIE}}

We report the performance of APE against other ResNet pretrained algorithm in Table \ref{table:pieg}.
Results indicates that APE have better learning efficiency than other pretrained methods on both 100K and 500K environment step benchmarks and achieves comparable performance with SAC:state \cite{SAC} at 100K environment step. Results are averaged over at least 3 runs.


% \begin{figure}[t]
% \centering
% \includegraphics[width=0.7\columnwidth]{figs/boxing_re1.pdf} % Reduce the figure size so that it is slightly narrower than the column. Don't use precise values for figure width.This setup will avoid overfull boxes.
% \caption{Results on task with multi-item observations.}
% \label{fig:img1}

% \end{figure}





\begin{table*}[h!]
% \scalebox{0.95}{
% \vspace{-10pt}
  \begin{center}
    % \setlength{\abovecaptionskip}{0pt} # 调整间距
    % \setlength{\belowcaptionskip}{3pt}
    
    
    
    % \vspace{2pt}
    \begin{tabular}{l|c|cc|ccc|cc} % <-- Alignments: 1st column left, 2nd middle and 3rd right, with vertical lines in between
      \toprule  %添加表格头部粗线
      Task &\makecell[c]{SAC:\\state}& \makecell[c]{SAC:\\vision} & RRL & DrQ-v2& {PIE-G} & \makecell[c]{APE\\(DrQ-v2)} & DreamerV3 & \makecell[c]{APE\\(DreamerV3)}\\
      \midrule
      \midrule
      \multicolumn{3}{l}{\textit{100K Environment Step}} \\
      \midrule
      Walker Walk & \textbf{891} & 28 & 63 & 169.6 & 336.9 & 428.2 & 635.1 & \textbf{877.2}\\
      Finger Spin  & \textbf{811} & 158.8 & 135 & 325.2 & 539.9& 518.4 & 330 & 716.1\\
       Cup Catch & 746 & 177.5 & 261& 359& 587.9& 734& 410.8 & \textbf{916.8}\\
       \midrule
       \rowcolor{gray!20}
       Mean & \textbf{816}& 121.4& 153& 284.6& 488.2& 560.2& 458.6& \textbf{836.7} \\
       % ($\uparrow$ 134.2 \%)\\
       \midrule
        \multicolumn{3}{l}{\textit{500K Environment Step}} \\
        \midrule
        Walker Walk& \textbf{948} & 34.3 & 148& 704.7&689 & 680.5& \textbf{950.4} & \textbf{943.8}\\
        Finger Spin  & \textbf{923} & 296.8 & 422& 788.6 & \textbf{963.7} & 894.9& 439.2& 742.2\\
        Cup Catch & \textbf{974} & 639.4 & 447& 825.9 & \textbf{947.4}& \textbf{955.8}& 857.6 & \textbf{962.4}\\
        \midrule
        \rowcolor{gray!20}
        Mean & \textbf{948.3} & 323.5& 339& 773.1& 866.7 & 843.7& 749.1& 882.8\\
      \bottomrule %添加表格底部粗线
    \end{tabular}
  \end{center}
  % }
  % \vspace{-10pt}
  \caption{Comparison of APE against other ResNet pretrained algorithms (RRL \cite{RRL} and PIE-G) and their baselines (SAC:vision and DrQ-v2 \cite{Yarats2021MasteringVC}), together with SAC:state, which learns on proprioceptive observations.}
  \label{table:pieg} %table的标号必须在caption后面
\end{table*}

\subsection{DMC Results}

Table \ref{tabledmc} shows the score of APE on DMC control tasks under 1M environment steps, 
compared with other state-of-the-art methods. The results of SAC, CURL, DrQ-v2, and DreamerV3
are from the paper of DreamerV3 \cite{dreamerv3} except for those used for visualization, whose "best" scores are reported, representing the best performance during training.
\begin{table*}[h!]
\centering

\begin{tabular}{l|ccccc}
\toprule
Tasks & SAC & CURL & DrQ-v2 & DreamerV3 & APE(Ours)\\
\midrule
\midrule
% Acrobot Swingup & 5.1 & 5.1 & 128.4  & 210.0\\
Cartpole Balance & \textbf{963.1} & \textbf{979} & \textbf{991.5} & $\textbf{999.8}$ & \textbf{998.8}\\
Cartpole Balance Sparse & \textbf{950.8} & \textbf{981} & \textbf{996.2} & \textbf{1000} & \textbf{1000}\\
Cartpole Swingup & 692.1 & 762.7 & \textbf{858.9} & 819.1 & \textbf{874}\\
Cartpole Swingup Sparse & \textbf{830.5} & 774.3 & 706.9 & $771.3$ & \textbf{845.2}\\
Cartpole Two Poles & 238& 255.4 & 295.8 & \textbf{437.6} & \textbf{482.8}\\
Cheetah Run & 27.2 & 474.3 & \textbf{691} & \textbf{728.7} & \textbf{688.6}\\
Cup Catch & 918.8 & \textbf{982.8} & 931.8 & $\textbf{981}$ & \textbf{985.5}\\
Finger Spin & 350.7 & 399.5 & 846.7 & $588.1$ & \textbf{969.9}\\
Finger Turn Easy & 176.7 & 338 & 448.4 & \textbf{787.7} & 721.6\\
Finger Turn Hard & 70.5 & 215.6 & 220 & \textbf{810.8} & \textbf{772.4}\\
Pendulum Swingup  & 560.1 & 376.4 & \textbf{839.7} &  \textbf{806.3} & \textbf{840.6}\\
% Quadruped Run & 50.5 & 141.5 & 407.0 & 352.3\\
% Quadruped Walk & 49.7 & 123.7 & 660.3 & 352.6 & \textbf{340.4}\\
Reacher Easy & 86.5 & 609.3 & \textbf{910.2} & 898.9 & \textbf{949.9}\\
Reacher Hard &  9.1 &  400.2 & \textbf{572.9} &  499.2 & 386\\
Walker Run & 26.9 & 376.2 & 517.1 & \textbf{757.8} & \textbf{758.2}\\
Walker Stand & 159.3 & 463.5 & \textbf{974.1} & \textbf{976.7} & \textbf{986.6}\\
Walker Walk & 268.9 & 909.4 & 762.9 & $\textbf{979}$ & \textbf{987.5}\\
\midrule
\rowcolor{gray!20}
Mean & 395.6 & 581.1 & 722.8 & \textbf{802.6} & \textbf{828}\\
\bottomrule
\end{tabular}
%}
\caption{DMC scores for visual inputs after 1M environment steps.  }
\label{tabledmc}
\end{table*}

\subsection{Comparisons with ViT-Based Pretrained Encoder}
APE's efficacy lies in its augmentation strategy, outperforming methods merely rely on larger models or datasets. We finetuned MAE \cite{He2021MaskedAA}, a widely pretrained ViT encoder with diverse augmentations, to show APE’s effectiveness in three DMC tasks. Notably, APE achieved better results with much lower training time (19 GPU hours vs. 127.2 GPU hours for MAE). Results are shown in Table \ref{table:mae} (averaged over 3 runs).

\begin{table}[h!]
% \vspace{-10pt}
  \begin{center}
    % \setlength{\abovecaptionskip}{0pt} # 调整间距
    % \setlength{\belowcaptionskip}{3pt}
    
    % \vspace{2pt}
    \begin{tabular}{l|cc} % <-- Alignments: 1st column left, 2nd middle and 3rd right, with vertical lines in between
      \toprule  %添加表格头部粗线
      Task & MAE (ViT)	 & APE (ResNet)\\
      \midrule
      \midrule
      DMC Mean 100K & 783.6 &\textbf{836.7} \\
      DMC Mean 500K & 809.1 & \textbf{882.8}\\

      \bottomrule %添加表格底部粗线
    \end{tabular}
  \end{center}
  % \vspace{-10pt}
  \caption{Comparisons with ViT-based pretrained encoder.}
    \label{table:mae} %table的标号必须在caption后面
\end{table}

\subsection{Visualization of Reconstructions}
As illustrated in Fig. \ref{fig: recon}, APE helps to perform more accurate predictions in the beginning of policy learning period (shown in Stage 1), enabling the agent to learn successful behaviors with fewer environment steps: APE manages to walk in Stage 2 while DreamerV3 struggles until Stage 3.


\begin{figure*}[h]
\centering
\includegraphics[width=0.65\textwidth]{figs/recon1.pdf} % Reduce the figure size so that it is slightly narrower than the column.
\caption{Visualization of reconstructions in different phases during policy learning period of \texttt{DMC walker walk}. The first row in each stage shows the real states of the agent, while the second row depicts the predictions reconstructed by the latent dynamics. The third row displays the prediction accuracy by comparing the actual states' outline with the predicted ones.}
\label{fig: recon}
\end{figure*}



\subsection{Further Exploration on Atari Benchmarks}
\begin{wrapfigure}{R}{8cm}
\centering
\includegraphics[width=0.25\textwidth]{figs/boxing_re1.pdf}
\caption{Results on task with multi-item observations.}
\label{fig:img1}
\end{wrapfigure}
We further explore the slight performance decrease of APE on several Atari tasks, e.g., \texttt{Atari Boxing} (shown in Fig. \ref{fig:img1}), where the agent is tasked to fight an opponent in a boxing ring. As illustrated in Fig. \ref{fig: boxing}, we visualize the features of different types of pretraining strategy to explore the generality of image classification models. 
For tasks with such challenging domain gap, APE achieves competitive results as supervised pretrained model, which is trained with a larger variety of images, i.e., ImageNet-1k. 
However, model with random initialization shows to be more adaptive to distributional shifts, since ImageNet-trained models are biased towards classifying single items instead of recognising multi-item observations, which are common in Atari tasks. In this case, the agent tends to overlook its opponents or targets, leading to a decline in Atari performance. We leave the improvement of multi-item detection in future work.



\begin{figure*}[h]
\centering
\includegraphics[width=0.8\textwidth]{figs/boxing1.pdf} % Reduce the figure size so that it is slightly narrower than the column.
\caption{Visualization of different initialization of ResNet-18 model using LayerCAM \cite{9462463}.}
\label{fig: boxing}
\end{figure*}




% \clearpage

\section{C \quad Implementation Details}
\subsection{Agent Learning}
The actor and critic networks learn behaviors completely from the representations predicted by the latent dynamics, which produces a imagined sequence of states $s_t$, actions $a_t$, and continuation flags $c_t$. With $T$ represent the imagination horizon, the $\lambda$-return $G_t^\lambda$ \cite{Zhang2023STORMES} is computed as:
\begin{equation}\label{ec-9}
\begin{array}{l}
G_t^\lambda  \buildrel\textstyle.\over= {{r}_t} + \gamma {{ c}_t}\left[ {(1 - \lambda ){V_\psi }({s_{t + 1}}) + \lambda G_{t + 1}^\lambda } \right]\\
G_T^\lambda  \buildrel\textstyle.\over= {V_\psi }({s_T})
\end{array}
\end{equation}

We adopt the agent learning setting of DreamerV3 with the overall loss of the actor-critic algorithm remain unchanged, which can be described as follows \cite{Zhang2023STORMES}:
\begin{equation}
\begin{aligned}
    \mathcal{L}(\phi) = \frac{1}{T} \sum\limits_{t = 1}^T \left[{-\rm sg} \left(\frac{{ {G_T^\lambda-{V_\psi }({s_t})} }}{{\max (1,S)}} \right){\ln \pi_\phi({a_t}\mid{s_t}) } 
    % \right. \\
    % \left.
    - \eta H( {{\pi _\phi}({a_t}\mid{s_t})} ) \right] \\
    \mathcal{L}(\psi ) = \frac{1}{{T}}{ {\sum\limits_{t = 1}^T {\left[ \left({{V_\psi }({s_t}) - \rm sg\left( {G_t^\lambda } \right)} \right)^2  + \left({V_\psi }({s_t})-\rm sg\left(V_\psi^{EMA}(s_t)\right)\right)^2\right]} } }
\end{aligned}
\label{ec-8}
\end{equation}

\noindent where $\eta$ represent the coefficient for entropy loss, $H(\cdot)$ denotes the entropy of the policy distribution. The scale $S$ is used to normalize returns by:
\begin{equation}
    S = Per(G_T^\lambda, 95) - Per(G_T^\lambda, 5)
\end{equation}
\noindent here $Per(\cdot)$ computes an exponentially decaying average of the batch percentile.

Exponential moving average (EMA) is applied on updating the value function to prevent overfitting, which is defined as:
\begin{equation}
    \psi_{t+1}^{EMA} = \sigma\psi_t^{EMA} + (1-\sigma\psi_t)
\end{equation}
\noindent here $\sigma$ denotes the decay rate.
% A discreet regression approach for learning the critic is 
\subsection{Hyper Parameters and Setup for APE}
The pretext task trains for 200 epochs on 4 Nvidia Tesla A40 (48G) GPU servers while the evaluation runs for 100 epochs on 2 Nvidia Tesla A40 (48G) GPU servers. The RL agent is trained on one Nvidia Tesla A40 (48G) GPU server. Both the pretraining and policy learning algorithms are implemented using PyTorch’s packages. 

APE is pretrained on ImageNet-100, which is a subset of the common ImageNet-1k dataset \cite{Deng2009ImageNetAL}. It consists of 100 classes with a total of around 130,000 natural images, with each class containing roughly 1,000 images. This subset is often used for benchmarking and evaluating computer vision algorithms and models due to its diverse range of object categories and large number of images.




\begin{table}[h!]
% \vspace{-10pt}
  \begin{center}
    % \setlength{\abovecaptionskip}{0pt} # 调整间距
    % \setlength{\belowcaptionskip}{3pt}
    
    % \vspace{2pt}
    \begin{tabular}{l|cc} % <-- Alignments: 1st column left, 2nd middle and 3rd right, with vertical lines in between
      \toprule  %添加表格头部粗线
      Environment & Action Repeat & Train Ratio\\
      \midrule
      \midrule
      DeepMind Control (DMC) & 2 &512 \\
      Memory Maze & 2 & 512\\
       Atari 100k & 4 & 1024\\

      \bottomrule %添加表格底部粗线
    \end{tabular}
  \end{center}
  % \vspace{-10pt}
  \caption{APE list of hyperparameters for each task.}
    \label{table:repeat} %table的标号必须在caption后面
\end{table}

\begin{table}[h!]
% \vspace{-10pt}
  \begin{center}
    % \setlength{\abovecaptionskip}{0pt} # 调整间距
    % \setlength{\belowcaptionskip}{3pt}
    
    % \vspace{2pt}
    \begin{tabular}{l|c} % <-- Alignments: 1st column left, 2nd middle and 3rd right, with vertical lines in between
      \toprule  %添加表格头部粗线
      {Hyperparameter} & Setting\\
      \midrule
      \midrule
     Input dimension & $3\times 224 \times 224$ \\
      Optimizer & SGD\\
       Learning rate & Res18\\
       \midrule
        \multicolumn{2}{l}{\textit{Pretext task}}\\
       \midrule
        Batch size & 128\\
        Learning rate & 3e-2\\
        Momentum & 0.999\\
        Weight decay & 1e-4\\
        Temperature & 0.2\\
        Queue & 65536\\
        \midrule
        \multicolumn{2}{l}{\textit{Linear Classification}}\\
       \midrule
       Batch size & 256\\
       Learning rate & 30\\
       Weight decay & 0\\
        \midrule
        \multicolumn{2}{l}{\textit{Data Augmentation}}\\
       \midrule
        $f_{Jitter}$ & 0.6, 0.7, 0.8 (default: 0.8)\\
        $f_{Blur}$ & 0, 0.2, 0.4, 0.5, 0.6, 0.8, 1 (default: 0.5)\\
        $f_{Flip}$ & 0.5 (default: 0.5)\\
        $f_{gray}$ & 0.2 (default: 0.2)\\
        Brightness delta& 0.4\\
        Contrast delta& 0.4 \\
        saturation delta& 0.4\\
        Hue delta & 0.1\\
      \bottomrule %添加表格底部粗线
    \end{tabular}
  \end{center}
  % \vspace{-10pt}
  \caption{APE list of hyperparameters in pretraining period.}
    \label{table:pre} %table的标号必须在caption后面
\end{table}

\begin{table}[h!]
% \vspace{-10pt}
  \begin{center}
    % \setlength{\abovecaptionskip}{0pt} # 调整间距
    % \setlength{\belowcaptionskip}{3pt}

    % \vspace{2pt}
    \begin{tabular}{l|c} % <-- Alignments: 1st column left, 2nd middle and 3rd right, with vertical lines in between
      \toprule  %添加表格头部粗线
      {Hyperparameter} & Setting\\
      \midrule
      \midrule
      Replay capacity & 1e6\\
     Input dimension & $3\times 64 \times 64$ \\
      Optimizer & Adam\\
      Batch size & 16\\
      Batch length & 64\\
      Policy and reward MPL number of layers & 2\\
      Policy and reward MPL number of units & 512\\
      Strides of the fourth layer for Res18 & 1, 1, 1, 1\\
      Strides of the fourth layer for Res50 & 1, 1, 1, 2\\
       \midrule
        \multicolumn{2}{l}{\textit{World Model}}\\
       \midrule
        RSSM number of units & 512\\
        Learning rate & 1e-4\\
        Adam epsilon & 1e-8\\
        Gradient clipping & 1000\\
        \midrule
        \multicolumn{2}{l}{\textit{Actor Critic}}\\
       \midrule
       Imagination horizon & 15\\
       Learning rate & 3e-5\\
       Adam epsilon & 1e-5\\
       Gradient clipping & 100\\
      \bottomrule %添加表格底部粗线
    \end{tabular}
  \end{center}
  % \vspace{-10pt}
  \caption{APE list of hyperparameters in policy learning period.}
    \label{table:policy} %table的标号必须在caption后面
\end{table}

\begin{algorithm}[tb]
\caption{APE’s main training algorithm}
\label{alg:algorithm}
\textcolor{gray}{\texttt{//Adaptive Pretraining period}} \\
Initialize sampling probabilities ${\{p_i\}}_{i=1}^N$:

\qquad \qquad \qquad${p_1} = {p_2} = ... = {p_N}$

\begin{algorithmic}[1]
% \For{\rm{\textbf{all}} \rm{training epoch}}\\
% \STATE compute

% \ENDFOR
\FORALL {training epoch}
    \STATE compute the size of each sub-batch:\\
    $numbe{r_ - }dat{a_i} = soft\max (\alpha {p_i}) \times nu{m_ - }X$
    \STATE update samplers and resample sub-batches;
    \FORALL {sub-batches}
    \STATE draw two augmentation functions ${\Gamma _i}$ and ${{\Gamma '}_i}$;
    \STATE transform and map the training example;
    \STATE compute ${\mathcal{L}_z}$ and measure similarity;
    \STATE update networks to minimize ${\mathcal{L}_z}$;
    \STATE save the pretext task accuracy $ac{c_i}$;
    \ENDFOR
    \STATE update sampling probability for each sub-batch:\\
\qquad    $p_i^{t + 1} = Softmax(\alpha(1 - Acc_i^t))$
\ENDFOR 
\end{algorithmic}

\textcolor{gray}{\texttt{// Policy learning period}}\\
%Initialize world model parameters $\theta$, actor parameters $\phi$, critic parameters $\psi$.
Initialize critic ${V_\psi }$ and actor ${\pi _w}$ and model ${M^\Delta }$

Loading pretrained encoder ${Encoder }$ with parameters $\varphi$
\begin{algorithmic}[1] %[1] enables line numbers

\FORALL {$e=1,\cdots,E$}
            \STATE get initial state ${s_1} = {Encoder_\varphi }({o_1})$
            \FORALL {$t=1,\cdots,T$}
                \STATE obtain the latent feature ${s_t} = {Encoder_\varphi }({o_t})$
                \STATE apply action ${a_t} \sim {\pi _w}({a_t}|{s_t})$
                \STATE observe $s_{t+1}$ and $r_t$
                \STATE save transition $({s_t},{a_t},{s_{t + 1}},{r_t})$ in $\Re $
                \STATE generate $B$ random imaginary transitions of length $D$ starting from $s_t$ using ${M^\Delta }$
                \STATE store the imaginary transitions in $I$
                
                \FORALL {$k=1,\cdots,U_M$}
                    \STATE train ${M^\Delta }$ on minibatch from $\Re $
                \ENDFOR
                
                \FORALL {$k=1,\cdots,U_I$}
                    
                    \STATE train $\psi$ and $w$ on minibatch from $I $
                \ENDFOR       
            \ENDFOR
        \ENDFOR  


\end{algorithmic}
\end{algorithm}
The DreamerV3-based APE is bulit upon the PyTorch DreamerV3 codebase\footnote{https://github.com/NM512/dreamerv3-torch} while the DrQ-v2-based APE is bulit upon the official PIE-G codebase\footnote{https://github.com/gemcollector/PIE-G/tree/master}.
Algorithm \ref{alg:algorithm} summarizes the training phase of APE. 
Hyperparameters for each task is provided in Table \ref{table:repeat}.
Moreover, we list the hyperparameters of the pretraining period and the policy learning period in Table \ref{table:pre} and Table \ref{table:policy} respectively.

\subsection{Hyper Parameters and Setup for Baselines}

Our PyTorch SAC implementation is based off of the official codebase\footnote{https://github.com/denisyarats/pytorch\_sac\_ae} of SAC+AE without decoder and thus achieves better performance than the common pixel SAC. The size of replay buffer for PIE-G is decreased to 50000 due to limited computational resources. The result of MoCo v2 with ResNet18 is bulit upon the official MoCo v2 codebase\footnote{https://github.com/facebookresearch/moco}.
\end{document}

