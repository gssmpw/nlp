
\def\thefootnote{\arabic{footnote}}
\setcounter{footnote}{0}

\section{Introduction}
\label{sec:intro}


Autoregressive (AR) models aim to learn the probability distribution of the next token, offering great flexibility by generating tokens of any length. This design brings significant advancements in the field of Natural Language Processing (NLP), demonstrating satisfactory generality and transferability \cite{gpt3, chatgpt, gpt4}.
Concurrently, the computer vision field has been striving to develop large autoregressive models \cite{lu2022unified, lu2024unified, bai2024sequential, team2024chameleon, luo2024open, ren2025videoworld}. These models employ visual tokenizers to discretize images into a series of 1D tokens \cite{razavi2019generating, lee2022autoregressive, yu2021vector, zheng2022movq, llamagen} or 2D scales \cite{var, zhang2024var, tang2024hart, ren2024m, li2024imagefolder} and then utilize AR to model the next unit.
However, these image autoregressive models typically output images at a single resolution, \textbf{the flexibility of AR has not yet been realized.}


Recently, in image generation, VAR \cite{var} has pioneered scale-wise autoregressive modeling, completing image autoregression based on 2D sequences. This approach predicts the next scale rather than the next token, thereby preserving the 2D structure of images and mitigating the issue of limited receptive fields in 1D causal transformers.
Specifically, VAR predicts the ground-truth (GT)\footnote{To avoid confusion, we use ground-truth (GT) to represent image latent feature, and residual to represent residual latent feature.} of the smallest scale in the first step. Subsequently, at each step, it predicts the residuals of the current scale and the prior one. Finally, the outputs of each scale are upsampled to a uniform size and undergo weighted summation to generate the final output, as illustrated in Fig. \ref{fig:intro-var}(a). Successors \cite{zhang2024var, tang2024hart, ren2024m, li2024imagefolder} have all adopted the residual design, assuming it to be effective.
Although this technique achieves commendable performance, it encounters a primary challenge: The residual prediction relies on a rigid step design, restricting the flexibility to generate images with varying resolutions and aspect ratios, thus limiting the adaptability and flexibility of image generation.
Meanwhile, residuals at different scales often lack semantic continuity, and this implicit prediction approach may limit the model's capacity to represent diverse image variations.


In this work, we examine the necessity of residual prediction
in visual autoregressive modeling. Our intuition is that, 
in scale-wise autoregressive modeling, the ground-truth value of the current scale can be reliably estimated from the prior series of scales, rendering residual prediction (\textit{i.e.}, predicting the bias between the current scale and the preceding one) unnecessary.
Notably, predicting GT ensures semantic coherence between adjacent scales, making it more conducive for modeling the probability distribution of the scale. Additionally, this structure can output reasonable results at any step, breaking the rigid step design of the residual prediction and endowing autoregressive modeling with great flexibility.

% However, \textit{directly transferring the residual prediction to ground-truth prediction is not feasible}: 
% Existing VQVAE tokenizers usually do not support scaling the latent features in VQVAE to a range of small to large scales, nor do they provide reliable prior information for visual autoregressive modeling.
% % existing VQVAE tokenizers (e.g., \cite{llamagen, aim}) usually do not support image reconstruction at arbitrary scales and typically exhibit good reconstruction quality only at one single scale. 
% In Fig. \ref{fig:intro-vae}, we follow the next-scale-prediction paradigm to reconstruct multi-scale images by scaling the latent features in VQVAE tokenizers. The VQVAE tokenizer from Llamagen \cite{llamagen} only shows excellent reconstruction performance at one single resolution. When scaling in the latent features, the reconstruction quality significantly degrades, indicating that it is not feasible to use existing VQVAE tokenizers for visual autoregressive modeling.

\section{Introduction}


\begin{figure}[t]
\centering
\includegraphics[width=0.6\columnwidth]{figures/evaluation_desiderata_V5.pdf}
\vspace{-0.5cm}
\caption{\systemName is a platform for conducting realistic evaluations of code LLMs, collecting human preferences of coding models with real users, real tasks, and in realistic environments, aimed at addressing the limitations of existing evaluations.
}
\label{fig:motivation}
\end{figure}

\begin{figure*}[t]
\centering
\includegraphics[width=\textwidth]{figures/system_design_v2.png}
\caption{We introduce \systemName, a VSCode extension to collect human preferences of code directly in a developer's IDE. \systemName enables developers to use code completions from various models. The system comprises a) the interface in the user's IDE which presents paired completions to users (left), b) a sampling strategy that picks model pairs to reduce latency (right, top), and c) a prompting scheme that allows diverse LLMs to perform code completions with high fidelity.
Users can select between the top completion (green box) using \texttt{tab} or the bottom completion (blue box) using \texttt{shift+tab}.}
\label{fig:overview}
\end{figure*}

As model capabilities improve, large language models (LLMs) are increasingly integrated into user environments and workflows.
For example, software developers code with AI in integrated developer environments (IDEs)~\citep{peng2023impact}, doctors rely on notes generated through ambient listening~\citep{oberst2024science}, and lawyers consider case evidence identified by electronic discovery systems~\citep{yang2024beyond}.
Increasing deployment of models in productivity tools demands evaluation that more closely reflects real-world circumstances~\citep{hutchinson2022evaluation, saxon2024benchmarks, kapoor2024ai}.
While newer benchmarks and live platforms incorporate human feedback to capture real-world usage, they almost exclusively focus on evaluating LLMs in chat conversations~\citep{zheng2023judging,dubois2023alpacafarm,chiang2024chatbot, kirk2024the}.
Model evaluation must move beyond chat-based interactions and into specialized user environments.



 

In this work, we focus on evaluating LLM-based coding assistants. 
Despite the popularity of these tools---millions of developers use Github Copilot~\citep{Copilot}---existing
evaluations of the coding capabilities of new models exhibit multiple limitations (Figure~\ref{fig:motivation}, bottom).
Traditional ML benchmarks evaluate LLM capabilities by measuring how well a model can complete static, interview-style coding tasks~\citep{chen2021evaluating,austin2021program,jain2024livecodebench, white2024livebench} and lack \emph{real users}. 
User studies recruit real users to evaluate the effectiveness of LLMs as coding assistants, but are often limited to simple programming tasks as opposed to \emph{real tasks}~\citep{vaithilingam2022expectation,ross2023programmer, mozannar2024realhumaneval}.
Recent efforts to collect human feedback such as Chatbot Arena~\citep{chiang2024chatbot} are still removed from a \emph{realistic environment}, resulting in users and data that deviate from typical software development processes.
We introduce \systemName to address these limitations (Figure~\ref{fig:motivation}, top), and we describe our three main contributions below.


\textbf{We deploy \systemName in-the-wild to collect human preferences on code.} 
\systemName is a Visual Studio Code extension, collecting preferences directly in a developer's IDE within their actual workflow (Figure~\ref{fig:overview}).
\systemName provides developers with code completions, akin to the type of support provided by Github Copilot~\citep{Copilot}. 
Over the past 3 months, \systemName has served over~\completions suggestions from 10 state-of-the-art LLMs, 
gathering \sampleCount~votes from \userCount~users.
To collect user preferences,
\systemName presents a novel interface that shows users paired code completions from two different LLMs, which are determined based on a sampling strategy that aims to 
mitigate latency while preserving coverage across model comparisons.
Additionally, we devise a prompting scheme that allows a diverse set of models to perform code completions with high fidelity.
See Section~\ref{sec:system} and Section~\ref{sec:deployment} for details about system design and deployment respectively.



\textbf{We construct a leaderboard of user preferences and find notable differences from existing static benchmarks and human preference leaderboards.}
In general, we observe that smaller models seem to overperform in static benchmarks compared to our leaderboard, while performance among larger models is mixed (Section~\ref{sec:leaderboard_calculation}).
We attribute these differences to the fact that \systemName is exposed to users and tasks that differ drastically from code evaluations in the past. 
Our data spans 103 programming languages and 24 natural languages as well as a variety of real-world applications and code structures, while static benchmarks tend to focus on a specific programming and natural language and task (e.g. coding competition problems).
Additionally, while all of \systemName interactions contain code contexts and the majority involve infilling tasks, a much smaller fraction of Chatbot Arena's coding tasks contain code context, with infilling tasks appearing even more rarely. 
We analyze our data in depth in Section~\ref{subsec:comparison}.



\textbf{We derive new insights into user preferences of code by analyzing \systemName's diverse and distinct data distribution.}
We compare user preferences across different stratifications of input data (e.g., common versus rare languages) and observe which affect observed preferences most (Section~\ref{sec:analysis}).
For example, while user preferences stay relatively consistent across various programming languages, they differ drastically between different task categories (e.g. frontend/backend versus algorithm design).
We also observe variations in user preference due to different features related to code structure 
(e.g., context length and completion patterns).
We open-source \systemName and release a curated subset of code contexts.
Altogether, our results highlight the necessity of model evaluation in realistic and domain-specific settings.







Motivated by this, we systematically design the paradigm of visual autoregressive modeling without residual prediction, referred to as FlexVAR. Within FlexVAR, the ground-truth is predicted at each step instead of the residuals. 
Specifically, we design a scalable VQVAE tokenizer with multi-scale constraints, enhancing the VQVAE's robustness to various latent scales and thereby enabling image reconstruction at arbitrary resolutions. Then, the FlexVAR Transformer learns the probability distribution of a series of multi-scale latent features, modeling the ground-truth of the next scale, as shown in Fig. \ref{fig:intro-var}(b).
Additionally, we propose scalable 2D positional embeddings, which incorporate 2D learnable queries initialized with 2D sin-cosine weights. This approach enables the scale-wise autoregressive modeling to be extended to various resolutions/steps, including those beyond the resolutions/steps used during training, as shown in Fig. \ref{fig:abs}. 


In a nutshell, this non-residual modeling approach ensures continuous semantic representation between adjacent scales. Simultaneously, it avoids the rigid step design inherent in residual prediction, significantly expanding the flexibility of image generation. FlexVAR can (1) generate images of various resolutions and aspect ratios, even exceeding the training resolutions; (2) support image-to-image tasks such as in/out-painting, image refinement, and image expansion without the need for fine-tuning; (3) enjoy flexible inference steps, allowing for accelerated inference with fewer steps or improved image quality with more steps.
% refer fig1


% vae trade-off
