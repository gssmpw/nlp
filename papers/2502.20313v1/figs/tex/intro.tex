
\begin{figure}[t]
\begin{center}
   \includegraphics[width=0.99\linewidth]{figs/pdf/intro-all.pdf}
\end{center}
\vspace{-5mm}
   \caption{
    Comparison between VAR \cite{var} and our FlexVAR. VAR predicts the GT\textsuperscript{\ref{sec:intro}} in step 1 and then predicts the residuals relative to the GT in all subsequent steps. Our FlexVAR predicts the GT at each step.
   %  % Comparison between VAR \cite{var} and our FlexVAR. VAR predicts the ground truth (GT) in step 1 and then predicts the residuals relative to the GT in all subsequent steps. Our FlexVAR predicts the GT at each step
   }
\label{fig:intro-var}
\end{figure}

% \begin{figure*}[t]
%     \centering
%     \begin{minipage}{0.48\textwidth}
%         \centering
%         \includegraphics[width=\textwidth]{figs/pdf/intro-var.pdf}
%         \caption{
%         Comparison between VAR \cite{var} and our FlexVAR. VAR predicts the ground truth (GT) in step 1 and then predicts the residuals relative to the GT in all subsequent steps. Our FlexVAR predicts the GT at each step.
%         }
%         \label{fig:intro-var}
%     \end{minipage}
%     \hfill
%     \begin{minipage}{0.48\textwidth}
%         \centering
%         \includegraphics[width=\textwidth]{figs/pdf/intro-vae.pdf}
%         \caption{
%         Compared with VQVAE tokenizers \cite{var, llamagen} for multi-scale reconstructing images, we downsample the latent features in VQVAE to multiple scales and then use the VQVAE Decoder to reconstruct images. We upsample images $<$ 100 pixels using bilinear interpolation for better visual quality.
%         }
%         \label{fig:intro-vae}
%     \end{minipage}
% \end{figure*}


