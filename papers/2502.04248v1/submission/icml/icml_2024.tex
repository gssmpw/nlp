%%%%%%%% ICML 2024 EXAMPLE LATEX SUBMISSION FILE %%%%%%%%%%%%%%%%%

\documentclass{article}

% Recommended, but optional, packages for figures and better typesetting:
\usepackage{microtype}
\usepackage{graphicx}
\usepackage{subfigure}
\usepackage{booktabs} % for professional tables

% hyperref makes hyperlinks in the resulting PDF.
% If your build breaks (sometimes temporarily if a hyperlink spans a page)
% please comment out the following usepackage line and replace
% \usepackage{icml2024} with \usepackage[nohyperref]{icml2024} above.
\usepackage{hyperref}


% Attempt to make hyperref and algorithmic work together better:
\newcommand{\theHalgorithm}{\arabic{algorithm}}

% Use the following line for the initial blind version submitted for review:
\usepackage{icml2024}

% If accepted, instead use the following line for the camera-ready submission:
% \usepackage[accepted]{icml2024}

% For theorems and such
\usepackage{amsmath}
\usepackage{amssymb}
\usepackage{mathtools}
\usepackage{amsthm}
\usepackage{subcaption}


% if you use cleveref..
\usepackage[capitalize,noabbrev]{cleveref}

%%%%%%%%%%%%%%%%%%%%%%%%%%%%%%%%
% THEOREMS
%%%%%%%%%%%%%%%%%%%%%%%%%%%%%%%%
\theoremstyle{plain}
\newtheorem{theorem}{Theorem}[section]
\newtheorem{proposition}[theorem]{Proposition}
\newtheorem{lemma}[theorem]{Lemma}
\newtheorem{corollary}[theorem]{Corollary}
\theoremstyle{definition}
\newtheorem{definition}[theorem]{Definition}
\newtheorem{assumption}[theorem]{Assumption}
\theoremstyle{remark}
\newtheorem{remark}[theorem]{Remark}


% Todonotes is useful during development; simply uncomment the next line
%    and comment out the line below the next line to turn off comments
%\usepackage[disable,textsize=tiny]{todonotes}
\usepackage[textsize=tiny]{todonotes}
\newcommand{\sophie}[1]{\textcolor{blue}{{\bf sophie:}  #1}}
\newcommand{\probname}{Continual Adaptive Robustness}
\newcommand{\probabbrv}{CAR}
\newcommand{\christian}[1]{\textcolor{green}{{\bf christian:} #1}}

\DeclareMathOperator*{\argmax}{arg\,max}
\DeclareMathOperator*{\argmin}{arg\,min}
\usepackage{enumitem}
\setlist[itemize]{leftmargin=*}

% The \icmltitle you define below is probably too long as a header.
% Therefore, a short form for the running title is supplied here:
\icmltitlerunning{Submission and Formatting Instructions for ICML 2024}

\begin{document}

\twocolumn[
\icmltitle{Adapting to New Test-time Attacks with Effective Pre-training and Fine-tuning}

% Continual Adaptation to New Test-Time Attacks

% It is OKAY to include author information, even for blind
% submissions: the style file will automatically remove it for you
% unless you've provided the [accepted] option to the icml2024
% package.

% List of affiliations: The first argument should be a (short)
% identifier you will use later to specify author affiliations
% Academic affiliations should list Department, University, City, Region, Country
% Industry affiliations should list Company, City, Region, Country

% You can specify symbols, otherwise they are numbered in order.
% Ideally, you should not use this facility. Affiliations will be numbered
% in order of appearance and this is the preferred way.
\icmlsetsymbol{equal}{*}

\begin{icmlauthorlist}
\icmlauthor{Firstname1 Lastname1}{equal,yyy}
\icmlauthor{Firstname2 Lastname2}{equal,yyy,comp}
\icmlauthor{Firstname3 Lastname3}{comp}
\icmlauthor{Firstname4 Lastname4}{sch}
\icmlauthor{Firstname5 Lastname5}{yyy}
\icmlauthor{Firstname6 Lastname6}{sch,yyy,comp}
\icmlauthor{Firstname7 Lastname7}{comp}
%\icmlauthor{}{sch}
\icmlauthor{Firstname8 Lastname8}{sch}
\icmlauthor{Firstname8 Lastname8}{yyy,comp}
%\icmlauthor{}{sch}
%\icmlauthor{}{sch}
\end{icmlauthorlist}

\icmlaffiliation{yyy}{Department of XXX, University of YYY, Location, Country}
\icmlaffiliation{comp}{Company Name, Location, Country}
\icmlaffiliation{sch}{School of ZZZ, Institute of WWW, Location, Country}

\icmlcorrespondingauthor{Firstname1 Lastname1}{first1.last1@xxx.edu}
\icmlcorrespondingauthor{Firstname2 Lastname2}{first2.last2@www.uk}

% You may provide any keywords that you
% find helpful for describing your paper; these are used to populate
% the "keywords" metadata in the PDF but will not be shown in the document
\icmlkeywords{Machine Learning, ICML}

\vskip 0.3in
]

% this must go after the closing bracket ] following \twocolumn[ ...

% This command actually creates the footnote in the first column
% listing the affiliations and the copyright notice.
% The command takes one argument, which is text to display at the start of the footnote.
% The \icmlEqualContribution command is standard text for equal contribution.
% Remove it (just {}) if you do not need this facility.

%\printAffiliationsAndNotice{}  % leave blank if no need to mention equal contribution
\printAffiliationsAndNotice{\icmlEqualContribution} % otherwise use the standard text.

\begin{abstract}
\begin{abstract}

LLMs have immense potential for generating plans, transforming an initial world state into a desired goal state. A large body of research has explored the use of LLMs for various planning tasks, from web navigation to travel planning and database querying. However, many of these systems are tailored to specific problems, making it challenging to compare them or determine the best approach for new tasks. There is also a lack of clear and consistent evaluation criteria. Our survey aims to offer a comprehensive overview of current LLM planners to fill this gap. It builds on foundational work by Kartam and Wilkins (1990) and examines six key performance criteria: completeness, executability, optimality, representation, generalization, and efficiency. For each, we provide a thorough analysis of representative works and highlight their strengths and weaknesses. Our paper also identifies crucial future directions, making it a valuable resource for both practitioners and newcomers interested in leveraging LLM planning to support agentic workflows.

\end{abstract}
\end{abstract}

\section{Introduction}
Research on designing attacks against image classification has led to a wide variety of attack types\citep{madry2017towards, XiaoZ0HLS18, LaidlawF19, laidlaw2020perceptual, wasserstein_attacks, wu2020stronger, brown2017adversarial}, such as $\ell_p$-bounded attacks \citep{madry2017towards}, spatial transformation based attacks \citep{XiaoZ0HLS18}, and color shift based attacks \citep{LaidlawF19}.  However, the scope of research on defending against multiple attacks simultaneously is limited in comparison. Existing work on defending against adversarial examples has focused primarily on achieving robustness against a single attack type (primarily $\ell_p$ bounded attacks) \citep{madry2017towards, zhang2019theoretically, croce2020robustbench, gowal2020uncovering, cohen2019certified,  gowal2021improving, sehwag2021robust}.  While a few works have looked at achieving robustness against multiple attacks \citep{MainiWK20, TB19, madaan2020learning, Croce020,croce2022adversarial}, a property we refer to as \emph{multi-robustness} for short, these works all assume that every attack we would like to be robust against is known \textit{a priori} by the defender.  This is unrealistic as it is difficult to model the space of perturbations we would like to be robust to, and over time, new perturbation types may be introduced.  Thus, it is important to consider a temporal dimension to the problem of multi-robustness; specifically, new attacks are introduced \textit{sequentially} and the defender can continuously update the deployed model to be robust against the new attack, while maintaining robustness on the previous attacks.  This dynamic setting of robustness is known as \textit{continual adaptive robustness} (CAR) \citep{dai2024position}.

\textbf{How can continual adaptive robustness be obtained? (\cref{sec: setup})} In CAR, it is important that models can be adapted quickly to new attacks when they are introduced.  While this setting was introduced by \citet{dai2024position}, to the best of our knowledge, no prior works have analyzed and proposed defenses in this setting.  We propose \textit{continual robust training} (CRT) as a natural solution for CAR.  CRT (Figure \ref{fig:overview}) consists of two main parts: (1) initial adversarial training, where we perform adversarial training to obtain robustness against the set of attacks known before deployment, and (2) iterative robust fine-tuning where each time a new attack is introduced post-deployment, we fine-tune the model to gain robustness against the new attack.

%Robustness to adversarial examples has emerged as one of the key requirements for the safe deployment of machine learning (ML) models. Most work on creating robust models has focused on adversarial examples generated via a single attack. Recently, there has been a push to make ML models robust to adversarial examples generated by multiple attacks \citep{MainiWK20, TB19, madaan2020learning, Croce020,croce2022adversarial}, which we refer to as \emph{multi-robustness} for short. Multi-robustness is critical for the practical deployment of robust models since practitioners will be unwilling to deploy an expensively trained model only for it to be vulnerable to the next attack that emerges. \christian{Minor point but I think plenty of practitioners are willing to deploy non-robust models.}

%\textbf{Paragraph 1}: Why should we care about multi-robustness - this is the only way robustness can be made practical. It is also intellectually challenging as it is harder than achieving singly robust models and can inform their design. prior work has however focused on the setting where all the attacks to be defended against are known simultaneoulsy

\begin{figure*}[!t]
	\centering
	\includegraphics[width=\textwidth]{figures/overview_figure_2.pdf}
	\caption{An overview of the problem of adapting to new adversaries (Continual Adaptive Robustness) and our solution framework (Regularized Continual Robust Training).  In this problem, the defender learns about the existence of new attacks sequentially, and at time $t$ aims to achieve robustness against all attacks seen at times $\le t$.  The model is deployed at time $0$ to be robust against an initial set of attacks, new attacks are introduced at times $t_1$, $t_2$, and $t_3$. We propose to performing initial robust training when the first attack (or set of attacks) is available and then use fine-tuning to adapt the model against future attacks within time $\Delta t$.}
	\label{fig:overview}
	\vspace{-20pt}
\end{figure*}

%\arjun{Paragraph below needs a rewrite. Start with multi-robustness being important, and then its relation to the problem of CAR and why we should care about it. Then, propose CRT as a natural solution. But, then ask, how is it best done? That's where our theory and RCRT comes in.}

%Not only does multi-robustness introduce extra computational considerations, it is often unclear whether joint robustness is even possible for attacks with very different constraint sets. Most prior work has focused on training models that are jointly robust against different $\ell_p$ adversaries \citep{MainiWK20, TB19, madaan2020learning}. The critical assumption here is that all the attacks that need to be defended against are known \emph{a priori}. In this paper, we consider the scenario where models are trained to be multi-robust one attack at a time, a paradigm known as Continual Adaptive Robustness \citep{dai2024position}. This is of interest for two reasons: i) new attacks are likely be revealed over time requiring model trainers to have a training strategy that accounts for them; ii) continual training improves efficiency by obviating the need to train from scratch with all attacks for each new attack, and can leverage the benefits of residual robustness, if any. Thus, the key question we ask in this paper is:

%\begin{center}
%\emph{How can a model trainer effectively and efficiently adapt to new attacks?}
%\end{center}

%\textbf{Paragraph 2}: Perhaps a more practical scenario is where different attacks reveal themselves over time. even if you know all attacks, you can benefit from residual robustness and reduce overall training time, and also not hit capacity by trying to train in too many attacks at the same time. Given this, we consider the paradigm of CAR, where the defender trains a model to be robust one attack at a time. The key question we ask is: how can a defender efficiently and effectively adapt to new attacks?

\noindent \textbf{Theoretical improvements via regularization (\cref{subsec: theory}):} A natural question to ask is: what changes to the training procedure can we make to best perform CRT?  The key challenges here are to obtain robust representations before each new attack is introduced that provide as much robustness to unseen attacks as possible, and to fine-tune with minimal degradation in robustness against previous attacks. Our theoretical analysis shows that the difference in robust losses with respect to a pair of attacks is upper bounded by a function of the \textit{adversarial sensitivity} of a model's representation function: the maximal $\ell_2$ distance between the representations of a data point and a perturbed version of that data point, given the two adversarial constraints.
% \sophie{unclear what adversarial sensitivity means, might help to mention L2 to make the connection to next paragraph more clear} 
As a corollary, we show the difference between the robust loss over the union of two attacks and the benign loss has the same upper bound. These results motivate regularizing robust representations to avoid loss degradation against previous attacks and to have acceptable performance on benign data.

\textbf{Regularization Methods (\cref{subsec: methods}):} Our theoretical results motivate the use of regularization in CRT, and thus, we propose Regularized CRT (RCRT) using \emph{adversarial $\ell_2$ regularization} during both robust pre-training and fine-tuning. This regularization term, motivated by our theoretical results, promotes closeness in terms of Euclidean distance between adversarial and benign representations. We also consider two potential variants to improve efficiency and robustness: i) \emph{Random Noise Regularization:} the representations used are generated using \emph{random noise} instead of worst-case adversarial perturbations; ii) \emph{Variation Regularization \citep{dai2022formulating}:} the Euclidean distance between representations resulting from different instantiations of the adversarial perturbation is minimized.

\noindent \textbf{Empirical Findings (\cref{sec: results}):} We conduct experiments on hundreds of attack combinations on 3 vision datasets to rigorously determine the multi-robustness of models trained using RCRT.  Interestingly, we observe that when used in initial training, all variants of regularization generally improve robustness on new attacks.  For example, using regularization based on uniform noise, we can improve robust accuracy on a spatial attack (StAdv \citep{XiaoZ0HLS18}) from 0.79\% to 26.22\% while training only with $\ell_2$ attacks on CIFAR-10.  We further observe that variants of regularization using adversarial perturbations (adversarial $\ell_2$ and variation regularization) generally improve robustness when used during fine-tuning.  For example, when fine-tuning a model initially robust to $\ell_2$ attacks, using adversarial $\ell_2$ regularization can improve robustness on the snow attack~\citep{kaufmann2019testing} by 7.85\% on ImageNette \citep{imagenette} (from 27.64\% to 35.49\%).%We compare these models to those trained using 3 baselines for continual robust training and 3 for simultaneous multi-robust training from prior work. While no single method dominates in all settings, we find that RCRT provides a good balance between performance and efficiency compared to existing methods for CRT. In addition, we find that CRT methods in general approach or even surpass simultaneous multi-robust training, while being much more efficient. \arjun{Maybe add a couple of illustrative numbers here Sophie?}

\noindent \textbf{Looking ahead (\cref{sec: discussion}):} We show how CAR can be effectively achieved using iterative fine-tuning and regularization techniques. However, future work is necessary to better understand how interactions between sets of adversarial constraints impact the joint learnability of different attacks. We hope that our RCRT method will help practitioners ensure that their models remain robust against changing real-world threats, and may be adapted to ensure other desirable properties, such as compliance with changing standards for fairness or privacy. 
% \sophie{privacy and fairness definitions don't change over time as it does with adversarial examples}


%\textbf{Paragraph 3:} The simple answer to this question is to just pre-train with a specific attack and then fine-tune . However, the robust loss on different attacks for the same classifier, even if jointly trained on both attacks, can be quite different. To control how large this term is, we first prove that the difference in the robust loss for different attacks is bounded by the distance between of the learned robust features. 

%\arjun{The direct way to use the theory is to just consider the classifier trained jointly on both type of adversaries, and then the motivation for regularization is clear. But what would be nicer is to say that if you regularize in pre-training, then you reduce the difference in loss for any subsequent classifier}

%\arjun{Why do continual over joint in practice should be the only argument we make. We should not explicitly say that we are trying to beat joint, just get in the ballpark.}



%In this work, we study a new direction in adversarial robustness called \probname ~(\probabbrv) \sophie{add citation once position paper is up on arxiv}.  In \probabbrv, attacks are introduced in a sequential manner after the defender's model is deployed.  The goal of the defender is to quickly adapt the deployed model to be robust against the new attack while retaining robustness against all previous attacks in the sequence.

%To address this problem, we propose using an initial training and fine-tuning framework which involves first performing adversarial training with the initial attack and then fine-tuning for a few epochs on examples using new attacks once they are introduced (Figure \ref{fig:overview}).  We demonstrate that regularization is crucial for this problem \sophie{add more details here}

% Old text
%While existing machine learning (ML) models can achieve high accuracy, they are vulnerable to imperceptible perturbations in the input space called adversarial examples \citep{szegedy2013intriguing}.  These perturbations can be exploited by an attacker in order to cause misclassification, which may be harmful in safety critical areas.

%Prior works in adversarial robustness consider designing defenses for achieving robustness against a single attack type.  For example, many works propose defenses for achieving robustness against an attacker using a bounded $\ell_\infty$ or $\ell_2$ attack \citep{cohen2019certified, madry2017towards, zhang2019theoretically}.  However, these attacks can only capture a small portion of the space of possible perturbations; many works in attack literature have proposed other forms of attacks that are not captured by the $\ell_p$ threat model \citep{XiaoZ0HLS18, LaidlawF19, laidlaw2020perceptual, kaufmann2019testing}.  This motivates the design of a defense that can be efficiently adapted over time to new attacks.

\section{Related Works}

\textbf{Adversarial Attacks and Defenses:} ML models are vulnerable to input-space perturbations known as adversarial examples \citep{szegedy2013intriguing}.  These attacks come in different formulations including $\ell_p$-norm bounded attacks \citep{madry2017towards, carlini2017towards}, spatial transformations \citep{XiaoZ0HLS18}, color shifts \citep{LaidlawF19}, JPEG compression and weather changes \citep{kaufmann2019testing}, bounded Wasserstein distance \citep{wasserstein_attacks, wu2020stronger} as well as attacks based on distances that are more aligned with human perception such as SSIM \citep{GRAGNANIELLO2021142} and LPIPS distances \citep{laidlaw2020perceptual, ghazanfari2023r}.

Despite the wide variety of attacks that have been introduced, defenses against adversarial examples focus mainly on $\ell_{\infty}$ or $\ell_2$-norm bounded perturbations \citep{cohen2019certified, zhang2020towards, madry2017towards, zhang2019theoretically, croce2020robustbench}.  Of existing defenses, adversarial training \citep{madry2017towards}, an approach that uses adversarial examples generated by the attack of interest during training, can most easily be adjusted to different attacks.  In our work, we build off of adversarial training in order to adapt to new adversaries.

\textbf{Training Techniques for Multi-Robustness:}A few prior works have studied the problem of achieving robustness against multiple attacks, under the assumption that all attacks are known a priori.  These include training based approaches \citep{MainiWK20, TB19, madaan2020learning} which incorporate adversarial examples from the threat models of interest (usually the combination of $\ell_1$, $\ell_2$, and $\ell_\infty$ norm bounded attacks) during training.  \citet{Croce020} provides a robustness certificate of all $\ell_p$ norms given certified robustness against $\ell_\infty$ and $\ell_1$ attacks.

Another line of works has looked at defending against attacks that are not known by the defender, which is a problem known as unforeseen robustness.  These techniques are all training-based and include \citet{laidlaw2020perceptual} which proposes training based on LPIPS \citep{zhang2018lpips}, a metric more aligned with human perception than $\ell_p$ distances, and \citet{dai2022formulating, jin2020manifold} which use regularization during training in order to obtain better generalization to unforeseen attacks.  \citet{dai2023multirobustbench} provides a comprehensive leaderboard for the performance of existing defenses against a large variety of attacks at different attack strengths.

Our work differs from these lines of works since we assume that while the defender may not know all attacks a priori, they are allowed to adjust their defense when they become aware of new attacks.  The work most similar to ours is \citet{croce2022adversarial}, which proposes fine-tuning a model robust against one $\ell_p$ attack to be robust against the union of $\ell_p$ attacks.  Specifically, they demonstrate that we can achieve simultaneous multiattack robustness for the union of $\ell_p$ attacks by obtaining robustness against $\ell_1$ and $\ell_{\infty}$ attacks, and thus propose fine-tuning with $\ell_1$ and $\ell_{\infty}$ attacks to achieve this efficiently.  Our work differs from this work since we explore adapting to attacks outside of $\ell_p$ attacks, investigate ways of improving the initial state of the model prior to fine-tuning, and consider adapting to sequences of attacks.

\textbf{Continual Learning:} A similar direction of research is continual learning (CL) in which a set of tasks are learned sequentially with the goal of performing as well as if they were learned simultaneously \citep{wang2023comprehensive}.  Few works have studied CL in conjunction with adversarial ML.  Of these, most works focus on evaluating or improving the robustness of models trained in the CL framework \citep{bai2023towards,9892970, khan2022susceptibility}. The most similar to our work is \citet{wang2023continual} which treats different attacks as tasks and uses approaches in CL in order to sequentially adapt a model against new attacks.  The attacks they consider follow the same threat model (ie. $\ell_{\infty}$ attacks using different optimization procedures to find the adversarial example).  In our work, we investigate adapting to new threat models.

%\textbf{Representation Similarity } \sophie{depending on how large of a portion CKA results make up the experimental section and how much space we have, can move this section into the appendix}
%\citep{cianfarani2022understanding}


\section{\probname}
In this section, we introduce the problem of continual adaptive robustness (CAR)~\cite{dai2024position}, which aims to achieve robustness against new attacks as they are sequentially discovered. We survey existing approaches to this problem, with additional related work included in \cref{appsec: add_rel_work}.  CAR is visualized in Figure \ref{fig:overview}.

\subsection{A Motivating Example}
Consider an entity that wants to deploy a robust ML system. The entity uses recent techniques (\emph{e.g.} adversarial training) to defend their model  against existing attack types (such as $\ell_p$ perturbations) and deploys their model at time $t=0$. At a later time $t_1$, a research group publishes a paper about a new attack type (\emph{e.g.} spatial perturbations \citep{XiaoZ0HLS18}) against which the entity's model is not robust.  Since the ML system has been deployed, the entity would want to \textit{quickly} modify the model to be robust against the new attack while maintaining robustness against previous attacks.  Having a quick update procedure would minimize the time that an attacker can exploit this vulnerability.  Quick adaptation to new attacks is the foundation of continual adaptive robustness (CAR), a problem setting first introduced in a position paper \citep{dai2024position}.  We propose and analyze the first dedicated defense for CAR in this paper.%, and any adversary can take advantage of the new attack type to break the defended model. Thus, the entity needs to develop a new defense.
%Consider an entity which has deployed an ML system involving an image classifier and wants their classifier to predict robustly even in the presence of an adversary. This entity follows the adversarial robustness literature and uses a technique such as adversarial training to secure their model against $\ell_p$ perturbations and deploys their model at time $t=0$.  At some later time $t_1$, a research group publishes a paper about a new attack type (for example, spatial perturbations \citep{XiaoZ0HLS18}) that the entity's image classifier is not robust against.  In this case, what should the entity do?

%The commonly used definition of adversarial robustness is unable to capture these goals since it does not consider the dimension of time in its formulation. \citet{dai2024position} \edit{is a position paper which} introduces a problem setting called continual adaptive robustness (CAR) which models a dynamic setting of robustness where new attack types are \emph{revealed} to the defender sequentially and the defender has the ability to update their model with information of the new attack.  We now provide the problem formulation for CAR and demonstrate how it models the example described in this section.  \edit{In later sections, we will introduce a novel defense framework for this problem which we call continual robust training (CRT) and theoretically and empirically demonstrate how regularized CRT can help improve performance in CAR.}%. Other possible models are discussed in \cref{sec: discussion}.

\subsection{Problem Formulation}
\noindent
\textbf{Notation:} $\mathcal{D} = X \times Y$ denotes a data distribution where $X$ and $Y$ are the support of inputs and labels, respectively.  $\mathcal{H}$ denotes the hypothesis class.  We use $C:X \to \tilde{X}$ to define an adversarial constraint where $\tilde{X}$ is the space of adversarial examples.  $\ell: Y \times Y \to \mathbb{R}$ denotes the loss function.

\noindent
\textbf{Attack sequences:} In CAR~\cite{dai2024position}, different test-time attacks are introduced sequentially (Figure \ref{fig:overview}). Each attack (perturbed input) can be formulated as the maximizer of the loss $P_C(x, y, h) = \argmax_{x' \in C(x)} \ell(h(x), y)$ (within the constraint $C$) and has a corresponding time $T(P_C)$ at which it is discovered by the defender.  We call the set of attacks known by the defender at a given time $t$ the \textit{knowledge set} at time $t$: $K(t) = \{P\ | T(P) \le t\}$. The expansion of $K$ over time models settings such as research groups or a security team discovering new attack types. 

\noindent
\textbf{Goals in CAR:}  A defender in CAR uses a defense algorithm $\mathcal{A}_{\text{CAR}}$ to deploy a model $h_t = \mathcal{A}_{\text{CAR}}(\mathcal{D},K(t),\mathcal{H})$ at each time step $t$. Performance at time $t$ is measured by Union robust loss 
% \anote{Say that this is the Union loss quantity we care about in general, good robustness on all known attack} a
across the knowledge set:
$\mathcal{L}(h, t) = \mathbb{E}_{(x,y)\sim \mathcal{D}} \max_{P \in K(t)} [\ell(P(x,y, h), y)]$.%Overall, there are 3 goals of the defender: (1) achieve good robustness on the set of known attacks, (2) achieve some robustness on previously unseen attacks, and (3) recover quickly from recently introduced attacks.  We now provide a more formal description of what it means for a defense to achieve CAR: \anote{The 3 goals are defined again below the definition, also are these and the definition taken from the position paper? If yes, then clarify}

\begin{definition}[Continual Adaptive Robustness \citep{dai2024position}] Given loss tolerances $\delta_{\text{known}}$ and $\delta_{\text{unknown}}$ with $0 < \delta_{\text{known}} < \delta_{\text{unknown}}$ and grace period $\Delta t$ for recovering from a new attack, a defense algorithm $\mathcal{A}_{\text{CAR}}$ achieves CAR if for all $t > 0$:
\begin{itemize}
    \item When $t - T(P) < \Delta t$ for any attack $P$ and $T(P) < t$, $h_t$ satisfies $\mathcal{L}(h_t, t) \le \delta_{\text{unknown}}$
    \item Otherwise, $\mathcal{L}(h_t, t) \le \delta_{\text{known}}$.
\end{itemize}
\end{definition}

These criteria capture 3 distinct goals for the defender:  (1) The model at time $t$ must \textit{achieve good robustness if no attacks have been introduced recently} (within $\Delta t$ time). This is due to the $\delta_{\text{known}}$ threshold on the robust loss in the second criterion;  (2) If a new attack has occurred within $\Delta t$ period before the current time $t$, the model at time $t$ must \textit{achieve some robustness against the new attack}.  This is modeled by the $\delta_{\text{unknown}}$ threshold in the first criterion.  Since $0 < \delta_{\text{known}} < \delta_{\text{unknown}}$, CAR tolerates a degradation in robustness between the 2 cases; (3) The defense is expected to \textit{recover robustness quickly after new attacks}.  This is modeled by the $\Delta t$ time window; $\Delta t$ time after the introduction of a new attack, the loss threshold changes from $\delta_{\text{unknown}}$ to $\delta_{\text{known}}$.
%\begin{enumerate}
%    \item \textit{Achieve good robustness when all attacks are known}: without the introduction of new attacks, the defender has a lower error tolerance $\delta_{\text{known}}$ for $\mathcal{L}(h_t, K(t))$.  We also note that $\mathcal{L}$ measured is the worst-case over all attacks which ensures that the model must perform well over all attacks.
%    \item \textit{Recover robustness quickly after the introduction of a new attack}: After a new attack is introduced, the defender must recover robustness within $\Delta t$ time since after the $\Delta t$ grace period, the defender must achieve $\mathcal{L}(h_t, K(t)) \le \delta_{\text{known}}$ again.  
%    \item \textit{Residual robustness when a new attack is introduced}: Within the $\Delta t$ grace period after a new attack is introduced, the defender should have $\mathcal{L}(h_t, K(t)) \le \delta_{\text{unknown}}$.  While $\delta_{\text{unknown}}$ is larger than $\delta_{\text{known}}$, meaning we do not expect as much robustness as on the set of known attacks (used by the defense), we still have an upper bound on the allowed error so we hope to have some robustness against the new attack as well.
%\end{enumerate}

%The goal of the defender is to ensure robustness against the set of attacks known by the defender at any time $t\ge 0$ (where $t = 0$ denotes the time at which the initial model is deployed): $K(t) = \{P\ | T(P) \le t\}$.  We call $K(t)$ the defender's knowledge set, and the expansion of $K$ over time models settings such as research groups or a security team discovering new attack types.

%\noindent
%\textbf{Multi-robustness:} Let $$\mathcal{L}(h, K(t)) = \mathbb{E}_{(x,y)\sim \mathcal{D}} \max_{P \in K(t)} [h(P(x)) \ne y]$$ be the adversarial 0-1 loss
%of model $h \in \mathcal{H}$ to the union of attacks in $K(t)$.  In the original formulation of adversarial robustness, the goal of the defender is to choose an algorithm $\mathcal{A}_{\text{AR}}$ which chooses a model $h$ such that $\forall t \ge 0$, $\mathcal{L}(h, K(t)) \le \delta$, where $\delta > 0$ is a small value representing the amount of error that can be tolerated by the application of interest; which would then lead to the defender failing this meet this objective for the first new attack $P$ for which $T(P) > 0$.

%\noindent \textbf{Robust model sequence:} To resolve this, firstly, a defender in CAR chooses an algorithm that not only gives a single model but rather gives a sequence of models $h_t$ for each time step.  Additionally, rather having a single error threshold $\delta$, the CAR problem has two separate error tolerances representing the tolerance when all attacks are known to the defender $\delta_{\text{known}}$ and when a new attack is introduced $\delta_{\text{unknown}}$ with $0 < \delta_{\text{known}} < \delta_{\text{unknown}}$.  $\delta_{\text{unknown}}$ is in effect for a small time window $\Delta t$ representing the grace period for recovering after a new attack is introduced.  More formally,
%\begin{itemize}
%    \item When $t - T(P) < \Delta t$ for any attack $P$ and $T(P) < t$, the defender's algorithm $\mathcal{A}_{\text{CAR}}$ aims to give a $h_t$ such that $\mathcal{L}(h_t, K(t)) \le \delta_{\text{unknown}}$
%    \item Otherwise, $\mathcal{L}(h_t, K(t)) \le \delta_{\text{known}}$.
%\end{itemize}



%Then, the goal of the defender is choose an algorithm $\mathcal{A}$ which $\forall t \ge 0$ outputs a $h_t$ such that, $\mathcal{L}(h_t, K(t)) < \delta$, where $\delta > 0$ is a small value representing the amount of error that can be tolerated by the application of interest. Unlike existing defenses that output a single robust model, $\mathcal{A}$ outputs a model at each time step and is able to use information from $K(t)$ and previous models $\{h_{i}\}_{i < t}$ in order to generate $h_{t + \Delta t}$, where $\Delta t$ is the the time taken to update the model.

%\begin{enumerate}
%    \item Achieve robustness on the new attack while maintaining robustness on previous attacks - given that the objective of the defender is to minimize $\mathcal{L}(h_t, K(t)) = \mathbb{E}_{(x,y)\sim \mathcal{D}} \max_{P \in K(t)} [h_t(P(x)) \ne y]$, the defender needs to optimize for a model that has good performance across the new attack and previous attacks.
%    \item Quickly update vulnerable models - 
%    \item Have robustness to unforeseen attacks - 
%\end{enumerate}

% Unlike existing defense algorithms that focus on outputting a single defended model, a defense algorithm for the CAR problem outputs a model $h_t \in \mathcal{H}$ for each time step.  To choose $h_t$, the defender's algorithm $\mathcal{A}$ can only 

%\textbf{Why is CAR important? } For domains such as vision, it is hard to define the space of all possible attacks, due to which new attack types are periodically discovered \citep{kaufmann2019testing, XiaoZ0HLS18, LaidlawF19}, bypassing previous defenses. It is thus difficult to ensure that a single defended model is robust to all possible future attacks. CAR can be thought of as a framework with the goal of resilience: even if a new attack is introduced, we should be able to quickly update the model to recover from this new attack. A defense for CAR can also be applied with multiple known attacks for efficient training of multi-robust models. %\arjun{Also important even when we know attacks already, we can train one-by-one instead of all at once for efficiency, which could be useful}

\subsection{Baseline approaches to CAR}
\noindent
\textbf{CAR through multiattack robustness (MAR). }Prior works for multiattack robustness often involve training with multiple attacks simultaneously \citep{TB19, MainiWK20}, which can be computationally expensive. A trivial (but expensive) defense algorithm for CAR is to use these training-based techniques and retrain a model from scratch on $K(t)$ every time it changes.  However, this would require us to tolerate larger values of $\Delta t$. % For CAR, inefficiency in updating a model with each new attack is harmful since the adversary can continue to exploit the new attack while the model is being updated.% \arjun{This paragraph should say that sMAR is the framework that all previous work has basically operated in}

\noindent
\textbf{CAR through unforeseen attack robustness (UAR). } Defenses for unforeseen attack robustness (UAR) aim to attacks outside of those used in the design of the defense. Another trivial defense algorithm for CAR is to use a UAR defense \citep{laidlaw2020perceptual,dai2022formulating} to get a model $h$ and use $h$ for all time steps.  This approach is efficient as no time is spent updating the model but would require much higher values of $\delta_{\text{known}}$ as these methods do not obtain high robustness across attacks \citep{dai2023multirobustbench}.%  Prior work  demonstrates that current techniques for UAR performs poorly when evaluated on a wide variety of attacks with best performing approaches achieving only 3\% when considering the worst-case attack of the set. 
% \arjun{Worried about the phrasing here, we might get asked why we don't compare to UAR based defenses}
%\arjun{There needs to be more motivation either here or in the next section about why we ask the theoretical question we do ask. }

\section{Bounds on Joint Robustness}
% In order to achieve a balance between efficiency and robust performance, we propose using an initial robust training and iterative robust fine-tuning pipeline which we call Continual Robust Training (CRT).  We study the impact of using regularization with CRT (RCRT), diagrammed in Figure \ref{fig:overview}.  In this pipeline, we perform (regularized) adversarial training in order to gain robustness against the initial space of known attacks and then utilize fine-tuning every time a new attack is introduced in order to achieve robustness against the new attack. Our approach of RCRT is most similar to that of \citet{croce2022adversarial}, which uses finetuning (without regularization) to gain robustness to unions of $\ell_p$ norms. However, we demonstrate that \textit{the use of regularization in both the initial adversarial training and fine-tuning steps can improve robust performance} and we expand the scope of our experimentation beyond pairs of $\ell_p$ bounded attacks.

% Finetuning (without regularization) for robustness to unions of $\ell_p$ norms has been utilized by \citet{croce2022adversarial}, and the conjunction of adversarial training on the initial set of attacks and \citet{croce2022adversarial}'s finetuning approach can be thought of as an instantiation of CRT.  We distinguish ourselves from this work by demonstrating that \textit{the use of regularization in both the initial adversarial training and finetuning steps can improve robust performance}.  In addition, we study the effectiveness of adapting to attacks outside of $\ell_p$ bounded attacks and look at longer sequences of attacks (\citet{croce2022adversarial} look at finetuning for sequences of 2 attacks).

% We will now show that there is theoretical justification for such an approach. In recent work, \citet{nern2023transfer} present theoretical results supporting the use of regularization in robust learning with a single adversary, showing that the maximum increase in loss incurred by adversarial perturbations is proportional to the adversarial sensitivity of a model's internal representations. Following Nern et al., adversarial sensitivity refers to the adversarial robustness of the representation function (i.e. $\max_{x' \in C(x)}\|g(x') - g(x)\|_2$).
% Building on this framework, we show that these results extend to the setting of multi-robustness, justifying RCRT as valid approach for continual adaptive robustness.

%To approach CAR, we propose Continual Robust Training (CRT), which involves initial robust training against a single attack followed by iterative robust fine-tuning against new attacks. 
% \arjun{Do we want to say something like: `For this to hold, the difference in robust loss with respect to different attacks must be small. We theoretically show...'}
% This dynamic would have the effect of reducing the gap between the adversarial losses of the previously seen attacks and the adversarial loss of the new attack at each iteration.
% We theoretically show that these gaps can be bounded above by a function of the adversarial sensitivity of a model's internal representation function. This suggests a form of regularization which directly minimizes adversarial sensitivity, thereby shrinking the adversarial loss gaps. 
%  Our theoretical results motivate a regularization term which we incorporate into CRT.
%We then propose regularized CRT (RCRT), incorporating a regularization term in both the initial training and fine-tuning losses to minimize the model's sensitivity to each attack, as an enhanced approach for achieving CAR.

% \arjun{These two paragraphs can be shrunk a fair bit: start by saying we approach CAR by proposing CRT. But then we ask, how to improve it? We want to bound the change in loss between different constraints to prevent forgetting etc. We show this change is bounded by representation sensitivity to adversarial perturbations, building on Nern et al, so we ultimately we propose RCRT}

%%%%%%%%%%%%%%%%%%%%%%%%%%%%

% We now present justification for our use of regularization techniques to achieve continual adaptive robustness. 

%It is not immediately clear whether it is possible to train a model that is suitably robust to both types of attack. We provide a theoretical intuition for why joint robustness of this form is achievable.
In this section, we introduce continual robust training (CRT) and provide theoretical results to demonstrate that adding a regularization term bounding adversarial logit distances can help balance performance across a set of adversaries.
\subsection{Continual Robust Training}
We now introduce continual robust training (CRT).  CRT consists of 2 parts, \textit{initial training} and \textit{iterative fine-tuning} (Figure \ref{fig:overview}).  The output of initial training is deployed at $t=0$ while fine-tuning is used as new attacks are introduced.

% \anote{Okay but why can CRT meet the defender's goals? The fine-tuning reduces $\Delta t$, and can satisfy the $\delta_{\text{known}}$ condition given enough time.}

At time $t=0$, the goal of the defender is to minimize the initial training objective: $\mathcal{L}(h,0) = \frac{1}{m}\sum_{i=1}^m\ell(h(P_{C_\text{init}}(x_i, y_i, h)), y_i)$
where $\{(x_i, y_i)\}_{i=1}^m$ is the training dataset and $P_{C_\text{init}}$ is the initial attack. 
Notably, using standard training in this stage yields a high $\delta_\text{unknown}$.
% The goal of the defender is to minimize $L_{\text{init}}$ in initial training.

At $t>0$, as new attacks are introduced, we use a fine-tuning strategy $F$ to select the attack from $K(t)$ to use for each example.  Specifically, we formulate this as:
$    \mathcal{L}(h, t) = \frac{1}{m}\sum_{i=1}^m \ell(h(P_{C}(x_i, y_i, h)), y_i)$ where $P_{C}= F(K(t), (x_i, y_i))$.
Fine-tuning strategies include picking the attack that maximizes $\ell(x_i, y_i)$, randomly sampling from $K(t)$, and using the newest attack.  A good fine-tuning strategy would be able to quickly adapt the model to new attacks, allowing it to satisfy a small $\Delta t$ threshold.
However, naive fine-tuning does not guarantee good performance across all attacks and may require large values of $\delta_\text{known}$. 
% \anote{Cast this in terms of the requirements: too much degradation will cause an unacceptable $\delta_{\text{known}}$} 
As illustrated in \cref{fig:summary}, a model may lose robustness to the initial attack after the fine-tuning stage. We now discuss how such degradation can be addressed through regularization.
% \sophie{point to \cref{fig:summary}, naively doing CRT doesn't guarantee good robust performance for CAR - poor generalization to unforeseen attacks, forgetting of initial attack. Then connect with next section.}

\subsection{Bounding the difference in adversarial losses}\label{subsec: theory}
% \anote{Motivate the terms we consider: we want L1-L2 to be small for two reasons: some unforeseen robustness, which helps meet the $\delta_{\text{unknown}}$ requirement; and prevent degradation on known attacks, which helps meet the $\delta_{\text{known}}$ requirement}

In order for CRT to be a practical solution for CAR, it is important that CRT enhances robustness to new attacks without losing robustness against attacks we have already learned. 
We now relate the gap in robust loss between two attacks to how far each attack can perturb the logits for a given example,
which suggests that regularization in CRT may improve decrease the drop in robustness across attacks. 
%Such a bound can be useful in designing new defenses, as it could correspond to an upper bound on the increase in adversarial loss on a previously learned attack after a new attack is introduced.
%Ideally, we would want to ensure not only that the gap between the losses is minimal, but also that the individual losses are small in absolute terms~\cite{yin2019rademacher}. While our results do not directly ensure low individual losses, they hold for any model that follows our assumptions, including those that perform well against both attacks. Moreover, they suggest that a model whose \edit{logits} are highly sensitive to perturbations will not perform well against both attacks.

% Let $h:\mathbb{R}^d \rightarrow \mathbb{R}^c$ be a $c$ class neural network classification model with a final linear layer (i.e. $h(x) = Wg(x)$, where $g: \mathbb{R}^d \rightarrow \mathbb{R}^r$ is a representation function and $W \in \mathbb{R}^{c \times r}$). 
\edit{Let $h:\mathbb{R}^d \rightarrow \mathbb{R}^k$ be a $k$ class neural network classification model.}
To simplify the problem, we focus on the state of the model when attacks $P_{C_1}$ and $P_{C_2}$ (with corresponding adversarial constraints $C_1$ and $C_2$) are known to the defender. %, 
%although the bounds we derive hold for any model of the above form.
Consider the following two adversarial loss functions:
$\mathcal{L}_1(h) := \mathbb{E}_{\mathcal{D}}\left[\ell(h(P_{C_1}(x,y)),y)\right]$
and
$\mathcal{L}_2(h) := \mathbb{E}_{\mathcal{D}}\left[\ell(h(P_{C_2}(x,y)),y)\right].$
Without loss of generality, assume that $\mathcal{L}_1(h)
\geq \mathcal{L}_2(h)$. We can then bound the difference between $\mathcal{L}_1(h)$ and $\mathcal{L}_2(h)$, \edit{adapting a result from}~\citet{nern2023transfer}, as follows\footnote{\edit{As stated, these results hold for loss functions that are Lipschitz with respect to the $\ell_2$ norm. We note that similar bounds can be derived for other norms by applying a constant scaling factor to the first term of the bound (i.e. for losses Lipschitz with respect to the $\ell_1$ norm, the scaling factor would be $\sqrt{c}$).}}:
\vspace{-3pt}
%We arrive at the following theorem, building on Nern \textit{et. al.}~\cite{nern2023transfer}, bounding the difference between $\mathcal{L}_1(h)$ and $\mathcal{L}_2(h)$:

%In order for CRT to be effective, it must be possible for a model to achieve acceptable loss on multiple attacks. Therefore, we demonstrate theoretically that the loss of a model on additional attacks is bounded by an objective that can be directly minimized. 

%For ease of notation, we will refer to $P_1$ as $\psi$ and $P_2$ as $\omega$.

% , which mirrors Theorem 1 from \cite{nern2023transfer} \sophie{do we need citation here?}:
\begin{theorem}
    \label{thm:robustness}
    Assume that loss $\ell(\hat{y},y)$ is $M_1$-Lipschitz in $\|\cdot\|_2$, for $\hat{y} \in h(X)$ with $M_1 > 0$ and bounded by $M_2 > 0$ \footnote{We note that surrogate losses such as the cross-entropy used during training are not bounded, but the $0-1$ loss which is often the key quantity of interest \emph{is bounded}.}, i.e. $0 \leq \ell(\hat{y},y) \leq M_2$ $\forall \hat{y} \in h(X)$. Then, for a subset $\mathbb{X} = \{x_i\}_{i=1}^n$ independently drawn from $\mathcal{D}$, the following holds with probability at least $1-\rho$:
    \begin{align*}
        \mathcal{L}_1(h) - \mathcal{L}_2(h) \leq \;&M_1 \frac{1}{n}\sum_{i=1}^n\biggl(\max_{x' \in C_1(x_i)}\|h(x') - h(x_i)\|_2 \\
        &+ \max_{x' \in C_2(x_i)}\|h(x') - h(x_i)\|_2\biggl) + D,
    \end{align*}
    where $D = M_2\sqrt{\frac{\log(\rho/2)}{-2n}}$. 
\end{theorem}
This result suggests that regularization with respect to a single attack (say, in pre-training) will give the model greater resiliency against unforeseen attacks and help meet the $\delta_\text{unknown}$ threshold. Using regularization when fine-tuning on a new attack could also prevent degradations in robustness against previously seen attacks, helping to meet the $\delta_\text{known}$ threshold.
Using similar reasoning, we can also bound the gap between Union and clean loss:
\begin{corollary}
\label{thm:corollary}
Let $\mathcal{L}_{1,2}(h) := \mathbb{E}_{\mathcal{D}}\left[\max{(\ell(h(P_{C_1}(x,y,h)),y),\ell(h(P_{C_2}(x,y,h)),y)})\right]$. Then, with probability at least $1 - \rho$,
\begin{align*}
        \mathcal{L}_{1,2}&(h) - \mathcal{L}(h) 
        \leq M_1 \frac{1}{n}\sum_{i=1}^n\biggl(\max_{x' \in C_1(x_i)}\|h(x') - h(x_i)\|_2 \\
        &+ \max_{x' \in C_2(x_i)}\|h(x') - h(x_i)\|_2\biggl) + D.
    \end{align*}
\end{corollary}
This corollary helps characterize the trade-off between clean and robust loss in our setting. Proofs of Theorem \ref{thm:robustness} and Corollary \ref{thm:corollary} are present in Appendix \ref{sec:proof}.
% \begin{theorem}
%     \label{thm:robustness}
%     Assume that loss $\ell(\hat{y},y)$ is $M_1$-Lipschitz in $\|\cdot\|_\alpha$ for $\alpha \in \{1,2,\infty\}$, for $\hat{y} \in h(X)$ with $M_1 > 0$ and bounded by $M_2 > 0$, i.e. $0 \leq \ell(\hat{y},y) \leq M_2$ $\forall \hat{y} \in h(X)$. Then, for a subset $\mathbb{X} = \{x_i\}_{i=1}^n$ independently drawn from $\mathcal{D}$, the following holds with probability at least $1-\rho$:
%     \begin{align*}
%         \mathcal{L}_1&(h) - \mathcal{L}_2(h) \\
%         &\leq L_\alpha(W)M_1 \frac{1}{n}\sum_{i=1}^n\biggl(\max_{x' \in C_1(x_i)}\|g(x') - g(x_i)\|_2 \\
%         &+ \max_{x' \in C_2(x_i)}\|g(x') - g(x_i)\|_2\biggl) + D,
%     \end{align*}
%     % \begin{align*}
%     %     \mathcal{L}_\psi(f_{W,\theta}) &- \mathcal{L}_\omega(f_{W,\theta}) \\
%     %     &\leq L_\alpha(W)C_1 \frac{1}{n}\sum_{i=1}^n\biggl(\max_{\|\delta_\psi\|_\psi \leq \epsilon_\psi}\|g_\theta(x_i + \delta_{\psi}) - g_\theta(x_i)\|_2 \\
%     %     &+ \max_{\|\delta_\omega\|_\omega \leq \epsilon_\omega}\|g_\theta(x_i + \delta_{\omega}) - g_\theta(x_i)\|_2\biggl) + C_2\sqrt{\frac{\log(\rho/2)}{-2n}},
%     % \end{align*}
%     where $D = M_2\sqrt{\frac{\log(\rho/2)}{-2n}}$ and
%     \[
%     L_\alpha(W) := \begin{cases}
%     \|W\|_2 &, \text{if } \|\cdot\|_\alpha = \|\cdot\|_2, \\
%     \sum_i\|W_i\|_2 &, \text{if } \|\cdot\|_\alpha = \|\cdot\|_1, \\
%     \max_i\|W_i\|_2 &, \text{if } \|\cdot\|_\alpha = \|\cdot\|_\infty. 
%     \end{cases}
%     \]
% \end{theorem}

%As a corollary, we can bound the difference in the adversarial loss over the union of two attacks and the benign loss:

%From this result, we can derive a similar bound for the difference in loss on the union of two attacks and loss on unperturbed data:
% \begin{corollary}
% \label{thm:corollary}
% Let $\mathcal{L}_{1,2}(h) := \mathbb{E}_{\mathcal{D}}\left[\max{(\ell(h(P_{C_1}(x,y,h)),y),\ell(h(P_{C_2}(x,y,h)),y)})\right]$. Then, with probability at least $1 - \rho$,
% \begin{align*}
%         \mathcal{L}_{1,2}&(h) - \mathcal{L}(h) \\
%         &\leq L_\alpha(W)M_1 \frac{1}{n}\sum_{i=1}^n\biggl(\max_{x' \in C_1(x_i)}\|g(x') - g(x_i)\|_2 \\
%         &+ \max_{x' \in C_2(x_i)}\|g(x') - g(x_i)\|_2\biggl) + D.
%     \end{align*}
% \end{corollary}

%\begin{corollary}
%\label{thm:corollary}
%Let $\mathcal{L}_{1,2}(h) := \mathbb{E}_{\mathcal{D}}\left[\max{(\ell(h(P_{C_1}(x,y,h)),y),\ell(h(P_{C_2}(x,y,h)),y)})\right]$. Then, with probability at least $1 - \rho$,
%\begin{align*}
%        \mathcal{L}_{1,2}(h) - \mathcal{L}(h) \leq \;&\edit{M_1 \frac{1}{n}\sum_{i=1}^n\biggl(\max_{x' \in C_1(x_i)}\|h(x') - h(x_i)\|_2} \\
%        &\edit{+ \max_{x' \in C_2(x_i)}\|h(x') - h(x_i)\|_2\biggl)} + D.
%    \end{align*}
%\end{corollary}

%\edit{An additional \cref{corollary: deeper_reps} extends these results to bounds in terms of the difference of penultimate layer representations (see Appendix~\ref{sec:proof})}. Put simply, Theorem~\ref{thm:robustness} shows that the gap between $\mathcal{L}_1(h)$ and $\mathcal{L}_2(h)$ is upper bounded by the \textit{adversarial sensitivity} ($\ell_2$ distance between benign and perturbed \edit{logits}) of the model to perturbations satisfying the constraint sets $C_1$ and $C_2$ 
%Corollary~\ref{thm:corollary} shows that the gap between the union robust loss ($\mathcal{L}_{1,2}$) and the benign loss is bounded above by the same term. 

% Our results point towards regularization as a promising approach for achieving robustness to multiple perturbations. 
%In principle, regularization can reduce the distances between learned robust representations for different attack types, directly minimizing the upper bound in the theorem statement. We further discuss regularization in the next section. 

\noindent \textbf{Comparison to \citet{dai2022formulating}: } We note that \citet[Theorem 4.2]{dai2022formulating} derive a related bound on the adversarial loss gap between two attacks in the context of UAR. However, their formulation assumes that the constraint set of the target attack is a strict superset of that of the source attack, whereas we make no assumptions about the relationship between the two constraint sets.
%\edit{
%\begin{remark}
   
%\end{remark}
%}
% Theorem~\ref{thm:robustness} therefore suggests that regularization will decrease the upper bound on the pairwise difference in performance on multiple attacks. Corollary~\ref{thm:corollary} implies this will also hold for the model's performance on the union of two attacks and its performance on benign examples.


% \christian{Maybe remove rest of this paragraph}
% In removing this restriction, our bounds more directly imply that regularization will reduce the robust loss gap both during pretraining (i.e. regularizing just with respect to attack 1) and during fine-tuning to a previously unknown attack (i.e. simultaneously regularizing with respect to attacks 1 and 2). 
% From Theorem~\ref{thm:robustness}, we see that when using a $M$-Lipschitz loss function 
% (which includes common loss functions like softmax cross-entropy loss), the pairwise 
% difference between adversarial losses is bounded above by a function 
% of $g$'s robustness to each of the perturbations, as
% well as the norm of the final linear layer $W$. 
%Beyond providing bounds on the gap in adversarial robustness, this finding suggests a framework for understanding how close existing training methods are to achieving optimal joint robustness.
% This theoretical result points towards regularization as a promising approach for achieving robustness to multiple perturbations. \sophie{needs more intuition here}

% \begin{table}[]
% \centering
% \begin{tabular}{@{}lllll@{}}
% \toprule
%                   & $P_1$ acc & $P_2$ acc & $\ell_2$ distance & CKA   \\ \midrule
% $\ell_2$ (Single) & 0.726        & 0.018     & 439,479.13         & 0.445 \\ \midrule
% MAX (Joint)       & 0.637        & 0.544     & 103,696.29         & 0.958 \\
% AVG (Joint)       & 0.649        & 0.497     & 116,405.75         & 0.930 \\
% RAND (Joint)      & 0.645        & 0.504     & 117,666.86         & 0.947 \\ \bottomrule
% \end{tabular}
% \label{tab:similarity}
% \caption{\christian{Preliminary results} Robust accuracy, $\ell_2$ distance between robust representations of $P_1$ and $P_2$, and CKA similarity between robust representations of $P_1$ and $P_2$ for multi-robust training methods. Here, the size of the adaptive knowledge set is 2, $P_1$ represents an $\ell_2$ attack with $\epsilon = 0.5$ and $P_2$ represents a spatial attack with $\epsilon = 0.05$.}
% \end{table}

% Furthermore, this distance appears roughly correlated with the CKA similarity between the two representations, supporting the assumption that multi-robust training is capable of enforcing similarity between representations of different perturbation types in the same way that normal adversarial training is capable of enforcing similarity between adversarial and benign representations. 
% Lastly, the lowest $\ell_2$ distance and highest CKA similarity is seen in the model that has the smallest gap between $P_1$ and $P_2$ accuracy, providing reassurance that the relevant metrics from our theoretical results and prior work on adversarial robustness correlate with meaningful downstream metrics in the \probabbrv\ setting.

% \sophie{should also discuss a comparison to theoretical results in \citet{dai2022formulating}.}

\section{Regularization for Improving \probabbrv}
\label{sec:regularization}
Theorem \ref{thm:robustness} suggests that reducing the sensitivity of logits to \textit{either} attack has the potential to reduce the performance gap between attacks (see \cref{fig:loss_gap} in the Appendix for an empirical validation of this effect). To this end, we propose incorporating regularization into both training stages.  Specifically, we adopt modified training objective $\mathcal{L}_{\text{reg}}(h,t) = \mathcal{L}(h,t) + \lambda R(h, K(t))$, 
% andx\anote{I think this is now $\mathcal{L}(h,0)$, and respectively for the fine-tuning loss} $L_{\text{fine-tune-reg}}(h, t) = L_{\text{fine-tune}}(h) + \lambda R(h, K(t))$, 
where $\lambda$ is the regularization strength and $R(h)$ is the regularization term used. We will now discuss several forms of regularization.


%We adopt a modified training objective $L(h) = L_{\text{init}}(h) + \lambda R(h)$,
%where $\lambda$ is the regularization strength and $R(h)$ is the regularization term used. 
%\edit{\st{Following the notation of our theoretical results, we define $h$ to be a composition of a feature extractor and a final linear layer (i.e. $h(x) = Wg(x)$).}}
% Notably, we introduce regularization terms in both pretraining and fine-tuning, which is supported both by our theoretical results and prior work showing that regularization can improve model performance on unseen attacks \cite{dai2022formulating}. 
%As regularization with respect to \textit{either} attack has the potential to directly reduce the upper bound in \ref{thm:robustness}, we would expect the modified objective to be beneficial in both the pre-training stage (when only the initial attack is known) and the fine-tuning stage (when both attacks are known).


\noindent
\textbf{Adversarial $\ell_2$ regularization. (ALR)}
Driven by our theoretical results, we first introduce adversarial $\ell_2$ regularization: $R_{\text{ALR}}(h, K(t)) =  \frac{1}{m} \sum_{i=1}^m \max_{x' \in C(x_i)} \|h(x') - h(x_i)\|_2$ where $C = C_{\text{init}}$ in initial training and corresponds to attack $P_C = F(K(t), (x_i, y_i))$ chosen by the fine-tuning strategy.  $\ell_2$ regularization penalizes the maximum distance between a sample's \edit{logits} and the furthest adversarially perturbed \edit{logits} within that sample's neighborhood. Using this regularization term would directly minimize the upper bounds in Theorem~\ref{thm:robustness} and Corollary~\ref{thm:corollary}.
We note that while ALR is similar in form to  TRADES~\cite{zhang2019theoretically}, it uses a Euclidean distance instead of the KL-divergence. Our paper is the first to show that this form of regularization is beneficial for CAR.

\noindent
\textbf{Efficiently approximating ALR. } Computing ALR uses multi-step optimization which can be costly to compute in practice. To improve efficiency in experiments, we consider (1) using single step optimization for ALR and (2) using randomly sampled, unoptimized perturbations can help with CAR. For (2), we consider Gaussian noise regularization (GR) and Uniform noise regularization (UR), specifically:
$R_{\text{GR}}(h, K(t)) = \frac{1}{m} \sum_{i=1}^m \|\edit{h}(x') - \edit{h}(x_i)\|_2$ where $x' \sim \mathcal{N}(0, \sigma^2)$ and 
$R_{\text{UR}}(h, K(t)) = \frac{1}{m} \sum_{i=1}^m \|\edit{h}(x') - \edit{h}(x_i)\|_2$ where $x' \sim \mathcal{U}(-\sigma, \sigma)$.

\noindent
\textbf{Other Baselines.} We compare to variation regularization (VR), which has been shown to improve generalization to unforeseen attacks \citep{dai2022formulating}. VR is defined as: $R_{\text{VR}}(h, K(t)) = \frac{1}{m} \sum_{i=1}^m \max_{x', x'' \in C(x_i)} \|\edit{h}(x') - \edit{h}(x'')\|_2$ where $C = C_{\text{init}}$ in initial training.  We also consider VR in finetuning with $C$ corresponding to attack $P_C = F(K(t), (x_i, y_i))$.
The link between VR and ALR is discussed in \cref{app:alr_vr}.
% \sophie{add link to Appendix, mention that the terms are connected.}
%\edit{We note that since VR maximizes over 2 perturbations and $x_i \in C(x_i)$, we can lower bound VR with ALR:} $R_\text{ALR}(h) \leq R_\text{VR}(h)$.  \edit{Additionally, by adding and subtracting $h(x_i)$ from within the distance computation and applying triangle inequality, we have that }$R_\text{VR}(h) \le 2R_\text{ALR}(h)$.
%Since during the initial training phase, the CAR problem reduces to an unforeseen robustness problem, variation regularization may provide a good starting point in terms of robustness.
%Algorithmically, \citet{dai2022formulating} estimate the maximization within $R_{\text{VR}}(h)$ by optimizing both $x'$ and $x''$ simultaneously with PGD.





\section{Experimental Results}
%\begin{table*}[ht]
%\centering
%\scalebox{0.77}{
%\begin{tabular}{|c|c|l|l|c|cccc|cc|cc|c|}
%\hline
%\multicolumn{1}{|c|}{\begin{tabular}[c]{@{}c@{}}Time\\ Step \end{tabular}} & &Procedure & Threat Models & \multicolumn{1}{c|}{Clean} & \multicolumn{1}{c}{$\ell_2$} & \multicolumn{1}{c}{StAdv} & \multicolumn{1}{c}{$\ell_\infty$} & \multicolumn{1}{c|}{Recolor} & \multicolumn{1}{c}{\begin{tabular}[c]{@{}c@{}}Avg\\ (known)\end{tabular}} & \multicolumn{1}{c|}{\begin{tabular}[c]{@{}c@{}}Union\\ (known)\end{tabular}} & \multicolumn{1}{c}{\begin{tabular}[c]{@{}c@{}}Avg\\ (all)\end{tabular}} & \multicolumn{1}{c|}{\begin{tabular}[c]{@{}c@{}}Union\\ (all)\end{tabular}} & \multicolumn{1}{c|}{\begin{tabular}[c]{@{}c@{}}Time\\ (hrs)\end{tabular}} \\ \hline
%\multirow{ 2}{*}{0} & \parbox[t]{2mm}{\multirow{2}{*}{\rotatebox[origin=c]{90}{Init}}}& AT & $\ell_2$ & \textbf{91.17} & \cellcolor[HTML]{B7E1CD}69.7 & 2.08 & 28.41 & 44.94 & 69.7 & 69.7 & 36.28 & 1.24 & 8.68 \\ %16.69
%& & AT + ALR ($\lambda=1$) & $\ell_2$ & 89.43 & \cellcolor[HTML]{B7E1CD}\textbf{69.84} & \textbf{48.23} & \textbf{34.00} & \textbf{65.46} & \textbf{69.84} & \textbf{69.84} & \textbf{54.38} & \textbf{31.27} & 17.17\\ %22.29
%\hline 
%\multirow{ 8}{*}{1} & \parbox[t]{2mm}{\multirow{3}{*}{\rotatebox[origin=c]{90}{Scratch}}} & AVG & $\ell_2$, StAdv & \bf{87.74} & \cellcolor[HTML]{B7E1CD}62.17 & \cellcolor[HTML]{B7E1CD}50.92 & 17.17 & 45.47 & 56.55 & 47.55 & 43.93 & 15.92 & 23.72\\ %47.44
%& & MAX & $\ell_2$, StAdv & 86.18 & \cellcolor[HTML]{B7E1CD}58.65 & \cellcolor[HTML]{B7E1CD}57.21 & 11.21 & 43.07 & 57.93 & 51.72 & 42.54 & 11.03 &  23.69 \\ %47.37
%& & Random & $\ell_2$, StAdv & 84.91 & \cellcolor[HTML]{B7E1CD}57.77 & \cellcolor[HTML]{B7E1CD}59.74 & 14.05 & 44.88 & 58.76 & 52.15 & 44.11 & 13.68 & 10.92 \\ %9.36
%\cdashline{2-14}
% & \parbox[t]{2mm}{\multirow{5}{*}{\rotatebox[origin=c]{90}{Finetune}}}& FT MAX &  $\ell_2$, StAdv & 83.73 & \cellcolor[HTML]{B7E1CD}57.07 & \cellcolor[HTML]{B7E1CD}58.67 & 12.51 & 49.03 & 57.87 & 51.32 & 44.32 & 12.36 & 4.00 \\
 %& & FT Croce & $\ell_2$, StAdv & 84.7 & \cellcolor[HTML]{B7E1CD}57.88 & \cellcolor[HTML]{B7E1CD}54.27 & 14.38 & 51.08 & 56.07 & 48.13 & 44.4 & 13.8 & 2.40\\
 %& & FT Single & $\ell_2$, StAdv &  80.89 & \cellcolor[HTML]{B7E1CD}45.45& \cellcolor[HTML]{B7E1CD}54.5 & 6.09 & 41.98 & 49.98 & 41.05 & 37.0 & 5.87 & 2.78 \\
 %& & FT Single + ALR & $\ell_2$, StAdv & \underline{87.24} & \cellcolor[HTML]{B7E1CD}\textbf{\underline{62.22}}& \cellcolor[HTML]{B7E1CD}61.5 & \textbf{\underline{21.4}} & \textbf{\underline{70.87}} & 61.86 & 55.04 & \textbf{\underline{54.0}} & \textbf{\underline{21.14}} & 4.24 \\
%   & & FT Croce + ALR & $\ell_2$, StAdv & 86.03 & \cellcolor[HTML]{B7E1CD}59.18 &\cellcolor[HTML]{B7E1CD}\textbf{\underline{65.14}} & 15.36  & 63.31 & \textbf{\underline{62.16}}& \textbf{\underline{55.83}} & 50.75 & 15.29 & 3.47 \\

%\hline

%\multirow{ 8}{*}{2}&\parbox[t]{2mm}{\multirow{3}{*}{\rotatebox[origin=c]{90}{Scratch}}} & AVG & $\ell_2$, StAdv, $\ell_\infty$ & 85.98 & \cellcolor[HTML]{B7E1CD}67.60 & \cellcolor[HTML]{B7E1CD}45.81 & \cellcolor[HTML]{B7E1CD}\textbf{42.39} & 62.43 & 51.93 & 34.05 & 54.56 & 33.39 & 33.12 \\
%& & MAX & $\ell_2$, StAdv, $\ell_\infty$ & 84.54 & \cellcolor[HTML]{B7E1CD}54.87 & \cellcolor[HTML]{B7E1CD}52.33 & \cellcolor[HTML]{B7E1CD}38.23 & 55.90 & 48.48 & 35.25 & 50.33 & 34.08 & 45.84 \\
%& &Random & $\ell_2$, StAdv, $\ell_\infty$ & 84.63 & \cellcolor[HTML]{B7E1CD}67.46 & \cellcolor[HTML]{B7E1CD}47.35 & \cellcolor[HTML]{B7E1CD}42.12 & 63.61 & 52.31 & 35.46 & 55.13 & 34.79 & 11.03 \\ \cdashline{2-14}
 %&\parbox[t]{2mm}{\multirow{5}{*}{\rotatebox[origin=c]{90}{Finetune}}} &FT MAX & $\ell_2$, StAdv, $\ell_\infty$ & 83.16 &  \cellcolor[HTML]{B7E1CD}65.63 &  \cellcolor[HTML]{B7E1CD}56.68 &  \cellcolor[HTML]{B7E1CD}36.9 & 65.69 & 53.07 & 35.18 & 56.23 & 34.83 &	5.62 \\

 %& &FT Croce & $\ell_2$, StAdv, $\ell_{\infty}$ & 85.05 &  \cellcolor[HTML]{B7E1CD}67.3 &  \cellcolor[HTML]{B7E1CD}48.07 &  \cellcolor[HTML]{B7E1CD}33.38 & 62.52 & 49.58 & 28.96 & 52.82 & 28.63 & 2.27 \\ 

  %& &FT Single & $\ell_2$, StAdv, $\ell_{\infty}$ & 87.99 & \cellcolor[HTML]{B7E1CD}\textbf{\underline{70.53}} & \cellcolor[HTML]{B7E1CD}11.17 & \cellcolor[HTML]{B7E1CD}41.63 & 63.46 &41.11 & 7.95 & 46.7 & 7.74  & 1.57 \\
 %& &FT Single + ALR & $\ell_2$, StAdv, $\ell_{\infty}$ & \textbf{\underline{88.74}} & \cellcolor[HTML]{B7E1CD}69.15 & \cellcolor[HTML]{B7E1CD}47.33 & \cellcolor[HTML]{B7E1CD}\underline{42.08} & 68.62 & 52.85 & \textbf{\underline{36.66}} & 56.8 & \textbf{\underline{36.62}} & 2.26 \\
 %& &FT Croce + ALR & $\ell_2$, StAdv, $\ell_{\infty}$ & 86.57 & \cellcolor[HTML]{B7E1CD}67.99 & \cellcolor[HTML]{B7E1CD}\textbf{\underline{61.55}} & \cellcolor[HTML]{B7E1CD}36.59 & \textbf{\underline{72.16}} & \textbf{\underline{55.38}} & 35.68 & \textbf{\underline{59.57}} & 35.52 & 2.87 \\

 %\hline

%\multirow{ 8}{*}{3} &\parbox[t]{2mm}{\multirow{3}{*}{\rotatebox[origin=c]{90}{Scratch}}}& AVG & $\ell_2$, StAdv, $\ell_\infty$, Recolor & 87.77 & \cellcolor[HTML]{B7E1CD}\textbf{68.55} & \cellcolor[HTML]{B7E1CD}39.55 & \cellcolor[HTML]{B7E1CD}\textbf{41.97} & \cellcolor[HTML]{B7E1CD}67.93 & 54.5 & 30.39 & 54.5 & 30.39 & 51.55 \\
%& &MAX & $\ell_2$, StAdv, $\ell_\infty$, Recolor & 84.3 & \cellcolor[HTML]{B7E1CD}57.62 & \cellcolor[HTML]{B7E1CD}52.3 & \cellcolor[HTML]{B7E1CD}41.69 & \cellcolor[HTML]{B7E1CD}65.1 & 54.18 & \textbf{37.44} & 54.18 & \textbf{37.44} & 61.09 \\
%& &Random & $\ell_2$, StAdv, $\ell_\infty$, Recolor & 86.32 & \cellcolor[HTML]{B7E1CD}65.87 & \cellcolor[HTML]{B7E1CD}47.82 & \cellcolor[HTML]{B7E1CD}35.04 & \cellcolor[HTML]{B7E1CD}68.35 & 54.27 & 30.76 & 54.27 & 30.76 & 13.15 \\ \cdashline{2-14}
%&\parbox[t]{2mm}{\multirow{5}{*}{\rotatebox[origin=c]{90}{Finetune}}}&FT MAX & $\ell_2$, StAdv, $\ell_{\infty}$, Recolor & 83.64 & \cellcolor[HTML]{B7E1CD}66.21 & \cellcolor[HTML]{B7E1CD}57.53 & \cellcolor[HTML]{B7E1CD}\underline{37.77} & \cellcolor[HTML]{B7E1CD}69.32 & 57.71 & \underline{36.02} & 57.71 & \underline{36.02} & 8.45 \\
%& &FT Croce & $\ell_2$, StAdv, $\ell_{\infty}$, Recolor & 86.64 & \cellcolor[HTML]{B7E1CD}\underline{68.76} & \cellcolor[HTML]{B7E1CD}44.81 & \cellcolor[HTML]{B7E1CD}36.02 & \cellcolor[HTML]{B7E1CD}68.05 & 54.41 & 29.44 & 54.41 & 29.44 & 2.34 \\
 %  & &FT Single & $\ell_2$, StAdv, $\ell_{\infty}$, Recolor &  90.41& \cellcolor[HTML]{B7E1CD}66.47 & \cellcolor[HTML]{B7E1CD}3.93 & \cellcolor[HTML]{B7E1CD}29.6& \cellcolor[HTML]{B7E1CD}69.03 & 42.26 & 2.49 & 42.26 & 2.49 & 3.11 \\
 %& &FT Single + ALR & $\ell_2$, StAdv, $\ell_{\infty}$, Recolor & \textbf{\underline{90.45}} & \cellcolor[HTML]{B7E1CD}61.58 & \cellcolor[HTML]{B7E1CD}25.77 & \cellcolor[HTML]{B7E1CD}27.43 & \cellcolor[HTML]{B7E1CD}69.26 & 46.01 & 19.2 & 46.01 & 19.2 & 4.24 \\
 % & &FT Croce + ALR & $\ell_2$, StAdv, $\ell_{\infty}$, Recolor &87.62 & \cellcolor[HTML]{B7E1CD}68.14 & \cellcolor[HTML]{B7E1CD}\underline{\textbf{58.5}} & \cellcolor[HTML]{B7E1CD}36.39 & \cellcolor[HTML]{B7E1CD}\textbf{\underline{72.35}} & \textbf{\underline{58.85}} & 34.92 & \textbf{\underline{58.85}} & 34.92 & 3.35 \\
%\hline
%\end{tabular}}
%\vspace{-5pt}
%\caption{\textbf{Continual Robust Training on CIFAR-10.} The learner knows about $\ell_2$ attacks and over time, is sequentially introduced to StAdv, $\ell_\infty$, and ReColor attacks. We report clean accuracy, accuracy on individual attacks, and average and union accuracies.  The ``Threat Models" column specifies known attacks and accuracies on these attacks are highlighted in green. Initial adversarial training occurs at time $t=0$, and the model is updated either through training from scratch or through fine-tuning the model at the previous time step.
%The ``Avg (known)" and ``Union (known)" columns represent average and union accuracies on known attacks while ``Avg (all)" and ``Union (all)" columns performance across all four attacks.  Additionally, we report non-cumulative training times in the ``Time" column.  Best performance for each time step out of all procedures are \textbf{bolded}, while best performance across fine-tuning based procedures \underline{underlined}.
%\anote{push this table to next page, move the last sentence to the beginning, add info about the green highlighting}}
%\label{tab:main_results_cifar}
%\vspace{-10pt}
%\end{table*}


%\begin{figure*}[t!]
%    \centering
%    \begin{subfigure}[t]{0.5\textwidth}
%        \centering
%        \includegraphics[width=0.95\textwidth]{figures/imagenette_finetune_froml2_union.pdf}
%        \vspace{-10pt}
%        \caption{Adversarial $\ell_2$ regularization\\ ($\lambda=0.5$)}
%    \end{subfigure}%
%    \begin{subfigure}[t]{0.5\textwidth}
%        \centering
%        \includegraphics[width=0.95\textwidth]{figures/imagenette_finetune_fromuniform_union.pdf}
%        \vspace{-10pt}
%        \caption{Uniform Regularization\\
%        ($\sigma=2,\lambda=1$)}
%    \end{subfigure}
%    \caption{\textbf{Ablation 1: Change in union robust accuracy after fine-tuning from checkpoint initially trained with regularization.}  We fine-tune models on Imagenette across 144 pairs of initial attack and new attack.  The initial attack corresponds to the row of each grid and new attack corresponds to each column.  Values represent differences between the accuracy measured on a model \emph{fine-tuned with and without regularization}.  Gains in accuracy of at least 1\% are highlighted in green, while drops in accuracy of at least 1\% in red. Further results are in Appendix~\ref{app:init_train}.}
%    \label{fig:fine-tune_abl_main_text_from_init}
%    \vspace{-10pt}
%\end{figure*}
%\sophie{add discussion of feature reg vs logit reg} \christian{added to training procedure}
In this section, we empirically demonstrate that using regularization in CRT helps improve robustness when attacks are introduced sequentially.  This section is organized as follows: (i) experimental setup \S (\ref{sec:exp_setup}), (ii) overall results for using regularization in CRT (\S \ref{sec:car_reg}), (iii) ablations in initial training (\S \ref{sec:init_train_impact}) and (iv) ablations in fine-tuning (\S \ref{sec:fine-tuning_impact}).% of the CRT pipeline.  \anote{shorten by using inline bullets like i) experimental setup, ii) overall results using RCRT, ...}
%We begin by looking at the impact of regularization on different parts of the CRT pipeline; we discuss impact on initial training in \S\ref{sec:init_train_impact} and iterative fine-tuning in \S\ref{sec:fine-tuning_impact}.  We then analyze the performance of fine-tuning with existing approaches \citep{croce2022adversarial,TB19} within a CRT framework, and investigate how regularization within both stages of CRT performs on sequences of up to 4 attacks.
\vspace{-5pt}
\subsection{Experimental setup}
\label{sec:exp_setup}

\noindent
\textbf{Datasets. } We experiment with CIFAR-10, CIFAR-100 \citep{krizhevsky2009learning}, and ImageNette \citep{howardimagenette}, a 10-class subset of ImageNet \citep{deng2009imagenet}.%and 10-class and 100-class ImageNet subsets\citep{deng2009imagenet} (ImageNette \citep{howardimagenette} and ImageNet-100 respectively).

\noindent
\textbf{Architectures. } For CIFAR-10 and CIFAR-100, we use WideResnet-28-10 (WRN-28-10) architecture \citep{zagoruyko2016wide} and ResNet-18 for ImageNette. % For Imagenet subsets, we use ResNet architectures \citep{he2016deep}; we use ResNet-18 for ImageNette and Resnet-50 for ImageNet-100.

\noindent
\textbf{Attacks. }We include results for $\ell_2$, $\ell_\infty$, StAdv \citep{XiaoZ0HLS18}, ReColor attacks \citep{LaidlawF19}, and the 8 core attacks of Imagenet-UA \citep{kaufmann2019testing}. For $\ell_2$ attacks, we use a bound $\epsilon= 0.5$ for CIFAR datasets and $\epsilon=1$ for ImageNette.  For $\ell_2$ attacks, we use $\epsilon = \frac{8}{255}$, and for StAdv and ReColor attacks, we use the same bounds as used in their original papers \citet{XiaoZ0HLS18} ($\epsilon=0.05$) and \citet{LaidlawF19} ($\epsilon=0.06$) respectively. For ImageNet-UA attacks, we use the \emph{medium distortion} strength bounds used by \citet{kaufmann2019testing}.  For experiments investigating the impact of regularization in the fine-tuning step of CRT (\S\ref{sec:fine-tuning_impact}), we include results for fine-tuning to the same attack type but with larger attack bounds.  For these experiments, the larger bounds are given by $\epsilon = 1$ for $\ell_2$, $\epsilon=\frac{12}{255}$ for $\ell_\infty$, $\epsilon=0.07$ for StAdv, $\epsilon=0.08$ for ReColor, and high distortion strength bounds for ImageNet-UA attacks. 

\noindent\textbf{Training from scratch baselines. }We consider the following baselines for training from scratch:
\begin{itemize}
    %\item \textit{Adversarial Training (AT)}  \citep{madry2017towards}: when only a single attack is known, we consider adversarial training a baseline for comparison.  We also consider using this AT model as the initialization for different fine-tuning techniques within a CRT framework.
    \item \textit{Training with AVG and MAX objectives} \citep{TB19}: \citet{TB19} propose two different training objectives, AVG ($L_{\text{AVG}}(h, t) = \frac{1}{m|K(t)|} \sum_{i=1}^{m} \sum_{P_C \in K(t)} \ell(h(P_C(x_i, y_i)), y_i)$) and MAX ($L_{\text{MAX}}(h, t) = \frac{1}{m} \sum_{i=1}^{m} \max_{P_C \in K(t)} \ell(h(P_C(x_i, y_i)), y_i)$), for robustness against multiple known attacks.
    \item \textit{Randomly sampling attacks} \citep{madaan2020learning}: 
    AVG and MAX require generating adversarial examples with all attacks for each image. For a more efficient baseline, we consider randomly sampling an attack for each batch for use in adversarial training.
\end{itemize}

\noindent
\textbf{CRT Baselines. }For CRT, we use PGD adversarial training (AT) \citep{madry2017towards} for initial training and then fine-tune the model using several different fine-tuning strategies:
\begin{itemize}
    \item \textit{MAX objective fine-tuning} (FT-MAX) \citep{TB19}: We use the MAX objective for fine-tuning when a new attack is introduced.
    \item \textit{\citet{croce2022adversarial} fine-tuning} (FT Croce): \citet{croce2022adversarial} introduce a fine-tuning technique for use with $\ell_{\infty}$ and $\ell_1$ attacks which we \emph{generalize to training with arbitrary attacks}. This approach samples a single attack per batch. The probability that an attack $P_C$ is sampled is given by $\frac{\text{err}(P_C)}{\sum_{P \in K(t)} \text{err}(P)}$ where $\text{err}(P)$ denotes the running average of robust loss with respect to attack $P$ computed across batches of each attack.%sampling of different attack types during training such that only 1 attack is used per batch, making it more efficient than MAX fine-tuning for which all attacks are generated per batch.  The sampling procedure is based on computing a running average across batches of errors with respect to each attack.  Each attack $P_C$ is sampled with probability $\frac{\text{err}(P_C)}{\sum_{P \in K(t)} \text{err}(P)}$ where $\text{err}(P)$ denotes the running average of loss computed across batches of each attack.
    \item \textit{Single attack fine-tuning} (FT Single): We also consider fine-tuning with \emph{only the newly introduced attack}, allowing us to determine the extent to which previous attacks are forgotten. The previous two fine-tuning techniques involve replaying previous attacks.
\end{itemize}
We then investigate incorporating regularization into the initial training and fine-tuning phases of CRT.

\noindent
\textbf{Training and Fine-tuning Procedures. } During training, we use 10-step Projected Gradient Descent~\cite{madry2017towards} to generate adversarial examples. For the regularization terms (\cref{subsec: methods}), VR and ALR use single step optimization to reduce time overhead, while UR and GR use $\sigma=2$ and $\sigma=0.2$, respectively.  Results for additional values of $\sigma$ are in \cref{app:random_noise_var_abl}. We train models for 100 epochs for initial training and 10 epochs  for fine-tuning (results with 25 epochs in \cref{app:exp_seq}). We include additional details about the training procedure in \cref{app:exp_setup}.%We train models using adversarial training with and without regularization (\S\ref{sec:regularization}) with 10-step PGD adversarial training for $\ell_p$ attacks \citep{madry2017towards}.  We similarly use 10-steps to generate adversarial examples for adversarial training with non-$\ell_p$ attack types. For variation regularization and adversarial $\ell_2$ regularization, we optimize the objective using iterative optimization (in the same manner in which adversarial examples are generated for each attack type) with a single iteration to reduce the time overhead of training.  For uniform regularization and gaussian regularization, we use $\sigma=2$ and $\sigma=0.2$, respectively.  We include results for additional $\sigma$ in Appendix \ref{app:random_noise_var_abl}. When performing initial training, we train models for 100 epochs and when performing fine-tuning, we train models for 10 epochs (we also include results for other numbers of epochs in Appendix \ref{app:exp_seq}. We include additional details about the training procedure in Appendix \ref{app:exp_setup}.

%For the purposes of our experiments, we modified the regularization terms from Section~\ref{sec: theory_methods} to penalize distance in the \textit{logit} space (i.e. the output of $h$) rather than the \textit{feature} space (i.e. the output of $g$). We found that this change led to more stable training and better robust performance, likely owing to the high-dimensional nature of the internal representations. See Table~\ref{tab:main_results_cifar_epochs} in the Appendix for an ablation on the choice of layer used in regularization.

\noindent
\textbf{Evaluation Attacks and Metrics. } Our main results in Table \ref{tab:main_results_cifar} and additional ones in Appendix \ref{app:exp_seq} use full AutoAttack \citep{croce2020reliable} for evaluating $\ell_p$ robustness. For ablations, we restrict to APGD-T and FAB-T from AutoAttack to reduce evaluation time.  We use 20-step optimization when evaluating StAdv and ReColor attacks and the default evaluation hyperparameters for ImageNet-UA attacks in \citet{kaufmann2019testing}. We report \textit{accuracy on each attack}, \textit{Union accuracy} (overall accuracy when the worst case attack is chosen for each test example), \textit{Average accuracy} (average over accuracy on each attack), and \textit{training time} (in hours). Metrics are reported for the epoch $E^*$ with best performance on the set of known attacks. For training from scratch, the reported training time is scaled by fraction of training for the best epoch (\textit{i.e.} we report $\frac{E^*}{100} \times \text{training time for 100 epochs}$). For fine-tuning we report training time for the full 10 epochs. This allows us to see how much faster fine-tuning is to optimal early stopping when re-training from scratch.

%Beyond loss on each attack, we also report \textit{union robust accuracy} (\edit{overall accuracy if for each test example, we chose the worst case attack from the set of attacks}) and \textit{average robust accuracy} (average over individual robust accuracies). \edit{We note that union accuracy is equal to $1 - L_{MAX}$ and average accuracy is equal to $1 - L_{AVG}$ when $\ell$ is taken to be the 0-1 loss in Equations \ref{eq:avg} and \ref{eq:max}.}
%Since the time that is taken for fine-tuning is also important, we report training and fine-tuning times in hours.  


\begin{table*}[ht]
\centering
\scalebox{0.77}{
\begin{tabular}{|c|c|l|l|c|cccc|cc|cc|c|}
\hline
\multicolumn{1}{|c|}{\begin{tabular}[c]{@{}c@{}}Time\\ Step \end{tabular}} & &Procedure & Threat Models & \multicolumn{1}{c|}{Clean} & \multicolumn{1}{c}{$\ell_2$} & \multicolumn{1}{c}{StAdv} & \multicolumn{1}{c}{$\ell_\infty$} & \multicolumn{1}{c|}{Recolor} & \multicolumn{1}{c}{\begin{tabular}[c]{@{}c@{}}Avg\\ (known)\end{tabular}} & \multicolumn{1}{c|}{\begin{tabular}[c]{@{}c@{}}Union\\ (known)\end{tabular}} & \multicolumn{1}{c}{\begin{tabular}[c]{@{}c@{}}Avg\\ (all)\end{tabular}} & \multicolumn{1}{c|}{\begin{tabular}[c]{@{}c@{}}Union\\ (all)\end{tabular}} & \multicolumn{1}{c|}{\begin{tabular}[c]{@{}c@{}}Time\\ (hrs)\end{tabular}} \\ \hline
\multirow{ 2}{*}{0} & \parbox[t]{2mm}{\multirow{2}{*}{\rotatebox[origin=c]{90}{Init}}}& AT & $\ell_2$ & \textbf{91.17} & \cellcolor[HTML]{B7E1CD}69.7 & 2.08 & 28.41 & 44.94 & 69.7 & 69.7 & 36.28 & 1.24 & 8.68 \\ %16.69
& & AT + ALR ($\lambda=1$) & $\ell_2$ & 89.43 & \cellcolor[HTML]{B7E1CD}\textbf{69.84} & \textbf{48.23} & \textbf{34.00} & \textbf{65.46} & \textbf{69.84} & \textbf{69.84} & \textbf{54.38} & \textbf{31.27} & 17.17\\ %22.29
\hline 
%\multirow{ 8}{*}{1} & \parbox[t]{2mm}{\multirow{3}{*}{\rotatebox[origin=c]{90}{Scratch}}} & AVG & $\ell_2$, StAdv & \bf{87.74} & \cellcolor[HTML]{B7E1CD}62.17 & \cellcolor[HTML]{B7E1CD}50.92 & 17.17 & 45.47 & 56.55 & 47.55 & 43.93 & 15.92 & 23.72\\ %47.44
%& & MAX & $\ell_2$, StAdv & 86.18 & \cellcolor[HTML]{B7E1CD}58.65 & \cellcolor[HTML]{B7E1CD}57.21 & 11.21 & 43.07 & 57.93 & 51.72 & 42.54 & 11.03 &  23.69 \\ %47.37
%& & Random & $\ell_2$, StAdv & 84.91 & \cellcolor[HTML]{B7E1CD}57.77 & \cellcolor[HTML]{B7E1CD}59.74 & 14.05 & 44.88 & 58.76 & 52.15 & 44.11 & 13.68 & 10.92 \\ %9.36
%\cdashline{2-14}
\multirow{ 5}{*}{1} & \parbox[t]{2mm}{\multirow{5}{*}{\rotatebox[origin=c]{90}{Finetune}}}& FT MAX &  $\ell_2$, StAdv & 83.73 & \cellcolor[HTML]{B7E1CD}57.07 & \cellcolor[HTML]{B7E1CD}58.67 & 12.51 & 49.03 & 57.87 & 51.32 & 44.32 & 12.36 & 4.00 \\
 %& FT MAX (25 ep) & $\ell_2$, StAdv & 84.85 & \cellcolor[HTML]{B7E1CD}56.44 & \cellcolor[HTML]{B7E1CD}\textbf{\underline{61.34}} & 10.35 & 48.08 & 58.89 & 52.52 & 44.05 & 10.24 & 10 \\ %
 
 & & FT Single & $\ell_2$, StAdv &  80.89
& \cellcolor[HTML]{B7E1CD}45.45& \cellcolor[HTML]{B7E1CD}54.5
& 6.09 & 41.98 & 49.98 & 41.05 & 37.0 & 5.87 & 2.78 \\
 & & FT Croce & $\ell_2$, StAdv & 84.7 & \cellcolor[HTML]{B7E1CD}57.88 & \cellcolor[HTML]{B7E1CD}54.27 & 14.38 & 51.08 & 56.07 & 48.13 & 44.4 & 13.8 & 2.40\\

 & & FT Single + ALR & $\ell_2$, StAdv & \textbf{87.24} & \cellcolor[HTML]{B7E1CD}\textbf{62.22}& \cellcolor[HTML]{B7E1CD}61.5
& \textbf{21.4} & \textbf{70.87} & 61.86 & 55.04 & \textbf{54.0} & \textbf{21.14} & 4.24 \\
   & & FT Croce + ALR & $\ell_2$, StAdv & 86.03 & \cellcolor[HTML]{B7E1CD}59.18 &\cellcolor[HTML]{B7E1CD}\textbf{65.14} & 15.36  & 63.31 & \textbf{62.16}& \textbf{55.83} & 50.75 & 15.29 & 3.47 \\

\hline

%\multirow{ 8}{*}{2}&\parbox[t]{2mm}{\multirow{3}{*}{\rotatebox[origin=c]{90}{Scratch}}} & AVG & $\ell_2$, StAdv, $\ell_\infty$ & 85.98 & \cellcolor[HTML]{B7E1CD}67.60 & \cellcolor[HTML]{B7E1CD}45.81 & \cellcolor[HTML]{B7E1CD}\textbf{42.39} & 62.43 & 51.93 & 34.05 & 54.56 & 33.39 & 33.12 \\
%& & MAX & $\ell_2$, StAdv, $\ell_\infty$ & 84.54 & \cellcolor[HTML]{B7E1CD}54.87 & \cellcolor[HTML]{B7E1CD}52.33 & \cellcolor[HTML]{B7E1CD}38.23 & 55.90 & 48.48 & 35.25 & 50.33 & 34.08 & 45.84 \\
%& &Random & $\ell_2$, StAdv, $\ell_\infty$ & 84.63 & \cellcolor[HTML]{B7E1CD}67.46 & \cellcolor[HTML]{B7E1CD}47.35 & \cellcolor[HTML]{B7E1CD}42.12 & 63.61 & 52.31 & 35.46 & 55.13 & 34.79 & 11.03 \\ \cdashline{2-14}
 \multirow{ 5}{*}{2}&\parbox[t]{2mm}{\multirow{5}{*}{\rotatebox[origin=c]{90}{Finetune}}} &FT MAX & $\ell_2$, StAdv, $\ell_\infty$ & 83.16 &  \cellcolor[HTML]{B7E1CD}65.63 &  \cellcolor[HTML]{B7E1CD}56.68 &  \cellcolor[HTML]{B7E1CD}36.9 & 65.69 & 53.07 & 35.18 & 56.23 & 34.83 &	5.62 \\
  & &FT Single & $\ell_2$, StAdv, $\ell_{\infty}$ & 87.99 & \cellcolor[HTML]{B7E1CD}\textbf{70.53} & \cellcolor[HTML]{B7E1CD}11.17
 & \cellcolor[HTML]{B7E1CD}41.63 & 63.46 &41.11 & 7.95 & 46.7 & 7.74  & 1.57 \\
 & &FT Croce & $\ell_2$, StAdv, $\ell_{\infty}$ & 85.05 &  \cellcolor[HTML]{B7E1CD}67.3 &  \cellcolor[HTML]{B7E1CD}48.07 &  \cellcolor[HTML]{B7E1CD}33.38 & 62.52 & 49.58 & 28.96 & 52.82 & 28.63 & 2.27 \\ 


 & &FT Single + ALR & $\ell_2$, StAdv, $\ell_{\infty}$ & \textbf{88.74} & \cellcolor[HTML]{B7E1CD}69.15 & \cellcolor[HTML]{B7E1CD}47.33 & \cellcolor[HTML]{B7E1CD}\textbf{42.08} & 68.62 & 52.85 & \textbf{36.66} & 56.8 & \textbf{36.62} & 2.26 \\
 & &FT Croce + ALR & $\ell_2$, StAdv, $\ell_{\infty}$ & 86.57 & \cellcolor[HTML]{B7E1CD}67.99 & \cellcolor[HTML]{B7E1CD}\textbf{61.55} & \cellcolor[HTML]{B7E1CD}36.59 & \textbf{72.16} & \textbf{55.38} & 35.68 & \textbf{59.57} & 35.52 & 2.87 \\

 \hline

%\multirow{ 8}{*}{3} &\parbox[t]{2mm}{\multirow{3}{*}{\rotatebox[origin=c]{90}{Scratch}}}& AVG & $\ell_2$, StAdv, $\ell_\infty$, Recolor & 87.77 & \cellcolor[HTML]{B7E1CD}\textbf{68.55} & \cellcolor[HTML]{B7E1CD}39.55 & \cellcolor[HTML]{B7E1CD}\textbf{41.97} & \cellcolor[HTML]{B7E1CD}67.93 & 54.5 & 30.39 & 54.5 & 30.39 & 51.55 \\
%& &MAX & $\ell_2$, StAdv, $\ell_\infty$, Recolor & 84.3 & \cellcolor[HTML]{B7E1CD}57.62 & \cellcolor[HTML]{B7E1CD}52.3 & \cellcolor[HTML]{B7E1CD}41.69 & \cellcolor[HTML]{B7E1CD}65.1 & 54.18 & \textbf{37.44} & 54.18 & \textbf{37.44} & 61.09 \\
%& &Random & $\ell_2$, StAdv, $\ell_\infty$, Recolor & 86.32 & \cellcolor[HTML]{B7E1CD}65.87 & \cellcolor[HTML]{B7E1CD}47.82 & \cellcolor[HTML]{B7E1CD}35.04 & \cellcolor[HTML]{B7E1CD}68.35 & 54.27 & 30.76 & 54.27 & 30.76 & 13.15 \\ \cdashline{2-14}
\multirow{ 5}{*}{3}&\parbox[t]{2mm}{\multirow{5}{*}{\rotatebox[origin=c]{90}{Finetune}}}&FT MAX & $\ell_2$, StAdv, $\ell_{\infty}$, Recolor & 83.64 & \cellcolor[HTML]{B7E1CD}66.21 & \cellcolor[HTML]{B7E1CD}57.53 & \cellcolor[HTML]{B7E1CD}\textbf{37.77} & \cellcolor[HTML]{B7E1CD}69.32 & 57.71 & \textbf{36.02} & 57.71 & \textbf{36.02} & 8.45 \\
   & &FT Single & $\ell_2$, StAdv, $\ell_{\infty}$, Recolor &  90.41& \cellcolor[HTML]{B7E1CD}66.47 & \cellcolor[HTML]{B7E1CD}3.93 & \cellcolor[HTML]{B7E1CD}29.6& \cellcolor[HTML]{B7E1CD}69.03 & 42.26 & 2.49 & 42.26 & 2.49 & 3.11 \\
& &FT Croce & $\ell_2$, StAdv, $\ell_{\infty}$, Recolor & 86.64 & \cellcolor[HTML]{B7E1CD}\textbf{68.76} & \cellcolor[HTML]{B7E1CD}44.81 & \cellcolor[HTML]{B7E1CD}36.02 & \cellcolor[HTML]{B7E1CD}68.05 & 54.41 & 29.44 & 54.41 & 29.44 & 2.34 \\
 & &FT Single + ALR & $\ell_2$, StAdv, $\ell_{\infty}$, Recolor & \textbf{90.45} & \cellcolor[HTML]{B7E1CD}61.58 & \cellcolor[HTML]{B7E1CD}25.77 & \cellcolor[HTML]{B7E1CD}27.43 & \cellcolor[HTML]{B7E1CD}69.26 & 46.01 & 19.2 & 46.01 & 19.2 & 4.24 \\
  & &FT Croce + ALR & $\ell_2$, StAdv, $\ell_{\infty}$, Recolor &87.62 & \cellcolor[HTML]{B7E1CD}68.14 & \cellcolor[HTML]{B7E1CD}\textbf{58.5} & \cellcolor[HTML]{B7E1CD}36.39 & \cellcolor[HTML]{B7E1CD}\textbf{72.35} & \textbf{58.85} & 34.92 & \textbf{58.85} & 34.92 & 3.35 \\
\hline
\end{tabular}}
\vspace{-5pt}
\caption{\textbf{Continual Robust Training on CIFAR-10.} Best performance for each time step are \textbf{bolded}. The defender initially knows about $\ell_2$ attacks and over time, is sequentially introduced to StAdv, $\ell_\infty$, and ReColor attacks. We report clean accuracy, accuracy on individual attacks, and average and union accuracies.  The ``Threat Models" column specifies known attacks at the current time step, and accuracies on these attacks are in {\color[HTML]{B7E1CD} green cells}. Initial adversarial training occurs at time step 0, and the model is updated through fine-tuning the model from the previous time step.  ``Avg (known)" and ``Union (known)" columns represent average and union accuracies on known attacks while ``Avg (all)" and ``Union (all)" columns report performance across all four attacks.  We report training time for each time step in the ``Time" column.}
\label{tab:main_results_cifar}
\vspace{-10pt}
\end{table*}

\begin{table}[ht]
    \centering
    \scalebox{0.85}{
    \begin{tabular}{|l|c|cc|c|}
    \hline 
    Procedure & Clean & Avg & Union & Time \\ \hline
       MAX & 84.3 &  54.18 & 37.44 &61.09 \\
       AVG  & 87.77 &  54.5 & 30.39 & 51.55\\
       Random & 86.32 & 54.27 & 30.76 & 13.15\\\hline
       CRT + ALR & 87.62 & 58.85 & 34.92 & 26.86\\ \hline
    \end{tabular}}
    \caption{Regularized CRT (using \citet{croce2020reliable} fine-tuning strategy) compared to training from scratch on $\ell_2$, StAdv, $\ell_{\infty}$, and Recolor attacks on CIFAR-10.}
    \label{tab:training_from_scratch}
    \vspace{-20pt}
\end{table}


\begin{table*}[]
\centering
{\renewcommand{\arraystretch}{1.2}
\scalebox{0.7}{
\begin{tabular}{|l|c|c|c|cccccccccccc|cc|}
\hline
\begin{tabular}[c]{@{}l@{}}Initial\\Attack\end{tabular} & \begin{tabular}[c]{@{}c@{}}Reg\\Type\end{tabular} & $\lambda$ & Clean & $\ell_2$ & $\ell_\infty$ & StAdv & ReColor & Gabor & Snow & Pixel & JPEG & Elastic & Wood & Glitch & \begin{tabular}[c]{@{}c@{}}Kaleid-\\oscope\end{tabular} & Avg & Union \\ \hline
$\ell_2$ & None & 0 & 91.08 & 70.02 & 29.38 & 0.79 & 33.69 & 66.93 & 24.59 & 14.99 & 64.22 & 45.13 & 70.85 & 80.30 & 30.08 & 44.25 & 0.10 \\
$\ell_2$ & VR & 0.2 & \cellcolor[HTML]{F4C7C3}89.99 & 70.38 & \cellcolor[HTML]{B7E1CD}34.56 & \cellcolor[HTML]{B7E1CD}13.41 & \cellcolor[HTML]{B7E1CD}48.99 & 67.64 & \cellcolor[HTML]{B7E1CD}29.09 & \cellcolor[HTML]{B7E1CD}22.57 & \cellcolor[HTML]{B7E1CD}66.64 & \cellcolor[HTML]{B7E1CD}48.38 & \cellcolor[HTML]{B7E1CD}73.31 & 80.07 & \cellcolor[HTML]{B7E1CD}32.33 & \cellcolor[HTML]{B7E1CD}48.94 & \cellcolor[HTML]{B7E1CD}5.40 \\ 
$\ell_2$ & ALR & 0.5 & \cellcolor[HTML]{F4CCCC}89.57 & 70.29 & \cellcolor[HTML]{B7E1CD}34.16 & \cellcolor[HTML]{B7E1CD}17.44 & \cellcolor[HTML]{B7E1CD}51.04 & \cellcolor[HTML]{F4CCCC}65.63 & \cellcolor[HTML]{B7E1CD}28.71 & \cellcolor[HTML]{B7E1CD}22.50 & \cellcolor[HTML]{B7E1CD}66.76 & \cellcolor[HTML]{B7E1CD}48.80 & \cellcolor[HTML]{B7E1CD}73.24 & 79.66 & \cellcolor[HTML]{F4CCCC}28.83 & \cellcolor[HTML]{B7E1CD}48.92 & \cellcolor[HTML]{B7E1CD}5.94 \\ 
$\ell_2$ & UR & 5  & \cellcolor[HTML]{F4CCCC}88.34 & \cellcolor[HTML]{F4CCCC}66.66   & \cellcolor[HTML]{F4CCCC}27.41   & \cellcolor[HTML]{B7E1CD}26.22 & \cellcolor[HTML]{B7E1CD}60.22 & \cellcolor[HTML]{B7E1CD}69.16   & \cellcolor[HTML]{B7E1CD}26.67 & \cellcolor[HTML]{B7E1CD}22.57 & 64.08  & \cellcolor[HTML]{B7E1CD}46.83   & 71.14  & \cellcolor[HTML]{F4CCCC}77.60    & \cellcolor[HTML]{B7E1CD}31.36    & \cellcolor[HTML]{B7E1CD}49.16 & \cellcolor[HTML]{B7E1CD}6.23 \\
$\ell_2$ & GR & 0.5& \cellcolor[HTML]{F4CCCC}86.89 & \cellcolor[HTML]{F4CCCC}68.19   & \cellcolor[HTML]{B7E1CD}32.02 & \cellcolor[HTML]{B7E1CD}16.54 & \cellcolor[HTML]{B7E1CD}58.32 & \cellcolor[HTML]{B7E1CD}74.85 & \cellcolor[HTML]{B7E1CD}25.69 & \cellcolor[HTML]{B7E1CD}21.26 & \cellcolor[HTML]{B7E1CD}65.32   & \cellcolor[HTML]{B7E1CD}46.82 & \cellcolor[HTML]{B7E1CD}74.08   & \cellcolor[HTML]{F4CCCC}76.99   & \cellcolor[HTML]{B7E1CD}31.93    & \cellcolor[HTML]{B7E1CD}49.33 & \cellcolor[HTML]{B7E1CD}4.18 \\
\hline
$\ell_\infty$ & None & 0 & 85.53 & 59.36 & 50.98 & 6.34 & 56.27 & 68.94 & 36.79 & 20.57 & 54.02 & 51.00 & 64.24 & 75.94 & 39.44 & 48.66 & 1.31 \\
$\ell_\infty$ & VR & 0.2 & \cellcolor[HTML]{F4C7C3}82.58 & 58.36 & 51.53 & \cellcolor[HTML]{B7E1CD}18.98 & \cellcolor[HTML]{B7E1CD}62.12 & \cellcolor[HTML]{F4C7C3}67.18 & \cellcolor[HTML]{B7E1CD}39.22 & \cellcolor[HTML]{B7E1CD}23.62 & 54.73 & \cellcolor[HTML]{B7E1CD}52 & 63.35 & \cellcolor[HTML]{F4C7C3}71.72 & \cellcolor[HTML]{B7E1CD}43.18 & \cellcolor[HTML]{B7E1CD}50.50 & \cellcolor[HTML]{B7E1CD}5.08 \\ 
$\ell_\infty$ & ALR &  0.5 & \cellcolor[HTML]{F4C7C3}83.18 & \cellcolor[HTML]{F4C7C3}58.21 & 51.47 & \cellcolor[HTML]{B7E1CD}19.50 & \cellcolor[HTML]{B7E1CD}61.02 & 68.75 & \cellcolor[HTML]{B7E1CD}37.94 & \cellcolor[HTML]{B7E1CD}22.78 & 53.89 & \cellcolor[HTML]{F4C7C3}49.82 & 63.47 & \cellcolor[HTML]{F4C7C3}73.57 & 39.88 & \cellcolor[HTML]{B7E1CD}50.02 & \cellcolor[HTML]{B7E1CD}5.52 \\ 
 $\ell_\infty$    & UR  & 5  & \cellcolor[HTML]{F4C7C3}78.04 & 60.28  & \cellcolor[HTML]{F4C7C3}40.59   & \cellcolor[HTML]{B7E1CD}42.25 & \cellcolor[HTML]{B7E1CD}70.00    & \cellcolor[HTML]{F4C7C3}67.06   & \cellcolor[HTML]{F4C7C3}33.40    & \cellcolor[HTML]{B7E1CD}26.57 & \cellcolor[HTML]{B7E1CD}60.07   & \cellcolor[HTML]{F4C7C3}49.21   & 64.61  & \cellcolor[HTML]{F4C7C3}67.08   & \cellcolor[HTML]{F4C7C3}38.43      & \cellcolor[HTML]{B7E1CD}51.63 & \cellcolor[HTML]{B7E1CD}8.36 \\ 
$\ell_\infty$      & GR & 0.5& \cellcolor[HTML]{F4C7C3}80.65 & 59.74  & \cellcolor[HTML]{F4C7C3}46.12   & \cellcolor[HTML]{B7E1CD}34.57 & \cellcolor[HTML]{B7E1CD}70.49 & 68.33  & 35.80   & \cellcolor[HTML]{B7E1CD}26.04 & \cellcolor[HTML]{B7E1CD}57.28   & 51.98  & \cellcolor[HTML]{B7E1CD}65.46   & \cellcolor[HTML]{F4C7C3}70.73   & \cellcolor[HTML]{F4C7C3}38.21      & \cellcolor[HTML]{B7E1CD}52.06 & \cellcolor[HTML]{B7E1CD}6.28 \\ \hline
\end{tabular}
}}
\vspace{-5pt}
\caption{\textbf{Impact of Regularization on Unforeseen Robustness.} We consider the setting where the defender is only aware of a single attack and performs training with and without different types of regularization: variation regularization (VR), adversarial $\ell_2$ regularization (ALR), uniform regularization (UR), and Gaussian regularization (GR) at regularization strength $\lambda$.  We report clean accuracy and robust accuracies on a range of attacks. {\color[HTML]{B7E1CD} Green cells} represent an improvement of at least 1\%  while {\color[HTML]{F4C7C3} red cells} represent a drop of at least 1\% in comparison to no regularization.}
\label{tab:reg_unforeseen_rob}
\vspace{-10pt}
\end{table*}



\subsection{Improving CRT with Regularization}
\label{sec:car_reg}
We now analyze the robustness of models trained using CRT with and without regularization. For simplicity, we focus on ALR with other methods analyzed in \cref{sec:init_train_impact}.  To model a CAR setting, we consider a sequence of 4 attacks: $\ell_2 \to$ StAdv $\to \ell_{\infty} \to$ Recolor.  The first attack is the initially known attack while other attacks are introduced at later time steps. %The defender learns about these attacks at different time steps, with the first attack of the sequence known prior to deployment while subsequent attacks are sequentially discovered post-deployment. 
We present results for CIFAR-10 in Table \ref{tab:main_results_cifar}.  We include results in Appendix \ref{app:exp_seq} for Imagenette and CIFAR-100 as well as additional results for longer duration of fine-tuning (25 epochs) and a separate sequence of attacks: $\ell_\infty \to$ StAdv $\to$ Recolor $\to \ell_2$.  For these experiments, we use $\lambda=0.5$ unless specified otherwise.

\noindent\textbf{Regularization reduces degradation on previous attacks. } From Table \ref{tab:main_results_cifar}, we observe that fine-tuning with only the new attack (FT Single) can lead to degradation of robustness against previous attacks.  The incorporation of ALR significantly decreases this drop in robustness.  For example, when fine-tuning from an $\ell_2$ robust model with StAdv attacks (time step 1 in Table \ref{tab:main_results_cifar}), FT Single incurs a 24.25\% drop (from 69.7\% to 45.45\%) in $\ell_2$ accuracy from the initial checkpoint (AT at time step 0).  Meanwhile FT Single + ALR only experiences a 7.62\% drop (from 69.84\% to 62.22\%) in $\ell_2$ accuracy from the initial checkpoint (AT + ALR at time step 0).  Similarly, after the introduction of $\ell_\infty$ attack at time step 2, the accuracy of FT Single on StAdv attacks drops 43.42\% (from 54.5\% to 11.17\%) while FT Single + ALR only experiences a 14.17\% drop (from 61.5\% to 47.33\%). These results align with Theorem~\ref{thm:robustness}: when incorporating ALR into training, the gap in loss on the two attacks is lessened.
%For example, we find that FT MAX and our approach is able to consistently outperform training from scratch with AVG, MAX, and Random procedures in terms of average and union accuracy.  This suggests that representations learned on the attack used in initial training ($\ell_2$ attack in Table \ref{tab:main_results_cifar}) can be a useful starting point for robustness on other attacks.  This also suggests that existing algorithms for achieving simultaneous multi-robustness may be suboptimal since we would expect the performance of these methods to serve as an upper bound for CAR.

\begin{figure*}[t!]
    \centering
    \begin{subfigure}[t]{0.45\textwidth}
        \centering
        \includegraphics[width=0.95\textwidth]{figures/imagenette_finetune_l2_reg_union.pdf}
        \vspace{-15pt}
        \caption{Adversarial $\ell_2$ regularization ($\lambda=0.5$)}
    \end{subfigure}%
    \begin{subfigure}[t]{0.45\textwidth}
        \centering
        \includegraphics[width=0.95\textwidth]{figures/imagenette_finetune_uniform_union.pdf}
        \vspace{-15pt}
        \caption{Uniform Regularization
        ($\sigma=2,\lambda=1$)}
    \end{subfigure}
    %\begin{subfigure}[t]{0.5\textwidth}
    %    \centering
    %    \includegraphics[width=0.95\textwidth]{figures/imagenette_finetune_varreg_union.pdf}
    %    \vspace{-10pt}
    %    \caption{Variation Regularization\\ ($\lambda = 0.5$)}
    %\end{subfigure}%
    %\begin{subfigure}[t]{0.5\textwidth}
    %    \centering
    %    \includegraphics[width=0.95\textwidth]{figures/imagenette_finetune_gaussian_union.pdf}
     %   \vspace{-10pt}
     %   \caption{Gaussian Regularization\\ ($\sigma=0.2, \lambda = 0.5$)}
    %\end{subfigure}
    \vspace{-10pt}
    \caption{\textbf{Ablation 2: Change in union robust accuracy after fine-tuning with regularization (initial model does not use regularization).}  We fine-tune models on Imagenette across 144 pairs of initial attack and new attack.  The initial attack corresponds to the row of each grid and new attack corresponds to each column.  Values represent differences between the accuracy measured on a model \emph{fine-tuned with and without regularization}.  Gains in accuracy of at least 1\% are highlighted in {\color[HTML]{0E7003} green}, while drops in accuracy of at least 1\% in {\color[HTML]{FC0006} red}. Further results are in Appendix~\ref{app:fine-tuning}.}
    \label{fig:fine-tune_abl_main_text}
    \vspace{-15pt}
\end{figure*}

\noindent\textbf{Regularization improves performance on held out (unforeseen) attacks. }We observe that regularized CRT leads to higher robustness on attacks held out from training.  For example, at time step 1 in Table \ref{tab:main_results_cifar}, which trains with $\ell_2$ and StAdv attacks, the best accuracy on Recolor attacks out of unregularized CRT methods is 51.08\%, while FT Single + ALR achieves 70.87\% accuracy on Recolor attacks and FT Croce + ALR achieves 63.31\% accuracy on Recolor attacks.  The improvement in robustness on unforeseen attacks aligns with \cref{thm:corollary} as regularization helps decrease the drop in accuracy between clean inputs and perturbed inputs. This also aligns with CAR's goal of having a small $\delta_{\text{unknown}}$.%\sophie{does this make sense?}

%This suggests that regularized CRT can also lead to improvements in unforeseen robustness. \anote{last sentence does not add any information, what is the intuition? The reg. brings the logits close to benign ones, so helps against unseen attacks as well...}% as well as robustness on the set of known attacks.% This gain also aligns with the goal of also having some unforeseen robustness for when a new attack is just introduced and time is taken to fine-tune the model before it can be redeployed for use.

\noindent\textbf{Regularization balances performance and efficiency. } Our proposed regularization term adds a small computational overhead over other FT approaches but generally improves union performance on the set of known attacks. For example, when considering the sequence of $\ell_2$ and StAdv attacks (time step 1 in Table \ref{tab:main_results_cifar}), FT Croce + ALR improves union accuracy over FT Croce by 7.7\% while adding a time overhead of 1.07 hours.  Additionally, when considering the sequence of 3 attacks ($\ell_2$, StAdv, and $\ell_\infty$ attacks), FT Croce + ALR improves union accuracy over FT Croce by 6.72\% while adding a time overhead of 0.6 hours.  This increase in time complexity is much smaller than FT MAX which takes 1.6 hours longer than FT Croce for $\ell_2$ and StAdv and 3.35 hours longer for $\ell_2$, StAdv, and $\ell_\infty$. With respect to goals in CAR, regularization balances $\delta_{\text{known}}$ and $\Delta t$.

\noindent\textbf{Comparison to training from scratch.} In Table \ref{tab:training_from_scratch}, we report clean, average, and union accuracies along with total training times for using training from scratch on all 4 attacks compared to training sequentially with regularized CRT on CIFAR-10.  We observe that regularized CRT is significantly more efficient than MAX and AVG training (taking a total of 26.86 hours while AVG and MAX take over 50 hours of training time).  Surprisingly, we find that on CIFAR-10, regularized CRT can outperform training from scratch methods, achieving 4.35\% higher average accuracy compared to the best achieved by training from scratch.  This suggests that transferable robustness between carefully chosen attacks can improve MAR as a whole.  However, we note that the ability to outperform training from scratch seems to be specific to CIFAR-10; for ImageNette and CIFAR-100 (Appendix \ref{app:exp_seq}) training from scratch outperforms using fine-tuning in CAR.

\noindent\textbf{Impact of dataset and attack sequence. }In Appendix \ref{app:exp_seq}, we provide results on ImageNette and CIFAR-100 as well as for attack sequence $\ell_\infty \to$ StAdv $\to$ Recolor $\to \ell_2$.  Overall, we observe that trends such as improved robustness to unforeseen and the union of attacks are generally consistent. However, but the extent to which regularization improves performance over FT Croce varies.  The choice of the initial attack seems to play a role in subsequent robustness, and if defenders are aware of multiple attacks, choosing the right one to start with is an interesting open question.  %We discuss this direction more in Section \ref{sec: discussion}.} 
%Overall, we find that the degree to which ALR helps in CRT to be dependent on dataset and attack sequence.  For ImageNette and sequence starting with $\ell_2$, we find that FT Croce + ALR can lead to an increase of at least 2.58\% in union (known attack) accuracy over FT Croce across time steps in the sequence.  However, for ImageNette with $\ell_{\infty}$ sequence, we find that the gain in performance of FT Croce + ALR over FT Croce can be quite small after the first time step; At time step 1, there is a gain of 3.19\%, 0.81\% at time step 2, and 1.4\% at time step 3. Meanwhile for CIFAR-100 with the $\ell_2$ attack sequence, we find that the gain in performance of FT Croce + ALR over FT Croce to be 2.6\% at time step 1 and 2.13\% at time step 2, but there is a 2.46\% drop in accuracy over FT Croce at time step 3.  Additionally for the $\ell_\infty$ attack sequence on CIFAR-100, improvements in through ALR can also be quite small, with a 1.21\% increase at time step 1, 1.26\% at time step 2, and 0.13\% at time step 3.  


%\begin{froval}
\begin{tcolorbox}[myboxstyle]
\begin{cfinding}
CRT+ALR improves robustness on both known and unforeseen attacks, and reduces drop in robustness on previous attacks with only a small overhead in fine-tuning time compared to unregularized CRT.\end{cfinding}
\end{tcolorbox}
%\end{froval}


\subsection{Ablation 1: Regularization in Initial Training}
\label{sec:init_train_impact}
We now study the impact of regularization \textit{only} in the initial training phase of CRT.  In Table \ref{tab:reg_unforeseen_rob}, we present results for robust accuracies of models initially trained on $\ell_2$ and $\ell_{\infty}$ attacks with different forms of regularization.  We present results for different regularization strengths and initial attack choices in \cref{app:init_train_different_attack_types}.

\noindent\textbf{Regularization improves robustness on unforeseen attacks.} Interestingly, we find that all regularization types including random noise-based regularization can improve unforeseen robustness.  For example, at $\lambda=5$, UR improves union accuracy across all attacks by 6.13\% for $\ell_2$ initial attack and by 7.05\% for $\ell_\infty$ initial attack compared to the model trained without regularization.  Improved unforeseen robustness provides a better starting point for fine-tuning, which we demonstrate experimentally in Appendix \ref{app:fine-tuning_pairs_init_train}.

\noindent\textbf{Trade-offs for clean and different attack accuracies. }We observe that all regularization types generally exhibit a trade-off with clean accuracy and trade-offs with a few attack types such as Glitch.  This trade-off aligns with \cref{thm:corollary} which states that the gap between clean loss and loss over the union of attacks is decreased via regularization. We also find that random noise based regularization (UR and GR) generally exhibits trade-off with the robust accuracy on the initial attack.  This is generally not the case for adversarial regularization (ALR and VR) which maintains performance on the initial attack. 

\noindent\textbf{Regularized initial models are better starting points for fine-tuning. } In Appendix \ref{app:fine-tuning_pairs_init_train}, we present results for fine-tuning with a new attack from models using regularization in only initial training.  We observe that for all regularization types, regularization in initial training can improve the robustness on the union of attacks after fine-tuning, but this trend is more consistent with adversarial regularization types (ALR and VR) compared to random regularization types (UR and GR).
% We now ask the question, can regularization in initial training help with robust accuracies after fine-tuning to a new attack?  To understand this, we consider initially training with one attack and then fine-tuning to a new attack. In Figure \ref{fig:fine-tune_abl_main_text_from_init}, we plot the difference in accuracies between a model fine-tuned from a checkpoint initially trained with regularization compared to a model fine-tuned from a checkpoint without regularization in initial training \edit{for ALR and UR.  We provide results for VR and GR in Appendix \ref{app:fine-tuning_pairs}.}  The attack in the row represents the initial attack and the attack in the column represents the new attack. For diagonal entries where the initial attack and new attack have the same attack type, we use a larger attack strength for the new attack (see experimental setup in \S\ref{sec:exp_setup} for full details). Gains in robustness from fine-tuning at least 1\% are colored in green while drops in robustness are colored in red at least 1\%.  We observe that that for all regularization types, regularization in initial training can improve the robustness on the union of attacks after fine-tuning, but this trend is more consistent with adversarial regularization types (ALR and VR) compared to random regularization types (UR and GR).  We also observe that the magnitude of these changes can be quite large; for example, when the initial attack is the Kaleidoscope attack and the new attack is Snow, incorporating ALR in initial training improves the resulting model's robust accuracy on the union of these attacks by 16.72\%.

\begin{tcolorbox}[myboxstyle]
    \begin{cfinding}\label{cfind: justification} Adversarial and random noise regularization in initial training improves performance on unforeseen attacks.  Fine-tuning on a new attack from a regularized model boosts resulting Union accuracy.
    \end{cfinding}
\end{tcolorbox}

\subsection{Ablation 2: Regularization during Fine-tuning}
\label{sec:fine-tuning_impact}

We now investigate whether regularization within just the the fine-tuning phase can improve CAR.  We initially train models on a single initial attack using adversarial training (\emph{without regularization}) and then fine-tune with \citet{croce2022adversarial}'s fine-tuning approach both with and without regularization on a new attack.  In Figure \ref{fig:fine-tune_abl_main_text}, we present grids representing differences in Union accuracy between regularized and unregularized fine-tuning.  Rows represent the initial attack used to adversarially train the model (without regularization), columns represent the new attack.  We provide corresponding plots detailing differences in average accuracy, initial attack accuracy, new attack accuracy, and clean accuracy in Appendix \ref{app:fine-tuning_pairs}.  

\noindent\textbf{Adversarial regularization can improve union accuracy in fine-tuning. } We find that across different initial and new attack pairs, using ALR in fine-tuning generally improves union accuracy as most cells in Figure \ref{fig:fine-tune_abl_main_text}(a) are green.  These increases in robustness can be quite large; for example, when the initial attack is StAdv \citep{XiaoZ0HLS18} and the new attack is Kaleidoscope \citep{kaufmann2019testing}, ALR improves robustness on the union by 8.66\%.  Additionally, when the initial attack is $\ell_2$ and the new attack is Snow \citep{kaufmann2019testing}, ALR improves robustness on the Union of both attacks by 7.85\%.  We find same trend holds for VR (\cref{app:fine-tuning_pairs}).

\noindent\textbf{Random noise based regularization is harmful when used in fine-tuning. }Although random noise based regularization can improve robustness when used in the initial training phase of CRT, Figure \ref{fig:fine-tune_abl_main_text}(b) demonstrates that UR in fine-tuning hurts union accuracy for many initial and new attack pairs (corresponding results for GR are present in \cref{app:fine-tuning_pairs}). This suggests that while random noise based regularization can be used to perform initial training more efficiently, they should not be used during fine-tuning.  Since we found that UR and GR trade off accuracy on the initial attack when used in initial training in \cref{sec:init_train_impact}, this suggests that UR and GR generally trade off performance on attacks that are used in training or fine-tuning.

\begin{tcolorbox}[myboxstyle]
    \begin{cfinding}\label{cfind: justification} In fine-tuning, adversarial regularization (ALR and VR) can improve Union accuracy significantly (up to $\sim 7\%$) while random noise-based regularization hurts Union accuracy. %Adversarial regularization (ALR and VR) can help improve union accuracy when used in fine-tuning. In contrast, we find that random noise based regularization can harm performance when used in fine-tuning across many pairs attack types.
    \end{cfinding}
\end{tcolorbox}

\section{Discussion, Limitations, and Conclusion}
\section{Discussion}
 
In the following sections, we first discuss the use of AI-RPM technologies in the remote monitoring of patient mental health symptoms.
Then, we talked about collaborative design in AI-RPM systems for enhancing usability and effectiveness.
Finally, we conclude with collaborative emergency management with AI-RPM systems. 
Our work contributes to CSCW and HCI research at the intersection of health, RPM, and human-AI decision-making.


\subsection{Unraveling Patients' Intertwined Mental Health and Concussion Symptoms}

In clinical scenarios, it is not uncommon to have multiple illnesses intertwined with each other.
For instance, ~\citet{wu2024cardioai} discovered that cancer oncologists need to collaborate with cardiologists to adjust the cancer treatment plan and be constantly aware of patients' cardiotoxicity.
%during cancer treatment.
Nevertheless, the concussion-mental health scenario and the cancer-cardiotoxicity scenario are fundamentally different.
Concussion clinicians, according to our study, do not collaborate with mental health specialists, and neither do they diagnose mental health disorders. 
In most clinical practices, concussion clinicians assess whether a patient's mental health symptoms exceed a certain severity threshold during in-person visits. Based on this assessment, they adjust the treatment plan accordingly.




We believe that AI-RPM technologies have the potential to give clinicians insights into the severity of patients' mental health symptoms.
First, LLM-based CAs could provide insight into patients' mental health symptoms, such as the level of anxiety. 
Moreover, AI-based predictive models could help assess the likelihood of developing mental health sequelae, which can offer valuable and actionable insights into patients' mental health conditions.
As suggested by \citet{bennett2012ehrs, zhang2024rethinking}, utilizing AI predictive models to provide actionable insights can increase the usefulness of AI in clinicians' workflow. 
Future research could explore the design of AI-RPM technologies for collecting and distinguishing youth patients’ overlapping symptoms.




%\iffalse
%\subsubsection{Remote Monitoring of Objective Patient Data}

%Our study revealed significant challenges for concussion clinicians in assessing mental health conditions among youth concussion patients. 
%Some youth patients and their parents often avoid openly discussing mental health concerns due to the stigma of being labeled with mental health disorders. 
%Consequently, clinicians use mental health self-evaluation questionnaires or direct observation of patients' behaviors during clinical visits to assess their mental health condition. 
%However, if patients do not want to disclose their mental health conditions, the validity of questionnaires can be compromised, leaving observation during clinical visits as the only viable option. 
%Yet this in-clinic observation is inherently limited from the geographical perspective.
%Once patients leave the clinic, concussion clinicians can hardly track patients' mental health status in a timely and effective manner.
% they are beyond the clinician's ability to monitor. 
%Even if patients are willing to faithfully and openly discuss their at-home information about mental health, it's different for patients to report their information (e.g., sleep patterns and heart rate) in a detailed and accurate way. 
% about following recommendations—such as maintaining sufficient daily activity and sleep—tends to be inaccurate.


%We suggest that AI-RPM technologies, particularly wearable devices like smartwatches, could help address the aforementioned limitations.
%Wearable devices can collect and provide key physiological metrics such as sleep patterns, physical activities, and heart rate to clinicians in real-time.
%Prior research demonstrated the feasibility and advantages of using wearable in monitoring objective patient physiological metrics outside the clinic~\cite{mendel2024advice, frazier2023six, su2019novel, wu2024cardioai}. 
%These metrics are objective and can faithfully reflect patients' mental health conditions.
%For instance, elevated anxiety levels may be inferred through reduced physical activity or disrupted sleep patterns, while a stable resting heart rate could indicate recovery progress. 
%Concussion clinicians could use such data for clinical decision-making such as scheduling timely follow-up visits or initiating referrals to mental health specialists as needed.
%Moreover, given that the patients in our setting are youth aged 13 to 17, wearable devices can monitor patients' status in a non-intrusive way, and they can be seamlessly integrated into patients' daily routines.
%This will not only prevent patients from experiencing physical pain but also minimize reminders of their concussion as much as possible, which is good for their recovery.

%\fi




\subsection{Augmenting Objective Data With Subjective Experience}
Traditional RPM technologies, such as wearables, have already been used to remotely monitor concussion patients' objective data~\cite{yang2020bidirectional}.
However, collecting objective data alone is not enough in our scenarios.
%Our study focuses uniquely on mental health sequelae in youth patients after a concussion.
As we discussed in the last section, it is critical to monitor the severity of mental health symptoms, such as anxiety and suicidal ideation.
However, these symptoms are more subjective experiences that can hardly be captured by traditional RPM technologies.
%because the data collected by wearables are standardized objective physiological measurements.
To address this gap, we propose using LLM-based CAs to gather patients' anxiety levels and other mental health conditions at home. 
%LLM-powered CAs have the ability to engage in free-form natural language conversations with patients about their subjective experiences.
LLM-based CAs could automatically detect key psychological risk factors such as suicidal thoughts from patients' conversations and generate a summary reportfor the day, which has been demoed in various prior research works in other scenarios~\cite{bartle2023machine, bartle2022second, wang2023enabling, simpson2020daisy, ma2024understanding, yang2023integrating}.


We believe that subjective information collected from LLM-powered CAs serves not only as supplementary information but also as contextual information for objective data (i.e., heart rate).
Prior HCI and CSCW research has highlighted the principle that technology design should incorporate context into data~\cite{dourish2004we, yoo2024missed}.
By combining subjective information and objective data in AI-RPM systems, clinicians could use the system to assess patients' situations more accurately and avoid unnecessary concern over what might otherwise appear to be worrisome in objective metrics alone. 
%For example, if wearable data shows that a patient had very low physical activity on a given day, the information from the LLM-powered CA might reveal that the low activity was only due to a family gathering at home. 
%Future work can explore the use of LLM-powered CAs and wearables to remotely monitor patients' subjective as well as objective data in other clinical settings to assist clinicians and caregivers.


%Compared with traditional RPM approaches, natural conversation significantly reduces the requirement of patients' technical literacy to engage with the devices, enables flexible patient self-reporting, and can be further designed to conduct regular health check-ins through pre-set question lists. 




%%%%%\subsubsection{Patient Privacy Consideration With AI-RPM Technologies}
%When designing such an AI-RPM system, it is essential to underscore protecting patient privacy and ensuring data security. 
%This issue has been widely discussed in many studies and is recognized as a core challenge in AI system design~\cite{gawankar2024patient, bala2024ensuring, korobenko2024towards}. 
%However, in our research context, the target population comprises youth patients aged 11–17, whose privacy needs are uniquely sensitive. 
%The privacy of youth patients is not only a concern for the patients themselves but is also critical to their parents, who often play a key role in the medical process. 
%Parents typically expect to have transparency and informed consent about how their child's data is collected and used~\cite{sisk2020parental, haley2024attitudes}. 
%Therefore, future research could explore methods to effectively safeguard youth health data while simultaneously increasing engagement and building trust among parents in the data collection and use process. 








\subsection{Collaborative Design in AI-RPM Systems for Enhanced Usability and Effectiveness}

When presenting objective and subjective data in an AI-RPM system, it is important to prioritize the data presentation to the usability of the system. 
Concussion clinicians are already overwhelmed by existing EHR systems, and introducing an AI-RPM system with additional data could further increase their workload and strain limited healthcare resources.
Nevertheless, clinicians expressed a willingness to briefly review AI-RPM data before each visit to better prepare for patient consultations.
%Introducing another AI-RPM system with additional data could significantly increase clinicians' workload and reduce the number of available weekly appointment slots, which further strain already limited healthcare resources. 
%However, concussion clinicians expressed the willingness to quickly review AI-RPM systems before each clinical visit to better prepare for patient consultations.
To provide information without increasing workload, designers need to collaborate with clinicians to prioritize data based on the importance of AI-RPM systems. 
%customized and personalized~\cite{amershi2019guidelines}
Key data should be upfront while keeping secondary information hidden but accessible.
Our work offers a refined system design that presents a clear information hierarchy in a visually intuitive manner (Fig. \ref{fig:refined_system_design}), serving as a reference for future development.




Moreover, domain knowledge is essential for defining thresholds of mental health symptom severity in AI-RPM system design.
Without clear definitions of severity, AI-RPM systems may either underreport or overemphasize symptoms, leading to delayed interventions or unnecessary concerns. 
Thus, researcher and designers should closely collaborate with clinicians to establish clear thresholds for the severity of mental health symptoms so that clinicians get alert by the system only when necessary.
Such a collaborative approach among clinicians, researchers, and designers could enhance the usability and effectiveness of clinician-facing AI-RPM systems.
In the future, researchers and designers should focus on the key thresholds that influence clinician decisions.













\subsection{Collaborative Emergency Management With AI-RPM Systems}
Beyond the design of AI-RPM systems, it is also important to consider how the systems support clinicians during emergency situations in real-world practice.

In our study, concussion clinicians view AI as an important collaborative partner to support their decisions, which aligns with precious research in HCI and CSCW communities~\cite{zhang2024rethinking, hao2024advancing, yang2019unremarkable, zhang2020effect}, 
%Our study highlighted that concussion clinicians view AI as an important collaborative partner to support their decisions,
%rather than relying solely on AI-generated results, 
%which aligns with the research in the HCI and CSCW communities ~\cite{zhang2024rethinking, hao2024advancing, yang2019unremarkable, zhang2020effect}.
However, they expressed two main concerns in emergencies.
The first concern is the potential false alert of "suicidal ideation" in the system. 
%While clinicians recognize the importance of such alerts in handling emergencies, they are concerned about the accuracy of the alert.
%—not only in terms of technical capabilities but also regarding the reliability of self-reported data from youth concussion patients. 
Youth patients may often experience mood swings and express emotions with exaggeration, which could lead to false alarms.
%If a youth patient says, “I want to hurt myself,” it may not reflect true intent, but the AI-RPM system could misclassify it as a high-risk alert, leading to false alarms.
%The implications of such false alarms are multifaceted. 
%First, 
As a result, a false alert could waste clinicians' valuable time and critical medical resources that could be allocated to other patients. 
Moreover, if false alarms occur frequently, clinicians could lose trust in the system, which eventually leads to the abandonment of such a system~\cite{liao2020questioning, amershi2019guidelines}. 
Another concern is the ambiguity of accountability that AI-RPM systems introduce. 
%This issue was also raised in our study. 
When an AI-RPM system triggers an emergency alert (e.g., suicidal ideation), clinicians may face time or resource constraints that delay their response. 
Such delays can lead to serious outcomes and raise questions about accountability, which may affect system adoption.

% This lack of clarity in the allocation of responsibility leaves clinicians feeling uneasy, as they remain solely liable for the consequences despite relying on AI assistance.


In high-risk, uncertain cases involving youth, it’s challenging to verify alerts and balance response with accountability.
We believe that addressing these two challenges requires collaboration among a broader set of stakeholders and clarifying accountability~\cite{goodman2017european} before implementing the system. 
%Although the previous studies have highlighted the phenomenon where clinicians, as final decision-makers, have full responsibility for medical outcomes, we suggest that responsibility needs to depend on different clinical situations and discussion with other departments such as legal or leadership teams.
Previous HCI research has primarily focused on involving clinicians and patients in emergency management with AI systems~\cite{wu2024cardioai, hao2024advancing}. 
We propose involving the youth patient’s family in the decision-making process alongside clinicians to manage emergencies. 
If suicidal ideation is detected, both the clinician and the patient’s family should be notified. 
Since family members are often nearby, they can respond quickly, while clinicians provide timely professional guidance.
%For example, if a patient's suicidal ideation is detected, the AI-RPM system should notify not only the clinician but also the patient’s family immediately.
%This allows family members to respond quickly, as they are often nearby or have better access to the patient.
%During this process, clinicians could collaborate with the family by providing timely professional guidance. 
Such collaboration helps mitigate risks from delayed responses while ensuring shared responsibility in emergencies.
Future research could explore parent-facing systems that integrate with clinicians' systems to enhance remote youth patient care.




% Acknowledgements should only appear in the accepted version.
%\section*{Acknowledgements}


% In the unusual situation where you want a paper to appear in the
% references without citing it in the main text, use \nocite
%\nocite{langley00}

\bibliography{example_paper}
\bibliographystyle{icml2024}


%%%%%%%%%%%%%%%%%%%%%%%%%%%%%%%%%%%%%%%%%%%%%%%%%%%%%%%%%%%%%%%%%%%%%%%%%%%%%%%
%%%%%%%%%%%%%%%%%%%%%%%%%%%%%%%%%%%%%%%%%%%%%%%%%%%%%%%%%%%%%%%%%%%%%%%%%%%%%%%
% APPENDIX
%%%%%%%%%%%%%%%%%%%%%%%%%%%%%%%%%%%%%%%%%%%%%%%%%%%%%%%%%%%%%%%%%%%%%%%%%%%%%%%
%%%%%%%%%%%%%%%%%%%%%%%%%%%%%%%%%%%%%%%%%%%%%%%%%%%%%%%%%%%%%%%%%%%%%%%%%%%%%%%
\newpage
\appendix
\onecolumn
\subsection{Lloyd-Max Algorithm}
\label{subsec:Lloyd-Max}
For a given quantization bitwidth $B$ and an operand $\bm{X}$, the Lloyd-Max algorithm finds $2^B$ quantization levels $\{\hat{x}_i\}_{i=1}^{2^B}$ such that quantizing $\bm{X}$ by rounding each scalar in $\bm{X}$ to the nearest quantization level minimizes the quantization MSE. 

The algorithm starts with an initial guess of quantization levels and then iteratively computes quantization thresholds $\{\tau_i\}_{i=1}^{2^B-1}$ and updates quantization levels $\{\hat{x}_i\}_{i=1}^{2^B}$. Specifically, at iteration $n$, thresholds are set to the midpoints of the previous iteration's levels:
\begin{align*}
    \tau_i^{(n)}=\frac{\hat{x}_i^{(n-1)}+\hat{x}_{i+1}^{(n-1)}}2 \text{ for } i=1\ldots 2^B-1
\end{align*}
Subsequently, the quantization levels are re-computed as conditional means of the data regions defined by the new thresholds:
\begin{align*}
    \hat{x}_i^{(n)}=\mathbb{E}\left[ \bm{X} \big| \bm{X}\in [\tau_{i-1}^{(n)},\tau_i^{(n)}] \right] \text{ for } i=1\ldots 2^B
\end{align*}
where to satisfy boundary conditions we have $\tau_0=-\infty$ and $\tau_{2^B}=\infty$. The algorithm iterates the above steps until convergence.

Figure \ref{fig:lm_quant} compares the quantization levels of a $7$-bit floating point (E3M3) quantizer (left) to a $7$-bit Lloyd-Max quantizer (right) when quantizing a layer of weights from the GPT3-126M model at a per-tensor granularity. As shown, the Lloyd-Max quantizer achieves substantially lower quantization MSE. Further, Table \ref{tab:FP7_vs_LM7} shows the superior perplexity achieved by Lloyd-Max quantizers for bitwidths of $7$, $6$ and $5$. The difference between the quantizers is clear at 5 bits, where per-tensor FP quantization incurs a drastic and unacceptable increase in perplexity, while Lloyd-Max quantization incurs a much smaller increase. Nevertheless, we note that even the optimal Lloyd-Max quantizer incurs a notable ($\sim 1.5$) increase in perplexity due to the coarse granularity of quantization. 

\begin{figure}[h]
  \centering
  \includegraphics[width=0.7\linewidth]{sections/figures/LM7_FP7.pdf}
  \caption{\small Quantization levels and the corresponding quantization MSE of Floating Point (left) vs Lloyd-Max (right) Quantizers for a layer of weights in the GPT3-126M model.}
  \label{fig:lm_quant}
\end{figure}

\begin{table}[h]\scriptsize
\begin{center}
\caption{\label{tab:FP7_vs_LM7} \small Comparing perplexity (lower is better) achieved by floating point quantizers and Lloyd-Max quantizers on a GPT3-126M model for the Wikitext-103 dataset.}
\begin{tabular}{c|cc|c}
\hline
 \multirow{2}{*}{\textbf{Bitwidth}} & \multicolumn{2}{|c|}{\textbf{Floating-Point Quantizer}} & \textbf{Lloyd-Max Quantizer} \\
 & Best Format & Wikitext-103 Perplexity & Wikitext-103 Perplexity \\
\hline
7 & E3M3 & 18.32 & 18.27 \\
6 & E3M2 & 19.07 & 18.51 \\
5 & E4M0 & 43.89 & 19.71 \\
\hline
\end{tabular}
\end{center}
\end{table}

\subsection{Proof of Local Optimality of LO-BCQ}
\label{subsec:lobcq_opt_proof}
For a given block $\bm{b}_j$, the quantization MSE during LO-BCQ can be empirically evaluated as $\frac{1}{L_b}\lVert \bm{b}_j- \bm{\hat{b}}_j\rVert^2_2$ where $\bm{\hat{b}}_j$ is computed from equation (\ref{eq:clustered_quantization_definition}) as $C_{f(\bm{b}_j)}(\bm{b}_j)$. Further, for a given block cluster $\mathcal{B}_i$, we compute the quantization MSE as $\frac{1}{|\mathcal{B}_{i}|}\sum_{\bm{b} \in \mathcal{B}_{i}} \frac{1}{L_b}\lVert \bm{b}- C_i^{(n)}(\bm{b})\rVert^2_2$. Therefore, at the end of iteration $n$, we evaluate the overall quantization MSE $J^{(n)}$ for a given operand $\bm{X}$ composed of $N_c$ block clusters as:
\begin{align*}
    \label{eq:mse_iter_n}
    J^{(n)} = \frac{1}{N_c} \sum_{i=1}^{N_c} \frac{1}{|\mathcal{B}_{i}^{(n)}|}\sum_{\bm{v} \in \mathcal{B}_{i}^{(n)}} \frac{1}{L_b}\lVert \bm{b}- B_i^{(n)}(\bm{b})\rVert^2_2
\end{align*}

At the end of iteration $n$, the codebooks are updated from $\mathcal{C}^{(n-1)}$ to $\mathcal{C}^{(n)}$. However, the mapping of a given vector $\bm{b}_j$ to quantizers $\mathcal{C}^{(n)}$ remains as  $f^{(n)}(\bm{b}_j)$. At the next iteration, during the vector clustering step, $f^{(n+1)}(\bm{b}_j)$ finds new mapping of $\bm{b}_j$ to updated codebooks $\mathcal{C}^{(n)}$ such that the quantization MSE over the candidate codebooks is minimized. Therefore, we obtain the following result for $\bm{b}_j$:
\begin{align*}
\frac{1}{L_b}\lVert \bm{b}_j - C_{f^{(n+1)}(\bm{b}_j)}^{(n)}(\bm{b}_j)\rVert^2_2 \le \frac{1}{L_b}\lVert \bm{b}_j - C_{f^{(n)}(\bm{b}_j)}^{(n)}(\bm{b}_j)\rVert^2_2
\end{align*}

That is, quantizing $\bm{b}_j$ at the end of the block clustering step of iteration $n+1$ results in lower quantization MSE compared to quantizing at the end of iteration $n$. Since this is true for all $\bm{b} \in \bm{X}$, we assert the following:
\begin{equation}
\begin{split}
\label{eq:mse_ineq_1}
    \tilde{J}^{(n+1)} &= \frac{1}{N_c} \sum_{i=1}^{N_c} \frac{1}{|\mathcal{B}_{i}^{(n+1)}|}\sum_{\bm{b} \in \mathcal{B}_{i}^{(n+1)}} \frac{1}{L_b}\lVert \bm{b} - C_i^{(n)}(b)\rVert^2_2 \le J^{(n)}
\end{split}
\end{equation}
where $\tilde{J}^{(n+1)}$ is the the quantization MSE after the vector clustering step at iteration $n+1$.

Next, during the codebook update step (\ref{eq:quantizers_update}) at iteration $n+1$, the per-cluster codebooks $\mathcal{C}^{(n)}$ are updated to $\mathcal{C}^{(n+1)}$ by invoking the Lloyd-Max algorithm \citep{Lloyd}. We know that for any given value distribution, the Lloyd-Max algorithm minimizes the quantization MSE. Therefore, for a given vector cluster $\mathcal{B}_i$ we obtain the following result:

\begin{equation}
    \frac{1}{|\mathcal{B}_{i}^{(n+1)}|}\sum_{\bm{b} \in \mathcal{B}_{i}^{(n+1)}} \frac{1}{L_b}\lVert \bm{b}- C_i^{(n+1)}(\bm{b})\rVert^2_2 \le \frac{1}{|\mathcal{B}_{i}^{(n+1)}|}\sum_{\bm{b} \in \mathcal{B}_{i}^{(n+1)}} \frac{1}{L_b}\lVert \bm{b}- C_i^{(n)}(\bm{b})\rVert^2_2
\end{equation}

The above equation states that quantizing the given block cluster $\mathcal{B}_i$ after updating the associated codebook from $C_i^{(n)}$ to $C_i^{(n+1)}$ results in lower quantization MSE. Since this is true for all the block clusters, we derive the following result: 
\begin{equation}
\begin{split}
\label{eq:mse_ineq_2}
     J^{(n+1)} &= \frac{1}{N_c} \sum_{i=1}^{N_c} \frac{1}{|\mathcal{B}_{i}^{(n+1)}|}\sum_{\bm{b} \in \mathcal{B}_{i}^{(n+1)}} \frac{1}{L_b}\lVert \bm{b}- C_i^{(n+1)}(\bm{b})\rVert^2_2  \le \tilde{J}^{(n+1)}   
\end{split}
\end{equation}

Following (\ref{eq:mse_ineq_1}) and (\ref{eq:mse_ineq_2}), we find that the quantization MSE is non-increasing for each iteration, that is, $J^{(1)} \ge J^{(2)} \ge J^{(3)} \ge \ldots \ge J^{(M)}$ where $M$ is the maximum number of iterations. 
%Therefore, we can say that if the algorithm converges, then it must be that it has converged to a local minimum. 
\hfill $\blacksquare$


\begin{figure}
    \begin{center}
    \includegraphics[width=0.5\textwidth]{sections//figures/mse_vs_iter.pdf}
    \end{center}
    \caption{\small NMSE vs iterations during LO-BCQ compared to other block quantization proposals}
    \label{fig:nmse_vs_iter}
\end{figure}

Figure \ref{fig:nmse_vs_iter} shows the empirical convergence of LO-BCQ across several block lengths and number of codebooks. Also, the MSE achieved by LO-BCQ is compared to baselines such as MXFP and VSQ. As shown, LO-BCQ converges to a lower MSE than the baselines. Further, we achieve better convergence for larger number of codebooks ($N_c$) and for a smaller block length ($L_b$), both of which increase the bitwidth of BCQ (see Eq \ref{eq:bitwidth_bcq}).


\subsection{Additional Accuracy Results}
%Table \ref{tab:lobcq_config} lists the various LOBCQ configurations and their corresponding bitwidths.
\begin{table}
\setlength{\tabcolsep}{4.75pt}
\begin{center}
\caption{\label{tab:lobcq_config} Various LO-BCQ configurations and their bitwidths.}
\begin{tabular}{|c||c|c|c|c||c|c||c|} 
\hline
 & \multicolumn{4}{|c||}{$L_b=8$} & \multicolumn{2}{|c||}{$L_b=4$} & $L_b=2$ \\
 \hline
 \backslashbox{$L_A$\kern-1em}{\kern-1em$N_c$} & 2 & 4 & 8 & 16 & 2 & 4 & 2 \\
 \hline
 64 & 4.25 & 4.375 & 4.5 & 4.625 & 4.375 & 4.625 & 4.625\\
 \hline
 32 & 4.375 & 4.5 & 4.625& 4.75 & 4.5 & 4.75 & 4.75 \\
 \hline
 16 & 4.625 & 4.75& 4.875 & 5 & 4.75 & 5 & 5 \\
 \hline
\end{tabular}
\end{center}
\end{table}

%\subsection{Perplexity achieved by various LO-BCQ configurations on Wikitext-103 dataset}

\begin{table} \centering
\begin{tabular}{|c||c|c|c|c||c|c||c|} 
\hline
 $L_b \rightarrow$& \multicolumn{4}{c||}{8} & \multicolumn{2}{c||}{4} & 2\\
 \hline
 \backslashbox{$L_A$\kern-1em}{\kern-1em$N_c$} & 2 & 4 & 8 & 16 & 2 & 4 & 2  \\
 %$N_c \rightarrow$ & 2 & 4 & 8 & 16 & 2 & 4 & 2 \\
 \hline
 \hline
 \multicolumn{8}{c}{GPT3-1.3B (FP32 PPL = 9.98)} \\ 
 \hline
 \hline
 64 & 10.40 & 10.23 & 10.17 & 10.15 &  10.28 & 10.18 & 10.19 \\
 \hline
 32 & 10.25 & 10.20 & 10.15 & 10.12 &  10.23 & 10.17 & 10.17 \\
 \hline
 16 & 10.22 & 10.16 & 10.10 & 10.09 &  10.21 & 10.14 & 10.16 \\
 \hline
  \hline
 \multicolumn{8}{c}{GPT3-8B (FP32 PPL = 7.38)} \\ 
 \hline
 \hline
 64 & 7.61 & 7.52 & 7.48 &  7.47 &  7.55 &  7.49 & 7.50 \\
 \hline
 32 & 7.52 & 7.50 & 7.46 &  7.45 &  7.52 &  7.48 & 7.48  \\
 \hline
 16 & 7.51 & 7.48 & 7.44 &  7.44 &  7.51 &  7.49 & 7.47  \\
 \hline
\end{tabular}
\caption{\label{tab:ppl_gpt3_abalation} Wikitext-103 perplexity across GPT3-1.3B and 8B models.}
\end{table}

\begin{table} \centering
\begin{tabular}{|c||c|c|c|c||} 
\hline
 $L_b \rightarrow$& \multicolumn{4}{c||}{8}\\
 \hline
 \backslashbox{$L_A$\kern-1em}{\kern-1em$N_c$} & 2 & 4 & 8 & 16 \\
 %$N_c \rightarrow$ & 2 & 4 & 8 & 16 & 2 & 4 & 2 \\
 \hline
 \hline
 \multicolumn{5}{|c|}{Llama2-7B (FP32 PPL = 5.06)} \\ 
 \hline
 \hline
 64 & 5.31 & 5.26 & 5.19 & 5.18  \\
 \hline
 32 & 5.23 & 5.25 & 5.18 & 5.15  \\
 \hline
 16 & 5.23 & 5.19 & 5.16 & 5.14  \\
 \hline
 \multicolumn{5}{|c|}{Nemotron4-15B (FP32 PPL = 5.87)} \\ 
 \hline
 \hline
 64  & 6.3 & 6.20 & 6.13 & 6.08  \\
 \hline
 32  & 6.24 & 6.12 & 6.07 & 6.03  \\
 \hline
 16  & 6.12 & 6.14 & 6.04 & 6.02  \\
 \hline
 \multicolumn{5}{|c|}{Nemotron4-340B (FP32 PPL = 3.48)} \\ 
 \hline
 \hline
 64 & 3.67 & 3.62 & 3.60 & 3.59 \\
 \hline
 32 & 3.63 & 3.61 & 3.59 & 3.56 \\
 \hline
 16 & 3.61 & 3.58 & 3.57 & 3.55 \\
 \hline
\end{tabular}
\caption{\label{tab:ppl_llama7B_nemo15B} Wikitext-103 perplexity compared to FP32 baseline in Llama2-7B and Nemotron4-15B, 340B models}
\end{table}

%\subsection{Perplexity achieved by various LO-BCQ configurations on MMLU dataset}


\begin{table} \centering
\begin{tabular}{|c||c|c|c|c||c|c|c|c|} 
\hline
 $L_b \rightarrow$& \multicolumn{4}{c||}{8} & \multicolumn{4}{c||}{8}\\
 \hline
 \backslashbox{$L_A$\kern-1em}{\kern-1em$N_c$} & 2 & 4 & 8 & 16 & 2 & 4 & 8 & 16  \\
 %$N_c \rightarrow$ & 2 & 4 & 8 & 16 & 2 & 4 & 2 \\
 \hline
 \hline
 \multicolumn{5}{|c|}{Llama2-7B (FP32 Accuracy = 45.8\%)} & \multicolumn{4}{|c|}{Llama2-70B (FP32 Accuracy = 69.12\%)} \\ 
 \hline
 \hline
 64 & 43.9 & 43.4 & 43.9 & 44.9 & 68.07 & 68.27 & 68.17 & 68.75 \\
 \hline
 32 & 44.5 & 43.8 & 44.9 & 44.5 & 68.37 & 68.51 & 68.35 & 68.27  \\
 \hline
 16 & 43.9 & 42.7 & 44.9 & 45 & 68.12 & 68.77 & 68.31 & 68.59  \\
 \hline
 \hline
 \multicolumn{5}{|c|}{GPT3-22B (FP32 Accuracy = 38.75\%)} & \multicolumn{4}{|c|}{Nemotron4-15B (FP32 Accuracy = 64.3\%)} \\ 
 \hline
 \hline
 64 & 36.71 & 38.85 & 38.13 & 38.92 & 63.17 & 62.36 & 63.72 & 64.09 \\
 \hline
 32 & 37.95 & 38.69 & 39.45 & 38.34 & 64.05 & 62.30 & 63.8 & 64.33  \\
 \hline
 16 & 38.88 & 38.80 & 38.31 & 38.92 & 63.22 & 63.51 & 63.93 & 64.43  \\
 \hline
\end{tabular}
\caption{\label{tab:mmlu_abalation} Accuracy on MMLU dataset across GPT3-22B, Llama2-7B, 70B and Nemotron4-15B models.}
\end{table}


%\subsection{Perplexity achieved by various LO-BCQ configurations on LM evaluation harness}

\begin{table} \centering
\begin{tabular}{|c||c|c|c|c||c|c|c|c|} 
\hline
 $L_b \rightarrow$& \multicolumn{4}{c||}{8} & \multicolumn{4}{c||}{8}\\
 \hline
 \backslashbox{$L_A$\kern-1em}{\kern-1em$N_c$} & 2 & 4 & 8 & 16 & 2 & 4 & 8 & 16  \\
 %$N_c \rightarrow$ & 2 & 4 & 8 & 16 & 2 & 4 & 2 \\
 \hline
 \hline
 \multicolumn{5}{|c|}{Race (FP32 Accuracy = 37.51\%)} & \multicolumn{4}{|c|}{Boolq (FP32 Accuracy = 64.62\%)} \\ 
 \hline
 \hline
 64 & 36.94 & 37.13 & 36.27 & 37.13 & 63.73 & 62.26 & 63.49 & 63.36 \\
 \hline
 32 & 37.03 & 36.36 & 36.08 & 37.03 & 62.54 & 63.51 & 63.49 & 63.55  \\
 \hline
 16 & 37.03 & 37.03 & 36.46 & 37.03 & 61.1 & 63.79 & 63.58 & 63.33  \\
 \hline
 \hline
 \multicolumn{5}{|c|}{Winogrande (FP32 Accuracy = 58.01\%)} & \multicolumn{4}{|c|}{Piqa (FP32 Accuracy = 74.21\%)} \\ 
 \hline
 \hline
 64 & 58.17 & 57.22 & 57.85 & 58.33 & 73.01 & 73.07 & 73.07 & 72.80 \\
 \hline
 32 & 59.12 & 58.09 & 57.85 & 58.41 & 73.01 & 73.94 & 72.74 & 73.18  \\
 \hline
 16 & 57.93 & 58.88 & 57.93 & 58.56 & 73.94 & 72.80 & 73.01 & 73.94  \\
 \hline
\end{tabular}
\caption{\label{tab:mmlu_abalation} Accuracy on LM evaluation harness tasks on GPT3-1.3B model.}
\end{table}

\begin{table} \centering
\begin{tabular}{|c||c|c|c|c||c|c|c|c|} 
\hline
 $L_b \rightarrow$& \multicolumn{4}{c||}{8} & \multicolumn{4}{c||}{8}\\
 \hline
 \backslashbox{$L_A$\kern-1em}{\kern-1em$N_c$} & 2 & 4 & 8 & 16 & 2 & 4 & 8 & 16  \\
 %$N_c \rightarrow$ & 2 & 4 & 8 & 16 & 2 & 4 & 2 \\
 \hline
 \hline
 \multicolumn{5}{|c|}{Race (FP32 Accuracy = 41.34\%)} & \multicolumn{4}{|c|}{Boolq (FP32 Accuracy = 68.32\%)} \\ 
 \hline
 \hline
 64 & 40.48 & 40.10 & 39.43 & 39.90 & 69.20 & 68.41 & 69.45 & 68.56 \\
 \hline
 32 & 39.52 & 39.52 & 40.77 & 39.62 & 68.32 & 67.43 & 68.17 & 69.30  \\
 \hline
 16 & 39.81 & 39.71 & 39.90 & 40.38 & 68.10 & 66.33 & 69.51 & 69.42  \\
 \hline
 \hline
 \multicolumn{5}{|c|}{Winogrande (FP32 Accuracy = 67.88\%)} & \multicolumn{4}{|c|}{Piqa (FP32 Accuracy = 78.78\%)} \\ 
 \hline
 \hline
 64 & 66.85 & 66.61 & 67.72 & 67.88 & 77.31 & 77.42 & 77.75 & 77.64 \\
 \hline
 32 & 67.25 & 67.72 & 67.72 & 67.00 & 77.31 & 77.04 & 77.80 & 77.37  \\
 \hline
 16 & 68.11 & 68.90 & 67.88 & 67.48 & 77.37 & 78.13 & 78.13 & 77.69  \\
 \hline
\end{tabular}
\caption{\label{tab:mmlu_abalation} Accuracy on LM evaluation harness tasks on GPT3-8B model.}
\end{table}

\begin{table} \centering
\begin{tabular}{|c||c|c|c|c||c|c|c|c|} 
\hline
 $L_b \rightarrow$& \multicolumn{4}{c||}{8} & \multicolumn{4}{c||}{8}\\
 \hline
 \backslashbox{$L_A$\kern-1em}{\kern-1em$N_c$} & 2 & 4 & 8 & 16 & 2 & 4 & 8 & 16  \\
 %$N_c \rightarrow$ & 2 & 4 & 8 & 16 & 2 & 4 & 2 \\
 \hline
 \hline
 \multicolumn{5}{|c|}{Race (FP32 Accuracy = 40.67\%)} & \multicolumn{4}{|c|}{Boolq (FP32 Accuracy = 76.54\%)} \\ 
 \hline
 \hline
 64 & 40.48 & 40.10 & 39.43 & 39.90 & 75.41 & 75.11 & 77.09 & 75.66 \\
 \hline
 32 & 39.52 & 39.52 & 40.77 & 39.62 & 76.02 & 76.02 & 75.96 & 75.35  \\
 \hline
 16 & 39.81 & 39.71 & 39.90 & 40.38 & 75.05 & 73.82 & 75.72 & 76.09  \\
 \hline
 \hline
 \multicolumn{5}{|c|}{Winogrande (FP32 Accuracy = 70.64\%)} & \multicolumn{4}{|c|}{Piqa (FP32 Accuracy = 79.16\%)} \\ 
 \hline
 \hline
 64 & 69.14 & 70.17 & 70.17 & 70.56 & 78.24 & 79.00 & 78.62 & 78.73 \\
 \hline
 32 & 70.96 & 69.69 & 71.27 & 69.30 & 78.56 & 79.49 & 79.16 & 78.89  \\
 \hline
 16 & 71.03 & 69.53 & 69.69 & 70.40 & 78.13 & 79.16 & 79.00 & 79.00  \\
 \hline
\end{tabular}
\caption{\label{tab:mmlu_abalation} Accuracy on LM evaluation harness tasks on GPT3-22B model.}
\end{table}

\begin{table} \centering
\begin{tabular}{|c||c|c|c|c||c|c|c|c|} 
\hline
 $L_b \rightarrow$& \multicolumn{4}{c||}{8} & \multicolumn{4}{c||}{8}\\
 \hline
 \backslashbox{$L_A$\kern-1em}{\kern-1em$N_c$} & 2 & 4 & 8 & 16 & 2 & 4 & 8 & 16  \\
 %$N_c \rightarrow$ & 2 & 4 & 8 & 16 & 2 & 4 & 2 \\
 \hline
 \hline
 \multicolumn{5}{|c|}{Race (FP32 Accuracy = 44.4\%)} & \multicolumn{4}{|c|}{Boolq (FP32 Accuracy = 79.29\%)} \\ 
 \hline
 \hline
 64 & 42.49 & 42.51 & 42.58 & 43.45 & 77.58 & 77.37 & 77.43 & 78.1 \\
 \hline
 32 & 43.35 & 42.49 & 43.64 & 43.73 & 77.86 & 75.32 & 77.28 & 77.86  \\
 \hline
 16 & 44.21 & 44.21 & 43.64 & 42.97 & 78.65 & 77 & 76.94 & 77.98  \\
 \hline
 \hline
 \multicolumn{5}{|c|}{Winogrande (FP32 Accuracy = 69.38\%)} & \multicolumn{4}{|c|}{Piqa (FP32 Accuracy = 78.07\%)} \\ 
 \hline
 \hline
 64 & 68.9 & 68.43 & 69.77 & 68.19 & 77.09 & 76.82 & 77.09 & 77.86 \\
 \hline
 32 & 69.38 & 68.51 & 68.82 & 68.90 & 78.07 & 76.71 & 78.07 & 77.86  \\
 \hline
 16 & 69.53 & 67.09 & 69.38 & 68.90 & 77.37 & 77.8 & 77.91 & 77.69  \\
 \hline
\end{tabular}
\caption{\label{tab:mmlu_abalation} Accuracy on LM evaluation harness tasks on Llama2-7B model.}
\end{table}

\begin{table} \centering
\begin{tabular}{|c||c|c|c|c||c|c|c|c|} 
\hline
 $L_b \rightarrow$& \multicolumn{4}{c||}{8} & \multicolumn{4}{c||}{8}\\
 \hline
 \backslashbox{$L_A$\kern-1em}{\kern-1em$N_c$} & 2 & 4 & 8 & 16 & 2 & 4 & 8 & 16  \\
 %$N_c \rightarrow$ & 2 & 4 & 8 & 16 & 2 & 4 & 2 \\
 \hline
 \hline
 \multicolumn{5}{|c|}{Race (FP32 Accuracy = 48.8\%)} & \multicolumn{4}{|c|}{Boolq (FP32 Accuracy = 85.23\%)} \\ 
 \hline
 \hline
 64 & 49.00 & 49.00 & 49.28 & 48.71 & 82.82 & 84.28 & 84.03 & 84.25 \\
 \hline
 32 & 49.57 & 48.52 & 48.33 & 49.28 & 83.85 & 84.46 & 84.31 & 84.93  \\
 \hline
 16 & 49.85 & 49.09 & 49.28 & 48.99 & 85.11 & 84.46 & 84.61 & 83.94  \\
 \hline
 \hline
 \multicolumn{5}{|c|}{Winogrande (FP32 Accuracy = 79.95\%)} & \multicolumn{4}{|c|}{Piqa (FP32 Accuracy = 81.56\%)} \\ 
 \hline
 \hline
 64 & 78.77 & 78.45 & 78.37 & 79.16 & 81.45 & 80.69 & 81.45 & 81.5 \\
 \hline
 32 & 78.45 & 79.01 & 78.69 & 80.66 & 81.56 & 80.58 & 81.18 & 81.34  \\
 \hline
 16 & 79.95 & 79.56 & 79.79 & 79.72 & 81.28 & 81.66 & 81.28 & 80.96  \\
 \hline
\end{tabular}
\caption{\label{tab:mmlu_abalation} Accuracy on LM evaluation harness tasks on Llama2-70B model.}
\end{table}

%\section{MSE Studies}
%\textcolor{red}{TODO}


\subsection{Number Formats and Quantization Method}
\label{subsec:numFormats_quantMethod}
\subsubsection{Integer Format}
An $n$-bit signed integer (INT) is typically represented with a 2s-complement format \citep{yao2022zeroquant,xiao2023smoothquant,dai2021vsq}, where the most significant bit denotes the sign.

\subsubsection{Floating Point Format}
An $n$-bit signed floating point (FP) number $x$ comprises of a 1-bit sign ($x_{\mathrm{sign}}$), $B_m$-bit mantissa ($x_{\mathrm{mant}}$) and $B_e$-bit exponent ($x_{\mathrm{exp}}$) such that $B_m+B_e=n-1$. The associated constant exponent bias ($E_{\mathrm{bias}}$) is computed as $(2^{{B_e}-1}-1)$. We denote this format as $E_{B_e}M_{B_m}$.  

\subsubsection{Quantization Scheme}
\label{subsec:quant_method}
A quantization scheme dictates how a given unquantized tensor is converted to its quantized representation. We consider FP formats for the purpose of illustration. Given an unquantized tensor $\bm{X}$ and an FP format $E_{B_e}M_{B_m}$, we first, we compute the quantization scale factor $s_X$ that maps the maximum absolute value of $\bm{X}$ to the maximum quantization level of the $E_{B_e}M_{B_m}$ format as follows:
\begin{align}
\label{eq:sf}
    s_X = \frac{\mathrm{max}(|\bm{X}|)}{\mathrm{max}(E_{B_e}M_{B_m})}
\end{align}
In the above equation, $|\cdot|$ denotes the absolute value function.

Next, we scale $\bm{X}$ by $s_X$ and quantize it to $\hat{\bm{X}}$ by rounding it to the nearest quantization level of $E_{B_e}M_{B_m}$ as:

\begin{align}
\label{eq:tensor_quant}
    \hat{\bm{X}} = \text{round-to-nearest}\left(\frac{\bm{X}}{s_X}, E_{B_e}M_{B_m}\right)
\end{align}

We perform dynamic max-scaled quantization \citep{wu2020integer}, where the scale factor $s$ for activations is dynamically computed during runtime.

\subsection{Vector Scaled Quantization}
\begin{wrapfigure}{r}{0.35\linewidth}
  \centering
  \includegraphics[width=\linewidth]{sections/figures/vsquant.jpg}
  \caption{\small Vectorwise decomposition for per-vector scaled quantization (VSQ \citep{dai2021vsq}).}
  \label{fig:vsquant}
\end{wrapfigure}
During VSQ \citep{dai2021vsq}, the operand tensors are decomposed into 1D vectors in a hardware friendly manner as shown in Figure \ref{fig:vsquant}. Since the decomposed tensors are used as operands in matrix multiplications during inference, it is beneficial to perform this decomposition along the reduction dimension of the multiplication. The vectorwise quantization is performed similar to tensorwise quantization described in Equations \ref{eq:sf} and \ref{eq:tensor_quant}, where a scale factor $s_v$ is required for each vector $\bm{v}$ that maps the maximum absolute value of that vector to the maximum quantization level. While smaller vector lengths can lead to larger accuracy gains, the associated memory and computational overheads due to the per-vector scale factors increases. To alleviate these overheads, VSQ \citep{dai2021vsq} proposed a second level quantization of the per-vector scale factors to unsigned integers, while MX \citep{rouhani2023shared} quantizes them to integer powers of 2 (denoted as $2^{INT}$).

\subsubsection{MX Format}
The MX format proposed in \citep{rouhani2023microscaling} introduces the concept of sub-block shifting. For every two scalar elements of $b$-bits each, there is a shared exponent bit. The value of this exponent bit is determined through an empirical analysis that targets minimizing quantization MSE. We note that the FP format $E_{1}M_{b}$ is strictly better than MX from an accuracy perspective since it allocates a dedicated exponent bit to each scalar as opposed to sharing it across two scalars. Therefore, we conservatively bound the accuracy of a $b+2$-bit signed MX format with that of a $E_{1}M_{b}$ format in our comparisons. For instance, we use E1M2 format as a proxy for MX4.

\begin{figure}
    \centering
    \includegraphics[width=1\linewidth]{sections//figures/BlockFormats.pdf}
    \caption{\small Comparing LO-BCQ to MX format.}
    \label{fig:block_formats}
\end{figure}

Figure \ref{fig:block_formats} compares our $4$-bit LO-BCQ block format to MX \citep{rouhani2023microscaling}. As shown, both LO-BCQ and MX decompose a given operand tensor into block arrays and each block array into blocks. Similar to MX, we find that per-block quantization ($L_b < L_A$) leads to better accuracy due to increased flexibility. While MX achieves this through per-block $1$-bit micro-scales, we associate a dedicated codebook to each block through a per-block codebook selector. Further, MX quantizes the per-block array scale-factor to E8M0 format without per-tensor scaling. In contrast during LO-BCQ, we find that per-tensor scaling combined with quantization of per-block array scale-factor to E4M3 format results in superior inference accuracy across models. 


%%%%%%%%%%%%%%%%%%%%%%%%%%%%%%%%%%%%%%%%%%%%%%%%%%%%%%%%%%%%%%%%%%%%%%%%%%%%%%%
%%%%%%%%%%%%%%%%%%%%%%%%%%%%%%%%%%%%%%%%%%%%%%%%%%%%%%%%%%%%%%%%%%%%%%%%%%%%%%%


\end{document}


% This document was modified from the file originally made available by
% Pat Langley and Andrea Danyluk for ICML-2K. This version was created
% by Iain Murray in 2018, and modified by Alexandre Bouchard in
% 2019 and 2021 and by Csaba Szepesvari, Gang Niu and Sivan Sabato in 2022.
% Modified again in 2023 and 2024 by Sivan Sabato and Jonathan Scarlett.
% Previous contributors include Dan Roy, Lise Getoor and Tobias
% Scheffer, which was slightly modified from the 2010 version by
% Thorsten Joachims & Johannes Fuernkranz, slightly modified from the
% 2009 version by Kiri Wagstaff and Sam Roweis's 2008 version, which is
% slightly modified from Prasad Tadepalli's 2007 version which is a
% lightly changed version of the previous year's version by Andrew
% Moore, which was in turn edited from those of Kristian Kersting and
% Codrina Lauth. Alex Smola contributed to the algorithmic style files.
