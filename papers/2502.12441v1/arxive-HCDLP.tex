\documentclass[12pt,a4paper]{article}
\usepackage[a4paper]{geometry}
\usepackage{amssymb,latexsym,amsmath,amsfonts,amsthm}
\usepackage{graphicx}
\usepackage{epstopdf}
\usepackage{float,amsmath,amssymb,mathrsfs,bm,multirow,graphics}
\usepackage{tikz}
\usepackage{pgflibraryshapes}
\usetikzlibrary{arrows,decorations.markings}
\usepackage{subfigure}
\usepackage{overpic}
\usepackage{fullpage}
%\usepackage{arydshln}
%\usepackage{epsfig}
%\usepackage{times}
%\usepackage{mathtime}
\usepackage{comment,verbatim}
%\usepackage{graphicx,ebezier}
\usepackage{hyperref}
\usepackage{color}
\usepackage{mathrsfs}
%\usepackage{refcheck}
\usepackage[title,titletoc]{appendix}
\usepackage{ulem}
\usepackage{enumitem}

\usepackage{graphicx}% Include figure files
\usepackage{amssymb}
\usepackage{amsmath}
\usepackage{bm}
%\usepackage{amsthm}
\usepackage{color}
\usepackage{epstopdf}
\usepackage{algorithm}
\usepackage{algorithmic}
\usepackage{multicol}
%\usepackage{setspace}
\usepackage[justification=centering]{caption}
%\usepackage[utf8]{inputenc}
\RequirePackage{fix-cm}
\renewcommand{\algorithmicrequire}{ \textbf{Inputs:}}
\renewcommand{\algorithmicensure}{ \textbf{Output:}}
\newtheorem{lma}{Lemma}
\newtheorem{thm}[lma]{Theorem}
\DeclareMathOperator{\diag}{diag}
\DeclareMathOperator*{\Res}{Res}
\DeclareMathOperator{\Airy}{Airy}
\DeclareMathOperator{\CHF}{CHF}
\newcommand{\Bes}{\mathrm{Bes}}
\newcommand{\De}{\Delta}
\newcommand{\DKM}{\mathrm{DKM}}
\newcommand{\DG}{\mathrm{DG}}
\newcommand{\er}{\mathbb{R}}
\newcommand{\cee}{\mathbb{C}}
\newcommand{\zet}{\mathbb{Z}}
\newcommand{\enn}{\mathbb{N}}
\newcommand{\R}{\mathbb{R}}
\newcommand{\C}{\mathbb{C}}
\newcommand{\Z}{\mathbb{Z}}
\newcommand{\lam}{\lambda}
\newcommand{\Lam}{\Lambda}
\newcommand{\epsi}{\epsilon}
\newcommand{\til}{\tilde}
\DeclareMathOperator\tr{{tr}}
\newcommand{\bol}{\hfill\square\\}
\renewcommand{\Re}{\mathrm{Re}\,}
\renewcommand{\Im}{\mathrm{Im}\,}
\newcommand{\Ai}{\mathrm{Ai}}
\newcommand{\Pe}{\mathrm{Pe}}
\newcommand{\vece}{\mathbf{e}}
\renewcommand{\vec}{\mathbf}
\newcommand{\ud}{\,\mathrm{d}}
\newcommand{\om}{\omega}
\newcommand{\wtil}{\widetilde}
\newcommand{\err}{\mathcal{R}}
\def\supp{\mathop{\mathrm{supp}}\nolimits}
\newcommand{\what}{\widehat}
\def\Kcr{\mathop{K^{\mathrm{cr}}}\nolimits}
\def\Ktac{\mathop{K^{\mathrm{tac}}}\nolimits}
%\newtheorem{prop}[thm]{Proposition}
\newcommand{\crit}{\textrm{crit}}
\newcommand{\neww}{\textrm{new}}
\newcommand{\Pee}{{\cal P}}
\newcommand{\q}{{\mathsf{q}}}
\newcommand{\ep}{\epsilon}
\newcommand{\ds}{\displaystyle}
\newcommand{\cum}{\mathrm{cum}}
\newcommand{\msf}{\mathsf}
\newcommand{\ii}{\mathrm{i}}

\newcommand{\ccal}{\mathcal{C}}
\newcommand{\Mdec}{M^{(1)}}
\newcommand{\Mmod}{M^{(2)}}
\newcommand{\Mhom}{M^{(3)}}
\newcommand{\Mhomdef}{M^{(4)}}
\newcommand{\Mhomtris}{M^{(5)}}
\newcommand{\sq}{\textrm{sq}}
\newcommand{\ti}{\tau}
\newcommand{\re}{\Re}
\newcommand{\im}{\Im}
\newcommand{\Boh}{\mathcal{O}}
\newcommand{\sgn}{\mathrm{sgn}}
\def\red{\color{red}}
\def\blue{\color{blue}}
\def\olive{\color{olive}}
%\newcommand\restr[2]{{#1}\raisebox{-.5ex}{$|$}_{#2}}


%\newtheorem{theorem}{Theorem}[section]
%\newtheorem{lemma}[theorem]{Lemma}
%\newtheorem{proposition}[theorem]{Proposition}
%\newtheorem{propdef}[theorem]{Proposition/Definition}
%\newtheorem{example}{Example}
%\newtheorem{corollary}[theorem]{Corollary}
%\newtheorem{algorithm}[theorem]{Algorithm}
%\newtheorem{rhp}[theorem]{RH problem}
%\newtheorem{assumption}[theorem]{Assumption}
%\newcommand{\tac}{\mathrm{tac}}




%\theoremstyle{definition}
%\newtheorem{definition}[theorem]{Definition}
%
%
%\theoremstyle{remark}
%\newtheorem{note}[theorem]{Note}
%\newtheorem{remark}[theorem]{Remark}
%\newtheorem{problem}[theorem]{Problem}
%\newtheorem{convention}[theorem]{Convention}

\numberwithin{equation}{section}

\hyphenation{pa-ra-me-tri-za-tion}

%\usepackage{showkeys}

\begin{document}
%\title{The multiplicative constant in large gap asymptotics at the singular edge points for random matrix models }
\title{Choosing Coordinate Forms for Solving  ECDLP Using Shor's Algorithm }

\author{Yan Huang\footnotemark[1]\ \footnotemark[2],~ Fangguo Zhang\footnotemark[3],~ Fei Gao\footnotemark[4],~ Zijian Zhou\footnotemark[1],~ Longjiang Qu \footnotemark[1]\ \footnotemark[5]}

\renewcommand{\thefootnote}{\fnsymbol{footnote}}
\footnotetext[1]{College of Liberal Arts and Sciences, National University of Defense Technology, Changsha 410083, China.}
\footnotetext[2]{School of Mathematics and Computational Science, Hunan University of Science and Technology, Xiangtan 411201, China. } 
\footnotetext[3]{School of Computer Science and Engineering, Sun Yat-sen University, Guangzhou {\rm 510006}, China.}
	\footnotetext[4]{State Key Laboratory of Networking and Switching Technology,
Beijing University of Posts and Telecommunications, Beijing {\rm 100876}, China. }
\footnotetext[5]{Corresponding author: \texttt{ljqu\_happy@hotmail.com}}
\date{\today}

\maketitle


\begin{abstract}
Shor's algorithm is well-known for its capability to address the elliptic curve discrete logarithm problem (ECDLP) in polynomial time. The enhancement of its quantum resources continues to be a crucial focus of research. Nevertheless, the application of projective coordinates for quantum resource optimization remains an unresolved issue, mainly because the representation of projective coordinates lacks uniqueness without employing modular division operations. Our study reveals that projective coordinates do not provide the same advantages as affine coordinates when utilizing Shor's method to tackle the ECDLP.



\end{abstract}

\textbf{Keywords:} {Discrete logarithm problem, Shor's quantum algorithm, Projective coordinates, Quantum circuit}


\section{Introduction}
Shor's algorithm is well-known for its ability to efficiently tackle large integer factorization and discrete logarithm problems (DLP) in polynomial time ~\cite{Shor94}. In 2003, Zalka et al. were pioneers in proposing a quantum algorithm specifically designed to address discrete logarithm issues on Weierstrass curves (ECDLP) \cite{PZ03}. Roetteler et al. conducted a thorough examination of the quantum resources necessary for solving discrete logarithm problems over an $n$-bit prime field, determining that a maximum of $9n+2\lceil \log_{2}n\rceil+10$ qubits and $448n^{3}\log_{2}n+4090n^{3}$ Toffoli gates are required \cite{Rotteler2017}. Furthermore, they introduced specific quantum algorithms for essential arithmetic operations over finite fields and the reversible point-addition operation for the first time in 2017.



In 2020, H\"{a}ner et al. made significant advancements in the utilization of quantum resources for addressing discrete logarithmic problems over $n$-bit prime fields, optimizing the approach across three key aspects: quantum gates, the number of qubits used, and the depth of the quantum circuit\cite{Jaques2020}. When dealing with a limited number of qubits, their method could be applied using around $8n+10.2\lceil \log_{2}n\rceil-1$ qubits alongside $436n^{3}-1.05\cdot 2^{26}$ T gates. In situations where the goal was to minimize T-gate usage, the algorithm required $1115n^3/\lg n-1.08\cdot 2^{24}$ T gates, resulting in a T-depth of $389n^3/\lg n-1.70\cdot 2^{22}$. Conversely, to reduce circuit depth, the algorithm needed a T-depth of $285n^2-1.54\cdot 2^{17}$, though this increased the qubit requirement to $11n+3.9\lfloor\lg n\rfloor +16.5$. Furthermore, H\"{a}ner et al. performed a comparative study of various curve models for reversible point addition in the affine coordinate system, concluding that the Weierstrass curve is the most effective model for executing this operation.

As for classical elliptic curve cryptography (ECC), the use of projective coordinate forms and  curve models are  pivotal strategies for enhancing ECC, particularly with respect to the computation of scalar multiplication. Bernstein and Lange presented the point addition and doubling  formulas  in projective coordinates and analyzed their calculations for various curves such as Weierstrass curve, Montgomery Curve, Edwards curve, Huff curve, Hessian curve, et al\cite{Daniel}.  The projective coordinates of Edwards curves are most efficient for the computation of point addition and doubling\cite{Bernstein2008},  thus used  such as Bitcoin system\cite{Monero}\cite{Upgrade}, RFC 8032\cite{Josefsson} and more.
 


Currently, the optimization of quantum resources for solving the ECDLP using projective coordinates on quantum computers has not yet been realized. This is primarily because projective coordinates do not provide a unique representation for each point. As a result, the problem of obtaining a unique representation of points in projective coordinates without resorting to division arithmetic has been identified as an unresolved issue, as noted by H\"{a}ner et al. Cheung has suggested a technique for uniquely representing points on an elliptic curve in projective coordinates within finite fields of characteristic 2.






\textbf{Contributions:} Assuming it is feasible to uniquely represent a point in projective coordinates, we introduce a reversible in-place point addition operation for projective coordinates on both Weierstrass and Edwards curves, while also examining the quantum resources required for a single reversible point addition within Shor's framework. Our research shows that using projective coordinates on a quantum computer necessitates a greater number of quantum gates, increased quantum depth, and more qubits than utilizing affine coordinates, as previously outlined in \cite{Jaques2020}.
 

 
 

 
 \subsection{Roadmap}
The remainder of the paper is structured as follows: Section 2 discusses essential preliminaries regarding point addition on Weierstrass and Edwards curves using projective coordinates, as well as an overview of Shor's algorithm. In Section 3, we develop reversible point additions for both the Weierstrass and Edwards curves in projective coordinates. Section 4 presents a detailed analysis of the quantum resources required for a single reversible point addition over an $n$-bit prime field. Finally, Section 5 summarizes our findings and suggests possible directions for future research.
\section{Preliminaries}\label{Preliminaries}
The initial section of this chapter will focus on reviewing Shor's algorithm for addressing the ECDLP, followed by an introduction to the point addition process using projective coordinates on both Weierstrass and Edwards curves.
\subsection{Shor's algorithm for  ECDLP}
Let \( P \) be a predetermined generator of a cyclic group denoted as \( G = \langle P \rangle \), where the group's order is specified as \( ord(G) = r \). Let \( Q \) be a specific element within the subgroup \( \langle P \rangle \) generated by \( P \). The objective is to identify a unique integer \( s \in \{1, \ldots, r\} \) such that \( Q = sP \). The steps of Shor's algorithm are as follows. Initially, two registers, each consisting of \( n+1 \) qubits, are established, and all qubits are set to the state \( |0\rangle \). A Hadamard transformation \( H \) is then applied to each qubit, resulting in the state \( \frac{1}{2^{n+1}} \sum_{k,l=0}^{2^{n+1}-1} |k,l\rangle \). Subsequently, based on the values in the register containing the labels \( k \) or \( l \), the respective multiples of \( P \) and \( Q \) are added, effectively implementing the transformation 
\[
\frac{1}{2^{n+1}} \sum_{k,l=0}^{2^{n+1}-1} |k,l\rangle \mapsto \frac{1}{2^{n+1}} \sum_{k,l=0}^{2^{n+1}-1} |k,l\rangle |[k]P + [l]Q\rangle.
\]
After discarding the third register, two inverse quantum Fourier transforms \( QFT_{2^{n+1}} \) are applied to the remaining registers. Finally, the states of the first two registers are measured. By employing the technique of continued fractions, the value of \( s \) can be determined. The corresponding quantum circuit is illustrated in Figure \ref{fig:circuit of the quantum algorithm for ECDLP}.
\begin{figure}[ht]
   \centering
   \includegraphics[scale=0.20]{Circuit.pdf}
   \caption{ The circuit of the quantum algorithm for ECDLP.}\label{fig:circuit of the quantum algorithm for ECDLP}
\end{figure}
\subsection{Projective coordinates on Weierstrass curve and Edwards curve}
In Shor's algorithm, the U operation is responsible for executing the point addition $P + Q$, where $Q$ is a precomputed point. Therefore, we will now examine the mixed point addition formulas, considering $P$ in projective coordinates and $Q$ in affine coordinates for both the Weierstrass curve and the Edwards curve.


Notice that the existing optimal point-addition formulas utilize Kummer coordinates, such as those found in the Montgomery curve, the Edwards curve, and the Huff curve, all of which necessitate 4 multiplications and 2 squarings for point addition in Kummer's projective coordinates. However, these formulas cannot be employed for the \(U\) operation in Shor's algorithm because they involve the introduction of \( P-Q \). While calculating \( P+Q \) is straightforward given the points \( P, Q, P-Q \), the process becomes complicated when attempting to eliminate point \( P \) using \( Q, Q-P, Q+P \). Therefore, we are limited to curves that possess direct addition formulas.

Renes, Costello, and Batina introduced the complete addition formulas for prime order Weierstrass curves in projective coordinates \cite{Renes2016}. For two points \( P=(X_{1}, Y_{1}, Z_{1}) \) and \( Q=(x_{2}, y_{2}, 1) \) on the curve \( E/k: Y^2Z=X^3+aXZ^2+bZ^3 \) where \( char(k) \neq 2,3 \), the resulting sum \( (X_{3}, Y_{3}, Z_{3})=P+Q \) is expressed as follows:
\begin{equation*}
\begin{split}
X_{3}=&(X_{1}y_{2}+x_{2}Y_{1})(Y_{1}y_{2}-a(X_{1}+x_{2}Z_{1})-3bZ_{1})\\
&-(Y_{1}+y_{2}Z_{1})(aX_{1}x_{2}+3b(X_{1}+x_{2}Z_{1})-a^2Z_{1}),\\
Y_{3}=&(3X_{1}x_{2}+aZ_{1})(aX_{1}x_{2}+3b(X_{1}+x_{2}Z_{1})-a^2Z_{1})\\
&+(Y_{1}y_{2}-a(X_{1}+x_{2}Z_{1})-3bZ_{1})(Y_{1}y_{2}+a(X_{1}+x_{2}Z_{1})+3bZ_{1}),\\
Z_{3}=&(Y_{1}+y_{2}Z_{1})(Y_{1}y_{2}+a(X_{1}+x_{2}Z_{1})+3bZ_{1})+(X_{1}y_{2}+x_{2}Y_{1})(3X_{1}x_{2}+aZ_{1}).
\end{split}
\end{equation*} 
The addition operation needs 16 multiplications and 17 additions.

 Next, we examine the mixed addition operation represented in projective coordinates on the Edwards curve. For the Twisted Edwards curve \( E_{a,d}: ax^2 + y^2 = 1 + dx^2y^2 \), a point \( P = (x, y) \in E_{a,d} \) can be expressed in projective coordinates by introducing the coordinate \( t = xy \). Thus, we have \( P = (X, Y, T, Z) \) where \( x = \frac{X}{Z}, y = \frac{Y}{Z}, T = \frac{XY}{Z} \) \cite{Hisil2008}. Given two points \( P = (X_{1}, Y_{1}, T_{1}, Z_{1}) \) and \( Q = (x_{2}, y_{2}, x_{2}y_{2}, 1) \) with \( Z_{1} \neq 0, Z_{2} \neq 0 \), the unified mixed addition \( P + Q \) can be performed as follows:
\begin{equation*}
\begin{split}
&X_{3}=(X_{1}y_{2}+Y_{1}x_{2})(Z_{1}-dT_{1}x_{2}y_{2}),Y_{3}=(Y_{1}y_{2}-aX_{1}x_{2})(Z_{1}+dT_{1}x_{2}y_{2}),\\
&T_{3}=(Y_{1}y_{2}-aX_{1}x_{2})(X_{1}y_{2}+Y_{1}x_{2}),Z_{3}=(Z_{1}-dT_{1}T_{2})(Z_{1}+dT_{1}x_{2}y_{2}).
\end{split}
\end{equation*}
%The formulas  also work for special point addition such as $P+(0,1,0,1)$, $P+P$, where $(0,1,0,1)$ is the identity  element. On classical computers, 

The  addition operation needs 8 multiplcations and 6 additions. When $a = -1$, the number of multiplication operations can be reduced by one.
\begin{algorithm}[H]
\caption{: Reversible, Weierstrass curve out-of-place point addition. This algorithm operates on a quantum register holding the point $|\overline{P}\rangle=|X_{1}\rangle|Y_{1}\rangle|Z_{1}\rangle$ which is stored in registers $X_{1}, Y_{1},  Z_{1}$,   the second point $Q=(x_{2}, y_{2}, x_{2}y_{2}, 1)$ is assumed to be a precomputed classical constant. The output point $|\overline{P+Q}\rangle =|X_{3}\rangle|Y_{3}\rangle|Z_{3}\rangle$ is stored in $X_{3}, Y_{3}, Z_{3}$ registers, respectively. }
\label{Algorithm 1:}
\begin{algorithmic}[1]
\begin{multicols}{2}
\STATE mul\_modp $X_{1}, x_{2}, t_{0}$;      
\STATE mul\_modp $Y_{1}, y_{2}, t_{1}$;
\STATE add\_modp $ X_{1},Y_{1}$;
\STATE mul\_modp $X_{1}, x_{2}+y_{2}, t_{2}$;
\STATE add\_modp $ t_{0},t_{1}$;
\STATE sub\_modp $ t_{2},t_{0};$//$t_{2}=X_{1}y_{2}+Y_{1}x_{2}$
\STATE mul\_modp $x_{2}, Z_{1}, t_{3}$;
\STATE sub\_modp $ X_{1},Y_{1};$
\STATE add\_modp $ t_{3},X_{1}$;//$t_{3}=x_{2}Z_{1}+X_{1}$
\STATE mul\_modp $y_{2}, Z_{1}, t_{4}$;
\STATE add\_modp $ t_{4},Y_{1}$;//$t_{4}=y_{2}Z_{1}+Y_{1}$
\STATE mul\_modp $t_{3}, a, t_{5}$;
\STATE mul\_modp $b_{3}, Z_{1}, t_{6}$;
\STATE add\_modp $ t_{5},t_{6}$;
\STATE add\_modp $t_{5}, t_{1}$;
\STATE add\_modp $t_{1}, t_{1}$;
\STATE sub\_modp $ t_{1},t_{5}$;
\STATE mul\_modp $t_{5}, t_{1}, Y_{3}$;
\STATE mul\_modp $t_{2}, t_{1}, X_{3}$;
\STATE mul\_modp $t_{5}, t_{4}, Z_{3}$;
\STATE add\_modp $ t_{1},t_{5}$;
\STATE hlv\_modp $t_{1}, t_{1}$;
\STATE sub\_modp $ t_{5},t_{1}$;
\STATE sub\_modp $ t_{5},t_{6}$;
\STATE mul\_modp $a, t_{3}, t_{5}$;//$t_{5}=0$
\STATE mul\_modp $b_{3}, Z_{1}, t_{6}$;//$t_{6}=0$
\STATE mul\_modp $a, Z_{1}, t_{5}$;
\STATE triple\_modp $ t_{0}$;//$t_{0}=3X_{1}x_{2}$
\STATE add\_modp $ t_{0},t_{5}$;//$t_{0}=3X_{1}x_{2}+aZ_{1}$
\STATE mul\_modp $t_{0}, t_{2}, t_{6}$;
\STATE add\_modp $ Z_{3},t_{6};$
\STATE mul\_modp $t_{0}, t_{2}, t_{6}$;//$t_{6}=0$
\STATE mul\_modp $a, t_{5}, t_{6}$;
\STATE sub\_modp $ t_{0},t_{5}$;
\STATE one-third\_modp $ t_{0}$;
\STATE mul\_modp $a, t_{0}, t_{7}$;
\STATE add\_modp $ t_{7},t_{6}$;//$t_{7}=aX_{1}x_{2}-a^2Z_{1}$
\STATE mul\_modp $b_{3}, t_{3}, t_{9}$;
\STATE add\_modp $ t_{7},t_{9}$;
\STATE mul\_modp $t_{4}, t_{7}, t_{8}$;
\STATE sub\_modp $ X_{3},t_{8};$
\STATE mul\_modp $t_{4}, t_{7}, t_{8}$;//$t_{8}=0$
\STATE triple\_modp $ t_{0}$;//$t_{0}=3X_{1}x_{2}$
\STATE add\_modp $ t_{0},t_{5}$;
\STATE mul\_modp $t_{0}, t_{7}, t_{8}$;
\STATE add\_modp $ Y_{3},t_{8};$
\STATE mul\_modp $t_{0}, t_{7}, t_{8}$;//$t_{8}=0$
\STATE sub\_modp $ t_{7},t_{9}$;
\STATE add\_modp $ t_{7},t_{6}$;
\STATE sub\_modp $ t_{0},t_{5}$;
\STATE one-third\_modp $ t_{0}$;
\STATE mul\_modp $a, t_{0}, t_{7}$;//$t_{7}=0$
\STATE mul\_modp $b_{3}, t_{3}, t_{9}$;//$t_{9}=0$
\STATE mul\_modp $a, t_{5}, t_{6}$;//$t_{6}=0$
\STATE mul\_modp $a, Z_{1}, t_{5}$;//$t_{5}=0$
\STATE sub\_modp $ t_{4},Y_{1}$;
\STATE mul\_modp $y_{2}, Z_{1}, t_{4}$;//$t_{4}=0$
\STATE sub\_modp $ t_{3},X_{1}$;
\STATE mul\_modp $x_{2}, Z_{1}, t_{3}$;//$t_{3}=0$
\STATE add\_modp $ t_{0},t_{1}$;
\STATE add\_modp $ t_{2},t_{0}$;
\STATE add\_modp $ X_{1},Y_{1}$;
\STATE mul\_modp $X_{1}, x_{2}+y_{2}, t_{2}$;//$t_{2}=0$
\STATE sub\_modp $ t_{0},t_{1}$;
\STATE sub\_modp $ X_{1},Y_{1}$;
\STATE mul\_modp $Y_{1}, y_{2}, t_{1}$; //$t_{1}=0$
\STATE mul\_modp $X_{1}, x_{2}, t_{0}$; //$t_{0}=0$

\end{multicols}
 \end{algorithmic}
\end{algorithm}
 
\section{Reversible point addition on Weierstrass curves and Edwards curves in projective coordinates }\label{point-addition}
In this section, we initially introduce an out-of-place point addition method utilizing projective coordinates on both the Weierstrass and Edwards curves, under the premise that points on an elliptic curve can be distinctly represented in a quantum computing environment using projective coordinates. Subsequently, we apply Bennett's technique to convert the out-of-place point addition into an in-place point addition and analyze the quantum resources involved.


In Shor's algorithm for addressing the ECDLP, the use of projective coordinates does not provide a unique representation of a point. This lack of uniqueness compromises the history independence of the algorithm and affects the interference within the superposition state. Typically, projective coordinates are converted into affine coordinates through division operations. However, in the context of quantum algorithms, performing division over finite fields is also resource-intensive. Consequently, finding a way to uniquely represent a point in projective coordinates without resorting to division remains an unresolved challenge.

Cheung proposed a method over finite fields of characteristic 2\cite{Cheung2008}. That is, given a point $P=(x,y)$, they represented the  point uinquely in the following form 
\begin{equation*}
\begin{split}
|P(x,y)\rangle=\frac{1}{\sqrt{2^n}}\sum_{z\in \mathbf{F}_{p}}|xz\rangle|yz\rangle|z\rangle.
\end{split}
\end{equation*}
 Assuming we can represent points on an elliptic curve uniquely in a manner similar to that over a general prime field, the next step to solve the ECDLP using Shor's algorithm is to develop quantum algorithms for in-place point addition in projective coordinates.

We develop reversible out-of-place point addition operations utilizing projective coordinates in Algorithm 1 for Weierstrass curves and Algorithm 2 for Edwards curves, based on their respective point addition formulas. The functions \textbf{add\_modp}, \textbf{sub\_modp}, \textbf{mul\_modp}, \textbf{hlv\_modp}, \textbf{triple\_modp}, and \textbf{one-third\_modp} specified in both algorithms correspond to addition, subtraction, multiplication, halving, tripling, and one-third operations, respectively. To provide a clearer understanding of these algorithms, we focus on Algorithm 2 and illustrate its quantum circuit in Fig. 2 and Fig. 3. The symbols $+, -, M$ represent addition, subtraction, and multiplication operations within a finite field.


\begin{algorithm}[htb]
\caption{: Out-of-place point addition on Edwards Curves. This algorithm operates on a quantum register holding the point $|\overline{P}\rangle=|X_{1}\rangle|Y_{1}\rangle|T_{1}\rangle|Z_{1}\rangle$ which is stored in registers $X_{1}, Y_{1}, T_{1}, Z_{1}$,   the second point $Q=(x_{2}, y_{2}, x_{2}y_{2})$ is assumed to be a precomputed classical constant. The output point $|\overline{P+Q}\rangle =|X_{3}\rangle|Y_{3}\rangle|T_{3}\rangle|Z_{3}\rangle$ is stored in $X_{3}, Y_{3}, T_{3}, Z_{3}$ registers, respectively. }
\label{Algorithm 2:}
\begin{algorithmic}[1]
\begin{multicols}{2}
\STATE mul\_modp $X_{1}, x_{2}, t_{0}$;      
\STATE mul\_modp $Y_{1}, y_{2}, t_{1}$; 
\STATE mul\_modp $T_{1}, x_{2}y_{2}, t_{2}$;
\STATE mul\_modp $d, t_{2},  t_{3}$;
\STATE add\_modp $  Y_{1},X_{1}$;//$X_{1}=X_{1}+Y_{1}$
\STATE mul\_modp $X_{1}, x_{2}+y_{2}, t_{4}$;
\STATE sub\_modp $ t_{4},t_{0}$;
\STATE sub\_modp $ t_{4},t_{1}$;//$t_{4}=X_{1}y_{2}+X_{2}y_{1}$
\STATE sub\_modp $ Z_{1}, t_{3}$;//$Z_{1}=Z_{1}-dT_{1}x_{2}y_{2}$
\STATE add\_modp $t_{3}, t_{3}$;
\STATE add\_modp $  Z_{1},t_{3}$;//$t_{3}=Z_{1}+dT_{1}x_{2}y_{2}$
\STATE mul\_modp $ Z_{1},t_{4}, X_{3}$;
\STATE mul\_modp $ Z_{1},t_{3}, Z_{3}$;
\STATE mul\_modp $ t_{0},a, t_{5}$;
\STATE sub\_modp $ t_{1},t_{5}$;//$t_{1}=Y_{1}y_{2}-aX_{1}x_{2}$
\STATE mul\_modp $ t_{1},t_{3}, Y_{3}$;
\STATE mul\_modp $ t_{1}, t_{4}, T_{3}$;
\STATE add\_modp $ t_{5}, t_{1}$;
\STATE mul\_modp $ t_{0},a, t_{5}$;//$t_{5}=0$
\STATE add\_modp $  t_{0},t_{4}$;
\STATE add\_modp $ t_{1},t_{4}$;
\STATE mul\_modp $X_{1}, x_{2}+y_{2}, t_{4}$;//$t_{4}=0$
\STATE sub\_modp $ X_{1},Y_{1}$;
\STATE mul\_modp $X_{1}, x_{2}, t_{0}$;   //$t_{0}=0$   
\STATE mul\_modp $Y_{1}, y_{2}, t_{1}$;//$t_{1}=0$
\STATE sub\_modp $t_{3},  Z_{1}$;
\STATE hlv\_modp $t_{3}, t_{3}$;
\STATE add\_modp $t_{3},  Z_{1}$;
\STATE mul\_modp $d, t_{2},  t_{3}$;//$t_{3}=0$
\STATE mul\_modp $T_{1}, x_{2}y_{2}, t_{2}$.//$t_{2}=0$
\end{multicols}
 \end{algorithmic}
\end{algorithm}






\begin{figure}[ht]
   \centering
   \includegraphics[scale=0.20]{ed1.pdf}
   \caption{ Quantum circuit for $P+Q$ from Step 1 to Step 15.}\label{fig:GeometricalGroupLawstep1-14}
\end{figure}

\begin{figure}[ht]
   \centering
   \includegraphics[scale=0.20]{ed2.pdf}
   \caption{ Quantum circuit for $P+Q$ from Step 16 to Step 30.}\label{fig:GeometricalGroupLawstep15-28}
\end{figure}




When performing a reversible point addition, Algorithm 1 needs 31 multiplications,  36 additions, while  Algorithm 2 needs  16 multiplications, 14 additions  for $n$-bit prime field. 



Noting that both Algorithm 1 and Algorithm 2 are out-of-place circuits, we can  use the Bennett's method\cite{Bennett} to construct corresponding in-place circuits as follows:
\begin{equation*}
\begin{split}
&| \overline{P}\rangle|0^{4n}\rangle\stackrel{U}{\longrightarrow}| \overline{P}\rangle| \overline{P+Q}\rangle\stackrel{U^{\dagger}}{\longrightarrow}|0^{4n}\rangle| \overline{P+Q}\rangle\stackrel{SWAP}{\longrightarrow}| \overline{P+Q}\rangle|0^{4n}\rangle,
\end{split}
\end{equation*}
where $U$ is an operation for Algorithm 1 or Algorithm 2. Thus, a reversible in-place point addition needs about 62 multiplications, 72 additions and $6n$ SWAP gates for Algorithm 1 and  32 multiplications, 28 additions and $4n$ SWAP gates for Algorithm 2.

\begin{table}[h]
%\scriptsize
\begin{center}
\caption{ The  number of operations for  reversible point addition . }
\label{tab11}
\setlength{\tabcolsep}{0.1cm}
\begin{tabular*}{\linewidth}{cccccc}
\hline
 Curve model & Inversion & Division & Multiplication & Squaring & Addition  \\ \hline
 Weierstrass \cite{Jaques2020}(affine )& 0 & 2&2&1 &9  \\ 
 Weierstrass(projective )& 0 & 0&62&0 &72  \\ 
 Edwards curve(projective )&0&0 & 32 &0 & 28\\ 
 \hline
\end{tabular*}
\end{center}
\end{table}


\begin{table}[h]
\caption{ Quantum resources for  a reversible point addition . }
\label{tab11}
%\renewcommand{\arrayrulewidth}{1pt}
%\resizebox{\textwidth}{1mm}
\begin{tabular*}{\linewidth}{cccccc}
\hline
 Curve model & depth & gates \\ \hline
 Weierstrass \cite{Jaques2020}(affine )& $432n^2+1.07\cdot 2^{14}$ & $1182n^2+1.41\cdot 2^{16}$    \\ 
 Weierstrass(projective )&$1326.8n^2+1008n+1.32\cdot 2^{16}$  &$2343.6n^2+2160n+1.36\cdot 2^{18}$    \\ 
 Edwards curve(projective )& $684.8n^2+392n+1.35\cdot 2^{15}$ &  $1209.6n^2+840n+1.4\cdot 2^{17}$   \\ 
 \hline
\end{tabular*}
%\end{center}
\end{table}

\begin{table}[h]
\caption{Quantum resources for a reversible point addition. }
\label{tab11}
%\renewcommand{\arrayrulewidth}{1pt}
%\resizebox{\textwidth}{1mm}
\begin{tabular*}{\linewidth}{cc}
\hline
 Curve model & qubits\\ \hline
 Weierstrass \cite{Jaques2020}(affine ) & $11n+18.9$   \\ 
 Weierstrass(projective ) &$19n+12.2$  \\ 
 Edwards curve(projective )& $18n+12.2$   \\ 
 \hline
\end{tabular*}
%\end{center}
\end{table}

\section{Efficiency analysis}\label{efficience}
We evaluate the quantum resource requirements for a single unitary operation of Shor's algorithm aimed at solving the Elliptic Curve Discrete Logarithm Problem (ECDLP) using both affine and projective coordinates. For fundamental algebraic tasks such as modular addition, modular multiplication, modular squaring, and modular division over the base field, we reference the optimized results by H\"{a}ner et al. regarding the minimal number of T-gates\cite{Jaques2020}. Table 1 outlines the essential algebraic operations over a finite field required for in-place point addition. Table 2-3 examines the quantum resources necessary for a reversible point addition, detailing aspects such as quantum depth, the number of T-gates, and the number of qubits. Our findings indicate that utilizing projective coordinates does not offer benefits concerning quantum depth and gate count, and in fact requires a greater number of qubits compared to affine coordinates. When factoring in the costs associated with unique representation and the retrieval of pre-computed table values, it becomes clear that employing Shor's framework for solving ECDLP in projective coordinates demands more quantum resources than in affine coordinates.



\section{Conclusion and discussion}\label{conclusion}
This paper examines the quantum resources required for reversible point addition in projective coordinates and contrasts them with those in affine coordinates. Our findings indicate that utilizing projective coordinates does not provide any advantages. This conclusion can be primarily attributed to two factors: (1) The reversibility of the point operation necessitates the restoration of auxiliary registers to the initial state $|0\rangle$, which results in the need for extra multiplications. (2) The in-place point operation can solely be constructed using Bennett's method, meaning that the initial out-of-place point operation must be performed twice to achieve the U operation in Shor's algorithm, effectively doubling the quantum resources required for the original out-of-place point addition.

This also suggests avenues for our future research. We aim to optimize the fundamental algebraic operations over finite fields while also striving to develop direct in-place reversible point addition operations.


\section*{Acknowledgments}
This work is supported by the  National Natural Science Foundation of China (Nos. 12401696,    62272491,  62202475),  Guangdong Provincial Key Laboratory of Information Security Technology (Grant No. 2023B1212060026) and Hunan Provincial Natural Science Foundation of China (No. 2022JJ40154).  

\begin{thebibliography}{99}
\bibitem{Shor94}
Shor, P. Algorithms for Quantum Computation: Discrete Logarithms and Factoring.  FOCS 1994,  IEEE, 124-134, 1994.
\bibitem{PZ03}
Zalka, C., Proos J. Shor's discrete logarithm quantum algorithm for elliptic curves. Quantum Inf. Comput.   3(4), 317-344, 2003.
\bibitem{Rotteler2017}
Roetteler, M., Naehrig, M., Svore, K.M., Lauter, K.  Quantum Resource Estimates for Computing Elliptic Curve Discrete Logarithms.   ASIACRYPT 2017, Springer LNCS 10625, pp.  241-270, 2017.
\bibitem{Jaques2020}
H\"{a}ner, T., Jaques, S., Naehrig, M., Roettelerm, M., Soeken, M. Improved quantum circuits for elliptic curve discrete logarithms. PQC 2020, Springer LNCS 12100, pp.425-444, 2020.
\bibitem{Daniel}
Daniel J. Bernstein and Tanja Lange. Explicit-formulas database (2007). http://www.hyperelliptic.org/EFD
\bibitem{Bernstein2008}
Bernstein, D.J., Birkner, P., Joye, M., Lange, T., Peters, C. Twisted Edwards curves.  AFRICACRYPT 2008. Springer LNCS 5023, pp. 389-405, 2008.
\bibitem{Monero}
Monero documentation - edwards25519 elliptic curve.
https://monerodocs.org/cryptography/asymmetric/edwards25519/.
\bibitem{Upgrade}
Upgrade libsodium to 1.0.18.Website, 2016. https:\//github.com/zcash/zcash/pull/4359.
\bibitem{Josefsson}
Josefsson, S.,   Liusvaara, I. Edwards-curve digital signature algorithm (EdDSA).  Internet Research Task Force, Crypto Forum Research Group, RFC, vol. 8032, 2017.
\bibitem{Cheung2008}
Cheung, D., Maslov, D., Mathew, J., Pradhan, D.K.  On the Design and Optimization of a Quantum Polynomial-Time Attack on Elliptic Curve Cryptography.  TQC 2008, Springer LNCS 5106. 
\bibitem{Costello20117}
Costello, C., Hisil, H. A simple and compact algorithm for SIDH with arbitrary degree isogenies.  ASIACRYPT 2017, Springer  LNCS 10625, pp. 303–329, 2017.
\bibitem{Kim2019}
Kim, S., Yoon, K., Park, Y.-H., Hong, S. Optimized method for computing odd-degree isogenies on Edwards curves.  ASIACRYPT 2019, Springer  LNCS 11922, pp. 273-292, 2019.
\bibitem{HUANGY2020}
Huang, Y., Zhang, F. G., Hu, Z., Liu, Z. J. Optimized Arithmetic Operations for Isogeny-based Cryptography on Huff Curves. ACISP 2020, Springer LNCS 12248, pp.23-40, 2020.
\bibitem{Renes2016}
Renes, J., Costello, C., Batina, L.  Complete Addition Formulas for Prime Order Elliptic Curves.  EUROCRYPT 2016. Springer  LNCS 9665, pp. 403-428, 2016.
\bibitem{Hisil2008}
Hisil, H., Wong, K. K. H.,  Carter, G., Dawson, E. Twisted Edwards Curves Revisited.   ASIACRYPT 2008, Springer LNCS 5350, pp. 326-343, 2008.	
\bibitem{Bennett}
Bennett, C. H. Logical reversibility of computation. IBM Journal of Research and Development, 17(6): 525-532, 1973.
\end{thebibliography}

\end{document}


