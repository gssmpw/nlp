\section{Experiment}\label{sec:exp}
% \begin{table*}[t]
% \centering
% \caption{Comparison of efficiency and visual quality.}
% \label{table:Comparison of efficiency and visual quality}
% \resizebox{0.95\textwidth}{!}{%
% \begin{tabular}{|l|c|c|c|c|c|c|c|}
% \hline
% \rowcolor[HTML]{C0C0C0} 
% Method & \multicolumn{3}{c|}{Efficiency} & \multicolumn{4}{c|}{Visual Quality} \\ \cline{2-8} 
% \rowcolor[HTML]{C0C0C0} 
%  & MACs (P) $\downarrow$ & Speedup $\uparrow$ & Latency (s) $\downarrow$ & VBench $\uparrow$ & LPIPS $\downarrow$ & SSIM $\uparrow$ & PSNR $\uparrow$ \\ \hline
% \multicolumn{8}{|c|}{Open-Sora 1.2 (49 frames, 480P)} \\ 
%  \hline
%     Open-Sora 1.2 ($T=30$) & 5.59 & 1$\times$ & 54.38& 78.69\% & - & - & - \\ \hline
%     VideoSys ($T=30$) & 4.81 & 1.34$\times$ & 40.59 & 78.21\% & 0.1020 & 0.8821 & 26.43 \\ \hline
%     FasterCache ($T=30$) & 4.13 & 1.57$\times$ & 34.64 & 77.69\% & 0.0937 & 0.8830 & 26.57 \\ \hline
%     Ours (E1) ($T=30$)& \textbf{0} & \textbf{0$\times$} & \textbf{0} & \textbf{0\%} & \textbf{\textcolor{blue}{0}} & \textbf{\textcolor{blue}{0}} & \textbf{\textcolor{blue}{0}} \\ \hline
%     Ours (E2) ($T=30$)& \textbf{0} & \textbf{0$\times$} & \textbf{0} & \textbf{0\%} & \textbf{\textcolor{blue}{0}} & \textbf{\textcolor{blue}{0}} & \textbf{\textcolor{blue}{0}} \\ \hline
%     Ours (E3) ($T=30$)& \textbf{0} & \textbf{0$\times$} & \textbf{0} & \textbf{0\%} & \textbf{\textcolor{blue}{0}} & \textbf{\textcolor{blue}{0}} & \textbf{\textcolor{blue}{0}} \\ \hline
%     Ours (E4) ($T=30$)& \textbf{0} & \textbf{0$\times$} & \textbf{0} & \textbf{0\%} & \textbf{\textcolor{blue}{0}} & \textbf{\textcolor{blue}{0}} & \textbf{\textcolor{blue}{0}} \\ \hline
%     Ours (E5) ($T=30$)& \textbf{4.09} & \textbf{1.59$\times$} & \textbf{34.20} & \textbf{77.34\%} & \textbf{\textcolor{blue}{0.0892}} & \textbf{\textcolor{blue}{0.8874}} & \textbf{\textcolor{blue}{26.37}} \\ \hline
% \multicolumn{8}{|c|}{Latte (16 frames, 512x512)} \\  \hline
%     Latte ($T=50$) & 3.05 & 1$\times$ & 28.71 & 77.23\% & - & - & - \\ \hline
%     VideoSys ($T=50$) & 2.24 & 1.32$\times$ & 21.76 & 77.01\% & 0.2837 & 0.7152 & 19.20 \\ \hline
%     FasterCache ($T=50$) & 1.97 & 1.59$\times$ & 18.06 & 76.92\% & 0.0952 & 0.8548 & 23.75 \\ \hline
%     Ours (E1) ($T=50$)& \textbf{0} & \textbf{0$\times$} & \textbf{0} & \textbf{0\%} & \textbf{\textcolor{blue}{0}} & \textbf{\textcolor{blue}{0}} & \textbf{\textcolor{blue}{0}} \\ \hline
%     Ours (E2) ($T=50$)& \textbf{0} & \textbf{0$\times$} & \textbf{0} & \textbf{0\%} & \textbf{\textcolor{blue}{0}} & \textbf{\textcolor{blue}{0}} & \textbf{\textcolor{blue}{0}} \\ \hline
%     Ours (E3) ($T=50$)& \textbf{0} & \textbf{0$\times$} & \textbf{0} & \textbf{0\%} & \textbf{\textcolor{blue}{0}} & \textbf{\textcolor{blue}{0}} & \textbf{\textcolor{blue}{0}} \\ \hline
%     Ours (E4) ($T=50$)& \textbf{0} & \textbf{0$\times$} & \textbf{0} & \textbf{0\%} & \textbf{\textcolor{blue}{0}} & \textbf{\textcolor{blue}{0}} & \textbf{\textcolor{blue}{0}} \\ \hline
%     Ours (E5) ($T=50$)& \textbf{1.96} & \textbf{1.61$\times$} & \textbf{17.83} & \textbf{76.82\%} & \textbf{\textcolor{blue}{0.0978}} & \textbf{\textcolor{blue}{0.8471}} & \textbf{\textcolor{blue}{23.65}} \\ \hline
% \multicolumn{8}{|c|}{CogVideoX (49frames, 480x720)} \\ \hline
%     CogVideoX ($T=50$) & 6.03 & 1$\times$ & 85.74 & 81.18\% & - & - & - \\ \hline
%     VideoSys ($T=50$) & 4.45 & 1.32$\times$ & 65.06 & 79.85\% & 0.0872 & 0.9463 & 28,51 \\ \hline
%     FasterCache ($T=50$) & 3.71 & 1.60$\times$ & 53.61 & 78.34\% & 0.0850 & 0.9572 & 28.66 \\ \hline
%     Ours (E1) ($T=50$)& \textbf{0} & \textbf{0$\times$} & \textbf{0} & \textbf{0\%} & \textbf{\textcolor{blue}{0}} & \textbf{\textcolor{blue}{0}} & \textbf{\textcolor{blue}{0}} \\ \hline
%     Ours (E2) ($T=50$)& \textbf{0} & \textbf{0$\times$} & \textbf{0} & \textbf{0\%} & \textbf{\textcolor{blue}{0}} & \textbf{\textcolor{blue}{0}} & \textbf{\textcolor{blue}{0}} \\ \hline
%     Ours (E3) ($T=50$)& \textbf{0} & \textbf{0$\times$} & \textbf{0} & \textbf{0\%} & \textbf{\textcolor{blue}{0}} & \textbf{\textcolor{blue}{0}} & \textbf{\textcolor{blue}{0}} \\ \hline
%     Ours (E4) ($T=50$)& \textbf{0} & \textbf{0$\times$} & \textbf{0} & \textbf{0\%} & \textbf{\textcolor{blue}{0}} & \textbf{\textcolor{blue}{0}} & \textbf{\textcolor{blue}{0}} \\ \hline
%     Ours (E5) ($T=50$)& \textbf{0} & \textbf{0$\times$} & \textbf{0} & \textbf{0\%} & \textbf{\textcolor{blue}{0}} & \textbf{\textcolor{blue}{0}} & \textbf{\textcolor{blue}{0}} \\ \hline



    
% % \multicolumn{8}{|c|}{Open-Sora-Plan (65frames, 512x512)} \\ \hline
% %     Open-Sora-Plan ($T=50$) & 10.30 & 1$\times$ & 107.63 & 80.38\% & - & - & - \\ \hline
% %     VideoSys ($T=50$) & 7.39 & 1.31$\times$ &  82.01& 79.17\% & 0.2231 & 0.7856 & 21.31 \\ \hline
% %     FasterCache ($T=50$) & 5.51 & 1.65$\times$ & 65.01 & 79.55\% & 0.1537 & 0.7762 & 24.33 \\ \hline
% %     Ours & \textbf{0} & \textbf{0$\times$} & \textbf{0} & \textbf{0\%} & \textbf{\textcolor{blue}{0}} & \textbf{\textcolor{blue}{0}} & \textbf{\textcolor{blue}{0}} \\ \hline
% \end{tabular}%
% }
% \end{table*}


\begin{table*}[ht]
\centering
\caption{Performance of UniCP across Different Video Generation Models. We assessed UniCP under varying error thresholds. The top three results are distinguished by color: blue indicates the first rank, red the second, and green the third.}
\label{table:Comparison of efficiency and visual quality}

    \begin{tabular}{ccccccccc}
        \toprule
        model & method & MACs (P) $\downarrow$ & Speedup $\uparrow$ & Latency (s) $\downarrow$ & VBench$\uparrow$ & LPIPS$\downarrow$ & SSIM $\uparrow$ & PSNR $\uparrow$ \\ 
        \midrule
        \multirow{4}{*}{Open-Sora} 
        & origin & 5.59 & 1$\times$ & 54.38& 78.69\% & - & - & - \\
        & PAB \cite{zhao2024pab} & 4.81 & 1.34$\times$ & 40.59 & \textcolor{blue}{78.21\%} & 0.1020 & 0.8821 & 26.43 \\
        & FasterCache \cite{lv2024fastercache} &  \textcolor{red}{4.13} & \textcolor{red}{1.57$\times$} & \textcolor{red}{34.64} & 77.69\% & 0.0937 & 0.8830 & 26.57 \\
        \hline
        &UniCP (E1) & 5.23 & 1.11$\times$ & 49.07 & \textcolor{red}{78.17\%} & \textcolor{blue}{0.0847} & \textcolor{blue}{0.9017} & \textcolor{blue}{27.15} \\
        &UniCP (E2) &5.05&1.16$\times$&46.72&\textcolor{cyan}{78.03\%}&\textcolor{red}{0.0857}&\textcolor{red}{0.8970}&\textcolor{red}{26.99}\\
        &UniCP (E3) &4.82&1.29$\times$&42.29&77.92\%&\textcolor{cyan}{0.0879}&\textcolor{cyan}{0.8917}&\textcolor{cyan}{26.57}\\
        &UniCP (E4) &\textcolor{cyan}{4.47}&\textcolor{cyan}{1.42} &\textcolor{cyan}{38.33}&77.41\%&0.0893&0.8871&26.43\\
        &UniCP (E5) & \textcolor{blue}{4.09} & \textcolor{blue}{1.59$\times$} & \textcolor{blue}{34.20} & 77.34\% & 0.0892 & 0.8874 & 26.37 \\
        \midrule
        \multirow{4}{*}{Latte}
        & origin &3.05 & 1$\times$ & 28.71 & 77.23\% & - & - & - \\
        & PAB \cite{zhao2024pab} & 2.24 & 1.32$\times$ & 21.76 & \textcolor{cyan}{77.01\%} & 0.2837 & 0.7152 & 19.20 \\ 
        & FasterCache \cite{lv2024fastercache} & \textcolor{red}{1.97} & \textcolor{red}{1.59$\times$} & \textcolor{red}{18.06} & 76.92\% & 0.0952 & 0.8548 & 23.75 \\ 
        \hline
        &UniCP (E1) & 2.79 & 1.10$\times$ & 26.03 & \textcolor{blue}{77.13\%} & \textcolor{blue}{0.0827} & \textcolor{blue}{0.9019} & \textcolor{blue}{25.34} \\
        &UniCP (E2) &2.67&1.19$\times$&24.03&\textcolor{red}{77.05\%} &\textcolor{red}{0.0878}&\textcolor{red}{0.8872}&\textcolor{red}{24.89}\\
        &UniCP (E3) &2.45&1.29$\times$&22.11&76.96\%&\textcolor{cyan}{0.0912}&\textcolor{cyan}{0.8734}&\textcolor{cyan}{24.46}\\
        &UniCP (E4) &\textcolor{cyan}{2.21}&\textcolor{cyan}{1.44$\times$}&\textcolor{cyan}{19.87}&76.89\%&0.0934&0.8541&24.05\\
        &UniCP (E5) & \textcolor{blue}{1.96} & \textcolor{blue}{1.61$\times$} & \textcolor{blue}{17.83} & 76.82\% & 0.0978 & 0.8471 & 23.65 \\
        \midrule
        \multirow{4}{*}{CogVideoX}
        & origin & 6.03 & 1$\times$ & 85.74 & 81.18\% & - & - & - \\ 
        & PAB \cite{zhao2024pab} & 4.45 & 1.32$\times$ & 65.06 & \textcolor{red}{79.85\%} & 0.0872 & \textcolor{cyan}{0.9463} & 28,51 \\
        & FasterCache \cite{lv2024fastercache} & \textcolor{red}{3.71} & \textcolor{red}{1.60$\times$} & \textcolor{red}{53.61} & 78.34\% & 0.0850 & \textcolor{blue}{0.9572} & \textcolor{red}{28.66} \\
        \hline
        &UniCP (E1) & 5.35 & 1.12$\times$ & 76.91&\textcolor{blue}{79.97\%} & \textcolor{blue}{0.0835} & \textcolor{red}{0.9479} & \textcolor{blue}{28.87} \\
        &UniCP (E2) &4.96&1.21$\times$&70.88&\textcolor{cyan}{79.25\%}&\textcolor{red}{0.0840}&0.9411&28.61\\
        &UniCP (E3) &4.54&1.33$\times$&64.51&78.72\%&\textcolor{cyan}{0.0845}&0.9385&\textcolor{cyan}{28.63}\\
        &UniCP (E4) &\textcolor{cyan}{4.11}&\textcolor{cyan}{1.49$\times$}&\textcolor{cyan}{57.46}&78.39\%&0.0860&0.9299&28.44\\
        &UniCP (E5) &\textcolor{blue}{3.61}&\textcolor{blue}{1.64$\times$}&\textcolor{blue}{50.37}&77.72\%&0.0866&0.9237&28.37\\
        
        \bottomrule
    \end{tabular}

\end{table*}
In this section, we describe experimental setup and present the results and key findings from experiments.
\subsection{Experimental Setup}
Our method is integrated into state-of-the-art DIT-based video generation models, including OpenSora 1.2, Latte, and CogVideoX These models serve as the foundation for our experiments, allowing us to assess the effectiveness of our proposed approach. For baseline comparisons, we employ PAB \cite{zhao2024pab} and FasterCache \cite{lv2024fastercache}, both of which are based on caching frameworks. Additionally, we utilize the prompts provided by the VBench as our evaluation dataset to comprehensively evaluate performance. All experiments were conducted on NVIDIA A800 80GB GPUs.

\subsection{Evaluation Metrics} To assess the visual quality of generated videos, we utilize several metrics, including VBench\cite{huang2024vbench}, LPIPS\cite{zhang2018unreasonable}, SSIM\cite{wang2004image}, and PSNR\cite{korhonen2012peak}. VBench provides a standardized benchmarking framework for comparing various algorithms. LPIPS measures perceptual similarity by computing distances in the image feature space using pretrained convolutional neural networks. SSIM evaluates image similarity by considering luminance, contrast, and structural information. PSNR  quantifies video quality by measuring the error between video sequences, offering a precise indication of their differences. Additionally, to evaluate latency and computational complexity, we use Latency (inference time) and Multiply-Accumulate Operations (MACs). These metrics are essential for quantifying the computational cost during inference process and are robust indicators of acceleration method effectiveness.

\begin{figure}
    \centering
    \includegraphics[width=\linewidth]{figures/Mainexp.pdf}
    \caption{Video generation samples under various error thresholds. UniCP demonstrates stable performance across various error thresholds, with only minimal quality degradation.}
    \label{fig:maxinexp}
    \vspace{-5mm}
\end{figure}
\subsection{Quantitative Experiments}
% In summary, utilizing these diverse evaluation metrics provides a comprehensive analysis of the visual quality and performance of our video generation models, ensuring robust and reliable assessments across multiple dimensions.
Quantitative experiments with state-of-the-art methods are presented in Table \ref{table:Comparison of efficiency and visual quality}. We synthesize videos using prompts provided by VBench and employ these synthesized videos to compute the VBench metrics. Additionally, we calculate LPIPS, SSIM, and PSNR using videos generated by the original models. We denote these threshold settings as E1 (δ=0.025), E2 (δ=0.05), E3 (δ=0.75), E4 (δ=0.125), E5 (δ=0.175). The results indicate that UniCP maintains stable performance across various error thresholds. As the threshold increases, it significantly reduces computational complexity and latency while largely preserving video quality.



% (i) Our method achieves significant acceleration while maintaining high-quality output, enhancing efficiency with minimal impact on quality.

% (ii) Our method performs excellently across different models, demonstrating the strong generalizability of error-guided caching and multi-strategy acceleration in model generation.

\subsection{Qualitative Experiments} Consistent with the experimental setup described earlier, we visualize the video results generated under different error thresholds (E1, E2, \ldots, E5). The generated images are presented in Fig.~\ref{fig:maxinexp}. In these visual comparisons, our method demonstrates a remarkable ability to maintain video quality, particularly in terms of color accuracy and detail preservation.



% \begin{table*}[h]
% \centering
% \caption{VISUAL QUALITY VISUALIZATION RESULTS.}
% \label{table:VISUAL QUALITY VISUALIZATION RESULTS}
%    \resizebox{\textwidth}{!}{%
%     \begin{tabular}{cccccc}

%         \multicolumn{2}{c}{Open-Sora} & \multicolumn{2}{c}{CogVideoX} & \multicolumn{2}{c}{Latte} \\
%         Original & Ours & Original & Ours & Original & Ours \\
        
%         \includegraphics[width=0.15\textwidth]{ICME/figures/paper_compare/opensora/The timelapse of tos_baseline.jpg} &
%         \includegraphics[width=0.15\textwidth]{ICME/figures/paper_compare/opensora/The timelapse of tos.jpg} &
%         \includegraphics[width=0.15\textwidth]{ICME/figures/paper_compare/cog/The timelapse ocog_baseline.jpg} &
%         \includegraphics[width=0.15\textwidth]{ICME/figures/paper_compare/cog/The timelapse of tcog.jpg} &
%         \includegraphics[width=0.15\textwidth]{ICME/figures/paper_compare/latte/The timelapse latte_baseline.jpg} &
%         \includegraphics[width=0.15\textwidth]{ICME/figures/paper_compare/latte/The timelapse of tlatte.jpg} \\

%         \multicolumn{6}{p{\linewidth}}{\small prompt: The timelapse of the northern lights dancing across the Arctic sky, stars twinkling, snow-covered landscape.} \\

%         \includegraphics[width=0.15\textwidth]{ICME/figures/paper_compare/opensora/A festive Christmasos_baseline.jpg} &
%         \includegraphics[width=0.15\textwidth]{ICME/figures/paper_compare/opensora/A festive Christmasos.jpg} &
%         \includegraphics[width=0.15\textwidth]{ICME/figures/paper_compare/cog/A festive Christcog_baseline.jpg} &
%         \includegraphics[width=0.15\textwidth]{ICME/figures/paper_compare/cog/A festive Christmascog.jpg} &
%         \includegraphics[width=0.15\textwidth]{ICME/figures/paper_compare/latte/A festive Christmas atte_baseline.jpg} &
%         \includegraphics[width=0.15\textwidth]{ICME/figures/paper_compare/latte/A festive Christmaslatte.jpg} \\

%         \multicolumn{6}{p{\linewidth}}{\small prompt: A festive Christmas scene with a decorated tree, presents, and a warm fireplace.} \\

%         \includegraphics[width=0.15\textwidth]{ICME/figures/paper_compare/opensora/A whimsical scene ios_baseline.jpg} &
%         \includegraphics[width=0.15\textwidth]{ICME/figures/paper_compare/opensora/A whimsical scene ios.jpg} &
%         \includegraphics[width=0.15\textwidth]{ICME/figures/paper_compare/cog/A whimsical scencog_baseline.jpg} &
%         \includegraphics[width=0.15\textwidth]{ICME/figures/paper_compare/cog/A whimsical scene icog.jpg} &
%         \includegraphics[width=0.15\textwidth]{ICME/figures/paper_compare/latte/A whimsical latte_baseline.jpg} &
%         \includegraphics[width=0.15\textwidth]{ICME/figures/paper_compare/latte/A whimsical scene ilatte.jpg} \\

%         \multicolumn{6}{p{\linewidth}}{\small prompt: A whimsical scene in an enchanted forest, where sunlight filters through the leaves, casting a glow on a hidden, sparkling fairy path.} \\

%         \includegraphics[width=0.15\textwidth]{ICME/figures/paper_compare/opensora/A serene sunset oveos_baseline.jpg} &
%         \includegraphics[width=0.15\textwidth]{ICME/figures/paper_compare/opensora/A serene sunset oveos.jpg} &
%         \includegraphics[width=0.15\textwidth]{ICME/figures/paper_compare/cog/A serene sunset cog_baseline.jpg} &
%         \includegraphics[width=0.15\textwidth]{ICME/figures/paper_compare/cog/A serene sunset ovecog.jpg} &
%         \includegraphics[width=0.15\textwidth]{ICME/figures/paper_compare/latte/A serene sunset latte_baseline.jpg} &
%         \includegraphics[width=0.15\textwidth]{ICME/figures/paper_compare/latte/A serene sunset ovelatte.jpg} \\

%         \multicolumn{6}{p{\linewidth}}{\small prompt: A serene sunset over a tranquil lake with a distant mountain range in the background.} \\

%     \end{tabular}%
% }
% \end{table*}

% Compared to baseline models, our approach captures complex light and shadow variations without loss during acceleration, resulting in highly realistic effects in scenes such as the aurora and forest. Additionally, in festive scenes, our method preserves the warmth and holiday atmosphere of the images. These results highlight the robust capability of our error-aware approach in maintaining video quality.

\subsection{In-depth Analysis}
To investigate the acceleration potential and characteristics of our strategy, we conducted extensive ablation experiments. In the following experiments, we deployed Open-Sora 1.2 as the base model and used a single Nvidia A800 GPU to generate 49-frames videos.

\begin{figure}[t]
    \centering
    \includegraphics[width=\linewidth]{figures/aba1.pdf}
    \caption{Visualization of generated video quality, latency, and computational complexity under different error thresholds.}
    \label{fig:flops_comparison}
    \vspace{-4mm}
\end{figure}
\begin{table}[t]
\caption{Quantitative analysis of different caching strategies.}
\label{table:Comparison of different strategies}
\centering
\resizebox{\columnwidth}{!}{
\begin{tabular}{lccc}
\toprule
Strategy         & Latency (s) & $\Delta$ & VBench (\%) $\uparrow$ \\ \hline
attention output & 49.19       & 5.19     & \textbf{78.23}                  \\
attention map    & 49.27       & 5.11     & 78.26                  \\
dynamic select   & \textbf{49.07}       & \textbf{5.31}     & 78.17                  \\ \bottomrule
\end{tabular}
}
\vspace{-4mm}
\end{table}
\noindent{\textbf{Error Thereshold Analysis.}} We evaluated the computational complexity, latency, and video quality of models compressed with UniCP on OpenSora across different error thresholds (Fig. \ref{fig:flops_comparison}). Results demonstrate that increasing the error threshold leads to a significant reduction in both computational complexity and latency, while the generated video quality remains largely stable, exhibiting only minor decreases.
\begin{figure}[t]
    \centering
    \includegraphics[width=0.98\linewidth]{figures/aba_slice.pdf}
    \caption{Visual results across various slice ratios.}
    \label{fig:slice1}
    \vspace{-4mm}
\end{figure}

% \subsubsection{Speed up}
% We extended the model's inference process to multiple GPUs, with specific experimental results shown in the Fig\ref{fig:latency_comparison}. Based on extensive analysis, it is demonstrated that the model's dynamic caching mechanism performs well under multi-GPU conditions. The acceleration achieved by our caching method shows an almost linear increase with a smaller number of GPUs, reaching nearly an 8-fold speedup when using 8 GPUs.
\noindent{\textbf{Caching Strategy Analysis.}} Caching entire blocks typically induces significant errors. To address this, we developed dynamic caching strategies for the attention output and attention map. TABLE \ref{table:Comparison of different strategies} presents the quantitative compression results on OpenSora with an error threshold of 0.025. The results show that caching the attention map achieves greater computational savings but leads to more video quality degradation. In contrast, our proposed strategies reduce computational overhead while maintaining video quality.

% \begin{table}[t]
%     \caption{Strategy evaluation.}
%     \label{table:Comparison of different strategies}
%     \centering
%     \begin{tabular}{lccc}
%         \toprule
%         strategy & latency (s) & $\Delta$ & VBench (\%) $\uparrow$ \\
%         \midrule
%         Dynamic Cache & 0 & 0 & 0 \\
%         Window Attention Cache & 0 & 0 & 0 \\

%         All Strategies & 0 & -- & 0 \\
%         \bottomrule
%     \end{tabular}
% \end{table}
\noindent{\textbf{Slice Ratio Analyse.}} Fig. \ref{fig:slice1} illustrates the performance of the PCAS strategy across different partition ratios. High image quality is maintained when the partition ratio is below 0.4. In this work, we dynamically adjust the partition ratio within the range of 0.1 to 0.4, adhering to the defined error threshold.
% \label{sec:intro}


% \begin{table*}[h]
%     \centering
%     \begin{tabular}{c c c c c c c}
%         & \textbf{Raw} & \textbf{D1} & \textbf{D2} & \textbf{D3} & \textbf{D4} & \textbf{D5}  \\
%         \rotatebox[origin=c]{90}{\parbox[c]{2cm}{\centering Open-Sora}} &
%         \includegraphics[width=0.13\textwidth]{ICME/figures/paper_picture/A whimsical scene ios.jpg} &
%         \includegraphics[width=0.13\textwidth]{ICME/figures/paper_picture/A whimsical scene iosD1.jpg} &
%         \includegraphics[width=0.13\textwidth]{ICME/figures/paper_picture/A whimsical scene iosD2.jpg} &
%         \includegraphics[width=0.13\textwidth]{ICME/figures/paper_picture/A whimsical scene iosD3.jpg} &
%         \includegraphics[width=0.13\textwidth]{ICME/figures/paper_picture/A whimsical scene iosD4.jpg} &
%         \includegraphics[width=0.13\textwidth]{ICME/figures/paper_picture/A whimsical scene iosD5.jpg} \\

%        \rotatebox[origin=c]{90}{\parbox[c]{2cm}{\centering Latte}} &
%         \includegraphics[width=0.13\textwidth]{ICME/figures/paper_picture/The timelapse of tlatte.jpg} &
%         \includegraphics[width=0.13\textwidth]{ICME/figures/paper_picture/The timelapse of tlatteD1.jpg} &
%         \includegraphics[width=0.13\textwidth]{ICME/figures/paper_picture/The timelapse of tlatteD2.jpg} &
%         \includegraphics[width=0.13\textwidth]{ICME/figures/paper_picture/The timelapse of tlatteD3.jpg} &
%         \includegraphics[width=0.13\textwidth]{ICME/figures/paper_picture/The timelapse of tlatteD4.jpg} &
%         \includegraphics[width=0.13\textwidth]{ICME/figures/paper_picture/The timelapse of tlatteD5.jpg} \\


%         \rotatebox[origin=c]{90}{\parbox[c]{2cm}{\centering CogVideoX}} &
%         \includegraphics[width=0.13\textwidth]{ICME/figures/A detailed depiccog.jpg} &
%         \includegraphics[width=0.13\textwidth]{ICME/figures/paper_picture/A detailed depictiocogD1.jpg} &
%         \includegraphics[width=0.13\textwidth]{ICME/figures/paper_picture/A detailed depictiocogD2.jpg} &
%         \includegraphics[width=0.13\textwidth]{ICME/figures/paper_picture/A detailed depictiocogD3.jpg} &
%         \includegraphics[width=0.13\textwidth]{ICME/figures/paper_picture/A detailed depictiocogD4.jpg} &
%         \includegraphics[width=0.13\textwidth]{ICME/figures/paper_picture/A detailed depictiocogD5.jpg} \\

%     \end{tabular}
%     \caption{Image generation samples at various image resolutions under various compression ratios.}
%     \label{fig:image_samples}
% \end{table*}