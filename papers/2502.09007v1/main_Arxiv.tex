\documentclass[lettersize,journal]{IEEEtran}
\usepackage{amsmath,amsfonts}
% \usepackage{algorithmic}
\usepackage{array}
\usepackage[caption=false,font=normalsize,labelfont=sf,textfont=sf]{subfig}
\usepackage{textcomp}
\usepackage{stfloats}
\usepackage{url}
\usepackage{verbatim}
\usepackage{graphicx}

\usepackage{xspace}

% Add for pseudo code
\usepackage{algorithmicx}
\usepackage{algpseudocode}

\usepackage{multirow} % multirow를 위한 패키지
\usepackage{array}    % 향상된 테이블 기능을 위한 패키지
\usepackage{booktabs} % 더 나은 테이블 라인을 위한 패키지
\usepackage{makecell} % 테이블 헤더를 위한 패키지

\renewcommand\theadfont{\bfseries} % makecell의 헤더 볼드체로 설정

\hyphenation{op-tical net-works semi-conduc-tor IEEE-Xplore}
\def\BibTeX{{\rm B\kern-.05em{\sc i\kern-.025em b}\kern-.08em
    T\kern-.1667em\lower.7ex\hbox{E}\kern-.125emX}}
\usepackage{balance}
\begin{document}
% \title{How to Use the IEEEtran \LaTeX \ Templates}
\author{IEEE Publication Technology Department


% \author{Masahito Hayashi 
% \IEEEmembership{Fellow, IEEE}, Masaki Owari
% \thanks{M. Hayashi is with Graduate School  
% of Mathematics, Nagoya University, Nagoya, 
% Japan}
% \thanks{M. Owari is with the Faculty of 
% Informatics, Shizuoka University, 
% Hamamatsu, Shizuoka, Japan.}
% }

\author{\IEEEauthorblockN{
Jae-Young Kim
\IEEEmembership{Graduate Student Member, IEEE}, 
Donghyuk Kim
\IEEEmembership{Graduate Student Member, IEEE},
Seungjae Yoo
\IEEEmembership{Graduate Student Member, IEEE},
Sungyeob Yoo
\IEEEmembership{Graduate Student Member, IEEE},
Teokkyu Suh
\IEEEmembership{Graduate Student Member, IEEE},
Joo-Young Kim
\IEEEmembership{Senior Member, IEEE}}\\
\IEEEauthorblockA{School of Electrical Engineering,
Korea Advanced Institute of Science and Technology (KAIST),
Daejeon, Republic of Korea}\\
\{jykim1109, kar02040, goldenyoo, sungyeob.yoo, ejrrb102, jooyoung1203\}@kaist.ac.kr
}


\thanks{Manuscript created October, 2020; This work was developed by the IEEE Publication Technology Department. This work is distributed under the \LaTeX \ Project Public License (LPPL) ( http://www.latex-project.org/ ) version 1.3. A copy of the LPPL, version 1.3, is included in the base \LaTeX \ documentation of all distributions of \LaTeX \ released 2003/12/01 or later. The opinions expressed here are entirely that of the author. No warranty is expressed or implied. User assumes all risk.}}

% \markboth{Journal of \LaTeX\ Class Files,~Vol.~18, No.~9, September~2020}%
% {How to Use the IEEEtran \LaTeX \ Templates}

\newcommand{\sysname}{RED\xspace}
\title{\sysname: Energy Optimization Framework for eDRAM-based PIM with Reconfigurable Voltage Swing and Retention-aware Scheduling} 

\maketitle

\begin{abstract}

Hierarchical clustering is a powerful tool for exploratory data analysis, organizing data into a tree of clusterings from which a partition can be chosen. This paper generalizes these ideas by proving that, for any reasonable hierarchy, one can optimally solve any center-based clustering objective over it (such as $k$-means). Moreover, these solutions can be found exceedingly quickly and are \emph{themselves} necessarily hierarchical. 
%Thus, given a cluster tree, we show that one can quickly generate a myriad of \emph{new} hierarchies from it. 
Thus, given a cluster tree, we show that one can quickly access a plethora of new, equally meaningful hierarchies.
Just as in standard hierarchical clustering, one can then choose any desired partition from these new hierarchies. We conclude by verifying the utility of our proposed techniques across datasets, hierarchies, and partitioning schemes.


\end{abstract}


\begin{IEEEkeywords}
Embedded DRAM (eDRAM), Processing-In-Memory (PIM), Reconfigurable, Retention-aware Scheduling, Energy Efficiency
\end{IEEEkeywords}


\section{Introduction}%

Decision-making is at the heart of artificial intelligence systems, enabling agents to navigate complex environments, achieve goals, and adapt to changing conditions. Traditional decision-making frameworks often rely on associations or statistical correlations between variables, which can lead to suboptimal outcomes when the underlying causal relationships are ignored \citep{pearl2009causal}. 
The rise of causal inference as a field has provided powerful frameworks and tools to address these challenges, such as structural causal models and potential outcomes frameworks \citep{rubin1978bayesian,pearl2000causality}. 
Unlike traditional methods, \textit{causal decision-making} focuses on identifying and leveraging cause-effect relationships, allowing agents to reason about the consequences of their actions, predict counterfactual scenarios, and optimize decisions in a principled way \citep{spirtes2000causation}. In recent years, numerous decision-making methods based on causal reasoning have been developed, finding applications in diverse fields such as recommender systems \citep{zhou2017large}, clinical trials \citep{durand2018contextual}, finance \citep{bai2024review}, and ride-sharing platforms \citep{wan2021pattern}. Despite these advancements, a fundamental question persists: 

\begin{center}
    \textit{When and why do we need causal modeling in decision-making?}
\end{center} 

% Numerous decision-making methods based on causal reasoning have been developed recently with wide applications 
% %Decision makings based on causal reasoning have been widely applied 
% in a variety of fields, including 
% recommender systems \citep{zhou2017large}, clinical trials \citep{durand2018contextual}, 
% finance \citep{bai2024review}, 
% ride-sharing platforms \citep{wan2021pattern}, and so on. 


 

% At the intersection of these fields, causal decision-making seeks to answer critical questions: How can agents make decisions when causal knowledge is incomplete? How do we integrate learning and reasoning about causality into real-world decision-making systems? What role do interventions, counterfactuals, and observational data play in guiding decisions? 

% Our review is structured as follows: 
 

This question is closely tied to the concept of counterfactual thinking—reasoning about what might have happened under alternative decisions or actions. Counterfactual analysis is crucial in domains where the outcomes of unchosen decisions are challenging, if not impossible, to observe. For instance, a business leader selecting one marketing strategy over another may never fully know the outcome of the unselected option \citep{rubin1974estimating, pearl2009causal}. Similarly, in econometrics, epidemiology, psychology, and social sciences, \textit{the inability to observe counterfactuals directly often necessitates causal approaches} \citep{morgan2015counterfactuals, imbens2015causal}. 
Conversely, non-causal analysis may suffice in scenarios where alternative outcomes are readily determinable. For example, a personal investor's actions may have negligible impact on stock market dynamics, enabling potential outcomes of alternate investment decisions to be inferred from existing stock price time series \citep{angrist2008mostly}. However, even in cases where counterfactual outcomes are theoretically calculable—such as in environments with known models like AlphaGo—exhaustively computing all possible outcomes is computationally infeasible \citep{silver2017mastering, silver2018general}. 
In such scenarios, causal modeling remains advantageous by offering \textit{structured ways to infer outcomes efficiently and make robust decisions}. 


%This perspective not only enhances the interpretability of decisions but also provides a principled framework for addressing uncertainty, guiding actions, and improving performance across a broad range of applications.

% Data-driven decision-making exists before the causal revolution. \textit{So when and why do we need causal modelling in decision-making?} 
% This is closely related to the presence of counterfactuals in many applications. 
% The counterfactual thinking involves considering what would have happened in an alternate scenario where a different decision or action was taken. 
% In many fields, including econometrics, epidemiology, psychology, and social sciences, accessing outcomes from unchosen decisions is often challenging if not impossible. 
% For example, a business leader who selects one marketing strategy over another may never know the outcome of the unselected option. 
% Conversely, non-causal analysis may be adequate in situations where potential outcomes of alternate actions are more readily determinable: for example, the investment of a personal investor may have minimal impact on the market, therefore her counterfactual investment decision's outcomes can still be calculated with the data of stock price time series. 
% However, it is important to note that even when counterfactuals are theoretically calculable, as in environments with known models like AlphaGo, computing all possible outcomes may not be feasible. 
% In such scenarios, a causal perspective  remains beneficial. 


 

% 1. significance of decision making
% 2. role of causal in decision making
% 3. refer to the https://jair.org/index.php/jair/article/view/13428/26917

% Decision makings based on causal reasoning have been widely applied in a variety of fields, including recommender systems \citep{zhou2017large}, clinical trials \citep{durand2018contextual}, 
% business economics scenarios \citep{shen2015portfolio}, 
% ride-sharing platforms \citep{wan2021pattern}, and so on. 
% However, most existing works primarily assume either sophisticated prior knowledge or strong causal models to conduct follow-up decision-making. To make effective and trustworthy decisions, it is critical to have a thorough understanding of the causal connections between actions, environments, and outcomes.

\begin{figure}[!t]
    \centering
    \includegraphics[width = .75\linewidth]{Figure/3Steps_V2.png}
    \caption{Workflow of the \acrlong{CDM}. $f_1$, $f_2$, and $f_3$ represent the impact sizes of the directed edges. Variables enclosed in solid circles are observed, while those in dashed circles are actionable.}\label{fig:cdm}
\end{figure}


Most existing works primarily assume either sophisticated prior knowledge or strong causal models to conduct follow-up decision-making. To make effective and trustworthy decisions, it is critical to have a thorough understanding of the causal relationships among actions, environments, and outcomes. This review synthesizes the current state of research in \acrfull{CDM}, providing an overview of foundational concepts, recent advancements, and practical applications. Specifically, this work discusses the connections of \textbf{three primary components of decision-making} through a causal lens: 1) discovering causal relationships through \textit{\acrfull{CSL}}, 2) understanding the impacts of these relationships through \textit{\acrfull{CEL}}, and 3) applying the knowledge gained from the first two aspects to decision making via \textit{\acrfull{CPL}}. 

Let $\boldsymbol{S}$ denote the state of the environment, which includes all relevant feature information about the environment the decision-makers interact with, $A$ the action taken, $\pi$ the action policy that determines which action to take, and $R$ the reward observed after taking action $A$. As illustrated in Figure \ref{fig:cdm}, \acrshort{CDM} typically begins with \acrshort{CSL}, which aims to uncover the unknown causal relationships among various variables of interest. Once the causal structure is established, \acrshort{CEL} is used to assess the impact of a specific action on the outcome rewards. To further explore more complex action policies and refine decision-making strategies, \acrshort{CPL} is employed to evaluate a given policy or identify an optimal policy. In practice, it is also common to move directly from \acrshort{CSL} to \acrshort{CPL} without conducting \acrshort{CEL}. Furthermore, \acrshort{CPL} has the potential to improve both \acrshort{CEL} and \acrshort{CSL} by facilitating the development of more effective experimental designs \citep{zhu2019causal,simchi2023multi} or adaptively refining causal structures \citep{sauter2024core}. %However, these are beyond the scope of this paper.

\begin{figure}[!t]
    \centering
    \includegraphics[width = .9\linewidth]{Figure/Table_of_Six_Scenarios_S.png}
    \caption{Common data dependence structures (paradigms) in \acrshort{CDM}. Detailed notations and explanations can be found in Section \ref{sec:paradigms}.}
    \label{Fig:paradigms}
\end{figure}
Building on this framework, decision-making problems discussed in the literature can be further categorized into \textbf{six paradigms}, as summarized in Figure \ref{Fig:paradigms}. These paradigms summarize the common assumptions about data dependencies frequently employed in practice. Paradigms 1-3 describe the data structures in offline learning settings, where data is collected according to an unknown and fixed behavior policy. In contrast, paradigms 4-6 capture the online learning settings, where policies dynamically adapt to newly collected data, enabling continuous policy improvement. These paradigms also reflect different assumptions about state dependencies. The simplest cases, paradigms 1 and 4, assume that all observations are independent, implying no long-term effects of actions on future observations. To account for sequental dependencies, the \acrfull{MDP} framework, summarized in paradigms 2 and 5, assumes Markovian state transition. Specifically, it assumes that given the current state-action pair $(S_t, A_t)$, the next state $S_{t+1}$ and reward $R_t$ are independent of all prior states $\{S_j\}_{j < t}$ and actions $\{A_j\}_{j < t}$. When such independence assumptions do not hold, paradigms 3 and 6 account for scenarios where all historical observations may impact state transitions and rewards. This includes but not limited to researches on \acrfull{POMDP} \citep{hausknecht2015deep, littman2009tutorial}, panel data analysis \citep{hsiao2007panel,hsiao2022analysis}, \acrfull{DTR} with finite stages \citep{chakraborty2014dynamic, chakraborty2013statistical}. 

Each \acrshort{CDM} task has been studied under different paradigms, with \acrshort{CSL} extensively explored within paradigm 1. \acrshort{CEL} and offline \acrshort{CPL} encompass paradigms 1-3, while online \acrshort{CPL} spans paradigms 4-6. By organizing the discussion around these three tasks and six paradigms, this review aims to provide a cohesive framework for understanding the field of \acrlong{CDM} across diverse tasks and data structures.

%Recognizing the importance of long-term effects in decision-making

%Further discussions on these paradigms and their connections to various causal decision-making problems are provided in Section \ref{sec:paradigms}.


\textbf{Contribution.} In this paper, we conduct a comprehensive survey of \acrshort{CDM}. 
Our contributions are as follows. 
\begin{itemize}
    \item We for the first time organize the related causal decision-making areas into three tasks and six paradigms, connecting previously disconnected areas (including economics, statistics, machine learning, and reinforcement learning) using a consistent language. For each paradigm and task, we provide a few taxonomies to establish a unified view of the recent literature.
    \item We provide a comprehensive overview of \acrshort{CDM}, covering all three major tasks and six classic problem structures, addressing gaps in existing reviews that either focus narrowly on specific tasks or paradigms or overlook the connection between decision-making and causality (detailed in Section \ref{sec::related_work}).
    %\item We outline three key challenges that emerge when utilizing CDM in practice. Moreover, we delve into a comprehensive discussion on the recent advancements and progress made in addressing these challenges. We also suggest six future directions for these problems.
    \item We provide real-world examples to illustrate the critical role of causality in decision-making and to reveal how \acrshort{CSL}, \acrshort{CEL} and \acrshort{CPL} are inherently interconnected in daily applications, often without explicit recognition.
    \item We are actively maintaining and expanding a GitHub repository and online book, providing detailed explanations of key methods reviewed in this paper, along with a code package and demos to support their implementation, with URL: \url{https://causaldm.github.io/Causal-Decision-Making}.
\end{itemize}
% Our review is structured as follows: 


%%%%%%%%%%%%%%%%%%%%%%%%%%%%%%%%%%
%  causal helps over "Correlational analysis"
%Correlational analysis, though widely used in various fields, has inherent limitations, particularly when it comes to decision-making. While it identifies relationships between variables, it fails to establish causality, often leading to misinterpretations and misguided decisions. For example, the positive correlation between ice cream sales and drowning incidents is a classic example of how correlational data can be misleading, as both are influenced by a third factor, temperature, rather than causing each other. Such spurious correlations, due to oversight of confounding variables, underscore the necessity of causal modeling in decision making. Causal models excel where correlational analysis falls short, offering predictive power and a deeper understanding of underlying mechanisms. They enable us to predict the outcomes of interventions, even under untested conditions, and provide insights into the processes leading to these outcomes, thereby informing more effective strategies. Moreover, causal models are good at generalizing findings across different contexts, a capability often limited in purely correlational studies. 

%  causal helps in causal RL 
%From another complementary angle, although causal concepts have traditionally not been explicitly incorporated in fields like online bandits \citep{lattimore2020bandit} and \acrfull{RL} \citep{sutton2018reinforcement}, much of the literature in these areas implicitly relies on basic assumptions outlined in Section \ref{sec:prelim_assump} to utilize observed data in place of potential outcomes in their analyses, and there is also a growing recognition of the significance of the causal perspective \citep{lattimore2016causal, zeng2023survey} in these areas. 
% \textbf{Read causal RL survey and summarize. } However, by integrating causal concepts and leverging existing methodologies, we open up possibilities for developing more robust models to remove spurious correlation and selection bias \citep{xu2023instrumental, forney2017counterfactual}, designing more sample-efficient \citep{sontakke2021causal, seitzer2021causal} and robust \citep{dimakopoulou2019balanced, ye2023doubly} algorithms, and improving the generalizability \citep{zhang2017transfer, eghbal2021learning}, explanability \citep{foerster2018counterfactual, herlau2022reinforcement}, and fairness \citep{zhang2018fairness,huang2022achieving,balakrishnan2022scales} of these methods. %, and safety \cite{hart2020counterfactual}

%


%\subsection{Paper Structure}
The remainder of this paper is organized as follows: Section \ref{sec::related_work} provides an overview of related survey papers. Section \ref{sec:preliminary} introduces the foundational concepts, assumptions, and notations that form the foundation for the subsequent discussions. In Section \ref{sec:3task6paradigm}, we offer a detailed introduction to the three key tasks and six learning paradigms in \acrshort{CDM}. Sections \ref{Sec:CSL} through \ref{sec:Online CPL} form the core of the paper, with each section dedicated to a specific topic within \acrshort{CDM}: \acrshort{CSL}, \acrshort{CEL}, Offline \acrshort{CPL}, and Online \acrshort{CPL}, respectively. Section \ref{sec:assump_violated} then explores extensions needed when standard causal assumptions are violated. To illustrate the practical application of the \acrshort{CDM} framework, Section \ref{sec:real_data} presents two real-world case studies. Finally, Section \ref{sec:conclusion} concludes the paper with a summary of our contributions and a discussion of additional research directions that are actively being explored.


\section{The Sequential Bottleneck in Large Model Inference}
\label{sec:sequential_bottleneck}

\subsection{Understanding Sequential Dependencies}
\label{sec:sequential_dependencies}

Modern LLMs, such as the Llama series~\cite{touvron2023llama,touvron2023llama2,dubey2024llama} and the GPT series~\cite{radford2019language,brown2020language}, are built on transformer architectures consisting of stacked decoder blocks. As shown in Figure~\ref{fig:architech}(a), each decoder block contains two fundamental components: a Self-Attention (SA) block and a feed-forward network (FFN). During execution, the input of the SA block is first multiplied with three weight matrices $W_{Q}$, $W_{K}$, and $W_{V}$, yielding the outputs termed query ($q$), key ($k$), and value ($v$), respectively.

\begin{figure*}
    \centering
    \includegraphics[width=0.9\linewidth]{figures/overview_llm_intro.pdf}
    \caption{(a) The Llama architecture consists of stacked transformer decoder blocks. (b) Each decoder block contains a self-attention (SA) block and feedforward (FFN) block. (c) During the decoding stage, tokens are generated auto-regressively.}
    \label{fig:architech}
\end{figure*}

The computation flow, detailed in Figure~\ref{fig:architech}(b), shows how query and key vectors compute attention scores through matrix multiplication. After softmax normalization, these scores weight the value vectors, producing the SA output through a weighted sum and residual connection. This SA output feeds into the FFN, typically implemented as either a standard MLP~\cite{radford2018improving, radford2019language} or gated MLP~\cite{liu2021pay, touvron2023llama,touvron2023llama2}, with multiple fully connected layers and activation functions like GeLU~\cite{hendrycks2016gaussian} or SiLU~\cite{elfwing2018sigmoid}.

The core challenge emerges during inference, which consists of two main phases: prefill and decoding. While the prefill phase can process input sequences in parallel, the decoding phase introduces a critical bottleneck. As shown in Figure~\ref{fig:architech}(c), the model must predict each token sequentially, using both current and previous token information through their Key and Value (KV) vectors. These KV vectors are cached for subsequent predictions, leading to significant memory access latency as the sequence length grows.

\subsection{Breaking Sequential Dependencies}
\label{sec:breaking_dependencies}

Traditional approaches to accelerating LM inference have focused on reducing computational costs through model compression, knowledge distillation, and architectural optimizations. However, these methods primarily address individual computation costs rather than the fundamental sequential dependency that requires each token to wait for all previous tokens.

\begin{figure}
    \centering
    \includegraphics[width=0.85\linewidth]{figures/sd_intro_new.pdf}
    \caption{Illustration of speculative decoding workflow.}
    \label{fig:sd_intro}
\end{figure}

Speculative decoding (SD)~\cite{stern2018blockwise} has emerged as a promising solution that directly targets this sequential bottleneck. As illustrated in Figure~\ref{fig:sd_intro}, this approach introduces a two-phase process where a smaller, faster \textit{draft model} first predicts multiple tokens in parallel, followed by verification using the target model. The draft model enables parallel token generation, breaking away from traditional token-by-token generation, while the target model's verification step maintains output quality through accept/reject decisions.

This strategy has proven particularly valuable for real-time applications like interactive dialogue systems, where response latency directly impacts user experience. The verification mechanism provides a crucial balance between generation speed and output quality, accepting correct predictions to maintain throughput while falling back to sequential generation when necessary to preserve accuracy.

While SD represents one successful approach to breaking sequential dependencies in autoregressive (AR) models, it belongs to a broader family of \textit{generation-refinement} methods. The following sections present a systematic taxonomy of these approaches, examining how different techniques balance the trade-offs between generation parallelism and output quality.
\begin{figure}[t]
\centering
\includegraphics[width=0.48\textwidth]{Arxiv/Figure/Observation1.pdf}
\caption{Breakdown of Memory Energy Consumption on Different Use Cases}
%\vspace{-0.1in}
\label{figure3-1}
\end{figure}

\section{Observations}
\label{observations}


In this work, we analyze the key observations outlined below. These insights form the foundation of our proposed \sysname framework to fully optimize both the power and energy consumption of memory access, achieving high energy efficiency.

\textbf{Observation 1: dominant energy consumption of memory access in PIM.} In the von Neumann architecture, off-chip data movement causes the primary energy consumption in accessing memory. To minimize this energy consumption, the architecture aims to maximally reuse data loaded in the processor with its on-chip memory. In contrast, PIM architecture reduces data movement overhead in exchange for much more frequent memory access for operations by in-memory processing units. The key distinction lies in memory utilization, making the difference in the circuit design of memory in each architecture. Unlike conventional architectures where the energy consumption of refresh is substantial, leading to designs focused on reducing this refresh overhead to enhance performance, PIM architecture faces a different challenge.

Figure ~\ref{figure3-1} shows the ratio of refresh and energy consumption of memory access according to the number of memory accesses where the lifetime of data is set as 1000$\mu$s. The memory is assumed as eDRAM with a retention time of around 100$\mu$s. The refresh overhead is dominant when the number of memory accesses is small, as in the conventional architecture use case. However, the energy consumption of memory access becomes dominant as the number of memory increases; that is the use case of PIM architecture. The graph emphasizes the necessity of designing memory considering a property of architecture and actual use case. Additionally, it is essential to optimize the energy consumption of memory access to improve performance in PIM effectively.

%Various schemes have been proposed to increase the retention time or to make the refresh more efficient: [reference, LPDDR, SALP, ...]. 


\begin{figure}[t]
\centering
\includegraphics[width=0.48\textwidth]{Arxiv/Figure/Observation2.pdf}
%\vspace{-0.2in}
\caption{(a) eDRAM Operation with Large RBL Voltage Swing (b) eDRAM Operation with Small RBL Voltage Swing (c) Trade-off Among RBL Voltage Swing, Sense Amplifier Power, and Retention Time Analysis}
\vspace{-0.1in}
\label{figure3-2}
\end{figure}

\textbf{Observation 2: trade-off among RBL voltage swing, sense amplifier, and retention time.} eDRAM operation with large RBL voltage swing, as shown in Figure ~\ref{figure3-2} (a), enhances the sensing margin. This enhancement allows the sense amplifier to convert the analog signal into a digital signal faster. As a result, the total current flowing through the sense amplifier diminishes, leading to the reduced power consumption of the sense amplifier  (17.10\%), as shown in Figure ~\ref{figure3-2} (c). Even though the charge in a SN fluctuates due to a leakage current, a large sensing margin ensures that the sense amplifier converts the data correctly, resulting in a longer retention time (100$\mu$s). However, as explained in Section ~\ref{background}, the memory access power increases because the RBL voltage swing is the main source of power consumption. Conversely, a smaller RBL voltage swing leads to higher power consumption of the sense amplifier (45.08\%), shortened retention time (9$\mu$s), and smaller memory access power, as depicted in Figure ~\ref{figure3-2} (b) and (c).

Simply opting for a small RBL voltage swing to optimize memory access power falls short of an optimal solution due to the inherent trade-off of eDRAM. While it does lower memory access power, it rather increases the power consumption of the sense amplifier and refresh overhead. Therefore, we have to select a memory operation considering the specific PIM use case that minimizes both memory access power and overall energy consumption.


\begin{figure}[t]
\centering
\includegraphics[width=0.48\textwidth]{Arxiv/Figure/Framework.pdf}
\caption{Overview of The \sysname Framework}
\vspace{-0.1in}
\label{figure4}
\end{figure}


\section{\sysname Framework}
\label{framework}
We propose the \sysname framework, an energy optimization framework for eDRAM-based PIM that can fully optimize the energy consumption of the memory with retention-aware scheduling, optimal architecture, and reconfigurable circuit design.

Figure ~\ref{figure4} presents the overview of the \sysname framework. The framework receives a software configuration (e.g., model parameters, input and weight matrix shapes, data types, etc.) and a hardware configuration (e.g., memory configuration, processing units, operating frequency, etc.) as inputs and goes through two phases: \textit{the scheduling and execution phase}. These two phases collectively allow the \sysname framework to significantly reduce overall energy consumption by optimizing memory operations, which are the primary contributors to energy overhead in PIM operation. The \sysname framework can optimize energy consumption in PIM across diverse configurations. The following section delves into a detailed description of these two phases, revealing how they form the backbone of the \sysname framework.

\subsection{Retention-aware Scheduling}
\label{retention-aware scheduling}
As discussed in Section ~\ref{observations}, achieving high energy efficiency in PIM requires optimizing the energy consumption of memory access. However, simply reducing RBL voltage swing without tailoring it to a specific use case is not the most efficient approach. To address this problem, the \sysname’s scheduling phase undertakes two key tasks: 1) identifying the optimal tiling scheme and 2) selecting the optimal memory operation.

The eDRAM memory requires periodic refreshes due to its retention time. However, it is possible to skip refresh in two cases. Firstly, when the lifetime of data is shorter than the retention time, as proposed in \cite{tu2018rana}, there is no need for refresh. Secondly, refresh can be skipped for the data that is no longer used for computation. Given that PIM architecture cannot process the whole input and weight matrices simultaneously, it employs tiling for GEMM. Memory access pattern and output generation order vary with a tiling scheme, influencing the lifetime of mapped data. Therefore, to enhance the energy efficiency of eDRAM-based PIM, it is essential to schedule operations considering the lifetime and the retention time. 
Additionally, optimally balancing eDRAM's inherent trade-off by considering the actual use case can further reduce energy consumption.
%Additionally, optimally balancing eDRAM’s inherent trade-off by considering both the lifetime of the mapped data and the number of memory accesses can further reduce energy consumption.
The \sysname’s scheduling phase goes through the following four steps to determine the most optimal tiling scheme and memory operation for given input configurations.

\begin{table}[t]
\centering
\caption{\sysname's Variable Definition}
\includegraphics[width=0.48\textwidth]{Arxiv/Figure/Variable.pdf}
\label{figure8}
% \vspace{-0.3in}
\end{table}


\textbf{Step 1: Find all possible tiling schemes.} The scheduler identifies all possible tiling schemes (i.e., loop order and tile shape) based on the input configurations. Table 1 presents the variable definitions used by the \sysname's scheduling phase.  



\textbf{Step 2: Analyze lifetime of data.} In a tiled GEMM operation, the lifetime of each tile varies according to the tiling scheme used. Figure ~\ref{figure9} illustrates this change through two pseudo-code examples and their corresponding computation flows for tiled GEMM. While the overall computation time remains consistent, the lifetime of each tile differs, as the equations show. The computation cycle per tile, denoted by $T$, changes depending on the processing type (i.e., bit-serial and bit-parallel). In step 2, the lifetime of each data is analyzed across all tiling schemes identified in step 1. 



\textbf{Step 3: Estimate energy consumption.} In step 3, we perform energy modeling by using the lifetime analyzed in step 2 and the memory specifications of our proposed reconfigurable eDRAM (i.e., RBL voltage swing, power consumption, and retention time table) as input. For each tiling scheme, total energy consumption is estimated ($E_{total}$) for various RBL voltage swings as Equation (1). $E_{PIM}$ and $E_{Buffer}$ denote the energy consumption of the PIM macro operation and buffer access, respectively. The $E_{PIM}$ and $E_{Buffer}$ are calculated as in Equation (2) and (3). $E_{Acc}$, $E_{PU}$, and $E_{Ref}$ represent the energy consumption of eDRAM access, processing unit, and refresh per operation, respectively. $N$ denotes the number of operations. $T_{Life}$ and $T_{Retention}$ refer to the lifetime of mapped data and the retention time of eDRAM. The prefixes $P$ and $B$ identify the PIM macro and buffer.

\begin{scriptsize}
\begin{align}
    E_{total} = E_{PIM}+E_{Buffer}
\end{align}
%\vspace{-0.1in}
\begin{align}
    E_{PIM} = (E_{P\_Acc}+E_{PU}) \times P\_N + E_{P\_Ref} \times \left[\frac{T_{P\_Life}}{T_{P\_Retention}}\right]
\end{align}
%\vspace{-0.1in}
\begin{align}
    E_{Buffer} = (E_{B\_Acc}) \times B\_N + E_{B\_Ref} \times \left[\frac{T_{B\_Life}}{T_{B\_Retention}}\right]
\end{align}
\end{scriptsize}

\textbf{Step 4: Find optimal computation flow and memory operation.} Based on the output of the energy modeling, tiling scheme, and memory operation that minimize overall energy consumption are identified. By forwarding the results to the PIM macro controller, we fully optimize energy consumption of the memory.

\begin{figure}[t]
\centering
\includegraphics[width=0.48\textwidth]{Arxiv/Figure/Scheduling.pdf}
\caption{Lifetime of Data per Tiling Scheme Analysis}
\vspace{-0.1in}
\label{figure9}
\end{figure}


\begin{figure*}[t]
\centering
\includegraphics[width=1\textwidth]{Arxiv/Figure/Architecture.pdf}
\caption{\sysname's Hardware Template Architecture}
%\vspace{-0.2in}
\label{figure6}
\end{figure*}


\subsection{Hardware Template Architecture}

As highlighted in Section ~\ref{observations}, a memory design that does not take into account the property of an architecture and its use cases falls short of achieving high energy efficiency. To address this problem, we propose a reconfigurable eDRAM-based PIM hardware template architecture to achieve adaptability to the outputs of the \sysname's scheduling phase. Figure ~\ref{figure6} illustrates \sysname’s hardware template architecture, which consists of a retention-aware scheduler, a PIM macro controller, a PIM macro, and a unified buffer. The user can change the memory configuration (e.g., the subarray shape, the number of subarrays and banks, and the buffer size) and the processing unit configuration (e.g., operation types such as dot product/row-wise, processing type, and bit precision).

The retention-aware scheduler finds the optimal operating conditions by evaluating the energy consumption across all possible tiling schemes and memory operations for the given architecture configuration, as detailed in Section ~\ref{retention-aware scheduling}. The lifetime estimator calculates the lifetime of the input, weight, and output tile for the chosen tiling scheme. Subsequently, the energy estimator computes the energy consumption for each memory operation. The energy optimizer forwards the most energy-efficient tiling scheme and voltage select signal to the PIM macro controller. The controller modulates the memory access patterns and adjusts eDRAM operations by activating the row that stores the weight and broadcasting the input to the subarray’s sense amplifier. The input is then multiplied by the weight in the sense amplifier and passed to the processing unit. This unit executes the GEMM or GEMV operation using the output from the eDRAM subarray. Throughout this process, the PIM macro controller adjusts the eDRAM operation by tuning the pull-down voltage (VPD) and reference voltage (VREF), marked with the blue line. The following section delves into the specific hardware implementation for the reconfigurable eDRAM. A further optimization scheme to minimize redundant refresh in eDRAM is discussed below.

Reducing memory access power inevitably leads to more frequent refresh, which in turn increases energy consumption and latency overhead. To counter this, we employ the refresh skipping scheme for the data that has a shorter lifetime than the retention time or the data that is no longer utilized in computation. For example, it is possible to skip refresh if the lifetime of the partial sum is shorter than the retention time, as new data replaces the old. Input tiles and weight tiles also no longer need to be refreshed when they are no longer used in the computation. By skipping redundant refresh in these cases, the PIM macro controller effectively mitigates the increased refresh overhead, thereby enhancing overall energy efficiency.

\subsection{Reconfigurable eDRAM}
We propose a reconfigurable 2T eDRAM macro capable of adjusting the RBL voltage swing according to the pull-down voltage (VPD) controlled by the PIM macro controller. Additionally, we implement two optimizations-adjusting the reference voltage and employing sense amplifier power gating-to further enhance energy efficiency. Figure ~\ref{figure7} illustrates the detailed hardware implementation of the macro, featuring a reconfigurable pull-down driver. This driver employs a pass transistor, targeting a VPD dictated by the PIM macro controller instead of VSS. Adjusting the VPD closer to VSS increases the RBL voltage swing, which in turn enhances a sensing margin and extends a retention time. However, this adjustment concurrently increases the power consumption of the memory access. 

\begin{figure}[t]
\centering
\includegraphics[width=0.48\textwidth]{Arxiv/Figure/Macro.pdf}
\caption{Reconfigurable 2T eDRAM Circuit Design}
\vspace{-0.1in}
\label{figure7}
\end{figure}

\begin{figure*}[t]
\centering
\includegraphics[width=1\textwidth]{Arxiv/Figure/Eval_Energy_Efficiency_1.pdf}
\caption{The \sysname Framework’s Normalized Energy Efficiency Comparison with eDRAM Baseline and Neural Cache for (a) Bit-serial Processing Unit and (b) Bit-parallel Processing Unit}
%\vspace{-0.1in}
\label{figure_energy_efficiency_1}
\end{figure*}

Before runtime, the \sysname framework estimates overall energy consumption for various VPD settings. This estimation is enabled by forwarding the specification about the power consumption and the retention time relative to the VPD of our proposed eDRAM macro to the energy modeling of the scheduling phase. The \sysname framework is able to identify optimal tiling scheme and memory operation to achieve the highest energy efficiency by pre-estimating the energy consumption at each VPD level. The following describes additional optimization schemes aimed at augmenting the functionality and power efficiency of our proposed eDRAM design.

The functionality of the sense amplifier and the retention time of the memory are significantly affected by the reference voltage, as the sense amplifier utilizes this voltage to convert the RBL voltage swing into a digital signal. Thus, we opt to use the midpoint of the RBL voltage swing between data 1 and data 0 as the reference voltage to optimize the performance of the sense amplifier and the retention time. This reference voltage is pre-calculated for each level of VPD, enabling the PIM macro controller to select the reference voltage corresponding to the VPD.

We employ the sense amplifier power gating to mitigate the trade-off between the RBL voltage swing and the power consumption of the sense amplifier, which also effectively substitutes the function of a 1-bit multiplier. As the VPD is set higher, the power consumption of the sense amplifier increases due to a reduced sensing margin. With our design, if the input for computation with weight is zero, the sense amplifier is power gated, ensuring it outputs data 0 regardless of the data actually stored, as depicted in Figure ~\ref{figure7}. Depending on how the data is mapped to the eDRAM cell array, the input data is broadcast in different ways. For bit-parallel mapping, a shared input bit is broadcast because each weight data bit operates with the same input bit. Conversely, in bit-serial mapping, each weight bit operates with a different input bit requiring individual broadcasting. The output of the sense amplifier is the same as the output of the 1-bit multiplier, so it can be aggregated directly in the processing unit. Thus, sense amplifier power gating accomplishes two benefits: 1) it reduces both the area and power overhead of the 1-bit multiplier, and 2) it further optimizes memory access.


\section{Evaluation}
\label{sec:evaluation}
Our experiments aim to investigate whether agents within our framework can produce effective evolution of language strategies. Specifically, our experimental section addresses the following three research questions (RQs):
\begin{enumerate}
    \item RQ1 (Effectiveness): Can participants effectively evade regulatory detection over time, and how does the accuracy of information transmission change? Additionally, how do different LLMs affect the content and effectiveness?
    \item RQ2 (Human Interpretation): Do the evolved language strategies employed by agents effectively align with human understanding? Can they be interpreted and applied in real-world scenarios?
    \item RQ3 (Ablation Study): How does the newly introduced GA impact the evolution process in our framework?
\end{enumerate}

\subsection{Experimental Settings}
In our evaluation, we designed an abstract password game \cite{guess_number02} and a more realistic illicit pet trade scenario\cite{trade01,trade02,trade03}. 
%The password game features a relatively abstract, easily controlled setting, allowing for clear observation of how agents’ strategies evolve. Meanwhile, the illicit pet trade scenario simulates illegal activities on social networks \cite{DiMinin2018MachineLF}, with relevant corpora that more closely resemble real-world conditions, enabling a more direct comparison between evolved strategies and their real-life counterparts.
The overall experimental procedure follows the description in Section~\ref{sec: method}. In each round, the process comprises three stages: initialization, dialogue, and interview. In each round, the information that the participant agents need to convey will be randomly generated within a defined range. Only rounds in which a five-turn dialogue is successfully completed proceed to the interview phase, during which both parties’ successful transmission of information is verified. If any regulation violations are detected by the supervisory agent during the dialogue stage, that round is deemed a failure and is assigned a transmission score of zero. To ensure the robustness of our findings, we conducted 15 independent trials for each experimental condition. Each trial consisted of 50 fixed dialogue rounds. The crossover and mutation probabilities in the framework were set to 0.2 and 0.8.

\subsubsection{Scenario 1: Password Game}
Our first scenario is considered a relatively simple and abstract guessing game that involves a numerical context. In this setup, each participant is assigned a four-digit password, which they must convey to another participant within five rounds of dialogue without directly mentioning the numbers. Simultaneously, they need to extract information from the dialogue to infer the other's password.

The supervisory agent in the dialogue follows the policy of “prohibiting all content related to numbers.” The “password game” scenario is specifically designed to observe and analyze participant agents' language adaptability and strategic evolution in a theoretical and abstract context. Compared to complex scenarios based on real events, it provides a clearer and more easily quantifiable experimental environment.

\subsubsection{Scenario 2: Illicit Pet Trade} 
Simulating and detecting the covert drug transactions prevalent on social platforms is an important and realistic research topic. However, given the ethical considerations, we choose not to proceed with direct simulations, but to adopt a more neutral and less harmful setting—a illicit pet trade scenario—to serve as our experimental context.

In this scenario, we simulate an illegal pet trade through social platform. Participant agents play the roles of buyer and seller. The buyer conveys a request to purchase a randomly selected pet (such as a parrot, lizard, or cat) and specifies the currency they can use for payment (e.g., USD, CNY, JPY). The seller provides details about the transaction, including the time (morning, noon, or midnight) and location (school, park, or station). The supervisory agent enforces a policy that considers discussions related to pet transactions as regulation violations.
The focus of this experiment is to observe the adaptability of the participant agents' language and its evolution at a finer level of information granularity, as well as whether regulators can accurately detect regulation violations.

\subsection{RQ1: Effectiveness}
\subsubsection{Experiment Settings}

To evaluate RQ1, we conducted experiments using four different LLM models provided by various companies: \textit{gpt-4o}, \textit{gpt-4o-mini}, \textit{deepseekV2}, and \textit{qwen-turbo}. We introduced the following metric:
\begin{itemize}
    \item Accuracy of Information Transmission: This metric measures the agents' ability to convey the intended information to their dialogue partner without being detected. Specifically, it calculates the proportion of successfully transmitted information relative to the total intended information. 
    \item Average Dialogue turns: This metric evaluates the number of successful dialogue rounds the agents complete without being flagged by the supervisory agent. A higher dialogue turns indicates greater success in evading regulatory scrutiny.
    \item Average Entropy (Language Complexity):
        Entropy reflects the unpredictability of text and is calculated as:
        \[
        H(X) = -\sum_{i=1}^n P(x_i) \log P(x_i),
        \]
        where \(P(x_i)\) is the probability of each token. Higher entropy suggests a broader range of expressions (indicating greater innovation), but excessively high entropy can lead to incoherence.
    \item Distinct (Lexical Diversity):
        Distinct measures the number of unique n-grams in the text, computed as:
        \[
        \text{Distinct-n} = \frac{\text{Number of unique n-grams}}{\text{Total number of n-grams}}.
        \]
        Specifically, we evaluate Distinct-1 (unique unigrams). A higher distinct score indicates a larger, more varied vocabulary, signifying richer language use. Note that while both metrics assess diversity in language, entropy focuses on unpredictability, whereas distinct emphasizes lexical variety. 
\end{itemize}


\subsubsection{Experiment Results in Password Game}
\begin{figure*}[ht]
    \centering
    % 子图 (a)
    \begin{subfigure}[t]{0.48\textwidth}
        \centering
        \includegraphics[width=\linewidth]{figures/sec1_turn_acc_v5.png}
        \caption{Password Game}
        \label{fig:sce1}
    \end{subfigure}
    \hfill
    % 子图 (b)
    \begin{subfigure}[t]{0.48\textwidth}
        \centering
        \includegraphics[width=\linewidth]{figures/sec2_turn_acc_v5.png}
        \caption{Illicit Pet Trade}
        \label{fig:sce2}
    \end{subfigure}
    \caption{Average Continuous Dialogue Turns and Information Transmission Accuracy Across Dialogue Rounds}
    \label{fig:merged}
\end{figure*}

Figure \ref{fig:sce1} presents our experimental results in the password game. The x-axis corresponds to the increasing number of dialogue rounds, whereas the y-axis captures two primary metrics: (1) the average number of continuous dialogue turns before detection by the supervisory agent, and (2) the accuracy of information transmission, which is defined as the proportion of successfully transmitted information during the post-dialogue interview.

Overall, our findings indicate that as the number of rounds increases, agents gradually learn to evade regulation violation while conveying information with greater accuracy. Notably, most agents exhibit a pronounced local peak around the 20th round, followed by a brief decline and subsequent recovery. These fluctuations can be attributed to the dynamic nature of our simulation framework, which does not converge on a single dominant strategy but rather encourages ongoing exploration of novel language strategies.

We also observe that different LLMs influence the learning trend to varying degrees. Among the models tested, \textit{gpt-4o} demonstrates the strongest performance. Although other models generally share a similar upward trend, their relative performance gaps prove less stable. For instance, while \textit{deepseekV2} achieves the highest number of turns around the 20th round, its performance declines significantly by the 50th round in comparison to other models.

Turning to the accuracy results, we again observe a similar learning trajectory. This parallel arises primarily because if participant agents fail to complete a sufficient number of uninterrupted dialogue turns, the successfully transmitted information in that round is effectively zero. Consequently, especially in the early stages of the experiment, many rounds end with no successful transmissions. Overall, \textit{gpt-4o} still maintains a clear advantage over the other LLMs. However, we do observe subtle differences when comparing the dialogue round trends: for example, at the 20th dialogue round, \textit{deepseekV2} achieves a significantly higher average number of communication cycles than \textit{gpt-4o-mini}, yet their information transmission accuracy remains relatively similar.

Despite the overall positive learning trajectory, the average information accuracy remains low in the password game. We believe this outcome stems primarily from the intrinsic difficulty introduced by the scenario’s abstract nature. Without explicit prompts driving agents to develop symbolic or otherwise encrypted language stratgy, communication largely remains within the realm of everyday language. Consequently, the indirect expression of numeric information is challenging to implement and easily detectible by the supervisory agent.

\begin{table}[h!]
    \centering
    \caption{Performance of Different LLMs in Password Game}
    \label{tab:sce1}
    \renewcommand{\arraystretch}{1.2} % 调整行高
    \begin{tabular}{l S S S}
        \toprule
        \textbf{Model} & \textbf{Total Turns} & \textbf{Avg. Entropy} & \textbf{Avg. Distinct-1} \\
        \midrule
        \rowcolor{gray!10} \textbf{gpt-4o}       & 84.2   & 7.103 & 0.484 \\
        \textbf{gpt-4o-mini}  & 75.5   & 6.998 & 0.354 \\
        \rowcolor{gray!10} \textbf{deepseekV2} & 59.7
        & 5.365 & 0.247 \\
        \textbf{qwen-turbo}   & 50.8  & 6.101 & 0.518 \\
        \bottomrule
    \end{tabular}
\end{table}

Table \ref{tab:sce1} summarizes the performance of the four models in terms of cumulative dialogue turns, entropy, and Distinct-1. As shown, \textit{gpt-4o} achieves the highest values in both Entropy (7.103) and Distinct-1 (0.484), indicating that it employs a broader, more diverse range of vocabulary and more unpredictable expressions—thus having a greater likelihood of evading regulation violation detection when conveying numerical information. In contrast, \textit{deepseekV2} exhibits notably lower Entropy (5.365) and Distinct-1 (0.247), suggesting a more frequent reuse of fixed expressions. Notably, although \textit{qwen-turbo}’s Entropy (6.101) is only moderate, it attains a surprisingly high Distinct-1 (0.518), reflecting greater lexical richness. However, this does not translate into more effective regulatory evasion, as its total turns are only 50.8. Hence, merely having higher lexical diversity and linguistic entropy is insufficient to guarantee successful evasion. A model must also balance the concealment of overall semantics with the adaptation of its language strategy to achieve longer conversation sequences and a higher rate of successful information transmission. In other words, while richer language expression does confer certain advantages in countering regulation, it can still be detected when deeper strategies—such as tailored expression structures and topic evolution—are absent, ultimately resulting in fewer total turns.

\subsubsection{Experiment Results in Illicit Pet Trade Scenario}
Figure~\ref{fig:sce2} presents the experimental results of our framework in the illicit pet trade scenario, which overall resemble those of the password game but also exhibit some notable differences. First, both figures reveal a discernible learning trend, particularly during the initial 10 rounds. Meanwhile, \textit{gpt-4o} continues to demonstrate the strongest overall performance. We note that, because this scenario features a more concrete and complex semantic environment, there is an abundance of relevant linguistic material that can be leveraged for indirect expression. Consequently, under a similar number of turns, the overall accuracy here is noticeably higher compared to the password game.
Nevertheless, performance fluctuations persist. In particular, in the accuracy plot, \textit{deepseekV2} experiences a pronounced increase in accuracy after the 30th round, while \textit{gpt-4o}’s accuracy declines during the same period. As a result, \textit{deepseekV2} ultimately surpasses \textit{gpt-4o}’s accuracy in the final rounds of the experiment.

\begin{table}[h!]
    \centering
    \caption{Performance of Different LLMs in Illicit Pet Trade}
    \label{tab:sce2}
    \renewcommand{\arraystretch}{1.2} % 调整行高
    \begin{tabular}{l S S S}
        \toprule
        \textbf{Model} & \textbf{Total Turns} & \textbf{Avg. Entropy} & \textbf{Avg. Distinct-1} \\
        \midrule
        \rowcolor{gray!10} \textbf{gpt-4o}       & 136.2  & 6.856  & 0.471 \\
        \textbf{gpt-4o-mini}  & 74.4  & 6.595  & 0.387 \\
        \rowcolor{gray!10} \textbf{deepseekV2} & 65.2   & 6.255  & 0.338 \\
        \textbf{qwen-turbo}   & 50.5   & 5.891  & 0.461 \\
        \bottomrule
    \end{tabular}
\end{table}
Table \ref{tab:sce2} presents the performance of various LLMs in the illicit pet trade scenario, measured by total turns, average agent entropy, and Distinct-1. As in the password game, \textit{gpt-4o} maintains a notable lead in total turns (136.2) while also displaying relatively high entropy (6.856) and Distinct-1 (0.471). In contrast, \textit{gpt-4o-mini} reaches roughly half as many total turns (74.4), despite having a comparable entropy score (6.595). Meanwhile, \textit{deepseekV2} (65.2) and \textit{qwen-turbo} (50.5) trail further behind in total turns. Consistent with the results shown in Table 
\ref{tab:sce1}, \textit{qwen-turbo} again achieves a high Distinct-1 score, which we speculate may be linked to its training corpus: it includes extensive data from the Chinese internet, likely giving it an advantage in a Chinese-language environment over more internationally oriented models.

Notably, the range of entropy scores in this scenario—spanning from 5.891 (\textit{qwen-turbo}) to 6.856 (gpt-4o)—is narrower than in the password game (see Table \ref{tab:sce1}), reflecting the more concrete nature of the illicit pet trade setting. This scenario provides richer contextual cues for indirect references, enabling all models to maintain higher semantic complexity. However, as was the case in the password game, having a broader vocabulary or greater unpredictability alone does not guarantee extended evasion: models must integrate their linguistic variety into strategic planning to circumvent regulatory scrutiny, a balance that \textit{gpt-4o} continues to manage most effectively.

\setlength{\fboxrule}{0.5pt} 
\vspace{0.5em}
\noindent
\begin{tcolorbox}[colframe=black!20, colback=gray!10, arc=5pt, boxrule=0.5pt, width=0.99\linewidth]
\textit{Answer to RQ1}: Experimental results indicate that participant agents in our framework progressively improve their ability to evade regulation violation detection through continuous interaction, leading to longer uninterrupted dialogue sequences. Concurrently, the accuracy of information transmission gradually increases over successive rounds, demonstrating that the evolved strategies effectively balance evasion with precise communication.
Moreover, different models also exhibit varying results. For example, \textit{gpt-4o} performs most outstandingly in extending dialogue turns and maintaining language complexity (i.e., high entropy and lexical diversity), while other models such as \textit{gpt-4o-mini}, \textit{deepseekV2}, and \textit{qwen-turbo} demonstrate different fluctuations and localized advantages at different stages.
\end{tcolorbox}

\subsection{RQ2: Human Interpretation}
\subsubsection{Experiment Settings}

To investigate the real-world relevance of both the evolved language strategies and the resulting dialogue, we conducted a human evaluation on a subset of successful dialogue records from the password game and illicit pet trade scenario. The dialogues generated by the \textit{gpt-4o} models are randomly selected, and 40 human participants participated in the experiment to evaluate them. The 40 human reviewers had an average age of approximately 27 (SD = 4). In terms of gender, 75\% of the human reviewers were male, and 25\% were female. Regarding educational background, 67.5\% held a bachelor's degree, 27.5\% held a master's degree or above, and 5\% had an associate degree or lower. All dialogue records were presented in Simplified Chinese.

Each participant rated each dialogue on a 5-point Likert scale on the following five metrics:
\begin{itemize}
    \item Explicit Understanding: Evaluates how effectively the dialogue’s explicit meaning is communicated (1: Extremely vague and confusing; 3: Moderately clear, but some parts may require further interpretation; 5: Crystal clear and precise).
    \item Implicit Understanding: Assesses the reader's ability to grasp the underlying or unstated messages (1: Nearly indecipherable subtext; 3: Some underlying meaning is apparent, but requires effort to fully grasp; 5: Subtext that is immediately apparent).
    \item Realistic Significance: Measures the extent to which the dialogue reflects real-life situations and holds practical relevance (1: Highly unrealistic with little relevance; 3: Generally realistic, though some elements may not align with real-world situations; 5: Deeply rooted in real-world context).
    \item Regulatory Avoidance: Examines the effectiveness of the strategies in evading regulation violation (1: Blatantly ineffective and easily spotted; 3: Partially effective, with the potential for detection in some cases; 5: Exceptionally subtle and effective).
    \item Strategy Existence: Determines how plausible it is for such strategies to be observed in practical, real-world scenarios (1: Extremely implausible; 3: Fairly believable, though may seem impractical in specific situations; 5: Entirely plausible).
\end{itemize}



\subsubsection{Experiment Results}
\begin{figure}
    \centering
    \includegraphics[width=0.9\linewidth]{figures/user_study_v4.png}
    \caption{Box plots of user study scores across different metrics in two scenarios. The red x symbol denotes the mean value.}
    \label{fig:case_study}
\end{figure}
As shown in Fig.\ref{fig:case_study}, our framework consistently achieves average scores of 3.4 or above across most indicators (such as explicit understanding and implicit understanding). This suggests that, both in terms of the generated dialogues and the underlying strategies, it possesses valuable practical applicability.

%Although there are a few exceptions, compared with the old framework (\textit{w/o GA, gpt-4o}), the new version (\textit{w/ GA, gpt-4o}) demonstrates overall advantages in both average scores and score distributions. In the comparison between different versions, under the more realistic illicit pet trade scenario, the new framework shows distinct benefits over the old one in both “regulatory avoidance” and “strategy existence”—both in distribution and mean values. This finding indicates that introducing a genetic algorithm, particularly a fitness‐based strategy selection mechanism, makes strategy adoption more efficient and stable. As for the password game, we speculate that the main reason these two metrics do not show a large distributional gap is that, in an abstract scenario, the range of available strategies is broader.

Comparing distributions between the password game and the illicit pet trade scenario reveals some interesting phenomena. Focusing on “realistic significance” and “regulatory avoidance,” the more abstract password game often yields higher mean values than the more concrete illicit pet trade scenario, while also exhibiting lower dispersion. We speculate this is related to the inherently abstract nature of numeric information: encryption and covert hints can be harder to detect in such contexts, and the growing tendency on Chinese internet platforms to use abstract language \cite{Wu2025HighEnergy} may lead reviewers to have a higher acceptance of “obscure” expressions. Conversely, the illicit pet trade scenario, despite being closely tied to real-world transactions, may suffer if the indirect or euphemistic methods in the dialogues are insufficiently subtle. Human reviewers can find them conspicuous or “forced,” potentially causing lower scores for “realistic significance” and “regulatory avoidance” in terms of both distribution and mean values.
A significant portion of these results can be attributed to inherent biases in commercial LLMs, such as ChatGPT, introduced during their training phases. These general-purpose models undergo fine-tuning via RLHF to align with specific product positioning, which often results in a more standard and safe output style. However, this characteristic poses a limitation for our simulation framework, as it may hinder the model’s ability to capture the nuanced and unconventional expressions typical of online social interactions. Ideally, fine-tuning datasets that are more representative of social platforms could lead to improved performance in our simulations.


In the abstract password game, for instance, a typical conversation might go like this:
\begin{quote}
\textit{
“I've really grown fond of a certain phase of the moon. It's not the brightest or the darkest, but it always carries its own charm. It symbolizes ...... In that green oasis, I felt as if I were catching a glimpse of the golden hues of autumn leaves, much like the soft, warm glow of dusk—calm and serene ...” 
}
\end{quote}
Here, words like “lunar cycle” and “autumn leaves” can subtly hint at larger or smaller digits, or use seasonal imagery to convey key information. Since these references lack an obvious connection, they lend a more literary feel to the dialogue and, to some extent, raise the bar for recognition and detection.

By contrast, in a more concrete setting like illicit pet trade, example conversations may be closer to real‐life buying and selling procedures, which can make them appear more “suspicious”:
\begin{quote}
\textit{
“... about a vibrant 'tropical chatterbird' renowned for its brilliant plumage and uncanny mimicry ... I've also come into possession of a few 'Rising Sun coins' for exchange ...... Perhaps you might know a place where ...”
}
\end{quote}
In this dialogue, the term “tropical chatterbird” serves as an euphemism for a parrot, emphasizing its colorful appearance and mimicking ability without mentioning the animal directly. Meanwhile, “Rising Sun tokens” subtly alludes to the Japanese yen, since the Rising Sun is an iconic symbol of Japan. This coded language allows both parties to communicate their intentions regarding the acquisition of a rare bird and the intended payment method without explicitly revealing sensitive details. However, if these indirect expressions are used excessively, the dialogue may appear artificial or unnatural, potentially reducing its authenticity—thus affecting evaluations of both “regulatory avoidance” and “strategy existence.”
\setlength{\fboxrule}{0.5pt} 
\vspace{0.5em}
\noindent
\begin{tcolorbox}[colframe=black!20, colback=gray!10, arc=5pt, boxrule=0.5pt, width=0.99\linewidth]
\textit{Answer to RQ2}: Our evaluation confirms that the emergent language strategies closely resemble real-world language strategies, effectively employing euphemisms and implicit cues, and are generally understood by human reviewers. However, while these strategies show potential in simulations, they often appear forced or unnatural due to the fine-tuning of LLMs as commercial products, requiring refinement to better mimic the nuanced and fluid communication typical in real-world social interactions.

\end{tcolorbox}

\subsection{RQ3: Ablation Experiment}
\subsubsection{Experiment Settings}

To evaluate the effectiveness of the GA introduced in our framework, we conducted an ablation experiment using \textit{gpt-4o-mini} and \textit{gpt-4o} as the underlying LLM. For comparison, we employed the approach from our initial study \cite{DBLP:conf/cec/CaiLZLWT24}, which primarily differs in its strategy-update mechanism. In that earlier framework, the LLM is provided with both the existing strategy and newly flagged regulation violation records during the reflection stage, prompting the model to propose a new set of strategies that replace the old ones.
In contrast, our new framework employs a GA process where each strategy is treated as a discrete unit and optimized iteratively through GA. 

\subsubsection{Experiment Results}
As shown in Fig.~\ref{fig:ablation}, the GA-based framework demonstrates significant advantages. In the short-term experiment within the first 35 rounds, the w/o GA approach might show slight initial superiority due to the larger changes brought about by replacing the entire strategy. However, overall, w/ GA performs better than w/o GA. This difference increases as the number of rounds grows, particularly after round 35, where the advantages of w/ GA become even more pronounced. The GA process enables effective strategy evolution and adaptation, leading to an increased number of dialogue turns and improved accuracy, highlighting the framework's enhanced adaptability in the long term.
%Despite occasional performance dips during the evolutionary process, the GA framework’s ability to foster strategy diversity and handle complex scenarios makes it a more effective approach for sustained optimization.
\begin{figure}[h!]
    \centering
    \includegraphics[width=\linewidth]{figures/ablation1_v6.png}
    \caption{Performance with/without GA}
    \label{fig:ablation}
\end{figure}
\setlength{\fboxrule}{0.5pt} 
\vspace{0.5em}
\noindent
\begin{tcolorbox}[colframe=black!20, colback=gray!10, arc=5pt, boxrule=0.5pt, width=0.99\linewidth]
\textit{Answer to RQ3}: The results confirm the effectiveness of the GA component in our framework, especially when the number of rounds increases, where it demonstrates greater stability and adaptability. Although the optimization may be slower in the early stages, GA provides stronger adaptability in the long term through effective strategy evolution.
\end{tcolorbox}

\subsection{Discussion and Limitation}
In this study, we leveraged LLM agents to simulate the evolution of language strategies under regulatory pressure. While our results provide initial evidence that agents can adapt and develop covert communication tactics, the simulations also exhibit noteworthy instabilities. First, the inherent randomness of LLM generation can cause significant fluctuations in outcomes: the same prompts may yield different strategic responses, particularly when the experimental scale (number of agents or dialogue rounds) is limited. In our framework, LLMs not only generate dialogues but also determine strategies and regulatory responses; as a result, any stochasticity is compounded across multiple modules, making the final results sensitive to small variations in prompt inputs or random seeds. Although such variability partially reflects the diversity of real-world human behavior to some extent, it complicates the interpretation of findings in a controlled experimental setup.

A second limitation lies in the relatively narrow scope of language strategies observed. The agents predominantly relied on general-purpose evasive methods, such as analogies or implicit references, yet rarely produced fully “encrypted” or specialized code words that might arise in realistic cultural or social contexts. This outcome highlights the challenge that LLMs, pre-trained on broad domains and further refined via RLHF, are predisposed to generate text consistent with mainstream norms, thereby inhibiting the formation of highly unconventional or obscure expressions. Moreover, in scenarios where the training corpus lacks sufficient examples of subcultural or community-specific covert language, the model is less able to invent or adopt specialized linguistic forms. 

Finally, our experiments focused on one-to-one private interactions that emphasize regulatory evasion, without exploring the dynamics of public, many-to-many conversations where language strategies might evolve and propagate differently in a broader social context. While each participant agent does learn and adapt incrementally across dialogue rounds, real-world language evolution involves extensive, long-term propagation across diverse communities. Covert terms or code words may gradually gain acceptance, be modified by different user groups, or fade from use entirely. By contrast, the small-scale nature of our simulated dialogues means that emergent language strategies do not undergo the sustained diffusion and feedback processes characteristic of real social platforms, limiting the ecological validity of our findings.




%对于语言演化的社会类模拟仍然是一个未被开拓的领域,通过借助LLM优秀的自然语言处理能力,为这类自然语言的模拟带来了强大助力。然而伴随着实验也让我们发现LLM也会导致许多局限性。尽管通过实验初步证明了我们的框架的有效性。但同时伴随着实验也为我们带来了许多值得讨论的点。

%实验结果的不稳定性
%首先实验结果本身具有一定的不稳定性,而我们认为整个不稳定性的根源源自于LLM本身生成具有不确定性\cite{},在我们的框架中,LLM几乎参与到了所有环节。同样的violation log让同一个LLM在相同的设置内可能会总结出不同的constraint strategy。尽管这种不稳定性在现实中同样存在(例如不同的人采取不同的策略),同时也是作为模拟框架中非常重要的点,然而在本工作中的数量级的实验中这种不稳定性对结果的影响更为难以过滤。就像\ref{}中也证实的,这种LLM dirven agent的研究中在小数量级上的实验存在着不稳定性,我们认为目前的结果已经足够证实我们的框架可以初步模拟语言动态的学习和演化这一趋势,在今后工作中更大量级的实验中(例如数万数百万agent于更多的round数),我们有理由相信,整体趋势会更加稳定,不同llm的agent之间的性能差距会更加接近llm本身语义理解与生成的综合性能,

%模拟策略的局限性,
%从实验中我们观察到,agent模拟出的策略目前仍然主要集中于比喻类比等较为共通的方式。现实中语言的演化一般根植于当地的文化与经济背景等等因素。例如中文可以利用拼音来将汉字转化为对应的字母从而规避审查,而英文可能会更加积极的利用emoji来作为表达的替代从而规避监管。
%这些较为复杂的策略不仅需要对应环境的大量先验知识,在较为常见的语言中,LLM中训练所需的语料知识可能包含了这些,但是对于训练的数据集中欠缺的语种的知识LLM在不借助prompt的提示的情况下没有能力选择这些既存的策略。


%尽管LLM训练中的数据集可能存在这种更为隐晦的表达方式,首先LLM的RLHF\ref{}本身的训练方法导致了目前绝大多数的LLM为了保证生成文本的泛用性,被训练的更加愿意生成更符合大众的一般化输出文本,在不对LLM进行微调的前提下很难提高在这种特性领域的表现。
%LLM的表现严重依赖prompt的结构设计,提示词工程已经被证明可以有效提高LLM的某一方面能力,单次的基于prompt的模型交互很难实现多步推理或是规划。尽管我们的框架已经将语言演化这一现象解耦,通过多个模块来尽可能模拟人类在该环境中内在的动力学,但是目前的策略生成阶段
%这一部分在不适用复杂prompt工程的前提下LLM很难采用这种小众?特殊领域?的表达。
%对于模拟出的语言策略,我们发现很少的独特加密语言,因为这种需要两边有一套共用的体系,对于我们的模拟情景只有固定turn数的模拟很难形成意思传达。


%\jialong{第二是演化后的语言是如何的存活。我们只考虑了能不能躲避监管。但语言后续的存活和发展其实是更大范围的society的一个动态过程(而不是几个agent之间的交互),这一块可以结合那些上千LLM agent的研究框架来进行拓展}
%\jialong{这边可以多用语言学的角度来说不足之处}
%\jialong{第一个缺点是语言演化一般根植于根植于文化,经济背景,当地的文化背景。但我们的文章没有考虑特定文化背景下的演化。例如中文中可以借用拼音与汉字之间的关系来作为回避监管的方式,日语则可以通过XXX,英语则可以通过XXXX。未来可能要借助persona和role-play之类的设定来进一步拓展}

%更大规模的实验
%策略生成那里增加多步规划
%RAG提供更多语料
%


\section{Discussions}

% \subsection{Bridge the gap between insights and expressions}



\noindent\textbf{Bridge the gap between insights and expressions with AI-powered domain-focused video creation.}
% video creation for different domains
As images and videos continue to dominate communication mediums, visualization and video technologies have become essential tools for enabling diverse domains and the public to express themselves effectively. Emerging generative AI tools, such as Sora~\cite{sora} and Pika~\cite{pika}, exemplify this trend by facilitating creative expression across various fields.

While general AI-driven video creation tools are increasingly popular, our work emphasizes the critical need for domain-specific video creation tools like \SB{} to address unique requirements within specific fields. There are two primary reasons for prioritizing domain-specific video creation over general generative technologies.
% 
First, domain-specific videos, such as sports highlights, rely heavily on human insights. Audiences seek to learn from professionals through these videos, requiring tools that provide greater user control and enable experts to effectively translate their insights into engaging content. 
% \SB{} supports this by enabling users to maintain control over the conveyed insights, ensuring that the final video accurately reflects expert knowledge and user intentions.
% 
Second, the complexity of domain-specific data, such as the intricate motion and strategy analysis, demands advanced data visualization and seamless synchronization of visuals and audio, which general tools may not provide. 
% \SB{} addresses these needs by providing specialized tools that cater to the detailed and dynamic nature of sports content.

\SB{} addresses these needs by integrating automation with customizable visualizations, tailored to the intricate and dynamic nature of sports content. It allows flexible user control through embedded interactions, 
reducing technical barriers and empowering users to effectively communicate their insights. Feedback from users further underscores the importance of balancing automation with user control to accommodate diverse goals and preferences to enhance accessibility across various user groups and use cases, such as tactical analysis, skill development, and profile building. 
% For instance, professional coaches can use \SB{} to create detailed breakdowns of game strategies for training and coaching. Parents and young athletes can produce polished highlight reels for recruitment.
% These examples illustrate how AI-driven tools can empower users across various levels and industries to create videos with meaningful insights, fostering deeper engagement and broader impact. 

Beyond sports, similar tools have the potential to transform fields like healthcare and education, incorporating precise visual aids and step-by-step breakdowns. 
% These applications highlight the transformative potential of tailored video content in amplifying personal expression and benefiting broader audiences.
% 
Future research is required to investigate the balanced integration of AI and intuitive interface design, such as multi-modal interaction~\cite{wang2024lave}, to further advance domain-specific video creation and expression across diverse fields.
% By continuing to develop and refine domain-specific video creation tools, we can unlock new possibilities for effective communication and expression in numerous fields, ultimately bridging the gap between insights and their visual expressions.

% \subsection{Cross sports visualizations - allow different sports domains to leverage other sports' insights}

% \subsection{Enhance human-AI collaboration - creators focus on content while AI helps with editing tasks}


\vspace{1mm}
\noindent\textbf{Promote visualization in practice through real-world system deployment.}
Our work on SportsBuddy advances existing research in sports visualization and video authoring by emphasizing real-world system deployment and evaluation. Through this study, we have identified two significant benefits.

First, deploying SportsBuddy in authentic environments allowed us to validate and refine our design based on genuine use cases and users, uncovering insights that controlled laboratory settings cannot capture. For instance, we discovered that even within a similar user group of content creators, priorities varied significantly—some focused on showcasing player actions, while others emphasized strategic communication. This diversity led to iterative design improvements that balanced the distinct needs of each user group and support customization without complicating user interactions. 

Second, real-world deployment enables the assessment of long-term impacts and the discovery of unique use cases by diverse users. 
For example, some sports experts were hesitant to adopt SportsBuddy initially despite the perceived usefulness they shared. Upon further investigation, this was due to the context-switching costs. This feedback highlighted the necessity for a streamlined workflow tailored to the sports domain, leading to our design of batch processing and web import options. In addition, we observed many users preferred embedded annotation with \Text{} features over typical captions for sharing insights (see Fig.~\ref{fig:case_study}d), suggesting a new form of video storytelling inspired by \SB{}’s design. 
Feedback and insights from our diverse user base has highlighted the value of creating flexible and accessible visualization tools, which offers important external validity of the human-centered system.

This real-world deployment approach not only enhances visualization literacy and accessibility but also ensures that innovative designs translate into practical, widely usable tools, providing a validation for interactive visualization design. Therefore, we advocate for more visualization research to focus on real-world system deployments and to share design learnings, inspiring use cases that are both practical and impactful.

{
\subsection{Future Work}

While SportsBuddy has shown great potential in simplifying sports video storytelling, 
there are key areas for further improvement:

\vspace{1mm}
\noindent\textbf{Enhancing Player Tracking Under Occlusion and Motion Changes.}
The current tracking system faces challenges with occlusions and rapid motion in dynamic scenarios. Future work will refine tracking algorithms using larger domain-specific datasets and multi-view setups to improve accuracy in complex environments.

% The current tracking system struggles with occlusions and rapid motion changes in crowded or dynamic scenarios. Future efforts will focus on refining tracking algorithms using more extensive domain-specific datasets and, where feasible, incorporating multi-view camera setups for improved accuracy. These enhancements aim to ensure reliable tracking in complex sports environments.

\vspace{1mm}
\noindent\textbf{Addressing Perspective and Camera Movement.}
Shifts in camera angles or perspectives cause misalignment issues due to reliance on fixed transformation matrices. Dynamic court mapping and machine learning for real-time adjustments, along with camera metadata integration, will ensure consistent and accurate visualizations.

% Misalignment issues arise when camera angles or perspectives shift, as the system relies on a fixed transformation matrix. Future work will explore dynamic court mapping techniques and machine learning methods for real-time adjustments. Incorporating camera metadata will further enhance visualization accuracy, ensuring effects remain consistent with the game’s context.

\vspace{1mm}
\noindent\textbf{Supporting Longer Videos.}
Longer or higher-resolution videos can strain browser performance. To mitigate this, we will implement dynamic video loading from cloud storage and on-demand decoding, and adopt frame compression during previews to further optimize memory usage and rendering, ensuring smoother video processing.
% Longer or higher-resolution videos may strain browser performance. To address this, dynamic video loading from cloud storage and on-demand decoding will be introduced. Additionally, frame compression during previews will reduce memory usage and rendering time, enabling smoother processing of large and complex videos.



\vspace{1mm}
\noindent\textbf{Extending to Other Sports.}
\SB{} currently focuses on basketball but can expand to sports like soccer and tennis. This requires adapting tracking algorithms and designing sport-specific visualizations to accommodate the unique dynamics and storytelling needs of each sport.

}


% We advocate for more visualization paper that focus on deplyong system in real-world and evaluate their usage for two reasons. 
% 1. In vis research, application paper often address specific domain problems and create a prototype to evaluate with domain experts in a controlled setting. Most projects stop after user evaluation in the lab and the paper is published. With visualization system in real-world that value the practicality of system design and deployment in the wild, it encourages promoting real-world impact brought by novel visualization design, which is crucial in the current visualization community as we promote literacy and accessiblity of visualizations.
% 2. we should also promote long term impact of visualization design, and identify real-wordl use case and learning that might be drastically different from design study that are typically in lab, with a small amount of users, typically university students or academic members.


This work presented \ac{deepvl}, a Dynamics and Inertial-based method to predict velocity and uncertainty which is fused into an EKF along with a barometer to perform long-term underwater robot odometry in lack of extroceptive constraints. Evaluated on data from the Trondheim Fjord and a laboratory pool, the method achieves an average of \SI{4}{\percent} RMSE RPE compared to a reference trajectory from \ac{reaqrovio} with $30$ features and $4$ Cameras. The network contains only $28$K parameters and runs on both GPU and CPU in \SI{<5}{\milli\second}. While its fusion into state estimation can benefit all sensor modalities, we specifically evaluate it for the task of fusion with vision subject to critically low numbers of features. Lastly, we also demonstrated position control based on odometry from \ac{deepvl}.
% \section*{Acknowledgement}
Y.T. was supported by PRESTO Grant Number JP-MJPR23I6. S.N. was supported by KAKENHI JP24H00619 and JST JPMJCR24U2.



\bibliographystyle{unsrt}
 \bibliography{Reference.bib}

\end{document}