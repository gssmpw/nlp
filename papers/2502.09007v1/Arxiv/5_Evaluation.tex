



\section{Evaluation}
\label{evaluation}


\subsection{Experimental Setup}
\subsubsection{Benchmark}
To evaluate the energy efficiency of our \sysname framework, we use four Transformer models: BERT-Base, BERT-Large, T5-Base, and T5-Large. We utilize MRPC, SQuAD2.0, and WikiText-103 for datasets, each with an input token length of 1024. The data types of activation and weight data are configured to INT8. To evaluate our proposed sense amplifier power gating, we use bit-wise sparsity of activation for each layer.

\begin{figure}[t]
\centering
\includegraphics[width=0.48\textwidth]{Arxiv/Figure/Eval_Energy_Efficiency_2.pdf}
\caption{The \sysname Framework’s Normalized Energy Efficiency with Different Matrix Configurations for (a) Bit-serial Processing Unit and (b) Bit-parallel Processing Unit}
\vspace{-0.1in}
\label{figure_energy_efficiency_2}
\end{figure}


\subsubsection{Hardware Configuration}
To validate reconfigurable eDRAM, we measure the power consumption across six eDRAM subarray configurations: 32$\times$128, 32$\times$256, 32$\times$512, 64$\times$128, 64$\times$256, and 64$\times$512. We evaluate our framework with a 60KB unified buffer and 16KB PIM macro. The PIM macro features four banks, each configured with a number of subarrays to achieve a 4KB capacity per bank. We use a weight stationary data flow in our architecture, processing inputs in a bit-serial manner. For Weight, we evaluate two methods: bit-serial and bit-parallel. The processing unit consists of an adder tree and an accumulator for MAC operation.

\subsubsection{Simulation Methodology}

We develop C++ based in-house simulator for the scheduling phase of \sysname, incorporating the power consumption of retention-aware scheduler, PIM macro controller, eDRAM access, and processing unit. It estimates overall energy consumption and energy efficiency. The energy modeling of the hardware template are from post-layout simulation in HSPICE for eDRAM memory, and from Synopsys Design Compiler for scheduler, controller, and processing unit with Samsung’s 28nm technology. The hardware operates under conditions of a 200MHz frequency, 1V voltage, and room temperature. We simulate our reconfigurable eDRAM across four different VPDs: 200mV, 300mV, 400mV, and 500mV. 


Since this is the first framework that optimizes the energy consumption of eDRAM macro as a digital PIM device to the best of our knowledge, we do not compare it against previous analog circuit-based eDRAM PIMs. Instead, we compare the \sysname framework with two baselines: an eDRAM baseline and Neural Cache ~\cite{eckert2018neural}. Firstly, we use an eDRAM baseline operating under fixed memory operation without any of the proposed optimization schemes. We configure the baseline to maintain the highest RBL voltage swing, aiming to minimize the refresh overhead as normal memory design usually adopts. Secondly, we compare our framework against Neural Cache, prior SRAM-based PIM, because previous digital PIM macros usually adopt SRAM as memory. We scale Neural Cache to the same operational conditions for consistency.

\begin{figure}[t]
\centering
\includegraphics[width=0.48\textwidth]{Arxiv/Figure/Eval_Breakdown.pdf}
\caption{Power Consumption Breakdown and Reduction}
\vspace{-0.1in}
\label{figure_breakdown}
\end{figure}



\subsection{Energy Efficiency of \sysname}
\textbf{Comparison with various transformer models}
Figure ~\ref{figure_energy_efficiency_1} shows the normalized energy efficiency with four Transformer models and three benchmarks for six memory configurations, using a processing unit with (a) bit-serial operators and (b) bit-parallel operators. Energy efficiency values are normalized with the minimum value of the eDRAM baseline. The graph shows the range from maximum to minimum normalized energy efficiency of all possible operating conditions configured by the \sysname framework. Notably, the framework consistently outperforms both the eDRAM baseline and Neural Cache in the best case identified through retention-aware scheduling. These performance improvements come from \sysname’s capability to identify the optimal scheme through retention-aware scheduling and control memory operation optimally for the actual use case. Furthermore, mitigating the increased overhead from the sense amplifier and refresh fully optimizes the energy consumption of eDRAM macro, ensuring peak energy efficiency. There are some cases where the worst-case energy efficiency falls below that of Neural Cache despite the lower access power of eDRAM compared to SRAM. This is because suboptimal memory control and tiling schemes lead to increased refresh overhead, underscoring the importance of memory control and scheduling in adopting eDRAM as a PIM device. With the bit-serial processing unit, the \sysname framework achieves 1.59$\times$-2.66$\times$, 1.52$\times$-3.05$\times$, and 2.87$\times$-8.16$\times$ higher energy efficiency than worst point, Neural Cache, and eDRAM baseline, respectively, at the optimal point. The bit-parallel processing unit demonstrates performance enhancements similar to those of the bit-serial processing unit, as both processing units with the same throughput exhibit similar operating power.

\begin{figure}[t]
\centering
\includegraphics[width=0.48\textwidth]{Arxiv/Figure/Eval_Optimization.pdf}
\caption{Energy Reduction per Scheme}
\vspace{-0.1in}
\label{figure_optimization}
\end{figure}

\textbf{Comparison with different matrix configurations}
Figure ~\ref{figure_energy_efficiency_2} illustrates the impact of varying input and weight matrix shapes on the energy efficiency across six memory configurations with the BERT-Base and MRPC dataset. While changing the matrix shape influences the lifetime of the tile and thereby affects tiling schemes, our framework adeptly identifies the most energy-efficient operating conditions. Regardless of the workload, \sysname consistently finds the optimal tiling scheme and memory operation by utilizing the SW configuration as input and conducting energy modeling before runtime. With the bit-serial processing unit, our framework achieves energy efficiency improvements ranging from 1.63$\times$ to 2.93$\times$, 1.52$\times$ to 2.95$\times$, and 2.87$\times$ to 5.85$\times$ compared to the worst point, Neural Cache, and eDRAM baseline, respectively. Similarly, with the bit-parallel processing unit, the framework achieves 1.62$\times$-2.78$\times$, 1.51$\times$-2.76$\times$, and 2.84$\times$-5.38$\times$ higher energy efficiency than worst point, Neural Cache, and eDRAM baseline, respectively.


\subsection{Power Consumption Comparison}

Figure ~\ref{figure_breakdown} shows the power consumption of memory access and its breakdown for the five different VPDs with the six memory configurations, respectively. The data shows that higher VPD reduces power consumption of the RWL/RBL voltage swing and the pull-down driver, resulting in a lower power consumption. However, a higher VPD level results in increased power consumption by the sense amplifier. For example, the power consumption of memory access reduces by 71.31\% when VPD elevates from 0 to 500mV for eDRAM macro with a cell array size of 32$\times$512. However, this adjustment causes the sense amplifier’s portion in total power consumption to increase from 6.82\% to 50.14\%. This shift indicates the need to optimize the sense amplifier’s power consumption to reduce memory access power further.



\subsection{Energy Consumption Optimization}
Figure ~\ref{figure_optimization} shows changes of the eDRAM macro energy consumption normalized to the eDRAM baseline by gradually adding each of the proposed schemes. Each indicates reconfigurable eDRAM, sense amplifier power gating, and retention-aware scheduling with refresh skipping, respectively. As shown in the graph, since memory access energy consumption is dominant in PIM, we can optimize energy consumption by 54.04\% on average by reducing memory access power with reconfigurable eDRAM. Furthermore, addressing the increased power consumption of the sense amplifier with power gating reduces the energy consumption by 58.00\% on average.

Although the reconfigurable eDRAM can reduce memory access power by up to 71.31\%, the energy optimization only by the eDRAM is 66.57\%. This is because the refresh overhead is far more increased due to the shortened retention time. The scheduling phase of the \sysname framework enables the best use of our proposed hardware design by identifying the optimal tiling scheme and memory operation through energy modeling. Employing this retention-aware scheduling, the \sysname framework effectively optimizes the energy consumption of eDRAM macro compared to the baseline by 56.43\%, 66.95\%, 74.88\%, 52.97\%, 65.76\%, and 74.28\% in each memory configurations.

\begin{figure}[t]
\centering
\includegraphics[width=0.48\textwidth]{Arxiv/Figure/Eval_Area_Breakdown.pdf}
\caption{Area Breakdown of the \sysname Framework}
\label{figure_area_breakdown}
\end{figure}

\subsection{Area and Energy Breakdown}
The \sysname framework's area breakdown is depicted in Figure ~\ref{figure_area_breakdown}. The graph shows the breakdown when using bit-serial and bit-parallel operators as processing units for each of the six memory configurations. As the graph shows, for both types of processing units, the proposed retention-aware scheduler and PIM macro controller take up less than 2.5\% and 1\% of the total area, respectively, that is marginal. We can see that most of the area overhead comes from the unified buffer (around 70\%) and bank (around 20\%). The proposed reconfigurable driver takes up 13.7\%, 8.41\%, 4.74\%, 17.24\%, 10.58\%, and 5.97\% of the subarray area for the six configurations, respectively.

Figure ~\ref{figure_energy_breakdown} shows the energy breakdown for the same experimental conditions. The data shows that the retention-aware scheduler and PIM macro controller both account for less than 1\% energy consumption in the overall operation, regardless of the type of processing unit. The PIM macro, which performs the actual computation, accounts for most of the energy consumption, with the bank accounting for about 60\% and the processing unit accounting for about 20\%.

\begin{figure}[t]
\centering
\includegraphics[width=0.48\textwidth]{Arxiv/Figure/Eval_Energy_Breakdown.pdf}
\caption{Energy Breakdown of the \sysname Framework}
\vspace{-0.1in}
\label{figure_energy_breakdown}
\end{figure}