
\begin{figure}[t]
\centering
\includegraphics[width=0.48\textwidth]{Arxiv/Figure/Framework.pdf}
\caption{Overview of The \sysname Framework}
\vspace{-0.1in}
\label{figure4}
\end{figure}


\section{\sysname Framework}
\label{framework}
We propose the \sysname framework, an energy optimization framework for eDRAM-based PIM that can fully optimize the energy consumption of the memory with retention-aware scheduling, optimal architecture, and reconfigurable circuit design.

Figure ~\ref{figure4} presents the overview of the \sysname framework. The framework receives a software configuration (e.g., model parameters, input and weight matrix shapes, data types, etc.) and a hardware configuration (e.g., memory configuration, processing units, operating frequency, etc.) as inputs and goes through two phases: \textit{the scheduling and execution phase}. These two phases collectively allow the \sysname framework to significantly reduce overall energy consumption by optimizing memory operations, which are the primary contributors to energy overhead in PIM operation. The \sysname framework can optimize energy consumption in PIM across diverse configurations. The following section delves into a detailed description of these two phases, revealing how they form the backbone of the \sysname framework.

\subsection{Retention-aware Scheduling}
\label{retention-aware scheduling}
As discussed in Section ~\ref{observations}, achieving high energy efficiency in PIM requires optimizing the energy consumption of memory access. However, simply reducing RBL voltage swing without tailoring it to a specific use case is not the most efficient approach. To address this problem, the \sysname’s scheduling phase undertakes two key tasks: 1) identifying the optimal tiling scheme and 2) selecting the optimal memory operation.

The eDRAM memory requires periodic refreshes due to its retention time. However, it is possible to skip refresh in two cases. Firstly, when the lifetime of data is shorter than the retention time, as proposed in \cite{tu2018rana}, there is no need for refresh. Secondly, refresh can be skipped for the data that is no longer used for computation. Given that PIM architecture cannot process the whole input and weight matrices simultaneously, it employs tiling for GEMM. Memory access pattern and output generation order vary with a tiling scheme, influencing the lifetime of mapped data. Therefore, to enhance the energy efficiency of eDRAM-based PIM, it is essential to schedule operations considering the lifetime and the retention time. 
Additionally, optimally balancing eDRAM's inherent trade-off by considering the actual use case can further reduce energy consumption.
%Additionally, optimally balancing eDRAM’s inherent trade-off by considering both the lifetime of the mapped data and the number of memory accesses can further reduce energy consumption.
The \sysname’s scheduling phase goes through the following four steps to determine the most optimal tiling scheme and memory operation for given input configurations.

\begin{table}[t]
\centering
\caption{\sysname's Variable Definition}
\includegraphics[width=0.48\textwidth]{Arxiv/Figure/Variable.pdf}
\label{figure8}
% \vspace{-0.3in}
\end{table}


\textbf{Step 1: Find all possible tiling schemes.} The scheduler identifies all possible tiling schemes (i.e., loop order and tile shape) based on the input configurations. Table 1 presents the variable definitions used by the \sysname's scheduling phase.  



\textbf{Step 2: Analyze lifetime of data.} In a tiled GEMM operation, the lifetime of each tile varies according to the tiling scheme used. Figure ~\ref{figure9} illustrates this change through two pseudo-code examples and their corresponding computation flows for tiled GEMM. While the overall computation time remains consistent, the lifetime of each tile differs, as the equations show. The computation cycle per tile, denoted by $T$, changes depending on the processing type (i.e., bit-serial and bit-parallel). In step 2, the lifetime of each data is analyzed across all tiling schemes identified in step 1. 



\textbf{Step 3: Estimate energy consumption.} In step 3, we perform energy modeling by using the lifetime analyzed in step 2 and the memory specifications of our proposed reconfigurable eDRAM (i.e., RBL voltage swing, power consumption, and retention time table) as input. For each tiling scheme, total energy consumption is estimated ($E_{total}$) for various RBL voltage swings as Equation (1). $E_{PIM}$ and $E_{Buffer}$ denote the energy consumption of the PIM macro operation and buffer access, respectively. The $E_{PIM}$ and $E_{Buffer}$ are calculated as in Equation (2) and (3). $E_{Acc}$, $E_{PU}$, and $E_{Ref}$ represent the energy consumption of eDRAM access, processing unit, and refresh per operation, respectively. $N$ denotes the number of operations. $T_{Life}$ and $T_{Retention}$ refer to the lifetime of mapped data and the retention time of eDRAM. The prefixes $P$ and $B$ identify the PIM macro and buffer.

\begin{scriptsize}
\begin{align}
    E_{total} = E_{PIM}+E_{Buffer}
\end{align}
%\vspace{-0.1in}
\begin{align}
    E_{PIM} = (E_{P\_Acc}+E_{PU}) \times P\_N + E_{P\_Ref} \times \left[\frac{T_{P\_Life}}{T_{P\_Retention}}\right]
\end{align}
%\vspace{-0.1in}
\begin{align}
    E_{Buffer} = (E_{B\_Acc}) \times B\_N + E_{B\_Ref} \times \left[\frac{T_{B\_Life}}{T_{B\_Retention}}\right]
\end{align}
\end{scriptsize}

\textbf{Step 4: Find optimal computation flow and memory operation.} Based on the output of the energy modeling, tiling scheme, and memory operation that minimize overall energy consumption are identified. By forwarding the results to the PIM macro controller, we fully optimize energy consumption of the memory.

\begin{figure}[t]
\centering
\includegraphics[width=0.48\textwidth]{Arxiv/Figure/Scheduling.pdf}
\caption{Lifetime of Data per Tiling Scheme Analysis}
\vspace{-0.1in}
\label{figure9}
\end{figure}


\begin{figure*}[t]
\centering
\includegraphics[width=1\textwidth]{Arxiv/Figure/Architecture.pdf}
\caption{\sysname's Hardware Template Architecture}
%\vspace{-0.2in}
\label{figure6}
\end{figure*}


\subsection{Hardware Template Architecture}

As highlighted in Section ~\ref{observations}, a memory design that does not take into account the property of an architecture and its use cases falls short of achieving high energy efficiency. To address this problem, we propose a reconfigurable eDRAM-based PIM hardware template architecture to achieve adaptability to the outputs of the \sysname's scheduling phase. Figure ~\ref{figure6} illustrates \sysname’s hardware template architecture, which consists of a retention-aware scheduler, a PIM macro controller, a PIM macro, and a unified buffer. The user can change the memory configuration (e.g., the subarray shape, the number of subarrays and banks, and the buffer size) and the processing unit configuration (e.g., operation types such as dot product/row-wise, processing type, and bit precision).

The retention-aware scheduler finds the optimal operating conditions by evaluating the energy consumption across all possible tiling schemes and memory operations for the given architecture configuration, as detailed in Section ~\ref{retention-aware scheduling}. The lifetime estimator calculates the lifetime of the input, weight, and output tile for the chosen tiling scheme. Subsequently, the energy estimator computes the energy consumption for each memory operation. The energy optimizer forwards the most energy-efficient tiling scheme and voltage select signal to the PIM macro controller. The controller modulates the memory access patterns and adjusts eDRAM operations by activating the row that stores the weight and broadcasting the input to the subarray’s sense amplifier. The input is then multiplied by the weight in the sense amplifier and passed to the processing unit. This unit executes the GEMM or GEMV operation using the output from the eDRAM subarray. Throughout this process, the PIM macro controller adjusts the eDRAM operation by tuning the pull-down voltage (VPD) and reference voltage (VREF), marked with the blue line. The following section delves into the specific hardware implementation for the reconfigurable eDRAM. A further optimization scheme to minimize redundant refresh in eDRAM is discussed below.

Reducing memory access power inevitably leads to more frequent refresh, which in turn increases energy consumption and latency overhead. To counter this, we employ the refresh skipping scheme for the data that has a shorter lifetime than the retention time or the data that is no longer utilized in computation. For example, it is possible to skip refresh if the lifetime of the partial sum is shorter than the retention time, as new data replaces the old. Input tiles and weight tiles also no longer need to be refreshed when they are no longer used in the computation. By skipping redundant refresh in these cases, the PIM macro controller effectively mitigates the increased refresh overhead, thereby enhancing overall energy efficiency.

\subsection{Reconfigurable eDRAM}
We propose a reconfigurable 2T eDRAM macro capable of adjusting the RBL voltage swing according to the pull-down voltage (VPD) controlled by the PIM macro controller. Additionally, we implement two optimizations-adjusting the reference voltage and employing sense amplifier power gating-to further enhance energy efficiency. Figure ~\ref{figure7} illustrates the detailed hardware implementation of the macro, featuring a reconfigurable pull-down driver. This driver employs a pass transistor, targeting a VPD dictated by the PIM macro controller instead of VSS. Adjusting the VPD closer to VSS increases the RBL voltage swing, which in turn enhances a sensing margin and extends a retention time. However, this adjustment concurrently increases the power consumption of the memory access. 

\begin{figure}[t]
\centering
\includegraphics[width=0.48\textwidth]{Arxiv/Figure/Macro.pdf}
\caption{Reconfigurable 2T eDRAM Circuit Design}
\vspace{-0.1in}
\label{figure7}
\end{figure}

\begin{figure*}[t]
\centering
\includegraphics[width=1\textwidth]{Arxiv/Figure/Eval_Energy_Efficiency_1.pdf}
\caption{The \sysname Framework’s Normalized Energy Efficiency Comparison with eDRAM Baseline and Neural Cache for (a) Bit-serial Processing Unit and (b) Bit-parallel Processing Unit}
%\vspace{-0.1in}
\label{figure_energy_efficiency_1}
\end{figure*}

Before runtime, the \sysname framework estimates overall energy consumption for various VPD settings. This estimation is enabled by forwarding the specification about the power consumption and the retention time relative to the VPD of our proposed eDRAM macro to the energy modeling of the scheduling phase. The \sysname framework is able to identify optimal tiling scheme and memory operation to achieve the highest energy efficiency by pre-estimating the energy consumption at each VPD level. The following describes additional optimization schemes aimed at augmenting the functionality and power efficiency of our proposed eDRAM design.

The functionality of the sense amplifier and the retention time of the memory are significantly affected by the reference voltage, as the sense amplifier utilizes this voltage to convert the RBL voltage swing into a digital signal. Thus, we opt to use the midpoint of the RBL voltage swing between data 1 and data 0 as the reference voltage to optimize the performance of the sense amplifier and the retention time. This reference voltage is pre-calculated for each level of VPD, enabling the PIM macro controller to select the reference voltage corresponding to the VPD.

We employ the sense amplifier power gating to mitigate the trade-off between the RBL voltage swing and the power consumption of the sense amplifier, which also effectively substitutes the function of a 1-bit multiplier. As the VPD is set higher, the power consumption of the sense amplifier increases due to a reduced sensing margin. With our design, if the input for computation with weight is zero, the sense amplifier is power gated, ensuring it outputs data 0 regardless of the data actually stored, as depicted in Figure ~\ref{figure7}. Depending on how the data is mapped to the eDRAM cell array, the input data is broadcast in different ways. For bit-parallel mapping, a shared input bit is broadcast because each weight data bit operates with the same input bit. Conversely, in bit-serial mapping, each weight bit operates with a different input bit requiring individual broadcasting. The output of the sense amplifier is the same as the output of the 1-bit multiplier, so it can be aggregated directly in the processing unit. Thus, sense amplifier power gating accomplishes two benefits: 1) it reduces both the area and power overhead of the 1-bit multiplier, and 2) it further optimizes memory access.

