\begin{figure}[t]
\centering
\includegraphics[width=0.48\textwidth]{Arxiv/Figure/Observation1.pdf}
\caption{Breakdown of Memory Energy Consumption on Different Use Cases}
%\vspace{-0.1in}
\label{figure3-1}
\end{figure}

\section{Observations}
\label{observations}


In this work, we analyze the key observations outlined below. These insights form the foundation of our proposed \sysname framework to fully optimize both the power and energy consumption of memory access, achieving high energy efficiency.

\textbf{Observation 1: dominant energy consumption of memory access in PIM.} In the von Neumann architecture, off-chip data movement causes the primary energy consumption in accessing memory. To minimize this energy consumption, the architecture aims to maximally reuse data loaded in the processor with its on-chip memory. In contrast, PIM architecture reduces data movement overhead in exchange for much more frequent memory access for operations by in-memory processing units. The key distinction lies in memory utilization, making the difference in the circuit design of memory in each architecture. Unlike conventional architectures where the energy consumption of refresh is substantial, leading to designs focused on reducing this refresh overhead to enhance performance, PIM architecture faces a different challenge.

Figure ~\ref{figure3-1} shows the ratio of refresh and energy consumption of memory access according to the number of memory accesses where the lifetime of data is set as 1000$\mu$s. The memory is assumed as eDRAM with a retention time of around 100$\mu$s. The refresh overhead is dominant when the number of memory accesses is small, as in the conventional architecture use case. However, the energy consumption of memory access becomes dominant as the number of memory increases; that is the use case of PIM architecture. The graph emphasizes the necessity of designing memory considering a property of architecture and actual use case. Additionally, it is essential to optimize the energy consumption of memory access to improve performance in PIM effectively.

%Various schemes have been proposed to increase the retention time or to make the refresh more efficient: [reference, LPDDR, SALP, ...]. 


\begin{figure}[t]
\centering
\includegraphics[width=0.48\textwidth]{Arxiv/Figure/Observation2.pdf}
%\vspace{-0.2in}
\caption{(a) eDRAM Operation with Large RBL Voltage Swing (b) eDRAM Operation with Small RBL Voltage Swing (c) Trade-off Among RBL Voltage Swing, Sense Amplifier Power, and Retention Time Analysis}
\vspace{-0.1in}
\label{figure3-2}
\end{figure}

\textbf{Observation 2: trade-off among RBL voltage swing, sense amplifier, and retention time.} eDRAM operation with large RBL voltage swing, as shown in Figure ~\ref{figure3-2} (a), enhances the sensing margin. This enhancement allows the sense amplifier to convert the analog signal into a digital signal faster. As a result, the total current flowing through the sense amplifier diminishes, leading to the reduced power consumption of the sense amplifier  (17.10\%), as shown in Figure ~\ref{figure3-2} (c). Even though the charge in a SN fluctuates due to a leakage current, a large sensing margin ensures that the sense amplifier converts the data correctly, resulting in a longer retention time (100$\mu$s). However, as explained in Section ~\ref{background}, the memory access power increases because the RBL voltage swing is the main source of power consumption. Conversely, a smaller RBL voltage swing leads to higher power consumption of the sense amplifier (45.08\%), shortened retention time (9$\mu$s), and smaller memory access power, as depicted in Figure ~\ref{figure3-2} (b) and (c).

Simply opting for a small RBL voltage swing to optimize memory access power falls short of an optimal solution due to the inherent trade-off of eDRAM. While it does lower memory access power, it rather increases the power consumption of the sense amplifier and refresh overhead. Therefore, we have to select a memory operation considering the specific PIM use case that minimizes both memory access power and overall energy consumption.
