

\section{Background}
\label{background}


%\subsection{Embedded DRAM}
Embedded DRAM (eDRAM) cell varies by the number of transistors and structure, such as 1T1C, 2T, 3T, and 4T. While 1T1C provides the highest memory density, due to its destructive read issue as described in~\cite{xie2022gain}, it is hard to utilize the multi-row activation scheme, which is commonly employed in PIM. Hence, this paper adopts a 2T eDRAM cell, which has high memory density and the capability for non-destructive read.

%The structure of an eDRAM macro and 2T eDRAM cell is shown in Figure~\ref{figure2} (a) illustrating write and read operations. 

Figure~\ref{figure2} (a) depicts the structure of an eDRAM macro and a 2T eDRAM cell, along with the write and read operations. The macro consists of a controller, a decoder, a wordline (WL) driver, a cell array, a pull-down driver, and a sense amplifier. For write operation, data is stored in the storage node (SN) by driving write wordline (WWL) to a negative voltage (-$V_{th}$) and driving the data through write bitline (WBL). In a read operation, the pull-down driver drives read wordline (RWL) to ground (VSS). This operation causes read bitline (RBL), initially precharged to VDD, to fluctuate in response to a storage node (SN) voltage. In this work, the sensing margin denotes the voltage level difference between the reference voltage and data 0, 1. The sense amplifier then converts this RBL voltage swing to a digital value by comparing it with the reference voltage. However, as shown in  Figure~\ref{figure2} (a) with the red arrow, a leakage current flows through the PMOS transistor. This current changes the charge stored in the SN, leading the RBL voltage swing to deviate from the ideal case. The retention time means the duration until the sense amplifier accurately converts the data. Because of the non-ideality, eDRAM requires periodic data refresh, which imposes large energy and latency overhead at the system level. Figure~\ref{figure2} (b) shows the power breakdown of an eDRAM macro with 32$\times$512 cell array. Due to the high capacitance of long metal wires composing the cell array, the majority of power consumption is from RWL/RBL voltage swing, pull-down driver, and sense amplifier, occupying 56.93\%, 22.14\%, and 16.87\%, respectively. It implies that optimizing these components can maximally reduce the memory access power.