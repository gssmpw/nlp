





\section{Quantitative Experiment on Edge Bundling}
\label{sec:appendixA}

\begin{table}[b]
  \setlength{\abovecaptionskip}{2mm}
  \centering
  \tabcolsep=0.75mm
  \caption{Comparison between different edge bundling algorithms, including divided edge bundling (DEB)~\cite{selassie2011divided}, multilevel agglomerative edge bundling (MAEB)~\cite{gansner2011MINGLE}, and our algorithm (RouteFlow). Lower values are better for both metrics.\looseness=-1
  }
  \label{tab:bundling_result} % Optional: adds a label for referencing the table elsewhere
  \resizebox{0.5\textwidth}{!}{
   % \scriptsize
    \begin{tabular}{c|p{7mm}cccccccccccccc}
       \toprule
        \multirow{2}{*}{\textbf{Dataset}}
        & \multicolumn{3}{c}{Deviation}
        & \multicolumn{3}{c}{Ink ratio}\\
        \cmidrule(lr){2-4} \cmidrule(lr){5-7}
        & {DEB} & {MAEB} & {RouteFlow}
        & {DEB} & {MAEB} & {RouteFlow} \\
        \midrule

        Taxi~\cite{yuan2011drive, yuan2010tdrive} &
        0.023 &	0.017 &	\textbf{0.012} &
        0.387 & \textbf{0.349} & 0.378 \\

        BirdMap~\cite{BirdMap} & 
        0.048 &	0.035 &	\textbf{0.016} &
        0.237 &	\textbf{0.235} &	0.239 \\

        Railway~\cite{railway} &
        0.038 &	0.031 &	\textbf{0.018} &
        0.243 &	0.309 &	\textbf{0.234} \\

        MEIBook~\cite{mei} & 
        0.034 &	0.018 & \textbf{0.017} &
        0.397 &	\textbf{0.327} & 0.388 \\

        OpenSkyAirline~\cite{opensky} &
        0.043 & \textbf{0.014} & \textbf{0.014} &
        0.393 & 0.342 & \textbf{0.339} \\

        US Migration~\cite{holten2009force} & 
        0.045 &	0.023 &	\textbf{0.019} &
        0.432 &	0.272 &	\textbf{0.250} \\

        DanishAIS~\cite{DanishAIS} &
        0.071 &	0.076 &	\textbf{0.024} &
        0.213 &	\textbf{0.201} & 0.252 \\

        \midrule
        
        \textbf{Average} & 
        0.043 &	0.030 &	\textbf{0.017} &
        0.329 &	\textbf{0.291} &	0.296 \\ 
        
        \bottomrule
    \end{tabular}
    }
\end{table}



We conducted a quantitative experiment to compare the effectiveness of our trajectory-driven path generation algorithm with two representative edge bundling algorithms.
The experiment aims to assess 1) the deviation of the bundled paths from the original ones, and 2) the efficiency in reducing the total path length. 
The same datasets used in Sec.~\ref{sec:quantitative} were employed for this experiment.


\noindent\textbf{Baseline algorithms.} We selected two representative edge bundling algorithms for comparison.
The first, divided edge bundling (DEB)~\cite{selassie2011divided}, employs the attraction and spring forces to bundle edges with similar positions and directions.
The second, multilevel agglomerative edge bundling (MAEB)~\cite{gansner2011MINGLE}, hierarchically bundles similar edges to minimize the total path length.
For both algorithms, we used the default parameters reported in their papers.

\noindent\textbf{Evaluation criteria.}
We adopted two metrics, deviation and ink ratio~\cite{wallinger2022edgepath}, which assess the deviation from the original paths and the efficiency in reducing the total path length, respectively. 
Let the original path set be $S=\{s_1,\dots, s_n\}$ and the bundled path set be $S'=\{s_1',\dots, s_n'\}$.


\myparagraph{Deviation} measures the misalignment of the bundled paths with the original ones.
Following the common practice~\cite{Francois2011globaldtw,Tao2021comparative}, the misalignment of a bundled path $s_i'$ and its original path $s_i$ is measured by their dynamic time warping distance DTW($s_i$, $s_i'$).
The deviation is then calculated as the average dynamic time warping distance across all pairs of bundled and original paths:
\begin{equation}
\text{Deviation}(S,S') = \frac{1}{n}\sum^n_{i=1}\text{DTW}(s_i,s_i')
\notag
\end{equation}

\myparagraph{Ink ratio} measures the efficiency in reducing the total path length.
It is calculated as the ink of the bundled path set $S'$ divided by that of the original path set $S$:
\begin{equation}
\text{Ink ratio}(S,S') = \frac{\text{Ink}(S')}{\text{Ink}(S)},
\notag
\end{equation}
where $\text{Ink}(S)$ is the number of pixels occupied by the paths in $S$~\cite{wallinger2022edgepath}.\looseness=-1


\begin{figure}[t]
  \centering
  \includegraphics[width=1.0\linewidth]{figures/danish_new.pdf}
  \put(-186,133){(a)}
  \put(-66,133){(b)}
  \put(-186,3){(c)}
  \put(-66,3){(d)}
  % \newcaption
  \caption{
  Original paths and the edge bundling results on the DanishAIS dataset: (a) original paths; (b) divided edge bundling (DEB)~\cite{selassie2011divided}; (c) multilevel agglomerative edge bundling (MAEB)~\cite{gansner2011MINGLE}; (d) RouteFlow.
  }
  \label{fig:danish}
  \Description{Original paths and the edge bundling results on the DanishAIS dataset: (a) original paths; (b) divided edge bundling (DEB)~\cite{selassie2011divided}; (c) multilevel agglomerative edge bundling (MAEB)~\cite{gansner2011MINGLE}; (d) RouteFlow.}
\end{figure}

% \vspace{1mm}
\noindent\textbf{Results}

Table~\ref{tab:bundling_result} shows the comparison results on seven datasets. 
These results indicate that our algorithm achieves the lowest deviation and performs comparably with the baseline algorithms in terms of ink ratio.






\begin{figure}[t]
  \centering
  \setlength{\abovecaptionskip}{1.2mm}
  \includegraphics[width=0.9\linewidth]{figures/birdmap_new.pdf}
  \put(-173,136){(a)}
  \put(-65,136){(b)}
  \put(-171,3){(c)}
  \put(-63,3){(d)}
  % \newcaption
  \caption{
  Original paths and the edge bundling results on the BirdMap dataset: (a) original paths; (b) divided edge bundling (DEB)~\cite{selassie2011divided}; (c) multilevel agglomerative edge bundling (MAEB)~\cite{gansner2011MINGLE}; (d) RouteFlow.
  }
  \label{fig:birdmap}
  \Description{Original paths and the edge bundling results on the BirdMap dataset: (a) original paths; (b) divided edge bundling (DEB)~\cite{selassie2011divided}; (c) multilevel agglomerative edge bundling (MAEB)~\cite{gansner2011MINGLE}; (d) RouteFlow.}
\end{figure}
\myparagraph{Deviation}.
The baseline algorithms perform worse in terms of deviation because they do not explicitly preserve the original paths and often excessively aggregate paths. 
This leads to larger deviation from the original paths and hinders the identification of local hotspots.
For example, as shown in Fig.~\ref{fig:danish}(a), the DanishAIS maritime transportation dataset contains two chokepoints that significantly impact maritime traffic efficiency, making them strategic hotspots for route optimization.
However, the baseline algorithms fail to preserve these chokepoints (Figs.~\ref{fig:danish}(b) and (c)).
In contrast, RouteFlow reduces the deviation from the original paths using an anchor force and effectively preserves these two chokepoints (Fig.~\ref{fig:danish}(d)).
This preservation enables more accurate and effective route optimization.


\myparagraph{Ink ratio}.
Out of the seven datasets, MAEB achieves the lowest ink ratio in four, while RouteFlow achieves the lowest in three.
MAEB's better performance is due to its focus on optimizing ink ratio, but this comes at the cost of higher deviation from the original paths.
In contrast, RouteFlow balances ink ratio and deviation, sacrificing a small amount of ink ratio to better align the bundled paths with the original ones.
This balance is important in generating clear and reliable animation.
For example, as shown in Fig.~\ref{fig:birdmap}(a), migratory birds in the BirdMap dataset take detour paths along the Great Rift Valley to access water or suitable habitats during their journey from Europe to Africa.
These detour paths are critical to understanding the birds' movement patterns in the context of geographical features.
However, MAEB distorts the original paths and bundles them into straight paths to minimize the ink ratio, which obscures the birds' actual movement patterns (Fig.~\ref{fig:birdmap}(c)).
Although DEB does not explicitly minimize the ink ratio, it also bundles the paths into straight paths and obscures the movement patterns in this example (Fig.~\ref{fig:birdmap}(b)).
In contrast, RouteFlow effectively bundles the paths while preserving the detour paths (Fig.~\ref{fig:birdmap}(d)).
This provides a clearer and more reliable visual summary of the birds' movements.\looseness=-1