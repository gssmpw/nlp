\section{Introduction}

\begin{figure*}[t]
\setcounter{figure}{0}
  \centering
  \setlength{\abovecaptionskip}{1.5mm}
 \includegraphics[width=\linewidth]{figures/motivation.pdf}
  \put(-432,3){(a)}
  \put(-309,3){(b)}
  \put(-186,3){(c)}
  \put(-63,3){(d)}
  \caption{A comparison of three animated transition methods on a bird migration example:
  (a) trajectory data;
  (b) the focus+context grouping method~\cite{zheng2018focus+};
  (c) the vector-field-based method~\cite{wang2017vector};
  (d) RouteFlow.
  }
  \Description{A comparision of our methods with Focus+Context Grouping~\cite{zheng2018focus+}, Vector Field~\cite{wang2017vector} on a bird migration example.}
  \label{fig:motivation}
   % \vspace{-1mm}
\end{figure*}

% Animating the objects' movements along their trajectories is pervasive in real-world applications.
Animating objects’ movements is widely used to facilitate tracking changes and observing both the global trend and local hotspots where objects converge or diverge~\cite{cui2011textflow,tao2017hotspot}.
For example, by animating bird migration data~\cite{birdmapdemo,birdvideo}, users can observe birds' movements, understand the migration trend, and identify highly active locations where birds converge to cross the straits or diverge to bypass mountains.
Here, the global trend provides valuable insights into broader movement patterns, while the local hotspots serve as strategic locations for observation and analysis~\cite{li2023vectortrajectory,liu2014survey}.
% For example, in maritime transportation, local hotspots, such as major ports or chokepoints, serve as strategic locations for route optimization.

% \change{For example, maritime transportation animation~\cite{vesselvideo} highlights the congested, strategic locations, such as major ports or chokepoints like canals.
% Such local hotspots serve as strategic locations for route optimization. 
% % These animations not only facilitate tracking objects' movements but also reveal critical insights into complex trajectory data, including the global trend and local hotspots where many objects converge or diverge.
% }


% 第一段的最开始版本:
% Animating the objects' movements along their trajectories is pervasive in real-world applications. 
% For example, by animating bird migration data~\cite{birdmapdemo,birdvideo}, users can observe birds' movements, understand the migration trend, and identify the locations where birds converge to cross the straits or diverge to bypass mountains.
% These animations not only facilitate tracking objects' movements but also reveal the global trend and local hotspots where many objects converge or diverge~\cite{tao2017hotspot,li2023vectortrajectory,deng2018road}.



Many research efforts have been directed toward developing techniques for animated transitions, aimed at helping users track objects' movements.
They mainly focus on adjusting various animation parameters from temporal (e.g., speed~\cite{dragicevic2011distortion}, staging~\cite{heer2007animated}, staggering~\cite{chevalier2014not}) and spatial (e.g., animation paths~\cite{du2015trajectory,wang2017vector}) perspectives.
% Notably, 
Recent studies have further advanced these techniques.
Zheng~\etal~\cite{zheng2018focus+} divided transitions into groups and animated them sequentially, thereby breaking down complex animations into simpler ones (Fig.~\ref{fig:motivation}(b)).
Wang~\etal~\cite{wang2017vector} used vector fields to coordinate group movements and reduce occlusion by spatially separating animation paths (Fig.~\ref{fig:motivation}(c)).
% However, these methods only consider the start and end positions in the objects' trajectories.
% Although they can effectively convey the global trend, they often obscure critical local hotspots along the movement trajectories (Fig,~\ref{fig:motivation}(a)).
% While these methods can effectively convey global trends by considering the start and end positions of the trajectories, they often obscure critical local hotspots along the movement trajectories (Fig.\ref{fig:motivation}(a)).
However, all these methods only consider the start and end positions in the objects' trajectories.
Although effective in conveying the global trends, they often obscure critical local hotspots along the movement trajectories (Fig,~\ref{fig:motivation}(a)).
% \looseness=-1

Recognizing this gap, we aim to design an animated transition method that considers the movement trajectories of objects.
% These animations should reveal both the global trend and local hotspots while assisting users in tracking objects' movements.
% Recognizing this gap, we design an animated transition method that considers movement trajectories. 
By using these trajectories, the animations can effectively reveal both the global trend and local hotspots.
Thus, our method provides a clearer understanding of local areas of high activity in their global context.
However, designing such animations is non-trivial. 
First, balancing the global trend and local hotspots in animation remains challenging. 
Local hotspots are highly active points where many objects usually converge or diverge~\cite{deng2018road, li2023vectortrajectory, tao2017hotspot}.
Overemphasizing local hotspots may result in excessive branching areas, impeding the identification of the global trend.
Conversely, stressing the global trend heavily may obscure important local hotspots.
Second, reducing occlusion in animated transitions is imperative yet difficult, especially when multiple objects move simultaneously. 
They may occlude each other, thus significantly increasing the difficulty of tracking their movements.
Occlusion becomes even more severe in local hotspots, where many objects converge or diverge. 
% % Formulation + Method

To address these challenges, we use a real-world bus route analogy: groups of passengers board the same bus at different stops, travel together along the shared routes, and disembark at designated stops.
This analogy guides our animation design.
We regard objects as passengers traveling together, with local hotspots representing various bus stops where these passengers get on and off.
Based on this, we animate objects following the shared paths, converging or diverging at local hotspots. 
As such, users can observe the global trend and identify local hotspots, similar to observing overall bus routes and identifying frequently visited stops.
In this analogy, we regard 1) achieving a balance between the global trend and local hotspots as planning bus routes for efficiency and effectiveness. 
These bus routes should not only be of minimal length but also meet passengers' travel demands.
Besides, we consider 2) reducing occlusion by allocating passengers to respective seats in the process.
Consequently, we formulate the problem of designing animated transitions for trajectory data as a sequential optimization of two sub-problems: bus routing and seat allocation.

Based on this formulation, we propose RouteFlow, a trajectory-aware animated transition method comprising two steps: trajectory-driven path generation and object layout generation. 
As shown in Fig.~\ref{fig:motivation}(d), we create ``bundled'' animation paths for groups of objects that share similar movement trajectories. 
These animation paths are generated by a bottom-up hierarchical edge bundling algorithm, which progressively bundles similar trajectories, level by level, effectively capturing both the global trend and local hotspots. 
To minimize occlusion, we apply an incremental circle packing algorithm, sequentially generating the layout at each local hotspot.
The animation is then rendered using an interpolation-based method.

We evaluate RouteFlow through a quantitative experiment on real-world data and a controlled user study. 
The results indicate that compared with the state-of-the-art methods, RouteFlow better facilitates identifying the global trend and locating local hotspots while performing comparably in tracking objects' movements.
% Contribution
The main contributions of our work include:
\begin{itemize}
[itemsep=2pt,topsep=0pt,parsep=0pt]
    \item A formulation of designing animated transitions as the sequential optimization of bus routing and seat allocation problems.
    \item RouteFlow, a trajectory-aware animated transition method that consists of a bottom-up hierarchical edge bundling algorithm and an incremental circle packing algorithm.
    The open-source implementation is available at \url{https://github.com/Trajectory-Anim/Trajectory-Aware-Animated-Transitions}.
    \item A quantitative experiment and a user study evaluating performance on tracking objects' movements, identifying the global trend, and locating local hotspots.
\end{itemize}

