\section{Problem Formulation}


In this section, we introduce the problem formulation, including the bus route analogy and two sub-problems derived from this analogy.

\subsection{The Bus Route Analogy}

We illustrate the objects' movements in animation using the real-world bus route analogy, where passengers travel along different bus routes to reach their destinations.
As shown in Fig.~\ref{fig:teaser}(a), 
passengers gather at bus stops and board the same bus (A$_1$).
They then travel together along shared routes (A$_2$).
Eventually, they disembark at designated stops when approaching their destinations or transferring to other routes (A$_3$).
% Following this analogy, we organize the objects' movements in animation to mirror the process of passengers navigating bus routes.
As such, we apply this analogy to guide the design of our animation.
As shown in Fig.~\ref{fig:teaser}(b), groups of objects with similar movement trajectories converge at local hotspots (B$_1$), analogous to bus stops, and then move together along the shared animation paths (B$_2$), much like passengers on the same bus.
As the animation progresses, these objects may diverge to reach their destinations separately (B$_3$).



Based on this analogy, we design our animation, RouteFlow, to capture both the global trend and local hotspots.
By grouping objects with similar trajectories and moving them along shared animation paths, we reveal the global trend, just as the bus routes that passengers travel along. 
Meanwhile, objects converge or diverge at specific local hotspots, similar to passengers boarding and disembarking at bus stops.
This allows us to simplify complex and cluttered trajectories in animation while ensuring that critical convergence and divergence points are preserved. 



Our animation leverages the Gestalt principles of \textit{Common Fate} and \textit{Proximity}~\cite{todorovic2008gestalt, wagemans2012century} to shape the perception of grouping. 
The \textit{Common Fate} principle states that visual elements moving together are perceived as a group~\cite{chalbi2020common}.
Accordingly, objects moving together along the same animation path are interpreted as a cohesive group.
The \textit{Proximity} principle states that visual elements close to one another are perceived as part of the same group~\cite{wagemans2012century}.
In this case, we position objects with similar trajectories in close proximity, simulating passengers on the same bus. 

\begin{figure*}[ht]
  \centering
  \setlength{\abovecaptionskip}{1.2mm}
  \includegraphics[width=0.9\linewidth]{figures/pipeline.pdf}
  \put(-417,3){(a)}
  \put(-252,3){(b)}
  \put(-64,3){(c)}
  \caption{The pipeline of our method:
  (a) trajectory data;
  (b) bottom-up hierarchical edge bundling algorithm;
  (c) incremental circle packing algorithm.
  }
  \label{fig:pipeline}
  \Description{The pipeline of our method.}
\end{figure*}

To create the animation, we should generate the animation paths in a way that is similar to planning bus routes. 
Furthermore, since multiple objects often move simultaneously along the same animation path, we should minimize occlusion in animation, ensuring that each object has its own position, like passengers having individual seats on a bus.
In this process, the seat allocation depends on the bus routes, as the bus routes determine which passengers are on the bus and where they board and disembark.
This dependency naturally lends itself to sequential optimization~\cite{aubry2018sequential}.
In sequential optimization, the overall problem is decomposed into smaller, manageable sub-problems that are solved in sequence.
The solution to each sub-problem then informs and serves as the input for the subsequent one, ensuring a cohesive and efficient resolution of the entire problem.
Accordingly, we decompose the problem into two sub-problems: trajectory-driven path generation (bus routing) and object layout generation (seat allocation).
Next, we detail these two sub-problems and their respective optimization goals.



\subsection{Trajectory-Driven Path Generation}
\label{sec:formulation1}

In the context of the bus routing problem, there are two main optimization goals: efficiency and effectiveness~\cite{li2002school}. 
Efficiency involves minimizing operational costs, such as reducing the total length of the bus routes. 
One of the most effective strategies is route aggregation.
This strategy encourages passengers to share the same route as much as possible during their journeys.
The key is to identify groups of passengers with similar trajectories and then design routes that accommodate these shared trajectories.
Likewise, in animation, we should group objects with similar trajectories and share their animation paths to reduce the total path length.
This consolidates similar movements, allowing users to easily perceive the global trend in animation.
However, overemphasizing efficiency can lead to excessive route aggregation, forcing some passengers to deviate far from their intended trajectories and causing significant detours. 
On the other hand, effectiveness focuses on meeting passengers' travel demands by ensuring they can successfully reach their destination without excessive detours.
This requires minimizing the deviation between the aggregated route and each passenger's intended trajectory.
One solution is to set up proper bus stops in high-demand locations to satisfy more passengers' demands.  
In animation, the resulting animation paths should align closely with the input trajectories of the objects and thus better reveal critical local hotspots where objects converge or diverge.

As such, we derive two primary optimization goals for trajectory-driven path generation: 
% As such, we derive two main optimization goals:  
\begin{itemize}
[itemsep=2pt,topsep=0pt,parsep=0pt]

    \item \textbf{Minimize the total animation path length} by aggregating animation paths for groups of objects with similar trajectories. 
    \item \textbf{Minimize the deviation from the input trajectories} by constraining the distance between the input trajectories to their aggregated paths.  
\end{itemize}


\subsection{Object Layout Generation}
\label{sec:formulation2}
There are two main optimization goals when allocating seats: maximize capacity and avoid overcrowding.
The first optimization goal is to maximize capacity.
Similar to buses efficiently filling seats, the object layout should be designed to reduce empty space to enhance compactness.
To achieve this, we strive to position similar objects close together to reduce gaps between them.  
This not only optimizes space utilization but also fosters a sense of group cohesion among closely placed objects, aligning with the \textit{Proximity} principle.
The second optimization goal is to avoid overcrowding, which can be addressed through three strategies.
First, when passengers are on the same bus, each should have their own seat to avoid interfering with others.
Second, co-travelers who board or disembark together should sit close to maintain group cohesion and avoid mixing with the crowd.
Third, passengers who disembark first should sit closest to the exit (the principle of ``first out, closest to the exit''), facilitating a smoother queueing process and mitigating potential overcrowding during disembarkation.
% the seat allocation should align with the order and position of disembarkation, i.e., passengers who get off first seated closer to the exit.
Correspondingly, in our animation: 1) objects moving along the same path should remain visible and not overlap; 2) objects that converge or diverge together should be grouped closely to avoid mixing with other groups;
and 
3) objects should be placed based on their disembarking order and positions.
Based on the analysis above, this problem involves two primary optimization goals: 
\begin{itemize}
[itemsep=2pt,topsep=0pt,parsep=0pt]
    \item \textbf{Maximizing compactness} by reducing empty space in layout.
    \item \textbf{Minimizing occlusion} by 1) applying the non-overlapping constraint within a group of objects moving together, 2) keeping objects that converge or diverge together as a group, and 3) following the principle of ``first out, closest to the exit.''
\end{itemize}




