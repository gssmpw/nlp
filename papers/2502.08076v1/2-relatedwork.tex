\section{Related Work}
There are two main tasks for animated transitions: \textbf{tracking objects' movements} and \textbf{identifying the trend}.
Most existing efforts focus on \textbf{tracking objects' movements} from temporal and spatial perspectives.
The temporal perspective includes adjustments such as refining movement speed~\cite{dragicevic2011distortion}, staging~\cite{guilmaine2012hierarchy,heer2007animated,zheng2018focus+}, and staggering~\cite{chevalier2014not}.
The spatial perspective focuses on the animation paths of objects~\cite{du2015trajectory,heer2007animated,wang2017vector,yee2001animated}.
Our work falls into the latter perspective.

\begin{figure*}[hbpt]
  \centering
  \setlength{\abovecaptionskip}{1mm}
  \includegraphics[width=\linewidth]{figures/formulation.pdf}
  \put(-388,3){(a)}
  \put(-116,3){(b)}
  \caption{Illustration of our analogy:
  (a) passengers gather and board the same bus, travel together, and disembark;
  (b) objects converge, move together, and diverge in RouteFlow animation.}
  \Description{An illustration of our analogy.}
  \label{fig:teaser}
\end{figure*}

Animation paths play a crucial role in \textbf{tracking objects' movements}~\cite{fencsik2007role,mutsumi2006trajectory}.
According to Heer and Robertson~\cite{heer2007animated}, simple trajectories are effective in minimizing confusion and enhancing predictability, thereby making it easier for users to track objects' movements.
A direct method to achieve simplicity is to use straight lines connecting the start and end positions of the movement~\cite{chang1993animation}.
To provide more natural and engaging movements while maintaining simplicity, later research used smooth curves, such as arcs~\cite{dragicevic2011gliimpse,yee2001animated} and B-splines~\cite{cao2011dicon}. 
Building on these advancements, Du~\etal~\cite{du2015trajectory} explored bundling animation paths to coordinate group movements, which improved group tracking
but could introduce the occlusion issue.
To address this, Wang~\etal~\cite{wang2017vector} utilized vector fields to coordinate movements for each group and separated the animation paths mutually to reduce occlusion among them (Fig.~\ref{fig:motivation}(c)).
However, this separation can cause much deviation from the input trajectories.
In contrast, RouteFlow creates bundled animation paths for objects with similar trajectories and reduces occlusion by applying non-overlapping constraints on the object layout.



In addition to tracking objects, animated transitions are widely used to \textbf{identify the global trend}~\cite{roberson2008trend}.
Empirical studies have discussed the potential of careful animation designs for trend identification~\cite{brehmer2020smart,chalbi2020common}. 
Recently, Zheng~\etal~\cite{zheng2018focus+} proposed the focus+context grouping method for animated transition to simultaneously track objects and identify the global trend. 
This method grouped objects with similar trends together and animated these groups sequentially (Fig.~\ref{fig:motivation}(b)).
It simplifies complex animated transitions by dividing them into a sequence of simpler groups, facilitating easier tracking of objects while also revealing the global trend.
However, this method groups transitions solely based on the start and end positions of the objects, ignoring their trajectories.
As a result, it may fail to capture important patterns throughout trajectories, e.g., the local hotspots where objects converge or diverge.
To overcome this limitation, RouteFlow considers the movement trajectories of objects, aiming to balance both the global trend and local hotspots. 