\section{Method}

\begin{figure*}[b]
\setcounter{figure}{4}
  \centering
  \setlength{\abovecaptionskip}{1.0mm}
  \includegraphics[width=0.75\linewidth]{figures/dtw.pdf}
  \put(-317,3){(a)}
  \put(-202,3){(b)}
  \put(-91,3){(c)}
  
  \caption{
  Illustration of our bottom-up hierarchical edge bundling algorithm: 
  (a) trajectory data; 
  (b) level-1; 
  (c) level-2.
  }
  \label{fig:global}
  \Description{Illustration of bottom-up hierarchical edge bundling algorithm: 
  (a) trajectory data; 
  (b) level-1; 
  (c) level-2.
  }
\end{figure*}

Fig.~\ref{fig:pipeline} shows the pipeline of our method.
Given trajectory data as input, it consists of two modules: \textbf{trajectory-driven path generation} and \textbf{object layout generation}.

\begin{figure}[t]
\setcounter{figure}{3}
  \centering
  \setlength{\abovecaptionskip}{1.5mm}
  \includegraphics[width=0.9\linewidth]{figures/forces.pdf}
  \caption{Illustration of three types of forces in our algorithm.}
  \label{fig:forces}
  \Description{Illustration of forces in our algorithm.}
  % \vspace{-3mm}
\end{figure}



\subsection{Trajectory-Driven Path Generation}




An edge bundling algorithm can effectively minimize both the total path length and the deviation from the input trajectories in generating the animation paths.
However, aggregating all the trajectories simultaneously presents two issues.
First, it fails to identify local hotspots at multiple levels of granularity, which are pervasive in real-world applications~\cite{martino2019granular}.
Second, the real-time animated transitions require high scalability of the algorithm.
To address these issues, we develop a bottom-up hierarchical edge bundling algorithm that progressively bundles similar trajectories, level by level.
As shown in Fig~\ref{fig:pipeline}(b), it captures local hotspots across multiple levels of granularity while revealing the global trend.


At each level, we adopt a force-directed strategy~\cite{holten2009force, selassie2011divided} to bundle the edges.
The core of our algorithm lies in the design of the forces that drive the bundling process, along with a bottom-up bundling that progressively bundles trajectories.

\subsubsection{Force Design}
Existing force-directed edge bundling algorithms model trajectories as a series of control points and apply forces to adjust their positions~\cite{holten2009force,selassie2011divided}.
They typically adopt two types of forces: attraction force and spring force.
However, they often fail to preserve local hotspots because these forces ignore the original positions of these input trajectories.
To address this issue, we introduce a new force, the anchor force, to reduce deviation from the input trajectories.
Fig.~\ref{fig:forces} illustrates how our algorithm incorporates these three types of forces.
Given the trajectory set $S$ and a pair of trajectories $u$ and $v$, the three types of forces are defined as follows:



\begin{itemize}
    \item \textbf{Attraction force} ($F_{att}$) is applied between control points on different trajectories to draw them closer together.
    This force bundles similar trajectories.
    According to Selassie~\etal~\cite{selassie2011divided}, $F_{att}$ is defined as:
    \begin{equation}
        \begin{aligned}
            F_{att}(u_i,v_j)=\frac{\eta(v_j-u_i)}{C_v(\eta^2+||u_i-v_j||^2)^2},
        \end{aligned}
    \end{equation}
    where $u_i$ and $v_j$ represent the $i$-th and $j$-th control points on these trajectories, and $||u_i - v_j||$ denotes the Euclidean distance between them.
    The weighting parameter $\eta$ controls the rate at which the force diminishes with increasing distance.
    A larger $\eta$ causes $F_{att}$ to decrease slower, thereby extending its influence range.
    $C_v$ denotes the number of control points on trajectory $v$.
    \item \textbf{Spring force} ($F_{spr}$) is applied between adjacent control points on the same trajectory.
    This force promotes uniform distribution of control points along the trajectory and avoids highly curved trajectories.
    According to Holten~\etal~\cite{holten2009force}, $F_{spr}$ is defined as:
    \begin{equation}
        \begin{aligned}
            F_{spr}(u_i) = C_u(u_{i+1}+u_{i-1}-2u_i),
        \end{aligned}
    \end{equation}
    where $C_u$ denotes the number of control points on trajectory $u$.
    \item \textbf{Anchor force} ($F_{anc}$) is applied to each control point, pulling it back toward its position in the input trajectories.
    This force prevents the current trajectories from deviating too far from the input trajectories. $F_{anc}$ is defined as:
    \begin{equation}
        \begin{aligned}
            F_{anc}(u_i)=||u_i'-u_i||^2\cdot \frac{u_i'-u_i}{||u_i'-u_i||},
        \end{aligned}
    \end{equation}
    where $u_i'$ denotes the original position of $u_i$, and $\frac{u_i'-u_i}{||u_i'-u_i||}$ is a unit vector indicating the direction of the force.
\end{itemize}



Based on the above force analysis, the resultant force on the $i$-th control point of trajectory $u$ is calculated as:
\begin{equation}
    \begin{aligned}
        F(u_i) = (\sum_{v\in \Gamma_{u}}\sum^{C_v}_{j=1}F_{att}(u_i,v_j))+\alpha F_{spr}(u_i) +\beta F_{anc}(u_i).
    \end{aligned}
\end{equation}
Here, $\Gamma_{u}$ denotes the set of top-$k$ similar trajectories of $u$.
The parameters $\alpha$ and $\beta$ balance the three types of forces. In our implementation, they are determined as 5 and 1 through a grid search.


\begin{figure*}[b]
  \setcounter{figure}{6}
  \centering
  \setlength{\abovecaptionskip}{1.2mm}
  \includegraphics[width=0.8\linewidth]{figures/packingpipeline.pdf}
  \put(-307,3){(a)}
  \put(-115,3){(b)}
  \caption{
  Illustration of our incremental circle packing algorithm:
  (a) construct a directed acyclic graph to model the order of local hotspots;
  (b) use forces to incrementally generate the layout at each local hotspots
  }
  \label{fig:packingpipeline}
  \Description{Illustration of our incremental circle packing algorithm:
  (a) construct a directed acyclic graph to model the sequential relationships between local hotspots;
  (b) use forces to compact objects.}
\end{figure*}

\subsubsection{Bottom-Up Hierarchical Edge Bundling}
Progressively bundling similar trajectories at each level of granularity involves two key aspects.
The first is how to select the most similar trajectories to consider when applying forces at each level.
Existing edge bundling algorithms assess edge similarities through compatibility metrics, which consider factors such as topology~\cite{selassie2011divided} and importance~\cite{Quan2012tgieb} but often fail to capture trajectory similarities.
To better capture trajectory similarities, we design our compatibility metric based on dynamic time warping (DTW)~\cite{muller2007dtw}, a widely accepted metric for assessing trajectory similarity~\cite{zheng2015trajectory}.
DTW calculates the distance between two trajectories by finding the optimal alignment between points on them, thereby capturing the overall similarity between the entire trajectories~\cite{muller2007dtw}.
Given two trajectories ($u$, $v$) and their DTW distance (DTW$(u,v)$), the compatibility between $u$ and $v$ is defined as:
\begin{equation}
    \begin{aligned}
    \text{compatibility}(u,v)=1 - \text{norm}(\text{DTW}(u,v)),
    \end{aligned}
\end{equation}
where norm$(\cdot)$ denotes the min-max normalization to scale the distance into the range [0,1].
To reduce computational complexity, we only consider the top-$k$ similar trajectories according to the compatibility metrics.
In our implementation, $k$ is a user-specified parameter that is set as five by default.

The second is how to hierarchically bundle these trajectories.
At each level, similar trajectories are bundled, revealing local hotspots where they converge and diverge.
To identify local hotspots across multiple levels of granularity, it is crucial to preserve the convergence and divergence identified at lower levels.
Therefore, we determine the bundled portions of trajectories at each level by measuring the distances between control points, as shown in Fig.~\ref{fig:global}.
Each bundled portion will be merged into a single trajectory, which serves as input for the next level, where they are further bundled.
Throughout this process, the identified local hotspots remain unchanged, as they are excluded from the bundled portions and unaffected by further bundling.
At each level, the three types of forces are applied.
When moving to the next level, the attraction force is increased tenfold to adapt to the sparser distribution of trajectories.



To evaluate the effectiveness of our algorithm, we compare it with two representative edge bundling algorithms, divided edge bundling (DEB)~\cite{selassie2011divided} and multilevel agglomerative edge bundling (MAEB)~\cite{gansner2011MINGLE}. 
We assess these methods based on their efficiency in reducing total path length and deviation from the original paths.
The results show that our algorithm achieves the lowest deviation and performs comparably with the baseline algorithms in terms of ink ratio.
Details can be found in Appendix~\ref{sec:appendixA}.




\subsection{Object Layout Generation}

\begin{figure}[t]
\setcounter{figure}{5}
  \centering
  \setlength{\abovecaptionskip}{1mm}
  \includegraphics[width=0.9\linewidth]{figures/packinggoal.pdf}
  \caption{Illustration of the goals in our object layout generation.}
  \label{fig:packinggoal}
  \Description{Illustration of goals in our object layout generation.}
   % \vspace{-7mm}
\end{figure}
The optimization goals described in Sec.~\ref{sec:formulation2} are achieved in three ways.
First, to reduce the empty space and satisfy the non-overlapping constraint for a group of objects moving together (Fig.~\ref{fig:packinggoal}A), we use a circle packing algorithm~\cite{yuan2023visual} to generate the object layout.
Second, to keep objects that converge or diverge together as a group (Fig.~\ref{fig:packinggoal}B), an incremental circle packing algorithm is developed.
Third, to follow the principle of ``first out, closest to the exit"(Fig.~\ref{fig:packinggoal}C), we place objects based on their disembarking order and positions. 


Just as passengers only adjust their seats when boarding or disembarking along the bus route, we update the layout incrementally only at the local hotspots.
To achieve this, we first establish the order of local hotspots for layout generation by constructing a directed acyclic graph (Fig.~\ref{fig:packingpipeline}(a)) and then incrementally generate the layout at each local hotspot (Fig.~\ref{fig:packingpipeline}(b)).
In the directed acyclic graph, nodes represent local hotspots, and directed edges indicate the movements of objects between these local hotspots.
We perform a reverse topological sort on the graph to generate the order of local hotspots.
Then, we incrementally generate the layout at local hotspots according to their order.
The basic idea of generating the layout at each local hotspot is to generate a layout for each newly arriving group and then pack these new layouts with those of previous groups.
As shown in Fig.~\ref{fig:packingpipeline}, when packing the objects at a given local hotspot A, they are placed to preserve their relative positions.
This prevents occlusion during disembarking.
Inspired by G\"ortler~\etal~\cite{gortler2018bubble}, we adopt a force-directed algorithm and apply two types of attraction forces (Fig.~\ref{fig:packingpipeline}(b)).
The first type of force moves all objects/groups toward the current local hotspot.
The second type of force attracts neighboring objects/groups together.
To avoid occlusion, we model objects/groups as rigid bodies and use the Box2D engine~\cite{catto2010box2d} for implementation.




\begin{table*}[b]
  \centering
  \setlength{\abovecaptionskip}{1.2mm}
  \caption{Comparison between different methods, including the focus+context grouping method (F+C), the vector-field-based method (VF), and RouteFlow. For all four metrics, lower values are better.}
  \label{tab:metric_result} % Optional: adds a label for referencing the table elsewhere
  \resizebox{\textwidth}{!}{
   % \scriptsize
    \begin{tabular}{c|ccccccccccccccccccccc}
       \toprule
        \multirow{2}{*}{\textbf{Dataset}}
        & \multicolumn{3}{c}{Overall occlusion}
        & \multicolumn{3}{c}{Within-group occlusion}
        & \multicolumn{3}{c}{Deformation}
        & \multicolumn{3}{c}{Dispersion}\\
        \cmidrule(lr){2-4} \cmidrule(lr){5-7} \cmidrule(lr){8-10} \cmidrule(lr){11-13}
        & {F+C} & {VF} & {RouteFlow}
        & {F+C} & {VF} & {RouteFlow}
        & {F+C} & {VF} & {RouteFlow}
        & {F+C} & {VF} & {RouteFlow} \\
        \midrule

        Taxi~\cite{yuan2011drive, yuan2010tdrive} &
        0.00123 &	0.00180 &	\textbf{0.00048} &
        0.01963 &	0.00626 &	\textbf{0.00606} &
        0.00081 &	0.00107 &	\textbf{0.00062} &
        0.05948 &	0.06766 &	\textbf{0.02606} \\

        BirdMap~\cite{BirdMap} & 
        0.00277 &	0.00958 &	\textbf{0.00213} &
        0.02773 &	0.02607 &	\textbf{0.00904} &
        0.00152 &	0.00160 &	\textbf{0.00064} &
        0.13762 &	0.12494 &	\textbf{0.03258} \\

        Railway~\cite{railway} &
        0.00229 &	0.01351 &	\textbf{0.00142} &
        0.11960 &	0.10420 &	\textbf{0.02875} &
        0.00128 &	0.00140 &	\textbf{0.00063} &
        0.10629 &	0.10950 &	\textbf{0.03090} \\


        MEIBook~\cite{mei} & 
        0.00132 &	0.00566 &	\textbf{0.00066} &
        0.01362 &	0.01196 &	\textbf{0.00675} &
        0.00079 &	0.00105 &	\textbf{0.00064} &
        0.06699 &	0.06817 &	\textbf{0.02430} \\

        OpenSkyAirline~\cite{opensky} &
        0.00076	&   0.00394	&   \textbf{0.00055} &
        0.01934	&   0.01814 &  	\textbf{0.00571} &
        0.00082	&   0.00117	&   \textbf{0.00061} &
        0.08047	&   0.08216	&   \textbf{0.03007} \\

        US Migration~\cite{holten2009force} & 
        0.00115 &	0.00408 &	\textbf{0.00072} &
        0.01107 &	0.01027 &	\textbf{0.00295} &
        0.00097 &	0.00147 &	\textbf{0.00066} &
        0.13015 &	0.13345 &	\textbf{0.03802} \\

        DanishAIS~\cite{DanishAIS} &
        0.00047 &	0.00120 &	\textbf{0.00012} &
        0.01435 &	0.00460 &	\textbf{0.00266} &
        0.00122 &	0.00089 &	\textbf{0.00087} &
        0.11688 &	0.12816 &	\textbf{0.08611} \\


        \midrule
        
        \textbf{Average} & 
        0.00148 &	0.00561 &	\textbf{0.00088} &
        0.03097 &	0.02469 &	\textbf{0.00842} &
        0.00122 &	0.00141 &	\textbf{0.00065} &
        0.10627 &	0.10858 &	\textbf{0.03727} \\ 

 
        
        \bottomrule
    \end{tabular}
    }
\end{table*}


\subsection{Implementation}
After generating the animation paths and object layout at all the local hotspots, we use an interpolation-based method to render smooth animations.
This method relies on the timing of local hotspots and the objects' start and end points.
In the following, we will use ``point'' as a substitute for both local hotspots and the objects' start and end points.
We ensure that 1) groups of objects that converge or diverge together arrive at or leave the local hotspots at the same time and 2) excessively fast speeds are avoided.
As shown in Fig.~\ref{fig:scanline}, we use a scan line that moves through all points, assigning their timing as when they intersect with the scan line.
The movement direction of the scan line is determined by the vector formed between the average start and end positions of all animation paths.
However, paths that form a large angle with the scan line's movement direction can result in excessively high speeds of objects.
To address this, we iteratively adjust the timing of points until the maximum speed is less than twice the minimum speed.
If we detect excessively high speed between two connected points, we adjust the timing by either delaying the latter point or advancing the former at random.
We interpolate between points to render the animation after completing the iterative adjustment.
To further enhance smoothness, we apply the slow-in, slow-out technique~\cite{dragicevic2011distortion}.
\begin{figure}[t]
\setcounter{figure}{7}
  \centering
  \setlength{\abovecaptionskip}{1.2mm}
  \includegraphics[width=0.8\linewidth]{figures/scanline.pdf}
  \caption{
  % Illustration of our object layout.
  Illustration of the scan line moving through all the local hotspots and the objects' start and end points.
  }
  \label{fig:scanline}
  \Description{Illustration of our object layout.}
  % \vspace{-3mm}
\end{figure}




