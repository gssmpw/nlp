\section{Detailed Animation Result Analysis}
\label{sec:appendixB}
To facilitate interpreting the magnitudes of the measured values, Fig.~\ref{fig:metrics} compares the animation results with different measured values, which are generated by three methods:  RouteFlow, the focus+context grouping method (F+C) and the vector-field-based method (VF).
These methods are the same as in Sec.~\ref{sec:quantitative}. 
Fig.~\ref{fig:metrics}(a) shows the positions of all objects, the same as in Fig.~\ref{fig:quan-result}. 
For easy comparison, we select a subset of 10 objects in a group (with the orange contour), and calculate their values for each metric in this frame (Figs.~\ref{fig:metrics}(c)-(e)), except for overall occlusion, which is calculated on all objects (Fig.~\ref{fig:metrics}(b)).

\begin{figure}[t]
  \centering
  \includegraphics[width=\linewidth]{figures/Metric-Compare.pdf}
  \put(-127,394){(a)}
  \put(-127,300){(b)}
  \put(-127,191){(c)}
  \put(-127,97){(d)}
  \put(-127,3){(e)}
  % \newcaption
  \caption{Comparisons of the three animation results with different metric values, which are generated by the three methods: (a) the same object positions as Fig.~\ref{fig:quan-result}, and a subset of 10 objects for detailed analysis; (b) overall occlusion (c) within-group occlusion (d) deformation; (e) dispersion. Here, (b)-(e) show the different metric values of the 10 objects.}
  \label{fig:metrics}
  \Description{Illustration of how the metric values change with different object positions.}
\end{figure}

\myparagraph{Occlusion}.
Figs.~\ref{fig:metrics}(b) and (c) compare the animation results generated by the three methods in terms of the overall and within-group occlusion.
RouteFlow prevents overlap between objects when they move together as a group, thereby achieving an occlusion score of $0.00000$ for both metrics at this frame.
However, overlaps remain unavoidable at local hotspots where objects converge and diverge.
In contrast, the focus+context grouping method and the vector-field-based method result in object overlap to different degrees.
For example, the focus+context grouping method achieves an overall occlusion score of $0.00124$, with 31 out of 51 objects overlapping, and a within-group occlusion score of $0.13333$, with $7$ of the $10$ objects overlapping.

\myparagraph{Deformation}.
Fig.~\ref{fig:metrics}(d) compares the animation results generated by the three methods in terms of deformation. 
This metric measures the changes in distance between objects from the previous frame (grey) to the current frame (dark grey).
RouteFlow organizes the objects that move together into a cohesive group, with their relative positions changing only at local hotspots.
As a result, the deformation score is $0.00000$, indicating no changes in the distance between objects.
In contrast, the focus+context grouping method and the vector-field-based method do not explicitly manage the object layout, leading to larger deformation scores.
For example, the focus+context grouping method achieves a deformation score of $0.00850$, indicating that the average distance between objects has changed by approximately $1.46$ times the object's radius.



\myparagraph{Dispersion}.
Fig.~\ref{fig:metrics}(e) compares the animation results generated by the three methods in terms of dispersion.
RouteFlow closely groups the objects that move together, resulting in the lowest dispersion score of $0.01573$, with the objects occupying approximately $80\%$ of the space within their convex hull.
% thus achieving a better balance between occlusion and dispersion.
In contrast, the focus+context grouping method and the vector-field-based method exhibit higher dispersion scores.
For example, the focus+context grouping method achieves a dispersion score of $0.06006$, with the objects occupying approximately $20\%$ of the space within their convex hull.