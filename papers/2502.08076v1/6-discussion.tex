\section{Discussion and Future Work}
\label{sec:discuss}
As demonstrated by the results from both the quantitative evaluation and user study, the primary benefit of RouteFlow lies in its capacity to effectively reveal the global trend and identify local hotspots.
It also maintains comparable performance in tracking objects.
The participants generally gave positive feedback on its usability.
Nonetheless, they also pointed out several limitations that deserve further investigation.

\noindent\textbf{Introducing additional visual channels.}
RouteFlow delivers an elaborate animation by planning animation paths and generating a compact and non-occluded object layout. 
In the user study, P10, who majored in information design, suggested including more visual channels to enhance the expressiveness of animations.
For instance, using different colors and brightness can help users track the targeted objects~\cite{hu2016spot}.
Moreover, objects can be encoded in varying sizes based on their importance, making them more easily distinguishable.
A recent study has also validated that the variations in object brightness and size have a positive impact on perceiving groups~\cite{chalbi2020common}.
In the future, we can apply a dynamic color palette to our generated object layout to improve the animation~\cite{chen2024dynamic}.


\noindent\textbf{Interactive animation adjustment.}
Our method automatically generates animations based on the input trajectories, easing the effort required for manual design. 
On top of the automated method, RouteFlow can be enhanced by supporting users to interactively design and refine animations, addressing their unique needs~\cite{wang2017vector}.
For example, users can directly drag to refine the animation paths or modify the corresponding object groups and layout, and our method can propagate the changes to the whole animation. 
Therefore, users do not need to manipulate each object in detail.
Besides, our method can be integrated with dynamic parameter adjustment to balance different types of forces in our hierarchical edge bundling for animation path generation. 
As such, we can analyze data features such as trajectory density, local hotspot distribution, or occlusion levels and adjust parameters on targeted datasets or areas to identify different local hotspots. 
Users can also interactively specify parameters on demands. 



\noindent\textbf{Supporting more trajectory patterns.}
While our current method effectively highlights the global trend and local hotspots, trajectory data often contains other patterns that can provide valuable insights~\cite{zheng2015trajectory}.
For instance, periodic patterns represent movements that repeat at regular intervals, such as daily commutes or seasonal migrations~\cite{cao2007periodic}.
Anomalies refer to movements that deviate significantly from the norm, representing unusual or rare behaviors, such as a migratory bird taking an abnormal path due to environmental disruptions~\cite{liu2014fraud,liu2011anomaly}.
A practical solution is integrating existing pattern detection techniques~\cite{chandola2009anomaly} and designing animations to communicate these patterns effectively to users.
For example, trajectories that exhibit periodic patterns could be animated with pulsating effects or cyclic color changes~\cite{aigner2011visualization}, clearly highlighting their repetitive nature over time.

\noindent\textbf{Scalability.}
RouteFlow faces scalability issues due to both algorithmic capabilities and visual constraints.
From the algorithmic perspective, RouteFlow can process hundreds of moving objects in real time.
When scaled to one thousand objects, RouteFlow completes processing in around 10 seconds, making real-time animation impractical at this scale.
From the visual perspective, the number of displayed objects is constrained by both screen space and human perception.
Especially, previous studies have shown that people can only track a limited subset of moving objects~\cite{feria2012effects, franconeri2010tracking},
making human perception a more limiting factor.
A potential solution to address these limitations is to combine a sampling method with our animation method. 
The key challenge is to identify a set of representative samples that effectively capture both the global trend and the local hotspots.


