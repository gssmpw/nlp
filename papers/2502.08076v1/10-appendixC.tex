\section{User Study Settings}
\label{sec:appendixC}
\subsection{Synthetic Dataset Generation}
As illustrated in Fig.~\ref{fig:datageneration}, our dataset generation process involves three steps: generate the global trend, determine local hotspots, and create trajectories.

Since the global trend can be depicted by representative trajectories~\cite{zheng2015trajectory}, we generate a smooth trajectory using B-splines as the ground truth.
We include one or two bends in these B-splines, leading to 2 types of the global trend in our dataset. 
To achieve this, we randomly select two points as the start and end points and sample one (for one bend) or two (for two bends) intermediate points as B-spline control points within the region between them.
To avoid highly curved trajectories, we ensure that the angle between any three consecutive points (start, control points, and end) is greater than 135 degrees.

Next, to determine local hotspots, we randomly sample points along the generated B-spline and designate these points as ground truth for local hotspots.
Feedback from our pilot study indicates that 
to balance complexity and diversity, the number of local hotspots should be limited to two or three, with each dataset containing at least one of the local hotspot types: convergence or divergence.
Each local hotspot is then randomly assigned as either a converging or diverging point, resulting in three possible assignments: 1 convergence + 1 divergence, 2 convergences + 1 divergence, and 1 convergence + 2 divergences.
Considering the types of the global trend and the local hotspot assignments, we generate 6 dataset types (2 types of the global trend $\times$ 3 local hotspot assignments).
To simplify the evaluation, we 1) limit the number of branches at the local hotspots to 2; 2) select one B-spline as the global trend and generate another branch at the local hotspots along this B-spline; and 3) avoid crossing between the two branches of each local hotspot. 


Finally, we generate the trajectories for the objects.
Following the user feedback and the setting of previous studies~\cite{zheng2018focus+}, we set the number of trajectories to 30.
The trajectories are generated by adding random perturbations to the global trend and the branches of the local hotspots.

\subsection{Method Counterbalance}
In our user study, we counterbalanced the order of different methods.
We divided 15 participants into five groups, with three participants in each group. 
Within each group, we used an expanded Latin square and applied a cyclic shift to the method order for each participant.
The three methods are denoted as A, B, and C.
In the test scenario, the order in which these methods were presented to the participants in each group was as follows:
\begin{itemize}
    \item A B C, B C A, C A B, A B C, B C A, C A B, A B C, B C A, C A B, A B C, B C A, C A B
    \item B C A, C A B, A B C, B C A, C A B, A B C, B C A, C A B, A B C, B C A, C A B, A B C
    \item C A B, A B C, B C A, C A B, A B C, B C A, C A B, A B C, B C A, C A B, A B C, B C A
\end{itemize}