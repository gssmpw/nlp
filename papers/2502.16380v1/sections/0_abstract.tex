
% Recourse is important

% Existing methods can miss recourse, hard to find reason data points don't have recourse, slow if have to check all data points
Machine learning models routinely support or automate decisions that affect individuals -- deciding who receives loans, a job interview, or an organ transparent. In these settings, models can assign \emph{fixed predictions} that individuals cannot change that individuals cannot change. 
%
%Existing methods can only verify recourse exists for individual data points. To verify recourse over an entire population to which a model is being deployed, these individualized procedures need to be run on a dataset sampled from the population which can be computationally demanding, miss out-of-sample instances without recourse, require access to (possibly) sensitive data, or yield uninterpretable results. 
%
In this work, we introduce a approach based on mixed-integer quadratically-constrained programming that can certify that an entire population has recourse under a linear classifier, or identify interpretable regions with no recourse. Our approach handles a broad class of constraints on actions, and runs in a under a second on real-world datasets. 
%
We conduct a comprehensive empirical study of recourse verification on datasets from consumer finance. Our results highlight individualized recourse methods may misleadingly certify recourse for a population and emphasize the importance of population-based approaches.

% Comically long abstract
%Machine learning (ML) is increasingly being used in high stakes domains to decide who receives loans, jobs, or even livers. In these settings, it is critical that ML models offer individuals \textit{recourse}, the ability to change the decision of a model.%, or else they risk precluding access to credit, employment, and healthcare. While we typically want to certify recourse over the population to a which a ML model could be applied, existing methods can only verify recourse on individual data points. Running these procedures on a dataset sampled from the population can be computationally demanding, miss out-of-sample instances without recourse, and yield results that are difficult to interpret for model developers. In this work, we introduce a novel approach that is able to certify whether an entire population has actionable recourse under a linear classifier. Our method leverages a Mixed-Integer Quadratically Constrained Programming model that is able to either certify recourse over the entire population, or generate an interpretable region within the population with no recourse. Our approach handles a broad class of actionability constraints, and runs in a fraction of the time it takes to audit recourse for every data point individually within the population. We conduct a comprehensive empirical study of our approach on datasets from consumer finance. Our results highlight that existing approaches may misleadingly certify recourse for a population, emphasizing the need for tools that audit recourse beyond individual data points.
