\section{Concluding Remarks}
Our paper introduces a new paradigm for algorithmic recourse that seeks to characterize fixed predictions by finding confined regions, areas in the feature space where a model is not responsive to individuals' actions. This work highlights that characterizing confined regions can help model developers pre-empt harms that arise from deploying models with fixed predictions. Our work introduces the first method to tackle this problem by leveraging tools from MIQCP to find confined \emph{boxes} within a region of the feature space. Our method provides interpretable descriptions of confined regions, can be run in seconds for real-world datasets, and enables data-free auditing of model responsiveness. However, these methods should be extended to address the following limitations:
\vspace{-1em}
\begin{itemize}[leftmargin=*, itemsep=0pt]
    \item Our methods are designed to work with linear classifiers. In principle, our methods can be extended to any model that can be represented by a MILP. However, many MILP-representable models (e.g., tree ensembles) would require solving a large, computationally-intractable MIQCP and need new algorithmic approaches.
    \item \us{} finds confined \emph{boxes}. Boxes are an interpretable way to characterize regions but have limited expressive power. Future work should explore whether more expressive classes capture confined regions with fewer items.
\end{itemize}