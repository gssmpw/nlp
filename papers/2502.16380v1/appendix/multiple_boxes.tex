\section{Generating multiple confined boxes} \label{app:multi_boxes}
Solving an instance of the RVP can generate \emph{single} confined box. However in practice, a given region may contain multiple confined boxes. To provide model developers or stakeholders with a holistic view of fixed predictions, the RVP can be run sequentially to enumerate multiple (or all) confined boxes with the region. We do so by adding \textit{no-good cuts} to the FCP that exclude previously discovered regions. This is equivalent to re-solving the RVP for a new region ${\cal R}' \subset {\cal R}$ that excludes existing confined boxes.

Let $\bar{\mathbf{u}}, \bar{\mathbf{l}}$ be an existing confined box. Let decision variables $\mathbf{z}_{u}, \mathbf{z}_{l} \in \{0, 1\}^d$ track whether the new box $\mathbf{u}, \mathbf{l}$ is outside the existing box. We model the no-good cuts that exclude $\bar{\mathbf{u}}, \bar{\mathbf{l}}$ via the following linear constraints:

\begin{subequations}
\begin{align}
	 &u_d \leq \bar{l}_d - 1 + (U_d - \bar{l}_d - 1)(1-z_{ud}) ~~ \forall d \in [d] \label{const:ub}\\
	& l_d \geq \bar{u}_d + 1 + (L_d - \bar{u}_d + 1)(1-z_{ld}) ~~ \forall d \in [d] \label{const:lb}\\
    &\sum_d z_{ud} + z_{ld} \geq 1 \label{const:z_sum} \\
    & \mathbf{z}_{u}, \mathbf{z}_{l} \in \{0, 1\}^d \label{const:z_bin} 
\end{align}
\end{subequations}

Constraint \eqref{const:ub} checks whether the upper bound for feature $d$ in the new box is less than the lower bound for feature $d$ in the existing box. Note that these constraints are enforced if $z_{ud} = 1$. Constraint \eqref{const:lb} checks an analogous condition for the lower bound of feature $d$. Constraints \eqref{const:z_sum}-\eqref{const:z_bin} ensure that at least one constraint of type \eqref{const:ub} or \eqref{const:lb} are active. In other words, it ensures that the new region must fall outside the previous region by ensuring that at least one feature differs in the new box. We exclude constraints for bounds that are tight with the population bounds (i.e., $u_d = U_d$ or $l_d = L_d$) as they are implied in the FCP. Given that our procedure tends to generate sparse regions (i.e., regions that change few features from their population upper and lower bounds), these no good-cuts typically add a very small number of binary variables and constraints to the full formulation.
