\section{REP Integer Programming Formulation} \label{app:rep_form}
Recall the following:
\begin{itemize}
    \item $\mathbf{x} \in \mathbb{R}^{d-q} \times \mathbb{Z}^q$ is a decision variable representing an  individual.
    \item $\mathbf{a} \in \mathbb{R}^{d-q} \times \mathbb{Z}^q$ is a decision variable representing an action.
    \item  $f(\mathbf{x}) = \text{sign}(\mathbf{w}^\top x + b)$ is the linear classifier where $f(\mathbf{x}) = 1$ is the desirable outcome.
    \item ${\cal X}$ represents the feature space.
    \item ${\cal R}$ represents the region of interest.
    \item $A(\mathbf{x})$ represents a set of feasible actions for a given individual $\mathbf{x}$ under the set of actionability constraints.
\end{itemize}

We assume that ${\cal X}$, ${\cal R}$, and $A$ can all be represented via a set of constraints in a MILP optimization model. We formulate the REP for a given region of interest ${\cal R}$ as the following MILP:

\begin{subequations}
\begin{align}
	&& \mathbf{w}^\top(\mathbf{x} + \mathbf{a}) &\geq b ~~&& \label{const:d_classification}\\
	&& \mathbf{x} + \mathbf{a} &\in {\cal X} && \label{const:d_feasible_end} \\
	&& \mathbf{x} &\in {\cal R} && \label{const:d_feasible_x} \\
	&& \mathbf{a} &\in A(x) && \label{const:d_feasible_action} \\
	&& \mathbf{x} &\in B(\mathbf{u}, \mathbf{l}) && \label{const:d_box_constr} \\
	%        \sum_{d} y_d \leq \beta \label{const:sparsity}\\
	&& \mathbf{x}, \mathbf{a} &\in \mathbb{R}^{d-m} \times \mathbb{Z}^m \label{const:d_binary}
\end{align}
\end{subequations}
Note that this problem has no objective as it is a \emph{feasibility} problem. 
Constraint \eqref{const:d_classification} ensures that the data point $\mathbf{x}$, after taking action $\mathbf{a}$, is classified with the desirable outcome. 
Constraints \eqref{const:d_feasible_x} and \eqref{const:d_feasible_end} ensure that $\mathbf{x}+\mathbf{a}$ if a valid feature vector (i.e., included in ${\cal X}$), and $\mathbf{x}$ is part of the region of interest. 
Constraint \eqref{const:d_feasible_action} ensures that the action $\mathbf{a}$ satisfies the actionability constraints. Finally Constraint \eqref{const:d_box_constr} ensures that $x$ is included in the box $B(u, l)$. Note that any solution to the REP represents a feasible $x \in B(u, l)$ with recourse. Thus the region $B(u,l)$ is \emph{confined} if and only if the REP is infeasible. 

\subsection{Continuous Restriction}
Recall that ${\cal C}$ is the set of \emph{continuous restrictions} of the REP, where a continuous restriction $c \in {\cal C}$ corresponds to the REP with fixed values for the discrete variables (e.g., $x_1 = 1, x_2 = 2$ for discrete variables ${\cal J}_D = \{x_1, x_2\}$). Let $v_j \in {\cal J}_D$ represent a discrete variable, and $s_j$ be its fixed value in continuous restriction $c$. We can incorporate continuous restrictions into the MILP formulation by adding the following additional constraints that fix the values of discrete variables in the formulation:
$$
v_j = s_j ~~\forall v_j \in {\cal J}_D
$$
Note that since all discrete values are fixed, this is not a \emph{linear} program over the $d-q$ continuous variables.

%Without loss of generality we represent the inequalities \eqref{const:classification}-\eqref{const:box_constr} in the following standard form:
%$$
%A\mathbf{x} + B\mathbf{a} \leq b(\mathbf{u}, \mathbf{l})
%$$
%where $A, B$ are $m \times d$ matrices and $b(u,l)$ is a $m$-dimensional vector. Here $m$ represents the number of constraints in the full PREP.

%\subsection{Auditing Individual Data Points}
%In this section, we simplify the PREP formulation to audit whether a given individual has recourse. 