\section{Practical Guidelines for Selecting Continuous Restrictions for the FCP} \label{app:relax_disc}

Unfortunately, Theorem \ref{thm:tum} does not hold directly under a broader set of constraints. Consider the following simple example over two features $x_1, x_2$. Let $x_1$ be a binary variable, and $x_2$ be an integer variable bounded between $0$ and $10$. Add one directional linkage constraint such that increasing $x_1$ by one unit causes $x_2$ to decrease by 10 units. The classifier assigns the desirable outcome if $x_1 + a_1 \geq 0.5$. Consider the region ${\cal R} = x_1 \times x_2 = \{0\} \times [5,10]$. Every individual in the region has \emph{continuous recourse} by setting $a_1 = 0.5$. However, in the discrete version of the REP, which requires $a_1 \in \{0,1\}$, no individual has recourse because setting $a_1 = 1$ would require violating the bound constraint $x_2 + a_2 \geq 0$.

One strategy to handle general constraints is to restrict a subset of discrete variables in the REP, such that each resulting continuous restriction meets the conditions of \cref{thm:tum}.
In practice, this strategy leads to a small number of continuous restrictions. For example, consider the \texttt{heloc} dataset used in our experiments, which has 42 binary features, one integer feature, and 20 non-separable constraints. It contains two constraints that are not linear recourse constraints:
\begin{itemize}
    \item Actions on \textfn{YearsSinceLastDelqTrade$\leq$3} will a 3 unit change in \textfn{YearsOfAccountHistory}
    \item Actions on \textfn{YearsSinceLastDelqTrade$\leq$5} will a 5 unit change in \textfn{YearsOfAccountHistory}
\end{itemize}
Note that these are not linear recourse constraints because a unit change in one variable results in a non-unit change in another. There are over $2^{42}$ possible continuous restrictions which makes solving the full FCP over every restriction computationally intractable. However, restricting \textfn{YearsSinceLastDelqTrade$\leq$3} and \textfn{YearsSinceLastDelqTrade$\leq$5} (i.e., fixing each variable to either $0$ or $1$) transforms the two violating constraints into integer bound constraints. This only generates four continuous restrictions ($2 \times 2$) but still allows the FCP to leverage \cref{thm:tum}. 

This strategy is sufficient for many actionability constraints available with public implementations. For example, all constraints implemented in the \textds{Reach} package \cite{kothari2023prediction} (e.g., if-then constraints, directional linkage constraints with non-unit scaling factors) only require restricting one discrete variable. An alternative approach in settings with a large number of non-separable constraints over discrete variables is to enumerate feasible feature vectors for the inter-connected discrete variables and re-formulate them as one feature which represents a categorical encoding over all possible values \citep[similar to the approach detailed in ][]{ustun2019actionable}. 

In settings where this is not possible or does not dramatically reduce the number of discrete variables our approach can also be run with a single continuous restriction corresponding to a linear relaxation of the original REP. Solving this \emph{relaxed} FCP can be used to provide weaker guarantees:
\begin{itemize}
    \item If the relaxed version of the FCP returns a confined box, this box is confined for the discrete version of the problem (any individual without continuous recourse also does not have recourse in the discrete problem).
    \item If the relaxed FCP certifies that the entire region is confined (i.e., all data points are assigned fixed predictions), then the entire region is guaranteed to be confined in the discrete version of the problem.
    \item Unfortunately, if the relaxed version of the FCP is infeasible there is no guarantee that the entire region is responsive. 
\end{itemize}
