\section{Supplementary Material for Experiments} \label{app:experiments}
\newlist{constraints}{enumerate}{3}
\setlist[constraints]{label={\arabic*.}}

In this section we present all the actionability constraints for the datasets used in our experiments. Note that the feature space ${\cal X}$ for each dataset has the same upper and lower bounds as well as non-separable constraints as the action set. All regions we audit in this paper are represented by boxes with fixed values for immutable features.

\subsection{Actionability Constraints for the \textds{heloc} Dataset} 

We show a list of all features and their separable actionability constraints in \cref{Table::IndividualActionSetHeloc}.
\begin{table}[!h]
\centering
\fontsize{9pt}{9pt}\selectfont
\resizebox{0.75\linewidth}{!}{\begin{tabular}{llllll}
\toprule
\textheader{Name} & \textheader{Type} & \textheader{LB} & \textheader{UB} & \textheader{Actionability} & \textheader{Sign} \\
\midrule
\textfn{ExternalRiskEstimate$\geq$40} & $\{0,1\}$ & 0 & 1 & No &  \\
\textfn{ExternalRiskEstimate$\geq$50} & $\{0,1\}$ & 0 & 1 & No &  \\
\textfn{ExternalRiskEstimate$\geq$60} & $\{0,1\}$ & 0 & 1 & No &  \\
\textfn{ExternalRiskEstimate$\geq$70} & $\{0,1\}$ & 0 & 1 & No &  \\
\textfn{ExternalRiskEstimate$\geq$80} & $\{0,1\}$ & 0 & 1 & No &  \\
\textfn{YearsOfAccountHistory} & $\mathbb{Z}$ & 0 & 50 & No &  \\
\textfn{AvgYearsInFile$\geq$3} & $\{0,1\}$ & 0 & 1 & Yes & + \\
\textfn{AvgYearsInFile$\geq$5} & $\{0,1\}$ & 0 & 1 & Yes & + \\
\textfn{AvgYearsInFile$\geq$7} & $\{0,1\}$ & 0 & 1 & Yes & + \\
\textfn{MostRecentTradeWithinLastYear} & $\{0,1\}$ & 0 & 1 & Yes &  \\
\textfn{MostRecentTradeWithinLast2Years} & $\{0,1\}$ & 0 & 1 & Yes &  \\
\textfn{AnyDerogatoryComment} & $\{0,1\}$ & 0 & 1 & No &  \\
\textfn{AnyTrade120DaysDelq} & $\{0,1\}$ & 0 & 1 & No &  \\
\textfn{AnyTrade90DaysDelq} & $\{0,1\}$ & 0 & 1 & No &  \\
\textfn{AnyTrade60DaysDelq} & $\{0,1\}$ & 0 & 1 & No &  \\
\textfn{AnyTrade30DaysDelq} & $\{0,1\}$ & 0 & 1 & No &  \\
\textfn{NoDelqEver} & $\{0,1\}$ & 0 & 1 & No &  \\
\textfn{YearsSinceLastDelqTrade$\leq$1} & $\{0,1\}$ & 0 & 1 & Yes &  \\
\textfn{YearsSinceLastDelqTrade$\leq$3} & $\{0,1\}$ & 0 & 1 & Yes &  \\
\textfn{YearsSinceLastDelqTrade$\leq$5} & $\{0,1\}$ & 0 & 1 & Yes &  \\
\textfn{NumInstallTrades$\geq$2} & $\{0,1\}$ & 0 & 1 & Yes & + \\
\textfn{NumInstallTrades$\geq$3} & $\{0,1\}$ & 0 & 1 & Yes & + \\
\textfn{NumInstallTrades$\geq$5} & $\{0,1\}$ & 0 & 1 & Yes & + \\
\textfn{NumInstallTrades$\geq$7} & $\{0,1\}$ & 0 & 1 & Yes & + \\
\textfn{NumInstallTradesWBalance$\geq$2} & $\{0,1\}$ & 0 & 1 & Yes & + \\
\textfn{NumInstallTradesWBalance$\geq$3} & $\{0,1\}$ & 0 & 1 & Yes & + \\
\textfn{NumInstallTradesWBalance$\geq$5} & $\{0,1\}$ & 0 & 1 & Yes & + \\
\textfn{NumInstallTradesWBalance$\geq$7} & $\{0,1\}$ & 0 & 1 & Yes & + \\
\textfn{NumRevolvingTrades$\geq$2} & $\{0,1\}$ & 0 & 1 & Yes & + \\
\textfn{NumRevolvingTrades$\geq$3} & $\{0,1\}$ & 0 & 1 & Yes & + \\
\textfn{NumRevolvingTrades$\geq$5} & $\{0,1\}$ & 0 & 1 & Yes & + \\
\textfn{NumRevolvingTrades$\geq$7} & $\{0,1\}$ & 0 & 1 & Yes & + \\
\textfn{NumRevolvingTradesWBalance$\geq$2} & $\{0,1\}$ & 0 & 1 & Yes & + \\
\textfn{NumRevolvingTradesWBalance$\geq$3} & $\{0,1\}$ & 0 & 1 & Yes & + \\
\textfn{NumRevolvingTradesWBalance$\geq$5} & $\{0,1\}$ & 0 & 1 & Yes & + \\
\textfn{NumRevolvingTradesWBalance$\geq$7} & $\{0,1\}$ & 0 & 1 & Yes & + \\
\textfn{NetFractionInstallBurden$\geq$10} & $\{0,1\}$ & 0 & 1 & Yes & + \\
\textfn{NetFractionInstallBurden$\geq$20} & $\{0,1\}$ & 0 & 1 & Yes & + \\
\textfn{NetFractionInstallBurden$\geq$50} & $\{0,1\}$ & 0 & 1 & Yes & + \\
\textfn{NetFractionRevolvingBurden$\geq$10} & $\{0,1\}$ & 0 & 1 & Yes &  \\
\textfn{NetFractionRevolvingBurden$\geq$20} & $\{0,1\}$ & 0 & 1 & Yes &  \\
\textfn{NetFractionRevolvingBurden$\geq$50} & $\{0,1\}$ & 0 & 1 & Yes &  \\
\textfn{NumBank2NatlTradesWHighUtilization$\geq$2} & $\{0,1\}$ & 0 & 1 & Yes & + \\
\bottomrule
\end{tabular}
}
\caption{Separable actionability constraints for the \textds{heloc} dataset.}
\label{Table::IndividualActionSetHeloc}
\end{table}

The non-separable actionability constraints for this dataset include:
\begin{constraints}
\item DirectionalLinkage: Actions on \textfn{NumRevolvingTradesWBalance$\geq$2} will induce to actions on \textfn{NumRevolvingTrades$\geq$2}.Each unit change in \textfn{NumRevolvingTradesWBalance$\geq$2} leads to:1.00-unit change in \textfn{NumRevolvingTrades$\geq$2}
\item DirectionalLinkage: Actions on \textfn{NumInstallTradesWBalance$\geq$2} will induce to actions on \textfn{NumInstallTrades$\geq$2}.Each unit change in \textfn{NumInstallTradesWBalance$\geq$2} leads to:1.00-unit change in \textfn{NumInstallTrades$\geq$2}
\item DirectionalLinkage: Actions on \textfn{NumRevolvingTradesWBalance$\geq$3} will induce to actions on \textfn{NumRevolvingTrades$\geq$3}.Each unit change in \textfn{NumRevolvingTradesWBalance$\geq$3} leads to:1.00-unit change in \textfn{NumRevolvingTrades$\geq$3}
\item DirectionalLinkage: Actions on \textfn{NumInstallTradesWBalance$\geq$3} will induce to actions on \textfn{NumInstallTrades$\geq$3}. Each unit change in \textfn{NumInstallTradesWBalance$\geq$3} leads to:1.00-unit change in \textfn{NumInstallTrades$\geq$3}
\item DirectionalLinkage: Actions on \textfn{NumRevolvingTradesWBalance$\geq$5} will induce to actions on \textfn{NumRevolvingTrades$\geq$5}.Each unit change in \textfn{NumRevolvingTradesWBalance$\geq$5} leads to:1.00-unit change in \textfn{NumRevolvingTrades$\geq$5}
\item DirectionalLinkage: Actions on \textfn{NumInstallTradesWBalance$\geq$5} will induce to actions on \textfn{NumInstallTrades$\geq$5}.Each unit change in \textfn{NumInstallTradesWBalance$\geq$5} leads to:1.00-unit change in \textfn{NumInstallTrades$\geq$5}
\item DirectionalLinkage: Actions on \textfn{NumRevolvingTradesWBalance$\geq$7} will induce to actions on \textfn{NumRevolvingTrades$\geq$7}.Each unit change in \textfn{NumRevolvingTradesWBalance$\geq$7} leads to:1.00-unit change in \textfn{NumRevolvingTrades$\geq$7}
\item DirectionalLinkage: Actions on \textfn{NumInstallTradesWBalance$\geq$7} will induce to actions on \textfn{NumInstallTrades$\geq$7}.Each unit change in \textfn{NumInstallTradesWBalance$\geq$7} leads to:1.00-unit change in \textfn{NumInstallTrades$\geq$7}
\item DirectionalLinkage: Actions on \textfn{YearsSinceLastDelqTrade$\leq$1} will induce to actions on \textfn{YearsOfAccountHistory}. Each unit change in \textfn{YearsSinceLastDelqTrade$\leq$1} leads to:-1.00-unit change in \textfn{YearsOfAccountHistory}
\item DirectionalLinkage: Actions on \textfn{YearsSinceLastDelqTrade$\leq$3} will induce to actions on \textfn{YearsOfAccountHistory}. Each unit change in \textfn{YearsSinceLastDelqTrade$\leq$3} leads to:-3.00-unit change in \textfn{YearsOfAccountHistory}
\item DirectionalLinkage: Actions on \textfn{YearsSinceLastDelqTrade$\leq$5} will induce to actions on \textfn{YearsOfAccountHistory}. Each unit change in \textfn{YearsSinceLastDelqTrade$\leq$5} leads to:-5.00-unit change in \textfn{YearsOfAccountHistory}
\item ThermometerEncoding: Actions on [\textfn{YearsSinceLastDelqTrade$\leq$1}, \textfn{YearsSinceLastDelqTrade$\leq$3}, \textfn{YearsSinceLastDelqTrade$\leq$5}] must preserve thermometer encoding., which can only decrease.Actions can only turn off higher-level dummies that are on, where \textfn{YearsSinceLastDelqTrade$\leq$1} is the lowest-level dummy and \textfn{YearsSinceLastDelqTrade$\leq$5} is the highest-level-dummy.
\item ThermometerEncoding: Actions on [\textfn{MostRecentTradeWithinLast2Years}, \textfn{MostRecentTradeWithinLastYear}] must preserve thermometer encoding.
\item ThermometerEncoding: Actions on [\textfn{AvgYearsInFile$\geq$3}, \textfn{AvgYearsInFile$\geq$5}, \textfn{AvgYearsInFile$\geq$7}] must preserve thermometer encoding., which can only increase.Actions can only turn on higher-level dummies that are off, where \textfn{AvgYearsInFile$\geq$3} is the lowest-level dummy and \textfn{AvgYearsInFile$\geq$7} is the highest-level-dummy.
\item ThermometerEncoding: Actions on [\textfn{NetFractionRevolvingBurden$\geq$10}, \textfn{NetFractionRevolvingBurden$\geq$20}, \textfn{NetFractionRevolvingBurden$\geq$50}] must preserve thermometer encoding., which can only decrease.Actions can only turn off higher-level dummies that are on, where \textfn{NetFractionRevolvingBurden$\geq$10} is the lowest-level dummy and \textfn{NetFractionRevolvingBurden$\geq$50} is the highest-level-dummy.
\item ThermometerEncoding: Actions on [\textfn{NetFractionInstallBurden$\geq$10}, \textfn{NetFractionInstallBurden$\geq$20}, \textfn{NetFractionInstallBurden$\geq$50}] must preserve thermometer encoding., which can only decrease.Actions can only turn off higher-level dummies that are on, where \textfn{NetFractionInstallBurden$\geq$10} is the lowest-level dummy and \textfn{NetFractionInstallBurden$\geq$50} is the highest-level-dummy.
\item ThermometerEncoding: Actions on [\textfn{NumRevolvingTradesWBalance$\geq$2}, \textfn{NumRevolvingTradesWBalance$\geq$3}, \textfn{NumRevolvingTradesWBalance$\geq$5}, \textfn{NumRevolvingTradesWBalance$\geq$7}] must preserve thermometer encoding., which can only decrease.Actions can only turn off higher-level dummies that are on, where \textfn{NumRevolvingTradesWBalance$\geq$2} is the lowest-level dummy and \textfn{NumRevolvingTradesWBalance$\geq$7} is the highest-level-dummy.
\item ThermometerEncoding: Actions on [\textfn{NumRevolvingTrades$\geq$2}, \textfn{NumRevolvingTrades$\geq$3}, \textfn{NumRevolvingTrades$\geq$5}, \textfn{NumRevolvingTrades$\geq$7}] must preserve thermometer encoding., which can only decrease.Actions can only turn off higher-level dummies that are on, where \textfn{NumRevolvingTrades$\geq$2} is the lowest-level dummy and \textfn{NumRevolvingTrades$\geq$7} is the highest-level-dummy.
\item ThermometerEncoding: Actions on [\textfn{NumInstallTradesWBalance$\geq$2}, \textfn{NumInstallTradesWBalance$\geq$3}, \textfn{NumInstallTradesWBalance$\geq$5}, \textfn{NumInstallTradesWBalance$\geq$7}] must preserve thermometer encoding., which can only decrease.Actions can only turn off higher-level dummies that are on, where \textfn{NumInstallTradesWBalance$\geq$2} is the lowest-level dummy and \textfn{NumInstallTradesWBalance$\geq$7} is the highest-level-dummy.
\item ThermometerEncoding: Actions on [\textfn{NumInstallTrades$\geq$2}, \textfn{NumInstallTrades$\geq$3}, \textfn{NumInstallTrades$\geq$5}, \textfn{NumInstallTrades$\geq$7}] must preserve thermometer encoding., which can only decrease.Actions can only turn off higher-level dummies that are on, where \textfn{NumInstallTrades$\geq$2} is the lowest-level dummy and \textfn{NumInstallTrades$\geq$7} is the highest-level-dummy.
\end{constraints}
\clearpage
\subsection{Actionability Constraints for the \textds{givemecredit} Dataset} 

We present a list of all features and their separable actionability constraints in \cref{Table::IndividualActionSetGiveMeCredit}.
\begin{table}[!h]
\centering
\fontsize{9pt}{9pt}\selectfont
\resizebox{0.75\linewidth}{!}{\begin{tabular}{llllll}
\toprule
\textheader{Name} & \textheader{Type} & \textheader{LB} & \textheader{UB} & \textheader{Actionability} & \textheader{Sign} \\
\midrule
\textfn{Age} & $\mathbb{Z}$ & 21 & 103 & No &  \\
\textfn{NumberOfDependents} & $\mathbb{Z}$ & 0 & 20 & No &  \\
\textfn{DebtRatio$\geq$1} & $\{0,1\}$ & 0 & 1 & Yes & + \\
\textfn{MonthlyIncome$\geq$3K} & $\{0,1\}$ & 0 & 1 & Yes & + \\
\textfn{MonthlyIncome$\geq$5K} & $\{0,1\}$ & 0 & 1 & Yes & + \\
\textfn{MonthlyIncome$\geq$10K} & $\{0,1\}$ & 0 & 1 & Yes & + \\
\textfn{CreditLineUtilization$\geq$10} & $\{0,1\}$ & 0 & 1 & Yes &  \\
\textfn{CreditLineUtilization$\geq$20} & $\{0,1\}$ & 0 & 1 & Yes &  \\
\textfn{CreditLineUtilization$\geq$50} & $\{0,1\}$ & 0 & 1 & Yes &  \\
\textfn{CreditLineUtilization$\geq$70} & $\{0,1\}$ & 0 & 1 & Yes &  \\
\textfn{CreditLineUtilization$\geq$100} & $\{0,1\}$ & 0 & 1 & Yes &  \\
\textfn{AnyRealEstateLoans} & $\{0,1\}$ & 0 & 1 & Yes & + \\
\textfn{MultipleRealEstateLoans} & $\{0,1\}$ & 0 & 1 & Yes & + \\
\textfn{AnyCreditLinesAndLoans} & $\{0,1\}$ & 0 & 1 & Yes & + \\
\textfn{MultipleCreditLinesAndLoans} & $\{0,1\}$ & 0 & 1 & Yes & + \\
\textfn{HistoryOfLatePayment} & $\{0,1\}$ & 0 & 1 & No &  \\
\textfn{HistoryOfDelinquency} & $\{0,1\}$ & 0 & 1 & No &  \\
\bottomrule
\end{tabular}
}
\caption{Separable actionability constraints for the \textds{givemecredit} dataset.}
\label{Table::IndividualActionSetGiveMeCredit}
\end{table}

The non-separable actionability constraints for this dataset include:
\begin{constraints}
\item ThermometerEncoding: Actions on [\textfn{MonthlyIncome$\geq$3K}, \textfn{MonthlyIncome$\geq$5K}, \textfn{MonthlyIncome$\geq$10K}] must preserve thermometer encoding., which can only increase.Actions can only turn on higher-level dummies that are off, where \textfn{MonthlyIncome$\geq$3K} is the lowest-level dummy and \textfn{MonthlyIncome$\geq$10K} is the highest-level-dummy.
\item ThermometerEncoding: Actions on [\textfn{CreditLineUtilization$\geq$10}, \textfn{CreditLineUtilization$\geq$20}, \textfn{CreditLineUtilization$\geq$50}, \textfn{CreditLineUtilization$\geq$70}, \textfn{CreditLineUtilization$\geq$100}] must preserve thermometer encoding., which can only decrease.Actions can only turn off higher-level dummies that are on, where \textfn{CreditLineUtilization$\geq$10} is the lowest-level dummy and \textfn{CreditLineUtilization$\geq$100} is the highest-level-dummy.
\item ThermometerEncoding: Actions on [\textfn{AnyRealEstateLoans}, \textfn{MultipleRealEstateLoans}] must preserve thermometer encoding., which can only decrease.Actions can only turn off higher-level dummies that are on, where \textfn{AnyRealEstateLoans} is the lowest-level dummy and \textfn{MultipleRealEstateLoans} is the highest-level-dummy.
\item ThermometerEncoding: Actions on [\textfn{AnyCreditLinesAndLoans}, \textfn{MultipleCreditLinesAndLoans}] must preserve thermometer encoding., which can only decrease.Actions can only turn off higher-level dummies that are on, where \textfn{AnyCreditLinesAndLoans} is the lowest-level dummy and \textfn{MultipleCreditLinesAndLoans} is the highest-level-dummy.
\end{constraints}

\clearpage
\subsection{Actionability Constraints for the \textds{twitterbot} Dataset} 

We present a list of all features and their separable actionability constraints in \cref{Table::IndividualActionSetTwitterbot}.
\begin{table}[!h]
\centering
\fontsize{9pt}{9pt}\selectfont
\resizebox{0.75\linewidth}{!}{\begin{tabular}{llllll}
\toprule
\textheader{Name} & \textheader{Type} & \textheader{LB} & \textheader{UB} & \textheader{Actionability} & \textheader{Sign} \\
\midrule
\textfn{SourceAutomation} & $\{0,1\}$ & 0 & 1 & No &  \\
\textfn{SourceOther} & $\{0,1\}$ & 0 & 1 & No &  \\
\textfn{SourceBranding} & $\{0,1\}$ & 0 & 1 & No &  \\
\textfn{SourceMobile} & $\{0,1\}$ & 0 & 1 & No &  \\
\textfn{SourceWeb} & $\{0,1\}$ & 0 & 1 & No &  \\
\textfn{SourceApp} & $\{0,1\}$ & 0 & 1 & No &  \\
\textfn{FollowerFriendRatio$\geq$1} & $\{0,1\}$ & 0 & 1 & No &  \\
\textfn{FollowerFriendRatio$\geq$10} & $\{0,1\}$ & 0 & 1 & No &  \\
\textfn{FollowerFriendRatio$\geq$100} & $\{0,1\}$ & 0 & 1 & No &  \\
\textfn{FollowerFriendRatio$\geq$1000} & $\{0,1\}$ & 0 & 1 & No &  \\
\textfn{FollowerFriendRatio$\geq$10000} & $\{0,1\}$ & 0 & 1 & No &  \\
\textfn{FollowerFriendRatio$\geq$100000} & $\{0,1\}$ & 0 & 1 & No &  \\
\textfn{AgeOfAccountInDays$\geq$365} & $\{0,1\}$ & 0 & 1 & Yes &  \\
\textfn{AgeOfAccountInDays$\geq$730} & $\{0,1\}$ & 0 & 1 & Yes &  \\
\textfn{UserReplied$\geq$10} & $\{0,1\}$ & 0 & 1 & Yes &  \\
\textfn{UserReplied$\geq$100} & $\{0,1\}$ & 0 & 1 & Yes &  \\
\textfn{UserFavourited$\geq$1000} & $\{0,1\}$ & 0 & 1 & Yes &  \\
\textfn{UserFavourited$\geq$10000} & $\{0,1\}$ & 0 & 1 & Yes &  \\
\textfn{UserRetweeted$\geq$1} & $\{0,1\}$ & 0 & 1 & Yes &  \\
\textfn{UserRetweeted$\geq$10} & $\{0,1\}$ & 0 & 1 & Yes &  \\
\textfn{UserRetweeted$\geq$100} & $\{0,1\}$ & 0 & 1 & Yes &  \\
\bottomrule
\end{tabular}
}
\caption{Separable actionability constraints for the \textds{Twitterbot} dataset.}
\label{Table::IndividualActionSetTwitterbot}
\end{table}

The non-separable actionability constraints for this dataset include:
\begin{constraints}
\item ThermometerEncoding: Actions on [\textfn{FollowerFriendRatio$\geq$1}, \textfn{FollowerFriendRatio$\geq$10}, \textfn{FollowerFriendRatio$\geq$100}, \textfn{FollowerFriendRatio$\geq$1000}, \textfn{FollowerFriendRatio$\geq$10000}, 

\textfn{FollowerFriendRatio$\geq$100000}] must preserve thermometer encoding.
\item ThermometerEncoding: Actions on [\textfn{AgeOfAccountInDays$\geq$365}, \textfn{AgeOfAccountInDays$\geq$730}] must preserve thermometer encoding.
\item ThermometerEncoding: Actions on [\textfn{UserReplied$\geq$10}, \textfn{UserReplied$\geq$100}] must preserve thermometer encoding.
\item ThermometerEncoding: Actions on [\textfn{UserFavourited$\geq$1000}, \textfn{UserFavourited$\geq$10000}] must preserve thermometer encoding.
\item ThermometerEncoding: Actions on [\textfn{UserRetweeted$\geq$1}, \textfn{UserRetweeted$\geq$10}, \textfn{UserRetweeted$\geq$100}] must preserve thermometer encoding.
\end{constraints}

\subsection{Computing Infrastructure}
We run all experiments on a personal computer with an Apple M1 Pro chip and 32 GB of RAM. All MILP and MIQCP problems were solved using Gurobi 9.0 \cite{achterberg2019gurobi} with default settings.



\subsection{Additional Results}
\cref{tab:full_results} shows an extended version of our main results that include three additional metrics:
\begin{itemize}
    \item \textbf{Realized Blindspot Rate}: The fraction of total regions that are predicted to be responsive but contain individuals with fixed predictions in the test dataset.
    \item \textbf{Realized Loophole Rate}: The fraction of total regions that are predicted to be confined but contain individuals with recourse in the test dataset.
    \item \textbf{Computation Time}: The average computation time in seconds over all regions for each dataset.
\end{itemize}

As expected, the \emph{realized} blindspot and loophole rates are lower than the true blindspot and loophole rate. Recourse verification over regions safeguards against rare events (i.e., new individuals with fixed predictions that were not part of the training dataset), which makes the risks less likely to be realized over a small test dataset. However in real-world applications, models are typically deployed and make predictions on datasets much larger than its original training dataset --- increasing the probability of blindspots and loopholes being realized. 

Our extended results also highlight that our approach runs incredibly quickly across all three datasets, auditing a region in under 5 seconds on average. In two out of three datasets, our approach is quicker than the \baseline{Region} which runs pointwise recourse on 100 data points.

\begin{table}[t!]
    \centering
    %\resizebox{0.9\textwidth}{!}{

% \begin{table*}[!th]
%     \centering
%     \resizebox{0.9\textwidth}{!}{
%     \begin{tabular}{|l|r|r|r|l|r|r|r|} 
%        \cline{1-4} \cline{6-8}
%         & \multicolumn{3}{c|}{Retrieve} & &\multicolumn{3}{c|}{Retrieve + Rerank} \\ 
%                \cline{2-4} \cline{6-8}
%                &  \multicolumn{3}{c|}{Hits@10}   & & \multicolumn{2}{c|}{Hits@10}  & Speedup \\
% Dataset  &  Baseline  &  DE-2 Rand  & DE-2 CE  & & DE-2 CE &  Baseline &  \\
%        \cline{1-4} \cline{6-8}
% all nli \cite{bowman-etal-2015-large},\cite{N18-1101}  & 0.77 & 0.59 & 0.75 &  &\textbf{ 0.84} & 0.83 & 4.5x\\
% eli5 \cite{fan-etal-2019-eli5} & 0.43 & 0.12 & 0.29 &  & 0.43 & 0.49 & 5.4x \\
% gooaq \cite{Khashabi2021GooAQOQ} & 0.75 & 0.42 & 0.64 &  & 0.77 & 0.80 & 5.8x\\
% msmarco \cite{nguyen2016ms} & 0.95 & 0.81 & 0.89 &  & 0.96 & 0.98 & 5.1x \\
% \href{https://quoradata.quora.com/First-Quora-Dataset-Release-Question-Pairs}{quora duplicates}  & 0.68 & 0.48 & 0.65 &  &\textbf{ 0.68} & 0.68 & 5.1x\\
% natural ques. \cite{47761} & 0.77 & 0.37 & 0.61 &  & 0.61 & 0.64 & 5.4x \\
% sentence comp \cite{filippova-altun-2013-overcoming} & 0.95 & 0.83 & 0.93 &  & \textbf{0.97 }& 0.97 & 6.6x\\
% simplewiki \cite{coster-kauchak-2011-simple} & 0.97 & 0.93 & 0.97 &  &\textbf{ 0.97 }& 0.96 & 4.8x\\
% stsb \cite{cer-etal-2017-semeval} & 0.97 & 0.87 & 0.97 &  & \textbf{0.98} & 0.98 & 4.7x \\
% zeshel \cite{logeswaran2019zero}  & 0.22 & 0.16 & 0.20 &  & \textbf{0.21} & 0.19 & 4.8x\\
%        \cline{1-4} \cline{6-8}
%     \end{tabular}
%     }

% Please add the following required packages to your document preamble:
% \usepackage[table,xcdraw]{xcolor}
% Beamer presentation requires \usepackage{colortbl} instead of \usepackage[table,xcdraw]{xcolor}
\begin{table*}[!th]
\centering
     \resizebox{1\textwidth}{!}{ \begin{tabular}{|l|llllll|l|lllll|}
\cline{1-7} \cline{9-13}
\textbf{}                  & \multicolumn{6}{c|}{\textbf{Retrieve}}                                                                                                                                                                                         & \textbf{} & \multicolumn{5}{c|}{\textbf{Retrieve + Rerank}}                                                                                                                                    \\ 
\cline{2-7} \cline{9-13}
{\textbf{dataset}}           & \multicolumn{3}{c|}{\textbf{Hits@10}}                                                                                   & \multicolumn{3}{c|}{\textbf{MRR@10}}                                                                & \textbf{} & \multicolumn{2}{c|}{\textbf{Hits@10}}                                         & \multicolumn{2}{c|}{\textbf{MRR@10}}                                           & \textbf{Speedup} \\ 

                           
\cline{2-7} \cline{9-13} 
                           & \multicolumn{1}{l|}{\textbf{Baseline}} & \multicolumn{1}{l|}{\textbf{DE-2-Rand}} & \multicolumn{1}{l|}{\textbf{DE-2-CE}} & \multicolumn{1}{l|}{\textbf{Baseline}} & \multicolumn{1}{l|}{\textbf{DE-2-Rand}} & \textbf{DE-2-CE} & \textbf{} & \multicolumn{1}{l|}{\textbf{Baseline}} & \multicolumn{1}{l|}{\textbf{DE-2-CE}} & \multicolumn{1}{l|}{\textbf{Baseline}} & \multicolumn{1}{l|}{\textbf{DE-2-CE}} & \textbf{}        \\ 
                       \cline{1-7} \cline{9-13}
                       
msmarco/dev/small  & \multicolumn{1}{l|}{0.59}              & \multicolumn{1}{l|}{0.37}               & \multicolumn{1}{l|}{0.45}             & \multicolumn{1}{l|}{0.32}              & \multicolumn{1}{l|}{0.19}               & 0.24             &           & \multicolumn{1}{l|}{0.66}              & \multicolumn{1}{l|}{0.59}             & \multicolumn{1}{l|}{0.38}              & \multicolumn{1}{l|}{0.35}             & 5.1x             \\ \cline{1-7} \cline{9-13}
beir/quora/dev             & \multicolumn{1}{l|}{0.95}              & \multicolumn{1}{l|}{0.90}               & \multicolumn{1}{l|}{0.93}             & \multicolumn{1}{l|}{0.85}              & \multicolumn{1}{l|}{0.76}               & 0.81             &           & \multicolumn{1}{l|}{\textbf{0.96}}     & \multicolumn{1}{l|}{\textbf{0.96}}    & \multicolumn{1}{l|}{\textbf{0.74}}     & \multicolumn{1}{l|}{\textbf{0.74}}    & 5.3x             \\ \cline{1-7} \cline{9-13}
beir/scifact/test          & \multicolumn{1}{l|}{0.63}              & \multicolumn{1}{l|}{0.53}               & \multicolumn{1}{l|}{0.55}             & \multicolumn{1}{l|}{0.42}              & \multicolumn{1}{l|}{0.30}               & 0.36             &           & \multicolumn{1}{l|}{0.72}              & \multicolumn{1}{l|}{0.69}             & \multicolumn{1}{l|}{0.57}              & \multicolumn{1}{l|}{0.55}             & 4.7x             \\ \cline{1-7} \cline{9-13}
beir/fiqa/dev              & \multicolumn{1}{l|}{0.50}              & \multicolumn{1}{l|}{0.22}               & \multicolumn{1}{l|}{0.32}             & \multicolumn{1}{l|}{0.32}              & \multicolumn{1}{l|}{0.12}               & 0.18             &           & \multicolumn{1}{l|}{0.59}              & \multicolumn{1}{l|}{0.46}             & \multicolumn{1}{l|}{0.28}              & \multicolumn{1}{l|}{0.23}             & 5.3x             \\ \cline{1-7} \cline{9-13}
zeshel/test                & \multicolumn{1}{l|}{0.22}              & \multicolumn{1}{l|}{0.16}               & \multicolumn{1}{l|}{0.20}             & \multicolumn{1}{l|}{0.12}              & \multicolumn{1}{l|}{0.09}               & 0.11             &           & \multicolumn{1}{l|}{\textbf{0.20}}     & \multicolumn{1}{l|}{\textbf{0.19}}    & \multicolumn{1}{l|}{\textbf{0.11}}     & \multicolumn{1}{l|}{\textbf{0.10}}    & 4.8x             \\ \cline{1-7} \cline{9-13}
stsb/train                 & \multicolumn{1}{l|}{0.97}              & \multicolumn{1}{l|}{0.95}               & \multicolumn{1}{l|}{0.97}             & \multicolumn{1}{l|}{0.85}              & \multicolumn{1}{l|}{0.83}               & 0.85             &           & \multicolumn{1}{l|}{\textbf{0.98}}     & \multicolumn{1}{l|}{\textbf{0.98}}    & \multicolumn{1}{l|}{\textbf{0.88}}     & \multicolumn{1}{l|}{\textbf{0.88}}    & 4.7x             \\ \cline{1-7} \cline{9-13}
all-nli/train              & \multicolumn{1}{l|}{0.49}              & \multicolumn{1}{l|}{0.41}               & \multicolumn{1}{l|}{0.47}             & \multicolumn{1}{l|}{0.39}              & \multicolumn{1}{l|}{0.33}               & 0.36             &           & \multicolumn{1}{l|}{\textbf{0.55}}     & \multicolumn{1}{l|}{\textbf{0.54}}    & \multicolumn{1}{l|}{\textbf{0.47}}     & \multicolumn{1}{l|}{\textbf{0.46}}    & 4.5x             \\ \cline{1-7} \cline{9-13}
simplewiki/train           & \multicolumn{1}{l|}{0.97}              & \multicolumn{1}{l|}{0.98}               & \multicolumn{1}{l|}{0.98}             & \multicolumn{1}{l|}{0.92}              & \multicolumn{1}{l|}{0.93}               & 0.94             &           & \multicolumn{1}{l|}{\textbf{0.96}}     & \multicolumn{1}{l|}{\textbf{0.97}}    & \multicolumn{1}{l|}{\textbf{0.91}}     & \multicolumn{1}{l|}{\textbf{0.91}}    & 4.8x             \\ \cline{1-7} \cline{9-13}
natural-questions/train    & \multicolumn{1}{l|}{0.77}              & \multicolumn{1}{l|}{0.59}               & \multicolumn{1}{l|}{0.65}             & \multicolumn{1}{l|}{0.51}              & \multicolumn{1}{l|}{0.37}               & 0.42             &           & \multicolumn{1}{l|}{0.65}              & \multicolumn{1}{l|}{0.61}             & \multicolumn{1}{l|}{0.39}              & \multicolumn{1}{l|}{0.37}             & 5.4x             \\ \cline{1-7} \cline{9-13}
eli5/train                 & \multicolumn{1}{l|}{0.32}              & \multicolumn{1}{l|}{0.14}               & \multicolumn{1}{l|}{0.22}             & \multicolumn{1}{l|}{0.19}              & \multicolumn{1}{l|}{0.08}               & 0.12             &           & \multicolumn{1}{l|}{0.39}              & \multicolumn{1}{l|}{0.32}             & \multicolumn{1}{l|}{0.26}              & \multicolumn{1}{l|}{0.22}             & 5.4x             \\ \cline{1-7} \cline{9-13}
sentence compression/train & \multicolumn{1}{l|}{0.93}              & \multicolumn{1}{l|}{0.89}               & \multicolumn{1}{l|}{0.93}             & \multicolumn{1}{l|}{0.85}              & \multicolumn{1}{l|}{0.79}               & 0.84             &           & \multicolumn{1}{l|}{\textbf{0.96}}     & \multicolumn{1}{l|}{\textbf{0.96}}    & \multicolumn{1}{l|}{\textbf{0.93}}     & \multicolumn{1}{l|}{\textbf{0.94}}    & 6.6x             \\ \cline{1-7} \cline{9-13}
gooaq/train                & \multicolumn{1}{l|}{0.73}              & \multicolumn{1}{l|}{0.68}               & \multicolumn{1}{l|}{0.72}             & \multicolumn{1}{l|}{0.58}              & \multicolumn{1}{l|}{0.55}               & 0.58             &           & \multicolumn{1}{l|}{0.80}              & \multicolumn{1}{l|}{0.76}             & \multicolumn{1}{l|}{0.64}              & \multicolumn{1}{l|}{0.62}             & 5.2x             \\ \cline{1-7} \cline{9-13}
\end{tabular}
}
    \caption{Comparison of CE infused DE. \textbf{Baseline} is the pre-trained DE SBERT model.  \textbf{DE-2 Rand} is a trained two-layered DE initialized with random initial weights. \textbf{DE-2 CE} is a two-layered DE infused with initial weights from CE as explained in Fig~\ref{fig:dual_cross} and trained similar to DE-2. All the above models are trained on msmarco. Bold face numbers for DE-2 CE show where performance is at least within .01 of the baseline DE. \textbf{Speedup} is inference time gain for DE-2 CE over Baseline.}
    % \textbf{Retrieve + Rerank}: Columns 5-6 present Accuracy@10 and columns 7-8 present number of documents encoded per sec. Here, the documents retrieved using baseline and our approach are reranked using a CE.}
    \label{tab:comp}
\end{table*} }
    \resizebox{0.7\textwidth}{!}{\begin{table}[t]
\centering
\footnotesize
\resizebox{\columnwidth}{!}{
\begin{tabular}{lccccc}
\toprule
 & \multicolumn{5}{c}{HotpotQA} \\
Model & 8K & 16K & 32K & 64K & 128K \\
\midrule
Llama3.1-8B               & 63.3 & 56.7 & 61.1 & 47.8 & 40.0 \\
Llama3.1-8B+step-by-step  & 60.0 & 66.7 & 56.7 & 58.9 & 56.7 \\
Llama3.1-8B+plan\&solve   & 71.1 & 66.7 & 72.2 & 62.2 & 50.0 \\
Llama3.1-8B+fact\&reflect & 58.9 & 58.9 & 62.2 & 61.1 & 48.9 \\
ProLong-8B                & 62.2 & 65.6 & 57.8 & 53.3 & 58.9 \\
Llama3.1-8B+LongRAG       & 61.1 & 58.9 & 73.3 & 56.7 & 57.8 \\
\method-8B                & 81.1 & 75.6 & 78.9 & 75.6 & 71.1 \\
\bottomrule
\end{tabular}}
\caption{Long-context performance on HotpotQA.}
\label{tab:hotpotqa}
\end{table}

\begin{table}[t]
\centering
\footnotesize
\resizebox{\columnwidth}{!}{
\begin{tabular}{lccccc}
\toprule
 & \multicolumn{5}{c}{Nature Questions} \\
Model & 8K & 16K & 32K & 64K & 128K \\
\midrule
Llama3.1-8B               & 71.7 & 69.4 & 70.6 & 73.9 & 56.1 \\
Llama3.1-8B+step-by-step  & 66.7 & 66.1 & 58.9 & 55.6 & 38.9 \\
Llama3.1-8B+plan\&solve   & 67.8 & 71.7 & 66.7 & 62.2 & 50.6 \\
Llama3.1-8B+fact\&reflect & 63.3 & 63.3 & 61.7 & 59.4 & 40.0 \\
ProLong-8B                & 83.3 & 82.2 & 83.9 & 90.0 & 77.8 \\
Llama3.1-8B+LongRAG       & 65.6 & 76.1 & 79.4 & 77.2 & 73.9 \\
\method-8B                & 91.7 & 91.1 & 85.0 & 85.0 & 77.8 \\
\bottomrule
\end{tabular}}
\caption{Long-context performance on Nature Questions.}
\label{tab:nature_questions}
\end{table}


\begin{table}[t]
\centering
\footnotesize
\resizebox{\columnwidth}{!}{
\begin{tabular}{lccccc}
\toprule
 & \multicolumn{5}{c}{TriviaQA} \\
Model & 8K & 16K & 32K & 64K & 128K \\
\midrule
Llama3.1-8B               & 82.8 & 86.7 & 85.6 & 81.1 & 80.6 \\
Llama3.1-8B+step-by-step  & 84.4 & 86.1 & 90.0 & 82.2 & 57.2 \\
Llama3.1-8B+plan\&solve   & 78.9 & 88.3 & 89.4 & 87.2 & 86.7 \\
Llama3.1-8B+fact\&reflect & 87.8 & 83.9 & 84.4 & 86.7 & 84.4 \\
ProLong-8B                & 71.1 & 88.3 & 78.9 & 82.8 & 78.3 \\
Llama3.1-8B+LongRAG       & 77.2 & 79.4 & 83.9 & 83.9 & 83.3 \\
\method-8B                & 88.3 & 92.2 & 91.1 & 93.3 & 88.3 \\
\bottomrule
\end{tabular}}
\caption{Long-context performance on TriviaQA.}
\label{tab:triviaqa}
\end{table}

\begin{table}[t]
\centering
\footnotesize
\resizebox{\columnwidth}{!}{
\begin{tabular}{lccccc}
\toprule
 & \multicolumn{5}{c}{PopQA} \\
Model & 8K & 16K & 32K & 64K & 128K \\
\midrule
Llama3.1-8B               & 61.1 & 62.8 & 57.2 & 58.3 & 56.1 \\
Llama3.1-8B+step-by-step  & 61.7 & 58.9 & 55.0 & 58.9 & 60.6 \\
Llama3.1-8B+plan\&solve   & 62.2 & 63.3 & 58.9 & 55.0 & 61.1 \\
Llama3.1-8B+fact\&reflect & 65.0 & 64.4 & 58.9 & 53.3 & 65.0 \\
ProLong-8B                & 67.8 & 68.3 & 70.0 & 64.4 & 65.6 \\
Llama3.1-8B+LongRAG       & 47.8 & 54.4 & 54.4 & 57.2 & 50.6 \\
\method-8B                & 82.2 & 82.2 & 78.3 & 76.7 & 65.6 \\
\bottomrule
\end{tabular}}
\caption{Long-context performance on PopQA.}
\label{tab:popqa}
\end{table}

\begin{table}[t]
\centering
\footnotesize
\resizebox{\columnwidth}{!}{
\begin{tabular}{lccccc}
\toprule
 & \multicolumn{5}{c}{NarrativeQA} \\
Model & 8K & 16K & 32K & 64K & 128K \\
\midrule
Llama3.1-8B               & 15.0 & 19.0 & 27.0 & 35.0 & 38.0 \\
Llama3.1-8B+step-by-step  & 23.0 & 30.0 & 36.0 & 51.0 & 43.0 \\
Llama3.1-8B+plan\&solve   & 22.0 & 25.0 & 38.0 & 41.0 & 39.0 \\
Llama3.1-8B+fact\&reflect & 18.0 & 35.0 & 37.0 & 42.0 & 46.0 \\
ProLong-8B                & 18.0 & 27.0 & 28.0 & 38.0 & 42.0 \\
Llama3.1-8B+LongRAG       & 23.3 & 23.3 & 50.0 & 50.0 & 46.7 \\
\method-8B                & 27.0 & 35.0 & 41.0 & 49.0 & 56.0 \\
\bottomrule
\end{tabular}}
\caption{Long-context performance on NarrativeQA.}
\label{tab:narrativeqa}
\end{table}


\begin{table}[t]
\centering
\footnotesize
\resizebox{\columnwidth}{!}{
\begin{tabular}{lccccc}
\toprule
 & \multicolumn{5}{c}{InfbenchQA} \\
Model & 8K & 16K & 32K & 64K & 128K \\
\midrule
Llama3.1-8B               & 17.0 & 31.0 & 36.0 & 40.0 & 48.0 \\
Llama3.1-8B+step-by-step  & 21.0 & 36.0 & 36.0 & 45.0 & 43.0 \\
Llama3.1-8B+plan\&solve   & 17.0 & 26.0 & 32.0 & 41.0 & 40.0 \\
Llama3.1-8B+fact\&reflect & 19.0 & 30.0 & 40.0 & 42.0 & 37.0 \\
ProLong-8B                & 16.0 & 31.0 & 29.0 & 31.0 & 45.0 \\
Llama3.1-8B+LongRAG       & 16.7 & 23.3 & 36.7 & 43.3 & 36.7 \\
\method-8B                & 25.0 & 39.0 & 42.0 & 47.0 & 50.0 \\
\bottomrule
\end{tabular}}
\caption{Long-context performance on InfbenchQA.}
\label{tab:infbenchqa}
\end{table}

\begin{table}[t]
\centering
\footnotesize
\resizebox{\columnwidth}{!}{
\begin{tabular}{lccccc}
\toprule
 & \multicolumn{5}{c}{InfbenchChoice} \\
Model & 8K & 16K & 32K & 64K & 128K \\
\midrule
Llama3.1-8B               & 9.0  & 12.0 & 24.0 & 39.0 & 55.0 \\
Llama3.1-8B+step-by-step  & 15.0 & 13.0 & 41.0 & 41.0 & 44.0 \\
Llama3.1-8B+plan\&solve   & 27.0 & 15.0 & 48.0 & 55.0 & 58.0 \\
Llama3.1-8B+fact\&reflect & 20.0 & 14.0 & 38.0 & 51.0 & 56.0 \\
ProLong-8B                & 22.0 & 27.0 & 37.0 & 48.0 & 58.0 \\
Llama3.1-8B+LongRAG       & 16.7 & 30.0 & 43.3 & 53.3 & 63.3 \\
\method-8B                & 45.0 & 46.0 & 47.0 & 64.0 & 68.0 \\
\bottomrule
\end{tabular}}
\caption{Long-context performance on InfbenchChoice.}
\label{tab:infbenchchoice}
\end{table}
}
    \caption{Overview of results for all datasets, regions, and methods. For each dataset, we include the number of regions we audit ($|\Omega|$), and the fraction of data points with fixed predictions ($p$). }
    \vspace{-1em}
    \label{tab:full_results}
\end{table}
