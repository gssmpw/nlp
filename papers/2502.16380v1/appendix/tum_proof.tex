\section{Proof of Theorem \ref{thm:tum}} \label{app:tum_pf}
%
We prove this result by showing that the polyhedron defining feasible individuals $\mathbf{x}$ and actions $\mathbf{a}$ under linear recourse constraints is \emph{totally unimodular}, which means that all extreme points of the polyhedron are integral. 
Consequently, the linear relaxation of the REP is feasible 
if and only the discrete REP is feasible.

We start by (re)-introducing important notation:
\begin{itemize}
    \item $\mathbf{x}$ is a decision variable corresponding to an individual with the region ${\cal R}$.
    \item $\mathbf{a}$ is a decision variable corresponding to an action which must be \emph{feasible} under the action set $A(x)$.
    \item $v_j$ represents a set of variables corresponding to feature $j$ (i.e., $x_j, a_j$, or $x_j+a_j$). 
\end{itemize}

Consider the following mixed-integer polyhedron that represents all feature space, region, and actionability constraints. 

\begin{subequations} \label{lr_polyhedron}
\begin{align}
	&& \mathbf{x} + \mathbf{a} &\in {\cal X} && \\
	&& \mathbf{x} &\in {\cal R} && \\
	&& \mathbf{x} &\in B(\mathbf{u}, \mathbf{l}) && \ \\
	&& \mathbf{a} &\in A(x) &&  \\
	&& \mathbf{x}, \mathbf{a} &\in \mathbb{R}^{d-m} \times \mathbb{Z}^m 
\end{align}
\end{subequations}

Recall that for \cref{thm:tum}, we consider a subset of possible constraints used in ${\cal X}, {\cal R}, A$ called \emph{linear recourse} constraints which were introduced in \cref{sec:scaling}. We denote the polyhedron \ref{lr_polyhedron} with only linear recourse constraints as the \emph{linear recourse polyhedron}. We repeat these constraint types here to keep this section self-contained:
\begin{itemize}
\item \textbf{$K$-Hot Constraints:} Let $J_i$ be the set of variables that participate in a K-hot constraint $i$. A K-Hot constraint is:
$$
\sum_{j \in J_i}  \pm~v_j \leq K.
$$
\item \textbf{Unit Directional Linkage Constraints:} This constraint acts on two sets of variables $v_i$, $v_k$ and requires:
$$
v_{j} \leq v_{k}.
$$
\item \textbf{Integer Bound Constraints: } Act on a single variable $v_j$ and require:
$$
L_j \leq  v_j \quad\text{or}\quad v_j \geq U_j
$$
\end{itemize}

Note that in all these constraints, all variable coefficients in these constraints are in $\{0, \pm 1\}$, and thus any matrix comprised of linear recourse constraints is a $\{0, \pm 1\}$-matrix. Let ${\cal J} = \{J_i\}$ be the set of $K$-hot constraints. We define an undirected graph which captures all directional implications between variable indices, which we call the \emph{implication graph}. For every variable index $i$ we create one node in the graph $n_i$ in the implication graph. We create an edge from node $n_j$ to $n_k$ if there exists a unit-directional linkage that acts on $j$ and $k$. Note that this implication graph is could include multiple connected (and potentially cyclic) components. 


We now formally define assumptions \ref{a1:onehot}-\ref{a3:cycles}:
\begin{enumerate}[label={A.\arabic*}, itemsep=0pt]
\item No variable appears in more than one $K$-hot constraint. Formally, $$ |J_i \cap J_k| = \emptyset, \quad \forall J_i, J_k \in {\cal J}.$$
\item The directional linkage constraints do not enforce relationships between variables appearing in $K$-hot constraints. Formally, each connected component in the implication graph contains at most one node $n_i$ corresponding to a variable in ${\cup_{J_i \in \cal J} J_i}$.
\item The directional linkage constraints do not imply any bi-directional implications between variables. Formally, the implication graph is acyclic. 
\end{enumerate}
% Note that under assumption \ref{a3:cycles} the implication graph is acyclic. 

We start by proving the key lemma for our result which states the linear recourse polyhedron is totally unimodular.
\begin{lemma} \label{lemma:tum}
    The linear recourse polyhedron is totally unimodular.
\end{lemma}
\begin{proof}
We prove this result by showing that every column submatrix of the linear recourse polyhedron admits an equitable bicoloring \citep{Ipref}. We consider a linear recourse polyhedron comprised exclusively of constraints where $v_j = x_j + a_j$, which we denote with the matrix Z. Note that any constraint where $v_j = x_j$ or $v_j = a_j$ uses a subset of the variables $\{x_j, a_j\}$ and thus represents a column sub-matrix of this general case. In other words, proving our result in this special case also proves the result for settings where constraints may have $v_j = x_j$ or $v_j = a_j$.

Consider the following coloring scheme for any column submatrix of Z. 
\begin{enumerate}
\item For each index $j$, if columns corresponding to both $x_j$ and $a_j$ are included in the submatrix, color the column corresponding to $x_j$ red and the column corresponding to $a_j$ blue. 

\item For each $K$-hot constraint $i$, let $\bar{J}_i \subseteq J_i$ be the remaining variables in the constraint, whose columns are included in the submatrix but have not yet been colored. Alternate coloring variables in $\bar{J}_i$ red and blue.

\item Consider the implication graph created by the directional constraints. Remove any node corresponding to an index $j$ that contains no variables selected in the submatrix, and any index $j$ where all variables corresponding to the index that are present in the submatrix have already been colored in Step 1. Note that every node $n_j$ now corresponds to exactly one column (i.e., $x_j$ or $a_j$). For each connected component of the revised implication graph, pick an initial node as follows. If there is a node $n_j$ corresponding to a variable in $\cup_{J_i \in {\cal J}} J_i $ in the component, select it as the initial node. Note that $n_j$ must have already been assigned a color in Step 2. If no such node exists, select an arbitrary node as the initial node. If the initial node corresponds corresponds to a column that has not yet been colored, color the corresponding column arbitrarily. We now traverse the connected component coloring columns as follows. Given a current node $n_j$ with color $c$, color all nodes $n_k$ connected to it that are uncolored with the opposite color. Repeat until all nodes in the connected component are colored.

%\connor{need nice way to say we can always color remaining variables here} %Recall that the implication graph is a directed graph where each node corresponds to a variable (i.e., column) in the matrix Z, and has an outgoing directed edge to at most one other node. Consider a subgraph of the implication graph that corresponds to all variables selected in the submatrix and has removed all nodes corresponding to variables whose columns have been colored in Step 1. We now color the columns corresponding to remaining nodes in this subgraph as follows. For each connected component in the implication graph, if there is a node $n_j \in {\cal N}_{K}$ corresponding to a variable that participates in a $K$-hot constraint set the next n. Otherwise, pick an arbitrary initial node. Starting at the initial node traverse the implication graph (e), for each node alternate coloring it red or blue until 

\item Note that any remaining uncolored columns must correspond to a single variable $x_j$ or $a_j$ that only participate in integer bound constraints. Color each column arbitrarily.
\end{enumerate}


Note that assumptions \ref{a1:onehot}- \ref{a3:cycles} ensure that this is a valid coloring (i.e., we always assign a column exactly one column). Specifically, \cref{a1:onehot} ensures that a column corresponding to a variable $x_j$ or $a_j$ is never assigned more than one color in Step 2. Similarly, \cref{a2:directional_linkage} and \cref{a3:cycles} ensures that each node in the revised implication graph is assigned exactly one color, and that every pair of columns connected with associated nodes in the revised implication is assigned different colors.

We now show that this coloring is equitable (i.e., the sum of the columns colored red minus the sum of the columns colored blue differ by at most one for each row). We denote the sum of the columns colored red minus the sum of the columns colored blue for each row as the \emph{row sum}. Since each row corresponds to a single constraint, we show this result for each constraint class separately:
\begin{itemize}
    \item \textbf{$K$-hot Constraints}: This follows directly from Step 1 and Step 2.
    \item \textbf{Unit-Directional Linkage Constraints}: Consider a generic sub-matrix of $Z$. If the row corresponding to a directional linkage constraint $i$ that operates on index $j$ and $k$ has columns corresponding to $x_j$ and $a_j$ ($x_k$ and $a_k$) in the the submatrix, Step 1 ensures the net sum for columns for variables corresponding to $j$ ($k$) is 0. If after Step 1, there is exactly one remaining uncolored column in the row of the submatrix then the entire row sum is $\pm 1$. If there are two uncolored columns then it must correspond to one variable (i.e., $x_j$ or $a_j$) from index $j$ and one from index $k$. Step 3 ensures that these are assigned different colors. Thus in all cases, the row sum is $0$ or $\pm 1$.
    \item \textbf{Integer Bound Constraints}: Every corresponding to an integer bound constraint in the submatrix includes columns correspond to either $x_j$ AND $a_j$, $x_j$, or $a_j$. In the first case, Step 1 ensures that the row sum is $0$. In the latter case, the row sum will be $\pm 1$ depending on the color selected in Step 4.
\end{itemize}

\end{proof}

We now prove the full result of Theorem \ref{thm:tum}. 
\begin{proof}
By \cref{lemma:tum}, under Assumptions \ref{a1:onehot}- \ref{a3:cycles} the linear recourse polyhedron is totally unimodular. This means that every extreme point of the polyhedron is integral and corresponds to feasible \emph{integer} vectors $\mathbf{x}, \mathbf{a}$. 

The REP with linear recourse constraints corresponds to the linear recourse polyhedron with an addition linear constraint (i.e., the linear recourse polyhedron intersected with a half-space). We now show that the REP is feasible iff the linear relaxation of the REP is feasible.

\paragraph{$\Rightarrow$ (REP is feasible implies the linear relaxation of the REP is feasible)}
This follows from the fact that the latter problem is a relaxation of the first problem.

\paragraph{$\Leftarrow$ (The linear relaxation of the REP is feasible implies the REP is feasible)}
Intuitively, this follows from the fact that the REP is the linear recourse polyhedron intersected with a halfspace. If there are any feasible data points in the REP there must be at least one feasible extreme point (which represents a solution to the REP).

Formally, consider a feasible solution to the relaxed REP $\mathbf{v}$ (note that this includes both $\mathbf{x}, \mathbf{a}$ for ease of notation). This solution must be a feasible point in the linear recourse polyhedron (otherwise it would contradict it being a feasible solution). We can represent any feasible point in the linear recourse polyhedron as a convex combination of extreme points of the polyhedron $\{\mu_k\}_k$ by the Minkowski representation theorem:
$$
\mathbf{v} = \sum_k \lambda_k \mu_k \quad\text{s.t. } \sum_k \lambda_k = 1, \lambda_k \geq 0 ~\forall k
$$
Constraint \eqref{const:d_classification} in the REP implies:
\begin{align*}
b \leq \mathbf{w}^\top \mathbf{v} 
= \mathbf{w}^\top (\sum_k \lambda_k \mu_k) 
= \sum_k \lambda_k \mathbf{w}^\top \mu_k
\end{align*}
Since $\lambda_k \geq 0$ this means that there is at least one extreme point $\mu_k$ such that $ \mathbf{w}^\top \mu_k \geq b$, and thus at least one feasible solution to the discrete REP.
\end{proof}