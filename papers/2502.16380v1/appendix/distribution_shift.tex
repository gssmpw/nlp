\section{On Out-of-Sample Robustness} \label{app:distribution_shift}
Auditing recourse over regions, rather than individuals, allows practitioners to find individuals with fixed predictions beyond a training dataset and is robust to distribution shifts. We demonstrate this capability by considering \emph{realized blindspots}, regions that are predicted to be responsive but contain individuals with fixed predictions in the test distribution. We evaluate the performance of the \baseline{Data} baseline, which certifies recourse by looking at individuals within a training dataset, in two regimes: (1) where the test distribution is the same as the train distribution of data, (2) where the train distribution is more likely to include individuals predicted to receive the desirable outcome. We simulate this distribution shift by training a logistic regression classifier on the entire dataset that predicts the likelihood of receiving the desirable outcome. We then construct the training dataset by sampling individuals with a probability proportional to their predicted score in the linear classifier using a soft-max function with a temperature of 1. In Figure \ref{fig:outofsample} we plot the realized blindspot rate for the \baseline{Data} baseline in all datasets with and without distribution shift. Our results show that \emph{even under the same distribution} the \baseline{Data} baseline can fail to catch instances of fixed predictions in the test dataset. This problem is further exacerbated by distribution shift, with the realized blindspot rate increasing across all three datasets. Note that by design \us{} has 0 realized blindspots because the regions themselves remain fixed under both train and test. Overall, these results highlight the importance of auditing regions as a tool to robustly foresee future fixed predictions, even under distribution shift.

\begin{figure}[ht]
      \centering
      \includegraphics[width=0.49\textwidth]{figures/out-of-sample.pdf}
  \caption {\label{fig:outofsample}  Realized blindspot rate for \baseline{Data} baseline with and without a distribution shift. Realized blindspot rate is the percentage of regions predicted to be responsive that contain individuals with fixed predictions in test dataset.}
\end{figure}

