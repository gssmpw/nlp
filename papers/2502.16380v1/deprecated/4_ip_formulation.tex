\section{Verification via MIQCP} \label{sec:method}
Towards formally verifying recourse over an entire population, we formulate a MIQCP to solve the interpretable PRVP. We start by formulating the related problem of auditing whether a fixed region $B(\mathbf{u}, \mathbf{l})$ contains any data points with recourse, which we denote the \emph{Population Recourse Existence Problem (PREP)}. 

Let $\mathbf{x} \in \mathbb{R}^{d-m} \times \mathbb{Z}^m$ be a decision variable representing an  point, and $\mathbf{a} \in \mathbb{R}^{d-m} \times \mathbb{Z}^m$ represent an action. We formulate the PREP as a mixed-integer linear program (MILP) (see Appendix \ref{app:prep_form} for details). For ease of presentation, we summarize the PREP MILP formulation using the following standard form:
$$
A\mathbf{x} + B\mathbf{a} \leq b(\mathbf{u}, \mathbf{l})
$$
where $A$ and $B$ are $m \times d$ matrices and $b(u,l)$ is a $m$-dimensional vector that is a linear function of the box region upper and lower bounds $\mathbf{u}, \mathbf{l}$. Here $m$ represents the number of constraints in the full PREP.

Our goal is to find a region $B(\mathbf{u}, \mathbf{l})$ where no data points have recourse, and thus the PREP is infeasible. We start by considering the \emph{linear relaxation} of the problem that relaxes all integer variables to be continuous, which we denote as the \emph{Linear} PREP (L-PREP). To formulate the problem of finding $B(u, l)$ we leverage a classical result from linear optimization called Farkas' lemma: 

\begin{theorem}[\citet{farkas}]
Let $A \in \mathbb{R}^{m \times n}$ and $b \in \mathbb{R}^m$. Then exactly one of the following two assertions is true:
\begin{itemize}
    \item There exists $x \in \mathbb{R}^n$ such that $Ax \leq b$
    \item There exists $y \geq 0$ such that $A^T y = 0$ and $b^T y = -1$
\end{itemize}
\end{theorem}

Farkas' lemma states that we can certify that a system of inequalities $Ax \leq b$ is infeasible by finding a certificate $y \geq 0$ such that $A^\top y = 0$ and $b^\top y = -1$. In our context, we can thus view the problem of finding a region without recourse as a joint problem of selecting a region and ensuring there is an associated certificate of infeasibility for the L-PREP. 

Let $y \in \mathbb{R}^{m}$ be decision variables representing the certificate of infeasibility.  Note that there is one variable for every constraint in the original problem. We can now find the largest region $B(u, l)$ with associated certificate of infeasibility $y$ for the L-PREP using the following MIQCP formulation, which we denote the \emph{Linear PRVP (L-PRVP)}:

\begin{align}
	\textbf{max}~~&& \sum_d \frac{u_d - l_d}{U_d - L_d} \label{obj:size}\\[.1cm]
	\textbf{s.t.}~~&& b(\mathbf{u}, \mathbf{l})^T \mathbf{y} &= -1 ~~&& \label{const:neg_ray}\\
	&& A^T \mathbf{y} &= 0 && \label{const:dual_feas_a} \\
	&& B^T \mathbf{y} &= 0 && \label{const:dual_feas_b} \\
	&& \mathbf{y} &\geq 0 && \label{const:non_neg_y} \\
	&& \mathbf{L} \leq \mathbf{l} \leq \mathbf{u} &\leq \mathbf{U} && \label{const:box_bounds} \\
	&& \mathbf{u}, \mathbf{l} &\in \mathbb{Z}^d \label{const:ul_int}
\end{align}

The objective of the problem is to maximize the size of $B(\mathbf{u}, \mathbf{l})$. Constraints \eqref{const:neg_ray}-\eqref{const:non_neg_y} follow from Farkas' lemma and ensure that $y$ is a valid certificate of infeasibility for the linear relaxation of the PREP. Constraints \eqref{const:box_bounds} and \eqref{const:ul_int} ensure we generate a valid region $B(\mathbf{u}, \mathbf{l})$. We restrict $\mathbf{u}, \mathbf{l}$ to be integer variables to prevent numerical precision issues when solving the MIQCP. This is not an onerous assumption as any continuous variable $x_j$ with a desired precision $10^{-p}$ can be re-scaled and rounded to an integer variable $10^p x_j$. The L-PRVP is quadratically constrained due to the inner product of $b(u,l)$ and $y$. While MIQCPs are known to be much more computationally demanding than mixed-integer linear programming, we find that in practice we are able to solve the problem in seconds on real-world tabular datasets. This is because the problem only scales with the number of features and actionability constraints, which are typically small, rather than the number of data points in the data set. %Note that if the LRAP problem is \emph{infeasible}, this certifies that every data point in ${\cal X}$ has recourse.


\subsection{Discrete Setting}

So far, we have shown how to construct a region $B(u,l)$ where all data points have no recourse in the L-PREP. Note that any region $B(u,l)$ that is feasible for the L-PRVP also contains no \textit{integer} data points with recourse, as its a relaxation of the discrete problem. However, it may fail to find regions where there is recourse in the relaxed version of the problem, but no feasible recourse in the discrete problem. In this case, the L-PRVP being infeasible cannot certify that all data points have recourse in the discrete problem.

To extend our approach to the discrete setting, we consider a simple strategy that solves the L-PRVP over all possible values of the discrete variables, which we call scenarios. Note that if we can certify recourse for the entire population on every scenario then we have have certified recourse for the discrete version of the problem. Let ${\cal J}_D$ be the set of discrete variables in the PREP. These can include variables $x_j$, $a_j$, or any auxiliary discrete variables that are needed to enforce certain actionability constraints. 

Let ${\cal S}$ be the set of scenarios, where a scenario $s \in {\cal S}$ corresponds to fixed values for the discrete variables (e.g., $x_1 = 1, x_2 = 2$ for ${\cal J}_D = \{1,2\}$). Note that the set ${\cal S}$ grows exponentially with respect to the number of discrete variables, which may become computationally intractable for large problems. However, we prove in Section \ref{sec:scaling} that under very general constraints and minimal assumptions we can relax many of the discrete variables in the PRVP. Under these new theoretical results, the set of scenarios that the algorithm must consider is incredibly small (e.g., $|{\cal S}| \leq 4$) for real-world datasets considered in prior work \cite{kothari2023prediction}. 

We integrate these scenarios into the PREP MILP formulation (see Appendix \ref{app:prep_form}), giving a revised formulation:
$$
A(s)\mathbf{x} + B(s)\mathbf{a} \leq b(\mathbf{u}, \mathbf{l},s)
$$
where $A, B,$ and $b$ now depend on the scenario $s$. To verify recourse over the entire population, we now need to find a certificate of infeasibility $\mathbf{y}_s$ for all scenarios $s \in {\cal S}$, giving rise to the following MIQCP which captures the original PRVP:
\begin{align}
	\textbf{max}~~&& \sum_d \frac{u_d - l_d}{U_d - L_d} \label{obj:f_size}\\[.1cm]
	\textbf{s.t.}~~&& b(\mathbf{u}, \mathbf{l}, s)^T \mathbf{y}_s &= -1 ~~&& \forall s \in {\cal S} \label{const:f_neg_ray}\\
	&& A(s)^T \mathbf{y}_s &= 0 && \forall s \in {\cal S} \label{const:f_dual_feas_a} \\
	&& B(s)^T \mathbf{y}_s &= 0 && \forall s \in {\cal S}\label{const:f_dual_feas_b} \\
	&& \mathbf{y}_s &\geq 0 && \forall s \in {\cal S}\label{const:f_non_neg_y} \\
	&& \mathbf{L} \leq \mathbf{l} \leq \mathbf{u} &\leq \mathbf{U} && \label{const:f_box_bounds} \\
	&& \mathbf{u}, \mathbf{l} &\in \mathbb{Z}^d \label{const:f_ul_int}
\end{align}

Note that when auditing recourse over a \emph{fixed} region $B(\mathbf{u}, \mathbf{l})$ (i.e., the decision-variant of the problem) we can decompose the PRVP into $|{\cal S}|$ problems (i.e., audit recourse for each scenario independently). If the PRVP is infeasible for any scenario (i.e., there exists an individual without recourse), then the PRVP is infeasible for the fixed region. Otherwise, we can certify recourse for the entire population over the region. However, in the interpretable PRVP (i.e., where we aim to find the largest reason) the problem cannot be decomposed as the box upper and lower bound variables link all the sub-problems. 


%\item Scales badly but in next section we show we can relax almost all the variables.

\subsection{Generating multiple regions}
In practice, it may be beneficial to generate multiple regions of data points without recourse to capture diverse sources of preclusion. Our framework can be easily extended to a sequential setting where we generate new regions, or certify that there are no additional regions without recourse. We do so by adding \textit{no-good cuts} to the RAP that excludes previously discovered regions.

Let $\bar{\mathbf{u}}, \bar{\mathbf{l}}$ be an existing region with no recourse. We add decision variables $\mathbf{z}_{u}, \mathbf{z}_{l} \in \{0, 1\}^d$ that track whether our new region $\mathbf{u}, \mathbf{l}$ is outside the existing region. We can model the no-good cuts that exclude $\bar{\mathbf{u}}, \bar{\mathbf{l}}$ using the following linear constraints:

\begin{align}
	 &u_d \leq \bar{l}_d - 1 + (U_d - \bar{l}_d - 1)(1-z_{ud}) ~~ \forall d \in [d] \label{const:ub}\\
	& l_d \geq \bar{u}_d + 1 + (L_d - \bar{u}_d + 1)(1-z_{ld}) ~~ \forall d \in [d] \label{const:lb}\\
    &\sum_d z_{ud} + z_{ld} \geq 1 \label{const:z_sum} \\
    & \mathbf{z}_{u}, \mathbf{z}_{l} \in \{0, 1\}^d \label{const:z_bin} 
\end{align}

Constraint \eqref{const:ub} checks whether the upper bound for feature $d$ in our new region is less than the lower bound for feature $d$ in the existing region. Note that these constraints are enforced if $z_{ud} = 1$. Constraint \eqref{const:lb} checks an analogous condition for the lower bound of feature d. Constraints \eqref{const:z_sum}-\eqref{const:z_bin} ensure that at least one constraint of type \eqref{const:ub} or \eqref{const:lb} are active. In other words, it ensures that the new region must fall outside the previous region by ensuring that in at least one feature the boxes do not coincide. Note that we can exclude constraints for bounds that are tight with the population bounds (i.e., $u_d = U_d$ or $l_d = L_d$). Given that our procedure tends to generate sparse regions (i.e., regions that change few features from their population upper and lower bounds), these no good-cuts typically add a very small number of binary variables and constraints to the full formulation.

%We can generate a set of regions by repeatedly solving the RAP and cutting off previous regions by adding new constraint blocks \eqref{const:ub}-\eqref{const:z_bin}. Given our assumption that the input data is integer, this procedure can in theory enumerate all possible regions with no recourse. In practice this is only feasible in settings where there are a small number of regions without recourse.
