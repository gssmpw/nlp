\section{Recourse Verification over Populations}
We consider a binary classification task where the goal is to predict a label $y \in \{0,1\}$ from a set of $d$ features $\mathbf{x} = [x_1, x_2, \dots, x_d ] \in {\cal X}$ where ${\cal X} \subseteq \mathbb{R}^{d+1}$ is a bounded set representing a population of individuals. We assume that the population ${\cal X}$ can be represented by a set of linear constraints over a mixed-integer set (see Fig. X for an example). This is an incredibly general assumption that encompasses a variety of potential populations ranging from regions as simple as a box to complex disjunctions over the convex hulls of different clusters of data. Throughout this paper we use boldface variables (e.g., $\mathbf{x}$) to denote vectors, and standard text with subscripts (e.g., $x_d$) to denote a specific element of a vector. We assume we have access to a linear classifier $f(x) = \text{sign}(\mathbf{w}^T\mathbf{x} + b)$ where $\mathbf{w} \in \mathbb{R}^d$ is a vector of coefficients and $b \in \mathbb{R}$ is the intercept of the classifier. Without loss of generality, we assume $f(x) = 1$ is the desired outcome (e.g., receiving a loan or job interview).

The recourse verification problem \cite{kothari2023prediction} tests whether an individual $\mathbf{x}$ can obtain the desired outcome of a model by changing their features, which we call \textit{actions}. Each action is a vector $\mathbf{a} \in \mathbb{R}^{d}$ that shifts the features of the individual to $\mathbf{x} + \mathbf{a} = \mathbf{x}' \in {\cal X}'$. We refer to the set of all actions an individual $\mathbf{x}$ can take as the \emph{action set} $A(\mathbf{x})$. In practice, an action set is represented by a set of constraints ~\citep[see][for details]{kothari2023prediction}. Note that in our setting, the population of potential individuals after taking an action ${\cal X}'$ may be different than the original population. For instance, we may want to verify recourse for individuals coming from a sub-population within a larger dataset (e.g., job applicants under 18 years old), but only want to constrain that the resulting feature vector is within the original population (e.g., resembles a realistic job application). We denote the \emph{recourse set} $R(x,f, A) = \{a: f(x + a) = 1, a \in A(x)\}$ as the set of feasible actions for an individual $x$ that lead to the desired outcome.

We study the problem of verifying recourse over the entire population of individuals ${\cal X}$. Formally, given a population ${\cal X}$, a classifier $f$, and a set of actionability constraints ${\cal A}$, the \textit{Population Recourse Verification Problem (PRVP) } checks whether there $\exists x \in {\cal X}: R(x, f, A) = \emptyset$. Note that this generalizes the recourse verification problem introduced in \citet{kothari2023prediction} to operate on an entire population of individuals. 

As a decision problem, the PRVP returns a yes or no response to whether a region has recourse. This binary response can make it difficult for model developers to interpret the root cause of the preclusion. To enhance interpretability, we also study an optimization variant of the PRVP that aims to find the largest region within the population without recourse. We focus on \textit{box} regions that are defined by simple upper and lower bound constraints on each feature. Let $U_j = \max_{x \in {\cal X}}x_j$, $L_j = \min_{x \in {\cal X}}x_j$ be the upper and lower bound for each feature $j$ in the population. Formally, for a given upper bound, $\mathbf{u} \in \mathbb{R}^d: \mathbf{u} \leq \mathbf{U}$, and lower bound, $\mathbf{l} \in \mathbb{R}^d: \mathbf{l} \leq \mathbf{L}$, a box region $B_{\cal X}(\mathbf{u},\mathbf{l})$ is defined as:
$$
B_{\cal X}(\mathbf{u},\mathbf{l}) = \{\mathbf{x} \in {\cal X}: \mathbf{l} \leq x \leq \mathbf{u}\}
$$
Box regions can be viewed as a decision rule (i.e., equivalent to the leaf node of a decision tree), which have been widely studied for their interpretability within the broader ML community (e.g., \cite{lawless2023interpretable, lawless2022interpretable, lawless2023cluster}).
For ease of notation we drop the explicit dependence on ${\cal X}$ and refer to box regions as $B(\mathbf{u}, \mathbf{l})$.
Note that setting $u_j = U_j$ ($l_j = L_j$) is equivalent to enforcing no upper bound (lower bound) on $x$ for feature $j$, and thus we can often express box regions with a small number of non-trivial constraints. We define the size of a box region $B(\mathbf{u},\mathbf{l})$ as the sum of the normalized ranges of each feature. Formally, the size of a region $B(\mathbf{u},\mathbf{l})$ is: $$\text{Size}(B(\mathbf{u}, \mathbf{l})) = \sum_{j=1}^d \frac{u_j - l_j}{U_j - L_j}$$ Note that this value is bounded between $0$ and $d$. We experimented with maximizing the area of the box (i.e., use a product instead of a sum) but found it to be much more computationally demanding empirically due to its non-linearity. In practice, we also found that maximizing the normalized range produced \emph{sparse} regions where only a few features were adjusted from their upper and lower bounds.

Given a population ${\cal X}$, a classifier $f$, and action set $A$, the \textit{Interpretable} PRVP aims to solve the following optimization problem:
\begin{align*}
\begin{split}
        \max\quad& \text{Size}(B(\mathbf{u}, \mathbf{l}))\\
        \textnormal{s.t.}\quad & \forall x \in B(\mathbf{u}, \mathbf{l}): R(x, f, A) = \emptyset\\
        & \mathbf{L} \leq \mathbf{l} \leq \mathbf{u} \leq \mathbf{U}
\end{split}
\end{align*}

Note that the output of the interpretable PRVP can be used to provide solutions to the original PRVP. For instance, the PVRP problem is a no-instance (i.e., certifies recourse for the entire population) if and only if the Interpretable PRVP is infeasible. Note that if the optimal value of the Interpretable PRVP $\text{Size}(B(\mathbf{u}^*, \mathbf{l}^*)) = \text{Size}(B(\mathbf{U}, \mathbf{L}))$ then the entire population has no recourse and is precluded from receiving the desirable outcome. We can also add additional constraints on the problem to answer other related questions. For instance, by constraining $\mathbf{u} = \mathbf{l}$ the PRVP will return a single data point within a population without recourse. 

\textbf{Use Cases:} Population-level recourse verification can be used in similar contexts as individualized recourse verification but provides stronger guarantees for out-of-sample deployment. %Moreover, identifying interpretable regions of preclusion can also aid in model development.

\emph{Detecting Preclusion.} In applications like lending and hiring where ML models need to safeguard against preclusion, our tools can be used to certify that all individuals in the population have recourse. Unlike existing approaches, these guarantees hold for out-of-sample data and can guarantee the absence of preclusion within the population on which a ML model is being deployed. 

\emph{Certifying Robustness.} Our approach can also certify the lack of recourse over a population, providing strong guarantees on model robustness to adversarial perturbations. For instance, in content moderation our tools can formally verify that malicious messages cannot bypass a filter by gaming their features. 

\emph{Characterizing Preclusion.} By iteratively using our tool to generate regions without recourse, our approach can enumerate precluded regions within a population. These regions are simple to understand and can be used to help debug ML models or provide a high-level summary of preclusion for stakeholders. 


%\textbf{Why is this task challenging?} A natural approach to auditing recourse over a population would be to verify the recourse of the worst-case member of the population (i.e., the individual $\mathbf{x}$ with the lowest score $\mathbf{w}^{\top} \mathbf{x}$). However, under complex actionability constraints there may be precluded individuals in the interior of the population --- even when the worst-case member has recourse (see Fig. X for an example). Therefore, heuristics that verify recourse for the worst-case member of the population cannot \emph{certify} recourse over the entire population.

%Therefore to \emph{formally certify} recourse over the population, individualized approaches would need to audit the recourse of every individual!

%connor{A corollary of our TUM result is that under linear recourse actionability constraints, if a population has an individual without recourse it'll be at \textit{an} extreme point. }

