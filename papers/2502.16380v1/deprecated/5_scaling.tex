\section{Relaxing Discrete Variables} \label{sec:scaling}
In the preceding section, we showed that we could solve the PRVP with discrete variables by enumerating all scenarios (i.e., fixed values for the discrete variables) and solving a MIQCP over all scenarios. However, this approach scales exponentially with respect to the number of discrete variables in the PREP. In this section we show that under a very broad set of actionability constraints and general assumptions we can relax all the discrete variables in the PREP problem and still certify population-level recourse.

We consider the following set of candidate constraints. We allow these constraints to act on either $x$ variables, $a$ variables, or both. Let $v_j$ represent a set of variables corresponding to feature $j$ (i.e., $x_j, a_j$, or $x_j+a_j$). 

\textbf{K-Hot Constraints:} These constraints capture that the unweighted sum of variables is exactly, at most, or at least $K \in \mathbb{Z}$. Let $J_c$ be the set of variables that participate in a k-hot constraint $c$. As an example, a K-Hot constraint can be represented by:
$$
\sum_{j \in J_c}  \pm~v_j \leq K
$$
Note that this constraint generalizes the popular \emph{one-hot encoding} for categorical variables.


\textbf{Unit Directional Linkage Constraints:} These constraints encode that a unit change in one feature, $v_{j}$ implies a unit change in another feature $v_{k}$. When applied to $\mathbf{x}$ variables it implies that one feature is always greater than or equal to another.
$$
v_{j} \leq v_{k}
$$
This family of constraints can ensure a broad class of non-separable constraints including thermometer encoding, and deterministic causal constraints (e.g., increasing years of account history implies a commensurate increase in Age). 

\textbf{Integer Bound Constraints:} These constraints put an integer upper or lower bounds on a variable. 
$$
L_j \leq  v_j \leq U_j
$$
This simple constraint class encompasses a wide range of separable constraints including monotonicity, actionability, and bounds on the action step size \cite{kothari2023prediction}.

We denote an action set comprised of these constraints as \textit{linear recourse constraints}. Linear recourse constraints encompass a very broad set of actionability constraints considered in previous literature including categorical encodings, thermometer encodings, montonocity, actionability, and action step sizes. This encompasses all the constraints in \citet{ustun2019actionable, russell2019efficient}, and all but 2 of the 100+ constraints used in the experiments in \citet{kothari2023prediction}. We now show that with two additional general assumptions, we can relax all the discrete variables in the PREP.

\begin{assumption}[A1, Informal\label{a1:onehot}] 
All variables participate in at most one $k$-hot constraint.
\end{assumption}

\begin{assumption}[A2, Informal\label{a2:directional_linkage}] 
The set of directional linkage constraints do not imply any relationships between variables participating in k-hot constraints.
\end{assumption}

Under these assumptions, the following result shows that solving the L-PREP is equivalent to solve the PREP. Practically, this means that we can relax all discrete variables in the problem and audit the model using the L-PRVP. 

\begin{theorem}\label{thm:tum}
Under Assumptions \ref{a1:onehot} and \ref{a2:directional_linkage}, the L-PRVP with linear recourse constraints is feasible iff the PRVP is feasible.
\end{theorem}

\textit{Proof Sketch.} At a high-level, we prove this result by showing that the polyhedron defining feasible $x$ and recourse $a$ under linear recourse constraints is \emph{totally unimodular}, which means that all extreme points of the polyhedron are integral. Consequently, a solution to the L-PREP exists if and only if there is at least one solution to the PREP, and by extension the PRVP. For a full proof, and formal definitions of the assumptions please see Appendix X.


