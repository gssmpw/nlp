\textbf{K-Hot Constraints:} These constraints capture that the unweighted sum of variables is exactly, at most, or at least $K \in \mathbb{Z}$. Let $J_i$ be the set of variables that participate in a K-hot constraint $i$. A K-Hot constraint can be written as
$$
\sum_{j \in J_i}  \pm~v_j \leq K.
$$
This constraint generalizes the popular \emph{one-hot encoding} for categorical variables.

\textbf{Unit Directional Linkage Constraints:} These constraints encode that a unit change in one feature, $v_{j}$ implies a unit change in another feature $v_{k}$: 
$$
v_{j} \leq v_{k}.
$$
This constraint on $\mathbf{x}$ variables implies that one feature is always greater than or equal to another.
This family of constraints can ensure a broad class of non-separable constraints (i.e., constraints that act on multiple features) including thermometer encodings, and deterministic causal constraints (e.g., increasing years of account history implies a commensurate increase in Age). 

\textbf{Integer Bound Constraints:} These constraints put an integer upper or lower bound on a variable:
$$
L_j \leq  v_j \leq U_j.
$$
This simple constraint class encompasses a wide range of separable constraints including monotonicity, actionability, and bounds on the action step size \cite{kothari2023prediction}.
