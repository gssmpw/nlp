\subsection{Results}
The classification task in this research does not align with a typical binary classification scenario, where classes are usually defined as "normal" and "abnormal." Instead, we work with two specific classes: \textit{Anastrepha fraterculus} and \textit{Ceratitis capitata}, whose correct classification is crucial in the context of our study.

Since we need to evaluate the classifier's performance for both classes accurately, it is essential to employ metrics that provide detailed information about the model's behavior for each of them. While accuracy and recall are useful, they are not sufficient on their own to correctly assess the model's performance, as we need to consider reducing both false positive and false negative rates. This is especially important due to the critical nature of correctly classifying both species and the meticulous work carried out by SENASA experts to identify the morphological characteristics of each species.

Minimizing false positives and false negatives is essential to ensure accuracy in the reports on the population of each species in a given area. This guarantees the integrity of the results and prevents misinterpretation of the data, which could impact the population analysis of the studied species.

A thorough analysis of our results must be conducted using the F1-score metric, which provides a balance between precision and recall. However, since we need an overall result that reflects the performance across both classes, we use the "macro average," as recommended by the documentation in \cite{scikitlearn}. This approach calculates the average of the metrics for each class, giving equal weight to both, which allows us to obtain a balanced view of the classifier's performance. The metrics calculated with the results from the classification of the test data for the VGG16 model are presented in Table \ref{tab:resultadosvgg16}, for the VGG19 model in Table \ref{tab:resultadosvgg19}, and for the Inception-V3 model in Table \ref{tab:resultadosinception}.
\begin{table}[htbp]
  \centering
  \caption{Metrics of the VGG16 Model}
  \label{tab:resultadosvgg16}
  \begin{tabular}{|l|c|c|c|}
  \hline
     & \textbf{Precision} & \textbf{Recall} & \textbf{F1-score} \\
    \hline
    \textbf{\textit{A. fraterculus}} & 0.96 & 0.68 & 0.79 \\
    \hline
    \textbf{\textit{C. capitata}} & 0.75 & 0.97 & 0.85 \\
    \hline
    \textbf{Macro average} & 0.86 & 0.82 & \textbf{0.82} \\
    \hline
  \end{tabular}
\end{table}

\begin{table}[htbp]
  \centering
  \caption{Metrics of the VGG19 Model}
  \label{tab:resultadosvgg19}
  \begin{tabular}{|l|c|c|c|}
  \hline
     & \textbf{Precision} & \textbf{Recall} & \textbf{F1-score} \\
    \hline
    \textbf{\textit{A. fraterculus}} & 0.90 & 0.72 & 0.80 \\
    \hline
    \textbf{\textit{C. capitata}} & 0.76 & 0.92 & 0.83 \\
    \hline
    \textbf{Macro average} & 0.83 & 0.82 & \textbf{0.82} \\
    \hline
  \end{tabular}
\end{table}

\begin{table}[htbp]
	\centering
	\caption{Metrics of the Inception-v3 Model}
	\label{tab:resultadosinception}
	\begin{tabular}{|l|c|c|c|}
		\hline
		& \textbf{Precision} & \textbf{Recall} & \textbf{F1-score} \\
		\hline
		\textbf{\textit{A. fraterculus}} & 0.96 & 0.89 & 0.92 \\
		\hline
		\textbf{\textit{C. capitata}} & 0.90 & 0.96 & 0.93 \\
		\hline
		\textbf{Macro average} & 0.93 & 0.93 & \textbf{0.93} \\
		\hline
	\end{tabular}
\end{table}

When analyzing the results obtained by our three models, we observed that we achieved good performance in classifying the two species of fruit flies. However, as mentioned in Section \ref{sec:moscafruta}, there are specific morphological characteristics for each species that the models must be able to correctly identify.

\subsection{Analysis of Model Explainability.}
\label{sec:cam}
To better understand which visual aspects the models are considering when making classifications, we employed the Grad-CAM technique. This technique allows us to generate a visual representation of the areas of the images that the models consider most relevant when assigning them to a specific class. For this, we used two images from our test set, as shown in (Figure \ref{fig:moscaprueba}).

\begin{figure}[htbp]
    \centering
    \includegraphics[width=0.3\textwidth]{imagenes/discusion/mosca_prueba.png}
    \caption{Test images with areas that our models should focus on: a. \textit{Anastrepha fraterculus}. b. \textit{Ceratitis capitata}}
    \scriptsize \textit{Source:} Author's own creation
    \label{fig:moscaprueba}
\end{figure}

In Section \ref{sec:moscafruta}, we describe the key anatomical sections of fruit flies, such as the thorax, wings, and ovipositor. The ovipositor, which is a crucial feature for classifying these species, is excluded as a morphological feature for classification because it is not possible to accurately differentiate this structure in our images, and it will not be considered an area of interest for our models.

In the images shown in Figure \ref{fig:moscaprueba}.a and \ref{fig:moscaprueba}.b, the areas that should be considered of interest by our models are highlighted with black circles.

Figures \ref{fig:camvgg16}, \ref{fig:camvgg19}, \ref{fig:caminception} display the areas of interest identified by each of our models for classifying the same images from Figure \ref{fig:moscaprueba}. These results allow us to analyze which visual features were key for the models to assign a class to each image.

\begin{figure}[htbp]
    \centering
    \includegraphics[width=0.4\textwidth]{imagenes/discusion/camvgg16.png}
    \caption{Grad-Cam of the VGG16 model. a) \textit{Anastrepha fraterculus}. b) \textit{Ceratitis capitata}.}
    \scriptsize \textit{Source:} Author's own creation
    \label{fig:camvgg16}
\end{figure}

\begin{figure}[htbp]
    \centering
    \includegraphics[width=0.4\textwidth]{imagenes/discusion/camvgg19.png}
    \caption{Grad-Cam of the VGG19 model. a) \textit{Anastrepha fraterculus}. b) \textit{Ceratitis capitata}.}
    \scriptsize \textit{Source:} Author's own creation
    \label{fig:camvgg19}
\end{figure}

\begin{figure}[htbp]
    \centering
    \includegraphics[width=0.4\textwidth]{imagenes/discusion/caminception.png}
    \caption{Grad-Cam of the Inception-V3 model. a) \textit{Anastrepha fraterculus}. b) \textit{Ceratitis capitata}.}
    \scriptsize \textit{Source:} Author's own creation
    \label{fig:caminception}
\end{figure}

The Grad-CAM method allows us to identify the areas of the images that the models consider most relevant for making their predictions. In this analysis, we used images in which all three models classified correctly, which allowed us to evaluate the activated regions in each case.

In Figures \ref{fig:camvgg16} and \ref{fig:camvgg19}, corresponding to the VGG16 and VGG19 models, we observe that both share similar patterns when selecting areas of interest. The activated regions are more widely distributed, focusing on multiple subareas within the main parts of the object, such as the wings and thorax. This suggests that both models tend to consider more diverse information when making their predictions, utilizing various features present in the image.

On the other hand, in Figure \ref{fig:caminception}, the Inception V3 model shows a different approach by focusing its attention on a single predominant region. This area encompasses essential features, such as the wings and thorax, which are determinant for classification. This behavior could explain the model's excellent performance when evaluated with the test data, achieving 93\% in the F1-score metric, as it effectively utilizes the most relevant features to make its decisions.

\subsection{Criteria for Model Selection.}
The criteria considered for model selection include classification time, performance metrics, areas of interest identified by the models, and their performance in uncontrolled scenarios. These criteria are based on the results obtained from evaluating the models with test images, consisting of 74 images per class under study.

\subsubsection{Evaluation of Classification Time.}
When comparing classification times, the Inception-V3 model proved to be the most efficient, with a total time of 15.04 seconds. It was followed by the VGG19 model, with a time of 40.83 seconds, while VGG16 recorded a time of 45.13 seconds. These results position Inception-V3 as the most efficient model in terms of classification speed.

\subsubsection{Evaluation of Performance Metrics.}
In terms of performance metrics, Inception-V3 excelled with an F1-score of 93\%, reflecting its superior ability to learn and generalize the patterns associated with the studied species. In contrast, VGG16 and VGG19 achieved F1-scores of 0.82\%. These findings suggest that Inception-V3 offers higher precision and robustness in classification.

\subsubsection{Evaluation Using Grad-CAM.}
In Section \ref{sec:cam}, the graphical representations of the image regions used by the models to make their predictions are presented, using the Grad-CAM technique. This technique highlights the areas of the image that are relevant for each prediction and is shown in the third column of Figures \ref{fig:camvgg16}, \ref{fig:camvgg19}, and \ref{fig:caminception}, where Grad-CAM is overlaid on the original image, facilitating the visualization of the areas on which each model focuses its attention.

Upon analyzing the results, it is observed that Inception-V3 covers a greater amount of information, including regions with key morphological characteristics. This demonstrates that Inception-V3 is the model that best utilizes these features to make its classifications.

The goal of this work is to develop a model capable of replicating the criteria used by experts in species classification. The results obtained confirm that Inception-V3 is the most suitable model for identifying relevant morphological areas, validating its suitability for this task.

\subsubsection{Evaluation in Uncontrolled Scenarios.}

For the evaluation of uncontrolled scenarios, images were selected from the internet with the goal of more thoroughly assessing the robustness and generalization capability of our models. We used a set of images manually collected from various online sources. These images represent real-world scenarios, including flies perched on fruits or fabrics, as well as images with uniform single-color backgrounds.

The final dataset consisted of 36 images, divided equally into 18 images for each class. This test allowed us to analyze how the models managed to generalize the specific features of the species under study when faced with data that were not part of the initial training. The results obtained are crucial for evaluating the performance of the models in uncontrolled contexts.

The results of the models with this dataset are shown in Tables \ref{tab:ivgg16}, \ref{tab:ivgg19}, and \ref{tab:iv3}.
\begin{table}[htbp]  
	\centering  
	\caption{Results with internet data from the VGG16 model.}  
	\label{tab:ivgg16}  
	\begin{tabular}{|l|c|c|c|}  
		\hline  
		& \textbf{Precision} & \textbf{Recall} & \textbf{F1-score} \\  
		\hline  
		\textbf{\textit{A. fraterculus}} & 0.4 & 0.11 & 0.17 \\  
		\hline  
		\textbf{\textit{C. capitata}} & 0.48 & 0.83 & 0.61 \\  
		\hline  
		\textbf{Macro average} & 0.44 & 0.47 & \textbf{0.39} \\  
		\hline  
	\end{tabular}  
\end{table}
\begin{table}[htbp]  
	\centering  
	\caption{Results with internet data from the VGG19 model.}  
	\label{tab:ivgg19}  
	\begin{tabular}{|l|c|c|c|}  
		\hline  
		& \textbf{Precision} & \textbf{Recall} & \textbf{F1-score} \\  
		\hline  
		\textbf{\textit{A. fraterculus}} & - & 0 & - \\  
		\hline  
		\textbf{\textit{C. capitata}} & 0.5 & 1.00 & 0.67 \\  
		\hline  
		\textbf{Macro average} & - & 0.5 & - \\  
		\hline  
	\end{tabular}  
\end{table}
\begin{table}[htbp]  
	\centering  
	\caption{Results with internet data from the Inception-V3 model.}  
	\label{tab:iv3}  
	\begin{tabular}{|l|c|c|c|}  
		\hline  
		& \textbf{Precision} & \textbf{Recall} & \textbf{F1-score} \\  
		\hline  
		\textbf{\textit{A. fraterculus}} & 0.73 & 0.61 & 0.67 \\  
		\hline  
		\textbf{\textit{C. capitata}} & 0.67 & 0.78 & 0.72 \\  
		\hline  
		\textbf{Macro average} & 0.7 & 0.69 & \textbf{0.69} \\  
		\hline  
	\end{tabular}  
\end{table}

This study aims to develop a robust model that demonstrates its effectiveness in species classification. The results obtained highlight Inception-V3 as the model with the highest robustness and best classification time, successfully performing the task of fruit fly species classification.