\begin{abstract}
This study develops a transfer learning model for the automated classification of two species of fruit flies, \textit{Anastrepha fraterculus} and \textit{Ceratitis capitata}, in a controlled laboratory environment. The research addresses the need to optimize identification and classification, which are currently performed manually by experts, being affected by human factors and facing time challenges. The methodological process of this study includes the capture of high-quality images using a mobile phone camera and a stereo microscope, followed by segmentation to reduce size and focus on relevant morphological areas. The images were carefully labeled and preprocessed to ensure the quality and consistency of the dataset used to train the pre-trained convolutional neural network models VGG16, VGG19, and Inception-v3. The results were evaluated using the F1-score, achieving 82\% for VGG16 and VGG19, while Inception-v3 reached an F1-score of 93\%. Inception-v3's reliability was verified through model testing in uncontrolled environments, with positive results, complemented by the Grad-CAM technique, demonstrating its ability to capture essential morphological features. These findings indicate that Inception-v3 is an effective and replicable approach for classifying \textit{A. fraterculus} and \textit{C. capitata}, with potential for implementation in automated monitoring systems.
\end{abstract}

\begin{keyword}
	Transfer Learning \sep Data Augmentation \sep Grad-CAM \sep Image Segmentation \sep Morphological Features
\end{keyword}
