\section{Morphological Characteristics.}
\label{sec:moscafruta}
Understanding the morphological characteristics of different species of fruit flies is crucial for appreciating the work of experts in the identification and classification of these pests in the laboratory. The accuracy in species identification depends on a thorough analysis of their morphological traits, among which the wings, thorax, and ovipositor are key elements. These characteristics allow differentiation of species that, although visually similar, exhibit subtle variations in their anatomical structures, essential for their taxonomic classification.

\subsection{Sections and Venations of the Fruit Fly Wing.}
The wings of the fruit fly provide key information through their sections and venations, which are crucial for the classification of different species. This information is represented in specific patterns and structural differences that allow effective identification and differentiation between species.

In Figure \ref{fig:taxonomiaala}, the distribution of the venations and sections present in the wing is shown. Some venations provide relevant information for species differentiation, such as the $R_{4+5}$, $bm{-}cu$, $dm{-}cu$ veins, among others. Similarly, specific sections such as $dm$, $r_{4+5}$, $bm$, among others, are identified. For the two species studied in this research, the venations and sections that allow their differentiation will be described in detail.

\begin{figure}[htbp]
    \centering
    \includegraphics[width=0.5\textwidth]{imagenes/mosca_fruta/secciones_ala.png}
    \caption{Sections of the wing.}
    \scriptsize \textit{Adapted from:} \cite{norrbom}.
    \label{fig:taxonomiaala}
\end{figure}

\subsection{Parts of the Fruit Fly Thorax.}
The thorax of the fruit fly constitutes another significant source of information due to its structure and the coloration patterns it exhibits. These characteristics are essential for distinguishing between various species, enabling precise classification.

In Figure \ref{fig:taxonomiatorax}, the distribution of the parts that make up the thorax is presented. Some anatomical structures provide key information for species differentiation, such as the scutum (Esd), scuto-scutellar suture (Ses-es), scutellum (Esl), subscutellum (Sbe), and mediotergite (Mdt). These anatomical features of the thorax are fundamental for the classification of fruit fly species. In this study, these specific structures will be described in detail for the two species under investigation, highlighting those that allow for their differentiation.


\begin{figure}[htbp]
    \centering
    \includegraphics[width=0.4\textwidth]{imagenes/mosca_fruta/secciones_torax.png}
    \caption{Parts of the Thorax}
    \scriptsize \textit{Adapted from:} \cite{senasaTephritidae, especiesImEco}.
    \label{fig:taxonomiatorax}
\end{figure}

For this research, the selection of species to be classified was made following the recommendation of the laboratory expert, who considered the relevance of the population dynamics of \textit{Anastrepha fraterculus} and \textit{Ceratitis capitata}. Initially, a larger number of species was proposed for analysis; however, the complexity associated with separating each species into individual vials would have increased the expert's workload. To ensure accuracy in processing and avoid interference with their tasks, the study was limited to these two species.

The selected species are:
\begin{itemize}
	\item \textit{Anastrepha fraterculus}, known as the South American fruit fly.
	\item \textit{Ceratitis capitata}, known as the Mediterranean fruit fly.
\end{itemize}
%\subsection{Comparación de Características Morfológicas entre \textit{Anastrepha fraterculus} y \textit{Ceratitis capitata}.}

Knowing the sections of the wings and the parts of the thorax, we will proceed to identify the morphological characteristics that differentiate the two selected species. These differences can be observed in Figures \ref{fig:alas} and \ref{fig:toraxs}, and are described in Tables \ref{tab:diferenciasalas} and \ref{tab:diferenciastorax}.

\begin{figure*}[htbp]
    \centering
    \includegraphics[width=0.8\textwidth]{imagenes/mosca_fruta/alas.png}
    \caption{a. Wing of \textit{Anastrepha fraterculus}. b. Wing of \textit{Ceratitis capitata}.}
    \scriptsize \textit{Source:} \cite{especiesImEco}.
    \label{fig:alas}
\end{figure*}

\begin{table*}[htbp]
	\centering
	\caption{Differences in wings. \cite{senasaTephritidae, senasaAnastrepha, especiesImEco}}
	\label{tab:diferenciasalas}
	\begin{tabular}{|p{0.4\textwidth}|p{0.05\textwidth}|p{0.4\textwidth}|p{0.05\textwidth}|}
		\hline
		\textbf{\textit{Anastrepha fraterculus}} & \textbf{Figure} & \textbf{\textit{Ceratitis capitata}} & \textbf{Figure} \\
		\hline
		Complete and slightly wide S-band at its apical portion & \ref{fig:alas}.a.1 & Wing with a pattern of yellow stripes & \ref{fig:alas}.b \\
		\hline
		C and S bands fully connected & \ref{fig:alas}.a.2 & Wing with dark spots in cells bc, c, br, bm, and bcu & \ref{fig:alas}.b.1 \\
		\hline
		Hyaline spot at the apex of vein $R_1$ & \ref{fig:alas}.a.3 & r-m vein near the middle of the dm cell within the discal band including the pterostigma & \ref{fig:alas}.b.2 \\
		\hline
		S and V bands generally connected, sometimes with a slight separation & \ref{fig:alas}.a.4 & Costal band pigmenting vein dm-cu does not cross the $r_{4+5}$ cell & \ref{fig:alas}.b.3 \\
		\hline
		Complete V band at its upper portion & \ref{fig:alas}.a.5 & Posteroapical lobe of cell bcu thinner at the base than in the middle & \ref{fig:alas}.b.4 \\
		\hline
		$R_{4+5}$ vein almost straight & \ref{fig:alas}.a.6 & Discal stripe & \ref{fig:alas}.b.5 \\
		\hline
		Moderate apical curvature of vein M & \ref{fig:alas}.a.7 & Costal stripe extended to the apical margin of the wing & \ref{fig:alas}.b.6 \\
		\hline
	\end{tabular}
\end{table*}

\begin{figure*}[htbp]
    \centering
    \includegraphics[width=0.8\textwidth]{imagenes/mosca_fruta/toraxs.png}
    \caption{a. Thorax of \textit{Anastrepha fraterculus} b. Thorax of \textit{Ceratitis capitata}.}
    \scriptsize \textit{Source:} \cite{especiesImEco}
    \label{fig:toraxs}
\end{figure*}

\begin{table*}[htbp]
	\centering
	\caption{Differences in the thorax. \cite{senasaTephritidae, senasaAnastrepha, especiesImEco}}
	\label{tab:diferenciastorax}
	\begin{tabular}{|p{0.4\textwidth}|p{0.05\textwidth}|p{0.4\textwidth}|p{0.05\textwidth}|}
		\hline
		\textbf{\textit{Anastrepha fraterculus}} & \textbf{Figure} & \textbf{\textit{Ceratitis capitata}} & \textbf{Figure} \\
		\hline
		Scuto-scutellar suture (Ses-es) mark generally present and extending to the sides & \ref{fig:toraxs}.a.1 & Scutum with irregular yellow spots & \ref{fig:toraxs}.b.1 \\
		\hline
		Scutellum (sbe) with a black spot on each side extending to the mediotergite (mdt) & \ref{fig:toraxs}.a.2 & Yellow mark near the scuto-scutellar suture & \ref{fig:toraxs}.b.2 \\
		\hline
		& & Globose or swollen scutellum with a shiny black color & \ref{fig:toraxs}.b.3 \\
		\hline
	\end{tabular}
\end{table*}
