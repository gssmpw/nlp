%\subsection{Comparación de Características Morfológicas entre \textit{Anastrepha fraterculus} y \textit{Ceratitis capitata}.}

Knowing the sections of the wings and the parts of the thorax, we will proceed to identify the morphological characteristics that differentiate the two selected species. These differences can be observed in Figures \ref{fig:alas} and \ref{fig:toraxs}, and are described in Tables \ref{tab:diferenciasalas} and \ref{tab:diferenciastorax}.

\begin{figure*}[htbp]
    \centering
    \includegraphics[width=0.8\textwidth]{imagenes/mosca_fruta/alas.png}
    \caption{a. Wing of \textit{Anastrepha fraterculus}. b. Wing of \textit{Ceratitis capitata}.}
    \scriptsize \textit{Source:} \cite{especiesImEco}.
    \label{fig:alas}
\end{figure*}

\begin{table*}[htbp]
	\centering
	\caption{Differences in wings. \cite{senasaTephritidae, senasaAnastrepha, especiesImEco}}
	\label{tab:diferenciasalas}
	\begin{tabular}{|p{0.4\textwidth}|p{0.05\textwidth}|p{0.4\textwidth}|p{0.05\textwidth}|}
		\hline
		\textbf{\textit{Anastrepha fraterculus}} & \textbf{Figure} & \textbf{\textit{Ceratitis capitata}} & \textbf{Figure} \\
		\hline
		Complete and slightly wide S-band at its apical portion & \ref{fig:alas}.a.1 & Wing with a pattern of yellow stripes & \ref{fig:alas}.b \\
		\hline
		C and S bands fully connected & \ref{fig:alas}.a.2 & Wing with dark spots in cells bc, c, br, bm, and bcu & \ref{fig:alas}.b.1 \\
		\hline
		Hyaline spot at the apex of vein $R_1$ & \ref{fig:alas}.a.3 & r-m vein near the middle of the dm cell within the discal band including the pterostigma & \ref{fig:alas}.b.2 \\
		\hline
		S and V bands generally connected, sometimes with a slight separation & \ref{fig:alas}.a.4 & Costal band pigmenting vein dm-cu does not cross the $r_{4+5}$ cell & \ref{fig:alas}.b.3 \\
		\hline
		Complete V band at its upper portion & \ref{fig:alas}.a.5 & Posteroapical lobe of cell bcu thinner at the base than in the middle & \ref{fig:alas}.b.4 \\
		\hline
		$R_{4+5}$ vein almost straight & \ref{fig:alas}.a.6 & Discal stripe & \ref{fig:alas}.b.5 \\
		\hline
		Moderate apical curvature of vein M & \ref{fig:alas}.a.7 & Costal stripe extended to the apical margin of the wing & \ref{fig:alas}.b.6 \\
		\hline
	\end{tabular}
\end{table*}

\begin{figure*}[htbp]
    \centering
    \includegraphics[width=0.8\textwidth]{imagenes/mosca_fruta/toraxs.png}
    \caption{a. Thorax of \textit{Anastrepha fraterculus} b. Thorax of \textit{Ceratitis capitata}.}
    \scriptsize \textit{Source:} \cite{especiesImEco}
    \label{fig:toraxs}
\end{figure*}

\begin{table*}[htbp]
	\centering
	\caption{Differences in the thorax. \cite{senasaTephritidae, senasaAnastrepha, especiesImEco}}
	\label{tab:diferenciastorax}
	\begin{tabular}{|p{0.4\textwidth}|p{0.05\textwidth}|p{0.4\textwidth}|p{0.05\textwidth}|}
		\hline
		\textbf{\textit{Anastrepha fraterculus}} & \textbf{Figure} & \textbf{\textit{Ceratitis capitata}} & \textbf{Figure} \\
		\hline
		Scuto-scutellar suture (Ses-es) mark generally present and extending to the sides & \ref{fig:toraxs}.a.1 & Scutum with irregular yellow spots & \ref{fig:toraxs}.b.1 \\
		\hline
		Scutellum (sbe) with a black spot on each side extending to the mediotergite (mdt) & \ref{fig:toraxs}.a.2 & Yellow mark near the scuto-scutellar suture & \ref{fig:toraxs}.b.2 \\
		\hline
		& & Globose or swollen scutellum with a shiny black color & \ref{fig:toraxs}.b.3 \\
		\hline
	\end{tabular}
\end{table*}