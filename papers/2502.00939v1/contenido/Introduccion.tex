\section{Introduction}
Fruit flies, belonging to the Order Diptera and the Family Tephritidae, are one of the main pests affecting fruit production worldwide. There are approximately 4,000 species of fruit flies, distributed in genera of great economic importance such as \textit{Anastrepha}, \textit{Bactrocera}, \textit{Ceratitis}, \textit{Dacus}, \textit{Rhagoletis}, and \textit{Toxotrypana} \cite{3ortiz2021, delgado2009moscas}. In Peru, several genera are present, including \textit{Anastrepha} (with 44 species), \textit{Ceratitis} (1 species), \textit{Rhagoletis} (5 species), and \textit{Toxotrypana} (1 species) \cite{iica1997}. The species of greatest economic relevance, such as \textit{Anastrepha distincta}, \textit{Anastrepha fraterculus}, \textit{Ceratitis capitata}, and \textit{Anastrepha striata}, are found in various regions of the country, from the south and central areas to the north \cite{senasa2011}.

The biological cycle of these pests causes significant damage to fruit production. Females deposit their eggs under the skin of the fruit, where the larvae develop and consume the pulp, accelerating ripening, decomposition, and falling, leading to substantial economic losses. In Latin America, fruit flies are estimated to cause annual losses of around 35 million dollars, and in Andean Group countries (Bolivia, Colombia, Ecuador, and Peru), these losses exceed 30\% of the value of fruit and vegetable production \cite{1municipalidad_echarati}. In Peru, the damage caused by these pests reaches up to 60\% of the production \cite{2senasa_erradicacion}.

To mitigate these losses, ``El Servicio Nacional de Sanidad Agraria del Perú'' (SENASA) has implemented fruit fly detection and monitoring projects. This system relies on the use of McPhail traps with food attractants, which allow for the collection of specimens for laboratory analysis. Specifically, in the province of ``La Convención'' in the Cusco region, specialists identify and classify species by analyzing specific morphological characteristics, and weekly reports are generated to evaluate the population of these species in the monitored areas.

Currently, species identification is performed manually by experts, which requires speed and precision. However, both factors are compromised: the average identification time per fly is around 10 seconds, and the accuracy of the analysis can be affected by human factors such as fatigue and subjectivity. This manual process, although necessary, is not optimal during peak fly population periods, resulting in delays in decision-making.

Recent studies have shown that computational systems based on artificial intelligence (AI), particularly those using deep learning techniques, are capable of improving accuracy in similar classification processes \cite{4Das2018, 22Toke2020}. The literature suggests that by using datasets combined with machine learning and deep learning models, accuracy rates exceeding 85\% can be achieved \cite{7Faria2014, 9Martins2019, 13Leonardo2017}. However, to date, these methods have not been specifically applied to \textit{Anastrepha fraterculus} and \textit{Ceratitis capitata} species, nor has the potential of a dataset based on images taken with a mobile phone through a stereomicroscope been explored.

This study proposes the development of a computational model for the identification and classification of \textit{Anastrepha fraterculus} and \textit{Ceratitis capitata} through image analysis. The main contributions include the creation of an original dataset and the development of an automated classification model. The dataset was obtained from samples selected by an expert, using a stereomicroscope and a mobile device. The proposed model is based on the analysis of key morphological areas of each species, replicating the identification process followed by specialists in the laboratory. This methodology has the potential to be implemented in AI-assisted taxonomic classification environments, thereby optimizing monitoring and control systems for this pest.

The organization of this paper is as follows: Section 2 reviews related works; Section 3 describes the morphological characteristics of the studied species; Section 4 presents the materials and methods; Section 5 details the experiments and results obtained; Section 6 discusses the findings; and finally, Section 7 presents the conclusions and analyzes the study's findings.