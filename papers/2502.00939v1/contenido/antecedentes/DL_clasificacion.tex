\subsection{Deep Learning Techniques for Fruit Fly Classification}

The classification of \textit{Ceratitis capitata} and \textit{Grapholita molesta} in smart traps was addressed by \cite{9Martins2019} using pre-trained models, with ResNet18 standing out, initially achieving an accuracy of 84.28\%. The accuracy improved to 93.55\% and 91.28\%, respectively, after applying data augmentation techniques (flipping, rotation, and random erasing). Given the goal of implementing a model on the Raspberry Pi v2, SqueezeNet was selected, achieving an accuracy of 88.56\% for \textit{Ceratitis capitata} and 90.60\% for \textit{Grapholita molesta} with the same data augmentation techniques.

\cite{13Leonardo2017} focused on the identification of fruit fly species, using a dataset containing wing images of \textit{Anastrepha fraterculus}, \textit{Anastrepha obliqua}, and \textit{Anastrepha sororcula}.

Feature descriptors and detectors were employed to train nine machine learning techniques. The best performance was achieved with the Multilayer Perceptron (MLP), which reached an average accuracy of 88.9\% in classifying all species.

The research by \cite{15Gonzalez-Lopez2022} addressed the determination of the age of fruit fly pupae with the goal of making them sterile. The dataset was prepared by cleaning a section of the pupae to expose the fly’s eyes.

The Multi-Template Matching technique was applied alongside an Inception-v1 model. This combination of image preprocessing with templates and the Inception-v1 model achieved an accuracy of 72.22\% for \textit{Anastrepha ludens} and 83.17\% for \textit{Ceratitis capitata}.
