\subsection{Deep Learning Techniques Combined with Machine Learning for Fruit Fly Classification}

The research by \cite{10Peng2020} addressed the classification of \textit{Bactrocera dorsalis}, \textit{Bactrocera cucurbitae}, \textit{Bactrocera tau}, and \textit{Bactrocera scutellata} in images with complex backgrounds. Initially, they tested different hyperparameter configurations in a CNN but faced overfitting issues. To resolve this, they replaced the final Softmax layer of the CNN with machine learning models: SVM, KNN, AdaBoost, and Random Forest. The CNN-SVM combination was the most effective, achieving an accuracy of 92.4\%.

In their work, \cite{12Leonardo2018} successfully classified \textit{Anastrepha} species: \textit{A. fraterculus}, \textit{A. obliqua}, and \textit{A. sororcula}, using pre-trained models (ResNet, VGG16, VGG19, Xception, Inception) to extract features. These features were evaluated with machine learning models such as Decision Tree, KNN, MLP, Naive Bayes, SGD, and SVM. Using cross-validation, VGG16-SVM achieved the highest accuracy, reaching 95.68\%.

The proposal by \cite{14Molina-Rotger2023} focused on the development of a monitoring system using sticky traps, with two datasets: one distinguishing "Olive Fly" and "Others" (including debris), and another differentiating "Fruit Fly" and debris. They used models such as Random Forest, SVM, Decision Trees, and deep networks like VGG16, MobileNet, and Xception. For the first dataset, RF and SVM achieved accuracies of 62.1\% and 86.4\%. In the second dataset, accuracies increased to 91.9\% (RF) and 94.5\% (SVM).