\subsection{Machine Learning Techniques for Fruit Fly Classification}

In the study by \cite{7Faria2014}, the classification of the fruit fly species \textit{Anastrepha fraterculus}, \textit{Anastrepha obliqua}, and \textit{Anastrepha sororcula} was investigated using a set of wing and ovipositor images. During preprocessing, Otsu's thresholding and morphological dilation were employed to segment the images, extracting color and texture features to train multiple classifiers. Accuracy was optimized by combining up to 33 individual classifiers with a Support Vector Machines (SVM) meta-classifier and merging the datasets, achieving a final accuracy of 98.8\%.

\cite{8Remboski2018} proposed the development of a classification system designed to identify fruit fly species using smart traps, utilizing two datasets: one with four classes (\textit{Anastrepha fraterculus}, \textit{Ceratitis capitata}, "Other insects," and "Residues"), and another with three classes, consolidating the last two into "Others." The images were converted to grayscale, adaptive thresholding and morphological opening were applied to remove unwanted areas. They used Bag of Visual Words (BOVW) for feature extraction and evaluated various classification models, with SVM being the most effective, achieving accuracies of 84.56\% and 86.38\% in each dataset.
