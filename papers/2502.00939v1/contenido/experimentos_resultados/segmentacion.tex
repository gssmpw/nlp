\subsection{Segmentation.}

The dataset was generated from the manual capture of images using the eyepieces of a stereomicroscope, which resulted in variations in contrast, lighting, and brightness of the images. During the segmentation process, it was observed that some of these images were not correctly segmented. In Figure \ref{fig:segmentacioncorrecta}.b, which shows the segmentation of Figure \ref{fig:segmentacioncorrecta}.a, it can be seen that the image retains parts of the background, which are considered noise. This noise could affect the predictions of the models, preventing them from being properly trained. On the other hand, Figure \ref{fig:segmentacioncorrecta}.d shows correct segmentation, without background remnants or noise, ensuring higher prediction accuracy.

To ensure proper training of the models, images that were not correctly segmented were manually separated. These included 50 discarded images per class, which reduced the dataset size to 236 images per class.
\begin{figure}[htbp]
    \centering
    \includegraphics[width=0.4\textwidth]{imagenes/experimentos_resultados/segmentacion/segmentacion.png}
    \caption{a) Original image. b) Incorrectly segmented image. c) Original image. d) Correctly segmented image.}
    \scriptsize \textit{Source:} Author's own creation
    \label{fig:segmentacioncorrecta}
\end{figure}