\subsection{Results.}
After running experiments with each selected model, the following results were obtained:

\textbf{VGG16 Model.} The best performance for this model was achieved in epoch 91, with a total training time of 4750.1 seconds. For the test set, consisting of 148 images, the classification time was \textbf{28.88} seconds for \textit{Anastrepha fraterculus} samples and \textbf{16.25} seconds for \textit{Ceratitis capitata} samples, with a total classification time of \textbf{45.13} seconds.

\textbf{VGG19 Model.} The VGG19 model achieved its best performance in epoch 79, with a total training time of 5831.29 seconds. In the test set of 148 images, the classification time was \textbf{16.81} seconds for \textit{Anastrepha fraterculus} samples and \textbf{24.02} seconds for \textit{Ceratitis capitata} samples, with a total classification time of \textbf{40.83} seconds.

\textbf{Inception-v3 Model.} The best performance for the Inception-v3 model was achieved in epoch 41, with a total training time of 806.56 seconds. In the test set of 148 images, the classification time was \textbf{7.68} seconds for \textit{Anastrepha fraterculus} samples and \textbf{7.36} seconds for \textit{Ceratitis capitata} samples, with a total classification time of \textbf{15.04} seconds.

Table \ref{tab:resultados_experimentos} presents the confusion matrices corresponding to the classification performance of VGG16, VGG19, and Inception-V3. Each matrix is structured to display the classification results for both species.

\begin{table}[htbp]
	\centering
	\caption{Confusion matrix result. TP = True Positives FP = False Positives FN = False Negatives TN = True Negatives}
	\label{tab:resultados_experimentos}
	\begin{tabular}{|l|c|c|c|c|}
		\hline
		\textbf{MODELS} & \textbf{TP} & \textbf{FN} & \textbf{FP} & \textbf{TN}  \\
		   \hline
		\textbf{Vgg16 \textit{A. fraterculus}} & 50 & 24 & 2 & 72 \\
		\hline
		\textbf{Vgg16 \textit{C. capitata}} & 72 & 2 & 24 & 50 \\
		\hline
		\textbf{Vgg19 \textit{A. fraterculus}} & 53 & 21 & 6 & 68 \\
		\hline
		\textbf{Vgg19 \textit{C. capitata}} & 68 & 6 & 21 & 53 \\
		\hline
		\textbf{Inception-v3 \textit{A. fraterculus}} & 66 & 8 & 3 & 71 \\
		\hline
		\textbf{Inception-v3 \textit{C. capitata}} & 71 & 3 & 8 & 66 \\
		\hline
	\end{tabular}
\end{table}

In Figure \ref{fig:performance}, the performance curves of each model are shown, evaluated through accuracy and loss metrics:

\begin{itemize}
	\item \textbf{Train Accuracy:} Indicates how effectively the models have learned the patterns present in the training dataset.
	\item \textbf{Validation Accuracy:} Reflects how the model's accuracy improves when classifying new, unseen images, evaluating its generalization ability.
	\item \textbf{Train Loss:} Shows how the model adjusts its parameters to reduce errors on the training dataset.
	\item \textbf{Validation Loss:} Measures how the model minimizes errors on the validation data, essential for assessing its performance on unseen data.
\end{itemize}

Analyzing these curves, a similar behavior is observed across all three models. During the initial epochs, the accuracy curves start at low levels. It is notable that the Validation Accuracy curve quickly surpasses the Train Accuracy curve, indicating that the models manage to generalize well from early stages. Subsequently, the Validation Accuracy stabilizes and consistently stays above the Train Accuracy, suggesting that the models did not suffer from overfitting and were effectively trained.

Complementing this analysis, the loss curves also provide relevant insights. Initially, both Train Loss and Validation Loss curves present high values, indicating a significant amount of initial errors. However, as the epochs progress, these curves gradually decrease, showing that the models effectively adjusted their parameters, reducing errors in both the training and validation data. This behavior reinforces the idea that the models did not memorize the training data, avoiding overfitting and correctly generalizing to unseen images.

\begin{figure*}[htb]
	\centering
	\includegraphics[width=1\textwidth]{imagenes/performance.png}
	\caption{Performance curves of model training.}
	\scriptsize \textit{Source:} Author's own creation
	\label{fig:performance}
\end{figure*}