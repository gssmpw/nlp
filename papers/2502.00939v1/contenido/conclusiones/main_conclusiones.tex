\section{Conclusions.}  
In this study, various classification models were evaluated with the goal of classifying fruit fly species in segmented images. Among the models tested, Inception-V3 demonstrated the highest efficiency, achieving a total classification time of 15.04 seconds for 148 samples and an F1-score of 93\%, indicating its suitability for species classification in terms of both accuracy and speed.

Although Inception-V3 showed strong performance in controlled conditions and with poorly segmented images, its generalization capability on a set of randomly selected images from the internet was limited, yielding an F1-score of 69\%. This result emphasizes the importance of incorporating more diverse conditions in future studies to enhance the model's applicability in real-world environments.

The visual analysis conducted with Grad-Cam revealed that Inception-V3 was able to identify key morphological features associated with the species, further supporting its ability to extract relevant characteristics for classification. This, combined with its performance on well-segmented data and additional tests, justifies its selection as the most appropriate model for the task.

In conclusion, Inception-V3 emerges as a promising model for species classification due to its balance of accuracy and speed, along with its partial adaptability to diverse scenarios. However, the findings also highlight the necessity of continuing to explore strategies to improve the model's generalization in uncontrolled conditions, expanding its potential for real-world applications.