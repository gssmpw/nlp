\subsection{Image Acquisition}
Two main categories are considered for the image capture setup: laboratory environments and field environments \cite{18Martineau2017}.

Since SENASA is the competent authority in the identification of fruit fly species, access was requested to one of their facilities. This allowed image capture in a laboratory environment, which follows a standard protocol for capturing, identifying, classifying, and reporting species.

The laboratory protocol is as follows:
\begin{itemize}
    \item Official SENASA McPhail traps (Figure \ref{fig:multilure}) are placed in different districts the province of "La Convención".
        \begin{figure}[htbp]
            \centering
            \includegraphics[width=0.2\textwidth]{imagenes/materiales_metodos/adquisicion_imagenes/mcphail.png}
            \caption{McPhail trap.}
            \scriptsize \textit{Source:} \cite{Safer2022}.
            \label{fig:multilure}
        \end{figure}
    \item The fruit fly species are attracted to the traps using a food attractant made from Torula yeast. The flies that land on this attractant become stuck and trapped.
    \item To collect the flies, the food attractant, along with the trapped flies, is poured into jars labeled according to the district, zone, and specific trap where they were found.
    \item The jars with the samples are handed over to the expert responsible for the identification and classification of the species.
    \item The information about the identified and classified species is recorded in an Official Trap Record (ROT) format, which allows tracking of the traps installed nationwide \cite{6senasa_manual_vigilancia}.
\end{itemize}

During this stage, a dataset of images corresponding to the species \textit{Anastrepha fraterculus} and \textit{Ceratitis capitata} was requested from the entity. However, it was reported that they did not have a dataset with the required characteristics to meet the objectives of this research. Due to this limitation, the creation of a custom dataset was proposed, specifically designed to meet the needs of the study.

In collaboration with the expert responsible for classifying the flies from the traps, the separation of the flies belonging to the species \textit{Anastrepha fraterculus} and \textit{Ceratitis capitata} was coordinated.

Since the essential morphological characteristics for the classification of these species are the wings and thorax, the optimal position to capture the most information from these two structures was identified. After a detailed analysis, it was concluded that the dorsal position provides the maximum visibility of the relevant coloration and sections. However, during the collection process, issues arose with some samples because the flies, when trapped, had their morphological features altered due to the position in which they were caught. For this reason, each sample was carefully handled.

\textbf{Sample Handling.}
To ensure the accuracy of photographic captures of the samples, it is essential to handle each fly in a way that maximizes the visibility of the areas of interest for the correct classification of each species. The following steps were taken to adjust the position of the flies during the handling process:

\begin{enumerate}
	\item First, the fly is observed in the dorsal position (Figure \ref{fig:manipulacion}-a). In this position, the left wing is retracted, preventing full visualization of the area of interest required for classification.
	\item Next, the fly is shown in the ventral position (Figure \ref{fig:manipulacion}-b). In this view, the abdomen is contracted, and the legs are retracted towards the thorax, which hinders proper observation from the dorsal position.
	\item With the fly in the ventral position, the abdomen is manipulated to expand it, and the legs are adjusted to extend from the thorax (Figure \ref{fig:manipulacion}-c). This adjustment allows for better visualization of the relevant structures.
	\item Finally, the fly is returned to the dorsal position (Figure \ref{fig:manipulacion}-d). In this position, the wings are fully extended, unlike what was observed in Figure \ref{fig:manipulacion}-a. Additionally, at the end of the abdomen, the ovipositor is visible, which was not visible in Figure \ref{fig:manipulacion}-a.
\end{enumerate}

\begin{figure}[htbp]
    \centering
    \includegraphics[width=0.4\textwidth]{imagenes/materiales_metodos/adquisicion_imagenes/manipulacion.png}
    \caption{Fly Handling.}
    \scriptsize \textit{Source:} Author's own creation. 
    \label{fig:manipulacion}
\end{figure}

Subsequently, the process of capturing photographs of the species \textit{Anastrepha fraterculus} and \textit{Ceratitis capitata} was carried out. The images were taken in the dorsal position using the camera of a Xiaomi Mi 11 Lite 5G NE smartphone, equipped with autofocus. The captures were made through the ocular lenses of the Leica EZ4 stereomicroscope, as shown in Figure \ref{fig:refcapturaimagen}.

\begin{figure}[htbp]
    \centering
    \includegraphics[width=0.3\textwidth]{imagenes/materiales_metodos/adquisicion_imagenes/ref_captura.png}
    \caption{Reference image of the image capture process.}
    \scriptsize \textit{Source:} Author's own creation.
    \label{fig:refcapturaimagen}
\end{figure}

The captured images were stored in JPG format, with resolutions ranging from 2088 x 4640 pixels to 3472 x 4640 pixels. A total of 689 images of \textit{Anastrepha fraterculus} and 286 images of \textit{Ceratitis capitata} were obtained. To balance the number of images per class, 286 images were randomly selected from the \textit{Anastrepha fraterculus} set, thus equalizing the number of images for both species. This resulted in a balanced dataset of 572 images in total.

It is important to emphasize the privacy of this dataset. The obtained images are authored by SENASA and have been personally collected by me for this research. Access to these images is strictly limited to the involved parties, ensuring their controlled and protected use in accordance with the confidentiality and security principles established in the research.
 