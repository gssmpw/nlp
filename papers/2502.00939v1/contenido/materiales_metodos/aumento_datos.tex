\subsection{Data Augmentation.}

Given the need for large amounts of training data, the collection and labeling process being slow, complex, and costly \cite{22Toke2020}, it is imperative to use techniques that allow for model training with smaller datasets. In this context, data augmentation presents itself as an effective solution in the field of deep learning.

For our training dataset, we will apply basic image manipulation methods such as random rotation, horizontal and vertical flipping, as shown in Figure \ref{fig:aumento_datos}. These techniques have proven to be effective in previous studies \cite{9Martins2019, 14Molina-Rotger2023, 15Gonzalez-Lopez2022, 16Mamdouh2021}, providing significant improvements in training quality and model performance.

\begin{figure}[htbp]
    \centering
    \includegraphics[width=0.4\textwidth]{imagenes/materiales_metodos/aumento_datos/aumento_datos.png}
    \caption{Example of data augmentation.}
    \scriptsize \textit{Source:} Author's own creation
    \label{fig:aumento_datos}
\end{figure}
