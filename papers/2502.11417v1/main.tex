\pdfoutput=1
\documentclass[11pt]{article}
\usepackage[final]{ACL2023}

\usepackage{times}
\usepackage{latexsym}
\usepackage[T1]{fontenc}
\usepackage[utf8]{inputenc}
\usepackage{microtype}
\usepackage{inconsolata}
\usepackage{amsmath}
\usepackage{amsthm} 
\usepackage{graphicx}
\usepackage{subfigure}
\usepackage{booktabs}
\usepackage{xcolor}
\usepackage{listings}
\usepackage{tabularx}
\usepackage{array}
\usepackage{makecell}
\usepackage{multirow}
\usepackage{enumitem}
\usepackage{algorithm}
\usepackage{algorithmic}
\usepackage{amssymb}
\usepackage{url}

\newenvironment{denseitemize}{
\begin{itemize}[topsep=2.5pt, partopsep=0pt, leftmargin=1.5em]
 \setlength{\itemsep}{2.5pt}
 \setlength{\parskip}{0pt}
 \setlength{\parsep}{0pt}
}{\end{itemize}}

\newenvironment{denseenum}{
\begin{enumerate}[topsep=2pt, partopsep=0pt, leftmargin=1.5em]
 \setlength{\itemsep}{2pt}
 \setlength{\parskip}{0pt}
 \setlength{\parsep}{0pt}
}{\end{enumerate}}

\newcommand{\disco}{{\emph{DiSCo}}}

\lstset{
   language=Python,
   basicstyle=\small\ttfamily,
   identifierstyle=\color{black},
   keywordstyle=\color{black},
   stringstyle=\color{black},
   commentstyle=\color{black},
   morekeywords={keeper_service},
   keywordstyle=\color{purple},
   breaklines=true,
   frame=single,
   columns=flexible,
   xleftmargin=0pt,
}

\newtheorem{lemma}{Lemma}
\newtheorem{theorem}{Theorem}

\setlength\titlebox{5cm}

\title{\disco: Device-Server Collaborative LLM-Based Text Streaming Services}

\author{Ting Sun$^{1}$,
Penghan Wang$^{1}$, 
Fan Lai$^{1}$
\\
\textsuperscript{1} University of Illinois Urbana-Champaign, United States \\
\texttt{fanlai@illinois.edu}
}

\begin{document}
\maketitle

\begin{abstract}  
Test time scaling is currently one of the most active research areas that shows promise after training time scaling has reached its limits.
Deep-thinking (DT) models are a class of recurrent models that can perform easy-to-hard generalization by assigning more compute to harder test samples.
However, due to their inability to determine the complexity of a test sample, DT models have to use a large amount of computation for both easy and hard test samples.
Excessive test time computation is wasteful and can cause the ``overthinking'' problem where more test time computation leads to worse results.
In this paper, we introduce a test time training method for determining the optimal amount of computation needed for each sample during test time.
We also propose Conv-LiGRU, a novel recurrent architecture for efficient and robust visual reasoning. 
Extensive experiments demonstrate that Conv-LiGRU is more stable than DT, effectively mitigates the ``overthinking'' phenomenon, and achieves superior accuracy.
\end{abstract}  
\section{Introduction}
\label{sec:introduction}
The business processes of organizations are experiencing ever-increasing complexity due to the large amount of data, high number of users, and high-tech devices involved \cite{martin2021pmopportunitieschallenges, beerepoot2023biggestbpmproblems}. This complexity may cause business processes to deviate from normal control flow due to unforeseen and disruptive anomalies \cite{adams2023proceddsriftdetection}. These control-flow anomalies manifest as unknown, skipped, and wrongly-ordered activities in the traces of event logs monitored from the execution of business processes \cite{ko2023adsystematicreview}. For the sake of clarity, let us consider an illustrative example of such anomalies. Figure \ref{FP_ANOMALIES} shows a so-called event log footprint, which captures the control flow relations of four activities of a hypothetical event log. In particular, this footprint captures the control-flow relations between activities \texttt{a}, \texttt{b}, \texttt{c} and \texttt{d}. These are the causal ($\rightarrow$) relation, concurrent ($\parallel$) relation, and other ($\#$) relations such as exclusivity or non-local dependency \cite{aalst2022pmhandbook}. In addition, on the right are six traces, of which five exhibit skipped, wrongly-ordered and unknown control-flow anomalies. For example, $\langle$\texttt{a b d}$\rangle$ has a skipped activity, which is \texttt{c}. Because of this skipped activity, the control-flow relation \texttt{b}$\,\#\,$\texttt{d} is violated, since \texttt{d} directly follows \texttt{b} in the anomalous trace.
\begin{figure}[!t]
\centering
\includegraphics[width=0.9\columnwidth]{images/FP_ANOMALIES.png}
\caption{An example event log footprint with six traces, of which five exhibit control-flow anomalies.}
\label{FP_ANOMALIES}
\end{figure}

\subsection{Control-flow anomaly detection}
Control-flow anomaly detection techniques aim to characterize the normal control flow from event logs and verify whether these deviations occur in new event logs \cite{ko2023adsystematicreview}. To develop control-flow anomaly detection techniques, \revision{process mining} has seen widespread adoption owing to process discovery and \revision{conformance checking}. On the one hand, process discovery is a set of algorithms that encode control-flow relations as a set of model elements and constraints according to a given modeling formalism \cite{aalst2022pmhandbook}; hereafter, we refer to the Petri net, a widespread modeling formalism. On the other hand, \revision{conformance checking} is an explainable set of algorithms that allows linking any deviations with the reference Petri net and providing the fitness measure, namely a measure of how much the Petri net fits the new event log \cite{aalst2022pmhandbook}. Many control-flow anomaly detection techniques based on \revision{conformance checking} (hereafter, \revision{conformance checking}-based techniques) use the fitness measure to determine whether an event log is anomalous \cite{bezerra2009pmad, bezerra2013adlogspais, myers2018icsadpm, pecchia2020applicationfailuresanalysispm}. 

The scientific literature also includes many \revision{conformance checking}-independent techniques for control-flow anomaly detection that combine specific types of trace encodings with machine/deep learning \cite{ko2023adsystematicreview, tavares2023pmtraceencoding}. Whereas these techniques are very effective, their explainability is challenging due to both the type of trace encoding employed and the machine/deep learning model used \cite{rawal2022trustworthyaiadvances,li2023explainablead}. Hence, in the following, we focus on the shortcomings of \revision{conformance checking}-based techniques to investigate whether it is possible to support the development of competitive control-flow anomaly detection techniques while maintaining the explainable nature of \revision{conformance checking}.
\begin{figure}[!t]
\centering
\includegraphics[width=\columnwidth]{images/HIGH_LEVEL_VIEW.png}
\caption{A high-level view of the proposed framework for combining \revision{process mining}-based feature extraction with dimensionality reduction for control-flow anomaly detection.}
\label{HIGH_LEVEL_VIEW}
\end{figure}

\subsection{Shortcomings of \revision{conformance checking}-based techniques}
Unfortunately, the detection effectiveness of \revision{conformance checking}-based techniques is affected by noisy data and low-quality Petri nets, which may be due to human errors in the modeling process or representational bias of process discovery algorithms \cite{bezerra2013adlogspais, pecchia2020applicationfailuresanalysispm, aalst2016pm}. Specifically, on the one hand, noisy data may introduce infrequent and deceptive control-flow relations that may result in inconsistent fitness measures, whereas, on the other hand, checking event logs against a low-quality Petri net could lead to an unreliable distribution of fitness measures. Nonetheless, such Petri nets can still be used as references to obtain insightful information for \revision{process mining}-based feature extraction, supporting the development of competitive and explainable \revision{conformance checking}-based techniques for control-flow anomaly detection despite the problems above. For example, a few works outline that token-based \revision{conformance checking} can be used for \revision{process mining}-based feature extraction to build tabular data and develop effective \revision{conformance checking}-based techniques for control-flow anomaly detection \cite{singh2022lapmsh, debenedictis2023dtadiiot}. However, to the best of our knowledge, the scientific literature lacks a structured proposal for \revision{process mining}-based feature extraction using the state-of-the-art \revision{conformance checking} variant, namely alignment-based \revision{conformance checking}.

\subsection{Contributions}
We propose a novel \revision{process mining}-based feature extraction approach with alignment-based \revision{conformance checking}. This variant aligns the deviating control flow with a reference Petri net; the resulting alignment can be inspected to extract additional statistics such as the number of times a given activity caused mismatches \cite{aalst2022pmhandbook}. We integrate this approach into a flexible and explainable framework for developing techniques for control-flow anomaly detection. The framework combines \revision{process mining}-based feature extraction and dimensionality reduction to handle high-dimensional feature sets, achieve detection effectiveness, and support explainability. Notably, in addition to our proposed \revision{process mining}-based feature extraction approach, the framework allows employing other approaches, enabling a fair comparison of multiple \revision{conformance checking}-based and \revision{conformance checking}-independent techniques for control-flow anomaly detection. Figure \ref{HIGH_LEVEL_VIEW} shows a high-level view of the framework. Business processes are monitored, and event logs obtained from the database of information systems. Subsequently, \revision{process mining}-based feature extraction is applied to these event logs and tabular data input to dimensionality reduction to identify control-flow anomalies. We apply several \revision{conformance checking}-based and \revision{conformance checking}-independent framework techniques to publicly available datasets, simulated data of a case study from railways, and real-world data of a case study from healthcare. We show that the framework techniques implementing our approach outperform the baseline \revision{conformance checking}-based techniques while maintaining the explainable nature of \revision{conformance checking}.

In summary, the contributions of this paper are as follows.
\begin{itemize}
    \item{
        A novel \revision{process mining}-based feature extraction approach to support the development of competitive and explainable \revision{conformance checking}-based techniques for control-flow anomaly detection.
    }
    \item{
        A flexible and explainable framework for developing techniques for control-flow anomaly detection using \revision{process mining}-based feature extraction and dimensionality reduction.
    }
    \item{
        Application to synthetic and real-world datasets of several \revision{conformance checking}-based and \revision{conformance checking}-independent framework techniques, evaluating their detection effectiveness and explainability.
    }
\end{itemize}

The rest of the paper is organized as follows.
\begin{itemize}
    \item Section \ref{sec:related_work} reviews the existing techniques for control-flow anomaly detection, categorizing them into \revision{conformance checking}-based and \revision{conformance checking}-independent techniques.
    \item Section \ref{sec:abccfe} provides the preliminaries of \revision{process mining} to establish the notation used throughout the paper, and delves into the details of the proposed \revision{process mining}-based feature extraction approach with alignment-based \revision{conformance checking}.
    \item Section \ref{sec:framework} describes the framework for developing \revision{conformance checking}-based and \revision{conformance checking}-independent techniques for control-flow anomaly detection that combine \revision{process mining}-based feature extraction and dimensionality reduction.
    \item Section \ref{sec:evaluation} presents the experiments conducted with multiple framework and baseline techniques using data from publicly available datasets and case studies.
    \item Section \ref{sec:conclusions} draws the conclusions and presents future work.
\end{itemize}
\section{Background}\label{sec:backgrnd}

\subsection{Cold Start Latency and Mitigation Techniques}

Traditional FaaS platforms mitigate cold starts through snapshotting, lightweight virtualization, and warm-state management. Snapshot-based methods like \textbf{REAP} and \textbf{Catalyzer} reduce initialization time by preloading or restoring container states but require significant memory and I/O resources, limiting scalability~\cite{dong_catalyzer_2020, ustiugov_benchmarking_2021}. Lightweight virtualization solutions, such as \textbf{Firecracker} microVMs, achieve fast startup times with strong isolation but depend on robust infrastructure, making them less adaptable to fluctuating workloads~\cite{agache_firecracker_2020}. Warm-state management techniques like \textbf{Faa\$T}~\cite{romero_faa_2021} and \textbf{Kraken}~\cite{vivek_kraken_2021} keep frequently invoked containers ready, balancing readiness and cost efficiency under predictable workloads but incurring overhead when demand is erratic~\cite{romero_faa_2021, vivek_kraken_2021}. While these methods perform well in resource-rich cloud environments, their resource intensity challenges applicability in edge settings.

\subsubsection{Edge FaaS Perspective}

In edge environments, cold start mitigation emphasizes lightweight designs, resource sharing, and hybrid task distribution. Lightweight execution environments like unikernels~\cite{edward_sock_2018} and \textbf{Firecracker}~\cite{agache_firecracker_2020}, as used by \textbf{TinyFaaS}~\cite{pfandzelter_tinyfaas_2020}, minimize resource usage and initialization delays but require careful orchestration to avoid resource contention. Function co-location, demonstrated by \textbf{Photons}~\cite{v_dukic_photons_2020}, reduces redundant initializations by sharing runtime resources among related functions, though this complicates isolation in multi-tenant setups~\cite{v_dukic_photons_2020}. Hybrid offloading frameworks like \textbf{GeoFaaS}~\cite{malekabbasi_geofaas_2024} balance edge-cloud workloads by offloading latency-tolerant tasks to the cloud and reserving edge resources for real-time operations, requiring reliable connectivity and efficient task management. These edge-specific strategies address cold starts effectively but introduce challenges in scalability and orchestration.

\subsection{Predictive Scaling and Caching Techniques}

Efficient resource allocation is vital for maintaining low latency and high availability in serverless platforms. Predictive scaling and caching techniques dynamically provision resources and reduce cold start latency by leveraging workload prediction and state retention.
Traditional FaaS platforms use predictive scaling and caching to optimize resources, employing techniques (OFC, FaasCache) to reduce cold starts. However, these methods rely on centralized orchestration and workload predictability, limiting their effectiveness in dynamic, resource-constrained edge environments.



\subsubsection{Edge FaaS Perspective}

Edge FaaS platforms adapt predictive scaling and caching techniques to constrain resources and heterogeneous environments. \textbf{EDGE-Cache}~\cite{kim_delay-aware_2022} uses traffic profiling to selectively retain high-priority functions, reducing memory overhead while maintaining readiness for frequent requests. Hybrid frameworks like \textbf{GeoFaaS}~\cite{malekabbasi_geofaas_2024} implement distributed caching to balance resources between edge and cloud nodes, enabling low-latency processing for critical tasks while offloading less critical workloads. Machine learning methods, such as clustering-based workload predictors~\cite{gao_machine_2020} and GRU-based models~\cite{guo_applying_2018}, enhance resource provisioning in edge systems by efficiently forecasting workload spikes. These innovations effectively address cold start challenges in edge environments, though their dependency on accurate predictions and robust orchestration poses scalability challenges.

\subsection{Decentralized Orchestration, Function Placement, and Scheduling}

Efficient orchestration in serverless platforms involves workload distribution, resource optimization, and performance assurance. While traditional FaaS platforms rely on centralized control, edge environments require decentralized and adaptive strategies to address unique challenges such as resource constraints and heterogeneous hardware.



\subsubsection{Edge FaaS Perspective}

Edge FaaS platforms adopt decentralized and adaptive orchestration frameworks to meet the demands of resource-constrained environments. Systems like \textbf{Wukong} distribute scheduling across edge nodes, enhancing data locality and scalability while reducing network latency. Lightweight frameworks such as \textbf{OpenWhisk Lite}~\cite{kravchenko_kpavelopenwhisk-light_2024} optimize resource allocation by decentralizing scheduling policies, minimizing cold starts and latency in edge setups~\cite{benjamin_wukong_2020}. Hybrid solutions like \textbf{OpenFaaS}~\cite{noauthor_openfaasfaas_2024} and \textbf{EdgeMatrix}~\cite{shen_edgematrix_2023} combine edge-cloud orchestration to balance resource utilization, retaining latency-sensitive functions at the edge while offloading non-critical workloads to the cloud. While these approaches improve flexibility, they face challenges in maintaining coordination and ensuring consistent performance across distributed nodes.


In this section, we discuss the physical effects observed in the testbed, such as the involved fluid mechanics and different forms of \ac{ISI} present including those that occur exclusively in closed-loop systems.
%
\scaleSubsection
\subsection{Signaling Molecule Propagation}\label{subsec:propagation}
\scaleSubsectionBelow
%
In our testbed, the propagation of the \ac{GFPD} signaling molecules is mainly affected by advection, facilitated by the bulk background flow of the buffer liquid. In the following, we analyze the flow characteristics and its parameters in the testbed.

When a fluid of kinematic viscosity $\nu$ flows with average flow speed $v_{\mathrm{eff}}$ through a pipe of radius $r_\mathrm{T}$, either laminar flow or turbulent flow occurs. 
Which type of flow is dominant can be predicted by the Reynolds number $\text{Re} = \frac{2 r_\mathrm{T} \veff}{\nu}$ \cite[p. 14]{darby2017chemical}. For increasing Reynolds numbers, the transition from the laminar to the turbulent flow regime occurs around $\text{Re} \approx 2300$ \cite[p. 12]{schlichting2016boundary}. The viscosity of the buffer solution in our testbed equals the viscosity of water \cite{hink2000structural}, i.e., $\nu = 1.037\, \times 10^{-6} \,\si{ \meter \squared \per \second}$ (at $21\, \si{\celsius}$). 
Here, for the parameter regime in which the testbed is operated, we obtain $\text{Re} = 167$, i.e., laminar flow is dominant. This value of $\text{Re}$ corresponds to a medium-sized artery, where the Reynolds number is typically between 100 and 1000 \cite{caro2012mechanics}.
Furthermore, we determine the relative influence of diffusion on the fluid transport, which can be characterized by the dimensionless Péclet number $\text{Pe} = \frac{r_\mathrm{T} \veff}{\D}$ \cite[eq. (4.44)]{tabeling2023introduction}. Here, utilizing the diffusion coefficient $ \D = 1\, \times 10^{-10} \,\si{\meter\squared \per \second}$ of \ac{GFPD} \cite{Junghans2016DiffusionGFPD}, $\text{Pe} = 1.11\, \times 10^{6} \gg 1$ follows. Therefore, flow dominates over diffusion in the system, and the latter can be neglected as a result.
%
\scaleSubsection
\subsection{ISI Caused by Closed-Loop Operation}\label{SubSec:ISI}
\scaleSubsectionBelow
%
The closed-loop character of our testbed gives rise to four distinct forms of \ac{ISI}: \textit{channel \ac{ISI}}, \textit{inter-loop \ac{ISI}}, \textit{offset \ac{ISI}}, and \textit{permanent \ac{ISI}}. The latter three forms can only be observed in closed-loop systems. 
All forms of \ac{ISI} are discussed in detail in the following. 
%
\scaleSubsubsection
\subsubsection{Channel ISI}\label{subsubsec:channel_isi}
\scaleSubsubsectionBelow
Within the channel, the propagation of OFF state \ac{GFPD} molecules from \ac{TX} to \ac{RX} may result in the overlap of consecutive transmitted symbols, leading to \ac{ISI}.
The existence of channel \ac{ISI} has been demonstrated with existing \ac{MC} testbeds, e.g., \cite{grebenstein2018biological, wang2020understanding}.
Channel \ac{ISI} can be mitigated by introducing a guard interval $T_\mathrm{G}$, during which the \ac{TX} is inactive, where the effectiveness of \ac{ISI} mitigation is contingent on the duration of the guard interval. However, increasing $T_\mathrm{G}$ also results in a reduction of the achievable data rate.
\scaleSubsubsection
\subsubsection{Inter-Loop ISI}\label{subsubsec:loop_isi}
\scaleSubsubsectionBelow
Due to the closed-loop design of the testbed, inter-loop \ac{ISI} occurs, as the OFF state \ac{GFPD} molecules remain in the system and reenter the \ac{RX} after completing one loop. Hence, these molecules interfere not only with neighboring symbols (as is the case for channel \ac{ISI}) but also with symbols transmitted much later in time. In particular, OFF state \ac{GFPD} molecules, which are switched
to the OFF state by the \ac{TX}, may travel several times through the loop before they spontaneously switch back to the ON state. Thus, these signaling molecules are likely to affect the received signal multiple times, while the intensity, timing, and number of recurrences of inter-loop \ac{ISI} depend on the length of the loop and the half-life $T_{1/2}$ of \ac{GFPD}. The intensity of inter-loop \ac{ISI} can be reduced by using the \ac{EX}.
%
\scaleSubsubsection
\subsubsection{Offset ISI}\label{subsubsec:offset_isi}
\scaleSubsubsectionBelow
As \ac{GFPD} molecules in the OFF state disperse in the system over multiple loops, the inter-loop \ac{ISI}, characterized by distinct drops in fluorescence intensity, undergoes a gradual transformation to a temporal offset in the observed fluorescence intensity, which we refer to as offset \ac{ISI}. The intensity of offset \ac{ISI} depends on the irradiation intensity used at the \ac{TX}, symbol duration $\Ti$, and half-life $T_{1/2}$ of \ac{GFPD}. The intensity of offset \ac{ISI} can be reduced by using the \ac{EX}.
%
\scaleSubsubsection
\subsubsection{Permanent ISI}\label{subsubsec:permanent_isi}
\scaleSubsubsectionBelow
After prolonged exposure to light, \ac{GFPD} molecules may undergo an irreversible degradation as a consequence of photobleaching effects (cf. \Section{par:photobleaching}). While the decrease in fluorescence intensity resulting from offset \ac{ISI} is reversible, photobleaching is not. Thus, photobleaching causes an irreversible reduction of the number of functional \ac{GFPD} molecules available for modulation. Consequently, this leads to an overall decrease in the received fluorescence signal intensity over time. As a result, we observe a deterioration of the communication performance.
Since \ac{GFPD} molecules are irradiated by the \ac{EX}, the \ac{RX}, and the \ac{TX} in the testbed, all three light sources contribute to photobleaching. Therefore, we refer to the fractions caused by the \ac{TX}, \ac{EX}, and \ac{RX} as permanent \ac{ISI}, \ac{EX} bleaching, and measurement bleaching, respectively. Note that only photobleaching caused by the \ac{TX} is interpreted as \ac{ISI}, since it can be varied by varying the transmitted signal, whereas the photobleaching caused by \ac{EX} and \ac{RX} is independent of the transmitted sequence.
\section{\disco{} Policies}
\label{sec:disco_policy}

\disco{} optimizes both QoE and cost through (1) dispatch control that determines where to initiate token generation, and (2) migration control that enables dynamic handoff during generation. The dispatch controller optimizes TTFT by strategically routing requests, while the migration controller maintains consistent TBT while reducing costs.

\subsection{Problem Formulation}
\label{subsec:model}
We propose a unified cost model combining both monetary bills from on-server inference and energy bills from on-device inference. Let $c^p_s$ and $c^d_s$ denote the per-token monetary costs for server prefill and decode phases respectively, while $c^p_d$ and $c^d_d$ represent the per-token energy costs for device prefill and decode phases. Converting between energy and monetary costs is done by a dynamic exchange rate $\lambda$, adjusted by users to reflect their preferences. We offer a user-friendly tunable budget ratio $b \in [0,1]$, representing the additional cost allowance beyond baseline costs. Our optimization objectives focus on: (1) minimizing both mean and tail TTFT, and (2) maintaining consistent token delivery at a specified pace (i.e., stable TBT).

\subsection{Dispatch Controller: Cost-Aware Request Routing}
\label{subsec:ttft_opt}
Based on our analysis in \S\ref{sec:characteristics}, server-side TTFT shows weak correlation with prompt length due to various factors (network delay, request queueing, etc.). We model server TTFTs as a known distribution, obtained either from server-provided information or device-side profiling. In contrast, device-side TTFT exhibits a linear relationship with prompt length, with the coefficient determined through offline profiling. 

Our key insight is that the optimization problem naturally decomposes into two scenarios based on dominant cost factors: device-constrained scenarios where energy consumption is the primary bottleneck, and server-constrained scenarios where API monetary costs dominate. This decomposition enables efficient solutions. Pseudocode for the dispatch controller is in Appendix~\ref{appendix:pseudocode_scheduling}.

\paragraph{Device-Constrained Optimization.} 

When device costs dominate ($\min(c^p_d, c^d_d) > \max(c^p_s, c^d_s)$), we need to carefully manage device resource usage under a budget constraint $\mathbb{E}[I_d(l)l] \leq b \cdot \mathbb{E}[l]$, where $l$ is the prompt length and $I_d(l)$ indicates device execution. The key challenge is balancing between two goals: leveraging device execution to bound worst-case latency while conserving energy on shorter prompts where possible.

Our solution uses a wait-time strategy: for each prompt of length $l$, we first try server execution and wait for time $w(l)$ before potentially starting device execution. This conserves device energy when the server responds quickly. We determine the optimal wait time through a two-phase approach:

\begin{denseitemize}
    \item \textbf{Phase 1 (Tail Protection):} We reserve budget portion $\alpha$ for worst-case scenarios by setting a maximum wait time $w_{tail} = F^{-1}(1-\min(\alpha,b))$, where $F(\cdot)$ is the server TTFT distribution. This ensures we have device resources ready when server latency exceeds its $(1-\min(\alpha,b))$-th percentile.
    
    \item \textbf{Phase 2 (Average Case):} With the remaining budget $(b-\alpha)$, we set length-dependent wait times:
    \begin{equation}
        w(l) = \begin{cases}
            0 & \text{if } l \leq l_{th} \\
            \min(\beta l, w_{tail}) & \text{otherwise}
        \end{cases}
    \end{equation}
    where $l_{th}$ is a threshold below which we start device execution immediately, and $\beta$ is chosen to satisfy:
    \begin{equation}
        \int_{l_{th}}^{\infty} (1-F(\beta l)) \cdot c^p_d \cdot l \cdot p(l)dl = (b-\alpha) \cdot \mathbb{E}[l]
    \end{equation}
\end{denseitemize}

This design guarantees worst-case TTFT through $w_{tail}$ while optimizing average performance by adaptively adjusting wait times based on prompt length. Whichever endpoint (server or device) generates the first token continues to the decode phase, while the other terminates.

\paragraph{Server-Constrained Optimization.}
When server costs dominate ($\max(c^p_s, c^d_s) > \min(c^p_d, c^d_d)$), we need to carefully manage server resource usage under a budget constraint $\mathbb{E}[I_s(l)l] \leq b \cdot \mathbb{E}[l]$, where $I_s(l)$ indicates server execution. Our analysis in \S\ref{sec:characteristics} shows that device TTFT scales linearly with prompt length as $T_d(l) = kl + c$, while server TTFT has minimal length correlation. This suggests a length-based routing strategy: short prompts run on device to conserve server budget, while long prompts use both endpoints to minimize TTFT.

We determine the length threshold $l_{th}$ by:
\begin{equation}
  \int_0^{l_{th}} l \cdot p(l) dl = (1-b) \cdot \mathbb{E}[l]
\end{equation}
This ensures prompts shorter than $l_{th}$ consume exactly $(1-b)$ fraction of total expected tokens through device-only execution, leaving the remaining longer prompts with sufficient server budget for concurrent execution on both endpoints.

\subsection{Migration Controller: Cost-Efficient Token Delivery}\label{sec:migration}
When both endpoints process a request, the constrained endpoint may win the prefill phase but incur higher decode costs. In such cases, we can migrate token generation to the other endpoint to reduce total cost while maintaining quality.

\paragraph{Efficient Token Transfer.}

When endpoints share the same vocabulary, we transmit token IDs rather than complete token representations. Additionally, we avoid transferring intermediate states (e.g., attention key-value cache) for two practical reasons: (1) endpoints often employ different model architectures optimized for their respective hardware, making state transfer incompatible, and (2) intermediate state transfer would incur significant network overhead. Migration triggers when projected cost savings exceed overhead:
\begin{equation}
    C_{migration} = \Delta c^d_{decode} \times l_{remaining}
\end{equation}
where $\Delta c^d_{decode} = |c^d_s - c^d_d|$ and $l_{remaining}$ denote the per-token decode cost difference between endpoints, and the expected remaining sequence length, respectively.

\paragraph{Buffer-Based Migration Protocol.}
To ensure smooth token delivery during migration, we introduce a token buffer that leverages the natural gap between token generation speed ($r_g$ tokens/s) and human consumption rate ($r_c$ tokens/s, typically $r_g > r_c$). The buffer size is set to:
\begin{equation}
    B = r_c \times t_m
\end{equation}
where $t_m$ is the estimated migration overhead time. Migration begins only when the buffer contains enough tokens ($B$) to mask the migration latency.

As shown in Figure~\ref{fig:cost_saving}, this design enables seamless handoff: the source endpoint (Row A) continues generation until the target endpoint (Row B) is ready, ensuring uninterrupted token delivery to users despite the underlying endpoint transition.

\begin{figure}[t]
    \centering
    \includegraphics[width=0.45\textwidth]{figs/migration.pdf}
    \caption{Token generation migration between endpoints. Row A shows the original sequence on the source endpoint, while Row B shows the sequence after migration to the target endpoint, maintaining consistent token delivery while reducing cost.}
    \label{fig:cost_saving}
\end{figure}

\definecolor{darkgreen}{rgb}{0.0, 0.5, 0.0}
\definecolor{violet}{rgb}{0.56, 0.0, 1.0}
\section{Evaluation}
We apply our methodology to derive counterfactual policies for various MDPs, addressing three main research questions: (1) how does our policy's performance compare to the Gumbel-max SCM approach; (2) how do the counterfactual stability and monotonicity assumptions impact the probability bounds; and (3) how fast is our approach compared with the Gumbel-max SCM method?

\begin{figure*}
    \centering
    %
    \resizebox{0.6\textwidth}{!}{
        \begin{tikzpicture}[scale=1.0, every node/.style={scale=1.0}]
            \draw[thick, black] (-3, -0.25) rectangle (10, 0.25);
            %
            \draw[black, line width=1pt] (-2.5, 0.0) -- (-2,0.0);
            \fill[black] (-2.25,0.0) circle (2pt); %
            \node[right] at (-2,0.0) {\small Observed Path};
            
            %
            \draw[blue, line width=1pt] (1.0,0.0) -- (1.5,0.0);
            \node[draw=blue, circle, minimum size=4pt, inner sep=0pt] at (1.25,0.0) {}; %
            \node[right] at (1.5,0.0) {\small Interval CFMDP Policy};
            
            %
            \draw[red, line width=1pt] (5.5,0) -- (6,0);
            \node[red] at (5.75,0) {$\boldsymbol{\times}$}; %
            \node[right] at (6,0) {\small Gumbel-max SCM Policy};
        \end{tikzpicture}
    }\\
    %
    \subfigure[\footnotesize Lowest cumulative reward: Interval CFMDP ($312$), Gumbel-max SCM ($312$)]{%
        \resizebox{0.76\columnwidth}{!}{
             \begin{tikzpicture}
                \begin{axis}[
                    xlabel={$t$},
                    ylabel={Mean reward at time step $t$},
                    title={Optimal Path},
                    grid=both,
                    width=20cm, height=8.5cm,
                    every axis/.style={font=\Huge},
                    %
                ]
                \addplot[
                    color=black, %
                    mark=*, %
                    line width=2pt,
                    mark size=3pt,
                    error bars/.cd,
                    y dir=both, %
                    y explicit, %
                    error bar style={line width=1pt,solid},
                    error mark options={line width=1pt,mark size=4pt,rotate=90}
                ]
                coordinates {
                    (0, 0.0)  +- (0, 0.0)
                    (1, 0.0)  +- (0, 0.0) 
                    (2, 1.0)  +- (0, 0.0) 
                    (3, 1.0)  +- (0, 0.0)
                    (4, 2.0)  +- (0, 0.0)
                    (5, 3.0) +- (0, 0.0)
                    (6, 5.0) +- (0, 0.0)
                    (7, 100.0) +- (0, 0.0)
                    (8, 100.0) +- (0, 0.0)
                    (9, 100.0) +- (0, 0.0)
                };
                %
                \addplot[
                    color=blue, %
                    mark=o, %
                    line width=2pt,
                    mark size=3pt,
                    error bars/.cd,
                    y dir=both, %
                    y explicit, %
                    error bar style={line width=1pt,solid},
                    error mark options={line width=1pt,mark size=4pt,rotate=90}
                ]
                 coordinates {
                    (0, 0.0)  +- (0, 0.0)
                    (1, 0.0)  +- (0, 0.0) 
                    (2, 1.0)  +- (0, 0.0) 
                    (3, 1.0)  +- (0, 0.0)
                    (4, 2.0)  +- (0, 0.0)
                    (5, 3.0) +- (0, 0.0)
                    (6, 5.0) +- (0, 0.0)
                    (7, 100.0) +- (0, 0.0)
                    (8, 100.0) +- (0, 0.0)
                    (9, 100.0) +- (0, 0.0)
                };
                %
                \addplot[
                    color=red, %
                    mark=x, %
                    line width=2pt,
                    mark size=6pt,
                    error bars/.cd,
                    y dir=both, %
                    y explicit, %
                    error bar style={line width=1pt,solid},
                    error mark options={line width=1pt,mark size=4pt,rotate=90}
                ]
                coordinates {
                    (0, 0.0)  +- (0, 0.0)
                    (1, 0.0)  +- (0, 0.0) 
                    (2, 1.0)  +- (0, 0.0) 
                    (3, 1.0)  +- (0, 0.0)
                    (4, 2.0)  +- (0, 0.0)
                    (5, 3.0) +- (0, 0.0)
                    (6, 5.0) +- (0, 0.0)
                    (7, 100.0) +- (0, 0.0)
                    (8, 100.0) +- (0, 0.0)
                    (9, 100.0) +- (0, 0.0)
                };
                \end{axis}
            \end{tikzpicture}
         }
    }
    \hspace{1cm}
    \subfigure[\footnotesize Lowest cumulative reward: Interval CFMDP ($19$), Gumbel-max SCM ($-88$)]{%
         \resizebox{0.76\columnwidth}{!}{
            \begin{tikzpicture}
                \begin{axis}[
                    xlabel={$t$},
                    ylabel={Mean reward at time step $t$},
                    title={Slightly Suboptimal Path},
                    grid=both,
                    width=20cm, height=8.5cm,
                    every axis/.style={font=\Huge},
                    %
                ]
                \addplot[
                    color=black, %
                    mark=*, %
                    line width=2pt,
                    mark size=3pt,
                    error bars/.cd,
                    y dir=both, %
                    y explicit, %
                    error bar style={line width=1pt,solid},
                    error mark options={line width=1pt,mark size=4pt,rotate=90}
                ]
              coordinates {
                    (0, 0.0)  +- (0, 0.0)
                    (1, 1.0)  +- (0, 0.0) 
                    (2, 1.0)  +- (0, 0.0) 
                    (3, 1.0)  +- (0, 0.0)
                    (4, 2.0)  +- (0, 0.0)
                    (5, 3.0) +- (0, 0.0)
                    (6, 3.0) +- (0, 0.0)
                    (7, 2.0) +- (0, 0.0)
                    (8, 2.0) +- (0, 0.0)
                    (9, 4.0) +- (0, 0.0)
                };
                %
                \addplot[
                    color=blue, %
                    mark=o, %
                    line width=2pt,
                    mark size=3pt,
                    error bars/.cd,
                    y dir=both, %
                    y explicit, %
                    error bar style={line width=1pt,solid},
                    error mark options={line width=1pt,mark size=4pt,rotate=90}
                ]
              coordinates {
                    (0, 0.0)  +- (0, 0.0)
                    (1, 1.0)  +- (0, 0.0) 
                    (2, 1.0)  +- (0, 0.0) 
                    (3, 1.0)  +- (0, 0.0)
                    (4, 2.0)  +- (0, 0.0)
                    (5, 3.0) +- (0, 0.0)
                    (6, 3.0) +- (0, 0.0)
                    (7, 2.0) +- (0, 0.0)
                    (8, 2.0) +- (0, 0.0)
                    (9, 4.0) +- (0, 0.0)
                };
                %
                \addplot[
                    color=red, %
                    mark=x, %
                    line width=2pt,
                    mark size=6pt,
                    error bars/.cd,
                    y dir=both, %
                    y explicit, %
                    error bar style={line width=1pt,solid},
                    error mark options={line width=1pt,mark size=4pt,rotate=90}
                ]
                coordinates {
                    (0, 0.0)  +- (0, 0.0)
                    (1, 1.0)  +- (0, 0.0) 
                    (2, 1.0)  +- (0, 0.0) 
                    (3, 1.0)  +- (0, 0.0)
                    (4, 2.0)  += (0, 0.0)
                    (5, 3.0)  += (0, 0.0)
                    (6, 3.17847) += (0, 0.62606746) -= (0, 0.62606746)
                    (7, 2.5832885) += (0, 1.04598233) -= (0, 1.04598233)
                    (8, 5.978909) += (0, 17.60137623) -= (0, 17.60137623)
                    (9, 5.297059) += (0, 27.09227512) -= (0, 27.09227512)
                };
                \end{axis}
            \end{tikzpicture}
         }
    }\\[-1.5pt]
    \subfigure[\footnotesize Lowest cumulative reward: Interval CFMDP ($14$), Gumbel-max SCM ($-598$)]{%
         \resizebox{0.76\columnwidth}{!}{
             \begin{tikzpicture}
                \begin{axis}[
                    xlabel={$t$},
                    ylabel={Mean reward at time step $t$},
                    title={Almost Catastrophic Path},
                    grid=both,
                    width=20cm, height=8.5cm,
                    every axis/.style={font=\Huge},
                    %
                ]
                \addplot[
                    color=black, %
                    mark=*, %
                    line width=2pt,
                    mark size=3pt,
                    error bars/.cd,
                    y dir=both, %
                    y explicit, %
                    error bar style={line width=1pt,solid},
                    error mark options={line width=1pt,mark size=4pt,rotate=90}
                ]
                coordinates {
                    (0, 0.0)  +- (0, 0.0)
                    (1, 1.0)  +- (0, 0.0) 
                    (2, 2.0)  +- (0, 0.0) 
                    (3, 1.0)  +- (0, 0.0)
                    (4, 0.0)  +- (0, 0.0)
                    (5, 1.0) +- (0, 0.0)
                    (6, 2.0) +- (0, 0.0)
                    (7, 2.0) +- (0, 0.0)
                    (8, 3.0) +- (0, 0.0)
                    (9, 2.0) +- (0, 0.0)
                };
                %
                \addplot[
                    color=blue, %
                    mark=o, %
                    line width=2pt,
                    mark size=3pt,
                    error bars/.cd,
                    y dir=both, %
                    y explicit, %
                    error bar style={line width=1pt,solid},
                    error mark options={line width=1pt,mark size=4pt,rotate=90}
                ]
                coordinates {
                    (0, 0.0)  +- (0, 0.0)
                    (1, 1.0)  +- (0, 0.0) 
                    (2, 2.0)  +- (0, 0.0) 
                    (3, 1.0)  +- (0, 0.0)
                    (4, 0.0)  +- (0, 0.0)
                    (5, 1.0) +- (0, 0.0)
                    (6, 2.0) +- (0, 0.0)
                    (7, 2.0) +- (0, 0.0)
                    (8, 3.0) +- (0, 0.0)
                    (9, 2.0) +- (0, 0.0)
                };
                %
                \addplot[
                    color=red, %
                    mark=x, %
                    line width=2pt,
                    mark size=6pt,
                    error bars/.cd,
                    y dir=both, %
                    y explicit, %
                    error bar style={line width=1pt,solid},
                    error mark options={line width=1pt,mark size=4pt,rotate=90}
                ]
                coordinates {
                    (0, 0.0)  +- (0, 0.0)
                    (1, 0.7065655)  +- (0, 0.4553358) 
                    (2, 1.341673)  +- (0, 0.67091621) 
                    (3, 1.122926)  +- (0, 0.61281824)
                    (4, -1.1821935)  +- (0, 13.82444042)
                    (5, -0.952399)  +- (0, 15.35195457)
                    (6, -0.72672) +- (0, 20.33508414)
                    (7, -0.268983) +- (0, 22.77861454)
                    (8, -0.1310835) +- (0, 26.31013314)
                    (9, 0.65806) +- (0, 28.50670214)
                };
                %
            %
            %
            %
            %
            %
            %
            %
            %
            %
            %
            %
            %
            %
            %
            %
            %
            %
            %
                \end{axis}
            \end{tikzpicture}
         }
    }
    \hspace{1cm}
    \subfigure[\footnotesize Lowest cumulative reward: Interval CFMDP ($-698$), Gumbel-max SCM ($-698$)]{%
         \resizebox{0.76\columnwidth}{!}{
            \begin{tikzpicture}
                \begin{axis}[
                    xlabel={$t$},
                    ylabel={Mean reward at time step $t$},
                    title={Catastrophic Path},
                    grid=both,
                    width=20cm, height=8.5cm,
                    every axis/.style={font=\Huge},
                    %
                ]
                \addplot[
                    color=black, %
                    mark=*, %
                    line width=2pt,
                    mark size=3pt,
                    error bars/.cd,
                    y dir=both, %
                    y explicit, %
                    error bar style={line width=1pt,solid},
                    error mark options={line width=1pt,mark size=4pt,rotate=90}
                ]
                coordinates {
                    (0, 1.0)  +- (0, 0.0)
                    (1, 2.0)  +- (0, 0.0) 
                    (2, -100.0)  +- (0, 0.0) 
                    (3, -100.0)  +- (0, 0.0)
                    (4, -100.0)  +- (0, 0.0)
                    (5, -100.0) +- (0, 0.0)
                    (6, -100.0) +- (0, 0.0)
                    (7, -100.0) +- (0, 0.0)
                    (8, -100.0) +- (0, 0.0)
                    (9, -100.0) +- (0, 0.0)
                };
                %
                \addplot[
                    color=blue, %
                    mark=o, %
                    line width=2pt,
                    mark size=3pt,
                    error bars/.cd,
                    y dir=both, %
                    y explicit, %
                    error bar style={line width=1pt,solid},
                    error mark options={line width=1pt,mark size=4pt,rotate=90}
                ]
                coordinates {
                    (0, 0.0)  +- (0, 0.0)
                    (1, 0.504814)  +- (0, 0.49997682) 
                    (2, 0.8439835)  +- (0, 0.76831917) 
                    (3, -8.2709165)  +- (0, 28.93656754)
                    (4, -9.981082)  +- (0, 31.66825363)
                    (5, -12.1776325) +- (0, 34.53463233)
                    (6, -13.556076) +- (0, 38.62845372)
                    (7, -14.574418) +- (0, 42.49603359)
                    (8, -15.1757075) +- (0, 46.41913968)
                    (9, -15.3900395) +- (0, 50.33563368)
                };
                %
                \addplot[
                    color=red, %
                    mark=x, %
                    line width=2pt,
                    mark size=6pt,
                    error bars/.cd,
                    y dir=both, %
                    y explicit, %
                    error bar style={line width=1pt,solid},
                    error mark options={line width=1pt,mark size=4pt,rotate=90}
                ]
                coordinates {
                    (0, 0.0)  +- (0, 0.0)
                    (1, 0.701873)  +- (0, 0.45743556) 
                    (2, 1.1227805)  +- (0, 0.73433129) 
                    (3, -8.7503255)  +- (0, 30.30257976)
                    (4, -10.722092)  +- (0, 33.17618589)
                    (5, -13.10721)  +- (0, 36.0648089)
                    (6, -13.7631645) +- (0, 40.56553451)
                    (7, -13.909043) +- (0, 45.23829402)
                    (8, -13.472517) +- (0, 49.96270296)
                    (9, -12.8278835) +- (0, 54.38618735)
                };
                %
            %
            %
            %
            %
            %
            %
            %
            %
            %
            %
            %
            %
            %
            %
            %
            %
            %
            %
                \end{axis}
            \end{tikzpicture}
         }
    }
    \caption{Average instant reward of CF paths induced by policies on GridWorld $p=0.4$.}
    \label{fig: reward p=0.4}
\end{figure*}

\subsection{Experimental Setup}
To compare policy performance, we measure the average rewards of counterfactual paths induced by our policy and the Gumbel-max policy by uniformly sampling $200$ counterfactual MDPs from the ICFMDP and generating $10,000$ counterfactual paths over each sampled CFMDP. \jl{Since the interval CFMDP depends on the observed path, we select $4$  paths of varying optimality to evaluate how the observed path impacts the performance of both policies: an optimal path, a slightly suboptimal path that could reach the optimal reward with a few changes, a catastrophic path that enters a catastrophic, terminal state with low reward, and an almost catastrophic path that was close to entering a catastrophic state.} When measuring the average probability bound widths and execution time needed to generate the ICFMDPs, we averaged over $20$ randomly generated observed paths
\footnote{Further training details are provided in Appendix \ref{app: training details}, and the code is provided at \href{https://github.com/ddv-lab/robust-cf-inference-in-MDPs}{https://github.com/ddv-lab/robust-cf-inference-in-MDPs}
%
%
.}.

\subsection{GridWorld}
\jl{The GridWorld MDP is a $4 \times 4$ grid where an agent must navigate from the top-left corner to the goal state in the bottom-right corner, avoiding a dangerous terminal state in the centre. At each time step, the agent can move up, down, left, or right, but there is a small probability (controlled by hyper-parameter $p$) of moving in an unintended direction. As the agent nears the goal, the reward for each state increases, culminating in a reward of $+100$ for reaching the goal. Entering the dangerous state results in a penalty of $-100$. We use two versions of GridWorld: a less stochastic version with $p=0.9$ (i.e., $90$\% chance of moving in the chosen direction) and a more stochastic version with $p=0.4$.}

\paragraph{GridWorld ($p=0.9$)}
When $p=0.9$, the counterfactual probability bounds are typically narrow (see Table \ref{tab:nonzero_probs} for average measurements). Consequently, as shown in Figure \ref{fig: reward p=0.9}, both policies are nearly identical and perform similarly well across the optimal, slightly suboptimal, and catastrophic paths.
%
However, for the almost catastrophic path, the interval CFMDP path is more conservative and follows the observed path more closely (as this is where the probability bounds are narrowest), which typically requires one additional step to reach the goal state than the Gumbel-max SCM policy.
%

\paragraph{GridWorld ($p=0.4$)}
\jl{When $p=0.4$, the GridWorld environment becomes more uncertain, increasing the risk of entering the dangerous state even if correct actions are chosen. Thus, as shown in Figure \ref{fig: reward p=0.4}, the interval CFMDP policy adopts a more conservative approach, avoiding deviation from the observed policy if it cannot guarantee higher counterfactual rewards (see the slightly suboptimal and almost catastrophic paths), whereas the Gumbel-max SCM is inconsistent: it can yield higher rewards, but also much lower rewards, reflected in the wide error bars.} For the catastrophic path, both policies must deviate from the observed path to achieve a higher reward and, in this case, perform similarly.
%
%
%
%
\subsection{Sepsis}
The Sepsis MDP \citep{oberst2019counterfactual} simulates trajectories of Sepsis patients. Each state consists of four vital signs (heart rate, blood pressure, oxygen concentration, and glucose levels), categorised as low, normal, or high.
and three treatments that can be toggled on/off at each time step (8 actions in total). Unlike \citet{oberst2019counterfactual}, we scale rewards based on the number of out-of-range vital signs, between $-1000$ (patient dies) and $1000$ (patient discharged). \jl{Like the GridWorld $p=0.4$ experiment, the Sepsis MDP is highly uncertain, as many states are equally likely to lead to optimal and poor outcomes. Thus, as shown in Figure \ref{fig: reward sepsis}, both policies follow the observed optimal and almost catastrophic paths to guarantee rewards are no worse than the observation.} However, improving the catastrophic path requires deviating from the observation. Here, the Gumbel-max SCM policy, on average, performs better than the interval CFMDP policy. But, since both policies have lower bounds clipped at $-1000$, neither policy reliably improves over the observation. In contrast, for the slightly suboptimal path, the interval CFMDP policy performs significantly better, shown by its higher lower bounds. 
Moreover, in these two cases, the worst-case counterfactual path generated by the interval CFMDP policy is better than that of the Gumbel-max SCM policy,
indicating its greater robustness.
%
\begin{figure*}
    \centering
     \resizebox{0.6\textwidth}{!}{
        \begin{tikzpicture}[scale=1.0, every node/.style={scale=1.0}]
            \draw[thick, black] (-3, -0.25) rectangle (10, 0.25);
            %
            \draw[black, line width=1pt] (-2.5, 0.0) -- (-2,0.0);
            \fill[black] (-2.25,0.0) circle (2pt); %
            \node[right] at (-2,0.0) {\small Observed Path};
            
            %
            \draw[blue, line width=1pt] (1.0,0.0) -- (1.5,0.0);
            \node[draw=blue, circle, minimum size=4pt, inner sep=0pt] at (1.25,0.0) {}; %
            \node[right] at (1.5,0.0) {\small Interval CFMDP Policy};
            
            %
            \draw[red, line width=1pt] (5.5,0) -- (6,0);
            \node[red] at (5.75,0) {$\boldsymbol{\times}$}; %
            \node[right] at (6,0) {\small Gumbel-max SCM Policy};
        \end{tikzpicture}
    }\\
    \subfigure[\footnotesize Lowest cumulative reward: Interval CFMDP ($8000$), Gumbel-max SCM ($8000$)]{%
         \resizebox{0.76\columnwidth}{!}{
             \begin{tikzpicture}
                \begin{axis}[
                    xlabel={$t$},
                    ylabel={Mean reward at time step $t$},
                    title={Optimal Path},
                    grid=both,
                    width=20cm, height=8.5cm,
                    every axis/.style={font=\Huge},
                    %
                ]
                \addplot[
                    color=black, %
                    mark=*, %
                    line width=2pt,
                    mark size=3pt,
                ]
                coordinates {
                    (0, -50.0)
                    (1, 50.0)
                    (2, 1000.0)
                    (3, 1000.0)
                    (4, 1000.0)
                    (5, 1000.0)
                    (6, 1000.0)
                    (7, 1000.0)
                    (8, 1000.0)
                    (9, 1000.0)
                };
                %
                \addplot[
                    color=blue, %
                    mark=o, %
                    line width=2pt,
                    mark size=3pt,
                    error bars/.cd,
                    y dir=both, %
                    y explicit, %
                    error bar style={line width=1pt,solid},
                    error mark options={line width=1pt,mark size=4pt,rotate=90}
                ]
                coordinates {
                    (0, -50.0)  +- (0, 0.0)
                    (1, 50.0)  +- (0, 0.0) 
                    (2, 1000.0)  +- (0, 0.0) 
                    (3, 1000.0)  +- (0, 0.0)
                    (4, 1000.0)  +- (0, 0.0)
                    (5, 1000.0) +- (0, 0.0)
                    (6, 1000.0) +- (0, 0.0)
                    (7, 1000.0) +- (0, 0.0)
                    (8, 1000.0) +- (0, 0.0)
                    (9, 1000.0) +- (0, 0.0)
                };
                %
                \addplot[
                    color=red, %
                    mark=x, %
                    line width=2pt,
                    mark size=6pt,
                    error bars/.cd,
                    y dir=both, %
                    y explicit, %
                    error bar style={line width=1pt,solid},
                    error mark options={line width=1pt,mark size=4pt,rotate=90}
                ]
                coordinates {
                    (0, -50.0)  +- (0, 0.0)
                    (1, 50.0)  +- (0, 0.0) 
                    (2, 1000.0)  +- (0, 0.0) 
                    (3, 1000.0)  +- (0, 0.0)
                    (4, 1000.0)  +- (0, 0.0)
                    (5, 1000.0) +- (0, 0.0)
                    (6, 1000.0) +- (0, 0.0)
                    (7, 1000.0) +- (0, 0.0)
                    (8, 1000.0) +- (0, 0.0)
                    (9, 1000.0) +- (0, 0.0)
                };
                %
                \end{axis}
            \end{tikzpicture}
         }
    }
    \hspace{1cm}
    \subfigure[\footnotesize Lowest cumulative reward: Interval CFMDP ($-5980$), Gumbel-max SCM ($-8000$)]{%
         \resizebox{0.76\columnwidth}{!}{
            \begin{tikzpicture}
                \begin{axis}[
                    xlabel={$t$},
                    ylabel={Mean reward at time step $t$},
                    title={Slightly Suboptimal Path},
                    grid=both,
                    width=20cm, height=8.5cm,
                    every axis/.style={font=\Huge},
                    %
                ]
               \addplot[
                    color=black, %
                    mark=*, %
                    line width=2pt,
                    mark size=3pt,
                ]
                coordinates {
                    (0, -50.0)
                    (1, 50.0)
                    (2, -50.0)
                    (3, -50.0)
                    (4, -1000.0)
                    (5, -1000.0)
                    (6, -1000.0)
                    (7, -1000.0)
                    (8, -1000.0)
                    (9, -1000.0)
                };
                %
                \addplot[
                    color=blue, %
                    mark=o, %
                    line width=2pt,
                    mark size=3pt,
                    error bars/.cd,
                    y dir=both, %
                    y explicit, %
                    error bar style={line width=1pt,solid},
                    error mark options={line width=1pt,mark size=4pt,rotate=90}
                ]
                coordinates {
                    (0, -50.0)  +- (0, 0.0)
                    (1, 50.0)  +- (0, 0.0) 
                    (2, -50.0)  +- (0, 0.0) 
                    (3, 20.0631)  +- (0, 49.97539413)
                    (4, 71.206585)  +- (0, 226.02033693)
                    (5, 151.60797) +- (0, 359.23292559)
                    (6, 200.40593) +- (0, 408.86185176)
                    (7, 257.77948) +- (0, 466.10372804)
                    (8, 299.237465) +- (0, 501.82579506)
                    (9, 338.9129) +- (0, 532.06124996)
                };
                %
                \addplot[
                    color=red, %
                    mark=x, %
                    line width=2pt,
                    mark size=6pt,
                    error bars/.cd,
                    y dir=both, %
                    y explicit, %
                    error bar style={line width=1pt,solid},
                    error mark options={line width=1pt,mark size=4pt,rotate=90}
                ]
                coordinates {
                    (0, -50.0)  +- (0, 0.0)
                    (1, 20.00736)  +- (0, 49.99786741) 
                    (2, -12.282865)  +- (0, 267.598755) 
                    (3, -47.125995)  +- (0, 378.41755832)
                    (4, -15.381965)  +- (0, 461.77616558)
                    (5, 41.15459) +- (0, 521.53189262)
                    (6, 87.01595) +- (0, 564.22243126 )
                    (7, 132.62376) +- (0, 607.31338037)
                    (8, 170.168145) +- (0, 641.48013693)
                    (9, 201.813135) +- (0, 667.29441777)
                };
                %
                %
                %
                %
                %
                %
                %
                %
                %
                %
                %
                %
                %
                %
                %
                %
                %
                %
                %
                \end{axis}
            \end{tikzpicture}
         }
    }\\[-1.5pt]
    \subfigure[\footnotesize Lowest cumulative reward: Interval CFMDP ($100$), Gumbel-max SCM ($100$)]{%
         \resizebox{0.76\columnwidth}{!}{
             \begin{tikzpicture}
                \begin{axis}[
                    xlabel={$t$},
                    ylabel={Mean reward at time step $t$},
                    title={Almost Catastrophic Path},
                    grid=both,
                    every axis/.style={font=\Huge},
                    width=20cm, height=8.5cm,
                    %
                ]
               \addplot[
                    color=black, %
                    mark=*, %
                    line width=2pt,
                    mark size=3pt,
                ]
                coordinates {
                    (0, -50.0)
                    (1, 50.0)
                    (2, 50.0)
                    (3, 50.0)
                    (4, -50.0)
                    (5, 50.0)
                    (6, -50.0)
                    (7, 50.0)
                    (8, -50.0)
                    (9, 50.0)
                };
                %
                %
                \addplot[
                    color=blue, %
                    mark=o, %
                    line width=2pt,
                    mark size=3pt,
                    error bars/.cd,
                    y dir=both, %
                    y explicit, %
                    error bar style={line width=1pt,solid},
                    error mark options={line width=1pt,mark size=4pt,rotate=90}
                ]
                coordinates {
                    (0, -50.0)  +- (0, 0.0)
                    (1, 50.0)  +- (0, 0.0) 
                    (2, 50.0)  +- (0, 0.0) 
                    (3, 50.0)  +- (0, 0.0)
                    (4, -50.0)  +- (0, 0.0)
                    (5, 50.0) +- (0, 0.0)
                    (6, -50.0) +- (0, 0.0)
                    (7, 50.0) +- (0, 0.0)
                    (8, -50.0) +- (0, 0.0)
                    (9, 50.0) +- (0, 0.0)
                };
                %
                \addplot[
                    color=red, %
                    mark=x, %
                    line width=2pt,
                    mark size=6pt,
                    error bars/.cd,
                    y dir=both, %
                    y explicit, %
                    error bar style={line width=1pt,solid},
                    error mark options={line width=1pt,mark size=4pt,rotate=90}
                ]
                coordinates {
                    (0, -50.0)  +- (0, 0.0)
                    (1, 50.0)  +- (0, 0.0) 
                    (2, 50.0)  +- (0, 0.0) 
                    (3, 50.0)  +- (0, 0.0)
                    (4, -50.0)  +- (0, 0.0)
                    (5, 50.0) +- (0, 0.0)
                    (6, -50.0) +- (0, 0.0)
                    (7, 50.0) +- (0, 0.0)
                    (8, -50.0) +- (0, 0.0)
                    (9, 50.0) +- (0, 0.0)
                };
                %
                %
                %
                %
                %
                %
                %
                %
                %
                %
                %
                %
                %
                %
                %
                %
                %
                %
                %
                \end{axis}
            \end{tikzpicture}
         }
    }
    \hspace{1cm}
    \subfigure[\footnotesize Lowest cumulative reward: Interval CFMDP ($-7150$), Gumbel-max SCM ($-9050$)]{%
         \resizebox{0.76\columnwidth}{!}{
            \begin{tikzpicture}
                \begin{axis}[
                    xlabel={$t$},
                    ylabel={Mean reward at time step $t$},
                    title={Catastrophic Path},
                    grid=both,
                    width=20cm, height=8.5cm,
                    every axis/.style={font=\Huge},
                    %
                ]
               \addplot[
                    color=black, %
                    mark=*, %
                    line width=2pt,
                    mark size=3pt,
                ]
                coordinates {
                    (0, -50.0)
                    (1, -50.0)
                    (2, -1000.0)
                    (3, -1000.0)
                    (4, -1000.0)
                    (5, -1000.0)
                    (6, -1000.0)
                    (7, -1000.0)
                    (8, -1000.0)
                    (9, -1000.0)
                };
                %
                %
                \addplot[
                    color=blue, %
                    mark=o, %
                    line width=2pt,
                    mark size=3pt,
                    error bars/.cd,
                    y dir=both, %
                    y explicit, %
                    error bar style={line width=1pt,solid},
                    error mark options={line width=1pt,mark size=4pt,rotate=90}
                ]
                coordinates {
                    (0, -50.0)  +- (0, 0.0)
                    (1, -50.0)  +- (0, 0.0) 
                    (2, -50.0)  +- (0, 0.0) 
                    (3, -841.440725)  += (0, 354.24605512) -= (0, 158.559275)
                    (4, -884.98225)  += (0, 315.37519669) -= (0, 115.01775)
                    (5, -894.330425) += (0, 304.88572805) -= (0, 105.669575)
                    (6, -896.696175) += (0, 301.19954514) -= (0, 103.303825)
                    (7, -897.4635) += (0, 299.61791279) -= (0, 102.5365)
                    (8, -897.77595) += (0, 298.80392585) -= (0, 102.22405)
                    (9, -897.942975) += (0, 298.32920557) -= (0, 102.057025)
                };
                %
                \addplot[
                    color=red, %
                    mark=x, %
                    line width=2pt,
                    mark size=6pt,
                    error bars/.cd,
                    y dir=both, %
                    y explicit, %
                    error bar style={line width=1pt,solid},
                    error mark options={line width=1pt,mark size=4pt,rotate=90}
                ]
            coordinates {
                    (0, -50.0)  +- (0, 0.0)
                    (1, -360.675265)  +- (0, 479.39812699) 
                    (2, -432.27629)  +- (0, 510.38620897) 
                    (3, -467.029545)  += (0, 526.36009628) -= (0, 526.36009628)
                    (4, -439.17429)  += (0, 583.96638919) -= (0, 560.82571)
                    (5, -418.82704) += (0, 618.43027478) -= (0, 581.17296)
                    (6, -397.464895) += (0, 652.67322574) -= (0, 602.535105)
                    (7, -378.49052) += (0, 682.85407033) -= (0, 621.50948)
                    (8, -362.654195) += (0, 707.01412023) -= (0, 637.345805)
                    (9, -347.737935) += (0, 729.29076479) -= (0, 652.262065)
                };
                %
                %
                %
                %
                %
                %
                %
                %
                %
                %
                %
                %
                %
                %
                %
                %
                %
                %
                %
                \end{axis}
            \end{tikzpicture}
         }
    }
    \caption{Average instant reward of CF paths induced by policies on Sepsis.}
    \label{fig: reward sepsis}
\end{figure*}

%
%
%
\subsection{Interval CFMDP Bounds}
%
%
Table \ref{tab:nonzero_probs} presents the mean counterfactual probability bound widths (excluding transitions where the upper bound is $0$) for each MDP, averaged over 20 observed paths. We compare the bounds under counterfactual stability (CS) and monotonicity (M) assumptions, CS alone, and no assumptions. This shows that the assumptions marginally reduce the bound widths, indicating the assumptions tighten the bounds without excluding too many causal models, as intended.
\renewcommand{\arraystretch}{1}

\begin{table}
\centering
\caption{Mean width of counterfactual probability bounds}
\resizebox{0.8\columnwidth}{!}{%
\begin{tabular}{|c|c|c|c|}
\hline
\multirow{2}{*}{\textbf{Environment}} & \multicolumn{3}{c|}{\textbf{Assumptions}} \\ \cline{2-4}
 & \textbf{CS + M} & \textbf{CS} & \textbf{None\tablefootnote{\jl{Equivalent to \citet{li2024probabilities}'s bounds (see Section \ref{sec: equivalence with Li}).}}} \\ \hline
\textbf{GridWorld} ($p=0.9$) & 0.0817 & 0.0977 & 0.100 \\ \hline
\textbf{GridWorld} ($p=0.4$) & 0.552  & 0.638  & 0.646 \\ \hline
\textbf{Sepsis} & 0.138 & 0.140 & 0.140 \\ \hline
\end{tabular}
}
\label{tab:nonzero_probs}
\end{table}


\subsection{Execution Times}
Table \ref{tab: times} compares the average time needed to generate the interval CFMDP vs.\ the Gumbel-max SCM CFMDP for 20 observations.
The GridWorld algorithms were run single-threaded, while the Sepsis experiments were run in parallel.
Generating the interval CFMDP is significantly faster as it uses exact analytical bounds, whereas the Gumbel-max CFMDP requires sampling from the Gumbel distribution to estimate counterfactual transition probabilities. \jl{Since constructing the counterfactual MDP models is the main bottleneck in both approaches, ours is more efficient overall and suitable for larger MDPs.}
\begin{table}
\centering
\caption{Mean execution time to generate CFMDPs}
\resizebox{0.99\columnwidth}{!}{%
\begin{tabular}{|c|c|c|}
\hline
\multirow{2}{*}{\textbf{Environment}} & \multicolumn{2}{c|}{\textbf{Mean Execution Time (s)}} \\ \cline{2-3} 
                                      & \textbf{Interval CFMDP} & \textbf{Gumbel-max CFMDP} \\ \hline
\textbf{GridWorld ($p=0.9$) }                  & 0.261                   & 56.1                      \\ \hline
\textbf{GridWorld ($p=0.4$)  }                 & 0.336                   & 54.5                      \\ \hline
\textbf{Sepsis}                                 & 688                     & 2940                      \\ \hline
\end{tabular}%
}
\label{tab: times}
\end{table}

\section{Conclusion}
In this work, we propose a simple yet effective approach, called SMILE, for graph few-shot learning with fewer tasks. Specifically, we introduce a novel dual-level mixup strategy, including within-task and across-task mixup, for enriching the diversity of nodes within each task and the diversity of tasks. Also, we incorporate the degree-based prior information to learn expressive node embeddings. Theoretically, we prove that SMILE effectively enhances the model's generalization performance. Empirically, we conduct extensive experiments on multiple benchmarks and the results suggest that SMILE significantly outperforms other baselines, including both in-domain and cross-domain few-shot settings.

\section{Limitations}
While \disco{} demonstrates significant improvements in LLM serving efficiency, we acknowledge several important limitations of our current work:

\paragraph{Model Coverage.} We focus on scenarios where on-device LLMs achieve sufficient accuracy for target applications. While this covers many common use cases, \disco{} may not be suitable for applications requiring complex reasoning.

\paragraph{Energy Modeling.} For device energy consumption, we use a linear energy model based on FLOPs. Real-world device energy consumption patterns can be more complex, varying with factors such as battery state, temperature, and concurrent workloads.

\paragraph{Scalability Considerations.} Our current implementation and evaluation focus on single-device scenarios. Extending \disco{} to handle multi-device collaborative serving presents additional challenges in terms of coordination overhead and resource allocation that warrant further investigation.

\section{Ethical Considerations}
Our work focuses solely on optimizing the efficiency of LLM serving systems through device-server collaboration and does not introduce new language generation capabilities or content. All experiments were conducted using publicly available models and datasets. While our work may indirectly benefit the accessibility of LLM services by reducing costs and improving performance, we acknowledge that broader ethical considerations around LLM deployment and usage are important but outside the scope of this technical contribution.

\bibliography{main}
\bibliographystyle{acl_natbib}

\appendix
\section{RELATED WORK}
\label{sec:relatedwork}
In this section, we describe the previous works related to our proposal, which are divided into two parts. In Section~\ref{sec:relatedwork_exoplanet}, we present a review of approaches based on machine learning techniques for the detection of planetary transit signals. Section~\ref{sec:relatedwork_attention} provides an account of the approaches based on attention mechanisms applied in Astronomy.\par

\subsection{Exoplanet detection}
\label{sec:relatedwork_exoplanet}
Machine learning methods have achieved great performance for the automatic selection of exoplanet transit signals. One of the earliest applications of machine learning is a model named Autovetter \citep{MCcauliff}, which is a random forest (RF) model based on characteristics derived from Kepler pipeline statistics to classify exoplanet and false positive signals. Then, other studies emerged that also used supervised learning. \cite{mislis2016sidra} also used a RF, but unlike the work by \citet{MCcauliff}, they used simulated light curves and a box least square \citep[BLS;][]{kovacs2002box}-based periodogram to search for transiting exoplanets. \citet{thompson2015machine} proposed a k-nearest neighbors model for Kepler data to determine if a given signal has similarity to known transits. Unsupervised learning techniques were also applied, such as self-organizing maps (SOM), proposed \citet{armstrong2016transit}; which implements an architecture to segment similar light curves. In the same way, \citet{armstrong2018automatic} developed a combination of supervised and unsupervised learning, including RF and SOM models. In general, these approaches require a previous phase of feature engineering for each light curve. \par

%DL is a modern data-driven technology that automatically extracts characteristics, and that has been successful in classification problems from a variety of application domains. The architecture relies on several layers of NNs of simple interconnected units and uses layers to build increasingly complex and useful features by means of linear and non-linear transformation. This family of models is capable of generating increasingly high-level representations \citep{lecun2015deep}.

The application of DL for exoplanetary signal detection has evolved rapidly in recent years and has become very popular in planetary science.  \citet{pearson2018} and \citet{zucker2018shallow} developed CNN-based algorithms that learn from synthetic data to search for exoplanets. Perhaps one of the most successful applications of the DL models in transit detection was that of \citet{Shallue_2018}; who, in collaboration with Google, proposed a CNN named AstroNet that recognizes exoplanet signals in real data from Kepler. AstroNet uses the training set of labelled TCEs from the Autovetter planet candidate catalog of Q1–Q17 data release 24 (DR24) of the Kepler mission \citep{catanzarite2015autovetter}. AstroNet analyses the data in two views: a ``global view'', and ``local view'' \citep{Shallue_2018}. \par


% The global view shows the characteristics of the light curve over an orbital period, and a local view shows the moment at occurring the transit in detail

%different = space-based

Based on AstroNet, researchers have modified the original AstroNet model to rank candidates from different surveys, specifically for Kepler and TESS missions. \citet{ansdell2018scientific} developed a CNN trained on Kepler data, and included for the first time the information on the centroids, showing that the model improves performance considerably. Then, \citet{osborn2020rapid} and \citet{yu2019identifying} also included the centroids information, but in addition, \citet{osborn2020rapid} included information of the stellar and transit parameters. Finally, \citet{rao2021nigraha} proposed a pipeline that includes a new ``half-phase'' view of the transit signal. This half-phase view represents a transit view with a different time and phase. The purpose of this view is to recover any possible secondary eclipse (the object hiding behind the disk of the primary star).


%last pipeline applies a procedure after the prediction of the model to obtain new candidates, this process is carried out through a series of steps that include the evaluation with Discovery and Validation of Exoplanets (DAVE) \citet{kostov2019discovery} that was adapted for the TESS telescope.\par
%



\subsection{Attention mechanisms in astronomy}
\label{sec:relatedwork_attention}
Despite the remarkable success of attention mechanisms in sequential data, few papers have exploited their advantages in astronomy. In particular, there are no models based on attention mechanisms for detecting planets. Below we present a summary of the main applications of this modeling approach to astronomy, based on two points of view; performance and interpretability of the model.\par
%Attention mechanisms have not yet been explored in all sub-areas of astronomy. However, recent works show a successful application of the mechanism.
%performance

The application of attention mechanisms has shown improvements in the performance of some regression and classification tasks compared to previous approaches. One of the first implementations of the attention mechanism was to find gravitational lenses proposed by \citet{thuruthipilly2021finding}. They designed 21 self-attention-based encoder models, where each model was trained separately with 18,000 simulated images, demonstrating that the model based on the Transformer has a better performance and uses fewer trainable parameters compared to CNN. A novel application was proposed by \citet{lin2021galaxy} for the morphological classification of galaxies, who used an architecture derived from the Transformer, named Vision Transformer (VIT) \citep{dosovitskiy2020image}. \citet{lin2021galaxy} demonstrated competitive results compared to CNNs. Another application with successful results was proposed by \citet{zerveas2021transformer}; which first proposed a transformer-based framework for learning unsupervised representations of multivariate time series. Their methodology takes advantage of unlabeled data to train an encoder and extract dense vector representations of time series. Subsequently, they evaluate the model for regression and classification tasks, demonstrating better performance than other state-of-the-art supervised methods, even with data sets with limited samples.

%interpretation
Regarding the interpretability of the model, a recent contribution that analyses the attention maps was presented by \citet{bowles20212}, which explored the use of group-equivariant self-attention for radio astronomy classification. Compared to other approaches, this model analysed the attention maps of the predictions and showed that the mechanism extracts the brightest spots and jets of the radio source more clearly. This indicates that attention maps for prediction interpretation could help experts see patterns that the human eye often misses. \par

In the field of variable stars, \citet{allam2021paying} employed the mechanism for classifying multivariate time series in variable stars. And additionally, \citet{allam2021paying} showed that the activation weights are accommodated according to the variation in brightness of the star, achieving a more interpretable model. And finally, related to the TESS telescope, \citet{morvan2022don} proposed a model that removes the noise from the light curves through the distribution of attention weights. \citet{morvan2022don} showed that the use of the attention mechanism is excellent for removing noise and outliers in time series datasets compared with other approaches. In addition, the use of attention maps allowed them to show the representations learned from the model. \par

Recent attention mechanism approaches in astronomy demonstrate comparable results with earlier approaches, such as CNNs. At the same time, they offer interpretability of their results, which allows a post-prediction analysis. \par


\section{Cold Start Evaluation}\label{appendix:cold_start}

This section presents cold start performance measurements for the Qwen-2.5 model series across different hardware configurations.
The experiments were conducted on two platforms: Windows 10 with NVIDIA RTX 3060 12GB and Linux with NVIDIA A40 48GB. A fixed prompt "How to use GitHub?" was used throughout all experiments. We measured two critical metrics: model loading time and TTFT for Qwen-2.5 models ranging from 0.5B to 7B parameters, all using FP16 precision. The experimental setup consisted of 10 measurement runs, with 2 additional warmup runs to ensure measurement stability. It is worth noting that such warmups can potentially mask the true gap between model loading and prompt prefill time due to various optimizations, including OS page cache. To maintain authentic cold start conditions, we explicitly cleared the CUDA cache and performed garbage collection before each run.

The results revealed several significant patterns. On the RTX 3060, the loading time exhibits an approximately linear increase with model size, ranging from 1.29s for the 0.5B model to 4.45s for the 3B model. While TTFT follows a similar trend, the processing time is substantially lower, ranging from 0.051s to 0.145s. On the A40 GPU, despite observing longer loading times, TTFT is significantly reduced across all models, maintaining a remarkably consistent value regardless of model size. These findings indicate that while model loading remains more resource-intensive on our Linux setup, the inference performance benefits substantially from the A40's superior computational capabilities.

\begin{table}[ht]
    \centering
    \footnotesize
    \vskip 0.1in
    \begin{tabular}{p{2.5cm}cccc}
    \toprule
    \textbf{Metric} & \textbf{0.5B} & \textbf{1.5B} & \textbf{3B} & \textbf{7B} \\
    \midrule
    \multicolumn{5}{c}{\textbf{Windows 10 (NVIDIA RTX 3060 12GB)}} \\
    \midrule
    Load Time (s) & 1.29 & 2.48 & 4.45 & - \\
    TTFT (s) & 0.051 & 0.105 & 0.145 & - \\
    \midrule
    \multicolumn{5}{c}{\textbf{Linux (NVIDIA A40 48GB)}} \\
    \midrule
    Load Time (s) & 1.53 & 3.12 & 5.72 & 13.43 \\
    TTFT (s) & 0.025 & 0.026 & 0.033 & 0.033 \\
    \bottomrule
    \end{tabular}
    \caption{Model loading time during cold start can significantly slow down TTFT. Average Qwen-2.5 model performance over 10 runs. The 7B model exceeds the memory capacity of the RTX 3060 and thus cannot be evaluated.}
    \label{tab:cold-start}
\end{table}

\section{Posterior Predictive Distributions of PCTM}

$\mathbf{W}_{iq}$ is a vector of length $n_{iq}$, the number of words in paragraph $q$ of document $i$, and its $l$ th element, $W_{iql}$, is the $l$ the word in paragraph $p$ of document $i$.
$\mathbf{D}_{iq}$ is a vector of length $i-1$, the number of all possible documents to be cited, and its $j$ th element, $D_{iqj}$ is 1 if paragraph $p$ of document $i$ cited document $j$; 0 otherwise.
We use $\mathbf{W}_{iq}$ and $\mathbf{D}_{iq}$ as the test data to be predicted, and all words and citations in other paragraph than $q$ in document $i$ as well as all data in previous documents as the training data.
Thus, $\mathbf{W}^{train}$ be the set of $\mathbf{W}_{ir, r \neq q}$ and $\mathbf{W}_{jp}$ for all $j < i$ and $p \in \{1 , \ldots , n_j\}$ where $n_j$ is the number of paragraphs in document $j$.
Likewise, $\mathbf{D}^{train}$ be the set of $\mathbf{D}_{ir, r \neq q}$ and $\mathbf{D}_{jp}$ for all $j < i$ and $p \in \{1 , \ldots , n_j\}$.

\begin{equation}
\begin{split}
  &p(\mathbf{W}_{iq}, \mathbf{D}_{iq} \vert \mathbf{W}^{train}, \mathbf{D}^{train}) \\
  &\propto \int_{\eta, \Psi, \tau} \sum_{\mathbf{Z}} p(\mathbf{W}_{iq}, \mathbf{D}_{iq} \vert \mathbf{Z}, \eta, \Psi, \tau) 
  \times p(\mathbf{Z}, \eta, \Psi, \tau, \vert \mathbf{W}^{train}, \mathbf{D}^{train}) d\eta d\Psi d\tau \\
  &\propto \int_{\eta, \Psi, \tau} \sum_{\mathbf{Z}} p(\mathbf{W}_{iq}, \mathbf{D}_{iq} \vert \mathbf{Z}, \eta, \Psi, \tau) 
  \times p(\mathbf{Z} \vert \eta, \Psi, \tau, \mathbf{W}^{train}, \mathbf{D}^{train}) p( \eta, \Psi, \tau \vert \mathbf{W}^{train}, \mathbf{D}^{train}) d\eta d\Psi d\tau \\
  &\approx \sum_{Z_{iq}} p(\mathbf{W}_{iq}, \mathbf{D}_{iq} \vert Z_{iq}, \hat{\mathbf{Z}}^{train}, \hat{\eta}, \hat{\Psi}, \hat{\tau}) \times p(Z_{iq} \vert \hat{\eta})  \\
  &= \sum_{k=1}^K \Big\{ p(\mathbf{W}_{iq}, \mathbf{D}_{iq} \vert Z_{iq} = k, \hat{\mathbf{Z}}^{train}, \hat{\eta}, \hat{\Psi}, \hat{\tau}) \times p(Z_{iq} = k \vert \hat{\eta}) \Big\} \\
  &= \sum_{k=1}^K \Big\{ p(\mathbf{W}_{iq} \vert Z_{iq} = k, \hat{\Psi})
  \times \prod_{j=1}^{i-1} p(D_{iqj} \vert \hat{\tau}, \hat{\eta}, Z_{ip})
  \times p(Z_{iq} = k \vert \hat{\eta}) \Big\} \\
  &= \sum_{k=1}^K \Big\{ p(\mathbf{W}_{iq} \vert Z_{iq} = k, \hat{\Psi})
  \times \prod_{j=1}^{i-1} \mathbb{P}(D_{iqj}^* > 0 \vert \hat{\tau}, \hat{\eta}, Z_{ip} = k)^{\mathbb{I}\{D_{iqj}=1\}}\mathbb{P}(D_{iqj}^* < 0 \vert \hat{\tau}, \hat{\eta}, Z_{ip} = k)^{\mathbb{I}\{D_{iqj}=0\}} \\
  &\quad \times p(Z_{iq} = k \vert \hat{\eta}) \Big\} \\
  &\propto \sum_{k=1}^K \Bigg\{ \prod_{v=1}^V \Psi_{vk}^{W_{iqv}} 
  \times \prod_{j=1}^{i-1} \Big[\int_{t=0}^{\infty} p(D_{iqj}^* = t | \tau_0 + \tau_1 \kappa_j^{(i)} + \tau_2\eta_{jk}) dt\Big]^{\mathbb{I}\{D_{iqj}=1\}} \\
  &\quad \times \Big[\int_{t=-\infty}^{0} p(D_{iqj}^* = t | \tau_0 + \tau_1 \kappa_j^{(i)} + \tau_2\eta_{jk}) dt \Big]^{\mathbb{I}\{D_{iqj}=0\}} 
  \times \frac{\exp(\eta_{ik})}{\sum_{k'=1}^K \exp(\eta_{ik'})} \Bigg\} \\
\end{split}
\end{equation}

The approximation part can use Monte Carlo simulation - We draw parameters $\eta, \tau$ from the posterior distributions and take the average.
Alternatively, we could use the point estimate of $\eta, \tau$ to match the computation in RTM and LDA+logistic.

\section{Response Quality}\label{appendix:accuracy-eval}

This section examines the quality of responses generated by \disco{}, with a particular focus on quality preservation during endpoint transitions. We first establish bounds on generation quality, then present our evaluation methodology, and finally demonstrate through extensive experiments that \disco{} maintains consistent quality across different model configurations and tasks.

\subsection{Quality Bounds}

A critical aspect of \disco{} is maintaining generation quality during endpoint transitions. We employ a systematic approach to quality preservation~\cite{diba2017weakly,gupta2022semi,chen2023frugalgpt}. Specifically, for endpoints A and B with quality metrics $Q_A$ and $Q_B$ (measured by LLM scores or ROUGE scores), we find that any migrated sequence M with quality $Q_M$ satisfies:

\begin{equation}
    \min(Q_A, Q_B) \leq Q_M \leq \max(Q_A, Q_B)
\end{equation}

This bound ensures that migration does not degrade quality beyond the capabilities of individual endpoints.

\subsection{Evaluation Methodology}

\paragraph{Evaluation Framework}
We establish a comprehensive assessment framework encompassing both automated metrics and LLM-based evaluation. Our framework evaluates two distinct tasks:

\begin{itemize}
    \item \textbf{Instruction Following}: We evaluate 500 data items from the Alpaca dataset~\cite{alpaca} using our structured prompt template, with quality assessment performed by multiple LLM judges: Gemini1.5-pro, GPT-4o, and QWen2.5-72b-instruct.
    
    \item \textbf{Translation Quality}: We assess Chinese-to-English translation on 500 data items from Flores\_zho\_Hans-eng\_Latn dataset~\cite{nllb2022,goyal2022flores} using the ROUGE-1 metric.
\end{itemize}

These two tasks are popular on end-user devices. Understandably, for complex tasks such as advanced math reasoning, we notice DisCo can lead to accuracy drops compared to the on-server model due to the limited capability of the on-device models, yet still achieves better performance than the on-device counterpart. 

\paragraph{Experimental Setup}
We configure our experiments with:
\begin{itemize}
    \item A fixed maximum generation length of 256 tokens
    \item First endpoint's maximum generation length varied through [0, 4, 16, 64, 256] tokens
    \item Four model combinations: 0.5B-7B, 3B-7B, 7B-0.5B, and 7B-3B (prefix and suffix denote the model sizes of first and second endpoints respectively)
\end{itemize}

The generation transitions to the second endpoint when the first endpoint reaches its length limit without producing an end-of-generation token, creating natural boundary conditions for analysis.

For instruction-following tasks, we employ the following structured evaluation template:

\begin{footnotesize}
\begin{verbatim}
JUDGE_PROMPT = """Strictly evaluate the 
quality of the following answer on a scale
of 1-10 (1 being the worst, 10 being the 
best). First briefly point out the problem
of the answer, then give a total rating in
the following format.

Question: {question}

Answer: {answer}

Evaluation: (your rationale for the rating, 
as a brief text)

Total rating: (your rating, as a number 
between 1 and 10)
"""
\end{verbatim}
\end{footnotesize}


\subsection{Results and Analysis}

\subsubsection{Quality Metrics}
Our comprehensive evaluation reveals several key findings:
\begin{itemize}
    \item \textbf{Bounded Quality}: The combined sequence quality consistently remains bounded between individual model performance levels
    \item \textbf{Translation Performance}: ROUGE-1 scores maintain stability between 0.23 and 0.26
    \item \textbf{Instruction Following}: Scores show consistent ranges from 4 to 6
\end{itemize}

\begin{figure}[t]
    \centering
    \includegraphics[width=0.45\textwidth]{figs/acc_translation.pdf}
    \\[12pt]
    \includegraphics[width=0.45\textwidth]{figs/acc_instruction-05b7b.pdf}
    \caption{Quality evaluation results of \disco{}. The top figure shows translation quality evaluation using ROUGE-1 scores, demonstrating that \disco{} consistently achieves higher quality than the on-device baseline. The bottom figure presents evaluation scores from different LLM judges on instruction-following capabilities, where each subplot represents a different model pair comparison with varied first-endpoint model's maximum sequence length. The consistent patterns across different LLM judges demonstrate the robustness of our evaluation framework.}
    \label{fig:accuracy}
\end{figure}

\section{Experiment Settings for End-to-end Cost}
\label{appendix:unified_cost}

For on-device LLMs, we quantify cost using FLOPs (floating-point operations). For on-server LLM services, we use their respective pricing rates at the time of experimentation. We set the energy-to-monetary conversion ratio (\textit{energy\_to\_money}) to 0.3 \$ per million FLOPs for server-constrained experiments and 5 \$ per million FLOPs for device-constrained experiments. To establish a comprehensive cost model that enables direct comparison between device and server computation costs, we analyze both the computational complexity of on-device models through detailed FLOPs calculations (Section~\ref{sec:flops_analysis}) and the pricing structures of commercial LLM services (Section~\ref{appendix:llm_pricing}). The generation length limit is set to 128.


\subsection{FLOPs of On-Device LLMs}
\label{sec:flops_analysis}

To accurately quantify the computational cost per token in both prefill and decode stages, we conduct a detailed FLOPs analysis using three representative models: BLOOM-1.1B, BLOOM-560M, and Qwen1.5-0.5B. All models share a 24-layer architecture but differ in other parameters: BLOOM-1.1B ($d_{\text{model}}=1024$, 16 heads, FFN dim=4096), BLOOM-560M ($d_{\text{model}}=512$, 8 heads, FFN dim=2048), and Qwen1.5-0.5B ($d_{\text{model}}=768$, 12 heads, FFN dim=2048).

\paragraph{Per-token FLOPs computation.}
The total FLOPs for processing each token consist of five components:
\begin{align}
    \text{FLOPs}_{\text{total}} &= \text{FLOPs}_{\text{attn}} + \text{FLOPs}_{\text{ffn}} \nonumber \\
    &\quad+ \text{FLOPs}_{\text{ln}} + \text{FLOPs}_{\text{emb}} + \text{FLOPs}_{\text{out}}
\end{align}

For a sequence of length $L$, the attention computation differs between stages. In prefill:
\begin{align}
    \text{FLOPs}_{\text{attn}} &= n_{\text{layers}} \cdot \Big(3d_{\text{model}}^2 + \frac{L^2 d_{\text{model}}}{n_{\text{heads}}} \nonumber \\
    &\quad+ L d_{\text{model}} + d_{\text{model}}^2\Big)
\end{align}

While in decode, KV caching eliminates the quadratic term:
\begin{align}
    \text{FLOPs}_{\text{attn}} &= n_{\text{layers}} \cdot \Big(3d_{\text{model}}^2 + \frac{L d_{\text{model}}}{n_{\text{heads}}} \nonumber \\
    &\quad+ L d_{\text{model}} + d_{\text{model}}^2\Big)
\end{align}

Table~\ref{tab:prefill-decode} presents the total FLOPs across different sequence lengths. The decode phase maintains constant FLOPs regardless of sequence length due to KV caching, while prefill phase FLOPs increase with sequence length. A breakdown of computational cost by component (Table~\ref{tab:components}) reveals that embedding and output projection operations account for the majority of FLOPs, particularly in models with large vocabularies.

\begin{table}[t]
    \centering
    \footnotesize
    \begin{tabularx}{\linewidth}{lccc}
    \toprule
    \textbf{Length} & \textbf{BLOOM-1.1B} & \textbf{BLOOM-560M} & \textbf{Qwen-0.5B} \\
    \midrule
    \multicolumn{4}{l}{\textit{Prefill Phase}} \\
    L = 32 & 0.85 & 0.45 & 0.39 \\
    L = 64 & 0.93 & 0.50 & 0.45 \\
    L = 128 & 1.25 & 0.65 & 0.69 \\
    \midrule
    \multicolumn{4}{l}{\textit{Decode Phase}} \\
    L = 32 & 0.82 & 0.42 & 0.37 \\
    L = 64 & 0.82 & 0.42 & 0.37 \\
    L = 128 & 0.82 & 0.42 & 0.37 \\
    \bottomrule
    \end{tabularx}
    \caption{Prefill and Decode FLOPs (billions)}
    \vskip -0.1in
    \label{tab:prefill-decode}
\end{table}

\begin{table}[t]
    \centering
    \footnotesize
    \begin{tabularx}{\linewidth}{lccc}
    \toprule
    \textbf{Component} & \textbf{BLOOM-1.1B} & \textbf{BLOOM-560M} & \textbf{Qwen-0.5B} \\
    \midrule
    Embedding & 31.24 & 25.00 & 31.51 \\
    Attention & 13.01 & 10.00 & 16.56 \\
    FFN & 24.48 & 20.00 & 20.38 \\
    LayerNorm & 0.02 & 0.02 & 0.04 \\
    Output & 31.24 & 25.00 & 31.51 \\
    \bottomrule
    \end{tabularx}
    \caption{Component Ratios at L=128 (\%)}
    \vskip -0.1in
    \label{tab:components}
\end{table}

\subsection{LLM Service Pricing}\label{appendix:llm_pricing}
This section provides further details on the pricing of LLM services. Table \ref{tab:model-pricing} presents the pricing models for several commercial Large Language Models (LLMs) as of October 28, 2024. The pricing structure follows a dual-rate model, differentiating between input (prompt) and output (generation) tokens. These rates represent the public pricing tiers available to general users, excluding any enterprise-specific arrangements or volume-based discounts.

\begin{table}[t]
    \centering
    \footnotesize
    \begin{tabularx}{\linewidth}{lcccc}
    \toprule
    \textbf{Model} & \textbf{Vendor} & \textbf{Input price} & \textbf{Output price} \\
    \midrule
    DeepSeek-V2.5     & DeepSeek   & 0.14 & 0.28 \\
    GPT-4o-mini       & OpenAI     & 0.15 & 0.60 \\
    LLaMa-3.1-70b     & Hyperbolic & 0.40 & 0.40 \\
    LLaMa-3.1-70b     & Amazon     & 0.99 & 0.99 \\
    Command           & Cohere     & 1.25 & 2.00 \\
    GPT-4o            & OpenAI     & 2.50 & 10.0 \\
    Claude-3.5-Sonnet & Anthropic  & 3.00 & 15.0 \\
    o1-preview        & OpenAI     & 15.0 & 60.0 \\
    \bottomrule
    \end{tabularx}
    \caption{LLM service pricing (USD per 1M Tokens). Input prices refer to tokens in the prompt, while output prices apply to generated tokens.}
    \label{tab:model-pricing}
\end{table}

\section{Pseudocode for Cost-Aware Adaptive Request Scheduling}\label{appendix:pseudocode_scheduling}

The request scheduling algorithm consists of three key components. Algorithm~\ref{alg:definitions} defines the input parameters and determines whether the scenario is device-constrained or server-constrained based on the relative costs. For device-constrained scenarios, Algorithm~\ref{alg:device_constrained} implements a wait-time strategy to protect tail latency while conserving device energy when possible. For server-constrained scenarios, Algorithm~\ref{alg:server_constrained} employs a length-based routing approach to optimize TTFT while maintaining the server budget constraint. These algorithms work together to achieve the dual objectives of minimizing latency and managing costs.

\begin{algorithm}[ht]
\caption{Variable Definitions and Constraints}
\label{alg:definitions}
\begin{algorithmic}[1]
\REQUIRE
    \STATE $p(l)$: Length distribution
    \STATE $F(t)$: TTFT CDF of server
    \STATE $b \in [0,1]$: Budget ratio 
    \STATE $c^p_d,c^d_d$: Device prefill/decode costs
    \STATE $c^p_s,c^d_s$: Server prefill/decode costs
    \STATE $\alpha \in (0,1)$: Tail ratio
\ENSURE Policy type based on cost constraints
\STATE \textbf{if} $\min(c^p_d,c^d_d) > \max(c^p_s,c^d_s)$ \textbf{then} Device-constrained
\STATE \textbf{else} Server-constrained
\end{algorithmic}
\end{algorithm}

\begin{algorithm}[ht]
\caption{Device-constrained Scheduling}
\label{alg:device_constrained}
\begin{algorithmic}[1]
\REQUIRE Variables from Algorithm \ref{alg:definitions}
\STATE // Phase 1: Set maximum wait time for tail protection
\STATE $w_{tail} \leftarrow F^{-1}(1 - \min(\alpha, b))$ 
\STATE // Initialize wait times for all prompt lengths
\STATE $W \leftarrow \{l: w_{tail} \text{ for all } l\}$

\IF{$b \leq \alpha$} 
    \RETURN $W$ \COMMENT{Use max wait time for all lengths}
\ENDIF

\STATE // Phase 2: Optimize wait times with remaining budget
\STATE available\_budget $\leftarrow b - \alpha$
\FOR{$l \in$ sort(support($p(l)$))}
    \STATE length\_cost $\leftarrow p(l) \cdot l \cdot (1-\alpha)$ 
    \IF{available\_budget $\geq$ length\_cost}
        \STATE $W[l] \leftarrow 0$ \COMMENT{Start device immediately}
        \STATE available\_budget $\leftarrow$ available\_budget - length\_cost
    \ELSE
        \STATE // Find optimal wait time that meets budget
        \STATE Find $w^* \in [0, w_{tail}]$ where:
        \STATE $F(w^*) \cdot$ length\_cost + (b - available\_budget) = $b$
        \STATE $W[l] \leftarrow w^*$
        \STATE \textbf{break}
    \ENDIF
\ENDFOR
\RETURN $W$ \COMMENT{Map from prompt lengths to wait times}
\end{algorithmic}
\end{algorithm}

\begin{algorithm}[ht]
\caption{Server-constrained Scheduling}
\label{alg:server_constrained}
\begin{algorithmic}[1]
\REQUIRE Variables from Algorithm \ref{alg:definitions}
\STATE // Find length threshold to split execution modes
\STATE Compute $l_{th}$ where: $\int_0^{l_{th}} l \cdot p(l) dl = (1-b) \cdot \int_0^\infty l \cdot p(l) dl$

\STATE // Initialize execution policy map
\STATE $P \leftarrow \emptyset$ 
\FOR{$l \in$ support($p(l)$)}
    \IF{$l < l_{th}$}
        \STATE $P[l] \leftarrow$ (1, 0) \COMMENT{$(I_d,I_s)$: Device only}
    \ELSE
        \STATE $P[l] \leftarrow$ (1, 1) \COMMENT{$(I_d,I_s)$: Concurrent execution}
    \ENDIF
\ENDFOR
\RETURN $P$ \COMMENT{Map from lengths to execution indicators}
\end{algorithmic}
\end{algorithm}

\end{document}