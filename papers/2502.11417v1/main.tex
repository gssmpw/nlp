\pdfoutput=1
\documentclass[11pt]{article}
\usepackage[final]{ACL2023}

\usepackage{times}
\usepackage{latexsym}
\usepackage[T1]{fontenc}
\usepackage[utf8]{inputenc}
\usepackage{microtype}
\usepackage{inconsolata}
\usepackage{amsmath}
\usepackage{amsthm} 
\usepackage{graphicx}
\usepackage{subfigure}
\usepackage{booktabs}
\usepackage{xcolor}
\usepackage{listings}
\usepackage{tabularx}
\usepackage{array}
\usepackage{makecell}
\usepackage{multirow}
\usepackage{enumitem}
\usepackage{algorithm}
\usepackage{algorithmic}
\usepackage{amssymb}
\usepackage{url}

\newenvironment{denseitemize}{
\begin{itemize}[topsep=2.5pt, partopsep=0pt, leftmargin=1.5em]
 \setlength{\itemsep}{2.5pt}
 \setlength{\parskip}{0pt}
 \setlength{\parsep}{0pt}
}{\end{itemize}}

\newenvironment{denseenum}{
\begin{enumerate}[topsep=2pt, partopsep=0pt, leftmargin=1.5em]
 \setlength{\itemsep}{2pt}
 \setlength{\parskip}{0pt}
 \setlength{\parsep}{0pt}
}{\end{enumerate}}

\newcommand{\disco}{{\emph{DiSCo}}}

\lstset{
   language=Python,
   basicstyle=\small\ttfamily,
   identifierstyle=\color{black},
   keywordstyle=\color{black},
   stringstyle=\color{black},
   commentstyle=\color{black},
   morekeywords={keeper_service},
   keywordstyle=\color{purple},
   breaklines=true,
   frame=single,
   columns=flexible,
   xleftmargin=0pt,
}

\newtheorem{lemma}{Lemma}
\newtheorem{theorem}{Theorem}

\setlength\titlebox{5cm}

\title{\disco: Device-Server Collaborative LLM-Based Text Streaming Services}

\author{Ting Sun$^{1}$,
Penghan Wang$^{1}$, 
Fan Lai$^{1}$
\\
\textsuperscript{1} University of Illinois Urbana-Champaign, United States \\
\texttt{fanlai@illinois.edu}
}

\begin{document}
\maketitle

\begin{abstract}
Retrieval-Augmented Generation (RAG) is often used with Large Language Models (LLMs) to infuse domain knowledge or user-specific information. In RAG, given a user query, a retriever extracts chunks of relevant text from a knowledge base. These chunks are sent to an LLM as part of the input prompt. Typically, any given chunk is repeatedly retrieved across user questions. However, currently, for every question, attention-layers in LLMs fully compute the key values (KVs) repeatedly for the input chunks, as state-of-the-art methods cannot reuse KV-caches when chunks appear at arbitrary locations with arbitrary contexts. Naive reuse leads to output quality degradation.  This leads to potentially redundant computations on expensive GPUs and increases latency. In this work, we propose \sys, a system for managing and reusing precomputed KVs corresponding to the text chunks (we call \textit{chunk-caches}) in RAG-based systems. We present how to identify \hl{\textit{chunk-caches} that are reusable}, how to efficiently perform a small fraction of recomputation to \textit{fix} the cache to maintain output quality, and how to efficiently store and evict \textit{chunk-caches} in the hardware for maximizing reuse while masking any overheads. With real production workloads as well as synthetic datasets, we show that \sys reduces redundant computation by \textbf{51\%} over SOTA prefix-caching and \textbf{75\%} over full recomputation.
\hl{Additionally, with continuous batching on a real production workload, we get a \textbf{1.6$\times$} speedup in throughput and a \textbf{2$\times$} reduction in end-to-end response latency over prefix-caching while maintaining quality, for both the \llama-3-8B and \llama-3-70B models. 
}
\end{abstract}





\documentclass[../main.tex]{subfiles}
\graphicspath{{../images/}}
\makeatletter
\def\input@path{{../images/}}
\makeatother
\begin{document}
\section{Introduction}
\begin{figure}
\centering
\begin{tikzpicture}
\node[inner sep=0pt] (ws) at (0, 0) {
\includegraphics[height=.4\textwidth, trim={10cm 0 10cm 0},clip]{world_space.png}};
\node[inner sep=0pt] (cs) at (6,0) {\includegraphics[height=.4\textwidth, trim={10cm 1cm 10cm 4cm},clip]{conf_space.png}};
\end{tikzpicture}
\vspace{-5pt}
\label{fig:pbrm_intro}
\caption{\textbf{Left}: Shows world space obstacles as grey spheres. Robots start and goal configuration is colored red and green, respectively. Configurations along the computed path are colored transparent blue. \textbf{Right:} Mapped world space scenario to configuration space. Obstacle region is the grey mesh. Red spheres are collision-free regions computed by the neural SCDF. The optimized shortest path in the convex corridor is the blue curve.}
\vspace{-25pt}
\end{figure}
Motion planning is the problem of finding a collision-free trajectory that connects a given start and goal configuration. The planning takes place in the configuration space of the robot. For single body robots, like mobile robots or drones, the configuration space and the world space are usually the same. This simplifies the planning, since explicit obstacle representations are available which enables geometrical tools like separating hyperplanes, smallest distance to obstacles etc., to be used when designing motion planning algorithms. For multi-body robots like manipulators, the situation is completely different. The world space obstacles are usually mapped to non-convex regions, and to make the problem even harder, the mapping is usually not known. Forming explicit representations of the obstacle region in the configuration space is usually too expensive or intractable. Despite all of this, sampling based planners are used with great success, which mainly is due to their use of implicit representations of the obstacle region. The basic idea is to construct a graph in the configuration space that covers and connects the collision-free region. From this graph, a path can be extracted that connects a given start and goal configuration. The approach is computationally expensive, since the graph is constructed with the smallest geometrical building block available, points, which represents a collision-check. Furthermore, the extracted paths from the graph are non-smooth and jagged due to the stochastic nature of the approach. This adds an additional post-processing step to the process, where the paths are shortcutted and smoothened, before the path can be used for tracking. Clearly a lot of time is invested to form this graph and produce smooth paths. Thus, if the obstacles start to move, then all of this work is done in no use, since all points that make up this graph need to be re-verified, which is simply too time consuming to be done in real time.
\\\\
In this work, we want to address the existing drawbacks of the sampling based planners. Our main contribution is an improved motion planner where each vertex in the graph covers a collision-free region in the form of a sphere instead of a point and where the edges are formed with neighboring intersecting spheres. This representation has the advantage of instead of returning piecewise linear paths, returning a sequence of overlapping spheres, i.e. a convex corridor, that connects a given start and goal configuration, illustrated in Figure \ref{fig:pbrm_intro}. This convex corridor allows us to use convex optimization to produce smooth trajectories, instead of computationally expensive post-processing methods. The representation further allows us to estimate the coverage of the collision-free space, which gives us awareness and feedback in the offline roadmap construction phase. Finally, our representation is simple to adapt to moving obstacles, simply requery for the new radii and recheck for intersections. 
\\\\
The spherical collision-free regions are formed using a signed distance function (SDF), which is a function that returns the smallest distance from an arbitrary point to the boundary of an obstacle. As the name implies, the distance is signed, thus if the point is inside the obstacle it is negative otherwise positive. If the distance is positive, a sphere with radius equal to the distance is guaranteed to cover a collision-free region. Using an SDF in motion planning is not new, but what is novel about our approach is that we express the distance in the configuration space instead of the world space and by doing so allows us to form these convex collision-free regions. We refer to the resulting SDF as a signed configuration distance function (SCDF). Computing an SCDF analytically is non-trivial, our approach is therefore to parameterize the SCDF with a deep neural network and learn the mapping by supervised learning. Our resulting neural SCDF can compute distances for different parameter values of obstacle shapes and we also show how multiple distances can be combined, thus making our approach flexible.
\section{Related work}
Motion planning algorithms can roughly be divided into three families, grid-based, sampling based and optimization based methods. Grid-based methods (GBM) discretize the planning space from which a graph is then compiled. A standard search method is A$^\star$ \citep{a_star}, which is classified as an \textit{informed} search method, since it employs a heuristic function to speed up the search. A$^\star$ guarantees to return an optimal path at the level of discretization used. GBMs usually discretize the planning space by a regular lattice and this limits the GBMs to problems with low dimensionality due to the curse of dimensionality. Thus, GBMs are usually limited to single-body robots where the degrees of freedom (DOF) are low. To overcome the inherent scaling problem with the GBMs, stochastic methods are usually used for multi-body robots. These methods are termed as sampling-based methods (SBM) and core members within this family are the rapidly-exploring random trees (RRT) \citep{rrt} and the probabilistic roadmap (PRM) \citep{prm}. RRT grows a tree from the start configuration and explores the collision-free region in a rapid way until it is able to connect to the goal region. RRT is usually improved by bi-directional planning \citep{rrt_connect}, i.e. an additional tree is grown from the goal configuration and the trees are tested for connection after any tree has been expanded. RRT is a single-query method, thus it searches for a path from scratch each time it is queried. Contrary to this, PRM is a multi-query method, which solves for multiple queries without starting from scratch. PRM does this by creating a roadmap (graph) that covers the collision-free space as an offline step. The graph is then used to solve for multiple queries. PRMs are used in cases where the environment does not change since the extra offline step is too computationally costly and needs to be re-done if the environment is changed. In our work, we address this inherent issue by using a different roadmap representation. Our vertices in the graph cover a collision-free region in the form of spheres and we form the edges by checking for intersecting spheres. If something in the environment changes, we recompute the spheres radii and recheck the intersections, without relying on collision detection. We use a trained neural network to compute the sphere radius, therefore querying for the radius can be done fast, hence our representation enables the PRM for dynamic environments.
\\\\
In the recent decades, optimization based methods (OBM) \citep{chomp, schulman, itomp, stomp} have been introduced as an alternative to SBM for multi-body robots. Like the SBM, the OBMs scale well to higher dimensional problems and produce smoother motion. It is common to use a SDF in the optimization since it is a smooth function, thus enabling gradient-based methods. However, the standard way of expressing the SDF is in world space. The distance therefore needs to be mapped to the configuration space by the forward kinematics. This mapping makes the optimization problem a non-linear program (NLP), which is computationally expensive to solve. Recently, a different approach has been proposed. In \cite{mp_gcs} motion planning is formulated as a convex optimization problem by using the graph of convex sets framework \citep{gcs}. The underlying idea is to decompose the collision-free space into intersecting convex sets from which a convex optimization problem is formulated. In cases where an explicit representation of the obstacles in the configuration space exists, like for single-body robots, creating collision-free convex regions can be done fast \citep{iris}. For multi-body robots, this is non-trivial. Existing work does this successfully \citep{iris_nlp, iris_c} by an optimization based approach, but the methods are still too time consuming to be used in the presence of moving obstacles. Our approach is instead to use deep learning to learn an SDF expressed in the configuration space. With this, we can query for shortest distances to the collision boundary, which allows us to expand spherical regions which are collision-free. Our approach is fast and therefore enables our suggested roadmap planner to be used in dynamic environments.
\\\\
Recent research has focused on learning collision detection \citep{fk_kernel_distance, diffco, graphdistnet} by predicting the signed distance between the robot links and the surrounding obstacles in the world space. The learned SDF is used in trajectory optimization but since the distance is expressed in the world space, the problem becomes an NLP and therefore takes a long time to solve. We take a novel approach and suggest to instead express the signed distance in the configuration space. This allows us to improve the PRM at the same time as it enables convex optimization for trajectory optimization, which runs faster and is more reliable than NLP solvers. In \cite{cspf} a learned signed distance function in the configuration space is proposed similar to our approach. However, their approach is restricted to point cloud representations, while we propose to represent the obstacles as parameterized geometric shapes, e.g. spheres. Furthermore, we also show how to use our learned SCDF to improve an existing roadmap planner.
\section{Problem formulation}
A robot is located in the world space, $\W \subset \R^3 $. The unique location of the robot is given by its configuration $\q \in \C$, where $\C$ is the configuration space. The set of points covered by the robots bodies at a certain configuration is expressed as $\B(\q) \subset \W$. The robot is surrounded by $\NrObst$ obstacles $\O = \bigcup_{i=1}^{\NrObst} \O_i$, where  $\O_i \subset \W$. The representation of the obstacle in the configuration space is the set $\C\O_i = \{\q \in \C \: |\: \B(\q) \cap \O_i \neq \emptyset \}$. The obstacle space is formed as $\Co = \bigcup_{i=1}^{\NrObst} \C \O_i$. The complement is referred to as the free space, $\Cf = \C \setminus \Co$. The path planning problem is a tuple, ($\Cf$, $\qStart$, $\qGoal$), where we want to connect a query pair, consisting of a start, $\qStart$, and goal configuration, $\qGoal$, with a geometric path, $\q(s): [0, 1] \mapsto \Cf$, such that $\q(0)=\qStart$ and $\q(1)=\qGoal$, or report correctly when such a path does not exist.
\end{document}

\section{Basic Background: Supervised Learning and the PAC Model}
\label{sec:background}

At this point almost everyone has heard of machine learning (ML). Anyone likely to stumble upon this article will have also heard of its most influential special case, supervised learning, and those theoretically inclined will also be familiar with the PAC model. Nonetheless, I will set the stage by  recapping the basics.

\subsection{Basics of Supervised Learning}%Let's set the stage in any case

\emph{Supervised Learning} is the task of ``coming up'' with a function $f: \X \to \Y$ to ``explain'' or ``fit'' a sequence of input/output examples   $(x_1,y_1), \ldots, (x_n,y_n)$, with $x_i \in \X$ and $y_i \in \Y$.  Here $\X$ is a \emph{data domain} consisting of \emph{datapoints} $x \in \X$, $\Y$ is a \emph{label set} consisting of \emph{labels} $y \in \Y$, and the sequence $(x_1,y_1),\ldots,(x_n,y_n)$ is the \emph{training data} consisting of \emph{labeled examples (a.k.a. samples)}~$(x_i,y_i)$.  I~will refer to the chosen function $f$ as a \emph{predictor}, and to $n$ as the \emph{sample size}. A \emph{learning algorithm} takes as input training data, and outputs (some representation of) a predictor $f \in \Y^\X$.\footnote{Note that this describes the usual \emph{batch}, a.k.a.~\emph{offline}, setting of supervised learning. I do not discuss other paradigms such as online or active learning in this article.} 



Success in supervised learning is defined as \emph{generalization} to  future examples: For a typical \emph{test example}  $(x_{\tst},y_{\tst})$, the predicted label $y'_{\tst}=f(x_{\tst})$ should ``equal'' $y_{\tst}$, perhaps approximately. We usually assume the test example is drawn from the same  ``source'' as the training data  --- commonly, i.i.d.~from the same distribution. The quality of the prediction is quantified by $\ell(y'_{\tst},y_{\tst})$, where $\ell:~\Y~\times~\Y \to \RR_{\geq 0}$ is a \emph{loss function} chosen as part of the problem definition. Common loss functions include the 0-1 loss $\ell_{0-1}(y',y) = [y' \neq y]$ for \emph{classification} problems,\footnote{The notation $[P]$ denotes $1$ when predicate $P$ is true, and denotes $0$ when $P$ is false.} as well as the absolute loss $|y'-y|$ or squared loss $(y'-y)^2$ for \emph{regression problems} featuring $\Y  \sse \RR$.

Nontrivial generalization properties are typically only possible if one assumes something about the data.\footnote{The need for such an assumption is formalized by the  \emph{no free lunch theorems} of supervised learning \cite{wolpert_connection_1992,wolpert_lack_1996,schaffer_conservation_1994}.} The Bayesian approach to  machine learning, common in many applications, assumes some parametric form for the distribution generating the data, and postulates a prior on the parameters. This is not the approach I will take in this article. Instead, I will focus on the frequentist --- and some would say ``worst-case'' or ``adversarial'' ---  approach that is common in the computational learning theory community, embodied by the PAC model. Here we assume that the (training and test) data can be explained, perhaps approximately, by a function in some ``simple enough to learn'' class of functions $\H \sse \Y^\X$, often called the \emph{hypotheses}. Equivalently, we  seek a predictor which explains the unseen data roughly  as well as the best hypothesis $h^* \in \H$, whether or not we assume that $h^*$ itself provides a perfect explanation.



 \paragraph{Common Algorithmic Templates.} Perhaps the best known general-purpose supervised learning algorithm is \emph{empirical risk minimization (ERM)}, which chooses as its predictor a hypothesis $f \in \H$ minimizing $\frac{1}{n} \sum_{i=1}^n \ell(f(x_i),y_i)$ --- a quantity called the \emph{training error}, \emph{empirical error}, or \emph{empirical risk} of $f$. %\footnote{When multiple hypotheses minimize the empirical risk, we assume ERM breaks ties arbitrarily.}
A common template for generalizing ERM involves adding a \emph{regularization term} $\psi(f)$ to the  objective function, typically chosen to measure some notion of ``hypothesis complexity.'' An algorithm instantiating this template is known as a \emph{structural risk minimizer (SRM)}, and chooses as its predictor the hypothesis $f \in \H$ minimizing the \emph{structural risk} $\frac{1}{n} \sum_{i=1}^n \ell(f(x_i),y_i) + \psi(f)$. Other well-known algorithms, such as gradient descent and its variations,  can frequently be interpreted as approximate implementations of ERM or SRM.


\paragraph{Proper vs Improper Learning.} A learning algorithm is said to be \emph{proper} if its predictor $f$ is always chosen from the hypothesis class, i.e., $f \in \H$, otherwise it is said to be \emph{improper}. ERM  is an example of a proper learning algorithm, as are SRM algorithms of the form described above.  In the \emph{proper regime} of learning, algorithms are required to be proper. This article will be concerned with the more flexible \emph{improper regime} (a.k.a \emph{representation-independent learning}), where no such constraint is placed on the learner. In other words, all we care about is predictive power at test time, rather than any insights derived from the functional form or representation of the predictor~itself.


\subsection{The PAC Model}
A standard mathematical setup for evaluation of supervised learning algorithms, at least in the theoretical computer science community, is Valiant's \emph{Probably Approximately Correct (PAC) model} of learning (see e.g.~\cite{kearns_introduction_1994,mohri_foundations_2018}). Here, we assume there is an unknown distribution $\D$ on $\X \times \Y$ from which training and test data are  drawn.  Specifically, the labeled datapoints of the training set  $(x_1,y_1), \ldots, (x_n,y_n)$, as well as the test data  $(x_\tst,y_\tst)$, are i.i.d.~from $\D$. Often it is assumed that $\D$ lies in some class of distributions of interest. The \emph{true expected loss}, or simply \emph{loss}, of a predictor $f: \X \to \Y$ is the expected loss it incurs on draws from $\D$, written $L_\D(f) = \Ex_{(x,y) \sim \D} \ell(f(x),y)$.


There are two main ``settings'' in PAC learning. The  \emph{realizable setting} only requires that the data be perfectly explained by some hypothesis in $\H$. More generally, the \emph{agnostic setting} makes no assumption relating the data to the hypotheses, but shifts the goalposts as necessary to allow nontrivial guarantees: the expected loss at test time is evaluated only ``relative'' to that of the best hypothesis $h^* \in \H$. There are other settings which make more nuanced assumptions, such as $\D$ being of a particular parametric form or its support living in some (unknown) lower-dimensional space, etc. I will mostly discuss the realizable and agnostic settings in this article, those being the simplest and most studied from a theoretical perspective. %TODO:We will briefly discuss other settings in Section ??

The PAC model demands high probability guarantees of learners, in the worst case over distributions of interest. Consider first the realizable setting, where $\D$ is such that $\min_{h \in \H} L_{\D}(h) = 0$. A PAC learner has \emph{error} $\epsilon=\epsilon(n)$ and \emph{confidence} $\delta=\delta(n)$ if, when training data consists of $n$ i.i.d~samples from a realizable distribution $\D$, it produces a predictor $f$  satisfying $L_\D(f) \leq \epsilon$ with probability at least $1-\delta$. In the agnostic setting, where $\D$ can be arbitrary, we require $L_\D(f) - \min_{h \in \H} L_\D(h) \leq \epsilon$ with probability $1-\delta$.

In both the realizable and agnostic settings, we look for PAC learners with small $\epsilon$ and $\delta$ as a function of the sample size $n$. An equivalent perspective looks at the sample complexity $m(\epsilon,\delta)$, which is the minimum sample size which guarantees error  at most $\epsilon$ with probability at least $1-\delta$. We say a problem is \emph{PAC learnable} if its PAC sample complexity is finite whenever $\epsilon,\delta > 0$.

For most PAC learning problems, learnability and sample complexity are characterized in terms of a  ``dimension'' of the hypothesis class. Most prominently this is the \emph{VC dimension} for binary classification, the \emph{fat shattering dimension} for agnostic regression, and the \emph{DS dimension} for multiclass classification (see \cite{anthony_neural_1999,daniely_optimal_2014,brukhim_characterization_2022}). Treatment of these is beyond the scope of this article. The unfamiliar reader need not worry, however,  as dimensions will feature only tangentially in our~discussion.




%\paragraph{Learning settings: Realizable, Agnostic, etc.} In learning theory, evaluating a supervised learning algorithm requires specifying a data model and an objective. We will leave the details of the data model flexible for now, to allow for both the PAC model and the adversarial transductive model. Nonetheless we will describe two variations, which we call ``settings'', which cut across different models. The  \emph{realizable setting}  requires only that the data be perfectly explained by some hypothesis $h \in \H$ --- i.e., there exists a hypothesis which is guaranteed to suffer a loss of $0$ on training and test data. The performance of the learning algorithm is its expected loss at test time for some ``worst case'' realizable instance. More generally, the \emph{agnostic setting} makes no assumption relating the data to the hypotheses, but shifts the goalposts as necessary to allow nontrivial guarantees: the expected loss at test time is evaluated only ``relative'' to that of the best hypothesis $h^* \in \H$, again for some ``worst case'' instance. There are other settings which make more nuanced assumptions about the data, such as it is drawn from a distribution of a particular parametric form, or that it lives in some (unknown) lower-dimensional space, etc. We will mostly discuss the realizable and agnostic settings, those being the simplest and most studied from a theoretical perspective.




%%% Local Variables:
%%% mode: latex
%%% TeX-master: "learning_matching"
%%% End:

\section{Characterizing LLM Inference}
\label{sec:characteristics}
This section characterizes LLM inference performance in on-server and on-device paradigms, which informs our design.

We evaluate four commercial streaming LLM APIs: OpenAI's GPT-4o-mini~\citep{gpt-4o-mini}, DeepSeek's DeepSeek-V2.5~\citep{deepseek-v2_5}, Cohere's Command~\citep{command}, and Hyperbolic-hosted LLaMA-3-70b-Instruct~\citep{llama3-70b}. For on-device analysis, we deploy Qwen-2.5-7B-Instruct~\citep{qwen2_5} and Llama-3.1-8B-Instruct~\citep{grattafiori2024llama3herdmodels} on both server-grade (NVIDIA A40, 48GB) and consumer-grade (dual NVIDIA RTX 3080, denoted as 3080x2) GPUs. 
We sample 1,000 requests from the Alpaca dataset~\citep{alpaca}, following a Poisson distribution with a mean request arrival interval of 30 seconds.

\begin{figure}
    \subfigure[On-Server TTFTs.]{\includegraphics[width=0.5\columnwidth]{figs/ttft_analysis_server.pdf}}\hfill
    \subfigure[On-Device TTFTs.]{\includegraphics[width=0.5\columnwidth]{figs/ttft_analysis_device.pdf}}
    \caption{On-device TTFT performance is more stable.}
    \label{fig:ttft_repeat}
\end{figure}

\paragraph{TTFT characteristics.} 
Our measurements reveal contrasting TTFT patterns between on-device and on-server inference. As shown in Figure~\ref{fig:ttft_repeat}, on-device inference exhibits stable TTFTs when processing identical prompts at 60-second intervals, primarily reflecting the prefill duration due to dedicated local hardware resources. In contrast, on-server inference experiences high variations and significant tail latency, attributed to network delays, request queuing, and resource contention.

We summarize the TTFT performance of 1,000 requests in Table~\ref{tab:correlation-analysis}. We observe that on-device TTFT scales linearly with prompt length due to hardware constraints~\citep{edgebenchmark}, while on-server TTFT shows minimal prompt-length sensitivity through advanced resource scaling~\citep{distserve,splitwise,memserve}.

\begin{table}[t]
    \centering
    \footnotesize
    \begin{tabularx}{\linewidth}{lcc}
    \toprule
    \textbf{Model} & \textbf{Deployment} & \textbf{Pearson Coef.} \\
    \midrule 
    Command & Server & 0.0142 \\
    GPT-4o-mini & Server & 0.0236 \\
    DeepSeek-V2.5 & Server & -0.0273 \\
    LLaMA-3-70b-Instruct & Server & 0.0402 \\
    \hline
    LLaMA-3.1-8b-Instruct & Device & 0.8424 \\
    \bottomrule
    \end{tabularx}
    \caption{Pearson coefficient between prompt length and TTFT in on-server deployment is weak.}
    \vskip -0.1in
    \label{tab:correlation-analysis}
\end{table}

\paragraph{TBT characteristics.}
TBT characterizes the I/O-bound decode stage latency. Analysis of temporal samples and distributions across six setups (Figure~\ref{fig:tbt_analysis}) reveals higher TBT variability in on-server inference compared to on-device execution. More importantly, both deployment approaches achieve generation speeds exceeding user consumption rates~(\S \ref{sec:llm_applications}), making cooperative serving practical.

\begin{figure}[t]
    \centering
    \includegraphics[width=0.49\textwidth]{figs/tbt_analysis.pdf}
    \vskip -0.1in
    \caption{On-device TBT performance is more stable. \footnotemark[1]{}
    }
    \vskip -0.1in
    \label{fig:tbt_analysis}
\end{figure}

\footnotetext[1]{On-server inference, such as in GPT, streams tokens with each packet containing multiple tokens, resulting in near-zero perceived TBTs.}

\paragraph{Opportunities and challenges.}
Our studies further reveal that as on-device models continue to improve---often fine-tuned for specific tasks~\citep{appleintelligence,liu2024mobilellm}---their performance increasingly matches that of on-server models in popular applications like instruction-following and translation (detailed in \S\ref{sec:evaluation} and Appendix~\ref{appendix:accuracy-eval}). However, deploying these models on-device introduces challenges such as long prefilling latency and startup overhead.

On the other hand, our real-world studies of conversational workloads highlight key opportunities: (i) on-server TTFT is largely unpredictable and shows minimal correlation with prompt length, whereas on-device TTFT scales nearly linearly with prompt length and is highly predictable; and (ii) both paradigms achieve token generation speeds that exceed typical user consumption rates.

Taking these findings together---particularly the predictable performance of on-device inference and the elastic scaling capabilities of server-based inference---we observe opportunities for optimization in cost-constrained device-server cooperative serving. Dynamic request migration between server and device endpoints during response generation can yield significant cost savings. 

\section{\disco{} Policies}
\label{sec:disco_policy}

\disco{} optimizes both QoE and cost through (1) dispatch control that determines where to initiate token generation, and (2) migration control that enables dynamic handoff during generation. The dispatch controller optimizes TTFT by strategically routing requests, while the migration controller maintains consistent TBT while reducing costs.

\subsection{Problem Formulation}
\label{subsec:model}
We propose a unified cost model combining both monetary bills from on-server inference and energy bills from on-device inference. Let $c^p_s$ and $c^d_s$ denote the per-token monetary costs for server prefill and decode phases respectively, while $c^p_d$ and $c^d_d$ represent the per-token energy costs for device prefill and decode phases. Converting between energy and monetary costs is done by a dynamic exchange rate $\lambda$, adjusted by users to reflect their preferences. We offer a user-friendly tunable budget ratio $b \in [0,1]$, representing the additional cost allowance beyond baseline costs. Our optimization objectives focus on: (1) minimizing both mean and tail TTFT, and (2) maintaining consistent token delivery at a specified pace (i.e., stable TBT).

\subsection{Dispatch Controller: Cost-Aware Request Routing}
\label{subsec:ttft_opt}
Based on our analysis in \S\ref{sec:characteristics}, server-side TTFT shows weak correlation with prompt length due to various factors (network delay, request queueing, etc.). We model server TTFTs as a known distribution, obtained either from server-provided information or device-side profiling. In contrast, device-side TTFT exhibits a linear relationship with prompt length, with the coefficient determined through offline profiling. 

Our key insight is that the optimization problem naturally decomposes into two scenarios based on dominant cost factors: device-constrained scenarios where energy consumption is the primary bottleneck, and server-constrained scenarios where API monetary costs dominate. This decomposition enables efficient solutions. Pseudocode for the dispatch controller is in Appendix~\ref{appendix:pseudocode_scheduling}.

\paragraph{Device-Constrained Optimization.} 

When device costs dominate ($\min(c^p_d, c^d_d) > \max(c^p_s, c^d_s)$), we need to carefully manage device resource usage under a budget constraint $\mathbb{E}[I_d(l)l] \leq b \cdot \mathbb{E}[l]$, where $l$ is the prompt length and $I_d(l)$ indicates device execution. The key challenge is balancing between two goals: leveraging device execution to bound worst-case latency while conserving energy on shorter prompts where possible.

Our solution uses a wait-time strategy: for each prompt of length $l$, we first try server execution and wait for time $w(l)$ before potentially starting device execution. This conserves device energy when the server responds quickly. We determine the optimal wait time through a two-phase approach:

\begin{denseitemize}
    \item \textbf{Phase 1 (Tail Protection):} We reserve budget portion $\alpha$ for worst-case scenarios by setting a maximum wait time $w_{tail} = F^{-1}(1-\min(\alpha,b))$, where $F(\cdot)$ is the server TTFT distribution. This ensures we have device resources ready when server latency exceeds its $(1-\min(\alpha,b))$-th percentile.
    
    \item \textbf{Phase 2 (Average Case):} With the remaining budget $(b-\alpha)$, we set length-dependent wait times:
    \begin{equation}
        w(l) = \begin{cases}
            0 & \text{if } l \leq l_{th} \\
            \min(\beta l, w_{tail}) & \text{otherwise}
        \end{cases}
    \end{equation}
    where $l_{th}$ is a threshold below which we start device execution immediately, and $\beta$ is chosen to satisfy:
    \begin{equation}
        \int_{l_{th}}^{\infty} (1-F(\beta l)) \cdot c^p_d \cdot l \cdot p(l)dl = (b-\alpha) \cdot \mathbb{E}[l]
    \end{equation}
\end{denseitemize}

This design guarantees worst-case TTFT through $w_{tail}$ while optimizing average performance by adaptively adjusting wait times based on prompt length. Whichever endpoint (server or device) generates the first token continues to the decode phase, while the other terminates.

\paragraph{Server-Constrained Optimization.}
When server costs dominate ($\max(c^p_s, c^d_s) > \min(c^p_d, c^d_d)$), we need to carefully manage server resource usage under a budget constraint $\mathbb{E}[I_s(l)l] \leq b \cdot \mathbb{E}[l]$, where $I_s(l)$ indicates server execution. Our analysis in \S\ref{sec:characteristics} shows that device TTFT scales linearly with prompt length as $T_d(l) = kl + c$, while server TTFT has minimal length correlation. This suggests a length-based routing strategy: short prompts run on device to conserve server budget, while long prompts use both endpoints to minimize TTFT.

We determine the length threshold $l_{th}$ by:
\begin{equation}
  \int_0^{l_{th}} l \cdot p(l) dl = (1-b) \cdot \mathbb{E}[l]
\end{equation}
This ensures prompts shorter than $l_{th}$ consume exactly $(1-b)$ fraction of total expected tokens through device-only execution, leaving the remaining longer prompts with sufficient server budget for concurrent execution on both endpoints.

\subsection{Migration Controller: Cost-Efficient Token Delivery}\label{sec:migration}
When both endpoints process a request, the constrained endpoint may win the prefill phase but incur higher decode costs. In such cases, we can migrate token generation to the other endpoint to reduce total cost while maintaining quality.

\paragraph{Efficient Token Transfer.}

When endpoints share the same vocabulary, we transmit token IDs rather than complete token representations. Additionally, we avoid transferring intermediate states (e.g., attention key-value cache) for two practical reasons: (1) endpoints often employ different model architectures optimized for their respective hardware, making state transfer incompatible, and (2) intermediate state transfer would incur significant network overhead. Migration triggers when projected cost savings exceed overhead:
\begin{equation}
    C_{migration} = \Delta c^d_{decode} \times l_{remaining}
\end{equation}
where $\Delta c^d_{decode} = |c^d_s - c^d_d|$ and $l_{remaining}$ denote the per-token decode cost difference between endpoints, and the expected remaining sequence length, respectively.

\paragraph{Buffer-Based Migration Protocol.}
To ensure smooth token delivery during migration, we introduce a token buffer that leverages the natural gap between token generation speed ($r_g$ tokens/s) and human consumption rate ($r_c$ tokens/s, typically $r_g > r_c$). The buffer size is set to:
\begin{equation}
    B = r_c \times t_m
\end{equation}
where $t_m$ is the estimated migration overhead time. Migration begins only when the buffer contains enough tokens ($B$) to mask the migration latency.

As shown in Figure~\ref{fig:cost_saving}, this design enables seamless handoff: the source endpoint (Row A) continues generation until the target endpoint (Row B) is ready, ensuring uninterrupted token delivery to users despite the underlying endpoint transition.

\begin{figure}[t]
    \centering
    \includegraphics[width=0.45\textwidth]{figs/migration.pdf}
    \caption{Token generation migration between endpoints. Row A shows the original sequence on the source endpoint, while Row B shows the sequence after migration to the target endpoint, maintaining consistent token delivery while reducing cost.}
    \label{fig:cost_saving}
\end{figure}

\section{Evaluation}
We provide three sets of insights into this section, organised as \textit{findings (F*)}. We quantitatively study the effect of the adversarial and counterfactual perturbations on the performance of informal reasoners and autoformalisation methods. Then, we dive deeper into method variants. Finally, 
we analyse the nature of formalisation errors made by the models.

\subsection{Robustness Analysis}
\paragraph{\textbf{\emph{F1: Noise perturbations have a stronger effect on formalisation methods than informal \ac{LLM} reasoners.}}}
Table~\ref{tab:distraction_k4_formalisation} shows that, on average, the accuracy of both direct and \ac{CoT} informal reasoning remains between $73\%$ and $74\%$ in the face of added noise. While the autoformalisation method performs similarly to informal reasoners on the original dataset, its performance decreases between $4\%$ and $11\%$. The accuracy drops especially with logical (L) and tautological (T) distractions, whose logical language formats trick the \ac{LLM} into formalizing the noisy clauses. On the other hand, the linguistically complex and more natural sentences of encyclopedic distractions show a minor effect, suggesting that \acp{LLM} successfully avoids formalizing the more complicated sentences.

\paragraph{\textbf{\emph{F2: All \ac{LLM}-based reasoning methods suffer a drop for counterfactual perturbations.}}} % influence .}}}
Table~\ref{tab:distraction_k4_formalisation} shows that counterfactual statements cause a significant decrease in performance for both the informal reasoners and autoformalisation methods of between $12\%$ and $13\%$ on average. 
Moreover, this observation also holds for all tested models, i.e., none are robust towards counterfactual perturbations across every evaluated dimension. Even the strongest model, GPT 4o-mini, yields a performance of 63-68\%, which is relatively close to the random performance of 50\%. The high impact of counterfactual statements (the single ``not'' inserted) could be due to the inability of \acp{LLM} to overwrite prior knowledge with explicitly stated information or memorization of the answers. We study the error sources further in §\ref{subsec:errors}.  

\noindent \paragraph{\textbf{\emph{F3: Introducing multiple noise sentences has an effect only for logical distractions.}}}
We show the impact of introducing between one and four sentences for the two top-performing autoformalisation models in Figure~\ref{fig:length_distraction}. The figure shows similar trends with and without counterfactual perturbations.
As additional logical distractions are introduced, the model performance consistently decreases. Tautological (T) distractions lead to a decline in accuracy with a single disruptive sentence, yet adding more noise does not worsen the outcome. 
The tautological corpus introduces truth constants for all sentences as a persistent unseen logical construct. Given that this leads only to a decrease for a single occurrence, we can assume that a model can consistently handle the same unseen logical construct. In contrast, the logical corpus increases the chance of adding text, requiring new, previously unseen reasoning constructs for each added sentence. The impact of encyclopedic noise remains negligible, generalising F1 to $k$ sentences. Similarly, counterfactual perturbations remain much more effective for all settings, generalising F2.

\begin{table}[!t]
\small
\setlength{\modelspacing}{2pt}
\setlength{\tabcolsep}{1.7pt} % Default value: 6pt
\setlength{\belowrulesep}{4pt}
\begin{threeparttable}
    \centering
    \begin{tabular}{cc l r rrr @{\quad} rrrr}
\toprule
\multirow{2}{*}{} & \multirow{2}{*}{} & Reasoning & \multirow{2}{*}{O} & \multicolumn{3}{c}{Distraction} & \multicolumn{4}{c}{Counterfactual} \\
 & & Format & & E& L & T & $\text{O}_C$ & $\text{E}_C$& $\text{L}_C$ & $\text{T}_C$\\
\midrule
\multirow{6}{*}{\rotatebox{90}{Gemma-2}} & \multirow{3}{*}{\rotatebox{90}{9b}}
   & Informal (direct) & \textbf{0.78} & \textbf{0.80} & \textbf{0.79} & \textbf{0.77} & 0.58 & 0.52 & 0.50 & 0.59 \\
 & & Informal (CoT) & 0.72 & 0.78 & 0.73 & 0.76 & 0.61 & \textbf{0.57} & \textbf{0.60} & \textbf{0.66} \\
 & & Formal (FOL) & 0.62 & 0.58 & 0.52 & 0.53 & \textbf{0.63} & 0.52 & 0.46 & 0.46 \\[\modelspacing]
\cmidrule{2-11}
 & \multirow{3}{*}{\rotatebox{90}{27b}} 
   & Informal (direct) & 0.71 & 0.69 & \textbf{0.66} & \textbf{0.68} & 0.59 & 0.51 & 0.54 & 0.59 \\
 & & Informal (CoT) & 0.66 & 0.65 & 0.64 & 0.63 & 0.62 & 0.58 & \textbf{0.62} & \textbf{0.64} \\
 & & Formal (FOL) & \textbf{0.74} & \textbf{0.74} & 0.61 & 0.61 & \underline{\textbf{0.72}} & \underline{\textbf{0.67}} & 0.58 & 0.51 \\[\modelspacing]
\midrule
\multirow{6}{*}{\rotatebox{90}{Mistral}} & \multirow{3}{*}{\rotatebox{90}{7B}} 
   & Informal (direct) & 0.77 & \textbf{0.77} & 0.75 & \textbf{0.79} & \textbf{0.63} & \textbf{0.54} & \textbf{0.54} & \textbf{0.66} \\
 & & Informal (CoT) & \textbf{0.79} & 0.75 & \textbf{0.77} & 0.78 & 0.55 & 0.52 & \textbf{0.54} & 0.58 \\
 & & Formal (FOL) & 0.62 & 0.58 & 0.54 & 0.57 & 0.50 & \textbf{0.54} & 0.51 & 0.52 \\[\modelspacing]
\cmidrule{2-11}
 & \multirow{3}{*}{\rotatebox{90}{Small}} 
   & Informal (direct) & \textbf{0.77} & \textbf{0.76} & \textbf{0.76} & \textbf{0.75} & 0.61 & 0.51 & 0.56 & 0.59 \\
 & & Informal (CoT) & 0.72 & 0.72 & 0.72 & 0.71 & \textbf{0.62} & \textbf{0.59} & \textbf{0.62} & \textbf{0.68} \\
 & & Formal (FOL) & 0.68 & 0.59 & 0.53 & 0.64 & 0.54 & 0.55 & 0.49 & 0.51 \\[\modelspacing]
\midrule
\multirow{6}{*}{\rotatebox{90}{Llama-3.1}} & \multirow{3}{*}{\rotatebox{90}{8B}} 
   & Informal (direct) & 0.63 & 0.61 & 0.64 & 0.66 & 0.61 & \textbf{0.62} & 0.59 & 0.61 \\
 & & Informal (CoT) & 0.73 & \textbf{0.73} & \textbf{0.71} & \textbf{0.72} & \textbf{0.62} & 0.59 & \textbf{0.61} & \textbf{0.65} \\
 & & Formal (FOL) & \textbf{0.77} & 0.71 & 0.63 & 0.52 & 0.60 & 0.58 & 0.55 & 0.52 \\[\modelspacing]
\cmidrule{2-11}
 & \multirow{3}{*}{\rotatebox{90}{70B}} 
   & Informal (direct) & 0.77 & 0.74 & 0.74 & 0.73 & 0.62 & 0.53 & 0.56 & 0.64 \\
 & & Informal (CoT) & \textbf{0.78} & \textbf{0.75} & \textbf{0.76} & \textbf{0.76} & 0.64 & 0.61 & \textbf{0.66} & \underline{\textbf{0.73}} \\
 & & Formal (FOL) & 0.74 & 0.73 & 0.71 & 0.71 & \textbf{0.66} & \textbf{0.62} & 0.59 & 0.57 \\[\modelspacing]
 \midrule
\multirow{3}{*}{\rotatebox{90}{GPT}} & \multirow{3}{*}{\rotatebox{90}{4o-mini}} 
   & Informal (direct) & 0.78 & 0.77 & 0.79 & 0.79 & 0.64 & 0.61 & 0.61 & 0.63 \\
 & & Informal (CoT) & 0.80 & 0.80 & \underline{\textbf{0.81}} & \underline{\textbf{0.82}} & \textbf{0.68} & \textbf{0.63} & \underline{\textbf{0.68}} & \textbf{0.64} \\
 & & Formal (FOL) & \underline{\textbf{0.84}} & \underline{\textbf{0.82}} & 0.73 & 0.79 & 0.63 & 0.62 & 0.57 & 0.54 \\[\modelspacing]
 \midrule
\multicolumn{2}{c}{\multirow{3}{*}{\textbf{Avg}}} 
 & Informal (direct) & 0.74 & 0.73 & 0.73 & 0.73 & 0.61 & 0.55 & 0.56 & 0.62 \\
 & & Informal (CoT) & 0.74 & 0.74 & 0.73 & 0.74 & 0.62 & 0.58 & 0.62 & 0.65 \\
  & & Formal (FOL) & 0.72 & 0.68 &	0.61 & 0.62 & 0.61 & 0.59 & 0.54 & 0.52 \\
\bottomrule
\end{tabular}
\caption{Accuracies of informal and autoformalisation-based deductive reasoners. The best overall model per dataset is underlined; the best model version is marked in bold.}
\label{tab:distraction_k4_formalisation}
\end{threeparttable}
\end{table} 

\begin{figure}[!t]
    \centering
    \scriptsize
    \begin{tikzpicture}
        \begin{axis}[name=gpt,
            title={GPT-4o-mini},
            width=0.6\linewidth,
            height=0.6\linewidth,
            xlabel={\# Noise sentences},
            ylabel={Accuracy},
            xmin=-0.1, xmax=4.1,
            ymin=0.5, ymax=0.9,
            xtick={1,2,4},
            ytick={0.55, 0.6, 0.65, 0.75, 0.8, 0.85},
            title style={yshift=-0.6em},
            legend style={at={(1,-0.15)},
	           anchor=north,legend columns=-1},
            x label style={at={(axis description cs:1,-0.05)},anchor=north},
            y label style={at={(axis description cs:-0.15,0.5)},anchor=south},
            ymajorgrids=true,
            grid style=dashed,
        ]
            \addplot[color=blue, mark=square,]
                coordinates {
                (0,0.848076939582825)(1,0.823076903820038)(2,0.826923072338104)(4,0.821153819561005)
                };
            \addplot[color=red, mark=triangle,]
                coordinates {
                (0,0.848076939582825)(1,0.817307710647583)(2,0.801923096179962)(4,0.759615361690521)
                };
            \addplot[color=green, mark=diamond,] 
                coordinates {
                (0,0.848076939582825)(1,0.767307698726654)(2,0.769230782985687)(4,0.803846180438995)
                };
            \addplot[color=blue, mark=square*] 
                coordinates {
                (0,0.627777755260468)(1,0.622222244739533)(2,0.600000023841858)(4,0.633333325386047)
                };
            \addplot[color=red, mark=triangle*,] 
                coordinates {
                (0,0.627777755260468)(1,0.611111104488373)(2,0.611111104488373)(4,0.594444453716278)
                };
            \addplot[color=green, mark=diamond*,] 
                coordinates {
                (0,0.627777755260468)(1,0.572222232818604)(2,0.538888871669769)(4,0.555555582046509)
                };
                \legend{E,L,T,$\text{E}_C$, $\text{L}_C$ , $\text{T}_C$}
        \end{axis}

        \begin{axis}[name=llama, at={($(gpt.east)+(0.1cm,0)$)},anchor=west,
            title={Llama 3.1 70b},
            width=0.6\linewidth,
            height=0.6\linewidth,
            xmin=-0.1,, xmax=4.1,
            ymin=0.5, ymax=0.9,
            xtick={1,2,4},
            ytick={0.55, 0.6, 0.65, 0.75, 0.8, 0.85},
            title style={yshift=-0.6em},
            yticklabel=\empty,
            ymajorgrids=true,
            grid style=dashed,
        ]
            \addplot[color=blue, mark=square,]
                coordinates {
                (0,0.838461518287659)(1,0.817307710647583)(2,0.805769205093384)(4,0.817307710647583)
                };
            \addplot[color=red, mark=triangle,]
                coordinates {
                (0,0.838461518287659)(1,0.819230794906616)(2,0.803846180438995)(4,0.771153867244721)
                };
            \addplot[color=green, mark=diamond,]
                coordinates {
                (0,0.838461518287659)(1,0.803846180438995)(2,0.807692289352417)(4,0.805769205093384)
                };
            \addplot[color=blue, mark=square*]
                coordinates {
                (0,0.627777755260468)(1,0.622222244739533)(2,0.577777802944183)(4,0.594444453716278)
                };
            \addplot[color=red, mark=triangle*,]
                coordinates {
                (0,0.627777755260468)(1,0.583333313465118)(2,0.561111092567444)(4,0.577777802944183)
                };
            \addplot[color=green, mark=diamond*,]
                coordinates {
                (0,0.627777755260468)(1,0.627777755260468)(2,0.566666662693024)(4,0.577777802944183)
                };
        \end{axis}
    \end{tikzpicture}
    \caption{Influence of the number of noisy sentences for FOL.}
    \label{fig:length_distraction}
\end{figure}



\subsection{Impact of Method Design}
\paragraph{\textbf{\emph{F4: \ac{CoT} prompting is most impactful when both noise and counterfactual perturbations are applied.}}}
The accuracies for the individual \acp{LLM} in Table~\ref{tab:distraction_k4_formalisation} show that the impact of \ac{CoT} is negligible for noise-only datasets (first four columns). Meanwhile, the benefit from \ac{CoT} is most pronounced in the datasets that combine noise and counterfactual perturbations.
The better-performing informal prompting strategy for a model remains stable for all types of distractions. Still, the decline in performance due to counterfactuals leads to a less consistent preference for a specific prompting style.

\paragraph{\textbf{\emph{F5: The best-performing grammar differs per model and is unstable across data versions.}}}

The evaluation of different logical forms for formal \ac{LLM}-based reasoning in Table~\ref{tab:distraction_k4_logical_form} shows the preference of some models for specific syntactic formats.
Llama 3.1 70B has a considerable improvement of $12\%$ with TPTP syntax on the original set, while Llama 3.1 8B benefits from the R-FOL syntax. However, all grammars show a declining accuracy trend and increased syntax errors for noise perturbations, where the best grammar loses its advantage over the rest. 
When comparing the grammars on the counterfactual partitions, we observe that TPTP is consistently more robust than the standard first-order logic grammar. Here, GPT 4o-mini shows a reduction from $O$ to $O_C$ of $20\%$ for FOL and only $12\%$ for the TPTP grammar. Since this does not correlate with fewer syntax errors, the formalisation in TPTP prevents semantical errors for counterfactual premises. 
A positive reading of these results, especially the minor differences between FOL and R-FOL, is that autoformalisation \acp{LLM} can adapt to the grammar syntax prescribed in the prompt without further loss in performance.

\begin{table}[!t]
\small
\setlength{\modelspacing}{2pt}
\setlength{\tabcolsep}{1.7pt} % Default value: 6pt
\setlength{\belowrulesep}{4pt}
\begin{threeparttable}
    \centering
    \begin{tabular}{cc l r rrr @{\quad} rrrr}
\toprule
\multirow{2}{*}{} & \multirow{2}{*}{} & Grammar & \multirow{2}{*}{O} & \multicolumn{3}{c}{Distraction} & \multicolumn{4}{c}{Counterfactual} \\
 & & Syntax & & E& L & T & $\text{O}_C$ & $\text{E}_C$& $\text{L}_C$ & $\text{T}_C$\\
\midrule
\multirow{6}{*}{\rotatebox{90}{Llama-3.1}} & \multirow{3}{*}{\rotatebox{90}{8B}} 
   & FOL & 0.77 & \textbf{0.71} & 0.61 & \textbf{0.53} & 0.58 & \textbf{0.55} & 0.52 & \textbf{0.56} \\
 & & R-FOL & \textbf{0.78} & 0.69 & \textbf{0.62} & \textbf{0.53} & 0.58 & \textbf{0.55} & \textbf{0.54} & 0.52 \\
 & & TPTP & 0.73 & 0.67 & 0.55 & 0.51 & \textbf{0.68} & 0.54 & 0.46 & 0.51 \\[\modelspacing]
\cmidrule{2-11}
 & \multirow{3}{*}{\rotatebox{90}{70B}} 
   & FOL & 0.76 & 0.73 & 0.71 & \textbf{0.72} & 0.67 & 0.57 & 0.63 & 0.56 \\
 & & R-FOL & 0.76 & 0.73 & 0.67 & 0.71 & 0.64 & 0.57 & 0.53 & 0.64 \\
 & & TPTP & \underline{\textbf{0.88}} & \underline{\textbf{0.84}} & \underline{\textbf{0.81}} & \textbf{0.72} & \underline{\textbf{0.81}} & \underline{\textbf{0.68}} & \underline{\textbf{0.67}} & \underline{\textbf{0.68}} \\[\modelspacing]
\midrule
\multirow{3}{*}{\rotatebox{90}{GPT}} & \multirow{3}{*}{\rotatebox{90}{4o-mini}} 
   & FOL & \textbf{0.84} & \textbf{0.82} & \textbf{0.72} & \underline{\textbf{0.78}} & 0.64 & \textbf{0.63} & \textbf{0.61} & 0.51 \\
 & & R-FOL & \textbf{0.84} & 0.77 & 0.70 & \underline{\textbf{0.78}} & \textbf{0.72} & 0.56 & 0.54 & \textbf{0.63} \\
 & & TPTP & 0.83 & \textbf{0.82} & 0.71 & 0.71 & 0.69 & \textbf{0.63} & 0.57 & 0.57 \\
\bottomrule
\end{tabular}
\caption{Accuracies of different formalisation grammars for autoformalisation.}
\label{tab:distraction_k4_logical_form}
\end{threeparttable}
\end{table} 

\paragraph{\textbf{\emph{F6: Feedback does not help \acp{LLM} self-correct to mitigate robustness issues.}}}
\autoref{tab:distraction_k4_feedback} shows the results with different error recovery mechanisms. The results indicate that no feedback strategy emerges as a winner in the different datasets. 
All feedback variants reduce syntax errors for noise perturbations, but given the lack of a consistent increase in accuracy, the corrected formalisations are most likely to contain semantic errors still. 
The type of feedback message only has a minor influence on correcting syntax errors, whereas Llama 3.1 70b and GPT 4o-mini correct slightly more syntax errors with specific error messages. This finding aligns with \cite{huang2023large}, who also found that \acp{LLM} cannot consistently self-correct their reasoning after receiving relevant feedback.

\begin{table}[!ht]
\small
\setlength{\modelspacing}{2pt}
\setlength{\tabcolsep}{1.7pt} % Default value: 6pt
\setlength{\belowrulesep}{4pt}
\begin{threeparttable}
    \centering
    \begin{tabular}{cc l r rrr @{\quad} rrrr}
\toprule
\multirow{2}{*}{} & \multirow{2}{*}{} & \multirow{2}{*}{Feedback} & \multirow{2}{*}{O} & \multicolumn{3}{c}{Distraction} & \multicolumn{4}{c}{Counterfactual} \\
 & & & & E& L & T & $\text{O}_C$ & $\text{E}_C$& $\text{L}_C$ & $\text{T}_C$\\
\midrule
\multirow{8}{*}{\rotatebox{90}{Llama-3.1}} & \multirow{4}{*}{\rotatebox{90}{8B}} 
   & No recovery & 0.77 & \textbf{0.72} & 0.62 & 0.53 & 0.59 & 0.58 & 0.56 & \textbf{0.56} \\
 & & Error type & \textbf{0.79} & 0.71 & 0.63 & \textbf{0.56} & \textbf{0.66} & 0.54 & 0.52 & 0.51 \\
 & & Error message & 0.78 & 0.71 & \textbf{0.67} & 0.55 & 0.59 & 0.53 & \underline{\textbf{0.64}} & 0.49 \\
 & & Warning & 0.74 & 0.66 & 0.58 & 0.55 & 0.55 & \textbf{0.60} & 0.49 & 0.49 \\[\modelspacing]
\cmidrule{2-11}
 & \multirow{4}{*}{\rotatebox{90}{70B}} 
   & No recovery & \textbf{0.77} & \textbf{0.72} & \textbf{0.73} & 0.71 & \textbf{0.64} & 0.59 & \textbf{0.61} & 0.56 \\
 & & Error type & 0.72 & 0.70 & 0.72 & \textbf{0.73} & 0.62 & 0.56 & 0.60 & 0.58 \\
 & & Error message & 0.71 & 0.70 & \textbf{0.73} & 0.71 & \textbf{0.64} & 0.59 & 0.54 & \underline{\textbf{0.64}} \\
 & & Warning & 0.69 & \textbf{0.72} & 0.72 & 0.72 & 0.62 & \underline{\textbf{0.65}} & \textbf{0.61} & 0.63 \\[\modelspacing]
\midrule
\multirow{4}{*}{\rotatebox{90}{GPT}} & \multirow{4}{*}{\rotatebox{90}{4o-mini}} 
   & No recovery & \underline{\textbf{0.84}} & \underline{\textbf{0.82}} & 0.73 & 0.79 & 0.64 & \textbf{0.62} & 0.56 & \textbf{0.56} \\
 & & Error type & 0.83 & 0.79 & 0.74 & 0.76 & 0.67 & 0.57 & 0.56 & \textbf{0.56} \\
 & & Error message & \underline{\textbf{0.84}} & 0.78 & \underline{\textbf{0.77}} & \underline{\textbf{0.80}} & 0.62 & 0.59 & 0.56 & \textbf{0.56} \\
 & & Warning & \underline{\textbf{0.84}} & 0.75 & 0.73 & 0.76 & \underline{\textbf{0.70}} & 0.61 & \textbf{0.61} & 0.55 \\
 \bottomrule
\end{tabular}
\caption{Accuracies of error recovery strategies.}
\label{tab:distraction_k4_feedback}
\end{threeparttable}
\end{table} 

\subsection{Error Analysis}
\label{subsec:errors}
\paragraph{\textbf{\emph{F7: Autoformalisation increases syntax errors for noise perturbations.}}}
The low performance for noise perturbations correlates with more syntax errors for all models and distraction categories (cf. execution rates in Table~\ref{tab:appendix_k4_formalisation_exec}). The three worst-performing models (both Mistral models, Gemma-2 9b) generate, at best, for $37\%$  and, at worst, for only $4\%$ of the samples, a valid logical form.
Gemma-2 9b and Llama3.1 8b produce more syntax errors than the larger counterparts, suggesting that larger models are more robust towards noise perturbations. 
The accuracy of syntactically valid samples is higher than the informal reasoning methods for most distractions (Table~\ref{tab:appendix_k4_formalisation_vacc}), motivating informal reasoning as a backup strategy for formal reasoning. The error message feedback reveals two common syntax errors: 1) errors by models with an initial low execution rate exhibit issues with the template structure, including using incorrect keywords or adding conversational phrases;
2) perturbation-related errors, the most common of which is using undefined truth constants as part of tautological distractions. 

\paragraph{\textbf{\emph{F8: Autoformalisation increases semantic errors for counterfactuals.}}}
Unlike the introduced noise, counterfactual perturbations do not lead to more syntax errors. The execution rate in Table~\ref{tab:appendix_k4_formalisation_exec} is stable or improves for counterfactuals. However, we see a drop in accuracy for the counterfactual column $\text{O}_C$ in Table~\ref{tab:distraction_k4_formalisation} and can conclude that the number of logical forms with semantic errors has to increase. This suggests that the introduced negation is not correctly formalised. Looking at the warnings generated by the feedback mechanism, for GPT 4o-mini, $161$ warning messages are generated on the unperturbed data. $54$ of these were fixed with a single iteration. Not considering predicates and individuals as part of the context is the most frequent warning across all models. 
\section*{Conclusion}
This paper aims to enhance our understanding of the computational complexity of computing various Shapley value variants. We found that for various ML models --- including decision trees, regression tree ensembles, weighted automata, and linear regression --- both local and global interventional and baseline SHAP can be computed in polynomial time under HMM modeled distributions. This extends popular algorithms, such as TreeSHAP, beyond their empirical distributional scope. We also establish strict complexity gaps between the various SHAP variants (baseline, interventional, and conditional) and prove the intractability of computing SHAP for tree ensembles and neural networks in simplified scenarios. Overall, we present SHAP as a versatile framework whose complexity depends on four key factors: \begin{inparaenum}[(i)] \item model type, \item SHAP variant, \item distribution modeling approach, \item and local vs. global explanations\end{inparaenum}. We believe this perspective provides deeper insight into the computational complexity of SHAP, paving the way for future work.




%We believe that our framework provides a more intricate understanding of SHAP computation complexity across different models, distributions, and variants, paving the way for further research.

Our work opens promising directions for future research. First, expanding our computational analysis to other SHAP-related metrics, such as asymmetric SHAP~\citep{frye20} and SAGE~\citep{covert2020understanding}, would be valuable. Additionally, we aim to explore more expressive distribution classes and relaxed assumptions beyond those in Section \ref{sec:tractable} while maintaining tractable SHAP computation. Finally, when exact computation is intractable (Section \ref{sec:intractable}), investigating the approximability of SHAP metrics through approximation and parameterized complexity theory~\citep{downey2012parameterized} is an important direction.

%Our work opens several promising avenues for future research on the computational properties of explainable AI methods, with a particular focus on SHAP. First, it would be interesting to broaden the computational analysis conducted in this work to include other popular SHAP-related metrics in the literature, such as asymmetric SHAP \cite{frye20} and SAGE \cite{covert2020understanding}. Also, in the future, we aim to explore more expressive distribution classes and relaxed distributional assumptions—extending beyond those examined in Section \ref{sec:tractable} —that still yield tractable SHAP computation. Finally, when exact computation proves intractable (Section \ref{sec:intractable}), it is worthwhile to theoretically investigate the question of the approximability of computing the SHAP metrics across various configurations, through the lens of approximation and parametrized complexity theory \cite{arora2009computational}.

%This paper aims to deepen our understanding of the computational complexity involved in obtaining different Shapley value variants. We found that for a variety of ML models, including decision trees, tree ensembles for regression, weighted automata, and linear regression models — computing both local and global interventional and baseline SHAP can be done in polynomial time when distributions are modeled by HMMs. This extends the distributional scope of popular algorithms like TreeSHAP, which is limited to empirical distributions. Additionally, we demonstrate a strict complexity gap between SHAP variants, showing that interventional and baseline SHAP can be strictly easier to compute than conditional SHAP. Despite these positive results, we uncovered intractability for various SHAP variants in neural networks and tree ensembles. Finally, we provided generalized complexity relations across SHAP variants. We believe that our framework offers a deeper understanding of the complexity involved in computing SHAP across various variants, models, distributions, as well as in both local and global computations, laying the groundwork for future research.

\section{Limitations}
While \disco{} demonstrates significant improvements in LLM serving efficiency, we acknowledge several important limitations of our current work:

\paragraph{Model Coverage.} We focus on scenarios where on-device LLMs achieve sufficient accuracy for target applications. While this covers many common use cases, \disco{} may not be suitable for applications requiring complex reasoning.

\paragraph{Energy Modeling.} For device energy consumption, we use a linear energy model based on FLOPs. Real-world device energy consumption patterns can be more complex, varying with factors such as battery state, temperature, and concurrent workloads.

\paragraph{Scalability Considerations.} Our current implementation and evaluation focus on single-device scenarios. Extending \disco{} to handle multi-device collaborative serving presents additional challenges in terms of coordination overhead and resource allocation that warrant further investigation.

\section{Ethical Considerations}
Our work focuses solely on optimizing the efficiency of LLM serving systems through device-server collaboration and does not introduce new language generation capabilities or content. All experiments were conducted using publicly available models and datasets. While our work may indirectly benefit the accessibility of LLM services by reducing costs and improving performance, we acknowledge that broader ethical considerations around LLM deployment and usage are important but outside the scope of this technical contribution.

\bibliography{main}
\bibliographystyle{acl_natbib}

\appendix
\section{Related Work}
% \subsection{Vision Language Model}
% 시각장애인에서 상황을 설명할 DB가 없으니 만들었다. 그리고 이를 VLM에 튜닝했다.
\subsection{Technical approaches for assisting the visually-impaired}


\subsection{Datasets for visual instruction tuning}

\section{Cold Start Evaluation}\label{appendix:cold_start}

This section presents cold start performance measurements for the Qwen-2.5 model series across different hardware configurations.
The experiments were conducted on two platforms: Windows 10 with NVIDIA RTX 3060 12GB and Linux with NVIDIA A40 48GB. A fixed prompt "How to use GitHub?" was used throughout all experiments. We measured two critical metrics: model loading time and TTFT for Qwen-2.5 models ranging from 0.5B to 7B parameters, all using FP16 precision. The experimental setup consisted of 10 measurement runs, with 2 additional warmup runs to ensure measurement stability. It is worth noting that such warmups can potentially mask the true gap between model loading and prompt prefill time due to various optimizations, including OS page cache. To maintain authentic cold start conditions, we explicitly cleared the CUDA cache and performed garbage collection before each run.

The results revealed several significant patterns. On the RTX 3060, the loading time exhibits an approximately linear increase with model size, ranging from 1.29s for the 0.5B model to 4.45s for the 3B model. While TTFT follows a similar trend, the processing time is substantially lower, ranging from 0.051s to 0.145s. On the A40 GPU, despite observing longer loading times, TTFT is significantly reduced across all models, maintaining a remarkably consistent value regardless of model size. These findings indicate that while model loading remains more resource-intensive on our Linux setup, the inference performance benefits substantially from the A40's superior computational capabilities.

\begin{table}[ht]
    \centering
    \footnotesize
    \vskip 0.1in
    \begin{tabular}{p{2.5cm}cccc}
    \toprule
    \textbf{Metric} & \textbf{0.5B} & \textbf{1.5B} & \textbf{3B} & \textbf{7B} \\
    \midrule
    \multicolumn{5}{c}{\textbf{Windows 10 (NVIDIA RTX 3060 12GB)}} \\
    \midrule
    Load Time (s) & 1.29 & 2.48 & 4.45 & - \\
    TTFT (s) & 0.051 & 0.105 & 0.145 & - \\
    \midrule
    \multicolumn{5}{c}{\textbf{Linux (NVIDIA A40 48GB)}} \\
    \midrule
    Load Time (s) & 1.53 & 3.12 & 5.72 & 13.43 \\
    TTFT (s) & 0.025 & 0.026 & 0.033 & 0.033 \\
    \bottomrule
    \end{tabular}
    \caption{Model loading time during cold start can significantly slow down TTFT. Average Qwen-2.5 model performance over 10 runs. The 7B model exceeds the memory capacity of the RTX 3060 and thus cannot be evaluated.}
    \label{tab:cold-start}
\end{table}

\section{Prediction-based Model Selection}\label{appendix:prediction}

This section provides a comparative analysis of several TTFT prediction methods. For selecting the endpoint with a lower TTFT for each request, TTFT prediction is imperative. For on-device inference, TTFT prediction is straightforward, as TTFT exhibits a linear relationship with prompt length. Conversely, on-server inference TTFT is characterized by high variability, rendering prediction challenging. Moreover, the prediction method itself must be computationally efficient, as its overhead also contributes to end-to-end TTFT.

Table~\ref{tab:model-comparison} presents a comparative analysis of four common lightweight time-series-based prediction methods applied to traces collected from three prevalent LLM services. Our correlation analysis (Table~\ref{tab:correlation-analysis}) revealed no significant correlation between prompt length and TTFT; thus, prompt length is omitted as a feature in these prediction methods. We demonstrate that none of these methods offers sufficient accuracy for TTFT prediction.

\begin{table}[t]
    \centering
    \footnotesize
    \begin{tabular}{p{3.5cm}cc}
    \toprule
    \textbf{Model} & \textbf{MAPE(\%)} & \textbf{MAE(s)} \\
    \midrule
    \multicolumn{3}{c}{\textbf{Command}} \\
    \midrule
    Moving Average & 39.40 & 0.0899 \\
    ExponentialSmoothing & 53.51 & 0.1047 \\
    Random Forest & 39.33 & 0.0966 \\
    XGBoost & 35.43 & 0.0905 \\
    \midrule
    \multicolumn{3}{c}{\textbf{DeepSeek-V2.5}} \\
    \midrule
    Moving Average & 27.80 & 0.3959 \\
    ExponentialSmoothing & 27.39 & 0.3771 \\
    Random Forest & 32.97 & 0.4745 \\
    XGBoost & 27.51 & 0.4001 \\
    \midrule
    \multicolumn{3}{c}{\textbf{GPT-4o-mini}} \\
    \midrule
    Moving Average & 24.55 & 0.0995 \\
    ExponentialSmoothing & 20.88 & 0.0844 \\
    Random Forest & 28.68 & 0.1128 \\
    XGBoost & 24.83 & 0.0997 \\
    \midrule
    \multicolumn{3}{c}{\textbf{LLaMA-3-70b-Instruct}} \\
    \midrule
    Moving Average & 42.18 & 0.3312 \\
    ExponentialSmoothing & 40.27 & 0.3154 \\
    Random Forest & 49.67 & 0.3875 \\
    XGBoost & 43.94 & 0.3451 \\
    \bottomrule
    \end{tabular}
    \caption{Comparative analysis of Moving Average, Exponential Smoothing, Random Forest, and XGBoost prediction models across Command, DeepSeek, GPT, and LLaMA model traces. Metrics include Mean Absolute Percentage Error (MAPE) and Mean Absolute Error (MAE).}
    \label{tab:model-comparison}
\end{table}
\section{Response Quality}\label{appendix:accuracy-eval}

This section examines the quality of responses generated by \disco{}, with a particular focus on quality preservation during endpoint transitions. We first establish bounds on generation quality, then present our evaluation methodology, and finally demonstrate through extensive experiments that \disco{} maintains consistent quality across different model configurations and tasks.

\subsection{Quality Bounds}

A critical aspect of \disco{} is maintaining generation quality during endpoint transitions. We employ a systematic approach to quality preservation~\cite{diba2017weakly,gupta2022semi,chen2023frugalgpt}. Specifically, for endpoints A and B with quality metrics $Q_A$ and $Q_B$ (measured by LLM scores or ROUGE scores), we find that any migrated sequence M with quality $Q_M$ satisfies:

\begin{equation}
    \min(Q_A, Q_B) \leq Q_M \leq \max(Q_A, Q_B)
\end{equation}

This bound ensures that migration does not degrade quality beyond the capabilities of individual endpoints.

\subsection{Evaluation Methodology}

\paragraph{Evaluation Framework}
We establish a comprehensive assessment framework encompassing both automated metrics and LLM-based evaluation. Our framework evaluates two distinct tasks:

\begin{itemize}
    \item \textbf{Instruction Following}: We evaluate 500 data items from the Alpaca dataset~\cite{alpaca} using our structured prompt template, with quality assessment performed by multiple LLM judges: Gemini1.5-pro, GPT-4o, and QWen2.5-72b-instruct.
    
    \item \textbf{Translation Quality}: We assess Chinese-to-English translation on 500 data items from Flores\_zho\_Hans-eng\_Latn dataset~\cite{nllb2022,goyal2022flores} using the ROUGE-1 metric.
\end{itemize}

These two tasks are popular on end-user devices. Understandably, for complex tasks such as advanced math reasoning, we notice DisCo can lead to accuracy drops compared to the on-server model due to the limited capability of the on-device models, yet still achieves better performance than the on-device counterpart. 

\paragraph{Experimental Setup}
We configure our experiments with:
\begin{itemize}
    \item A fixed maximum generation length of 256 tokens
    \item First endpoint's maximum generation length varied through [0, 4, 16, 64, 256] tokens
    \item Four model combinations: 0.5B-7B, 3B-7B, 7B-0.5B, and 7B-3B (prefix and suffix denote the model sizes of first and second endpoints respectively)
\end{itemize}

The generation transitions to the second endpoint when the first endpoint reaches its length limit without producing an end-of-generation token, creating natural boundary conditions for analysis.

For instruction-following tasks, we employ the following structured evaluation template:

\begin{footnotesize}
\begin{verbatim}
JUDGE_PROMPT = """Strictly evaluate the 
quality of the following answer on a scale
of 1-10 (1 being the worst, 10 being the 
best). First briefly point out the problem
of the answer, then give a total rating in
the following format.

Question: {question}

Answer: {answer}

Evaluation: (your rationale for the rating, 
as a brief text)

Total rating: (your rating, as a number 
between 1 and 10)
"""
\end{verbatim}
\end{footnotesize}


\subsection{Results and Analysis}

\subsubsection{Quality Metrics}
Our comprehensive evaluation reveals several key findings:
\begin{itemize}
    \item \textbf{Bounded Quality}: The combined sequence quality consistently remains bounded between individual model performance levels
    \item \textbf{Translation Performance}: ROUGE-1 scores maintain stability between 0.23 and 0.26
    \item \textbf{Instruction Following}: Scores show consistent ranges from 4 to 6
\end{itemize}

\begin{figure}[t]
    \centering
    \includegraphics[width=0.45\textwidth]{figs/acc_translation.pdf}
    \\[12pt]
    \includegraphics[width=0.45\textwidth]{figs/acc_instruction-05b7b.pdf}
    \caption{Quality evaluation results of \disco{}. The top figure shows translation quality evaluation using ROUGE-1 scores, demonstrating that \disco{} consistently achieves higher quality than the on-device baseline. The bottom figure presents evaluation scores from different LLM judges on instruction-following capabilities, where each subplot represents a different model pair comparison with varied first-endpoint model's maximum sequence length. The consistent patterns across different LLM judges demonstrate the robustness of our evaluation framework.}
    \label{fig:accuracy}
\end{figure}

\section{Experiment Settings for End-to-end Cost}
\label{appendix:unified_cost}

For on-device LLMs, we quantify cost using FLOPs (floating-point operations). For on-server LLM services, we use their respective pricing rates at the time of experimentation. We set the energy-to-monetary conversion ratio (\textit{energy\_to\_money}) to 0.3 \$ per million FLOPs for server-constrained experiments and 5 \$ per million FLOPs for device-constrained experiments. To establish a comprehensive cost model that enables direct comparison between device and server computation costs, we analyze both the computational complexity of on-device models through detailed FLOPs calculations (Section~\ref{sec:flops_analysis}) and the pricing structures of commercial LLM services (Section~\ref{appendix:llm_pricing}). The generation length limit is set to 128.


\subsection{FLOPs of On-Device LLMs}
\label{sec:flops_analysis}

To accurately quantify the computational cost per token in both prefill and decode stages, we conduct a detailed FLOPs analysis using three representative models: BLOOM-1.1B, BLOOM-560M, and Qwen1.5-0.5B. All models share a 24-layer architecture but differ in other parameters: BLOOM-1.1B ($d_{\text{model}}=1024$, 16 heads, FFN dim=4096), BLOOM-560M ($d_{\text{model}}=512$, 8 heads, FFN dim=2048), and Qwen1.5-0.5B ($d_{\text{model}}=768$, 12 heads, FFN dim=2048).

\paragraph{Per-token FLOPs computation.}
The total FLOPs for processing each token consist of five components:
\begin{align}
    \text{FLOPs}_{\text{total}} &= \text{FLOPs}_{\text{attn}} + \text{FLOPs}_{\text{ffn}} \nonumber \\
    &\quad+ \text{FLOPs}_{\text{ln}} + \text{FLOPs}_{\text{emb}} + \text{FLOPs}_{\text{out}}
\end{align}

For a sequence of length $L$, the attention computation differs between stages. In prefill:
\begin{align}
    \text{FLOPs}_{\text{attn}} &= n_{\text{layers}} \cdot \Big(3d_{\text{model}}^2 + \frac{L^2 d_{\text{model}}}{n_{\text{heads}}} \nonumber \\
    &\quad+ L d_{\text{model}} + d_{\text{model}}^2\Big)
\end{align}

While in decode, KV caching eliminates the quadratic term:
\begin{align}
    \text{FLOPs}_{\text{attn}} &= n_{\text{layers}} \cdot \Big(3d_{\text{model}}^2 + \frac{L d_{\text{model}}}{n_{\text{heads}}} \nonumber \\
    &\quad+ L d_{\text{model}} + d_{\text{model}}^2\Big)
\end{align}

Table~\ref{tab:prefill-decode} presents the total FLOPs across different sequence lengths. The decode phase maintains constant FLOPs regardless of sequence length due to KV caching, while prefill phase FLOPs increase with sequence length. A breakdown of computational cost by component (Table~\ref{tab:components}) reveals that embedding and output projection operations account for the majority of FLOPs, particularly in models with large vocabularies.

\begin{table}[t]
    \centering
    \footnotesize
    \begin{tabularx}{\linewidth}{lccc}
    \toprule
    \textbf{Length} & \textbf{BLOOM-1.1B} & \textbf{BLOOM-560M} & \textbf{Qwen-0.5B} \\
    \midrule
    \multicolumn{4}{l}{\textit{Prefill Phase}} \\
    L = 32 & 0.85 & 0.45 & 0.39 \\
    L = 64 & 0.93 & 0.50 & 0.45 \\
    L = 128 & 1.25 & 0.65 & 0.69 \\
    \midrule
    \multicolumn{4}{l}{\textit{Decode Phase}} \\
    L = 32 & 0.82 & 0.42 & 0.37 \\
    L = 64 & 0.82 & 0.42 & 0.37 \\
    L = 128 & 0.82 & 0.42 & 0.37 \\
    \bottomrule
    \end{tabularx}
    \caption{Prefill and Decode FLOPs (billions)}
    \vskip -0.1in
    \label{tab:prefill-decode}
\end{table}

\begin{table}[t]
    \centering
    \footnotesize
    \begin{tabularx}{\linewidth}{lccc}
    \toprule
    \textbf{Component} & \textbf{BLOOM-1.1B} & \textbf{BLOOM-560M} & \textbf{Qwen-0.5B} \\
    \midrule
    Embedding & 31.24 & 25.00 & 31.51 \\
    Attention & 13.01 & 10.00 & 16.56 \\
    FFN & 24.48 & 20.00 & 20.38 \\
    LayerNorm & 0.02 & 0.02 & 0.04 \\
    Output & 31.24 & 25.00 & 31.51 \\
    \bottomrule
    \end{tabularx}
    \caption{Component Ratios at L=128 (\%)}
    \vskip -0.1in
    \label{tab:components}
\end{table}

\subsection{LLM Service Pricing}\label{appendix:llm_pricing}
This section provides further details on the pricing of LLM services. Table \ref{tab:model-pricing} presents the pricing models for several commercial Large Language Models (LLMs) as of October 28, 2024. The pricing structure follows a dual-rate model, differentiating between input (prompt) and output (generation) tokens. These rates represent the public pricing tiers available to general users, excluding any enterprise-specific arrangements or volume-based discounts.

\begin{table}[t]
    \centering
    \footnotesize
    \begin{tabularx}{\linewidth}{lcccc}
    \toprule
    \textbf{Model} & \textbf{Vendor} & \textbf{Input price} & \textbf{Output price} \\
    \midrule
    DeepSeek-V2.5     & DeepSeek   & 0.14 & 0.28 \\
    GPT-4o-mini       & OpenAI     & 0.15 & 0.60 \\
    LLaMa-3.1-70b     & Hyperbolic & 0.40 & 0.40 \\
    LLaMa-3.1-70b     & Amazon     & 0.99 & 0.99 \\
    Command           & Cohere     & 1.25 & 2.00 \\
    GPT-4o            & OpenAI     & 2.50 & 10.0 \\
    Claude-3.5-Sonnet & Anthropic  & 3.00 & 15.0 \\
    o1-preview        & OpenAI     & 15.0 & 60.0 \\
    \bottomrule
    \end{tabularx}
    \caption{LLM service pricing (USD per 1M Tokens). Input prices refer to tokens in the prompt, while output prices apply to generated tokens.}
    \label{tab:model-pricing}
\end{table}

\newcommand{\expkernel}{e^{- \Delta t / \lambda^2}}

\begin{figure}[h!]
% \centering
\SetKwInOut{Input}{Input}
\SetKwInOut{Output}{Output}
\Input{pre-trained diffusion model $\epsilon_\theta : \sR^{d_1 \times \dots \times d_k} \to \sR^{d_1 \times \dots \times d_k}$, target parameters $\psi \in \sR^d$, condition $c$, mapping function $g(\psi) : \sR^d \to \sR^{d_1 \times \dots \times d_k}$, time-dependent functions $w(t), \alpha(t)$, Monte Carlo sample size $N$.\\
}
\Output{$\psi^{*}$}
\begin{minipage}[t]{.45\textwidth}
    \centering
    \begin{algorithm}[H]
    \caption{Distillation via SDS} \label{alg:sds}
    \For{$k = 1,\hdots,steps$}{
        $x^k \gets g(\psi)$\\
        \For{$i = 1,\hdots,N$}{
            $t  \gets \text{U}(0, 1)$\\
            % $w(t) \gets 1 - \alpha_t$\\
            % $\epsilon_t \gets t \cdot \N(0,I)$\\
            $z_t \gets \alpha(t)  x^k + \epsilon_t$\\
            $y_i \gets w(t)  [ \epsilon_\theta(z_t, t,c) - \epsilon_t ]$
        }
        $\nabla_{\psi} \mathcal{L}_{SDS} \gets \frac{1}{N} \sum (y_i - x^k)$\\
        
        \Comment{Backpropagate $\nabla_{\psi} \mathcal{L}_{SDS}$, update $\psi$}
    }
    \end{algorithm}
\end{minipage}
\hspace{0.7cm}
\begin{minipage}[t]{.47\textwidth}
    \centering
    \begin{algorithm}[H]
    \caption{Distillation via \methodname (Ours)} \label{alg:msg}
    \Function{ODESolver($x, \lambda$) (eq~\ref{eq:kernel_guided_score})}{
        $z_T \gets \N(0,I)$\\
        \For{$t = T-1,\hdots,0$}
        {
            $z_t \gets \epsilon_\theta(z_t, t, c) - (x - z_t) / \lambda^2$\\
        }
        return $z_0$
    }
    \Function{ODESolver($x, \lambda$, stable) (eq~\ref{eq:integratingkernel})} {
        $\{z^*_t\}^T_{t=0} \gets inversion(x)$\\
        $z_T \gets  z^{*}_T  +  (\epsilon - z^{*}_T) \expkernel $\\
        % $z_T \gets  \expkernel \epsilon +  (1 - \expkernel) z^{*}_T $\\
        \For{$t = T-1,\hdots,0$}
        {
            $z_T \gets z^{*}_t +  (\epsilon_\theta(z_t, t, c) - z^{*}_t) \expkernel $\\
            % $z_T \gets \expkernel \epsilon_\theta(z_t, t, c) + z^{*}_t (1 - \expkernel)$\\
        }
        return $z_0$\\
    }
    \Comment{initialize $\lambda$, set $\lambda_{min}$}
    \For{$k = 1,\hdots,steps$}{
        $x^k \gets g(\psi)$ \\        
        \For{$i = 1,\hdots,N$}{
            $y_i \gets \text{ODESolver}(x^k, \lambda)$
        }
        $\nabla_{\psi} \mathcal{L}_{\methodname} \gets \frac{1}{N} \sum^N_i (y_i - x^k)$\\
        
        \Comment{Backpropagate $\nabla_{\psi} \mathcal{L}_{\methodname}$, update $\psi$}
        \Comment{Anneal $\lambda$}
        \If{$\lambda < \lambda_{min}$}{
            \Comment{terminate}
        }
    }
    \end{algorithm}
\end{minipage}
\caption{
    Pseudocode of SDS and our procedure, \methodname. We additionally show the numerically stable solver, which is used for experiments with Stable Diffusion.
}
\label{alg:sds_vs_ours}
\end{figure}

% \begin{figure}[h!]
% \centering
% \begin{minipage}{.3\textwidth}
% % \centering
% \begin{algorithm}[H] \footnotesize
    % \caption{Distillation via SDS} \label{alg:sds}
    % \SetKwInOut{Input}{Input}
    % \SetKwInOut{Output}{Output}
    % \Input{pre-trained diffusion model $\epsilon_\theta : \sR^{d_1 \times \dots \times d_k} \to \sR^{d_1 \times \dots \times d_k}$, target parameters $\psi \in \sR^d$, condition $c$, mapping function $g(\psi) : \sR^d \to \sR^{d_1 \times \dots \times d_k}$, time-dependent functions $w(t), \alpha(t)$, sample size $N$.\\
    % }
    % \Output{$\psi^{*}$}
    % \For{$k = 1,\hdots,steps$}{
    %     $x^k \gets g(\psi)$\\
    %     \For{$i = 1,\hdots,N$}{
    %         $t  \gets \text{U}(0, 1)$\\
    %         % $w(t) \gets 1 - \alpha_t$\\
    %         % $\epsilon_t \gets t \cdot \N(0,I)$\\
    %         $z_t \gets \alpha(t)  x^k + \epsilon_t$\\
    %         $y_i \gets w(t)  [ \epsilon_\theta(z_t, t,c) - \epsilon_t ]$
    %     }
    %     $\nabla_{\psi} \mathcal{L}_{SDS} \gets \frac{1}{N} \sum y_i$\\
        
    %     \Comment{Backpropagate $\nabla_{\psi} \mathcal{L}_{SDS}$, update $\psi$}
    % }
% \end{algorithm}
% \end{minipage}%
% \begin{minipage}{.4\textwidth}
% \centering
% \begin{algorithm}[H] 
    % \caption{Distillation via \methodname (Ours)}
    % \label{alg:msg}
    %     \SetKwInOut{Input}{Input}
    %     \SetKwInOut{Output}{Output}
    %     \Input{pre-trained diffusion model $\epsilon_\theta : \sR^{d_1 \times \dots \times d_k} \to \sR^{d_1 \times \dots \times d_k}$, target parameters $\psi \in \sR^d$, condition $c$, mapping function $g(\psi) : \sR^d \to \sR^{d_1 \times \dots \times d_k}$, sample size $N$.\\
    %     }
    %     \Output{$\psi^{*}$}
        % \Function{ODESolver($x, \lambda$) (eq~\ref{eq:msd_guidance_2})}{
        %     $z_T \gets \N(0,I)$\\
        %     \For{$t = T-1,\hdots,0$}
        %     {
        %         $z_t \gets \epsilon_\theta(z_t, t, c) + \sigma_t (z_t - x) / \lambda^2$\\
        %     }
        %     return $z_0$
        % }
        % \Function{ODESolver($x, \lambda$, stable) (eq~\ref{eq:msd_guidance_2})} {
        %     $\{z^*_t\}^T_{t=0} \gets inversion(x)$\\
        %     $z_T \gets  z^{*}_T  +  (\epsilon - z^{*}_T) \expkernel $\\
        %     % $z_T \gets  \expkernel \epsilon +  (1 - \expkernel) z^{*}_T $\\
        %     \For{$t = T-1,\hdots,0$}
        %     {
        %         $z_T \gets z^{*}_t +  (\epsilon_\theta(z_t, t, c) - z^{*}_t) \expkernel $\\
        %         % $z_T \gets \expkernel \epsilon_\theta(z_t, t, c) + z^{*}_t (1 - \expkernel)$\\
        %     }
        %     return $z_0$\\
        % }
        % \Comment{initialize $\lambda$, set $\lambda_{min}$}
        % \For{$k = 1,\hdots,steps$}{
        %     $x^k \gets g(\psi)$ \\        
        %     \For{$i = 1,\hdots,N$}{
        %         $y_i \gets \text{ODESolver}(x^k, \lambda)$
        %     }
        %     $\nabla_{\psi} \mathcal{L}_{\methodname} \gets \frac{1}{N} \sum^N_i (y_i - x^k)$\\
            
        %     \Comment{Backpropagate $\nabla_{\psi} \mathcal{L}_{\methodname}$, update $\psi$}
        %     \Comment{Anneal $\lambda$}
        %     \If{$\lambda < \lambda_{min}$}{
        %         \Comment{terminate}
        %     }
        % }
% \end{algorithm}
% \end{minipage}
% \caption{
%     \VTtodo{add caption.}
% }
% \label{alg:comp}
% \end{figure}

% \Function{ODE\_Solver\_v2($x, \lambda$)}{
%     \Comment{Sample $t'$. Naive strategy: from a linearly decaying schedule.}
%     $z_{t'} \gets \epsilon_{t'} (\exp^{- \frac{\Delta t}{\lambda^2}}) + z^{*}_{t'} (1 - \exp^{- \frac{\Delta t}{\lambda^2}})$\\
%     \For{$t = t'-1,\hdots,0$}
%     {
%         $z_t \gets \textcolor{darkred}{\epsilon_\theta(z_t, t, \cdots)}$\\
%     }
%     return $z_0$
% }
% \Function{ODE\_Solver\_v3($x, \lambda$)}{
%     \Comment{Sample $t$.}
%     $z_{t} \gets \epsilon_{t} (\exp^{- \frac{\Delta t}{\lambda^2}}) + z^{*}_{t} (1 - \exp^{- \frac{\Delta t}{\lambda^2}})$\\
%     $z_{t-1} \gets \textcolor{darkred}{\epsilon_\theta(z_t, t, \cdots)}$\\
%     $z^{*}_0 \gets (z_{t-1} - \sqrt{1 - \alpha_{t-1}} \epsilon^{(t-1)}_\theta z_{t-1}) / \sqrt{\alpha_{t-1}}$\\
%     return $z_0$
% }
    % \Function{\methodname}{

\end{document}