%%%%%%%%%%%%%%%%%%%%%%%%%%%%%%%%%%%%%%%%%%%%%%%%%%%%%%%%%%%%%%%%%%%%%%%%%%
%%
%%	Author Submission Template for Transportation Science (TRSC)
%%	INFORMS, <informs@informs.org>
%%	Ver. 1.00, June 2024
%%
%%%%%%%%%%%%%%%%%%%%%%%%%%%%%%%%%%%%%%%%%%%%%%%%%%%%%%%%%%%%%%%%%%%%%%%%%%
%
% Use dblanonrev for Double Anonymous Review submission
% Use sglanonrev for Single Anonymous Review submission
% For example, submission to Operations Research, OPRE will have
% \documentclass[opre,dblanonrev]{informs4}

%%% TRUE SUBMISSION
% \documentclass[trsc,dblanonrev]{informs4}
% \usepackage{eqndefns-left} % For checking the display equation width and equation environment definitions %

%%% ARXIV VERSION
\documentclass[trsc,sglanonrev]{informs4_hide}

\RequirePackage{tgtermes}
\RequirePackage{newtxtext}
\RequirePackage{newtxmath}
\RequirePackage{multirow,multicol}
\RequirePackage{xspace,dsfont}
\RequirePackage{endnotes}
% \let\footnote=\endnote


%\OneAndAHalfSpacedXI
% \OneAndAHalfSpacedXII % Current default line spacing
\DoubleSpacedXI % Double-spacing for 11-point font (TSL)
% \DoubleSpacedXII

% Optional LaTeX Packages
\usepackage{algorithm}
\usepackage{algpseudocode}
\algrenewcommand\algorithmicrequire{\textbf{Input:}}
\algrenewcommand\algorithmicensure{\textbf{Output:}}
\usepackage{tikz}
%
\usepackage{xcolor}
\definecolor{dark1}{RGB}{157, 58, 103}
\definecolor{dark2}{RGB}{161, 67, 0}
\definecolor{dark3}{RGB}{115, 102, 0}
\definecolor{dark4}{RGB}{2, 120, 50}
\definecolor{dark5}{RGB}{0, 116, 122}
\definecolor{dark6}{RGB}{18, 100, 176}
\definecolor{dark7}{RGB}{116, 75, 163}
\definecolor{mid1}{RGB}{225, 119, 163}
\definecolor{mid2}{RGB}{228, 128, 77}
\definecolor{mid3}{RGB}{182, 158, 21}
\definecolor{mid4}{RGB}{87, 182, 109}
\definecolor{mid5}{RGB}{0, 181, 190}
\definecolor{mid6}{RGB}{88, 162, 242}
\definecolor{mid7}{RGB}{177, 135, 229}
\definecolor{light1}{RGB}{255, 187, 231}
\definecolor{light2}{RGB}{255, 198, 151}
\definecolor{light3}{RGB}{247, 223, 104}
\definecolor{light4}{RGB}{152, 248, 171}
\definecolor{light5}{RGB}{72, 249, 255}
\definecolor{light6}{RGB}{164, 219, 255}
\definecolor{light7}{RGB}{244, 192, 255}
\definecolor{lightyellow}{RGB}{255, 255, 204}

%% Comments
%%% Comments  will be removed if \Comments = 0 %%%
\newcount\Comments
\Comments = 0
\newcommand{\kibitz}[2]{\ifnum\Comments=1{\color{#1}{#2}}\fi}

\newcommand{\mr}[1]{\kibitz{mid1}{[Matias: #1]}}
\newcommand{\hma}[1]{\kibitz{mid4}{[HMa: #1]}}


\newcount\Drop  
\Drop = 0
\newcommand{\todrop}[2]{\ifnum\Drop=1{\color{#1}{#2}}\fi}
\newcommand{\drop}[1]{\todrop{mid5}{[Drop: #1]}}


%%%% Highlights that becomes black if \highlighttrue is commented out 
\newif\ifhighlight 
\highlighttrue % comment out this line to disable highlights

\newcommand{\mradd}[1]{{\ifhighlight\color{dark1}\fi#1}}
\newcommand{\hmadd}[1]{{\ifhighlight\color{dark4}\fi#1}}


\newif\iftodo
% \todotrue % comment out this line to hide the TODOs 

\newcommand{\todo}[1]{\iftodo{\color{mid1!50!gray}{[TODO: #1]}}\fi}

\usepackage{color}              % Need the color package
% \usepackage[suppress]{color-edits}
\usepackage{color-edits}
\addauthor{Will}{blue}
\addauthor{Hma}{dark4}
%
\usepackage[colorlinks,
	citecolor = dark6,
	urlcolor = black,
	linkcolor = dark6,
    hypertexnames=false]{hyperref}
\usepackage[nameinlink]{cleveref}
\crefname{subsection}{subsection}{subsections}
\usepackage[short]{optidef}
\usepackage[font=scriptsize]{subcaption}
\newcommand{\figWidth}{0.45}

% Private macros here (check that there is no clash with the style)
%%%%%%%%%%%%%%%%%%%%%%%%%% NOTATION %%%%%%%%%%%%%%%%%%%%%%%%

% bold lowercase letters, for vectors
\newcommand\vzero{{\bm 0}}
\newcommand\vzeron{{\bm 0}_n}
\newcommand\vone{{\bm 1}}
\newcommand\vonen{{\bm 1}_{n}}
% 
\newcommand\vpi{{\bm\pi}}
\newcommand\vphi{{\bm\phi}}
\newcommand\veta{{\bm\eta}}
\newcommand\vmu{{\bm\mu}}
% \newcommand\vtheta{{\bm\theta}}
\newcommand\vtheta{{\BFtheta}}
\newcommand\vdelta{{\bm\delta}}
\newcommand\va{{\bm a}}
\newcommand\vb{{\bm b}}
\newcommand\vc{{\bm c}}
\newcommand\vd{{\bm d}}
\newcommand\ve{{\bm e}}
\newcommand\vf{{\bm f}}
\newcommand\vg{{\bm g}}
\newcommand\vh{{\bm h}}
\newcommand\vi{{\bm i}}
\newcommand\vj{{\bm j}}
\newcommand\vk{{\bm k}}
\newcommand\vl{{\bm l}}
\newcommand\vm{{\bm m}}
\newcommand\vn{{\bm n}}
\newcommand\vo{{\bm o}}
\newcommand\vp{{\bm p}}
\newcommand\vq{{\bm q}}
\newcommand\vr{{\bm r}}
% \newcommand\vs{{\bm s}}
\newcommand\vt{{\bm t}}
\newcommand\vu{{\bm u}}
\def\vv{{\bm v}}
\newcommand\vw{{\bm w}}
\newcommand\vx{{\bm x}}
\newcommand\vy{{\bm y}}
\newcommand\vz{{\bm z}}


% Number sets
\newcommand{\C}{\mathbb{C}}
\newcommand{\R}{\mathbb{R}}
\newcommand{\Q}{\mathbb{Q}}
\newcommand{\N}{\mathbb{N}}
\newcommand{\Z}{\mathbb{Z}}

% Personal definitions
\renewcommand{\P}{\mathbb{P}}
\newcommand{\geom}[1]{\text{Geom}(#1)}
\newcommand{\E}{\mathbb{E}}
\newcommand{\var}{\text{Var}}
\newcommand{\normal}{\mathcal{N}}
\newcommand{\indicator}{\mathds{1}}
%\newcommand{\converges}[2][n \to \infty]{\xrightarrow[#1]{#2}}
\newcommand{\converges}[2][]{\xrightarrow[#1]{#2}}
\newcommand{\equalin}[1]{\stackrel{#1}{=}}
\newcommand{\loss}[1]{\depth({#1})}
% \DeclareMathOperator*{\argmin}{arg\,min}
% \DeclareMathOperator*{\argmax}{arg\,max}
\newcommand{\infsum}[1]{\sum_{#1\ge 1}}
\newcommand{\finsum}[2]{\sum_{#1=1}^{#2}}
\newcommand{\abs}[1]{\left|#1\right|}
\newcommand{\norm}[2][]{\left\|#2\right\|_{#1}}
\newcommand{\innerprod}[2][]{\left\langle #2 \right\rangle_{#1}}
\newcommand{\ceil}[1]{\left\lceil #1 \right\rceil}
\newcommand{\floor}[1]{\left\lfloor #1 \right\rfloor}
% \newcommand{\norm}[1]{\left\| #1 \right\|}
\newcommand{\indicatorof}[1]{\indicator\left\{#1\right\}}
\newcommand{\eps}{\varepsilon}

% Highlight a newly defined term
\newcommand{\newterm}[1]{\textit{#1}}
\newcommand{\red}[1]{{\leavevmode\color{red}#1}}
\newcommand{\blue}[1]{{\leavevmode\color{blue}#1}}
\newcommand{\nblue}[1]{{\leavevmode\color{nblue}#1}}
\newcommand{\violet}[1]{{\leavevmode\color{violet}#1}}
\newcommand{\purple}[1]{{\leavevmode\color{purple}#1}}
\newcommand{\ngreen}[1]{{\leavevmode\color{ngreen}#1}}

% macros for dynamic pooling
\newcommand{\Njob}{n}
\newcommand{\type}{\theta}
\newcommand{\typespace}{\Theta}
\newcommand{\instance}{\vtheta}
\newcommand{\sojourn}{d}
\newcommand{\potential}{p}
\newcommand{\ALG}{\mathsf{ALG}}
\newcommand{\OPT}{\mathsf{OPT}}
\newcommand{\OFF}{\mathsf{OFF}}
\newcommand{\regret}{\mathsf{Reg}}
\newcommand{\gap}{\mathsf{GAP}}
\newcommand{\gre}{\mathsf{GRE}}
\newcommand{\PB}{\mathsf{PB}}
\newcommand{\batching}{\mathsf{BAT}}
\newcommand{\rbatching}{\mathsf{R-BAT}}
\newcommand{\pot}{\mathsf{POT}}
\newcommand{\Potloss}{\mathsf{Loss}}
\newcommand{\LP}{\mathsf{LP}}
\newcommand{\DualLP}{\mathsf{DLP}}
\newcommand{\matchset}{\mathcal{M}}
\newcommand{\buffer}{A}
\newcommand{\indexf}{q}
\newcommand{\noutput}{m}
\newcommand{\order}[1]{(#1)}
\newcommand{\depth}{\ell}
\newcommand{\matchof}{m}
\newcommand{\ltype}{\alpha}
\newcommand{\rtype}{\beta}
\newcommand{\jobset}{J}
\newcommand{\width}{w}
\newcommand{\reward}{r}
\newcommand{\supI}{\mathrm{A}}
\newcommand{\supII}{\mathrm{B}}
\newcommand{\supIII}{\mathrm{C}}
\newcommand{\basecase}{\mathrm{BC}}
% \newcommand{\xorigin}{\vx_{\mathrm{ori}}}
\newcommand{\xorigin}{O}
% \newcommand{\xdestination}{\vx_{\mathrm{dest}}}
\newcommand{\xdestination}{D}
\newcommand{\yorigin}{y_{\mathrm{origin}}}
\newcommand{\ydestination}{y_{\mathrm{destination}}}
\newcommand{\marginalgain}{\mathsf{MG}}
\newcommand{\marginalloss}{\mathsf{ML}}
\newcommand{\randmarginalgain}{\mathsf{RMG}}
\newcommand{\randmarginalloss}{\mathsf{RML}}
\newcommand{\spacing}{S}
\newcommand{\dual}{\mathsf{HD}}
\newcommand{\discmap}{\varphi}
\newcommand{\averagedual}{\mathsf{AD}}
% \newcommand{\unif}{\text{Unif}}

% Natbib setup for author-number style
\usepackage{natbib}
 \bibpunct[, ]{(}{)}{,}{a}{}{,}%
 \def\bibfont{\small}%
 \def\bibsep{\smallskipamount}%
 \def\bibhang{24pt}%
 \def\newblock{\ }%
 \def\BIBand{and}%


%% Setup of the equation numbering system. Outcomment only one.
%% Preferred default is the first option.
\EquationsNumberedThrough    % Default: (1), (2), ...
%\EquationsNumberedBySection % (1.1), (1.2), ...

%% Setup of theorem styles. Outcomment only one.
%% Preferred default is the first option.
\TheoremsNumberedThrough     % Preferred (Theorem 1, Lemma 1, Theorem 2)
%\TheoremsNumberedByChapter  % (Theorem 1.1, Lema 1.1, Theorem 1.2)
\ECRepeatTheorems  %  

% For new submissions, leave this number blank.
% For revisions, input the manuscript number assigned by the on-line
% system along with a suffix ".Rx" where x is the revision number.
\MANUSCRIPTNO{TRSC-0001-2024.00}

%%%%%%%%%%%%%%%%
\begin{document}
%%%%%%%%%%%%%%%%

% Outcomment only when entries are known. Otherwise leave as is and
%   default values will be used.
%\setcounter{page}{1}
%\VOLUME{00}%
%\NO{0}%
%\MONTH{Xxxxx}% (month or a similar seasonal id)
%\YEAR{0000}% e.g., 2005
%\FIRSTPAGE{000}%
%\LASTPAGE{000}%
%\SHORTYEAR{00}% shortened year (two-digit)
%\ISSUE{0000} %
%\LONGFIRSTPAGE{0001} %
%\DOI{10.1287/xxxx.0000.0000}%

% Author's names for the running heads
% Sample depending on the number of authors;
% \RUNAUTHOR{Jones}
% \RUNAUTHOR{Jones and Wilson}
% \RUNAUTHOR{Jones, Miller, and Wilson}
% \RUNAUTHOR{Jones et al.} % for four or more authors
% Enter authors following the given pattern:
%\RUNAUTHOR{}
% \RUNAUTHOR{Smith and Johnson}

\RUNAUTHOR{Ma, Ma, and Romero}





% Title or shortened title suitable for running heads. Sample:
% \RUNTITLE{Predictive Maintenance in Manufacturing}
% Enter the (shortened) title:
% \RUNTITLE{Optimal Resource Allocation in Humanitarian Logistics}
\RUNTITLE{Dynamic Delivery Pooling}

% Full title. Sample:
% \TITLE{Optimal Resource Allocation in Humanitarian Logistics: A Stochastic Programming Approach}
% Enter the full title:
% \TITLE{Optimal Resource Allocation in Humanitarian Logistics: A Stochastic Programming Approach}
\TITLE{Potential-Based Greedy Matching for Dynamic Delivery Pooling} 

% Block of authors and their affiliations starts here:
% NOTE: Authors with same affiliation, if the order of authors allows,
%   should be entered in ONE field, separated by a comma.
%   \EMAIL field can be repeated if more than one author
\ARTICLEAUTHORS{%
%\AUTHOR{John Doe,\textsuperscript{a} Jane Smith,\textsuperscript{b}}
%\AFF{\textsuperscript{a}Department of Industrial Engineering, University of XYZ, \EMAIL{john.doe@xyz.edu; \textsuperscript{b}Department of Computer Science, University of ABC, \EMAIL{jane.smith@abc.edu}} 
% \AUTHOR{John Smith}
% \AFF{Department of Logistics,
% University of XYZ, \EMAIL{john.smith@xyz.edu}}

% \AUTHOR{Emily Johnson}
% \AFF{Department of Logistics,
% University of ABC, \EMAIL{emily.johnson@abc.edu}}

% Enter all authors
\AUTHOR{Hongyao Ma}
\AFF{Graduate School of Business, Columbia University, New York, NY 10027, \EMAIL{hongyao.ma@columbia.edu}}
\AUTHOR{Will Ma}
\AFF{Graduate School of Business, Columbia University, New York, NY 10027, \EMAIL{wm2428@gsb.columbia.edu}}
\AUTHOR{Matias Romero}
\AFF{Graduate School of Business, Columbia University, New York, NY 10027, \EMAIL{mer2262@gsb.columbia.edu}}
} % end of the block

\ABSTRACT{%
We study the problem of pooling together delivery orders into a single trip, a strategy widely adopted by platforms to reduce total travel distance.
% % 
Similar to other dynamic matching settings, the pooling decisions involve a trade-off between immediate reward and holding jobs for potentially better opportunities in the future. 
%
In this paper, we introduce a new heuristic dubbed potential-based greedy ($\PB$), which aims to keep longer-distance jobs in the system, as they have higher potential reward (distance savings) from being pooled with other jobs in the future. 
%
This algorithm is simple in that it depends solely on the topology of the space, and does not rely on forecasts or partial information about future demand arrivals.
%
We prove that $\PB$ significantly improves upon a naive greedy approach in terms of worst-case performance on the line. Moreover, we conduct extensive numerical experiments using both synthetic and real-world order-level data from the Meituan platform. 
%
Our simulations show that $\PB$ consistently outperforms not only the naive greedy heuristic but a number of benchmark algorithms, including (i) batching-based heuristics that are widely used in practice, and (ii) forecast-aware heuristics that are given the correct probability distributions (in synthetic data) or a best-effort forecast (in real data).
%
We attribute the surprising unbeatability of $\PB$ to the fact that it is specialized for rewards defined by distance saved in delivery pooling.
}%

% % \FUNDING{This research was supported by [grant number, funding agency].}

% %Supplemental Material:
% %Data Ethics & Reproducibility Note:

% % Sample
% %\KEYWORDS{Stochastic programming, Decision support,Uncertainty, Disaster response, Optimization}

% % Fill in data. If unknown, outcomment the field
\KEYWORDS{On-demand delivery, Platform operations, Online algorithms, Dynamic matching} 

% %\HISTORY{Received: Month DD, YYYY; Accepted: Month DD, YYYY; Published Online: Month DD, YYYY}

\maketitle
%%%%%%%%%%%%%%%%%%%%%%%%%%%%%%%%%%%%%%%%%%%%%%%%%%%%%%%%%%%%%%%%%%%%%%

\section{Introduction} \label{sec:intro}

% 
On-demand delivery platforms have become an integral part of modern life, transforming how consumers search for, purchase, and receive goods from restaurants and retailers.
%
Collectively, these platforms serve more than $1.5$ billion users worldwide, contributing to a global market valued at over \$250 billion \citep{statista2024}.
%
Growth has continued at a rapid pace--- major companies such as DoorDash in the United States and Meituan in China reported approximately 25\% year-over-year growth in transaction volume in 2023 \citep{curry2024doordash,scmp2024meituan}. 
%
Managing this ever-expanding stream of orders poses significant operational challenges, particularly as consumers demand increasingly faster deliveries, and fierce competition compels platforms to continuously improve both operational efficiency and cost-effectiveness.

%
One strategy for improving efficiency and reducing labor costs is to \emph{pool} into a single trip multiple orders from the same or nearby restaurants.
% 
This is referred to as stacked, grouped, or batched orders, and is advertised to drivers as opportunities for increasing earnings and efficiency~\citep{doordash2024batched}.
% 
The strategy has been generally successful and very widely adopted~\citep{deliveroo2022,uberEatsMultiple,grabGroupedJobs}.
%
For example, data from Meituan, made public by the 2024 INFORMS TSL Data-Driven Research Challenge, show that more than 85\% of orders are pooled, rising to 95\% during peak hours and consistently remaining above 50\% at all other times (see \Cref{fig:meituan_pooled_orders_per_hour}).
%

\begin{figure}[hpbt]
    \centering
    \includegraphics[width=0.9\linewidth]{Simulation_Results/Meituan/City/meituan_fraction_of_pooled_orders_how.png}
    % 
    \caption{Fraction of pooled orders by hour-of-week in Meituan data. 
    The dataset includes $8$ days of order-level data from one city (see \Cref{sec:sim_meituan} for more details). The gray shade indicates peak lunch hours (10:30am-1:30pm), while the green shade indicates peak dinner hours (5pm-8pm). 
    }
    \label{fig:meituan_pooled_orders_per_hour}
\end{figure}

The delivery pooling approach introduces a highly complex decision-making problem, as orders arrive dynamically to the market, and need to be dispatched within a few minutes --- orders in the Meituan data are offered to delivery drivers an average of around five minutes after being placed, often before meal preparation is complete (see \Cref{fig:meituan_order_to_first_dispatch_how}).
% 
%
During this limited window of each order, the platform must determine which (if any) of the available orders to pool with, carefully weighing immediate efficiency gains against the uncertain, differential benefits of holding each candidate for future pooling opportunities.
%
Consequently, there remains a pressing need for more efficient and robust solution methods to inform the platform's delivery pooling decisions.


%
Similar problems have been studied in the context of various marketplaces such as ride-sharing platforms and kidney exchange programs, in a large body of work known as dynamic matching.
%
However, we argue that the reward structure for delivery pooling is fundamentally different. 
% 
First, many problems studied in the literature involves connecting two sides of a market, e.g. ridesharing platforms matching drivers to riders. 
% 
In such bipartite settings, the presence of a large number of agents of identical or similar types (e.g., many riders requesting trips originating from the same area) typically leads to less efficient outcomes.
% 
While the kidney exchange problem is not bipartite, long queues of ``hard-to-match'' patient-donor pairs can still form, when many pairs share the same combination of incompatible tissue types and blood types.

In stark contrast, it is unlikely for long queues to build up for delivery pooling, especially in dense\footnote{
Indeed, today's major platforms observe extremely large volumes. Meituan, for example, processes over 12 thousand 
orders per hour in one market during peak lunch periods (see \Cref{fig:meituan_orders_per_hour}).} markets, in that given enough orders, two of them will have closely situated origins and destinations, resulting in a good match.
%
In fact, many customers requesting deliveries originating from the same location is actually \textit{desirable}, since this reduces travel distance and parking stops.
% 
We find that delivery orders tend to indeed be highly concentrated in space, with most requests originating from popular restaurants in busy city centers and ending in residential neighborhoods (see \Cref{appx:meituan_spatial_distribution}). 
%
We emphasize that we are not studying the problem of matching (pooled) orders to couriers, but delivery pooling is a relevant first step regardless, as pooling orders efficiently before matching with couriers will effectively increase the amount of courier capacity.

In this work, we study delivery pooling and exploit its distinct \emph{reward topology}, that it is highly desirable to match two jobs of identical or very similar types, as described above.
%
We develop a simple "potential-based" greedy algorithm, which relies solely on the reward topology to determine the opportunity cost of dispatching each job.
% 
The intuition behind it is elementary --- long-distance jobs have higher \textit{potential} for cost savings when pooled with other jobs, and hence the online algorithm should generally prefer keeping them in the system, dispatching shorter deliveries first when it has a choice.

\subsection{Model Description}

Different models of arrivals and departures have been proposed in the dynamic matching literature (see \Cref{sec:dynMatchModels}), and we believe our insight about potential is relevant across all of them.
However, in this paper we focus on a single model well-suited for the delivery pooling application.

%
To elaborate, we consider a dynamic non-bipartite matching problem, with a total of $\Njob$ jobs arriving sequentially to be matched.
%
Jobs are characterized by (potentially infinite) types $\theta\in\Theta$ representing features (e.g. locations of origin and destination) that determine the reward for the platform.
%
Given the motivating application to pooled deliveries, we will always model jobs' types as belonging to some metric space.
%
Following \citet{ashlagi2019edge}, we assume that an unmatched job must be dispatched after $\sojourn$ new arrivals, which we interpret as a known \textit{internal} deadline imposed by the platform to incentivize timely service.
%
The platform is allowed to make a last-moment matching decision before the job leaves, termed as the job becoming \newterm{critical}.
%
At this point, the platform must decide either to match the critical job to another available one, collecting reward $r(\theta,\theta')$ where $\theta,\theta'$ are the types of the matched jobs and $r$ is a known reward function, or to dispatch the critical job on its own for zero reward.
%
The objective is to maximize the total reward collected from matching the $\Njob$ jobs.
%
We believe this to be an appropriate model for delivery platforms (cf. \Cref{sec:dynMatchModels}) because deadlines are known upon arrival and sudden departures (order cancellations) are rare.

Dynamic matching models can also be studied under different forms of information about the future arrivals.
%
Some papers \citep[e.g.][]{kerimov2024dynamic,aouad2020dynamic,eom2023batching,wei2023constant} develop sophisticated algorithms to leverage stochastic information, which is often necessary to derive theoretical guarantees under general matching rewards.
%
In contrast, our heuristic does not require any knowledge of future arrivals, and our theoretical results hold in an "adversarial" setting where no stochastic assumptions are made.
%
In our experiments, we consider arrivals generated both from stochastic distributions and real-world data, and find that our heuristic can outperform even algorithms that are given the correct stochastic distributions, under our specific reward topology.

\subsection{Main Contributions}

\paragraph{Notion of potential.}
% 
As mentioned above, we design a simple greedy-like algorithm that is based purely on topology and reward structure, which we term potential-based greedy ($\PB$). More precisely, $\PB$ defines the \newterm{potential} of a job type $\theta$ to be $p(\theta)=\sup_{\theta'\in\Theta} r(\theta,\theta')/2$, measuring the highest-possible reward obtainable from matching type $\theta$.  The potential acts as an opportunity cost, and the relevance of this notion arises in settings where the potential is heterogeneous across jobs. To illustrate, let $\Theta=[0,1]$ and $r(\theta,\theta')=\min\{\theta,\theta'\}$, a reward function used for delivery pooling as we will justify in \Cref{sec:model}.  Under this reward function, a job type (which is a real number) can never be matched for reward greater than its real value, which means that jobs with higher real value have greater potential.  Our $\PB$ algorithm matches a critical job (with type $\theta$) to the available job (with type $\theta'$) that maximizes
\begin{align} \label{eqn:potentialIntro}
r(\theta,\theta')-p(\theta')=\min\{\theta,\theta'\}-\frac12 \theta',
\end{align}
being dissuaded to use up job types $\theta'$ with high real values. Notably, this definition does not require any forecast or partial information of future arrivals, but rather assumes full knowledge of the universe of possible job types.

\paragraph{Theoretical results.}
Our theoretical results assume $\Theta=[0,1]$.  We compare $\PB$ to the naive greedy algorithm $\gre$, which selects $\theta'$ to maximize $r(\theta,\theta')$, instead of $r(\theta,\theta')-p(\theta')$ as in~\eqref{eqn:potentialIntro}.  We first consider an offline setting ($d=\infty$), showing that under reward function $r(\theta,\theta')=\min\{\theta,\theta'\}$, our algorithm $\PB$ achieves regret $O(\log n)$, whereas $\gre$ suffers regret $\Omega(n)$ compared to the optimal matching.
Our analysis of $\PB$ is tight, i.e.\ it has $\Theta(\log n)$ regret.
Building upon the offline analysis, we next consider the online setting ($d< n$), showing that $\PB$ has regret $\Theta(\frac nd\log d)$, which improves as $d$ increases. This can be interpreted as there being $\frac nd$ "batches" in the online setting, and our algorithm achieving a regret of $O(\log d)$ per batch.  Alternatively, it can be interpreted as the \newterm{regret per job} of our algorithm being $O(\frac d{\log d})$. By contrast, we show that the naive greedy algorithm $\gre$ suffers a total regret of $\Omega(n)$, regardless of $d$.

For comparison, we also analyze two other reward topologies, still assuming $\Theta=[0,1]$.
\begin{enumerate}
\item We consider the classical min-cost matching setting \citep{reingold1981greedy} where the goal is to minimize total match distance between points on a line, represented in our model by the reward function $r(\theta,\theta')=1-|\theta-\theta'|$.
%
For this reward function, both $\PB$ and $\gre$ have regret $\Theta(\log n)$; in fact, they are the same algorithm because all job types have the same potential. Like before, this translates into both algorithms having regret $\Theta(\frac nd \log d)$ in the online setting.
\item We consider reward function $r(\theta,\theta')=|\theta-\theta'|$, representing an opposite setting in which it is worst to match two jobs of the same type.  For this reward function, we show that any index-based matching policy must suffer regret $\Omega(n)$ in the offline setting, and that both $\PB$ and $\gre$ suffer regret $\Omega(n)$ (irrespective of $d$) in the online setting.
\end{enumerate}

Our theoretical results are summarized in \Cref{table:results}.
As our model is a special case of \citet{ashlagi2019edge}, their 1/4-competitive randomized online edge-weighted matching algorithm can be applied, which essentially translates in our setting to an $O(n)$ upper bound for regret under any definition of reward.
Our results show that it is possible to do much better ($O(\log n)$ instead of $O(n)$) for specific reward topologies, using a completely different algorithm and analysis.
We now outline how to prove our two main technical results, \Cref{thm:potential_log_upper_bound,thm: dynamic}.
%
\begin{table}[!t]
\centering
\begin{tabular}{|c|c|c|c|}
\hline
\updown $\theta,\theta'\in[0,1]$ & $r(\theta,\theta')=\min\{\theta,\theta'\}$ & $r(\theta,\theta')=1-|\theta-\theta'|$ & $r(\theta,\theta')=|\theta-\theta'|$ \\
\hline
\up\multirow{4}{*}{Regret of $\PB$} & Offline: $\Theta(\log n)$ & & \\
\down & (\Cref{thm:potential_log_upper_bound}, \Cref{prop:loglowerboundOffline}) & & \\
\up & Online: $\Theta(\frac nd \log d)$ & Offline: $\Theta(\log n)$ & Offline: $\Omega(n)$ \\
\down & (\Cref{thm: dynamic}, \Cref{prop:loglowerboundOnline}) & Online: $\Theta(\frac nd \log d)$ & Online: $\Omega(n)$ \\
\cline{1-2}
\up\multirow{4}{*}{Regret of $\gre$} & Offline: $\Omega(n)$ & (\Cref{sec:reward2}) & (\Cref{sec: reward 3}) \\
\down & (\Cref{prop:greedy_linear_lower_bound}) & & \\
\up & Online: $\Omega(n)$ & & \\
\down & (\Cref{prop:greedy_linear_lower_bound_dynamic}) & & \\
\hline
\end{tabular}
\caption{
Summary of theoretical results under different reward functions $r:[0,1]^2\to \R$.  The total number of jobs is denoted by $n$, and the batch size in the online setting is approximately $d$.
}
\label{table:results}
\end{table}
\paragraph{Proof techniques.}
%
We establish an upper bound on the regret of $\PB$ by comparing its performance to the sum of the potential of all jobs.
%
Under reward function $\reward(\type,\type')=\min\{\type,\type'\}$, the key driver of regret is the sum of the distances between jobs matched by $\PB$.
%
In \Cref{thm:potential_log_upper_bound}, we study this quantity by analyzing the intervals induced by matched jobs in the offline setting ($\sojourn=\infty$).
%
We show that the intervals formed by the matching output of $\PB$ constitute a \emph{laminar set family}; that is, every two intervals are either disjoint or one fully contains the other.
%
We then prove that more deeply nested intervals must be exponentially smaller in size. Finally, we use an LP to show that the sum of interval lengths remains bounded by a logarithmic function of the number of intervals.
%

In \Cref{thm: dynamic}, we partition the set of jobs in the online setting into roughly $\frac{\Njob}{d}$ "batches" and analyze the sum of the distances between matched jobs within a single batch.
%
Our main result is to show that we can use \Cref{thm:potential_log_upper_bound} to derive an upper bound for an arbitrary batch.
%
However, a direct application of the theorem on the offline instance defined by the batch would not yield a valid upper bound, because the resulting matching of $\PB$ in the offline setting could be inconsistent with its online matching decisions.
%
To overcome this challenge, we carefully construct a modified offline instance that allows for the online and offline decisions to be coupled, while ensuring that the total matching distance did not go down.  This allows us to upper-bound the regret per batch by $O(\log d)$, for a total regret of $O(\frac nd \log d)$.


\paragraph{Simulations on synthetic data.} We test $\PB$ on random instances in the setting of our theoretical results, except that job types are drawn uniformly at random from $[0,1]$ (instead of adversarial). 
%
We benchmark its performance against $\gre$, as well as more sophisticated algorithms, including (i) batching-based heuristics that are highly relevant both in theory and practice, and (ii) forecast-aware heuristics that use historical data to compute shadow prices.
%
Our extensive simulation results show that $\PB$ consistently outperforms all benchmarks starting from relatively low market densities, achieving over 95\% of the hindsight optimal solution that has full knowledge of arrivals.
%
This result is robust to different distributions of job types, and also two-dimensional locations.


\paragraph{Simulations on real data.}

Finally, we test the practical applicability of $\PB$ via extensive numerical experiments using order-level data from the Meituan platform, made available from the 2024 INFORMS TSL Data-Driven Research Challenge.\footnote{\url{https://connect.informs.org/tsl/tslresources/datachallenge}, accessed January 15, 2025.}
%
The key information we extract from this dataset is the exact timestamps for the creation of each request, and the geographic coordinates of pick-up and drop-off locations.
%
These aspects differ from our theoretical setting in that (i) requests may not be available to be pooled for a fixed number of new arrivals before being dispatched, and (ii) locations are two-dimensional with delivery orders having heterogeneous origins.
%
To address (i), we assume that the platform sets a fixed time window after which an order becomes critical, corresponding to each job being able to wait for at most a fixed sojourn time before being dispatched.
%
For (ii), we extend our definition of reward (that captures the travel distance saved) to two-dimensional heterogeneous origins.
%
Under this new reward definition, the main insight from our theoretical model remains true: longer deliveries have higher potential reward from being pooled with other jobs in the future, and thus $\PB$ aims to keep them in the system.
%

In contrast to our simulations with synthetic data, the real-life delivery locations may now be correlated and exhibit time-varying effects.
%
Regardless, we find that $\PB$ outperforms all tested heuristics (those without foreknowledge of the future), given that the platform is willing to wait up to one minute before dispatching each job, a fairly modest ask.
%
In this regime, $\PB$ achieves over 80\% of the maximum possible travel distance saved from pooling. $\PB$ also pools 10\% more jobs than the hindsight optimal matching (see \Cref{sec:match_rate}).
%


\paragraph{Explanation for the surprising unbeatability of $\PB$.}
Our potential-based greedy heuristic consistently performs at or near the best across a wide range of experimental setups, including both synthetic and real data.
%
This result is surprising to us, given that $\PB$ is a simple index-based rule that ignores forecast information.
%
To provide some explanation for this finding, we analyze a stylized setting in \Cref{sec: interpretation}, where we show that the "correct" shadow prices converge to our notion of potential as the market thickness increases.
%
That being said, we also end with a couple of caveats.
%
First, our findings about the effectiveness of $\PB$ are specific to our definition of matching reward for delivery pooling, based on travel distance saved, and may not extend to setting with different reward structures.
%
Second, while we carefully tuned the batching-based and dual-based heuristics that we compare against, it remains possible that more sophisticated algorithms, particularly those leveraging dynamic programming in stochastic settings, could outperform $\PB$.


\subsection{Further Related Work}

\subsubsection{Dynamic matching.} \label{sec:dynMatchModels}

Dynamic matching problems have received growing attention from different communities in economics, computer science, and operations research. We discuss different ways of modeling the trade-off between matching now vs.\ waiting for better matches, depending on the application that motivates the study. 


One possible model \citep{kerimov2024dynamic,kerimov2023optimality,wei2023constant} is to consider a setting with jobs that arrive stochastically in discrete time and remain in the market indefinitely, but use a notion of \newterm{all-time regret} that evaluates a matching policy, at every time period, against the best possible decisions until that moment, to disincentivize algorithms from trivially delaying until the end to make all matches.
%

A second possible model, motivated by the risk of cancellation in ride-sharing platforms, is to consider sudden departures modeled by jobs having heterogeneous \textit{sojourn times} representing the maximum time that they stay in the market \citep{aouad2020dynamic}. These sojourn times are unknown to the platform, and delaying too long risks many jobs being lost without a chance of being matched.
%
\citet{aouad2020dynamic} formulates an MDP with jobs that arrive stochastically in continuous time, and leave the system after an exponentially distributed sojourn time. They propose a policy that achieves a multiplicative factor of the hindsight optimal solution. 
%
Related work on the control of matching queues with abandonment includes \citet{collina2020dynamic}, \citet{castro2020matching}, \citet{wang2024demand}, \citet{kohlenberg2024cost}.

%
A third possible model \citep{huang2018match} also considers jobs that can leave the system at any period, but allows the platform to make a last-moment matching decision right before a job leaves, termed as the job becoming \newterm{critical}.  In this model, one can without loss assume that all matching decisions are made at times that jobs become critical.

Finally, the model we study also makes all decisions at times that jobs become critical, with the difference being that the sojourn times are known upon the arrival of a job, as studied in \citet{ashlagi2019edge, eom2023batching}. Under these assumptions, \citet{eom2023batching} assume stationary stochastic arrivals, while \citet{ashlagi2019edge} allow for arbitrary arrivals.
%
Our research focuses on deterministic greedy-like algorithms that can achieve good performance as the market thickness increases.
%
Moreover, our work differs from these papers in the description of the matching value. While they assume arbitrary matching rewards, we consider specific reward functions known to the platform in advance, and use this information to derive a simple greedy-like algorithm with good performance.
% 
This aligns with a stream of literature on online matching, that captures more specific features into the model to get stronger guarantees \citep[see][]{kanoria2021dynamic,chen2023feature,balkanski2023power}.


We should note that our paper also relates to a recent stream of work studying the effects of batching and delayed decisions in online matching \citep[e.g.][]{feng2024batching,xie2023benefits}, as well as works studying the relationship between market thickness and quality of online matches \citep[e.g.][]{ashlagi2021kidney,chen2021matchmaking}.


\subsubsection{Delivery operations.}

On-demand delivery operations have received special attention in the field of transportation and operations management. 
%
\citet{reyes2018meal} introduce the meal delivery routing problem, which falls in the class of dynamic vehicle routing problems (see e.g. \citet{psaraftis2016dynamic}); the authors develop heuristics to dynamically assign orders to vehicles. Similar efforts have been devoted to optimize detailed pooling and assignment strategies as customer orders arrive sequentially \citep{steever2019dynamic,ulmer2021restaurant}, mainly using approximate dynamic programming techniques.
%
These heuristics are shown to work well in extensive numerical experiments, but given the intricate nature of the model and techniques, it is very challenging to derive managerial insights or theoretical performance guarantees.
%
Closer to our research goal, \citet{chen2024courier} analyze the optimal dispatching policy on a stylized queueing model representing a disk service area centered at one restaurant. They show that delivering multiple orders per trip is beneficial when the service area is large.
%
\citet{cachon2023fast} study the interplay between the number of couriers and platform efficiency, assuming a one-dimensional geography with one single origin, which is also the primary model in our theoretical results.
%
A similar topology is studied in the game-theoretic model of \citet{keskin2024order} to capture the impact of delivery pooling on the interaction between the platform, customers, riders, and restaurants. 

\section{Preliminaries}
\label{sec:model}
% 
In this section, we introduce a dynamic non-bipartite matching model for the delivery pooling problem. A total of $\Njob$ jobs arrive to the platform sequentially.
% 
Each job $j \in [\Njob] = \{1, \ldots, \Njob\}$, indexed in the order of arrival, has a \emph{type} $\type_j \in \typespace$. 
% 
We adopt the criticality assumption from \citet{ashlagi2019edge}, in the sense that each job remains available to be matched for $\sojourn\ge 1$ new arrivals, after which it becomes \textit{critical}. The parameter $\sojourn$ can also be interpreted as the market density.
%
When a job becomes critical, the platform decides whether to (irrevocably) match it with another available job, in which case they are dispatched together (i.e. \emph{pooled})
and the platform collects a \emph{reward} given by a known function $\reward:\typespace^2\to\R$. If a critical job is not pooled with another job, it has to be dispatched by itself for zero reward.


\subsection{Reward Topology}
% 
Our modeling approach directly imposes structure on the type space, as well as the reward function that captures the benefits from pooling delivery orders together. Similar to previous numerical work on pooled trips in ride-sharing platforms \citep{eom2023batching, aouad2020dynamic}, we assume that the platform's goal is to reduce the total distance that needs to be traveled to complete all deliveries.
The reward of pooling two orders together is therefore the travel distance saved when they are delivered by the same driver in comparison to delivered separately. 
% 
Formally, we consider a \newterm{linear city model} where job types $\type \in \typespace = [0,1]$ represent destinations of the delivery orders, and assume that all orders need to be served from the origin 0 (similar to that analyzed in \citet{cachon2023fast})
% 
If a job of type $\type$ is dispatched by itself, the total travel distance from the origin 0 is exactly $\type$. If it is matched with another job of type $\type'$, they are pooled together on a single trip to the farthest destination $\max\{\type,\type'\}$. Thus, the distance saved by pooling is 
%
%
%
\begin{equation}
  \reward(\type,\type') = \type + \type' - \max\{\type,\type'\} = \min\{\type,\type'\}.
  \label{eq:defn_reward_A} 
\end{equation}

Although our main theoretical results leverage the structure of this reward topology, our proposed algorithm can be applied to other reward topologies. 
% 
In particular, we derive theoretical results for two other reward structures for $\typespace = [0,1]$. First, we consider 
% 
\begin{equation}
    \reward(\type,\type') = 1 - |\type - \type'|  \label{eq:defn_reward_B}
\end{equation}
% 
to capture the commonly-studied spatial matching setting~\citep[e.g.][]{kanoria2021dynamic,balkanski2023power}, where the reward is larger if the distance $|\theta-\theta'|$ is smaller. 
% 
We also consider 
\begin{equation}
    \reward(\type,\type') = |\type - \type'| \label{eq:defn_reward_C}
\end{equation}
% 
with the goal of matching types that are far away from each other, contrasting the other two reward functions.
% 
In addition, we perform numerical experiments for delivery pooling in two-dimensional (2D) space, where the reward function corresponds to the travel distance saved in the 2D setting.


\subsection{Benchmark Algorithms}

% 
Given market density $\sojourn$ and reward function $\reward$, if the platform had full information of the sequence of arrivals $\instance \in \typespace^\Njob$, the hindsight optimal $\OPT(\instance,\sojourn)$ can be computed by the integer program (IP) defined in \eqref{eq: OPT}. We denote the hindsight optimal matching solution as $\matchset_{\OPT} = \{(j,k):x_{j,k}^{\ast}=1\}$, where $x^\ast$ is an optimal solution of \eqref{eq: OPT}.
% 
\begin{maxi}
    {x}{ \sum_{j,k : j\neq k, |j-k|\le \sojourn} x_{jk} \reward(\type_j,\type_k)}
    {\label{eq: OPT}}{\OPT(\instance,\sojourn) =}
    \addConstraint{ \sum_{k:j\neq k} x_{jk}}{\le 1,}{j\in [\Njob]}
    \addConstraint{ x_{jk}}{\in\{0,1\},}{j,k\in [\Njob], j\neq k.}
\end{maxi}
%

An online matching algorithm operates over an instance $\instance\in\typespace^\Njob$ sequentially: when job $j\in [\Njob]$ becomes critical, the types of future arrivals $k > j + d$ are unknown, and any matching decision has to be made based on the currently available information.
% 
Given an algorithm $\ALG$, we denote by $\ALG(\instance,\sojourn)$ the total reward collected by an algorithm on such instance.
%
We analyze the performance of algorithms via \emph{regret}, as follows:
%
\[
    \regret_\ALG(\instance,\sojourn) = \OPT(\instance,\sojourn)-\ALG(\instance,\sojourn).
\]

We study a class of online matching algorithms that we call \newterm{index-based greedy matching algorithms}.
%
Each index-based greedy matching algorithm is specified by an \emph{index function} $\indexf:\typespace^2\to\R$, and only makes matching decisions when some job becomes critical (in our model, it is without loss of optimality to wait to match).
%
A general pseudocode is provided in \Cref{alg:dynamic}.
%
When a job $j \in [\Njob]$ becomes critical, the algorithm observes the set of available jobs in the system $A(j) \subseteq [\Njob]$ that have arrived but are not yet matched (this is the set $\buffer\setminus\{j\}$ in \Cref{alg:dynamic}), and chooses a match $\matchof(j) \in A(j)$ that maximizes the index function (even if negative), collecting a reward $\reward(\type_j,\type_{\matchof(j)})$. If there are multiple jobs that achieve the maximum, the algorithm breaks the tie arbitrarily.
%
The algorithm makes a set of matches $\matchset = \{ (j,m(j)) : j \in C \}$, where $C$ denotes the set of jobs that are matched when they become critical, and its total reward is
\[ \ALG(\instance,\sojourn) = \sum_{j\in C} \reward(\type_j,\type_{\matchof(j)}). \]
%
Note that $\matchof(j)$ is undefined if $j$ is either unmatched, or was not critical at the time it was matched.


\begin{algorithm}
\caption{Index-based Greedy Matching Algorithm}\label{alg:dynamic}
% 
\begin{algorithmic}[1]
\Require Instance $\instance$, density $\sojourn$, index function $\indexf$, reward function $\reward$
% 
\Ensure $C$, $m:C\to[n]$
% 
\State Initialize set of available jobs $\buffer=\emptyset$, and jobs that are critical when matched $C=\emptyset$
\For{job $t=1,\ldots,\Njob+\sojourn+1$}
    \If{$t\le\Njob$}
        \State $\buffer=\buffer\cup \{t\}$ \Comment{job $t$ arrives}
    \EndIf         
    \If {$t-d\in \buffer$} 
        \State $j=t-d$ \Comment{job $j=t-d$ becomes critical}
        \If{$\buffer \setminus\{j\} \neq\emptyset$}
            \State Choose $m(j) \in \argmax_{k \in \buffer \setminus\{j\} } \indexf(\type_{j}, \type_{k})$ \Comment{Ties are broken arbitrarily}
            \State $C\gets C\cup\{j\}$
            \State $\buffer \gets \buffer\setminus \{j,\matchof(j)\} $ \Comment{jobs $j,\matchof(j)$ are dispatched together}
        \Else
            \State $\buffer \gets \buffer\setminus \{j\} $ \Comment{job $j$ is dispatched by itself}
        \EndIf
    \EndIf
\EndFor
\State\Return $C, \{\matchof(j)\}_{j\in C}$
\end{algorithmic}
\end{algorithm}

As an example, the \newterm{naive greedy} algorithm ($\gre$) uses index function $\indexf_{\gre}(\type,\type') = \reward(\type,\type')$. When any job becomes critical, the algorithm chooses a match it with an available job to maximize the (instant) reward.
%
To illustrate the behavior of this algorithm under reward function $\reward(\type,\type') = \min\{\type,\type'\}$, we first make the following observation, that the algorithm always chooses to match each critical job with a higher type job when possible. 
%
\begin{remark}\label{deliverygreedy}
    When a job $j \in [\Njob]$ becomes critical, if the set $A_+(j) = \{k\in A(j):\type_k \ge \type_j\}$ is nonempty, then under the naive greedy algorithm $\gre$, we have $\matchof(j)\in A_+(j)$ and $\reward(\type_j,\type_{\matchof(j)}) = \type_j $. 
    % 
\end{remark}



\section{Potential-Based Greedy Algorithm}
% 
\label{sec:PB} 

We introduce in this section the potential-based greedy algorithm and prove that it substantially outperforms the naive greedy approach in terms of worst case regret.

The \newterm{potential-based greedy} algorithm ($\PB$) is an index-based greedy matching algorithm (as defined in \Cref{alg:dynamic}) whose index function is specified as
\begin{equation}
    \indexf_{\PB}(\type,\type') = \reward(\type, \type') - \potential(\type'),
\end{equation}
where $\potential(\type)$ is the \newterm{potential} of a job of type $\type$, formally defined as follows
%
\begin{equation}
    \potential(\type)=\frac{1}{2}\sup_{\type'\in\typespace} \reward(\type,\type'). \label{eq:defn_potential}
\end{equation} 
% 

This new notion of potential can be interpreted as an optimistic measure of the marginal value of holding on to a job that could be later matched with a job that maximizes instant reward.
%
Intuitively, this "ideal" matching outcome appears as the optimal matching solution when the density of the market is arbitrarily large. We provide a formal statement of this intuition in \Cref{sec: interpretation}.
%
Note that this definition of potential is quite simple in the sense that it relies on only in the knowledge of the reward topology: reward function and type space.
% 
In particular, for our topology of interest, $\reward(\type,\type')=\min\{\type,\type'\}$ on $\typespace=[0,1]$, the ideal scenario is achieved when two identical jobs are matched and the potential $\potential(\type)=\type/2$ is simply proportional to the length of a solo trip, capturing that longer deliveries have higher potential reward from being pooled with other jobs in the future.

We begin by giving a straightforward interpretation for this algorithm under the linear city model and our reward of interest, as we did for the naive greedy algorithm $\gre$ in \Cref{deliverygreedy}.

\begin{remark}\label{deliverypotential}
    Under the 1-dimensional type space $\Theta = [0,1]$ and the reward function $\reward(\type,\type')=\min\{\type,\type'\}$, 
    % 
    the potential-based greedy algorithm $\PB$ always matches each critical job to an available job that's the closest in space, since 
    % 
    \begin{align*}
        \matchof(j) 
        \in \argmax_{k\in A(j) } \indexf_\PB(\type_j,\type_k) 
        = \argmin_{k\in A(j) } |\type_j-\type_k|. 
    \end{align*}    % 
    This follows from the fact that $ |\type_j-\type_k| = \type_j + \type_k - 2\min\{\type_j, \type_k\} = \type_j - 2\indexf_\PB(\type_j,\type_k) $.
\end{remark}

In the rest of this section, we first assume $d = \infty$ and study the \textit{offline} performance of various index-based greedy matching algorithms under the reward function $\reward(\type,\type') = \min\{\type,\type'\}$.
% 
This will later help us analyze the algorithms' \emph{online} performances when $d<\infty$.
Results for the two alternative reward structures defined in \eqref{eq:defn_reward_B} and \eqref{eq:defn_reward_C} are reported in \Cref{sec:reward2} and  \Cref{sec: reward 3}, respectively.
%


\subsection{Offline Performance under Reward Function $\reward(\type,\type') = \min\{\type,\type'\}$}\label{sec: static}

Suppose $\sojourn=\infty$, meaning that all jobs are avilable to be matched when the first job becomes critical. We study the \emph{offline} performance of both the naive greedy algorithm $\gre$ and potential-based greedy algorithm. We show that the regret of $\gre$ grows linearly with the number of jobs, while the regret of $\PB$ is logarithmic.


\begin{proposition}[proof in \Cref{pf:greedy_linear_lower_bound}]\label{prop:greedy_linear_lower_bound}
Under reward function $\reward(\type,\type') = \min\{\type,\type'\}$, when the number of jobs $\Njob$ is divisible by 4, there exists an instance $\instance \in [0,1]^\Njob$ for which $\regret_\gre(\instance,\infty) \ge n/4$.
\end{proposition}
%

In particular, the \newterm{regret per job} of $\gre$, i.e.\ dividing the regret by $\Njob$, is constant in $\Njob$.
%
In contrast, we prove in the following theorem that $\PB$ performs substantially better. In fact, its regret per job gets better for larger market sizes $\Njob$.

\begin{theorem}
\label{thm:potential_log_upper_bound}
    Under reward function $\reward(\type,\type') = \min\{\type,\type'\}$, we have $\regret_\PB(\instance,\infty) \le 1 + \log_2(\Njob/2+1)/2$, for any $\Njob$, and any instance $\instance \in [0,1]^\Njob$.
\end{theorem}
%
\proof{Proof.}
Let $\instance \in \typespace^\Njob$, and $(C,\matchof(\cdot))$ be the output of $\PB$ on $\instance$ (generated as in \Cref{alg:dynamic}). In particular, 
$\PB(\instance,\infty) = \sum_{j \in C} \reward(\type_j,\type_{\matchof(j)})$.
On the other hand, since $\reward(\type,\type') \le \potential(\type) + \potential(\type')$ for all $\type,\type'
\in\typespace$, we have
\begin{align}
    \OPT(\instance,\infty) = \sum_{(j,k) \in \matchset_\OPT} \reward(\type_j,\type_k) \le \sum_{j=1}^\Njob \potential(\type_j) \le \sup_{\type\in[0,1]}\potential(\type) + \sum_{j\in C} \potential(\type_j) + \potential(\type_{\matchof(j)}),\nonumber
\end{align}
where the last inequality comes from the fact the $\PB$, and in fact any index-based matching algorithm, leaves at most one job unmatched (when $\Njob$ is odd).
%
Hence,
\begin{align}
\regret_\PB(\instance,\infty)&=
\OPT(\instance,\infty)-\PB(\instance,\infty) \\
&\le \sup_{\type\in \typespace}\potential(\type) + \sum_{j\in C} \left( \potential(\type_j) + \potential(\type_{\matchof(j)}) - \reward(\type_j,\type_{\matchof(j)})\right) \nonumber\\
&= \frac12 + \sum_{j \in C} \frac{\type_j}{2} + \frac{\type_{\matchof(j)}}{2} - \min\{\type_j,\type_{m(j)}\} \nonumber\\
&= \frac12 + \sum_{j \in C} \frac{|\type_j - \type_{\matchof(j)}|}{2}\label{eq: offline_regret_bound}. 
\end{align}
%
To bound the right-hand side, we first prove the following. For each $j\in C$, consider the interval $I_j=( \min\{\type_j,\type_{\matchof(j)}\},\max\{\type_j,\type_{\matchof(j)}\})\subseteq[0,1]$ and $\depth(j)=|\{k : I_j \subseteq I_k\}|$. We argue that
\begin{equation}\label{eq: distancebound}
    |\type_j - \type_{\matchof(j)}| \le 2^{1-\depth(j)}.
\end{equation}
%

First, note that $\{I_j\}_{j\in C}$ is a \newterm{laminar set family}: for every $j,k\in C$, the intersection of $I_j$ and $I_k$ is either empty, or equals $I_j$, or equals $I_k$. Indeed, without loss of generality, assume $j<k$. Then, when job $j$ becomes critical we have $\matchof(j),k,\matchof(k) \in A(j) $. Therefore, from \Cref{deliverypotential}, $\matchof(j)$ is the closest to $j$ and thus $k,\matchof(k)\notin I_j$, i.e.\ either $I_j \cap I_k = \emptyset$, or $I_j \cap I_k = I_j$.
%
Moreover, if $I_j\subsetneq I_k$ (i.e. $I_j\subsetneq I_k$ and $I_j\neq I_k$), then necessarily $j < k$, since otherwise $k$ becomes critical first and having $j,\matchof(j) \in A(k)$ closer in space, $\PB$ would not have chosen $\matchof(k)$.
%
Now, we can prove \eqref{eq: distancebound} by induction. If $\depth(j)=1=|\{j\}|$, clearly $|\type_j-\type_{\matchof(j)}|\le 1$. Suppose that the statement is true for $\depth(j)=\depth \ge 1$. If $\depth(j)=\depth+1$ then there exists $k$ such that $I_j\subsetneq I_k$ and $\depth(k)=\depth$. By the previous property, $j<k$ and since when job $j$ becomes critical we had $k,\matchof(k)\in A(j)$, it must be the case that
\begin{align*}
    |\type_j-\type_{\matchof(j)}| 
    &<\max\{ \min\{\type_j,\type_{m(j)}\} - \min\{\type_k,\type_{m(k)}\},\max\{\type_k,\type_{m(k)}\}-\max\{\type_j,\type_{m(j)}\}\}\\
    &\le \min\{\type_j,\type_{m(j)}\} - \min\{\type_k,\type_{m(k)}\} + \max\{\type_k,\type_{m(k)}\} - \max\{\type_j,\type_{m(j)}\} \\
    &= |\type_k-\type_{\matchof(k)}| - |\type_j-\type_{\matchof(j)}|.
\end{align*}
Thus, $|\type_j-\type_{\matchof(j)}|\le|\type_k-\type_{\matchof(k)}|/2\le 2^{1-(\depth+1)}$,
completing the induction.
%
    
Having established \eqref{eq: distancebound}, we use it to derive an upper bound for $\sum_{j\in C}|\type_j-\type_{\matchof(j)}|$. Note that since $\depth(j)\in \{1,\ldots,\floor{\Njob/2}\}$, we have
\begin{align} \label{eq: distanceboundbylayer}
\sum_{j \in C} |\type_j - \type_{\matchof(j)}| = \sum_{\depth = 1}^{\floor{\Njob/2}}\sum_{j\in C:\depth(j) = \depth} |\type_j - \type_{\matchof(j)}|.
\end{align}
%
For every $\depth$, we have $\sum_{j\in C:\depth(j) = \depth} |\type_j - \type_{\matchof(j)}|\le \min\{1,2^{1-\depth}|\{j:\depth(j)=\depth\}|\}$ by the laminar property and \eqref{eq: distancebound}.
%
Let $z_\depth$ denote $2^{1-\depth}|\{j:\depth(j)=\depth\}|$, where we note that 
\begin{align*}\label{eq: numofintervalsbound}
    \sum_{\depth=1}^{\floor{\Njob/2}} 2^{\depth-1}z_\depth = \sum_{\depth=1}^{\floor{\Njob/2}} |\{j:\depth(j)=\depth\}| = |C| \le \Njob/2.
\end{align*}
We can then use the following LP to upper-bound the value of~\eqref{eq: distanceboundbylayer} under an adversarial choice $\{z_\depth:\depth=1,\ldots,\lfloor n/2\rfloor\}$:
\begin{maxi}
    {z}{ \sum_{\depth=1}^{\floor{\Njob/2}} z_\depth }
    {\label{eq: knapsack}}{}
    \addConstraint{ \sum_{\depth=1}^{\floor{\Njob/2}} 2^{\depth-1}z_\depth }{\le \Njob/2}
    \addConstraint{0\le z_\depth }{\le 1,\quad }{ \depth = 1,\ldots,\floor{\Njob/2}.}
\end{maxi}
This is a fractional knapsack problem, whose optimal value is at most $\log_2({\Njob/2+1})$, and thus
\begin{equation}\label{dist_bound}
    \sum_{j \in C} |\type_j - \type_{\matchof(j)}| \le \log_2(\Njob/2+1).
\end{equation}
%
Substituting back into \eqref{eq: offline_regret_bound} yields $\regret_\PB(\instance,\infty) \le 1/2 + \log_2(\Njob/2+1)/2$, completing the proof.
\Halmos\endproof

The following result shows that this analysis of $\PB$ is tight, up to constants.

\begin{proposition}[proof in \Cref{pf:loglowerboundOffline}]
\label{prop:loglowerboundOffline}
Under reward function $\reward(\type,\type') = \min\{\type,\type'\}$, when the number of jobs is $\Njob=2^{k+3} - 4$ for some integer $k\ge 0$, there exists an instance $\instance \in [0,1]^\Njob$ for which $\regret_\PB(\instance,\infty) \ge (\log_2(n+4)-3)/4$.
\end{proposition}
%

%%%%%%%%%%%%%%%%%%%%%%%%%%%%%%%%%%%%%%%%%%%%%%%%%%%%%%%%

\subsection{Online Performance under Reward Function $\reward(\type,\type') = \min\{\type,\type'\}$}\label{sec:dynamic}

We now study the performance of algorithms in the online setting, i.e. when $\sojourn<\Njob$, and derive informative performance guarantees conditional on $\sojourn$, the number of new arrivals before a job becomes critical.
%
In fact, the regret per job of $\gre$ remains constant, whereas the one of $\PB$ decreases with $\sojourn$.
%
To build upon our results proved in \Cref{sec: static} for the offline setting, we partition the set of jobs into $b=\lceil\Njob/(\sojourn + 1) \rceil$ \newterm{batches}.
%
To be precise, we define the $t$-th batch as $B_{t} = \{(t-1)(\sojourn+1)+1,\ldots, t(\sojourn+1)\} \cap [\Njob]$, for each $t = 1, \ldots, b$.
%
Scaling $\sojourn$ while keeping the number of batches $b$ fixed can be interpreted as increasing the density of the market.
%
We first show that the regret under $\gre$ remains linear in the number of jobs $\Njob$, independent of $\sojourn$, which can be understood as suffering a regret of order $\Omega(\sojourn)$ per batch.

\begin{proposition}[proof in \Cref{pf:greedy_linear_lower_bound_dynamic}]
\label{prop:greedy_linear_lower_bound_dynamic}
    Under reward topology $\reward(\type,\type') = \min\{\type,\type'\}$, if $(\sojourn+1)$ is divisible by 4, then for any number of jobs $\Njob$ divisible by $(\sojourn+1)$, there exists an instance $\instance \in [0,1]^\Njob$ for which $\regret_\gre(\instance,\sojourn) \ge n/4$.
\end{proposition}

In contrast, the following theorem shows that the performance of $\PB$ improves as market density increases.

\begin{theorem}\label{thm: dynamic}
Under reward topology $\reward(\type,\type') = \min\{\type,\type'\}$, we have $\regret_\PB(\instance,\sojourn) \le 1/2 + (\frac{\Njob}{d+1}+1)(1+\log(\sojourn+2))/2$ for any $\Njob$, and any instance $\instance\in[0,1]^\Njob$.
\end{theorem}
%
\proof{Proof.}
Let $\instance\in \typespace^\Njob$ and $\sojourn\ge 1$. Let $(C,\matchof(\cdot))$ be the output of $\PB$ on $\instance$.
Similar to the proof of \Cref{thm:potential_log_upper_bound}, we have
\begin{align}
\regret_\PB(\instance,\sojourn)
& \le \sum_{j=1}^\Njob \potential(\type_j) - \PB(\instance,\sojourn)  \nonumber \\
& \le \sup_{\type\in\typespace}\potential(\type) + \sum_{j \in C} \left( p(\theta_j) + p(\theta_{m(j)}) - r(\theta_j,\theta_{m(j)}) \right) \nonumber \\ 
& = \sup_{\type\in\typespace}\potential(\type) + \sum_{t=1}^{b} \sum_{j\in  C\cap B_t} \left( p(\theta_j) + p(\theta_{m(j)}) - r(\theta_j,\theta_{m(j)}) \right) \nonumber \\
& = \frac12 + \frac12 \sum_{t=1}^{b} \left(\sum_{j\in  C\cap B_t} |\type_j-\type_{\matchof(j)}| \right), \label{eq:online_dist}
\end{align}
where we split the sum into the $b = \lceil \Njob/(\sojourn + 1) \rceil$ batches, defined as $B_{t} = \{(t-1)(\sojourn+1)+1,\ldots, t(\sojourn+1)\} \cap [\Njob]$ for $t = 1, \ldots, b$. We analyze the term in large parentheses for an arbitrary $t$. Intuitively, we would like to consider an offline instance consisting of jobs $(\type_j)_{j\in C \cap B_t}$ and their matches $(\type_{\matchof(j)})_{j\in C\cap B_t}$, and apply \Cref{thm:potential_log_upper_bound} on the offline instance which has size $2|C\cap B_t|\le 2(\sojourn+1)$.
However, since the offline setting allows jobs to observe the full instance, as opposed to only the next $\sojourn$ arrivals, the resulting matching of $\PB$ on the offline instance could be inconsistent with its output on the online instance. In particular, executing $\PB$ on the offline instance could match jobs with indices more than $d$ apart, which is not possible in the online setting. Thus, to derive a proper upper bound on regret, we need to construct a modified offline instance for which each resulting match of $\PB$ coincides with the matching output in the online counterpart, and in which matching distances were not decreased.

%
To construct this modified offline instance, first note that if $\matchof(j)\in B_t$ for all $j\in C\cap B_t$, then no modification is needed, since all matched jobs have indices less than $\sojourn$ apart.
%
However, if $\matchof(j)\in B_{t+1}$ for some $j\in C\cap B_t$, then $\matchof(j)$ can be available in the offline instance for some job $k<j$ with $|\matchof(j)-k|>\sojourn$ and be matched to $k$ in the offline instance even though this would not be possible in the online instance (see \Cref{fig:thm3_exm}).
%
Thus, in the modified offline instance, we "move" job $m(j)$ ensure job $k$ would not choose $m(j)$ for its match.
%
To do so, we distinguish three cases.
%
For each $j\in C\cap B_t$, let $A(j)$ be the set of jobs that the online algorithm could have chosen from to match with $j$, i.e.~$A_j$ is the set $A\setminus\{j\}$ in \Cref{alg:dynamic} right after job $j$ becomes critical.
%
If $\matchof(j)\in A(j)\cap B_t$, then both jobs $j,\matchof(j)$ are included in the offline instance we construct.
%
In the second case, if $\matchof(j)\in A(j)\cap B_{t+1}$ and $A(j)\cap B_t\neq\emptyset$, then we modify $\type_{m(j)}$ to be equal to the best available matching candidate within batch i.e.\ we "move" job $m(j)$ to location $\argmin \{ |\type_j - \type| : \type = \type_k, \ k\in A(j)\cap B_t \}$.
%
In the third case, if $\matchof(j)\in A(j)\cap B_{t+1}$ and $A(j)\cap B_t=\emptyset$, then we don't consider job $j$ or its match (if any) in the offline instance we construct. This can happen at most once per batch, since any other job $k<j$ would have $j\in A(k)$. 
%

To formally define the construction, fix an arbitrary batch $t$. Let
$\hat{C}=\{j\in C\cap B_t: A(j)\cap B_t\neq \emptyset\}$ and $\hat{n}= 2|\hat{C}|\le 2(\sojourn+1)$. 
%
$\hat{C}$ is the set of critical jobs in cases one or two above, and we include jobs $j\in\hat{C}$ and their matches $m(j)$ in the offline instance.  There are $\hat{n}$ jobs in the offline instance and the ordering of indices is consistent with the original instance.
Define modified locations in the offline instance as follows:
\begin{align*}
\hat{\type}_j &= \type_j &\text{for $j\in\hat{C}$}
\\ \hat{\type}_{m(j)} &= \type_{m(j)} &\text{for $j\in\hat{C}$, if $m(j)\in A(j)\cap B_t$ (case one)}
\\ \hat{\type}_{m(j)} &= \argmin \{ |\type_j - \type| : \type = \type_k, \ k\in A(j)\cap B_t \} &\text{for $j\in\hat{C}$, if $m(j)\in A(j)\cap B_{t+1}$ (case two)}
\end{align*}
%
%
\begin{figure}[H]
  \centering
  \subcaptionbox{Batch $B_t$ of the online instance $\instance$.
    \label{fig:thm3_exm_original}
    }[0.4 \textwidth]
    {
    \begin{tikzpicture}


% Background shading for periods 6t+1 to 6t+6
\fill[lightyellow] (1.8, 0) rectangle (6.2, 3.2);

\draw[->] (-0.5, 0) -- (8.5, 0) node[right] {$j$};
\draw[->] (0, -0.1) -- (0, 3.3) node[above] {$\type_j$};

\foreach \x in {1,...,8}
    \draw (\x, -0.1) -- (\x, 0.1);

\node[below] at (1, 0) {\tiny $5t$};
\node[below] at (2, 0) {\tiny $5t+1$};
\node[below] at (3, 0) {\tiny $5t+2$};
\node[below] at (4, 0) {\tiny $5t+3$};
\node[below] at (5, 0) {\tiny $5t+4$};
\node[below] at (6, 0) {\tiny $5t+5$};
\node[below] at (7, 0) {\tiny $5t+6$};
\node[below] at (8, 0) {\tiny $5t+7$};

\foreach \y in {0.1,0.2,0.3,0.4,0.5,0.6,0.7,0.8,0.9,1}
    \draw (-0.1, \y*3) -- (0.1, \y*3) node[left, xshift=-0.2cm] {\tiny \y};
    

\foreach \x/\y in {1/0.9, 2/0.1, 3/0.7, 4/0.3, 5/0.4, 6/1, 7/0.2, 8/0.8} 
{
    \fill[black] (\x, \y*3) circle (0.1);
    \draw[gray, dashed] (0, \y*3) -- (\x, \y*3);
    }
    

% Matches
\draw[thick] (1, 0.9*3) -- (3, 0.7*3);     
\draw[thick] (2, 0.1*3) -- (4, 0.3*3);
\draw[thick] (5, 0.4*3) -- (7, 0.2*3);     
\draw[thick] (6, 1*3) -- (8, 0.8*3);  

\end{tikzpicture}
    }
  \hfill
  \subcaptionbox{Modified offline instance $\hat{\instance}$. 
    \label{fig:thm3_exm_phantom}
    }[0.4 \textwidth]
    {
    \begin{tikzpicture}

\draw[->] (-0.5, 0) -- (4.5, 0) node[right] {$j$};
\draw[->] (0, -0.1) -- (0, 3.3) node[above] {$\type_j$};

\foreach \x in {1,...,4}
    \draw (\x, -0.1) -- (\x, 0.1);

\node[below] at (1, 0) {\tiny $5t+1$};
\node[below] at (2, 0) {\tiny $5t+3$};
\node[below] at (3, 0) {\tiny $5t+4$};
\node[below] at (4, 0) {\tiny $5t+6$};

\foreach \y in {0.1,0.2,0.3,0.4,0.5,0.6,0.7,0.8,0.9,1}
    \draw (-0.1, \y*3) -- (0.1, \y*3) node[left, xshift=-0.2cm] {\tiny \y};
    

\foreach \x/\y in {1/0.1, 2/0.3, 3/0.4, 4/1} 
{
    \fill[black] (\x, \y*3) circle (0.1);
    \draw[gray, dashed] (0, \y*3) -- (\x, \y*3);
    }
    

% Matches
\draw[thick] (1, 0.1*3) -- (2, 0.3*3);     
\draw[thick] (3, 0.4*3) -- (4, 1*3);


\end{tikzpicture}
    }
  % 
\caption{Illustration of the construction of the modified offline instance, with $\sojourn=4$. In \Cref{fig:thm3_exm_original}, job $5t+1$ chooses job $5t+3$ in the online execution of $\PB$, because only jobs up to $5t+5$ would have arrived when job $5t+1$ becomes critical.  However, in the offline execution of $\PB$, job $5t+1$ would choose job $5t+6$, because all jobs are available.  Our modified instance in \Cref{fig:thm3_exm_phantom} corrects the inconsistent decision.
}
  % 
  \label{fig:thm3_exm}
\end{figure}
%
Now, we show that the output of (offline) $\PB$ on $\hat{\instance}$ provides an upper bound for the expression $\sum_{j\in C\cap B_t}|\type_j-\type_{\matchof(j)}|$ in~\eqref{eq:online_dist}.
%
In particular, we show that the output of $\PB$ is exactly $(\hat{C},m(\cdot))$ when executed on the modified offline instance.
%
First, note that the construction potentially introduces ties if there is a job $k\in C\cap B_t$ with $\matchof(k)\in B_{t+1}$. However, since \Cref{thm:potential_log_upper_bound} holds under arbitrary tie-breaking, we assume $\PB$ breaks ties on $\hat{\instance}$ to maintain consistent decisions with $\PB$ on the original instance $\instance$. Thus, from \Cref{deliverypotential}, it suffices to show that for every $j\in \hat{C}$, $\matchof(j)$ minimizes the distance among available jobs.
%

%
Indeed, note that every job in the offline instance has a type that is identical to a job in $B_t$. Then, proceeding in the same order as in the online instance, the job types of the set of available jobs in the offline instance is a subset of the original available jobs. For a match in case one, we know that it minimizes the distance among the original available jobs, and hence it minimizes the distance among the offline available jobs. Moreover, we have $|\hat{\type}_j-\hat{\type}_{\matchof(j)}| = |\type_j - \type_{\matchof(j)}|$. In the second case, the match minimizes distance among offline available jobs by construction, and moreover $|\hat{\type}_j-\hat{\type}_{\matchof(j)}| \ge |\type_j - \type_{\matchof(j)}|$.
%
Consequently, we can upper bound 
\[ \sum_{j\in C\cap B_t } |\type_j-\type_{\matchof(j)}|
\le 1 + \sum_{j\in  \hat{C}} |\hat{\type}_j-\hat{\type}_{\matchof(j)}|. 
\]
%  
Then, recalling \eqref{dist_bound} in the proof of \Cref{thm:potential_log_upper_bound}, we have $\sum_{j\in  \hat{C}} |\hat{\type}_j-\hat{\type}_{\matchof(j)}| \le \log(\hat{n}/2 + 1) \le \log(\sojourn + 2)$, and then
\begin{align}
    \sum_{j\in C\cap B_t } |\type_j-\type_{\matchof(j)}|
\le 1 +  \log(\sojourn + 2).\label{eq: distance_dynamic}
\end{align}
Substituting back into~\eqref{eq:online_dist}, and using the fact that $b\le\frac n{d+1}+1$, we get $\regret_\PB(\instance,\sojourn)\le 1/2 + (\frac{\Njob}{d+1}+1)(1+\log(\sojourn+2))/2$, completing the proof. 
\Halmos\endproof
%
Lastly, we show that this online analysis of $\PB$ is also tight, up to constants.
%
\begin{proposition}[proof in \Cref{pf:loglowerboundOnline}]
\label{prop:loglowerboundOnline}
Under reward topology $\reward(\type,\type') = \min\{\type,\type'\}$, if $\sojourn+1 = 2^{k+3}-4$ for some $k\in\{0,1,\ldots\}$, then for any number of jobs $\Njob$ divisible by $(\sojourn+1)$, there exists an instance $\instance \in [0,1]^\Njob$ for which $\regret_\PB(\instance,\sojourn) \ge \frac{\Njob}{3(\sojourn+1)}(\log_2(\sojourn+5)-3)/4$.
\end{proposition}
%

%%%%%%%%%%%%%%%%%%%%%%%%%%%%%%%%%%%%%%%%%%%%%%%%%%%%%%

% \subsection{Interpretation of Potential}\label{sec: interpretation}

% When a job $j$ becomes critical, our potential-based greedy algorithm matches it to an available job $k$ maximizing $r(\theta_j,\theta_k)-p(\theta_k)$, where $p(\theta_k)$ can be interpreted as the opportunity cost of matching job $k$, with the specific definition $p(\theta_k)=\frac12 \sup_{\theta\in\Theta} r(\theta_k,\theta)$.
% This is an optimistic measure of opportunity cost because it assumes that job $k$ would otherwise be matched to an "ideal" type $\theta\in\Theta$ maximizing $r(\theta_k,\theta)$ (with half of this ideal reward $\sup_{\theta\in\Theta} r(\theta_k,\theta)$ attributed to job $k$).
% We now prove that this ideal reward can indeed be achieved under asymptotically-large market thickness, for a stochastic model under our topology of interest.

% \begin{definition}
% Let $\instance \in \typespace^\Njob$.
% For any job $j\in[\Njob]$, let $\instance^{-j}\in \typespace^{\Njob-1}$ be the same instance with the exception that job $j$ is not present. Meanwhile, let $\instance^{+j}\in \typespace^{\Njob+1}$ be the same instance with the exception that an additional copy of job $j$ is present.
% Consider the following definitions.
% \begin{enumerate}
% \item Marginal Loss: $\marginalloss_j(\instance) = \OPT(\instance) - \OPT(\instance^{-j})$
% \item Marginal Gain: $\marginalgain_j(\instance) = \OPT(\instance^{+j}) - \OPT(\instance)$
% \end{enumerate}
% \end{definition}

% Note that definitions $\marginalloss_j(\instance),\marginalgain_j(\instance)$ are based solely on offline matching, and we will use them as our definitions of opportunity cost if the future was known.  One could alternatively use shadow prices from the LP relaxation of the offline matching problem, but we note that the LP is not integral.  In either case, there is no ideal definition of opportunity cost that is guaranteed to lead to the optimal offline solution in matching problems \citep[see][]{cohen2016invisible}.

% We now establish the following \namecref{lem:marginal_as_interval} to help analyze the opportunity costs $\marginalloss_j(\instance),\marginalgain_j(\instance)$.


% \begin{lemma}\label{lem:marginal_as_interval}
% Let $\typespace=[0,1]$ and $\reward(\type,\type') = \min\{\type,\type'\}$. Consider an instance $\instance\in\typespace^n$ and relabel the indices to satisfy $\type_1 \ge \type_2 \ge \ldots \ge \type_n$. Then,
% \begin{align*}
% \marginalloss_j(\instance)
% &=\sum_{k\ge j,k\ \mathrm{even}}(\theta_k-\theta_{k+1}), 
% \\ \marginalgain_j(\instance)
% &=\sum_{k\ge j,k\ \mathrm{odd}}(\theta_k-\theta_{k+1}),
% \end{align*}
% where we consider $\theta_{n+1}=0$.
% \end{lemma}
% %
% \proof{Proof.}
%     Let $\instance \in \typespace^\Njob$ such that $\type_1 \ge \type_2 \ge \ldots \ge \type_n$. 
%     %
%     Then, $\OPT(\instance) =\sum_{k\ \mathrm{even}}\theta_k$, and moreover
%     \begin{align*}
%     \OPT(\instance^{-j}) &=\sum_{k<j, k\ \mathrm{even}}\theta_k+\sum_{k>j, k\ \mathrm{odd}}\theta_k
%     \\ \OPT(\instance^{+j}) &=\theta_j+\OPT(\instance^{-j})
%     \end{align*}
%     % 
%     Therefore,
%     % 
%     \begin{align*}
%     \marginalloss_j(\instance)
%     &=\sum_{k\ge j,k\ \mathrm{even}}\theta_k
%     -\sum_{k>j,k\ \mathrm{odd}}\theta_k
%     =\sum_{k\ge j,k\ \mathrm{even}}(\theta_k-\theta_{k+1})
%     \\ \marginalgain_j(\instance)
%     &=\theta_j-\sum_{k\ge j,k\ \mathrm{even}}(\theta_k-\theta_{k+1})
%     =\sum_{k\ge j,k\ \mathrm{odd}}(\theta_k-\theta_{k+1})
%     \end{align*}
%     % 
%     completing the proof.
% \Halmos\endproof

% Note that because $\OPT(\instance^{+j}) = \OPT(\instance^{-j}) + \type_j$ and $\potential(\type_j) = \reward(\type_j,\type_j)/2$ for this reward function, we immediately get the following \namecref{cor: marginal_average}.

% \begin{corollary}\label{cor: marginal_average}
%         If $\typespace=[0,1]$ and $\reward(\type,\type') = \min\{\type,\type'\}$, then for all jobs $j$,
%     \[ \potential(\type_j) = \frac{\marginalloss_j(\instance) + \marginalgain_j(\instance)}{2}. \]
% \end{corollary}

% We are now ready to prove our main result about the interpretation of potential, that the true opportunity costs $\marginalloss_j(\instance),\marginalgain_j(\instance)$ concentrate around $p(\theta_j)$ in a random uniform instance, assuming the market is sufficiently thick.  We without loss consider job $j=1$ and fix its type $\theta_1$.

% \begin{theorem} \label{thm:interpretation}
% Let $\typespace=[0,1]$ and $\reward(\type,\type') = \min\{\type,\type'\}$.
% Fix $\theta_1\in\typespace$ and suppose $\theta_2,\ldots,\theta_n$ are drawn IID from the uniform distribution over $[0,1]$, forming a random instance $\instance\in\typespace^n$, for some $n\ge 2$. Then, both $\mathbb{E}[\marginalloss_1(\instance)]$ and $\mathbb{E}[\marginalgain_1(\instance)]$ are within $O(1/n)$ of $\potential(\type_1)$ and moreover $\mathrm{Var}(\marginalloss_1(\instance))=\mathrm{Var}(\marginalgain_1(\instance)) = O\left(\frac{1}{n}\right)$.
% \end{theorem}
% %
% \proof{Proof.}
% We prove the statement only for $\marginalloss_1(\instance)$, and the analogous result for $\marginalgain_1(\instance)$ follows from \Cref{cor: marginal_average}.
% %
% If $\type_1=0$, then $\marginalloss_1(\instance)=0$ for any $\instance$, coinciding with $\potential(\type_1)=0$. Then, for the remainder of the proof, assume that $\type_1>0$.
% %
% Consider the random variable $N=|\{j:\theta_j<\theta_1\}|$, which counts the number of points between $0$ and $\theta_1$ and has distribution $\text{Binom}(\Njob-1,\type_1)$.
% %
% These $N$ points divide the interval $[0,\theta_1]$ into $N+1$ intervals with total length $\type_1$.
% %
% From \Cref{lem:marginal_as_interval}, the marginal loss $\marginalloss_1(\instance)$ is determined by computing the total length of a subset of these intervals. % Maybe add figure?
% %
% It is known that if $N$ random variables are drawn independently from $\text{Unif}[0,1]$, then the joint distribution of the induced interval lengths is $\text{Dirichlet}(1,1,\ldots,1)$ with $N+1$ parameters all equal to 1 (i.e., drawn uniformly from the simplex).
% %
% In particular, since the Dirichlet distribution is symmetric, the sum of any $k$ of these intervals is equal in distribution to the $k$-th smallest sample ($k$-th order statistic) of the $N$ uniform random variables, whose distribution is known to be $\text{Beta}(k,N+1-k)$.
% %
% Thus, conditional on $N$, with $N\ge 1$, since the distribution of each of the $N$ jobs to the left of $\type_1$ is $\text{Unif}[0,\type_1]$, the distribution of $\marginalloss_1(\instance)/\type_1$ is $\text{Beta}(k_L,N+1-k_L)$,
% where $k_L$ is either $\lceil\frac{N+1}2\rceil$ or $\floor{\frac{N+1}2}$.
% %
% To be precise, for $N\ge 1$, $k_L=\floor {N/2}+1=\lceil\frac{N+1}2\rceil$ if $n$ is even, and $k_L=\ceil{N/2}=\floor{\frac{N+1}2}$ if $n$ is odd.
% %
% Moreover, if $N=0$, then $\marginalloss_1(\instance)=\type_1$ if $n$ is even, and $\marginalgain_1(\instance)=0$ if $n$ is odd.
% %
% Hence,
% \begin{align*}
% \mathbb{E}[\marginalloss_1(\type) \mid N] 
% &= \type_1\frac{k_L}{N+1}
% % \label{eq: cond_exp}
% \end{align*}
% for all $N\in\{0,\ldots,n-1\}$.
% %
% Since $|k_L-(N+1)/2|\le 1$, then $|\mathbb{E}[\marginalloss_1(\type) \mid N]-\type_1/2|\le \type_1/(N+1)$, thus
% % 
% \begin{align*}
% &\left|\mathbb{E}[\marginalloss_1(\type)]-\frac{\type_1}{2}\right| 
% \le \type_1\mathbb{E}\left[\frac{1}{N+1}\right] \\
% &\qquad = \type_1\sum_{k=0}^{n-1} \frac{1}{k+1}\binom{n-1}{k}\type_1^k(1-\type_1)^{n-1-k} \\
% &\qquad = \frac{\type_1}{n}\sum_{k=0}^{n-1} \binom{n}{k+1}\type_1^k(1-\type_1)^{n-1-k} \\
% &\qquad = \frac{1}{n}\sum_{k=1}^{n} \binom{n}{k}\type_1^k(1-\type_1)^{n-k} \\
% &\qquad = \frac{1-(1-\type_1)^{n}}{n} = O\left(\frac{1}{n}\right).
% \end{align*}
% %
% This completes the proof of the statement about $\mathbb{E}[\marginalloss_1(\instance)]$.

% For the statement about variance, we know $\mathrm{Var}(\marginalloss_1(\type)) = \mathbb{E}[\mathrm{Var}(\marginalloss_1(\type) \mid N)] + \mathrm{Var}(\mathbb{E}[\marginalloss_1(\type)\mid N])$ by the law of total variance. For the first term, we have $\mathrm{Var}(\marginalloss_1(\type) \mid N) = \type_1^2\frac{k_L(N+1-k_L)}{(N+1)^2(N+2)}\indicator\{N\ge 1\}\le \frac{\type_1^2}{N+1}$, and then from the previous argument $\mathbb{E}[\mathrm{Var}(\marginalloss_1(\type) \mid N)]=O(1/n)$.
% %
% For the second term, 
% \begin{align*}
% \mathrm{Var}(\mathbb{E}[\marginalloss_1(\type)\mid N]) 
% &= \mathrm{Var}\left(\mathbb{E}[\marginalloss_1(\type)\mid N] - \frac{\type_1}{2}\right) \\
% &\le \mathbb{E} \left[\left(\mathbb{E}[\marginalloss_1(\type)\mid N] - \frac{\type_1}{2}\right)^2\right] \\
% % 
% &\le \mathbb{E}\left[\frac{\type_1^2}{(N+1)^2}\right] \\
% &\le \type_1^2\mathbb{E}\left[\frac{1}{N+1}\right] = O\left(\frac{1}{n}\right).
% \end{align*}
% Thus, $\mathrm{Var}(\marginalloss_1(\instance)) = O\left(1/n\right)$, completing the proof.
% \Halmos\endproof



\section{Numerical Experiments} \label{sec:numerical_experiments}

In this section, we show via numerical simulations that under the reward structure from delivery pooling, our proposed potential-based greedy algorithm ($\PB$) outperforms a number of benchmark algorithms, given sufficient density.
%
We first generate synthetic data under a setting more aligned with our theoretical results, and find that $\PB$ outperforms all benchmarks starting from very low market densities ($\sojourn\ge 5$).
%
We then apply all algorithms and benchmarks on real data from the Meituan platform.
%
Although the real-life setting differs from our theoretical model in a number of ways --- that we address in \Cref{sec:sim_meituan} --- the results remain qualitatively similar: $\PB$ performs the best, starting from the modest density level corresponding to ``each job is able to wait for one minute before being dispatched''.
%
Additional simulation results, including additional performance metrics and more general synthetic environments (e.g. two-dimensional locations, different spatial distributions, and different reward topologies) are provided in \Cref{sec: sim_results_extra}.


The benchmark algorithms that we compare with are categorized into two groups, \emph{forecast-agnostic}, and \emph{forecast-aware}, depending on the use of historical data/demand distributional information to make matching decisions. We now describe each of them.
%


\subsection{Benchmark Algorithms} \label{sec:sim_benchmarks}


As we have discussed earlier, the potential based greedy algorithm $\PB$ does not rely on any historical data/demand forecast.
% 
The same is true for naive greedy $\gre$, and we refer to heuristics with this property as \newterm{forecast-agnostic}. 
%
In this group, we consider batching policies as additional benchmarks, which are highly relevant both in theory and in practice.


\paragraph{Naive Batching.} 


Batching heuristics periodically compute optimal matching solutions given the available jobs.
% 
Since it is without loss in our model to wait to match, we consider batching algorithms that re-compute optimal matchings whenever any available job becomes critical.
%
% 
The \newterm{naive batching} algorithm ($\batching$) dispatches all available jobs according to this optimal solution. 
%
In particular, under the theoretical model introduced in \Cref{sec:model}, $\batching$ clears the market every $\sojourn + 1$ arrivals, which is equivalent to the one analyzed in \citet{ashlagi2019edge}. 
%
However, an obvious suboptimality of this algorithm is that there are pairs of jobs that are matched and dispatched before any of them becomes critical. 



\paragraph{Rolling Batching.} 

To address the previously mentioned issue of dispatching non-critical pairs of jobs, we also consider a modified batching heuristic, termed \newterm{rolling batching} ($\rbatching$).
%
$\rbatching$ also computes an optimal matching solution every time any available job becomes critical. Instead of dispatching all jobs, however, $\rbatching$ only dispatches the critical job and its match (if any) according to the optimal solution. 


\medskip 


In practice, delivery platforms typically have access to past order data that allows to predict patterns of future arrivals to some extent. 
%
In contrast to the aforementioned algorithms, we say that a policy is \newterm{forecast-aware} if it uses historical data to make pooling decisions. 
%
We consider dual-based benchmark algorithms that use optimal dual solutions (i.e. shadow prices) to approximate the opportunity cost of matching each job. 
% 
More specifically, given an instance from the historical data (i.e. a sequence of delivery requests with information about origin, destination, and timestamp), we can solve the dual program of a linear relaxation of the IP defined in \eqref{eq: OPT}.
% 
\footnote{We provide a detailed description of the primal and dual linear programs in \Cref{sec: LP}.}
% 
The optimal dual variables associated with the capacity constraint of each job will then provide a proxy for the marginal value of the job, which we use to construct two additional benchmark algorithms.


\paragraph{Hindsight Dual.}
% 
Ideally, if we had access to the dual optimal solution for an instance in advance, we would use them as the opportunity cost of matching each job.
%
In an attempt to measure the value of this hindsight information, we simulate a \newterm{hindsight-dual} algorithm ($\dual$) that operates retrospectively on the same instance.
%
Formally, given an instance $\instance\in\typespace^\Njob$ and parameter $\sojourn$, let $\lambda_k=\lambda_k(\instance,\sojourn)$ be the optimal dual variable associated with the constraint $\sum_{k:j\neq k} x_{jk} \le 1$ in the linear program  (defined in \eqref{eq: LP}). The $\dual$ algorithm operates as an index-based greedy matching algorithm (as defined in \Cref{alg:dynamic}) where the index function is given by $\indexf_{\dual}(\type_j,\type_k) = \reward(\type_j,\type_k) - \lambda_k$, for all $j,k\in [\Njob]$, such that $j\neq k, |j-k| \le d$.
% 
Note that this index function is computed based on the entire realized instance $\instance$, thus this algorithm is only for benchmarking purposes and is not practically feasible.



\paragraph{Average Dual.}
%
A more realistic approach for estimating the opportunity costs is to average the hindsight duals of the jobs from historical data.
% 
However, this estimation is not free of challenges. 
% 
In particular, computing the dual solutions yields a shadow price for every observed type, which might not cover the entire type space and thus 
many future arrivals may not have an associated shadow price.
% 
To address this issue, we discretize the type space and estimate average duals for each discrete type as the empirical average of the shadow prices for jobs associated to the discrete type.
% 
To be precise, let $H$ be the set of all instances in the historical data, and consider a discretization $\{\typespace_1,\typespace_2,\ldots,\typespace_p\}$
% 
that partitions the space into $p$ cells (i.e. these cells are mutually exclusive and jointly exhaustive).
%
Each cell $i=1,2,\ldots,p$ is associated with a shadow price $\Bar{\lambda}_i$ that is the average shadow price of historical jobs lying in the same cell, i.e. $\Bar{\lambda}_i$ is the average of $\{\lambda(\instance,\sojourn) : \instance\in H, \type_j \in \typespace_i \}$.
%
Then, the \newterm{average-dual} algorithm
($\averagedual$) is an index-based greedy matching algorithm that uses the shadow prices for the cells, i.e.\ uses index function
$\indexf_{\averagedual}(\type,\type') = \reward(\type,\type') - \sum_{i=1}^p\Bar{\lambda}_{i}\indicator\{\type'\in\typespace_i\}.$

\subsection{Synthetic Data}\label{sec:sim_unif_1D}

We first consider the setting analyzed in our theoretical results, where job destinations are uniformly distributed on a one-dimensional space.
% 
We fix the total number of jobs at $\Njob=1000$, 
and compare algorithms at density levels $\sojourn\in\{5,10,15,20,25,30\}$.
%
For each $(\Njob, \sojourn)$ pair, we generate 100 instances $\instance \in[0,1]^\Njob$ uniformly at random, apply all algorithms on each of them, and compute the average regret and reward \newterm{ratio} ($\ALG(\instance,d)/\OPT(\instance,d)$) achieved by each algorithm and benchmark.
% 
Additional performance metrics, including the fraction of jobs that are pooled and the fraction of the total distance that is reduced by pooling, are presented in \Cref{sec:match_rate}. Similar results for the 1D, non-uniform setting are presented in \Cref{sec: nonunif_1D}.
% 

\subsubsection{Comparison with forecast-agnostic heuristics.}
%
\Cref{fig:1D} compares potential-based greedy $\PB$ with naive greedy $\gre$ and the two batching algorithms.
% 
As expected from our analysis, $\gre$ performs poorly even at high densities. To our surprise, $\PB$ outperforms both batching algorithms for all considered density values.
%
To assess the performance of the algorithms relative to the hindsight optimal, we also compute the \emph{ratio} for each instance, and plot its empirical average in \Cref{fig:1D_ratio}.
% 
We can see that even at relatively low density levels, $\PB$ achieves over 95\% of the maximum possible travel distance saved from pooling. $\gre$, on the other hand, performs worse relative to hindsight optimal as density increases--- despite the fact that the
%the fraction of
travel distance saved by $\gre$ increases with density (see \Cref{fig:1D_unif_saving_fraction} in \Cref{sec: 1D_extra}). 
%
\begin{figure}%[H]
  \centering
  \subcaptionbox{Average regret.%
    \label{fig:1D_regret}}[0.49 \textwidth]{\includegraphics[width = \figWidth \textwidth]{Simulation_Results/Synthetic/New/1D_Pooling/1D_Common_Origin/Forecast-Agnostic/regret.png}}
  \hfill
  \subcaptionbox{Average ratio.%
    \label{fig:1D_ratio}}[0.49 \textwidth]{\includegraphics[width = \figWidth \textwidth]{Simulation_Results/Synthetic/New/1D_Pooling/1D_Common_Origin/Forecast-Agnostic/frac_of_OPT.png}}
  % 
  \caption{Comparison of average regret and reward ratio for forecast-agnostic heuristics in random 1D instances.}
  % 
  \label{fig:1D}
\end{figure}
\subsubsection{Comparison with forecast-aware heuristics.}

We now compare $\PB$ with our two forecast-aware benchmarks, $\dual$ and $\averagedual$.
%
%
Under $\dual$, for each of the 100 instances generated for each market condition $(n,d)$, we compute a shadow price for each job via the dual LP.
%
As discussed in \Cref{sec:sim_benchmarks}, this algorithm is not practically feasible, and we present its performance here mainly to illustrate the value of this hindsight information used in this particular way.
%
%
For $\averagedual$, for each $(n,d)$, we first generate 400 \emph{historical instances} in the same way that the original 100 instances were drawn.
% 
%
We then compute optimal dual variables for each historical instance, and estimate average shadow prices $\Bar{\lambda}$ for a uniform discretization of the type space $[0,1]$ into 100 intervals.
% 

\Cref{fig:1D_forecastaware} compares the average regret and reward ratio on the original 100 instances.
%
We can see that $\PB$ performs on par with $\averagedual$ across all density parameters, without relying on any forecast/historical data. 
%
Notably, even with access to hindsight information, $\dual$ is consistently dominated by both $\averagedual$ and $\PB$, with the performance gap narrowing as $\sojourn$ increases.
%


\begin{figure}%[H]
  \centering
  \subcaptionbox{Average regret.%
    \label{fig:1D_forecastaware_regret}}[0.49 \textwidth]{\includegraphics[width = \figWidth \textwidth]{Simulation_Results/Synthetic/New/1D_Pooling/1D_Common_Origin/Forecast-Aware/regret.png}}
  \hfill
  \subcaptionbox{Average ratio.%
    \label{fig:1D_forecastaware_ratio}}[0.49 \textwidth]{\includegraphics[width = \figWidth \textwidth]{Simulation_Results/Synthetic/New/1D_Pooling/1D_Common_Origin/Forecast-Aware/frac_of_OPT.png}}
  % 
  \caption{Comparison of average regret and reward ratio for forecast-aware heuristics in random 1D instances. }
  % 
  \label{fig:1D_forecastaware}
\end{figure}




\subsection{Order-Level Data from Meituan Platform}\label{sec:sim_meituan}


In this \namecref{sec:sim_meituan}, we test the algorithms and benchmarks using data from the Meituan platform, made public for the 2024 INFORMS TSL Data-Driven Research Challenge.\footnote{\url{https://connect.informs.org/tsl/tslresources/datachallenge}, accessed January 15, 2025.
%
See \Cref{sec:market_dynamics} for more details on the dataset, as well as high-level illustrations of market dynamics over both space and time.
} 
%
The dataset provides detailed information on various aspects of the platform's delivery operations, for one city and a total of 8 days in October 2022.  
%
For each of the 569 million delivery orders, the dataset provides the pick-up and drop-off locations (the latitudes and longitudes were shifted for privacy considerations), and the timestamps at which orders are created, pushed into the dispatch system, etc.
% 
% 
This allows us to (i) construct instances of job arrivals and (ii) calculate travel distances and pooling opportunities.




Since the pick-up and drop-off locations now reside in a two-dimensional (2D) space, we first extend our definition of pooling reward to the 2D space, derive the potential of different job types, and provide the notion of a pooling window (i.e. the sojourn time) that determines which jobs can be pooled together.


\paragraph{Pooling reward.}
% 
Each job $j$ has type $\type_j = (\xorigin_j,\xdestination_j)$, 
% 
with $\xorigin_j,~\xdestination_j\in \R^2$ corresponding to the coordinates of the job origin and destination.
%
The travel distance saved by pooling two orders together is the difference between (i) the travel distance required to fulfill the jobs in two separate trips, and (ii) the minimum travel distance for fulfilling the two requests in a single, pooled trip.
%
% 
For the pooled trip, the travel distance always includes the distance between the two origins and that between the two destinations (both orders must be picked up before either of them is dropped off; otherwise the orders are effectively not pooled). 
% 
The remaining distance depends on the sequence in which the jobs are picked up and dropped off, resulting in four possible routes. The minimum travel distance is then determined by evaluating the total distance for all four routes and selecting the smallest value.
% 
Formally, letting $\norm{\cdot}$ denote the Euclidean norm, the reward function is
% 
\begin{align}
    \reward(\type,\type') = &  \norm{\xdestination-\xorigin} +  \norm{\xdestination'-\xorigin'}  \nonumber\\
    % 
    & -\left(\norm{\xorigin-\xorigin'} + \min\left\{\norm{\xdestination-\xorigin},\norm{\xdestination'-\xorigin'},\norm{\xdestination-\xorigin'},\norm{\xdestination'-\xorigin'}\right\}+\norm{\xdestination-\xdestination'}\right). \label{eq:2D_reward_part_2}
\end{align}
% 
Note that unlike the 1D common origin setting, the pooling reward in 2D space with heterogeneous origins is not guaranteed to be positive. As a result, we assume that all index-based greedy algorithms only pool jobs together when the reward is non-negative.

\paragraph{Potential.}
% 
For a given job type $\type=(\xorigin,\xdestination)$, the maximum reward a platform can achieve by pooling it with another job is the distance of the job, $\norm{\xdestination - \xorigin}$.\footnote{To see this, first observe that when a job $\type = (\xorigin,\xdestination)$ is pooled with another job of identical type, the travel distance saved is precisely $\norm{\xdestination - \xorigin}$.
%
On the other hand, let $m$ be the minimum in \eqref{eq:2D_reward_part_2}. Then, by triangle inequality $\|O'-D'\|\le \|O-O'\| + \|D-O\| + \|D-D'\|$, and thus $\reward(\type,\type')\le 2\|D-O\|-m \le \|D-O\|$, since $m\le \|D-O\|$.
% % 
To achieve an even higher pooling reward, it must be the case that the second part of the pooling reward \eqref{eq:2D_reward_part_2} is strictly smaller than $\norm{\xdestination' - \xorigin'}$. This implies that the total distance traveled to visit all four locations is smaller than the distance between two of them, which is impossible.}  

As a consequence, the potential of a job $\type$ is $\potential(\type) = \norm{\xdestination - \xorigin}/2$. %, 
% 
This is aligned with our 1D theoretical model, in which longer deliveries have higher potential reward from being pooled with other jobs in the future.
%

\paragraph{Pooling window.} 
% 
Instead of assuming that each job is available to be pooled for a fixed number of future arrivals, we take the actual timestamps of the jobs, and assume that the jobs are available for a fixed time window, at the end of which the order becomes critical.
%
Intuitively, the length of the pooling window corresponds to the ``patience level'' of the jobs.
% 
This deviates from our assumptions for the theoretical model, but it is straightforward to see how our algorithm as well as the benchmarks we compare with can be generalized and applied.


\subsubsection{Comparison with forecast-agnostic heuristics.}

We focus our analysis on a three-hour lunch period (10:30am to 1:30pm) for the 8 days of data provided by the challenge.
% 
During this lunch period, there are approximately $130$ orders per minute on average, thus $\Njob\approx 24,000$ orders in total (see \Cref{fig:meituan_orders_per_hour} in \Cref{sec:market_dynamics} for an illustration of order volume over time).
% 
If the platform allows a time window of one minute for each order before it has to be dispatched, the average density would be around $130$ jobs. 
% 
\Cref{fig:meituan} compares the forecast-agnostic heuristics, as the pooling window increases from 30 seconds to 5 minutes. We can see that a single minute is sufficient for $\PB$ to outperform every other forecast-agnostic algorithm, achieving around 80\% of the hindsight optimal reward.
% 
Moreover, the performance of $\PB$ improves steadily as the allowed matching window increases.
%

\begin{figure}%[H]
  \centering
  \subcaptionbox{Average regret.%
    \label{fig:meituan_regret}}[0.49 \textwidth]{\includegraphics[width = \figWidth \textwidth]{Simulation_Results/Meituan/City/regret_fagnostic.png}}
  \hfill
  \subcaptionbox{Average ratio.%
    \label{fig:meituan_ratio}}[0.49 \textwidth]{\includegraphics[width = \figWidth \textwidth]{Simulation_Results/Meituan/City/ratio_fagnostic.png}}
  % 
  \caption{Comparison of average regret and reward ratio for forecast-agnostic heuristics in Meituan data.}
  % 
  \label{fig:meituan}
\end{figure}


\subsubsection{Comparison with forecast-aware heuristics.}

We now compare our algorithm with the forecast-aware heuristics on data from the same three-hour lunch period, 10:30am to 1:30pm.
% 
To simulate the \emph{hindsight-dual} algorithm $\dual$ for each instance associated to each day, we compute the shadow prices via the dual LP program and then compute the ``online'' pooling outcomes using pooling reward minus these shadow prices as the index function.
%
% 
To estimate historical average shadow prices for the $\averagedual$ algorithm, for each of the 8 days, we (i) use the remaining 7 days as the historical data, and (ii) discretize the 2D space into discrete types based on job origin and destination using H3, the hexagonal spatial indexing system developed by Uber.\footnote{See {\url{https://www.uber.com/blog/h3/} (accessed January 7, 2024) for more details on the H3 package.} 
% 
% 
The system supports sixteen resolution levels, each tessellating the earth using hexagons of a particular size.
% %
For a given resolution, the set of all origin-destination (OD) hexagon pairs represents the partition of the type space $\typespace$, as described in \Cref{sec:sim_benchmarks}.
% %
% % 
To compute the average shadow price for a particular OD hexagon pair, we compute the average of optimal dual variables associated with all historical jobs that share the same hexagon pair.
% %
If there are no jobs for a given pair, however, then we consider the hexagon pairs at coarser resolutions (moving one or more levels up in the H3 hierarchy, with each level increasing the size of the hexagons by a factor of 7) until there exist associated historical jobs.
% %
There is a tradeoff between granularity and sparsity when we choose the baseline resolution level, as finer resolutions only average over trips with close-by origins and destinations, but may lack sufficient historical data for accurate estimation (see \Cref{fig:meituan_CDF_count_per_OD_pair} for the distribution of order volume by OD hexagon pairs).
% 
After testing all available resolution levels, we found that levels 8 and 9 perform the best, depending on the time window. The results presented in this section report the result corresponding to the best resolution for each time window.
}

\Cref{fig:meituan_forecastaware} compares the regret and reward ratio of $\PB$ to forecast-aware heuristics.
%
We first observe that in contrast to the synthetic setting from \Cref{sec:sim_unif_1D}, $\dual$ now incurs the smallest regret at all density levels. %
% 
This performance difference appears to be driven by the highly non-uniform spatial distribution of jobs --- additional simulations, presented in \Cref{sec:sim_unif_1D_2D}, demonstrate that (i) $\dual$ performs no better than $\PB$ if job origins and destinations are drawn uniformly at random from a unit square, and (ii) when jobs are generated from a non-uniform distribution, $\dual$ slightly outperforms $\PB$. Of course, we remind the reader that $\dual$ is not a realistic algorithm because it requires knowing the exact future.

The worse performance of $\averagedual$ in comparison to $\PB$ also highlights the challenges of estimating the opportunity costs of pooling different jobs from historical data in real-world settings. 
% 
Intuitively, taking historical data into consideration should be valuable in scenarios with highly non-uniform spatial distributions. In this case, however, the sparsity of the data for many if not most origin-destination pairs makes it difficult to properly estimate the opportunity costs, or requires a very coarse granularity for location.  $\PB$ bypasses this sparsity-granularity tradeoff by only considering the reward topology, i.e.\ the space of all possible types.
%


\begin{figure}%[H]
  \centering
  \subcaptionbox{Average regret.%
    \label{fig:meituan_forecastaware_regret}}[0.49 \textwidth]{\includegraphics[width = \figWidth \textwidth]{Simulation_Results/Meituan/City/regret_faware.png}}
  \hfill
  \subcaptionbox{Average ratio.%
    \label{fig:meituan_forecastaware_ratio}}[0.49 \textwidth]{\includegraphics[width = \figWidth \textwidth]{Simulation_Results/Meituan/City/ratio_faware.png}}
  % 
  \caption{Comparison of average regret and reward ratio for forecast-aware heuristics in Meituan data.}
  % 
  \label{fig:meituan_forecastaware}
\end{figure}


% \subsubsection{Match rate and saving fraction.}\label{sec:match_rate}
% % \paragraph{Match Rate and Saving Fraction.}

% In addition to regret and reward ratio, we consider in this section two additional performance metrics: the \newterm{match rate}, i.e. the fraction of jobs that were pooled instead of dispatched on their own, and  the \newterm{saving fraction}, i.e. the fraction of the total distance that is reduced by pooling, relative to the total travel distance without any pooling~\citep[see][]{aouad2020dynamic}.
% %
% % 
% For brevity, we focus only on the potential-based greedy algorithm $\PB$ and the hindsight optimum $\OPT$ in \Cref{fig:meituan_2}.
% % 
% A comprehensive comparison across all 
% algorithms and benchmarks is provided in \Cref{sec: meituan_extra}.


% \begin{figure}[H]
%   \centering
%   \subcaptionbox{Average match rate.%
%     \label{fig:meituan_2_match_rate}}[0.49 \textwidth]{\includegraphics[width = \figWidth \textwidth]{Simulation_Results/Meituan/City/match_rate.png}}
%   \hfill
%   \subcaptionbox{Average saving fraction.%
%     \label{fig:meituan_2_saving_fraction}}[0.49 \textwidth]{\includegraphics[width = \figWidth \textwidth]{Simulation_Results/Meituan/City/saving_rate.png} }
%   % 
%   \caption{Match rate and Saving Fraction, Meituan Order-Level Data.}
%   % 
%   \label{fig:meituan_2}
% \end{figure}

% Both regret and reward ratio are performance metrics that compare pooling algorithm relative to the hindsight optimal pooling outcome. The saving fraction
% % 
% as shown in \Cref{fig:meituan_2_saving_fraction} illustrates the benefit of delivery pooling in comparison to the total distance traveled, and is upper bounded by 0.5 (since reward from pooling a job cannot exceed its distance).
% % 
% We can see that if the platform allows a time window of at least $1$ minute for each order before it has to be dispatched, $\PB$ achieves a reduction in distance traveled by over 20\%. At $5$ minutes, this fraction increases to roughly 30\%.
% %
% Since the saving fraction is proportional to the total reward collected by the algorithm, our previous performance comparisons imply that the saving fraction achieved by $\PB$ is the best among all tested \emph{practical} heuristics, i.e. excluding $\dual$ and $\OPT$.

% \Cref{fig:meituan_2_match_rate} compares the match rate achieved under $\PB$ and $\OPT$. We can see that $\PB$ consistently pools more jobs than $\OPT$ (5-10\%). 
% %
% This highlights an additional desirable property of $\PB$ and, more generally, of \emph{index-based greedy matching} algorithms: they pool as many jobs as possible.
% % 
% This is desirable in practice, since drivers who are offered just only one order from a platform may try to pool orders from competing platforms \citep[see e.g.][]{reddit2022grubhub,reddit2023doordash}, leading to poor service reliability for customers.

% Note that for the one-dimensional settings, match rates are very close to 100\% under greedy policies since at most one job is left unmatched (see \Cref{fig:1D_unif_match_rate} in \Cref{sec: 1D_extra} for comparisons). 
% %
% In the real-data setting, match rates are generally lower since we pool the critical job with another only when it's possible to achieve a positive pooling reward. Nevertheless, \Cref{sec: meituan_extra} shows that greedy algorithms ($\gre$, $\PB$, $\dual$, $\averagedual$) still achieve substantially higher match rates in comparison to $\batching$, $\rbatching$, and $\OPT$.




\section{Concluding Remarks and Future Work} \label{sec:conclusion}

In this work, we study dynamic non-bipartite matching in the context of pooling delivery orders.
% 
Our modeling approach captures two key features of the problem through the reward topology and the assumption that jobs can be matched until they become \textit{critical}.
%
We propose a simple, forecast-agnostic potential-based greedy algorithm, specially designed for this reward topology, that performs surprisingly well both in synthetic and real data. It uses a new notion of potential as opportunity cost, that considers the value of keeping long-distance deliveries in the platform for better pooling opportunities in the future. We show that for our reward topology of focus, potential-based greedy improves upon greedy both in worst-case and empirical performance. Moreover, we showcase the robustness of potential-based greedy through numerical experiments in real data, outperforming several commonly used benchmark algorithms given enough market density.
%
We provide some intuition behind the success of this approach by showing through theory and experiments that different notions of opportunity cost are intimately related to our definition of potential, especially as density increases. 
%
Moreover, we believe that the main insight of long-distance jobs being more valuable to hold should apply even when pooling more than two orders into a single trip (see \citet{wei2023constant}); and we believe our notion of potential could be relevant even on non-metric reward structures, where it is a measure of e.g.\ the popularity of a volunteer position (see \citet{manshadi2022online}).
%
Formalizing and generalizing these results is an interesting avenue of future research.
%
Overall, these surprising results showcase the value of using the knowledge of reward topology to design application-specific algorithms. This general idea, which has received limited attention so far, could also be applied to other domains, such as ride-sharing or kidney exchange.

%\THEEndNotes
% \begingroup \parindent 0pt \parskip 0.0ex \def\enotesize{\normalsize} \theendnotes \endgroup

% Acknowledgments here
\ACKNOWLEDGMENT{
The authors would like to thank
Itai Ashlagi,
Omar Besbes,
Francisco Castro,
Yash Kanoria,
Jake Marcinek,
Rad Niazadeh,
Scott Rodilitz,
Daniela Saban, and
Alejandro Torrielo
for valuable comments and discussions.
}


% References here (outcomment the appropriate case)

% CASE 1: BiBTeX used to constantly update the references
%   (while the paper is being written).
% \bibliographystyle{informs2014trsc} % outcomment this and next line in Case 1
% \bibliography{DraftBib} % if more than one, comma separated

%\bibliographystyle{informs2014trsc} % outcomment this and next line in Case 1
%\bibliography{sample} % if more than one, comma separated

% CASE 2: BiBTeX used to generate mypaper.bbl (to be further fine tuned)
%%%%%%%%%%%%%%%%%%%%%%%%%%%%%%%%%%%%%%%%%%%%%%%%%%%%%%%%%%%%%%%%%%%%%%%%%%
%%
%%	Author Submission Template for Transportation Science (TRSC)
%%	INFORMS, <informs@informs.org>
%%	Ver. 1.00, June 2024
%%
%%%%%%%%%%%%%%%%%%%%%%%%%%%%%%%%%%%%%%%%%%%%%%%%%%%%%%%%%%%%%%%%%%%%%%%%%%
%
% Use dblanonrev for Double Anonymous Review submission
% Use sglanonrev for Single Anonymous Review submission
% For example, submission to Operations Research, OPRE will have
% \documentclass[opre,dblanonrev]{informs4}

%%% TRUE SUBMISSION
% \documentclass[trsc,dblanonrev]{informs4}
% \usepackage{eqndefns-left} % For checking the display equation width and equation environment definitions %

%%% ARXIV VERSION
\documentclass[trsc,sglanonrev]{informs4_hide}

\RequirePackage{tgtermes}
\RequirePackage{newtxtext}
\RequirePackage{newtxmath}
\RequirePackage{multirow,multicol}
\RequirePackage{xspace,dsfont}
\RequirePackage{endnotes}
% \let\footnote=\endnote


%\OneAndAHalfSpacedXI
% \OneAndAHalfSpacedXII % Current default line spacing
\DoubleSpacedXI % Double-spacing for 11-point font (TSL)
% \DoubleSpacedXII

% Optional LaTeX Packages
\usepackage{algorithm}
\usepackage{algpseudocode}
\algrenewcommand\algorithmicrequire{\textbf{Input:}}
\algrenewcommand\algorithmicensure{\textbf{Output:}}
\usepackage{tikz}
%
\usepackage{xcolor}
\definecolor{dark1}{RGB}{157, 58, 103}
\definecolor{dark2}{RGB}{161, 67, 0}
\definecolor{dark3}{RGB}{115, 102, 0}
\definecolor{dark4}{RGB}{2, 120, 50}
\definecolor{dark5}{RGB}{0, 116, 122}
\definecolor{dark6}{RGB}{18, 100, 176}
\definecolor{dark7}{RGB}{116, 75, 163}
\definecolor{mid1}{RGB}{225, 119, 163}
\definecolor{mid2}{RGB}{228, 128, 77}
\definecolor{mid3}{RGB}{182, 158, 21}
\definecolor{mid4}{RGB}{87, 182, 109}
\definecolor{mid5}{RGB}{0, 181, 190}
\definecolor{mid6}{RGB}{88, 162, 242}
\definecolor{mid7}{RGB}{177, 135, 229}
\definecolor{light1}{RGB}{255, 187, 231}
\definecolor{light2}{RGB}{255, 198, 151}
\definecolor{light3}{RGB}{247, 223, 104}
\definecolor{light4}{RGB}{152, 248, 171}
\definecolor{light5}{RGB}{72, 249, 255}
\definecolor{light6}{RGB}{164, 219, 255}
\definecolor{light7}{RGB}{244, 192, 255}
\definecolor{lightyellow}{RGB}{255, 255, 204}

%% Comments
%%% Comments  will be removed if \Comments = 0 %%%
\newcount\Comments
\Comments = 0
\newcommand{\kibitz}[2]{\ifnum\Comments=1{\color{#1}{#2}}\fi}

\newcommand{\mr}[1]{\kibitz{mid1}{[Matias: #1]}}
\newcommand{\hma}[1]{\kibitz{mid4}{[HMa: #1]}}


\newcount\Drop  
\Drop = 0
\newcommand{\todrop}[2]{\ifnum\Drop=1{\color{#1}{#2}}\fi}
\newcommand{\drop}[1]{\todrop{mid5}{[Drop: #1]}}


%%%% Highlights that becomes black if \highlighttrue is commented out 
\newif\ifhighlight 
\highlighttrue % comment out this line to disable highlights

\newcommand{\mradd}[1]{{\ifhighlight\color{dark1}\fi#1}}
\newcommand{\hmadd}[1]{{\ifhighlight\color{dark4}\fi#1}}


\newif\iftodo
% \todotrue % comment out this line to hide the TODOs 

\newcommand{\todo}[1]{\iftodo{\color{mid1!50!gray}{[TODO: #1]}}\fi}

\usepackage{color}              % Need the color package
% \usepackage[suppress]{color-edits}
\usepackage{color-edits}
\addauthor{Will}{blue}
\addauthor{Hma}{dark4}
%
\usepackage[colorlinks,
	citecolor = dark6,
	urlcolor = black,
	linkcolor = dark6,
    hypertexnames=false]{hyperref}
\usepackage[nameinlink]{cleveref}
\crefname{subsection}{subsection}{subsections}
\usepackage[short]{optidef}
\usepackage[font=scriptsize]{subcaption}
\newcommand{\figWidth}{0.45}

% Private macros here (check that there is no clash with the style)
%%%%%%%%%%%%%%%%%%%%%%%%%% NOTATION %%%%%%%%%%%%%%%%%%%%%%%%

% bold lowercase letters, for vectors
\newcommand\vzero{{\bm 0}}
\newcommand\vzeron{{\bm 0}_n}
\newcommand\vone{{\bm 1}}
\newcommand\vonen{{\bm 1}_{n}}
% 
\newcommand\vpi{{\bm\pi}}
\newcommand\vphi{{\bm\phi}}
\newcommand\veta{{\bm\eta}}
\newcommand\vmu{{\bm\mu}}
% \newcommand\vtheta{{\bm\theta}}
\newcommand\vtheta{{\BFtheta}}
\newcommand\vdelta{{\bm\delta}}
\newcommand\va{{\bm a}}
\newcommand\vb{{\bm b}}
\newcommand\vc{{\bm c}}
\newcommand\vd{{\bm d}}
\newcommand\ve{{\bm e}}
\newcommand\vf{{\bm f}}
\newcommand\vg{{\bm g}}
\newcommand\vh{{\bm h}}
\newcommand\vi{{\bm i}}
\newcommand\vj{{\bm j}}
\newcommand\vk{{\bm k}}
\newcommand\vl{{\bm l}}
\newcommand\vm{{\bm m}}
\newcommand\vn{{\bm n}}
\newcommand\vo{{\bm o}}
\newcommand\vp{{\bm p}}
\newcommand\vq{{\bm q}}
\newcommand\vr{{\bm r}}
% \newcommand\vs{{\bm s}}
\newcommand\vt{{\bm t}}
\newcommand\vu{{\bm u}}
\def\vv{{\bm v}}
\newcommand\vw{{\bm w}}
\newcommand\vx{{\bm x}}
\newcommand\vy{{\bm y}}
\newcommand\vz{{\bm z}}


% Number sets
\newcommand{\C}{\mathbb{C}}
\newcommand{\R}{\mathbb{R}}
\newcommand{\Q}{\mathbb{Q}}
\newcommand{\N}{\mathbb{N}}
\newcommand{\Z}{\mathbb{Z}}

% Personal definitions
\renewcommand{\P}{\mathbb{P}}
\newcommand{\geom}[1]{\text{Geom}(#1)}
\newcommand{\E}{\mathbb{E}}
\newcommand{\var}{\text{Var}}
\newcommand{\normal}{\mathcal{N}}
\newcommand{\indicator}{\mathds{1}}
%\newcommand{\converges}[2][n \to \infty]{\xrightarrow[#1]{#2}}
\newcommand{\converges}[2][]{\xrightarrow[#1]{#2}}
\newcommand{\equalin}[1]{\stackrel{#1}{=}}
\newcommand{\loss}[1]{\depth({#1})}
% \DeclareMathOperator*{\argmin}{arg\,min}
% \DeclareMathOperator*{\argmax}{arg\,max}
\newcommand{\infsum}[1]{\sum_{#1\ge 1}}
\newcommand{\finsum}[2]{\sum_{#1=1}^{#2}}
\newcommand{\abs}[1]{\left|#1\right|}
\newcommand{\norm}[2][]{\left\|#2\right\|_{#1}}
\newcommand{\innerprod}[2][]{\left\langle #2 \right\rangle_{#1}}
\newcommand{\ceil}[1]{\left\lceil #1 \right\rceil}
\newcommand{\floor}[1]{\left\lfloor #1 \right\rfloor}
% \newcommand{\norm}[1]{\left\| #1 \right\|}
\newcommand{\indicatorof}[1]{\indicator\left\{#1\right\}}
\newcommand{\eps}{\varepsilon}

% Highlight a newly defined term
\newcommand{\newterm}[1]{\textit{#1}}
\newcommand{\red}[1]{{\leavevmode\color{red}#1}}
\newcommand{\blue}[1]{{\leavevmode\color{blue}#1}}
\newcommand{\nblue}[1]{{\leavevmode\color{nblue}#1}}
\newcommand{\violet}[1]{{\leavevmode\color{violet}#1}}
\newcommand{\purple}[1]{{\leavevmode\color{purple}#1}}
\newcommand{\ngreen}[1]{{\leavevmode\color{ngreen}#1}}

% macros for dynamic pooling
\newcommand{\Njob}{n}
\newcommand{\type}{\theta}
\newcommand{\typespace}{\Theta}
\newcommand{\instance}{\vtheta}
\newcommand{\sojourn}{d}
\newcommand{\potential}{p}
\newcommand{\ALG}{\mathsf{ALG}}
\newcommand{\OPT}{\mathsf{OPT}}
\newcommand{\OFF}{\mathsf{OFF}}
\newcommand{\regret}{\mathsf{Reg}}
\newcommand{\gap}{\mathsf{GAP}}
\newcommand{\gre}{\mathsf{GRE}}
\newcommand{\PB}{\mathsf{PB}}
\newcommand{\batching}{\mathsf{BAT}}
\newcommand{\rbatching}{\mathsf{R-BAT}}
\newcommand{\pot}{\mathsf{POT}}
\newcommand{\Potloss}{\mathsf{Loss}}
\newcommand{\LP}{\mathsf{LP}}
\newcommand{\DualLP}{\mathsf{DLP}}
\newcommand{\matchset}{\mathcal{M}}
\newcommand{\buffer}{A}
\newcommand{\indexf}{q}
\newcommand{\noutput}{m}
\newcommand{\order}[1]{(#1)}
\newcommand{\depth}{\ell}
\newcommand{\matchof}{m}
\newcommand{\ltype}{\alpha}
\newcommand{\rtype}{\beta}
\newcommand{\jobset}{J}
\newcommand{\width}{w}
\newcommand{\reward}{r}
\newcommand{\supI}{\mathrm{A}}
\newcommand{\supII}{\mathrm{B}}
\newcommand{\supIII}{\mathrm{C}}
\newcommand{\basecase}{\mathrm{BC}}
% \newcommand{\xorigin}{\vx_{\mathrm{ori}}}
\newcommand{\xorigin}{O}
% \newcommand{\xdestination}{\vx_{\mathrm{dest}}}
\newcommand{\xdestination}{D}
\newcommand{\yorigin}{y_{\mathrm{origin}}}
\newcommand{\ydestination}{y_{\mathrm{destination}}}
\newcommand{\marginalgain}{\mathsf{MG}}
\newcommand{\marginalloss}{\mathsf{ML}}
\newcommand{\randmarginalgain}{\mathsf{RMG}}
\newcommand{\randmarginalloss}{\mathsf{RML}}
\newcommand{\spacing}{S}
\newcommand{\dual}{\mathsf{HD}}
\newcommand{\discmap}{\varphi}
\newcommand{\averagedual}{\mathsf{AD}}
% \newcommand{\unif}{\text{Unif}}

% Natbib setup for author-number style
\usepackage{natbib}
 \bibpunct[, ]{(}{)}{,}{a}{}{,}%
 \def\bibfont{\small}%
 \def\bibsep{\smallskipamount}%
 \def\bibhang{24pt}%
 \def\newblock{\ }%
 \def\BIBand{and}%


%% Setup of the equation numbering system. Outcomment only one.
%% Preferred default is the first option.
\EquationsNumberedThrough    % Default: (1), (2), ...
%\EquationsNumberedBySection % (1.1), (1.2), ...

%% Setup of theorem styles. Outcomment only one.
%% Preferred default is the first option.
\TheoremsNumberedThrough     % Preferred (Theorem 1, Lemma 1, Theorem 2)
%\TheoremsNumberedByChapter  % (Theorem 1.1, Lema 1.1, Theorem 1.2)
\ECRepeatTheorems  %  

% For new submissions, leave this number blank.
% For revisions, input the manuscript number assigned by the on-line
% system along with a suffix ".Rx" where x is the revision number.
\MANUSCRIPTNO{TRSC-0001-2024.00}

%%%%%%%%%%%%%%%%
\begin{document}
%%%%%%%%%%%%%%%%

% Outcomment only when entries are known. Otherwise leave as is and
%   default values will be used.
%\setcounter{page}{1}
%\VOLUME{00}%
%\NO{0}%
%\MONTH{Xxxxx}% (month or a similar seasonal id)
%\YEAR{0000}% e.g., 2005
%\FIRSTPAGE{000}%
%\LASTPAGE{000}%
%\SHORTYEAR{00}% shortened year (two-digit)
%\ISSUE{0000} %
%\LONGFIRSTPAGE{0001} %
%\DOI{10.1287/xxxx.0000.0000}%

% Author's names for the running heads
% Sample depending on the number of authors;
% \RUNAUTHOR{Jones}
% \RUNAUTHOR{Jones and Wilson}
% \RUNAUTHOR{Jones, Miller, and Wilson}
% \RUNAUTHOR{Jones et al.} % for four or more authors
% Enter authors following the given pattern:
%\RUNAUTHOR{}
% \RUNAUTHOR{Smith and Johnson}

\RUNAUTHOR{Ma, Ma, and Romero}





% Title or shortened title suitable for running heads. Sample:
% \RUNTITLE{Predictive Maintenance in Manufacturing}
% Enter the (shortened) title:
% \RUNTITLE{Optimal Resource Allocation in Humanitarian Logistics}
\RUNTITLE{Dynamic Delivery Pooling}

% Full title. Sample:
% \TITLE{Optimal Resource Allocation in Humanitarian Logistics: A Stochastic Programming Approach}
% Enter the full title:
% \TITLE{Optimal Resource Allocation in Humanitarian Logistics: A Stochastic Programming Approach}
\TITLE{Potential-Based Greedy Matching for Dynamic Delivery Pooling} 

% Block of authors and their affiliations starts here:
% NOTE: Authors with same affiliation, if the order of authors allows,
%   should be entered in ONE field, separated by a comma.
%   \EMAIL field can be repeated if more than one author
\ARTICLEAUTHORS{%
%\AUTHOR{John Doe,\textsuperscript{a} Jane Smith,\textsuperscript{b}}
%\AFF{\textsuperscript{a}Department of Industrial Engineering, University of XYZ, \EMAIL{john.doe@xyz.edu; \textsuperscript{b}Department of Computer Science, University of ABC, \EMAIL{jane.smith@abc.edu}} 
% \AUTHOR{John Smith}
% \AFF{Department of Logistics,
% University of XYZ, \EMAIL{john.smith@xyz.edu}}

% \AUTHOR{Emily Johnson}
% \AFF{Department of Logistics,
% University of ABC, \EMAIL{emily.johnson@abc.edu}}

% Enter all authors
\AUTHOR{Hongyao Ma}
\AFF{Graduate School of Business, Columbia University, New York, NY 10027, \EMAIL{hongyao.ma@columbia.edu}}
\AUTHOR{Will Ma}
\AFF{Graduate School of Business, Columbia University, New York, NY 10027, \EMAIL{wm2428@gsb.columbia.edu}}
\AUTHOR{Matias Romero}
\AFF{Graduate School of Business, Columbia University, New York, NY 10027, \EMAIL{mer2262@gsb.columbia.edu}}
} % end of the block

\ABSTRACT{%
We study the problem of pooling together delivery orders into a single trip, a strategy widely adopted by platforms to reduce total travel distance.
% % 
Similar to other dynamic matching settings, the pooling decisions involve a trade-off between immediate reward and holding jobs for potentially better opportunities in the future. 
%
In this paper, we introduce a new heuristic dubbed potential-based greedy ($\PB$), which aims to keep longer-distance jobs in the system, as they have higher potential reward (distance savings) from being pooled with other jobs in the future. 
%
This algorithm is simple in that it depends solely on the topology of the space, and does not rely on forecasts or partial information about future demand arrivals.
%
We prove that $\PB$ significantly improves upon a naive greedy approach in terms of worst-case performance on the line. Moreover, we conduct extensive numerical experiments using both synthetic and real-world order-level data from the Meituan platform. 
%
Our simulations show that $\PB$ consistently outperforms not only the naive greedy heuristic but a number of benchmark algorithms, including (i) batching-based heuristics that are widely used in practice, and (ii) forecast-aware heuristics that are given the correct probability distributions (in synthetic data) or a best-effort forecast (in real data).
%
We attribute the surprising unbeatability of $\PB$ to the fact that it is specialized for rewards defined by distance saved in delivery pooling.
}%

% % \FUNDING{This research was supported by [grant number, funding agency].}

% %Supplemental Material:
% %Data Ethics & Reproducibility Note:

% % Sample
% %\KEYWORDS{Stochastic programming, Decision support,Uncertainty, Disaster response, Optimization}

% % Fill in data. If unknown, outcomment the field
\KEYWORDS{On-demand delivery, Platform operations, Online algorithms, Dynamic matching} 

% %\HISTORY{Received: Month DD, YYYY; Accepted: Month DD, YYYY; Published Online: Month DD, YYYY}

\maketitle
%%%%%%%%%%%%%%%%%%%%%%%%%%%%%%%%%%%%%%%%%%%%%%%%%%%%%%%%%%%%%%%%%%%%%%

\section{Introduction} \label{sec:intro}

% 
On-demand delivery platforms have become an integral part of modern life, transforming how consumers search for, purchase, and receive goods from restaurants and retailers.
%
Collectively, these platforms serve more than $1.5$ billion users worldwide, contributing to a global market valued at over \$250 billion \citep{statista2024}.
%
Growth has continued at a rapid pace--- major companies such as DoorDash in the United States and Meituan in China reported approximately 25\% year-over-year growth in transaction volume in 2023 \citep{curry2024doordash,scmp2024meituan}. 
%
Managing this ever-expanding stream of orders poses significant operational challenges, particularly as consumers demand increasingly faster deliveries, and fierce competition compels platforms to continuously improve both operational efficiency and cost-effectiveness.

%
One strategy for improving efficiency and reducing labor costs is to \emph{pool} into a single trip multiple orders from the same or nearby restaurants.
% 
This is referred to as stacked, grouped, or batched orders, and is advertised to drivers as opportunities for increasing earnings and efficiency~\citep{doordash2024batched}.
% 
The strategy has been generally successful and very widely adopted~\citep{deliveroo2022,uberEatsMultiple,grabGroupedJobs}.
%
For example, data from Meituan, made public by the 2024 INFORMS TSL Data-Driven Research Challenge, show that more than 85\% of orders are pooled, rising to 95\% during peak hours and consistently remaining above 50\% at all other times (see \Cref{fig:meituan_pooled_orders_per_hour}).
%

\begin{figure}[hpbt]
    \centering
    \includegraphics[width=0.9\linewidth]{Simulation_Results/Meituan/City/meituan_fraction_of_pooled_orders_how.png}
    % 
    \caption{Fraction of pooled orders by hour-of-week in Meituan data. 
    The dataset includes $8$ days of order-level data from one city (see \Cref{sec:sim_meituan} for more details). The gray shade indicates peak lunch hours (10:30am-1:30pm), while the green shade indicates peak dinner hours (5pm-8pm). 
    }
    \label{fig:meituan_pooled_orders_per_hour}
\end{figure}

The delivery pooling approach introduces a highly complex decision-making problem, as orders arrive dynamically to the market, and need to be dispatched within a few minutes --- orders in the Meituan data are offered to delivery drivers an average of around five minutes after being placed, often before meal preparation is complete (see \Cref{fig:meituan_order_to_first_dispatch_how}).
% 
%
During this limited window of each order, the platform must determine which (if any) of the available orders to pool with, carefully weighing immediate efficiency gains against the uncertain, differential benefits of holding each candidate for future pooling opportunities.
%
Consequently, there remains a pressing need for more efficient and robust solution methods to inform the platform's delivery pooling decisions.


%
Similar problems have been studied in the context of various marketplaces such as ride-sharing platforms and kidney exchange programs, in a large body of work known as dynamic matching.
%
However, we argue that the reward structure for delivery pooling is fundamentally different. 
% 
First, many problems studied in the literature involves connecting two sides of a market, e.g. ridesharing platforms matching drivers to riders. 
% 
In such bipartite settings, the presence of a large number of agents of identical or similar types (e.g., many riders requesting trips originating from the same area) typically leads to less efficient outcomes.
% 
While the kidney exchange problem is not bipartite, long queues of ``hard-to-match'' patient-donor pairs can still form, when many pairs share the same combination of incompatible tissue types and blood types.

In stark contrast, it is unlikely for long queues to build up for delivery pooling, especially in dense\footnote{
Indeed, today's major platforms observe extremely large volumes. Meituan, for example, processes over 12 thousand 
orders per hour in one market during peak lunch periods (see \Cref{fig:meituan_orders_per_hour}).} markets, in that given enough orders, two of them will have closely situated origins and destinations, resulting in a good match.
%
In fact, many customers requesting deliveries originating from the same location is actually \textit{desirable}, since this reduces travel distance and parking stops.
% 
We find that delivery orders tend to indeed be highly concentrated in space, with most requests originating from popular restaurants in busy city centers and ending in residential neighborhoods (see \Cref{appx:meituan_spatial_distribution}). 
%
We emphasize that we are not studying the problem of matching (pooled) orders to couriers, but delivery pooling is a relevant first step regardless, as pooling orders efficiently before matching with couriers will effectively increase the amount of courier capacity.

In this work, we study delivery pooling and exploit its distinct \emph{reward topology}, that it is highly desirable to match two jobs of identical or very similar types, as described above.
%
We develop a simple "potential-based" greedy algorithm, which relies solely on the reward topology to determine the opportunity cost of dispatching each job.
% 
The intuition behind it is elementary --- long-distance jobs have higher \textit{potential} for cost savings when pooled with other jobs, and hence the online algorithm should generally prefer keeping them in the system, dispatching shorter deliveries first when it has a choice.

\subsection{Model Description}

Different models of arrivals and departures have been proposed in the dynamic matching literature (see \Cref{sec:dynMatchModels}), and we believe our insight about potential is relevant across all of them.
However, in this paper we focus on a single model well-suited for the delivery pooling application.

%
To elaborate, we consider a dynamic non-bipartite matching problem, with a total of $\Njob$ jobs arriving sequentially to be matched.
%
Jobs are characterized by (potentially infinite) types $\theta\in\Theta$ representing features (e.g. locations of origin and destination) that determine the reward for the platform.
%
Given the motivating application to pooled deliveries, we will always model jobs' types as belonging to some metric space.
%
Following \citet{ashlagi2019edge}, we assume that an unmatched job must be dispatched after $\sojourn$ new arrivals, which we interpret as a known \textit{internal} deadline imposed by the platform to incentivize timely service.
%
The platform is allowed to make a last-moment matching decision before the job leaves, termed as the job becoming \newterm{critical}.
%
At this point, the platform must decide either to match the critical job to another available one, collecting reward $r(\theta,\theta')$ where $\theta,\theta'$ are the types of the matched jobs and $r$ is a known reward function, or to dispatch the critical job on its own for zero reward.
%
The objective is to maximize the total reward collected from matching the $\Njob$ jobs.
%
We believe this to be an appropriate model for delivery platforms (cf. \Cref{sec:dynMatchModels}) because deadlines are known upon arrival and sudden departures (order cancellations) are rare.

Dynamic matching models can also be studied under different forms of information about the future arrivals.
%
Some papers \citep[e.g.][]{kerimov2024dynamic,aouad2020dynamic,eom2023batching,wei2023constant} develop sophisticated algorithms to leverage stochastic information, which is often necessary to derive theoretical guarantees under general matching rewards.
%
In contrast, our heuristic does not require any knowledge of future arrivals, and our theoretical results hold in an "adversarial" setting where no stochastic assumptions are made.
%
In our experiments, we consider arrivals generated both from stochastic distributions and real-world data, and find that our heuristic can outperform even algorithms that are given the correct stochastic distributions, under our specific reward topology.

\subsection{Main Contributions}

\paragraph{Notion of potential.}
% 
As mentioned above, we design a simple greedy-like algorithm that is based purely on topology and reward structure, which we term potential-based greedy ($\PB$). More precisely, $\PB$ defines the \newterm{potential} of a job type $\theta$ to be $p(\theta)=\sup_{\theta'\in\Theta} r(\theta,\theta')/2$, measuring the highest-possible reward obtainable from matching type $\theta$.  The potential acts as an opportunity cost, and the relevance of this notion arises in settings where the potential is heterogeneous across jobs. To illustrate, let $\Theta=[0,1]$ and $r(\theta,\theta')=\min\{\theta,\theta'\}$, a reward function used for delivery pooling as we will justify in \Cref{sec:model}.  Under this reward function, a job type (which is a real number) can never be matched for reward greater than its real value, which means that jobs with higher real value have greater potential.  Our $\PB$ algorithm matches a critical job (with type $\theta$) to the available job (with type $\theta'$) that maximizes
\begin{align} \label{eqn:potentialIntro}
r(\theta,\theta')-p(\theta')=\min\{\theta,\theta'\}-\frac12 \theta',
\end{align}
being dissuaded to use up job types $\theta'$ with high real values. Notably, this definition does not require any forecast or partial information of future arrivals, but rather assumes full knowledge of the universe of possible job types.

\paragraph{Theoretical results.}
Our theoretical results assume $\Theta=[0,1]$.  We compare $\PB$ to the naive greedy algorithm $\gre$, which selects $\theta'$ to maximize $r(\theta,\theta')$, instead of $r(\theta,\theta')-p(\theta')$ as in~\eqref{eqn:potentialIntro}.  We first consider an offline setting ($d=\infty$), showing that under reward function $r(\theta,\theta')=\min\{\theta,\theta'\}$, our algorithm $\PB$ achieves regret $O(\log n)$, whereas $\gre$ suffers regret $\Omega(n)$ compared to the optimal matching.
Our analysis of $\PB$ is tight, i.e.\ it has $\Theta(\log n)$ regret.
Building upon the offline analysis, we next consider the online setting ($d< n$), showing that $\PB$ has regret $\Theta(\frac nd\log d)$, which improves as $d$ increases. This can be interpreted as there being $\frac nd$ "batches" in the online setting, and our algorithm achieving a regret of $O(\log d)$ per batch.  Alternatively, it can be interpreted as the \newterm{regret per job} of our algorithm being $O(\frac d{\log d})$. By contrast, we show that the naive greedy algorithm $\gre$ suffers a total regret of $\Omega(n)$, regardless of $d$.

For comparison, we also analyze two other reward topologies, still assuming $\Theta=[0,1]$.
\begin{enumerate}
\item We consider the classical min-cost matching setting \citep{reingold1981greedy} where the goal is to minimize total match distance between points on a line, represented in our model by the reward function $r(\theta,\theta')=1-|\theta-\theta'|$.
%
For this reward function, both $\PB$ and $\gre$ have regret $\Theta(\log n)$; in fact, they are the same algorithm because all job types have the same potential. Like before, this translates into both algorithms having regret $\Theta(\frac nd \log d)$ in the online setting.
\item We consider reward function $r(\theta,\theta')=|\theta-\theta'|$, representing an opposite setting in which it is worst to match two jobs of the same type.  For this reward function, we show that any index-based matching policy must suffer regret $\Omega(n)$ in the offline setting, and that both $\PB$ and $\gre$ suffer regret $\Omega(n)$ (irrespective of $d$) in the online setting.
\end{enumerate}

Our theoretical results are summarized in \Cref{table:results}.
As our model is a special case of \citet{ashlagi2019edge}, their 1/4-competitive randomized online edge-weighted matching algorithm can be applied, which essentially translates in our setting to an $O(n)$ upper bound for regret under any definition of reward.
Our results show that it is possible to do much better ($O(\log n)$ instead of $O(n)$) for specific reward topologies, using a completely different algorithm and analysis.
We now outline how to prove our two main technical results, \Cref{thm:potential_log_upper_bound,thm: dynamic}.
%
\begin{table}[!t]
\centering
\begin{tabular}{|c|c|c|c|}
\hline
\updown $\theta,\theta'\in[0,1]$ & $r(\theta,\theta')=\min\{\theta,\theta'\}$ & $r(\theta,\theta')=1-|\theta-\theta'|$ & $r(\theta,\theta')=|\theta-\theta'|$ \\
\hline
\up\multirow{4}{*}{Regret of $\PB$} & Offline: $\Theta(\log n)$ & & \\
\down & (\Cref{thm:potential_log_upper_bound}, \Cref{prop:loglowerboundOffline}) & & \\
\up & Online: $\Theta(\frac nd \log d)$ & Offline: $\Theta(\log n)$ & Offline: $\Omega(n)$ \\
\down & (\Cref{thm: dynamic}, \Cref{prop:loglowerboundOnline}) & Online: $\Theta(\frac nd \log d)$ & Online: $\Omega(n)$ \\
\cline{1-2}
\up\multirow{4}{*}{Regret of $\gre$} & Offline: $\Omega(n)$ & (\Cref{sec:reward2}) & (\Cref{sec: reward 3}) \\
\down & (\Cref{prop:greedy_linear_lower_bound}) & & \\
\up & Online: $\Omega(n)$ & & \\
\down & (\Cref{prop:greedy_linear_lower_bound_dynamic}) & & \\
\hline
\end{tabular}
\caption{
Summary of theoretical results under different reward functions $r:[0,1]^2\to \R$.  The total number of jobs is denoted by $n$, and the batch size in the online setting is approximately $d$.
}
\label{table:results}
\end{table}
\paragraph{Proof techniques.}
%
We establish an upper bound on the regret of $\PB$ by comparing its performance to the sum of the potential of all jobs.
%
Under reward function $\reward(\type,\type')=\min\{\type,\type'\}$, the key driver of regret is the sum of the distances between jobs matched by $\PB$.
%
In \Cref{thm:potential_log_upper_bound}, we study this quantity by analyzing the intervals induced by matched jobs in the offline setting ($\sojourn=\infty$).
%
We show that the intervals formed by the matching output of $\PB$ constitute a \emph{laminar set family}; that is, every two intervals are either disjoint or one fully contains the other.
%
We then prove that more deeply nested intervals must be exponentially smaller in size. Finally, we use an LP to show that the sum of interval lengths remains bounded by a logarithmic function of the number of intervals.
%

In \Cref{thm: dynamic}, we partition the set of jobs in the online setting into roughly $\frac{\Njob}{d}$ "batches" and analyze the sum of the distances between matched jobs within a single batch.
%
Our main result is to show that we can use \Cref{thm:potential_log_upper_bound} to derive an upper bound for an arbitrary batch.
%
However, a direct application of the theorem on the offline instance defined by the batch would not yield a valid upper bound, because the resulting matching of $\PB$ in the offline setting could be inconsistent with its online matching decisions.
%
To overcome this challenge, we carefully construct a modified offline instance that allows for the online and offline decisions to be coupled, while ensuring that the total matching distance did not go down.  This allows us to upper-bound the regret per batch by $O(\log d)$, for a total regret of $O(\frac nd \log d)$.


\paragraph{Simulations on synthetic data.} We test $\PB$ on random instances in the setting of our theoretical results, except that job types are drawn uniformly at random from $[0,1]$ (instead of adversarial). 
%
We benchmark its performance against $\gre$, as well as more sophisticated algorithms, including (i) batching-based heuristics that are highly relevant both in theory and practice, and (ii) forecast-aware heuristics that use historical data to compute shadow prices.
%
Our extensive simulation results show that $\PB$ consistently outperforms all benchmarks starting from relatively low market densities, achieving over 95\% of the hindsight optimal solution that has full knowledge of arrivals.
%
This result is robust to different distributions of job types, and also two-dimensional locations.


\paragraph{Simulations on real data.}

Finally, we test the practical applicability of $\PB$ via extensive numerical experiments using order-level data from the Meituan platform, made available from the 2024 INFORMS TSL Data-Driven Research Challenge.\footnote{\url{https://connect.informs.org/tsl/tslresources/datachallenge}, accessed January 15, 2025.}
%
The key information we extract from this dataset is the exact timestamps for the creation of each request, and the geographic coordinates of pick-up and drop-off locations.
%
These aspects differ from our theoretical setting in that (i) requests may not be available to be pooled for a fixed number of new arrivals before being dispatched, and (ii) locations are two-dimensional with delivery orders having heterogeneous origins.
%
To address (i), we assume that the platform sets a fixed time window after which an order becomes critical, corresponding to each job being able to wait for at most a fixed sojourn time before being dispatched.
%
For (ii), we extend our definition of reward (that captures the travel distance saved) to two-dimensional heterogeneous origins.
%
Under this new reward definition, the main insight from our theoretical model remains true: longer deliveries have higher potential reward from being pooled with other jobs in the future, and thus $\PB$ aims to keep them in the system.
%

In contrast to our simulations with synthetic data, the real-life delivery locations may now be correlated and exhibit time-varying effects.
%
Regardless, we find that $\PB$ outperforms all tested heuristics (those without foreknowledge of the future), given that the platform is willing to wait up to one minute before dispatching each job, a fairly modest ask.
%
In this regime, $\PB$ achieves over 80\% of the maximum possible travel distance saved from pooling. $\PB$ also pools 10\% more jobs than the hindsight optimal matching (see \Cref{sec:match_rate}).
%


\paragraph{Explanation for the surprising unbeatability of $\PB$.}
Our potential-based greedy heuristic consistently performs at or near the best across a wide range of experimental setups, including both synthetic and real data.
%
This result is surprising to us, given that $\PB$ is a simple index-based rule that ignores forecast information.
%
To provide some explanation for this finding, we analyze a stylized setting in \Cref{sec: interpretation}, where we show that the "correct" shadow prices converge to our notion of potential as the market thickness increases.
%
That being said, we also end with a couple of caveats.
%
First, our findings about the effectiveness of $\PB$ are specific to our definition of matching reward for delivery pooling, based on travel distance saved, and may not extend to setting with different reward structures.
%
Second, while we carefully tuned the batching-based and dual-based heuristics that we compare against, it remains possible that more sophisticated algorithms, particularly those leveraging dynamic programming in stochastic settings, could outperform $\PB$.


\subsection{Further Related Work}

\subsubsection{Dynamic matching.} \label{sec:dynMatchModels}

Dynamic matching problems have received growing attention from different communities in economics, computer science, and operations research. We discuss different ways of modeling the trade-off between matching now vs.\ waiting for better matches, depending on the application that motivates the study. 


One possible model \citep{kerimov2024dynamic,kerimov2023optimality,wei2023constant} is to consider a setting with jobs that arrive stochastically in discrete time and remain in the market indefinitely, but use a notion of \newterm{all-time regret} that evaluates a matching policy, at every time period, against the best possible decisions until that moment, to disincentivize algorithms from trivially delaying until the end to make all matches.
%

A second possible model, motivated by the risk of cancellation in ride-sharing platforms, is to consider sudden departures modeled by jobs having heterogeneous \textit{sojourn times} representing the maximum time that they stay in the market \citep{aouad2020dynamic}. These sojourn times are unknown to the platform, and delaying too long risks many jobs being lost without a chance of being matched.
%
\citet{aouad2020dynamic} formulates an MDP with jobs that arrive stochastically in continuous time, and leave the system after an exponentially distributed sojourn time. They propose a policy that achieves a multiplicative factor of the hindsight optimal solution. 
%
Related work on the control of matching queues with abandonment includes \citet{collina2020dynamic}, \citet{castro2020matching}, \citet{wang2024demand}, \citet{kohlenberg2024cost}.

%
A third possible model \citep{huang2018match} also considers jobs that can leave the system at any period, but allows the platform to make a last-moment matching decision right before a job leaves, termed as the job becoming \newterm{critical}.  In this model, one can without loss assume that all matching decisions are made at times that jobs become critical.

Finally, the model we study also makes all decisions at times that jobs become critical, with the difference being that the sojourn times are known upon the arrival of a job, as studied in \citet{ashlagi2019edge, eom2023batching}. Under these assumptions, \citet{eom2023batching} assume stationary stochastic arrivals, while \citet{ashlagi2019edge} allow for arbitrary arrivals.
%
Our research focuses on deterministic greedy-like algorithms that can achieve good performance as the market thickness increases.
%
Moreover, our work differs from these papers in the description of the matching value. While they assume arbitrary matching rewards, we consider specific reward functions known to the platform in advance, and use this information to derive a simple greedy-like algorithm with good performance.
% 
This aligns with a stream of literature on online matching, that captures more specific features into the model to get stronger guarantees \citep[see][]{kanoria2021dynamic,chen2023feature,balkanski2023power}.


We should note that our paper also relates to a recent stream of work studying the effects of batching and delayed decisions in online matching \citep[e.g.][]{feng2024batching,xie2023benefits}, as well as works studying the relationship between market thickness and quality of online matches \citep[e.g.][]{ashlagi2021kidney,chen2021matchmaking}.


\subsubsection{Delivery operations.}

On-demand delivery operations have received special attention in the field of transportation and operations management. 
%
\citet{reyes2018meal} introduce the meal delivery routing problem, which falls in the class of dynamic vehicle routing problems (see e.g. \citet{psaraftis2016dynamic}); the authors develop heuristics to dynamically assign orders to vehicles. Similar efforts have been devoted to optimize detailed pooling and assignment strategies as customer orders arrive sequentially \citep{steever2019dynamic,ulmer2021restaurant}, mainly using approximate dynamic programming techniques.
%
These heuristics are shown to work well in extensive numerical experiments, but given the intricate nature of the model and techniques, it is very challenging to derive managerial insights or theoretical performance guarantees.
%
Closer to our research goal, \citet{chen2024courier} analyze the optimal dispatching policy on a stylized queueing model representing a disk service area centered at one restaurant. They show that delivering multiple orders per trip is beneficial when the service area is large.
%
\citet{cachon2023fast} study the interplay between the number of couriers and platform efficiency, assuming a one-dimensional geography with one single origin, which is also the primary model in our theoretical results.
%
A similar topology is studied in the game-theoretic model of \citet{keskin2024order} to capture the impact of delivery pooling on the interaction between the platform, customers, riders, and restaurants. 

\section{Preliminaries}
\label{sec:model}
% 
In this section, we introduce a dynamic non-bipartite matching model for the delivery pooling problem. A total of $\Njob$ jobs arrive to the platform sequentially.
% 
Each job $j \in [\Njob] = \{1, \ldots, \Njob\}$, indexed in the order of arrival, has a \emph{type} $\type_j \in \typespace$. 
% 
We adopt the criticality assumption from \citet{ashlagi2019edge}, in the sense that each job remains available to be matched for $\sojourn\ge 1$ new arrivals, after which it becomes \textit{critical}. The parameter $\sojourn$ can also be interpreted as the market density.
%
When a job becomes critical, the platform decides whether to (irrevocably) match it with another available job, in which case they are dispatched together (i.e. \emph{pooled})
and the platform collects a \emph{reward} given by a known function $\reward:\typespace^2\to\R$. If a critical job is not pooled with another job, it has to be dispatched by itself for zero reward.


\subsection{Reward Topology}
% 
Our modeling approach directly imposes structure on the type space, as well as the reward function that captures the benefits from pooling delivery orders together. Similar to previous numerical work on pooled trips in ride-sharing platforms \citep{eom2023batching, aouad2020dynamic}, we assume that the platform's goal is to reduce the total distance that needs to be traveled to complete all deliveries.
The reward of pooling two orders together is therefore the travel distance saved when they are delivered by the same driver in comparison to delivered separately. 
% 
Formally, we consider a \newterm{linear city model} where job types $\type \in \typespace = [0,1]$ represent destinations of the delivery orders, and assume that all orders need to be served from the origin 0 (similar to that analyzed in \citet{cachon2023fast})
% 
If a job of type $\type$ is dispatched by itself, the total travel distance from the origin 0 is exactly $\type$. If it is matched with another job of type $\type'$, they are pooled together on a single trip to the farthest destination $\max\{\type,\type'\}$. Thus, the distance saved by pooling is 
%
%
%
\begin{equation}
  \reward(\type,\type') = \type + \type' - \max\{\type,\type'\} = \min\{\type,\type'\}.
  \label{eq:defn_reward_A} 
\end{equation}

Although our main theoretical results leverage the structure of this reward topology, our proposed algorithm can be applied to other reward topologies. 
% 
In particular, we derive theoretical results for two other reward structures for $\typespace = [0,1]$. First, we consider 
% 
\begin{equation}
    \reward(\type,\type') = 1 - |\type - \type'|  \label{eq:defn_reward_B}
\end{equation}
% 
to capture the commonly-studied spatial matching setting~\citep[e.g.][]{kanoria2021dynamic,balkanski2023power}, where the reward is larger if the distance $|\theta-\theta'|$ is smaller. 
% 
We also consider 
\begin{equation}
    \reward(\type,\type') = |\type - \type'| \label{eq:defn_reward_C}
\end{equation}
% 
with the goal of matching types that are far away from each other, contrasting the other two reward functions.
% 
In addition, we perform numerical experiments for delivery pooling in two-dimensional (2D) space, where the reward function corresponds to the travel distance saved in the 2D setting.


\subsection{Benchmark Algorithms}

% 
Given market density $\sojourn$ and reward function $\reward$, if the platform had full information of the sequence of arrivals $\instance \in \typespace^\Njob$, the hindsight optimal $\OPT(\instance,\sojourn)$ can be computed by the integer program (IP) defined in \eqref{eq: OPT}. We denote the hindsight optimal matching solution as $\matchset_{\OPT} = \{(j,k):x_{j,k}^{\ast}=1\}$, where $x^\ast$ is an optimal solution of \eqref{eq: OPT}.
% 
\begin{maxi}
    {x}{ \sum_{j,k : j\neq k, |j-k|\le \sojourn} x_{jk} \reward(\type_j,\type_k)}
    {\label{eq: OPT}}{\OPT(\instance,\sojourn) =}
    \addConstraint{ \sum_{k:j\neq k} x_{jk}}{\le 1,}{j\in [\Njob]}
    \addConstraint{ x_{jk}}{\in\{0,1\},}{j,k\in [\Njob], j\neq k.}
\end{maxi}
%

An online matching algorithm operates over an instance $\instance\in\typespace^\Njob$ sequentially: when job $j\in [\Njob]$ becomes critical, the types of future arrivals $k > j + d$ are unknown, and any matching decision has to be made based on the currently available information.
% 
Given an algorithm $\ALG$, we denote by $\ALG(\instance,\sojourn)$ the total reward collected by an algorithm on such instance.
%
We analyze the performance of algorithms via \emph{regret}, as follows:
%
\[
    \regret_\ALG(\instance,\sojourn) = \OPT(\instance,\sojourn)-\ALG(\instance,\sojourn).
\]

We study a class of online matching algorithms that we call \newterm{index-based greedy matching algorithms}.
%
Each index-based greedy matching algorithm is specified by an \emph{index function} $\indexf:\typespace^2\to\R$, and only makes matching decisions when some job becomes critical (in our model, it is without loss of optimality to wait to match).
%
A general pseudocode is provided in \Cref{alg:dynamic}.
%
When a job $j \in [\Njob]$ becomes critical, the algorithm observes the set of available jobs in the system $A(j) \subseteq [\Njob]$ that have arrived but are not yet matched (this is the set $\buffer\setminus\{j\}$ in \Cref{alg:dynamic}), and chooses a match $\matchof(j) \in A(j)$ that maximizes the index function (even if negative), collecting a reward $\reward(\type_j,\type_{\matchof(j)})$. If there are multiple jobs that achieve the maximum, the algorithm breaks the tie arbitrarily.
%
The algorithm makes a set of matches $\matchset = \{ (j,m(j)) : j \in C \}$, where $C$ denotes the set of jobs that are matched when they become critical, and its total reward is
\[ \ALG(\instance,\sojourn) = \sum_{j\in C} \reward(\type_j,\type_{\matchof(j)}). \]
%
Note that $\matchof(j)$ is undefined if $j$ is either unmatched, or was not critical at the time it was matched.


\begin{algorithm}
\caption{Index-based Greedy Matching Algorithm}\label{alg:dynamic}
% 
\begin{algorithmic}[1]
\Require Instance $\instance$, density $\sojourn$, index function $\indexf$, reward function $\reward$
% 
\Ensure $C$, $m:C\to[n]$
% 
\State Initialize set of available jobs $\buffer=\emptyset$, and jobs that are critical when matched $C=\emptyset$
\For{job $t=1,\ldots,\Njob+\sojourn+1$}
    \If{$t\le\Njob$}
        \State $\buffer=\buffer\cup \{t\}$ \Comment{job $t$ arrives}
    \EndIf         
    \If {$t-d\in \buffer$} 
        \State $j=t-d$ \Comment{job $j=t-d$ becomes critical}
        \If{$\buffer \setminus\{j\} \neq\emptyset$}
            \State Choose $m(j) \in \argmax_{k \in \buffer \setminus\{j\} } \indexf(\type_{j}, \type_{k})$ \Comment{Ties are broken arbitrarily}
            \State $C\gets C\cup\{j\}$
            \State $\buffer \gets \buffer\setminus \{j,\matchof(j)\} $ \Comment{jobs $j,\matchof(j)$ are dispatched together}
        \Else
            \State $\buffer \gets \buffer\setminus \{j\} $ \Comment{job $j$ is dispatched by itself}
        \EndIf
    \EndIf
\EndFor
\State\Return $C, \{\matchof(j)\}_{j\in C}$
\end{algorithmic}
\end{algorithm}

As an example, the \newterm{naive greedy} algorithm ($\gre$) uses index function $\indexf_{\gre}(\type,\type') = \reward(\type,\type')$. When any job becomes critical, the algorithm chooses a match it with an available job to maximize the (instant) reward.
%
To illustrate the behavior of this algorithm under reward function $\reward(\type,\type') = \min\{\type,\type'\}$, we first make the following observation, that the algorithm always chooses to match each critical job with a higher type job when possible. 
%
\begin{remark}\label{deliverygreedy}
    When a job $j \in [\Njob]$ becomes critical, if the set $A_+(j) = \{k\in A(j):\type_k \ge \type_j\}$ is nonempty, then under the naive greedy algorithm $\gre$, we have $\matchof(j)\in A_+(j)$ and $\reward(\type_j,\type_{\matchof(j)}) = \type_j $. 
    % 
\end{remark}



\section{Potential-Based Greedy Algorithm}
% 
\label{sec:PB} 

We introduce in this section the potential-based greedy algorithm and prove that it substantially outperforms the naive greedy approach in terms of worst case regret.

The \newterm{potential-based greedy} algorithm ($\PB$) is an index-based greedy matching algorithm (as defined in \Cref{alg:dynamic}) whose index function is specified as
\begin{equation}
    \indexf_{\PB}(\type,\type') = \reward(\type, \type') - \potential(\type'),
\end{equation}
where $\potential(\type)$ is the \newterm{potential} of a job of type $\type$, formally defined as follows
%
\begin{equation}
    \potential(\type)=\frac{1}{2}\sup_{\type'\in\typespace} \reward(\type,\type'). \label{eq:defn_potential}
\end{equation} 
% 

This new notion of potential can be interpreted as an optimistic measure of the marginal value of holding on to a job that could be later matched with a job that maximizes instant reward.
%
Intuitively, this "ideal" matching outcome appears as the optimal matching solution when the density of the market is arbitrarily large. We provide a formal statement of this intuition in \Cref{sec: interpretation}.
%
Note that this definition of potential is quite simple in the sense that it relies on only in the knowledge of the reward topology: reward function and type space.
% 
In particular, for our topology of interest, $\reward(\type,\type')=\min\{\type,\type'\}$ on $\typespace=[0,1]$, the ideal scenario is achieved when two identical jobs are matched and the potential $\potential(\type)=\type/2$ is simply proportional to the length of a solo trip, capturing that longer deliveries have higher potential reward from being pooled with other jobs in the future.

We begin by giving a straightforward interpretation for this algorithm under the linear city model and our reward of interest, as we did for the naive greedy algorithm $\gre$ in \Cref{deliverygreedy}.

\begin{remark}\label{deliverypotential}
    Under the 1-dimensional type space $\Theta = [0,1]$ and the reward function $\reward(\type,\type')=\min\{\type,\type'\}$, 
    % 
    the potential-based greedy algorithm $\PB$ always matches each critical job to an available job that's the closest in space, since 
    % 
    \begin{align*}
        \matchof(j) 
        \in \argmax_{k\in A(j) } \indexf_\PB(\type_j,\type_k) 
        = \argmin_{k\in A(j) } |\type_j-\type_k|. 
    \end{align*}    % 
    This follows from the fact that $ |\type_j-\type_k| = \type_j + \type_k - 2\min\{\type_j, \type_k\} = \type_j - 2\indexf_\PB(\type_j,\type_k) $.
\end{remark}

In the rest of this section, we first assume $d = \infty$ and study the \textit{offline} performance of various index-based greedy matching algorithms under the reward function $\reward(\type,\type') = \min\{\type,\type'\}$.
% 
This will later help us analyze the algorithms' \emph{online} performances when $d<\infty$.
Results for the two alternative reward structures defined in \eqref{eq:defn_reward_B} and \eqref{eq:defn_reward_C} are reported in \Cref{sec:reward2} and  \Cref{sec: reward 3}, respectively.
%


\subsection{Offline Performance under Reward Function $\reward(\type,\type') = \min\{\type,\type'\}$}\label{sec: static}

Suppose $\sojourn=\infty$, meaning that all jobs are avilable to be matched when the first job becomes critical. We study the \emph{offline} performance of both the naive greedy algorithm $\gre$ and potential-based greedy algorithm. We show that the regret of $\gre$ grows linearly with the number of jobs, while the regret of $\PB$ is logarithmic.


\begin{proposition}[proof in \Cref{pf:greedy_linear_lower_bound}]\label{prop:greedy_linear_lower_bound}
Under reward function $\reward(\type,\type') = \min\{\type,\type'\}$, when the number of jobs $\Njob$ is divisible by 4, there exists an instance $\instance \in [0,1]^\Njob$ for which $\regret_\gre(\instance,\infty) \ge n/4$.
\end{proposition}
%

In particular, the \newterm{regret per job} of $\gre$, i.e.\ dividing the regret by $\Njob$, is constant in $\Njob$.
%
In contrast, we prove in the following theorem that $\PB$ performs substantially better. In fact, its regret per job gets better for larger market sizes $\Njob$.

\begin{theorem}
\label{thm:potential_log_upper_bound}
    Under reward function $\reward(\type,\type') = \min\{\type,\type'\}$, we have $\regret_\PB(\instance,\infty) \le 1 + \log_2(\Njob/2+1)/2$, for any $\Njob$, and any instance $\instance \in [0,1]^\Njob$.
\end{theorem}
%
\proof{Proof.}
Let $\instance \in \typespace^\Njob$, and $(C,\matchof(\cdot))$ be the output of $\PB$ on $\instance$ (generated as in \Cref{alg:dynamic}). In particular, 
$\PB(\instance,\infty) = \sum_{j \in C} \reward(\type_j,\type_{\matchof(j)})$.
On the other hand, since $\reward(\type,\type') \le \potential(\type) + \potential(\type')$ for all $\type,\type'
\in\typespace$, we have
\begin{align}
    \OPT(\instance,\infty) = \sum_{(j,k) \in \matchset_\OPT} \reward(\type_j,\type_k) \le \sum_{j=1}^\Njob \potential(\type_j) \le \sup_{\type\in[0,1]}\potential(\type) + \sum_{j\in C} \potential(\type_j) + \potential(\type_{\matchof(j)}),\nonumber
\end{align}
where the last inequality comes from the fact the $\PB$, and in fact any index-based matching algorithm, leaves at most one job unmatched (when $\Njob$ is odd).
%
Hence,
\begin{align}
\regret_\PB(\instance,\infty)&=
\OPT(\instance,\infty)-\PB(\instance,\infty) \\
&\le \sup_{\type\in \typespace}\potential(\type) + \sum_{j\in C} \left( \potential(\type_j) + \potential(\type_{\matchof(j)}) - \reward(\type_j,\type_{\matchof(j)})\right) \nonumber\\
&= \frac12 + \sum_{j \in C} \frac{\type_j}{2} + \frac{\type_{\matchof(j)}}{2} - \min\{\type_j,\type_{m(j)}\} \nonumber\\
&= \frac12 + \sum_{j \in C} \frac{|\type_j - \type_{\matchof(j)}|}{2}\label{eq: offline_regret_bound}. 
\end{align}
%
To bound the right-hand side, we first prove the following. For each $j\in C$, consider the interval $I_j=( \min\{\type_j,\type_{\matchof(j)}\},\max\{\type_j,\type_{\matchof(j)}\})\subseteq[0,1]$ and $\depth(j)=|\{k : I_j \subseteq I_k\}|$. We argue that
\begin{equation}\label{eq: distancebound}
    |\type_j - \type_{\matchof(j)}| \le 2^{1-\depth(j)}.
\end{equation}
%

First, note that $\{I_j\}_{j\in C}$ is a \newterm{laminar set family}: for every $j,k\in C$, the intersection of $I_j$ and $I_k$ is either empty, or equals $I_j$, or equals $I_k$. Indeed, without loss of generality, assume $j<k$. Then, when job $j$ becomes critical we have $\matchof(j),k,\matchof(k) \in A(j) $. Therefore, from \Cref{deliverypotential}, $\matchof(j)$ is the closest to $j$ and thus $k,\matchof(k)\notin I_j$, i.e.\ either $I_j \cap I_k = \emptyset$, or $I_j \cap I_k = I_j$.
%
Moreover, if $I_j\subsetneq I_k$ (i.e. $I_j\subsetneq I_k$ and $I_j\neq I_k$), then necessarily $j < k$, since otherwise $k$ becomes critical first and having $j,\matchof(j) \in A(k)$ closer in space, $\PB$ would not have chosen $\matchof(k)$.
%
Now, we can prove \eqref{eq: distancebound} by induction. If $\depth(j)=1=|\{j\}|$, clearly $|\type_j-\type_{\matchof(j)}|\le 1$. Suppose that the statement is true for $\depth(j)=\depth \ge 1$. If $\depth(j)=\depth+1$ then there exists $k$ such that $I_j\subsetneq I_k$ and $\depth(k)=\depth$. By the previous property, $j<k$ and since when job $j$ becomes critical we had $k,\matchof(k)\in A(j)$, it must be the case that
\begin{align*}
    |\type_j-\type_{\matchof(j)}| 
    &<\max\{ \min\{\type_j,\type_{m(j)}\} - \min\{\type_k,\type_{m(k)}\},\max\{\type_k,\type_{m(k)}\}-\max\{\type_j,\type_{m(j)}\}\}\\
    &\le \min\{\type_j,\type_{m(j)}\} - \min\{\type_k,\type_{m(k)}\} + \max\{\type_k,\type_{m(k)}\} - \max\{\type_j,\type_{m(j)}\} \\
    &= |\type_k-\type_{\matchof(k)}| - |\type_j-\type_{\matchof(j)}|.
\end{align*}
Thus, $|\type_j-\type_{\matchof(j)}|\le|\type_k-\type_{\matchof(k)}|/2\le 2^{1-(\depth+1)}$,
completing the induction.
%
    
Having established \eqref{eq: distancebound}, we use it to derive an upper bound for $\sum_{j\in C}|\type_j-\type_{\matchof(j)}|$. Note that since $\depth(j)\in \{1,\ldots,\floor{\Njob/2}\}$, we have
\begin{align} \label{eq: distanceboundbylayer}
\sum_{j \in C} |\type_j - \type_{\matchof(j)}| = \sum_{\depth = 1}^{\floor{\Njob/2}}\sum_{j\in C:\depth(j) = \depth} |\type_j - \type_{\matchof(j)}|.
\end{align}
%
For every $\depth$, we have $\sum_{j\in C:\depth(j) = \depth} |\type_j - \type_{\matchof(j)}|\le \min\{1,2^{1-\depth}|\{j:\depth(j)=\depth\}|\}$ by the laminar property and \eqref{eq: distancebound}.
%
Let $z_\depth$ denote $2^{1-\depth}|\{j:\depth(j)=\depth\}|$, where we note that 
\begin{align*}\label{eq: numofintervalsbound}
    \sum_{\depth=1}^{\floor{\Njob/2}} 2^{\depth-1}z_\depth = \sum_{\depth=1}^{\floor{\Njob/2}} |\{j:\depth(j)=\depth\}| = |C| \le \Njob/2.
\end{align*}
We can then use the following LP to upper-bound the value of~\eqref{eq: distanceboundbylayer} under an adversarial choice $\{z_\depth:\depth=1,\ldots,\lfloor n/2\rfloor\}$:
\begin{maxi}
    {z}{ \sum_{\depth=1}^{\floor{\Njob/2}} z_\depth }
    {\label{eq: knapsack}}{}
    \addConstraint{ \sum_{\depth=1}^{\floor{\Njob/2}} 2^{\depth-1}z_\depth }{\le \Njob/2}
    \addConstraint{0\le z_\depth }{\le 1,\quad }{ \depth = 1,\ldots,\floor{\Njob/2}.}
\end{maxi}
This is a fractional knapsack problem, whose optimal value is at most $\log_2({\Njob/2+1})$, and thus
\begin{equation}\label{dist_bound}
    \sum_{j \in C} |\type_j - \type_{\matchof(j)}| \le \log_2(\Njob/2+1).
\end{equation}
%
Substituting back into \eqref{eq: offline_regret_bound} yields $\regret_\PB(\instance,\infty) \le 1/2 + \log_2(\Njob/2+1)/2$, completing the proof.
\Halmos\endproof

The following result shows that this analysis of $\PB$ is tight, up to constants.

\begin{proposition}[proof in \Cref{pf:loglowerboundOffline}]
\label{prop:loglowerboundOffline}
Under reward function $\reward(\type,\type') = \min\{\type,\type'\}$, when the number of jobs is $\Njob=2^{k+3} - 4$ for some integer $k\ge 0$, there exists an instance $\instance \in [0,1]^\Njob$ for which $\regret_\PB(\instance,\infty) \ge (\log_2(n+4)-3)/4$.
\end{proposition}
%

%%%%%%%%%%%%%%%%%%%%%%%%%%%%%%%%%%%%%%%%%%%%%%%%%%%%%%%%

\subsection{Online Performance under Reward Function $\reward(\type,\type') = \min\{\type,\type'\}$}\label{sec:dynamic}

We now study the performance of algorithms in the online setting, i.e. when $\sojourn<\Njob$, and derive informative performance guarantees conditional on $\sojourn$, the number of new arrivals before a job becomes critical.
%
In fact, the regret per job of $\gre$ remains constant, whereas the one of $\PB$ decreases with $\sojourn$.
%
To build upon our results proved in \Cref{sec: static} for the offline setting, we partition the set of jobs into $b=\lceil\Njob/(\sojourn + 1) \rceil$ \newterm{batches}.
%
To be precise, we define the $t$-th batch as $B_{t} = \{(t-1)(\sojourn+1)+1,\ldots, t(\sojourn+1)\} \cap [\Njob]$, for each $t = 1, \ldots, b$.
%
Scaling $\sojourn$ while keeping the number of batches $b$ fixed can be interpreted as increasing the density of the market.
%
We first show that the regret under $\gre$ remains linear in the number of jobs $\Njob$, independent of $\sojourn$, which can be understood as suffering a regret of order $\Omega(\sojourn)$ per batch.

\begin{proposition}[proof in \Cref{pf:greedy_linear_lower_bound_dynamic}]
\label{prop:greedy_linear_lower_bound_dynamic}
    Under reward topology $\reward(\type,\type') = \min\{\type,\type'\}$, if $(\sojourn+1)$ is divisible by 4, then for any number of jobs $\Njob$ divisible by $(\sojourn+1)$, there exists an instance $\instance \in [0,1]^\Njob$ for which $\regret_\gre(\instance,\sojourn) \ge n/4$.
\end{proposition}

In contrast, the following theorem shows that the performance of $\PB$ improves as market density increases.

\begin{theorem}\label{thm: dynamic}
Under reward topology $\reward(\type,\type') = \min\{\type,\type'\}$, we have $\regret_\PB(\instance,\sojourn) \le 1/2 + (\frac{\Njob}{d+1}+1)(1+\log(\sojourn+2))/2$ for any $\Njob$, and any instance $\instance\in[0,1]^\Njob$.
\end{theorem}
%
\proof{Proof.}
Let $\instance\in \typespace^\Njob$ and $\sojourn\ge 1$. Let $(C,\matchof(\cdot))$ be the output of $\PB$ on $\instance$.
Similar to the proof of \Cref{thm:potential_log_upper_bound}, we have
\begin{align}
\regret_\PB(\instance,\sojourn)
& \le \sum_{j=1}^\Njob \potential(\type_j) - \PB(\instance,\sojourn)  \nonumber \\
& \le \sup_{\type\in\typespace}\potential(\type) + \sum_{j \in C} \left( p(\theta_j) + p(\theta_{m(j)}) - r(\theta_j,\theta_{m(j)}) \right) \nonumber \\ 
& = \sup_{\type\in\typespace}\potential(\type) + \sum_{t=1}^{b} \sum_{j\in  C\cap B_t} \left( p(\theta_j) + p(\theta_{m(j)}) - r(\theta_j,\theta_{m(j)}) \right) \nonumber \\
& = \frac12 + \frac12 \sum_{t=1}^{b} \left(\sum_{j\in  C\cap B_t} |\type_j-\type_{\matchof(j)}| \right), \label{eq:online_dist}
\end{align}
where we split the sum into the $b = \lceil \Njob/(\sojourn + 1) \rceil$ batches, defined as $B_{t} = \{(t-1)(\sojourn+1)+1,\ldots, t(\sojourn+1)\} \cap [\Njob]$ for $t = 1, \ldots, b$. We analyze the term in large parentheses for an arbitrary $t$. Intuitively, we would like to consider an offline instance consisting of jobs $(\type_j)_{j\in C \cap B_t}$ and their matches $(\type_{\matchof(j)})_{j\in C\cap B_t}$, and apply \Cref{thm:potential_log_upper_bound} on the offline instance which has size $2|C\cap B_t|\le 2(\sojourn+1)$.
However, since the offline setting allows jobs to observe the full instance, as opposed to only the next $\sojourn$ arrivals, the resulting matching of $\PB$ on the offline instance could be inconsistent with its output on the online instance. In particular, executing $\PB$ on the offline instance could match jobs with indices more than $d$ apart, which is not possible in the online setting. Thus, to derive a proper upper bound on regret, we need to construct a modified offline instance for which each resulting match of $\PB$ coincides with the matching output in the online counterpart, and in which matching distances were not decreased.

%
To construct this modified offline instance, first note that if $\matchof(j)\in B_t$ for all $j\in C\cap B_t$, then no modification is needed, since all matched jobs have indices less than $\sojourn$ apart.
%
However, if $\matchof(j)\in B_{t+1}$ for some $j\in C\cap B_t$, then $\matchof(j)$ can be available in the offline instance for some job $k<j$ with $|\matchof(j)-k|>\sojourn$ and be matched to $k$ in the offline instance even though this would not be possible in the online instance (see \Cref{fig:thm3_exm}).
%
Thus, in the modified offline instance, we "move" job $m(j)$ ensure job $k$ would not choose $m(j)$ for its match.
%
To do so, we distinguish three cases.
%
For each $j\in C\cap B_t$, let $A(j)$ be the set of jobs that the online algorithm could have chosen from to match with $j$, i.e.~$A_j$ is the set $A\setminus\{j\}$ in \Cref{alg:dynamic} right after job $j$ becomes critical.
%
If $\matchof(j)\in A(j)\cap B_t$, then both jobs $j,\matchof(j)$ are included in the offline instance we construct.
%
In the second case, if $\matchof(j)\in A(j)\cap B_{t+1}$ and $A(j)\cap B_t\neq\emptyset$, then we modify $\type_{m(j)}$ to be equal to the best available matching candidate within batch i.e.\ we "move" job $m(j)$ to location $\argmin \{ |\type_j - \type| : \type = \type_k, \ k\in A(j)\cap B_t \}$.
%
In the third case, if $\matchof(j)\in A(j)\cap B_{t+1}$ and $A(j)\cap B_t=\emptyset$, then we don't consider job $j$ or its match (if any) in the offline instance we construct. This can happen at most once per batch, since any other job $k<j$ would have $j\in A(k)$. 
%

To formally define the construction, fix an arbitrary batch $t$. Let
$\hat{C}=\{j\in C\cap B_t: A(j)\cap B_t\neq \emptyset\}$ and $\hat{n}= 2|\hat{C}|\le 2(\sojourn+1)$. 
%
$\hat{C}$ is the set of critical jobs in cases one or two above, and we include jobs $j\in\hat{C}$ and their matches $m(j)$ in the offline instance.  There are $\hat{n}$ jobs in the offline instance and the ordering of indices is consistent with the original instance.
Define modified locations in the offline instance as follows:
\begin{align*}
\hat{\type}_j &= \type_j &\text{for $j\in\hat{C}$}
\\ \hat{\type}_{m(j)} &= \type_{m(j)} &\text{for $j\in\hat{C}$, if $m(j)\in A(j)\cap B_t$ (case one)}
\\ \hat{\type}_{m(j)} &= \argmin \{ |\type_j - \type| : \type = \type_k, \ k\in A(j)\cap B_t \} &\text{for $j\in\hat{C}$, if $m(j)\in A(j)\cap B_{t+1}$ (case two)}
\end{align*}
%
%
\begin{figure}[H]
  \centering
  \subcaptionbox{Batch $B_t$ of the online instance $\instance$.
    \label{fig:thm3_exm_original}
    }[0.4 \textwidth]
    {
    \begin{tikzpicture}


% Background shading for periods 6t+1 to 6t+6
\fill[lightyellow] (1.8, 0) rectangle (6.2, 3.2);

\draw[->] (-0.5, 0) -- (8.5, 0) node[right] {$j$};
\draw[->] (0, -0.1) -- (0, 3.3) node[above] {$\type_j$};

\foreach \x in {1,...,8}
    \draw (\x, -0.1) -- (\x, 0.1);

\node[below] at (1, 0) {\tiny $5t$};
\node[below] at (2, 0) {\tiny $5t+1$};
\node[below] at (3, 0) {\tiny $5t+2$};
\node[below] at (4, 0) {\tiny $5t+3$};
\node[below] at (5, 0) {\tiny $5t+4$};
\node[below] at (6, 0) {\tiny $5t+5$};
\node[below] at (7, 0) {\tiny $5t+6$};
\node[below] at (8, 0) {\tiny $5t+7$};

\foreach \y in {0.1,0.2,0.3,0.4,0.5,0.6,0.7,0.8,0.9,1}
    \draw (-0.1, \y*3) -- (0.1, \y*3) node[left, xshift=-0.2cm] {\tiny \y};
    

\foreach \x/\y in {1/0.9, 2/0.1, 3/0.7, 4/0.3, 5/0.4, 6/1, 7/0.2, 8/0.8} 
{
    \fill[black] (\x, \y*3) circle (0.1);
    \draw[gray, dashed] (0, \y*3) -- (\x, \y*3);
    }
    

% Matches
\draw[thick] (1, 0.9*3) -- (3, 0.7*3);     
\draw[thick] (2, 0.1*3) -- (4, 0.3*3);
\draw[thick] (5, 0.4*3) -- (7, 0.2*3);     
\draw[thick] (6, 1*3) -- (8, 0.8*3);  

\end{tikzpicture}
    }
  \hfill
  \subcaptionbox{Modified offline instance $\hat{\instance}$. 
    \label{fig:thm3_exm_phantom}
    }[0.4 \textwidth]
    {
    \begin{tikzpicture}

\draw[->] (-0.5, 0) -- (4.5, 0) node[right] {$j$};
\draw[->] (0, -0.1) -- (0, 3.3) node[above] {$\type_j$};

\foreach \x in {1,...,4}
    \draw (\x, -0.1) -- (\x, 0.1);

\node[below] at (1, 0) {\tiny $5t+1$};
\node[below] at (2, 0) {\tiny $5t+3$};
\node[below] at (3, 0) {\tiny $5t+4$};
\node[below] at (4, 0) {\tiny $5t+6$};

\foreach \y in {0.1,0.2,0.3,0.4,0.5,0.6,0.7,0.8,0.9,1}
    \draw (-0.1, \y*3) -- (0.1, \y*3) node[left, xshift=-0.2cm] {\tiny \y};
    

\foreach \x/\y in {1/0.1, 2/0.3, 3/0.4, 4/1} 
{
    \fill[black] (\x, \y*3) circle (0.1);
    \draw[gray, dashed] (0, \y*3) -- (\x, \y*3);
    }
    

% Matches
\draw[thick] (1, 0.1*3) -- (2, 0.3*3);     
\draw[thick] (3, 0.4*3) -- (4, 1*3);


\end{tikzpicture}
    }
  % 
\caption{Illustration of the construction of the modified offline instance, with $\sojourn=4$. In \Cref{fig:thm3_exm_original}, job $5t+1$ chooses job $5t+3$ in the online execution of $\PB$, because only jobs up to $5t+5$ would have arrived when job $5t+1$ becomes critical.  However, in the offline execution of $\PB$, job $5t+1$ would choose job $5t+6$, because all jobs are available.  Our modified instance in \Cref{fig:thm3_exm_phantom} corrects the inconsistent decision.
}
  % 
  \label{fig:thm3_exm}
\end{figure}
%
Now, we show that the output of (offline) $\PB$ on $\hat{\instance}$ provides an upper bound for the expression $\sum_{j\in C\cap B_t}|\type_j-\type_{\matchof(j)}|$ in~\eqref{eq:online_dist}.
%
In particular, we show that the output of $\PB$ is exactly $(\hat{C},m(\cdot))$ when executed on the modified offline instance.
%
First, note that the construction potentially introduces ties if there is a job $k\in C\cap B_t$ with $\matchof(k)\in B_{t+1}$. However, since \Cref{thm:potential_log_upper_bound} holds under arbitrary tie-breaking, we assume $\PB$ breaks ties on $\hat{\instance}$ to maintain consistent decisions with $\PB$ on the original instance $\instance$. Thus, from \Cref{deliverypotential}, it suffices to show that for every $j\in \hat{C}$, $\matchof(j)$ minimizes the distance among available jobs.
%

%
Indeed, note that every job in the offline instance has a type that is identical to a job in $B_t$. Then, proceeding in the same order as in the online instance, the job types of the set of available jobs in the offline instance is a subset of the original available jobs. For a match in case one, we know that it minimizes the distance among the original available jobs, and hence it minimizes the distance among the offline available jobs. Moreover, we have $|\hat{\type}_j-\hat{\type}_{\matchof(j)}| = |\type_j - \type_{\matchof(j)}|$. In the second case, the match minimizes distance among offline available jobs by construction, and moreover $|\hat{\type}_j-\hat{\type}_{\matchof(j)}| \ge |\type_j - \type_{\matchof(j)}|$.
%
Consequently, we can upper bound 
\[ \sum_{j\in C\cap B_t } |\type_j-\type_{\matchof(j)}|
\le 1 + \sum_{j\in  \hat{C}} |\hat{\type}_j-\hat{\type}_{\matchof(j)}|. 
\]
%  
Then, recalling \eqref{dist_bound} in the proof of \Cref{thm:potential_log_upper_bound}, we have $\sum_{j\in  \hat{C}} |\hat{\type}_j-\hat{\type}_{\matchof(j)}| \le \log(\hat{n}/2 + 1) \le \log(\sojourn + 2)$, and then
\begin{align}
    \sum_{j\in C\cap B_t } |\type_j-\type_{\matchof(j)}|
\le 1 +  \log(\sojourn + 2).\label{eq: distance_dynamic}
\end{align}
Substituting back into~\eqref{eq:online_dist}, and using the fact that $b\le\frac n{d+1}+1$, we get $\regret_\PB(\instance,\sojourn)\le 1/2 + (\frac{\Njob}{d+1}+1)(1+\log(\sojourn+2))/2$, completing the proof. 
\Halmos\endproof
%
Lastly, we show that this online analysis of $\PB$ is also tight, up to constants.
%
\begin{proposition}[proof in \Cref{pf:loglowerboundOnline}]
\label{prop:loglowerboundOnline}
Under reward topology $\reward(\type,\type') = \min\{\type,\type'\}$, if $\sojourn+1 = 2^{k+3}-4$ for some $k\in\{0,1,\ldots\}$, then for any number of jobs $\Njob$ divisible by $(\sojourn+1)$, there exists an instance $\instance \in [0,1]^\Njob$ for which $\regret_\PB(\instance,\sojourn) \ge \frac{\Njob}{3(\sojourn+1)}(\log_2(\sojourn+5)-3)/4$.
\end{proposition}
%

%%%%%%%%%%%%%%%%%%%%%%%%%%%%%%%%%%%%%%%%%%%%%%%%%%%%%%

% \subsection{Interpretation of Potential}\label{sec: interpretation}

% When a job $j$ becomes critical, our potential-based greedy algorithm matches it to an available job $k$ maximizing $r(\theta_j,\theta_k)-p(\theta_k)$, where $p(\theta_k)$ can be interpreted as the opportunity cost of matching job $k$, with the specific definition $p(\theta_k)=\frac12 \sup_{\theta\in\Theta} r(\theta_k,\theta)$.
% This is an optimistic measure of opportunity cost because it assumes that job $k$ would otherwise be matched to an "ideal" type $\theta\in\Theta$ maximizing $r(\theta_k,\theta)$ (with half of this ideal reward $\sup_{\theta\in\Theta} r(\theta_k,\theta)$ attributed to job $k$).
% We now prove that this ideal reward can indeed be achieved under asymptotically-large market thickness, for a stochastic model under our topology of interest.

% \begin{definition}
% Let $\instance \in \typespace^\Njob$.
% For any job $j\in[\Njob]$, let $\instance^{-j}\in \typespace^{\Njob-1}$ be the same instance with the exception that job $j$ is not present. Meanwhile, let $\instance^{+j}\in \typespace^{\Njob+1}$ be the same instance with the exception that an additional copy of job $j$ is present.
% Consider the following definitions.
% \begin{enumerate}
% \item Marginal Loss: $\marginalloss_j(\instance) = \OPT(\instance) - \OPT(\instance^{-j})$
% \item Marginal Gain: $\marginalgain_j(\instance) = \OPT(\instance^{+j}) - \OPT(\instance)$
% \end{enumerate}
% \end{definition}

% Note that definitions $\marginalloss_j(\instance),\marginalgain_j(\instance)$ are based solely on offline matching, and we will use them as our definitions of opportunity cost if the future was known.  One could alternatively use shadow prices from the LP relaxation of the offline matching problem, but we note that the LP is not integral.  In either case, there is no ideal definition of opportunity cost that is guaranteed to lead to the optimal offline solution in matching problems \citep[see][]{cohen2016invisible}.

% We now establish the following \namecref{lem:marginal_as_interval} to help analyze the opportunity costs $\marginalloss_j(\instance),\marginalgain_j(\instance)$.


% \begin{lemma}\label{lem:marginal_as_interval}
% Let $\typespace=[0,1]$ and $\reward(\type,\type') = \min\{\type,\type'\}$. Consider an instance $\instance\in\typespace^n$ and relabel the indices to satisfy $\type_1 \ge \type_2 \ge \ldots \ge \type_n$. Then,
% \begin{align*}
% \marginalloss_j(\instance)
% &=\sum_{k\ge j,k\ \mathrm{even}}(\theta_k-\theta_{k+1}), 
% \\ \marginalgain_j(\instance)
% &=\sum_{k\ge j,k\ \mathrm{odd}}(\theta_k-\theta_{k+1}),
% \end{align*}
% where we consider $\theta_{n+1}=0$.
% \end{lemma}
% %
% \proof{Proof.}
%     Let $\instance \in \typespace^\Njob$ such that $\type_1 \ge \type_2 \ge \ldots \ge \type_n$. 
%     %
%     Then, $\OPT(\instance) =\sum_{k\ \mathrm{even}}\theta_k$, and moreover
%     \begin{align*}
%     \OPT(\instance^{-j}) &=\sum_{k<j, k\ \mathrm{even}}\theta_k+\sum_{k>j, k\ \mathrm{odd}}\theta_k
%     \\ \OPT(\instance^{+j}) &=\theta_j+\OPT(\instance^{-j})
%     \end{align*}
%     % 
%     Therefore,
%     % 
%     \begin{align*}
%     \marginalloss_j(\instance)
%     &=\sum_{k\ge j,k\ \mathrm{even}}\theta_k
%     -\sum_{k>j,k\ \mathrm{odd}}\theta_k
%     =\sum_{k\ge j,k\ \mathrm{even}}(\theta_k-\theta_{k+1})
%     \\ \marginalgain_j(\instance)
%     &=\theta_j-\sum_{k\ge j,k\ \mathrm{even}}(\theta_k-\theta_{k+1})
%     =\sum_{k\ge j,k\ \mathrm{odd}}(\theta_k-\theta_{k+1})
%     \end{align*}
%     % 
%     completing the proof.
% \Halmos\endproof

% Note that because $\OPT(\instance^{+j}) = \OPT(\instance^{-j}) + \type_j$ and $\potential(\type_j) = \reward(\type_j,\type_j)/2$ for this reward function, we immediately get the following \namecref{cor: marginal_average}.

% \begin{corollary}\label{cor: marginal_average}
%         If $\typespace=[0,1]$ and $\reward(\type,\type') = \min\{\type,\type'\}$, then for all jobs $j$,
%     \[ \potential(\type_j) = \frac{\marginalloss_j(\instance) + \marginalgain_j(\instance)}{2}. \]
% \end{corollary}

% We are now ready to prove our main result about the interpretation of potential, that the true opportunity costs $\marginalloss_j(\instance),\marginalgain_j(\instance)$ concentrate around $p(\theta_j)$ in a random uniform instance, assuming the market is sufficiently thick.  We without loss consider job $j=1$ and fix its type $\theta_1$.

% \begin{theorem} \label{thm:interpretation}
% Let $\typespace=[0,1]$ and $\reward(\type,\type') = \min\{\type,\type'\}$.
% Fix $\theta_1\in\typespace$ and suppose $\theta_2,\ldots,\theta_n$ are drawn IID from the uniform distribution over $[0,1]$, forming a random instance $\instance\in\typespace^n$, for some $n\ge 2$. Then, both $\mathbb{E}[\marginalloss_1(\instance)]$ and $\mathbb{E}[\marginalgain_1(\instance)]$ are within $O(1/n)$ of $\potential(\type_1)$ and moreover $\mathrm{Var}(\marginalloss_1(\instance))=\mathrm{Var}(\marginalgain_1(\instance)) = O\left(\frac{1}{n}\right)$.
% \end{theorem}
% %
% \proof{Proof.}
% We prove the statement only for $\marginalloss_1(\instance)$, and the analogous result for $\marginalgain_1(\instance)$ follows from \Cref{cor: marginal_average}.
% %
% If $\type_1=0$, then $\marginalloss_1(\instance)=0$ for any $\instance$, coinciding with $\potential(\type_1)=0$. Then, for the remainder of the proof, assume that $\type_1>0$.
% %
% Consider the random variable $N=|\{j:\theta_j<\theta_1\}|$, which counts the number of points between $0$ and $\theta_1$ and has distribution $\text{Binom}(\Njob-1,\type_1)$.
% %
% These $N$ points divide the interval $[0,\theta_1]$ into $N+1$ intervals with total length $\type_1$.
% %
% From \Cref{lem:marginal_as_interval}, the marginal loss $\marginalloss_1(\instance)$ is determined by computing the total length of a subset of these intervals. % Maybe add figure?
% %
% It is known that if $N$ random variables are drawn independently from $\text{Unif}[0,1]$, then the joint distribution of the induced interval lengths is $\text{Dirichlet}(1,1,\ldots,1)$ with $N+1$ parameters all equal to 1 (i.e., drawn uniformly from the simplex).
% %
% In particular, since the Dirichlet distribution is symmetric, the sum of any $k$ of these intervals is equal in distribution to the $k$-th smallest sample ($k$-th order statistic) of the $N$ uniform random variables, whose distribution is known to be $\text{Beta}(k,N+1-k)$.
% %
% Thus, conditional on $N$, with $N\ge 1$, since the distribution of each of the $N$ jobs to the left of $\type_1$ is $\text{Unif}[0,\type_1]$, the distribution of $\marginalloss_1(\instance)/\type_1$ is $\text{Beta}(k_L,N+1-k_L)$,
% where $k_L$ is either $\lceil\frac{N+1}2\rceil$ or $\floor{\frac{N+1}2}$.
% %
% To be precise, for $N\ge 1$, $k_L=\floor {N/2}+1=\lceil\frac{N+1}2\rceil$ if $n$ is even, and $k_L=\ceil{N/2}=\floor{\frac{N+1}2}$ if $n$ is odd.
% %
% Moreover, if $N=0$, then $\marginalloss_1(\instance)=\type_1$ if $n$ is even, and $\marginalgain_1(\instance)=0$ if $n$ is odd.
% %
% Hence,
% \begin{align*}
% \mathbb{E}[\marginalloss_1(\type) \mid N] 
% &= \type_1\frac{k_L}{N+1}
% % \label{eq: cond_exp}
% \end{align*}
% for all $N\in\{0,\ldots,n-1\}$.
% %
% Since $|k_L-(N+1)/2|\le 1$, then $|\mathbb{E}[\marginalloss_1(\type) \mid N]-\type_1/2|\le \type_1/(N+1)$, thus
% % 
% \begin{align*}
% &\left|\mathbb{E}[\marginalloss_1(\type)]-\frac{\type_1}{2}\right| 
% \le \type_1\mathbb{E}\left[\frac{1}{N+1}\right] \\
% &\qquad = \type_1\sum_{k=0}^{n-1} \frac{1}{k+1}\binom{n-1}{k}\type_1^k(1-\type_1)^{n-1-k} \\
% &\qquad = \frac{\type_1}{n}\sum_{k=0}^{n-1} \binom{n}{k+1}\type_1^k(1-\type_1)^{n-1-k} \\
% &\qquad = \frac{1}{n}\sum_{k=1}^{n} \binom{n}{k}\type_1^k(1-\type_1)^{n-k} \\
% &\qquad = \frac{1-(1-\type_1)^{n}}{n} = O\left(\frac{1}{n}\right).
% \end{align*}
% %
% This completes the proof of the statement about $\mathbb{E}[\marginalloss_1(\instance)]$.

% For the statement about variance, we know $\mathrm{Var}(\marginalloss_1(\type)) = \mathbb{E}[\mathrm{Var}(\marginalloss_1(\type) \mid N)] + \mathrm{Var}(\mathbb{E}[\marginalloss_1(\type)\mid N])$ by the law of total variance. For the first term, we have $\mathrm{Var}(\marginalloss_1(\type) \mid N) = \type_1^2\frac{k_L(N+1-k_L)}{(N+1)^2(N+2)}\indicator\{N\ge 1\}\le \frac{\type_1^2}{N+1}$, and then from the previous argument $\mathbb{E}[\mathrm{Var}(\marginalloss_1(\type) \mid N)]=O(1/n)$.
% %
% For the second term, 
% \begin{align*}
% \mathrm{Var}(\mathbb{E}[\marginalloss_1(\type)\mid N]) 
% &= \mathrm{Var}\left(\mathbb{E}[\marginalloss_1(\type)\mid N] - \frac{\type_1}{2}\right) \\
% &\le \mathbb{E} \left[\left(\mathbb{E}[\marginalloss_1(\type)\mid N] - \frac{\type_1}{2}\right)^2\right] \\
% % 
% &\le \mathbb{E}\left[\frac{\type_1^2}{(N+1)^2}\right] \\
% &\le \type_1^2\mathbb{E}\left[\frac{1}{N+1}\right] = O\left(\frac{1}{n}\right).
% \end{align*}
% Thus, $\mathrm{Var}(\marginalloss_1(\instance)) = O\left(1/n\right)$, completing the proof.
% \Halmos\endproof



\section{Numerical Experiments} \label{sec:numerical_experiments}

In this section, we show via numerical simulations that under the reward structure from delivery pooling, our proposed potential-based greedy algorithm ($\PB$) outperforms a number of benchmark algorithms, given sufficient density.
%
We first generate synthetic data under a setting more aligned with our theoretical results, and find that $\PB$ outperforms all benchmarks starting from very low market densities ($\sojourn\ge 5$).
%
We then apply all algorithms and benchmarks on real data from the Meituan platform.
%
Although the real-life setting differs from our theoretical model in a number of ways --- that we address in \Cref{sec:sim_meituan} --- the results remain qualitatively similar: $\PB$ performs the best, starting from the modest density level corresponding to ``each job is able to wait for one minute before being dispatched''.
%
Additional simulation results, including additional performance metrics and more general synthetic environments (e.g. two-dimensional locations, different spatial distributions, and different reward topologies) are provided in \Cref{sec: sim_results_extra}.


The benchmark algorithms that we compare with are categorized into two groups, \emph{forecast-agnostic}, and \emph{forecast-aware}, depending on the use of historical data/demand distributional information to make matching decisions. We now describe each of them.
%


\subsection{Benchmark Algorithms} \label{sec:sim_benchmarks}


As we have discussed earlier, the potential based greedy algorithm $\PB$ does not rely on any historical data/demand forecast.
% 
The same is true for naive greedy $\gre$, and we refer to heuristics with this property as \newterm{forecast-agnostic}. 
%
In this group, we consider batching policies as additional benchmarks, which are highly relevant both in theory and in practice.


\paragraph{Naive Batching.} 


Batching heuristics periodically compute optimal matching solutions given the available jobs.
% 
Since it is without loss in our model to wait to match, we consider batching algorithms that re-compute optimal matchings whenever any available job becomes critical.
%
% 
The \newterm{naive batching} algorithm ($\batching$) dispatches all available jobs according to this optimal solution. 
%
In particular, under the theoretical model introduced in \Cref{sec:model}, $\batching$ clears the market every $\sojourn + 1$ arrivals, which is equivalent to the one analyzed in \citet{ashlagi2019edge}. 
%
However, an obvious suboptimality of this algorithm is that there are pairs of jobs that are matched and dispatched before any of them becomes critical. 



\paragraph{Rolling Batching.} 

To address the previously mentioned issue of dispatching non-critical pairs of jobs, we also consider a modified batching heuristic, termed \newterm{rolling batching} ($\rbatching$).
%
$\rbatching$ also computes an optimal matching solution every time any available job becomes critical. Instead of dispatching all jobs, however, $\rbatching$ only dispatches the critical job and its match (if any) according to the optimal solution. 


\medskip 


In practice, delivery platforms typically have access to past order data that allows to predict patterns of future arrivals to some extent. 
%
In contrast to the aforementioned algorithms, we say that a policy is \newterm{forecast-aware} if it uses historical data to make pooling decisions. 
%
We consider dual-based benchmark algorithms that use optimal dual solutions (i.e. shadow prices) to approximate the opportunity cost of matching each job. 
% 
More specifically, given an instance from the historical data (i.e. a sequence of delivery requests with information about origin, destination, and timestamp), we can solve the dual program of a linear relaxation of the IP defined in \eqref{eq: OPT}.
% 
\footnote{We provide a detailed description of the primal and dual linear programs in \Cref{sec: LP}.}
% 
The optimal dual variables associated with the capacity constraint of each job will then provide a proxy for the marginal value of the job, which we use to construct two additional benchmark algorithms.


\paragraph{Hindsight Dual.}
% 
Ideally, if we had access to the dual optimal solution for an instance in advance, we would use them as the opportunity cost of matching each job.
%
In an attempt to measure the value of this hindsight information, we simulate a \newterm{hindsight-dual} algorithm ($\dual$) that operates retrospectively on the same instance.
%
Formally, given an instance $\instance\in\typespace^\Njob$ and parameter $\sojourn$, let $\lambda_k=\lambda_k(\instance,\sojourn)$ be the optimal dual variable associated with the constraint $\sum_{k:j\neq k} x_{jk} \le 1$ in the linear program  (defined in \eqref{eq: LP}). The $\dual$ algorithm operates as an index-based greedy matching algorithm (as defined in \Cref{alg:dynamic}) where the index function is given by $\indexf_{\dual}(\type_j,\type_k) = \reward(\type_j,\type_k) - \lambda_k$, for all $j,k\in [\Njob]$, such that $j\neq k, |j-k| \le d$.
% 
Note that this index function is computed based on the entire realized instance $\instance$, thus this algorithm is only for benchmarking purposes and is not practically feasible.



\paragraph{Average Dual.}
%
A more realistic approach for estimating the opportunity costs is to average the hindsight duals of the jobs from historical data.
% 
However, this estimation is not free of challenges. 
% 
In particular, computing the dual solutions yields a shadow price for every observed type, which might not cover the entire type space and thus 
many future arrivals may not have an associated shadow price.
% 
To address this issue, we discretize the type space and estimate average duals for each discrete type as the empirical average of the shadow prices for jobs associated to the discrete type.
% 
To be precise, let $H$ be the set of all instances in the historical data, and consider a discretization $\{\typespace_1,\typespace_2,\ldots,\typespace_p\}$
% 
that partitions the space into $p$ cells (i.e. these cells are mutually exclusive and jointly exhaustive).
%
Each cell $i=1,2,\ldots,p$ is associated with a shadow price $\Bar{\lambda}_i$ that is the average shadow price of historical jobs lying in the same cell, i.e. $\Bar{\lambda}_i$ is the average of $\{\lambda(\instance,\sojourn) : \instance\in H, \type_j \in \typespace_i \}$.
%
Then, the \newterm{average-dual} algorithm
($\averagedual$) is an index-based greedy matching algorithm that uses the shadow prices for the cells, i.e.\ uses index function
$\indexf_{\averagedual}(\type,\type') = \reward(\type,\type') - \sum_{i=1}^p\Bar{\lambda}_{i}\indicator\{\type'\in\typespace_i\}.$

\subsection{Synthetic Data}\label{sec:sim_unif_1D}

We first consider the setting analyzed in our theoretical results, where job destinations are uniformly distributed on a one-dimensional space.
% 
We fix the total number of jobs at $\Njob=1000$, 
and compare algorithms at density levels $\sojourn\in\{5,10,15,20,25,30\}$.
%
For each $(\Njob, \sojourn)$ pair, we generate 100 instances $\instance \in[0,1]^\Njob$ uniformly at random, apply all algorithms on each of them, and compute the average regret and reward \newterm{ratio} ($\ALG(\instance,d)/\OPT(\instance,d)$) achieved by each algorithm and benchmark.
% 
Additional performance metrics, including the fraction of jobs that are pooled and the fraction of the total distance that is reduced by pooling, are presented in \Cref{sec:match_rate}. Similar results for the 1D, non-uniform setting are presented in \Cref{sec: nonunif_1D}.
% 

\subsubsection{Comparison with forecast-agnostic heuristics.}
%
\Cref{fig:1D} compares potential-based greedy $\PB$ with naive greedy $\gre$ and the two batching algorithms.
% 
As expected from our analysis, $\gre$ performs poorly even at high densities. To our surprise, $\PB$ outperforms both batching algorithms for all considered density values.
%
To assess the performance of the algorithms relative to the hindsight optimal, we also compute the \emph{ratio} for each instance, and plot its empirical average in \Cref{fig:1D_ratio}.
% 
We can see that even at relatively low density levels, $\PB$ achieves over 95\% of the maximum possible travel distance saved from pooling. $\gre$, on the other hand, performs worse relative to hindsight optimal as density increases--- despite the fact that the
%the fraction of
travel distance saved by $\gre$ increases with density (see \Cref{fig:1D_unif_saving_fraction} in \Cref{sec: 1D_extra}). 
%
\begin{figure}%[H]
  \centering
  \subcaptionbox{Average regret.%
    \label{fig:1D_regret}}[0.49 \textwidth]{\includegraphics[width = \figWidth \textwidth]{Simulation_Results/Synthetic/New/1D_Pooling/1D_Common_Origin/Forecast-Agnostic/regret.png}}
  \hfill
  \subcaptionbox{Average ratio.%
    \label{fig:1D_ratio}}[0.49 \textwidth]{\includegraphics[width = \figWidth \textwidth]{Simulation_Results/Synthetic/New/1D_Pooling/1D_Common_Origin/Forecast-Agnostic/frac_of_OPT.png}}
  % 
  \caption{Comparison of average regret and reward ratio for forecast-agnostic heuristics in random 1D instances.}
  % 
  \label{fig:1D}
\end{figure}
\subsubsection{Comparison with forecast-aware heuristics.}

We now compare $\PB$ with our two forecast-aware benchmarks, $\dual$ and $\averagedual$.
%
%
Under $\dual$, for each of the 100 instances generated for each market condition $(n,d)$, we compute a shadow price for each job via the dual LP.
%
As discussed in \Cref{sec:sim_benchmarks}, this algorithm is not practically feasible, and we present its performance here mainly to illustrate the value of this hindsight information used in this particular way.
%
%
For $\averagedual$, for each $(n,d)$, we first generate 400 \emph{historical instances} in the same way that the original 100 instances were drawn.
% 
%
We then compute optimal dual variables for each historical instance, and estimate average shadow prices $\Bar{\lambda}$ for a uniform discretization of the type space $[0,1]$ into 100 intervals.
% 

\Cref{fig:1D_forecastaware} compares the average regret and reward ratio on the original 100 instances.
%
We can see that $\PB$ performs on par with $\averagedual$ across all density parameters, without relying on any forecast/historical data. 
%
Notably, even with access to hindsight information, $\dual$ is consistently dominated by both $\averagedual$ and $\PB$, with the performance gap narrowing as $\sojourn$ increases.
%


\begin{figure}%[H]
  \centering
  \subcaptionbox{Average regret.%
    \label{fig:1D_forecastaware_regret}}[0.49 \textwidth]{\includegraphics[width = \figWidth \textwidth]{Simulation_Results/Synthetic/New/1D_Pooling/1D_Common_Origin/Forecast-Aware/regret.png}}
  \hfill
  \subcaptionbox{Average ratio.%
    \label{fig:1D_forecastaware_ratio}}[0.49 \textwidth]{\includegraphics[width = \figWidth \textwidth]{Simulation_Results/Synthetic/New/1D_Pooling/1D_Common_Origin/Forecast-Aware/frac_of_OPT.png}}
  % 
  \caption{Comparison of average regret and reward ratio for forecast-aware heuristics in random 1D instances. }
  % 
  \label{fig:1D_forecastaware}
\end{figure}




\subsection{Order-Level Data from Meituan Platform}\label{sec:sim_meituan}


In this \namecref{sec:sim_meituan}, we test the algorithms and benchmarks using data from the Meituan platform, made public for the 2024 INFORMS TSL Data-Driven Research Challenge.\footnote{\url{https://connect.informs.org/tsl/tslresources/datachallenge}, accessed January 15, 2025.
%
See \Cref{sec:market_dynamics} for more details on the dataset, as well as high-level illustrations of market dynamics over both space and time.
} 
%
The dataset provides detailed information on various aspects of the platform's delivery operations, for one city and a total of 8 days in October 2022.  
%
For each of the 569 million delivery orders, the dataset provides the pick-up and drop-off locations (the latitudes and longitudes were shifted for privacy considerations), and the timestamps at which orders are created, pushed into the dispatch system, etc.
% 
% 
This allows us to (i) construct instances of job arrivals and (ii) calculate travel distances and pooling opportunities.




Since the pick-up and drop-off locations now reside in a two-dimensional (2D) space, we first extend our definition of pooling reward to the 2D space, derive the potential of different job types, and provide the notion of a pooling window (i.e. the sojourn time) that determines which jobs can be pooled together.


\paragraph{Pooling reward.}
% 
Each job $j$ has type $\type_j = (\xorigin_j,\xdestination_j)$, 
% 
with $\xorigin_j,~\xdestination_j\in \R^2$ corresponding to the coordinates of the job origin and destination.
%
The travel distance saved by pooling two orders together is the difference between (i) the travel distance required to fulfill the jobs in two separate trips, and (ii) the minimum travel distance for fulfilling the two requests in a single, pooled trip.
%
% 
For the pooled trip, the travel distance always includes the distance between the two origins and that between the two destinations (both orders must be picked up before either of them is dropped off; otherwise the orders are effectively not pooled). 
% 
The remaining distance depends on the sequence in which the jobs are picked up and dropped off, resulting in four possible routes. The minimum travel distance is then determined by evaluating the total distance for all four routes and selecting the smallest value.
% 
Formally, letting $\norm{\cdot}$ denote the Euclidean norm, the reward function is
% 
\begin{align}
    \reward(\type,\type') = &  \norm{\xdestination-\xorigin} +  \norm{\xdestination'-\xorigin'}  \nonumber\\
    % 
    & -\left(\norm{\xorigin-\xorigin'} + \min\left\{\norm{\xdestination-\xorigin},\norm{\xdestination'-\xorigin'},\norm{\xdestination-\xorigin'},\norm{\xdestination'-\xorigin'}\right\}+\norm{\xdestination-\xdestination'}\right). \label{eq:2D_reward_part_2}
\end{align}
% 
Note that unlike the 1D common origin setting, the pooling reward in 2D space with heterogeneous origins is not guaranteed to be positive. As a result, we assume that all index-based greedy algorithms only pool jobs together when the reward is non-negative.

\paragraph{Potential.}
% 
For a given job type $\type=(\xorigin,\xdestination)$, the maximum reward a platform can achieve by pooling it with another job is the distance of the job, $\norm{\xdestination - \xorigin}$.\footnote{To see this, first observe that when a job $\type = (\xorigin,\xdestination)$ is pooled with another job of identical type, the travel distance saved is precisely $\norm{\xdestination - \xorigin}$.
%
On the other hand, let $m$ be the minimum in \eqref{eq:2D_reward_part_2}. Then, by triangle inequality $\|O'-D'\|\le \|O-O'\| + \|D-O\| + \|D-D'\|$, and thus $\reward(\type,\type')\le 2\|D-O\|-m \le \|D-O\|$, since $m\le \|D-O\|$.
% % 
To achieve an even higher pooling reward, it must be the case that the second part of the pooling reward \eqref{eq:2D_reward_part_2} is strictly smaller than $\norm{\xdestination' - \xorigin'}$. This implies that the total distance traveled to visit all four locations is smaller than the distance between two of them, which is impossible.}  

As a consequence, the potential of a job $\type$ is $\potential(\type) = \norm{\xdestination - \xorigin}/2$. %, 
% 
This is aligned with our 1D theoretical model, in which longer deliveries have higher potential reward from being pooled with other jobs in the future.
%

\paragraph{Pooling window.} 
% 
Instead of assuming that each job is available to be pooled for a fixed number of future arrivals, we take the actual timestamps of the jobs, and assume that the jobs are available for a fixed time window, at the end of which the order becomes critical.
%
Intuitively, the length of the pooling window corresponds to the ``patience level'' of the jobs.
% 
This deviates from our assumptions for the theoretical model, but it is straightforward to see how our algorithm as well as the benchmarks we compare with can be generalized and applied.


\subsubsection{Comparison with forecast-agnostic heuristics.}

We focus our analysis on a three-hour lunch period (10:30am to 1:30pm) for the 8 days of data provided by the challenge.
% 
During this lunch period, there are approximately $130$ orders per minute on average, thus $\Njob\approx 24,000$ orders in total (see \Cref{fig:meituan_orders_per_hour} in \Cref{sec:market_dynamics} for an illustration of order volume over time).
% 
If the platform allows a time window of one minute for each order before it has to be dispatched, the average density would be around $130$ jobs. 
% 
\Cref{fig:meituan} compares the forecast-agnostic heuristics, as the pooling window increases from 30 seconds to 5 minutes. We can see that a single minute is sufficient for $\PB$ to outperform every other forecast-agnostic algorithm, achieving around 80\% of the hindsight optimal reward.
% 
Moreover, the performance of $\PB$ improves steadily as the allowed matching window increases.
%

\begin{figure}%[H]
  \centering
  \subcaptionbox{Average regret.%
    \label{fig:meituan_regret}}[0.49 \textwidth]{\includegraphics[width = \figWidth \textwidth]{Simulation_Results/Meituan/City/regret_fagnostic.png}}
  \hfill
  \subcaptionbox{Average ratio.%
    \label{fig:meituan_ratio}}[0.49 \textwidth]{\includegraphics[width = \figWidth \textwidth]{Simulation_Results/Meituan/City/ratio_fagnostic.png}}
  % 
  \caption{Comparison of average regret and reward ratio for forecast-agnostic heuristics in Meituan data.}
  % 
  \label{fig:meituan}
\end{figure}


\subsubsection{Comparison with forecast-aware heuristics.}

We now compare our algorithm with the forecast-aware heuristics on data from the same three-hour lunch period, 10:30am to 1:30pm.
% 
To simulate the \emph{hindsight-dual} algorithm $\dual$ for each instance associated to each day, we compute the shadow prices via the dual LP program and then compute the ``online'' pooling outcomes using pooling reward minus these shadow prices as the index function.
%
% 
To estimate historical average shadow prices for the $\averagedual$ algorithm, for each of the 8 days, we (i) use the remaining 7 days as the historical data, and (ii) discretize the 2D space into discrete types based on job origin and destination using H3, the hexagonal spatial indexing system developed by Uber.\footnote{See {\url{https://www.uber.com/blog/h3/} (accessed January 7, 2024) for more details on the H3 package.} 
% 
% 
The system supports sixteen resolution levels, each tessellating the earth using hexagons of a particular size.
% %
For a given resolution, the set of all origin-destination (OD) hexagon pairs represents the partition of the type space $\typespace$, as described in \Cref{sec:sim_benchmarks}.
% %
% % 
To compute the average shadow price for a particular OD hexagon pair, we compute the average of optimal dual variables associated with all historical jobs that share the same hexagon pair.
% %
If there are no jobs for a given pair, however, then we consider the hexagon pairs at coarser resolutions (moving one or more levels up in the H3 hierarchy, with each level increasing the size of the hexagons by a factor of 7) until there exist associated historical jobs.
% %
There is a tradeoff between granularity and sparsity when we choose the baseline resolution level, as finer resolutions only average over trips with close-by origins and destinations, but may lack sufficient historical data for accurate estimation (see \Cref{fig:meituan_CDF_count_per_OD_pair} for the distribution of order volume by OD hexagon pairs).
% 
After testing all available resolution levels, we found that levels 8 and 9 perform the best, depending on the time window. The results presented in this section report the result corresponding to the best resolution for each time window.
}

\Cref{fig:meituan_forecastaware} compares the regret and reward ratio of $\PB$ to forecast-aware heuristics.
%
We first observe that in contrast to the synthetic setting from \Cref{sec:sim_unif_1D}, $\dual$ now incurs the smallest regret at all density levels. %
% 
This performance difference appears to be driven by the highly non-uniform spatial distribution of jobs --- additional simulations, presented in \Cref{sec:sim_unif_1D_2D}, demonstrate that (i) $\dual$ performs no better than $\PB$ if job origins and destinations are drawn uniformly at random from a unit square, and (ii) when jobs are generated from a non-uniform distribution, $\dual$ slightly outperforms $\PB$. Of course, we remind the reader that $\dual$ is not a realistic algorithm because it requires knowing the exact future.

The worse performance of $\averagedual$ in comparison to $\PB$ also highlights the challenges of estimating the opportunity costs of pooling different jobs from historical data in real-world settings. 
% 
Intuitively, taking historical data into consideration should be valuable in scenarios with highly non-uniform spatial distributions. In this case, however, the sparsity of the data for many if not most origin-destination pairs makes it difficult to properly estimate the opportunity costs, or requires a very coarse granularity for location.  $\PB$ bypasses this sparsity-granularity tradeoff by only considering the reward topology, i.e.\ the space of all possible types.
%


\begin{figure}%[H]
  \centering
  \subcaptionbox{Average regret.%
    \label{fig:meituan_forecastaware_regret}}[0.49 \textwidth]{\includegraphics[width = \figWidth \textwidth]{Simulation_Results/Meituan/City/regret_faware.png}}
  \hfill
  \subcaptionbox{Average ratio.%
    \label{fig:meituan_forecastaware_ratio}}[0.49 \textwidth]{\includegraphics[width = \figWidth \textwidth]{Simulation_Results/Meituan/City/ratio_faware.png}}
  % 
  \caption{Comparison of average regret and reward ratio for forecast-aware heuristics in Meituan data.}
  % 
  \label{fig:meituan_forecastaware}
\end{figure}


% \subsubsection{Match rate and saving fraction.}\label{sec:match_rate}
% % \paragraph{Match Rate and Saving Fraction.}

% In addition to regret and reward ratio, we consider in this section two additional performance metrics: the \newterm{match rate}, i.e. the fraction of jobs that were pooled instead of dispatched on their own, and  the \newterm{saving fraction}, i.e. the fraction of the total distance that is reduced by pooling, relative to the total travel distance without any pooling~\citep[see][]{aouad2020dynamic}.
% %
% % 
% For brevity, we focus only on the potential-based greedy algorithm $\PB$ and the hindsight optimum $\OPT$ in \Cref{fig:meituan_2}.
% % 
% A comprehensive comparison across all 
% algorithms and benchmarks is provided in \Cref{sec: meituan_extra}.


% \begin{figure}[H]
%   \centering
%   \subcaptionbox{Average match rate.%
%     \label{fig:meituan_2_match_rate}}[0.49 \textwidth]{\includegraphics[width = \figWidth \textwidth]{Simulation_Results/Meituan/City/match_rate.png}}
%   \hfill
%   \subcaptionbox{Average saving fraction.%
%     \label{fig:meituan_2_saving_fraction}}[0.49 \textwidth]{\includegraphics[width = \figWidth \textwidth]{Simulation_Results/Meituan/City/saving_rate.png} }
%   % 
%   \caption{Match rate and Saving Fraction, Meituan Order-Level Data.}
%   % 
%   \label{fig:meituan_2}
% \end{figure}

% Both regret and reward ratio are performance metrics that compare pooling algorithm relative to the hindsight optimal pooling outcome. The saving fraction
% % 
% as shown in \Cref{fig:meituan_2_saving_fraction} illustrates the benefit of delivery pooling in comparison to the total distance traveled, and is upper bounded by 0.5 (since reward from pooling a job cannot exceed its distance).
% % 
% We can see that if the platform allows a time window of at least $1$ minute for each order before it has to be dispatched, $\PB$ achieves a reduction in distance traveled by over 20\%. At $5$ minutes, this fraction increases to roughly 30\%.
% %
% Since the saving fraction is proportional to the total reward collected by the algorithm, our previous performance comparisons imply that the saving fraction achieved by $\PB$ is the best among all tested \emph{practical} heuristics, i.e. excluding $\dual$ and $\OPT$.

% \Cref{fig:meituan_2_match_rate} compares the match rate achieved under $\PB$ and $\OPT$. We can see that $\PB$ consistently pools more jobs than $\OPT$ (5-10\%). 
% %
% This highlights an additional desirable property of $\PB$ and, more generally, of \emph{index-based greedy matching} algorithms: they pool as many jobs as possible.
% % 
% This is desirable in practice, since drivers who are offered just only one order from a platform may try to pool orders from competing platforms \citep[see e.g.][]{reddit2022grubhub,reddit2023doordash}, leading to poor service reliability for customers.

% Note that for the one-dimensional settings, match rates are very close to 100\% under greedy policies since at most one job is left unmatched (see \Cref{fig:1D_unif_match_rate} in \Cref{sec: 1D_extra} for comparisons). 
% %
% In the real-data setting, match rates are generally lower since we pool the critical job with another only when it's possible to achieve a positive pooling reward. Nevertheless, \Cref{sec: meituan_extra} shows that greedy algorithms ($\gre$, $\PB$, $\dual$, $\averagedual$) still achieve substantially higher match rates in comparison to $\batching$, $\rbatching$, and $\OPT$.




\section{Concluding Remarks and Future Work} \label{sec:conclusion}

In this work, we study dynamic non-bipartite matching in the context of pooling delivery orders.
% 
Our modeling approach captures two key features of the problem through the reward topology and the assumption that jobs can be matched until they become \textit{critical}.
%
We propose a simple, forecast-agnostic potential-based greedy algorithm, specially designed for this reward topology, that performs surprisingly well both in synthetic and real data. It uses a new notion of potential as opportunity cost, that considers the value of keeping long-distance deliveries in the platform for better pooling opportunities in the future. We show that for our reward topology of focus, potential-based greedy improves upon greedy both in worst-case and empirical performance. Moreover, we showcase the robustness of potential-based greedy through numerical experiments in real data, outperforming several commonly used benchmark algorithms given enough market density.
%
We provide some intuition behind the success of this approach by showing through theory and experiments that different notions of opportunity cost are intimately related to our definition of potential, especially as density increases. 
%
Moreover, we believe that the main insight of long-distance jobs being more valuable to hold should apply even when pooling more than two orders into a single trip (see \citet{wei2023constant}); and we believe our notion of potential could be relevant even on non-metric reward structures, where it is a measure of e.g.\ the popularity of a volunteer position (see \citet{manshadi2022online}).
%
Formalizing and generalizing these results is an interesting avenue of future research.
%
Overall, these surprising results showcase the value of using the knowledge of reward topology to design application-specific algorithms. This general idea, which has received limited attention so far, could also be applied to other domains, such as ride-sharing or kidney exchange.

%\THEEndNotes
% \begingroup \parindent 0pt \parskip 0.0ex \def\enotesize{\normalsize} \theendnotes \endgroup

% Acknowledgments here
\ACKNOWLEDGMENT{
The authors would like to thank
Itai Ashlagi,
Omar Besbes,
Francisco Castro,
Yash Kanoria,
Jake Marcinek,
Rad Niazadeh,
Scott Rodilitz,
Daniela Saban, and
Alejandro Torrielo
for valuable comments and discussions.
}


% References here (outcomment the appropriate case)

% CASE 1: BiBTeX used to constantly update the references
%   (while the paper is being written).
% \bibliographystyle{informs2014trsc} % outcomment this and next line in Case 1
% \bibliography{DraftBib} % if more than one, comma separated

%\bibliographystyle{informs2014trsc} % outcomment this and next line in Case 1
%\bibliography{sample} % if more than one, comma separated

% CASE 2: BiBTeX used to generate mypaper.bbl (to be further fine tuned)
%%%%%%%%%%%%%%%%%%%%%%%%%%%%%%%%%%%%%%%%%%%%%%%%%%%%%%%%%%%%%%%%%%%%%%%%%%
%%
%%	Author Submission Template for Transportation Science (TRSC)
%%	INFORMS, <informs@informs.org>
%%	Ver. 1.00, June 2024
%%
%%%%%%%%%%%%%%%%%%%%%%%%%%%%%%%%%%%%%%%%%%%%%%%%%%%%%%%%%%%%%%%%%%%%%%%%%%
%
% Use dblanonrev for Double Anonymous Review submission
% Use sglanonrev for Single Anonymous Review submission
% For example, submission to Operations Research, OPRE will have
% \documentclass[opre,dblanonrev]{informs4}

%%% TRUE SUBMISSION
% \documentclass[trsc,dblanonrev]{informs4}
% \usepackage{eqndefns-left} % For checking the display equation width and equation environment definitions %

%%% ARXIV VERSION
\documentclass[trsc,sglanonrev]{informs4_hide}

\RequirePackage{tgtermes}
\RequirePackage{newtxtext}
\RequirePackage{newtxmath}
\RequirePackage{multirow,multicol}
\RequirePackage{xspace,dsfont}
\RequirePackage{endnotes}
% \let\footnote=\endnote


%\OneAndAHalfSpacedXI
% \OneAndAHalfSpacedXII % Current default line spacing
\DoubleSpacedXI % Double-spacing for 11-point font (TSL)
% \DoubleSpacedXII

% Optional LaTeX Packages
\usepackage{algorithm}
\usepackage{algpseudocode}
\algrenewcommand\algorithmicrequire{\textbf{Input:}}
\algrenewcommand\algorithmicensure{\textbf{Output:}}
\usepackage{tikz}
%
\usepackage{xcolor}
\definecolor{dark1}{RGB}{157, 58, 103}
\definecolor{dark2}{RGB}{161, 67, 0}
\definecolor{dark3}{RGB}{115, 102, 0}
\definecolor{dark4}{RGB}{2, 120, 50}
\definecolor{dark5}{RGB}{0, 116, 122}
\definecolor{dark6}{RGB}{18, 100, 176}
\definecolor{dark7}{RGB}{116, 75, 163}
\definecolor{mid1}{RGB}{225, 119, 163}
\definecolor{mid2}{RGB}{228, 128, 77}
\definecolor{mid3}{RGB}{182, 158, 21}
\definecolor{mid4}{RGB}{87, 182, 109}
\definecolor{mid5}{RGB}{0, 181, 190}
\definecolor{mid6}{RGB}{88, 162, 242}
\definecolor{mid7}{RGB}{177, 135, 229}
\definecolor{light1}{RGB}{255, 187, 231}
\definecolor{light2}{RGB}{255, 198, 151}
\definecolor{light3}{RGB}{247, 223, 104}
\definecolor{light4}{RGB}{152, 248, 171}
\definecolor{light5}{RGB}{72, 249, 255}
\definecolor{light6}{RGB}{164, 219, 255}
\definecolor{light7}{RGB}{244, 192, 255}
\definecolor{lightyellow}{RGB}{255, 255, 204}

%% Comments
%%% Comments  will be removed if \Comments = 0 %%%
\newcount\Comments
\Comments = 0
\newcommand{\kibitz}[2]{\ifnum\Comments=1{\color{#1}{#2}}\fi}

\newcommand{\mr}[1]{\kibitz{mid1}{[Matias: #1]}}
\newcommand{\hma}[1]{\kibitz{mid4}{[HMa: #1]}}


\newcount\Drop  
\Drop = 0
\newcommand{\todrop}[2]{\ifnum\Drop=1{\color{#1}{#2}}\fi}
\newcommand{\drop}[1]{\todrop{mid5}{[Drop: #1]}}


%%%% Highlights that becomes black if \highlighttrue is commented out 
\newif\ifhighlight 
\highlighttrue % comment out this line to disable highlights

\newcommand{\mradd}[1]{{\ifhighlight\color{dark1}\fi#1}}
\newcommand{\hmadd}[1]{{\ifhighlight\color{dark4}\fi#1}}


\newif\iftodo
% \todotrue % comment out this line to hide the TODOs 

\newcommand{\todo}[1]{\iftodo{\color{mid1!50!gray}{[TODO: #1]}}\fi}

\usepackage{color}              % Need the color package
% \usepackage[suppress]{color-edits}
\usepackage{color-edits}
\addauthor{Will}{blue}
\addauthor{Hma}{dark4}
%
\usepackage[colorlinks,
	citecolor = dark6,
	urlcolor = black,
	linkcolor = dark6,
    hypertexnames=false]{hyperref}
\usepackage[nameinlink]{cleveref}
\crefname{subsection}{subsection}{subsections}
\usepackage[short]{optidef}
\usepackage[font=scriptsize]{subcaption}
\newcommand{\figWidth}{0.45}

% Private macros here (check that there is no clash with the style)
%%%%%%%%%%%%%%%%%%%%%%%%%% NOTATION %%%%%%%%%%%%%%%%%%%%%%%%

% bold lowercase letters, for vectors
\newcommand\vzero{{\bm 0}}
\newcommand\vzeron{{\bm 0}_n}
\newcommand\vone{{\bm 1}}
\newcommand\vonen{{\bm 1}_{n}}
% 
\newcommand\vpi{{\bm\pi}}
\newcommand\vphi{{\bm\phi}}
\newcommand\veta{{\bm\eta}}
\newcommand\vmu{{\bm\mu}}
% \newcommand\vtheta{{\bm\theta}}
\newcommand\vtheta{{\BFtheta}}
\newcommand\vdelta{{\bm\delta}}
\newcommand\va{{\bm a}}
\newcommand\vb{{\bm b}}
\newcommand\vc{{\bm c}}
\newcommand\vd{{\bm d}}
\newcommand\ve{{\bm e}}
\newcommand\vf{{\bm f}}
\newcommand\vg{{\bm g}}
\newcommand\vh{{\bm h}}
\newcommand\vi{{\bm i}}
\newcommand\vj{{\bm j}}
\newcommand\vk{{\bm k}}
\newcommand\vl{{\bm l}}
\newcommand\vm{{\bm m}}
\newcommand\vn{{\bm n}}
\newcommand\vo{{\bm o}}
\newcommand\vp{{\bm p}}
\newcommand\vq{{\bm q}}
\newcommand\vr{{\bm r}}
% \newcommand\vs{{\bm s}}
\newcommand\vt{{\bm t}}
\newcommand\vu{{\bm u}}
\def\vv{{\bm v}}
\newcommand\vw{{\bm w}}
\newcommand\vx{{\bm x}}
\newcommand\vy{{\bm y}}
\newcommand\vz{{\bm z}}


% Number sets
\newcommand{\C}{\mathbb{C}}
\newcommand{\R}{\mathbb{R}}
\newcommand{\Q}{\mathbb{Q}}
\newcommand{\N}{\mathbb{N}}
\newcommand{\Z}{\mathbb{Z}}

% Personal definitions
\renewcommand{\P}{\mathbb{P}}
\newcommand{\geom}[1]{\text{Geom}(#1)}
\newcommand{\E}{\mathbb{E}}
\newcommand{\var}{\text{Var}}
\newcommand{\normal}{\mathcal{N}}
\newcommand{\indicator}{\mathds{1}}
%\newcommand{\converges}[2][n \to \infty]{\xrightarrow[#1]{#2}}
\newcommand{\converges}[2][]{\xrightarrow[#1]{#2}}
\newcommand{\equalin}[1]{\stackrel{#1}{=}}
\newcommand{\loss}[1]{\depth({#1})}
% \DeclareMathOperator*{\argmin}{arg\,min}
% \DeclareMathOperator*{\argmax}{arg\,max}
\newcommand{\infsum}[1]{\sum_{#1\ge 1}}
\newcommand{\finsum}[2]{\sum_{#1=1}^{#2}}
\newcommand{\abs}[1]{\left|#1\right|}
\newcommand{\norm}[2][]{\left\|#2\right\|_{#1}}
\newcommand{\innerprod}[2][]{\left\langle #2 \right\rangle_{#1}}
\newcommand{\ceil}[1]{\left\lceil #1 \right\rceil}
\newcommand{\floor}[1]{\left\lfloor #1 \right\rfloor}
% \newcommand{\norm}[1]{\left\| #1 \right\|}
\newcommand{\indicatorof}[1]{\indicator\left\{#1\right\}}
\newcommand{\eps}{\varepsilon}

% Highlight a newly defined term
\newcommand{\newterm}[1]{\textit{#1}}
\newcommand{\red}[1]{{\leavevmode\color{red}#1}}
\newcommand{\blue}[1]{{\leavevmode\color{blue}#1}}
\newcommand{\nblue}[1]{{\leavevmode\color{nblue}#1}}
\newcommand{\violet}[1]{{\leavevmode\color{violet}#1}}
\newcommand{\purple}[1]{{\leavevmode\color{purple}#1}}
\newcommand{\ngreen}[1]{{\leavevmode\color{ngreen}#1}}

% macros for dynamic pooling
\newcommand{\Njob}{n}
\newcommand{\type}{\theta}
\newcommand{\typespace}{\Theta}
\newcommand{\instance}{\vtheta}
\newcommand{\sojourn}{d}
\newcommand{\potential}{p}
\newcommand{\ALG}{\mathsf{ALG}}
\newcommand{\OPT}{\mathsf{OPT}}
\newcommand{\OFF}{\mathsf{OFF}}
\newcommand{\regret}{\mathsf{Reg}}
\newcommand{\gap}{\mathsf{GAP}}
\newcommand{\gre}{\mathsf{GRE}}
\newcommand{\PB}{\mathsf{PB}}
\newcommand{\batching}{\mathsf{BAT}}
\newcommand{\rbatching}{\mathsf{R-BAT}}
\newcommand{\pot}{\mathsf{POT}}
\newcommand{\Potloss}{\mathsf{Loss}}
\newcommand{\LP}{\mathsf{LP}}
\newcommand{\DualLP}{\mathsf{DLP}}
\newcommand{\matchset}{\mathcal{M}}
\newcommand{\buffer}{A}
\newcommand{\indexf}{q}
\newcommand{\noutput}{m}
\newcommand{\order}[1]{(#1)}
\newcommand{\depth}{\ell}
\newcommand{\matchof}{m}
\newcommand{\ltype}{\alpha}
\newcommand{\rtype}{\beta}
\newcommand{\jobset}{J}
\newcommand{\width}{w}
\newcommand{\reward}{r}
\newcommand{\supI}{\mathrm{A}}
\newcommand{\supII}{\mathrm{B}}
\newcommand{\supIII}{\mathrm{C}}
\newcommand{\basecase}{\mathrm{BC}}
% \newcommand{\xorigin}{\vx_{\mathrm{ori}}}
\newcommand{\xorigin}{O}
% \newcommand{\xdestination}{\vx_{\mathrm{dest}}}
\newcommand{\xdestination}{D}
\newcommand{\yorigin}{y_{\mathrm{origin}}}
\newcommand{\ydestination}{y_{\mathrm{destination}}}
\newcommand{\marginalgain}{\mathsf{MG}}
\newcommand{\marginalloss}{\mathsf{ML}}
\newcommand{\randmarginalgain}{\mathsf{RMG}}
\newcommand{\randmarginalloss}{\mathsf{RML}}
\newcommand{\spacing}{S}
\newcommand{\dual}{\mathsf{HD}}
\newcommand{\discmap}{\varphi}
\newcommand{\averagedual}{\mathsf{AD}}
% \newcommand{\unif}{\text{Unif}}

% Natbib setup for author-number style
\usepackage{natbib}
 \bibpunct[, ]{(}{)}{,}{a}{}{,}%
 \def\bibfont{\small}%
 \def\bibsep{\smallskipamount}%
 \def\bibhang{24pt}%
 \def\newblock{\ }%
 \def\BIBand{and}%


%% Setup of the equation numbering system. Outcomment only one.
%% Preferred default is the first option.
\EquationsNumberedThrough    % Default: (1), (2), ...
%\EquationsNumberedBySection % (1.1), (1.2), ...

%% Setup of theorem styles. Outcomment only one.
%% Preferred default is the first option.
\TheoremsNumberedThrough     % Preferred (Theorem 1, Lemma 1, Theorem 2)
%\TheoremsNumberedByChapter  % (Theorem 1.1, Lema 1.1, Theorem 1.2)
\ECRepeatTheorems  %  

% For new submissions, leave this number blank.
% For revisions, input the manuscript number assigned by the on-line
% system along with a suffix ".Rx" where x is the revision number.
\MANUSCRIPTNO{TRSC-0001-2024.00}

%%%%%%%%%%%%%%%%
\begin{document}
%%%%%%%%%%%%%%%%

% Outcomment only when entries are known. Otherwise leave as is and
%   default values will be used.
%\setcounter{page}{1}
%\VOLUME{00}%
%\NO{0}%
%\MONTH{Xxxxx}% (month or a similar seasonal id)
%\YEAR{0000}% e.g., 2005
%\FIRSTPAGE{000}%
%\LASTPAGE{000}%
%\SHORTYEAR{00}% shortened year (two-digit)
%\ISSUE{0000} %
%\LONGFIRSTPAGE{0001} %
%\DOI{10.1287/xxxx.0000.0000}%

% Author's names for the running heads
% Sample depending on the number of authors;
% \RUNAUTHOR{Jones}
% \RUNAUTHOR{Jones and Wilson}
% \RUNAUTHOR{Jones, Miller, and Wilson}
% \RUNAUTHOR{Jones et al.} % for four or more authors
% Enter authors following the given pattern:
%\RUNAUTHOR{}
% \RUNAUTHOR{Smith and Johnson}

\RUNAUTHOR{Ma, Ma, and Romero}





% Title or shortened title suitable for running heads. Sample:
% \RUNTITLE{Predictive Maintenance in Manufacturing}
% Enter the (shortened) title:
% \RUNTITLE{Optimal Resource Allocation in Humanitarian Logistics}
\RUNTITLE{Dynamic Delivery Pooling}

% Full title. Sample:
% \TITLE{Optimal Resource Allocation in Humanitarian Logistics: A Stochastic Programming Approach}
% Enter the full title:
% \TITLE{Optimal Resource Allocation in Humanitarian Logistics: A Stochastic Programming Approach}
\TITLE{Potential-Based Greedy Matching for Dynamic Delivery Pooling} 

% Block of authors and their affiliations starts here:
% NOTE: Authors with same affiliation, if the order of authors allows,
%   should be entered in ONE field, separated by a comma.
%   \EMAIL field can be repeated if more than one author
\ARTICLEAUTHORS{%
%\AUTHOR{John Doe,\textsuperscript{a} Jane Smith,\textsuperscript{b}}
%\AFF{\textsuperscript{a}Department of Industrial Engineering, University of XYZ, \EMAIL{john.doe@xyz.edu; \textsuperscript{b}Department of Computer Science, University of ABC, \EMAIL{jane.smith@abc.edu}} 
% \AUTHOR{John Smith}
% \AFF{Department of Logistics,
% University of XYZ, \EMAIL{john.smith@xyz.edu}}

% \AUTHOR{Emily Johnson}
% \AFF{Department of Logistics,
% University of ABC, \EMAIL{emily.johnson@abc.edu}}

% Enter all authors
\AUTHOR{Hongyao Ma}
\AFF{Graduate School of Business, Columbia University, New York, NY 10027, \EMAIL{hongyao.ma@columbia.edu}}
\AUTHOR{Will Ma}
\AFF{Graduate School of Business, Columbia University, New York, NY 10027, \EMAIL{wm2428@gsb.columbia.edu}}
\AUTHOR{Matias Romero}
\AFF{Graduate School of Business, Columbia University, New York, NY 10027, \EMAIL{mer2262@gsb.columbia.edu}}
} % end of the block

\ABSTRACT{%
We study the problem of pooling together delivery orders into a single trip, a strategy widely adopted by platforms to reduce total travel distance.
% % 
Similar to other dynamic matching settings, the pooling decisions involve a trade-off between immediate reward and holding jobs for potentially better opportunities in the future. 
%
In this paper, we introduce a new heuristic dubbed potential-based greedy ($\PB$), which aims to keep longer-distance jobs in the system, as they have higher potential reward (distance savings) from being pooled with other jobs in the future. 
%
This algorithm is simple in that it depends solely on the topology of the space, and does not rely on forecasts or partial information about future demand arrivals.
%
We prove that $\PB$ significantly improves upon a naive greedy approach in terms of worst-case performance on the line. Moreover, we conduct extensive numerical experiments using both synthetic and real-world order-level data from the Meituan platform. 
%
Our simulations show that $\PB$ consistently outperforms not only the naive greedy heuristic but a number of benchmark algorithms, including (i) batching-based heuristics that are widely used in practice, and (ii) forecast-aware heuristics that are given the correct probability distributions (in synthetic data) or a best-effort forecast (in real data).
%
We attribute the surprising unbeatability of $\PB$ to the fact that it is specialized for rewards defined by distance saved in delivery pooling.
}%

% % \FUNDING{This research was supported by [grant number, funding agency].}

% %Supplemental Material:
% %Data Ethics & Reproducibility Note:

% % Sample
% %\KEYWORDS{Stochastic programming, Decision support,Uncertainty, Disaster response, Optimization}

% % Fill in data. If unknown, outcomment the field
\KEYWORDS{On-demand delivery, Platform operations, Online algorithms, Dynamic matching} 

% %\HISTORY{Received: Month DD, YYYY; Accepted: Month DD, YYYY; Published Online: Month DD, YYYY}

\maketitle
%%%%%%%%%%%%%%%%%%%%%%%%%%%%%%%%%%%%%%%%%%%%%%%%%%%%%%%%%%%%%%%%%%%%%%

\section{Introduction} \label{sec:intro}

% 
On-demand delivery platforms have become an integral part of modern life, transforming how consumers search for, purchase, and receive goods from restaurants and retailers.
%
Collectively, these platforms serve more than $1.5$ billion users worldwide, contributing to a global market valued at over \$250 billion \citep{statista2024}.
%
Growth has continued at a rapid pace--- major companies such as DoorDash in the United States and Meituan in China reported approximately 25\% year-over-year growth in transaction volume in 2023 \citep{curry2024doordash,scmp2024meituan}. 
%
Managing this ever-expanding stream of orders poses significant operational challenges, particularly as consumers demand increasingly faster deliveries, and fierce competition compels platforms to continuously improve both operational efficiency and cost-effectiveness.

%
One strategy for improving efficiency and reducing labor costs is to \emph{pool} into a single trip multiple orders from the same or nearby restaurants.
% 
This is referred to as stacked, grouped, or batched orders, and is advertised to drivers as opportunities for increasing earnings and efficiency~\citep{doordash2024batched}.
% 
The strategy has been generally successful and very widely adopted~\citep{deliveroo2022,uberEatsMultiple,grabGroupedJobs}.
%
For example, data from Meituan, made public by the 2024 INFORMS TSL Data-Driven Research Challenge, show that more than 85\% of orders are pooled, rising to 95\% during peak hours and consistently remaining above 50\% at all other times (see \Cref{fig:meituan_pooled_orders_per_hour}).
%

\begin{figure}[hpbt]
    \centering
    \includegraphics[width=0.9\linewidth]{Simulation_Results/Meituan/City/meituan_fraction_of_pooled_orders_how.png}
    % 
    \caption{Fraction of pooled orders by hour-of-week in Meituan data. 
    The dataset includes $8$ days of order-level data from one city (see \Cref{sec:sim_meituan} for more details). The gray shade indicates peak lunch hours (10:30am-1:30pm), while the green shade indicates peak dinner hours (5pm-8pm). 
    }
    \label{fig:meituan_pooled_orders_per_hour}
\end{figure}

The delivery pooling approach introduces a highly complex decision-making problem, as orders arrive dynamically to the market, and need to be dispatched within a few minutes --- orders in the Meituan data are offered to delivery drivers an average of around five minutes after being placed, often before meal preparation is complete (see \Cref{fig:meituan_order_to_first_dispatch_how}).
% 
%
During this limited window of each order, the platform must determine which (if any) of the available orders to pool with, carefully weighing immediate efficiency gains against the uncertain, differential benefits of holding each candidate for future pooling opportunities.
%
Consequently, there remains a pressing need for more efficient and robust solution methods to inform the platform's delivery pooling decisions.


%
Similar problems have been studied in the context of various marketplaces such as ride-sharing platforms and kidney exchange programs, in a large body of work known as dynamic matching.
%
However, we argue that the reward structure for delivery pooling is fundamentally different. 
% 
First, many problems studied in the literature involves connecting two sides of a market, e.g. ridesharing platforms matching drivers to riders. 
% 
In such bipartite settings, the presence of a large number of agents of identical or similar types (e.g., many riders requesting trips originating from the same area) typically leads to less efficient outcomes.
% 
While the kidney exchange problem is not bipartite, long queues of ``hard-to-match'' patient-donor pairs can still form, when many pairs share the same combination of incompatible tissue types and blood types.

In stark contrast, it is unlikely for long queues to build up for delivery pooling, especially in dense\footnote{
Indeed, today's major platforms observe extremely large volumes. Meituan, for example, processes over 12 thousand 
orders per hour in one market during peak lunch periods (see \Cref{fig:meituan_orders_per_hour}).} markets, in that given enough orders, two of them will have closely situated origins and destinations, resulting in a good match.
%
In fact, many customers requesting deliveries originating from the same location is actually \textit{desirable}, since this reduces travel distance and parking stops.
% 
We find that delivery orders tend to indeed be highly concentrated in space, with most requests originating from popular restaurants in busy city centers and ending in residential neighborhoods (see \Cref{appx:meituan_spatial_distribution}). 
%
We emphasize that we are not studying the problem of matching (pooled) orders to couriers, but delivery pooling is a relevant first step regardless, as pooling orders efficiently before matching with couriers will effectively increase the amount of courier capacity.

In this work, we study delivery pooling and exploit its distinct \emph{reward topology}, that it is highly desirable to match two jobs of identical or very similar types, as described above.
%
We develop a simple "potential-based" greedy algorithm, which relies solely on the reward topology to determine the opportunity cost of dispatching each job.
% 
The intuition behind it is elementary --- long-distance jobs have higher \textit{potential} for cost savings when pooled with other jobs, and hence the online algorithm should generally prefer keeping them in the system, dispatching shorter deliveries first when it has a choice.

\subsection{Model Description}

Different models of arrivals and departures have been proposed in the dynamic matching literature (see \Cref{sec:dynMatchModels}), and we believe our insight about potential is relevant across all of them.
However, in this paper we focus on a single model well-suited for the delivery pooling application.

%
To elaborate, we consider a dynamic non-bipartite matching problem, with a total of $\Njob$ jobs arriving sequentially to be matched.
%
Jobs are characterized by (potentially infinite) types $\theta\in\Theta$ representing features (e.g. locations of origin and destination) that determine the reward for the platform.
%
Given the motivating application to pooled deliveries, we will always model jobs' types as belonging to some metric space.
%
Following \citet{ashlagi2019edge}, we assume that an unmatched job must be dispatched after $\sojourn$ new arrivals, which we interpret as a known \textit{internal} deadline imposed by the platform to incentivize timely service.
%
The platform is allowed to make a last-moment matching decision before the job leaves, termed as the job becoming \newterm{critical}.
%
At this point, the platform must decide either to match the critical job to another available one, collecting reward $r(\theta,\theta')$ where $\theta,\theta'$ are the types of the matched jobs and $r$ is a known reward function, or to dispatch the critical job on its own for zero reward.
%
The objective is to maximize the total reward collected from matching the $\Njob$ jobs.
%
We believe this to be an appropriate model for delivery platforms (cf. \Cref{sec:dynMatchModels}) because deadlines are known upon arrival and sudden departures (order cancellations) are rare.

Dynamic matching models can also be studied under different forms of information about the future arrivals.
%
Some papers \citep[e.g.][]{kerimov2024dynamic,aouad2020dynamic,eom2023batching,wei2023constant} develop sophisticated algorithms to leverage stochastic information, which is often necessary to derive theoretical guarantees under general matching rewards.
%
In contrast, our heuristic does not require any knowledge of future arrivals, and our theoretical results hold in an "adversarial" setting where no stochastic assumptions are made.
%
In our experiments, we consider arrivals generated both from stochastic distributions and real-world data, and find that our heuristic can outperform even algorithms that are given the correct stochastic distributions, under our specific reward topology.

\subsection{Main Contributions}

\paragraph{Notion of potential.}
% 
As mentioned above, we design a simple greedy-like algorithm that is based purely on topology and reward structure, which we term potential-based greedy ($\PB$). More precisely, $\PB$ defines the \newterm{potential} of a job type $\theta$ to be $p(\theta)=\sup_{\theta'\in\Theta} r(\theta,\theta')/2$, measuring the highest-possible reward obtainable from matching type $\theta$.  The potential acts as an opportunity cost, and the relevance of this notion arises in settings where the potential is heterogeneous across jobs. To illustrate, let $\Theta=[0,1]$ and $r(\theta,\theta')=\min\{\theta,\theta'\}$, a reward function used for delivery pooling as we will justify in \Cref{sec:model}.  Under this reward function, a job type (which is a real number) can never be matched for reward greater than its real value, which means that jobs with higher real value have greater potential.  Our $\PB$ algorithm matches a critical job (with type $\theta$) to the available job (with type $\theta'$) that maximizes
\begin{align} \label{eqn:potentialIntro}
r(\theta,\theta')-p(\theta')=\min\{\theta,\theta'\}-\frac12 \theta',
\end{align}
being dissuaded to use up job types $\theta'$ with high real values. Notably, this definition does not require any forecast or partial information of future arrivals, but rather assumes full knowledge of the universe of possible job types.

\paragraph{Theoretical results.}
Our theoretical results assume $\Theta=[0,1]$.  We compare $\PB$ to the naive greedy algorithm $\gre$, which selects $\theta'$ to maximize $r(\theta,\theta')$, instead of $r(\theta,\theta')-p(\theta')$ as in~\eqref{eqn:potentialIntro}.  We first consider an offline setting ($d=\infty$), showing that under reward function $r(\theta,\theta')=\min\{\theta,\theta'\}$, our algorithm $\PB$ achieves regret $O(\log n)$, whereas $\gre$ suffers regret $\Omega(n)$ compared to the optimal matching.
Our analysis of $\PB$ is tight, i.e.\ it has $\Theta(\log n)$ regret.
Building upon the offline analysis, we next consider the online setting ($d< n$), showing that $\PB$ has regret $\Theta(\frac nd\log d)$, which improves as $d$ increases. This can be interpreted as there being $\frac nd$ "batches" in the online setting, and our algorithm achieving a regret of $O(\log d)$ per batch.  Alternatively, it can be interpreted as the \newterm{regret per job} of our algorithm being $O(\frac d{\log d})$. By contrast, we show that the naive greedy algorithm $\gre$ suffers a total regret of $\Omega(n)$, regardless of $d$.

For comparison, we also analyze two other reward topologies, still assuming $\Theta=[0,1]$.
\begin{enumerate}
\item We consider the classical min-cost matching setting \citep{reingold1981greedy} where the goal is to minimize total match distance between points on a line, represented in our model by the reward function $r(\theta,\theta')=1-|\theta-\theta'|$.
%
For this reward function, both $\PB$ and $\gre$ have regret $\Theta(\log n)$; in fact, they are the same algorithm because all job types have the same potential. Like before, this translates into both algorithms having regret $\Theta(\frac nd \log d)$ in the online setting.
\item We consider reward function $r(\theta,\theta')=|\theta-\theta'|$, representing an opposite setting in which it is worst to match two jobs of the same type.  For this reward function, we show that any index-based matching policy must suffer regret $\Omega(n)$ in the offline setting, and that both $\PB$ and $\gre$ suffer regret $\Omega(n)$ (irrespective of $d$) in the online setting.
\end{enumerate}

Our theoretical results are summarized in \Cref{table:results}.
As our model is a special case of \citet{ashlagi2019edge}, their 1/4-competitive randomized online edge-weighted matching algorithm can be applied, which essentially translates in our setting to an $O(n)$ upper bound for regret under any definition of reward.
Our results show that it is possible to do much better ($O(\log n)$ instead of $O(n)$) for specific reward topologies, using a completely different algorithm and analysis.
We now outline how to prove our two main technical results, \Cref{thm:potential_log_upper_bound,thm: dynamic}.
%
\begin{table}[!t]
\centering
\begin{tabular}{|c|c|c|c|}
\hline
\updown $\theta,\theta'\in[0,1]$ & $r(\theta,\theta')=\min\{\theta,\theta'\}$ & $r(\theta,\theta')=1-|\theta-\theta'|$ & $r(\theta,\theta')=|\theta-\theta'|$ \\
\hline
\up\multirow{4}{*}{Regret of $\PB$} & Offline: $\Theta(\log n)$ & & \\
\down & (\Cref{thm:potential_log_upper_bound}, \Cref{prop:loglowerboundOffline}) & & \\
\up & Online: $\Theta(\frac nd \log d)$ & Offline: $\Theta(\log n)$ & Offline: $\Omega(n)$ \\
\down & (\Cref{thm: dynamic}, \Cref{prop:loglowerboundOnline}) & Online: $\Theta(\frac nd \log d)$ & Online: $\Omega(n)$ \\
\cline{1-2}
\up\multirow{4}{*}{Regret of $\gre$} & Offline: $\Omega(n)$ & (\Cref{sec:reward2}) & (\Cref{sec: reward 3}) \\
\down & (\Cref{prop:greedy_linear_lower_bound}) & & \\
\up & Online: $\Omega(n)$ & & \\
\down & (\Cref{prop:greedy_linear_lower_bound_dynamic}) & & \\
\hline
\end{tabular}
\caption{
Summary of theoretical results under different reward functions $r:[0,1]^2\to \R$.  The total number of jobs is denoted by $n$, and the batch size in the online setting is approximately $d$.
}
\label{table:results}
\end{table}
\paragraph{Proof techniques.}
%
We establish an upper bound on the regret of $\PB$ by comparing its performance to the sum of the potential of all jobs.
%
Under reward function $\reward(\type,\type')=\min\{\type,\type'\}$, the key driver of regret is the sum of the distances between jobs matched by $\PB$.
%
In \Cref{thm:potential_log_upper_bound}, we study this quantity by analyzing the intervals induced by matched jobs in the offline setting ($\sojourn=\infty$).
%
We show that the intervals formed by the matching output of $\PB$ constitute a \emph{laminar set family}; that is, every two intervals are either disjoint or one fully contains the other.
%
We then prove that more deeply nested intervals must be exponentially smaller in size. Finally, we use an LP to show that the sum of interval lengths remains bounded by a logarithmic function of the number of intervals.
%

In \Cref{thm: dynamic}, we partition the set of jobs in the online setting into roughly $\frac{\Njob}{d}$ "batches" and analyze the sum of the distances between matched jobs within a single batch.
%
Our main result is to show that we can use \Cref{thm:potential_log_upper_bound} to derive an upper bound for an arbitrary batch.
%
However, a direct application of the theorem on the offline instance defined by the batch would not yield a valid upper bound, because the resulting matching of $\PB$ in the offline setting could be inconsistent with its online matching decisions.
%
To overcome this challenge, we carefully construct a modified offline instance that allows for the online and offline decisions to be coupled, while ensuring that the total matching distance did not go down.  This allows us to upper-bound the regret per batch by $O(\log d)$, for a total regret of $O(\frac nd \log d)$.


\paragraph{Simulations on synthetic data.} We test $\PB$ on random instances in the setting of our theoretical results, except that job types are drawn uniformly at random from $[0,1]$ (instead of adversarial). 
%
We benchmark its performance against $\gre$, as well as more sophisticated algorithms, including (i) batching-based heuristics that are highly relevant both in theory and practice, and (ii) forecast-aware heuristics that use historical data to compute shadow prices.
%
Our extensive simulation results show that $\PB$ consistently outperforms all benchmarks starting from relatively low market densities, achieving over 95\% of the hindsight optimal solution that has full knowledge of arrivals.
%
This result is robust to different distributions of job types, and also two-dimensional locations.


\paragraph{Simulations on real data.}

Finally, we test the practical applicability of $\PB$ via extensive numerical experiments using order-level data from the Meituan platform, made available from the 2024 INFORMS TSL Data-Driven Research Challenge.\footnote{\url{https://connect.informs.org/tsl/tslresources/datachallenge}, accessed January 15, 2025.}
%
The key information we extract from this dataset is the exact timestamps for the creation of each request, and the geographic coordinates of pick-up and drop-off locations.
%
These aspects differ from our theoretical setting in that (i) requests may not be available to be pooled for a fixed number of new arrivals before being dispatched, and (ii) locations are two-dimensional with delivery orders having heterogeneous origins.
%
To address (i), we assume that the platform sets a fixed time window after which an order becomes critical, corresponding to each job being able to wait for at most a fixed sojourn time before being dispatched.
%
For (ii), we extend our definition of reward (that captures the travel distance saved) to two-dimensional heterogeneous origins.
%
Under this new reward definition, the main insight from our theoretical model remains true: longer deliveries have higher potential reward from being pooled with other jobs in the future, and thus $\PB$ aims to keep them in the system.
%

In contrast to our simulations with synthetic data, the real-life delivery locations may now be correlated and exhibit time-varying effects.
%
Regardless, we find that $\PB$ outperforms all tested heuristics (those without foreknowledge of the future), given that the platform is willing to wait up to one minute before dispatching each job, a fairly modest ask.
%
In this regime, $\PB$ achieves over 80\% of the maximum possible travel distance saved from pooling. $\PB$ also pools 10\% more jobs than the hindsight optimal matching (see \Cref{sec:match_rate}).
%


\paragraph{Explanation for the surprising unbeatability of $\PB$.}
Our potential-based greedy heuristic consistently performs at or near the best across a wide range of experimental setups, including both synthetic and real data.
%
This result is surprising to us, given that $\PB$ is a simple index-based rule that ignores forecast information.
%
To provide some explanation for this finding, we analyze a stylized setting in \Cref{sec: interpretation}, where we show that the "correct" shadow prices converge to our notion of potential as the market thickness increases.
%
That being said, we also end with a couple of caveats.
%
First, our findings about the effectiveness of $\PB$ are specific to our definition of matching reward for delivery pooling, based on travel distance saved, and may not extend to setting with different reward structures.
%
Second, while we carefully tuned the batching-based and dual-based heuristics that we compare against, it remains possible that more sophisticated algorithms, particularly those leveraging dynamic programming in stochastic settings, could outperform $\PB$.


\subsection{Further Related Work}

\subsubsection{Dynamic matching.} \label{sec:dynMatchModels}

Dynamic matching problems have received growing attention from different communities in economics, computer science, and operations research. We discuss different ways of modeling the trade-off between matching now vs.\ waiting for better matches, depending on the application that motivates the study. 


One possible model \citep{kerimov2024dynamic,kerimov2023optimality,wei2023constant} is to consider a setting with jobs that arrive stochastically in discrete time and remain in the market indefinitely, but use a notion of \newterm{all-time regret} that evaluates a matching policy, at every time period, against the best possible decisions until that moment, to disincentivize algorithms from trivially delaying until the end to make all matches.
%

A second possible model, motivated by the risk of cancellation in ride-sharing platforms, is to consider sudden departures modeled by jobs having heterogeneous \textit{sojourn times} representing the maximum time that they stay in the market \citep{aouad2020dynamic}. These sojourn times are unknown to the platform, and delaying too long risks many jobs being lost without a chance of being matched.
%
\citet{aouad2020dynamic} formulates an MDP with jobs that arrive stochastically in continuous time, and leave the system after an exponentially distributed sojourn time. They propose a policy that achieves a multiplicative factor of the hindsight optimal solution. 
%
Related work on the control of matching queues with abandonment includes \citet{collina2020dynamic}, \citet{castro2020matching}, \citet{wang2024demand}, \citet{kohlenberg2024cost}.

%
A third possible model \citep{huang2018match} also considers jobs that can leave the system at any period, but allows the platform to make a last-moment matching decision right before a job leaves, termed as the job becoming \newterm{critical}.  In this model, one can without loss assume that all matching decisions are made at times that jobs become critical.

Finally, the model we study also makes all decisions at times that jobs become critical, with the difference being that the sojourn times are known upon the arrival of a job, as studied in \citet{ashlagi2019edge, eom2023batching}. Under these assumptions, \citet{eom2023batching} assume stationary stochastic arrivals, while \citet{ashlagi2019edge} allow for arbitrary arrivals.
%
Our research focuses on deterministic greedy-like algorithms that can achieve good performance as the market thickness increases.
%
Moreover, our work differs from these papers in the description of the matching value. While they assume arbitrary matching rewards, we consider specific reward functions known to the platform in advance, and use this information to derive a simple greedy-like algorithm with good performance.
% 
This aligns with a stream of literature on online matching, that captures more specific features into the model to get stronger guarantees \citep[see][]{kanoria2021dynamic,chen2023feature,balkanski2023power}.


We should note that our paper also relates to a recent stream of work studying the effects of batching and delayed decisions in online matching \citep[e.g.][]{feng2024batching,xie2023benefits}, as well as works studying the relationship between market thickness and quality of online matches \citep[e.g.][]{ashlagi2021kidney,chen2021matchmaking}.


\subsubsection{Delivery operations.}

On-demand delivery operations have received special attention in the field of transportation and operations management. 
%
\citet{reyes2018meal} introduce the meal delivery routing problem, which falls in the class of dynamic vehicle routing problems (see e.g. \citet{psaraftis2016dynamic}); the authors develop heuristics to dynamically assign orders to vehicles. Similar efforts have been devoted to optimize detailed pooling and assignment strategies as customer orders arrive sequentially \citep{steever2019dynamic,ulmer2021restaurant}, mainly using approximate dynamic programming techniques.
%
These heuristics are shown to work well in extensive numerical experiments, but given the intricate nature of the model and techniques, it is very challenging to derive managerial insights or theoretical performance guarantees.
%
Closer to our research goal, \citet{chen2024courier} analyze the optimal dispatching policy on a stylized queueing model representing a disk service area centered at one restaurant. They show that delivering multiple orders per trip is beneficial when the service area is large.
%
\citet{cachon2023fast} study the interplay between the number of couriers and platform efficiency, assuming a one-dimensional geography with one single origin, which is also the primary model in our theoretical results.
%
A similar topology is studied in the game-theoretic model of \citet{keskin2024order} to capture the impact of delivery pooling on the interaction between the platform, customers, riders, and restaurants. 

\section{Preliminaries}
\label{sec:model}
% 
In this section, we introduce a dynamic non-bipartite matching model for the delivery pooling problem. A total of $\Njob$ jobs arrive to the platform sequentially.
% 
Each job $j \in [\Njob] = \{1, \ldots, \Njob\}$, indexed in the order of arrival, has a \emph{type} $\type_j \in \typespace$. 
% 
We adopt the criticality assumption from \citet{ashlagi2019edge}, in the sense that each job remains available to be matched for $\sojourn\ge 1$ new arrivals, after which it becomes \textit{critical}. The parameter $\sojourn$ can also be interpreted as the market density.
%
When a job becomes critical, the platform decides whether to (irrevocably) match it with another available job, in which case they are dispatched together (i.e. \emph{pooled})
and the platform collects a \emph{reward} given by a known function $\reward:\typespace^2\to\R$. If a critical job is not pooled with another job, it has to be dispatched by itself for zero reward.


\subsection{Reward Topology}
% 
Our modeling approach directly imposes structure on the type space, as well as the reward function that captures the benefits from pooling delivery orders together. Similar to previous numerical work on pooled trips in ride-sharing platforms \citep{eom2023batching, aouad2020dynamic}, we assume that the platform's goal is to reduce the total distance that needs to be traveled to complete all deliveries.
The reward of pooling two orders together is therefore the travel distance saved when they are delivered by the same driver in comparison to delivered separately. 
% 
Formally, we consider a \newterm{linear city model} where job types $\type \in \typespace = [0,1]$ represent destinations of the delivery orders, and assume that all orders need to be served from the origin 0 (similar to that analyzed in \citet{cachon2023fast})
% 
If a job of type $\type$ is dispatched by itself, the total travel distance from the origin 0 is exactly $\type$. If it is matched with another job of type $\type'$, they are pooled together on a single trip to the farthest destination $\max\{\type,\type'\}$. Thus, the distance saved by pooling is 
%
%
%
\begin{equation}
  \reward(\type,\type') = \type + \type' - \max\{\type,\type'\} = \min\{\type,\type'\}.
  \label{eq:defn_reward_A} 
\end{equation}

Although our main theoretical results leverage the structure of this reward topology, our proposed algorithm can be applied to other reward topologies. 
% 
In particular, we derive theoretical results for two other reward structures for $\typespace = [0,1]$. First, we consider 
% 
\begin{equation}
    \reward(\type,\type') = 1 - |\type - \type'|  \label{eq:defn_reward_B}
\end{equation}
% 
to capture the commonly-studied spatial matching setting~\citep[e.g.][]{kanoria2021dynamic,balkanski2023power}, where the reward is larger if the distance $|\theta-\theta'|$ is smaller. 
% 
We also consider 
\begin{equation}
    \reward(\type,\type') = |\type - \type'| \label{eq:defn_reward_C}
\end{equation}
% 
with the goal of matching types that are far away from each other, contrasting the other two reward functions.
% 
In addition, we perform numerical experiments for delivery pooling in two-dimensional (2D) space, where the reward function corresponds to the travel distance saved in the 2D setting.


\subsection{Benchmark Algorithms}

% 
Given market density $\sojourn$ and reward function $\reward$, if the platform had full information of the sequence of arrivals $\instance \in \typespace^\Njob$, the hindsight optimal $\OPT(\instance,\sojourn)$ can be computed by the integer program (IP) defined in \eqref{eq: OPT}. We denote the hindsight optimal matching solution as $\matchset_{\OPT} = \{(j,k):x_{j,k}^{\ast}=1\}$, where $x^\ast$ is an optimal solution of \eqref{eq: OPT}.
% 
\begin{maxi}
    {x}{ \sum_{j,k : j\neq k, |j-k|\le \sojourn} x_{jk} \reward(\type_j,\type_k)}
    {\label{eq: OPT}}{\OPT(\instance,\sojourn) =}
    \addConstraint{ \sum_{k:j\neq k} x_{jk}}{\le 1,}{j\in [\Njob]}
    \addConstraint{ x_{jk}}{\in\{0,1\},}{j,k\in [\Njob], j\neq k.}
\end{maxi}
%

An online matching algorithm operates over an instance $\instance\in\typespace^\Njob$ sequentially: when job $j\in [\Njob]$ becomes critical, the types of future arrivals $k > j + d$ are unknown, and any matching decision has to be made based on the currently available information.
% 
Given an algorithm $\ALG$, we denote by $\ALG(\instance,\sojourn)$ the total reward collected by an algorithm on such instance.
%
We analyze the performance of algorithms via \emph{regret}, as follows:
%
\[
    \regret_\ALG(\instance,\sojourn) = \OPT(\instance,\sojourn)-\ALG(\instance,\sojourn).
\]

We study a class of online matching algorithms that we call \newterm{index-based greedy matching algorithms}.
%
Each index-based greedy matching algorithm is specified by an \emph{index function} $\indexf:\typespace^2\to\R$, and only makes matching decisions when some job becomes critical (in our model, it is without loss of optimality to wait to match).
%
A general pseudocode is provided in \Cref{alg:dynamic}.
%
When a job $j \in [\Njob]$ becomes critical, the algorithm observes the set of available jobs in the system $A(j) \subseteq [\Njob]$ that have arrived but are not yet matched (this is the set $\buffer\setminus\{j\}$ in \Cref{alg:dynamic}), and chooses a match $\matchof(j) \in A(j)$ that maximizes the index function (even if negative), collecting a reward $\reward(\type_j,\type_{\matchof(j)})$. If there are multiple jobs that achieve the maximum, the algorithm breaks the tie arbitrarily.
%
The algorithm makes a set of matches $\matchset = \{ (j,m(j)) : j \in C \}$, where $C$ denotes the set of jobs that are matched when they become critical, and its total reward is
\[ \ALG(\instance,\sojourn) = \sum_{j\in C} \reward(\type_j,\type_{\matchof(j)}). \]
%
Note that $\matchof(j)$ is undefined if $j$ is either unmatched, or was not critical at the time it was matched.


\begin{algorithm}
\caption{Index-based Greedy Matching Algorithm}\label{alg:dynamic}
% 
\begin{algorithmic}[1]
\Require Instance $\instance$, density $\sojourn$, index function $\indexf$, reward function $\reward$
% 
\Ensure $C$, $m:C\to[n]$
% 
\State Initialize set of available jobs $\buffer=\emptyset$, and jobs that are critical when matched $C=\emptyset$
\For{job $t=1,\ldots,\Njob+\sojourn+1$}
    \If{$t\le\Njob$}
        \State $\buffer=\buffer\cup \{t\}$ \Comment{job $t$ arrives}
    \EndIf         
    \If {$t-d\in \buffer$} 
        \State $j=t-d$ \Comment{job $j=t-d$ becomes critical}
        \If{$\buffer \setminus\{j\} \neq\emptyset$}
            \State Choose $m(j) \in \argmax_{k \in \buffer \setminus\{j\} } \indexf(\type_{j}, \type_{k})$ \Comment{Ties are broken arbitrarily}
            \State $C\gets C\cup\{j\}$
            \State $\buffer \gets \buffer\setminus \{j,\matchof(j)\} $ \Comment{jobs $j,\matchof(j)$ are dispatched together}
        \Else
            \State $\buffer \gets \buffer\setminus \{j\} $ \Comment{job $j$ is dispatched by itself}
        \EndIf
    \EndIf
\EndFor
\State\Return $C, \{\matchof(j)\}_{j\in C}$
\end{algorithmic}
\end{algorithm}

As an example, the \newterm{naive greedy} algorithm ($\gre$) uses index function $\indexf_{\gre}(\type,\type') = \reward(\type,\type')$. When any job becomes critical, the algorithm chooses a match it with an available job to maximize the (instant) reward.
%
To illustrate the behavior of this algorithm under reward function $\reward(\type,\type') = \min\{\type,\type'\}$, we first make the following observation, that the algorithm always chooses to match each critical job with a higher type job when possible. 
%
\begin{remark}\label{deliverygreedy}
    When a job $j \in [\Njob]$ becomes critical, if the set $A_+(j) = \{k\in A(j):\type_k \ge \type_j\}$ is nonempty, then under the naive greedy algorithm $\gre$, we have $\matchof(j)\in A_+(j)$ and $\reward(\type_j,\type_{\matchof(j)}) = \type_j $. 
    % 
\end{remark}



\section{Potential-Based Greedy Algorithm}
% 
\label{sec:PB} 

We introduce in this section the potential-based greedy algorithm and prove that it substantially outperforms the naive greedy approach in terms of worst case regret.

The \newterm{potential-based greedy} algorithm ($\PB$) is an index-based greedy matching algorithm (as defined in \Cref{alg:dynamic}) whose index function is specified as
\begin{equation}
    \indexf_{\PB}(\type,\type') = \reward(\type, \type') - \potential(\type'),
\end{equation}
where $\potential(\type)$ is the \newterm{potential} of a job of type $\type$, formally defined as follows
%
\begin{equation}
    \potential(\type)=\frac{1}{2}\sup_{\type'\in\typespace} \reward(\type,\type'). \label{eq:defn_potential}
\end{equation} 
% 

This new notion of potential can be interpreted as an optimistic measure of the marginal value of holding on to a job that could be later matched with a job that maximizes instant reward.
%
Intuitively, this "ideal" matching outcome appears as the optimal matching solution when the density of the market is arbitrarily large. We provide a formal statement of this intuition in \Cref{sec: interpretation}.
%
Note that this definition of potential is quite simple in the sense that it relies on only in the knowledge of the reward topology: reward function and type space.
% 
In particular, for our topology of interest, $\reward(\type,\type')=\min\{\type,\type'\}$ on $\typespace=[0,1]$, the ideal scenario is achieved when two identical jobs are matched and the potential $\potential(\type)=\type/2$ is simply proportional to the length of a solo trip, capturing that longer deliveries have higher potential reward from being pooled with other jobs in the future.

We begin by giving a straightforward interpretation for this algorithm under the linear city model and our reward of interest, as we did for the naive greedy algorithm $\gre$ in \Cref{deliverygreedy}.

\begin{remark}\label{deliverypotential}
    Under the 1-dimensional type space $\Theta = [0,1]$ and the reward function $\reward(\type,\type')=\min\{\type,\type'\}$, 
    % 
    the potential-based greedy algorithm $\PB$ always matches each critical job to an available job that's the closest in space, since 
    % 
    \begin{align*}
        \matchof(j) 
        \in \argmax_{k\in A(j) } \indexf_\PB(\type_j,\type_k) 
        = \argmin_{k\in A(j) } |\type_j-\type_k|. 
    \end{align*}    % 
    This follows from the fact that $ |\type_j-\type_k| = \type_j + \type_k - 2\min\{\type_j, \type_k\} = \type_j - 2\indexf_\PB(\type_j,\type_k) $.
\end{remark}

In the rest of this section, we first assume $d = \infty$ and study the \textit{offline} performance of various index-based greedy matching algorithms under the reward function $\reward(\type,\type') = \min\{\type,\type'\}$.
% 
This will later help us analyze the algorithms' \emph{online} performances when $d<\infty$.
Results for the two alternative reward structures defined in \eqref{eq:defn_reward_B} and \eqref{eq:defn_reward_C} are reported in \Cref{sec:reward2} and  \Cref{sec: reward 3}, respectively.
%


\subsection{Offline Performance under Reward Function $\reward(\type,\type') = \min\{\type,\type'\}$}\label{sec: static}

Suppose $\sojourn=\infty$, meaning that all jobs are avilable to be matched when the first job becomes critical. We study the \emph{offline} performance of both the naive greedy algorithm $\gre$ and potential-based greedy algorithm. We show that the regret of $\gre$ grows linearly with the number of jobs, while the regret of $\PB$ is logarithmic.


\begin{proposition}[proof in \Cref{pf:greedy_linear_lower_bound}]\label{prop:greedy_linear_lower_bound}
Under reward function $\reward(\type,\type') = \min\{\type,\type'\}$, when the number of jobs $\Njob$ is divisible by 4, there exists an instance $\instance \in [0,1]^\Njob$ for which $\regret_\gre(\instance,\infty) \ge n/4$.
\end{proposition}
%

In particular, the \newterm{regret per job} of $\gre$, i.e.\ dividing the regret by $\Njob$, is constant in $\Njob$.
%
In contrast, we prove in the following theorem that $\PB$ performs substantially better. In fact, its regret per job gets better for larger market sizes $\Njob$.

\begin{theorem}
\label{thm:potential_log_upper_bound}
    Under reward function $\reward(\type,\type') = \min\{\type,\type'\}$, we have $\regret_\PB(\instance,\infty) \le 1 + \log_2(\Njob/2+1)/2$, for any $\Njob$, and any instance $\instance \in [0,1]^\Njob$.
\end{theorem}
%
\proof{Proof.}
Let $\instance \in \typespace^\Njob$, and $(C,\matchof(\cdot))$ be the output of $\PB$ on $\instance$ (generated as in \Cref{alg:dynamic}). In particular, 
$\PB(\instance,\infty) = \sum_{j \in C} \reward(\type_j,\type_{\matchof(j)})$.
On the other hand, since $\reward(\type,\type') \le \potential(\type) + \potential(\type')$ for all $\type,\type'
\in\typespace$, we have
\begin{align}
    \OPT(\instance,\infty) = \sum_{(j,k) \in \matchset_\OPT} \reward(\type_j,\type_k) \le \sum_{j=1}^\Njob \potential(\type_j) \le \sup_{\type\in[0,1]}\potential(\type) + \sum_{j\in C} \potential(\type_j) + \potential(\type_{\matchof(j)}),\nonumber
\end{align}
where the last inequality comes from the fact the $\PB$, and in fact any index-based matching algorithm, leaves at most one job unmatched (when $\Njob$ is odd).
%
Hence,
\begin{align}
\regret_\PB(\instance,\infty)&=
\OPT(\instance,\infty)-\PB(\instance,\infty) \\
&\le \sup_{\type\in \typespace}\potential(\type) + \sum_{j\in C} \left( \potential(\type_j) + \potential(\type_{\matchof(j)}) - \reward(\type_j,\type_{\matchof(j)})\right) \nonumber\\
&= \frac12 + \sum_{j \in C} \frac{\type_j}{2} + \frac{\type_{\matchof(j)}}{2} - \min\{\type_j,\type_{m(j)}\} \nonumber\\
&= \frac12 + \sum_{j \in C} \frac{|\type_j - \type_{\matchof(j)}|}{2}\label{eq: offline_regret_bound}. 
\end{align}
%
To bound the right-hand side, we first prove the following. For each $j\in C$, consider the interval $I_j=( \min\{\type_j,\type_{\matchof(j)}\},\max\{\type_j,\type_{\matchof(j)}\})\subseteq[0,1]$ and $\depth(j)=|\{k : I_j \subseteq I_k\}|$. We argue that
\begin{equation}\label{eq: distancebound}
    |\type_j - \type_{\matchof(j)}| \le 2^{1-\depth(j)}.
\end{equation}
%

First, note that $\{I_j\}_{j\in C}$ is a \newterm{laminar set family}: for every $j,k\in C$, the intersection of $I_j$ and $I_k$ is either empty, or equals $I_j$, or equals $I_k$. Indeed, without loss of generality, assume $j<k$. Then, when job $j$ becomes critical we have $\matchof(j),k,\matchof(k) \in A(j) $. Therefore, from \Cref{deliverypotential}, $\matchof(j)$ is the closest to $j$ and thus $k,\matchof(k)\notin I_j$, i.e.\ either $I_j \cap I_k = \emptyset$, or $I_j \cap I_k = I_j$.
%
Moreover, if $I_j\subsetneq I_k$ (i.e. $I_j\subsetneq I_k$ and $I_j\neq I_k$), then necessarily $j < k$, since otherwise $k$ becomes critical first and having $j,\matchof(j) \in A(k)$ closer in space, $\PB$ would not have chosen $\matchof(k)$.
%
Now, we can prove \eqref{eq: distancebound} by induction. If $\depth(j)=1=|\{j\}|$, clearly $|\type_j-\type_{\matchof(j)}|\le 1$. Suppose that the statement is true for $\depth(j)=\depth \ge 1$. If $\depth(j)=\depth+1$ then there exists $k$ such that $I_j\subsetneq I_k$ and $\depth(k)=\depth$. By the previous property, $j<k$ and since when job $j$ becomes critical we had $k,\matchof(k)\in A(j)$, it must be the case that
\begin{align*}
    |\type_j-\type_{\matchof(j)}| 
    &<\max\{ \min\{\type_j,\type_{m(j)}\} - \min\{\type_k,\type_{m(k)}\},\max\{\type_k,\type_{m(k)}\}-\max\{\type_j,\type_{m(j)}\}\}\\
    &\le \min\{\type_j,\type_{m(j)}\} - \min\{\type_k,\type_{m(k)}\} + \max\{\type_k,\type_{m(k)}\} - \max\{\type_j,\type_{m(j)}\} \\
    &= |\type_k-\type_{\matchof(k)}| - |\type_j-\type_{\matchof(j)}|.
\end{align*}
Thus, $|\type_j-\type_{\matchof(j)}|\le|\type_k-\type_{\matchof(k)}|/2\le 2^{1-(\depth+1)}$,
completing the induction.
%
    
Having established \eqref{eq: distancebound}, we use it to derive an upper bound for $\sum_{j\in C}|\type_j-\type_{\matchof(j)}|$. Note that since $\depth(j)\in \{1,\ldots,\floor{\Njob/2}\}$, we have
\begin{align} \label{eq: distanceboundbylayer}
\sum_{j \in C} |\type_j - \type_{\matchof(j)}| = \sum_{\depth = 1}^{\floor{\Njob/2}}\sum_{j\in C:\depth(j) = \depth} |\type_j - \type_{\matchof(j)}|.
\end{align}
%
For every $\depth$, we have $\sum_{j\in C:\depth(j) = \depth} |\type_j - \type_{\matchof(j)}|\le \min\{1,2^{1-\depth}|\{j:\depth(j)=\depth\}|\}$ by the laminar property and \eqref{eq: distancebound}.
%
Let $z_\depth$ denote $2^{1-\depth}|\{j:\depth(j)=\depth\}|$, where we note that 
\begin{align*}\label{eq: numofintervalsbound}
    \sum_{\depth=1}^{\floor{\Njob/2}} 2^{\depth-1}z_\depth = \sum_{\depth=1}^{\floor{\Njob/2}} |\{j:\depth(j)=\depth\}| = |C| \le \Njob/2.
\end{align*}
We can then use the following LP to upper-bound the value of~\eqref{eq: distanceboundbylayer} under an adversarial choice $\{z_\depth:\depth=1,\ldots,\lfloor n/2\rfloor\}$:
\begin{maxi}
    {z}{ \sum_{\depth=1}^{\floor{\Njob/2}} z_\depth }
    {\label{eq: knapsack}}{}
    \addConstraint{ \sum_{\depth=1}^{\floor{\Njob/2}} 2^{\depth-1}z_\depth }{\le \Njob/2}
    \addConstraint{0\le z_\depth }{\le 1,\quad }{ \depth = 1,\ldots,\floor{\Njob/2}.}
\end{maxi}
This is a fractional knapsack problem, whose optimal value is at most $\log_2({\Njob/2+1})$, and thus
\begin{equation}\label{dist_bound}
    \sum_{j \in C} |\type_j - \type_{\matchof(j)}| \le \log_2(\Njob/2+1).
\end{equation}
%
Substituting back into \eqref{eq: offline_regret_bound} yields $\regret_\PB(\instance,\infty) \le 1/2 + \log_2(\Njob/2+1)/2$, completing the proof.
\Halmos\endproof

The following result shows that this analysis of $\PB$ is tight, up to constants.

\begin{proposition}[proof in \Cref{pf:loglowerboundOffline}]
\label{prop:loglowerboundOffline}
Under reward function $\reward(\type,\type') = \min\{\type,\type'\}$, when the number of jobs is $\Njob=2^{k+3} - 4$ for some integer $k\ge 0$, there exists an instance $\instance \in [0,1]^\Njob$ for which $\regret_\PB(\instance,\infty) \ge (\log_2(n+4)-3)/4$.
\end{proposition}
%

%%%%%%%%%%%%%%%%%%%%%%%%%%%%%%%%%%%%%%%%%%%%%%%%%%%%%%%%

\subsection{Online Performance under Reward Function $\reward(\type,\type') = \min\{\type,\type'\}$}\label{sec:dynamic}

We now study the performance of algorithms in the online setting, i.e. when $\sojourn<\Njob$, and derive informative performance guarantees conditional on $\sojourn$, the number of new arrivals before a job becomes critical.
%
In fact, the regret per job of $\gre$ remains constant, whereas the one of $\PB$ decreases with $\sojourn$.
%
To build upon our results proved in \Cref{sec: static} for the offline setting, we partition the set of jobs into $b=\lceil\Njob/(\sojourn + 1) \rceil$ \newterm{batches}.
%
To be precise, we define the $t$-th batch as $B_{t} = \{(t-1)(\sojourn+1)+1,\ldots, t(\sojourn+1)\} \cap [\Njob]$, for each $t = 1, \ldots, b$.
%
Scaling $\sojourn$ while keeping the number of batches $b$ fixed can be interpreted as increasing the density of the market.
%
We first show that the regret under $\gre$ remains linear in the number of jobs $\Njob$, independent of $\sojourn$, which can be understood as suffering a regret of order $\Omega(\sojourn)$ per batch.

\begin{proposition}[proof in \Cref{pf:greedy_linear_lower_bound_dynamic}]
\label{prop:greedy_linear_lower_bound_dynamic}
    Under reward topology $\reward(\type,\type') = \min\{\type,\type'\}$, if $(\sojourn+1)$ is divisible by 4, then for any number of jobs $\Njob$ divisible by $(\sojourn+1)$, there exists an instance $\instance \in [0,1]^\Njob$ for which $\regret_\gre(\instance,\sojourn) \ge n/4$.
\end{proposition}

In contrast, the following theorem shows that the performance of $\PB$ improves as market density increases.

\begin{theorem}\label{thm: dynamic}
Under reward topology $\reward(\type,\type') = \min\{\type,\type'\}$, we have $\regret_\PB(\instance,\sojourn) \le 1/2 + (\frac{\Njob}{d+1}+1)(1+\log(\sojourn+2))/2$ for any $\Njob$, and any instance $\instance\in[0,1]^\Njob$.
\end{theorem}
%
\proof{Proof.}
Let $\instance\in \typespace^\Njob$ and $\sojourn\ge 1$. Let $(C,\matchof(\cdot))$ be the output of $\PB$ on $\instance$.
Similar to the proof of \Cref{thm:potential_log_upper_bound}, we have
\begin{align}
\regret_\PB(\instance,\sojourn)
& \le \sum_{j=1}^\Njob \potential(\type_j) - \PB(\instance,\sojourn)  \nonumber \\
& \le \sup_{\type\in\typespace}\potential(\type) + \sum_{j \in C} \left( p(\theta_j) + p(\theta_{m(j)}) - r(\theta_j,\theta_{m(j)}) \right) \nonumber \\ 
& = \sup_{\type\in\typespace}\potential(\type) + \sum_{t=1}^{b} \sum_{j\in  C\cap B_t} \left( p(\theta_j) + p(\theta_{m(j)}) - r(\theta_j,\theta_{m(j)}) \right) \nonumber \\
& = \frac12 + \frac12 \sum_{t=1}^{b} \left(\sum_{j\in  C\cap B_t} |\type_j-\type_{\matchof(j)}| \right), \label{eq:online_dist}
\end{align}
where we split the sum into the $b = \lceil \Njob/(\sojourn + 1) \rceil$ batches, defined as $B_{t} = \{(t-1)(\sojourn+1)+1,\ldots, t(\sojourn+1)\} \cap [\Njob]$ for $t = 1, \ldots, b$. We analyze the term in large parentheses for an arbitrary $t$. Intuitively, we would like to consider an offline instance consisting of jobs $(\type_j)_{j\in C \cap B_t}$ and their matches $(\type_{\matchof(j)})_{j\in C\cap B_t}$, and apply \Cref{thm:potential_log_upper_bound} on the offline instance which has size $2|C\cap B_t|\le 2(\sojourn+1)$.
However, since the offline setting allows jobs to observe the full instance, as opposed to only the next $\sojourn$ arrivals, the resulting matching of $\PB$ on the offline instance could be inconsistent with its output on the online instance. In particular, executing $\PB$ on the offline instance could match jobs with indices more than $d$ apart, which is not possible in the online setting. Thus, to derive a proper upper bound on regret, we need to construct a modified offline instance for which each resulting match of $\PB$ coincides with the matching output in the online counterpart, and in which matching distances were not decreased.

%
To construct this modified offline instance, first note that if $\matchof(j)\in B_t$ for all $j\in C\cap B_t$, then no modification is needed, since all matched jobs have indices less than $\sojourn$ apart.
%
However, if $\matchof(j)\in B_{t+1}$ for some $j\in C\cap B_t$, then $\matchof(j)$ can be available in the offline instance for some job $k<j$ with $|\matchof(j)-k|>\sojourn$ and be matched to $k$ in the offline instance even though this would not be possible in the online instance (see \Cref{fig:thm3_exm}).
%
Thus, in the modified offline instance, we "move" job $m(j)$ ensure job $k$ would not choose $m(j)$ for its match.
%
To do so, we distinguish three cases.
%
For each $j\in C\cap B_t$, let $A(j)$ be the set of jobs that the online algorithm could have chosen from to match with $j$, i.e.~$A_j$ is the set $A\setminus\{j\}$ in \Cref{alg:dynamic} right after job $j$ becomes critical.
%
If $\matchof(j)\in A(j)\cap B_t$, then both jobs $j,\matchof(j)$ are included in the offline instance we construct.
%
In the second case, if $\matchof(j)\in A(j)\cap B_{t+1}$ and $A(j)\cap B_t\neq\emptyset$, then we modify $\type_{m(j)}$ to be equal to the best available matching candidate within batch i.e.\ we "move" job $m(j)$ to location $\argmin \{ |\type_j - \type| : \type = \type_k, \ k\in A(j)\cap B_t \}$.
%
In the third case, if $\matchof(j)\in A(j)\cap B_{t+1}$ and $A(j)\cap B_t=\emptyset$, then we don't consider job $j$ or its match (if any) in the offline instance we construct. This can happen at most once per batch, since any other job $k<j$ would have $j\in A(k)$. 
%

To formally define the construction, fix an arbitrary batch $t$. Let
$\hat{C}=\{j\in C\cap B_t: A(j)\cap B_t\neq \emptyset\}$ and $\hat{n}= 2|\hat{C}|\le 2(\sojourn+1)$. 
%
$\hat{C}$ is the set of critical jobs in cases one or two above, and we include jobs $j\in\hat{C}$ and their matches $m(j)$ in the offline instance.  There are $\hat{n}$ jobs in the offline instance and the ordering of indices is consistent with the original instance.
Define modified locations in the offline instance as follows:
\begin{align*}
\hat{\type}_j &= \type_j &\text{for $j\in\hat{C}$}
\\ \hat{\type}_{m(j)} &= \type_{m(j)} &\text{for $j\in\hat{C}$, if $m(j)\in A(j)\cap B_t$ (case one)}
\\ \hat{\type}_{m(j)} &= \argmin \{ |\type_j - \type| : \type = \type_k, \ k\in A(j)\cap B_t \} &\text{for $j\in\hat{C}$, if $m(j)\in A(j)\cap B_{t+1}$ (case two)}
\end{align*}
%
%
\begin{figure}[H]
  \centering
  \subcaptionbox{Batch $B_t$ of the online instance $\instance$.
    \label{fig:thm3_exm_original}
    }[0.4 \textwidth]
    {
    \begin{tikzpicture}


% Background shading for periods 6t+1 to 6t+6
\fill[lightyellow] (1.8, 0) rectangle (6.2, 3.2);

\draw[->] (-0.5, 0) -- (8.5, 0) node[right] {$j$};
\draw[->] (0, -0.1) -- (0, 3.3) node[above] {$\type_j$};

\foreach \x in {1,...,8}
    \draw (\x, -0.1) -- (\x, 0.1);

\node[below] at (1, 0) {\tiny $5t$};
\node[below] at (2, 0) {\tiny $5t+1$};
\node[below] at (3, 0) {\tiny $5t+2$};
\node[below] at (4, 0) {\tiny $5t+3$};
\node[below] at (5, 0) {\tiny $5t+4$};
\node[below] at (6, 0) {\tiny $5t+5$};
\node[below] at (7, 0) {\tiny $5t+6$};
\node[below] at (8, 0) {\tiny $5t+7$};

\foreach \y in {0.1,0.2,0.3,0.4,0.5,0.6,0.7,0.8,0.9,1}
    \draw (-0.1, \y*3) -- (0.1, \y*3) node[left, xshift=-0.2cm] {\tiny \y};
    

\foreach \x/\y in {1/0.9, 2/0.1, 3/0.7, 4/0.3, 5/0.4, 6/1, 7/0.2, 8/0.8} 
{
    \fill[black] (\x, \y*3) circle (0.1);
    \draw[gray, dashed] (0, \y*3) -- (\x, \y*3);
    }
    

% Matches
\draw[thick] (1, 0.9*3) -- (3, 0.7*3);     
\draw[thick] (2, 0.1*3) -- (4, 0.3*3);
\draw[thick] (5, 0.4*3) -- (7, 0.2*3);     
\draw[thick] (6, 1*3) -- (8, 0.8*3);  

\end{tikzpicture}
    }
  \hfill
  \subcaptionbox{Modified offline instance $\hat{\instance}$. 
    \label{fig:thm3_exm_phantom}
    }[0.4 \textwidth]
    {
    \begin{tikzpicture}

\draw[->] (-0.5, 0) -- (4.5, 0) node[right] {$j$};
\draw[->] (0, -0.1) -- (0, 3.3) node[above] {$\type_j$};

\foreach \x in {1,...,4}
    \draw (\x, -0.1) -- (\x, 0.1);

\node[below] at (1, 0) {\tiny $5t+1$};
\node[below] at (2, 0) {\tiny $5t+3$};
\node[below] at (3, 0) {\tiny $5t+4$};
\node[below] at (4, 0) {\tiny $5t+6$};

\foreach \y in {0.1,0.2,0.3,0.4,0.5,0.6,0.7,0.8,0.9,1}
    \draw (-0.1, \y*3) -- (0.1, \y*3) node[left, xshift=-0.2cm] {\tiny \y};
    

\foreach \x/\y in {1/0.1, 2/0.3, 3/0.4, 4/1} 
{
    \fill[black] (\x, \y*3) circle (0.1);
    \draw[gray, dashed] (0, \y*3) -- (\x, \y*3);
    }
    

% Matches
\draw[thick] (1, 0.1*3) -- (2, 0.3*3);     
\draw[thick] (3, 0.4*3) -- (4, 1*3);


\end{tikzpicture}
    }
  % 
\caption{Illustration of the construction of the modified offline instance, with $\sojourn=4$. In \Cref{fig:thm3_exm_original}, job $5t+1$ chooses job $5t+3$ in the online execution of $\PB$, because only jobs up to $5t+5$ would have arrived when job $5t+1$ becomes critical.  However, in the offline execution of $\PB$, job $5t+1$ would choose job $5t+6$, because all jobs are available.  Our modified instance in \Cref{fig:thm3_exm_phantom} corrects the inconsistent decision.
}
  % 
  \label{fig:thm3_exm}
\end{figure}
%
Now, we show that the output of (offline) $\PB$ on $\hat{\instance}$ provides an upper bound for the expression $\sum_{j\in C\cap B_t}|\type_j-\type_{\matchof(j)}|$ in~\eqref{eq:online_dist}.
%
In particular, we show that the output of $\PB$ is exactly $(\hat{C},m(\cdot))$ when executed on the modified offline instance.
%
First, note that the construction potentially introduces ties if there is a job $k\in C\cap B_t$ with $\matchof(k)\in B_{t+1}$. However, since \Cref{thm:potential_log_upper_bound} holds under arbitrary tie-breaking, we assume $\PB$ breaks ties on $\hat{\instance}$ to maintain consistent decisions with $\PB$ on the original instance $\instance$. Thus, from \Cref{deliverypotential}, it suffices to show that for every $j\in \hat{C}$, $\matchof(j)$ minimizes the distance among available jobs.
%

%
Indeed, note that every job in the offline instance has a type that is identical to a job in $B_t$. Then, proceeding in the same order as in the online instance, the job types of the set of available jobs in the offline instance is a subset of the original available jobs. For a match in case one, we know that it minimizes the distance among the original available jobs, and hence it minimizes the distance among the offline available jobs. Moreover, we have $|\hat{\type}_j-\hat{\type}_{\matchof(j)}| = |\type_j - \type_{\matchof(j)}|$. In the second case, the match minimizes distance among offline available jobs by construction, and moreover $|\hat{\type}_j-\hat{\type}_{\matchof(j)}| \ge |\type_j - \type_{\matchof(j)}|$.
%
Consequently, we can upper bound 
\[ \sum_{j\in C\cap B_t } |\type_j-\type_{\matchof(j)}|
\le 1 + \sum_{j\in  \hat{C}} |\hat{\type}_j-\hat{\type}_{\matchof(j)}|. 
\]
%  
Then, recalling \eqref{dist_bound} in the proof of \Cref{thm:potential_log_upper_bound}, we have $\sum_{j\in  \hat{C}} |\hat{\type}_j-\hat{\type}_{\matchof(j)}| \le \log(\hat{n}/2 + 1) \le \log(\sojourn + 2)$, and then
\begin{align}
    \sum_{j\in C\cap B_t } |\type_j-\type_{\matchof(j)}|
\le 1 +  \log(\sojourn + 2).\label{eq: distance_dynamic}
\end{align}
Substituting back into~\eqref{eq:online_dist}, and using the fact that $b\le\frac n{d+1}+1$, we get $\regret_\PB(\instance,\sojourn)\le 1/2 + (\frac{\Njob}{d+1}+1)(1+\log(\sojourn+2))/2$, completing the proof. 
\Halmos\endproof
%
Lastly, we show that this online analysis of $\PB$ is also tight, up to constants.
%
\begin{proposition}[proof in \Cref{pf:loglowerboundOnline}]
\label{prop:loglowerboundOnline}
Under reward topology $\reward(\type,\type') = \min\{\type,\type'\}$, if $\sojourn+1 = 2^{k+3}-4$ for some $k\in\{0,1,\ldots\}$, then for any number of jobs $\Njob$ divisible by $(\sojourn+1)$, there exists an instance $\instance \in [0,1]^\Njob$ for which $\regret_\PB(\instance,\sojourn) \ge \frac{\Njob}{3(\sojourn+1)}(\log_2(\sojourn+5)-3)/4$.
\end{proposition}
%

%%%%%%%%%%%%%%%%%%%%%%%%%%%%%%%%%%%%%%%%%%%%%%%%%%%%%%

% \subsection{Interpretation of Potential}\label{sec: interpretation}

% When a job $j$ becomes critical, our potential-based greedy algorithm matches it to an available job $k$ maximizing $r(\theta_j,\theta_k)-p(\theta_k)$, where $p(\theta_k)$ can be interpreted as the opportunity cost of matching job $k$, with the specific definition $p(\theta_k)=\frac12 \sup_{\theta\in\Theta} r(\theta_k,\theta)$.
% This is an optimistic measure of opportunity cost because it assumes that job $k$ would otherwise be matched to an "ideal" type $\theta\in\Theta$ maximizing $r(\theta_k,\theta)$ (with half of this ideal reward $\sup_{\theta\in\Theta} r(\theta_k,\theta)$ attributed to job $k$).
% We now prove that this ideal reward can indeed be achieved under asymptotically-large market thickness, for a stochastic model under our topology of interest.

% \begin{definition}
% Let $\instance \in \typespace^\Njob$.
% For any job $j\in[\Njob]$, let $\instance^{-j}\in \typespace^{\Njob-1}$ be the same instance with the exception that job $j$ is not present. Meanwhile, let $\instance^{+j}\in \typespace^{\Njob+1}$ be the same instance with the exception that an additional copy of job $j$ is present.
% Consider the following definitions.
% \begin{enumerate}
% \item Marginal Loss: $\marginalloss_j(\instance) = \OPT(\instance) - \OPT(\instance^{-j})$
% \item Marginal Gain: $\marginalgain_j(\instance) = \OPT(\instance^{+j}) - \OPT(\instance)$
% \end{enumerate}
% \end{definition}

% Note that definitions $\marginalloss_j(\instance),\marginalgain_j(\instance)$ are based solely on offline matching, and we will use them as our definitions of opportunity cost if the future was known.  One could alternatively use shadow prices from the LP relaxation of the offline matching problem, but we note that the LP is not integral.  In either case, there is no ideal definition of opportunity cost that is guaranteed to lead to the optimal offline solution in matching problems \citep[see][]{cohen2016invisible}.

% We now establish the following \namecref{lem:marginal_as_interval} to help analyze the opportunity costs $\marginalloss_j(\instance),\marginalgain_j(\instance)$.


% \begin{lemma}\label{lem:marginal_as_interval}
% Let $\typespace=[0,1]$ and $\reward(\type,\type') = \min\{\type,\type'\}$. Consider an instance $\instance\in\typespace^n$ and relabel the indices to satisfy $\type_1 \ge \type_2 \ge \ldots \ge \type_n$. Then,
% \begin{align*}
% \marginalloss_j(\instance)
% &=\sum_{k\ge j,k\ \mathrm{even}}(\theta_k-\theta_{k+1}), 
% \\ \marginalgain_j(\instance)
% &=\sum_{k\ge j,k\ \mathrm{odd}}(\theta_k-\theta_{k+1}),
% \end{align*}
% where we consider $\theta_{n+1}=0$.
% \end{lemma}
% %
% \proof{Proof.}
%     Let $\instance \in \typespace^\Njob$ such that $\type_1 \ge \type_2 \ge \ldots \ge \type_n$. 
%     %
%     Then, $\OPT(\instance) =\sum_{k\ \mathrm{even}}\theta_k$, and moreover
%     \begin{align*}
%     \OPT(\instance^{-j}) &=\sum_{k<j, k\ \mathrm{even}}\theta_k+\sum_{k>j, k\ \mathrm{odd}}\theta_k
%     \\ \OPT(\instance^{+j}) &=\theta_j+\OPT(\instance^{-j})
%     \end{align*}
%     % 
%     Therefore,
%     % 
%     \begin{align*}
%     \marginalloss_j(\instance)
%     &=\sum_{k\ge j,k\ \mathrm{even}}\theta_k
%     -\sum_{k>j,k\ \mathrm{odd}}\theta_k
%     =\sum_{k\ge j,k\ \mathrm{even}}(\theta_k-\theta_{k+1})
%     \\ \marginalgain_j(\instance)
%     &=\theta_j-\sum_{k\ge j,k\ \mathrm{even}}(\theta_k-\theta_{k+1})
%     =\sum_{k\ge j,k\ \mathrm{odd}}(\theta_k-\theta_{k+1})
%     \end{align*}
%     % 
%     completing the proof.
% \Halmos\endproof

% Note that because $\OPT(\instance^{+j}) = \OPT(\instance^{-j}) + \type_j$ and $\potential(\type_j) = \reward(\type_j,\type_j)/2$ for this reward function, we immediately get the following \namecref{cor: marginal_average}.

% \begin{corollary}\label{cor: marginal_average}
%         If $\typespace=[0,1]$ and $\reward(\type,\type') = \min\{\type,\type'\}$, then for all jobs $j$,
%     \[ \potential(\type_j) = \frac{\marginalloss_j(\instance) + \marginalgain_j(\instance)}{2}. \]
% \end{corollary}

% We are now ready to prove our main result about the interpretation of potential, that the true opportunity costs $\marginalloss_j(\instance),\marginalgain_j(\instance)$ concentrate around $p(\theta_j)$ in a random uniform instance, assuming the market is sufficiently thick.  We without loss consider job $j=1$ and fix its type $\theta_1$.

% \begin{theorem} \label{thm:interpretation}
% Let $\typespace=[0,1]$ and $\reward(\type,\type') = \min\{\type,\type'\}$.
% Fix $\theta_1\in\typespace$ and suppose $\theta_2,\ldots,\theta_n$ are drawn IID from the uniform distribution over $[0,1]$, forming a random instance $\instance\in\typespace^n$, for some $n\ge 2$. Then, both $\mathbb{E}[\marginalloss_1(\instance)]$ and $\mathbb{E}[\marginalgain_1(\instance)]$ are within $O(1/n)$ of $\potential(\type_1)$ and moreover $\mathrm{Var}(\marginalloss_1(\instance))=\mathrm{Var}(\marginalgain_1(\instance)) = O\left(\frac{1}{n}\right)$.
% \end{theorem}
% %
% \proof{Proof.}
% We prove the statement only for $\marginalloss_1(\instance)$, and the analogous result for $\marginalgain_1(\instance)$ follows from \Cref{cor: marginal_average}.
% %
% If $\type_1=0$, then $\marginalloss_1(\instance)=0$ for any $\instance$, coinciding with $\potential(\type_1)=0$. Then, for the remainder of the proof, assume that $\type_1>0$.
% %
% Consider the random variable $N=|\{j:\theta_j<\theta_1\}|$, which counts the number of points between $0$ and $\theta_1$ and has distribution $\text{Binom}(\Njob-1,\type_1)$.
% %
% These $N$ points divide the interval $[0,\theta_1]$ into $N+1$ intervals with total length $\type_1$.
% %
% From \Cref{lem:marginal_as_interval}, the marginal loss $\marginalloss_1(\instance)$ is determined by computing the total length of a subset of these intervals. % Maybe add figure?
% %
% It is known that if $N$ random variables are drawn independently from $\text{Unif}[0,1]$, then the joint distribution of the induced interval lengths is $\text{Dirichlet}(1,1,\ldots,1)$ with $N+1$ parameters all equal to 1 (i.e., drawn uniformly from the simplex).
% %
% In particular, since the Dirichlet distribution is symmetric, the sum of any $k$ of these intervals is equal in distribution to the $k$-th smallest sample ($k$-th order statistic) of the $N$ uniform random variables, whose distribution is known to be $\text{Beta}(k,N+1-k)$.
% %
% Thus, conditional on $N$, with $N\ge 1$, since the distribution of each of the $N$ jobs to the left of $\type_1$ is $\text{Unif}[0,\type_1]$, the distribution of $\marginalloss_1(\instance)/\type_1$ is $\text{Beta}(k_L,N+1-k_L)$,
% where $k_L$ is either $\lceil\frac{N+1}2\rceil$ or $\floor{\frac{N+1}2}$.
% %
% To be precise, for $N\ge 1$, $k_L=\floor {N/2}+1=\lceil\frac{N+1}2\rceil$ if $n$ is even, and $k_L=\ceil{N/2}=\floor{\frac{N+1}2}$ if $n$ is odd.
% %
% Moreover, if $N=0$, then $\marginalloss_1(\instance)=\type_1$ if $n$ is even, and $\marginalgain_1(\instance)=0$ if $n$ is odd.
% %
% Hence,
% \begin{align*}
% \mathbb{E}[\marginalloss_1(\type) \mid N] 
% &= \type_1\frac{k_L}{N+1}
% % \label{eq: cond_exp}
% \end{align*}
% for all $N\in\{0,\ldots,n-1\}$.
% %
% Since $|k_L-(N+1)/2|\le 1$, then $|\mathbb{E}[\marginalloss_1(\type) \mid N]-\type_1/2|\le \type_1/(N+1)$, thus
% % 
% \begin{align*}
% &\left|\mathbb{E}[\marginalloss_1(\type)]-\frac{\type_1}{2}\right| 
% \le \type_1\mathbb{E}\left[\frac{1}{N+1}\right] \\
% &\qquad = \type_1\sum_{k=0}^{n-1} \frac{1}{k+1}\binom{n-1}{k}\type_1^k(1-\type_1)^{n-1-k} \\
% &\qquad = \frac{\type_1}{n}\sum_{k=0}^{n-1} \binom{n}{k+1}\type_1^k(1-\type_1)^{n-1-k} \\
% &\qquad = \frac{1}{n}\sum_{k=1}^{n} \binom{n}{k}\type_1^k(1-\type_1)^{n-k} \\
% &\qquad = \frac{1-(1-\type_1)^{n}}{n} = O\left(\frac{1}{n}\right).
% \end{align*}
% %
% This completes the proof of the statement about $\mathbb{E}[\marginalloss_1(\instance)]$.

% For the statement about variance, we know $\mathrm{Var}(\marginalloss_1(\type)) = \mathbb{E}[\mathrm{Var}(\marginalloss_1(\type) \mid N)] + \mathrm{Var}(\mathbb{E}[\marginalloss_1(\type)\mid N])$ by the law of total variance. For the first term, we have $\mathrm{Var}(\marginalloss_1(\type) \mid N) = \type_1^2\frac{k_L(N+1-k_L)}{(N+1)^2(N+2)}\indicator\{N\ge 1\}\le \frac{\type_1^2}{N+1}$, and then from the previous argument $\mathbb{E}[\mathrm{Var}(\marginalloss_1(\type) \mid N)]=O(1/n)$.
% %
% For the second term, 
% \begin{align*}
% \mathrm{Var}(\mathbb{E}[\marginalloss_1(\type)\mid N]) 
% &= \mathrm{Var}\left(\mathbb{E}[\marginalloss_1(\type)\mid N] - \frac{\type_1}{2}\right) \\
% &\le \mathbb{E} \left[\left(\mathbb{E}[\marginalloss_1(\type)\mid N] - \frac{\type_1}{2}\right)^2\right] \\
% % 
% &\le \mathbb{E}\left[\frac{\type_1^2}{(N+1)^2}\right] \\
% &\le \type_1^2\mathbb{E}\left[\frac{1}{N+1}\right] = O\left(\frac{1}{n}\right).
% \end{align*}
% Thus, $\mathrm{Var}(\marginalloss_1(\instance)) = O\left(1/n\right)$, completing the proof.
% \Halmos\endproof



\section{Numerical Experiments} \label{sec:numerical_experiments}

In this section, we show via numerical simulations that under the reward structure from delivery pooling, our proposed potential-based greedy algorithm ($\PB$) outperforms a number of benchmark algorithms, given sufficient density.
%
We first generate synthetic data under a setting more aligned with our theoretical results, and find that $\PB$ outperforms all benchmarks starting from very low market densities ($\sojourn\ge 5$).
%
We then apply all algorithms and benchmarks on real data from the Meituan platform.
%
Although the real-life setting differs from our theoretical model in a number of ways --- that we address in \Cref{sec:sim_meituan} --- the results remain qualitatively similar: $\PB$ performs the best, starting from the modest density level corresponding to ``each job is able to wait for one minute before being dispatched''.
%
Additional simulation results, including additional performance metrics and more general synthetic environments (e.g. two-dimensional locations, different spatial distributions, and different reward topologies) are provided in \Cref{sec: sim_results_extra}.


The benchmark algorithms that we compare with are categorized into two groups, \emph{forecast-agnostic}, and \emph{forecast-aware}, depending on the use of historical data/demand distributional information to make matching decisions. We now describe each of them.
%


\subsection{Benchmark Algorithms} \label{sec:sim_benchmarks}


As we have discussed earlier, the potential based greedy algorithm $\PB$ does not rely on any historical data/demand forecast.
% 
The same is true for naive greedy $\gre$, and we refer to heuristics with this property as \newterm{forecast-agnostic}. 
%
In this group, we consider batching policies as additional benchmarks, which are highly relevant both in theory and in practice.


\paragraph{Naive Batching.} 


Batching heuristics periodically compute optimal matching solutions given the available jobs.
% 
Since it is without loss in our model to wait to match, we consider batching algorithms that re-compute optimal matchings whenever any available job becomes critical.
%
% 
The \newterm{naive batching} algorithm ($\batching$) dispatches all available jobs according to this optimal solution. 
%
In particular, under the theoretical model introduced in \Cref{sec:model}, $\batching$ clears the market every $\sojourn + 1$ arrivals, which is equivalent to the one analyzed in \citet{ashlagi2019edge}. 
%
However, an obvious suboptimality of this algorithm is that there are pairs of jobs that are matched and dispatched before any of them becomes critical. 



\paragraph{Rolling Batching.} 

To address the previously mentioned issue of dispatching non-critical pairs of jobs, we also consider a modified batching heuristic, termed \newterm{rolling batching} ($\rbatching$).
%
$\rbatching$ also computes an optimal matching solution every time any available job becomes critical. Instead of dispatching all jobs, however, $\rbatching$ only dispatches the critical job and its match (if any) according to the optimal solution. 


\medskip 


In practice, delivery platforms typically have access to past order data that allows to predict patterns of future arrivals to some extent. 
%
In contrast to the aforementioned algorithms, we say that a policy is \newterm{forecast-aware} if it uses historical data to make pooling decisions. 
%
We consider dual-based benchmark algorithms that use optimal dual solutions (i.e. shadow prices) to approximate the opportunity cost of matching each job. 
% 
More specifically, given an instance from the historical data (i.e. a sequence of delivery requests with information about origin, destination, and timestamp), we can solve the dual program of a linear relaxation of the IP defined in \eqref{eq: OPT}.
% 
\footnote{We provide a detailed description of the primal and dual linear programs in \Cref{sec: LP}.}
% 
The optimal dual variables associated with the capacity constraint of each job will then provide a proxy for the marginal value of the job, which we use to construct two additional benchmark algorithms.


\paragraph{Hindsight Dual.}
% 
Ideally, if we had access to the dual optimal solution for an instance in advance, we would use them as the opportunity cost of matching each job.
%
In an attempt to measure the value of this hindsight information, we simulate a \newterm{hindsight-dual} algorithm ($\dual$) that operates retrospectively on the same instance.
%
Formally, given an instance $\instance\in\typespace^\Njob$ and parameter $\sojourn$, let $\lambda_k=\lambda_k(\instance,\sojourn)$ be the optimal dual variable associated with the constraint $\sum_{k:j\neq k} x_{jk} \le 1$ in the linear program  (defined in \eqref{eq: LP}). The $\dual$ algorithm operates as an index-based greedy matching algorithm (as defined in \Cref{alg:dynamic}) where the index function is given by $\indexf_{\dual}(\type_j,\type_k) = \reward(\type_j,\type_k) - \lambda_k$, for all $j,k\in [\Njob]$, such that $j\neq k, |j-k| \le d$.
% 
Note that this index function is computed based on the entire realized instance $\instance$, thus this algorithm is only for benchmarking purposes and is not practically feasible.



\paragraph{Average Dual.}
%
A more realistic approach for estimating the opportunity costs is to average the hindsight duals of the jobs from historical data.
% 
However, this estimation is not free of challenges. 
% 
In particular, computing the dual solutions yields a shadow price for every observed type, which might not cover the entire type space and thus 
many future arrivals may not have an associated shadow price.
% 
To address this issue, we discretize the type space and estimate average duals for each discrete type as the empirical average of the shadow prices for jobs associated to the discrete type.
% 
To be precise, let $H$ be the set of all instances in the historical data, and consider a discretization $\{\typespace_1,\typespace_2,\ldots,\typespace_p\}$
% 
that partitions the space into $p$ cells (i.e. these cells are mutually exclusive and jointly exhaustive).
%
Each cell $i=1,2,\ldots,p$ is associated with a shadow price $\Bar{\lambda}_i$ that is the average shadow price of historical jobs lying in the same cell, i.e. $\Bar{\lambda}_i$ is the average of $\{\lambda(\instance,\sojourn) : \instance\in H, \type_j \in \typespace_i \}$.
%
Then, the \newterm{average-dual} algorithm
($\averagedual$) is an index-based greedy matching algorithm that uses the shadow prices for the cells, i.e.\ uses index function
$\indexf_{\averagedual}(\type,\type') = \reward(\type,\type') - \sum_{i=1}^p\Bar{\lambda}_{i}\indicator\{\type'\in\typespace_i\}.$

\subsection{Synthetic Data}\label{sec:sim_unif_1D}

We first consider the setting analyzed in our theoretical results, where job destinations are uniformly distributed on a one-dimensional space.
% 
We fix the total number of jobs at $\Njob=1000$, 
and compare algorithms at density levels $\sojourn\in\{5,10,15,20,25,30\}$.
%
For each $(\Njob, \sojourn)$ pair, we generate 100 instances $\instance \in[0,1]^\Njob$ uniformly at random, apply all algorithms on each of them, and compute the average regret and reward \newterm{ratio} ($\ALG(\instance,d)/\OPT(\instance,d)$) achieved by each algorithm and benchmark.
% 
Additional performance metrics, including the fraction of jobs that are pooled and the fraction of the total distance that is reduced by pooling, are presented in \Cref{sec:match_rate}. Similar results for the 1D, non-uniform setting are presented in \Cref{sec: nonunif_1D}.
% 

\subsubsection{Comparison with forecast-agnostic heuristics.}
%
\Cref{fig:1D} compares potential-based greedy $\PB$ with naive greedy $\gre$ and the two batching algorithms.
% 
As expected from our analysis, $\gre$ performs poorly even at high densities. To our surprise, $\PB$ outperforms both batching algorithms for all considered density values.
%
To assess the performance of the algorithms relative to the hindsight optimal, we also compute the \emph{ratio} for each instance, and plot its empirical average in \Cref{fig:1D_ratio}.
% 
We can see that even at relatively low density levels, $\PB$ achieves over 95\% of the maximum possible travel distance saved from pooling. $\gre$, on the other hand, performs worse relative to hindsight optimal as density increases--- despite the fact that the
%the fraction of
travel distance saved by $\gre$ increases with density (see \Cref{fig:1D_unif_saving_fraction} in \Cref{sec: 1D_extra}). 
%
\begin{figure}%[H]
  \centering
  \subcaptionbox{Average regret.%
    \label{fig:1D_regret}}[0.49 \textwidth]{\includegraphics[width = \figWidth \textwidth]{Simulation_Results/Synthetic/New/1D_Pooling/1D_Common_Origin/Forecast-Agnostic/regret.png}}
  \hfill
  \subcaptionbox{Average ratio.%
    \label{fig:1D_ratio}}[0.49 \textwidth]{\includegraphics[width = \figWidth \textwidth]{Simulation_Results/Synthetic/New/1D_Pooling/1D_Common_Origin/Forecast-Agnostic/frac_of_OPT.png}}
  % 
  \caption{Comparison of average regret and reward ratio for forecast-agnostic heuristics in random 1D instances.}
  % 
  \label{fig:1D}
\end{figure}
\subsubsection{Comparison with forecast-aware heuristics.}

We now compare $\PB$ with our two forecast-aware benchmarks, $\dual$ and $\averagedual$.
%
%
Under $\dual$, for each of the 100 instances generated for each market condition $(n,d)$, we compute a shadow price for each job via the dual LP.
%
As discussed in \Cref{sec:sim_benchmarks}, this algorithm is not practically feasible, and we present its performance here mainly to illustrate the value of this hindsight information used in this particular way.
%
%
For $\averagedual$, for each $(n,d)$, we first generate 400 \emph{historical instances} in the same way that the original 100 instances were drawn.
% 
%
We then compute optimal dual variables for each historical instance, and estimate average shadow prices $\Bar{\lambda}$ for a uniform discretization of the type space $[0,1]$ into 100 intervals.
% 

\Cref{fig:1D_forecastaware} compares the average regret and reward ratio on the original 100 instances.
%
We can see that $\PB$ performs on par with $\averagedual$ across all density parameters, without relying on any forecast/historical data. 
%
Notably, even with access to hindsight information, $\dual$ is consistently dominated by both $\averagedual$ and $\PB$, with the performance gap narrowing as $\sojourn$ increases.
%


\begin{figure}%[H]
  \centering
  \subcaptionbox{Average regret.%
    \label{fig:1D_forecastaware_regret}}[0.49 \textwidth]{\includegraphics[width = \figWidth \textwidth]{Simulation_Results/Synthetic/New/1D_Pooling/1D_Common_Origin/Forecast-Aware/regret.png}}
  \hfill
  \subcaptionbox{Average ratio.%
    \label{fig:1D_forecastaware_ratio}}[0.49 \textwidth]{\includegraphics[width = \figWidth \textwidth]{Simulation_Results/Synthetic/New/1D_Pooling/1D_Common_Origin/Forecast-Aware/frac_of_OPT.png}}
  % 
  \caption{Comparison of average regret and reward ratio for forecast-aware heuristics in random 1D instances. }
  % 
  \label{fig:1D_forecastaware}
\end{figure}




\subsection{Order-Level Data from Meituan Platform}\label{sec:sim_meituan}


In this \namecref{sec:sim_meituan}, we test the algorithms and benchmarks using data from the Meituan platform, made public for the 2024 INFORMS TSL Data-Driven Research Challenge.\footnote{\url{https://connect.informs.org/tsl/tslresources/datachallenge}, accessed January 15, 2025.
%
See \Cref{sec:market_dynamics} for more details on the dataset, as well as high-level illustrations of market dynamics over both space and time.
} 
%
The dataset provides detailed information on various aspects of the platform's delivery operations, for one city and a total of 8 days in October 2022.  
%
For each of the 569 million delivery orders, the dataset provides the pick-up and drop-off locations (the latitudes and longitudes were shifted for privacy considerations), and the timestamps at which orders are created, pushed into the dispatch system, etc.
% 
% 
This allows us to (i) construct instances of job arrivals and (ii) calculate travel distances and pooling opportunities.




Since the pick-up and drop-off locations now reside in a two-dimensional (2D) space, we first extend our definition of pooling reward to the 2D space, derive the potential of different job types, and provide the notion of a pooling window (i.e. the sojourn time) that determines which jobs can be pooled together.


\paragraph{Pooling reward.}
% 
Each job $j$ has type $\type_j = (\xorigin_j,\xdestination_j)$, 
% 
with $\xorigin_j,~\xdestination_j\in \R^2$ corresponding to the coordinates of the job origin and destination.
%
The travel distance saved by pooling two orders together is the difference between (i) the travel distance required to fulfill the jobs in two separate trips, and (ii) the minimum travel distance for fulfilling the two requests in a single, pooled trip.
%
% 
For the pooled trip, the travel distance always includes the distance between the two origins and that between the two destinations (both orders must be picked up before either of them is dropped off; otherwise the orders are effectively not pooled). 
% 
The remaining distance depends on the sequence in which the jobs are picked up and dropped off, resulting in four possible routes. The minimum travel distance is then determined by evaluating the total distance for all four routes and selecting the smallest value.
% 
Formally, letting $\norm{\cdot}$ denote the Euclidean norm, the reward function is
% 
\begin{align}
    \reward(\type,\type') = &  \norm{\xdestination-\xorigin} +  \norm{\xdestination'-\xorigin'}  \nonumber\\
    % 
    & -\left(\norm{\xorigin-\xorigin'} + \min\left\{\norm{\xdestination-\xorigin},\norm{\xdestination'-\xorigin'},\norm{\xdestination-\xorigin'},\norm{\xdestination'-\xorigin'}\right\}+\norm{\xdestination-\xdestination'}\right). \label{eq:2D_reward_part_2}
\end{align}
% 
Note that unlike the 1D common origin setting, the pooling reward in 2D space with heterogeneous origins is not guaranteed to be positive. As a result, we assume that all index-based greedy algorithms only pool jobs together when the reward is non-negative.

\paragraph{Potential.}
% 
For a given job type $\type=(\xorigin,\xdestination)$, the maximum reward a platform can achieve by pooling it with another job is the distance of the job, $\norm{\xdestination - \xorigin}$.\footnote{To see this, first observe that when a job $\type = (\xorigin,\xdestination)$ is pooled with another job of identical type, the travel distance saved is precisely $\norm{\xdestination - \xorigin}$.
%
On the other hand, let $m$ be the minimum in \eqref{eq:2D_reward_part_2}. Then, by triangle inequality $\|O'-D'\|\le \|O-O'\| + \|D-O\| + \|D-D'\|$, and thus $\reward(\type,\type')\le 2\|D-O\|-m \le \|D-O\|$, since $m\le \|D-O\|$.
% % 
To achieve an even higher pooling reward, it must be the case that the second part of the pooling reward \eqref{eq:2D_reward_part_2} is strictly smaller than $\norm{\xdestination' - \xorigin'}$. This implies that the total distance traveled to visit all four locations is smaller than the distance between two of them, which is impossible.}  

As a consequence, the potential of a job $\type$ is $\potential(\type) = \norm{\xdestination - \xorigin}/2$. %, 
% 
This is aligned with our 1D theoretical model, in which longer deliveries have higher potential reward from being pooled with other jobs in the future.
%

\paragraph{Pooling window.} 
% 
Instead of assuming that each job is available to be pooled for a fixed number of future arrivals, we take the actual timestamps of the jobs, and assume that the jobs are available for a fixed time window, at the end of which the order becomes critical.
%
Intuitively, the length of the pooling window corresponds to the ``patience level'' of the jobs.
% 
This deviates from our assumptions for the theoretical model, but it is straightforward to see how our algorithm as well as the benchmarks we compare with can be generalized and applied.


\subsubsection{Comparison with forecast-agnostic heuristics.}

We focus our analysis on a three-hour lunch period (10:30am to 1:30pm) for the 8 days of data provided by the challenge.
% 
During this lunch period, there are approximately $130$ orders per minute on average, thus $\Njob\approx 24,000$ orders in total (see \Cref{fig:meituan_orders_per_hour} in \Cref{sec:market_dynamics} for an illustration of order volume over time).
% 
If the platform allows a time window of one minute for each order before it has to be dispatched, the average density would be around $130$ jobs. 
% 
\Cref{fig:meituan} compares the forecast-agnostic heuristics, as the pooling window increases from 30 seconds to 5 minutes. We can see that a single minute is sufficient for $\PB$ to outperform every other forecast-agnostic algorithm, achieving around 80\% of the hindsight optimal reward.
% 
Moreover, the performance of $\PB$ improves steadily as the allowed matching window increases.
%

\begin{figure}%[H]
  \centering
  \subcaptionbox{Average regret.%
    \label{fig:meituan_regret}}[0.49 \textwidth]{\includegraphics[width = \figWidth \textwidth]{Simulation_Results/Meituan/City/regret_fagnostic.png}}
  \hfill
  \subcaptionbox{Average ratio.%
    \label{fig:meituan_ratio}}[0.49 \textwidth]{\includegraphics[width = \figWidth \textwidth]{Simulation_Results/Meituan/City/ratio_fagnostic.png}}
  % 
  \caption{Comparison of average regret and reward ratio for forecast-agnostic heuristics in Meituan data.}
  % 
  \label{fig:meituan}
\end{figure}


\subsubsection{Comparison with forecast-aware heuristics.}

We now compare our algorithm with the forecast-aware heuristics on data from the same three-hour lunch period, 10:30am to 1:30pm.
% 
To simulate the \emph{hindsight-dual} algorithm $\dual$ for each instance associated to each day, we compute the shadow prices via the dual LP program and then compute the ``online'' pooling outcomes using pooling reward minus these shadow prices as the index function.
%
% 
To estimate historical average shadow prices for the $\averagedual$ algorithm, for each of the 8 days, we (i) use the remaining 7 days as the historical data, and (ii) discretize the 2D space into discrete types based on job origin and destination using H3, the hexagonal spatial indexing system developed by Uber.\footnote{See {\url{https://www.uber.com/blog/h3/} (accessed January 7, 2024) for more details on the H3 package.} 
% 
% 
The system supports sixteen resolution levels, each tessellating the earth using hexagons of a particular size.
% %
For a given resolution, the set of all origin-destination (OD) hexagon pairs represents the partition of the type space $\typespace$, as described in \Cref{sec:sim_benchmarks}.
% %
% % 
To compute the average shadow price for a particular OD hexagon pair, we compute the average of optimal dual variables associated with all historical jobs that share the same hexagon pair.
% %
If there are no jobs for a given pair, however, then we consider the hexagon pairs at coarser resolutions (moving one or more levels up in the H3 hierarchy, with each level increasing the size of the hexagons by a factor of 7) until there exist associated historical jobs.
% %
There is a tradeoff between granularity and sparsity when we choose the baseline resolution level, as finer resolutions only average over trips with close-by origins and destinations, but may lack sufficient historical data for accurate estimation (see \Cref{fig:meituan_CDF_count_per_OD_pair} for the distribution of order volume by OD hexagon pairs).
% 
After testing all available resolution levels, we found that levels 8 and 9 perform the best, depending on the time window. The results presented in this section report the result corresponding to the best resolution for each time window.
}

\Cref{fig:meituan_forecastaware} compares the regret and reward ratio of $\PB$ to forecast-aware heuristics.
%
We first observe that in contrast to the synthetic setting from \Cref{sec:sim_unif_1D}, $\dual$ now incurs the smallest regret at all density levels. %
% 
This performance difference appears to be driven by the highly non-uniform spatial distribution of jobs --- additional simulations, presented in \Cref{sec:sim_unif_1D_2D}, demonstrate that (i) $\dual$ performs no better than $\PB$ if job origins and destinations are drawn uniformly at random from a unit square, and (ii) when jobs are generated from a non-uniform distribution, $\dual$ slightly outperforms $\PB$. Of course, we remind the reader that $\dual$ is not a realistic algorithm because it requires knowing the exact future.

The worse performance of $\averagedual$ in comparison to $\PB$ also highlights the challenges of estimating the opportunity costs of pooling different jobs from historical data in real-world settings. 
% 
Intuitively, taking historical data into consideration should be valuable in scenarios with highly non-uniform spatial distributions. In this case, however, the sparsity of the data for many if not most origin-destination pairs makes it difficult to properly estimate the opportunity costs, or requires a very coarse granularity for location.  $\PB$ bypasses this sparsity-granularity tradeoff by only considering the reward topology, i.e.\ the space of all possible types.
%


\begin{figure}%[H]
  \centering
  \subcaptionbox{Average regret.%
    \label{fig:meituan_forecastaware_regret}}[0.49 \textwidth]{\includegraphics[width = \figWidth \textwidth]{Simulation_Results/Meituan/City/regret_faware.png}}
  \hfill
  \subcaptionbox{Average ratio.%
    \label{fig:meituan_forecastaware_ratio}}[0.49 \textwidth]{\includegraphics[width = \figWidth \textwidth]{Simulation_Results/Meituan/City/ratio_faware.png}}
  % 
  \caption{Comparison of average regret and reward ratio for forecast-aware heuristics in Meituan data.}
  % 
  \label{fig:meituan_forecastaware}
\end{figure}


% \subsubsection{Match rate and saving fraction.}\label{sec:match_rate}
% % \paragraph{Match Rate and Saving Fraction.}

% In addition to regret and reward ratio, we consider in this section two additional performance metrics: the \newterm{match rate}, i.e. the fraction of jobs that were pooled instead of dispatched on their own, and  the \newterm{saving fraction}, i.e. the fraction of the total distance that is reduced by pooling, relative to the total travel distance without any pooling~\citep[see][]{aouad2020dynamic}.
% %
% % 
% For brevity, we focus only on the potential-based greedy algorithm $\PB$ and the hindsight optimum $\OPT$ in \Cref{fig:meituan_2}.
% % 
% A comprehensive comparison across all 
% algorithms and benchmarks is provided in \Cref{sec: meituan_extra}.


% \begin{figure}[H]
%   \centering
%   \subcaptionbox{Average match rate.%
%     \label{fig:meituan_2_match_rate}}[0.49 \textwidth]{\includegraphics[width = \figWidth \textwidth]{Simulation_Results/Meituan/City/match_rate.png}}
%   \hfill
%   \subcaptionbox{Average saving fraction.%
%     \label{fig:meituan_2_saving_fraction}}[0.49 \textwidth]{\includegraphics[width = \figWidth \textwidth]{Simulation_Results/Meituan/City/saving_rate.png} }
%   % 
%   \caption{Match rate and Saving Fraction, Meituan Order-Level Data.}
%   % 
%   \label{fig:meituan_2}
% \end{figure}

% Both regret and reward ratio are performance metrics that compare pooling algorithm relative to the hindsight optimal pooling outcome. The saving fraction
% % 
% as shown in \Cref{fig:meituan_2_saving_fraction} illustrates the benefit of delivery pooling in comparison to the total distance traveled, and is upper bounded by 0.5 (since reward from pooling a job cannot exceed its distance).
% % 
% We can see that if the platform allows a time window of at least $1$ minute for each order before it has to be dispatched, $\PB$ achieves a reduction in distance traveled by over 20\%. At $5$ minutes, this fraction increases to roughly 30\%.
% %
% Since the saving fraction is proportional to the total reward collected by the algorithm, our previous performance comparisons imply that the saving fraction achieved by $\PB$ is the best among all tested \emph{practical} heuristics, i.e. excluding $\dual$ and $\OPT$.

% \Cref{fig:meituan_2_match_rate} compares the match rate achieved under $\PB$ and $\OPT$. We can see that $\PB$ consistently pools more jobs than $\OPT$ (5-10\%). 
% %
% This highlights an additional desirable property of $\PB$ and, more generally, of \emph{index-based greedy matching} algorithms: they pool as many jobs as possible.
% % 
% This is desirable in practice, since drivers who are offered just only one order from a platform may try to pool orders from competing platforms \citep[see e.g.][]{reddit2022grubhub,reddit2023doordash}, leading to poor service reliability for customers.

% Note that for the one-dimensional settings, match rates are very close to 100\% under greedy policies since at most one job is left unmatched (see \Cref{fig:1D_unif_match_rate} in \Cref{sec: 1D_extra} for comparisons). 
% %
% In the real-data setting, match rates are generally lower since we pool the critical job with another only when it's possible to achieve a positive pooling reward. Nevertheless, \Cref{sec: meituan_extra} shows that greedy algorithms ($\gre$, $\PB$, $\dual$, $\averagedual$) still achieve substantially higher match rates in comparison to $\batching$, $\rbatching$, and $\OPT$.




\section{Concluding Remarks and Future Work} \label{sec:conclusion}

In this work, we study dynamic non-bipartite matching in the context of pooling delivery orders.
% 
Our modeling approach captures two key features of the problem through the reward topology and the assumption that jobs can be matched until they become \textit{critical}.
%
We propose a simple, forecast-agnostic potential-based greedy algorithm, specially designed for this reward topology, that performs surprisingly well both in synthetic and real data. It uses a new notion of potential as opportunity cost, that considers the value of keeping long-distance deliveries in the platform for better pooling opportunities in the future. We show that for our reward topology of focus, potential-based greedy improves upon greedy both in worst-case and empirical performance. Moreover, we showcase the robustness of potential-based greedy through numerical experiments in real data, outperforming several commonly used benchmark algorithms given enough market density.
%
We provide some intuition behind the success of this approach by showing through theory and experiments that different notions of opportunity cost are intimately related to our definition of potential, especially as density increases. 
%
Moreover, we believe that the main insight of long-distance jobs being more valuable to hold should apply even when pooling more than two orders into a single trip (see \citet{wei2023constant}); and we believe our notion of potential could be relevant even on non-metric reward structures, where it is a measure of e.g.\ the popularity of a volunteer position (see \citet{manshadi2022online}).
%
Formalizing and generalizing these results is an interesting avenue of future research.
%
Overall, these surprising results showcase the value of using the knowledge of reward topology to design application-specific algorithms. This general idea, which has received limited attention so far, could also be applied to other domains, such as ride-sharing or kidney exchange.

%\THEEndNotes
% \begingroup \parindent 0pt \parskip 0.0ex \def\enotesize{\normalsize} \theendnotes \endgroup

% Acknowledgments here
\ACKNOWLEDGMENT{
The authors would like to thank
Itai Ashlagi,
Omar Besbes,
Francisco Castro,
Yash Kanoria,
Jake Marcinek,
Rad Niazadeh,
Scott Rodilitz,
Daniela Saban, and
Alejandro Torrielo
for valuable comments and discussions.
}


% References here (outcomment the appropriate case)

% CASE 1: BiBTeX used to constantly update the references
%   (while the paper is being written).
% \bibliographystyle{informs2014trsc} % outcomment this and next line in Case 1
% \bibliography{DraftBib} % if more than one, comma separated

%\bibliographystyle{informs2014trsc} % outcomment this and next line in Case 1
%\bibliography{sample} % if more than one, comma separated

% CASE 2: BiBTeX used to generate mypaper.bbl (to be further fine tuned)
%%%%%%%%%%%%%%%%%%%%%%%%%%%%%%%%%%%%%%%%%%%%%%%%%%%%%%%%%%%%%%%%%%%%%%%%%%
%%
%%	Author Submission Template for Transportation Science (TRSC)
%%	INFORMS, <informs@informs.org>
%%	Ver. 1.00, June 2024
%%
%%%%%%%%%%%%%%%%%%%%%%%%%%%%%%%%%%%%%%%%%%%%%%%%%%%%%%%%%%%%%%%%%%%%%%%%%%
%
% Use dblanonrev for Double Anonymous Review submission
% Use sglanonrev for Single Anonymous Review submission
% For example, submission to Operations Research, OPRE will have
% \documentclass[opre,dblanonrev]{informs4}

%%% TRUE SUBMISSION
% \documentclass[trsc,dblanonrev]{informs4}
% \usepackage{eqndefns-left} % For checking the display equation width and equation environment definitions %

%%% ARXIV VERSION
\documentclass[trsc,sglanonrev]{informs4_hide}

\RequirePackage{tgtermes}
\RequirePackage{newtxtext}
\RequirePackage{newtxmath}
\RequirePackage{multirow,multicol}
\RequirePackage{xspace,dsfont}
\RequirePackage{endnotes}
% \let\footnote=\endnote


%\OneAndAHalfSpacedXI
% \OneAndAHalfSpacedXII % Current default line spacing
\DoubleSpacedXI % Double-spacing for 11-point font (TSL)
% \DoubleSpacedXII

% Optional LaTeX Packages
\usepackage{algorithm}
\usepackage{algpseudocode}
\algrenewcommand\algorithmicrequire{\textbf{Input:}}
\algrenewcommand\algorithmicensure{\textbf{Output:}}
\usepackage{tikz}
%
\usepackage{xcolor}
\definecolor{dark1}{RGB}{157, 58, 103}
\definecolor{dark2}{RGB}{161, 67, 0}
\definecolor{dark3}{RGB}{115, 102, 0}
\definecolor{dark4}{RGB}{2, 120, 50}
\definecolor{dark5}{RGB}{0, 116, 122}
\definecolor{dark6}{RGB}{18, 100, 176}
\definecolor{dark7}{RGB}{116, 75, 163}
\definecolor{mid1}{RGB}{225, 119, 163}
\definecolor{mid2}{RGB}{228, 128, 77}
\definecolor{mid3}{RGB}{182, 158, 21}
\definecolor{mid4}{RGB}{87, 182, 109}
\definecolor{mid5}{RGB}{0, 181, 190}
\definecolor{mid6}{RGB}{88, 162, 242}
\definecolor{mid7}{RGB}{177, 135, 229}
\definecolor{light1}{RGB}{255, 187, 231}
\definecolor{light2}{RGB}{255, 198, 151}
\definecolor{light3}{RGB}{247, 223, 104}
\definecolor{light4}{RGB}{152, 248, 171}
\definecolor{light5}{RGB}{72, 249, 255}
\definecolor{light6}{RGB}{164, 219, 255}
\definecolor{light7}{RGB}{244, 192, 255}
\definecolor{lightyellow}{RGB}{255, 255, 204}

%% Comments
%%% Comments  will be removed if \Comments = 0 %%%
\newcount\Comments
\Comments = 0
\newcommand{\kibitz}[2]{\ifnum\Comments=1{\color{#1}{#2}}\fi}

\newcommand{\mr}[1]{\kibitz{mid1}{[Matias: #1]}}
\newcommand{\hma}[1]{\kibitz{mid4}{[HMa: #1]}}


\newcount\Drop  
\Drop = 0
\newcommand{\todrop}[2]{\ifnum\Drop=1{\color{#1}{#2}}\fi}
\newcommand{\drop}[1]{\todrop{mid5}{[Drop: #1]}}


%%%% Highlights that becomes black if \highlighttrue is commented out 
\newif\ifhighlight 
\highlighttrue % comment out this line to disable highlights

\newcommand{\mradd}[1]{{\ifhighlight\color{dark1}\fi#1}}
\newcommand{\hmadd}[1]{{\ifhighlight\color{dark4}\fi#1}}


\newif\iftodo
% \todotrue % comment out this line to hide the TODOs 

\newcommand{\todo}[1]{\iftodo{\color{mid1!50!gray}{[TODO: #1]}}\fi}

\usepackage{color}              % Need the color package
% \usepackage[suppress]{color-edits}
\usepackage{color-edits}
\addauthor{Will}{blue}
\addauthor{Hma}{dark4}
%
\usepackage[colorlinks,
	citecolor = dark6,
	urlcolor = black,
	linkcolor = dark6,
    hypertexnames=false]{hyperref}
\usepackage[nameinlink]{cleveref}
\crefname{subsection}{subsection}{subsections}
\usepackage[short]{optidef}
\usepackage[font=scriptsize]{subcaption}
\newcommand{\figWidth}{0.45}

% Private macros here (check that there is no clash with the style)
%%%%%%%%%%%%%%%%%%%%%%%%%% NOTATION %%%%%%%%%%%%%%%%%%%%%%%%

% bold lowercase letters, for vectors
\newcommand\vzero{{\bm 0}}
\newcommand\vzeron{{\bm 0}_n}
\newcommand\vone{{\bm 1}}
\newcommand\vonen{{\bm 1}_{n}}
% 
\newcommand\vpi{{\bm\pi}}
\newcommand\vphi{{\bm\phi}}
\newcommand\veta{{\bm\eta}}
\newcommand\vmu{{\bm\mu}}
% \newcommand\vtheta{{\bm\theta}}
\newcommand\vtheta{{\BFtheta}}
\newcommand\vdelta{{\bm\delta}}
\newcommand\va{{\bm a}}
\newcommand\vb{{\bm b}}
\newcommand\vc{{\bm c}}
\newcommand\vd{{\bm d}}
\newcommand\ve{{\bm e}}
\newcommand\vf{{\bm f}}
\newcommand\vg{{\bm g}}
\newcommand\vh{{\bm h}}
\newcommand\vi{{\bm i}}
\newcommand\vj{{\bm j}}
\newcommand\vk{{\bm k}}
\newcommand\vl{{\bm l}}
\newcommand\vm{{\bm m}}
\newcommand\vn{{\bm n}}
\newcommand\vo{{\bm o}}
\newcommand\vp{{\bm p}}
\newcommand\vq{{\bm q}}
\newcommand\vr{{\bm r}}
% \newcommand\vs{{\bm s}}
\newcommand\vt{{\bm t}}
\newcommand\vu{{\bm u}}
\def\vv{{\bm v}}
\newcommand\vw{{\bm w}}
\newcommand\vx{{\bm x}}
\newcommand\vy{{\bm y}}
\newcommand\vz{{\bm z}}


% Number sets
\newcommand{\C}{\mathbb{C}}
\newcommand{\R}{\mathbb{R}}
\newcommand{\Q}{\mathbb{Q}}
\newcommand{\N}{\mathbb{N}}
\newcommand{\Z}{\mathbb{Z}}

% Personal definitions
\renewcommand{\P}{\mathbb{P}}
\newcommand{\geom}[1]{\text{Geom}(#1)}
\newcommand{\E}{\mathbb{E}}
\newcommand{\var}{\text{Var}}
\newcommand{\normal}{\mathcal{N}}
\newcommand{\indicator}{\mathds{1}}
%\newcommand{\converges}[2][n \to \infty]{\xrightarrow[#1]{#2}}
\newcommand{\converges}[2][]{\xrightarrow[#1]{#2}}
\newcommand{\equalin}[1]{\stackrel{#1}{=}}
\newcommand{\loss}[1]{\depth({#1})}
% \DeclareMathOperator*{\argmin}{arg\,min}
% \DeclareMathOperator*{\argmax}{arg\,max}
\newcommand{\infsum}[1]{\sum_{#1\ge 1}}
\newcommand{\finsum}[2]{\sum_{#1=1}^{#2}}
\newcommand{\abs}[1]{\left|#1\right|}
\newcommand{\norm}[2][]{\left\|#2\right\|_{#1}}
\newcommand{\innerprod}[2][]{\left\langle #2 \right\rangle_{#1}}
\newcommand{\ceil}[1]{\left\lceil #1 \right\rceil}
\newcommand{\floor}[1]{\left\lfloor #1 \right\rfloor}
% \newcommand{\norm}[1]{\left\| #1 \right\|}
\newcommand{\indicatorof}[1]{\indicator\left\{#1\right\}}
\newcommand{\eps}{\varepsilon}

% Highlight a newly defined term
\newcommand{\newterm}[1]{\textit{#1}}
\newcommand{\red}[1]{{\leavevmode\color{red}#1}}
\newcommand{\blue}[1]{{\leavevmode\color{blue}#1}}
\newcommand{\nblue}[1]{{\leavevmode\color{nblue}#1}}
\newcommand{\violet}[1]{{\leavevmode\color{violet}#1}}
\newcommand{\purple}[1]{{\leavevmode\color{purple}#1}}
\newcommand{\ngreen}[1]{{\leavevmode\color{ngreen}#1}}

% macros for dynamic pooling
\newcommand{\Njob}{n}
\newcommand{\type}{\theta}
\newcommand{\typespace}{\Theta}
\newcommand{\instance}{\vtheta}
\newcommand{\sojourn}{d}
\newcommand{\potential}{p}
\newcommand{\ALG}{\mathsf{ALG}}
\newcommand{\OPT}{\mathsf{OPT}}
\newcommand{\OFF}{\mathsf{OFF}}
\newcommand{\regret}{\mathsf{Reg}}
\newcommand{\gap}{\mathsf{GAP}}
\newcommand{\gre}{\mathsf{GRE}}
\newcommand{\PB}{\mathsf{PB}}
\newcommand{\batching}{\mathsf{BAT}}
\newcommand{\rbatching}{\mathsf{R-BAT}}
\newcommand{\pot}{\mathsf{POT}}
\newcommand{\Potloss}{\mathsf{Loss}}
\newcommand{\LP}{\mathsf{LP}}
\newcommand{\DualLP}{\mathsf{DLP}}
\newcommand{\matchset}{\mathcal{M}}
\newcommand{\buffer}{A}
\newcommand{\indexf}{q}
\newcommand{\noutput}{m}
\newcommand{\order}[1]{(#1)}
\newcommand{\depth}{\ell}
\newcommand{\matchof}{m}
\newcommand{\ltype}{\alpha}
\newcommand{\rtype}{\beta}
\newcommand{\jobset}{J}
\newcommand{\width}{w}
\newcommand{\reward}{r}
\newcommand{\supI}{\mathrm{A}}
\newcommand{\supII}{\mathrm{B}}
\newcommand{\supIII}{\mathrm{C}}
\newcommand{\basecase}{\mathrm{BC}}
% \newcommand{\xorigin}{\vx_{\mathrm{ori}}}
\newcommand{\xorigin}{O}
% \newcommand{\xdestination}{\vx_{\mathrm{dest}}}
\newcommand{\xdestination}{D}
\newcommand{\yorigin}{y_{\mathrm{origin}}}
\newcommand{\ydestination}{y_{\mathrm{destination}}}
\newcommand{\marginalgain}{\mathsf{MG}}
\newcommand{\marginalloss}{\mathsf{ML}}
\newcommand{\randmarginalgain}{\mathsf{RMG}}
\newcommand{\randmarginalloss}{\mathsf{RML}}
\newcommand{\spacing}{S}
\newcommand{\dual}{\mathsf{HD}}
\newcommand{\discmap}{\varphi}
\newcommand{\averagedual}{\mathsf{AD}}
% \newcommand{\unif}{\text{Unif}}

% Natbib setup for author-number style
\usepackage{natbib}
 \bibpunct[, ]{(}{)}{,}{a}{}{,}%
 \def\bibfont{\small}%
 \def\bibsep{\smallskipamount}%
 \def\bibhang{24pt}%
 \def\newblock{\ }%
 \def\BIBand{and}%


%% Setup of the equation numbering system. Outcomment only one.
%% Preferred default is the first option.
\EquationsNumberedThrough    % Default: (1), (2), ...
%\EquationsNumberedBySection % (1.1), (1.2), ...

%% Setup of theorem styles. Outcomment only one.
%% Preferred default is the first option.
\TheoremsNumberedThrough     % Preferred (Theorem 1, Lemma 1, Theorem 2)
%\TheoremsNumberedByChapter  % (Theorem 1.1, Lema 1.1, Theorem 1.2)
\ECRepeatTheorems  %  

% For new submissions, leave this number blank.
% For revisions, input the manuscript number assigned by the on-line
% system along with a suffix ".Rx" where x is the revision number.
\MANUSCRIPTNO{TRSC-0001-2024.00}

%%%%%%%%%%%%%%%%
\begin{document}
%%%%%%%%%%%%%%%%

% Outcomment only when entries are known. Otherwise leave as is and
%   default values will be used.
%\setcounter{page}{1}
%\VOLUME{00}%
%\NO{0}%
%\MONTH{Xxxxx}% (month or a similar seasonal id)
%\YEAR{0000}% e.g., 2005
%\FIRSTPAGE{000}%
%\LASTPAGE{000}%
%\SHORTYEAR{00}% shortened year (two-digit)
%\ISSUE{0000} %
%\LONGFIRSTPAGE{0001} %
%\DOI{10.1287/xxxx.0000.0000}%

% Author's names for the running heads
% Sample depending on the number of authors;
% \RUNAUTHOR{Jones}
% \RUNAUTHOR{Jones and Wilson}
% \RUNAUTHOR{Jones, Miller, and Wilson}
% \RUNAUTHOR{Jones et al.} % for four or more authors
% Enter authors following the given pattern:
%\RUNAUTHOR{}
% \RUNAUTHOR{Smith and Johnson}

\RUNAUTHOR{Ma, Ma, and Romero}





% Title or shortened title suitable for running heads. Sample:
% \RUNTITLE{Predictive Maintenance in Manufacturing}
% Enter the (shortened) title:
% \RUNTITLE{Optimal Resource Allocation in Humanitarian Logistics}
\RUNTITLE{Dynamic Delivery Pooling}

% Full title. Sample:
% \TITLE{Optimal Resource Allocation in Humanitarian Logistics: A Stochastic Programming Approach}
% Enter the full title:
% \TITLE{Optimal Resource Allocation in Humanitarian Logistics: A Stochastic Programming Approach}
\TITLE{Potential-Based Greedy Matching for Dynamic Delivery Pooling} 

% Block of authors and their affiliations starts here:
% NOTE: Authors with same affiliation, if the order of authors allows,
%   should be entered in ONE field, separated by a comma.
%   \EMAIL field can be repeated if more than one author
\ARTICLEAUTHORS{%
%\AUTHOR{John Doe,\textsuperscript{a} Jane Smith,\textsuperscript{b}}
%\AFF{\textsuperscript{a}Department of Industrial Engineering, University of XYZ, \EMAIL{john.doe@xyz.edu; \textsuperscript{b}Department of Computer Science, University of ABC, \EMAIL{jane.smith@abc.edu}} 
% \AUTHOR{John Smith}
% \AFF{Department of Logistics,
% University of XYZ, \EMAIL{john.smith@xyz.edu}}

% \AUTHOR{Emily Johnson}
% \AFF{Department of Logistics,
% University of ABC, \EMAIL{emily.johnson@abc.edu}}

% Enter all authors
\AUTHOR{Hongyao Ma}
\AFF{Graduate School of Business, Columbia University, New York, NY 10027, \EMAIL{hongyao.ma@columbia.edu}}
\AUTHOR{Will Ma}
\AFF{Graduate School of Business, Columbia University, New York, NY 10027, \EMAIL{wm2428@gsb.columbia.edu}}
\AUTHOR{Matias Romero}
\AFF{Graduate School of Business, Columbia University, New York, NY 10027, \EMAIL{mer2262@gsb.columbia.edu}}
} % end of the block

\ABSTRACT{%
We study the problem of pooling together delivery orders into a single trip, a strategy widely adopted by platforms to reduce total travel distance.
% % 
Similar to other dynamic matching settings, the pooling decisions involve a trade-off between immediate reward and holding jobs for potentially better opportunities in the future. 
%
In this paper, we introduce a new heuristic dubbed potential-based greedy ($\PB$), which aims to keep longer-distance jobs in the system, as they have higher potential reward (distance savings) from being pooled with other jobs in the future. 
%
This algorithm is simple in that it depends solely on the topology of the space, and does not rely on forecasts or partial information about future demand arrivals.
%
We prove that $\PB$ significantly improves upon a naive greedy approach in terms of worst-case performance on the line. Moreover, we conduct extensive numerical experiments using both synthetic and real-world order-level data from the Meituan platform. 
%
Our simulations show that $\PB$ consistently outperforms not only the naive greedy heuristic but a number of benchmark algorithms, including (i) batching-based heuristics that are widely used in practice, and (ii) forecast-aware heuristics that are given the correct probability distributions (in synthetic data) or a best-effort forecast (in real data).
%
We attribute the surprising unbeatability of $\PB$ to the fact that it is specialized for rewards defined by distance saved in delivery pooling.
}%

% % \FUNDING{This research was supported by [grant number, funding agency].}

% %Supplemental Material:
% %Data Ethics & Reproducibility Note:

% % Sample
% %\KEYWORDS{Stochastic programming, Decision support,Uncertainty, Disaster response, Optimization}

% % Fill in data. If unknown, outcomment the field
\KEYWORDS{On-demand delivery, Platform operations, Online algorithms, Dynamic matching} 

% %\HISTORY{Received: Month DD, YYYY; Accepted: Month DD, YYYY; Published Online: Month DD, YYYY}

\maketitle
%%%%%%%%%%%%%%%%%%%%%%%%%%%%%%%%%%%%%%%%%%%%%%%%%%%%%%%%%%%%%%%%%%%%%%

\section{Introduction} \label{sec:intro}

% 
On-demand delivery platforms have become an integral part of modern life, transforming how consumers search for, purchase, and receive goods from restaurants and retailers.
%
Collectively, these platforms serve more than $1.5$ billion users worldwide, contributing to a global market valued at over \$250 billion \citep{statista2024}.
%
Growth has continued at a rapid pace--- major companies such as DoorDash in the United States and Meituan in China reported approximately 25\% year-over-year growth in transaction volume in 2023 \citep{curry2024doordash,scmp2024meituan}. 
%
Managing this ever-expanding stream of orders poses significant operational challenges, particularly as consumers demand increasingly faster deliveries, and fierce competition compels platforms to continuously improve both operational efficiency and cost-effectiveness.

%
One strategy for improving efficiency and reducing labor costs is to \emph{pool} into a single trip multiple orders from the same or nearby restaurants.
% 
This is referred to as stacked, grouped, or batched orders, and is advertised to drivers as opportunities for increasing earnings and efficiency~\citep{doordash2024batched}.
% 
The strategy has been generally successful and very widely adopted~\citep{deliveroo2022,uberEatsMultiple,grabGroupedJobs}.
%
For example, data from Meituan, made public by the 2024 INFORMS TSL Data-Driven Research Challenge, show that more than 85\% of orders are pooled, rising to 95\% during peak hours and consistently remaining above 50\% at all other times (see \Cref{fig:meituan_pooled_orders_per_hour}).
%

\begin{figure}[hpbt]
    \centering
    \includegraphics[width=0.9\linewidth]{Simulation_Results/Meituan/City/meituan_fraction_of_pooled_orders_how.png}
    % 
    \caption{Fraction of pooled orders by hour-of-week in Meituan data. 
    The dataset includes $8$ days of order-level data from one city (see \Cref{sec:sim_meituan} for more details). The gray shade indicates peak lunch hours (10:30am-1:30pm), while the green shade indicates peak dinner hours (5pm-8pm). 
    }
    \label{fig:meituan_pooled_orders_per_hour}
\end{figure}

The delivery pooling approach introduces a highly complex decision-making problem, as orders arrive dynamically to the market, and need to be dispatched within a few minutes --- orders in the Meituan data are offered to delivery drivers an average of around five minutes after being placed, often before meal preparation is complete (see \Cref{fig:meituan_order_to_first_dispatch_how}).
% 
%
During this limited window of each order, the platform must determine which (if any) of the available orders to pool with, carefully weighing immediate efficiency gains against the uncertain, differential benefits of holding each candidate for future pooling opportunities.
%
Consequently, there remains a pressing need for more efficient and robust solution methods to inform the platform's delivery pooling decisions.


%
Similar problems have been studied in the context of various marketplaces such as ride-sharing platforms and kidney exchange programs, in a large body of work known as dynamic matching.
%
However, we argue that the reward structure for delivery pooling is fundamentally different. 
% 
First, many problems studied in the literature involves connecting two sides of a market, e.g. ridesharing platforms matching drivers to riders. 
% 
In such bipartite settings, the presence of a large number of agents of identical or similar types (e.g., many riders requesting trips originating from the same area) typically leads to less efficient outcomes.
% 
While the kidney exchange problem is not bipartite, long queues of ``hard-to-match'' patient-donor pairs can still form, when many pairs share the same combination of incompatible tissue types and blood types.

In stark contrast, it is unlikely for long queues to build up for delivery pooling, especially in dense\footnote{
Indeed, today's major platforms observe extremely large volumes. Meituan, for example, processes over 12 thousand 
orders per hour in one market during peak lunch periods (see \Cref{fig:meituan_orders_per_hour}).} markets, in that given enough orders, two of them will have closely situated origins and destinations, resulting in a good match.
%
In fact, many customers requesting deliveries originating from the same location is actually \textit{desirable}, since this reduces travel distance and parking stops.
% 
We find that delivery orders tend to indeed be highly concentrated in space, with most requests originating from popular restaurants in busy city centers and ending in residential neighborhoods (see \Cref{appx:meituan_spatial_distribution}). 
%
We emphasize that we are not studying the problem of matching (pooled) orders to couriers, but delivery pooling is a relevant first step regardless, as pooling orders efficiently before matching with couriers will effectively increase the amount of courier capacity.

In this work, we study delivery pooling and exploit its distinct \emph{reward topology}, that it is highly desirable to match two jobs of identical or very similar types, as described above.
%
We develop a simple "potential-based" greedy algorithm, which relies solely on the reward topology to determine the opportunity cost of dispatching each job.
% 
The intuition behind it is elementary --- long-distance jobs have higher \textit{potential} for cost savings when pooled with other jobs, and hence the online algorithm should generally prefer keeping them in the system, dispatching shorter deliveries first when it has a choice.

\subsection{Model Description}

Different models of arrivals and departures have been proposed in the dynamic matching literature (see \Cref{sec:dynMatchModels}), and we believe our insight about potential is relevant across all of them.
However, in this paper we focus on a single model well-suited for the delivery pooling application.

%
To elaborate, we consider a dynamic non-bipartite matching problem, with a total of $\Njob$ jobs arriving sequentially to be matched.
%
Jobs are characterized by (potentially infinite) types $\theta\in\Theta$ representing features (e.g. locations of origin and destination) that determine the reward for the platform.
%
Given the motivating application to pooled deliveries, we will always model jobs' types as belonging to some metric space.
%
Following \citet{ashlagi2019edge}, we assume that an unmatched job must be dispatched after $\sojourn$ new arrivals, which we interpret as a known \textit{internal} deadline imposed by the platform to incentivize timely service.
%
The platform is allowed to make a last-moment matching decision before the job leaves, termed as the job becoming \newterm{critical}.
%
At this point, the platform must decide either to match the critical job to another available one, collecting reward $r(\theta,\theta')$ where $\theta,\theta'$ are the types of the matched jobs and $r$ is a known reward function, or to dispatch the critical job on its own for zero reward.
%
The objective is to maximize the total reward collected from matching the $\Njob$ jobs.
%
We believe this to be an appropriate model for delivery platforms (cf. \Cref{sec:dynMatchModels}) because deadlines are known upon arrival and sudden departures (order cancellations) are rare.

Dynamic matching models can also be studied under different forms of information about the future arrivals.
%
Some papers \citep[e.g.][]{kerimov2024dynamic,aouad2020dynamic,eom2023batching,wei2023constant} develop sophisticated algorithms to leverage stochastic information, which is often necessary to derive theoretical guarantees under general matching rewards.
%
In contrast, our heuristic does not require any knowledge of future arrivals, and our theoretical results hold in an "adversarial" setting where no stochastic assumptions are made.
%
In our experiments, we consider arrivals generated both from stochastic distributions and real-world data, and find that our heuristic can outperform even algorithms that are given the correct stochastic distributions, under our specific reward topology.

\subsection{Main Contributions}

\paragraph{Notion of potential.}
% 
As mentioned above, we design a simple greedy-like algorithm that is based purely on topology and reward structure, which we term potential-based greedy ($\PB$). More precisely, $\PB$ defines the \newterm{potential} of a job type $\theta$ to be $p(\theta)=\sup_{\theta'\in\Theta} r(\theta,\theta')/2$, measuring the highest-possible reward obtainable from matching type $\theta$.  The potential acts as an opportunity cost, and the relevance of this notion arises in settings where the potential is heterogeneous across jobs. To illustrate, let $\Theta=[0,1]$ and $r(\theta,\theta')=\min\{\theta,\theta'\}$, a reward function used for delivery pooling as we will justify in \Cref{sec:model}.  Under this reward function, a job type (which is a real number) can never be matched for reward greater than its real value, which means that jobs with higher real value have greater potential.  Our $\PB$ algorithm matches a critical job (with type $\theta$) to the available job (with type $\theta'$) that maximizes
\begin{align} \label{eqn:potentialIntro}
r(\theta,\theta')-p(\theta')=\min\{\theta,\theta'\}-\frac12 \theta',
\end{align}
being dissuaded to use up job types $\theta'$ with high real values. Notably, this definition does not require any forecast or partial information of future arrivals, but rather assumes full knowledge of the universe of possible job types.

\paragraph{Theoretical results.}
Our theoretical results assume $\Theta=[0,1]$.  We compare $\PB$ to the naive greedy algorithm $\gre$, which selects $\theta'$ to maximize $r(\theta,\theta')$, instead of $r(\theta,\theta')-p(\theta')$ as in~\eqref{eqn:potentialIntro}.  We first consider an offline setting ($d=\infty$), showing that under reward function $r(\theta,\theta')=\min\{\theta,\theta'\}$, our algorithm $\PB$ achieves regret $O(\log n)$, whereas $\gre$ suffers regret $\Omega(n)$ compared to the optimal matching.
Our analysis of $\PB$ is tight, i.e.\ it has $\Theta(\log n)$ regret.
Building upon the offline analysis, we next consider the online setting ($d< n$), showing that $\PB$ has regret $\Theta(\frac nd\log d)$, which improves as $d$ increases. This can be interpreted as there being $\frac nd$ "batches" in the online setting, and our algorithm achieving a regret of $O(\log d)$ per batch.  Alternatively, it can be interpreted as the \newterm{regret per job} of our algorithm being $O(\frac d{\log d})$. By contrast, we show that the naive greedy algorithm $\gre$ suffers a total regret of $\Omega(n)$, regardless of $d$.

For comparison, we also analyze two other reward topologies, still assuming $\Theta=[0,1]$.
\begin{enumerate}
\item We consider the classical min-cost matching setting \citep{reingold1981greedy} where the goal is to minimize total match distance between points on a line, represented in our model by the reward function $r(\theta,\theta')=1-|\theta-\theta'|$.
%
For this reward function, both $\PB$ and $\gre$ have regret $\Theta(\log n)$; in fact, they are the same algorithm because all job types have the same potential. Like before, this translates into both algorithms having regret $\Theta(\frac nd \log d)$ in the online setting.
\item We consider reward function $r(\theta,\theta')=|\theta-\theta'|$, representing an opposite setting in which it is worst to match two jobs of the same type.  For this reward function, we show that any index-based matching policy must suffer regret $\Omega(n)$ in the offline setting, and that both $\PB$ and $\gre$ suffer regret $\Omega(n)$ (irrespective of $d$) in the online setting.
\end{enumerate}

Our theoretical results are summarized in \Cref{table:results}.
As our model is a special case of \citet{ashlagi2019edge}, their 1/4-competitive randomized online edge-weighted matching algorithm can be applied, which essentially translates in our setting to an $O(n)$ upper bound for regret under any definition of reward.
Our results show that it is possible to do much better ($O(\log n)$ instead of $O(n)$) for specific reward topologies, using a completely different algorithm and analysis.
We now outline how to prove our two main technical results, \Cref{thm:potential_log_upper_bound,thm: dynamic}.
%
\begin{table}[!t]
\centering
\begin{tabular}{|c|c|c|c|}
\hline
\updown $\theta,\theta'\in[0,1]$ & $r(\theta,\theta')=\min\{\theta,\theta'\}$ & $r(\theta,\theta')=1-|\theta-\theta'|$ & $r(\theta,\theta')=|\theta-\theta'|$ \\
\hline
\up\multirow{4}{*}{Regret of $\PB$} & Offline: $\Theta(\log n)$ & & \\
\down & (\Cref{thm:potential_log_upper_bound}, \Cref{prop:loglowerboundOffline}) & & \\
\up & Online: $\Theta(\frac nd \log d)$ & Offline: $\Theta(\log n)$ & Offline: $\Omega(n)$ \\
\down & (\Cref{thm: dynamic}, \Cref{prop:loglowerboundOnline}) & Online: $\Theta(\frac nd \log d)$ & Online: $\Omega(n)$ \\
\cline{1-2}
\up\multirow{4}{*}{Regret of $\gre$} & Offline: $\Omega(n)$ & (\Cref{sec:reward2}) & (\Cref{sec: reward 3}) \\
\down & (\Cref{prop:greedy_linear_lower_bound}) & & \\
\up & Online: $\Omega(n)$ & & \\
\down & (\Cref{prop:greedy_linear_lower_bound_dynamic}) & & \\
\hline
\end{tabular}
\caption{
Summary of theoretical results under different reward functions $r:[0,1]^2\to \R$.  The total number of jobs is denoted by $n$, and the batch size in the online setting is approximately $d$.
}
\label{table:results}
\end{table}
\paragraph{Proof techniques.}
%
We establish an upper bound on the regret of $\PB$ by comparing its performance to the sum of the potential of all jobs.
%
Under reward function $\reward(\type,\type')=\min\{\type,\type'\}$, the key driver of regret is the sum of the distances between jobs matched by $\PB$.
%
In \Cref{thm:potential_log_upper_bound}, we study this quantity by analyzing the intervals induced by matched jobs in the offline setting ($\sojourn=\infty$).
%
We show that the intervals formed by the matching output of $\PB$ constitute a \emph{laminar set family}; that is, every two intervals are either disjoint or one fully contains the other.
%
We then prove that more deeply nested intervals must be exponentially smaller in size. Finally, we use an LP to show that the sum of interval lengths remains bounded by a logarithmic function of the number of intervals.
%

In \Cref{thm: dynamic}, we partition the set of jobs in the online setting into roughly $\frac{\Njob}{d}$ "batches" and analyze the sum of the distances between matched jobs within a single batch.
%
Our main result is to show that we can use \Cref{thm:potential_log_upper_bound} to derive an upper bound for an arbitrary batch.
%
However, a direct application of the theorem on the offline instance defined by the batch would not yield a valid upper bound, because the resulting matching of $\PB$ in the offline setting could be inconsistent with its online matching decisions.
%
To overcome this challenge, we carefully construct a modified offline instance that allows for the online and offline decisions to be coupled, while ensuring that the total matching distance did not go down.  This allows us to upper-bound the regret per batch by $O(\log d)$, for a total regret of $O(\frac nd \log d)$.


\paragraph{Simulations on synthetic data.} We test $\PB$ on random instances in the setting of our theoretical results, except that job types are drawn uniformly at random from $[0,1]$ (instead of adversarial). 
%
We benchmark its performance against $\gre$, as well as more sophisticated algorithms, including (i) batching-based heuristics that are highly relevant both in theory and practice, and (ii) forecast-aware heuristics that use historical data to compute shadow prices.
%
Our extensive simulation results show that $\PB$ consistently outperforms all benchmarks starting from relatively low market densities, achieving over 95\% of the hindsight optimal solution that has full knowledge of arrivals.
%
This result is robust to different distributions of job types, and also two-dimensional locations.


\paragraph{Simulations on real data.}

Finally, we test the practical applicability of $\PB$ via extensive numerical experiments using order-level data from the Meituan platform, made available from the 2024 INFORMS TSL Data-Driven Research Challenge.\footnote{\url{https://connect.informs.org/tsl/tslresources/datachallenge}, accessed January 15, 2025.}
%
The key information we extract from this dataset is the exact timestamps for the creation of each request, and the geographic coordinates of pick-up and drop-off locations.
%
These aspects differ from our theoretical setting in that (i) requests may not be available to be pooled for a fixed number of new arrivals before being dispatched, and (ii) locations are two-dimensional with delivery orders having heterogeneous origins.
%
To address (i), we assume that the platform sets a fixed time window after which an order becomes critical, corresponding to each job being able to wait for at most a fixed sojourn time before being dispatched.
%
For (ii), we extend our definition of reward (that captures the travel distance saved) to two-dimensional heterogeneous origins.
%
Under this new reward definition, the main insight from our theoretical model remains true: longer deliveries have higher potential reward from being pooled with other jobs in the future, and thus $\PB$ aims to keep them in the system.
%

In contrast to our simulations with synthetic data, the real-life delivery locations may now be correlated and exhibit time-varying effects.
%
Regardless, we find that $\PB$ outperforms all tested heuristics (those without foreknowledge of the future), given that the platform is willing to wait up to one minute before dispatching each job, a fairly modest ask.
%
In this regime, $\PB$ achieves over 80\% of the maximum possible travel distance saved from pooling. $\PB$ also pools 10\% more jobs than the hindsight optimal matching (see \Cref{sec:match_rate}).
%


\paragraph{Explanation for the surprising unbeatability of $\PB$.}
Our potential-based greedy heuristic consistently performs at or near the best across a wide range of experimental setups, including both synthetic and real data.
%
This result is surprising to us, given that $\PB$ is a simple index-based rule that ignores forecast information.
%
To provide some explanation for this finding, we analyze a stylized setting in \Cref{sec: interpretation}, where we show that the "correct" shadow prices converge to our notion of potential as the market thickness increases.
%
That being said, we also end with a couple of caveats.
%
First, our findings about the effectiveness of $\PB$ are specific to our definition of matching reward for delivery pooling, based on travel distance saved, and may not extend to setting with different reward structures.
%
Second, while we carefully tuned the batching-based and dual-based heuristics that we compare against, it remains possible that more sophisticated algorithms, particularly those leveraging dynamic programming in stochastic settings, could outperform $\PB$.


\subsection{Further Related Work}

\subsubsection{Dynamic matching.} \label{sec:dynMatchModels}

Dynamic matching problems have received growing attention from different communities in economics, computer science, and operations research. We discuss different ways of modeling the trade-off between matching now vs.\ waiting for better matches, depending on the application that motivates the study. 


One possible model \citep{kerimov2024dynamic,kerimov2023optimality,wei2023constant} is to consider a setting with jobs that arrive stochastically in discrete time and remain in the market indefinitely, but use a notion of \newterm{all-time regret} that evaluates a matching policy, at every time period, against the best possible decisions until that moment, to disincentivize algorithms from trivially delaying until the end to make all matches.
%

A second possible model, motivated by the risk of cancellation in ride-sharing platforms, is to consider sudden departures modeled by jobs having heterogeneous \textit{sojourn times} representing the maximum time that they stay in the market \citep{aouad2020dynamic}. These sojourn times are unknown to the platform, and delaying too long risks many jobs being lost without a chance of being matched.
%
\citet{aouad2020dynamic} formulates an MDP with jobs that arrive stochastically in continuous time, and leave the system after an exponentially distributed sojourn time. They propose a policy that achieves a multiplicative factor of the hindsight optimal solution. 
%
Related work on the control of matching queues with abandonment includes \citet{collina2020dynamic}, \citet{castro2020matching}, \citet{wang2024demand}, \citet{kohlenberg2024cost}.

%
A third possible model \citep{huang2018match} also considers jobs that can leave the system at any period, but allows the platform to make a last-moment matching decision right before a job leaves, termed as the job becoming \newterm{critical}.  In this model, one can without loss assume that all matching decisions are made at times that jobs become critical.

Finally, the model we study also makes all decisions at times that jobs become critical, with the difference being that the sojourn times are known upon the arrival of a job, as studied in \citet{ashlagi2019edge, eom2023batching}. Under these assumptions, \citet{eom2023batching} assume stationary stochastic arrivals, while \citet{ashlagi2019edge} allow for arbitrary arrivals.
%
Our research focuses on deterministic greedy-like algorithms that can achieve good performance as the market thickness increases.
%
Moreover, our work differs from these papers in the description of the matching value. While they assume arbitrary matching rewards, we consider specific reward functions known to the platform in advance, and use this information to derive a simple greedy-like algorithm with good performance.
% 
This aligns with a stream of literature on online matching, that captures more specific features into the model to get stronger guarantees \citep[see][]{kanoria2021dynamic,chen2023feature,balkanski2023power}.


We should note that our paper also relates to a recent stream of work studying the effects of batching and delayed decisions in online matching \citep[e.g.][]{feng2024batching,xie2023benefits}, as well as works studying the relationship between market thickness and quality of online matches \citep[e.g.][]{ashlagi2021kidney,chen2021matchmaking}.


\subsubsection{Delivery operations.}

On-demand delivery operations have received special attention in the field of transportation and operations management. 
%
\citet{reyes2018meal} introduce the meal delivery routing problem, which falls in the class of dynamic vehicle routing problems (see e.g. \citet{psaraftis2016dynamic}); the authors develop heuristics to dynamically assign orders to vehicles. Similar efforts have been devoted to optimize detailed pooling and assignment strategies as customer orders arrive sequentially \citep{steever2019dynamic,ulmer2021restaurant}, mainly using approximate dynamic programming techniques.
%
These heuristics are shown to work well in extensive numerical experiments, but given the intricate nature of the model and techniques, it is very challenging to derive managerial insights or theoretical performance guarantees.
%
Closer to our research goal, \citet{chen2024courier} analyze the optimal dispatching policy on a stylized queueing model representing a disk service area centered at one restaurant. They show that delivering multiple orders per trip is beneficial when the service area is large.
%
\citet{cachon2023fast} study the interplay between the number of couriers and platform efficiency, assuming a one-dimensional geography with one single origin, which is also the primary model in our theoretical results.
%
A similar topology is studied in the game-theoretic model of \citet{keskin2024order} to capture the impact of delivery pooling on the interaction between the platform, customers, riders, and restaurants. 

\section{Preliminaries}
\label{sec:model}
% 
In this section, we introduce a dynamic non-bipartite matching model for the delivery pooling problem. A total of $\Njob$ jobs arrive to the platform sequentially.
% 
Each job $j \in [\Njob] = \{1, \ldots, \Njob\}$, indexed in the order of arrival, has a \emph{type} $\type_j \in \typespace$. 
% 
We adopt the criticality assumption from \citet{ashlagi2019edge}, in the sense that each job remains available to be matched for $\sojourn\ge 1$ new arrivals, after which it becomes \textit{critical}. The parameter $\sojourn$ can also be interpreted as the market density.
%
When a job becomes critical, the platform decides whether to (irrevocably) match it with another available job, in which case they are dispatched together (i.e. \emph{pooled})
and the platform collects a \emph{reward} given by a known function $\reward:\typespace^2\to\R$. If a critical job is not pooled with another job, it has to be dispatched by itself for zero reward.


\subsection{Reward Topology}
% 
Our modeling approach directly imposes structure on the type space, as well as the reward function that captures the benefits from pooling delivery orders together. Similar to previous numerical work on pooled trips in ride-sharing platforms \citep{eom2023batching, aouad2020dynamic}, we assume that the platform's goal is to reduce the total distance that needs to be traveled to complete all deliveries.
The reward of pooling two orders together is therefore the travel distance saved when they are delivered by the same driver in comparison to delivered separately. 
% 
Formally, we consider a \newterm{linear city model} where job types $\type \in \typespace = [0,1]$ represent destinations of the delivery orders, and assume that all orders need to be served from the origin 0 (similar to that analyzed in \citet{cachon2023fast})
% 
If a job of type $\type$ is dispatched by itself, the total travel distance from the origin 0 is exactly $\type$. If it is matched with another job of type $\type'$, they are pooled together on a single trip to the farthest destination $\max\{\type,\type'\}$. Thus, the distance saved by pooling is 
%
%
%
\begin{equation}
  \reward(\type,\type') = \type + \type' - \max\{\type,\type'\} = \min\{\type,\type'\}.
  \label{eq:defn_reward_A} 
\end{equation}

Although our main theoretical results leverage the structure of this reward topology, our proposed algorithm can be applied to other reward topologies. 
% 
In particular, we derive theoretical results for two other reward structures for $\typespace = [0,1]$. First, we consider 
% 
\begin{equation}
    \reward(\type,\type') = 1 - |\type - \type'|  \label{eq:defn_reward_B}
\end{equation}
% 
to capture the commonly-studied spatial matching setting~\citep[e.g.][]{kanoria2021dynamic,balkanski2023power}, where the reward is larger if the distance $|\theta-\theta'|$ is smaller. 
% 
We also consider 
\begin{equation}
    \reward(\type,\type') = |\type - \type'| \label{eq:defn_reward_C}
\end{equation}
% 
with the goal of matching types that are far away from each other, contrasting the other two reward functions.
% 
In addition, we perform numerical experiments for delivery pooling in two-dimensional (2D) space, where the reward function corresponds to the travel distance saved in the 2D setting.


\subsection{Benchmark Algorithms}

% 
Given market density $\sojourn$ and reward function $\reward$, if the platform had full information of the sequence of arrivals $\instance \in \typespace^\Njob$, the hindsight optimal $\OPT(\instance,\sojourn)$ can be computed by the integer program (IP) defined in \eqref{eq: OPT}. We denote the hindsight optimal matching solution as $\matchset_{\OPT} = \{(j,k):x_{j,k}^{\ast}=1\}$, where $x^\ast$ is an optimal solution of \eqref{eq: OPT}.
% 
\begin{maxi}
    {x}{ \sum_{j,k : j\neq k, |j-k|\le \sojourn} x_{jk} \reward(\type_j,\type_k)}
    {\label{eq: OPT}}{\OPT(\instance,\sojourn) =}
    \addConstraint{ \sum_{k:j\neq k} x_{jk}}{\le 1,}{j\in [\Njob]}
    \addConstraint{ x_{jk}}{\in\{0,1\},}{j,k\in [\Njob], j\neq k.}
\end{maxi}
%

An online matching algorithm operates over an instance $\instance\in\typespace^\Njob$ sequentially: when job $j\in [\Njob]$ becomes critical, the types of future arrivals $k > j + d$ are unknown, and any matching decision has to be made based on the currently available information.
% 
Given an algorithm $\ALG$, we denote by $\ALG(\instance,\sojourn)$ the total reward collected by an algorithm on such instance.
%
We analyze the performance of algorithms via \emph{regret}, as follows:
%
\[
    \regret_\ALG(\instance,\sojourn) = \OPT(\instance,\sojourn)-\ALG(\instance,\sojourn).
\]

We study a class of online matching algorithms that we call \newterm{index-based greedy matching algorithms}.
%
Each index-based greedy matching algorithm is specified by an \emph{index function} $\indexf:\typespace^2\to\R$, and only makes matching decisions when some job becomes critical (in our model, it is without loss of optimality to wait to match).
%
A general pseudocode is provided in \Cref{alg:dynamic}.
%
When a job $j \in [\Njob]$ becomes critical, the algorithm observes the set of available jobs in the system $A(j) \subseteq [\Njob]$ that have arrived but are not yet matched (this is the set $\buffer\setminus\{j\}$ in \Cref{alg:dynamic}), and chooses a match $\matchof(j) \in A(j)$ that maximizes the index function (even if negative), collecting a reward $\reward(\type_j,\type_{\matchof(j)})$. If there are multiple jobs that achieve the maximum, the algorithm breaks the tie arbitrarily.
%
The algorithm makes a set of matches $\matchset = \{ (j,m(j)) : j \in C \}$, where $C$ denotes the set of jobs that are matched when they become critical, and its total reward is
\[ \ALG(\instance,\sojourn) = \sum_{j\in C} \reward(\type_j,\type_{\matchof(j)}). \]
%
Note that $\matchof(j)$ is undefined if $j$ is either unmatched, or was not critical at the time it was matched.


\begin{algorithm}
\caption{Index-based Greedy Matching Algorithm}\label{alg:dynamic}
% 
\begin{algorithmic}[1]
\Require Instance $\instance$, density $\sojourn$, index function $\indexf$, reward function $\reward$
% 
\Ensure $C$, $m:C\to[n]$
% 
\State Initialize set of available jobs $\buffer=\emptyset$, and jobs that are critical when matched $C=\emptyset$
\For{job $t=1,\ldots,\Njob+\sojourn+1$}
    \If{$t\le\Njob$}
        \State $\buffer=\buffer\cup \{t\}$ \Comment{job $t$ arrives}
    \EndIf         
    \If {$t-d\in \buffer$} 
        \State $j=t-d$ \Comment{job $j=t-d$ becomes critical}
        \If{$\buffer \setminus\{j\} \neq\emptyset$}
            \State Choose $m(j) \in \argmax_{k \in \buffer \setminus\{j\} } \indexf(\type_{j}, \type_{k})$ \Comment{Ties are broken arbitrarily}
            \State $C\gets C\cup\{j\}$
            \State $\buffer \gets \buffer\setminus \{j,\matchof(j)\} $ \Comment{jobs $j,\matchof(j)$ are dispatched together}
        \Else
            \State $\buffer \gets \buffer\setminus \{j\} $ \Comment{job $j$ is dispatched by itself}
        \EndIf
    \EndIf
\EndFor
\State\Return $C, \{\matchof(j)\}_{j\in C}$
\end{algorithmic}
\end{algorithm}

As an example, the \newterm{naive greedy} algorithm ($\gre$) uses index function $\indexf_{\gre}(\type,\type') = \reward(\type,\type')$. When any job becomes critical, the algorithm chooses a match it with an available job to maximize the (instant) reward.
%
To illustrate the behavior of this algorithm under reward function $\reward(\type,\type') = \min\{\type,\type'\}$, we first make the following observation, that the algorithm always chooses to match each critical job with a higher type job when possible. 
%
\begin{remark}\label{deliverygreedy}
    When a job $j \in [\Njob]$ becomes critical, if the set $A_+(j) = \{k\in A(j):\type_k \ge \type_j\}$ is nonempty, then under the naive greedy algorithm $\gre$, we have $\matchof(j)\in A_+(j)$ and $\reward(\type_j,\type_{\matchof(j)}) = \type_j $. 
    % 
\end{remark}



\section{Potential-Based Greedy Algorithm}
% 
\label{sec:PB} 

We introduce in this section the potential-based greedy algorithm and prove that it substantially outperforms the naive greedy approach in terms of worst case regret.

The \newterm{potential-based greedy} algorithm ($\PB$) is an index-based greedy matching algorithm (as defined in \Cref{alg:dynamic}) whose index function is specified as
\begin{equation}
    \indexf_{\PB}(\type,\type') = \reward(\type, \type') - \potential(\type'),
\end{equation}
where $\potential(\type)$ is the \newterm{potential} of a job of type $\type$, formally defined as follows
%
\begin{equation}
    \potential(\type)=\frac{1}{2}\sup_{\type'\in\typespace} \reward(\type,\type'). \label{eq:defn_potential}
\end{equation} 
% 

This new notion of potential can be interpreted as an optimistic measure of the marginal value of holding on to a job that could be later matched with a job that maximizes instant reward.
%
Intuitively, this "ideal" matching outcome appears as the optimal matching solution when the density of the market is arbitrarily large. We provide a formal statement of this intuition in \Cref{sec: interpretation}.
%
Note that this definition of potential is quite simple in the sense that it relies on only in the knowledge of the reward topology: reward function and type space.
% 
In particular, for our topology of interest, $\reward(\type,\type')=\min\{\type,\type'\}$ on $\typespace=[0,1]$, the ideal scenario is achieved when two identical jobs are matched and the potential $\potential(\type)=\type/2$ is simply proportional to the length of a solo trip, capturing that longer deliveries have higher potential reward from being pooled with other jobs in the future.

We begin by giving a straightforward interpretation for this algorithm under the linear city model and our reward of interest, as we did for the naive greedy algorithm $\gre$ in \Cref{deliverygreedy}.

\begin{remark}\label{deliverypotential}
    Under the 1-dimensional type space $\Theta = [0,1]$ and the reward function $\reward(\type,\type')=\min\{\type,\type'\}$, 
    % 
    the potential-based greedy algorithm $\PB$ always matches each critical job to an available job that's the closest in space, since 
    % 
    \begin{align*}
        \matchof(j) 
        \in \argmax_{k\in A(j) } \indexf_\PB(\type_j,\type_k) 
        = \argmin_{k\in A(j) } |\type_j-\type_k|. 
    \end{align*}    % 
    This follows from the fact that $ |\type_j-\type_k| = \type_j + \type_k - 2\min\{\type_j, \type_k\} = \type_j - 2\indexf_\PB(\type_j,\type_k) $.
\end{remark}

In the rest of this section, we first assume $d = \infty$ and study the \textit{offline} performance of various index-based greedy matching algorithms under the reward function $\reward(\type,\type') = \min\{\type,\type'\}$.
% 
This will later help us analyze the algorithms' \emph{online} performances when $d<\infty$.
Results for the two alternative reward structures defined in \eqref{eq:defn_reward_B} and \eqref{eq:defn_reward_C} are reported in \Cref{sec:reward2} and  \Cref{sec: reward 3}, respectively.
%


\subsection{Offline Performance under Reward Function $\reward(\type,\type') = \min\{\type,\type'\}$}\label{sec: static}

Suppose $\sojourn=\infty$, meaning that all jobs are avilable to be matched when the first job becomes critical. We study the \emph{offline} performance of both the naive greedy algorithm $\gre$ and potential-based greedy algorithm. We show that the regret of $\gre$ grows linearly with the number of jobs, while the regret of $\PB$ is logarithmic.


\begin{proposition}[proof in \Cref{pf:greedy_linear_lower_bound}]\label{prop:greedy_linear_lower_bound}
Under reward function $\reward(\type,\type') = \min\{\type,\type'\}$, when the number of jobs $\Njob$ is divisible by 4, there exists an instance $\instance \in [0,1]^\Njob$ for which $\regret_\gre(\instance,\infty) \ge n/4$.
\end{proposition}
%

In particular, the \newterm{regret per job} of $\gre$, i.e.\ dividing the regret by $\Njob$, is constant in $\Njob$.
%
In contrast, we prove in the following theorem that $\PB$ performs substantially better. In fact, its regret per job gets better for larger market sizes $\Njob$.

\begin{theorem}
\label{thm:potential_log_upper_bound}
    Under reward function $\reward(\type,\type') = \min\{\type,\type'\}$, we have $\regret_\PB(\instance,\infty) \le 1 + \log_2(\Njob/2+1)/2$, for any $\Njob$, and any instance $\instance \in [0,1]^\Njob$.
\end{theorem}
%
\proof{Proof.}
Let $\instance \in \typespace^\Njob$, and $(C,\matchof(\cdot))$ be the output of $\PB$ on $\instance$ (generated as in \Cref{alg:dynamic}). In particular, 
$\PB(\instance,\infty) = \sum_{j \in C} \reward(\type_j,\type_{\matchof(j)})$.
On the other hand, since $\reward(\type,\type') \le \potential(\type) + \potential(\type')$ for all $\type,\type'
\in\typespace$, we have
\begin{align}
    \OPT(\instance,\infty) = \sum_{(j,k) \in \matchset_\OPT} \reward(\type_j,\type_k) \le \sum_{j=1}^\Njob \potential(\type_j) \le \sup_{\type\in[0,1]}\potential(\type) + \sum_{j\in C} \potential(\type_j) + \potential(\type_{\matchof(j)}),\nonumber
\end{align}
where the last inequality comes from the fact the $\PB$, and in fact any index-based matching algorithm, leaves at most one job unmatched (when $\Njob$ is odd).
%
Hence,
\begin{align}
\regret_\PB(\instance,\infty)&=
\OPT(\instance,\infty)-\PB(\instance,\infty) \\
&\le \sup_{\type\in \typespace}\potential(\type) + \sum_{j\in C} \left( \potential(\type_j) + \potential(\type_{\matchof(j)}) - \reward(\type_j,\type_{\matchof(j)})\right) \nonumber\\
&= \frac12 + \sum_{j \in C} \frac{\type_j}{2} + \frac{\type_{\matchof(j)}}{2} - \min\{\type_j,\type_{m(j)}\} \nonumber\\
&= \frac12 + \sum_{j \in C} \frac{|\type_j - \type_{\matchof(j)}|}{2}\label{eq: offline_regret_bound}. 
\end{align}
%
To bound the right-hand side, we first prove the following. For each $j\in C$, consider the interval $I_j=( \min\{\type_j,\type_{\matchof(j)}\},\max\{\type_j,\type_{\matchof(j)}\})\subseteq[0,1]$ and $\depth(j)=|\{k : I_j \subseteq I_k\}|$. We argue that
\begin{equation}\label{eq: distancebound}
    |\type_j - \type_{\matchof(j)}| \le 2^{1-\depth(j)}.
\end{equation}
%

First, note that $\{I_j\}_{j\in C}$ is a \newterm{laminar set family}: for every $j,k\in C$, the intersection of $I_j$ and $I_k$ is either empty, or equals $I_j$, or equals $I_k$. Indeed, without loss of generality, assume $j<k$. Then, when job $j$ becomes critical we have $\matchof(j),k,\matchof(k) \in A(j) $. Therefore, from \Cref{deliverypotential}, $\matchof(j)$ is the closest to $j$ and thus $k,\matchof(k)\notin I_j$, i.e.\ either $I_j \cap I_k = \emptyset$, or $I_j \cap I_k = I_j$.
%
Moreover, if $I_j\subsetneq I_k$ (i.e. $I_j\subsetneq I_k$ and $I_j\neq I_k$), then necessarily $j < k$, since otherwise $k$ becomes critical first and having $j,\matchof(j) \in A(k)$ closer in space, $\PB$ would not have chosen $\matchof(k)$.
%
Now, we can prove \eqref{eq: distancebound} by induction. If $\depth(j)=1=|\{j\}|$, clearly $|\type_j-\type_{\matchof(j)}|\le 1$. Suppose that the statement is true for $\depth(j)=\depth \ge 1$. If $\depth(j)=\depth+1$ then there exists $k$ such that $I_j\subsetneq I_k$ and $\depth(k)=\depth$. By the previous property, $j<k$ and since when job $j$ becomes critical we had $k,\matchof(k)\in A(j)$, it must be the case that
\begin{align*}
    |\type_j-\type_{\matchof(j)}| 
    &<\max\{ \min\{\type_j,\type_{m(j)}\} - \min\{\type_k,\type_{m(k)}\},\max\{\type_k,\type_{m(k)}\}-\max\{\type_j,\type_{m(j)}\}\}\\
    &\le \min\{\type_j,\type_{m(j)}\} - \min\{\type_k,\type_{m(k)}\} + \max\{\type_k,\type_{m(k)}\} - \max\{\type_j,\type_{m(j)}\} \\
    &= |\type_k-\type_{\matchof(k)}| - |\type_j-\type_{\matchof(j)}|.
\end{align*}
Thus, $|\type_j-\type_{\matchof(j)}|\le|\type_k-\type_{\matchof(k)}|/2\le 2^{1-(\depth+1)}$,
completing the induction.
%
    
Having established \eqref{eq: distancebound}, we use it to derive an upper bound for $\sum_{j\in C}|\type_j-\type_{\matchof(j)}|$. Note that since $\depth(j)\in \{1,\ldots,\floor{\Njob/2}\}$, we have
\begin{align} \label{eq: distanceboundbylayer}
\sum_{j \in C} |\type_j - \type_{\matchof(j)}| = \sum_{\depth = 1}^{\floor{\Njob/2}}\sum_{j\in C:\depth(j) = \depth} |\type_j - \type_{\matchof(j)}|.
\end{align}
%
For every $\depth$, we have $\sum_{j\in C:\depth(j) = \depth} |\type_j - \type_{\matchof(j)}|\le \min\{1,2^{1-\depth}|\{j:\depth(j)=\depth\}|\}$ by the laminar property and \eqref{eq: distancebound}.
%
Let $z_\depth$ denote $2^{1-\depth}|\{j:\depth(j)=\depth\}|$, where we note that 
\begin{align*}\label{eq: numofintervalsbound}
    \sum_{\depth=1}^{\floor{\Njob/2}} 2^{\depth-1}z_\depth = \sum_{\depth=1}^{\floor{\Njob/2}} |\{j:\depth(j)=\depth\}| = |C| \le \Njob/2.
\end{align*}
We can then use the following LP to upper-bound the value of~\eqref{eq: distanceboundbylayer} under an adversarial choice $\{z_\depth:\depth=1,\ldots,\lfloor n/2\rfloor\}$:
\begin{maxi}
    {z}{ \sum_{\depth=1}^{\floor{\Njob/2}} z_\depth }
    {\label{eq: knapsack}}{}
    \addConstraint{ \sum_{\depth=1}^{\floor{\Njob/2}} 2^{\depth-1}z_\depth }{\le \Njob/2}
    \addConstraint{0\le z_\depth }{\le 1,\quad }{ \depth = 1,\ldots,\floor{\Njob/2}.}
\end{maxi}
This is a fractional knapsack problem, whose optimal value is at most $\log_2({\Njob/2+1})$, and thus
\begin{equation}\label{dist_bound}
    \sum_{j \in C} |\type_j - \type_{\matchof(j)}| \le \log_2(\Njob/2+1).
\end{equation}
%
Substituting back into \eqref{eq: offline_regret_bound} yields $\regret_\PB(\instance,\infty) \le 1/2 + \log_2(\Njob/2+1)/2$, completing the proof.
\Halmos\endproof

The following result shows that this analysis of $\PB$ is tight, up to constants.

\begin{proposition}[proof in \Cref{pf:loglowerboundOffline}]
\label{prop:loglowerboundOffline}
Under reward function $\reward(\type,\type') = \min\{\type,\type'\}$, when the number of jobs is $\Njob=2^{k+3} - 4$ for some integer $k\ge 0$, there exists an instance $\instance \in [0,1]^\Njob$ for which $\regret_\PB(\instance,\infty) \ge (\log_2(n+4)-3)/4$.
\end{proposition}
%

%%%%%%%%%%%%%%%%%%%%%%%%%%%%%%%%%%%%%%%%%%%%%%%%%%%%%%%%

\subsection{Online Performance under Reward Function $\reward(\type,\type') = \min\{\type,\type'\}$}\label{sec:dynamic}

We now study the performance of algorithms in the online setting, i.e. when $\sojourn<\Njob$, and derive informative performance guarantees conditional on $\sojourn$, the number of new arrivals before a job becomes critical.
%
In fact, the regret per job of $\gre$ remains constant, whereas the one of $\PB$ decreases with $\sojourn$.
%
To build upon our results proved in \Cref{sec: static} for the offline setting, we partition the set of jobs into $b=\lceil\Njob/(\sojourn + 1) \rceil$ \newterm{batches}.
%
To be precise, we define the $t$-th batch as $B_{t} = \{(t-1)(\sojourn+1)+1,\ldots, t(\sojourn+1)\} \cap [\Njob]$, for each $t = 1, \ldots, b$.
%
Scaling $\sojourn$ while keeping the number of batches $b$ fixed can be interpreted as increasing the density of the market.
%
We first show that the regret under $\gre$ remains linear in the number of jobs $\Njob$, independent of $\sojourn$, which can be understood as suffering a regret of order $\Omega(\sojourn)$ per batch.

\begin{proposition}[proof in \Cref{pf:greedy_linear_lower_bound_dynamic}]
\label{prop:greedy_linear_lower_bound_dynamic}
    Under reward topology $\reward(\type,\type') = \min\{\type,\type'\}$, if $(\sojourn+1)$ is divisible by 4, then for any number of jobs $\Njob$ divisible by $(\sojourn+1)$, there exists an instance $\instance \in [0,1]^\Njob$ for which $\regret_\gre(\instance,\sojourn) \ge n/4$.
\end{proposition}

In contrast, the following theorem shows that the performance of $\PB$ improves as market density increases.

\begin{theorem}\label{thm: dynamic}
Under reward topology $\reward(\type,\type') = \min\{\type,\type'\}$, we have $\regret_\PB(\instance,\sojourn) \le 1/2 + (\frac{\Njob}{d+1}+1)(1+\log(\sojourn+2))/2$ for any $\Njob$, and any instance $\instance\in[0,1]^\Njob$.
\end{theorem}
%
\proof{Proof.}
Let $\instance\in \typespace^\Njob$ and $\sojourn\ge 1$. Let $(C,\matchof(\cdot))$ be the output of $\PB$ on $\instance$.
Similar to the proof of \Cref{thm:potential_log_upper_bound}, we have
\begin{align}
\regret_\PB(\instance,\sojourn)
& \le \sum_{j=1}^\Njob \potential(\type_j) - \PB(\instance,\sojourn)  \nonumber \\
& \le \sup_{\type\in\typespace}\potential(\type) + \sum_{j \in C} \left( p(\theta_j) + p(\theta_{m(j)}) - r(\theta_j,\theta_{m(j)}) \right) \nonumber \\ 
& = \sup_{\type\in\typespace}\potential(\type) + \sum_{t=1}^{b} \sum_{j\in  C\cap B_t} \left( p(\theta_j) + p(\theta_{m(j)}) - r(\theta_j,\theta_{m(j)}) \right) \nonumber \\
& = \frac12 + \frac12 \sum_{t=1}^{b} \left(\sum_{j\in  C\cap B_t} |\type_j-\type_{\matchof(j)}| \right), \label{eq:online_dist}
\end{align}
where we split the sum into the $b = \lceil \Njob/(\sojourn + 1) \rceil$ batches, defined as $B_{t} = \{(t-1)(\sojourn+1)+1,\ldots, t(\sojourn+1)\} \cap [\Njob]$ for $t = 1, \ldots, b$. We analyze the term in large parentheses for an arbitrary $t$. Intuitively, we would like to consider an offline instance consisting of jobs $(\type_j)_{j\in C \cap B_t}$ and their matches $(\type_{\matchof(j)})_{j\in C\cap B_t}$, and apply \Cref{thm:potential_log_upper_bound} on the offline instance which has size $2|C\cap B_t|\le 2(\sojourn+1)$.
However, since the offline setting allows jobs to observe the full instance, as opposed to only the next $\sojourn$ arrivals, the resulting matching of $\PB$ on the offline instance could be inconsistent with its output on the online instance. In particular, executing $\PB$ on the offline instance could match jobs with indices more than $d$ apart, which is not possible in the online setting. Thus, to derive a proper upper bound on regret, we need to construct a modified offline instance for which each resulting match of $\PB$ coincides with the matching output in the online counterpart, and in which matching distances were not decreased.

%
To construct this modified offline instance, first note that if $\matchof(j)\in B_t$ for all $j\in C\cap B_t$, then no modification is needed, since all matched jobs have indices less than $\sojourn$ apart.
%
However, if $\matchof(j)\in B_{t+1}$ for some $j\in C\cap B_t$, then $\matchof(j)$ can be available in the offline instance for some job $k<j$ with $|\matchof(j)-k|>\sojourn$ and be matched to $k$ in the offline instance even though this would not be possible in the online instance (see \Cref{fig:thm3_exm}).
%
Thus, in the modified offline instance, we "move" job $m(j)$ ensure job $k$ would not choose $m(j)$ for its match.
%
To do so, we distinguish three cases.
%
For each $j\in C\cap B_t$, let $A(j)$ be the set of jobs that the online algorithm could have chosen from to match with $j$, i.e.~$A_j$ is the set $A\setminus\{j\}$ in \Cref{alg:dynamic} right after job $j$ becomes critical.
%
If $\matchof(j)\in A(j)\cap B_t$, then both jobs $j,\matchof(j)$ are included in the offline instance we construct.
%
In the second case, if $\matchof(j)\in A(j)\cap B_{t+1}$ and $A(j)\cap B_t\neq\emptyset$, then we modify $\type_{m(j)}$ to be equal to the best available matching candidate within batch i.e.\ we "move" job $m(j)$ to location $\argmin \{ |\type_j - \type| : \type = \type_k, \ k\in A(j)\cap B_t \}$.
%
In the third case, if $\matchof(j)\in A(j)\cap B_{t+1}$ and $A(j)\cap B_t=\emptyset$, then we don't consider job $j$ or its match (if any) in the offline instance we construct. This can happen at most once per batch, since any other job $k<j$ would have $j\in A(k)$. 
%

To formally define the construction, fix an arbitrary batch $t$. Let
$\hat{C}=\{j\in C\cap B_t: A(j)\cap B_t\neq \emptyset\}$ and $\hat{n}= 2|\hat{C}|\le 2(\sojourn+1)$. 
%
$\hat{C}$ is the set of critical jobs in cases one or two above, and we include jobs $j\in\hat{C}$ and their matches $m(j)$ in the offline instance.  There are $\hat{n}$ jobs in the offline instance and the ordering of indices is consistent with the original instance.
Define modified locations in the offline instance as follows:
\begin{align*}
\hat{\type}_j &= \type_j &\text{for $j\in\hat{C}$}
\\ \hat{\type}_{m(j)} &= \type_{m(j)} &\text{for $j\in\hat{C}$, if $m(j)\in A(j)\cap B_t$ (case one)}
\\ \hat{\type}_{m(j)} &= \argmin \{ |\type_j - \type| : \type = \type_k, \ k\in A(j)\cap B_t \} &\text{for $j\in\hat{C}$, if $m(j)\in A(j)\cap B_{t+1}$ (case two)}
\end{align*}
%
%
\begin{figure}[H]
  \centering
  \subcaptionbox{Batch $B_t$ of the online instance $\instance$.
    \label{fig:thm3_exm_original}
    }[0.4 \textwidth]
    {
    \begin{tikzpicture}


% Background shading for periods 6t+1 to 6t+6
\fill[lightyellow] (1.8, 0) rectangle (6.2, 3.2);

\draw[->] (-0.5, 0) -- (8.5, 0) node[right] {$j$};
\draw[->] (0, -0.1) -- (0, 3.3) node[above] {$\type_j$};

\foreach \x in {1,...,8}
    \draw (\x, -0.1) -- (\x, 0.1);

\node[below] at (1, 0) {\tiny $5t$};
\node[below] at (2, 0) {\tiny $5t+1$};
\node[below] at (3, 0) {\tiny $5t+2$};
\node[below] at (4, 0) {\tiny $5t+3$};
\node[below] at (5, 0) {\tiny $5t+4$};
\node[below] at (6, 0) {\tiny $5t+5$};
\node[below] at (7, 0) {\tiny $5t+6$};
\node[below] at (8, 0) {\tiny $5t+7$};

\foreach \y in {0.1,0.2,0.3,0.4,0.5,0.6,0.7,0.8,0.9,1}
    \draw (-0.1, \y*3) -- (0.1, \y*3) node[left, xshift=-0.2cm] {\tiny \y};
    

\foreach \x/\y in {1/0.9, 2/0.1, 3/0.7, 4/0.3, 5/0.4, 6/1, 7/0.2, 8/0.8} 
{
    \fill[black] (\x, \y*3) circle (0.1);
    \draw[gray, dashed] (0, \y*3) -- (\x, \y*3);
    }
    

% Matches
\draw[thick] (1, 0.9*3) -- (3, 0.7*3);     
\draw[thick] (2, 0.1*3) -- (4, 0.3*3);
\draw[thick] (5, 0.4*3) -- (7, 0.2*3);     
\draw[thick] (6, 1*3) -- (8, 0.8*3);  

\end{tikzpicture}
    }
  \hfill
  \subcaptionbox{Modified offline instance $\hat{\instance}$. 
    \label{fig:thm3_exm_phantom}
    }[0.4 \textwidth]
    {
    \begin{tikzpicture}

\draw[->] (-0.5, 0) -- (4.5, 0) node[right] {$j$};
\draw[->] (0, -0.1) -- (0, 3.3) node[above] {$\type_j$};

\foreach \x in {1,...,4}
    \draw (\x, -0.1) -- (\x, 0.1);

\node[below] at (1, 0) {\tiny $5t+1$};
\node[below] at (2, 0) {\tiny $5t+3$};
\node[below] at (3, 0) {\tiny $5t+4$};
\node[below] at (4, 0) {\tiny $5t+6$};

\foreach \y in {0.1,0.2,0.3,0.4,0.5,0.6,0.7,0.8,0.9,1}
    \draw (-0.1, \y*3) -- (0.1, \y*3) node[left, xshift=-0.2cm] {\tiny \y};
    

\foreach \x/\y in {1/0.1, 2/0.3, 3/0.4, 4/1} 
{
    \fill[black] (\x, \y*3) circle (0.1);
    \draw[gray, dashed] (0, \y*3) -- (\x, \y*3);
    }
    

% Matches
\draw[thick] (1, 0.1*3) -- (2, 0.3*3);     
\draw[thick] (3, 0.4*3) -- (4, 1*3);


\end{tikzpicture}
    }
  % 
\caption{Illustration of the construction of the modified offline instance, with $\sojourn=4$. In \Cref{fig:thm3_exm_original}, job $5t+1$ chooses job $5t+3$ in the online execution of $\PB$, because only jobs up to $5t+5$ would have arrived when job $5t+1$ becomes critical.  However, in the offline execution of $\PB$, job $5t+1$ would choose job $5t+6$, because all jobs are available.  Our modified instance in \Cref{fig:thm3_exm_phantom} corrects the inconsistent decision.
}
  % 
  \label{fig:thm3_exm}
\end{figure}
%
Now, we show that the output of (offline) $\PB$ on $\hat{\instance}$ provides an upper bound for the expression $\sum_{j\in C\cap B_t}|\type_j-\type_{\matchof(j)}|$ in~\eqref{eq:online_dist}.
%
In particular, we show that the output of $\PB$ is exactly $(\hat{C},m(\cdot))$ when executed on the modified offline instance.
%
First, note that the construction potentially introduces ties if there is a job $k\in C\cap B_t$ with $\matchof(k)\in B_{t+1}$. However, since \Cref{thm:potential_log_upper_bound} holds under arbitrary tie-breaking, we assume $\PB$ breaks ties on $\hat{\instance}$ to maintain consistent decisions with $\PB$ on the original instance $\instance$. Thus, from \Cref{deliverypotential}, it suffices to show that for every $j\in \hat{C}$, $\matchof(j)$ minimizes the distance among available jobs.
%

%
Indeed, note that every job in the offline instance has a type that is identical to a job in $B_t$. Then, proceeding in the same order as in the online instance, the job types of the set of available jobs in the offline instance is a subset of the original available jobs. For a match in case one, we know that it minimizes the distance among the original available jobs, and hence it minimizes the distance among the offline available jobs. Moreover, we have $|\hat{\type}_j-\hat{\type}_{\matchof(j)}| = |\type_j - \type_{\matchof(j)}|$. In the second case, the match minimizes distance among offline available jobs by construction, and moreover $|\hat{\type}_j-\hat{\type}_{\matchof(j)}| \ge |\type_j - \type_{\matchof(j)}|$.
%
Consequently, we can upper bound 
\[ \sum_{j\in C\cap B_t } |\type_j-\type_{\matchof(j)}|
\le 1 + \sum_{j\in  \hat{C}} |\hat{\type}_j-\hat{\type}_{\matchof(j)}|. 
\]
%  
Then, recalling \eqref{dist_bound} in the proof of \Cref{thm:potential_log_upper_bound}, we have $\sum_{j\in  \hat{C}} |\hat{\type}_j-\hat{\type}_{\matchof(j)}| \le \log(\hat{n}/2 + 1) \le \log(\sojourn + 2)$, and then
\begin{align}
    \sum_{j\in C\cap B_t } |\type_j-\type_{\matchof(j)}|
\le 1 +  \log(\sojourn + 2).\label{eq: distance_dynamic}
\end{align}
Substituting back into~\eqref{eq:online_dist}, and using the fact that $b\le\frac n{d+1}+1$, we get $\regret_\PB(\instance,\sojourn)\le 1/2 + (\frac{\Njob}{d+1}+1)(1+\log(\sojourn+2))/2$, completing the proof. 
\Halmos\endproof
%
Lastly, we show that this online analysis of $\PB$ is also tight, up to constants.
%
\begin{proposition}[proof in \Cref{pf:loglowerboundOnline}]
\label{prop:loglowerboundOnline}
Under reward topology $\reward(\type,\type') = \min\{\type,\type'\}$, if $\sojourn+1 = 2^{k+3}-4$ for some $k\in\{0,1,\ldots\}$, then for any number of jobs $\Njob$ divisible by $(\sojourn+1)$, there exists an instance $\instance \in [0,1]^\Njob$ for which $\regret_\PB(\instance,\sojourn) \ge \frac{\Njob}{3(\sojourn+1)}(\log_2(\sojourn+5)-3)/4$.
\end{proposition}
%

%%%%%%%%%%%%%%%%%%%%%%%%%%%%%%%%%%%%%%%%%%%%%%%%%%%%%%

% \subsection{Interpretation of Potential}\label{sec: interpretation}

% When a job $j$ becomes critical, our potential-based greedy algorithm matches it to an available job $k$ maximizing $r(\theta_j,\theta_k)-p(\theta_k)$, where $p(\theta_k)$ can be interpreted as the opportunity cost of matching job $k$, with the specific definition $p(\theta_k)=\frac12 \sup_{\theta\in\Theta} r(\theta_k,\theta)$.
% This is an optimistic measure of opportunity cost because it assumes that job $k$ would otherwise be matched to an "ideal" type $\theta\in\Theta$ maximizing $r(\theta_k,\theta)$ (with half of this ideal reward $\sup_{\theta\in\Theta} r(\theta_k,\theta)$ attributed to job $k$).
% We now prove that this ideal reward can indeed be achieved under asymptotically-large market thickness, for a stochastic model under our topology of interest.

% \begin{definition}
% Let $\instance \in \typespace^\Njob$.
% For any job $j\in[\Njob]$, let $\instance^{-j}\in \typespace^{\Njob-1}$ be the same instance with the exception that job $j$ is not present. Meanwhile, let $\instance^{+j}\in \typespace^{\Njob+1}$ be the same instance with the exception that an additional copy of job $j$ is present.
% Consider the following definitions.
% \begin{enumerate}
% \item Marginal Loss: $\marginalloss_j(\instance) = \OPT(\instance) - \OPT(\instance^{-j})$
% \item Marginal Gain: $\marginalgain_j(\instance) = \OPT(\instance^{+j}) - \OPT(\instance)$
% \end{enumerate}
% \end{definition}

% Note that definitions $\marginalloss_j(\instance),\marginalgain_j(\instance)$ are based solely on offline matching, and we will use them as our definitions of opportunity cost if the future was known.  One could alternatively use shadow prices from the LP relaxation of the offline matching problem, but we note that the LP is not integral.  In either case, there is no ideal definition of opportunity cost that is guaranteed to lead to the optimal offline solution in matching problems \citep[see][]{cohen2016invisible}.

% We now establish the following \namecref{lem:marginal_as_interval} to help analyze the opportunity costs $\marginalloss_j(\instance),\marginalgain_j(\instance)$.


% \begin{lemma}\label{lem:marginal_as_interval}
% Let $\typespace=[0,1]$ and $\reward(\type,\type') = \min\{\type,\type'\}$. Consider an instance $\instance\in\typespace^n$ and relabel the indices to satisfy $\type_1 \ge \type_2 \ge \ldots \ge \type_n$. Then,
% \begin{align*}
% \marginalloss_j(\instance)
% &=\sum_{k\ge j,k\ \mathrm{even}}(\theta_k-\theta_{k+1}), 
% \\ \marginalgain_j(\instance)
% &=\sum_{k\ge j,k\ \mathrm{odd}}(\theta_k-\theta_{k+1}),
% \end{align*}
% where we consider $\theta_{n+1}=0$.
% \end{lemma}
% %
% \proof{Proof.}
%     Let $\instance \in \typespace^\Njob$ such that $\type_1 \ge \type_2 \ge \ldots \ge \type_n$. 
%     %
%     Then, $\OPT(\instance) =\sum_{k\ \mathrm{even}}\theta_k$, and moreover
%     \begin{align*}
%     \OPT(\instance^{-j}) &=\sum_{k<j, k\ \mathrm{even}}\theta_k+\sum_{k>j, k\ \mathrm{odd}}\theta_k
%     \\ \OPT(\instance^{+j}) &=\theta_j+\OPT(\instance^{-j})
%     \end{align*}
%     % 
%     Therefore,
%     % 
%     \begin{align*}
%     \marginalloss_j(\instance)
%     &=\sum_{k\ge j,k\ \mathrm{even}}\theta_k
%     -\sum_{k>j,k\ \mathrm{odd}}\theta_k
%     =\sum_{k\ge j,k\ \mathrm{even}}(\theta_k-\theta_{k+1})
%     \\ \marginalgain_j(\instance)
%     &=\theta_j-\sum_{k\ge j,k\ \mathrm{even}}(\theta_k-\theta_{k+1})
%     =\sum_{k\ge j,k\ \mathrm{odd}}(\theta_k-\theta_{k+1})
%     \end{align*}
%     % 
%     completing the proof.
% \Halmos\endproof

% Note that because $\OPT(\instance^{+j}) = \OPT(\instance^{-j}) + \type_j$ and $\potential(\type_j) = \reward(\type_j,\type_j)/2$ for this reward function, we immediately get the following \namecref{cor: marginal_average}.

% \begin{corollary}\label{cor: marginal_average}
%         If $\typespace=[0,1]$ and $\reward(\type,\type') = \min\{\type,\type'\}$, then for all jobs $j$,
%     \[ \potential(\type_j) = \frac{\marginalloss_j(\instance) + \marginalgain_j(\instance)}{2}. \]
% \end{corollary}

% We are now ready to prove our main result about the interpretation of potential, that the true opportunity costs $\marginalloss_j(\instance),\marginalgain_j(\instance)$ concentrate around $p(\theta_j)$ in a random uniform instance, assuming the market is sufficiently thick.  We without loss consider job $j=1$ and fix its type $\theta_1$.

% \begin{theorem} \label{thm:interpretation}
% Let $\typespace=[0,1]$ and $\reward(\type,\type') = \min\{\type,\type'\}$.
% Fix $\theta_1\in\typespace$ and suppose $\theta_2,\ldots,\theta_n$ are drawn IID from the uniform distribution over $[0,1]$, forming a random instance $\instance\in\typespace^n$, for some $n\ge 2$. Then, both $\mathbb{E}[\marginalloss_1(\instance)]$ and $\mathbb{E}[\marginalgain_1(\instance)]$ are within $O(1/n)$ of $\potential(\type_1)$ and moreover $\mathrm{Var}(\marginalloss_1(\instance))=\mathrm{Var}(\marginalgain_1(\instance)) = O\left(\frac{1}{n}\right)$.
% \end{theorem}
% %
% \proof{Proof.}
% We prove the statement only for $\marginalloss_1(\instance)$, and the analogous result for $\marginalgain_1(\instance)$ follows from \Cref{cor: marginal_average}.
% %
% If $\type_1=0$, then $\marginalloss_1(\instance)=0$ for any $\instance$, coinciding with $\potential(\type_1)=0$. Then, for the remainder of the proof, assume that $\type_1>0$.
% %
% Consider the random variable $N=|\{j:\theta_j<\theta_1\}|$, which counts the number of points between $0$ and $\theta_1$ and has distribution $\text{Binom}(\Njob-1,\type_1)$.
% %
% These $N$ points divide the interval $[0,\theta_1]$ into $N+1$ intervals with total length $\type_1$.
% %
% From \Cref{lem:marginal_as_interval}, the marginal loss $\marginalloss_1(\instance)$ is determined by computing the total length of a subset of these intervals. % Maybe add figure?
% %
% It is known that if $N$ random variables are drawn independently from $\text{Unif}[0,1]$, then the joint distribution of the induced interval lengths is $\text{Dirichlet}(1,1,\ldots,1)$ with $N+1$ parameters all equal to 1 (i.e., drawn uniformly from the simplex).
% %
% In particular, since the Dirichlet distribution is symmetric, the sum of any $k$ of these intervals is equal in distribution to the $k$-th smallest sample ($k$-th order statistic) of the $N$ uniform random variables, whose distribution is known to be $\text{Beta}(k,N+1-k)$.
% %
% Thus, conditional on $N$, with $N\ge 1$, since the distribution of each of the $N$ jobs to the left of $\type_1$ is $\text{Unif}[0,\type_1]$, the distribution of $\marginalloss_1(\instance)/\type_1$ is $\text{Beta}(k_L,N+1-k_L)$,
% where $k_L$ is either $\lceil\frac{N+1}2\rceil$ or $\floor{\frac{N+1}2}$.
% %
% To be precise, for $N\ge 1$, $k_L=\floor {N/2}+1=\lceil\frac{N+1}2\rceil$ if $n$ is even, and $k_L=\ceil{N/2}=\floor{\frac{N+1}2}$ if $n$ is odd.
% %
% Moreover, if $N=0$, then $\marginalloss_1(\instance)=\type_1$ if $n$ is even, and $\marginalgain_1(\instance)=0$ if $n$ is odd.
% %
% Hence,
% \begin{align*}
% \mathbb{E}[\marginalloss_1(\type) \mid N] 
% &= \type_1\frac{k_L}{N+1}
% % \label{eq: cond_exp}
% \end{align*}
% for all $N\in\{0,\ldots,n-1\}$.
% %
% Since $|k_L-(N+1)/2|\le 1$, then $|\mathbb{E}[\marginalloss_1(\type) \mid N]-\type_1/2|\le \type_1/(N+1)$, thus
% % 
% \begin{align*}
% &\left|\mathbb{E}[\marginalloss_1(\type)]-\frac{\type_1}{2}\right| 
% \le \type_1\mathbb{E}\left[\frac{1}{N+1}\right] \\
% &\qquad = \type_1\sum_{k=0}^{n-1} \frac{1}{k+1}\binom{n-1}{k}\type_1^k(1-\type_1)^{n-1-k} \\
% &\qquad = \frac{\type_1}{n}\sum_{k=0}^{n-1} \binom{n}{k+1}\type_1^k(1-\type_1)^{n-1-k} \\
% &\qquad = \frac{1}{n}\sum_{k=1}^{n} \binom{n}{k}\type_1^k(1-\type_1)^{n-k} \\
% &\qquad = \frac{1-(1-\type_1)^{n}}{n} = O\left(\frac{1}{n}\right).
% \end{align*}
% %
% This completes the proof of the statement about $\mathbb{E}[\marginalloss_1(\instance)]$.

% For the statement about variance, we know $\mathrm{Var}(\marginalloss_1(\type)) = \mathbb{E}[\mathrm{Var}(\marginalloss_1(\type) \mid N)] + \mathrm{Var}(\mathbb{E}[\marginalloss_1(\type)\mid N])$ by the law of total variance. For the first term, we have $\mathrm{Var}(\marginalloss_1(\type) \mid N) = \type_1^2\frac{k_L(N+1-k_L)}{(N+1)^2(N+2)}\indicator\{N\ge 1\}\le \frac{\type_1^2}{N+1}$, and then from the previous argument $\mathbb{E}[\mathrm{Var}(\marginalloss_1(\type) \mid N)]=O(1/n)$.
% %
% For the second term, 
% \begin{align*}
% \mathrm{Var}(\mathbb{E}[\marginalloss_1(\type)\mid N]) 
% &= \mathrm{Var}\left(\mathbb{E}[\marginalloss_1(\type)\mid N] - \frac{\type_1}{2}\right) \\
% &\le \mathbb{E} \left[\left(\mathbb{E}[\marginalloss_1(\type)\mid N] - \frac{\type_1}{2}\right)^2\right] \\
% % 
% &\le \mathbb{E}\left[\frac{\type_1^2}{(N+1)^2}\right] \\
% &\le \type_1^2\mathbb{E}\left[\frac{1}{N+1}\right] = O\left(\frac{1}{n}\right).
% \end{align*}
% Thus, $\mathrm{Var}(\marginalloss_1(\instance)) = O\left(1/n\right)$, completing the proof.
% \Halmos\endproof



\section{Numerical Experiments} \label{sec:numerical_experiments}

In this section, we show via numerical simulations that under the reward structure from delivery pooling, our proposed potential-based greedy algorithm ($\PB$) outperforms a number of benchmark algorithms, given sufficient density.
%
We first generate synthetic data under a setting more aligned with our theoretical results, and find that $\PB$ outperforms all benchmarks starting from very low market densities ($\sojourn\ge 5$).
%
We then apply all algorithms and benchmarks on real data from the Meituan platform.
%
Although the real-life setting differs from our theoretical model in a number of ways --- that we address in \Cref{sec:sim_meituan} --- the results remain qualitatively similar: $\PB$ performs the best, starting from the modest density level corresponding to ``each job is able to wait for one minute before being dispatched''.
%
Additional simulation results, including additional performance metrics and more general synthetic environments (e.g. two-dimensional locations, different spatial distributions, and different reward topologies) are provided in \Cref{sec: sim_results_extra}.


The benchmark algorithms that we compare with are categorized into two groups, \emph{forecast-agnostic}, and \emph{forecast-aware}, depending on the use of historical data/demand distributional information to make matching decisions. We now describe each of them.
%


\subsection{Benchmark Algorithms} \label{sec:sim_benchmarks}


As we have discussed earlier, the potential based greedy algorithm $\PB$ does not rely on any historical data/demand forecast.
% 
The same is true for naive greedy $\gre$, and we refer to heuristics with this property as \newterm{forecast-agnostic}. 
%
In this group, we consider batching policies as additional benchmarks, which are highly relevant both in theory and in practice.


\paragraph{Naive Batching.} 


Batching heuristics periodically compute optimal matching solutions given the available jobs.
% 
Since it is without loss in our model to wait to match, we consider batching algorithms that re-compute optimal matchings whenever any available job becomes critical.
%
% 
The \newterm{naive batching} algorithm ($\batching$) dispatches all available jobs according to this optimal solution. 
%
In particular, under the theoretical model introduced in \Cref{sec:model}, $\batching$ clears the market every $\sojourn + 1$ arrivals, which is equivalent to the one analyzed in \citet{ashlagi2019edge}. 
%
However, an obvious suboptimality of this algorithm is that there are pairs of jobs that are matched and dispatched before any of them becomes critical. 



\paragraph{Rolling Batching.} 

To address the previously mentioned issue of dispatching non-critical pairs of jobs, we also consider a modified batching heuristic, termed \newterm{rolling batching} ($\rbatching$).
%
$\rbatching$ also computes an optimal matching solution every time any available job becomes critical. Instead of dispatching all jobs, however, $\rbatching$ only dispatches the critical job and its match (if any) according to the optimal solution. 


\medskip 


In practice, delivery platforms typically have access to past order data that allows to predict patterns of future arrivals to some extent. 
%
In contrast to the aforementioned algorithms, we say that a policy is \newterm{forecast-aware} if it uses historical data to make pooling decisions. 
%
We consider dual-based benchmark algorithms that use optimal dual solutions (i.e. shadow prices) to approximate the opportunity cost of matching each job. 
% 
More specifically, given an instance from the historical data (i.e. a sequence of delivery requests with information about origin, destination, and timestamp), we can solve the dual program of a linear relaxation of the IP defined in \eqref{eq: OPT}.
% 
\footnote{We provide a detailed description of the primal and dual linear programs in \Cref{sec: LP}.}
% 
The optimal dual variables associated with the capacity constraint of each job will then provide a proxy for the marginal value of the job, which we use to construct two additional benchmark algorithms.


\paragraph{Hindsight Dual.}
% 
Ideally, if we had access to the dual optimal solution for an instance in advance, we would use them as the opportunity cost of matching each job.
%
In an attempt to measure the value of this hindsight information, we simulate a \newterm{hindsight-dual} algorithm ($\dual$) that operates retrospectively on the same instance.
%
Formally, given an instance $\instance\in\typespace^\Njob$ and parameter $\sojourn$, let $\lambda_k=\lambda_k(\instance,\sojourn)$ be the optimal dual variable associated with the constraint $\sum_{k:j\neq k} x_{jk} \le 1$ in the linear program  (defined in \eqref{eq: LP}). The $\dual$ algorithm operates as an index-based greedy matching algorithm (as defined in \Cref{alg:dynamic}) where the index function is given by $\indexf_{\dual}(\type_j,\type_k) = \reward(\type_j,\type_k) - \lambda_k$, for all $j,k\in [\Njob]$, such that $j\neq k, |j-k| \le d$.
% 
Note that this index function is computed based on the entire realized instance $\instance$, thus this algorithm is only for benchmarking purposes and is not practically feasible.



\paragraph{Average Dual.}
%
A more realistic approach for estimating the opportunity costs is to average the hindsight duals of the jobs from historical data.
% 
However, this estimation is not free of challenges. 
% 
In particular, computing the dual solutions yields a shadow price for every observed type, which might not cover the entire type space and thus 
many future arrivals may not have an associated shadow price.
% 
To address this issue, we discretize the type space and estimate average duals for each discrete type as the empirical average of the shadow prices for jobs associated to the discrete type.
% 
To be precise, let $H$ be the set of all instances in the historical data, and consider a discretization $\{\typespace_1,\typespace_2,\ldots,\typespace_p\}$
% 
that partitions the space into $p$ cells (i.e. these cells are mutually exclusive and jointly exhaustive).
%
Each cell $i=1,2,\ldots,p$ is associated with a shadow price $\Bar{\lambda}_i$ that is the average shadow price of historical jobs lying in the same cell, i.e. $\Bar{\lambda}_i$ is the average of $\{\lambda(\instance,\sojourn) : \instance\in H, \type_j \in \typespace_i \}$.
%
Then, the \newterm{average-dual} algorithm
($\averagedual$) is an index-based greedy matching algorithm that uses the shadow prices for the cells, i.e.\ uses index function
$\indexf_{\averagedual}(\type,\type') = \reward(\type,\type') - \sum_{i=1}^p\Bar{\lambda}_{i}\indicator\{\type'\in\typespace_i\}.$

\subsection{Synthetic Data}\label{sec:sim_unif_1D}

We first consider the setting analyzed in our theoretical results, where job destinations are uniformly distributed on a one-dimensional space.
% 
We fix the total number of jobs at $\Njob=1000$, 
and compare algorithms at density levels $\sojourn\in\{5,10,15,20,25,30\}$.
%
For each $(\Njob, \sojourn)$ pair, we generate 100 instances $\instance \in[0,1]^\Njob$ uniformly at random, apply all algorithms on each of them, and compute the average regret and reward \newterm{ratio} ($\ALG(\instance,d)/\OPT(\instance,d)$) achieved by each algorithm and benchmark.
% 
Additional performance metrics, including the fraction of jobs that are pooled and the fraction of the total distance that is reduced by pooling, are presented in \Cref{sec:match_rate}. Similar results for the 1D, non-uniform setting are presented in \Cref{sec: nonunif_1D}.
% 

\subsubsection{Comparison with forecast-agnostic heuristics.}
%
\Cref{fig:1D} compares potential-based greedy $\PB$ with naive greedy $\gre$ and the two batching algorithms.
% 
As expected from our analysis, $\gre$ performs poorly even at high densities. To our surprise, $\PB$ outperforms both batching algorithms for all considered density values.
%
To assess the performance of the algorithms relative to the hindsight optimal, we also compute the \emph{ratio} for each instance, and plot its empirical average in \Cref{fig:1D_ratio}.
% 
We can see that even at relatively low density levels, $\PB$ achieves over 95\% of the maximum possible travel distance saved from pooling. $\gre$, on the other hand, performs worse relative to hindsight optimal as density increases--- despite the fact that the
%the fraction of
travel distance saved by $\gre$ increases with density (see \Cref{fig:1D_unif_saving_fraction} in \Cref{sec: 1D_extra}). 
%
\begin{figure}%[H]
  \centering
  \subcaptionbox{Average regret.%
    \label{fig:1D_regret}}[0.49 \textwidth]{\includegraphics[width = \figWidth \textwidth]{Simulation_Results/Synthetic/New/1D_Pooling/1D_Common_Origin/Forecast-Agnostic/regret.png}}
  \hfill
  \subcaptionbox{Average ratio.%
    \label{fig:1D_ratio}}[0.49 \textwidth]{\includegraphics[width = \figWidth \textwidth]{Simulation_Results/Synthetic/New/1D_Pooling/1D_Common_Origin/Forecast-Agnostic/frac_of_OPT.png}}
  % 
  \caption{Comparison of average regret and reward ratio for forecast-agnostic heuristics in random 1D instances.}
  % 
  \label{fig:1D}
\end{figure}
\subsubsection{Comparison with forecast-aware heuristics.}

We now compare $\PB$ with our two forecast-aware benchmarks, $\dual$ and $\averagedual$.
%
%
Under $\dual$, for each of the 100 instances generated for each market condition $(n,d)$, we compute a shadow price for each job via the dual LP.
%
As discussed in \Cref{sec:sim_benchmarks}, this algorithm is not practically feasible, and we present its performance here mainly to illustrate the value of this hindsight information used in this particular way.
%
%
For $\averagedual$, for each $(n,d)$, we first generate 400 \emph{historical instances} in the same way that the original 100 instances were drawn.
% 
%
We then compute optimal dual variables for each historical instance, and estimate average shadow prices $\Bar{\lambda}$ for a uniform discretization of the type space $[0,1]$ into 100 intervals.
% 

\Cref{fig:1D_forecastaware} compares the average regret and reward ratio on the original 100 instances.
%
We can see that $\PB$ performs on par with $\averagedual$ across all density parameters, without relying on any forecast/historical data. 
%
Notably, even with access to hindsight information, $\dual$ is consistently dominated by both $\averagedual$ and $\PB$, with the performance gap narrowing as $\sojourn$ increases.
%


\begin{figure}%[H]
  \centering
  \subcaptionbox{Average regret.%
    \label{fig:1D_forecastaware_regret}}[0.49 \textwidth]{\includegraphics[width = \figWidth \textwidth]{Simulation_Results/Synthetic/New/1D_Pooling/1D_Common_Origin/Forecast-Aware/regret.png}}
  \hfill
  \subcaptionbox{Average ratio.%
    \label{fig:1D_forecastaware_ratio}}[0.49 \textwidth]{\includegraphics[width = \figWidth \textwidth]{Simulation_Results/Synthetic/New/1D_Pooling/1D_Common_Origin/Forecast-Aware/frac_of_OPT.png}}
  % 
  \caption{Comparison of average regret and reward ratio for forecast-aware heuristics in random 1D instances. }
  % 
  \label{fig:1D_forecastaware}
\end{figure}




\subsection{Order-Level Data from Meituan Platform}\label{sec:sim_meituan}


In this \namecref{sec:sim_meituan}, we test the algorithms and benchmarks using data from the Meituan platform, made public for the 2024 INFORMS TSL Data-Driven Research Challenge.\footnote{\url{https://connect.informs.org/tsl/tslresources/datachallenge}, accessed January 15, 2025.
%
See \Cref{sec:market_dynamics} for more details on the dataset, as well as high-level illustrations of market dynamics over both space and time.
} 
%
The dataset provides detailed information on various aspects of the platform's delivery operations, for one city and a total of 8 days in October 2022.  
%
For each of the 569 million delivery orders, the dataset provides the pick-up and drop-off locations (the latitudes and longitudes were shifted for privacy considerations), and the timestamps at which orders are created, pushed into the dispatch system, etc.
% 
% 
This allows us to (i) construct instances of job arrivals and (ii) calculate travel distances and pooling opportunities.




Since the pick-up and drop-off locations now reside in a two-dimensional (2D) space, we first extend our definition of pooling reward to the 2D space, derive the potential of different job types, and provide the notion of a pooling window (i.e. the sojourn time) that determines which jobs can be pooled together.


\paragraph{Pooling reward.}
% 
Each job $j$ has type $\type_j = (\xorigin_j,\xdestination_j)$, 
% 
with $\xorigin_j,~\xdestination_j\in \R^2$ corresponding to the coordinates of the job origin and destination.
%
The travel distance saved by pooling two orders together is the difference between (i) the travel distance required to fulfill the jobs in two separate trips, and (ii) the minimum travel distance for fulfilling the two requests in a single, pooled trip.
%
% 
For the pooled trip, the travel distance always includes the distance between the two origins and that between the two destinations (both orders must be picked up before either of them is dropped off; otherwise the orders are effectively not pooled). 
% 
The remaining distance depends on the sequence in which the jobs are picked up and dropped off, resulting in four possible routes. The minimum travel distance is then determined by evaluating the total distance for all four routes and selecting the smallest value.
% 
Formally, letting $\norm{\cdot}$ denote the Euclidean norm, the reward function is
% 
\begin{align}
    \reward(\type,\type') = &  \norm{\xdestination-\xorigin} +  \norm{\xdestination'-\xorigin'}  \nonumber\\
    % 
    & -\left(\norm{\xorigin-\xorigin'} + \min\left\{\norm{\xdestination-\xorigin},\norm{\xdestination'-\xorigin'},\norm{\xdestination-\xorigin'},\norm{\xdestination'-\xorigin'}\right\}+\norm{\xdestination-\xdestination'}\right). \label{eq:2D_reward_part_2}
\end{align}
% 
Note that unlike the 1D common origin setting, the pooling reward in 2D space with heterogeneous origins is not guaranteed to be positive. As a result, we assume that all index-based greedy algorithms only pool jobs together when the reward is non-negative.

\paragraph{Potential.}
% 
For a given job type $\type=(\xorigin,\xdestination)$, the maximum reward a platform can achieve by pooling it with another job is the distance of the job, $\norm{\xdestination - \xorigin}$.\footnote{To see this, first observe that when a job $\type = (\xorigin,\xdestination)$ is pooled with another job of identical type, the travel distance saved is precisely $\norm{\xdestination - \xorigin}$.
%
On the other hand, let $m$ be the minimum in \eqref{eq:2D_reward_part_2}. Then, by triangle inequality $\|O'-D'\|\le \|O-O'\| + \|D-O\| + \|D-D'\|$, and thus $\reward(\type,\type')\le 2\|D-O\|-m \le \|D-O\|$, since $m\le \|D-O\|$.
% % 
To achieve an even higher pooling reward, it must be the case that the second part of the pooling reward \eqref{eq:2D_reward_part_2} is strictly smaller than $\norm{\xdestination' - \xorigin'}$. This implies that the total distance traveled to visit all four locations is smaller than the distance between two of them, which is impossible.}  

As a consequence, the potential of a job $\type$ is $\potential(\type) = \norm{\xdestination - \xorigin}/2$. %, 
% 
This is aligned with our 1D theoretical model, in which longer deliveries have higher potential reward from being pooled with other jobs in the future.
%

\paragraph{Pooling window.} 
% 
Instead of assuming that each job is available to be pooled for a fixed number of future arrivals, we take the actual timestamps of the jobs, and assume that the jobs are available for a fixed time window, at the end of which the order becomes critical.
%
Intuitively, the length of the pooling window corresponds to the ``patience level'' of the jobs.
% 
This deviates from our assumptions for the theoretical model, but it is straightforward to see how our algorithm as well as the benchmarks we compare with can be generalized and applied.


\subsubsection{Comparison with forecast-agnostic heuristics.}

We focus our analysis on a three-hour lunch period (10:30am to 1:30pm) for the 8 days of data provided by the challenge.
% 
During this lunch period, there are approximately $130$ orders per minute on average, thus $\Njob\approx 24,000$ orders in total (see \Cref{fig:meituan_orders_per_hour} in \Cref{sec:market_dynamics} for an illustration of order volume over time).
% 
If the platform allows a time window of one minute for each order before it has to be dispatched, the average density would be around $130$ jobs. 
% 
\Cref{fig:meituan} compares the forecast-agnostic heuristics, as the pooling window increases from 30 seconds to 5 minutes. We can see that a single minute is sufficient for $\PB$ to outperform every other forecast-agnostic algorithm, achieving around 80\% of the hindsight optimal reward.
% 
Moreover, the performance of $\PB$ improves steadily as the allowed matching window increases.
%

\begin{figure}%[H]
  \centering
  \subcaptionbox{Average regret.%
    \label{fig:meituan_regret}}[0.49 \textwidth]{\includegraphics[width = \figWidth \textwidth]{Simulation_Results/Meituan/City/regret_fagnostic.png}}
  \hfill
  \subcaptionbox{Average ratio.%
    \label{fig:meituan_ratio}}[0.49 \textwidth]{\includegraphics[width = \figWidth \textwidth]{Simulation_Results/Meituan/City/ratio_fagnostic.png}}
  % 
  \caption{Comparison of average regret and reward ratio for forecast-agnostic heuristics in Meituan data.}
  % 
  \label{fig:meituan}
\end{figure}


\subsubsection{Comparison with forecast-aware heuristics.}

We now compare our algorithm with the forecast-aware heuristics on data from the same three-hour lunch period, 10:30am to 1:30pm.
% 
To simulate the \emph{hindsight-dual} algorithm $\dual$ for each instance associated to each day, we compute the shadow prices via the dual LP program and then compute the ``online'' pooling outcomes using pooling reward minus these shadow prices as the index function.
%
% 
To estimate historical average shadow prices for the $\averagedual$ algorithm, for each of the 8 days, we (i) use the remaining 7 days as the historical data, and (ii) discretize the 2D space into discrete types based on job origin and destination using H3, the hexagonal spatial indexing system developed by Uber.\footnote{See {\url{https://www.uber.com/blog/h3/} (accessed January 7, 2024) for more details on the H3 package.} 
% 
% 
The system supports sixteen resolution levels, each tessellating the earth using hexagons of a particular size.
% %
For a given resolution, the set of all origin-destination (OD) hexagon pairs represents the partition of the type space $\typespace$, as described in \Cref{sec:sim_benchmarks}.
% %
% % 
To compute the average shadow price for a particular OD hexagon pair, we compute the average of optimal dual variables associated with all historical jobs that share the same hexagon pair.
% %
If there are no jobs for a given pair, however, then we consider the hexagon pairs at coarser resolutions (moving one or more levels up in the H3 hierarchy, with each level increasing the size of the hexagons by a factor of 7) until there exist associated historical jobs.
% %
There is a tradeoff between granularity and sparsity when we choose the baseline resolution level, as finer resolutions only average over trips with close-by origins and destinations, but may lack sufficient historical data for accurate estimation (see \Cref{fig:meituan_CDF_count_per_OD_pair} for the distribution of order volume by OD hexagon pairs).
% 
After testing all available resolution levels, we found that levels 8 and 9 perform the best, depending on the time window. The results presented in this section report the result corresponding to the best resolution for each time window.
}

\Cref{fig:meituan_forecastaware} compares the regret and reward ratio of $\PB$ to forecast-aware heuristics.
%
We first observe that in contrast to the synthetic setting from \Cref{sec:sim_unif_1D}, $\dual$ now incurs the smallest regret at all density levels. %
% 
This performance difference appears to be driven by the highly non-uniform spatial distribution of jobs --- additional simulations, presented in \Cref{sec:sim_unif_1D_2D}, demonstrate that (i) $\dual$ performs no better than $\PB$ if job origins and destinations are drawn uniformly at random from a unit square, and (ii) when jobs are generated from a non-uniform distribution, $\dual$ slightly outperforms $\PB$. Of course, we remind the reader that $\dual$ is not a realistic algorithm because it requires knowing the exact future.

The worse performance of $\averagedual$ in comparison to $\PB$ also highlights the challenges of estimating the opportunity costs of pooling different jobs from historical data in real-world settings. 
% 
Intuitively, taking historical data into consideration should be valuable in scenarios with highly non-uniform spatial distributions. In this case, however, the sparsity of the data for many if not most origin-destination pairs makes it difficult to properly estimate the opportunity costs, or requires a very coarse granularity for location.  $\PB$ bypasses this sparsity-granularity tradeoff by only considering the reward topology, i.e.\ the space of all possible types.
%


\begin{figure}%[H]
  \centering
  \subcaptionbox{Average regret.%
    \label{fig:meituan_forecastaware_regret}}[0.49 \textwidth]{\includegraphics[width = \figWidth \textwidth]{Simulation_Results/Meituan/City/regret_faware.png}}
  \hfill
  \subcaptionbox{Average ratio.%
    \label{fig:meituan_forecastaware_ratio}}[0.49 \textwidth]{\includegraphics[width = \figWidth \textwidth]{Simulation_Results/Meituan/City/ratio_faware.png}}
  % 
  \caption{Comparison of average regret and reward ratio for forecast-aware heuristics in Meituan data.}
  % 
  \label{fig:meituan_forecastaware}
\end{figure}


% \subsubsection{Match rate and saving fraction.}\label{sec:match_rate}
% % \paragraph{Match Rate and Saving Fraction.}

% In addition to regret and reward ratio, we consider in this section two additional performance metrics: the \newterm{match rate}, i.e. the fraction of jobs that were pooled instead of dispatched on their own, and  the \newterm{saving fraction}, i.e. the fraction of the total distance that is reduced by pooling, relative to the total travel distance without any pooling~\citep[see][]{aouad2020dynamic}.
% %
% % 
% For brevity, we focus only on the potential-based greedy algorithm $\PB$ and the hindsight optimum $\OPT$ in \Cref{fig:meituan_2}.
% % 
% A comprehensive comparison across all 
% algorithms and benchmarks is provided in \Cref{sec: meituan_extra}.


% \begin{figure}[H]
%   \centering
%   \subcaptionbox{Average match rate.%
%     \label{fig:meituan_2_match_rate}}[0.49 \textwidth]{\includegraphics[width = \figWidth \textwidth]{Simulation_Results/Meituan/City/match_rate.png}}
%   \hfill
%   \subcaptionbox{Average saving fraction.%
%     \label{fig:meituan_2_saving_fraction}}[0.49 \textwidth]{\includegraphics[width = \figWidth \textwidth]{Simulation_Results/Meituan/City/saving_rate.png} }
%   % 
%   \caption{Match rate and Saving Fraction, Meituan Order-Level Data.}
%   % 
%   \label{fig:meituan_2}
% \end{figure}

% Both regret and reward ratio are performance metrics that compare pooling algorithm relative to the hindsight optimal pooling outcome. The saving fraction
% % 
% as shown in \Cref{fig:meituan_2_saving_fraction} illustrates the benefit of delivery pooling in comparison to the total distance traveled, and is upper bounded by 0.5 (since reward from pooling a job cannot exceed its distance).
% % 
% We can see that if the platform allows a time window of at least $1$ minute for each order before it has to be dispatched, $\PB$ achieves a reduction in distance traveled by over 20\%. At $5$ minutes, this fraction increases to roughly 30\%.
% %
% Since the saving fraction is proportional to the total reward collected by the algorithm, our previous performance comparisons imply that the saving fraction achieved by $\PB$ is the best among all tested \emph{practical} heuristics, i.e. excluding $\dual$ and $\OPT$.

% \Cref{fig:meituan_2_match_rate} compares the match rate achieved under $\PB$ and $\OPT$. We can see that $\PB$ consistently pools more jobs than $\OPT$ (5-10\%). 
% %
% This highlights an additional desirable property of $\PB$ and, more generally, of \emph{index-based greedy matching} algorithms: they pool as many jobs as possible.
% % 
% This is desirable in practice, since drivers who are offered just only one order from a platform may try to pool orders from competing platforms \citep[see e.g.][]{reddit2022grubhub,reddit2023doordash}, leading to poor service reliability for customers.

% Note that for the one-dimensional settings, match rates are very close to 100\% under greedy policies since at most one job is left unmatched (see \Cref{fig:1D_unif_match_rate} in \Cref{sec: 1D_extra} for comparisons). 
% %
% In the real-data setting, match rates are generally lower since we pool the critical job with another only when it's possible to achieve a positive pooling reward. Nevertheless, \Cref{sec: meituan_extra} shows that greedy algorithms ($\gre$, $\PB$, $\dual$, $\averagedual$) still achieve substantially higher match rates in comparison to $\batching$, $\rbatching$, and $\OPT$.




\section{Concluding Remarks and Future Work} \label{sec:conclusion}

In this work, we study dynamic non-bipartite matching in the context of pooling delivery orders.
% 
Our modeling approach captures two key features of the problem through the reward topology and the assumption that jobs can be matched until they become \textit{critical}.
%
We propose a simple, forecast-agnostic potential-based greedy algorithm, specially designed for this reward topology, that performs surprisingly well both in synthetic and real data. It uses a new notion of potential as opportunity cost, that considers the value of keeping long-distance deliveries in the platform for better pooling opportunities in the future. We show that for our reward topology of focus, potential-based greedy improves upon greedy both in worst-case and empirical performance. Moreover, we showcase the robustness of potential-based greedy through numerical experiments in real data, outperforming several commonly used benchmark algorithms given enough market density.
%
We provide some intuition behind the success of this approach by showing through theory and experiments that different notions of opportunity cost are intimately related to our definition of potential, especially as density increases. 
%
Moreover, we believe that the main insight of long-distance jobs being more valuable to hold should apply even when pooling more than two orders into a single trip (see \citet{wei2023constant}); and we believe our notion of potential could be relevant even on non-metric reward structures, where it is a measure of e.g.\ the popularity of a volunteer position (see \citet{manshadi2022online}).
%
Formalizing and generalizing these results is an interesting avenue of future research.
%
Overall, these surprising results showcase the value of using the knowledge of reward topology to design application-specific algorithms. This general idea, which has received limited attention so far, could also be applied to other domains, such as ride-sharing or kidney exchange.

%\THEEndNotes
% \begingroup \parindent 0pt \parskip 0.0ex \def\enotesize{\normalsize} \theendnotes \endgroup

% Acknowledgments here
\ACKNOWLEDGMENT{
The authors would like to thank
Itai Ashlagi,
Omar Besbes,
Francisco Castro,
Yash Kanoria,
Jake Marcinek,
Rad Niazadeh,
Scott Rodilitz,
Daniela Saban, and
Alejandro Torrielo
for valuable comments and discussions.
}


% References here (outcomment the appropriate case)

% CASE 1: BiBTeX used to constantly update the references
%   (while the paper is being written).
% \bibliographystyle{informs2014trsc} % outcomment this and next line in Case 1
% \bibliography{DraftBib} % if more than one, comma separated

%\bibliographystyle{informs2014trsc} % outcomment this and next line in Case 1
%\bibliography{sample} % if more than one, comma separated

% CASE 2: BiBTeX used to generate mypaper.bbl (to be further fine tuned)
\input{Submission_TS.bbl} % outcomment this line in Case 2

%If you don't use BiBTex, you can manually itemize references as shown below.

%\bibliographystyle{nonumber}

% \begin{thebibliography}{3}
% \providecommand{\natexlab}[1]{#1}
% \providecommand{\url}[1]{\texttt{#1}}
% \providecommand{\urlprefix}{URL }

% \bibitem[{Smith(2005)}]{smith2005}
% Smith J (2005) Optimal resource allocation in humanitarian logistics.
%   \emph{Journal of Operations Research} 30(2):123--135.
  
% \bibitem[{Jones(2010)}]{jones2010}
% Jones S (2010) Stochastic programming models for humanitarian logistics.
%   \emph{INFORMS Mathematics of Operations Research} 35(4):567--580.

% \bibitem[{Brown(2015)}]{brown2015}
% Brown D (2015) \emph{Introduction to Stochastic Programming} (Springer).

% \end{thebibliography}



% Appendix here
% Options are (1) APPENDIX (with or without general title) or
%             (2) APPENDICES (if it has more than one unrelated sections)
% Outcomment the appropriate case if necessary
%
\begin{APPENDIX}{}
We provide in \Cref{appx:proofs} proofs that are omitted from \Cref{sec:PB} of the paper.
% 
\Cref{sec: interpretation} shows % further details on 
an interpretation of potential in terms of the marginal value of jobs. 
% 
Theoretical guarantees under alternative reward topologies are stated and proved in \Cref{appx:alternative_reward_topologies}. 
%
\Cref{sec: LP} provides details on the LP formulation and dual variables used for the $\dual$ and $\averagedual$ benchmarks. 
%
Finally, we include in \Cref{sec:market_dynamics} high-level descriptions of market-dynamics observed in Meituan data, and in \Cref{sec: sim_results_extra} additional simulation results, both for settings studied in the body of the paper as well as more general settings with 2D locations, non-uniform spatial distributions, and different reward topologies. 

%
%   or
%
% \begin{APPENDICES}
\crefalias{section}{appendix}
\crefalias{subsection}{appendix}
\crefalias{subsubsection}{appendix}

\section{Deferred Proofs} \label{appx:proofs}

\subsection{Proof of \Cref{prop:greedy_linear_lower_bound}} \label{pf:greedy_linear_lower_bound}

Let $0<\eps<1/3$.
% 
Consider an instance $\instance \in [0,1]^\Njob$ such that $\type_{n/2}<\type_{n/2-1}<\ldots<\type_{1}<\eps$ (low-type jobs) and $\type_{n/2+1}=\type_{n/2+2}=\ldots=\type_{\Njob}=1$ (high-type jobs). 
%
Note that because of \Cref{deliverygreedy}, when the first job becomes critical, $\gre$ chooses to match with a high-type job, i.e. $\matchof(1)\in \{n/2+1,\ldots,\Njob\}$, collecting $\reward(\type_1,\type_{\matchof(1)})\le \eps$ independent of the actual choice of $\matchof(1)$.
%
In particular, the second job is not matched and is then the next to become critical.
%
Since there are still high-type jobs to be matched, then by the same argument $\matchof(2)\in \{n/2+1,\ldots,\Njob\}$ and $\reward(\type_2,\type_{\matchof(2)})\le \eps$.
%
By induction, since there are as many high-type jobs as low-type jobs, every job $j \in \{1,\ldots,n/2\}$ is matched with $\matchof(j)\in \{n/2+1,\ldots,\Njob\}$ and $\reward(\type_j,\type_{\matchof(j)}) \le \eps$. Therefore, ${\gre}(\instance,\infty) \le \eps n/2$. 

On the other hand, $\reward(\type_{j},\type_{j+1}) = 1$ for all $j \in \{\Njob/2+1,\ldots,\Njob-1\}$. Thus, an optimal matching solution would always match all $\Njob/2$ high-type jobs together, collecting $\OPT(\instance, \infty) \ge \Njob/4$. 
%
Hence,
%
\[ 
    \regret_\gre(\instance,\infty) = \OPT(\instance,\infty) - \gre(\instance, \infty) \ge \frac{(1-2\eps)}{4}\Njob, 
\]
independent of the tie-breaking rule. Since this inequality holds for all $0<\eps<1/3$, the proof is complete.\Halmos
% 

\subsection{Proof of \Cref{prop:loglowerboundOffline}}
\label{pf:loglowerboundOffline}

We construct a sequence of instances $\instance(k)\in [0,1]^{\Njob(k)}$ with $\Njob(k)=2^{k+3} - 4$ for every integer $k\ge 0$, and show that the regret gain at each iteration, $\regret_\PB(\instance(k+1),\infty)-\regret_\PB(\instance(k),\infty)$ is at least $1/4$. For $k=0$, consider $\instance(0)=(1/2,0,1,0)$. We have that $\OPT(\instance(0),\infty) = 1/2$. In turn, under $\PB$, when the first job becomes critical, all available jobs are equally close, resulting in ties. Assume ties are broken so that the resulting matching is $\{(1,2),(3,4)\}$. Then, $\PB(\instance(0),\infty) = 0$, and thus $\regret_\PB(\instance(0),\infty) = 1/2$.
%
For $k\ge 1$, we inductively construct an instance $\instance(k)$ by carefully adding $2^{k+2}$ new jobs to be processed before the instance considered in the previous step. In particular, we add $2^{k}$ copies of $\instance(0)$, each scaled by $1/2^{k+1}$ and shifted in space so that $\PB$ matches within each copy. 
%

Formally, for $s\in\{0,\ldots,2^{k}-1\}$ and $r\in\{1,2,3,4\}$, define
\[ \instance(k)_{4s+r} = \frac{\instance(0)_r}{2^{k+1}} + \frac{s}{2^k}. \]
Lastly, define $\instance(k)_j = \instance(k-1)_{j-2^{k+2}} $ for $j \in \{ 2^{k+2}+1,\ldots,  \Njob(k) \}$.
%
By construction, for each $s\in \{0,\ldots,2^{k}-1\}$, $\PB$ matches job $4s+1$ with $4s+2$ since it is the closest in space (see \Cref{deliverypotential}), and then matches $4s+3$ with $4s+4$ for the same reason. After all these jobs are matched, the remaining jobs represent $\instance(k-1)$ exactly.
%
On the other hand, $\OPT$ is at least as good as processing $\OPT$ separately on each of the $2^k$ copies, and then on $\instance(k-1)$.
%
Hence, the regret gain $\regret_\PB(\instance(k),\infty) - \regret_\PB(\instance(k-1),\infty)$ is at least the sum of the regret of $\PB$ on each copy.
%
Moreover, note the reward function $\reward(\type,\type') = \min\{\type,\type'\}$ satisfies $\reward(\alpha+\type/\beta,\alpha\type'/\beta) = \reward(\type,\type')/\beta$ for any scalars $\alpha,\beta$ such that $\alpha+\type/\beta,\alpha\type'/\beta\in[0,1]$. 
%
Thus, the regret of $\PB$ on each copy is exactly $\regret_\PB(\instance(0),\infty)/{2^{k+1}}$, and then
%
\[ \regret_\PB(\instance(k),\infty) - \regret_\PB(\instance(k-1),\infty) \ge 2^k\frac{\regret_\PB(\instance(0),\infty)}{2^{k+1}} \ge \frac{1}{4}, \]
%
completing the proof.
\Halmos

\subsection{Proof of \Cref{prop:greedy_linear_lower_bound_dynamic}}
\label{pf:greedy_linear_lower_bound_dynamic}

Note that if $(\sojourn+1)$ is divisible by 4, then for any number of jobs $\Njob$ divisible by $(\sojourn+1)$, we have an exact partition $b=\Njob/(\sojourn+1)\in\{1,2,\ldots\}$ batches, each of size $(\sojourn+1)$.
%
We construct an instance $\instance\in[0,1]^\Njob$ that can be analyzed separately on each batch in the following sense.
%
For each $t=1,\ldots,b$, let $\instance^{(t)} = (\type_{j})_{j\in B_t}\in[0,1]^{(\sojourn+1)}$.
%
It is straightforward to check that $\OPT(\instance,\sojourn) \ge \sum_{t=1}^b \OPT(\instance^{(t)},\infty) $, since the matching solution induced by the offline instances on the right-hand side is feasible in the online setting.
%
Thus, if $\instance$ is such that $\gre(\instance,\sojourn) = \sum_{t=1}^b \gre(\instance^{(t)},\infty) $, then
\[ \OPT(\instance,\sojourn) - \gre(\instance,\sojourn) \ge \sum_{t=1}^b \regret_\gre(\instance^{(t)},\infty). \]
%
Moreover, we construct $\instance$ such that each $\instance^{(t)}\in[0,1]^{(\sojourn+1)}$ resembles the worst-case instance analyzed in \Cref{prop:greedy_linear_lower_bound}.
%
To this end, let $0<\eps<1$. Consider $\instance\in[0,1]^\Njob$ such that for each $t=0,\ldots,b-1$,
%
$\eps/2^{t+1}<\type_{(\sojourn+1)t+(\sojourn+1)/2}<\ldots<\type_{(\sojourn+1)t+1}<\eps/2^t$ (low-type jobs) and $\type_{(\sojourn+1)t+(\sojourn+1)/2+1}=\ldots=\type_{(\sojourn+1)(t+1)} = 1$ (high-type jobs).
%
By construction, when job $1$ becomes critical, the available jobs $k\in A(1)\subseteq B_1$ have types $\type_k < \type_1$ or $\type_k=1$. Then, by \Cref{deliverygreedy}, $\gre$ chooses to match with a high-type job $\matchof(1)\in B_1$.
%
As job $2$ becomes critical next, the new set of available jobs is $A(2)=A(1)\cup\{(\sojourn+1)+1\}\setminus\{\matchof(1)\}$, with $\type_{(\sojourn+1)+1}<\eps/2<\type_2$. Again, $\gre$ chooses to match with a high-type job $\matchof(2)\in B_2$.
%
By repeating this argument inductively, $\gre$ outputs $(C,\{\matchof(j)\}_{j\in C})$ such that $C = \bigcup_{t=1}^{b-1}\{(d+1)t + 1,\ldots, (d+1)t + \sojourn+1)t+(\sojourn+1)/2 \}$ and $\matchof(j) \in B_t\setminus C$ for all $j\in C\cap B_t$. 
%
Thus, $\gre(\instance,\sojourn) = \sum_{t=1}^b \gre(\instance^{(t)},\infty) $, and since $\instance^{(t)}\in[0,1]^{(\sojourn+1)}$ is specified as in the proof of \Cref{prop:greedy_linear_lower_bound}, then  $\regret_\PB(\instance^{(t)},\infty)\le (\sojourn+1)(1-2\eps)/4$. 
%
Hence,
\[ \OPT(\instance,\sojourn) - \gre(\instance,\sojourn) \ge b\frac{(\sojourn+1)(1-2\eps)}{4} = \frac{\Njob}{4}(1-2\eps), \]
%
completing the proof.
\Halmos

\subsection{Proof of \Cref{prop:loglowerboundOnline}}
\label{pf:loglowerboundOnline}

Since $\Njob$ is divisible by $(\sojourn+1)$, we have an exact partition of the set of jobs into $b=n/(\sojourn+1)\in\{1,2,\ldots\}$ batches. 
%
As in \Cref{prop:greedy_linear_lower_bound_dynamic}, we construct an instance $\instance\in[0,1]^\Njob$ that can be analyzed separately on each batch, where we denote $\instance^{(t)}=(\type_j)_{j\in B_t}$ for $t=1,\ldots,b$.
%
By \Cref{prop:loglowerboundOffline}, there exists an instance $\instance^{(0)}\in[0,1]^{(\sojourn+1)}$ such that $\regret_\PB(\instance^{(0)},\infty) \ge (\log(\sojourn+5)-3)/4$.
%
Then, for each $t=1,\ldots,b$, let $\instance^{(t)}=\instance^{(0)}\cdot\frac13$ if $t$ is odd and $\instance^{(t)}=\instance^{(0)}\cdot\frac13+\frac23$ if $t$ is even.
%
Note that since each batch size is $\sojourn+1$ divisible by 4, it is possible to match within batch only. Moreover, by construction, every critical job has a matching candidate within batch, which is closer in space than any job on the next batch. Thus, because of \Cref{deliverypotential}, $\PB$ only matches within batch, and therefore $\PB(\instance,\sojourn) = \sum_{t=1}^b \PB(\instance^{(t)},\infty)$.
%
As in \Cref{prop:greedy_linear_lower_bound_dynamic}, we arrive at
\[ \OPT(\instance,\sojourn) - \PB(\instance,\sojourn) \ge \sum_{t=1}^b \regret_\PB(\instance^{(t)},\infty) = b\frac{ \regret_\PB(\instance^{(0)},\infty)}{3} \ge \frac{\Njob}{3(\sojourn+1)}(\log(\sojourn+5)-3)/4, \]
%
concluding the proof.
\Halmos

\section{Interpretation of Potential}\label{sec: interpretation}

When a job $j$ becomes critical, our potential-based greedy algorithm matches it to an available job $k$ maximizing $r(\theta_j,\theta_k)-p(\theta_k)$, where $p(\theta_k)$ can be interpreted as the opportunity cost of matching job $k$, with the specific definition $p(\theta_k)=\frac12 \sup_{\theta\in\Theta} r(\theta_k,\theta)$.
This is an optimistic measure of opportunity cost because it assumes that job $k$ would otherwise be matched to an "ideal" type $\theta\in\Theta$ maximizing $r(\theta_k,\theta)$ (with half of this ideal reward $\sup_{\theta\in\Theta} r(\theta_k,\theta)$ attributed to job $k$).
We now prove that this ideal reward can indeed be achieved under asymptotically-large market thickness, for a stochastic model under our topology of interest.

\begin{definition}
Let $\instance \in \typespace^\Njob$.
For any job $j\in[\Njob]$, let $\instance^{-j}\in \typespace^{\Njob-1}$ be the same instance with the exception that job $j$ is not present. Meanwhile, let $\instance^{+j}\in \typespace^{\Njob+1}$ be the same instance with the exception that an additional copy of job $j$ is present.
Consider the following definitions.
\begin{enumerate}
\item Marginal Loss: $\marginalloss_j(\instance) = \OPT(\instance) - \OPT(\instance^{-j})$
\item Marginal Gain: $\marginalgain_j(\instance) = \OPT(\instance^{+j}) - \OPT(\instance)$
\end{enumerate}
\end{definition}

Note that definitions $\marginalloss_j(\instance),\marginalgain_j(\instance)$ are based solely on offline matching, and we will use them as our definitions of opportunity cost if the future was known.  One could alternatively use shadow prices from the LP relaxation of the offline matching problem, but we note that the LP is not integral.  In either case, there is no ideal definition of opportunity cost that is guaranteed to lead to the optimal offline solution in matching problems \citep[see][]{cohen2016invisible}.

We now establish the following \namecref{lem:marginal_as_interval} to help analyze the opportunity costs $\marginalloss_j(\instance),\marginalgain_j(\instance)$.


\begin{lemma}\label{lem:marginal_as_interval}
Let $\typespace=[0,1]$ and $\reward(\type,\type') = \min\{\type,\type'\}$. Consider an instance $\instance\in\typespace^n$ and relabel the indices to satisfy $\type_1 \ge \type_2 \ge \ldots \ge \type_n$. Then,
\begin{align*}
\marginalloss_j(\instance)
&=\sum_{k\ge j,k\ \mathrm{even}}(\theta_k-\theta_{k+1}), 
\\ \marginalgain_j(\instance)
&=\sum_{k\ge j,k\ \mathrm{odd}}(\theta_k-\theta_{k+1}),
\end{align*}
where we consider $\theta_{n+1}=0$.
\end{lemma}
%
\proof{Proof.}
    Let $\instance \in \typespace^\Njob$ such that $\type_1 \ge \type_2 \ge \ldots \ge \type_n$. 
    %
    Then, $\OPT(\instance) =\sum_{k\ \mathrm{even}}\theta_k$, and moreover
    \begin{align*}
    \OPT(\instance^{-j}) &=\sum_{k<j, k\ \mathrm{even}}\theta_k+\sum_{k>j, k\ \mathrm{odd}}\theta_k
    \\ \OPT(\instance^{+j}) &=\theta_j+\OPT(\instance^{-j})
    \end{align*}
    % 
    Therefore,
    % 
    \begin{align*}
    \marginalloss_j(\instance)
    &=\sum_{k\ge j,k\ \mathrm{even}}\theta_k
    -\sum_{k>j,k\ \mathrm{odd}}\theta_k
    =\sum_{k\ge j,k\ \mathrm{even}}(\theta_k-\theta_{k+1})
    \\ \marginalgain_j(\instance)
    &=\theta_j-\sum_{k\ge j,k\ \mathrm{even}}(\theta_k-\theta_{k+1})
    =\sum_{k\ge j,k\ \mathrm{odd}}(\theta_k-\theta_{k+1})
    \end{align*}
    % 
    completing the proof.
\Halmos\endproof

Note that because $\OPT(\instance^{+j}) = \OPT(\instance^{-j}) + \type_j$ and $\potential(\type_j) = \reward(\type_j,\type_j)/2$ for this reward function, we immediately get the following \namecref{cor: marginal_average}.

\begin{corollary}\label{cor: marginal_average}
        If $\typespace=[0,1]$ and $\reward(\type,\type') = \min\{\type,\type'\}$, then for all jobs $j$,
    \[ \potential(\type_j) = \frac{\marginalloss_j(\instance) + \marginalgain_j(\instance)}{2}. \]
\end{corollary}

We are now ready to prove our main result about the interpretation of potential, that the true opportunity costs $\marginalloss_j(\instance),\marginalgain_j(\instance)$ concentrate around $p(\theta_j)$ in a random uniform instance, assuming the market is sufficiently thick.  We without loss consider job $j=1$ and fix its type $\theta_1$.

\begin{theorem} \label{thm:interpretation}
Let $\typespace=[0,1]$ and $\reward(\type,\type') = \min\{\type,\type'\}$.
Fix $\theta_1\in\typespace$ and suppose $\theta_2,\ldots,\theta_n$ are drawn IID from the uniform distribution over $[0,1]$, forming a random instance $\instance\in\typespace^n$, for some $n\ge 2$. Then, both $\mathbb{E}[\marginalloss_1(\instance)]$ and $\mathbb{E}[\marginalgain_1(\instance)]$ are within $O(1/n)$ of $\potential(\type_1)$ and moreover $\mathrm{Var}(\marginalloss_1(\instance))=\mathrm{Var}(\marginalgain_1(\instance)) = O\left(\frac{1}{n}\right)$.
\end{theorem}
%
\proof{Proof.}
We prove the statement only for $\marginalloss_1(\instance)$, and the analogous result for $\marginalgain_1(\instance)$ follows from \Cref{cor: marginal_average}.
%
If $\type_1=0$, then $\marginalloss_1(\instance)=0$ for any $\instance$, coinciding with $\potential(\type_1)=0$. Then, for the remainder of the proof, assume that $\type_1>0$.
%
Consider the random variable $N=|\{j:\theta_j<\theta_1\}|$, which counts the number of points between $0$ and $\theta_1$ and has distribution $\text{Binom}(\Njob-1,\type_1)$.
%
These $N$ points divide the interval $[0,\theta_1]$ into $N+1$ intervals with total length $\type_1$.
%
From \Cref{lem:marginal_as_interval}, the marginal loss $\marginalloss_1(\instance)$ is determined by computing the total length of a subset of these intervals. % Maybe add figure?
%
It is known that if $N$ random variables are drawn independently from $\text{Unif}[0,1]$, then the joint distribution of the induced interval lengths is $\text{Dirichlet}(1,1,\ldots,1)$ with $N+1$ parameters all equal to 1 (i.e., drawn uniformly from the simplex).
%
In particular, since the Dirichlet distribution is symmetric, the sum of any $k$ of these intervals is equal in distribution to the $k$-th smallest sample ($k$-th order statistic) of the $N$ uniform random variables, whose distribution is known to be $\text{Beta}(k,N+1-k)$.
%
Thus, conditional on $N$, with $N\ge 1$, since the distribution of each of the $N$ jobs to the left of $\type_1$ is $\text{Unif}[0,\type_1]$, the distribution of $\marginalloss_1(\instance)/\type_1$ is $\text{Beta}(k_L,N+1-k_L)$,
where $k_L$ is either $\lceil\frac{N+1}2\rceil$ or $\floor{\frac{N+1}2}$.
%
To be precise, for $N\ge 1$, $k_L=\floor {N/2}+1=\lceil\frac{N+1}2\rceil$ if $n$ is even, and $k_L=\ceil{N/2}=\floor{\frac{N+1}2}$ if $n$ is odd.
%
Moreover, if $N=0$, then $\marginalloss_1(\instance)=\type_1$ if $n$ is even, and $\marginalgain_1(\instance)=0$ if $n$ is odd.
%
Hence,
\begin{align*}
\mathbb{E}[\marginalloss_1(\type) \mid N] 
&= \type_1\frac{k_L}{N+1}
% \label{eq: cond_exp}
\end{align*}
for all $N\in\{0,\ldots,n-1\}$.
%
Since $|k_L-(N+1)/2|\le 1$, then $|\mathbb{E}[\marginalloss_1(\type) \mid N]-\type_1/2|\le \type_1/(N+1)$, thus
% 
\begin{align*}
&\left|\mathbb{E}[\marginalloss_1(\type)]-\frac{\type_1}{2}\right| 
\le \type_1\mathbb{E}\left[\frac{1}{N+1}\right] \\
&\qquad = \type_1\sum_{k=0}^{n-1} \frac{1}{k+1}\binom{n-1}{k}\type_1^k(1-\type_1)^{n-1-k} \\
&\qquad = \frac{\type_1}{n}\sum_{k=0}^{n-1} \binom{n}{k+1}\type_1^k(1-\type_1)^{n-1-k} \\
&\qquad = \frac{1}{n}\sum_{k=1}^{n} \binom{n}{k}\type_1^k(1-\type_1)^{n-k} \\
&\qquad = \frac{1-(1-\type_1)^{n}}{n} = O\left(\frac{1}{n}\right).
\end{align*}
%
This completes the proof of the statement about $\mathbb{E}[\marginalloss_1(\instance)]$.

For the statement about variance, we know $\mathrm{Var}(\marginalloss_1(\type)) = \mathbb{E}[\mathrm{Var}(\marginalloss_1(\type) \mid N)] + \mathrm{Var}(\mathbb{E}[\marginalloss_1(\type)\mid N])$ by the law of total variance. For the first term, we have $\mathrm{Var}(\marginalloss_1(\type) \mid N) = \type_1^2\frac{k_L(N+1-k_L)}{(N+1)^2(N+2)}\indicator\{N\ge 1\}\le \frac{\type_1^2}{N+1}$, and then from the previous argument $\mathbb{E}[\mathrm{Var}(\marginalloss_1(\type) \mid N)]=O(1/n)$.
%
For the second term, 
\begin{align*}
\mathrm{Var}(\mathbb{E}[\marginalloss_1(\type)\mid N]) 
&= \mathrm{Var}\left(\mathbb{E}[\marginalloss_1(\type)\mid N] - \frac{\type_1}{2}\right) \\
&\le \mathbb{E} \left[\left(\mathbb{E}[\marginalloss_1(\type)\mid N] - \frac{\type_1}{2}\right)^2\right] \\
% 
&\le \mathbb{E}\left[\frac{\type_1^2}{(N+1)^2}\right] \\
&\le \type_1^2\mathbb{E}\left[\frac{1}{N+1}\right] = O\left(\frac{1}{n}\right).
\end{align*}
Thus, $\mathrm{Var}(\marginalloss_1(\instance)) = O\left(1/n\right)$, completing the proof.
\Halmos\endproof
%%%%%%%%

\section{Algorithmic Performance under Alternative Reward Topologies} \label{appx:alternative_reward_topologies}


In this section, we study the performance of both the naive greedy algorithm $\gre$ and potential-based greedy algorithm $\PB$, under the two alternative reward topologies defined in \eqref{eq:defn_reward_B} and \eqref{eq:defn_reward_C}.

\subsection{Reward Function $\reward(\type,\type')=1-|\type-\type'|$}
\phantomsection
\label{sec:reward2}
%
This reward function represents the goal to minimize the total distance between matched jobs.
%
In this case, for any $\type\in[0,1]$, the potential of a job of type $\type$ is $\potential(\type)=1/2$ constant across job types. Thus, the description of $\PB$ is equivalent to $\gre$, since their index functions differ only in an additive constant.

\begin{remark}\label{rem:rewardB}
    Under the 1-dimensional type space $\typespace=[0,1]$ and the reward function $\reward(\type,\type')=1-|\type-\type'|$, both the naive greedy and potential-based greedy algorithm generate the same output $(C,\matchof(\cdot))$ as described in \Cref{alg:dynamic}. In particular, both algorithms always match each critical job to an available job that is the closest in space.
\end{remark}

Moreover, we show that their performance can be reduced to the one of $\PB$ under reward function $\reward(\type,\type')=\min\{\type,\type'\}$.
%
In particular, we first show that when $\sojourn=\infty$, the regret is at least logarithmic in the number of jobs.
\begin{proposition}\label{prop:loglowerboundOffline_rewardB}     
    Under reward function $\reward(\type,\type')=1-|\type-\type'|$, if $\Njob=2^{k+3}-4$ for some integer $k\ge0$, then there exists an instance $\instance\in[0,1]^\Njob$ for which $\regret_\PB(\instance,\infty) = \regret_\gre(\instance,\infty) \ge (\log_2(\Njob+4)-3)/2 $.
\end{proposition}

\proof{Proof.}
    %
    We prove the statement for $\gre$, since $\regret_\PB(\instance,\infty) = \regret_\gre(\instance,\infty)$ follows from \Cref{rem:rewardB}.
    %
    Let $\Njob=2^{k+3}-4$, for $k\in\{0,1,\ldots\}$. From \Cref{prop:loglowerboundOffline}, there exists an instance $\instance\in[0,1]^\Njob$ such that the regret of $\PB$ under reward topology $\reward(\type,\type')=\min\{\type,\type'\}$, which we denote by $R$ from here on, is at least $(\log_2(\Njob+4)-3)/4$.
    %
    We show that under $\reward(\type,\type')=1-|\type-\type'|$, we have $\regret_\gre(\instance,\infty) \ge 2R$.
    %
    First, note that since $|\type-\type'| = \type + \type' - 2\min\{\type,\type'\}$, we have
    %
    \[ \gre(\instance,\infty) 
         = \sum_{j \in C} 1 - |\type_j-\type_{m(j)}|  
         = |C| - \sum_{j\in C} (\type_j +\type_{\matchof(j)})  + 2\sum_{j\in C} \min\{\type,\type'\}. 
    \]
    % 
    Similarly, recall that $\matchset_\OPT$ is the set of matches in the hindsight optimal solution. Then,
    \[ \OPT(\instance,\infty) = |\matchset_\OPT| - \sum_{ (j,k)\in\matchset_\OPT} (\type_j +\type_{k})  + 2\sum_{(j,k)\in\matchset_\OPT} \min\{\type_j,\type_{k}\}. \]
    %
    Since $\reward(\type,\type')\ge 0$ for any $\type,\type'\in \typespace$ and the total number of jobs $\Njob$ is even, both $\gre$ and $\OPT$ match every job. Hence, $|C|=|\matchset_\OPT|=n/2$, and moreover
    \[ \sum_{(j,k)\in\matchset_\OPT} (\type_j +\type_{k}) = \sum_{j\in C} (\type_j +\type_{\matchof(j)}). \]
    Thus,
    \[ \OPT(\instance,\infty) - \gre(\instance,\infty) \ge 2\left( \sum_{(j,k)\in\matchset_\OPT} \min\{\type_j,\type_{k}\} - \sum_{j\in C} \min\{\type_j,\type_{\matchof(j)}\} \right). \]
    %
    We claim that the term in large parenthesis is exactly the regret of $\PB$ under reward topology $\reward(\type,\type')=\min\{\type,\type'\}$, which we denote by $R$.
    %
    Indeed, from \Cref{rem:rewardB}, $(C,\matchof(\cdot))$ coincides with the resulting matching of $\PB$ under $\reward(\type,\type')=\min\{\type,\type'\}$, since it is specified to always match to the closest job (\Cref{deliverypotential}).
    %
    On the other hand, since all jobs are matched, the set of matches $\matchset_\OPT$ must induce disjoint intervals, which coincides with the optimal matching solution under reward function $\reward(\type,\type') = \min\{\type,\type'\}$.
    %
    Thus, by \Cref{prop:loglowerboundOffline}, we get $ \regret_\gre(\instance,\infty) \ge 2R \ge (\log_2(\Njob+4)-3)/2$, completing the proof.
\Halmos\endproof

We now extend this result to the online setting ($\sojourn<n$).

\begin{proposition} \label{prop:loglowerboundOnline_rewardB}
    Under reward topology $\reward(\type,\type') = 1-|\type-\type'|$, if $\sojourn+1 = 2^{k+3}-4$ for some integer $k\ge 0$, then for any number of jobs $\Njob$ divisible by $(\sojourn+1)$, there exists an instance $\instance \in [0,1]^\Njob$ for which $\regret_\PB(\instance,\sojourn) = \regret_\gre(\instance,\sojourn) \ge \frac{\Njob}{3(\sojourn+1)}(\log_2(\sojourn+5)-3)/2$.
\end{proposition}

\proof{Proof.}
    % 
    This proof is analogous to the proof of \Cref{prop:loglowerboundOnline}, since (i) there exists an offline instance that achieves the desired regret for $\sojourn+1=n$ (\Cref{prop:loglowerboundOffline_rewardB}), and (ii) the algorithm matches a critical job with the closest available job (\Cref{rem:rewardB}).
    %
    We repeat the proof below for completeness.

    We have an exact partition of the set of jobs into $b=n/(\sojourn+1)\in\{1,2,\ldots\}$ batches, and construct an instance $\instance\in[0,1]^\Njob$ that can be analyzed separately on each batch, where we denote $\instance^{(t)}=(\type_j)_{j\in B_t}$ for $t=1,\ldots,b$.
    %
    By \Cref{prop:loglowerboundOffline_rewardB}, there exists an instance $\instance^{(0)}\in[0,1]^{(\sojourn+1)}$ such that $\regret_\PB(\instance^{(0)},\infty) \ge (\log(\sojourn+5)-3)/2$.
    %
    Then, for each $t=1,\ldots,b$, let $\instance^{(t)}=\instance^{(0)}\cdot\frac13$ if $t$ is odd and $\instance^{(t)}=\instance^{(0)}\cdot\frac13+\frac23$ if $t$ is even.
    %
    Note that since each batch size is $\sojourn+1$ divisible by 4, it is possible to match within batch only. Moreover, by construction, every critical job has a matching candidate within batch, which is closer in space than any job on the next batch. Thus, because of \Cref{rem:rewardB}, $\PB$ only matches within batch, and therefore $\PB(\instance,\sojourn) = \sum_{t=1}^b \PB(\instance^{(t)},\infty)$.
    %
    As in \Cref{prop:greedy_linear_lower_bound_dynamic}, we arrive at
    \[ \OPT(\instance,\sojourn) - \PB(\instance,\sojourn) \ge \sum_{t=1}^b \regret_\PB(\instance^{(t)},\infty) = b\frac{ \regret_\PB(\instance^{(0)},\infty)}{3} \ge \frac{\Njob}{3(\sojourn+1)}(\log(\sojourn+5)-3)/2, \]
    %
    concluding the proof.
\Halmos\endproof

Finally, we directly show the tight regret upper bound (up to constants) as an immediate consequence of the proof of \Cref{thm: dynamic}. Note that if $d+1=n$, then this yields the result for offline matching.

\begin{theorem}\label{thm:rewardB_dynamic}
Under reward topology $\reward(\type,\type') = 1-|\type-\type'|$, we have $\regret_\PB(\instance,\sojourn) = \regret_\gre(\instance,\sojourn) \le 1/2 + (\frac{\Njob}{d+1}+1)(1+\log(\sojourn+2))$ for any $\Njob$, and any instance $\instance\in[0,1]^\Njob$.
\end{theorem}


\proof{Proof.}
Let $\instance\in \typespace^\Njob$ and $\sojourn\ge 1$. Let $(C,\matchof(\cdot))$ be the output of $\gre$ on $\instance$. We have
\begin{align*}
\OPT(\instance,\sojourn)-\gre(\instance,\sojourn) 
& \le \sum_{j=1}^\Njob \potential(\type_j) - \gre(\instance,\sojourn)  \nonumber \\
& \le \sup_{\type\in\typespace}\potential(\type) + \sum_{j \in C} \left( p(\theta_j) + p(\theta_{m(j)}) - r(\theta_j,\theta_{m(j)}) \right) \nonumber \\ 
& = \frac12 +  \sum_{j \in C} \left( \frac12 + \frac12 - (1-|\theta_j-\theta_{m(j)}|) \right) \nonumber \\
& = \frac12 + \sum_{j \in C} |\type_j-\type_{\matchof(j)}|, 
\end{align*}
The second term is exactly the sum of distances between jobs matched by $\gre$. Recall that $(C,\matchof(\cdot))$ coincides with the matching output of $\PB$ under $\reward(\type,\type')=\min\{\type,\type'\}$, both are specified to always match with the closest job (\Cref{deliverypotential}, \Cref{rem:rewardB}). Thus, recalling \eqref{eq: distance_dynamic} in the proof of \Cref{thm: dynamic}, we have
\[ \sum_{j \in C} |\type_j-\type_{\matchof(j)}| = \sum_{t=1}^b \sum_{j \in C\cap B_t} |\type_j-\type_{\matchof(j)}|\le \left(\frac{n}{d+1}+1 \right)\left(1+\log(d+2)\right),\]
and the result follows.
\Halmos\endproof

\subsection{Reward Function $\reward(\type,\type')=|\type-\type'|$}\label{sec: reward 3}
%
Under this reward topology, it is worst to match two jobs of the same type, contrasting the two settings studied in \Cref{sec:PB} and \Cref{sec:reward2}.
%
In this case, the potential of a job type $\type\in[0,1]$ is $\potential(\type)=\max\{\type,1-\type\}/2$.
%
We first show that \emph{any} index-based matching policy must suffer regret constant regret per job in the offline setting.
%
\begin{proposition}\label{prop:linearLB_offline_rewardC}
    Under reward topology $\reward(\type,\type')=|\type-\type'|$, when the number of jobs $\Njob$ is divisible by 4, for any index-based greedy matching algorithm $\ALG$, there exists an instance $\instance\in[0,1]^\Njob$ for which $\regret_\ALG(\instance,\infty)\ge c_\ALG \Njob$, for some $c_\ALG\in (0,1)$. In particular, $c_\gre = c_\PB = 1/4$.
\end{proposition}

\proof{Proof.}
Let $\Njob$ be divisible by 4.
%
Let $\ALG$ be an index-based greedy matching algorithm specified by index function $\indexf:[0,1]^2\to \R$.
%
Define $\type^{c} = \sup\{ \type\in[0,1] : \indexf(\type,1)\ge \indexf(\type,0) \}$.
%
We first construct an instance in the case $\type^{c}>0$ (e.g. $\type^{c}=1/2$ for $\gre$ and $\PB$), and analyze the alternative case separately in the end.
%
Consider an instance $\instance\in\typespace^\Njob$ such that $\type_1,\ldots,\type_{\Njob/4} = \type^{c}$, $\type_{{\Njob/4}+1},\ldots,\type_{{3\Njob/4}}=0$, and $\type_{{3\Njob/4}+1},\ldots,\type_{\Njob}=1$. 
%
It is straightforward to check that $\OPT(\instance,\infty) \ge (1+\type^c){\Njob/4}$.
%
On the other hand, $\ALG$ chooses to match every $j=1,\ldots,n/4$ to $\matchof(j)\in \{3n/4+1,\ldots,n\}$.\footnote{In the presence of ties, such an output is achieved by some tie-breaking rule.}
%
Then,
\[ \ALG(\instance,\infty) = 0 + \sum_{j=1}^{\Njob/4} \reward(\type_j,\type_{3\Njob/4+j}) = \frac{\Njob(1-\type^c)}{4}. \]
%
and thus $\regret_\ALG(\instance,\infty) = \OPT(\instance,\infty) - \ALG(\instance,\infty) \ge {{\Njob\type^{c}}}/{2},$ proving the statement with $c_\ALG = \type^{c}/2$.

%
Now, if $\type^{c}\le 0$, then $\indexf(\type,0) < \indexf(\type,1)$ for all $\type\in(0,1]$.
%
In particular, for any $\eps>0$, consider the instance $\type_1,\ldots,\type_{\Njob/4} = \eps$, $\type_{{\Njob/4}+1},\ldots,\type_{{3\Njob/4}}=1$, and $\type_{{3\Njob/4}+1},\ldots,\type_{\Njob}=0$.
%
Then, $\OPT(\instance,\infty) \ge (1+(1-\eps))n/4$, and $\ALG(\instance,\infty) = \eps n/4$.
%
Hence, $\regret_\ALG(\instance,\infty)\ge (1-\eps)/2$, and the statement is true with $c_\ALG = (1-\eps)/2$ for every $\eps>0$.
\Halmos\endproof

The following result shows that, under $\PB$ and $\gre$, this lower bound extends to the online setting, losing a factor of 2.

\begin{proposition}\label{prop:linearLB_dynamic_rewardC}
    Under reward topology $\reward(\type,\type')=|\type-\type'|$, if $(\sojourn+1)$ is divisible by 4, then for any number of jobs $\Njob$ divisible by $2(\sojourn+1)$, there exists an instance $\instance \in [0,1]^\Njob$ for which $\regret_\gre(\instance,\sojourn)=\regret_\PB(\instance,\sojourn) \ge n/8$.
\end{proposition}

\proof{Proof.}
If $\Njob$ is divisible by $2(\sojourn+1)$, we have an exact partition of the set of jobs into $b=\Njob/(\sojourn+1)\in\{2,4,\ldots\}$ batches, defined as $B_{t+1} = \{t(\sojourn+1)+1,\ldots, (t+1)(\sojourn+1)\}$ for $t = 0, \ldots, b-1$.
%
We construct an instance $\instance\in[0,1]^\Njob$ for which $\gre$ (and $\PB$) "resets" every two batches, in the sense that all jobs in these two batches are matched among them.
%
To formally specify the construction, first fix $\eps\in(0,1/2)$. Define, for $t=0,\ldots,b/2-1$
%
\[
\type_{2t(\sojourn+1)+j} =
\begin{cases}
\frac12 & \text{if } j \le \frac{\sojourn+1}{4}, \\
\eps & \text{if } \frac{\sojourn+1}{4} < j \le \frac{3(\sojourn+1)}{4}, \\
1 & \text{if } \frac{3(\sojourn+1)}{4} < j \le \sojourn+1 \text{, or } \frac{3(\sojourn+1)}{2} < j \le 2(\sojourn+1), \\
0 & \text{if } \sojourn+1 < j \le \frac{3(\sojourn+1)}{2}.
\end{cases}
\]
%
Consider the notation $\instance^{(0)}=(\type_j)_{j\le 2(\sojourn+1)}$.
%
It is easy to check that
\[ 
\OPT(\instance^{(0)},\infty) = \left(\frac12-\eps\right)\frac{\sojourn+1}{4} + (1-\eps)\frac{\sojourn+1}{4} + \frac{\sojourn+1}{2} =  \left(\frac{7-4\eps}{8}\right)(\sojourn+1),
\]
and then $\OPT(\instance,\sojourn) \ge \frac{b}{2}\OPT(\instance^{(0)},\infty) = \frac{7-4\eps}{16}\Njob $.
%
On the other hand,
\[ 
\gre(\instance^{(0)},\infty) = \frac12\frac{\sojourn+1}{4} + \eps\frac{\sojourn+1}{4} + (1-\eps)\frac{\sojourn+1}{4} + \frac{\sojourn+1}{4} =  \frac58(\sojourn+1).
\]
%
We claim that $\gre(\instance,\sojourn) = \frac{b}{2}\gre(\instance^{(0)},\infty) = \frac{5}{16}\Njob $, and thus $\regret_\gre(\instance,\sojourn)\ge \frac{1-2\eps}{8}\Njob $.
%
Indeed, let $(C,\matchof(\cdot))$ be the output of $\gre$. Running $\gre$ on $\instance$ is as follows:
%
\begin{enumerate}
    \item For every job $j \le (\sojourn+1)/4$ of type $\type_j=1/2$, and then $3(\sojourn+1)/4 < \matchof(j) \le (\sojourn+1)$ with $\reward(\type_j,\type_{\matchof(j)})=1/2$.
    \item Then, for every job $\frac{\sojourn+1}{4} < j \le \frac{(\sojourn+1)}{2}$, the only available jobs are of type $\type\in\{0,\eps\}$, and then $ \sojourn+1 < \matchof(j) \le \frac{3(\sojourn+1)}{2}$ with $\reward(\type_j,\type_{\matchof(j)})=\eps$. 
    \item In turn, jobs $\frac{(\sojourn+1)}{2} < j \le \frac{3(\sojourn+1)}{4}$ get to observe jobs of type $\type=1$, and then $ \frac{3(\sojourn+1)}{2} < \matchof(j) \le 2(\sojourn+1)$ with $\reward(\type_j,\type_{\matchof(j)})=1-\eps$.
    \item Finally, the key observation is that the all remaining available jobs $j\le \frac{3(\sojourn+1)}{2}$ of type $\type_j=0$ observe jobs of type $1$, as well as jobs on a "third" batch. However, by construction every job $2(\sojourn+1) < k \le 2(\sojourn+1) + \frac{3(\sojourn+1)}{2} $ are of type $\type_k\in\{1/2,\eps\}$, and then each job $j\le \frac{3(\sojourn+1)}{2}$ is matched (within batch) to $ \frac{3(\sojourn+1)}{2} < \matchof(j) \le 2(\sojourn+1)$ with $\reward(\type_j,\type_{\matchof(j)})=1$.
\end{enumerate}

The proof for $\PB$ is analogous, since the matching decisions coincide with $\gre$ on this instance. To see this, note that
\begin{itemize}
    \item $\indexf_\PB(1/2,1) = \reward(1/2,1) - \potential(1) = 1/2 - 1/2 = 0$,
    \item $\indexf_\PB(1/2,\eps)=\reward(1/2,\eps) - \potential(\eps) = 1/2 - \eps - (1-\eps)/{2} = - \eps/2 < \indexf_\PB(1/2,1)$,
    \item $\indexf_\PB(\eps,0)=\reward(\eps,0) - \potential(0) = \eps - 1/2$,
    \item $\indexf_\PB(\eps,\eps)=\reward(\eps,\eps) - \potential(\eps) = 0 - (1-\eps)/{2}= \eps/{2}-1/2< \indexf_\PB(\eps,0)$,
    \item $\indexf_\PB(\eps,1)=\reward(\eps,1) - \potential(1) = 1 - \eps - 1/2 = 1/2 - \eps$,
    \item $\indexf_\PB(\eps,1/2)=\reward(\eps,1/2) - \potential(1/2) = 1/2 - \eps - 1/4 = 1/4 - \eps < \indexf_\PB(\eps,1) $, and
    \item $\indexf_\PB(0,\eps)= \eps - (1-\eps)/2 < \indexf_\PB(0,1/2) < \indexf_\PB(0,1)$.
\end{itemize}
Thus, running $\PB$ on $\instance$ follows the same four steps described by $\gre$, yielding the same regret.
\Halmos\endproof


\section{Linear Relaxation and Dual Variables}\label{sec: LP}

We present in this section the linear relaxation of the IP that describes the hindsight optimum defined in \eqref{eq: OPT}. Given market density $\sojourn\ge 1$ and full information of the sequence of arrivals $\instance\in\typespace$, consider the following linear program (LP)
%
\begin{maxi}
{x}{ \sum_{j,k : j\neq k, |j-k|\le \sojourn} x_{jk} \reward(\type_j,\type_k)}
{\label{eq: LP}}{}
\addConstraint{ \sum_{k:j\neq k} x_{jk}}{\le 1,}{j\in [\Njob]}
\addConstraint{ x_{jk}}{\ge 0,\quad }{ j,k\in [\Njob], j\neq k,}
\end{maxi}
%
and its dual
%
\begin{mini}
{\lambda}{ \sum_{j\in[\Njob]} \lambda_{j} }
{\label{eq: Dual-LP}}{}
\addConstraint{ \lambda_j + \lambda_k}{ \ge \reward(\type_j,\type_k),\quad}{j,k\in [\Njob], j\neq k, |j-k|\le \sojourn}
\addConstraint{ \lambda_{j} }{\ge 0 ,}{ j\in [\Njob].}
\end{mini}

It is known that the LP relaxation is integral only for bipartite graphs. However, we observe in \Cref{fig: LP_gap} that the integrality gap is small, and thus the value of the LP can be reasonably used as a benchmark.

\begin{figure}[H]
  \centering
  \subcaptionbox{Average Total Reward.%
    \label{fig:reward}}[0.49 \textwidth]{\includegraphics[width = \figWidth \textwidth]{Simulation_Results/Synthetic/New/1D_Pooling/1D_Common_Origin/Forecast-Agnostic/value.png}}
  \hfill
  \subcaptionbox{Fraction of LP value achieved by $\OPT$.%
    \label{fig:LP_Gap}}[0.49 \textwidth]{\includegraphics[width = \figWidth \textwidth]{Simulation_Results/Synthetic/New/1D_Pooling/1D_Common_Origin/Forecast-Agnostic/integrality_gap.png}}
  % 
  \caption{Comparison with LP value.
  }
  % 
  \label{fig: LP_gap}
\end{figure}

Also, note that the potential can always be used as a feasible (but not necessarily optimal) solution to the dual LP.


\begin{proposition}
    Let $\instance\in\typespace^\Njob$ and $\sojourn\ge 1$. Then, $\{\potential(\type_j)=\frac{1}{2}\sup_{\type'\in\typespace}\reward(\type_j,\type'):j\in[\Njob]\}$ is a feasible solution of the dual LP \eqref{eq: Dual-LP}.
\end{proposition}


\proof{Proof.}
    Non-negativity holds since $\reward\ge 0$. Moreover, for any $j,k$ we have
    \[ \reward(\type,\type')\le \frac{1}{2}\reward(\type,\type')+\frac{1}{2}\reward(\type,\type') \le \potential(\type)+\potential(\type'), \]
    verifying the dual feasibility constraints.
\Halmos\endproof

Interestingly, when we extract the shadow prices from the 400 instances considered as historical data in \Cref{sec:sim_unif_1D}, we can see that these concentrate around potential, as density increases (\Cref{fig: dual_converge}). This is aligned with our results in \Cref{sec: interpretation}.

\begin{figure}[H]
  \centering
  \subcaptionbox{Density $\sojourn=5$.%
    \label{fig:dual_conv_d5}}[0.4 \textwidth]{\includegraphics[width = \figWidth \textwidth]{Simulation_Results/Synthetic/New/1D_Pooling/1D_Common_Origin/Forecast-Aware/duals_at_d5.png}}
  % \hfill
  \hspace{1em}
  % 
  \subcaptionbox{Density $\sojourn=30$.%
    \label{fig:dual_conv_d30}}[0.4 \textwidth]{\includegraphics[width = \figWidth \textwidth]{Simulation_Results/Synthetic/New/1D_Pooling/1D_Common_Origin/Forecast-Aware/duals_at_d30.png}}
  % 
  \caption{Scatter plot of shadow prices, average, and potential.}
  % 
  \label{fig: dual_converge}
\end{figure}

\section{Meituan Data - Market Dynamics} \label{sec:market_dynamics}

In this appendix, we briefly describe the Meituan dataset made public by the INFORMS TSL Data-Driven Challenge, and provide high-level descriptions of the market-dynamics. 


\subsection{Temporal Dynamics} \label{appx:meituan_temporal_dynamics}

First, \Cref{fig:meituan_orders_per_hour} provides the average number of orders per hour by \emph{hour-of-week}.
Hour-of-week 0 corresponds to midnight to 1am on Mondays, and hour-of-week 1 corresponds to 1am to 2am on Mondays, and so on. 
%
The gray bars indicate the peak lunch hours (10:30am-1:30pm), while the green bars indicate peak dinner hours (5pm-8pm).
%
We can see that the platform observes highest order volume during lunch time, averaging around 200 order per minute between noon and 1pm.   

\newcommand{\FigWidthHOW}{0.85}

\begin{figure}
    \centering
    \includegraphics[width=\FigWidthHOW\linewidth]{Simulation_Results/Meituan/City/meituan_number_of_orders_how.png}
    \caption{Average number of orders by hour-of-week in Meituan data.}
    \label{fig:meituan_orders_per_hour}
\end{figure}



\Cref{fig:meituan_pooled_orders_per_hour} in \Cref{sec:intro} of the paper illustrates the fraction of orders placed during each hour-of-week that is pooled with at least one other order. 
%
This is computed by combining the order-level data (which contains the timestamp of each order and the order ID) and the ``wave level'' data, where each wave corresponds to a list of orders that were pooled and assigned to the same driver.\footnote{
For more details on waves, see \url{https://github.com/meituan/Meituan-INFORMS-TSL-Research-Challenge/blob/main/tsl_meituan_2024_data_report_20241015.pdf}, accessed January 30, 2024. Each wave may contain more than two orders, though some orders may have been dropped off by the driver before some other orders were picked up.   
}
The orders that are not pooled with any other order correspond to waves with only one order, and \Cref{fig:meituan_pooled_orders_per_hour} illustrates 1 minus this fraction of orders that are not pooled. 


\Cref{fig:meituan_grab_to_fetch_how} illustrates the average amount of time it takes the drivers to pick up the orders, after they received and then accepted the order from the platform. Specifically, this is the amount of time that elapses between (i) the grab\_time, i.e. when the order is accepted by a driver, and (ii) the fetch\_time, i.e. when the order is picked-up by the driver.
%
The average pick-up time across all orders is 10.1 minutes. 
%
We see from \Cref{fig:meituan_grab_to_fetch_how} that the the pick-up time is generally longer during lunch and dinner peaks, reaching a maximum of 12.0 minutes at noon on Tuesdays, but overall we do not see very substantial increases during peak hours (which may result from, for example, severe supply constraints in the system). 


\begin{figure}
    \centering
    \includegraphics[width=\FigWidthHOW\linewidth]{Simulation_Results/Meituan/City/meituan_grab_to_fetch_how.png}
    \caption{
    The average amount of time it takes the driver to pick-up the orders after accepting the dispatch from the platform, by hour-of-week. 
    }
    \label{fig:meituan_grab_to_fetch_how}
\end{figure}


Finally, \Cref{fig:meituan_order_to_first_dispatch_how} illustrates the amount of time that elapses, between platform\_order\_time, i.e. when each order is placed by the customer, and first dispatch\_time, i.e. when the order was offered to a delivery driver for the first time.
% 
%
Intuitively, this is the amount of time it takes the platform to start dispatching each order, and the average is around 4 minutes across all orders. 
% 
Despite the fact that it takes an estimated $12$ minutes for the restaurant to prepare the orders, the platform starts to dispatch the orders shortly after they are placed. This is potentially due to the fact that it takes time for drivers to get to the restaurants (see \Cref{fig:meituan_grab_to_fetch_how}), so orders are dispatched before they are ready in order to reduce the total wait time experienced by the customers.  


\begin{figure}
    \centering
    \includegraphics[width=\FigWidthHOW\linewidth]{Simulation_Results/Meituan/City/meituan_order_to_first_dispatch_how.png}
    \caption{
    The average amount of time (minutes) that elapses between (i) when an order is placed by the customer, and (ii) when the order is dispatched to a delivery driver for the first time.
    }
    \label{fig:meituan_order_to_first_dispatch_how}
\end{figure}



\subsection{Spatial Distribution} \label{appx:meituan_spatial_distribution}

We now visualize the distribution of orders in space, and provide the distribution of order volume by origin-destination hexagon pairs. We use resolution-9 hexagons from the h3 package, which is the resolution that performs the best for $\averagedual$ when the matching window is at least $2$ minutes. The average size of each hexagon is  $0.1 \mathrm{km}^2$, and the average edge length is around $0.2 \mathrm{km}$.\footnote{
\url{https://h3geo.org/docs/core-library/restable/}, accessed January 31, 2025. 
} 


First, \Cref{fig:meituan_heatmaps} provides heat-maps of trip volume by order origin and destination, respectively. We can see that the order origins are more concentrated than destinations, which is aligned with the more concentrated locations of restaurants in comparison to residential areas and office buildings. 
%
Note that both distributions are highly skewed. For hexagons with at least one originating order, the median, average, and maximum number of orders are 330, 960, and 17151, respectively. For hexagons that are the destination of at least one order, the median, average, and maximum order count are 71, 327 and 5167, respectively. 


\begin{figure}[t!]
  \centering
  \subcaptionbox{By Origin.%
    \label{fig:meituan_count_by_orig}}[0.49 \textwidth]{\includegraphics[width = \figWidth \textwidth]{Simulation_Results/Meituan/City/meituan_count_by_orig.png}}
  \hfill
  \subcaptionbox{By Destination.%
    \label{fig:meituan_count_by_dest}}[0.49 \textwidth]{\includegraphics[width = \figWidth \textwidth]{Simulation_Results/Meituan/City/meituan_count_by_dest.png}}
  % 
  \caption{Number of orders by order origin hexagon (left) and order destination hexagon (right).
  }
  % 
  \label{fig:meituan_heatmaps}
\end{figure}



Finally, \Cref{fig:meituan_CDF_count_per_OD_pair} presents the distribution order volume by origin and destination hexagon pairs. From the unweighted CDF on the left, we can see that the vast majority of OD pairs have fewer than 100 orders. Out of all OD hexagon pairs with at least one order, the median, average, and maximum number of orders are 3, 8.22, and 808, respectively.
% 
The CDF on the right is weighted by order count.
%
Overall, roughly 14\% of orders are associated with OD pairs with at least 100 orders, and 72.8\% of orders are from OD pairs with at least 10 orders.
%
% 
\begin{figure}
    \centering
    \includegraphics[width=0.75\linewidth]{Simulation_Results/Meituan/City/meituan_CDF_count_per_OD_pair.png}
    \caption{
    Distribution (CDF) of the number of orders per origin-destination hexagon pair (resolution-9), unweighted (left) and weighted by the number of orders (right). 
    }
    \label{fig:meituan_CDF_count_per_OD_pair}
\end{figure}


\section{Additional Experiments}\label{sec: sim_results_extra}

We first report additional performance metrics for settings studied in \Cref{sec:numerical_experiments} of the paper.
%
Additionally, we also consider more general synthetic environments, including two-dimensional locations, non-uniform spatial distributions, and different reward topologies.

\subsection{Match Rate and Saving Fraction}
\label{sec:match_rate}  

We consider in this section two additional performance metrics: the \newterm{match rate}, i.e. the fraction of jobs that were pooled instead of dispatched on their own, and  the \newterm{saving fraction}, i.e. the fraction of the total distance that is reduced by pooling, relative to the total travel distance without any pooling~\citep[see][]{aouad2020dynamic}.

Both regret and reward ratio are performance metrics that compare pooling algorithm relative to the hindsight optimal pooling outcome. 
%
The saving fraction illustrates the benefit of delivery pooling in comparison to the total distance traveled, and is upper bounded by 0.5 (since reward from pooling a job cannot exceed its distance).
%
The match rate, as shown in \Cref{fig:1D_unif_match_rate,fig:meituan_match_rate}, highlights an additional desirable property of $\PB$ and, more generally, of \emph{index-based greedy matching} algorithms: they pool as many jobs as possible.
% 
This is desirable in practice, since drivers who are offered just only one order from a platform may try to pool orders from competing platforms \citep[see e.g.][]{reddit2022grubhub,reddit2023doordash}, leading to poor service reliability for customers.

\subsubsection{Uniform one-dimensional case.} \label{sec: 1D_extra}


\Cref{fig:1D_extra} compares the match rate and the saving fraction achieved for the one-dimensional synthetic environment presented in \Cref{sec:sim_unif_1D}.
%
Recall that all jobs share the same origin at $0$ and destinations are drawn uniformly at random from $[0,1]$. 


%%
\Cref{fig:1D_unif_saving_fraction} shows that the reduction in travel distance can be remarkably high for this setting, with $\OPT$ saving more than 46\% even at low density levels.
%
Since the saving fraction is proportional to the total reward collected by the algorithm, our previous performance comparisons presented in \Cref{sec:numerical_experiments} imply that the saving fraction achieved by $\PB$ is the best among all benchmark algorithms, which translates into over 44\% of reduction in total travel distance.
%
We can also observe that the saving fraction under $\gre$ does not decrease with density.
%
Nonetheless, $\OPT$ improves at a much faster rate, resulting in a relative performance of $\gre$ against $\OPT$ that is decreasing with density (as observed in \Cref{fig:1D}).

\begin{figure}[H]
  \centering
  \subcaptionbox{Average match rate.%
    \label{fig:1D_unif_match_rate}}[0.49 \textwidth]{\includegraphics[width = \figWidth \textwidth]{Simulation_Results/Synthetic/New/1D_Pooling/1D_Common_Origin/match_rate.png}}
  \hfill
  \subcaptionbox{Average saving fraction.%
    \label{fig:1D_unif_saving_fraction}}[0.49 \textwidth]{\includegraphics[width = \figWidth \textwidth]{Simulation_Results/Synthetic/New/1D_Pooling/1D_Common_Origin/saving_rate_all.png}}
  % 
  \caption{Additional metrics in random 1D instances.}
  % 
  \label{fig:1D_extra}
\end{figure}

The match rate in \Cref{fig:1D_unif_match_rate} illustrates that all \emph{index-based greedy matching} algorithms match every one of the $\Njob = 1000$ jobs.
%
This is because (i) the reward from matching any two jobs is non-negative in this 1D setting, and (ii) the total number of jobs is even.
%
On the other hand, $\batching$ matches every job only if $\sojourn+1$ is even, since every time it computes an optimal matching, the \textit{batch size} is $\sojourn+1$ and it dispatches \textit{all} available jobs in pooled trips. 
%
This is not the case for $\rbatching$, since by dispatching only the critical job and its match (if any), the batch size for the next time it computes an optimal matching is potentially different from $\sojourn+1$.
%


\subsubsection{Meituan data.} \label{sec: meituan_extra}
%
\Cref{fig:meituan_extra} provides a comparison of the match rate and the saving fraction for the setting presented in \Cref{sec:sim_meituan}.
%
Recall that as we extend our definition of pooling reward to allow for 2D locations with heterogeneous origins, it is no longer guaranteed to be non-negative.
%
Consequently, both the matching rate and saving fraction of all algorithms are lower than in the one-dimensional setting.


Nonetheless, we can see that if the platform allows a time window of at least $1$ minute for each order before it has to be dispatched, $\PB$ achieves a reduction in distance traveled by over 20\%. At $5$ minutes, this fraction increases to roughly 30\%, which is the best among all tested \emph{practical} heuristics, i.e. excluding $\dual$ and $\OPT$.
%

%
\Cref{fig:meituan_match_rate} shows that greedy algorithms ($\gre$, $\PB$, $\dual$, $\averagedual$) still achieve substantially higher match rates in comparison to $\batching$, $\rbatching$, and $\OPT$.
%
In particular, $\PB$ consistently pools 5-10\% more jobs than $\OPT$.

\begin{figure}%[H]
  \centering
  \subcaptionbox{Average match rate.%
    \label{fig:meituan_match_rate}}[0.49 \textwidth]{\includegraphics[width = \figWidth \textwidth]{Simulation_Results/Meituan/City/match_rate_all.png}}
  \hfill
  \subcaptionbox{Average saving fraction.%
    \label{fig:meituan_saving_fraction}}[0.49 \textwidth]{\includegraphics[width = \figWidth \textwidth]{Simulation_Results/Meituan/City/saving_rate_all.png}}
  % 
  \caption{Match rate and saving fraction for Meituan Order-Level Data.}
  % 
  \label{fig:meituan_extra}
\end{figure}
%

\subsection{Non-Uniform One-Dimensional Case}\label{sec: nonunif_1D}
%
In this section, we show that our experimental results in one-dimensional synthetic data are robust to non-uniform distribution of types. We consider an analogous setting from \Cref{sec:sim_unif_1D}, but now each job type is drawn IID from a distribution Beta(0.5, 2). Overall, the results are consistent, with the main difference being that $\PB$ performs slightly worse than $\averagedual$ for low densities. Nonetheless, $\PB$ ends up outperforming all benchmarks starting from $\sojourn\ge 15$.

\begin{figure}[H]
  \centering
  \subcaptionbox{Average regret.%
    \label{fig:1D_nonunif_regret}}[0.49 \textwidth]{\includegraphics[width = \figWidth \textwidth]{Simulation_Results/Synthetic/New/1D_Pooling/1D_Common_Origin/Non-Uniform/regret.png}}
  \hfill
  \subcaptionbox{Average ratio.%
    \label{fig:1D_nonunif_ratio}}[0.49 \textwidth]{\includegraphics[width = \figWidth \textwidth]{Simulation_Results/Synthetic/New/1D_Pooling/1D_Common_Origin/Non-Uniform/frac_of_OPT.png}}
  % 
  \caption{Comparison of average regret and reward ratio for forecast-agnostic heuristics in non-uniform 1D instances.}
  % 
  \label{fig:1D_nonunif}
\end{figure}

\begin{figure}[H]
  \centering
  \subcaptionbox{Average regret.%
    \label{fig:1D_nonunif_forecastaware_regret}}[0.49 \textwidth]{\includegraphics[width = \figWidth \textwidth]{Simulation_Results/Synthetic/New/1D_Pooling/1D_Common_Origin/Non-Uniform/regret_dual.png}}
  \hfill
  \subcaptionbox{Average ratio.%
    \label{fig:1D_nonunif_forecastaware_ratio}}[0.49 \textwidth]{\includegraphics[width = \figWidth \textwidth]{Simulation_Results/Synthetic/New/1D_Pooling/1D_Common_Origin/Non-Uniform/frac_of_OPT_dual.png}}
  % 
  \caption{Comparison of average regret and reward ratio for forecast-aware heuristics in non-uniform 1D instances.}
  % 
  \label{fig:1D_nonunif_forecastaware}
\end{figure}




\subsection{Two-Dimensional Case with Heterogeneous Origins} \label{sec:sim_unif_1D_2D}


We simulate two-dimensional environments to supplement the simulations on Meituan data.

\subsubsection{Uniform spatial distribution.}

In this section, we consider two-dimensional locations with heterogeneous origins. Specifically, each location, origin and destination, is drawn from a uniform distribution in the unit square. 
%
Overall, we observe that although $\PB$ is far from best at low densities, it still outperforms every other tested benchmark for $\sojourn\ge 25$. We verify in \Cref{sec:sim_meituan} that achieving density values for which $\PB$ outperforms every other benchmark is realistic.


\begin{figure}[H]
  \centering
  \subcaptionbox{Average regret.%
    \label{fig:2Dhet_regret}}[0.49 \textwidth]{\includegraphics[width = \figWidth \textwidth]{Simulation_Results/Synthetic/New/2D_Different_Origins/regret.png}}
  \hfill
  \subcaptionbox{Average ratio.%
    \label{fig:2Dhet_ratio}}[0.49 \textwidth]{\includegraphics[width = \figWidth \textwidth]{Simulation_Results/Synthetic/New/2D_Different_Origins/frac_of_OPT.png}}
  % 
  \caption{Comparison of average regret and reward ratio for forecast-agnostic heuristics in random 2D instances.}
  % 
  \label{fig:2Dhet}
\end{figure}
\begin{figure}[H]
  \centering
  \subcaptionbox{Average regret.%
    \label{fig:2Dhet_forecastaware_regret}}[0.49 \textwidth]{\includegraphics[width = \figWidth \textwidth]{Simulation_Results/Synthetic/New/2D_Different_Origins/regret_dual.png}}
  \hfill
  \subcaptionbox{Average ratio.%
    \label{fig:2Dhet_forecastaware_ratio}}[0.49 \textwidth]{\includegraphics[width = \figWidth \textwidth]{Simulation_Results/Synthetic/New/2D_Different_Origins/frac_of_OPT_dual.png}}
  % 
  \caption{Comparison of average regret and reward ratio for forecast-aware heuristics in random 2D instances.}
  % 
  \label{fig:2Dhet_forecastaware}
\end{figure}


\subsubsection{Non-uniform spatial distribution.}

In addition, we consider an analogous two-dimensional setting with locations drawn from non-uniform distribution Beta(0.5,2). The results remain consistent: $\PB$ outperforms all \emph{practical} heuristics, with the \emph{unrealistic} $\dual$ benchmark being the only one achieving better performance.

\begin{figure}[H]
  \centering
  \subcaptionbox{Average regret.%
    \label{fig:2Dhet_nonunif_regret}}[0.49 \textwidth]{\includegraphics[width = \figWidth \textwidth]{Simulation_Results/Synthetic/New/2D_Different_Origins/Non-Uniform/regret.png}}
  \hfill
  \subcaptionbox{Average ratio.%
    \label{fig:2Dhet_nonunif_ratio}}[0.49 \textwidth]{\includegraphics[width = \figWidth \textwidth]{Simulation_Results/Synthetic/New/2D_Different_Origins/Non-Uniform/frac_of_OPT.png}}
  % 
  \caption{Empirical performance of forecast-agnostic heuristics in random 2D instances.}
  % 
  \label{fig:2Dhet_nonunif}
\end{figure}
\begin{figure}[H]
  \centering
  \subcaptionbox{Average regret.%
    \label{fig:2Dhet_nonunif_forecastaware_regret}}[0.49 \textwidth]{\includegraphics[width = \figWidth \textwidth]{Simulation_Results/Synthetic/New/2D_Different_Origins/Non-Uniform/regret_dual.png}}
  \hfill
  \subcaptionbox{Average ratio.%
    \label{fig:2Dhet_nonunif_forecastaware_ratio}}[0.49 \textwidth]{\includegraphics[width = \figWidth \textwidth]{Simulation_Results/Synthetic/New/2D_Different_Origins/Non-Uniform/frac_of_OPT_dual.png}}
  % 
  \caption{Empirical performance of forecast-aware heuristics in random 2D instances.}
  % 
  \label{fig:2Dhet_nonunif_forecastaware}
\end{figure}


\subsection{Different Reward Topology} \label{sec:sec:sim_reward3}

In this appendix, we present numerical results under the two alternative reward topologies defined in \eqref{eq:defn_reward_B} and \eqref{eq:defn_reward_C}.
%
This illustrates the importance of reward topology in determining the performance of algorithms.
%
As in \Cref{sec:sim_unif_1D}, we generate instances of size $\Njob=1000$, where each type is drawn IID from a uniform distribution on $[0,1]$.
%
Overall, we observe that $\PB$ performs on par with the best realistic benchmarks (i.e., outside $\dual$ and $\OPT$). This suggests that $\PB$ could be a leading heuristic even beyond the reward function $r(\type,\type')=\min\{\type,\type'\}$ from delivery pooling, although here we are still restricting to continuous distributions over (one-dimensional) metric spaces.

\subsubsection{Reward function $\reward(\type,\type')=1-|\type-\type'|$.} 

Recall that in this case the potential of a job is $1/2$, independent of the job type, and thus both $\PB$ and $\gre$ are the same algorithm (see \Cref{sec:reward2}).
%
\Cref{fig:R2} shows their superior performance compared to batching-based heuristics.
%
Moreover, as in \Cref{sec:sim_unif_1D}, forecast-aware heuristics are no better than $\PB$. 

\begin{figure}[H]
  \centering
  \subcaptionbox{Average regret.%
    \label{fig:R2_regret}}[0.49 \textwidth]{\includegraphics[width = \figWidth \textwidth]{Simulation_Results/Synthetic/New/1D_Reward_2/regret_all.png}}
  \hfill
  \subcaptionbox{Average ratio.%
    \label{fig:R2_ratio}}[0.49 \textwidth]{\includegraphics[width = \figWidth \textwidth]{Simulation_Results/Synthetic/New/1D_Reward_2/frac_of_OPT_all.png}}
  % 
  \caption{Comparison of average regret and reward ratio for forecast-agnostic heuristics in random 1D instances under $\reward(\type,\type')=1-|\type-\type'|$.}
  % 
  \label{fig:R2}
\end{figure}


\subsubsection{Reward function $\reward(\type,\type')=|\type-\type'|$.}

Consider $\typespace=[0,1]$ and $\reward(\type,\type') = |\type - \type'| $ (analyzed in \Cref{sec: reward 3}).
%
\Cref{fig:R3} highlights that greedy-like algorithms perform better than batching-based heuristics, with $\PB$ outperforming every practical matching algorithm (i.e., aside from $\dual$ and $\OPT$ that use hindsight information).
% 

\begin{figure}[H]
  \centering
  \subcaptionbox{Average regret.%
    \label{fig:R3_regret}}[0.49 \textwidth]{\includegraphics[width = \figWidth \textwidth]{Simulation_Results/Synthetic/New/1D_Matching/Uniform/regret_all.png}}
  \hfill
  \subcaptionbox{Average ratio.%
    \label{fig:R3_ratio}}[0.49 \textwidth]{\includegraphics[width = \figWidth \textwidth]{Simulation_Results/Synthetic/New/1D_Matching/Uniform/frac_of_OPT_all.png}}
  % 
  \caption{Comparison of average regret and reward ratio for forecast-agnostic heuristics in random 1D instances under $\reward(\type,\type')=|\type-\type'|$.}
  % 
  \label{fig:R3}
\end{figure}

% \end{APPENDICES}
\end{APPENDIX}




%%%%%%%%%%%%%%%%%
\end{document}
%%%%%%%%%%%%%%%%%



 % outcomment this line in Case 2

%If you don't use BiBTex, you can manually itemize references as shown below.

%\bibliographystyle{nonumber}

% \begin{thebibliography}{3}
% \providecommand{\natexlab}[1]{#1}
% \providecommand{\url}[1]{\texttt{#1}}
% \providecommand{\urlprefix}{URL }

% \bibitem[{Smith(2005)}]{smith2005}
% Smith J (2005) Optimal resource allocation in humanitarian logistics.
%   \emph{Journal of Operations Research} 30(2):123--135.
  
% \bibitem[{Jones(2010)}]{jones2010}
% Jones S (2010) Stochastic programming models for humanitarian logistics.
%   \emph{INFORMS Mathematics of Operations Research} 35(4):567--580.

% \bibitem[{Brown(2015)}]{brown2015}
% Brown D (2015) \emph{Introduction to Stochastic Programming} (Springer).

% \end{thebibliography}



% Appendix here
% Options are (1) APPENDIX (with or without general title) or
%             (2) APPENDICES (if it has more than one unrelated sections)
% Outcomment the appropriate case if necessary
%
\begin{APPENDIX}{}
We provide in \Cref{appx:proofs} proofs that are omitted from \Cref{sec:PB} of the paper.
% 
\Cref{sec: interpretation} shows % further details on 
an interpretation of potential in terms of the marginal value of jobs. 
% 
Theoretical guarantees under alternative reward topologies are stated and proved in \Cref{appx:alternative_reward_topologies}. 
%
\Cref{sec: LP} provides details on the LP formulation and dual variables used for the $\dual$ and $\averagedual$ benchmarks. 
%
Finally, we include in \Cref{sec:market_dynamics} high-level descriptions of market-dynamics observed in Meituan data, and in \Cref{sec: sim_results_extra} additional simulation results, both for settings studied in the body of the paper as well as more general settings with 2D locations, non-uniform spatial distributions, and different reward topologies. 

%
%   or
%
% \begin{APPENDICES}
\crefalias{section}{appendix}
\crefalias{subsection}{appendix}
\crefalias{subsubsection}{appendix}

\section{Deferred Proofs} \label{appx:proofs}

\subsection{Proof of \Cref{prop:greedy_linear_lower_bound}} \label{pf:greedy_linear_lower_bound}

Let $0<\eps<1/3$.
% 
Consider an instance $\instance \in [0,1]^\Njob$ such that $\type_{n/2}<\type_{n/2-1}<\ldots<\type_{1}<\eps$ (low-type jobs) and $\type_{n/2+1}=\type_{n/2+2}=\ldots=\type_{\Njob}=1$ (high-type jobs). 
%
Note that because of \Cref{deliverygreedy}, when the first job becomes critical, $\gre$ chooses to match with a high-type job, i.e. $\matchof(1)\in \{n/2+1,\ldots,\Njob\}$, collecting $\reward(\type_1,\type_{\matchof(1)})\le \eps$ independent of the actual choice of $\matchof(1)$.
%
In particular, the second job is not matched and is then the next to become critical.
%
Since there are still high-type jobs to be matched, then by the same argument $\matchof(2)\in \{n/2+1,\ldots,\Njob\}$ and $\reward(\type_2,\type_{\matchof(2)})\le \eps$.
%
By induction, since there are as many high-type jobs as low-type jobs, every job $j \in \{1,\ldots,n/2\}$ is matched with $\matchof(j)\in \{n/2+1,\ldots,\Njob\}$ and $\reward(\type_j,\type_{\matchof(j)}) \le \eps$. Therefore, ${\gre}(\instance,\infty) \le \eps n/2$. 

On the other hand, $\reward(\type_{j},\type_{j+1}) = 1$ for all $j \in \{\Njob/2+1,\ldots,\Njob-1\}$. Thus, an optimal matching solution would always match all $\Njob/2$ high-type jobs together, collecting $\OPT(\instance, \infty) \ge \Njob/4$. 
%
Hence,
%
\[ 
    \regret_\gre(\instance,\infty) = \OPT(\instance,\infty) - \gre(\instance, \infty) \ge \frac{(1-2\eps)}{4}\Njob, 
\]
independent of the tie-breaking rule. Since this inequality holds for all $0<\eps<1/3$, the proof is complete.\Halmos
% 

\subsection{Proof of \Cref{prop:loglowerboundOffline}}
\label{pf:loglowerboundOffline}

We construct a sequence of instances $\instance(k)\in [0,1]^{\Njob(k)}$ with $\Njob(k)=2^{k+3} - 4$ for every integer $k\ge 0$, and show that the regret gain at each iteration, $\regret_\PB(\instance(k+1),\infty)-\regret_\PB(\instance(k),\infty)$ is at least $1/4$. For $k=0$, consider $\instance(0)=(1/2,0,1,0)$. We have that $\OPT(\instance(0),\infty) = 1/2$. In turn, under $\PB$, when the first job becomes critical, all available jobs are equally close, resulting in ties. Assume ties are broken so that the resulting matching is $\{(1,2),(3,4)\}$. Then, $\PB(\instance(0),\infty) = 0$, and thus $\regret_\PB(\instance(0),\infty) = 1/2$.
%
For $k\ge 1$, we inductively construct an instance $\instance(k)$ by carefully adding $2^{k+2}$ new jobs to be processed before the instance considered in the previous step. In particular, we add $2^{k}$ copies of $\instance(0)$, each scaled by $1/2^{k+1}$ and shifted in space so that $\PB$ matches within each copy. 
%

Formally, for $s\in\{0,\ldots,2^{k}-1\}$ and $r\in\{1,2,3,4\}$, define
\[ \instance(k)_{4s+r} = \frac{\instance(0)_r}{2^{k+1}} + \frac{s}{2^k}. \]
Lastly, define $\instance(k)_j = \instance(k-1)_{j-2^{k+2}} $ for $j \in \{ 2^{k+2}+1,\ldots,  \Njob(k) \}$.
%
By construction, for each $s\in \{0,\ldots,2^{k}-1\}$, $\PB$ matches job $4s+1$ with $4s+2$ since it is the closest in space (see \Cref{deliverypotential}), and then matches $4s+3$ with $4s+4$ for the same reason. After all these jobs are matched, the remaining jobs represent $\instance(k-1)$ exactly.
%
On the other hand, $\OPT$ is at least as good as processing $\OPT$ separately on each of the $2^k$ copies, and then on $\instance(k-1)$.
%
Hence, the regret gain $\regret_\PB(\instance(k),\infty) - \regret_\PB(\instance(k-1),\infty)$ is at least the sum of the regret of $\PB$ on each copy.
%
Moreover, note the reward function $\reward(\type,\type') = \min\{\type,\type'\}$ satisfies $\reward(\alpha+\type/\beta,\alpha\type'/\beta) = \reward(\type,\type')/\beta$ for any scalars $\alpha,\beta$ such that $\alpha+\type/\beta,\alpha\type'/\beta\in[0,1]$. 
%
Thus, the regret of $\PB$ on each copy is exactly $\regret_\PB(\instance(0),\infty)/{2^{k+1}}$, and then
%
\[ \regret_\PB(\instance(k),\infty) - \regret_\PB(\instance(k-1),\infty) \ge 2^k\frac{\regret_\PB(\instance(0),\infty)}{2^{k+1}} \ge \frac{1}{4}, \]
%
completing the proof.
\Halmos

\subsection{Proof of \Cref{prop:greedy_linear_lower_bound_dynamic}}
\label{pf:greedy_linear_lower_bound_dynamic}

Note that if $(\sojourn+1)$ is divisible by 4, then for any number of jobs $\Njob$ divisible by $(\sojourn+1)$, we have an exact partition $b=\Njob/(\sojourn+1)\in\{1,2,\ldots\}$ batches, each of size $(\sojourn+1)$.
%
We construct an instance $\instance\in[0,1]^\Njob$ that can be analyzed separately on each batch in the following sense.
%
For each $t=1,\ldots,b$, let $\instance^{(t)} = (\type_{j})_{j\in B_t}\in[0,1]^{(\sojourn+1)}$.
%
It is straightforward to check that $\OPT(\instance,\sojourn) \ge \sum_{t=1}^b \OPT(\instance^{(t)},\infty) $, since the matching solution induced by the offline instances on the right-hand side is feasible in the online setting.
%
Thus, if $\instance$ is such that $\gre(\instance,\sojourn) = \sum_{t=1}^b \gre(\instance^{(t)},\infty) $, then
\[ \OPT(\instance,\sojourn) - \gre(\instance,\sojourn) \ge \sum_{t=1}^b \regret_\gre(\instance^{(t)},\infty). \]
%
Moreover, we construct $\instance$ such that each $\instance^{(t)}\in[0,1]^{(\sojourn+1)}$ resembles the worst-case instance analyzed in \Cref{prop:greedy_linear_lower_bound}.
%
To this end, let $0<\eps<1$. Consider $\instance\in[0,1]^\Njob$ such that for each $t=0,\ldots,b-1$,
%
$\eps/2^{t+1}<\type_{(\sojourn+1)t+(\sojourn+1)/2}<\ldots<\type_{(\sojourn+1)t+1}<\eps/2^t$ (low-type jobs) and $\type_{(\sojourn+1)t+(\sojourn+1)/2+1}=\ldots=\type_{(\sojourn+1)(t+1)} = 1$ (high-type jobs).
%
By construction, when job $1$ becomes critical, the available jobs $k\in A(1)\subseteq B_1$ have types $\type_k < \type_1$ or $\type_k=1$. Then, by \Cref{deliverygreedy}, $\gre$ chooses to match with a high-type job $\matchof(1)\in B_1$.
%
As job $2$ becomes critical next, the new set of available jobs is $A(2)=A(1)\cup\{(\sojourn+1)+1\}\setminus\{\matchof(1)\}$, with $\type_{(\sojourn+1)+1}<\eps/2<\type_2$. Again, $\gre$ chooses to match with a high-type job $\matchof(2)\in B_2$.
%
By repeating this argument inductively, $\gre$ outputs $(C,\{\matchof(j)\}_{j\in C})$ such that $C = \bigcup_{t=1}^{b-1}\{(d+1)t + 1,\ldots, (d+1)t + \sojourn+1)t+(\sojourn+1)/2 \}$ and $\matchof(j) \in B_t\setminus C$ for all $j\in C\cap B_t$. 
%
Thus, $\gre(\instance,\sojourn) = \sum_{t=1}^b \gre(\instance^{(t)},\infty) $, and since $\instance^{(t)}\in[0,1]^{(\sojourn+1)}$ is specified as in the proof of \Cref{prop:greedy_linear_lower_bound}, then  $\regret_\PB(\instance^{(t)},\infty)\le (\sojourn+1)(1-2\eps)/4$. 
%
Hence,
\[ \OPT(\instance,\sojourn) - \gre(\instance,\sojourn) \ge b\frac{(\sojourn+1)(1-2\eps)}{4} = \frac{\Njob}{4}(1-2\eps), \]
%
completing the proof.
\Halmos

\subsection{Proof of \Cref{prop:loglowerboundOnline}}
\label{pf:loglowerboundOnline}

Since $\Njob$ is divisible by $(\sojourn+1)$, we have an exact partition of the set of jobs into $b=n/(\sojourn+1)\in\{1,2,\ldots\}$ batches. 
%
As in \Cref{prop:greedy_linear_lower_bound_dynamic}, we construct an instance $\instance\in[0,1]^\Njob$ that can be analyzed separately on each batch, where we denote $\instance^{(t)}=(\type_j)_{j\in B_t}$ for $t=1,\ldots,b$.
%
By \Cref{prop:loglowerboundOffline}, there exists an instance $\instance^{(0)}\in[0,1]^{(\sojourn+1)}$ such that $\regret_\PB(\instance^{(0)},\infty) \ge (\log(\sojourn+5)-3)/4$.
%
Then, for each $t=1,\ldots,b$, let $\instance^{(t)}=\instance^{(0)}\cdot\frac13$ if $t$ is odd and $\instance^{(t)}=\instance^{(0)}\cdot\frac13+\frac23$ if $t$ is even.
%
Note that since each batch size is $\sojourn+1$ divisible by 4, it is possible to match within batch only. Moreover, by construction, every critical job has a matching candidate within batch, which is closer in space than any job on the next batch. Thus, because of \Cref{deliverypotential}, $\PB$ only matches within batch, and therefore $\PB(\instance,\sojourn) = \sum_{t=1}^b \PB(\instance^{(t)},\infty)$.
%
As in \Cref{prop:greedy_linear_lower_bound_dynamic}, we arrive at
\[ \OPT(\instance,\sojourn) - \PB(\instance,\sojourn) \ge \sum_{t=1}^b \regret_\PB(\instance^{(t)},\infty) = b\frac{ \regret_\PB(\instance^{(0)},\infty)}{3} \ge \frac{\Njob}{3(\sojourn+1)}(\log(\sojourn+5)-3)/4, \]
%
concluding the proof.
\Halmos

\section{Interpretation of Potential}\label{sec: interpretation}

When a job $j$ becomes critical, our potential-based greedy algorithm matches it to an available job $k$ maximizing $r(\theta_j,\theta_k)-p(\theta_k)$, where $p(\theta_k)$ can be interpreted as the opportunity cost of matching job $k$, with the specific definition $p(\theta_k)=\frac12 \sup_{\theta\in\Theta} r(\theta_k,\theta)$.
This is an optimistic measure of opportunity cost because it assumes that job $k$ would otherwise be matched to an "ideal" type $\theta\in\Theta$ maximizing $r(\theta_k,\theta)$ (with half of this ideal reward $\sup_{\theta\in\Theta} r(\theta_k,\theta)$ attributed to job $k$).
We now prove that this ideal reward can indeed be achieved under asymptotically-large market thickness, for a stochastic model under our topology of interest.

\begin{definition}
Let $\instance \in \typespace^\Njob$.
For any job $j\in[\Njob]$, let $\instance^{-j}\in \typespace^{\Njob-1}$ be the same instance with the exception that job $j$ is not present. Meanwhile, let $\instance^{+j}\in \typespace^{\Njob+1}$ be the same instance with the exception that an additional copy of job $j$ is present.
Consider the following definitions.
\begin{enumerate}
\item Marginal Loss: $\marginalloss_j(\instance) = \OPT(\instance) - \OPT(\instance^{-j})$
\item Marginal Gain: $\marginalgain_j(\instance) = \OPT(\instance^{+j}) - \OPT(\instance)$
\end{enumerate}
\end{definition}

Note that definitions $\marginalloss_j(\instance),\marginalgain_j(\instance)$ are based solely on offline matching, and we will use them as our definitions of opportunity cost if the future was known.  One could alternatively use shadow prices from the LP relaxation of the offline matching problem, but we note that the LP is not integral.  In either case, there is no ideal definition of opportunity cost that is guaranteed to lead to the optimal offline solution in matching problems \citep[see][]{cohen2016invisible}.

We now establish the following \namecref{lem:marginal_as_interval} to help analyze the opportunity costs $\marginalloss_j(\instance),\marginalgain_j(\instance)$.


\begin{lemma}\label{lem:marginal_as_interval}
Let $\typespace=[0,1]$ and $\reward(\type,\type') = \min\{\type,\type'\}$. Consider an instance $\instance\in\typespace^n$ and relabel the indices to satisfy $\type_1 \ge \type_2 \ge \ldots \ge \type_n$. Then,
\begin{align*}
\marginalloss_j(\instance)
&=\sum_{k\ge j,k\ \mathrm{even}}(\theta_k-\theta_{k+1}), 
\\ \marginalgain_j(\instance)
&=\sum_{k\ge j,k\ \mathrm{odd}}(\theta_k-\theta_{k+1}),
\end{align*}
where we consider $\theta_{n+1}=0$.
\end{lemma}
%
\proof{Proof.}
    Let $\instance \in \typespace^\Njob$ such that $\type_1 \ge \type_2 \ge \ldots \ge \type_n$. 
    %
    Then, $\OPT(\instance) =\sum_{k\ \mathrm{even}}\theta_k$, and moreover
    \begin{align*}
    \OPT(\instance^{-j}) &=\sum_{k<j, k\ \mathrm{even}}\theta_k+\sum_{k>j, k\ \mathrm{odd}}\theta_k
    \\ \OPT(\instance^{+j}) &=\theta_j+\OPT(\instance^{-j})
    \end{align*}
    % 
    Therefore,
    % 
    \begin{align*}
    \marginalloss_j(\instance)
    &=\sum_{k\ge j,k\ \mathrm{even}}\theta_k
    -\sum_{k>j,k\ \mathrm{odd}}\theta_k
    =\sum_{k\ge j,k\ \mathrm{even}}(\theta_k-\theta_{k+1})
    \\ \marginalgain_j(\instance)
    &=\theta_j-\sum_{k\ge j,k\ \mathrm{even}}(\theta_k-\theta_{k+1})
    =\sum_{k\ge j,k\ \mathrm{odd}}(\theta_k-\theta_{k+1})
    \end{align*}
    % 
    completing the proof.
\Halmos\endproof

Note that because $\OPT(\instance^{+j}) = \OPT(\instance^{-j}) + \type_j$ and $\potential(\type_j) = \reward(\type_j,\type_j)/2$ for this reward function, we immediately get the following \namecref{cor: marginal_average}.

\begin{corollary}\label{cor: marginal_average}
        If $\typespace=[0,1]$ and $\reward(\type,\type') = \min\{\type,\type'\}$, then for all jobs $j$,
    \[ \potential(\type_j) = \frac{\marginalloss_j(\instance) + \marginalgain_j(\instance)}{2}. \]
\end{corollary}

We are now ready to prove our main result about the interpretation of potential, that the true opportunity costs $\marginalloss_j(\instance),\marginalgain_j(\instance)$ concentrate around $p(\theta_j)$ in a random uniform instance, assuming the market is sufficiently thick.  We without loss consider job $j=1$ and fix its type $\theta_1$.

\begin{theorem} \label{thm:interpretation}
Let $\typespace=[0,1]$ and $\reward(\type,\type') = \min\{\type,\type'\}$.
Fix $\theta_1\in\typespace$ and suppose $\theta_2,\ldots,\theta_n$ are drawn IID from the uniform distribution over $[0,1]$, forming a random instance $\instance\in\typespace^n$, for some $n\ge 2$. Then, both $\mathbb{E}[\marginalloss_1(\instance)]$ and $\mathbb{E}[\marginalgain_1(\instance)]$ are within $O(1/n)$ of $\potential(\type_1)$ and moreover $\mathrm{Var}(\marginalloss_1(\instance))=\mathrm{Var}(\marginalgain_1(\instance)) = O\left(\frac{1}{n}\right)$.
\end{theorem}
%
\proof{Proof.}
We prove the statement only for $\marginalloss_1(\instance)$, and the analogous result for $\marginalgain_1(\instance)$ follows from \Cref{cor: marginal_average}.
%
If $\type_1=0$, then $\marginalloss_1(\instance)=0$ for any $\instance$, coinciding with $\potential(\type_1)=0$. Then, for the remainder of the proof, assume that $\type_1>0$.
%
Consider the random variable $N=|\{j:\theta_j<\theta_1\}|$, which counts the number of points between $0$ and $\theta_1$ and has distribution $\text{Binom}(\Njob-1,\type_1)$.
%
These $N$ points divide the interval $[0,\theta_1]$ into $N+1$ intervals with total length $\type_1$.
%
From \Cref{lem:marginal_as_interval}, the marginal loss $\marginalloss_1(\instance)$ is determined by computing the total length of a subset of these intervals. % Maybe add figure?
%
It is known that if $N$ random variables are drawn independently from $\text{Unif}[0,1]$, then the joint distribution of the induced interval lengths is $\text{Dirichlet}(1,1,\ldots,1)$ with $N+1$ parameters all equal to 1 (i.e., drawn uniformly from the simplex).
%
In particular, since the Dirichlet distribution is symmetric, the sum of any $k$ of these intervals is equal in distribution to the $k$-th smallest sample ($k$-th order statistic) of the $N$ uniform random variables, whose distribution is known to be $\text{Beta}(k,N+1-k)$.
%
Thus, conditional on $N$, with $N\ge 1$, since the distribution of each of the $N$ jobs to the left of $\type_1$ is $\text{Unif}[0,\type_1]$, the distribution of $\marginalloss_1(\instance)/\type_1$ is $\text{Beta}(k_L,N+1-k_L)$,
where $k_L$ is either $\lceil\frac{N+1}2\rceil$ or $\floor{\frac{N+1}2}$.
%
To be precise, for $N\ge 1$, $k_L=\floor {N/2}+1=\lceil\frac{N+1}2\rceil$ if $n$ is even, and $k_L=\ceil{N/2}=\floor{\frac{N+1}2}$ if $n$ is odd.
%
Moreover, if $N=0$, then $\marginalloss_1(\instance)=\type_1$ if $n$ is even, and $\marginalgain_1(\instance)=0$ if $n$ is odd.
%
Hence,
\begin{align*}
\mathbb{E}[\marginalloss_1(\type) \mid N] 
&= \type_1\frac{k_L}{N+1}
% \label{eq: cond_exp}
\end{align*}
for all $N\in\{0,\ldots,n-1\}$.
%
Since $|k_L-(N+1)/2|\le 1$, then $|\mathbb{E}[\marginalloss_1(\type) \mid N]-\type_1/2|\le \type_1/(N+1)$, thus
% 
\begin{align*}
&\left|\mathbb{E}[\marginalloss_1(\type)]-\frac{\type_1}{2}\right| 
\le \type_1\mathbb{E}\left[\frac{1}{N+1}\right] \\
&\qquad = \type_1\sum_{k=0}^{n-1} \frac{1}{k+1}\binom{n-1}{k}\type_1^k(1-\type_1)^{n-1-k} \\
&\qquad = \frac{\type_1}{n}\sum_{k=0}^{n-1} \binom{n}{k+1}\type_1^k(1-\type_1)^{n-1-k} \\
&\qquad = \frac{1}{n}\sum_{k=1}^{n} \binom{n}{k}\type_1^k(1-\type_1)^{n-k} \\
&\qquad = \frac{1-(1-\type_1)^{n}}{n} = O\left(\frac{1}{n}\right).
\end{align*}
%
This completes the proof of the statement about $\mathbb{E}[\marginalloss_1(\instance)]$.

For the statement about variance, we know $\mathrm{Var}(\marginalloss_1(\type)) = \mathbb{E}[\mathrm{Var}(\marginalloss_1(\type) \mid N)] + \mathrm{Var}(\mathbb{E}[\marginalloss_1(\type)\mid N])$ by the law of total variance. For the first term, we have $\mathrm{Var}(\marginalloss_1(\type) \mid N) = \type_1^2\frac{k_L(N+1-k_L)}{(N+1)^2(N+2)}\indicator\{N\ge 1\}\le \frac{\type_1^2}{N+1}$, and then from the previous argument $\mathbb{E}[\mathrm{Var}(\marginalloss_1(\type) \mid N)]=O(1/n)$.
%
For the second term, 
\begin{align*}
\mathrm{Var}(\mathbb{E}[\marginalloss_1(\type)\mid N]) 
&= \mathrm{Var}\left(\mathbb{E}[\marginalloss_1(\type)\mid N] - \frac{\type_1}{2}\right) \\
&\le \mathbb{E} \left[\left(\mathbb{E}[\marginalloss_1(\type)\mid N] - \frac{\type_1}{2}\right)^2\right] \\
% 
&\le \mathbb{E}\left[\frac{\type_1^2}{(N+1)^2}\right] \\
&\le \type_1^2\mathbb{E}\left[\frac{1}{N+1}\right] = O\left(\frac{1}{n}\right).
\end{align*}
Thus, $\mathrm{Var}(\marginalloss_1(\instance)) = O\left(1/n\right)$, completing the proof.
\Halmos\endproof
%%%%%%%%

\section{Algorithmic Performance under Alternative Reward Topologies} \label{appx:alternative_reward_topologies}


In this section, we study the performance of both the naive greedy algorithm $\gre$ and potential-based greedy algorithm $\PB$, under the two alternative reward topologies defined in \eqref{eq:defn_reward_B} and \eqref{eq:defn_reward_C}.

\subsection{Reward Function $\reward(\type,\type')=1-|\type-\type'|$}
\phantomsection
\label{sec:reward2}
%
This reward function represents the goal to minimize the total distance between matched jobs.
%
In this case, for any $\type\in[0,1]$, the potential of a job of type $\type$ is $\potential(\type)=1/2$ constant across job types. Thus, the description of $\PB$ is equivalent to $\gre$, since their index functions differ only in an additive constant.

\begin{remark}\label{rem:rewardB}
    Under the 1-dimensional type space $\typespace=[0,1]$ and the reward function $\reward(\type,\type')=1-|\type-\type'|$, both the naive greedy and potential-based greedy algorithm generate the same output $(C,\matchof(\cdot))$ as described in \Cref{alg:dynamic}. In particular, both algorithms always match each critical job to an available job that is the closest in space.
\end{remark}

Moreover, we show that their performance can be reduced to the one of $\PB$ under reward function $\reward(\type,\type')=\min\{\type,\type'\}$.
%
In particular, we first show that when $\sojourn=\infty$, the regret is at least logarithmic in the number of jobs.
\begin{proposition}\label{prop:loglowerboundOffline_rewardB}     
    Under reward function $\reward(\type,\type')=1-|\type-\type'|$, if $\Njob=2^{k+3}-4$ for some integer $k\ge0$, then there exists an instance $\instance\in[0,1]^\Njob$ for which $\regret_\PB(\instance,\infty) = \regret_\gre(\instance,\infty) \ge (\log_2(\Njob+4)-3)/2 $.
\end{proposition}

\proof{Proof.}
    %
    We prove the statement for $\gre$, since $\regret_\PB(\instance,\infty) = \regret_\gre(\instance,\infty)$ follows from \Cref{rem:rewardB}.
    %
    Let $\Njob=2^{k+3}-4$, for $k\in\{0,1,\ldots\}$. From \Cref{prop:loglowerboundOffline}, there exists an instance $\instance\in[0,1]^\Njob$ such that the regret of $\PB$ under reward topology $\reward(\type,\type')=\min\{\type,\type'\}$, which we denote by $R$ from here on, is at least $(\log_2(\Njob+4)-3)/4$.
    %
    We show that under $\reward(\type,\type')=1-|\type-\type'|$, we have $\regret_\gre(\instance,\infty) \ge 2R$.
    %
    First, note that since $|\type-\type'| = \type + \type' - 2\min\{\type,\type'\}$, we have
    %
    \[ \gre(\instance,\infty) 
         = \sum_{j \in C} 1 - |\type_j-\type_{m(j)}|  
         = |C| - \sum_{j\in C} (\type_j +\type_{\matchof(j)})  + 2\sum_{j\in C} \min\{\type,\type'\}. 
    \]
    % 
    Similarly, recall that $\matchset_\OPT$ is the set of matches in the hindsight optimal solution. Then,
    \[ \OPT(\instance,\infty) = |\matchset_\OPT| - \sum_{ (j,k)\in\matchset_\OPT} (\type_j +\type_{k})  + 2\sum_{(j,k)\in\matchset_\OPT} \min\{\type_j,\type_{k}\}. \]
    %
    Since $\reward(\type,\type')\ge 0$ for any $\type,\type'\in \typespace$ and the total number of jobs $\Njob$ is even, both $\gre$ and $\OPT$ match every job. Hence, $|C|=|\matchset_\OPT|=n/2$, and moreover
    \[ \sum_{(j,k)\in\matchset_\OPT} (\type_j +\type_{k}) = \sum_{j\in C} (\type_j +\type_{\matchof(j)}). \]
    Thus,
    \[ \OPT(\instance,\infty) - \gre(\instance,\infty) \ge 2\left( \sum_{(j,k)\in\matchset_\OPT} \min\{\type_j,\type_{k}\} - \sum_{j\in C} \min\{\type_j,\type_{\matchof(j)}\} \right). \]
    %
    We claim that the term in large parenthesis is exactly the regret of $\PB$ under reward topology $\reward(\type,\type')=\min\{\type,\type'\}$, which we denote by $R$.
    %
    Indeed, from \Cref{rem:rewardB}, $(C,\matchof(\cdot))$ coincides with the resulting matching of $\PB$ under $\reward(\type,\type')=\min\{\type,\type'\}$, since it is specified to always match to the closest job (\Cref{deliverypotential}).
    %
    On the other hand, since all jobs are matched, the set of matches $\matchset_\OPT$ must induce disjoint intervals, which coincides with the optimal matching solution under reward function $\reward(\type,\type') = \min\{\type,\type'\}$.
    %
    Thus, by \Cref{prop:loglowerboundOffline}, we get $ \regret_\gre(\instance,\infty) \ge 2R \ge (\log_2(\Njob+4)-3)/2$, completing the proof.
\Halmos\endproof

We now extend this result to the online setting ($\sojourn<n$).

\begin{proposition} \label{prop:loglowerboundOnline_rewardB}
    Under reward topology $\reward(\type,\type') = 1-|\type-\type'|$, if $\sojourn+1 = 2^{k+3}-4$ for some integer $k\ge 0$, then for any number of jobs $\Njob$ divisible by $(\sojourn+1)$, there exists an instance $\instance \in [0,1]^\Njob$ for which $\regret_\PB(\instance,\sojourn) = \regret_\gre(\instance,\sojourn) \ge \frac{\Njob}{3(\sojourn+1)}(\log_2(\sojourn+5)-3)/2$.
\end{proposition}

\proof{Proof.}
    % 
    This proof is analogous to the proof of \Cref{prop:loglowerboundOnline}, since (i) there exists an offline instance that achieves the desired regret for $\sojourn+1=n$ (\Cref{prop:loglowerboundOffline_rewardB}), and (ii) the algorithm matches a critical job with the closest available job (\Cref{rem:rewardB}).
    %
    We repeat the proof below for completeness.

    We have an exact partition of the set of jobs into $b=n/(\sojourn+1)\in\{1,2,\ldots\}$ batches, and construct an instance $\instance\in[0,1]^\Njob$ that can be analyzed separately on each batch, where we denote $\instance^{(t)}=(\type_j)_{j\in B_t}$ for $t=1,\ldots,b$.
    %
    By \Cref{prop:loglowerboundOffline_rewardB}, there exists an instance $\instance^{(0)}\in[0,1]^{(\sojourn+1)}$ such that $\regret_\PB(\instance^{(0)},\infty) \ge (\log(\sojourn+5)-3)/2$.
    %
    Then, for each $t=1,\ldots,b$, let $\instance^{(t)}=\instance^{(0)}\cdot\frac13$ if $t$ is odd and $\instance^{(t)}=\instance^{(0)}\cdot\frac13+\frac23$ if $t$ is even.
    %
    Note that since each batch size is $\sojourn+1$ divisible by 4, it is possible to match within batch only. Moreover, by construction, every critical job has a matching candidate within batch, which is closer in space than any job on the next batch. Thus, because of \Cref{rem:rewardB}, $\PB$ only matches within batch, and therefore $\PB(\instance,\sojourn) = \sum_{t=1}^b \PB(\instance^{(t)},\infty)$.
    %
    As in \Cref{prop:greedy_linear_lower_bound_dynamic}, we arrive at
    \[ \OPT(\instance,\sojourn) - \PB(\instance,\sojourn) \ge \sum_{t=1}^b \regret_\PB(\instance^{(t)},\infty) = b\frac{ \regret_\PB(\instance^{(0)},\infty)}{3} \ge \frac{\Njob}{3(\sojourn+1)}(\log(\sojourn+5)-3)/2, \]
    %
    concluding the proof.
\Halmos\endproof

Finally, we directly show the tight regret upper bound (up to constants) as an immediate consequence of the proof of \Cref{thm: dynamic}. Note that if $d+1=n$, then this yields the result for offline matching.

\begin{theorem}\label{thm:rewardB_dynamic}
Under reward topology $\reward(\type,\type') = 1-|\type-\type'|$, we have $\regret_\PB(\instance,\sojourn) = \regret_\gre(\instance,\sojourn) \le 1/2 + (\frac{\Njob}{d+1}+1)(1+\log(\sojourn+2))$ for any $\Njob$, and any instance $\instance\in[0,1]^\Njob$.
\end{theorem}


\proof{Proof.}
Let $\instance\in \typespace^\Njob$ and $\sojourn\ge 1$. Let $(C,\matchof(\cdot))$ be the output of $\gre$ on $\instance$. We have
\begin{align*}
\OPT(\instance,\sojourn)-\gre(\instance,\sojourn) 
& \le \sum_{j=1}^\Njob \potential(\type_j) - \gre(\instance,\sojourn)  \nonumber \\
& \le \sup_{\type\in\typespace}\potential(\type) + \sum_{j \in C} \left( p(\theta_j) + p(\theta_{m(j)}) - r(\theta_j,\theta_{m(j)}) \right) \nonumber \\ 
& = \frac12 +  \sum_{j \in C} \left( \frac12 + \frac12 - (1-|\theta_j-\theta_{m(j)}|) \right) \nonumber \\
& = \frac12 + \sum_{j \in C} |\type_j-\type_{\matchof(j)}|, 
\end{align*}
The second term is exactly the sum of distances between jobs matched by $\gre$. Recall that $(C,\matchof(\cdot))$ coincides with the matching output of $\PB$ under $\reward(\type,\type')=\min\{\type,\type'\}$, both are specified to always match with the closest job (\Cref{deliverypotential}, \Cref{rem:rewardB}). Thus, recalling \eqref{eq: distance_dynamic} in the proof of \Cref{thm: dynamic}, we have
\[ \sum_{j \in C} |\type_j-\type_{\matchof(j)}| = \sum_{t=1}^b \sum_{j \in C\cap B_t} |\type_j-\type_{\matchof(j)}|\le \left(\frac{n}{d+1}+1 \right)\left(1+\log(d+2)\right),\]
and the result follows.
\Halmos\endproof

\subsection{Reward Function $\reward(\type,\type')=|\type-\type'|$}\label{sec: reward 3}
%
Under this reward topology, it is worst to match two jobs of the same type, contrasting the two settings studied in \Cref{sec:PB} and \Cref{sec:reward2}.
%
In this case, the potential of a job type $\type\in[0,1]$ is $\potential(\type)=\max\{\type,1-\type\}/2$.
%
We first show that \emph{any} index-based matching policy must suffer regret constant regret per job in the offline setting.
%
\begin{proposition}\label{prop:linearLB_offline_rewardC}
    Under reward topology $\reward(\type,\type')=|\type-\type'|$, when the number of jobs $\Njob$ is divisible by 4, for any index-based greedy matching algorithm $\ALG$, there exists an instance $\instance\in[0,1]^\Njob$ for which $\regret_\ALG(\instance,\infty)\ge c_\ALG \Njob$, for some $c_\ALG\in (0,1)$. In particular, $c_\gre = c_\PB = 1/4$.
\end{proposition}

\proof{Proof.}
Let $\Njob$ be divisible by 4.
%
Let $\ALG$ be an index-based greedy matching algorithm specified by index function $\indexf:[0,1]^2\to \R$.
%
Define $\type^{c} = \sup\{ \type\in[0,1] : \indexf(\type,1)\ge \indexf(\type,0) \}$.
%
We first construct an instance in the case $\type^{c}>0$ (e.g. $\type^{c}=1/2$ for $\gre$ and $\PB$), and analyze the alternative case separately in the end.
%
Consider an instance $\instance\in\typespace^\Njob$ such that $\type_1,\ldots,\type_{\Njob/4} = \type^{c}$, $\type_{{\Njob/4}+1},\ldots,\type_{{3\Njob/4}}=0$, and $\type_{{3\Njob/4}+1},\ldots,\type_{\Njob}=1$. 
%
It is straightforward to check that $\OPT(\instance,\infty) \ge (1+\type^c){\Njob/4}$.
%
On the other hand, $\ALG$ chooses to match every $j=1,\ldots,n/4$ to $\matchof(j)\in \{3n/4+1,\ldots,n\}$.\footnote{In the presence of ties, such an output is achieved by some tie-breaking rule.}
%
Then,
\[ \ALG(\instance,\infty) = 0 + \sum_{j=1}^{\Njob/4} \reward(\type_j,\type_{3\Njob/4+j}) = \frac{\Njob(1-\type^c)}{4}. \]
%
and thus $\regret_\ALG(\instance,\infty) = \OPT(\instance,\infty) - \ALG(\instance,\infty) \ge {{\Njob\type^{c}}}/{2},$ proving the statement with $c_\ALG = \type^{c}/2$.

%
Now, if $\type^{c}\le 0$, then $\indexf(\type,0) < \indexf(\type,1)$ for all $\type\in(0,1]$.
%
In particular, for any $\eps>0$, consider the instance $\type_1,\ldots,\type_{\Njob/4} = \eps$, $\type_{{\Njob/4}+1},\ldots,\type_{{3\Njob/4}}=1$, and $\type_{{3\Njob/4}+1},\ldots,\type_{\Njob}=0$.
%
Then, $\OPT(\instance,\infty) \ge (1+(1-\eps))n/4$, and $\ALG(\instance,\infty) = \eps n/4$.
%
Hence, $\regret_\ALG(\instance,\infty)\ge (1-\eps)/2$, and the statement is true with $c_\ALG = (1-\eps)/2$ for every $\eps>0$.
\Halmos\endproof

The following result shows that, under $\PB$ and $\gre$, this lower bound extends to the online setting, losing a factor of 2.

\begin{proposition}\label{prop:linearLB_dynamic_rewardC}
    Under reward topology $\reward(\type,\type')=|\type-\type'|$, if $(\sojourn+1)$ is divisible by 4, then for any number of jobs $\Njob$ divisible by $2(\sojourn+1)$, there exists an instance $\instance \in [0,1]^\Njob$ for which $\regret_\gre(\instance,\sojourn)=\regret_\PB(\instance,\sojourn) \ge n/8$.
\end{proposition}

\proof{Proof.}
If $\Njob$ is divisible by $2(\sojourn+1)$, we have an exact partition of the set of jobs into $b=\Njob/(\sojourn+1)\in\{2,4,\ldots\}$ batches, defined as $B_{t+1} = \{t(\sojourn+1)+1,\ldots, (t+1)(\sojourn+1)\}$ for $t = 0, \ldots, b-1$.
%
We construct an instance $\instance\in[0,1]^\Njob$ for which $\gre$ (and $\PB$) "resets" every two batches, in the sense that all jobs in these two batches are matched among them.
%
To formally specify the construction, first fix $\eps\in(0,1/2)$. Define, for $t=0,\ldots,b/2-1$
%
\[
\type_{2t(\sojourn+1)+j} =
\begin{cases}
\frac12 & \text{if } j \le \frac{\sojourn+1}{4}, \\
\eps & \text{if } \frac{\sojourn+1}{4} < j \le \frac{3(\sojourn+1)}{4}, \\
1 & \text{if } \frac{3(\sojourn+1)}{4} < j \le \sojourn+1 \text{, or } \frac{3(\sojourn+1)}{2} < j \le 2(\sojourn+1), \\
0 & \text{if } \sojourn+1 < j \le \frac{3(\sojourn+1)}{2}.
\end{cases}
\]
%
Consider the notation $\instance^{(0)}=(\type_j)_{j\le 2(\sojourn+1)}$.
%
It is easy to check that
\[ 
\OPT(\instance^{(0)},\infty) = \left(\frac12-\eps\right)\frac{\sojourn+1}{4} + (1-\eps)\frac{\sojourn+1}{4} + \frac{\sojourn+1}{2} =  \left(\frac{7-4\eps}{8}\right)(\sojourn+1),
\]
and then $\OPT(\instance,\sojourn) \ge \frac{b}{2}\OPT(\instance^{(0)},\infty) = \frac{7-4\eps}{16}\Njob $.
%
On the other hand,
\[ 
\gre(\instance^{(0)},\infty) = \frac12\frac{\sojourn+1}{4} + \eps\frac{\sojourn+1}{4} + (1-\eps)\frac{\sojourn+1}{4} + \frac{\sojourn+1}{4} =  \frac58(\sojourn+1).
\]
%
We claim that $\gre(\instance,\sojourn) = \frac{b}{2}\gre(\instance^{(0)},\infty) = \frac{5}{16}\Njob $, and thus $\regret_\gre(\instance,\sojourn)\ge \frac{1-2\eps}{8}\Njob $.
%
Indeed, let $(C,\matchof(\cdot))$ be the output of $\gre$. Running $\gre$ on $\instance$ is as follows:
%
\begin{enumerate}
    \item For every job $j \le (\sojourn+1)/4$ of type $\type_j=1/2$, and then $3(\sojourn+1)/4 < \matchof(j) \le (\sojourn+1)$ with $\reward(\type_j,\type_{\matchof(j)})=1/2$.
    \item Then, for every job $\frac{\sojourn+1}{4} < j \le \frac{(\sojourn+1)}{2}$, the only available jobs are of type $\type\in\{0,\eps\}$, and then $ \sojourn+1 < \matchof(j) \le \frac{3(\sojourn+1)}{2}$ with $\reward(\type_j,\type_{\matchof(j)})=\eps$. 
    \item In turn, jobs $\frac{(\sojourn+1)}{2} < j \le \frac{3(\sojourn+1)}{4}$ get to observe jobs of type $\type=1$, and then $ \frac{3(\sojourn+1)}{2} < \matchof(j) \le 2(\sojourn+1)$ with $\reward(\type_j,\type_{\matchof(j)})=1-\eps$.
    \item Finally, the key observation is that the all remaining available jobs $j\le \frac{3(\sojourn+1)}{2}$ of type $\type_j=0$ observe jobs of type $1$, as well as jobs on a "third" batch. However, by construction every job $2(\sojourn+1) < k \le 2(\sojourn+1) + \frac{3(\sojourn+1)}{2} $ are of type $\type_k\in\{1/2,\eps\}$, and then each job $j\le \frac{3(\sojourn+1)}{2}$ is matched (within batch) to $ \frac{3(\sojourn+1)}{2} < \matchof(j) \le 2(\sojourn+1)$ with $\reward(\type_j,\type_{\matchof(j)})=1$.
\end{enumerate}

The proof for $\PB$ is analogous, since the matching decisions coincide with $\gre$ on this instance. To see this, note that
\begin{itemize}
    \item $\indexf_\PB(1/2,1) = \reward(1/2,1) - \potential(1) = 1/2 - 1/2 = 0$,
    \item $\indexf_\PB(1/2,\eps)=\reward(1/2,\eps) - \potential(\eps) = 1/2 - \eps - (1-\eps)/{2} = - \eps/2 < \indexf_\PB(1/2,1)$,
    \item $\indexf_\PB(\eps,0)=\reward(\eps,0) - \potential(0) = \eps - 1/2$,
    \item $\indexf_\PB(\eps,\eps)=\reward(\eps,\eps) - \potential(\eps) = 0 - (1-\eps)/{2}= \eps/{2}-1/2< \indexf_\PB(\eps,0)$,
    \item $\indexf_\PB(\eps,1)=\reward(\eps,1) - \potential(1) = 1 - \eps - 1/2 = 1/2 - \eps$,
    \item $\indexf_\PB(\eps,1/2)=\reward(\eps,1/2) - \potential(1/2) = 1/2 - \eps - 1/4 = 1/4 - \eps < \indexf_\PB(\eps,1) $, and
    \item $\indexf_\PB(0,\eps)= \eps - (1-\eps)/2 < \indexf_\PB(0,1/2) < \indexf_\PB(0,1)$.
\end{itemize}
Thus, running $\PB$ on $\instance$ follows the same four steps described by $\gre$, yielding the same regret.
\Halmos\endproof


\section{Linear Relaxation and Dual Variables}\label{sec: LP}

We present in this section the linear relaxation of the IP that describes the hindsight optimum defined in \eqref{eq: OPT}. Given market density $\sojourn\ge 1$ and full information of the sequence of arrivals $\instance\in\typespace$, consider the following linear program (LP)
%
\begin{maxi}
{x}{ \sum_{j,k : j\neq k, |j-k|\le \sojourn} x_{jk} \reward(\type_j,\type_k)}
{\label{eq: LP}}{}
\addConstraint{ \sum_{k:j\neq k} x_{jk}}{\le 1,}{j\in [\Njob]}
\addConstraint{ x_{jk}}{\ge 0,\quad }{ j,k\in [\Njob], j\neq k,}
\end{maxi}
%
and its dual
%
\begin{mini}
{\lambda}{ \sum_{j\in[\Njob]} \lambda_{j} }
{\label{eq: Dual-LP}}{}
\addConstraint{ \lambda_j + \lambda_k}{ \ge \reward(\type_j,\type_k),\quad}{j,k\in [\Njob], j\neq k, |j-k|\le \sojourn}
\addConstraint{ \lambda_{j} }{\ge 0 ,}{ j\in [\Njob].}
\end{mini}

It is known that the LP relaxation is integral only for bipartite graphs. However, we observe in \Cref{fig: LP_gap} that the integrality gap is small, and thus the value of the LP can be reasonably used as a benchmark.

\begin{figure}[H]
  \centering
  \subcaptionbox{Average Total Reward.%
    \label{fig:reward}}[0.49 \textwidth]{\includegraphics[width = \figWidth \textwidth]{Simulation_Results/Synthetic/New/1D_Pooling/1D_Common_Origin/Forecast-Agnostic/value.png}}
  \hfill
  \subcaptionbox{Fraction of LP value achieved by $\OPT$.%
    \label{fig:LP_Gap}}[0.49 \textwidth]{\includegraphics[width = \figWidth \textwidth]{Simulation_Results/Synthetic/New/1D_Pooling/1D_Common_Origin/Forecast-Agnostic/integrality_gap.png}}
  % 
  \caption{Comparison with LP value.
  }
  % 
  \label{fig: LP_gap}
\end{figure}

Also, note that the potential can always be used as a feasible (but not necessarily optimal) solution to the dual LP.


\begin{proposition}
    Let $\instance\in\typespace^\Njob$ and $\sojourn\ge 1$. Then, $\{\potential(\type_j)=\frac{1}{2}\sup_{\type'\in\typespace}\reward(\type_j,\type'):j\in[\Njob]\}$ is a feasible solution of the dual LP \eqref{eq: Dual-LP}.
\end{proposition}


\proof{Proof.}
    Non-negativity holds since $\reward\ge 0$. Moreover, for any $j,k$ we have
    \[ \reward(\type,\type')\le \frac{1}{2}\reward(\type,\type')+\frac{1}{2}\reward(\type,\type') \le \potential(\type)+\potential(\type'), \]
    verifying the dual feasibility constraints.
\Halmos\endproof

Interestingly, when we extract the shadow prices from the 400 instances considered as historical data in \Cref{sec:sim_unif_1D}, we can see that these concentrate around potential, as density increases (\Cref{fig: dual_converge}). This is aligned with our results in \Cref{sec: interpretation}.

\begin{figure}[H]
  \centering
  \subcaptionbox{Density $\sojourn=5$.%
    \label{fig:dual_conv_d5}}[0.4 \textwidth]{\includegraphics[width = \figWidth \textwidth]{Simulation_Results/Synthetic/New/1D_Pooling/1D_Common_Origin/Forecast-Aware/duals_at_d5.png}}
  % \hfill
  \hspace{1em}
  % 
  \subcaptionbox{Density $\sojourn=30$.%
    \label{fig:dual_conv_d30}}[0.4 \textwidth]{\includegraphics[width = \figWidth \textwidth]{Simulation_Results/Synthetic/New/1D_Pooling/1D_Common_Origin/Forecast-Aware/duals_at_d30.png}}
  % 
  \caption{Scatter plot of shadow prices, average, and potential.}
  % 
  \label{fig: dual_converge}
\end{figure}

\section{Meituan Data - Market Dynamics} \label{sec:market_dynamics}

In this appendix, we briefly describe the Meituan dataset made public by the INFORMS TSL Data-Driven Challenge, and provide high-level descriptions of the market-dynamics. 


\subsection{Temporal Dynamics} \label{appx:meituan_temporal_dynamics}

First, \Cref{fig:meituan_orders_per_hour} provides the average number of orders per hour by \emph{hour-of-week}.
Hour-of-week 0 corresponds to midnight to 1am on Mondays, and hour-of-week 1 corresponds to 1am to 2am on Mondays, and so on. 
%
The gray bars indicate the peak lunch hours (10:30am-1:30pm), while the green bars indicate peak dinner hours (5pm-8pm).
%
We can see that the platform observes highest order volume during lunch time, averaging around 200 order per minute between noon and 1pm.   

\newcommand{\FigWidthHOW}{0.85}

\begin{figure}
    \centering
    \includegraphics[width=\FigWidthHOW\linewidth]{Simulation_Results/Meituan/City/meituan_number_of_orders_how.png}
    \caption{Average number of orders by hour-of-week in Meituan data.}
    \label{fig:meituan_orders_per_hour}
\end{figure}



\Cref{fig:meituan_pooled_orders_per_hour} in \Cref{sec:intro} of the paper illustrates the fraction of orders placed during each hour-of-week that is pooled with at least one other order. 
%
This is computed by combining the order-level data (which contains the timestamp of each order and the order ID) and the ``wave level'' data, where each wave corresponds to a list of orders that were pooled and assigned to the same driver.\footnote{
For more details on waves, see \url{https://github.com/meituan/Meituan-INFORMS-TSL-Research-Challenge/blob/main/tsl_meituan_2024_data_report_20241015.pdf}, accessed January 30, 2024. Each wave may contain more than two orders, though some orders may have been dropped off by the driver before some other orders were picked up.   
}
The orders that are not pooled with any other order correspond to waves with only one order, and \Cref{fig:meituan_pooled_orders_per_hour} illustrates 1 minus this fraction of orders that are not pooled. 


\Cref{fig:meituan_grab_to_fetch_how} illustrates the average amount of time it takes the drivers to pick up the orders, after they received and then accepted the order from the platform. Specifically, this is the amount of time that elapses between (i) the grab\_time, i.e. when the order is accepted by a driver, and (ii) the fetch\_time, i.e. when the order is picked-up by the driver.
%
The average pick-up time across all orders is 10.1 minutes. 
%
We see from \Cref{fig:meituan_grab_to_fetch_how} that the the pick-up time is generally longer during lunch and dinner peaks, reaching a maximum of 12.0 minutes at noon on Tuesdays, but overall we do not see very substantial increases during peak hours (which may result from, for example, severe supply constraints in the system). 


\begin{figure}
    \centering
    \includegraphics[width=\FigWidthHOW\linewidth]{Simulation_Results/Meituan/City/meituan_grab_to_fetch_how.png}
    \caption{
    The average amount of time it takes the driver to pick-up the orders after accepting the dispatch from the platform, by hour-of-week. 
    }
    \label{fig:meituan_grab_to_fetch_how}
\end{figure}


Finally, \Cref{fig:meituan_order_to_first_dispatch_how} illustrates the amount of time that elapses, between platform\_order\_time, i.e. when each order is placed by the customer, and first dispatch\_time, i.e. when the order was offered to a delivery driver for the first time.
% 
%
Intuitively, this is the amount of time it takes the platform to start dispatching each order, and the average is around 4 minutes across all orders. 
% 
Despite the fact that it takes an estimated $12$ minutes for the restaurant to prepare the orders, the platform starts to dispatch the orders shortly after they are placed. This is potentially due to the fact that it takes time for drivers to get to the restaurants (see \Cref{fig:meituan_grab_to_fetch_how}), so orders are dispatched before they are ready in order to reduce the total wait time experienced by the customers.  


\begin{figure}
    \centering
    \includegraphics[width=\FigWidthHOW\linewidth]{Simulation_Results/Meituan/City/meituan_order_to_first_dispatch_how.png}
    \caption{
    The average amount of time (minutes) that elapses between (i) when an order is placed by the customer, and (ii) when the order is dispatched to a delivery driver for the first time.
    }
    \label{fig:meituan_order_to_first_dispatch_how}
\end{figure}



\subsection{Spatial Distribution} \label{appx:meituan_spatial_distribution}

We now visualize the distribution of orders in space, and provide the distribution of order volume by origin-destination hexagon pairs. We use resolution-9 hexagons from the h3 package, which is the resolution that performs the best for $\averagedual$ when the matching window is at least $2$ minutes. The average size of each hexagon is  $0.1 \mathrm{km}^2$, and the average edge length is around $0.2 \mathrm{km}$.\footnote{
\url{https://h3geo.org/docs/core-library/restable/}, accessed January 31, 2025. 
} 


First, \Cref{fig:meituan_heatmaps} provides heat-maps of trip volume by order origin and destination, respectively. We can see that the order origins are more concentrated than destinations, which is aligned with the more concentrated locations of restaurants in comparison to residential areas and office buildings. 
%
Note that both distributions are highly skewed. For hexagons with at least one originating order, the median, average, and maximum number of orders are 330, 960, and 17151, respectively. For hexagons that are the destination of at least one order, the median, average, and maximum order count are 71, 327 and 5167, respectively. 


\begin{figure}[t!]
  \centering
  \subcaptionbox{By Origin.%
    \label{fig:meituan_count_by_orig}}[0.49 \textwidth]{\includegraphics[width = \figWidth \textwidth]{Simulation_Results/Meituan/City/meituan_count_by_orig.png}}
  \hfill
  \subcaptionbox{By Destination.%
    \label{fig:meituan_count_by_dest}}[0.49 \textwidth]{\includegraphics[width = \figWidth \textwidth]{Simulation_Results/Meituan/City/meituan_count_by_dest.png}}
  % 
  \caption{Number of orders by order origin hexagon (left) and order destination hexagon (right).
  }
  % 
  \label{fig:meituan_heatmaps}
\end{figure}



Finally, \Cref{fig:meituan_CDF_count_per_OD_pair} presents the distribution order volume by origin and destination hexagon pairs. From the unweighted CDF on the left, we can see that the vast majority of OD pairs have fewer than 100 orders. Out of all OD hexagon pairs with at least one order, the median, average, and maximum number of orders are 3, 8.22, and 808, respectively.
% 
The CDF on the right is weighted by order count.
%
Overall, roughly 14\% of orders are associated with OD pairs with at least 100 orders, and 72.8\% of orders are from OD pairs with at least 10 orders.
%
% 
\begin{figure}
    \centering
    \includegraphics[width=0.75\linewidth]{Simulation_Results/Meituan/City/meituan_CDF_count_per_OD_pair.png}
    \caption{
    Distribution (CDF) of the number of orders per origin-destination hexagon pair (resolution-9), unweighted (left) and weighted by the number of orders (right). 
    }
    \label{fig:meituan_CDF_count_per_OD_pair}
\end{figure}


\section{Additional Experiments}\label{sec: sim_results_extra}

We first report additional performance metrics for settings studied in \Cref{sec:numerical_experiments} of the paper.
%
Additionally, we also consider more general synthetic environments, including two-dimensional locations, non-uniform spatial distributions, and different reward topologies.

\subsection{Match Rate and Saving Fraction}
\label{sec:match_rate}  

We consider in this section two additional performance metrics: the \newterm{match rate}, i.e. the fraction of jobs that were pooled instead of dispatched on their own, and  the \newterm{saving fraction}, i.e. the fraction of the total distance that is reduced by pooling, relative to the total travel distance without any pooling~\citep[see][]{aouad2020dynamic}.

Both regret and reward ratio are performance metrics that compare pooling algorithm relative to the hindsight optimal pooling outcome. 
%
The saving fraction illustrates the benefit of delivery pooling in comparison to the total distance traveled, and is upper bounded by 0.5 (since reward from pooling a job cannot exceed its distance).
%
The match rate, as shown in \Cref{fig:1D_unif_match_rate,fig:meituan_match_rate}, highlights an additional desirable property of $\PB$ and, more generally, of \emph{index-based greedy matching} algorithms: they pool as many jobs as possible.
% 
This is desirable in practice, since drivers who are offered just only one order from a platform may try to pool orders from competing platforms \citep[see e.g.][]{reddit2022grubhub,reddit2023doordash}, leading to poor service reliability for customers.

\subsubsection{Uniform one-dimensional case.} \label{sec: 1D_extra}


\Cref{fig:1D_extra} compares the match rate and the saving fraction achieved for the one-dimensional synthetic environment presented in \Cref{sec:sim_unif_1D}.
%
Recall that all jobs share the same origin at $0$ and destinations are drawn uniformly at random from $[0,1]$. 


%%
\Cref{fig:1D_unif_saving_fraction} shows that the reduction in travel distance can be remarkably high for this setting, with $\OPT$ saving more than 46\% even at low density levels.
%
Since the saving fraction is proportional to the total reward collected by the algorithm, our previous performance comparisons presented in \Cref{sec:numerical_experiments} imply that the saving fraction achieved by $\PB$ is the best among all benchmark algorithms, which translates into over 44\% of reduction in total travel distance.
%
We can also observe that the saving fraction under $\gre$ does not decrease with density.
%
Nonetheless, $\OPT$ improves at a much faster rate, resulting in a relative performance of $\gre$ against $\OPT$ that is decreasing with density (as observed in \Cref{fig:1D}).

\begin{figure}[H]
  \centering
  \subcaptionbox{Average match rate.%
    \label{fig:1D_unif_match_rate}}[0.49 \textwidth]{\includegraphics[width = \figWidth \textwidth]{Simulation_Results/Synthetic/New/1D_Pooling/1D_Common_Origin/match_rate.png}}
  \hfill
  \subcaptionbox{Average saving fraction.%
    \label{fig:1D_unif_saving_fraction}}[0.49 \textwidth]{\includegraphics[width = \figWidth \textwidth]{Simulation_Results/Synthetic/New/1D_Pooling/1D_Common_Origin/saving_rate_all.png}}
  % 
  \caption{Additional metrics in random 1D instances.}
  % 
  \label{fig:1D_extra}
\end{figure}

The match rate in \Cref{fig:1D_unif_match_rate} illustrates that all \emph{index-based greedy matching} algorithms match every one of the $\Njob = 1000$ jobs.
%
This is because (i) the reward from matching any two jobs is non-negative in this 1D setting, and (ii) the total number of jobs is even.
%
On the other hand, $\batching$ matches every job only if $\sojourn+1$ is even, since every time it computes an optimal matching, the \textit{batch size} is $\sojourn+1$ and it dispatches \textit{all} available jobs in pooled trips. 
%
This is not the case for $\rbatching$, since by dispatching only the critical job and its match (if any), the batch size for the next time it computes an optimal matching is potentially different from $\sojourn+1$.
%


\subsubsection{Meituan data.} \label{sec: meituan_extra}
%
\Cref{fig:meituan_extra} provides a comparison of the match rate and the saving fraction for the setting presented in \Cref{sec:sim_meituan}.
%
Recall that as we extend our definition of pooling reward to allow for 2D locations with heterogeneous origins, it is no longer guaranteed to be non-negative.
%
Consequently, both the matching rate and saving fraction of all algorithms are lower than in the one-dimensional setting.


Nonetheless, we can see that if the platform allows a time window of at least $1$ minute for each order before it has to be dispatched, $\PB$ achieves a reduction in distance traveled by over 20\%. At $5$ minutes, this fraction increases to roughly 30\%, which is the best among all tested \emph{practical} heuristics, i.e. excluding $\dual$ and $\OPT$.
%

%
\Cref{fig:meituan_match_rate} shows that greedy algorithms ($\gre$, $\PB$, $\dual$, $\averagedual$) still achieve substantially higher match rates in comparison to $\batching$, $\rbatching$, and $\OPT$.
%
In particular, $\PB$ consistently pools 5-10\% more jobs than $\OPT$.

\begin{figure}%[H]
  \centering
  \subcaptionbox{Average match rate.%
    \label{fig:meituan_match_rate}}[0.49 \textwidth]{\includegraphics[width = \figWidth \textwidth]{Simulation_Results/Meituan/City/match_rate_all.png}}
  \hfill
  \subcaptionbox{Average saving fraction.%
    \label{fig:meituan_saving_fraction}}[0.49 \textwidth]{\includegraphics[width = \figWidth \textwidth]{Simulation_Results/Meituan/City/saving_rate_all.png}}
  % 
  \caption{Match rate and saving fraction for Meituan Order-Level Data.}
  % 
  \label{fig:meituan_extra}
\end{figure}
%

\subsection{Non-Uniform One-Dimensional Case}\label{sec: nonunif_1D}
%
In this section, we show that our experimental results in one-dimensional synthetic data are robust to non-uniform distribution of types. We consider an analogous setting from \Cref{sec:sim_unif_1D}, but now each job type is drawn IID from a distribution Beta(0.5, 2). Overall, the results are consistent, with the main difference being that $\PB$ performs slightly worse than $\averagedual$ for low densities. Nonetheless, $\PB$ ends up outperforming all benchmarks starting from $\sojourn\ge 15$.

\begin{figure}[H]
  \centering
  \subcaptionbox{Average regret.%
    \label{fig:1D_nonunif_regret}}[0.49 \textwidth]{\includegraphics[width = \figWidth \textwidth]{Simulation_Results/Synthetic/New/1D_Pooling/1D_Common_Origin/Non-Uniform/regret.png}}
  \hfill
  \subcaptionbox{Average ratio.%
    \label{fig:1D_nonunif_ratio}}[0.49 \textwidth]{\includegraphics[width = \figWidth \textwidth]{Simulation_Results/Synthetic/New/1D_Pooling/1D_Common_Origin/Non-Uniform/frac_of_OPT.png}}
  % 
  \caption{Comparison of average regret and reward ratio for forecast-agnostic heuristics in non-uniform 1D instances.}
  % 
  \label{fig:1D_nonunif}
\end{figure}

\begin{figure}[H]
  \centering
  \subcaptionbox{Average regret.%
    \label{fig:1D_nonunif_forecastaware_regret}}[0.49 \textwidth]{\includegraphics[width = \figWidth \textwidth]{Simulation_Results/Synthetic/New/1D_Pooling/1D_Common_Origin/Non-Uniform/regret_dual.png}}
  \hfill
  \subcaptionbox{Average ratio.%
    \label{fig:1D_nonunif_forecastaware_ratio}}[0.49 \textwidth]{\includegraphics[width = \figWidth \textwidth]{Simulation_Results/Synthetic/New/1D_Pooling/1D_Common_Origin/Non-Uniform/frac_of_OPT_dual.png}}
  % 
  \caption{Comparison of average regret and reward ratio for forecast-aware heuristics in non-uniform 1D instances.}
  % 
  \label{fig:1D_nonunif_forecastaware}
\end{figure}




\subsection{Two-Dimensional Case with Heterogeneous Origins} \label{sec:sim_unif_1D_2D}


We simulate two-dimensional environments to supplement the simulations on Meituan data.

\subsubsection{Uniform spatial distribution.}

In this section, we consider two-dimensional locations with heterogeneous origins. Specifically, each location, origin and destination, is drawn from a uniform distribution in the unit square. 
%
Overall, we observe that although $\PB$ is far from best at low densities, it still outperforms every other tested benchmark for $\sojourn\ge 25$. We verify in \Cref{sec:sim_meituan} that achieving density values for which $\PB$ outperforms every other benchmark is realistic.


\begin{figure}[H]
  \centering
  \subcaptionbox{Average regret.%
    \label{fig:2Dhet_regret}}[0.49 \textwidth]{\includegraphics[width = \figWidth \textwidth]{Simulation_Results/Synthetic/New/2D_Different_Origins/regret.png}}
  \hfill
  \subcaptionbox{Average ratio.%
    \label{fig:2Dhet_ratio}}[0.49 \textwidth]{\includegraphics[width = \figWidth \textwidth]{Simulation_Results/Synthetic/New/2D_Different_Origins/frac_of_OPT.png}}
  % 
  \caption{Comparison of average regret and reward ratio for forecast-agnostic heuristics in random 2D instances.}
  % 
  \label{fig:2Dhet}
\end{figure}
\begin{figure}[H]
  \centering
  \subcaptionbox{Average regret.%
    \label{fig:2Dhet_forecastaware_regret}}[0.49 \textwidth]{\includegraphics[width = \figWidth \textwidth]{Simulation_Results/Synthetic/New/2D_Different_Origins/regret_dual.png}}
  \hfill
  \subcaptionbox{Average ratio.%
    \label{fig:2Dhet_forecastaware_ratio}}[0.49 \textwidth]{\includegraphics[width = \figWidth \textwidth]{Simulation_Results/Synthetic/New/2D_Different_Origins/frac_of_OPT_dual.png}}
  % 
  \caption{Comparison of average regret and reward ratio for forecast-aware heuristics in random 2D instances.}
  % 
  \label{fig:2Dhet_forecastaware}
\end{figure}


\subsubsection{Non-uniform spatial distribution.}

In addition, we consider an analogous two-dimensional setting with locations drawn from non-uniform distribution Beta(0.5,2). The results remain consistent: $\PB$ outperforms all \emph{practical} heuristics, with the \emph{unrealistic} $\dual$ benchmark being the only one achieving better performance.

\begin{figure}[H]
  \centering
  \subcaptionbox{Average regret.%
    \label{fig:2Dhet_nonunif_regret}}[0.49 \textwidth]{\includegraphics[width = \figWidth \textwidth]{Simulation_Results/Synthetic/New/2D_Different_Origins/Non-Uniform/regret.png}}
  \hfill
  \subcaptionbox{Average ratio.%
    \label{fig:2Dhet_nonunif_ratio}}[0.49 \textwidth]{\includegraphics[width = \figWidth \textwidth]{Simulation_Results/Synthetic/New/2D_Different_Origins/Non-Uniform/frac_of_OPT.png}}
  % 
  \caption{Empirical performance of forecast-agnostic heuristics in random 2D instances.}
  % 
  \label{fig:2Dhet_nonunif}
\end{figure}
\begin{figure}[H]
  \centering
  \subcaptionbox{Average regret.%
    \label{fig:2Dhet_nonunif_forecastaware_regret}}[0.49 \textwidth]{\includegraphics[width = \figWidth \textwidth]{Simulation_Results/Synthetic/New/2D_Different_Origins/Non-Uniform/regret_dual.png}}
  \hfill
  \subcaptionbox{Average ratio.%
    \label{fig:2Dhet_nonunif_forecastaware_ratio}}[0.49 \textwidth]{\includegraphics[width = \figWidth \textwidth]{Simulation_Results/Synthetic/New/2D_Different_Origins/Non-Uniform/frac_of_OPT_dual.png}}
  % 
  \caption{Empirical performance of forecast-aware heuristics in random 2D instances.}
  % 
  \label{fig:2Dhet_nonunif_forecastaware}
\end{figure}


\subsection{Different Reward Topology} \label{sec:sec:sim_reward3}

In this appendix, we present numerical results under the two alternative reward topologies defined in \eqref{eq:defn_reward_B} and \eqref{eq:defn_reward_C}.
%
This illustrates the importance of reward topology in determining the performance of algorithms.
%
As in \Cref{sec:sim_unif_1D}, we generate instances of size $\Njob=1000$, where each type is drawn IID from a uniform distribution on $[0,1]$.
%
Overall, we observe that $\PB$ performs on par with the best realistic benchmarks (i.e., outside $\dual$ and $\OPT$). This suggests that $\PB$ could be a leading heuristic even beyond the reward function $r(\type,\type')=\min\{\type,\type'\}$ from delivery pooling, although here we are still restricting to continuous distributions over (one-dimensional) metric spaces.

\subsubsection{Reward function $\reward(\type,\type')=1-|\type-\type'|$.} 

Recall that in this case the potential of a job is $1/2$, independent of the job type, and thus both $\PB$ and $\gre$ are the same algorithm (see \Cref{sec:reward2}).
%
\Cref{fig:R2} shows their superior performance compared to batching-based heuristics.
%
Moreover, as in \Cref{sec:sim_unif_1D}, forecast-aware heuristics are no better than $\PB$. 

\begin{figure}[H]
  \centering
  \subcaptionbox{Average regret.%
    \label{fig:R2_regret}}[0.49 \textwidth]{\includegraphics[width = \figWidth \textwidth]{Simulation_Results/Synthetic/New/1D_Reward_2/regret_all.png}}
  \hfill
  \subcaptionbox{Average ratio.%
    \label{fig:R2_ratio}}[0.49 \textwidth]{\includegraphics[width = \figWidth \textwidth]{Simulation_Results/Synthetic/New/1D_Reward_2/frac_of_OPT_all.png}}
  % 
  \caption{Comparison of average regret and reward ratio for forecast-agnostic heuristics in random 1D instances under $\reward(\type,\type')=1-|\type-\type'|$.}
  % 
  \label{fig:R2}
\end{figure}


\subsubsection{Reward function $\reward(\type,\type')=|\type-\type'|$.}

Consider $\typespace=[0,1]$ and $\reward(\type,\type') = |\type - \type'| $ (analyzed in \Cref{sec: reward 3}).
%
\Cref{fig:R3} highlights that greedy-like algorithms perform better than batching-based heuristics, with $\PB$ outperforming every practical matching algorithm (i.e., aside from $\dual$ and $\OPT$ that use hindsight information).
% 

\begin{figure}[H]
  \centering
  \subcaptionbox{Average regret.%
    \label{fig:R3_regret}}[0.49 \textwidth]{\includegraphics[width = \figWidth \textwidth]{Simulation_Results/Synthetic/New/1D_Matching/Uniform/regret_all.png}}
  \hfill
  \subcaptionbox{Average ratio.%
    \label{fig:R3_ratio}}[0.49 \textwidth]{\includegraphics[width = \figWidth \textwidth]{Simulation_Results/Synthetic/New/1D_Matching/Uniform/frac_of_OPT_all.png}}
  % 
  \caption{Comparison of average regret and reward ratio for forecast-agnostic heuristics in random 1D instances under $\reward(\type,\type')=|\type-\type'|$.}
  % 
  \label{fig:R3}
\end{figure}

% \end{APPENDICES}
\end{APPENDIX}




%%%%%%%%%%%%%%%%%
\end{document}
%%%%%%%%%%%%%%%%%



 % outcomment this line in Case 2

%If you don't use BiBTex, you can manually itemize references as shown below.

%\bibliographystyle{nonumber}

% \begin{thebibliography}{3}
% \providecommand{\natexlab}[1]{#1}
% \providecommand{\url}[1]{\texttt{#1}}
% \providecommand{\urlprefix}{URL }

% \bibitem[{Smith(2005)}]{smith2005}
% Smith J (2005) Optimal resource allocation in humanitarian logistics.
%   \emph{Journal of Operations Research} 30(2):123--135.
  
% \bibitem[{Jones(2010)}]{jones2010}
% Jones S (2010) Stochastic programming models for humanitarian logistics.
%   \emph{INFORMS Mathematics of Operations Research} 35(4):567--580.

% \bibitem[{Brown(2015)}]{brown2015}
% Brown D (2015) \emph{Introduction to Stochastic Programming} (Springer).

% \end{thebibliography}



% Appendix here
% Options are (1) APPENDIX (with or without general title) or
%             (2) APPENDICES (if it has more than one unrelated sections)
% Outcomment the appropriate case if necessary
%
\begin{APPENDIX}{}
We provide in \Cref{appx:proofs} proofs that are omitted from \Cref{sec:PB} of the paper.
% 
\Cref{sec: interpretation} shows % further details on 
an interpretation of potential in terms of the marginal value of jobs. 
% 
Theoretical guarantees under alternative reward topologies are stated and proved in \Cref{appx:alternative_reward_topologies}. 
%
\Cref{sec: LP} provides details on the LP formulation and dual variables used for the $\dual$ and $\averagedual$ benchmarks. 
%
Finally, we include in \Cref{sec:market_dynamics} high-level descriptions of market-dynamics observed in Meituan data, and in \Cref{sec: sim_results_extra} additional simulation results, both for settings studied in the body of the paper as well as more general settings with 2D locations, non-uniform spatial distributions, and different reward topologies. 

%
%   or
%
% \begin{APPENDICES}
\crefalias{section}{appendix}
\crefalias{subsection}{appendix}
\crefalias{subsubsection}{appendix}

\section{Deferred Proofs} \label{appx:proofs}

\subsection{Proof of \Cref{prop:greedy_linear_lower_bound}} \label{pf:greedy_linear_lower_bound}

Let $0<\eps<1/3$.
% 
Consider an instance $\instance \in [0,1]^\Njob$ such that $\type_{n/2}<\type_{n/2-1}<\ldots<\type_{1}<\eps$ (low-type jobs) and $\type_{n/2+1}=\type_{n/2+2}=\ldots=\type_{\Njob}=1$ (high-type jobs). 
%
Note that because of \Cref{deliverygreedy}, when the first job becomes critical, $\gre$ chooses to match with a high-type job, i.e. $\matchof(1)\in \{n/2+1,\ldots,\Njob\}$, collecting $\reward(\type_1,\type_{\matchof(1)})\le \eps$ independent of the actual choice of $\matchof(1)$.
%
In particular, the second job is not matched and is then the next to become critical.
%
Since there are still high-type jobs to be matched, then by the same argument $\matchof(2)\in \{n/2+1,\ldots,\Njob\}$ and $\reward(\type_2,\type_{\matchof(2)})\le \eps$.
%
By induction, since there are as many high-type jobs as low-type jobs, every job $j \in \{1,\ldots,n/2\}$ is matched with $\matchof(j)\in \{n/2+1,\ldots,\Njob\}$ and $\reward(\type_j,\type_{\matchof(j)}) \le \eps$. Therefore, ${\gre}(\instance,\infty) \le \eps n/2$. 

On the other hand, $\reward(\type_{j},\type_{j+1}) = 1$ for all $j \in \{\Njob/2+1,\ldots,\Njob-1\}$. Thus, an optimal matching solution would always match all $\Njob/2$ high-type jobs together, collecting $\OPT(\instance, \infty) \ge \Njob/4$. 
%
Hence,
%
\[ 
    \regret_\gre(\instance,\infty) = \OPT(\instance,\infty) - \gre(\instance, \infty) \ge \frac{(1-2\eps)}{4}\Njob, 
\]
independent of the tie-breaking rule. Since this inequality holds for all $0<\eps<1/3$, the proof is complete.\Halmos
% 

\subsection{Proof of \Cref{prop:loglowerboundOffline}}
\label{pf:loglowerboundOffline}

We construct a sequence of instances $\instance(k)\in [0,1]^{\Njob(k)}$ with $\Njob(k)=2^{k+3} - 4$ for every integer $k\ge 0$, and show that the regret gain at each iteration, $\regret_\PB(\instance(k+1),\infty)-\regret_\PB(\instance(k),\infty)$ is at least $1/4$. For $k=0$, consider $\instance(0)=(1/2,0,1,0)$. We have that $\OPT(\instance(0),\infty) = 1/2$. In turn, under $\PB$, when the first job becomes critical, all available jobs are equally close, resulting in ties. Assume ties are broken so that the resulting matching is $\{(1,2),(3,4)\}$. Then, $\PB(\instance(0),\infty) = 0$, and thus $\regret_\PB(\instance(0),\infty) = 1/2$.
%
For $k\ge 1$, we inductively construct an instance $\instance(k)$ by carefully adding $2^{k+2}$ new jobs to be processed before the instance considered in the previous step. In particular, we add $2^{k}$ copies of $\instance(0)$, each scaled by $1/2^{k+1}$ and shifted in space so that $\PB$ matches within each copy. 
%

Formally, for $s\in\{0,\ldots,2^{k}-1\}$ and $r\in\{1,2,3,4\}$, define
\[ \instance(k)_{4s+r} = \frac{\instance(0)_r}{2^{k+1}} + \frac{s}{2^k}. \]
Lastly, define $\instance(k)_j = \instance(k-1)_{j-2^{k+2}} $ for $j \in \{ 2^{k+2}+1,\ldots,  \Njob(k) \}$.
%
By construction, for each $s\in \{0,\ldots,2^{k}-1\}$, $\PB$ matches job $4s+1$ with $4s+2$ since it is the closest in space (see \Cref{deliverypotential}), and then matches $4s+3$ with $4s+4$ for the same reason. After all these jobs are matched, the remaining jobs represent $\instance(k-1)$ exactly.
%
On the other hand, $\OPT$ is at least as good as processing $\OPT$ separately on each of the $2^k$ copies, and then on $\instance(k-1)$.
%
Hence, the regret gain $\regret_\PB(\instance(k),\infty) - \regret_\PB(\instance(k-1),\infty)$ is at least the sum of the regret of $\PB$ on each copy.
%
Moreover, note the reward function $\reward(\type,\type') = \min\{\type,\type'\}$ satisfies $\reward(\alpha+\type/\beta,\alpha\type'/\beta) = \reward(\type,\type')/\beta$ for any scalars $\alpha,\beta$ such that $\alpha+\type/\beta,\alpha\type'/\beta\in[0,1]$. 
%
Thus, the regret of $\PB$ on each copy is exactly $\regret_\PB(\instance(0),\infty)/{2^{k+1}}$, and then
%
\[ \regret_\PB(\instance(k),\infty) - \regret_\PB(\instance(k-1),\infty) \ge 2^k\frac{\regret_\PB(\instance(0),\infty)}{2^{k+1}} \ge \frac{1}{4}, \]
%
completing the proof.
\Halmos

\subsection{Proof of \Cref{prop:greedy_linear_lower_bound_dynamic}}
\label{pf:greedy_linear_lower_bound_dynamic}

Note that if $(\sojourn+1)$ is divisible by 4, then for any number of jobs $\Njob$ divisible by $(\sojourn+1)$, we have an exact partition $b=\Njob/(\sojourn+1)\in\{1,2,\ldots\}$ batches, each of size $(\sojourn+1)$.
%
We construct an instance $\instance\in[0,1]^\Njob$ that can be analyzed separately on each batch in the following sense.
%
For each $t=1,\ldots,b$, let $\instance^{(t)} = (\type_{j})_{j\in B_t}\in[0,1]^{(\sojourn+1)}$.
%
It is straightforward to check that $\OPT(\instance,\sojourn) \ge \sum_{t=1}^b \OPT(\instance^{(t)},\infty) $, since the matching solution induced by the offline instances on the right-hand side is feasible in the online setting.
%
Thus, if $\instance$ is such that $\gre(\instance,\sojourn) = \sum_{t=1}^b \gre(\instance^{(t)},\infty) $, then
\[ \OPT(\instance,\sojourn) - \gre(\instance,\sojourn) \ge \sum_{t=1}^b \regret_\gre(\instance^{(t)},\infty). \]
%
Moreover, we construct $\instance$ such that each $\instance^{(t)}\in[0,1]^{(\sojourn+1)}$ resembles the worst-case instance analyzed in \Cref{prop:greedy_linear_lower_bound}.
%
To this end, let $0<\eps<1$. Consider $\instance\in[0,1]^\Njob$ such that for each $t=0,\ldots,b-1$,
%
$\eps/2^{t+1}<\type_{(\sojourn+1)t+(\sojourn+1)/2}<\ldots<\type_{(\sojourn+1)t+1}<\eps/2^t$ (low-type jobs) and $\type_{(\sojourn+1)t+(\sojourn+1)/2+1}=\ldots=\type_{(\sojourn+1)(t+1)} = 1$ (high-type jobs).
%
By construction, when job $1$ becomes critical, the available jobs $k\in A(1)\subseteq B_1$ have types $\type_k < \type_1$ or $\type_k=1$. Then, by \Cref{deliverygreedy}, $\gre$ chooses to match with a high-type job $\matchof(1)\in B_1$.
%
As job $2$ becomes critical next, the new set of available jobs is $A(2)=A(1)\cup\{(\sojourn+1)+1\}\setminus\{\matchof(1)\}$, with $\type_{(\sojourn+1)+1}<\eps/2<\type_2$. Again, $\gre$ chooses to match with a high-type job $\matchof(2)\in B_2$.
%
By repeating this argument inductively, $\gre$ outputs $(C,\{\matchof(j)\}_{j\in C})$ such that $C = \bigcup_{t=1}^{b-1}\{(d+1)t + 1,\ldots, (d+1)t + \sojourn+1)t+(\sojourn+1)/2 \}$ and $\matchof(j) \in B_t\setminus C$ for all $j\in C\cap B_t$. 
%
Thus, $\gre(\instance,\sojourn) = \sum_{t=1}^b \gre(\instance^{(t)},\infty) $, and since $\instance^{(t)}\in[0,1]^{(\sojourn+1)}$ is specified as in the proof of \Cref{prop:greedy_linear_lower_bound}, then  $\regret_\PB(\instance^{(t)},\infty)\le (\sojourn+1)(1-2\eps)/4$. 
%
Hence,
\[ \OPT(\instance,\sojourn) - \gre(\instance,\sojourn) \ge b\frac{(\sojourn+1)(1-2\eps)}{4} = \frac{\Njob}{4}(1-2\eps), \]
%
completing the proof.
\Halmos

\subsection{Proof of \Cref{prop:loglowerboundOnline}}
\label{pf:loglowerboundOnline}

Since $\Njob$ is divisible by $(\sojourn+1)$, we have an exact partition of the set of jobs into $b=n/(\sojourn+1)\in\{1,2,\ldots\}$ batches. 
%
As in \Cref{prop:greedy_linear_lower_bound_dynamic}, we construct an instance $\instance\in[0,1]^\Njob$ that can be analyzed separately on each batch, where we denote $\instance^{(t)}=(\type_j)_{j\in B_t}$ for $t=1,\ldots,b$.
%
By \Cref{prop:loglowerboundOffline}, there exists an instance $\instance^{(0)}\in[0,1]^{(\sojourn+1)}$ such that $\regret_\PB(\instance^{(0)},\infty) \ge (\log(\sojourn+5)-3)/4$.
%
Then, for each $t=1,\ldots,b$, let $\instance^{(t)}=\instance^{(0)}\cdot\frac13$ if $t$ is odd and $\instance^{(t)}=\instance^{(0)}\cdot\frac13+\frac23$ if $t$ is even.
%
Note that since each batch size is $\sojourn+1$ divisible by 4, it is possible to match within batch only. Moreover, by construction, every critical job has a matching candidate within batch, which is closer in space than any job on the next batch. Thus, because of \Cref{deliverypotential}, $\PB$ only matches within batch, and therefore $\PB(\instance,\sojourn) = \sum_{t=1}^b \PB(\instance^{(t)},\infty)$.
%
As in \Cref{prop:greedy_linear_lower_bound_dynamic}, we arrive at
\[ \OPT(\instance,\sojourn) - \PB(\instance,\sojourn) \ge \sum_{t=1}^b \regret_\PB(\instance^{(t)},\infty) = b\frac{ \regret_\PB(\instance^{(0)},\infty)}{3} \ge \frac{\Njob}{3(\sojourn+1)}(\log(\sojourn+5)-3)/4, \]
%
concluding the proof.
\Halmos

\section{Interpretation of Potential}\label{sec: interpretation}

When a job $j$ becomes critical, our potential-based greedy algorithm matches it to an available job $k$ maximizing $r(\theta_j,\theta_k)-p(\theta_k)$, where $p(\theta_k)$ can be interpreted as the opportunity cost of matching job $k$, with the specific definition $p(\theta_k)=\frac12 \sup_{\theta\in\Theta} r(\theta_k,\theta)$.
This is an optimistic measure of opportunity cost because it assumes that job $k$ would otherwise be matched to an "ideal" type $\theta\in\Theta$ maximizing $r(\theta_k,\theta)$ (with half of this ideal reward $\sup_{\theta\in\Theta} r(\theta_k,\theta)$ attributed to job $k$).
We now prove that this ideal reward can indeed be achieved under asymptotically-large market thickness, for a stochastic model under our topology of interest.

\begin{definition}
Let $\instance \in \typespace^\Njob$.
For any job $j\in[\Njob]$, let $\instance^{-j}\in \typespace^{\Njob-1}$ be the same instance with the exception that job $j$ is not present. Meanwhile, let $\instance^{+j}\in \typespace^{\Njob+1}$ be the same instance with the exception that an additional copy of job $j$ is present.
Consider the following definitions.
\begin{enumerate}
\item Marginal Loss: $\marginalloss_j(\instance) = \OPT(\instance) - \OPT(\instance^{-j})$
\item Marginal Gain: $\marginalgain_j(\instance) = \OPT(\instance^{+j}) - \OPT(\instance)$
\end{enumerate}
\end{definition}

Note that definitions $\marginalloss_j(\instance),\marginalgain_j(\instance)$ are based solely on offline matching, and we will use them as our definitions of opportunity cost if the future was known.  One could alternatively use shadow prices from the LP relaxation of the offline matching problem, but we note that the LP is not integral.  In either case, there is no ideal definition of opportunity cost that is guaranteed to lead to the optimal offline solution in matching problems \citep[see][]{cohen2016invisible}.

We now establish the following \namecref{lem:marginal_as_interval} to help analyze the opportunity costs $\marginalloss_j(\instance),\marginalgain_j(\instance)$.


\begin{lemma}\label{lem:marginal_as_interval}
Let $\typespace=[0,1]$ and $\reward(\type,\type') = \min\{\type,\type'\}$. Consider an instance $\instance\in\typespace^n$ and relabel the indices to satisfy $\type_1 \ge \type_2 \ge \ldots \ge \type_n$. Then,
\begin{align*}
\marginalloss_j(\instance)
&=\sum_{k\ge j,k\ \mathrm{even}}(\theta_k-\theta_{k+1}), 
\\ \marginalgain_j(\instance)
&=\sum_{k\ge j,k\ \mathrm{odd}}(\theta_k-\theta_{k+1}),
\end{align*}
where we consider $\theta_{n+1}=0$.
\end{lemma}
%
\proof{Proof.}
    Let $\instance \in \typespace^\Njob$ such that $\type_1 \ge \type_2 \ge \ldots \ge \type_n$. 
    %
    Then, $\OPT(\instance) =\sum_{k\ \mathrm{even}}\theta_k$, and moreover
    \begin{align*}
    \OPT(\instance^{-j}) &=\sum_{k<j, k\ \mathrm{even}}\theta_k+\sum_{k>j, k\ \mathrm{odd}}\theta_k
    \\ \OPT(\instance^{+j}) &=\theta_j+\OPT(\instance^{-j})
    \end{align*}
    % 
    Therefore,
    % 
    \begin{align*}
    \marginalloss_j(\instance)
    &=\sum_{k\ge j,k\ \mathrm{even}}\theta_k
    -\sum_{k>j,k\ \mathrm{odd}}\theta_k
    =\sum_{k\ge j,k\ \mathrm{even}}(\theta_k-\theta_{k+1})
    \\ \marginalgain_j(\instance)
    &=\theta_j-\sum_{k\ge j,k\ \mathrm{even}}(\theta_k-\theta_{k+1})
    =\sum_{k\ge j,k\ \mathrm{odd}}(\theta_k-\theta_{k+1})
    \end{align*}
    % 
    completing the proof.
\Halmos\endproof

Note that because $\OPT(\instance^{+j}) = \OPT(\instance^{-j}) + \type_j$ and $\potential(\type_j) = \reward(\type_j,\type_j)/2$ for this reward function, we immediately get the following \namecref{cor: marginal_average}.

\begin{corollary}\label{cor: marginal_average}
        If $\typespace=[0,1]$ and $\reward(\type,\type') = \min\{\type,\type'\}$, then for all jobs $j$,
    \[ \potential(\type_j) = \frac{\marginalloss_j(\instance) + \marginalgain_j(\instance)}{2}. \]
\end{corollary}

We are now ready to prove our main result about the interpretation of potential, that the true opportunity costs $\marginalloss_j(\instance),\marginalgain_j(\instance)$ concentrate around $p(\theta_j)$ in a random uniform instance, assuming the market is sufficiently thick.  We without loss consider job $j=1$ and fix its type $\theta_1$.

\begin{theorem} \label{thm:interpretation}
Let $\typespace=[0,1]$ and $\reward(\type,\type') = \min\{\type,\type'\}$.
Fix $\theta_1\in\typespace$ and suppose $\theta_2,\ldots,\theta_n$ are drawn IID from the uniform distribution over $[0,1]$, forming a random instance $\instance\in\typespace^n$, for some $n\ge 2$. Then, both $\mathbb{E}[\marginalloss_1(\instance)]$ and $\mathbb{E}[\marginalgain_1(\instance)]$ are within $O(1/n)$ of $\potential(\type_1)$ and moreover $\mathrm{Var}(\marginalloss_1(\instance))=\mathrm{Var}(\marginalgain_1(\instance)) = O\left(\frac{1}{n}\right)$.
\end{theorem}
%
\proof{Proof.}
We prove the statement only for $\marginalloss_1(\instance)$, and the analogous result for $\marginalgain_1(\instance)$ follows from \Cref{cor: marginal_average}.
%
If $\type_1=0$, then $\marginalloss_1(\instance)=0$ for any $\instance$, coinciding with $\potential(\type_1)=0$. Then, for the remainder of the proof, assume that $\type_1>0$.
%
Consider the random variable $N=|\{j:\theta_j<\theta_1\}|$, which counts the number of points between $0$ and $\theta_1$ and has distribution $\text{Binom}(\Njob-1,\type_1)$.
%
These $N$ points divide the interval $[0,\theta_1]$ into $N+1$ intervals with total length $\type_1$.
%
From \Cref{lem:marginal_as_interval}, the marginal loss $\marginalloss_1(\instance)$ is determined by computing the total length of a subset of these intervals. % Maybe add figure?
%
It is known that if $N$ random variables are drawn independently from $\text{Unif}[0,1]$, then the joint distribution of the induced interval lengths is $\text{Dirichlet}(1,1,\ldots,1)$ with $N+1$ parameters all equal to 1 (i.e., drawn uniformly from the simplex).
%
In particular, since the Dirichlet distribution is symmetric, the sum of any $k$ of these intervals is equal in distribution to the $k$-th smallest sample ($k$-th order statistic) of the $N$ uniform random variables, whose distribution is known to be $\text{Beta}(k,N+1-k)$.
%
Thus, conditional on $N$, with $N\ge 1$, since the distribution of each of the $N$ jobs to the left of $\type_1$ is $\text{Unif}[0,\type_1]$, the distribution of $\marginalloss_1(\instance)/\type_1$ is $\text{Beta}(k_L,N+1-k_L)$,
where $k_L$ is either $\lceil\frac{N+1}2\rceil$ or $\floor{\frac{N+1}2}$.
%
To be precise, for $N\ge 1$, $k_L=\floor {N/2}+1=\lceil\frac{N+1}2\rceil$ if $n$ is even, and $k_L=\ceil{N/2}=\floor{\frac{N+1}2}$ if $n$ is odd.
%
Moreover, if $N=0$, then $\marginalloss_1(\instance)=\type_1$ if $n$ is even, and $\marginalgain_1(\instance)=0$ if $n$ is odd.
%
Hence,
\begin{align*}
\mathbb{E}[\marginalloss_1(\type) \mid N] 
&= \type_1\frac{k_L}{N+1}
% \label{eq: cond_exp}
\end{align*}
for all $N\in\{0,\ldots,n-1\}$.
%
Since $|k_L-(N+1)/2|\le 1$, then $|\mathbb{E}[\marginalloss_1(\type) \mid N]-\type_1/2|\le \type_1/(N+1)$, thus
% 
\begin{align*}
&\left|\mathbb{E}[\marginalloss_1(\type)]-\frac{\type_1}{2}\right| 
\le \type_1\mathbb{E}\left[\frac{1}{N+1}\right] \\
&\qquad = \type_1\sum_{k=0}^{n-1} \frac{1}{k+1}\binom{n-1}{k}\type_1^k(1-\type_1)^{n-1-k} \\
&\qquad = \frac{\type_1}{n}\sum_{k=0}^{n-1} \binom{n}{k+1}\type_1^k(1-\type_1)^{n-1-k} \\
&\qquad = \frac{1}{n}\sum_{k=1}^{n} \binom{n}{k}\type_1^k(1-\type_1)^{n-k} \\
&\qquad = \frac{1-(1-\type_1)^{n}}{n} = O\left(\frac{1}{n}\right).
\end{align*}
%
This completes the proof of the statement about $\mathbb{E}[\marginalloss_1(\instance)]$.

For the statement about variance, we know $\mathrm{Var}(\marginalloss_1(\type)) = \mathbb{E}[\mathrm{Var}(\marginalloss_1(\type) \mid N)] + \mathrm{Var}(\mathbb{E}[\marginalloss_1(\type)\mid N])$ by the law of total variance. For the first term, we have $\mathrm{Var}(\marginalloss_1(\type) \mid N) = \type_1^2\frac{k_L(N+1-k_L)}{(N+1)^2(N+2)}\indicator\{N\ge 1\}\le \frac{\type_1^2}{N+1}$, and then from the previous argument $\mathbb{E}[\mathrm{Var}(\marginalloss_1(\type) \mid N)]=O(1/n)$.
%
For the second term, 
\begin{align*}
\mathrm{Var}(\mathbb{E}[\marginalloss_1(\type)\mid N]) 
&= \mathrm{Var}\left(\mathbb{E}[\marginalloss_1(\type)\mid N] - \frac{\type_1}{2}\right) \\
&\le \mathbb{E} \left[\left(\mathbb{E}[\marginalloss_1(\type)\mid N] - \frac{\type_1}{2}\right)^2\right] \\
% 
&\le \mathbb{E}\left[\frac{\type_1^2}{(N+1)^2}\right] \\
&\le \type_1^2\mathbb{E}\left[\frac{1}{N+1}\right] = O\left(\frac{1}{n}\right).
\end{align*}
Thus, $\mathrm{Var}(\marginalloss_1(\instance)) = O\left(1/n\right)$, completing the proof.
\Halmos\endproof
%%%%%%%%

\section{Algorithmic Performance under Alternative Reward Topologies} \label{appx:alternative_reward_topologies}


In this section, we study the performance of both the naive greedy algorithm $\gre$ and potential-based greedy algorithm $\PB$, under the two alternative reward topologies defined in \eqref{eq:defn_reward_B} and \eqref{eq:defn_reward_C}.

\subsection{Reward Function $\reward(\type,\type')=1-|\type-\type'|$}
\phantomsection
\label{sec:reward2}
%
This reward function represents the goal to minimize the total distance between matched jobs.
%
In this case, for any $\type\in[0,1]$, the potential of a job of type $\type$ is $\potential(\type)=1/2$ constant across job types. Thus, the description of $\PB$ is equivalent to $\gre$, since their index functions differ only in an additive constant.

\begin{remark}\label{rem:rewardB}
    Under the 1-dimensional type space $\typespace=[0,1]$ and the reward function $\reward(\type,\type')=1-|\type-\type'|$, both the naive greedy and potential-based greedy algorithm generate the same output $(C,\matchof(\cdot))$ as described in \Cref{alg:dynamic}. In particular, both algorithms always match each critical job to an available job that is the closest in space.
\end{remark}

Moreover, we show that their performance can be reduced to the one of $\PB$ under reward function $\reward(\type,\type')=\min\{\type,\type'\}$.
%
In particular, we first show that when $\sojourn=\infty$, the regret is at least logarithmic in the number of jobs.
\begin{proposition}\label{prop:loglowerboundOffline_rewardB}     
    Under reward function $\reward(\type,\type')=1-|\type-\type'|$, if $\Njob=2^{k+3}-4$ for some integer $k\ge0$, then there exists an instance $\instance\in[0,1]^\Njob$ for which $\regret_\PB(\instance,\infty) = \regret_\gre(\instance,\infty) \ge (\log_2(\Njob+4)-3)/2 $.
\end{proposition}

\proof{Proof.}
    %
    We prove the statement for $\gre$, since $\regret_\PB(\instance,\infty) = \regret_\gre(\instance,\infty)$ follows from \Cref{rem:rewardB}.
    %
    Let $\Njob=2^{k+3}-4$, for $k\in\{0,1,\ldots\}$. From \Cref{prop:loglowerboundOffline}, there exists an instance $\instance\in[0,1]^\Njob$ such that the regret of $\PB$ under reward topology $\reward(\type,\type')=\min\{\type,\type'\}$, which we denote by $R$ from here on, is at least $(\log_2(\Njob+4)-3)/4$.
    %
    We show that under $\reward(\type,\type')=1-|\type-\type'|$, we have $\regret_\gre(\instance,\infty) \ge 2R$.
    %
    First, note that since $|\type-\type'| = \type + \type' - 2\min\{\type,\type'\}$, we have
    %
    \[ \gre(\instance,\infty) 
         = \sum_{j \in C} 1 - |\type_j-\type_{m(j)}|  
         = |C| - \sum_{j\in C} (\type_j +\type_{\matchof(j)})  + 2\sum_{j\in C} \min\{\type,\type'\}. 
    \]
    % 
    Similarly, recall that $\matchset_\OPT$ is the set of matches in the hindsight optimal solution. Then,
    \[ \OPT(\instance,\infty) = |\matchset_\OPT| - \sum_{ (j,k)\in\matchset_\OPT} (\type_j +\type_{k})  + 2\sum_{(j,k)\in\matchset_\OPT} \min\{\type_j,\type_{k}\}. \]
    %
    Since $\reward(\type,\type')\ge 0$ for any $\type,\type'\in \typespace$ and the total number of jobs $\Njob$ is even, both $\gre$ and $\OPT$ match every job. Hence, $|C|=|\matchset_\OPT|=n/2$, and moreover
    \[ \sum_{(j,k)\in\matchset_\OPT} (\type_j +\type_{k}) = \sum_{j\in C} (\type_j +\type_{\matchof(j)}). \]
    Thus,
    \[ \OPT(\instance,\infty) - \gre(\instance,\infty) \ge 2\left( \sum_{(j,k)\in\matchset_\OPT} \min\{\type_j,\type_{k}\} - \sum_{j\in C} \min\{\type_j,\type_{\matchof(j)}\} \right). \]
    %
    We claim that the term in large parenthesis is exactly the regret of $\PB$ under reward topology $\reward(\type,\type')=\min\{\type,\type'\}$, which we denote by $R$.
    %
    Indeed, from \Cref{rem:rewardB}, $(C,\matchof(\cdot))$ coincides with the resulting matching of $\PB$ under $\reward(\type,\type')=\min\{\type,\type'\}$, since it is specified to always match to the closest job (\Cref{deliverypotential}).
    %
    On the other hand, since all jobs are matched, the set of matches $\matchset_\OPT$ must induce disjoint intervals, which coincides with the optimal matching solution under reward function $\reward(\type,\type') = \min\{\type,\type'\}$.
    %
    Thus, by \Cref{prop:loglowerboundOffline}, we get $ \regret_\gre(\instance,\infty) \ge 2R \ge (\log_2(\Njob+4)-3)/2$, completing the proof.
\Halmos\endproof

We now extend this result to the online setting ($\sojourn<n$).

\begin{proposition} \label{prop:loglowerboundOnline_rewardB}
    Under reward topology $\reward(\type,\type') = 1-|\type-\type'|$, if $\sojourn+1 = 2^{k+3}-4$ for some integer $k\ge 0$, then for any number of jobs $\Njob$ divisible by $(\sojourn+1)$, there exists an instance $\instance \in [0,1]^\Njob$ for which $\regret_\PB(\instance,\sojourn) = \regret_\gre(\instance,\sojourn) \ge \frac{\Njob}{3(\sojourn+1)}(\log_2(\sojourn+5)-3)/2$.
\end{proposition}

\proof{Proof.}
    % 
    This proof is analogous to the proof of \Cref{prop:loglowerboundOnline}, since (i) there exists an offline instance that achieves the desired regret for $\sojourn+1=n$ (\Cref{prop:loglowerboundOffline_rewardB}), and (ii) the algorithm matches a critical job with the closest available job (\Cref{rem:rewardB}).
    %
    We repeat the proof below for completeness.

    We have an exact partition of the set of jobs into $b=n/(\sojourn+1)\in\{1,2,\ldots\}$ batches, and construct an instance $\instance\in[0,1]^\Njob$ that can be analyzed separately on each batch, where we denote $\instance^{(t)}=(\type_j)_{j\in B_t}$ for $t=1,\ldots,b$.
    %
    By \Cref{prop:loglowerboundOffline_rewardB}, there exists an instance $\instance^{(0)}\in[0,1]^{(\sojourn+1)}$ such that $\regret_\PB(\instance^{(0)},\infty) \ge (\log(\sojourn+5)-3)/2$.
    %
    Then, for each $t=1,\ldots,b$, let $\instance^{(t)}=\instance^{(0)}\cdot\frac13$ if $t$ is odd and $\instance^{(t)}=\instance^{(0)}\cdot\frac13+\frac23$ if $t$ is even.
    %
    Note that since each batch size is $\sojourn+1$ divisible by 4, it is possible to match within batch only. Moreover, by construction, every critical job has a matching candidate within batch, which is closer in space than any job on the next batch. Thus, because of \Cref{rem:rewardB}, $\PB$ only matches within batch, and therefore $\PB(\instance,\sojourn) = \sum_{t=1}^b \PB(\instance^{(t)},\infty)$.
    %
    As in \Cref{prop:greedy_linear_lower_bound_dynamic}, we arrive at
    \[ \OPT(\instance,\sojourn) - \PB(\instance,\sojourn) \ge \sum_{t=1}^b \regret_\PB(\instance^{(t)},\infty) = b\frac{ \regret_\PB(\instance^{(0)},\infty)}{3} \ge \frac{\Njob}{3(\sojourn+1)}(\log(\sojourn+5)-3)/2, \]
    %
    concluding the proof.
\Halmos\endproof

Finally, we directly show the tight regret upper bound (up to constants) as an immediate consequence of the proof of \Cref{thm: dynamic}. Note that if $d+1=n$, then this yields the result for offline matching.

\begin{theorem}\label{thm:rewardB_dynamic}
Under reward topology $\reward(\type,\type') = 1-|\type-\type'|$, we have $\regret_\PB(\instance,\sojourn) = \regret_\gre(\instance,\sojourn) \le 1/2 + (\frac{\Njob}{d+1}+1)(1+\log(\sojourn+2))$ for any $\Njob$, and any instance $\instance\in[0,1]^\Njob$.
\end{theorem}


\proof{Proof.}
Let $\instance\in \typespace^\Njob$ and $\sojourn\ge 1$. Let $(C,\matchof(\cdot))$ be the output of $\gre$ on $\instance$. We have
\begin{align*}
\OPT(\instance,\sojourn)-\gre(\instance,\sojourn) 
& \le \sum_{j=1}^\Njob \potential(\type_j) - \gre(\instance,\sojourn)  \nonumber \\
& \le \sup_{\type\in\typespace}\potential(\type) + \sum_{j \in C} \left( p(\theta_j) + p(\theta_{m(j)}) - r(\theta_j,\theta_{m(j)}) \right) \nonumber \\ 
& = \frac12 +  \sum_{j \in C} \left( \frac12 + \frac12 - (1-|\theta_j-\theta_{m(j)}|) \right) \nonumber \\
& = \frac12 + \sum_{j \in C} |\type_j-\type_{\matchof(j)}|, 
\end{align*}
The second term is exactly the sum of distances between jobs matched by $\gre$. Recall that $(C,\matchof(\cdot))$ coincides with the matching output of $\PB$ under $\reward(\type,\type')=\min\{\type,\type'\}$, both are specified to always match with the closest job (\Cref{deliverypotential}, \Cref{rem:rewardB}). Thus, recalling \eqref{eq: distance_dynamic} in the proof of \Cref{thm: dynamic}, we have
\[ \sum_{j \in C} |\type_j-\type_{\matchof(j)}| = \sum_{t=1}^b \sum_{j \in C\cap B_t} |\type_j-\type_{\matchof(j)}|\le \left(\frac{n}{d+1}+1 \right)\left(1+\log(d+2)\right),\]
and the result follows.
\Halmos\endproof

\subsection{Reward Function $\reward(\type,\type')=|\type-\type'|$}\label{sec: reward 3}
%
Under this reward topology, it is worst to match two jobs of the same type, contrasting the two settings studied in \Cref{sec:PB} and \Cref{sec:reward2}.
%
In this case, the potential of a job type $\type\in[0,1]$ is $\potential(\type)=\max\{\type,1-\type\}/2$.
%
We first show that \emph{any} index-based matching policy must suffer regret constant regret per job in the offline setting.
%
\begin{proposition}\label{prop:linearLB_offline_rewardC}
    Under reward topology $\reward(\type,\type')=|\type-\type'|$, when the number of jobs $\Njob$ is divisible by 4, for any index-based greedy matching algorithm $\ALG$, there exists an instance $\instance\in[0,1]^\Njob$ for which $\regret_\ALG(\instance,\infty)\ge c_\ALG \Njob$, for some $c_\ALG\in (0,1)$. In particular, $c_\gre = c_\PB = 1/4$.
\end{proposition}

\proof{Proof.}
Let $\Njob$ be divisible by 4.
%
Let $\ALG$ be an index-based greedy matching algorithm specified by index function $\indexf:[0,1]^2\to \R$.
%
Define $\type^{c} = \sup\{ \type\in[0,1] : \indexf(\type,1)\ge \indexf(\type,0) \}$.
%
We first construct an instance in the case $\type^{c}>0$ (e.g. $\type^{c}=1/2$ for $\gre$ and $\PB$), and analyze the alternative case separately in the end.
%
Consider an instance $\instance\in\typespace^\Njob$ such that $\type_1,\ldots,\type_{\Njob/4} = \type^{c}$, $\type_{{\Njob/4}+1},\ldots,\type_{{3\Njob/4}}=0$, and $\type_{{3\Njob/4}+1},\ldots,\type_{\Njob}=1$. 
%
It is straightforward to check that $\OPT(\instance,\infty) \ge (1+\type^c){\Njob/4}$.
%
On the other hand, $\ALG$ chooses to match every $j=1,\ldots,n/4$ to $\matchof(j)\in \{3n/4+1,\ldots,n\}$.\footnote{In the presence of ties, such an output is achieved by some tie-breaking rule.}
%
Then,
\[ \ALG(\instance,\infty) = 0 + \sum_{j=1}^{\Njob/4} \reward(\type_j,\type_{3\Njob/4+j}) = \frac{\Njob(1-\type^c)}{4}. \]
%
and thus $\regret_\ALG(\instance,\infty) = \OPT(\instance,\infty) - \ALG(\instance,\infty) \ge {{\Njob\type^{c}}}/{2},$ proving the statement with $c_\ALG = \type^{c}/2$.

%
Now, if $\type^{c}\le 0$, then $\indexf(\type,0) < \indexf(\type,1)$ for all $\type\in(0,1]$.
%
In particular, for any $\eps>0$, consider the instance $\type_1,\ldots,\type_{\Njob/4} = \eps$, $\type_{{\Njob/4}+1},\ldots,\type_{{3\Njob/4}}=1$, and $\type_{{3\Njob/4}+1},\ldots,\type_{\Njob}=0$.
%
Then, $\OPT(\instance,\infty) \ge (1+(1-\eps))n/4$, and $\ALG(\instance,\infty) = \eps n/4$.
%
Hence, $\regret_\ALG(\instance,\infty)\ge (1-\eps)/2$, and the statement is true with $c_\ALG = (1-\eps)/2$ for every $\eps>0$.
\Halmos\endproof

The following result shows that, under $\PB$ and $\gre$, this lower bound extends to the online setting, losing a factor of 2.

\begin{proposition}\label{prop:linearLB_dynamic_rewardC}
    Under reward topology $\reward(\type,\type')=|\type-\type'|$, if $(\sojourn+1)$ is divisible by 4, then for any number of jobs $\Njob$ divisible by $2(\sojourn+1)$, there exists an instance $\instance \in [0,1]^\Njob$ for which $\regret_\gre(\instance,\sojourn)=\regret_\PB(\instance,\sojourn) \ge n/8$.
\end{proposition}

\proof{Proof.}
If $\Njob$ is divisible by $2(\sojourn+1)$, we have an exact partition of the set of jobs into $b=\Njob/(\sojourn+1)\in\{2,4,\ldots\}$ batches, defined as $B_{t+1} = \{t(\sojourn+1)+1,\ldots, (t+1)(\sojourn+1)\}$ for $t = 0, \ldots, b-1$.
%
We construct an instance $\instance\in[0,1]^\Njob$ for which $\gre$ (and $\PB$) "resets" every two batches, in the sense that all jobs in these two batches are matched among them.
%
To formally specify the construction, first fix $\eps\in(0,1/2)$. Define, for $t=0,\ldots,b/2-1$
%
\[
\type_{2t(\sojourn+1)+j} =
\begin{cases}
\frac12 & \text{if } j \le \frac{\sojourn+1}{4}, \\
\eps & \text{if } \frac{\sojourn+1}{4} < j \le \frac{3(\sojourn+1)}{4}, \\
1 & \text{if } \frac{3(\sojourn+1)}{4} < j \le \sojourn+1 \text{, or } \frac{3(\sojourn+1)}{2} < j \le 2(\sojourn+1), \\
0 & \text{if } \sojourn+1 < j \le \frac{3(\sojourn+1)}{2}.
\end{cases}
\]
%
Consider the notation $\instance^{(0)}=(\type_j)_{j\le 2(\sojourn+1)}$.
%
It is easy to check that
\[ 
\OPT(\instance^{(0)},\infty) = \left(\frac12-\eps\right)\frac{\sojourn+1}{4} + (1-\eps)\frac{\sojourn+1}{4} + \frac{\sojourn+1}{2} =  \left(\frac{7-4\eps}{8}\right)(\sojourn+1),
\]
and then $\OPT(\instance,\sojourn) \ge \frac{b}{2}\OPT(\instance^{(0)},\infty) = \frac{7-4\eps}{16}\Njob $.
%
On the other hand,
\[ 
\gre(\instance^{(0)},\infty) = \frac12\frac{\sojourn+1}{4} + \eps\frac{\sojourn+1}{4} + (1-\eps)\frac{\sojourn+1}{4} + \frac{\sojourn+1}{4} =  \frac58(\sojourn+1).
\]
%
We claim that $\gre(\instance,\sojourn) = \frac{b}{2}\gre(\instance^{(0)},\infty) = \frac{5}{16}\Njob $, and thus $\regret_\gre(\instance,\sojourn)\ge \frac{1-2\eps}{8}\Njob $.
%
Indeed, let $(C,\matchof(\cdot))$ be the output of $\gre$. Running $\gre$ on $\instance$ is as follows:
%
\begin{enumerate}
    \item For every job $j \le (\sojourn+1)/4$ of type $\type_j=1/2$, and then $3(\sojourn+1)/4 < \matchof(j) \le (\sojourn+1)$ with $\reward(\type_j,\type_{\matchof(j)})=1/2$.
    \item Then, for every job $\frac{\sojourn+1}{4} < j \le \frac{(\sojourn+1)}{2}$, the only available jobs are of type $\type\in\{0,\eps\}$, and then $ \sojourn+1 < \matchof(j) \le \frac{3(\sojourn+1)}{2}$ with $\reward(\type_j,\type_{\matchof(j)})=\eps$. 
    \item In turn, jobs $\frac{(\sojourn+1)}{2} < j \le \frac{3(\sojourn+1)}{4}$ get to observe jobs of type $\type=1$, and then $ \frac{3(\sojourn+1)}{2} < \matchof(j) \le 2(\sojourn+1)$ with $\reward(\type_j,\type_{\matchof(j)})=1-\eps$.
    \item Finally, the key observation is that the all remaining available jobs $j\le \frac{3(\sojourn+1)}{2}$ of type $\type_j=0$ observe jobs of type $1$, as well as jobs on a "third" batch. However, by construction every job $2(\sojourn+1) < k \le 2(\sojourn+1) + \frac{3(\sojourn+1)}{2} $ are of type $\type_k\in\{1/2,\eps\}$, and then each job $j\le \frac{3(\sojourn+1)}{2}$ is matched (within batch) to $ \frac{3(\sojourn+1)}{2} < \matchof(j) \le 2(\sojourn+1)$ with $\reward(\type_j,\type_{\matchof(j)})=1$.
\end{enumerate}

The proof for $\PB$ is analogous, since the matching decisions coincide with $\gre$ on this instance. To see this, note that
\begin{itemize}
    \item $\indexf_\PB(1/2,1) = \reward(1/2,1) - \potential(1) = 1/2 - 1/2 = 0$,
    \item $\indexf_\PB(1/2,\eps)=\reward(1/2,\eps) - \potential(\eps) = 1/2 - \eps - (1-\eps)/{2} = - \eps/2 < \indexf_\PB(1/2,1)$,
    \item $\indexf_\PB(\eps,0)=\reward(\eps,0) - \potential(0) = \eps - 1/2$,
    \item $\indexf_\PB(\eps,\eps)=\reward(\eps,\eps) - \potential(\eps) = 0 - (1-\eps)/{2}= \eps/{2}-1/2< \indexf_\PB(\eps,0)$,
    \item $\indexf_\PB(\eps,1)=\reward(\eps,1) - \potential(1) = 1 - \eps - 1/2 = 1/2 - \eps$,
    \item $\indexf_\PB(\eps,1/2)=\reward(\eps,1/2) - \potential(1/2) = 1/2 - \eps - 1/4 = 1/4 - \eps < \indexf_\PB(\eps,1) $, and
    \item $\indexf_\PB(0,\eps)= \eps - (1-\eps)/2 < \indexf_\PB(0,1/2) < \indexf_\PB(0,1)$.
\end{itemize}
Thus, running $\PB$ on $\instance$ follows the same four steps described by $\gre$, yielding the same regret.
\Halmos\endproof


\section{Linear Relaxation and Dual Variables}\label{sec: LP}

We present in this section the linear relaxation of the IP that describes the hindsight optimum defined in \eqref{eq: OPT}. Given market density $\sojourn\ge 1$ and full information of the sequence of arrivals $\instance\in\typespace$, consider the following linear program (LP)
%
\begin{maxi}
{x}{ \sum_{j,k : j\neq k, |j-k|\le \sojourn} x_{jk} \reward(\type_j,\type_k)}
{\label{eq: LP}}{}
\addConstraint{ \sum_{k:j\neq k} x_{jk}}{\le 1,}{j\in [\Njob]}
\addConstraint{ x_{jk}}{\ge 0,\quad }{ j,k\in [\Njob], j\neq k,}
\end{maxi}
%
and its dual
%
\begin{mini}
{\lambda}{ \sum_{j\in[\Njob]} \lambda_{j} }
{\label{eq: Dual-LP}}{}
\addConstraint{ \lambda_j + \lambda_k}{ \ge \reward(\type_j,\type_k),\quad}{j,k\in [\Njob], j\neq k, |j-k|\le \sojourn}
\addConstraint{ \lambda_{j} }{\ge 0 ,}{ j\in [\Njob].}
\end{mini}

It is known that the LP relaxation is integral only for bipartite graphs. However, we observe in \Cref{fig: LP_gap} that the integrality gap is small, and thus the value of the LP can be reasonably used as a benchmark.

\begin{figure}[H]
  \centering
  \subcaptionbox{Average Total Reward.%
    \label{fig:reward}}[0.49 \textwidth]{\includegraphics[width = \figWidth \textwidth]{Simulation_Results/Synthetic/New/1D_Pooling/1D_Common_Origin/Forecast-Agnostic/value.png}}
  \hfill
  \subcaptionbox{Fraction of LP value achieved by $\OPT$.%
    \label{fig:LP_Gap}}[0.49 \textwidth]{\includegraphics[width = \figWidth \textwidth]{Simulation_Results/Synthetic/New/1D_Pooling/1D_Common_Origin/Forecast-Agnostic/integrality_gap.png}}
  % 
  \caption{Comparison with LP value.
  }
  % 
  \label{fig: LP_gap}
\end{figure}

Also, note that the potential can always be used as a feasible (but not necessarily optimal) solution to the dual LP.


\begin{proposition}
    Let $\instance\in\typespace^\Njob$ and $\sojourn\ge 1$. Then, $\{\potential(\type_j)=\frac{1}{2}\sup_{\type'\in\typespace}\reward(\type_j,\type'):j\in[\Njob]\}$ is a feasible solution of the dual LP \eqref{eq: Dual-LP}.
\end{proposition}


\proof{Proof.}
    Non-negativity holds since $\reward\ge 0$. Moreover, for any $j,k$ we have
    \[ \reward(\type,\type')\le \frac{1}{2}\reward(\type,\type')+\frac{1}{2}\reward(\type,\type') \le \potential(\type)+\potential(\type'), \]
    verifying the dual feasibility constraints.
\Halmos\endproof

Interestingly, when we extract the shadow prices from the 400 instances considered as historical data in \Cref{sec:sim_unif_1D}, we can see that these concentrate around potential, as density increases (\Cref{fig: dual_converge}). This is aligned with our results in \Cref{sec: interpretation}.

\begin{figure}[H]
  \centering
  \subcaptionbox{Density $\sojourn=5$.%
    \label{fig:dual_conv_d5}}[0.4 \textwidth]{\includegraphics[width = \figWidth \textwidth]{Simulation_Results/Synthetic/New/1D_Pooling/1D_Common_Origin/Forecast-Aware/duals_at_d5.png}}
  % \hfill
  \hspace{1em}
  % 
  \subcaptionbox{Density $\sojourn=30$.%
    \label{fig:dual_conv_d30}}[0.4 \textwidth]{\includegraphics[width = \figWidth \textwidth]{Simulation_Results/Synthetic/New/1D_Pooling/1D_Common_Origin/Forecast-Aware/duals_at_d30.png}}
  % 
  \caption{Scatter plot of shadow prices, average, and potential.}
  % 
  \label{fig: dual_converge}
\end{figure}

\section{Meituan Data - Market Dynamics} \label{sec:market_dynamics}

In this appendix, we briefly describe the Meituan dataset made public by the INFORMS TSL Data-Driven Challenge, and provide high-level descriptions of the market-dynamics. 


\subsection{Temporal Dynamics} \label{appx:meituan_temporal_dynamics}

First, \Cref{fig:meituan_orders_per_hour} provides the average number of orders per hour by \emph{hour-of-week}.
Hour-of-week 0 corresponds to midnight to 1am on Mondays, and hour-of-week 1 corresponds to 1am to 2am on Mondays, and so on. 
%
The gray bars indicate the peak lunch hours (10:30am-1:30pm), while the green bars indicate peak dinner hours (5pm-8pm).
%
We can see that the platform observes highest order volume during lunch time, averaging around 200 order per minute between noon and 1pm.   

\newcommand{\FigWidthHOW}{0.85}

\begin{figure}
    \centering
    \includegraphics[width=\FigWidthHOW\linewidth]{Simulation_Results/Meituan/City/meituan_number_of_orders_how.png}
    \caption{Average number of orders by hour-of-week in Meituan data.}
    \label{fig:meituan_orders_per_hour}
\end{figure}



\Cref{fig:meituan_pooled_orders_per_hour} in \Cref{sec:intro} of the paper illustrates the fraction of orders placed during each hour-of-week that is pooled with at least one other order. 
%
This is computed by combining the order-level data (which contains the timestamp of each order and the order ID) and the ``wave level'' data, where each wave corresponds to a list of orders that were pooled and assigned to the same driver.\footnote{
For more details on waves, see \url{https://github.com/meituan/Meituan-INFORMS-TSL-Research-Challenge/blob/main/tsl_meituan_2024_data_report_20241015.pdf}, accessed January 30, 2024. Each wave may contain more than two orders, though some orders may have been dropped off by the driver before some other orders were picked up.   
}
The orders that are not pooled with any other order correspond to waves with only one order, and \Cref{fig:meituan_pooled_orders_per_hour} illustrates 1 minus this fraction of orders that are not pooled. 


\Cref{fig:meituan_grab_to_fetch_how} illustrates the average amount of time it takes the drivers to pick up the orders, after they received and then accepted the order from the platform. Specifically, this is the amount of time that elapses between (i) the grab\_time, i.e. when the order is accepted by a driver, and (ii) the fetch\_time, i.e. when the order is picked-up by the driver.
%
The average pick-up time across all orders is 10.1 minutes. 
%
We see from \Cref{fig:meituan_grab_to_fetch_how} that the the pick-up time is generally longer during lunch and dinner peaks, reaching a maximum of 12.0 minutes at noon on Tuesdays, but overall we do not see very substantial increases during peak hours (which may result from, for example, severe supply constraints in the system). 


\begin{figure}
    \centering
    \includegraphics[width=\FigWidthHOW\linewidth]{Simulation_Results/Meituan/City/meituan_grab_to_fetch_how.png}
    \caption{
    The average amount of time it takes the driver to pick-up the orders after accepting the dispatch from the platform, by hour-of-week. 
    }
    \label{fig:meituan_grab_to_fetch_how}
\end{figure}


Finally, \Cref{fig:meituan_order_to_first_dispatch_how} illustrates the amount of time that elapses, between platform\_order\_time, i.e. when each order is placed by the customer, and first dispatch\_time, i.e. when the order was offered to a delivery driver for the first time.
% 
%
Intuitively, this is the amount of time it takes the platform to start dispatching each order, and the average is around 4 minutes across all orders. 
% 
Despite the fact that it takes an estimated $12$ minutes for the restaurant to prepare the orders, the platform starts to dispatch the orders shortly after they are placed. This is potentially due to the fact that it takes time for drivers to get to the restaurants (see \Cref{fig:meituan_grab_to_fetch_how}), so orders are dispatched before they are ready in order to reduce the total wait time experienced by the customers.  


\begin{figure}
    \centering
    \includegraphics[width=\FigWidthHOW\linewidth]{Simulation_Results/Meituan/City/meituan_order_to_first_dispatch_how.png}
    \caption{
    The average amount of time (minutes) that elapses between (i) when an order is placed by the customer, and (ii) when the order is dispatched to a delivery driver for the first time.
    }
    \label{fig:meituan_order_to_first_dispatch_how}
\end{figure}



\subsection{Spatial Distribution} \label{appx:meituan_spatial_distribution}

We now visualize the distribution of orders in space, and provide the distribution of order volume by origin-destination hexagon pairs. We use resolution-9 hexagons from the h3 package, which is the resolution that performs the best for $\averagedual$ when the matching window is at least $2$ minutes. The average size of each hexagon is  $0.1 \mathrm{km}^2$, and the average edge length is around $0.2 \mathrm{km}$.\footnote{
\url{https://h3geo.org/docs/core-library/restable/}, accessed January 31, 2025. 
} 


First, \Cref{fig:meituan_heatmaps} provides heat-maps of trip volume by order origin and destination, respectively. We can see that the order origins are more concentrated than destinations, which is aligned with the more concentrated locations of restaurants in comparison to residential areas and office buildings. 
%
Note that both distributions are highly skewed. For hexagons with at least one originating order, the median, average, and maximum number of orders are 330, 960, and 17151, respectively. For hexagons that are the destination of at least one order, the median, average, and maximum order count are 71, 327 and 5167, respectively. 


\begin{figure}[t!]
  \centering
  \subcaptionbox{By Origin.%
    \label{fig:meituan_count_by_orig}}[0.49 \textwidth]{\includegraphics[width = \figWidth \textwidth]{Simulation_Results/Meituan/City/meituan_count_by_orig.png}}
  \hfill
  \subcaptionbox{By Destination.%
    \label{fig:meituan_count_by_dest}}[0.49 \textwidth]{\includegraphics[width = \figWidth \textwidth]{Simulation_Results/Meituan/City/meituan_count_by_dest.png}}
  % 
  \caption{Number of orders by order origin hexagon (left) and order destination hexagon (right).
  }
  % 
  \label{fig:meituan_heatmaps}
\end{figure}



Finally, \Cref{fig:meituan_CDF_count_per_OD_pair} presents the distribution order volume by origin and destination hexagon pairs. From the unweighted CDF on the left, we can see that the vast majority of OD pairs have fewer than 100 orders. Out of all OD hexagon pairs with at least one order, the median, average, and maximum number of orders are 3, 8.22, and 808, respectively.
% 
The CDF on the right is weighted by order count.
%
Overall, roughly 14\% of orders are associated with OD pairs with at least 100 orders, and 72.8\% of orders are from OD pairs with at least 10 orders.
%
% 
\begin{figure}
    \centering
    \includegraphics[width=0.75\linewidth]{Simulation_Results/Meituan/City/meituan_CDF_count_per_OD_pair.png}
    \caption{
    Distribution (CDF) of the number of orders per origin-destination hexagon pair (resolution-9), unweighted (left) and weighted by the number of orders (right). 
    }
    \label{fig:meituan_CDF_count_per_OD_pair}
\end{figure}


\section{Additional Experiments}\label{sec: sim_results_extra}

We first report additional performance metrics for settings studied in \Cref{sec:numerical_experiments} of the paper.
%
Additionally, we also consider more general synthetic environments, including two-dimensional locations, non-uniform spatial distributions, and different reward topologies.

\subsection{Match Rate and Saving Fraction}
\label{sec:match_rate}  

We consider in this section two additional performance metrics: the \newterm{match rate}, i.e. the fraction of jobs that were pooled instead of dispatched on their own, and  the \newterm{saving fraction}, i.e. the fraction of the total distance that is reduced by pooling, relative to the total travel distance without any pooling~\citep[see][]{aouad2020dynamic}.

Both regret and reward ratio are performance metrics that compare pooling algorithm relative to the hindsight optimal pooling outcome. 
%
The saving fraction illustrates the benefit of delivery pooling in comparison to the total distance traveled, and is upper bounded by 0.5 (since reward from pooling a job cannot exceed its distance).
%
The match rate, as shown in \Cref{fig:1D_unif_match_rate,fig:meituan_match_rate}, highlights an additional desirable property of $\PB$ and, more generally, of \emph{index-based greedy matching} algorithms: they pool as many jobs as possible.
% 
This is desirable in practice, since drivers who are offered just only one order from a platform may try to pool orders from competing platforms \citep[see e.g.][]{reddit2022grubhub,reddit2023doordash}, leading to poor service reliability for customers.

\subsubsection{Uniform one-dimensional case.} \label{sec: 1D_extra}


\Cref{fig:1D_extra} compares the match rate and the saving fraction achieved for the one-dimensional synthetic environment presented in \Cref{sec:sim_unif_1D}.
%
Recall that all jobs share the same origin at $0$ and destinations are drawn uniformly at random from $[0,1]$. 


%%
\Cref{fig:1D_unif_saving_fraction} shows that the reduction in travel distance can be remarkably high for this setting, with $\OPT$ saving more than 46\% even at low density levels.
%
Since the saving fraction is proportional to the total reward collected by the algorithm, our previous performance comparisons presented in \Cref{sec:numerical_experiments} imply that the saving fraction achieved by $\PB$ is the best among all benchmark algorithms, which translates into over 44\% of reduction in total travel distance.
%
We can also observe that the saving fraction under $\gre$ does not decrease with density.
%
Nonetheless, $\OPT$ improves at a much faster rate, resulting in a relative performance of $\gre$ against $\OPT$ that is decreasing with density (as observed in \Cref{fig:1D}).

\begin{figure}[H]
  \centering
  \subcaptionbox{Average match rate.%
    \label{fig:1D_unif_match_rate}}[0.49 \textwidth]{\includegraphics[width = \figWidth \textwidth]{Simulation_Results/Synthetic/New/1D_Pooling/1D_Common_Origin/match_rate.png}}
  \hfill
  \subcaptionbox{Average saving fraction.%
    \label{fig:1D_unif_saving_fraction}}[0.49 \textwidth]{\includegraphics[width = \figWidth \textwidth]{Simulation_Results/Synthetic/New/1D_Pooling/1D_Common_Origin/saving_rate_all.png}}
  % 
  \caption{Additional metrics in random 1D instances.}
  % 
  \label{fig:1D_extra}
\end{figure}

The match rate in \Cref{fig:1D_unif_match_rate} illustrates that all \emph{index-based greedy matching} algorithms match every one of the $\Njob = 1000$ jobs.
%
This is because (i) the reward from matching any two jobs is non-negative in this 1D setting, and (ii) the total number of jobs is even.
%
On the other hand, $\batching$ matches every job only if $\sojourn+1$ is even, since every time it computes an optimal matching, the \textit{batch size} is $\sojourn+1$ and it dispatches \textit{all} available jobs in pooled trips. 
%
This is not the case for $\rbatching$, since by dispatching only the critical job and its match (if any), the batch size for the next time it computes an optimal matching is potentially different from $\sojourn+1$.
%


\subsubsection{Meituan data.} \label{sec: meituan_extra}
%
\Cref{fig:meituan_extra} provides a comparison of the match rate and the saving fraction for the setting presented in \Cref{sec:sim_meituan}.
%
Recall that as we extend our definition of pooling reward to allow for 2D locations with heterogeneous origins, it is no longer guaranteed to be non-negative.
%
Consequently, both the matching rate and saving fraction of all algorithms are lower than in the one-dimensional setting.


Nonetheless, we can see that if the platform allows a time window of at least $1$ minute for each order before it has to be dispatched, $\PB$ achieves a reduction in distance traveled by over 20\%. At $5$ minutes, this fraction increases to roughly 30\%, which is the best among all tested \emph{practical} heuristics, i.e. excluding $\dual$ and $\OPT$.
%

%
\Cref{fig:meituan_match_rate} shows that greedy algorithms ($\gre$, $\PB$, $\dual$, $\averagedual$) still achieve substantially higher match rates in comparison to $\batching$, $\rbatching$, and $\OPT$.
%
In particular, $\PB$ consistently pools 5-10\% more jobs than $\OPT$.

\begin{figure}%[H]
  \centering
  \subcaptionbox{Average match rate.%
    \label{fig:meituan_match_rate}}[0.49 \textwidth]{\includegraphics[width = \figWidth \textwidth]{Simulation_Results/Meituan/City/match_rate_all.png}}
  \hfill
  \subcaptionbox{Average saving fraction.%
    \label{fig:meituan_saving_fraction}}[0.49 \textwidth]{\includegraphics[width = \figWidth \textwidth]{Simulation_Results/Meituan/City/saving_rate_all.png}}
  % 
  \caption{Match rate and saving fraction for Meituan Order-Level Data.}
  % 
  \label{fig:meituan_extra}
\end{figure}
%

\subsection{Non-Uniform One-Dimensional Case}\label{sec: nonunif_1D}
%
In this section, we show that our experimental results in one-dimensional synthetic data are robust to non-uniform distribution of types. We consider an analogous setting from \Cref{sec:sim_unif_1D}, but now each job type is drawn IID from a distribution Beta(0.5, 2). Overall, the results are consistent, with the main difference being that $\PB$ performs slightly worse than $\averagedual$ for low densities. Nonetheless, $\PB$ ends up outperforming all benchmarks starting from $\sojourn\ge 15$.

\begin{figure}[H]
  \centering
  \subcaptionbox{Average regret.%
    \label{fig:1D_nonunif_regret}}[0.49 \textwidth]{\includegraphics[width = \figWidth \textwidth]{Simulation_Results/Synthetic/New/1D_Pooling/1D_Common_Origin/Non-Uniform/regret.png}}
  \hfill
  \subcaptionbox{Average ratio.%
    \label{fig:1D_nonunif_ratio}}[0.49 \textwidth]{\includegraphics[width = \figWidth \textwidth]{Simulation_Results/Synthetic/New/1D_Pooling/1D_Common_Origin/Non-Uniform/frac_of_OPT.png}}
  % 
  \caption{Comparison of average regret and reward ratio for forecast-agnostic heuristics in non-uniform 1D instances.}
  % 
  \label{fig:1D_nonunif}
\end{figure}

\begin{figure}[H]
  \centering
  \subcaptionbox{Average regret.%
    \label{fig:1D_nonunif_forecastaware_regret}}[0.49 \textwidth]{\includegraphics[width = \figWidth \textwidth]{Simulation_Results/Synthetic/New/1D_Pooling/1D_Common_Origin/Non-Uniform/regret_dual.png}}
  \hfill
  \subcaptionbox{Average ratio.%
    \label{fig:1D_nonunif_forecastaware_ratio}}[0.49 \textwidth]{\includegraphics[width = \figWidth \textwidth]{Simulation_Results/Synthetic/New/1D_Pooling/1D_Common_Origin/Non-Uniform/frac_of_OPT_dual.png}}
  % 
  \caption{Comparison of average regret and reward ratio for forecast-aware heuristics in non-uniform 1D instances.}
  % 
  \label{fig:1D_nonunif_forecastaware}
\end{figure}




\subsection{Two-Dimensional Case with Heterogeneous Origins} \label{sec:sim_unif_1D_2D}


We simulate two-dimensional environments to supplement the simulations on Meituan data.

\subsubsection{Uniform spatial distribution.}

In this section, we consider two-dimensional locations with heterogeneous origins. Specifically, each location, origin and destination, is drawn from a uniform distribution in the unit square. 
%
Overall, we observe that although $\PB$ is far from best at low densities, it still outperforms every other tested benchmark for $\sojourn\ge 25$. We verify in \Cref{sec:sim_meituan} that achieving density values for which $\PB$ outperforms every other benchmark is realistic.


\begin{figure}[H]
  \centering
  \subcaptionbox{Average regret.%
    \label{fig:2Dhet_regret}}[0.49 \textwidth]{\includegraphics[width = \figWidth \textwidth]{Simulation_Results/Synthetic/New/2D_Different_Origins/regret.png}}
  \hfill
  \subcaptionbox{Average ratio.%
    \label{fig:2Dhet_ratio}}[0.49 \textwidth]{\includegraphics[width = \figWidth \textwidth]{Simulation_Results/Synthetic/New/2D_Different_Origins/frac_of_OPT.png}}
  % 
  \caption{Comparison of average regret and reward ratio for forecast-agnostic heuristics in random 2D instances.}
  % 
  \label{fig:2Dhet}
\end{figure}
\begin{figure}[H]
  \centering
  \subcaptionbox{Average regret.%
    \label{fig:2Dhet_forecastaware_regret}}[0.49 \textwidth]{\includegraphics[width = \figWidth \textwidth]{Simulation_Results/Synthetic/New/2D_Different_Origins/regret_dual.png}}
  \hfill
  \subcaptionbox{Average ratio.%
    \label{fig:2Dhet_forecastaware_ratio}}[0.49 \textwidth]{\includegraphics[width = \figWidth \textwidth]{Simulation_Results/Synthetic/New/2D_Different_Origins/frac_of_OPT_dual.png}}
  % 
  \caption{Comparison of average regret and reward ratio for forecast-aware heuristics in random 2D instances.}
  % 
  \label{fig:2Dhet_forecastaware}
\end{figure}


\subsubsection{Non-uniform spatial distribution.}

In addition, we consider an analogous two-dimensional setting with locations drawn from non-uniform distribution Beta(0.5,2). The results remain consistent: $\PB$ outperforms all \emph{practical} heuristics, with the \emph{unrealistic} $\dual$ benchmark being the only one achieving better performance.

\begin{figure}[H]
  \centering
  \subcaptionbox{Average regret.%
    \label{fig:2Dhet_nonunif_regret}}[0.49 \textwidth]{\includegraphics[width = \figWidth \textwidth]{Simulation_Results/Synthetic/New/2D_Different_Origins/Non-Uniform/regret.png}}
  \hfill
  \subcaptionbox{Average ratio.%
    \label{fig:2Dhet_nonunif_ratio}}[0.49 \textwidth]{\includegraphics[width = \figWidth \textwidth]{Simulation_Results/Synthetic/New/2D_Different_Origins/Non-Uniform/frac_of_OPT.png}}
  % 
  \caption{Empirical performance of forecast-agnostic heuristics in random 2D instances.}
  % 
  \label{fig:2Dhet_nonunif}
\end{figure}
\begin{figure}[H]
  \centering
  \subcaptionbox{Average regret.%
    \label{fig:2Dhet_nonunif_forecastaware_regret}}[0.49 \textwidth]{\includegraphics[width = \figWidth \textwidth]{Simulation_Results/Synthetic/New/2D_Different_Origins/Non-Uniform/regret_dual.png}}
  \hfill
  \subcaptionbox{Average ratio.%
    \label{fig:2Dhet_nonunif_forecastaware_ratio}}[0.49 \textwidth]{\includegraphics[width = \figWidth \textwidth]{Simulation_Results/Synthetic/New/2D_Different_Origins/Non-Uniform/frac_of_OPT_dual.png}}
  % 
  \caption{Empirical performance of forecast-aware heuristics in random 2D instances.}
  % 
  \label{fig:2Dhet_nonunif_forecastaware}
\end{figure}


\subsection{Different Reward Topology} \label{sec:sec:sim_reward3}

In this appendix, we present numerical results under the two alternative reward topologies defined in \eqref{eq:defn_reward_B} and \eqref{eq:defn_reward_C}.
%
This illustrates the importance of reward topology in determining the performance of algorithms.
%
As in \Cref{sec:sim_unif_1D}, we generate instances of size $\Njob=1000$, where each type is drawn IID from a uniform distribution on $[0,1]$.
%
Overall, we observe that $\PB$ performs on par with the best realistic benchmarks (i.e., outside $\dual$ and $\OPT$). This suggests that $\PB$ could be a leading heuristic even beyond the reward function $r(\type,\type')=\min\{\type,\type'\}$ from delivery pooling, although here we are still restricting to continuous distributions over (one-dimensional) metric spaces.

\subsubsection{Reward function $\reward(\type,\type')=1-|\type-\type'|$.} 

Recall that in this case the potential of a job is $1/2$, independent of the job type, and thus both $\PB$ and $\gre$ are the same algorithm (see \Cref{sec:reward2}).
%
\Cref{fig:R2} shows their superior performance compared to batching-based heuristics.
%
Moreover, as in \Cref{sec:sim_unif_1D}, forecast-aware heuristics are no better than $\PB$. 

\begin{figure}[H]
  \centering
  \subcaptionbox{Average regret.%
    \label{fig:R2_regret}}[0.49 \textwidth]{\includegraphics[width = \figWidth \textwidth]{Simulation_Results/Synthetic/New/1D_Reward_2/regret_all.png}}
  \hfill
  \subcaptionbox{Average ratio.%
    \label{fig:R2_ratio}}[0.49 \textwidth]{\includegraphics[width = \figWidth \textwidth]{Simulation_Results/Synthetic/New/1D_Reward_2/frac_of_OPT_all.png}}
  % 
  \caption{Comparison of average regret and reward ratio for forecast-agnostic heuristics in random 1D instances under $\reward(\type,\type')=1-|\type-\type'|$.}
  % 
  \label{fig:R2}
\end{figure}


\subsubsection{Reward function $\reward(\type,\type')=|\type-\type'|$.}

Consider $\typespace=[0,1]$ and $\reward(\type,\type') = |\type - \type'| $ (analyzed in \Cref{sec: reward 3}).
%
\Cref{fig:R3} highlights that greedy-like algorithms perform better than batching-based heuristics, with $\PB$ outperforming every practical matching algorithm (i.e., aside from $\dual$ and $\OPT$ that use hindsight information).
% 

\begin{figure}[H]
  \centering
  \subcaptionbox{Average regret.%
    \label{fig:R3_regret}}[0.49 \textwidth]{\includegraphics[width = \figWidth \textwidth]{Simulation_Results/Synthetic/New/1D_Matching/Uniform/regret_all.png}}
  \hfill
  \subcaptionbox{Average ratio.%
    \label{fig:R3_ratio}}[0.49 \textwidth]{\includegraphics[width = \figWidth \textwidth]{Simulation_Results/Synthetic/New/1D_Matching/Uniform/frac_of_OPT_all.png}}
  % 
  \caption{Comparison of average regret and reward ratio for forecast-agnostic heuristics in random 1D instances under $\reward(\type,\type')=|\type-\type'|$.}
  % 
  \label{fig:R3}
\end{figure}

% \end{APPENDICES}
\end{APPENDIX}




%%%%%%%%%%%%%%%%%
\end{document}
%%%%%%%%%%%%%%%%%



 % outcomment this line in Case 2

%If you don't use BiBTex, you can manually itemize references as shown below.

%\bibliographystyle{nonumber}

% \begin{thebibliography}{3}
% \providecommand{\natexlab}[1]{#1}
% \providecommand{\url}[1]{\texttt{#1}}
% \providecommand{\urlprefix}{URL }

% \bibitem[{Smith(2005)}]{smith2005}
% Smith J (2005) Optimal resource allocation in humanitarian logistics.
%   \emph{Journal of Operations Research} 30(2):123--135.
  
% \bibitem[{Jones(2010)}]{jones2010}
% Jones S (2010) Stochastic programming models for humanitarian logistics.
%   \emph{INFORMS Mathematics of Operations Research} 35(4):567--580.

% \bibitem[{Brown(2015)}]{brown2015}
% Brown D (2015) \emph{Introduction to Stochastic Programming} (Springer).

% \end{thebibliography}



% Appendix here
% Options are (1) APPENDIX (with or without general title) or
%             (2) APPENDICES (if it has more than one unrelated sections)
% Outcomment the appropriate case if necessary
%
\begin{APPENDIX}{}
We provide in \Cref{appx:proofs} proofs that are omitted from \Cref{sec:PB} of the paper.
% 
\Cref{sec: interpretation} shows % further details on 
an interpretation of potential in terms of the marginal value of jobs. 
% 
Theoretical guarantees under alternative reward topologies are stated and proved in \Cref{appx:alternative_reward_topologies}. 
%
\Cref{sec: LP} provides details on the LP formulation and dual variables used for the $\dual$ and $\averagedual$ benchmarks. 
%
Finally, we include in \Cref{sec:market_dynamics} high-level descriptions of market-dynamics observed in Meituan data, and in \Cref{sec: sim_results_extra} additional simulation results, both for settings studied in the body of the paper as well as more general settings with 2D locations, non-uniform spatial distributions, and different reward topologies. 

%
%   or
%
% \begin{APPENDICES}
\crefalias{section}{appendix}
\crefalias{subsection}{appendix}
\crefalias{subsubsection}{appendix}

\section{Deferred Proofs} \label{appx:proofs}

\subsection{Proof of \Cref{prop:greedy_linear_lower_bound}} \label{pf:greedy_linear_lower_bound}

Let $0<\eps<1/3$.
% 
Consider an instance $\instance \in [0,1]^\Njob$ such that $\type_{n/2}<\type_{n/2-1}<\ldots<\type_{1}<\eps$ (low-type jobs) and $\type_{n/2+1}=\type_{n/2+2}=\ldots=\type_{\Njob}=1$ (high-type jobs). 
%
Note that because of \Cref{deliverygreedy}, when the first job becomes critical, $\gre$ chooses to match with a high-type job, i.e. $\matchof(1)\in \{n/2+1,\ldots,\Njob\}$, collecting $\reward(\type_1,\type_{\matchof(1)})\le \eps$ independent of the actual choice of $\matchof(1)$.
%
In particular, the second job is not matched and is then the next to become critical.
%
Since there are still high-type jobs to be matched, then by the same argument $\matchof(2)\in \{n/2+1,\ldots,\Njob\}$ and $\reward(\type_2,\type_{\matchof(2)})\le \eps$.
%
By induction, since there are as many high-type jobs as low-type jobs, every job $j \in \{1,\ldots,n/2\}$ is matched with $\matchof(j)\in \{n/2+1,\ldots,\Njob\}$ and $\reward(\type_j,\type_{\matchof(j)}) \le \eps$. Therefore, ${\gre}(\instance,\infty) \le \eps n/2$. 

On the other hand, $\reward(\type_{j},\type_{j+1}) = 1$ for all $j \in \{\Njob/2+1,\ldots,\Njob-1\}$. Thus, an optimal matching solution would always match all $\Njob/2$ high-type jobs together, collecting $\OPT(\instance, \infty) \ge \Njob/4$. 
%
Hence,
%
\[ 
    \regret_\gre(\instance,\infty) = \OPT(\instance,\infty) - \gre(\instance, \infty) \ge \frac{(1-2\eps)}{4}\Njob, 
\]
independent of the tie-breaking rule. Since this inequality holds for all $0<\eps<1/3$, the proof is complete.\Halmos
% 

\subsection{Proof of \Cref{prop:loglowerboundOffline}}
\label{pf:loglowerboundOffline}

We construct a sequence of instances $\instance(k)\in [0,1]^{\Njob(k)}$ with $\Njob(k)=2^{k+3} - 4$ for every integer $k\ge 0$, and show that the regret gain at each iteration, $\regret_\PB(\instance(k+1),\infty)-\regret_\PB(\instance(k),\infty)$ is at least $1/4$. For $k=0$, consider $\instance(0)=(1/2,0,1,0)$. We have that $\OPT(\instance(0),\infty) = 1/2$. In turn, under $\PB$, when the first job becomes critical, all available jobs are equally close, resulting in ties. Assume ties are broken so that the resulting matching is $\{(1,2),(3,4)\}$. Then, $\PB(\instance(0),\infty) = 0$, and thus $\regret_\PB(\instance(0),\infty) = 1/2$.
%
For $k\ge 1$, we inductively construct an instance $\instance(k)$ by carefully adding $2^{k+2}$ new jobs to be processed before the instance considered in the previous step. In particular, we add $2^{k}$ copies of $\instance(0)$, each scaled by $1/2^{k+1}$ and shifted in space so that $\PB$ matches within each copy. 
%

Formally, for $s\in\{0,\ldots,2^{k}-1\}$ and $r\in\{1,2,3,4\}$, define
\[ \instance(k)_{4s+r} = \frac{\instance(0)_r}{2^{k+1}} + \frac{s}{2^k}. \]
Lastly, define $\instance(k)_j = \instance(k-1)_{j-2^{k+2}} $ for $j \in \{ 2^{k+2}+1,\ldots,  \Njob(k) \}$.
%
By construction, for each $s\in \{0,\ldots,2^{k}-1\}$, $\PB$ matches job $4s+1$ with $4s+2$ since it is the closest in space (see \Cref{deliverypotential}), and then matches $4s+3$ with $4s+4$ for the same reason. After all these jobs are matched, the remaining jobs represent $\instance(k-1)$ exactly.
%
On the other hand, $\OPT$ is at least as good as processing $\OPT$ separately on each of the $2^k$ copies, and then on $\instance(k-1)$.
%
Hence, the regret gain $\regret_\PB(\instance(k),\infty) - \regret_\PB(\instance(k-1),\infty)$ is at least the sum of the regret of $\PB$ on each copy.
%
Moreover, note the reward function $\reward(\type,\type') = \min\{\type,\type'\}$ satisfies $\reward(\alpha+\type/\beta,\alpha\type'/\beta) = \reward(\type,\type')/\beta$ for any scalars $\alpha,\beta$ such that $\alpha+\type/\beta,\alpha\type'/\beta\in[0,1]$. 
%
Thus, the regret of $\PB$ on each copy is exactly $\regret_\PB(\instance(0),\infty)/{2^{k+1}}$, and then
%
\[ \regret_\PB(\instance(k),\infty) - \regret_\PB(\instance(k-1),\infty) \ge 2^k\frac{\regret_\PB(\instance(0),\infty)}{2^{k+1}} \ge \frac{1}{4}, \]
%
completing the proof.
\Halmos

\subsection{Proof of \Cref{prop:greedy_linear_lower_bound_dynamic}}
\label{pf:greedy_linear_lower_bound_dynamic}

Note that if $(\sojourn+1)$ is divisible by 4, then for any number of jobs $\Njob$ divisible by $(\sojourn+1)$, we have an exact partition $b=\Njob/(\sojourn+1)\in\{1,2,\ldots\}$ batches, each of size $(\sojourn+1)$.
%
We construct an instance $\instance\in[0,1]^\Njob$ that can be analyzed separately on each batch in the following sense.
%
For each $t=1,\ldots,b$, let $\instance^{(t)} = (\type_{j})_{j\in B_t}\in[0,1]^{(\sojourn+1)}$.
%
It is straightforward to check that $\OPT(\instance,\sojourn) \ge \sum_{t=1}^b \OPT(\instance^{(t)},\infty) $, since the matching solution induced by the offline instances on the right-hand side is feasible in the online setting.
%
Thus, if $\instance$ is such that $\gre(\instance,\sojourn) = \sum_{t=1}^b \gre(\instance^{(t)},\infty) $, then
\[ \OPT(\instance,\sojourn) - \gre(\instance,\sojourn) \ge \sum_{t=1}^b \regret_\gre(\instance^{(t)},\infty). \]
%
Moreover, we construct $\instance$ such that each $\instance^{(t)}\in[0,1]^{(\sojourn+1)}$ resembles the worst-case instance analyzed in \Cref{prop:greedy_linear_lower_bound}.
%
To this end, let $0<\eps<1$. Consider $\instance\in[0,1]^\Njob$ such that for each $t=0,\ldots,b-1$,
%
$\eps/2^{t+1}<\type_{(\sojourn+1)t+(\sojourn+1)/2}<\ldots<\type_{(\sojourn+1)t+1}<\eps/2^t$ (low-type jobs) and $\type_{(\sojourn+1)t+(\sojourn+1)/2+1}=\ldots=\type_{(\sojourn+1)(t+1)} = 1$ (high-type jobs).
%
By construction, when job $1$ becomes critical, the available jobs $k\in A(1)\subseteq B_1$ have types $\type_k < \type_1$ or $\type_k=1$. Then, by \Cref{deliverygreedy}, $\gre$ chooses to match with a high-type job $\matchof(1)\in B_1$.
%
As job $2$ becomes critical next, the new set of available jobs is $A(2)=A(1)\cup\{(\sojourn+1)+1\}\setminus\{\matchof(1)\}$, with $\type_{(\sojourn+1)+1}<\eps/2<\type_2$. Again, $\gre$ chooses to match with a high-type job $\matchof(2)\in B_2$.
%
By repeating this argument inductively, $\gre$ outputs $(C,\{\matchof(j)\}_{j\in C})$ such that $C = \bigcup_{t=1}^{b-1}\{(d+1)t + 1,\ldots, (d+1)t + \sojourn+1)t+(\sojourn+1)/2 \}$ and $\matchof(j) \in B_t\setminus C$ for all $j\in C\cap B_t$. 
%
Thus, $\gre(\instance,\sojourn) = \sum_{t=1}^b \gre(\instance^{(t)},\infty) $, and since $\instance^{(t)}\in[0,1]^{(\sojourn+1)}$ is specified as in the proof of \Cref{prop:greedy_linear_lower_bound}, then  $\regret_\PB(\instance^{(t)},\infty)\le (\sojourn+1)(1-2\eps)/4$. 
%
Hence,
\[ \OPT(\instance,\sojourn) - \gre(\instance,\sojourn) \ge b\frac{(\sojourn+1)(1-2\eps)}{4} = \frac{\Njob}{4}(1-2\eps), \]
%
completing the proof.
\Halmos

\subsection{Proof of \Cref{prop:loglowerboundOnline}}
\label{pf:loglowerboundOnline}

Since $\Njob$ is divisible by $(\sojourn+1)$, we have an exact partition of the set of jobs into $b=n/(\sojourn+1)\in\{1,2,\ldots\}$ batches. 
%
As in \Cref{prop:greedy_linear_lower_bound_dynamic}, we construct an instance $\instance\in[0,1]^\Njob$ that can be analyzed separately on each batch, where we denote $\instance^{(t)}=(\type_j)_{j\in B_t}$ for $t=1,\ldots,b$.
%
By \Cref{prop:loglowerboundOffline}, there exists an instance $\instance^{(0)}\in[0,1]^{(\sojourn+1)}$ such that $\regret_\PB(\instance^{(0)},\infty) \ge (\log(\sojourn+5)-3)/4$.
%
Then, for each $t=1,\ldots,b$, let $\instance^{(t)}=\instance^{(0)}\cdot\frac13$ if $t$ is odd and $\instance^{(t)}=\instance^{(0)}\cdot\frac13+\frac23$ if $t$ is even.
%
Note that since each batch size is $\sojourn+1$ divisible by 4, it is possible to match within batch only. Moreover, by construction, every critical job has a matching candidate within batch, which is closer in space than any job on the next batch. Thus, because of \Cref{deliverypotential}, $\PB$ only matches within batch, and therefore $\PB(\instance,\sojourn) = \sum_{t=1}^b \PB(\instance^{(t)},\infty)$.
%
As in \Cref{prop:greedy_linear_lower_bound_dynamic}, we arrive at
\[ \OPT(\instance,\sojourn) - \PB(\instance,\sojourn) \ge \sum_{t=1}^b \regret_\PB(\instance^{(t)},\infty) = b\frac{ \regret_\PB(\instance^{(0)},\infty)}{3} \ge \frac{\Njob}{3(\sojourn+1)}(\log(\sojourn+5)-3)/4, \]
%
concluding the proof.
\Halmos

\section{Interpretation of Potential}\label{sec: interpretation}

When a job $j$ becomes critical, our potential-based greedy algorithm matches it to an available job $k$ maximizing $r(\theta_j,\theta_k)-p(\theta_k)$, where $p(\theta_k)$ can be interpreted as the opportunity cost of matching job $k$, with the specific definition $p(\theta_k)=\frac12 \sup_{\theta\in\Theta} r(\theta_k,\theta)$.
This is an optimistic measure of opportunity cost because it assumes that job $k$ would otherwise be matched to an "ideal" type $\theta\in\Theta$ maximizing $r(\theta_k,\theta)$ (with half of this ideal reward $\sup_{\theta\in\Theta} r(\theta_k,\theta)$ attributed to job $k$).
We now prove that this ideal reward can indeed be achieved under asymptotically-large market thickness, for a stochastic model under our topology of interest.

\begin{definition}
Let $\instance \in \typespace^\Njob$.
For any job $j\in[\Njob]$, let $\instance^{-j}\in \typespace^{\Njob-1}$ be the same instance with the exception that job $j$ is not present. Meanwhile, let $\instance^{+j}\in \typespace^{\Njob+1}$ be the same instance with the exception that an additional copy of job $j$ is present.
Consider the following definitions.
\begin{enumerate}
\item Marginal Loss: $\marginalloss_j(\instance) = \OPT(\instance) - \OPT(\instance^{-j})$
\item Marginal Gain: $\marginalgain_j(\instance) = \OPT(\instance^{+j}) - \OPT(\instance)$
\end{enumerate}
\end{definition}

Note that definitions $\marginalloss_j(\instance),\marginalgain_j(\instance)$ are based solely on offline matching, and we will use them as our definitions of opportunity cost if the future was known.  One could alternatively use shadow prices from the LP relaxation of the offline matching problem, but we note that the LP is not integral.  In either case, there is no ideal definition of opportunity cost that is guaranteed to lead to the optimal offline solution in matching problems \citep[see][]{cohen2016invisible}.

We now establish the following \namecref{lem:marginal_as_interval} to help analyze the opportunity costs $\marginalloss_j(\instance),\marginalgain_j(\instance)$.


\begin{lemma}\label{lem:marginal_as_interval}
Let $\typespace=[0,1]$ and $\reward(\type,\type') = \min\{\type,\type'\}$. Consider an instance $\instance\in\typespace^n$ and relabel the indices to satisfy $\type_1 \ge \type_2 \ge \ldots \ge \type_n$. Then,
\begin{align*}
\marginalloss_j(\instance)
&=\sum_{k\ge j,k\ \mathrm{even}}(\theta_k-\theta_{k+1}), 
\\ \marginalgain_j(\instance)
&=\sum_{k\ge j,k\ \mathrm{odd}}(\theta_k-\theta_{k+1}),
\end{align*}
where we consider $\theta_{n+1}=0$.
\end{lemma}
%
\proof{Proof.}
    Let $\instance \in \typespace^\Njob$ such that $\type_1 \ge \type_2 \ge \ldots \ge \type_n$. 
    %
    Then, $\OPT(\instance) =\sum_{k\ \mathrm{even}}\theta_k$, and moreover
    \begin{align*}
    \OPT(\instance^{-j}) &=\sum_{k<j, k\ \mathrm{even}}\theta_k+\sum_{k>j, k\ \mathrm{odd}}\theta_k
    \\ \OPT(\instance^{+j}) &=\theta_j+\OPT(\instance^{-j})
    \end{align*}
    % 
    Therefore,
    % 
    \begin{align*}
    \marginalloss_j(\instance)
    &=\sum_{k\ge j,k\ \mathrm{even}}\theta_k
    -\sum_{k>j,k\ \mathrm{odd}}\theta_k
    =\sum_{k\ge j,k\ \mathrm{even}}(\theta_k-\theta_{k+1})
    \\ \marginalgain_j(\instance)
    &=\theta_j-\sum_{k\ge j,k\ \mathrm{even}}(\theta_k-\theta_{k+1})
    =\sum_{k\ge j,k\ \mathrm{odd}}(\theta_k-\theta_{k+1})
    \end{align*}
    % 
    completing the proof.
\Halmos\endproof

Note that because $\OPT(\instance^{+j}) = \OPT(\instance^{-j}) + \type_j$ and $\potential(\type_j) = \reward(\type_j,\type_j)/2$ for this reward function, we immediately get the following \namecref{cor: marginal_average}.

\begin{corollary}\label{cor: marginal_average}
        If $\typespace=[0,1]$ and $\reward(\type,\type') = \min\{\type,\type'\}$, then for all jobs $j$,
    \[ \potential(\type_j) = \frac{\marginalloss_j(\instance) + \marginalgain_j(\instance)}{2}. \]
\end{corollary}

We are now ready to prove our main result about the interpretation of potential, that the true opportunity costs $\marginalloss_j(\instance),\marginalgain_j(\instance)$ concentrate around $p(\theta_j)$ in a random uniform instance, assuming the market is sufficiently thick.  We without loss consider job $j=1$ and fix its type $\theta_1$.

\begin{theorem} \label{thm:interpretation}
Let $\typespace=[0,1]$ and $\reward(\type,\type') = \min\{\type,\type'\}$.
Fix $\theta_1\in\typespace$ and suppose $\theta_2,\ldots,\theta_n$ are drawn IID from the uniform distribution over $[0,1]$, forming a random instance $\instance\in\typespace^n$, for some $n\ge 2$. Then, both $\mathbb{E}[\marginalloss_1(\instance)]$ and $\mathbb{E}[\marginalgain_1(\instance)]$ are within $O(1/n)$ of $\potential(\type_1)$ and moreover $\mathrm{Var}(\marginalloss_1(\instance))=\mathrm{Var}(\marginalgain_1(\instance)) = O\left(\frac{1}{n}\right)$.
\end{theorem}
%
\proof{Proof.}
We prove the statement only for $\marginalloss_1(\instance)$, and the analogous result for $\marginalgain_1(\instance)$ follows from \Cref{cor: marginal_average}.
%
If $\type_1=0$, then $\marginalloss_1(\instance)=0$ for any $\instance$, coinciding with $\potential(\type_1)=0$. Then, for the remainder of the proof, assume that $\type_1>0$.
%
Consider the random variable $N=|\{j:\theta_j<\theta_1\}|$, which counts the number of points between $0$ and $\theta_1$ and has distribution $\text{Binom}(\Njob-1,\type_1)$.
%
These $N$ points divide the interval $[0,\theta_1]$ into $N+1$ intervals with total length $\type_1$.
%
From \Cref{lem:marginal_as_interval}, the marginal loss $\marginalloss_1(\instance)$ is determined by computing the total length of a subset of these intervals. % Maybe add figure?
%
It is known that if $N$ random variables are drawn independently from $\text{Unif}[0,1]$, then the joint distribution of the induced interval lengths is $\text{Dirichlet}(1,1,\ldots,1)$ with $N+1$ parameters all equal to 1 (i.e., drawn uniformly from the simplex).
%
In particular, since the Dirichlet distribution is symmetric, the sum of any $k$ of these intervals is equal in distribution to the $k$-th smallest sample ($k$-th order statistic) of the $N$ uniform random variables, whose distribution is known to be $\text{Beta}(k,N+1-k)$.
%
Thus, conditional on $N$, with $N\ge 1$, since the distribution of each of the $N$ jobs to the left of $\type_1$ is $\text{Unif}[0,\type_1]$, the distribution of $\marginalloss_1(\instance)/\type_1$ is $\text{Beta}(k_L,N+1-k_L)$,
where $k_L$ is either $\lceil\frac{N+1}2\rceil$ or $\floor{\frac{N+1}2}$.
%
To be precise, for $N\ge 1$, $k_L=\floor {N/2}+1=\lceil\frac{N+1}2\rceil$ if $n$ is even, and $k_L=\ceil{N/2}=\floor{\frac{N+1}2}$ if $n$ is odd.
%
Moreover, if $N=0$, then $\marginalloss_1(\instance)=\type_1$ if $n$ is even, and $\marginalgain_1(\instance)=0$ if $n$ is odd.
%
Hence,
\begin{align*}
\mathbb{E}[\marginalloss_1(\type) \mid N] 
&= \type_1\frac{k_L}{N+1}
% \label{eq: cond_exp}
\end{align*}
for all $N\in\{0,\ldots,n-1\}$.
%
Since $|k_L-(N+1)/2|\le 1$, then $|\mathbb{E}[\marginalloss_1(\type) \mid N]-\type_1/2|\le \type_1/(N+1)$, thus
% 
\begin{align*}
&\left|\mathbb{E}[\marginalloss_1(\type)]-\frac{\type_1}{2}\right| 
\le \type_1\mathbb{E}\left[\frac{1}{N+1}\right] \\
&\qquad = \type_1\sum_{k=0}^{n-1} \frac{1}{k+1}\binom{n-1}{k}\type_1^k(1-\type_1)^{n-1-k} \\
&\qquad = \frac{\type_1}{n}\sum_{k=0}^{n-1} \binom{n}{k+1}\type_1^k(1-\type_1)^{n-1-k} \\
&\qquad = \frac{1}{n}\sum_{k=1}^{n} \binom{n}{k}\type_1^k(1-\type_1)^{n-k} \\
&\qquad = \frac{1-(1-\type_1)^{n}}{n} = O\left(\frac{1}{n}\right).
\end{align*}
%
This completes the proof of the statement about $\mathbb{E}[\marginalloss_1(\instance)]$.

For the statement about variance, we know $\mathrm{Var}(\marginalloss_1(\type)) = \mathbb{E}[\mathrm{Var}(\marginalloss_1(\type) \mid N)] + \mathrm{Var}(\mathbb{E}[\marginalloss_1(\type)\mid N])$ by the law of total variance. For the first term, we have $\mathrm{Var}(\marginalloss_1(\type) \mid N) = \type_1^2\frac{k_L(N+1-k_L)}{(N+1)^2(N+2)}\indicator\{N\ge 1\}\le \frac{\type_1^2}{N+1}$, and then from the previous argument $\mathbb{E}[\mathrm{Var}(\marginalloss_1(\type) \mid N)]=O(1/n)$.
%
For the second term, 
\begin{align*}
\mathrm{Var}(\mathbb{E}[\marginalloss_1(\type)\mid N]) 
&= \mathrm{Var}\left(\mathbb{E}[\marginalloss_1(\type)\mid N] - \frac{\type_1}{2}\right) \\
&\le \mathbb{E} \left[\left(\mathbb{E}[\marginalloss_1(\type)\mid N] - \frac{\type_1}{2}\right)^2\right] \\
% 
&\le \mathbb{E}\left[\frac{\type_1^2}{(N+1)^2}\right] \\
&\le \type_1^2\mathbb{E}\left[\frac{1}{N+1}\right] = O\left(\frac{1}{n}\right).
\end{align*}
Thus, $\mathrm{Var}(\marginalloss_1(\instance)) = O\left(1/n\right)$, completing the proof.
\Halmos\endproof
%%%%%%%%

\section{Algorithmic Performance under Alternative Reward Topologies} \label{appx:alternative_reward_topologies}


In this section, we study the performance of both the naive greedy algorithm $\gre$ and potential-based greedy algorithm $\PB$, under the two alternative reward topologies defined in \eqref{eq:defn_reward_B} and \eqref{eq:defn_reward_C}.

\subsection{Reward Function $\reward(\type,\type')=1-|\type-\type'|$}
\phantomsection
\label{sec:reward2}
%
This reward function represents the goal to minimize the total distance between matched jobs.
%
In this case, for any $\type\in[0,1]$, the potential of a job of type $\type$ is $\potential(\type)=1/2$ constant across job types. Thus, the description of $\PB$ is equivalent to $\gre$, since their index functions differ only in an additive constant.

\begin{remark}\label{rem:rewardB}
    Under the 1-dimensional type space $\typespace=[0,1]$ and the reward function $\reward(\type,\type')=1-|\type-\type'|$, both the naive greedy and potential-based greedy algorithm generate the same output $(C,\matchof(\cdot))$ as described in \Cref{alg:dynamic}. In particular, both algorithms always match each critical job to an available job that is the closest in space.
\end{remark}

Moreover, we show that their performance can be reduced to the one of $\PB$ under reward function $\reward(\type,\type')=\min\{\type,\type'\}$.
%
In particular, we first show that when $\sojourn=\infty$, the regret is at least logarithmic in the number of jobs.
\begin{proposition}\label{prop:loglowerboundOffline_rewardB}     
    Under reward function $\reward(\type,\type')=1-|\type-\type'|$, if $\Njob=2^{k+3}-4$ for some integer $k\ge0$, then there exists an instance $\instance\in[0,1]^\Njob$ for which $\regret_\PB(\instance,\infty) = \regret_\gre(\instance,\infty) \ge (\log_2(\Njob+4)-3)/2 $.
\end{proposition}

\proof{Proof.}
    %
    We prove the statement for $\gre$, since $\regret_\PB(\instance,\infty) = \regret_\gre(\instance,\infty)$ follows from \Cref{rem:rewardB}.
    %
    Let $\Njob=2^{k+3}-4$, for $k\in\{0,1,\ldots\}$. From \Cref{prop:loglowerboundOffline}, there exists an instance $\instance\in[0,1]^\Njob$ such that the regret of $\PB$ under reward topology $\reward(\type,\type')=\min\{\type,\type'\}$, which we denote by $R$ from here on, is at least $(\log_2(\Njob+4)-3)/4$.
    %
    We show that under $\reward(\type,\type')=1-|\type-\type'|$, we have $\regret_\gre(\instance,\infty) \ge 2R$.
    %
    First, note that since $|\type-\type'| = \type + \type' - 2\min\{\type,\type'\}$, we have
    %
    \[ \gre(\instance,\infty) 
         = \sum_{j \in C} 1 - |\type_j-\type_{m(j)}|  
         = |C| - \sum_{j\in C} (\type_j +\type_{\matchof(j)})  + 2\sum_{j\in C} \min\{\type,\type'\}. 
    \]
    % 
    Similarly, recall that $\matchset_\OPT$ is the set of matches in the hindsight optimal solution. Then,
    \[ \OPT(\instance,\infty) = |\matchset_\OPT| - \sum_{ (j,k)\in\matchset_\OPT} (\type_j +\type_{k})  + 2\sum_{(j,k)\in\matchset_\OPT} \min\{\type_j,\type_{k}\}. \]
    %
    Since $\reward(\type,\type')\ge 0$ for any $\type,\type'\in \typespace$ and the total number of jobs $\Njob$ is even, both $\gre$ and $\OPT$ match every job. Hence, $|C|=|\matchset_\OPT|=n/2$, and moreover
    \[ \sum_{(j,k)\in\matchset_\OPT} (\type_j +\type_{k}) = \sum_{j\in C} (\type_j +\type_{\matchof(j)}). \]
    Thus,
    \[ \OPT(\instance,\infty) - \gre(\instance,\infty) \ge 2\left( \sum_{(j,k)\in\matchset_\OPT} \min\{\type_j,\type_{k}\} - \sum_{j\in C} \min\{\type_j,\type_{\matchof(j)}\} \right). \]
    %
    We claim that the term in large parenthesis is exactly the regret of $\PB$ under reward topology $\reward(\type,\type')=\min\{\type,\type'\}$, which we denote by $R$.
    %
    Indeed, from \Cref{rem:rewardB}, $(C,\matchof(\cdot))$ coincides with the resulting matching of $\PB$ under $\reward(\type,\type')=\min\{\type,\type'\}$, since it is specified to always match to the closest job (\Cref{deliverypotential}).
    %
    On the other hand, since all jobs are matched, the set of matches $\matchset_\OPT$ must induce disjoint intervals, which coincides with the optimal matching solution under reward function $\reward(\type,\type') = \min\{\type,\type'\}$.
    %
    Thus, by \Cref{prop:loglowerboundOffline}, we get $ \regret_\gre(\instance,\infty) \ge 2R \ge (\log_2(\Njob+4)-3)/2$, completing the proof.
\Halmos\endproof

We now extend this result to the online setting ($\sojourn<n$).

\begin{proposition} \label{prop:loglowerboundOnline_rewardB}
    Under reward topology $\reward(\type,\type') = 1-|\type-\type'|$, if $\sojourn+1 = 2^{k+3}-4$ for some integer $k\ge 0$, then for any number of jobs $\Njob$ divisible by $(\sojourn+1)$, there exists an instance $\instance \in [0,1]^\Njob$ for which $\regret_\PB(\instance,\sojourn) = \regret_\gre(\instance,\sojourn) \ge \frac{\Njob}{3(\sojourn+1)}(\log_2(\sojourn+5)-3)/2$.
\end{proposition}

\proof{Proof.}
    % 
    This proof is analogous to the proof of \Cref{prop:loglowerboundOnline}, since (i) there exists an offline instance that achieves the desired regret for $\sojourn+1=n$ (\Cref{prop:loglowerboundOffline_rewardB}), and (ii) the algorithm matches a critical job with the closest available job (\Cref{rem:rewardB}).
    %
    We repeat the proof below for completeness.

    We have an exact partition of the set of jobs into $b=n/(\sojourn+1)\in\{1,2,\ldots\}$ batches, and construct an instance $\instance\in[0,1]^\Njob$ that can be analyzed separately on each batch, where we denote $\instance^{(t)}=(\type_j)_{j\in B_t}$ for $t=1,\ldots,b$.
    %
    By \Cref{prop:loglowerboundOffline_rewardB}, there exists an instance $\instance^{(0)}\in[0,1]^{(\sojourn+1)}$ such that $\regret_\PB(\instance^{(0)},\infty) \ge (\log(\sojourn+5)-3)/2$.
    %
    Then, for each $t=1,\ldots,b$, let $\instance^{(t)}=\instance^{(0)}\cdot\frac13$ if $t$ is odd and $\instance^{(t)}=\instance^{(0)}\cdot\frac13+\frac23$ if $t$ is even.
    %
    Note that since each batch size is $\sojourn+1$ divisible by 4, it is possible to match within batch only. Moreover, by construction, every critical job has a matching candidate within batch, which is closer in space than any job on the next batch. Thus, because of \Cref{rem:rewardB}, $\PB$ only matches within batch, and therefore $\PB(\instance,\sojourn) = \sum_{t=1}^b \PB(\instance^{(t)},\infty)$.
    %
    As in \Cref{prop:greedy_linear_lower_bound_dynamic}, we arrive at
    \[ \OPT(\instance,\sojourn) - \PB(\instance,\sojourn) \ge \sum_{t=1}^b \regret_\PB(\instance^{(t)},\infty) = b\frac{ \regret_\PB(\instance^{(0)},\infty)}{3} \ge \frac{\Njob}{3(\sojourn+1)}(\log(\sojourn+5)-3)/2, \]
    %
    concluding the proof.
\Halmos\endproof

Finally, we directly show the tight regret upper bound (up to constants) as an immediate consequence of the proof of \Cref{thm: dynamic}. Note that if $d+1=n$, then this yields the result for offline matching.

\begin{theorem}\label{thm:rewardB_dynamic}
Under reward topology $\reward(\type,\type') = 1-|\type-\type'|$, we have $\regret_\PB(\instance,\sojourn) = \regret_\gre(\instance,\sojourn) \le 1/2 + (\frac{\Njob}{d+1}+1)(1+\log(\sojourn+2))$ for any $\Njob$, and any instance $\instance\in[0,1]^\Njob$.
\end{theorem}


\proof{Proof.}
Let $\instance\in \typespace^\Njob$ and $\sojourn\ge 1$. Let $(C,\matchof(\cdot))$ be the output of $\gre$ on $\instance$. We have
\begin{align*}
\OPT(\instance,\sojourn)-\gre(\instance,\sojourn) 
& \le \sum_{j=1}^\Njob \potential(\type_j) - \gre(\instance,\sojourn)  \nonumber \\
& \le \sup_{\type\in\typespace}\potential(\type) + \sum_{j \in C} \left( p(\theta_j) + p(\theta_{m(j)}) - r(\theta_j,\theta_{m(j)}) \right) \nonumber \\ 
& = \frac12 +  \sum_{j \in C} \left( \frac12 + \frac12 - (1-|\theta_j-\theta_{m(j)}|) \right) \nonumber \\
& = \frac12 + \sum_{j \in C} |\type_j-\type_{\matchof(j)}|, 
\end{align*}
The second term is exactly the sum of distances between jobs matched by $\gre$. Recall that $(C,\matchof(\cdot))$ coincides with the matching output of $\PB$ under $\reward(\type,\type')=\min\{\type,\type'\}$, both are specified to always match with the closest job (\Cref{deliverypotential}, \Cref{rem:rewardB}). Thus, recalling \eqref{eq: distance_dynamic} in the proof of \Cref{thm: dynamic}, we have
\[ \sum_{j \in C} |\type_j-\type_{\matchof(j)}| = \sum_{t=1}^b \sum_{j \in C\cap B_t} |\type_j-\type_{\matchof(j)}|\le \left(\frac{n}{d+1}+1 \right)\left(1+\log(d+2)\right),\]
and the result follows.
\Halmos\endproof

\subsection{Reward Function $\reward(\type,\type')=|\type-\type'|$}\label{sec: reward 3}
%
Under this reward topology, it is worst to match two jobs of the same type, contrasting the two settings studied in \Cref{sec:PB} and \Cref{sec:reward2}.
%
In this case, the potential of a job type $\type\in[0,1]$ is $\potential(\type)=\max\{\type,1-\type\}/2$.
%
We first show that \emph{any} index-based matching policy must suffer regret constant regret per job in the offline setting.
%
\begin{proposition}\label{prop:linearLB_offline_rewardC}
    Under reward topology $\reward(\type,\type')=|\type-\type'|$, when the number of jobs $\Njob$ is divisible by 4, for any index-based greedy matching algorithm $\ALG$, there exists an instance $\instance\in[0,1]^\Njob$ for which $\regret_\ALG(\instance,\infty)\ge c_\ALG \Njob$, for some $c_\ALG\in (0,1)$. In particular, $c_\gre = c_\PB = 1/4$.
\end{proposition}

\proof{Proof.}
Let $\Njob$ be divisible by 4.
%
Let $\ALG$ be an index-based greedy matching algorithm specified by index function $\indexf:[0,1]^2\to \R$.
%
Define $\type^{c} = \sup\{ \type\in[0,1] : \indexf(\type,1)\ge \indexf(\type,0) \}$.
%
We first construct an instance in the case $\type^{c}>0$ (e.g. $\type^{c}=1/2$ for $\gre$ and $\PB$), and analyze the alternative case separately in the end.
%
Consider an instance $\instance\in\typespace^\Njob$ such that $\type_1,\ldots,\type_{\Njob/4} = \type^{c}$, $\type_{{\Njob/4}+1},\ldots,\type_{{3\Njob/4}}=0$, and $\type_{{3\Njob/4}+1},\ldots,\type_{\Njob}=1$. 
%
It is straightforward to check that $\OPT(\instance,\infty) \ge (1+\type^c){\Njob/4}$.
%
On the other hand, $\ALG$ chooses to match every $j=1,\ldots,n/4$ to $\matchof(j)\in \{3n/4+1,\ldots,n\}$.\footnote{In the presence of ties, such an output is achieved by some tie-breaking rule.}
%
Then,
\[ \ALG(\instance,\infty) = 0 + \sum_{j=1}^{\Njob/4} \reward(\type_j,\type_{3\Njob/4+j}) = \frac{\Njob(1-\type^c)}{4}. \]
%
and thus $\regret_\ALG(\instance,\infty) = \OPT(\instance,\infty) - \ALG(\instance,\infty) \ge {{\Njob\type^{c}}}/{2},$ proving the statement with $c_\ALG = \type^{c}/2$.

%
Now, if $\type^{c}\le 0$, then $\indexf(\type,0) < \indexf(\type,1)$ for all $\type\in(0,1]$.
%
In particular, for any $\eps>0$, consider the instance $\type_1,\ldots,\type_{\Njob/4} = \eps$, $\type_{{\Njob/4}+1},\ldots,\type_{{3\Njob/4}}=1$, and $\type_{{3\Njob/4}+1},\ldots,\type_{\Njob}=0$.
%
Then, $\OPT(\instance,\infty) \ge (1+(1-\eps))n/4$, and $\ALG(\instance,\infty) = \eps n/4$.
%
Hence, $\regret_\ALG(\instance,\infty)\ge (1-\eps)/2$, and the statement is true with $c_\ALG = (1-\eps)/2$ for every $\eps>0$.
\Halmos\endproof

The following result shows that, under $\PB$ and $\gre$, this lower bound extends to the online setting, losing a factor of 2.

\begin{proposition}\label{prop:linearLB_dynamic_rewardC}
    Under reward topology $\reward(\type,\type')=|\type-\type'|$, if $(\sojourn+1)$ is divisible by 4, then for any number of jobs $\Njob$ divisible by $2(\sojourn+1)$, there exists an instance $\instance \in [0,1]^\Njob$ for which $\regret_\gre(\instance,\sojourn)=\regret_\PB(\instance,\sojourn) \ge n/8$.
\end{proposition}

\proof{Proof.}
If $\Njob$ is divisible by $2(\sojourn+1)$, we have an exact partition of the set of jobs into $b=\Njob/(\sojourn+1)\in\{2,4,\ldots\}$ batches, defined as $B_{t+1} = \{t(\sojourn+1)+1,\ldots, (t+1)(\sojourn+1)\}$ for $t = 0, \ldots, b-1$.
%
We construct an instance $\instance\in[0,1]^\Njob$ for which $\gre$ (and $\PB$) "resets" every two batches, in the sense that all jobs in these two batches are matched among them.
%
To formally specify the construction, first fix $\eps\in(0,1/2)$. Define, for $t=0,\ldots,b/2-1$
%
\[
\type_{2t(\sojourn+1)+j} =
\begin{cases}
\frac12 & \text{if } j \le \frac{\sojourn+1}{4}, \\
\eps & \text{if } \frac{\sojourn+1}{4} < j \le \frac{3(\sojourn+1)}{4}, \\
1 & \text{if } \frac{3(\sojourn+1)}{4} < j \le \sojourn+1 \text{, or } \frac{3(\sojourn+1)}{2} < j \le 2(\sojourn+1), \\
0 & \text{if } \sojourn+1 < j \le \frac{3(\sojourn+1)}{2}.
\end{cases}
\]
%
Consider the notation $\instance^{(0)}=(\type_j)_{j\le 2(\sojourn+1)}$.
%
It is easy to check that
\[ 
\OPT(\instance^{(0)},\infty) = \left(\frac12-\eps\right)\frac{\sojourn+1}{4} + (1-\eps)\frac{\sojourn+1}{4} + \frac{\sojourn+1}{2} =  \left(\frac{7-4\eps}{8}\right)(\sojourn+1),
\]
and then $\OPT(\instance,\sojourn) \ge \frac{b}{2}\OPT(\instance^{(0)},\infty) = \frac{7-4\eps}{16}\Njob $.
%
On the other hand,
\[ 
\gre(\instance^{(0)},\infty) = \frac12\frac{\sojourn+1}{4} + \eps\frac{\sojourn+1}{4} + (1-\eps)\frac{\sojourn+1}{4} + \frac{\sojourn+1}{4} =  \frac58(\sojourn+1).
\]
%
We claim that $\gre(\instance,\sojourn) = \frac{b}{2}\gre(\instance^{(0)},\infty) = \frac{5}{16}\Njob $, and thus $\regret_\gre(\instance,\sojourn)\ge \frac{1-2\eps}{8}\Njob $.
%
Indeed, let $(C,\matchof(\cdot))$ be the output of $\gre$. Running $\gre$ on $\instance$ is as follows:
%
\begin{enumerate}
    \item For every job $j \le (\sojourn+1)/4$ of type $\type_j=1/2$, and then $3(\sojourn+1)/4 < \matchof(j) \le (\sojourn+1)$ with $\reward(\type_j,\type_{\matchof(j)})=1/2$.
    \item Then, for every job $\frac{\sojourn+1}{4} < j \le \frac{(\sojourn+1)}{2}$, the only available jobs are of type $\type\in\{0,\eps\}$, and then $ \sojourn+1 < \matchof(j) \le \frac{3(\sojourn+1)}{2}$ with $\reward(\type_j,\type_{\matchof(j)})=\eps$. 
    \item In turn, jobs $\frac{(\sojourn+1)}{2} < j \le \frac{3(\sojourn+1)}{4}$ get to observe jobs of type $\type=1$, and then $ \frac{3(\sojourn+1)}{2} < \matchof(j) \le 2(\sojourn+1)$ with $\reward(\type_j,\type_{\matchof(j)})=1-\eps$.
    \item Finally, the key observation is that the all remaining available jobs $j\le \frac{3(\sojourn+1)}{2}$ of type $\type_j=0$ observe jobs of type $1$, as well as jobs on a "third" batch. However, by construction every job $2(\sojourn+1) < k \le 2(\sojourn+1) + \frac{3(\sojourn+1)}{2} $ are of type $\type_k\in\{1/2,\eps\}$, and then each job $j\le \frac{3(\sojourn+1)}{2}$ is matched (within batch) to $ \frac{3(\sojourn+1)}{2} < \matchof(j) \le 2(\sojourn+1)$ with $\reward(\type_j,\type_{\matchof(j)})=1$.
\end{enumerate}

The proof for $\PB$ is analogous, since the matching decisions coincide with $\gre$ on this instance. To see this, note that
\begin{itemize}
    \item $\indexf_\PB(1/2,1) = \reward(1/2,1) - \potential(1) = 1/2 - 1/2 = 0$,
    \item $\indexf_\PB(1/2,\eps)=\reward(1/2,\eps) - \potential(\eps) = 1/2 - \eps - (1-\eps)/{2} = - \eps/2 < \indexf_\PB(1/2,1)$,
    \item $\indexf_\PB(\eps,0)=\reward(\eps,0) - \potential(0) = \eps - 1/2$,
    \item $\indexf_\PB(\eps,\eps)=\reward(\eps,\eps) - \potential(\eps) = 0 - (1-\eps)/{2}= \eps/{2}-1/2< \indexf_\PB(\eps,0)$,
    \item $\indexf_\PB(\eps,1)=\reward(\eps,1) - \potential(1) = 1 - \eps - 1/2 = 1/2 - \eps$,
    \item $\indexf_\PB(\eps,1/2)=\reward(\eps,1/2) - \potential(1/2) = 1/2 - \eps - 1/4 = 1/4 - \eps < \indexf_\PB(\eps,1) $, and
    \item $\indexf_\PB(0,\eps)= \eps - (1-\eps)/2 < \indexf_\PB(0,1/2) < \indexf_\PB(0,1)$.
\end{itemize}
Thus, running $\PB$ on $\instance$ follows the same four steps described by $\gre$, yielding the same regret.
\Halmos\endproof


\section{Linear Relaxation and Dual Variables}\label{sec: LP}

We present in this section the linear relaxation of the IP that describes the hindsight optimum defined in \eqref{eq: OPT}. Given market density $\sojourn\ge 1$ and full information of the sequence of arrivals $\instance\in\typespace$, consider the following linear program (LP)
%
\begin{maxi}
{x}{ \sum_{j,k : j\neq k, |j-k|\le \sojourn} x_{jk} \reward(\type_j,\type_k)}
{\label{eq: LP}}{}
\addConstraint{ \sum_{k:j\neq k} x_{jk}}{\le 1,}{j\in [\Njob]}
\addConstraint{ x_{jk}}{\ge 0,\quad }{ j,k\in [\Njob], j\neq k,}
\end{maxi}
%
and its dual
%
\begin{mini}
{\lambda}{ \sum_{j\in[\Njob]} \lambda_{j} }
{\label{eq: Dual-LP}}{}
\addConstraint{ \lambda_j + \lambda_k}{ \ge \reward(\type_j,\type_k),\quad}{j,k\in [\Njob], j\neq k, |j-k|\le \sojourn}
\addConstraint{ \lambda_{j} }{\ge 0 ,}{ j\in [\Njob].}
\end{mini}

It is known that the LP relaxation is integral only for bipartite graphs. However, we observe in \Cref{fig: LP_gap} that the integrality gap is small, and thus the value of the LP can be reasonably used as a benchmark.

\begin{figure}[H]
  \centering
  \subcaptionbox{Average Total Reward.%
    \label{fig:reward}}[0.49 \textwidth]{\includegraphics[width = \figWidth \textwidth]{Simulation_Results/Synthetic/New/1D_Pooling/1D_Common_Origin/Forecast-Agnostic/value.png}}
  \hfill
  \subcaptionbox{Fraction of LP value achieved by $\OPT$.%
    \label{fig:LP_Gap}}[0.49 \textwidth]{\includegraphics[width = \figWidth \textwidth]{Simulation_Results/Synthetic/New/1D_Pooling/1D_Common_Origin/Forecast-Agnostic/integrality_gap.png}}
  % 
  \caption{Comparison with LP value.
  }
  % 
  \label{fig: LP_gap}
\end{figure}

Also, note that the potential can always be used as a feasible (but not necessarily optimal) solution to the dual LP.


\begin{proposition}
    Let $\instance\in\typespace^\Njob$ and $\sojourn\ge 1$. Then, $\{\potential(\type_j)=\frac{1}{2}\sup_{\type'\in\typespace}\reward(\type_j,\type'):j\in[\Njob]\}$ is a feasible solution of the dual LP \eqref{eq: Dual-LP}.
\end{proposition}


\proof{Proof.}
    Non-negativity holds since $\reward\ge 0$. Moreover, for any $j,k$ we have
    \[ \reward(\type,\type')\le \frac{1}{2}\reward(\type,\type')+\frac{1}{2}\reward(\type,\type') \le \potential(\type)+\potential(\type'), \]
    verifying the dual feasibility constraints.
\Halmos\endproof

Interestingly, when we extract the shadow prices from the 400 instances considered as historical data in \Cref{sec:sim_unif_1D}, we can see that these concentrate around potential, as density increases (\Cref{fig: dual_converge}). This is aligned with our results in \Cref{sec: interpretation}.

\begin{figure}[H]
  \centering
  \subcaptionbox{Density $\sojourn=5$.%
    \label{fig:dual_conv_d5}}[0.4 \textwidth]{\includegraphics[width = \figWidth \textwidth]{Simulation_Results/Synthetic/New/1D_Pooling/1D_Common_Origin/Forecast-Aware/duals_at_d5.png}}
  % \hfill
  \hspace{1em}
  % 
  \subcaptionbox{Density $\sojourn=30$.%
    \label{fig:dual_conv_d30}}[0.4 \textwidth]{\includegraphics[width = \figWidth \textwidth]{Simulation_Results/Synthetic/New/1D_Pooling/1D_Common_Origin/Forecast-Aware/duals_at_d30.png}}
  % 
  \caption{Scatter plot of shadow prices, average, and potential.}
  % 
  \label{fig: dual_converge}
\end{figure}

\section{Meituan Data - Market Dynamics} \label{sec:market_dynamics}

In this appendix, we briefly describe the Meituan dataset made public by the INFORMS TSL Data-Driven Challenge, and provide high-level descriptions of the market-dynamics. 


\subsection{Temporal Dynamics} \label{appx:meituan_temporal_dynamics}

First, \Cref{fig:meituan_orders_per_hour} provides the average number of orders per hour by \emph{hour-of-week}.
Hour-of-week 0 corresponds to midnight to 1am on Mondays, and hour-of-week 1 corresponds to 1am to 2am on Mondays, and so on. 
%
The gray bars indicate the peak lunch hours (10:30am-1:30pm), while the green bars indicate peak dinner hours (5pm-8pm).
%
We can see that the platform observes highest order volume during lunch time, averaging around 200 order per minute between noon and 1pm.   

\newcommand{\FigWidthHOW}{0.85}

\begin{figure}
    \centering
    \includegraphics[width=\FigWidthHOW\linewidth]{Simulation_Results/Meituan/City/meituan_number_of_orders_how.png}
    \caption{Average number of orders by hour-of-week in Meituan data.}
    \label{fig:meituan_orders_per_hour}
\end{figure}



\Cref{fig:meituan_pooled_orders_per_hour} in \Cref{sec:intro} of the paper illustrates the fraction of orders placed during each hour-of-week that is pooled with at least one other order. 
%
This is computed by combining the order-level data (which contains the timestamp of each order and the order ID) and the ``wave level'' data, where each wave corresponds to a list of orders that were pooled and assigned to the same driver.\footnote{
For more details on waves, see \url{https://github.com/meituan/Meituan-INFORMS-TSL-Research-Challenge/blob/main/tsl_meituan_2024_data_report_20241015.pdf}, accessed January 30, 2024. Each wave may contain more than two orders, though some orders may have been dropped off by the driver before some other orders were picked up.   
}
The orders that are not pooled with any other order correspond to waves with only one order, and \Cref{fig:meituan_pooled_orders_per_hour} illustrates 1 minus this fraction of orders that are not pooled. 


\Cref{fig:meituan_grab_to_fetch_how} illustrates the average amount of time it takes the drivers to pick up the orders, after they received and then accepted the order from the platform. Specifically, this is the amount of time that elapses between (i) the grab\_time, i.e. when the order is accepted by a driver, and (ii) the fetch\_time, i.e. when the order is picked-up by the driver.
%
The average pick-up time across all orders is 10.1 minutes. 
%
We see from \Cref{fig:meituan_grab_to_fetch_how} that the the pick-up time is generally longer during lunch and dinner peaks, reaching a maximum of 12.0 minutes at noon on Tuesdays, but overall we do not see very substantial increases during peak hours (which may result from, for example, severe supply constraints in the system). 


\begin{figure}
    \centering
    \includegraphics[width=\FigWidthHOW\linewidth]{Simulation_Results/Meituan/City/meituan_grab_to_fetch_how.png}
    \caption{
    The average amount of time it takes the driver to pick-up the orders after accepting the dispatch from the platform, by hour-of-week. 
    }
    \label{fig:meituan_grab_to_fetch_how}
\end{figure}


Finally, \Cref{fig:meituan_order_to_first_dispatch_how} illustrates the amount of time that elapses, between platform\_order\_time, i.e. when each order is placed by the customer, and first dispatch\_time, i.e. when the order was offered to a delivery driver for the first time.
% 
%
Intuitively, this is the amount of time it takes the platform to start dispatching each order, and the average is around 4 minutes across all orders. 
% 
Despite the fact that it takes an estimated $12$ minutes for the restaurant to prepare the orders, the platform starts to dispatch the orders shortly after they are placed. This is potentially due to the fact that it takes time for drivers to get to the restaurants (see \Cref{fig:meituan_grab_to_fetch_how}), so orders are dispatched before they are ready in order to reduce the total wait time experienced by the customers.  


\begin{figure}
    \centering
    \includegraphics[width=\FigWidthHOW\linewidth]{Simulation_Results/Meituan/City/meituan_order_to_first_dispatch_how.png}
    \caption{
    The average amount of time (minutes) that elapses between (i) when an order is placed by the customer, and (ii) when the order is dispatched to a delivery driver for the first time.
    }
    \label{fig:meituan_order_to_first_dispatch_how}
\end{figure}



\subsection{Spatial Distribution} \label{appx:meituan_spatial_distribution}

We now visualize the distribution of orders in space, and provide the distribution of order volume by origin-destination hexagon pairs. We use resolution-9 hexagons from the h3 package, which is the resolution that performs the best for $\averagedual$ when the matching window is at least $2$ minutes. The average size of each hexagon is  $0.1 \mathrm{km}^2$, and the average edge length is around $0.2 \mathrm{km}$.\footnote{
\url{https://h3geo.org/docs/core-library/restable/}, accessed January 31, 2025. 
} 


First, \Cref{fig:meituan_heatmaps} provides heat-maps of trip volume by order origin and destination, respectively. We can see that the order origins are more concentrated than destinations, which is aligned with the more concentrated locations of restaurants in comparison to residential areas and office buildings. 
%
Note that both distributions are highly skewed. For hexagons with at least one originating order, the median, average, and maximum number of orders are 330, 960, and 17151, respectively. For hexagons that are the destination of at least one order, the median, average, and maximum order count are 71, 327 and 5167, respectively. 


\begin{figure}[t!]
  \centering
  \subcaptionbox{By Origin.%
    \label{fig:meituan_count_by_orig}}[0.49 \textwidth]{\includegraphics[width = \figWidth \textwidth]{Simulation_Results/Meituan/City/meituan_count_by_orig.png}}
  \hfill
  \subcaptionbox{By Destination.%
    \label{fig:meituan_count_by_dest}}[0.49 \textwidth]{\includegraphics[width = \figWidth \textwidth]{Simulation_Results/Meituan/City/meituan_count_by_dest.png}}
  % 
  \caption{Number of orders by order origin hexagon (left) and order destination hexagon (right).
  }
  % 
  \label{fig:meituan_heatmaps}
\end{figure}



Finally, \Cref{fig:meituan_CDF_count_per_OD_pair} presents the distribution order volume by origin and destination hexagon pairs. From the unweighted CDF on the left, we can see that the vast majority of OD pairs have fewer than 100 orders. Out of all OD hexagon pairs with at least one order, the median, average, and maximum number of orders are 3, 8.22, and 808, respectively.
% 
The CDF on the right is weighted by order count.
%
Overall, roughly 14\% of orders are associated with OD pairs with at least 100 orders, and 72.8\% of orders are from OD pairs with at least 10 orders.
%
% 
\begin{figure}
    \centering
    \includegraphics[width=0.75\linewidth]{Simulation_Results/Meituan/City/meituan_CDF_count_per_OD_pair.png}
    \caption{
    Distribution (CDF) of the number of orders per origin-destination hexagon pair (resolution-9), unweighted (left) and weighted by the number of orders (right). 
    }
    \label{fig:meituan_CDF_count_per_OD_pair}
\end{figure}


\section{Additional Experiments}\label{sec: sim_results_extra}

We first report additional performance metrics for settings studied in \Cref{sec:numerical_experiments} of the paper.
%
Additionally, we also consider more general synthetic environments, including two-dimensional locations, non-uniform spatial distributions, and different reward topologies.

\subsection{Match Rate and Saving Fraction}
\label{sec:match_rate}  

We consider in this section two additional performance metrics: the \newterm{match rate}, i.e. the fraction of jobs that were pooled instead of dispatched on their own, and  the \newterm{saving fraction}, i.e. the fraction of the total distance that is reduced by pooling, relative to the total travel distance without any pooling~\citep[see][]{aouad2020dynamic}.

Both regret and reward ratio are performance metrics that compare pooling algorithm relative to the hindsight optimal pooling outcome. 
%
The saving fraction illustrates the benefit of delivery pooling in comparison to the total distance traveled, and is upper bounded by 0.5 (since reward from pooling a job cannot exceed its distance).
%
The match rate, as shown in \Cref{fig:1D_unif_match_rate,fig:meituan_match_rate}, highlights an additional desirable property of $\PB$ and, more generally, of \emph{index-based greedy matching} algorithms: they pool as many jobs as possible.
% 
This is desirable in practice, since drivers who are offered just only one order from a platform may try to pool orders from competing platforms \citep[see e.g.][]{reddit2022grubhub,reddit2023doordash}, leading to poor service reliability for customers.

\subsubsection{Uniform one-dimensional case.} \label{sec: 1D_extra}


\Cref{fig:1D_extra} compares the match rate and the saving fraction achieved for the one-dimensional synthetic environment presented in \Cref{sec:sim_unif_1D}.
%
Recall that all jobs share the same origin at $0$ and destinations are drawn uniformly at random from $[0,1]$. 


%%
\Cref{fig:1D_unif_saving_fraction} shows that the reduction in travel distance can be remarkably high for this setting, with $\OPT$ saving more than 46\% even at low density levels.
%
Since the saving fraction is proportional to the total reward collected by the algorithm, our previous performance comparisons presented in \Cref{sec:numerical_experiments} imply that the saving fraction achieved by $\PB$ is the best among all benchmark algorithms, which translates into over 44\% of reduction in total travel distance.
%
We can also observe that the saving fraction under $\gre$ does not decrease with density.
%
Nonetheless, $\OPT$ improves at a much faster rate, resulting in a relative performance of $\gre$ against $\OPT$ that is decreasing with density (as observed in \Cref{fig:1D}).

\begin{figure}[H]
  \centering
  \subcaptionbox{Average match rate.%
    \label{fig:1D_unif_match_rate}}[0.49 \textwidth]{\includegraphics[width = \figWidth \textwidth]{Simulation_Results/Synthetic/New/1D_Pooling/1D_Common_Origin/match_rate.png}}
  \hfill
  \subcaptionbox{Average saving fraction.%
    \label{fig:1D_unif_saving_fraction}}[0.49 \textwidth]{\includegraphics[width = \figWidth \textwidth]{Simulation_Results/Synthetic/New/1D_Pooling/1D_Common_Origin/saving_rate_all.png}}
  % 
  \caption{Additional metrics in random 1D instances.}
  % 
  \label{fig:1D_extra}
\end{figure}

The match rate in \Cref{fig:1D_unif_match_rate} illustrates that all \emph{index-based greedy matching} algorithms match every one of the $\Njob = 1000$ jobs.
%
This is because (i) the reward from matching any two jobs is non-negative in this 1D setting, and (ii) the total number of jobs is even.
%
On the other hand, $\batching$ matches every job only if $\sojourn+1$ is even, since every time it computes an optimal matching, the \textit{batch size} is $\sojourn+1$ and it dispatches \textit{all} available jobs in pooled trips. 
%
This is not the case for $\rbatching$, since by dispatching only the critical job and its match (if any), the batch size for the next time it computes an optimal matching is potentially different from $\sojourn+1$.
%


\subsubsection{Meituan data.} \label{sec: meituan_extra}
%
\Cref{fig:meituan_extra} provides a comparison of the match rate and the saving fraction for the setting presented in \Cref{sec:sim_meituan}.
%
Recall that as we extend our definition of pooling reward to allow for 2D locations with heterogeneous origins, it is no longer guaranteed to be non-negative.
%
Consequently, both the matching rate and saving fraction of all algorithms are lower than in the one-dimensional setting.


Nonetheless, we can see that if the platform allows a time window of at least $1$ minute for each order before it has to be dispatched, $\PB$ achieves a reduction in distance traveled by over 20\%. At $5$ minutes, this fraction increases to roughly 30\%, which is the best among all tested \emph{practical} heuristics, i.e. excluding $\dual$ and $\OPT$.
%

%
\Cref{fig:meituan_match_rate} shows that greedy algorithms ($\gre$, $\PB$, $\dual$, $\averagedual$) still achieve substantially higher match rates in comparison to $\batching$, $\rbatching$, and $\OPT$.
%
In particular, $\PB$ consistently pools 5-10\% more jobs than $\OPT$.

\begin{figure}%[H]
  \centering
  \subcaptionbox{Average match rate.%
    \label{fig:meituan_match_rate}}[0.49 \textwidth]{\includegraphics[width = \figWidth \textwidth]{Simulation_Results/Meituan/City/match_rate_all.png}}
  \hfill
  \subcaptionbox{Average saving fraction.%
    \label{fig:meituan_saving_fraction}}[0.49 \textwidth]{\includegraphics[width = \figWidth \textwidth]{Simulation_Results/Meituan/City/saving_rate_all.png}}
  % 
  \caption{Match rate and saving fraction for Meituan Order-Level Data.}
  % 
  \label{fig:meituan_extra}
\end{figure}
%

\subsection{Non-Uniform One-Dimensional Case}\label{sec: nonunif_1D}
%
In this section, we show that our experimental results in one-dimensional synthetic data are robust to non-uniform distribution of types. We consider an analogous setting from \Cref{sec:sim_unif_1D}, but now each job type is drawn IID from a distribution Beta(0.5, 2). Overall, the results are consistent, with the main difference being that $\PB$ performs slightly worse than $\averagedual$ for low densities. Nonetheless, $\PB$ ends up outperforming all benchmarks starting from $\sojourn\ge 15$.

\begin{figure}[H]
  \centering
  \subcaptionbox{Average regret.%
    \label{fig:1D_nonunif_regret}}[0.49 \textwidth]{\includegraphics[width = \figWidth \textwidth]{Simulation_Results/Synthetic/New/1D_Pooling/1D_Common_Origin/Non-Uniform/regret.png}}
  \hfill
  \subcaptionbox{Average ratio.%
    \label{fig:1D_nonunif_ratio}}[0.49 \textwidth]{\includegraphics[width = \figWidth \textwidth]{Simulation_Results/Synthetic/New/1D_Pooling/1D_Common_Origin/Non-Uniform/frac_of_OPT.png}}
  % 
  \caption{Comparison of average regret and reward ratio for forecast-agnostic heuristics in non-uniform 1D instances.}
  % 
  \label{fig:1D_nonunif}
\end{figure}

\begin{figure}[H]
  \centering
  \subcaptionbox{Average regret.%
    \label{fig:1D_nonunif_forecastaware_regret}}[0.49 \textwidth]{\includegraphics[width = \figWidth \textwidth]{Simulation_Results/Synthetic/New/1D_Pooling/1D_Common_Origin/Non-Uniform/regret_dual.png}}
  \hfill
  \subcaptionbox{Average ratio.%
    \label{fig:1D_nonunif_forecastaware_ratio}}[0.49 \textwidth]{\includegraphics[width = \figWidth \textwidth]{Simulation_Results/Synthetic/New/1D_Pooling/1D_Common_Origin/Non-Uniform/frac_of_OPT_dual.png}}
  % 
  \caption{Comparison of average regret and reward ratio for forecast-aware heuristics in non-uniform 1D instances.}
  % 
  \label{fig:1D_nonunif_forecastaware}
\end{figure}




\subsection{Two-Dimensional Case with Heterogeneous Origins} \label{sec:sim_unif_1D_2D}


We simulate two-dimensional environments to supplement the simulations on Meituan data.

\subsubsection{Uniform spatial distribution.}

In this section, we consider two-dimensional locations with heterogeneous origins. Specifically, each location, origin and destination, is drawn from a uniform distribution in the unit square. 
%
Overall, we observe that although $\PB$ is far from best at low densities, it still outperforms every other tested benchmark for $\sojourn\ge 25$. We verify in \Cref{sec:sim_meituan} that achieving density values for which $\PB$ outperforms every other benchmark is realistic.


\begin{figure}[H]
  \centering
  \subcaptionbox{Average regret.%
    \label{fig:2Dhet_regret}}[0.49 \textwidth]{\includegraphics[width = \figWidth \textwidth]{Simulation_Results/Synthetic/New/2D_Different_Origins/regret.png}}
  \hfill
  \subcaptionbox{Average ratio.%
    \label{fig:2Dhet_ratio}}[0.49 \textwidth]{\includegraphics[width = \figWidth \textwidth]{Simulation_Results/Synthetic/New/2D_Different_Origins/frac_of_OPT.png}}
  % 
  \caption{Comparison of average regret and reward ratio for forecast-agnostic heuristics in random 2D instances.}
  % 
  \label{fig:2Dhet}
\end{figure}
\begin{figure}[H]
  \centering
  \subcaptionbox{Average regret.%
    \label{fig:2Dhet_forecastaware_regret}}[0.49 \textwidth]{\includegraphics[width = \figWidth \textwidth]{Simulation_Results/Synthetic/New/2D_Different_Origins/regret_dual.png}}
  \hfill
  \subcaptionbox{Average ratio.%
    \label{fig:2Dhet_forecastaware_ratio}}[0.49 \textwidth]{\includegraphics[width = \figWidth \textwidth]{Simulation_Results/Synthetic/New/2D_Different_Origins/frac_of_OPT_dual.png}}
  % 
  \caption{Comparison of average regret and reward ratio for forecast-aware heuristics in random 2D instances.}
  % 
  \label{fig:2Dhet_forecastaware}
\end{figure}


\subsubsection{Non-uniform spatial distribution.}

In addition, we consider an analogous two-dimensional setting with locations drawn from non-uniform distribution Beta(0.5,2). The results remain consistent: $\PB$ outperforms all \emph{practical} heuristics, with the \emph{unrealistic} $\dual$ benchmark being the only one achieving better performance.

\begin{figure}[H]
  \centering
  \subcaptionbox{Average regret.%
    \label{fig:2Dhet_nonunif_regret}}[0.49 \textwidth]{\includegraphics[width = \figWidth \textwidth]{Simulation_Results/Synthetic/New/2D_Different_Origins/Non-Uniform/regret.png}}
  \hfill
  \subcaptionbox{Average ratio.%
    \label{fig:2Dhet_nonunif_ratio}}[0.49 \textwidth]{\includegraphics[width = \figWidth \textwidth]{Simulation_Results/Synthetic/New/2D_Different_Origins/Non-Uniform/frac_of_OPT.png}}
  % 
  \caption{Empirical performance of forecast-agnostic heuristics in random 2D instances.}
  % 
  \label{fig:2Dhet_nonunif}
\end{figure}
\begin{figure}[H]
  \centering
  \subcaptionbox{Average regret.%
    \label{fig:2Dhet_nonunif_forecastaware_regret}}[0.49 \textwidth]{\includegraphics[width = \figWidth \textwidth]{Simulation_Results/Synthetic/New/2D_Different_Origins/Non-Uniform/regret_dual.png}}
  \hfill
  \subcaptionbox{Average ratio.%
    \label{fig:2Dhet_nonunif_forecastaware_ratio}}[0.49 \textwidth]{\includegraphics[width = \figWidth \textwidth]{Simulation_Results/Synthetic/New/2D_Different_Origins/Non-Uniform/frac_of_OPT_dual.png}}
  % 
  \caption{Empirical performance of forecast-aware heuristics in random 2D instances.}
  % 
  \label{fig:2Dhet_nonunif_forecastaware}
\end{figure}


\subsection{Different Reward Topology} \label{sec:sec:sim_reward3}

In this appendix, we present numerical results under the two alternative reward topologies defined in \eqref{eq:defn_reward_B} and \eqref{eq:defn_reward_C}.
%
This illustrates the importance of reward topology in determining the performance of algorithms.
%
As in \Cref{sec:sim_unif_1D}, we generate instances of size $\Njob=1000$, where each type is drawn IID from a uniform distribution on $[0,1]$.
%
Overall, we observe that $\PB$ performs on par with the best realistic benchmarks (i.e., outside $\dual$ and $\OPT$). This suggests that $\PB$ could be a leading heuristic even beyond the reward function $r(\type,\type')=\min\{\type,\type'\}$ from delivery pooling, although here we are still restricting to continuous distributions over (one-dimensional) metric spaces.

\subsubsection{Reward function $\reward(\type,\type')=1-|\type-\type'|$.} 

Recall that in this case the potential of a job is $1/2$, independent of the job type, and thus both $\PB$ and $\gre$ are the same algorithm (see \Cref{sec:reward2}).
%
\Cref{fig:R2} shows their superior performance compared to batching-based heuristics.
%
Moreover, as in \Cref{sec:sim_unif_1D}, forecast-aware heuristics are no better than $\PB$. 

\begin{figure}[H]
  \centering
  \subcaptionbox{Average regret.%
    \label{fig:R2_regret}}[0.49 \textwidth]{\includegraphics[width = \figWidth \textwidth]{Simulation_Results/Synthetic/New/1D_Reward_2/regret_all.png}}
  \hfill
  \subcaptionbox{Average ratio.%
    \label{fig:R2_ratio}}[0.49 \textwidth]{\includegraphics[width = \figWidth \textwidth]{Simulation_Results/Synthetic/New/1D_Reward_2/frac_of_OPT_all.png}}
  % 
  \caption{Comparison of average regret and reward ratio for forecast-agnostic heuristics in random 1D instances under $\reward(\type,\type')=1-|\type-\type'|$.}
  % 
  \label{fig:R2}
\end{figure}


\subsubsection{Reward function $\reward(\type,\type')=|\type-\type'|$.}

Consider $\typespace=[0,1]$ and $\reward(\type,\type') = |\type - \type'| $ (analyzed in \Cref{sec: reward 3}).
%
\Cref{fig:R3} highlights that greedy-like algorithms perform better than batching-based heuristics, with $\PB$ outperforming every practical matching algorithm (i.e., aside from $\dual$ and $\OPT$ that use hindsight information).
% 

\begin{figure}[H]
  \centering
  \subcaptionbox{Average regret.%
    \label{fig:R3_regret}}[0.49 \textwidth]{\includegraphics[width = \figWidth \textwidth]{Simulation_Results/Synthetic/New/1D_Matching/Uniform/regret_all.png}}
  \hfill
  \subcaptionbox{Average ratio.%
    \label{fig:R3_ratio}}[0.49 \textwidth]{\includegraphics[width = \figWidth \textwidth]{Simulation_Results/Synthetic/New/1D_Matching/Uniform/frac_of_OPT_all.png}}
  % 
  \caption{Comparison of average regret and reward ratio for forecast-agnostic heuristics in random 1D instances under $\reward(\type,\type')=|\type-\type'|$.}
  % 
  \label{fig:R3}
\end{figure}

% \end{APPENDICES}
\end{APPENDIX}




%%%%%%%%%%%%%%%%%
\end{document}
%%%%%%%%%%%%%%%%%



