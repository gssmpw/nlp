
% Example CSV file embedded in the LaTeX document
\begin{filecontents}{total_bill.csv}
Policy,MAY,JUN,JULY,AUG,SEP,OCT,NOV,DEC,JAN
ILP,6201.05,6713.33,7371.04,9308.91,7231.01,7640.64,6625.89,6079.85,6495.14
RL,6230.34,6878.66,7390.66,9467.51,7247.26,7765.61,6659.98,6410.16,6766.65
CHARGE FIRST LLF,6320.26,6920.76,7433.42,9537.53,7331.83,7800.05,6796.63,6344.38,6672.51
INFORMED GREEDY,6333.8,6955.63,7506.0,9570.76,7402.09,7844.11,6842.92,6393.12,6706.84
TRICKLE LLF,6465.3,7124.97,7546.71,9855.67,7433.71,8085.36,7059.31,6504.01,6859.74
\end{filecontents}

\pgfplotsset{compat=1.15} % Ensure compatibility with pgfplots version

\begin{figure}
    \centering
    \begin{tikzpicture}
    \begin{axis}[
        ybar,
        bar width=8pt,
        width=\textwidth,
        height=0.6\textwidth,
        symbolic x coords={MAY,JUN,JULY,AUG,SEP,OCT,NOV,DEC,JAN},
        xtick=data,
        ylabel={Total Bill (\$)},
        xlabel={Month},
        legend style={at={(0.5,-0.2)},anchor=north,legend columns=-1},
        ymin=6000,
        enlarge x limits=0.1,
        xticklabel style={rotate=45,anchor=east},
        nodes near coords, % Display the value of each bar on top
        every node near coord/.append style={font=\tiny}
    ]

    % Read CSV data
    \pgfplotstableread[col sep=comma]{total_bill.csv}\datatable
    
    % Plot for ILP
    \addplot+[
        color=blue, fill=blue!30
    ] table[x=Policy, y=MAY] {\datatable};

    \addplot+[
        color=red, fill=red!30
    ] table[x=Policy, y=JUN] {\datatable};

    \addplot+[
        color=green, fill=green!30
    ] table[x=Policy, y=JULY] {\datatable};

    \addplot+[
        color=orange, fill=orange!30
    ] table[x=Policy, y=AUG] {\datatable};

    \addplot+[
        color=purple, fill=purple!30
    ] table[x=Policy, y=SEP] {\datatable};

    \legend{ILP, RL, CHARGE FIRST LLF, INFORMED GREEDY, TRICKLE LLF}
    
    \end{axis}
    \end{tikzpicture}
    \caption{Comparison of Monthly Total Bills Across Policies}
\end{figure}


