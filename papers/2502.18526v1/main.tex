%%%%%%%%%%%%%%%%%%%%%%%%%%%%%%%%%%%%%%%%%%%%%%%%%%%%%%%%%%%%%%%%%%%%%%%%

%%% LaTeX Template for AAMAS-2025 (based on sample-sigconf.tex)
%%% Prepared by the AAMAS-2025 Program Chairs based on the version from AAMAS-2025. 

%%%%%%%%%%%%%%%%%%%%%%%%%%%%%%%%%%%%%%%%%%%%%%%%%%%%%%%%%%%%%%%%%%%%%%%%

%%% Start your document with the \documentclass command.


%%% == IMPORTANT ==
%%% Use the first variant below for the final paper (including author information).
%%% Use the second variant below to anonymize your submission (no author information shown).
%%% For further information on anonymity and double-blind reviewing, 
%%% please consult the call for paper information
%%% https://aamas2025.org/index.php/conference/calls/submission-instructions-main-technical-track/

%%%% For anonymized submission, use this
\documentclass[sigconf,nonacm]{acmart}

%%%% For camera-ready, use this
%\documentclass[sigconf]{aamas} 
%%% Load required packages here (note that many are included already).
\usepackage{pgfplots}
\usepackage{pgfplotstable}
%%% for tables %%%
\usepackage{siunitx}
\usepackage{booktabs}
%\usepackage{amssymb}
\usepackage{amsmath}
\usepackage{dblfloatfix}
% \usepackage{natbib}
\usepgfplotslibrary{statistics}
\pgfplotsset{compat=newest}
% \usepackage{filecontents}
\usepgfplotslibrary{fillbetween} % Load the fillbetween library
\usepgfplotslibrary{groupplots}
\usepackage{pgfgantt}

%% For alternative author names format
% \usepackage[noblocks]{authblk}

%%% for multi-header table %%%
\usepackage{multirow}
\usepackage{collcell}
\usepackage{colortbl}
\usepackage{xcolor}
\usepackage{datatool}
% \usepackage{natbib} % Include natbib package

% \usepackage{filecontents}  % For embedding the CSV file in the LaTeX document

\usepackage{balance} % for balancing columns on the final page

%%%%%%%%%%%%%%%%%%%% Removing the legend 2 bars for bar plot %%%%%%%%%%%%%%%%%%%%%%%
% \pgfplotsset{compat=newest,
%         /pgfplots/ybar legend/.style={
%         /pgfplots/legend image code/.code={%
%         %\draw[##1,/tikz/.cd,yshift=-0.25em]
%                 %(0cm,0cm) rectangle (3pt,0.8em);},
%         \draw[##1,/tikz/.cd,bar width=3pt,yshift=-0.2em,bar shift=0pt]
%                 plot coordinates {(0cm,0.8em)};},
% },
% }

%%%%%%%%%%%%%%%%%%%%%%%%%%%%%%%%%%%%%%%%%%%%%%%%%%%%%%%%%%%%%%%%%%%%%%%%

%%% AAMAS-2025 copyright block (do not change!)

% \makeatletter
% \gdef\@copyrightpermission{
%   \begin{minipage}{0.2\columnwidth}
%    \href{https://creativecommons.org/licenses/by/4.0/}{\includegraphics[width=0.90\textwidth]{by}}
%   \end{minipage}\hfill
%   \begin{minipage}{0.8\columnwidth}
%    \href{https://creativecommons.org/licenses/by/4.0/}{This work is licensed under a Creative Commons Attribution International 4.0 License.}
%   \end{minipage}
%   \vspace{5pt}
% }
% \makeatother

% \setcopyright{ifaamas}
\setcopyright{none}
% \acmConference[AAMAS '25]{Proc.\@ of the 24th International Conference
% on Autonomous Agents and Multiagent Systems (AAMAS 2025)}{May 19 -- 23, 2025}
% {Detroit, Michigan, USA}{Y.~Vorobeychik, S.~Das, A.~Nowé  (eds.)}
% \copyrightyear{2025}
% \acmYear{2025}
% \acmDOI{}
% \acmPrice{}
% \acmISBN{}

%%%%%%%%%%%%%%%%%%%%%%%%%%%%%%%%%%%%%%%%%%%%%%%%%%%%%%%%%%%%%%%%%%%%%%%%
% \acmSubmissionID{966}

%%%%%%%%%%%%%%%%%%%%%%%%%%%%%%%%%%%%%%%%%%%%%%%%%%%%%%%%%%%%%%%%%%%%%%%%
% FQ add+ for checking comments: 

% \usepackage{xcolor}
\usepackage{enumitem}
\usepackage{subcaption} % for subfigures
\usepackage{graphicx}
\usepackage{hyperref}
\usepackage{placeins}

 %\usepackage[paperwidth=12.5in, paperheight=11in, top=1in, bottom=1in, left=2in, right=2.8in]{geometry} 
\usepackage[textsize=tiny, colorinlistoftodos]{todonotes}
\newcommand{\ava}[1]{\todo[backgroundcolor=teal!20, linecolor=teal!85!black]{\textbf{Ava:} #1}}
\newcommand{\avai}[1]{\todo[backgroundcolor=teal!20, inline=true,linecolor=teal!85!black]{\textbf{Ava:} #1}}
\newcommand{\aaron}[1]{\todo[backgroundcolor=teal!20, linecolor=teal!85!black]{\textbf{Aaron:} #1}}

\newcommand{\ad}[1]{\todo[backgroundcolor=teal!20, linecolor=teal!85!black]{\textbf{AD:} #1}}
\newcommand{\iad}[1]{\todo[backgroundcolor=teal!20, inline=true, linecolor=teal!85!black]{\textbf{AD:} #1}}
\newcommand{\aaroni}[1]{\todo[backgroundcolor=teal!20, inline=true,linecolor=teal!85!black]{\textbf{Aaron: } #1}}

\newcommand\jpnote[1]{\textcolor{red}{#1}}
\newcommand{\jpt}[1]{\todo[backgroundcolor=teal!20, linecolor=teal!85!black]{\textbf{JP:} #1}}
\newcommand{\rishav}[1]{\todo[backgroundcolor=teal!20, linecolor=teal!85!black]{\textbf{RS:} #1}}

\newcommand{\K}{\mathcal{KWH}}
\newcommand{\Power}{\mathcal{KW}}
\newcommand{\PN}{{PN}_t}
% \usepackage[linesnumbered,ruled,noend,vlined]{algorithm2e} 
\usepackage[linesnumbered,ruled,vlined,noend]{algorithm2e}
% \newcommand\mycommfont[1]{\footnotesize\ttfamily\textcolor{blue}{#1}}

\usepackage{cleveref}
% Define a new command for comment font style
\newcommand{\mycommfont}[1]{\textcolor{blue}{#1}}

\SetCommentSty{mycommfont}
\usepackage{amsmath} 
\newcommand\nissan{Nissan}
% \newcommand\nissan{EV manufacturer}



\newcommand{\SOCR}{{{\it SOC}^{R}}}
\newcommand{\SOCI}{{{\it SOC}^{I}}} 
\newcommand{\SOC}{{{\it SOC}}} 
\newcommand{\SOCMAX}{{{\it SOC}^{max}}} 
\newcommand{\SOCMIN}{{{\it SOC}^{min}}} 
\newcommand{\PrdPeak}{\hat{P}^{max}} 
\newcommand{\CS}{\bar{\mathcal{C}}}
\newcommand{\PowerNeed}{{\it KWH^{R}}} % use R for required. KWH is energy
\newcommand{\ReTime}{\tau^R}
\newcommand{\MaskAction}{A'}

\newcommand{\Building}{\mathcal{B}}
\newcommand{\DepartureTime}{\mathcal{D}} 
\newcommand{\policyGuidanceRate}{{R^{PG}}}
\newcommand{\Mask}{{\it Mask}}
\newcommand{\MILP}{{\it MILP}} 
\newcommand{\Buffer}{\mathbf{BF}} 

% FQ TODO: Need add more! 



% Albation Study
\newcommand{\rlcluster}{{\bf RL\textbackslash{}500} }  % Use 500 training samples train model 
\newcommand{\rlrandom}{{\bf RL\textbackslash{}C} } % random select 60 samples training model 
\newcommand{\rlmorefeature}{{\bf RL\textbackslash{}F} } % useing 100 state features
\newcommand{\random}{{\bf Random\textbackslash{}A} } % random select action + action masking 

\newcommand{\rlnop}{{\bf RL\textbackslash{}P}}
\newcommand{\rlnoa}{{\bf RL\textbackslash{}A}}
\newcommand{\rlnoh}{{\bf RL\textbackslash{}H}}
\newcommand{\rlnoe}{{\bf RL\textbackslash{}E}}












%%% == IMPORTANT ==
%%% Use this command to specify your EasyChair submission number.
%%% In anonymous mode, it will be printed on the first page.
\DeclareMathOperator*{\argmin}{arg\,min}

\acmSubmissionID{966}

%%% Use this command to specify the title of your paper.

\title[AAMAS-2025 Formatting Instructions]{Reinforcement Learning-based Approach for Vehicle-to-Building Charging with Heterogeneous Agents and Long Term Rewards }
% \title[AAMAS-2025 Formatting Instructions]{An Online Approach for Vehicle-to-Building Problem with Heterogeneous Agents and Long Term Rewards }

%%% Provide names, affiliations, and email addresses for all authors.

\author{Fangqi Liu}
\affiliation{
  \institution{Vanderbilt University}
  \city{Nashville, TN}
  \country{USA}}
\email{fangqi.liu@vanderbilt.edu}

\author{Rishav Sen}
\affiliation{
  \institution{Vanderbilt University}
  \city{Nashville, TN}
  \country{USA}}
\email{rishav.sen@vanderbilt.edu}

\author{Jose Paolo Talusan}
\affiliation{
  \institution{Vanderbilt University}
  \city{Nashville, TN}
  \country{USA}}
\email{jose.paolo.talusan@vanderbilt.edu}

\author{Ava Pettet}
\affiliation{
  \institution{Nissan Advanced Technology Center - Silicon Valley}
  \city{Santa Clara, CA}
  \country{USA}}
\email{ava.pettet@nissan-usa.com}

\author{Aaron Kandel}
\affiliation{
  \institution{Nissan Advanced Technology Center - Silicon Valley}
  \city{Santa Clara, CA}
  \country{USA}}
\email{aaron.kandel@nissan-usa.com}

\author{Yoshinori Suzue}
\affiliation{
  \institution{Nissan Advanced Technology Center - Silicon Valley}
  \city{Santa Clara, CA}
  \country{USA}}
\email{yoshinori.suzue@nissan-usa.com}

\author{Ayan Mukhopadhyay}
\affiliation{
  \institution{Vanderbilt University}
  \city{Nashville, TN}
  \country{USA}}
\email{ayan.mukhopadhyay@vanderbilt.edu	}

\author{Abhishek Dubey}
\affiliation{
  \institution{Vanderbilt University}
  \city{Nashville, TN}
  \country{USA}}
\email{abhishek.dubey@vanderbilt.edu}

% \author{Fangqi Liu}
% \author{Rishav Sen}
% \author{Jose Paolo Talusan}
% \email{fangqi.liu@vanderbilt.edu}
% \email{rishav.sen@vanderbilt.edu}
% \email{jose.paolo.talusan@vanderbilt.edu}
% \affiliation{
%   \institution{Vanderbilt University}
%   \city{Nashville, TN}
%   \country{USA}}

% \author{Ava Pettet}
% \author{Aaron Kandel}
% \author{Yoshinori Suzue}
% \email{ava.pettet@nissan-usa.com}
% \email{aaron.kandel@nissan-usa.com}
% \email{yoshinori.suzue@nissan-usa.com}

% \affiliation{
%   \institution{Nissan Advanced Technology Center - Silicon Valley}
%   \city{Santa Clara, CA}
%   \country{USA}}


% \author{Ayan Mukhopadhyay}
% \author{Abhishek Dubey}
% \email{ayan.mukhopadhyay@vanderbilt.edu	}
% \email{abhishek.dubey@vanderbilt.edu}
% \affiliation{
%   \institution{Vanderbilt University}
%   \city{Nashville, TN}
%   \country{USA}}

%%% Use this environment to specify a short abstract for your paper.


\begin{abstract}
End-to-end imitation learning offers a promising approach for training robot policies. However, generalizing to new settings—such as unseen scenes, tasks, and object instances—remains a significant challenge. Although large-scale robot demonstration datasets have shown potential for inducing generalization, they are resource-intensive to scale. In contrast, human video data is abundant and diverse, presenting an attractive alternative. Yet, these human-video datasets lack action labels, complicating their use in imitation learning. Existing methods attempt to extract grounded action representations (e.g., hand poses), but resulting policies struggle to bridge the embodiment gap between human and robot actions.
% our approach
We propose an alternative approach: leveraging language-based reasoning from human videos - essential for guiding robot actions - to train generalizable robot policies. Building on recent advances in reasoning-based policy architectures, we introduce Reasoning through Action-free Data (RAD). RAD learns from both robot demonstration data (with reasoning and action labels) and action-free human video data (with only reasoning labels). The robot data teaches the model to map reasoning to low-level actions, while the action-free data enhances reasoning capabilities. Additionally, we will release a new dataset of 3,377 human-hand demonstrations compatible with the Bridge V2 benchmark. This dataset includes chain-of-thought reasoning annotations and hand-tracking data to help facilitate future work on reasoning-driven robot learning.
% experiments
Our experiments demonstrate that RAD enables effective transfer across the embodiment gap, allowing robots to perform tasks seen only in action-free data. Furthermore, scaling up action-free reasoning data significantly improves policy performance and generalization to novel tasks. These results highlight the promise of reasoning-driven learning from action-free datasets for advancing generalizable robot control. 
% releasing dataset
Website: \href{https://rad-generalization.github.io}{here}.



\end{abstract}

%%% The code below was generated by the tool at http://dl.acm.org/ccs.cfm.
%%% Please replace this example with code appropriate for your own paper.

\begin{CCSXML}
<ccs2012>
   <concept>
<concept_id>10010147.10010178.10010199.10010201</concept_id>
       <concept_desc>Computing methodologies~Planning under uncertainty</concept_desc>
       <concept_significance>500</concept_significance>
       </concept>
 </ccs2012>
\end{CCSXML}

\ccsdesc[500]{Computing methodologies~Planning under uncertainty}

% \ccsdesc[500]{Theory of computation~Algorithmic game theory}
% \ccsdesc[500]{Mathematics of computing~Matchings and factors}
% \ccsdesc[300]{Mathematics of computing~Combinatorial algorithms}
% \ccsdesc[300]{Theory of computation~Approximation algorithms analysis}

%%% Use this command to specify a few keywords describing your work.
%%% Keywords should be separated by commas.

\keywords{Reinforcement Learning; Optimization; Electric Vehicle Charging}

%%%%%%%%%%%%%%%%%%%%%%%%%%%%%%%%%%%%%%%%%%%%%%%%%%%%%%%%%%%%%%%%%%%%%%%%

%%% Include any author-defined commands here.
         
\newcommand{\BibTeX}{\rm B\kern-.05em{\sc i\kern-.025em b}\kern-.08em\TeX}

%%%%%%%%%%%%%%%%%%%%%%%%%%%%%%%%%%%%%%%%%%%%%%%%%%%%%%%%%%%%%%%%%%%%%%%%

\begin{document}
% \iad{please check the errors in compilation}
%%% The following commands remove the headers in your paper. For final 
%%% papers, these will be inserted during the pagination process.

\pagestyle{fancy}
\fancyhead{}

%%% The next command prints the information defined in the preamble. 
\maketitle 
%%%%%%%%%%%%%%%%%%%%%%%%%%%%%%%%%%%%%%%%%%%%%%%%%%%%%%%%%%%%%%%%%%%%%%%%

\begin{figure}[ht]
    \centering
    \includegraphics[width=0.8\linewidth]{graphs/greater_than_naive.pdf}
    \vspace{0.5cm}
    \includegraphics[width=0.8\linewidth]{graphs/p1_bottom.png}
    \vspace{-5pt}
    \caption{\textcolor{positional}{Positional} vs.\ \textcolor{nonpositional}{non-positional} circuits. In a \textcolor{nonpositional}{non-positional} circuit, the same edges must be included at all positions. A \textcolor{positional}{positional} circuit can distinguish between the same edge at different positions. This specificity yields better trade-offs between circuit size and faithfulness. It can also increase both precision and recall.}
    \label{fig:p1}
    \vspace{-5pt}
\end{figure}

\section{Introduction}

\looseness=-1
A primary goal of interpretability research is to characterize the internal mechanisms in language models (LMs) and other NLP models. 
A core approach in this area is \textbf{circuit discovery}---identifying the minimal subgraph within the model's computation graph that performs a specific task \citep{olah2021framework,olah-mech}.
Typically, the nodes of a circuit represent model components (e.g., attention heads, neurons, or layers).
While manual circuit discovery methods can yield position-specific insights \citep{wanginterpretability,goldowskydill2023localizingmodelbehaviorpath}, \emph{automatic methods often overlook positional information}, treating components as uniformly relevant across all input token positions \citep{conmytowards,syed2023attribution}. 
For instance, if an attention head is included in a circuit, it is assumed to contribute equally to the computation for every position in the input sequence.
The assumption that circuits are position-invariant ignores the fact that different positions often require distinct computations.
By ignoring positions, current methods limit their ability to capture mechanisms that operate across positions, such as interactions between attention heads across positions.

In this study, we start by demonstrating that positional agnosticism is a significant limitation (\S\ref{sec:motivating}). Then, to address these limitations, we introduce a new approach: position-aware edge attribution patching (PEAP; \S\ref{sec:full_circ_discovery}; Figure~\ref{fig:p1}). Current approaches  assume that if an edge is in a circuit, then the same edge will be in the circuit at all positions, thus leading to low precision. It is also assumed that an edge's importance should be aggregated across positions before deciding whether it should be included in the circuit; this can lead to cancellation effects, and thus low recall. PEAP instead allows us to compute the importance of cross-positional edges, and separately evaluates edge importance at each position. We show that this leads to smaller and more accurate circuits; see Figure~\ref{fig:p1}.

Incorporating positional information into circuit discovery is straightforward when inputs have the same length and structure across examples.

However, realistic datasets are not nearly this templatic.
How, then, can we incorporate positional information into automatic circuit discovery?
To address this challenge, we propose \textbf{schemas} (\S\ref{sec:schema}). 
Schemas assign semantic labels to spans of tokens, enabling information aggregation across examples even when the spans differ in length.

For example, in the input ``The \textcolor{positional}{war} lasted from 1453 to 14\underline{\hspace{1em}},'' the span ``\textcolor{positional}{war}'' could be labeled as ``\emph{Subject}''.
This enables handling spans with varying lengths: the phrase ``\textcolor{positional}{Black Plague}'' in another example can be treated as a single positional span with the same role as ``\textcolor{positional}{war}''.
In experiments with two LMs and three tasks, we find that circuits discovered using schemas achieve a better trade-off between circuit size and faithfulness to the model's behavior than position-agnostic circuits.
Importantly, position-aware circuits offer a more precise representation of the underlying mechanisms, providing a more concise foundation for mechanistic explanations.

We also present a fully automated pipeline for schema generation and application (\S\ref{sec:schema-generation}) using large language models (LLMs). 
We evaluate the quality of the generated schemas and their utility in discovering position-aware circuits (\S\ref{sec:schema-eval}).
Notably, circuits derived using automatically generated and applied schemas achieve comparable faithfulness scores to circuits discovered with human-designed and manually applied schemas.

We summarize our contributions as follows:
\begin{itemize}[noitemsep,leftmargin=*,topsep=1pt,parsep=1pt]
    \item Introduce a position-aware circuit discovery method, which obtains better faithfulness than position-agnostic discovery.  
    \item Introduce dataset schemas,  facilitating positional circuit discovery in more naturalistic settings. 
    \item Develop an automated schema generation and application pipeline with LLMs, yielding schemas that are comparable to manually-annotated ones.
\end{itemize}

\section{Problem Formulation}
\label{sec:problem_statement}

%

\makeatletter
\renewcommand*{\coloneq}{\mathrel{\rlap{%
  \raisebox{0.3ex}{$\m@th\cdot$}}%
  \raisebox{-0.3ex}{$\m@th\cdot$}}%
  =}
\makeatother


\ifx\coloneqq\undefined
  \makeatletter
  \newcommand*{\coloneqq}{\mathrel{{%
  \raisebox{0.110ex}{$\m@th::$}}}
  =}
  \makeatother
\else
  \makeatletter
  \renewcommand*{\coloneqq}{\mathrel{{%
  \raisebox{0.110ex}{$\m@th::$}}}
  =}
  \makeatother
\fi


\noindent \textbf{Charger and Time Intervals}: Consider the building has $N$ heterogeneous chargers $\mathcal{C} = \{C_1, C_2, \dots,C_N\}$. Each charger $C_i$ has limits on the charging rate, minimum $C_i^{min}$ and maximum $C_i^{max}$; $C_i^{min} < 0$ implies the charger $C_i$ is bi-directional and can discharge and $C_i^{min} = 0$ represents a unidirectional charger with no discharging. We assume that all chargers are designed to be able to charge at maximum rates simultaneously, i.e., $\sum_{i=1}^{i=N} C_i^{max} < \text{maximum rated capacity of the building} $.  
The planning horizon is one billing period, usually a month, which we divide into equal-sized fixed time intervals $\mathcal{T} = \{T_1, T_2, \dots\, T_{end}\}$, where $T_{j}-T_{j-1}=\delta$ (we use  $\delta$ = 0.25 hours). The choice of $\delta$ is user-specific and provides a stable decision epoch, preventing rapid changes in the charging rate.

\noindent \textbf{Charging Power}: Let us assume that the function $\mathcal{P}:  \mathcal{C} \times \mathcal{T}  \rightarrow \Re$ specifies the power consumed by the charger $C_i$ at time $T_j$. If the power is zero, the charger is not active, and if the power is negative, the charger discharges, acting as an energy source. Note that by construction $P(C_i,T_j) \in [C_i^{min},C_i^{max}]$. Let us also assume that function $\mathcal{B}: \mathcal{T}  \rightarrow \Re^{+} $ specifies the average building power consumed in $\delta$ time interval. 
Given the charger and the building power consumption, we can calculate the total cost for the billing period. The parts of the total cost are based on the property type, time of day, and state of the power grid and are based upon the rules and regulations set by the local transmission system operator (TSO) and distribution system operator (DSO). These parts include energy expenses for building power and charging, which vary with peak and off-peak hours, as well as demand charges based on the peak power draw over a longer-term period. 

Let the price of the energy consumed is given by $\theta_E : \mathcal{T}  \rightarrow \Re^{+}$ (in \$/kWh). In practice, the Time-of-Use (TOU) electricity rates do not vary continuously and are rather divided into two parts each day, i.e., a peak and a non-peak period. 
Then, the total cost of the energy consumed is  $\Theta_E(\mathcal{P})= \sum_{j=1}^{j=end} \left(\sum_{i=1}^{i=N} (P(C_i,T_j)) + \Building (T_j)\right) \times \theta_E  (T_j) \times \delta$. Effectively, $\Theta_E$ is a function of charging power  $\mathcal{P}=\{P(C_i, T_j)| C_i\in \mathcal{C}, T_j\in \mathcal{T}\}$.  
 
\noindent \textbf{Demand Charge}: The demand charge is calculated using the maximum (peak) power consumed during any time interval in the billing period, with the demand price denoted as $\theta_D$ (in \$/kW).
Let $P^{max} = \max_{j=1}^{j=end} (\sum_{i=1}^{i=N}$ $P(C_i,T_j)) + \Building(T_j)$ denote the maximum power consumed. The demand charge is given by $\Theta_D(\mathcal{P})= \theta_D \times P^{max} \times \delta$, which is a function of charging power  $\mathcal{P}$. Hence, the total cost of energy bought from the power grid is  $\Theta_E(\mathcal{P})+\Theta_D(\mathcal{P})$. To minimize the cost, we must reduce the net power usage when the cost $\theta_E$ is high and manage the power peaks to ensure $P^{max}$ remains as low as possible. Often, the demand charge is levied to ensure that the industrial buildings do not put excess burden on the power grid. In our problem, we use estimates of peak power and denote it by $\hat{P}^{max}$. It is important to note that the demand charge is typically applied during peak hours of the TOU electricity rate, as reflected in our formulation.
%The challenge with being able to lower the demand charge is the coupled temporal complexity. Assume if along a billing period of one month, it is not possible to reduce $P^{max}$, which will occur in the last week of the month, we may as well utilize high charging powers in early part of the month. 

%Let us now describe the decision problem given the structure above. 
\noindent \textbf{Electric Vehicle Sessions}: Assume that during the billing period $\mathcal{T}$, a set of electric vehicles, denoted as $\mathcal{V}$, are serviced at the building. Each EV $V$ is characterized by its arrival time $\mathcal{A}: \mathcal{V} \rightarrow  \mathcal{T}$ and departure time $\mathcal{D}: \mathcal{V} \rightarrow  \mathcal{T}$. Note that if the same vehicle arrives more than once, we will treat it as a separate session. If the EV arrives between time slots $[T_{i-1}, T_{i}]$, we consider its effective arrival time as $\mathcal{A}(V) = T_i$. Similarly, if the vehicle departs between $[T_{j}, T_{j+1}]$, we consider its effective departure time as $\mathcal{D}(V) = T_j$. EV sessions are contiguous, i.e., EV is expected to remain at the site between $\mathcal{A}(V)$ and $\mathcal{D}(V)$, for $\forall V \in \mathcal{V}$. 
%Additionally, it is important to emphasize that we may know the estimated arrival time $\hat{\mathcal{A}}(V)$ and departure time $\hat{\mathcal{D}}(V)$ for each session, but true arrival and departure times are unknown ahead of time and can only be observed once they happen. 
For each  $V$, we know the initial state of charge   $\SOCI: \mathcal{V} \rightarrow \Re^+$ and the required final state of charge (measured as a percentage of the battery capacity)   $\SOCR: \mathcal{V} \rightarrow \Re^+$ upon arrival. $\SOCMIN: \mathcal{V} \rightarrow \Re^+$ is the minimum allowed SoC for the car i.e., the car cannot be discharged below this value, and $\SOCMAX: \mathcal{V} \rightarrow \Re^+$  is the maximum allowed SoC for the car. The minimum and maximum bounds are specified by the EV manufacturer, considering the impact of charging and discharging on battery health. ${\it CAP}:\mathcal{V} \rightarrow \Re^+$ denotes the vehicle's battery capacity in kWh. We track the current SoC of the EV using ${\it SOC}$, where ${\it SOC}: \mathcal{V} \times \mathcal{T} \rightarrow \Re^+$ and it is defined later.



% For each vehicle session, we record 



% and are assigned to available chargers in $\mathcal{C}$, with each EV represented by $V$.   
% Each EV $V$ is characterized by its arrival time $\mathcal{A}(V)$, its scheduled departure time $\mathcal{D}(V)$, and its energy capacity $C(V)$. Additionally, each EV has a required state of charge, $\SOCR(V)$, and an initial state of charge, $\SOCI(V)$.
% We assume that EVs arrive at continuous time. If an EV arrives between time slots $[T_{j-1}, T_{j}]$, we consider its effective arrival time as $\mathcal{A}(V) = T_j$. Similarly, if an EV departs between $[T_{j}, T_{j+1}]$, we consider its effective departure time as $\mathcal{D}(V) = T_j$.



%lowering the demand charge though is that if we know in future there is no possibility of reducing demand charge then we can 


%We describe the problem as a series of problems starting from the basic V2B problem and expanding to the more complex problem of optimizing charging policy across a month. ~\Cref{table:notations} summarizes the key symbols utilized in the paper.
%
%These costs vary based on the property type, time of day, state of grid, and are based upon the rules and regulations set by the local transmission system operator (TSO) and distribution system operator (DSO). These costs include energy expenses for building power and charging, which vary with peak and off-peak hours, as well as demand charges based on the peak power draw over a longer-term period (e.g., one month). 


%To formalize this problem, we first provide~\Cref{table:notations} summarizing all notations used in this paper. 

% \subsection{Basic Vehicle-to-Building Problem} 
% % For any V2B problem we suppose that there are a set of mixed-mode chargers denoted as $\mathcal{C} = \{C^1, C^2, \dots\}$. Each charger $C_i$ is characterized by available charging powers within the range $[P^i_{\text{min}}, P^i_{\text{max}}]$, where $P^i_{\text{min}} < 0$ if charger $C_i$ is bi-directional. 
% %
% % Simple V2B 
% \textbf{Basic V2B Problem}: The objective of the Vehicle-to-Building (V2B) problem is to optimally schedule Electric Vehicle (EV) charging and discharging throughout a billing period to minimize the total electricity bill, while ensuring that each EV reaches its required state of charge (SoC) (\% of its capacity) by the time of departure. This required SoC represents the portion of the EV's total energy capacity that must be met before it leaves.
 
% %We focus on controlling the charging powers of EV-connected chargers online during the billing period $\mathcal{T}$. We adjust the charging power $P^i_t$ (in kilowatts) of each charger $C_i$ at each time slot $t$, indexed by $\{0, 1, \dots, t_{\text{end}}\}$, corresponding to a fixed time interval $\Delta t$ (in hours). The values $0$ and $t_{\text{end}}$ denote the start and end of the billing period, respectively. 

% % EVs arriving during the billing period $\mathcal{T}$ are assigned available chargers. We let $v$ represent an EV arriving during $\mathcal{T}$, which would be assigned to a charger in $ \mathcal{C}$. 
% %${\it V2C}(v)\in \mathcal{C}$. 
% % Inthe problem, each EV $v$ is specified by its arrival time $T_\mathcal{A}(V)$, predefined departure time $T_\mathcal{D}(V)$, and a power capacity ${\it Cap}(v)$. Additionally, each EV has a required State of Charge (SoC) ${\it SOC}^{req}(v)$ that indicates the desired energy ratio of its power capacity that must be reached before departure, and an initial SoC ${\it SoC^{in}}(v)$. 
% % Here we use $v_t^i$ to indicate the EV connected to charger $C_i$ at time slot $t$. 
% Assume that during the billing period $\mathcal{T}$, a set of EVs, denoted as $\mathcal{V}$, arrive and are assigned to available chargers in $\mathcal{C}$, with each EV represented by $V$.   
% Each EV $V$ is characterized by its arrival time $\mathcal{A}(V)$, its scheduled departure time $\mathcal{D}(V)$, and its energy capacity $C(V)$. Additionally, each EV has a required state of charge, $\SOCR(V)$, and an initial state of charge, $\SOCI(V)$.
% We assume that EVs arrive at continuous time. If an EV arrives between time slots $[T_{j-1}, T_{j}]$, we consider its effective arrival time as $\mathcal{A}(V) = T_j$. Similarly, if an EV departs between $[T_{j}, T_{j+1}]$, we consider its effective departure time as $\mathcal{D}(V) = T_j$.

%which indicates the target energy level (as a percentage of its capacity) that must be reached before its departure, 
%We use $v_t^i$ to represent the EV connected to charger $C_i$ at time slot $t$. The SoC of each charger-connected EV is tracked at each time slot, ${\it SOC}_t(v^i_t)$, based on the charging power of its connected charger, and initialized by ${\it SoC^{in}}(v)$. It is updated according to the following equation: 
\noindent \textbf{Charger Assignment}: 
{
Our approach employs a two-layer decision-making process for EV charging optimization. First, a heuristic assigns EVs to chargers upon arrival. Second, an RL-based policy optimizes charging rates at fixed intervals. 
}
We define an EV assignment function $\eta: \mathcal{V} \rightarrow \mathcal{C}$, where ($V \in \mathcal{V}$) $\eta(V) = C_i$ indicates the charger assigned to EV $V$. Correspondingly, we also maintain a charger-EV occupancy function $\phi: \mathcal{C} \times \mathcal{T} \rightarrow \mathcal{V}$, where $\phi(C_i, T_j) = V$, representing the connection of charger $C_i$ with EV $V$ at time $T_j$.  
The correlation of these two functions can be expressed as $\phi(\eta(V), T_j) = V, \ \text{s.t.}\ \mathcal{A}(V) \leq T_j \leq \mathcal{D}(V) $
indicating that if EV $V$ is assigned to charger $C_i$ through the function $\eta$, then at any time slot within its stay duration, it is confirmed that EV $V$ is connected to charger $C_i$. If no EV is connected to the charger at time $T_j$, the function may return a $\emptyset$ denoting an inactive state, expressed as $\phi(C_i, T_j) = \emptyset$. 
% This highlights the dynamic nature of charger assignments, ensuring that no two EVs share a charger simultaneously. Our FIFO policy prioritizes bidirectional chargers as the optimal strategy (see Table 5 in the Appendix), assigning EVs accordingly to optimize charging efficiency.  
This underscores the dynamic nature of charger assignments, which ensures that no two electric vehicles share a charger simultaneously. Our FIFO policy prioritizes bidirectional chargers as the optimal strategy (see ~\Cref{table:charger_assignment_policies} in the appendix\footnote{The full paper, including the appendix, is available on arXiv.}), enhancing charging efficiency. 
% emphasizing the dynamic nature of the connection function. Note that two vehicles cannot be connected to a charger simultaneously. We consider a first-in, first-out policy that assigns EVs to bidirectional chargers first. 
% , breaking ties assigning to later departing cars, 
% It is also important to emphasize that if the chargers are homogeneous and their count is greater than the number of vehicles, then the assignment problem will be trivial. Otherwise, the assignment problem is part of the decision process, as is our case. 
We also maintain the connection between the assigned charger and the EV until departure. For EV charging, we approximate a linear charging profile, following prior work~\cite{sundstrom2010optimization}. The SoC is updated at each time slot  $T_j$  using the following equation: 
\begin{equation}
{\it SOC}(V, T_{j+1}) = {\it SOC}(V, T_j) + \textstyle\frac{P(\eta(V), T_j)\times \delta} {{\it CAP}(V)}
\label{eq: soc}
\end{equation} 


\iffalse
For the EV assignment approach, we utilize a FIFO (First In, First Out) procedure that prioritizes bi-directional charging and charger ID. This ensures that the earliest arriving EVs are charged or discharged first while maximizing the use of bi-directional chargers, which offers the potential for peak shaving and demand charge reduction. Here, we use the function $\text{ID}(C_i)$ to indicate the charger ID of $C_i$, prioritizing EV assignment to bi-directional chargers by using smaller ID numbers. To enable this FIFO procedure, we set constraints to ensure the FIFO procedure: 
\begin{equation}
    \mathcal{A}(V) < A(V_{k+1}) \implies \text{ID}(\eta(V)) < \text{ID}(\eta(V_{k+1}))
\end{equation} 
This constraint indicates that if the arrival time of EV $V$ is earlier than that of EV $V_{k+1}$, then the ID of the charger assigned to $V$ must also be smaller than the ID of the charger assigned to $V_{k+1}$.  


We then track the state of charge (SoC) of EVs after they connect to chargers based on the charging power of the chargers. We define the function $SoC: \mathcal{V} \times \mathcal{T} \rightarrow [0, 1]$. The SoC update function is given by  
\begin{equation}
SoC(V, T_{I+1}) = SoC(V, T_j) + \frac{P(\eta(V), T_j) \times\delta }{C(V)},
\end{equation} 
with time slot $T_j \geq \mathcal{A}(V)$ and $T_{j+1} \leq \mathcal{D}(V)$, and $P(\eta(V), T_j)$ represents the charging power of the charger assigned to EV $V$ at time $T_j$, and $\Delta $ is the time interval.  
% We track the SoC of each charger-connected EV at each time slot, ${\it SOC}_t(v^i_t)$, based on its connected charger's charging power, initialized by ${\it SoC^{in}}(v)$. It is updated according to: 
% \begin{equation}
%     % {\it SOC}_{t+1}(v) = {\it SOC}_t(v) + P^{{\it V2C}(v)}_{t}\delta t / {\it Cap}(v), 
%     {\it SOC}_{t+1}(v_{t+1}^i) = {\it SOC}_{t}(v_t^i) + (P^i_t \times\delta t)/{\it Cap}(v_t^i)
% \label{eq: SoC}
% \end{equation}
% Here we use $v_t^i$ to indicate the EV connected to charger $C_i$ at time slot $t$.  
\fi 



\noindent \textbf{Feasibility}:
The set \textit{Feasible} indicates the feasible solutions that satisfy the following constraints:
\begin{align}
    & \forall C_i \in \mathcal{C}, \forall T_j \in \mathcal{T}: C_i^{min} \leq P(C_i, T_j) \leq C_i^{max} \label{eq:charging_rate} \\
    & \forall C_i \in \mathcal{C}, \forall T_j \in \mathcal{T}, \forall V \in \mathcal{V}: {\it SOC}(V, T_j) \geq \SOCMIN(V)\label{eq:soc_min} \\
    & \forall C_i \in \mathcal{C}, \forall T_j \in \mathcal{T}, \forall V \in \mathcal{V}: {\it SOC}(V, T_j)\leq \SOCMAX(V)\label{eq:soc_max} \\
    & \forall T_j \in \mathcal{T}: \textstyle\sum_{C_i \in \mathcal{C}} P(C_i, T_j) + \mathcal{B}(T_j) \geq 0  \label{eq:building_power}     
\end{align} 
Here, Constraint~(\ref{eq:charging_rate}) guarantees a valid charging action range, Constraints~(\ref{eq:soc_min} and \ref{eq:soc_max}) ensures that each EV's SoC remains within an acceptable range, and Constraint~(\ref{eq:building_power}) ensures that discharging power does not exceed building power. 




\noindent \textbf{Objectives}: 
% \SOCMIN(V) \geq \text{SoC}(V, T_j) \leq \SOCMAX(V), \forall C_i \in \mathcal{C}, \forall T_j \in \mathcal{T} \label{eq:soc_range} \\ 
% & \mathcal{B}(T_j) + \sum_{C_i \in \mathcal{C}} P(C_i, T_j) \geq 0, \quad \forall T_j \in \mathcal{T} \label{eq:building_power} 
% One of the primary objectives is to ensure the vehicles are charged to the requirement by the time they leave. The decision variable is the charger assignment and charging power per interval. % Thus, if $\eta$ is the charger assignment and $\mathcal{P}$ is the charging power decision per charger per interval, then % The objective of the energy trading problem is to maximize the amount of energy traded.
% Formally, an optimal solution to the energy trading problem is
% To fulfill charging requirements, one objective of this V2B problem is to decide the charger charging powers across all time slots, denoted as $\mathcal{P} = \{P(C_i, T_j)$| $C_i \in \mathcal{C}$ and $T_j \in \mathcal{T}\}$. The goal is to minimize the discrepancy between each EV's SoC at departure and its required SoC, denoted as $\Delta_{{\it SOC}}(\mathcal{P})$, computed by:
% \begin{equation}
% \arg\min_{\mathcal{P}}\delta_{{\it SOC}}(\mathcal{P}) = \sum_{V \in \mathcal{V}} \max\left( {\it SOC}^R(V) - {\it SOC}(V, \mathcal{D}(V)), 0 \right)
% \label{eq: soc}
% \end{equation}  
One of our objectives for the V2B problem is to minimize the total cost over the billing period, incorporating the Time-Of-Use (TOU) electricity rates and demand charges. This objective is expressed as:
% $\min_{(\eta,\mathcal{P}) \in \textit{Feasible}}$
% %(\mathcal{B}, \theta_E, \theta_D, \mathcal{V}, \SOCR, \SOCI, \SOCMIN,\SOCMAX, \mathcal{C})} \\  
% $\left( \Theta_E (\mathcal{P}) + \Theta_D(\mathcal{P}) \right)$
\begin{align}
\label{eq: billing}
\begin{split}
\min_{(\eta,\mathcal{P}) \, \in \textit{Feasible}}%(\mathcal{B}, \theta_E, \theta_D, \mathcal{V}, \SOCR, \SOCI, \SOCMIN,\SOCMAX, \mathcal{C})} \\  
\left( \Theta_E (\mathcal{P}) + \Theta_D(\mathcal{P}) \right)
\end{split}
\end{align}

The second objective ensures that vehicles are charged to their requirement, $\SOCR$, by the time they leave.
\begin{align}
\label{eq: soc_penalty}
\begin{split}
\min_{(\eta,\mathcal{P}) \in \textit{Feasible}} \textstyle\sum_{V \in \mathcal{V}} \max(\SOCR(V) - {\it SOC}(V,\mathcal{D}(V)), 0)
\end{split}
\end{align}
% The inner \texttt{max} function guarantees that all EV users' energy requirements are satisfied, even if this leads to overcharging the vehicle, as dictated by the problem's conditions.
The inner \texttt{max} function ensures EV users' energy requirements are met, even if overcharging occurs.
However, in practical scenarios, short stays may make meeting the SoC requirement impossible. To address this, we reformulate the objectives into a multi-weighted framework.
The optimal charger assignment and actions are then determined by optimizing these combined objectives.




% \begin{equation}
% \arg\min_{\mathcal{P}} \left( \Theta_E (\mathcal{P}) + \Theta_D(\mathcal{P}) \right)
% \label{eq: billing}
% \end{equation} 



% We consider two approaches to address the multi-objective aspect of this problem. First, we can set the SoC objective (as shown in Equation~(\ref{eq: soc})) as a constraint, ensuring that the SoC discrepancy is zero at the time of EV departure by requiring $\Delta_{{\it SoC}}(\mathcal{P}) = 0$. However, in practical scenarios, meeting the SoC requirement may be impossible due to short stay durations and high demand. Therefore, we opt for the second approach, which involves using a weighted sum of both objective functions:
% \begin{equation}
% \arg\min_{\mathcal{P}} \alpha \times \left( \Theta_E (\mathcal{P}) + \Theta_D(\mathcal{P}) \right) + \beta \times\delta_{{\it SoC}}(\mathcal{P}) 
% \label{eq: billing} 
% \end{equation} 
% In this method, we assign a high penalty coefficient $\beta$ for missing SoC to prioritize the SoC charging requirements.


% Moreover, we impose the following constraints on  $\mathcal{P}$:  \begin{subequations}

% Here, the set \(\{v_t^i \mid t \in \mathcal{T}, C_i \in \mathcal{C}\}\) represents all arriving EVs connected to chargers during the billing period \(\mathcal{T}\).


% V2B for Industrial Profiles
%Commercial and industrial complexes are often billed differently that residential houses. Their electricity rates often vary across a single day based on a rate structure used by electric utilities. 
% Our V2B scenario for commercial and industrial complexes
% work under Time-of-use (TOU) rates policies electricity bill. TOU rates incentives customers to use electricity during off-peak hours and reduce usage during peak hours. This pricing model reflects the varying cost of generating and delivering electricity at different times of the day, which is influenced by the overall demand on the electrical grid. 
%By adjusting their energy consumption patterns—such as running heavy machinery or charging electric vehicles during off-peak hours—industrial and commercial businesses can take advantage of lower rates and significantly reduce their overall electricity costs.

\iffalse
% \textbf{V2B for Commercial Buildings}: 
% This work is specifically designed for commercial and industrial smart buildings with EV chargers. By optimally controlling the charger charging power $\mathcal{P}$, the secondary objective of the V2B problem is to minimize the total electricity bill, considering both building power consumption and EV charging, {\color{black} which is governed by Time-of-Use (TOU) rate and demand charge policies.}
% TOU rates encourage customers to shift electricity usage to off-peak hours and reduce consumption during peak periods, reflecting the fluctuating costs of electricity generation and delivery based on grid demand.
% We denote the peak hour time slots during the billing period $\mathcal{T}$ as $\mathcal{T}_P$. 


% In addition to kilowatt-hour (kWh) consumption charges, businesses incur {\it demand charge} based on their peak power demand (measured in kilowatts, kW) during the billing period. %These charges ensure utilities maintain the capacity to meet maximum electricity needs. 
% For industrial businesses, managing demand charges is crucial for controlling energy costs. the V2B problem focuses on reducing peak demand by shifting loads, such as discharging EVs to supply building energy needs, thereby lowering overall grid power consumption. Here, we denote the demand price as  ${\it Pr}^{D}$ (in \$/kW).  

% Above all, we design the second objective of the V2B problem to optimize the charging power sequence $\mathcal{P}$ to minimize the total bill over the billing period, incorporating TOU rates and demand charges. This total cost, denoted as $\hat{Cost}(\mathcal{P})$, is computed by: 
% \begin{equation}
% \displaystyle
% \arg\min \hat{\Theta}(\mathcal{P}) = \Theta_E(\mathcal{P}) + \Theta^{D}(\mathcal{P})
% \label{eq:objective_1}
% \end{equation}

% where ${\it B}(T_j)$ denotes the building power at time slot $t$.  %and ${\it Pr}^E_t$ represents the electricity price, which varies over time and is typically higher during peak hours and lower during off-peak hours.  
% and ${\it Cost}^{DC}(\mathcal{P})$ denotes the demand charge, is determined based on the peak power consumption from both building power and charging. This charge can be calculated by: 
% \begin{equation}
%     \Theta_D(\mathcal{P}) =\max_{T_j\in\mathcal{T}_{P}} (\mathcal{B}(T_j)+\sum_{C_i\in\mathcal{C}}{\it P}(C_i, T_j))\times \theta_D 
% \end{equation}. 

% % V2B under uncertainty
% \textbf{V2B Under Uncertainty}: Uncertainty in the V2B problem arises from various factors. User behavior, such as the actual arrival and departure times of EVs, can deviate from the planned schedule due to traffic conditions or individual preferences, complicating predictions. The SoC required by EV users also varies among individuals under different circumstances. Additionally, building power demand can fluctuate significantly across seasons and may experience sudden changes, prompting Transmission System Operators (TSOs) to activate emergency load reduction programs to alleviate grid stress. Each of these uncertainties impact the overall objective, making their consideration essential in the V2B problem. 

%  \subsection{\nissan{} Use Case} 
% Our work aims to collaborate with our industry partner, \nissan{}, an EV manufacturer with a smart building equipped with both unidirectional and bidirectional chargers, as shown in ~\Cref{fig: EV chargers}. We are developing an online EV charging system to minimize the total electricity bill over a billing period of one month under a Time-of-Use (TOU) rate policy for their headquarters. 

% \noindent \textbf{Charger and EV availability}: Currently, {\color{black} Nissan Advanced Technology Center - Silicon Valley (NATCSV), is a smart building that offers 8 unidirectional and 5 bidirectional chargers for employees. Additionally, they manufacture EVs with bidirectional capabilities such as the Nissan Leaf.} These vehicles have battery sizes ranging from 42 kWh to 60 kWh. Thus, several of them are capable of providing substantial power back to the building when needed.

% \noindent \textbf{Service Classification:} Their headquarters fall within the bounds of Silicon Valley Power (SVP), a not-for-profit municipal electric utility owned and operated by the City of Santa Clara, California, United States. Due to their size and energy consumption, NATCSV is classified as a small industrial service under SVP.
% % For commercial and industrial customers whose energy use exceeds 8,000 kWh per month, but whose maximum electric demand does not exceed 4,000 (kW),
% This means that aside from the monthly electricity bills, they are also subject to time-of-use (TOU) rates and demand charges. The presence of these factors make NATCSV a candidate for V2B optimization. TOU introduces variations in the electricity rates, billing them higher during peak hours (6:00AM to 10:00PM) compared to off-peak hours. While demand charge adds a flat rate multiplier to their total bill based on the highest average kW delivery of the 15-minutes interval in which
% such delivery is greater than in any other 15-minute interval in the month.
% %
% The goal is to leverage the combination of bidirectional chargers, EVs, and a smart charging policy to potentially reduce electricity costs and demand charges. 

% \noindent \textbf{Data}:To leverage this opportunity, \nissan{} has collected building demand and EV telemetry data related to vehicle arrival/departure schedules and SoC needs from their headquarters and EVs. They have amassed data from May 2023 to January 2024 (detailed in Section~\ref{ssec:data}), which will be utilized in this paper to ensure the authenticity of the results. 

\fi
\section{Related Work}
\label{sec:related_work}
% \begin{table*}[ht]
\centering
\footnotesize
\caption{Comparison of state-of-the-art approaches for EV charging problem with our approach.}
\begin{tabular}
{|p{2.8cm}|p{2cm}|p{2cm}|p{2cm}|p{1.1cm}|p{1.1cm}|p{1.0cm}|p{0.6cm}|p{1.0cm}|}
\hline
    \textbf{Reference} & \textbf{Approach}  & \textbf{Objective} & \textbf{Action Space} & \textbf{Planning Horizon} & \textbf{Discharge} & \textbf{Mobility} & \textbf{Req. SoC}& \textbf{Demand Charge} \\  \hline
% \cite{alizadeh2014demand}  & Deep Q-network & 85 & \checkmark & \checkmark & & \\ \hline
% \cite{oconnell2010integration}  &  &  & & & \checkmark & \\ \hline
\citeauthor{AORC2013}~\cite{AORC2013} %\href{https://dl.acm.org/doi/pdf/10.1145/2487166.2487178?casa_token=09DlHwzE63wAAAAA:6bgeFK7d-LoIdgRAl30dA5gbZHE8eRdWRckS_m-6oCBGkEuy3XNbltlZhz3CZGt9TINQieD8wJrC}{2} 
& Distributed control algorithm & EV charging Fairness Allocation & Continuous power rate of 2 chargers & Single day &  &  & & \\ \hline 

\citeauthor{5986769}~\cite{5986769} %\href{https://ieeexplore.ieee.org/stamp/stamp.jsp?tp=&arnumber=5986769}{4} 
& Rule-based control & Minimize energy cost and grid energy losses & Continuous power rate & Permanent &  &   & &\\ \hline 

\citeauthor{9409126}~\cite{9409126} 
%\href{https://ieeexplore.ieee.org/document/9409126}{7}
& Scheduling algorithm & Minimize demand charge, total load variation, and capacity distribution fairness & Continuous power rate of 80 chargers & One month & & \checkmark & &\checkmark  \\ \hline 

\citeauthor{MMN2019}~\cite{MMN2019} %\href{https://ieeexplore.ieee.org/stamp/stamp.jsp?tp=&arnumber=8356086}{12} 
& Deep Q-learning, Deep Policy Gradient & Minimize building energy cost & Boolean decision for turn on/off 3 devices & Single day &  &   & &\checkmark  \\ \hline 

% \cite{MJG2015} & Adaptive scheduling algorithm &  & \checkmark & \checkmark & & & \checkmark \\ \hline 

\citeauthor{SNDJ2020}~\cite{SNDJ2020}
%\href{https://ieeexplore.ieee.org/stamp/stamp.jsp?tp=&arnumber=8727484}{18}  
& RL: off-policy value-iteration & Minimize power consumption and unfinished charging & Boolean decision (charge or not) on 50 charger stations & Single day &  &\checkmark &\checkmark  & \\ \hline

\citeauthor{NNM2024}~\cite{NNM2024}
%\href{https://dl.acm.org/doi/pdf/10.1145/3632410.3632421?casa_token=4mVExr-5ulAAAAAA:unTMsRbiehdQKfAyyNgGd2TOfnuVepjeC6c6sshMsV2-dgOq9Ny62E0P3kszzqRd_EQmRd2O8a2j}{14} 
& Deep RL: PPO & Minimize energy bill and satisfy user QoS  & Continuous power rate of an EV and a HVAC & Single day & \checkmark & \checkmark &\checkmark &  \\ \hline   

\citeauthor{ZJS2022}~\cite{ZJS2022} %\href{https://ieeexplore.ieee.org/stamp/stamp.jsp?tp=&arnumber=10008598}{23} 
& Federated RL: Soft Actor and Critic & Maximize EV user benefits and electricity prices & Continuous power rate of 3 chargers & One week & \checkmark & \checkmark &\checkmark &  \\ \hline

Our Approach & DDPG with action masking and policy guidance & Minimize demand charge, electricity cost  and missing SoC& Continuous power rate of 15 chargers & One month & \checkmark & \checkmark  &\checkmark &\checkmark \\ \hline
\end{tabular}
\label{tab:comparison}
\end{table*}

We highlight four major challenges of solving the V2B problem, namely: 1) the uncertainty of vehicles and SoC requirements; 2) Time-Of-Use (TOU) pricing, demand charges, and long-term rewards; 3) heterogeneous chargers and continuous action spaces; and 4) tracking real-world states and transitions. Below, we briefly cover prior work to tackle these challenges. \textit{A more detailed description of prior work is presented in~\Cref{tab:comparison} of the appendix.}  

\noindent\textbf{Uncertainty of vehicles and SoC requirements.} 
% Prior work by \citeauthor{MJG2015} considers mobility aspects like EV arrival/departure times and trip history for charging stations~\cite{MJG2015}.
Meta-heuristics and Model Predictive Control (MPC) have been used to solve the EV charging process, focusing on energy cost and user fairness in single-site or vehicle-to-grid (V2G) systems~\cite{AORC2013, 5986769, 9409126, MJG2015}. 
Studies by \citeauthor{richardson2011electric} analyze EV charging strategies' impact on grid stability, relevant to V2B systems~\cite{richardson2011electric}. \citeauthor{8274175} proposed a demand response framework for optimizing V2B systems amidst dynamic energy pricing~\cite{8274175}. Additionally, \citeauthor{oconnell2010integration} utilized Mixed Integer Linear Programming (MILP) to integrate renewable energy sources into grids~\cite{oconnell2010integration}.
% Additionally, there are empirical studies that have analyzed EV charging strategies and their impact on grid stability, which are closely related to V2B systems~\cite{richardson2011electric}.
However, many of these methods focus on unidirectional chargers and fail to fully account for all exogenous sources of uncertainty (e.g., uncertain arrival and departure times).


% {\color{black} Other approaches, including meta-heuristics and Model Predictive Control (MPC), have been explored to optimize the smart EV charging process for electric vehicles (EVs), focusing on energy cost and user fairness in single-site or vehicle-to-grid (V2G) systems~\cite{AORC2013, 5986769, 9409126, MJG2015}.} 
% However, many of these methods focus on unidirectional chargers and fail to fully account for uncertainty including, vehicle arrivals and departures~\cite{MJG2015}. 
% This uncertainty arise from the mobility of EVs, which include aspects such as their arrival and departure times of an EV at/from a charging station, trip history of EVs, and unplanned departure of EVs.

\noindent\textbf{Time of use pricing, demand charge, and long-term rewards.} 
% Time of use pricing and demand charge, Long-term rewards and planning horizon
% AORC2013, MMN2019, SNDJ2020
V2B optimization is difficult due to long billing periods. While prior work (barring some exceptions~\cite{9409126}) optimizes and plans for single-day horizons~\cite{AORC2013, MMN2019, SNDJ2020}, they fail to work for longer periods.

% \noindent\textbf{Heterogeneous chargers and continuous action spaces.} In practice, buildings develop EV infrastructure over time and have heterogeneous chargers, complicating the action space. While some prior work has successfully modeled charger heterogeneity~\cite{NNM2024,ZJS2022}, such work either does not capture long-term rewards (i.e., limit planning to a single day) or fails to account for demand charge, thereby failing to capture what is arguably the most critical real-world constraint of the V2B problem.
\noindent\textbf{Heterogeneous chargers and continuous action spaces.} 
%Approaches that solve EV charging without considering the ability of EVs to discharge, ignores even more potential savings. However, addressing this introduces further complexity to the system.
In practice, buildings develop EV infrastructure gradually, leading to heterogeneous chargers and a more complex action space.
While some prior work addresses charger heterogeneity~\cite{NNM2024,ZJS2022}, it often neglects long-term rewards (i.e., limit planning to a single day) or fails to account for demand charge, missing the key real-world constraint in the V2B problem.
\noindent\textbf{Tracking real-world state and transition.}
Existing solutions validate their approaches using simulations with limited interface with the real world (barring some exceptions~\cite{9409126}), thereby making simplistic assumptions that limit deployment.

%The integration of electric vehicles (EVs) into energy management systems, particularly through vehicle-to-grid (V2G) and vehicle-to-building (V2B) technologies, is increasingly recognized for its role in balancing energy demand~\cite{lund2008integration}. 
% Table~\ref{tab:comparison} summarizes related work, outlining each study's objectives, limitations in action spaces, and planning duration. It also highlights key features such as discharging, EV mobility, SoC requirements, and whether long-term demand charge cost reduction was considered. Here, mobility or mobility-aware charging is defined by~\citeauthor{MJG2015} as taking into consideration different mobility aspects such as the arrival/departure time of an EV at/from a charging station, trip history of EVs, and unplanned departure of EVs~\cite{MJG2015}.



%Additionally, offline methods and meta-heuristics often require extended runtimes, making them unsuitable for real-time decision-making in dynamic environments.
%TODO: Initial RL 2,4,13,

% To account for uncertainty, reinforcement learning methods have also been tried before.
% %Recent work has shifted toward machine learning, particularly reinforcement learning (RL), for optimizing energy systems, due to their ability to better generalize and also accommodate long-term rewards compared to meta-heuristics and MPC approaches.
% \citeauthor{mnih2015human} introduced Deep Q-Networks (DQN), which have been adapted for dynamic energy management in V2B systems~\cite{mnih2015human}. \citeauthor{MMN2019} explored Deep Q-learning and deep policy gradient methods for online optimization of building energy schedules~\cite{MMN2019}. However, these approaches still do not consider the mobility of the EVs. Instead the vehicles are treated as stationary loads with no temporal properties related to the
% % arrival and departure of the EVs. 
% %
% %TODO: mobility 7, 18 SNDJ2020 ,NNM2024
% \citeauthor{SNDJ2020} employed an RL-based approach control a set of charger stations to minimize energy consumption in smart grids~\cite{SNDJ2020}. Their approach only considers a boolean decision of turning the charger stations on or off, lacking the ability to discharge. \citeauthor{NNM2024}, improves upon this idea by extending the potential charger actions to include both discharging and charging actions. They applied deep reinforcement learning (Deep RL) to optimize EV and HVAC systems for daily energy cost minimization~\cite{NNM2024}. However, neither of these approaches incorporate the concept of demand charge into the problem and limit their planning horizon to a single day.
% %
% %
% %TODO: discharging 9, 23

% Improving upon these initial approaches, \citeauthor{ZJS2022} investigated federated RL for EV charger control, aiming to maximize user benefits~\cite{ZJS2022}, and minimize electricity prices. Their approach explores continuous action space of charger power rates and extends their planning horizon to an entire week. While their approach include both discharging and charging actions, they fail to capture the idea of demand charge into their problem which is critical for industrial loads. Additionally, when we consider demand charge the planning horizons have to increase to a month.
% %While many of these approaches utilize machine learning and RL-based approaches to the EV charging problem, many of these techniques are limited by small or discretized state and action spaces, consider limited chargers, and short-term (single-day) planning horizons.


%   %In this paper, we propose methods to enhance RL-based V2B optimization, exploring centralized control of multiple chargers, continuous charging decisions to meet diverse SoC requirements and building loads, and enabling long-term (monthly) planning in line with real-world charging constraints.

% \begin{figure*}[t]
%     \centering
%     \begin{subfigure}[b]{0.68\linewidth}
%         % \centering
%         \includegraphics[height=2.2in,keepaspectratio,trim=0.2cm 0.2cm 0.2cm 0.2cm,clip]{figures/framework.pdf}
%         % \vspace{-0.1in}
%         \caption{Reinforcement Learning Framework.}
%        \label{fig:framework}
%     \end{subfigure}
%     % \hfill
%     \begin{subfigure}[b]{0.28\linewidth}
%         % \centering
%         \includegraphics[height=2.2in,keepaspectratio,trim=0cm 0.5cm 0cm 0.5cm,clip]{figures/inference_pipeline.pdf}
%         \vspace{-0.15in}
%         \caption{Pipeline for Inference.}
%         \label{fig:pipeline}
%     \end{subfigure}
%     \caption{\color{black}(a) Our framework relies on daily samples along with an estimated monthly peak power. We use reinforcement learning, i.e., DDPG, and extend it with policy guidance and action masking, to learn a near-optimal policy, A reinforcement learning policy based on DDPG is trained with policy guidance and action masking. (b) At inference time, the model ingests data of connected cars, charger states, building power reading, and the estimated monthly peak power to make decisions.}
%     \label{fig:framework_and_pipeline}
% \end{figure*}


\section{Our Approach}
\label{sec:our_approach}
In this section, we discuss the different components in our  framework, shown in~\Cref{fig:framework}. 
%We present how our approach targets every critical issue we presented in~\Cref{sec:related_work}. 
% We begin by defining the simulator then we provide a description of the MDP.
%
% and our approach targets every critical issue we presented in~\Cref{sec:related_work}.
% , shown in~\Cref{fig:framework}. We present how our approach targets every critical issue we presented in~\Cref{sec:related_work}.
% 
% \noindent\textbf{Uncertainty of vehicles and SoC requirements.}
% %{\jpnote{
% % we first collect data across almost an year -- buildings and car, we use this data to empirically sample chains that can be used to train and validate our policy. This was used by our partners to train a poisson and conditional distribution models from which we were able to sample more chains.  Clearly define what a sample means...}}  
% To address the uncertainty and diversity of V2B scenarios, we collect real-world historical data 
% from \nissan{} from their fleet and chargers at their office building, covering the period from May 2023 to January 2024.  
% The dataset includes EV arrival and departure times, initial and required State of Charge (SoC), and building power demand readings at 15-minute intervals. Since the location is categorized as an industrial building, it is subject to time-of-use (TOU) electricity rates and demand charges during peak hours (6:00 AM to 10:00 PM). Demand charge adds a flat rate multiplier to the highest average power delivery across 15-minute intervals over the billing period and is added to the total bill.
% % . TOU introduces variations in the electricity rates, with higher prices during peak hours (6:00 AM to 10:00 PM). Demand charge adds a flat rate multiplier to the highest average power delivery across 15-minute intervals over the billing period, and is added to the total bill.
% Then, we use the data to learn a Poisson distribution for modeling EV arrival counts, and random forest models for SoC requirements, and building fluctuations based on historical data.
% % We then empirically sampled $1000$ one-month billing episodes samples for each month from May 2023 to January 2024. 
% 
% \noindent\textbf{Time of use pricing, demand charge, and long-term rewards.} 
% %\jpnote{so we break a month into daily episodes and utilize peak power prediction generated across chains using the optimal policy which is MILP. but why daily? constant environment daily, mininmal variance weekly}  
% To enhance training efficacy, we address the challenge of lengthy state-action episodes by splitting the monthly training dataset into daily episodes. This approach captures the diverse conditions observed across different weekdays, enabling the model to learn effectively from shorter episodes, which facilitate quicker adaptation to daily variations.  
% In addition, to incorporate monthly peak power considerations, we use an estimated monthly demand charge as an input feature, penalizing only daily demand charges that exceed this predicted value. This incentivizes the policy to minimize the monthly demand charge while training with daily episodes. We determine the minimum demand charge from optimal action sequences over the one-month billing period generated by MILP optimization (detailed in~\Cref{subsec:milpsolver}). By analyzing the peak power distribution from the MILP solutions, we estimate the lower bound of the 99\% confidence interval as the monthly demand charge, providing a conservative initial estimate. This input feature is further tuned by increasing it by 0\%, 5\%, and 10\% during RL training. 
% % We incorporate the estimated monthly peak power as an input feature and penalize only those peak power values exceeding this estimated value in each daily episode. This approach incentivizes the policy to minimize the monthly peak power while effectively training with daily episodes.   
% 
% During training, we varied the amount of samples used for training and observed that utilizing a larger number of training episodes results in longer convergence time and overall worse performance. We show the results of this in~\Cref{ssec:ablation}. Thus, we utilize a k-means clustering approach to down-sample. We used $k=5$ and clustered using the optimal demand charge derived from the MILP solution for each sample. From each cluster, we select 60 and 50 samples for the final training and testing datasets respectively, ensuring that these datasets are mutually exclusive. 
% 
% \noindent\textbf{Heterogeneous chargers and continuous action spaces.}
% % \jpnote{we model as MDP; for tractability we divide into 15 minute intervals. But our system can work with other discretizations. We use DDPG; but to help with training we use policy guidance (action masking and ILP); its too difficult when you consider heterogeneous and continuous, you need guidance and action masking for RL. only mention briefly. why we need, what we need.}  
% To address uncertainty in the V2B environment, we model the problem as a Markov Decision Process (MDP), dividing the time horizon into 15-minute intervals for tractability, although our system can accommodate other discretizations. We solve this problem using a  RL-based approach built on the Deep Deterministic Policy Gradient (DDPG) algorithm~\cite{lillicrap2015continuous}, which is well-suited for continuous actions and supports off-policy training. This capability allows the model to learn from diverse experiences across various scenarios, enhancing generalization. 
% However, traditional DDPG is limited in reaching global optima for particularly long horizons and large state-action spaces. To address these limitations, we introduce action masking~\cite{huang2020closer,kanervisto2020action}, ensuring that policy actions generated by the actor network are valid and reasonable during DDPG training. This constraint enhances training efficiency by preventing the actor network from exploring invalid actions, optimizing resource usage. 
% Additionally, we incorporate policy guidance techniques~\cite{pmlr-v28-levine13} into the RL training process, which aim to introduce optimal state-action transitions and improve local optima in DDPG. 
% 
% \noindent\textbf{Tracking real-world state and transition.}  
% % - first we downsample to help with generalization using a clustering approach. % why do we need do that. Is there a paper that did something similar.. 
% % \jpnote{we model a digital twin and provide rest APIs -- it has state and transitions and tracks the MDP that we will discuss in the next section. The simulator also is able to generate rewards that are used for training and are also discussed in the next section.} 
% We model a digital twin for the target environment and provide several interfaces that allows both simulated and real-world components to leverage our proposed approach. 
% We model the environment using a digital twin that holds a state representation of the world with transitions representing V2B behavior. 
% % This includes information on EVs, building, and the grid. This allows us to investigate how any action or decision can potentially impact the real world. 
% % Decisions are taken at the end of each set of events for any given time period. 
% %
% There are two main decisions that must be taken when solving the V2B charging problem. (1) Charger assignments and (2) Charger actions.
% % We address the charger assignment decision below and provide information on the charger action in~\Cref{sec:RL}.
% % \textbf{Environment updates and transitions.} The input episodes dictate the simulator's world view. Each event includes an event type and time, matching their real world trigger and occurrence. We identify several critical events in the episodes to serve as the triggers for the simulator. These include (1) EV arrivals, (2) EV departures, (3) building power readings, and (4) TOU rate changes. Events are placed in a queue, with each event triggering an update to the environment which modifies the state. Updates to the state, which include the charging or discharging of EVs, are based on the elapsed time between events. At each decision epoch, there are two decisions to be made, charger assignment and charger actions.
% % % \input{results/charger_assignments} 
% % \textbf{Charger assignment.} 
% We consider a first-in, first-out policy that assigns EVs to bidirectional chargers first, breaking ties assigning to later departing cars, \textit{a comparison of other approaches are available in the appendix}. 
% % %shows the different charger assignment and tie breaking policies tested.  
% % By comparing various charger assignment and tie-breaking policies (as shown in \Cref{table:charger_assignment_policies} in the Appendix), we observe that bidirectional charging assignments outperform all other policies. Tie-breaking strategies that prioritize later-departing vehicles show a marginal advantage. While these assignment policies could be further optimized, we chose to follow this heuristic and focus on the second decision problem: determining charger actions.
% %
% % \textbf{Charger actions.} 
% % We provide several policies with our simulator to contrast and compare with our proposed approach. 
% Charger action policies receive a state of the environment for a particular time and generate actions based on this. 
% % The simulator is stateless. Thus, it provides only a current representation of the world at that specific time to each policy. 
% We expand this idea and model the problem as an MDP.

% % \jpnote{Explain again that there are two decisions to be made; charger assignment and charging rates; but we use a heuristic to solve the charger assignment; we show quick results across the months in table 2 and we pick bidrection first and departure as our basic policy. Connect to next subsection on MDP}
\begin{figure*}[t]
    \centering
    \begin{subfigure}[b]{0.68\linewidth}
        % \centering
        \includegraphics[height=2.1in,keepaspectratio,trim=0.2cm 0.2cm 0.2cm 0.2cm,clip]{figures/framework.pdf}
        % \vspace{-0.1in}
        \caption{Reinforcement Learning Framework.}
       \label{fig:framework}
    \end{subfigure}
    % \hfill
    \begin{subfigure}[b]{0.3\linewidth}
        % \centering
        \includegraphics[height=2.1in,keepaspectratio,trim=0cm 0.5cm 0cm 0.5cm,clip]{figures/inference_pipeline.pdf}
        \caption{Pipeline for Inference.}
        \label{fig:pipeline}
    \end{subfigure}
    \caption{\color{black}(a) Our framework relies on daily samples and an estimated monthly peak power. We use RL, i.e., DDPG, and extend it with policy guidance and action masking, to learn a near-optimal policy. (b) At inference time, the model ingests data of connected cars, charger states, building power, and the estimated monthly peak power to make decisions.}
    \label{fig:framework_and_pipeline}
\end{figure*}

\subsection{Markov Decision Process Model}
\label{ssec:MDP}

We model the V2B problem as the following MDP. 

{\bf State.}
The complete state space for the problem can be described using features that capture historical, current, and future estimation at a given time $T_j$, which includes parameters for each vehicle, such as the current SoC, required SoC, departure time, and battery capacity for each EV, along with SoC boundaries across all chargers. Additionally, the current building power, time slot, day of the week, historical building power, and long-term peak power estimation value are included, resulting in approximately $100$ features. 
% While this state space is complete, it is not tractable to be used for the learning process. Therefore, 
We leverage domain-specific knowledge to abstract key information from these features, reducing the state space to the $37$ essential state elements.

These features are:
\textbf{1)} The current time slot, $T_j$. \textbf{2)} The current building power, denoted as ${B}(T_j)$. \textbf{3)} The power gap between the current building power and the estimated peak power for the billing period, given by $ \PrdPeak(T_j) - B(T_j)$, where $\PrdPeak(T_j)$ indicates the estimated peak power at $T_j$, initialized from a value derived from training data. This gap aids the RL model in estimating the optimal peak power for demand charge reduction. \textbf{4)} The mean peak building power over the previous 7 days, $\mu(B^H(T_j))$, where $B^H(T_j)$ represents the list of peak building power for the previous 7 days. \textbf{5)} {The variance of the peak building power over the previous 7 days, $\sigma^2(B^H(T_j))$, helps inform the model about the future building power use}. \textbf{6)} The day of the week for the current time slot, $T_j$, which helps the model distinguish daily patterns and enhance generalization. \textbf{7)} The number of EV arrivals up to time slot $T_j$, represented as $|\{V | V \in \mathcal{V}, A(V) \leq T_j \}|$ for tracking EV arrival status. \textbf{8)} The energy needed by each EV connected to a charger at time slot $T_j$, given by $[\PowerNeed(C_i, T_j)]_{C_i \in \mathcal{C}}$, which is initialized to $0$. This quantity represents the energy gap between required SoC ($\SOCR$) and current SoC ($\it{SOC}$) of the EV $V = \phi(C_i, T_j)$, defined as $\PowerNeed(C_i, T_j) = (\SOCR(V) - \SOC(V, T_j)) \times \text{CAP}(V)$.
    % \begin{equation} 
    %  \PowerNeed(C_i, T_j) = (\SOCR(V) - \SOC(V, T_j)) \times \text{CAP}(V)
    %  \label{eq:chargerstate}
    % \end{equation}
    % where $V=\phi(C_i, T_j)$ indicating the EV connected to $C_i$ at $T_j$. 
\textbf{9)} The remaining time until the departure of each EV connected to the chargers is given by $[\ReTime(C_i, T_j)]_{C_i \in \mathcal{C}}$, and is set to 0 when no cars are connected. Each term is computed as $\ReTime(C_i, T_j) = \DepartureTime(\phi(C_i, T_j)) - T_j$. 

% \begin{enumerate}[leftmargin=*]
%     \item The current time slot, $T_j$.
%     \item The current building power, denoted as ${B}(T_j)$.
%     \item The power gap between the current building power and the estimated peak power for the billing period, given by $ \PrdPeak(T_j) - B(T_j)$, where $\PrdPeak(T_j)$ indicates the estimated peak power at $T_j$, initialized from a value derived from training data. This gap aids the RL model in estimating the optimal peak power for demand charge reduction.
%     \item The mean peak building power over the previous 7 days, $\mu(B^H(T_j))$, where $B^H(T_j)$ represents the list of peak building powers from the previous 7 days. 
%     \item The variance of the peak building power over the previous 7 days, $\sigma^2(B^H(T_j))$. It informs the model about the future building power.  
%     \item The day of the week for the current time slot, $T_j$. It helps the model distinguish daily patterns and enhance generalization.
%     \item The number of EV arrivals up to time slot $T_j$, represented as $|\{V | V \in \mathcal{V}, A(V) \leq T_j \}|$ for tracking EV arrival status. 
%     \item The energy needed by each EV that is connected to a charger at time slot $T_j$, given by $[\PowerNeed(C_i, T_j)]_{C_i \in \mathcal{C}}$, and is initialized to $0$. This represents the energy gap between the required SoC and the current SoC of the EV connected to charger $C_i$ at time slot $T_j$, defined as: 
%     \begin{equation} 
%      \PowerNeed(C_i, T_j) = (\SOCR(V) - \SOC(V, T_j)) \times \text{CAP}(V)
%      \label{eq:chargerstate}
%     \end{equation}
%     where $V=\phi(C_i, T_j)$ indicating the EV connected to $C_i$ at $T_j$. 
%     \item The remaining time until the departure of each EV connected to the chargers is given by $[\ReTime(C_i, T_j)]_{C_i \in \mathcal{C}}$, and is set to 0 when no cars are connected. Each term is computed as $\ReTime(C_i, T_j) = \DepartureTime(\phi(C_i, T_j)) - T_j$. 
% \end{enumerate}

{\bf Actions.} We define the set of actions $\mathcal{A}$, which includes all actions at each time slot $T_j$ with $T_j \in \mathcal{T}$. In this MDP, $\mathcal{A}$ is continuous and specifies the power of all chargers at each time slot $T_j$, where $A(T_j) = [P(C_i, T_j)]_{C_i \in \mathcal{C}} $.
 

{\bf State Transition.} 
%The states of the building and chargers are updated based on actions taken at each time slot. 
% Building and charger states are updated based on actions and EV arrivals/departures at each time slot. To simulate state transitions, we designed an environment simulator that provides state features and manages these transitions. 
States are updated based on actions and EV arrivals/departures at each time slot. To simulate these transitions, we designed an environment simulator that provides and updates states. The state transition function is given as:
${\it Trans}(S(T_{j-1})$, $A(T_{j-1})) \mapsto S(T_j)$, with the following steps: 
\begin{enumerate}[leftmargin=*]
    \item Initialize the estimated peak power, $\PrdPeak(T_0)$, which can be derived from historical data 
    %\ad{does not make sense. We need to clearly describe the process of how this is estimated for training as well as inference. Please update. There is no mention in section 4.2} 
    (detailed in ~\Cref{sec:our_approach})
    , and update it by
    $
    \PrdPeak(T_{j}) = \max(\PrdPeak(T_{j-1})$, $ \Building(T_{j-1}) + \sum_{C_i \in \mathcal{C}} P(C_i, T_{j-1})),
    $
    which updates the estimated peak power depending on the previous estimate and the last peak power.
    \item Update SoC of EVs connected to all chargers: $\it{SOC}(\phi(C_i,T_{j}), T_{j})$ using action $A(T_{j-1})$ according to Equation~(\ref{eq: soc}). 
   \item Update the EV charger assignment $\phi(C_i, T_j)$ and $\eta(V)$ by first releasing chargers with departing EVs in the current time slot $T_j$ and then assigning new arrival EVs to idle chargers. 
   \item Update the energy requirement of all EVs connected to a charger: $[\PowerNeed(C_i, T_j)]_{C_i \in \mathcal{C}}$ by based on EV's current SoCs.
   \item Update the remaining time of all EVs connected to chargers: $[\ReTime(C_i, T_{j})]_{C_i \in \mathcal{C}}$ at time slot $T_{j}$.   
\end{enumerate}


{\bf Reward.} We define the function ${\it Reward}: \mathcal{S} \times \mathcal{A} \rightarrow \Re$, where ${\it Reward}(S(T_j), A(T_j))$ evaluates the reward for actions taken in a specific state, focusing on minimizing the total bill while satisfying SoC requirements. We express reward as $\lambda_{S} \cdot \mathit{r}_1 + \lambda_{E} \cdot \mathit{r}_2 + \lambda_{D} \cdot \mathit{r}_3$ where $\mathit{r}_1 =  \sum_{C_i\in\mathcal{C}} \max(0, \min(\PowerNeed(C_i, T_j), P(C_i, T_j) \times \delta))$, $\mathit{r}_2 = - P(C_i, T_j) \cdot \delta  \cdot \theta_E(T_j)$, and $\mathit{r}_3 = - \max(0, \Building(T_j) + \sum_{C_i \in \mathcal{C}} P(C_i, T_j) - \PrdPeak(T_j)) \cdot \theta_D$
% \begin{align*}
%    & \mathit{Reward}(S(T_j), A(T_j)) = \lambda_{S} \cdot \mathit{r}_1 + \lambda_{E} \cdot \mathit{r}_2 + \lambda_{D} \cdot \mathit{r}_3
% \end{align*}
.
% \begin{align*}
% \begin{aligned}
%     \mathit{r}_1 &=  \sum_{C_i\in\mathcal{C}} \max(0, \min(\PowerNeed(C_i, T_j), P(C_i, T_j) \cdot \delta)) \\
%     \mathit{r}_2 &= - P(C_i, T_j) \cdot \delta  \cdot \theta_E(T_j) \\
%     \mathit{r}_3 &= - \max(0, \Building(T_j) + \sum\limits_{C_i \in \mathcal{C}} P(C_i, T_j) - \PrdPeak(T_j)) \cdot \theta_D
% \end{aligned}
% \end{align*}
In this reward structure, $\mathit{r}_1$ promotes actions that charge EVs to reach their required SoC, as intended in Equation~(\ref{eq: soc_penalty}), while $\mathit{r}_2$ penalizes the energy cost for the charging actions taken. The third component, $\mathit{r}_3$, penalizes the increase in demand charges if peak power increases, aligning with our objective in Eq. (\ref{eq: billing}). These functions use three  coefficients, $\lambda_{S}$, $\lambda_{E}$, and $\lambda_{D}$ to balance trade-offs.
%between these reward factors.

% \subsection{MILP Solver}
% \label{subsec:milpsolver}



% When solving for a month, it has, on average, 20000 variables and 32000 constraints and takes 3.7 seconds to solve. However, the MILP solution is limited by its requirement to know the future completely, not accounting for any stochasticity in the future.



% We implement an optimization framework to generate optimal actions and use it to provide effective guidance to steer the search toward the global optimum. To get the optimal actions, we formulate the V2B problem using mixed-integer linear programming (MILP) and refer to the MILP framework as $\mathit{MILP(S(T_j), events_{future})}$. Here, $S(T_j)$ is the current state information, including the status of connected EVs, building's power, and $events_{future}$ covers all the future events from the input sample, which are EV arrivals and departures, building's power consumption, and electricity prices from the current time to the end of the billing period. The MILP solution provides the optimal power for each charger from the current time to the end of the billing period, maximizing the multi-objective weighted sum of the total cost (detailed in Equation~(\ref{eq: billing})) and penalties for missing SoC requirements (defined in Equation~(\ref{eq: soc_penalty})). 
% % It is designed to follow the multiple constraints related to the EV SoC update function following Equation~(\ref{eq: soc}) - (\ref{eq:building_power}).   

% As we discussed earlier, our problem deals with futures for up to a month, while some subproblems may need to be solved only for a day. 


 

% \begin{algorithm}[t]
\setcounter{AlgoLine}{0} %
\small
    \SetAlgoLined
    \KwIn{Initial policy parameters for actor network $\zeta_a$, critic parameters $\zeta_c$, target network parameters $\zeta_a', \zeta_c'$\\
Training parameters: $\mathit{actionNoise}$, $\policyGuidanceRate$, $\mathit{bufferSize}$, $\mathit{batchSize}$, maximum iterations: $M$,  training steps: ${\it trainStep}$; target network update steps: ${\it updateStep}$
}
    \KwOut{Trained policy $\pi_{\zeta_a}$}
    Initialize replay buffer $\Buffer$;  ${\it step}\gets 0$\\
    % Initialize target network weights $\zeta_a' \leftarrow \zeta_a$, $\zeta_c' \leftarrow \zeta_c$ \\
    \For{$1$ \KwTo $M$}{
        Input a sample into simulator to generate initial state $s_0$ 
        % Initialize a random process $N$ for action exploration 
        
        \For{each time slot $T_j \in \mathcal{T}$}{
        % \If{$T_j$ is during non-peak hours}{
        % Get action A(T_j) using h
        %}
        \tcp{Introducing policy guidance stochastically.}
            {\color{black} randomValue $\leftarrow randomBetween(0,1) $
            
            \If
            %(\tcp*[h]{Adding policy guidance stochastically}) $$
            { randomValue $\leq \policyGuidanceRate$}{
        Get action $A(T_j)$ by rerunning the MILP solver: $A(T_j)\leftarrow\MILP(S(T_j), {\it remainEpisode})$
                }
            \Else{
                % $A(T_j)\gets  \pi(S(T_j) | \zeta_a) + \mathit{actionNoise}$
                Get masked action using current policy,  $\mathit{actionNoise}$: \\
                $A(T_j) \gets \Mask\left(S(T_j), \pi(S(T_j) | \zeta_a)) + \mathit{actionNoise} \right)$
                % \tcp{Add action masking}
            }}
             State transition $S(T_{j+1})\leftarrow {\it Trans}(S(T_j), A(T_j))$ 
            %in Algorithm~\ref{alg: stateTrans}. 

            Get the action reward $R(T_j)\leftarrow {\it Reward}(S(T_j), A(T_j))$
            %by Algorithm~\ref{alg: reward}. 
            
            Store transition $(S(T_j), A(T_j), R(T_j), S(T_{j+1}))$ in $\Buffer$
            
            \If{${\it step} \bmod {\it trainStep} == 0$}{
            Sample batch $(S(T_i), A(T_i), R(T_i), S(T_{i+1})$ from $\Buffer$ 

           {\color{black} Get masked actions using target actor network:\\
            $A(T_{i+1})\leftarrow \Mask(S(T_{i+1}), \pi'(S(T_{i+1}) | \zeta_a'))$}  
            %using Algorithm~\ref{alg: mask}  \tcp{Add action masking}

            Set target $y_i\leftarrow R(T_j) + \gamma Q'(S(T_{i+1}), A(T_{i+1})
            | \zeta_c')$ 
            
            Update critic network by minimizing the loss: $L 
            \leftarrow \frac{1}{N} \sum_i (y_i - Q(S(T_i), A(T_i) | \zeta_c))^2$ 
            
             {\color{black} Get masked actions $A(T_i)$ at $S(T_i)$ using actor network: $A(T_i)\leftarrow \Mask( S(T_i), \pi(S(T_i) | \zeta_a))$ }%using Algorithm~\ref{alg: mask}.  
             % \tcp{Add action masking}
           % Mask action $A(T_i)\leftarrow \Mask(S(T_i), a_{i})$ 
           
            Update the actor policy using policy gradient:\\
            $\nabla_{\zeta_a} J \leftarrow \frac{1}{N} \sum_i \nabla_a Q(S(T_i), A(T_i) | \zeta_c) | \nabla_{\zeta_a} \pi(s | \zeta_a) |_{S(T_i)}$}
            
        \If{${\it step} \bmod {\it updateStep} == 0$}{
             Update the target networks: 
            $\zeta_a' \leftarrow \tau \zeta_a + (1 - \tau) \zeta_a'$;\quad 
            $\zeta_c' \leftarrow \tau \zeta_c + (1 - \tau) \zeta_c'$ \\
            }${\it step}\gets {\it step}+1$
        }
    }
    \caption{Improved DDPG with Action Masking and Policy Guidance.}
\label{alg:DDPG} 
\end{algorithm}
% \begin{algorithm}[t]
    % \SetAlgoNlRelativeSize{-1}
    \KwIn{$\textit{state: }S(T_j), \textit{action: } A(T_j) $}
    \KwOut{Masked action: $\MaskAction$} 
\small
Initializing: $\PowerNeed \gets [\PowerNeed(C_i, T_j)]_{C_i \in \mathbf{C}}$; \quad
$\ReTime \gets [\ReTime(\phi(C_i, T_j))]_{C_i \in \mathbf{C}};$
$\epsilon \gets 10^{-5}$; \quad
$C^{max} \gets [C^{max}_i]_{C_i \in \mathbf{C}}$; \quad
$C^{min} \gets [C^{min}_i]_{C_i \in \mathbf{C}}$

\tcp{Mask 1: Set action = 0 if no car is connected}
    $ \MaskAction \gets \frac{\ReTime}{\ReTime + \epsilon} \times A(T_j)$\
    
    \tcp{Mask 2: Stop charging when required SoC is reached for uni-directional chargers}
    $ \MaskAction_{tmp} \gets \MaskAction$; \quad
    $\MaskAction[\textit{uniIdx}] \gets \min(\MaskAction_{tmp}, \frac{\PowerNeed}{\delta})[\textit{uniIdx}]$
    % $  \gets \MaskAction_{tmp}$

    \tcp{Mask 3: Enforce charging to the req. SoC before departure. }
    $\overline{\Power(T_j)} \gets \frac{ \PowerNeed- (\ReTime - 1) \times C^{max} \times \delta }{\delta}$\
    $\overline{\Power(T_j)} \gets \min(\overline{\Power(T_j)}, C^{max})$; 
    $\MaskAction \gets \max(\MaskAction, \overline{\Power(T_j)})$\
    
    \tcp{Mask 4:  Bidirectional chargers discharge to req. SoC by departure.}
    $\Power^*(T_j) \gets \frac{\PowerNeed- (\ReTime - 1) \times C^{min} \times \delta }{\delta}$\
    $\Power^*(T_j) \gets \max(\Power^*_t, C^{min})$\
    
    $\MaskAction_{tmp}\gets \MaskAction$; \quad
    $ \MaskAction[\textit{biIdx}] \gets \min(\MaskAction_{tmp}, \Power^*_t)[\textit{biIdx}]$\

    \tcp{Mask 5: Power improvement strategy}
    $ \textit{powerGap} \gets \Building(T_j) - \PrdPeak(T_j)$\
    $ \textit{canIncrease} \gets \textit{ReLU}\left(\min\left(\frac{\PowerNeed}{\delta}, C^{max}\right) - \MaskAction \right)$
    
    $ \textit{toImprove} \gets \min\left(\textit{ReLU}(\textit{powerGap} - \sum \MaskAction), \sum \textit{canIncrease}\right)$
    
    $ \MaskAction \gets \MaskAction + \frac{\textit{toImprove} \times \textit{canIncrease}}{\sum(\textit{canIncrease}) + \epsilon}$

    \tcp{Mask 6: Do not discharge below building load}
    $ \textit{toImprove} \gets \max(-\Building(T_j) - \sum(\MaskAction), 0)$
    $ \textit{negAction} \gets \textit{ReLU}(\MaskAction \times -1) \times -1$
    
    % $ \textit{toIncrease} \gets \frac{\textit{toImprove} \times \textit{tmpAction}}{\sum(\textit{tmpAction}) + \epsilon}$\;
    $ \MaskAction \gets \MaskAction +  \frac{\textit{toImprove} \times \textit{negAction}}{\sum(\textit{negAction}) + \epsilon}$

    \caption{Action Masking: $\Mask(S(T_j), A(T_j))$.} 
    \label{alg: action_masking}
\end{algorithm} 

\subsection{Reinforcement Learning Approach}
\label{sec:RL}
In this section, we describe the entire reinforcement learning pipeline. We introduce the network structure, discuss how we use a simulator to gather state features and describe the different techniques, such as action masking and policy guidance, used to improve the performance of the V2B problem.  

To improve training efficiency, we address the challenge of long state-action sequences by splitting the monthly dataset into daily episodes. This allows the model to capture variations across different weekdays and learn more effectively from shorter episodes, adapting more quickly to daily changes. By incorporating estimated monthly peak power into the state features and reward function, the approach still accounts for monthly demand charges, helping to minimize long-term costs while staying aligned with our objective. 
% - make sure key hyperparameters are known.
% - make sure the confidence bound is explained as a hyperparameter.
% \jpnote{Will have to copy paste sections from~\Cref{sec:RL}. Talk about how the previous components are used to create our policy/model.}

%which is well-suited for continuous action spaces and supports off-policy training, allowing the model to learn from diverse experiences across various scenarios, thereby improving generalization.
%
\subsubsection{Enhanced Deep Deterministic Policy Gradient}
Our approach based on the DDPG framework~\cite{lillicrap2015continuous} uses an actor network for continuous actions.
During training, we interact with the simulator that provides state abstractions and transitions.
To improve RL performance in handling the limitations associated with large continuous action spaces and long-term reward optimization, we introduce action masking and policy guidance techniques. Details of the enhanced approach are in Algorithm~\ref{alg:DDPG} in the appendix. 
Action masking, denoted as $\Mask(S(T_j), A(T_j))$, refines the raw actions generated by the actor network by enforcing action validity and utilizing domain-specific knowledge, thereby improving policy performance. Additionally, policy guidance incorporates the MILP solver discussed earlier to provide optimal actions based on current and future information. These optimal actions are stochastically introduced during RL training into the replay buffer (i.e., tossing a biased coin) to mix high-quality actions given a deterministic trajectory with exploratory actions).  


\subsubsection{Action Masking}
Action masking ensures that the policy actions generated by the actor network are feasible during DDPG training. Findings from \cite{huang2020closer,kanervisto2020action} confirm that differentiable action masking does not interfere with the policy gradient backpropagation process. As a result, the learning process remains effective, while the imposed constraints on the action space prevent the policy from exploring invalid actions, thereby improving training efficiency and optimizing resource usage. 
\begin{algorithm}[t]
    % \SetAlgoNlRelativeSize{-1}
    \KwIn{$\textit{state: }S(T_j), \textit{action: } A(T_j) $}
    \KwOut{Masked action: $\MaskAction$} 
\small
Initializing: $\PowerNeed \gets [\PowerNeed(C_i, T_j)]_{C_i \in \mathbf{C}}$; \quad
$\ReTime \gets [\ReTime(\phi(C_i, T_j))]_{C_i \in \mathbf{C}};$
$\epsilon \gets 10^{-5}$; \quad
$C^{max} \gets [C^{max}_i]_{C_i \in \mathbf{C}}$; \quad
$C^{min} \gets [C^{min}_i]_{C_i \in \mathbf{C}}$

\tcp{Mask 1: Set action = 0 if no car is connected}
    $ \MaskAction \gets \frac{\ReTime}{\ReTime + \epsilon} \times A(T_j)$\
    
    \tcp{Mask 2: Stop charging when required SoC is reached for uni-directional chargers}
    $ \MaskAction_{tmp} \gets \MaskAction$; \quad
    $\MaskAction[\textit{uniIdx}] \gets \min(\MaskAction_{tmp}, \frac{\PowerNeed}{\delta})[\textit{uniIdx}]$
    % $  \gets \MaskAction_{tmp}$

    \tcp{Mask 3: Enforce charging to the req. SoC before departure. }
    $\overline{\Power(T_j)} \gets \frac{ \PowerNeed- (\ReTime - 1) \times C^{max} \times \delta }{\delta}$\
    $\overline{\Power(T_j)} \gets \min(\overline{\Power(T_j)}, C^{max})$; 
    $\MaskAction \gets \max(\MaskAction, \overline{\Power(T_j)})$\
    
    \tcp{Mask 4:  Bidirectional chargers discharge to req. SoC by departure.}
    $\Power^*(T_j) \gets \frac{\PowerNeed- (\ReTime - 1) \times C^{min} \times \delta }{\delta}$\
    $\Power^*(T_j) \gets \max(\Power^*_t, C^{min})$\
    
    $\MaskAction_{tmp}\gets \MaskAction$; \quad
    $ \MaskAction[\textit{biIdx}] \gets \min(\MaskAction_{tmp}, \Power^*_t)[\textit{biIdx}]$\

    \tcp{Mask 5: Power improvement strategy}
    $ \textit{powerGap} \gets \Building(T_j) - \PrdPeak(T_j)$\
    $ \textit{canIncrease} \gets \textit{ReLU}\left(\min\left(\frac{\PowerNeed}{\delta}, C^{max}\right) - \MaskAction \right)$
    
    $ \textit{toImprove} \gets \min\left(\textit{ReLU}(\textit{powerGap} - \sum \MaskAction), \sum \textit{canIncrease}\right)$
    
    $ \MaskAction \gets \MaskAction + \frac{\textit{toImprove} \times \textit{canIncrease}}{\sum(\textit{canIncrease}) + \epsilon}$

    \tcp{Mask 6: Do not discharge below building load}
    $ \textit{toImprove} \gets \max(-\Building(T_j) - \sum(\MaskAction), 0)$
    $ \textit{negAction} \gets \textit{ReLU}(\MaskAction \times -1) \times -1$
    
    % $ \textit{toIncrease} \gets \frac{\textit{toImprove} \times \textit{tmpAction}}{\sum(\textit{tmpAction}) + \epsilon}$\;
    $ \MaskAction \gets \MaskAction +  \frac{\textit{toImprove} \times \textit{negAction}}{\sum(\textit{negAction}) + \epsilon}$

    \caption{Action Masking: $\Mask(S(T_j), A(T_j))$.} 
    \label{alg: action_masking}
\end{algorithm} 

This procedure takes the RL raw action $A(T_j)$, an array of charging power $[P(C_i, T_j)]_{C_i\in\mathcal{C}}$ for all chargers, processes it through the following masking steps, and outputs the masked actions $A'$. Before starting the procedure, we need to obtain the following state features: the remaining power needed to reach the required SoC for all connected EVs ($\PowerNeed$), the time remaining for each EV ($\ReTime$), and the maximum ($C^{\max}$) and minimum ($C^{\min}$) power of all chargers (line 1 in Algorithm~\ref{alg: action_masking}). Also, for our case, since we work with both unidirectional and bidirectional, we denote ${\it uniIdx}$ and ${\it biIdx}$ as the indices for unidirectional and bidirectional chargers, respectively. All of the masking techniques referenced below are from Algorithm~\ref{alg: action_masking}.
% \jpnote{Shorten the descriptions and make it very concise.} 
\begin{itemize}[leftmargin=*]
    \item \textbf{Mask 1.} 
    We set the charging power $P(C_i, T_j)$ of charger $C_i$ to 0 if no EV is connected, i.e., $\ReTime(\phi(C_i, T_j))=0$. (line 2)
    %%%
     \item \textbf{Mask 2.} Overcharging unidirectional chargers is not beneficial since excess energy cannot be discharged. Thus, we limit the charging power to ensure the SoC of EVs connected to a unidirectional charger remains within their required SoC. 
    For each connected EV, the actions are masked to the minimum of the current charging power and the power needed to reach its required SoC $\left(\frac{\PowerNeed}{\delta}\right)$ (line 3). 
     
    \item \textbf{Mask 3.} 
    If necessary, we want to adjust actions such that it forces charging to the required SoC before departure to minimize missing SoC, as in Equation~(\ref{eq: soc_penalty}).
    We compute the critical power $\overline{\Power^*(T_j)}$, which is the minimum power required for all chargers at time $T_j$ to reach the required SoC of the connected EVs before departing (assuming maximum power $C^{max}$ is utilized in subsequent time slots). The raw action is adjusted if it falls below this value, especially in time slots leading up to the EV's departure (line 4).    
    \item \textbf{Mask 4.} 
    This mask is symmetrical to Mask 3 for force discharging.
    Overcharging bidirectional EVs is only advantageous if excess energy can be discharged during peak hours, but there is no benefit to overcharging just before departure. Using this mask, we force discharge EVs connected to bidirectional chargers, which have excess energy, and they reach the required SoC by departure.  Here, $\Power^*(T_j)$ denotes the minimum power to discharge for all chargers $C_i \in \mathcal{C}$ at time $T_j$ to guarantee EV can reduce to required SoC when departing (assuming the maximum discharging power $C^{min}$ is utilized subsequently) (lines 5, 6).    
    %%%
    \item \textbf{Mask 5.} 
    We increase charging power while ensuring the masked action stays within the estimated peak power $\PrdPeak(T_{j})$. This aims to charge EVs as much as possible towards their required SoC without raising demand charges, thereby avoiding forced charging just before departure, which could elevate peak power. 
    We calculate the ``power gap'' between estimated peak power and current building power, $\PrdPeak(T_j) - \Building(T_j)$. If the current power sum ($\Building(T_{j-1}) + \sum_{C_i \in \mathcal{C}} P(C_i, T_{j-1})$) is below this ``power gap'', we boost the current actions using the available ``power'' gap, constrained by $\min \left(\frac{\PowerNeed}{\delta}, C^{max}\right)$. (lines 7 to 9). 
    %%%
    \item \textbf{Mask 6.} We adjust the discharging power to prevent cumulatively discharging below the current building power $\Building(T_j)$, to satisfy Constraint~\ref{eq:building_power} by reducing the discharging power based on the current actions (lines 10 to 11).
\end{itemize} 
All of the action masking procedures utilize array computations and differentiable operations, such as ReLU \cite{rasamoelina2020review} and maximum/minimum operations, and the PyTorch framework~\cite{paszke2017automatic}. 

\subsubsection{Policy Guidance with MILP Solver}
Note that for a fixed sample, i.e., a fixed set of EV arrivals and departures, the V2B problem can be modeled as a single-shot mathematical program, i.e., a mixed-integer linear program (MILP), which can solved efficiently (at least, for our problem size) to retrieve the optimal actions. The objective of the MILP is maximizing the multi-objective weighted sum of the total rewards (detailed in Equations~\ref{eq: billing}, \eqref{eq: soc_penalty}), and the other properties of the V2B problem can be encoded as constraints. The fixed sample of arrivals and departures can be extracted from historical data. Naturally, this modeling paradigm does not solve the V2B problem in general---EV arrivals and departures are not known ahead of time---however, this strategy provides a set of optimal actions that the learning module can \textit{learn to imitate}. For our use case, the MILP problem can be solved reasonably fast. For example, for a planning horizon of a day with 15 cars, the problem size averages 800 variables and 1400 constraints and takes $0.05$ seconds to solve. 

We integrate a MILP solver based on CPLEX~\cite{cplex2009v12} as a policy guidance subroutine~\cite{pmlr-v28-levine13} in the RL training process. The solver, given the current state and future events, provides optimal charging actions.
{\color{black} Each training dataset contains complete episode data, enabling the MILP solver to account for future dynamics. During RL training, it generates optimal actions based on the current state and full future information of the episode (i.e., a full-month billing period).} The solver is stochastically triggered, and its outputs are added to the replay buffer with a predefined coefficient, $\policyGuidanceRate$ (see Algorithm~\ref{alg:DDPG} in the appendix). The next optimal action is computed as $\mathit{MILP(S(T_j), {\it remainEpisode})}$, considering factors such as EV arrivals, SoC requirements, and building power.
By blending MILP-generated actions with those from the RL actor network, the agent explores a more effective action space, improving its ability to handle large continuous action spaces and long-term rewards.% During training, the MILP solver is stochastically triggered to generate optimal actions based on the current state. These state-action transitions are added to the replay buffer with a predefined coefficient, $\policyGuidanceRate$ (see Algorithm~\ref{alg:DDPG} in the appendix). The next optimal action is obtained using $\mathit{MILP(S(T_j), {\it remainEpisode})}$, which accounts for remaining events like EV arrivals, SoC requirements, and building power. By blending MILP and RL actor network actions during training, the RL agent explores a more effective action space, improving performance in handling large continuous action spaces and maximizing long-term reward.

% \jpnote{Confirm with rishav that MILP above, talks about commented lines below}
%We add the optimal actions to the replay buffer, providing effective guidance to steer the search towards global optima. 
%
%
% We give the MILP solver the current state information, including the current EV status, charge usage, and all future events from the input sample (EV arrival/departure, building power flow, and electricity prices from the current time to the end of the billing period. The MILP solution provides the charging power for each charger from the current time to the end of the billing period, maximizing the multi-objective weighted sum of total cost (detailed in Equation~(\ref{eq: billing})) and penalties for missing SoC requirements (defined in Equation~(\ref{eq: soc_penalty})). Following multiple constraints related to the EV SoC update  function following Equation~(\ref{eq: soc}) and Constraints~(\ref{eq:charging_rate}) to (\ref{eq:building_power}). 
%
% During training, optimal actions are stochastically incorporated into the state transition process based on a predefined coefficient, denoted as $\policyGuidanceRate$, and stored in the replay buffer (see Algorithm~\ref{alg:DDPG}). We use $\mathit{MILP(S(T_j), {\it remainEpisode})}$ to get the next optimal action. Here, {\it remainEpisode} considers the remaining events in the fixed sample, such as upcoming EV arrivals, SoC requirements, and building power, to invoke the MILP solver. Using the MILP solver helps us maximize the long-term reward. 

\subsubsection{Actor-Critic Network Structure} 
Both the actor and critic networks are fully connected, having two hidden layers with 96 neurons each. Both feature a ReLU activation layer at the end. The critic network outputs a single Q-value estimate, while the actor network outputs the action, which represents the charging power of each charger.
%
To enhance convergence and improve generalization, we normalize all state variables to be within $[0, 1]$ before feeding them into neural networks. Time slot $T_j$ is normalized by division with the number of time slots in a day ($\frac{24}{\delta}$), while power-related variables such as building power $\Building(T_j)$, estimated peak power $\PrdPeak(T_j)$ are scaled by their respective statistical values from training data. Furthermore, we normalize the energy capacity $CAP(V)$ of each car by division with the maximum capacity among EVs, $\max(CAP(V))$.
For the action $A(T_j)=[P(C_i, T_j)]_{C^{i}\in \mathcal{C}}$, we constrain the output within the range $[-1, 1]$ using the $\tanh$ activation function. It is finally translated into the 
charging power range $[C_i^{min}, C_i^{max}]$ by scaling the value using a constant factor.

%We normalize the action values to the range of $[-1, 1]$ based on the charging power range $[C_i^{min}, C_i^{max}]$.
% by: 
% \begin{equation}
%     \hat{P}(C_i, T_j)=\frac{2\times(P(C_i, T_j)-C_i^{min})}{C_i^{max} - C_i^{min}}-1. 
% \label{eq: normalize}
% \end{equation} 
%The output of the actor network is constrained within the range $[-1, 1]$ using a $\tanh$ activation function, from which the unnormalized charging values are retrieved.
% The original charging power values can be obtained by computing the inverse of the normalization equation in Equation~(\ref{eq: normalize}). 

\subsubsection{Heuristics and Action Post Processing}

% for tractability the RL model focused on weekday and peak- billing period. For others including off-peaks and weekend we use heuristics. Here are the list of heuristics we use.  We will show later the results comparing to using heuristics for all periods as well.
% For tractability, the RL model focuses only on weekdays and peak-time billing periods. For off-peaks and weekends we use heuristics. We observe that off-peak hours offer lower electricity prices enabling EV charging at higher charging powers, also off-peak hours are not considered in the calculation of demand charge. Thus, heuristics can be used to optimize EV charging during these periods. Similarly, weekends often experience far fewer users than weekdays leading to low building power demand. Also, TSOs do not consider in the demand charge calculation, leading to minimal opportunities for optimization. During RL training, we use a charge first heuristic based on a least laxity task scheduling algorithm (Charge First LLF described in Section~\Cref{ssec:alternatives}) during off-peak hours and weekends that guarantees that EVs reach the required SoC before departure time. 

To enhance the ease of learning in this complex decision space, we use the RL model on weekdays and the peak hours of TOU price within each billing period (for both training and inference). For off-peak hours and weekends, we use a heuristic based on the least laxity task scheduling algorithm (described in \Cref{sec:experiments_and_results}) to ensure EVs achieve the required SoC before departure, calculating the minimum charge needed for each time slot. Off-peak hours offer lower electricity prices, allowing for higher EV charging rates, and are excluded from demand charge calculations, making heuristics effective for optimization. Similarly, weekends see fewer EV arrivals and lower power demand, with Transmission System Operators excluding them from demand charge assessments. 
% During training (and inference), we use 
% \jpnote{Cite paper for this}
% Finally, we condense the state features while accounting for the common minimum and maximum SoC boundaries of all EVs (with $\SOCMIN=0$ and $\SOCMAX=90\%$), we do not include SoC boundaries in the state representation, limiting the policy's ability to control actions. Thus, to maintain valid SoC boundaries, we apply a post-processing procedure~\cite{XXXX}, which differs from action masking. Post-processing is not integrated with the actor network and influenced by training backpropagation. This adjusts policy-generated actions before they are given to the environment, ensuring that charging EVs do not exceed $SoC^{\text{max}}$ and discharging prevent it to go below $SoC^{\text{min}}$. By employing this approach, we ensure that all policy-generated charging powers for charging EVs remain within the defined SoC boundaries, thereby satisfying Constraints~(\ref{eq:soc_min}) and~(\ref{eq:soc_max}).  
% Finally, we condense the state features while accounting for the common minimum and maximum SoC boundaries of all EVs, following the guidelines set by the EV manufacturer. To avoid limiting the policy's flexibility by including SoC boundaries in the state representation, we apply a post-processing procedure. This is a common approach in RL, utilized in AlphaGo~\cite{silver2016mastering}, where illegal actions are filtered before interacting with the environment to save training resources that would otherwise be spent on learning valid actions, which is not the final objective. Similarly, we adjust policy-generated actions before passing them to the environment, ensuring that EVs' SOC stays within the limits. Specifically, we stop discharging if the EV's SoC is below $SoC^{\text{min}}$, and stop charging if it exceeds $SoC^{\text{max}}$, satisfying Constraints~(\ref{eq:soc_min}) and~(\ref{eq:soc_max}). 
% Following the guidelines set by the EV manufacturer, we need to limit charging EVs to their SoC boundaries. We apply a post-processing procedure, a common approach in RL, utilized in AlphaGo~\cite{silver2016mastering}, where actions are adjusted to avoid illegal actions before interacting with the environment. This saves training resources that would otherwise be spent on learning valid actions, which is not the final objective. Similarly, considering the exceeding SoC boundary rarely happens during training and learning the validity of matining SoC limit is not our final objective which is to minimize total bill,  we adjust policy-generated actions before passing them to the environment, ensuring that the EVs' SoC stays within the specified limits. Specifically, we stop discharging if the EV's SoC is below $SoC^{\text{min}}$, and stop charging if it exceeds $SoC^{\text{max}}$, thereby satisfying Constraints~(\ref{eq:soc_min}) and~(\ref{eq:soc_max}).  
Following the EV manufacturer guidelines, we limit charging to SoC boundaries by clipping the actions of the learned policy within $[SoC^{\text{min}}, SoC^{\text{max}}]$ through post-processing to satisfy Constraints~(\ref{eq:soc_min}) and~(\ref{eq:soc_max})
% In this work, we assume all EVs share common minimum and maximum SoC limits. A post-processing procedure, similar to AlphaGo~\cite{silver2016mastering}, adjusts actions to prevent illegal steps during environmental interaction, conserving training resources. Since exceeding SoC boundaries is rare and maintaining SoC limits is not our main goal, which is minimizing the total bill, we adjust policy-generated actions after they are passed to the environment simulator. Specifically, we stop discharging if SoC is outside $[SoC^{\text{min}},SoC^{\text{max}}]$.

\subsection{Inference}
During execution, our RL-based policy, which is a trained actor network with the action masking procedure, operates at $\delta$ time intervals to determine the charging power for all chargers. At each time slot, the state features are generated from data captured from the environment, including charger status (connected EV's current SoC, expected departure time, and SoC), the building's current power and charging rate limits.
% , which are obtained through the simulator interface. 
% Additionally, we apply the heuristic approach: Fast Charge LLF (detailed in \Cref{ssec:results}) during off-peak hours and weekends. 
While we use the estimated peak power $\hat{P}^{max}$ as the state feature based on training samples, as shown in~\Cref{fig:pipeline}, it can be replaced by any data-driven forecasting or prediction model. Then, we input all the normalized state features, as described in~\Cref{ssec:MDP}, into the trained RL model to get the charging actions for the next time interval.

%{which can be derived from historical data or machine learning-based predictive models}.


% \subsection{Inference in Real-time}
% \jpnote{Write this in terms of how this will be used in the real world. Reference the digital twin.}
% write about inference diagram from workflow. we use the digital twin to track what the real-system state. We take a full chain of a month; we use the best heuristic for non-peak and weekend and for weekday and peak period we use RL policy. 


% \section{Approach}
\label{sec:approach} 



% \begin{figure*}[t]
%     \centering
%     \begin{subfigure}[b]{0.68\linewidth}
%         % \centering
%         \includegraphics[height=2.2in,keepaspectratio,trim=0.2cm 0.2cm 0.2cm 0.2cm,clip]{figures/framework.pdf}
%         % \vspace{-0.1in}
%         \caption{Reinforcement Learning Framework.}
%        \label{fig:framework}
%     \end{subfigure}
%     % \hfill
%     \begin{subfigure}[b]{0.28\linewidth}
%         % \centering
%         \includegraphics[height=2.2in,keepaspectratio,trim=0cm 0.5cm 0cm 0.5cm,clip]{figures/inference_pipeline.pdf}
%         \vspace{-0.15in}
%         \caption{Pipeline for Inference.}
%         \label{fig:pipeline}
%     \end{subfigure}
%     \caption{\color{black}(a) Our framework relies on daily samples along with an estimated monthly peak power. A reinforcement learning policy based on DDPG is trained with policy guidance and action masking. These techniques improve the performance of the model. Finally, it generates charger actions for every given state. (b) At inference time $T_j$, the model requires data of connected cars, charger states and building load reading. It then uses a monthly peak power estimator to forecast the peak power for the entire billing period. The result are the charger actions for time $T_{j+1}$.}
%     \label{fig:framework_and_pipeline}
% \end{figure*}

% \begin{figure*}[t]
% \centering
% % \includegraphics[width=0.95\textwidth]{figures/framework_only.pdf}
% \includegraphics[width=0.95\textwidth]{figures/framework_v2.pdf}
% \caption{(Left) Data Processing Pipeline, (right) Complete Framework of our RLApproach.}
% \label{fig:framework_and_pipeline}
% \Description{}{}
% \end{figure*}


%We employ the Deep Deterministic Policy Gradient (DDPG) algorithm~\cite{lillicrap2015continuous} to train the policy $\pi(S(T_j))$ for the V2B problem. DDPG is ideal for handling our continuous action space and supports off-policy training, allowing the RL model to learn from diverse experiences across various scenarios, thereby enhancing model generalization. To enhance the policy performance, we propose integrating action masking and policy guidance techniques alongside DDPG, as outlined below. Figure~\ref{fig:framework_and_pipeline}.  

%. This framework enables optimal decision-making regarding charger power rates and load management, allowing for effective responses to uncertainties in user behaviors and building loads in our working scenario. We aim to derive a policy for each time slot that minimizes total costs while ensuring that the SoC requirements of the EVs are met throughout the billing period. 

%Considering the uncertainty in the V2B problem, we model it as a sequential decision-making problem using a Markov Decision Process (MDP) to simulate state transitions and utilize an RL-based approach to train a policy for this MDP. 

In this section, we first outline the data pipeline, focusing on how we handle the monthly training data, generate input features, and execute the policy (Figure~\ref{fig:framework_and_pipeline}) and then detail the MDP formulation. \ad{while the figure talks about delayed update, its discussion clearly in a highlighted paragraph is missing. This is an important part and hence must be emphasized.}
%, then discuss the steps of the data processing pipeline, 
 

\subsection{Data Pipeline}
\label{ssec:pipeline}
%To generate training and testing data for the RL models in the V2B problem, we follow these steps for data collection and processing. 
% % \begin{enumerate}[leftmargin=*] 
% \textbf{Data Collection}: We gather data from \nissan{}'s EV fleet, including arrival and departure times, initial SoC, and required SoC. We also collect historical smart building load demand across different seasons from May 2023 to January 2024. Additionally, data from EV chargers is collected, including charging/discharging rates.
    
% \textbf{Modeling for User Behavior and Building Load Fluctuation}: Based on the collected data, we model multiple features. For instance, we use the Poisson distribution to model EV arrivals at each hour, considering inputs such as current time, month, and weather information. Similarly, we train models using random forest to generate estimateed EV stay duration, initial SoC, and required SoC, effectively modeling EV driver behaviors. To model the building load fluctuation, we utilize 9 months of real building energy consumption data collected from \nissan{}'s smart building, recorded at 15-minute intervals. The building load is modeled using a normal distribution, with the real building load as the mean and introducing a standard error of 2 kW.

% \textbf{Generating Sample Chains for Training and Testing}: We then generate training samples for one-month billing period episodes, including EV arrival/departure schedules along with their initial SoC and required SoC, using the developed models for user behaviors. Building load data is generated at every 15-minute interval over a month. Additionally, we incorporate data for 15 \nissan{} EV chargers, consisting of 5 bidirectional and 10 unidirectional chargers. Time-of-Use (ToU) electricity rate schedules and demand charge data from utility providers are also integrated to simulate realistic billing scenarios. We generate 110 samples for each month from May 2023 to January 2025. Each sample contains all the above information for each day of the month, and we use 60 samples for training and 50 for testing.  
{\color{black} 
We outline the data pipeline in Figure~\ref{fig:framework_and_pipeline}, which encompasses the preliminary processes for RL training, detailing data collection, sample generation, peak power estimation, and training episode segmentation. \ad{clarify how much data exists. Also, clarify how we reduce the monthly problem to daily problem}
% These are all essential steps for effective RL training. 
 %  \textbf{Data Collection.} We first gather comprehensive data from \nissan{}'s EV fleet, including arrival and departure times, initial and required State of Charge (SoC), as well as historical smart building load demand across seasons from May 2023 to January 2024. Additionally, we collect charging and discharging rates from EV chargers. Multiple features are modeled to capture these distributions using the Poisson distribution for EV arrivals, SoC requirements, and building fluctuations based on historical data.

% \textbf{Generating Sample Chains for Training and Testing}. We create training samples for one-month billing episodes that encompass EV schedules, initial SoC, and required SoC. Building load flows are generated at 15-minute intervals based on historical data. We also integrate Time-of-Use (ToU) electricity rate schedules and demand charge data from utility providers to simulate realistic billing scenarios. 
% , as well as historical smart building load demand across seasons from May 2023 to January 2024. 
% Additionally, we collect telemetry data from EVs and EV chargers.
{\bf Sample generation.}
Data was collected and provided by \nissan{} from their fleet and chargers at their office building over the period from May 2023 to January 2024. The dataset includes EV arrival and departure times and initial and required State of Charge (SoC) as well as building power demand readings at 15 minute intervals. Since the location is categorized as an industrial building, it is subject to time-of-use (TOU) electricity rates and demand charges. TOU introduces variations in the electricity rates, with higher prices during peak hours (6:00 AM to 10:00 PM). Demand charge adds a flat rate multiplier to the highest average power delivery across 15-minute intervals over the billing period, and is added to the total bill.
To capture the distributions of these features, we utilize the Poisson distribution for modeling EV arrivals, SoC requirements, and building fluctuations based on historical data. We then empirically sampled $1000$ one-month billing episodes samples for each month from May 2023 to January 2024.  
% The building did not have data before May 2023 and data collection stopped in Jan 2024.

%Additionally, we integrate Time-of-Use (ToU) electricity rate schedules and demand charge data from utility providers to simulate realistic billing scenarios. Using these data models, we generate 1,000 samples for each month from May 2023 to January 2024. 

\textbf{Splitting data into daily episodes.}
\jpt{Can we explain a bit more on how splitting monthly to daily help with the training?}
To enhance training efficacy, we address the challenge of lengthy state-action episodes by splitting the monthly training dataset into daily episodes.
We focus only on weekdays since employees are often not present at the office, resulting in minimal EV arrivals and low building loads, both of which do not significantly impact the overall monthly demand charge. We incorporate the estimated monthly peak power as an input feature and penalize only those peak power values exceeding this estimated value in each daily episode. This approach incentivizes the policy to minimize the monthly peak power while effectively training with daily episodes.

{{\bf Down-sampling Strategy}.
During training, we varied the amount of samples used for training and observed that utilizing a larger number of training episodes results in longer convergence time and overall worse performance. We show the results of this in~\Cref{ssec:ablation}.
% During training, we implement a down-sampling strategy from the 1,000 monthly samples. We observe that utilizing the complete dataset results in excessive training iterations, significantly increasing the training duration (up to 3 days on our computing systems) and complicating convergence, motivating us to perform down-sampling. 
Thus, we utilize a k-means clustering approach to down-sample. We used $k=5$ and clustered using the optimal demand charge derived from the MILP solution for each sample. 
% This optimal demand charge is closely correlated with sample characteristics, such as building load distribution and EV user's behavior, and significantly influences the total bill. 
% Specifically, We clustered 1,000 samples into five groups based on optimal demand charge using k-means. 
From each cluster, we select 60 and 50 samples for the final training and testing datasets respectively, ensuring that these datasets are mutually exclusive. 
% This method allows us to train and evaluate our RL model using fewer but representative training samples, thereby mitigating the loss of diversity and distribution among the samples.
}

\textbf{Peak power estimation.} 
% A key aspect of our state definition is the monthly peak power estimation. 
We employ a MILP solver to determine the minimum demand charge derived from optimal action sequences over the one-month billing period across all training samples. From the distribution of peak power from the MILP solutions, we set the lower bound of the 95\% confidence interval as the estimated peak power for the month, providing a conservative initial estimate.



%{\bf Data Collection}: We gather data from \nissan{}'s EV fleet, including arrival and departure times, initial and required State of Charge (SoC). Historical smart building load demand is collected across seasons from May 2023 to January 2024, along with charging and discharging rates from EV chargers. Multiple features are modeled to capture these distributions, employing the Poisson distribution for EV arrivals, SoC requirements, and building fluctuations based on historical data.
%{\bf Generating Sample Chains for Training and Testing}: We create training samples for one-month billing episodes, encompassing EV schedules, initial SoC, and required SoC, using the developed models. Building load flows are generated at 15-minute intervals based on historical data. Additionally, Time-of-Use (ToU) electricity rate schedules and demand charge data from utility providers are integrated to simulate realistic billing scenarios.

% \textbf{Feature Generation}: Based on the cleaned data, we generate input features such as building load profiles, EV arrival schedules, initial SoC, and departure SoC targets. These features serve as the input states for the RL model training and testing phases. 
% This process ensures that the RL models are trained on accurate and diverse data, providing robustness and generalization across various V2B scenarios.

%{\bf Generating samples.} To assess our RL-based V2B charging optimization approach, we used real-world data collected {\color{black} from an \nissan{}'s office building, along with charger meter readings in Santa Clara, California.} This data was used to derive statistics and develop generative models for generating samples for training and testing the RL model, which also serves as input to our simulator.

%\textbf{Monthly Peak Power estimation:} Our state definition includes an initial peak power estimation over the whole billing period as an input feature. We employ a Mixed-Integer Linear Programming (MILP) solver to determine the minimum demand charge derived from optimal action sequences over a one-month period across all training samples. For each month, we analyze the distribution of peak power from the MILP solutions and use the lower bound of the 95\% confidence interval as the estimated monthly demand charge, providing a conservative initial estimate. 
 
%{\bf Splitting Training Data into Daily Episodes:} During our training process, we encountered a challenge: utilizing a one-month billing period resulted in lengthy state-action episodes, hindering the policy's ability to learn the Q-value effectively. To address this, we reduced the training episodes to daily segments, focusing on weekdays. This decision was informed by our observation that weekends often have infrequent EV arrivals and low buidling load, which do not significantly impact the overall monthly demand charge.  To ensure that monthly peak power considerations are integrated, we incorporate the estimated monthly demand charge as an input feature. Additionally, the reward function penalizes only those demand charges that exceed this estimated value in each daily episode. This approach incentivizes the policy to minimize the monthly demand charge while effectively training with daily episodes.

{\bf Model inference.} Our RL-based policy operates at $delta$ intervals to determine current policy actions. To generate these actions, we require a set of input data, including the charger status (connected EV's current SoC, expected departure time, current building load, and charger information), the charging rate limits for each charger, and the estimated peak power. 
%{which can be derived from historical data or machine learning-based predictive models}.
In this paper, we utilize the estimated peak power based on training samples. With this current information, we abstract the input features for the RL model, enabling it to determine the power rate control actions for the upcoming time interval.
} 



\subsection{Environment Simulator}
Our approach uses stateless discrete event simulator that serves as the digital twin of our target environment. It holds a state that represents the entirety of the world. This includes information on EVs, building, and the grid. This allows us to investigate how any action or decision can potentially impact the real world. Decisions are taken at the end of each set of events for any given time period. There are two main decisions that must be taken when solving the V2B charging problem. (1) Charger assignments and (2) Charger actions. We address the charger assignment decision below and provide information on the charger action in~\Cref{sec:RL}.

\textbf{Environment updates.} The input episodes serve as the entirety of the simulator's world view. Each event includes an event type and time, matching their real world trigger and occurrence. We identify several critical events in the episodes to serve as the triggers for the simulator. These include (1) EV arrivals, (2) EV departures, (3) building power readings, and (4) TOU rate changes. Events are placed in a queue, with each event triggering an update to the environment which modifies the state. Updates to the state, which include the charging or discharging of EVs, are based on the elapsed time between events. Only EV arrival events trigger charger assignment decisions, while all events trigger a charger action decision.

\input{results/charger_assignments}
\textbf{Charger assignment.} In our approach we consider a first-in, first-out policy that prioritizes the assignment of EVs to bidirectional chargers. If multiple EVs arrive at the same time, then we break the ties randomly. \Cref{table:charger_assignment_policies} shows the different charger assignment and tie breaking policies tested. Bidirectional charging assignments outperform any other assignment policy. Tie breaking policies that favor latest departing cars have marginal advantage over others. While the assignment policies can be further optimized, we elected to follow this heuristic, focusing instead on the second decision problem of charger action.

\textbf{Charger actions.} We provide several policies with our simulator to contrast and compare with our proposed approach. Charger action policies receive a state of the environment for a particular time and generate actions based on this. The simulator is stateless. Thus, it provides only a current representation of the world at that specific time to each policy.


\subsection{Markov Decision Process Model}
\label{ssec:MDP}

%We model the V2B problem as a Markov Decision Process (MDP). This framework enables optimal decision-making regarding charger power rates and load management, allowing for effective responses to uncertainties in user behaviors and building loads in our working scenario. 
 % \color{black} We define the MDP as a tuple $(\mathcal{S}, \mathcal{A}, {\it Trans}(S(T_j),A(T_j))$, ${\it Reward}(S(T_j)$, $A(T_j))$, where $\mathcal{S} = \{S(T_j) \mid T_j \in \mathcal{T}\}$ is the set of states representing the system's state at the beginning of each time slot $T_j$, $\mathcal{A} = \{A(T_j) \mid T_j \in \mathcal{T}\}$ is the set of actions, with each action $A(T_j) \in \mathcal{A}$ representing a decision made at time slot $T_j$ and including continuous values indicating the power rate of all chargers. The state transition function $\text{Trans}(S(T_j), A(T_j))$ describes the transition from state $S(T_j)$ to the next state based on action $A(T_j)$, while the reward function $\text{Reward}(S(T_j), A(T_j))$ assigns a reward for taking action $A(T_j)$ in state $S(T_j)$. The key notations in the MDP are provided as follows: 

%We briefly describe the state, actions and transitions for the problem. 
% The key notations in the MDP for the V2B problem are as follows:
% \rishav{add small \\description of MDP, \\as a tuple\\ (S,A,P,T,$\gamma$, \\describing each \\in few words}
%Our objective is to determine the optimal power rate sequence $\mathcal{P}$ across all time slots $T_j \in \mathcal{T}$. 
%To address the uncertainly in our working scenario, we model the V2B problem as a Markov Decision Process (MDP), which effectively tracks changing conditions like fluctuating SoC, building load, and unpredictable EV arrivals. MDPs provide a framework for making optimal decisions regarding charger power rates and load management, ensuring effective responses to variations in user behavior and demand.    We model the V2B charging control process as a MDP, aiming to find the optimal power rate sequence \(\mathcal{P}\) for all time slots \(T_j \in \mathcal{T}\). Our goal is to derive the optimal policy for determining $P(C_i, T_j)$ for each time slot, minimizing the total cost while ensuring that the EVs' SoC requirements are met throughout the billing period.     




{\bf State.}
%At each time slot $T_j\in\mathcal{T}$, we define state $S(T_j) \in \mathcal{S}$ by using a combination of features that we identified through a mix of domain expertise and experiments.
The complete state space for the problem can be described using features that provide historical, current, and future estimation at a given time $T_j$.
%These features provide historical as well as current context and include estimated value of peak power that is used to calculate demand charge.  The complete state features of our V2B problem at each decision-making time 
This includes key input parameters for each vehicle such as the current SoC, required SoC, departure time, and battery capacity for each EV, along with SoC boundaries across 15 chargers. Additionally, the current building load, time slot, day of the week, past historical building load, and long-term peak power estimation value are included, resulting in approximately 100 features. While this state space is complete, it is not tractable to be used for the learning process. Therefore, we had to reduce the state space. For this,
%For improving practical performance, 
we leveraged domain-specific knowledge to abstract key information, reducing the state space to 37 essential state elements without compromising crucial data. We describe them below.
%abstracting the following features providing historical and current building load and arrival EVs, as well as features for future estimation:
    % \item The current building load is $\Building(T_j)$. 
    % \item The power gap between the current builidng load to the estimated peak power for the billing period is $\PrdPeak(T_j) - \Building(T_j)$. By incorporating this long-term peak power estimation, we provide the RL model with a feature that aids in estimating the optimal peak power for demand charge reduction.
    % \item The historical building load information which constrains the mean peak building load$\mu(B^H(T_j))$ and variance $\sigma^2(B^H(T_j))$  over the previous 7 days. This data can also inform estimations of future building load.  
    % \item The current time slot $T_j$ and day of the week are included to help the RL model distinguish daily patterns and manage feature diversity, enhancing generalization across different weekdays. 
    % \item Number of EV arrivals upto time slot $T_j$ for the current day, represented as $|\{V| V\in \mathcal{V}, A(V)\leq T_j \}|$, tracking the current EVs and their status. 
    % \item We include the status of all chargers $\CS(T_j) = \{(\PowerNeed(C_i, T_j)$, $\ReTime(C_i, T_j))\}_{C_i \in \mathcal{C}}$, where each charger status is defined by two components: (i) $\PowerNeed(C_i, T_j)$ representing the energy gap between the required SoC and the current SoC of the EV, where, 
    % \begin{equation*} 
    %  \PowerNeed(C_i, T_j) = ( \SOCR(V) - \SOC(V, T_j) ) \times {\it CAP}(V)
    %  \label{eq:chargerstate}
    % \end{equation*} 
    % with $V=\phi(C_i, T_j)$ indicating the EV connected to $C_i$ at time slot $T_j$
    % and, (ii) $\ReTime(C_i, T_j) = D(\phi(C_i, T_j)) - T_j$, specifying the time remaining till the departure of the EV.
   
    \begin{enumerate}[leftmargin=*]
    \item The current time slot, $T_j$.
    \item The current building load, denoted as ${B}(T_j)$.
    \item The power gap between the current building load and the estimated peak power for the billing period, $ \PrdPeak(T_j) - {B}(T_j)$. It aids the RL model in estimating optimal peak power for demand charge reduction.
    \item The mean peak building load over the previous 7 days, $\mu(B^H(T_j))$.
    \item The variance of the peak building load over the previous 7 days, $\sigma^2(B^H(T_j))$. It informs the model about the future building load.  
    \item The day of the week for the current time slot, $T_j$. It helps the RL model distinguish daily patterns and enhance generalization.
    \item The number of EV arrivals up to time slot $T_j$, represented as $|\{V | V \in \mathcal{V}, A(V) \leq T_j \}|$. This helps in tracking the number of EVs currently present.
    \item The energy needed by each EV that is connected to a charger at time slot $T_j$, given by $[\PowerNeed(C_i, T_j)]_{C_i \in \mathcal{C}}$, and is initialized to $0$. This represents the energy gap between the required SoC and the current SoC of the EV connected to charger $C_i$ at time slot $T_j$, defined as: 
    \begin{equation} 
     \PowerNeed(C_i, T_j) = (\SOCR(V) - \SOC(V, T_j)) \times \text{CAP}(V)
     \label{eq:chargerstate}
    \end{equation}
    where $V=\phi(C_i, T_j)$ indicating the EV connected to $C_i$ at time slot $T_j$. 
    \item The remaining time until the departure of each EV connected to the chargers is given by $[\ReTime(C_i, T_j)]_{C_i \in \mathcal{C}}$, and is set to 0 when no cars are connected. Each term is computed as $\ReTime(C_i, T_j) = \DepartureTime(\phi(C_i, T_j)) - T_j$. 
\end{enumerate} 
   % where $RT(C_i,T_j)$ represents the remaining time before the departure of the EV, i.e.,
    %The energy gap $\PowerNeed(C_i, T_j)$ is calculated based on the difference between the target SoC and the current SoC at time $T_j$.
   % Status of all chargers is given by $\CS(T_j) = \{ CS(C_i, T_j) \}_{C_i \in \mathcal{C}, T_j \in \mathcal{T}}$. Each charger status is specified by $CS(C_i, T_j) = (\PowerNeed(C_i, T_j)$ , $RT(C_i, T_j))$. Here, $\PowerNeed(C_i, T_j)$ represents the energy gap between the required SoC and the current SoC for the EV connected to $C_i$, calculated by: 
 %We set $CS_t^i = (0,0)$ if no EV is connected to charger $C^i$.

% {\bf Initial State.} The initial state $S(T_0)$ is defined at the beginning of the billing period. It consists of the starting building load $B(T_0)= 0$, the estimated peak power $\PrdPeak(T_0)$, which may be derived from historical data or forecasts (as detailed in Section~\ref{ssec:pipeline}), and historical load values from the preceding 7 days, represented as $B^{H}(T_0)$. The initial time is set to $T_0$, corresponding to the day of the week. At this initial time, the number of EV arrivals is $0$. Additionally, all parameters of the initial status of all chargers $\CS(T_0)$ are set to 0.  
%the initial status of all chargers, denoted as $\CS(T_0) = \{ CS(C_i, T_0) \}_{C^i \in \mathcal{C}}$, is initialized with $CS(C_i, T_0) = (0,0)$ for all chargers.  

{\bf Actions.} The actions $A(T_j) \in \mathcal{A}$ in this MDP are continuous and specify the power rates of all chargers at time $T_j$, where $A(T_j) = [P(C_i, T_j)]_{C^i \in \mathcal{C}}$.

{\bf State Transition.} We utilize a discrete event simulator to track the state transitions. \ad{Need to write a bit about the simulator.} The simulator is given a ``chain'' empirically sampled from the data provided by our partner representing a day in the monthly billing period. 
\jpt{Will connect it to the new subsection in Approach}
Each chain is accompanied by the estimate of peak power over the entire billing period, $\PrdPeak(T_0)$, based on the full sequence of daily chains across a month. This peak is generated by solving a MILP program that gives all the monthly chains as an input, and during training, this ``optimal peak'' across the month is used as an input for a given daily chain \ad{the whole concept of daily and monthly chains is a bit confusing and has to be cleared up. Basically how we reduce the monthly problem into a sequence of daily problems has to be described here clearly.}.


% Given the complexity of modeling state transition probabilities in the V2B problem, we conduct simulations for each billing period, referred to as a ``sample''. The simulator includes:
% \begin{itemize}[leftmargin=*] 
%     \item {The estimated peak power over entire billing period, $\PrdPeak(T_0)$ based on training samples.}
%     \item Electricity prices, $\theta_E(T_j)$ for $T_j \in \mathcal{T}$.
%     \item Building load, ${B}(T_j)$ for $T_j \in \mathcal{T}$.
%     \item EV arrival and departure schedules with SoC requirements, $\forall V \in \mathcal{V}: \SOCI, \SOCR, SOC, CAP$.
%     \item Available chargers and charging limits $\{C_i \in \mathcal{C}\}$.
% \end{itemize} 
The simulator updates the status of the building and chargers based on actions taken at each time slot. 
%The state transition function is defined as ${\it Trans}: \mathcal{S} \times \mathcal{A} \leftarrow \mathcal{S}$, with $(S(T_j), A(T_j)) \mapsto S(T_{j+1})$,  
%\rishav{change \\this and simulator\\ steps to use \\ ${T_j}$ \&$T_{j-1}$} 
%indicating how state features are updated.
This process includes the following steps: 

\begin{enumerate}[leftmargin=*]
    \item Initialize the estimated peak power, $\PrdPeak(T_0)$, which can be derived from historical data \ad{does not make sense. We need to clearly describe the process of how this is estimated for training as well as inference. Please update. There is no mention in section 4.2} (detailed in Section~\ref{ssec:pipeline}),
    , and update it by
    $
    \PrdPeak(T_{j+1}) = \max(\PrdPeak(T_j)$, $ \Building(T_j) + \sum_{C^i \in \mathcal{C}} P(C_i, T_j)),
    $
    which updates the estimated peak power depending on the previous estimate and the current peak power.
    \item Update SoC of EVs connected to all chargers: $\SOC(\phi(C_i,T_j), T_{j+1})$ using action $A(T_j)$ according to Equation~(\ref{eq: soc}). 
    %Additionally, we apply action post-processing to keep EV SoCs within valid boundaries by adjusting the power rate for stopping charging or discharging when they exceed $SoC^{\text{max}}(\phi(C_i, T_j))$ or drop below $SoC^{\text{min}}(\phi(C_i, T_j))$. 
   \item Update the EV charger assignment $\phi(C_i, T_j)$ and $\eta(V_k)$ by first releasing chargers with departing EVs in the next time slot $T_{j+1}$ and then assigning new arrival EVs to idle chargers, following the FIFO procedure and prioritizing the bi-directional chargers first. 
   \item Update the energy requirement of all EVs connected to a charger: $[\PowerNeed(C_i, T_{j+1})]_{C_i \in \mathcal{C}}$ (by Equation~(\ref{eq:chargerstate})) based on EV's current SoCs.
   \item Update the remaining time of all EVs connected to chargers: $[\ReTime(C_i, T_{j+1})]_{C_i \in \mathcal{C}}$ at time slot $T_{j+1}$.   
\end{enumerate}


{\bf Action Reward.} We define the function ${\it Reward}: \mathcal{S} \times \mathcal{A} \rightarrow \Re$, where ${\it Reward}(S(T_j), A(T_j))$ evaluates the reward for actions taken in a specific state, focusing on minimizing the total bill while satisfying SoC requirements. This function is expressed by: \ad{is the subscript index for actions is correct? }
\begin{align}
   & \mathit{Reward}(S(T_j), A(T_j)) = \lambda_{S} \cdot \mathit{Reward}_1 + \lambda_{E} \cdot \mathit{Reward}_2 + \lambda_{D} \cdot \mathit{Reward}_3
\end{align}
where, 
\begin{align*}
\begin{aligned}
    \mathit{Reward}_1 &=  \sum\limits_{C^i\in\mathcal{C}} \max(0, \min(\PowerNeed(C_i, T_j), P(C_i, T_j) \cdot \delta)) \\
    \mathit{Reward}_2 &= - P(C_i, T_j) \cdot \delta  \cdot \theta_E(T_j) \\
    \mathit{Reward}_3 &= - \max(0, \Building(T_j) + \sum\limits_{C^i \in \mathcal{C}} P(C_i, T_j) - \PrdPeak(T_j)) \cdot \theta_D
\end{aligned}
\end{align*}
% \begin{align}
%    & \mathit{Reward}(S(T_j), A(T_j)) = \nonumber \\
%    & \sum\limits_{C^i\in\mathcal{C}} \max(0, \min(\PowerNeed(C_i, T_j), P(C_i, T_j) \times\delta)) \times \lambda_{S} \nonumber \\
%    & - P(C_i, T_j) \times\delta  \times \theta_E(T_j) \times \lambda_{E} \nonumber \\
%    & - \max(0, \Building(T_j) + \sum\limits_{C^i \in \mathcal{C}} P(C_i, T_j) - \PrdPeak(T_j)) \times \theta_D \times \lambda_{D} 
% \end{align}
where $\mathit{Reward}_1$ promotes actions that charge EVs to reach their required SoC, as outlined in Eq. (\ref{eq: soc}). $\mathit{Reward}_2$ penalizes the energy costs generated by this action, and $\mathit{Reward}_3$ penalizes the increase in demand charges caused by rising peak power, aligning with our objective in Eq. (\ref{eq: billing}). These functions use three reward coefficients, $\lambda_{S}$, $\lambda_{E}$, and $\lambda_{D}$ to balance trade-offs between these reward factors. 
%The coefficients are adjustable during training to optimize model performance. 

% {\bf Policy}: Our objective is to develop the policy $\pi: \mathcal{S} \rightarrow \mathcal{A}$, where $\pi(S(T_j))$ generates the optimal charging actions for a given state, maximizing the overall action reward function throughout the billing period $T_j \in \mathcal{T}$.
 
% \rishav{Add discount factor \\short description}

% % \begin{algorithm}[t]
% \setcounter{AlgoLine}{0} %
% \small
%     \SetAlgoLined
%     \KwIn{Initial policy parameters for actor network $\zeta_a$, critic parameters $\zeta_c$, target network parameters $\zeta_a', \zeta_c'$\\
% Training parameters: $\mathit{actionNoise}$, $\policyGuidanceRate$, $\mathit{bufferSize}$, $\mathit{batchSize}$, maximum training iterations: $M$}
%     \KwOut{Trained policy $\pi_{\alpha}$}
%     Initialize replay buffer $\Buffer$ \\
%     % Initialize target network weights $\zeta_a' \leftarrow \zeta_a$, $\zeta_c' \leftarrow \zeta_c$ \\
%     \For{$1$ \KwTo $M$}{
%         $s_0$ \gets initial state from the simulator 
%         % Initialize a random process $N$ for action exploration 
        
%         \For{each time slot $T_j \in \mathcal{T}$}{
%         % \If{$T_j$ is during non-peak hours}{
%         % Get action A(T_j) using h
%         %}
%         \tcp{Introducing policy guidance stochastically.}
%             {\color{black} randomValue $\leftarrow randomBetween(0,1) $
            
%             \If
%             %(\tcp*[h]{Adding policy guidance stochastically}) $$
%             { randomValue $\leq \policyGuidanceRate$}{
%         Get action $A(T_j)$ by rerunning the ILP solver: $A(T_j)\leftarrow\MILP(S(T_j), {\it remainEpisode})$
%                 }
%             \Else{
%                 % $A(T_j)\gets  \pi(S(T_j) | \zeta_a) + \mathit{actionNoise}$
%                 Get masked action $A(T_j) \gets \Mask\left(S(T_j), \pi(S(T_j) | \zeta_a)) + \mathit{actionNoise} \right)$ by current policy and $\mathit{actionNoise}$  
%                 % \tcp{Add action masking}
%             }}
%              State transition $S(T_{j+1})\leftarrow {\it Trans}(S(T_j), A(T_j))$. 
%             %in Algorithm~\ref{alg: stateTrans}. 

%             Get the action reward $R(T_j)\leftarrow {\it Reward}(S(T_j), A(T_j))$. 
%             %by Algorithm~\ref{alg: reward}. 
            
%             Store transition $(S(T_j), A(T_j), R(T_j), S(T_{j+1}))$ in $\Buffer$

%             Sample a minibatch $(S(T_i), A(T_i), R(T_i), S(T_{j+1})$ from $\Buffer$ 

%            {\color{black} Get masked actions $A(T_{i+1})$ at $S(T_{i+1})$ using target actor network: $A(T_{i+1})\leftarrow \Mask(S(T_{i+1}), \pi'(S(T_{i+1}) | \zeta_a'))$}  
%             %using Algorithm~\ref{alg: mask}  \tcp{Add action masking}

%             Set target $y_i\leftarrow R(T_j) + \gamma Q'(S(T_{i+1}), A(T_{i+1})
%             | \zeta_c')$ 
            
%             Update critic network by minimizing the loss: $L 
%             \leftarrow \frac{1}{N} \sum_i (y_i - Q(S(T_i), A(T_i) | \zeta_c))^2$ 
            
%              {\color{black} Get masked actions $A(T_i)$ at $S(T_i)$ using actor network: $A(T_i)\leftarrow \Mask( S(T_i), \pi(S(T_i) | \zeta_a))$ }%using Algorithm~\ref{alg: mask}.  
%              % \tcp{Add action masking}
%            % Mask action $A(T_i)\leftarrow \Mask(S(T_i), a_{i})$ 
           
%             Update the actor policy by policy gradient: 
%             $\nabla_{\zeta_a} J \leftarrow \frac{1}{N} \sum_i \nabla_a Q(S(T_i), A(T_i) | \zeta_c) | \nabla_{\zeta_a} \pi(s | \zeta_a) |_{S(T_i)}$
            
%             Delayed Update the target networks: 
%             $\zeta_a' \leftarrow \tau \zeta_a + (1 - \tau) \zeta_a'$; 
%             $\zeta_c' \leftarrow \tau \zeta_c + (1 - \tau) \zeta_c'$ \\
%         }
%     }
%     \caption{Improved DDPG with Action Masking and Policy Guidance.}
% \label{alg:DDPG} 
% \end{algorithm}

%%%%%%%%%%%% ACTION MASKING %%%%%%%%%%%%
\begin{algorithm}[ht]
    \SetAlgoNlRelativeSize{-1}
    \KwIn{$\textit{state}, \textit{action: } A(T_j) $}
    \KwOut{Masked action: $\MaskAction$} 
\small
$\PowerNeed \gets [\PowerNeed(C_i, T_j)]_{C_i \in \mathbf{C}};$
$\ReTime \gets [\ReTime(\phi(C_i, T_j))]_{C_i \in \mathbf{C}};$
$\epsilon \gets 10^{-5};$
$C^{max} \gets [C^{max}_i]_{C_i \in \mathbf{C}};$
$C^{min} \gets [C^{min}_i]_{C_i \in \mathbf{C}};$

\tcp{Mask 1: Set action = 0 if no car is connected}
    $ \MaskAction \gets \frac{\ReTime}{\ReTime + \epsilon} \times A(T_j)$\;
    
    \tcp{Mask 2: Stop charging when required SoC is reached for uni-directional chargers}
    $ \MaskAction_{tmp} \gets \MaskAction$; 
    $\MaskAction_{tmp} \gets \min(\MaskAction_{tmp}, \frac{\PowerNeed}{\delta})$\;
    $ \MaskAction[\textit{uniIdx}] \gets \MaskAction_{tmp}[\textit{uniIdx}]$\;

    \tcp{Mask 3: Enforce charging to the required SoC before departure. }
    $\overline{\Power(T_j)} \gets \frac{ \PowerNeed- (\ReTime - 1) \times C^{max} \times \delta }{\delta}$\;
    $\overline{\Power(T_j)} \gets \min(\overline{\Power(T_j)}, C^{max})$\;
    $ \MaskAction \gets \max(\MaskAction, \overline{\Power(T_j)})$\;
    \tcp{Mask 4: Ensure bidirectional chargers discharge to the required SoC before departure.}
    $\Power^*(T_j) \gets \frac{\PowerNeed- (\ReTime - 1) \times P_{min} \times \delta }{\delta}$\;
    $\Power^*(T_j) \gets \max(\Power^*_t, P_{min})$\;
    
    $\MaskAction_{tmp}\gets \MaskAction$; 
    $ \MaskAction[\textit{biIdx}] \gets \min(\MaskAction_{tmp}, \Power^*_t)[\textit{biIdx}]$\;

    \tcp{Mask 5: Power improvement strategy}
    $ \textit{powerGap} \gets \Building(T_j) - \PrdPeak(T_j)$\;
    $ \textit{canIncrease} \gets \textit{RELU}\left(\max\left(\frac{\PowerNeed}{\delta}, C^{max}\right) - \MaskAction \right)$\;
    
    $ \textit{toImprove} \gets \min\left(\textit{RELU}(\textit{powerGap} - \sum \MaskAction), \sum \textit{canIncrease}\right)$
    
    $ \MaskAction \gets \MaskAction + \frac{\textit{toImprove} \times \textit{canIncrease}}{\sum(\textit{canIncrease}) + \epsilon}$\;

    \tcp{Mask 6: Do not discharge below building load}
    $ \textit{toImprove} \gets \max(-\Building(T_j) - \sum(\MaskAction), 0)$\;
    $ \textit{negAction} \gets \textit{RELU}(\MaskAction \times -1) \times -1$\;
    
    % $ \textit{toIncrease} \gets \frac{\textit{toImprove} \times \textit{tmpAction}}{\sum(\textit{tmpAction}) + \epsilon}$\;
    $ \MaskAction \gets \MaskAction +  \frac{\textit{toImprove} \times \textit{negAction}}{\sum(\textit{negAction}) + \epsilon}$\;

    \caption{Action Masking: $\Mask(S(T_j), A(T_j))$.} 
    \label{alg: action_masking}
\end{algorithm} 

% Based on the formulated MDP, we next use a RL approach to train a policy that maximizes long-term rewards by interacting with a custom environment simulator, which processes all data samples and handles state transitions.  
% \subsection{Environment Simulator}
% \jpnote{I will add this, just talk about how the transition is represented by the simulator. See Mike's paper.}



\section{Reinforcement Learning Policy}
\label{sec:RL}
% Figure~\ref{fig:framework_and_pipeline}. We input the the training data into our designed environment simulator, which handle the state generation and state transformations based on actions following our state transition function described in Section~\ref{ssec:MDP}.

% Based on the formulated MDP, we discuss the RL-based framework used for policy training to maximize the long-term reward of this MDP, as shown in \Cref{fig:framework}. This paper utilizes the Deep Deterministic Policy Gradient (DDPG) algorithm~\cite{lillicrap2015continuous} for policy training. DDPG is well-suited for handling continuous action spaces and supports off-policy training, enabling the model to learn from diverse experiences across various scenarios, thus improving generalization. 

% To further enhance policy performance, we integrate action masking and policy guidance techniques with DDPG. An overview of the RL-based framework shows that we input the training data into our designed environment simulator, which generates state transitions and transformations based on actions according to the state transition function described in Section~\ref{ssec:MDP}. In traditional DDPG, the actor network generates state and action trajectories through interaction with the environmental simulator, which are then stored in the replay buffer for training the critic and actor networks.  
% In this work, we enhance this process with a policy guidance approach. Instead of solely relying on the actor network for training trajectories, we incorporate a MILP solver to provide optimal actions based on current and future information, guiding the RL training away from local optima. Additionally, we implement an action masking procedure, represented as $\Mask(S(T_j), A(T_j))$, which refines the raw actions from the actor network by considering constraints for action validity and utilizing domain-specific knowledge to limit the exploration range of training actions and improve policy performance.  
% Our advanced DDPG method is described as follows. Algorithm~\ref{alg:DDPG} depicts our customized DDPG approach, incorporating action masking and policy guidance. This approach builds upon the classic DDPG algorithm by introducing policy guidance (lines 5 to 9) and action masking (lines 9, 14, and 17).
%Based on the formulated MDP, we present an RL-based framework for policy training aimed at maximizing long-term rewards, as illustrated in \Cref{fig:framework}. 
During training, data samples are input into our designed environment simulator \ad{as mentioned before we do not describe simulator enough, perhaps more information and statement that we will cite the simulator after blind review}, which provides the environment that abstracts state features for the RL models and manages state transitions based on the function described in Section~\ref{ssec:MDP}. The simulator operates based on actions (power rates) generated by the policies and dynamic events such as EV arrivals and departures. 
Our core approach is based on  Deep Deterministic Policy Gradient (DDPG) algorithm~\cite{lillicrap2015continuous}, which  is well-suited for continuous action spaces and supports off-policy training, allowing the model to learn from diverse experiences across various scenarios, thereby improving generalization. To enhance the policy performance, we integrate action masking and policy guidance techniques with DDPG as outlined in Algorithm~\ref{alg:DDPG}.  

Traditional DDPG relies on the actor network to generate actions (power rates in this work). The tuples of state, action, action reward, and next state are stored as state transitions in the replay buffer (see lines 9 to 12). In each training iteration, we batch state transitions from the replay buffer for model training (see line 13). Specifically, DDPG maintains target networks for both the actor and critic, which are used to generate the next state and compute Q-values essential for calculating the critic loss during training. The critic network is trained using gradient descent by minimizing the mean squared error between predicted Q-values and target Q-values derived from the Bellman equation (see lines 14 to 16). The critic learns Q-values for state-action pairs, which are then used to train the actor network through a policy gradient approach (see lines 17 and 18). These target networks are updated less frequently to stabilize the training process (see lines 18 and 19). 

To address the large state and action spaces in this RL model, we enhance DDPG by integrating an action masking procedure, denoted as $\Mask(S(T_j), A(T_j))$. This procedure refines the raw actions generated by the actor network by enforcing action validity and utilizing domain-specific knowledge, thereby improving policy performance. The action masking is applied after the actor network produces raw actions, acting as an additional layer (see lines 9, 14, 17). 
Additionally, we implement a policy guidance procedure (see lines 5 to 9) by incorporating a MILP solver to provide optimal actions through the function $\MILP(S(T_j), {\it remainEpisode})$, based on current and future information. These optimal actions are stochastically introduced during RL training into the replay buffer, mixing high-quality actions with the raw RL actions to enhance the training transition quality and guide the RL training in a beneficial direction. 

Next, we detail our RL framework by outlining the design of the environment simulator, the data normalization and network structure, the action masking and policy guidance procedures, and the heuristic approaches used to simplify our RL training. 
%which incorporates both policy guidance (lines 5 to 9) and action masking (lines 9, 14, and 17). 


% During training, optimal actions are stochastically incorporated into the process based on a predefined ratio coefficient, denoted as $\policyGuidanceRate$, and stored in the replay buffer. The function $\MILP(S(T_j), {\it remainEpisode})$ takes the current state as input and outputs the next optimal action. It can also access remaining events in the episode, including upcoming EV arrivals, SoC requirements, and building load, to invoke a MILP solver. This solver generates a sequence of optimal actions starting at timestamp $t$, returning the first action in the sequence, which serves as the optimal action for the current state to maximize long-term reward (lines 5 to 9).



\subsection{Actor-Critic Network Structure} 
%\jpnote{Add sentence description of the number of layers we are considering for the Actor Critic models}
To enhance convergence and improve generalization, we preprocess all state variables to be within the range of 0 and 1 before feeding them into neural networks. Timestamps are normalized by dividing the number of minutes in a day, while power-related variables such as building load $\Building(T_j)$, estimated peak power $\PrdPeak(T_j)$ are scaled by their respective statistical factors. Furthermore, we normalize the energy capacity $CAP(V_k)$ of each car by dividing it by the maximum capacity among EVs. 
For the action $A(T_j)=[P(C_i, T_j)]_{C^{i}\in \mathcal{C}}$. We normalize the action values to the range of $[-1, 1]$ based on the power rate range $[C_i^{min}, C_i^{max}]$ by Equation~(\ref{eq: normalize}).
\begin{equation}
    \hat{P}(C_i, T_j)=\frac{2\times(P(C_i, T_j)-C_i^{min})}{C_i^{max} - C_i^{min}}-1. 
\label{eq: normalize}
\end{equation}
% In DDPG, two neural networks are employed: the critic network evaluates state Q-values, while the actor network generates deterministic actions as the policy.
% The critic network processes state and action inputs through hidden layers (2 layers used in final model) with ReLU activation, outputting a single Q-value estimate.
Both critic and actor network have two hidden layers with 96 neurons each. Both feature a ReLU activation layer at the end. The critic network outputs a single Q-value estimate while the actor network outputs the action which represent the power rate of each charger.
% The exact size used in the model are described in~\Cref{tab:hyperparameters}.
% Simultaneously, the actor network maps the input state through multiple hidden layers (2 layers used in final model) with ReLU activations to produce the action output, representing the power rate of each charger.
This output is constrained within the range $[-1, 1]$ using a $\tanh$ activation function. The original charging power values can be obtained by computing the inverse of the normalization equation in Equation~(\ref{eq: normalize}).  \ad{this section needs clear revision. It should just tell how the network is structured and how data is normalized and input into the network. Also, clear description of the size and architecture is required.}
\jpt{I modified the text, it still needs the normalization part.}



\subsection{Action Masking}
\label{sec: DDPG} 

%\jpnote{Move the algorithm ahead of this and summarize your points while referring to the algorithm. You can add label commands to the lines of the algorithm to refer to it. Also rename your heuristic and constraints in the algorithm to just action masking N}

We propose action masking approach addresses the challenge of large continuous action spaces (detailed in Algorithm~\ref{alg: action_masking}), by ensuring that the policy actions generated by the actor network are valid and reasonable during DDPG training. This technique, based on findings from \cite{huang2020closer,kanervisto2020action}, confirms that differentiable action masking does not interfere with the policy gradient backpropagation process. As a result, the learning process remains effective, while the imposed constraints on the action space prevent the actor network from exploring invalid actions, thereby improving training efficiency and optimizing resource usage.

This procedure takes the RL raw action $A(T_j)$, an array of power rates $[P(C_i, T_j)]_{C_i\in\mathcal{C}}$ for all chargers, processes it through the following steps, and outputs the masked actions $A'$. Before computing, we first obtain the state features formatted as arrays: the remaining power needed to reach the required SoC for all connected EVs ($\PowerNeed$), their remaining time ($\ReTime$), and the maximum ($C^{\max}$) and minimum ($C^{\min}$) power rates of all chargers (see line 1). Also, we denote ${\it uniIdx}$ and ${\it biIdx}$ as the indices for unidirectional and bidirectional chargers, respectively.    
% \jpnote{Shorten the descriptions and make it very concise.} 
\begin{itemize}[leftmargin=*]
    \item \textbf{Mask 1.} 
    %We masks the action by setting the powe rate $P(C_i, T_j)$ of charger $C_i$ to 0 if no EV is plugged in. This method effectively masks all actions for idle chargers (where $\ReTime(\phi(C_i, T_j))=0$ indicates no EV is connected to charger $C_i$ at time slot $T_j$).
    We set the power rate $P(C_i, T_j)$ of charger $C_i$ to 0 if no EV is plugged in, effectively masking actions for idle chargers (where $\ReTime(\phi(C_i, T_j))=0$).
    %%%
    \item \textbf{Mask 2.} Overcharging unidirectional chargers is unbeneficial since excess energy cannot be discharged. Thus we limit the power rates to ensure the SoC of unidirectional chargers remains within required SoCs. 
    %by masking the actions using the minimum of the current power rate and the rate required for EVs to reach the required SoC after this time slot, calculated as $\PowerNeed/\delta$. 
    Actions are masked to the minimum of the current power rate and the rate needed for EVs to reach required SoC, calculated as $\PowerNeed/\delta$ (see lines 3 and 4).
    % The rationale behind this approach is that overcharging unidirectional chargers offers no benefit, as the excess energy cannot be discharged 
    %%%
    \item \textbf{Mask 3.} 
    %Action are masked to minimize the missing SoC for all EVs at the time of departure, as outlined in objective function~(\ref{eq: soc_penalty}).
    Actions are adjusted to forced charging to required SoC before departure if necessary, to minimize missing SoC as per objective function~(\ref{eq: soc_penalty}).  
    We compute the critical power rate $\overline{\Power^*(T_j)}$, representing the minimum required power rate for all chargers at time $T_j$ to their required SoC before departing (assuming maximum power is subsequently utilized). The raw action is adjusted if it falls below this rate, especially in final time slots (see lines 5 to 7).
    %At each time slot, we compute the critical power rate $\overline{\Power^*(T_j)}$, which represents the minimum power rate required for all chargers at time $T_j$ to ensure that all connected EVs reach the required SoC before departing (assuming maximum power is subsequently utilized). The raw action is adjusted if it falls below this critical power rate. 
    
    \item \textbf{Mask 4.} 
    Overcharging bidirectional EVs is only advantageous if excess energy can be discharged during peak hours; thus, there’s no benefit to overcharging just before departure. Thus, we adjust actions to prevent discharging EVs if the raw action would cause exceeding the required SoC before departure.  Here, $\Power^*(T_j)$ denotes the minimum power rate needed for all chargers $C_i \in \mathcal{C}$ at time $T_j$ (see lines 8 to 10).
    %We mask action to discharge EVs to reduce their SoC if the raw action would result in exceeding the required SoC before departure. The intuition behind this is that overcharging EVs in bidirectional chargers is advantageous only when the excess energy can be discharged during peak hours to mitigate demand charges; thus, there is no benefit to overcharging them just before departure. Here, $\Power^*(T_j)$ denotes the minimum power rate required for all chargers $C_i \in \mathcal{C}$ at time $T_j$ to ensure that all EVs connected to bidirectional chargers can discharge to the required SoC before leaving (see lines 8 to 10). 
    
    %%%
    \item \textbf{Mask 5.} 
    %We mask action to increase charging power rates while ensuring that the masked action remains within the estimated peak power limit. The goal is to encourage charging to the required SoC without raising the demand charge, thereby avoiding forced charging just before departure, which could elevate peak power. In this step, we calculate the power gap between the estimated peak power and the current building load, represented as $\PrdPeak(T_j) - \Building(T_j)$. If the current sum of action power is below this gap, we utilize the available power gap to enhance the current action by allocating the additional power needed. This allocation is constrained by the maximum power rate each entity can increase and the power required to reach the required SoC of the connected EVs, computed using ${\it canIncrease}$ (see lines 11 to 14). 
    We increase charging power rates while ensuring the masked action stays within the estimated peak power limit. This aims to charge EVs to their required SoC without raising demand charges, thereby avoiding forced charging just before departure, which could elevate peak power. 
    We calculate the power gap between estimated peak power and current building load, $\PrdPeak(T_j) - \Building(T_j)$. If the current power sum is below this gap, we enhance the action using the available power gap, constrained by the maximum rate each EV can increase to reach the required SoC, computed using ${\it canIncrease}$ (see lines 11 to 14). 
    %constrained by the maximum power increase and the required SoC of connected EVs (see lines 11 to 14).
    
    %%%
    \item \textbf{Mask 6} We adjust the discharging power rate to prevent discharging below the current building load $\Building(T_j)$ by increasing negative actions based on their current values (see lines 15 to 17). 
\end{itemize} 

All above action masking procedures utilize array computations and differentiable operations, such as ReLU \cite{rasamoelina2020review} and maximum/minimum operations, from the PyTorch library \cite{paszke2017automatic}. 







 % We then employ \textbf{Heuristic 2}, which ensures that all EVs' charging needs are met by enforcing minimum power rates $P(C_i, T_j)$ when the remaining time is insufficient to reach the required SoC ${\it SoC}^{\text{req}}(v)$ using the maximum power rate.

% We then employ \textbf{Heuristic 2}, which ensures that all EVs reach their required SoC at departure. This is achieved by enforcing maximum power rates \( P(C_i, T_j) \) when the remaining time is insufficient to reach the required SoC $ {\it SoC}^{\text{req}}(v) $ using the maximum power rate. 
% This step ensures that all chargers charge the EVs to the required SoC before departure, if feasible. Additionally, this forced charging is applied only in the final time slots when it is required.

%We denote ${\it uniIdx}$ and ${\it biIdx}$ as the indices of the uni-directional and bi-directional chargers, respectively.   
%We introduce \textbf{Heuristic 3}, which bounds discharging to allow EVs to reduce their SoC if it exceeds the required levels. The intuition is that there is no benefit to overcharging in bidirectional chargers before departure.






\subsection{MILP Policy Guidance} 

To address the challenge of local optima in DDPG, we  integrate a policy guidance approach~\cite{pmlr-v28-levine13} into the RL training process. This aims to improve performance by providing optimal actions to guide the training toward better outcomes. We implement an optimization framework to generate optimal actions and add them to the replay buffer, providing effective guidance to steer the search towards global optima. Specifically, we formulate the V2B problem using mixed-integer linear programming (MILP). We give the MILP solver the current state information, including the current EV status, charge usage, and all future events from the input sample (EV arrival/departure, building load flow, and electricity prices from the current time to the end of the billing period. The MILP solution provides the power rate for each charger from the current time to the end of the billing period, maximizing the multi-objective weighted sum of total cost (detailed in Equation~(\ref{eq: billing})) and penalties for missing SoC requirements (defined in Equation~(\ref{eq: soc_penalty})). Following multiple constraints related to the EV SoC update  function following Equation~(\ref{eq: soc}) and Constraints~(\ref{eq:charging_rate}) to (\ref{eq:building_power}).   

% During training, optimal actions are stochastically incorporated into state transitions process based on a predefined ratio coefficient, denoted as $\policyGuidanceRate$, and stored in the replay buffer (as shown in Algorithm~\ref{alg: DDPG}. The function $\MILP(S(T_j), {\it remainEpisode})$ takes the current state as input and outputs the next optimal action. It can also access remaining events in the episode, including upcoming EV arrivals, SoC requirements, and building load, to invoke a MILP solver. This solver generates a sequence of optimal actions starting at timestamp $t$, returning the first action in the sequence, which serves as the optimal action for the current state to maximize long-term reward (lines 5 to 9). 
During training, optimal actions are stochastically incorporated into the state transition process based on a predefined coefficient, denoted as $\policyGuidanceRate$, and stored in the replay buffer (see Algorithm~\ref{alg:DDPG}). The function $\MILP(S(T_j), {\it remainEpisode})$ takes the current state as input and outputs the next optimal action. It considers remaining events in the episode, such as upcoming EV arrivals, SoC requirements, and building load, to invoke a MILP solver. This solver generates a sequence of optimal actions starting at time slot $T_j$ and returns the optimal action for the next time slot, serving to maximize the long-term reward. 

%SoC requirement,
% which ensure EVs must reach their required SoC before departure.
% \[
% SoC_{T_d(v)}(v) \geq So
% \]   
% Using the MILP solution, we can generate actions based on any state to maximize the long-term reward. 
% {\color{black} 
% The MILP aims to minimize the total energy cost, including both consumption and demand charges, while ensuring that all EVs meet their SoC requirements before departure. The objective function is:
% \[
% \min \Cost^{\it EN}(\mathcal{P}) + Cost^{\it DC}(\mathcal{P})
% \]
% as shown in Equations~\ref{eq: objective_1, eq: objective_2}. 
% where \( g_t \) represents the energy usage cost at time slot \( t \in \mathcal{T} \), \( P_{\text{max}} \) is the maximum power consumed during peak hours.  
% The model includes binary variables for charger assignments (\( a_{v,cp,s} \)), and continuous variables for charging rates (\( c^v_s \)) and battery levels (\( e^v_s \)). Key constraints ensure that: 
% \begin{enumerate}
% \item \textbf{Charger Assignments}: Each EV \( v \) is assigned to at most one charger \( cp \) at any time slot \( s \), and charger capacities are not exceeded, and is denoted by the assignment variable $a \in \mathcal{A}$.
% \[
%  \sum_{C_i \in \mathcal{C}} a_{v,i,s} \leq 1
% \]
        
% \item \textbf{SoC Requirements}: EVs must reach their required SoC before departure.
% \[
% SoC_{T_d(v)}(v) \geq SoC^(v)
% \]  
% \item \textbf{Energy Balance}: The energy in the EV battery evolves according to the charging rate, where $C_i$ is the charger attached to car $v$ indicated by Equation~(\ref{eq: SoC}). 
%     % \[
%     % SoC_{t+1}(v) = SoC_{t}(v) + P(C_i, T_j)
%     % \]
% \item \textbf{Demand Charges}: Demand charges are incorporated by modeling the maximum power consumption, as indicated by Equation~\eqref{eq: objective_2}.
% \end{enumerate}
% By solving this MILP using current state information and future events (e.g., EV arrivals/departures, building load, electricity prices), we generate optimal action sequences that maximize the long-term reward. These optimal actions are then stochastically introduced into the replay buffer during DDPG training, effectively guiding the RL agent towards global optima and improving training outcomes.
% } 
% The final DDPG approach, incorporating action masking and policy guidance, is depicted in Algorithm~\ref{alg: DDPG}. 
% {\color{black} This approach builds upon the classic DDPG algorithm by introducing policy guidance (lines 5 to 9) and action masking (lines 9, 14, and 17). 
 %This procedure ensures valid and efficient action selection, leading to improved training outcomes in DDPG. 
%The lines in {\color{black}black} highlight the enhancements over the classic DDPG. These colored the sections involve policy guidance and action masking. During DDPG training, optimal actions are stochastically introduced based on a predefined ratio coefficient $\policyGuidanceRate$ and stored in the replay buffer. We define the $\MILP(S(T_j))$ function to invoke the MILP solver and generate a sequential optimal action sequence starting at timestamp $t$, returning the first generated action which represents the optimal action for the current state (see lines 5 to 9). This procedure ensures both valid and efficient action selection, leading to improved training outcomes in DDPG.  


% {\color{red} Add text connecting two subsections. }
% \begin{itemize}[leftmargin=*]
%     \item {\bf Least Laxity First (LLF)}: Least Laxity First is a dynaimc priority driven algorithm designed for scheduling multiprocessor real time tasks~\cite{leung1989new}. Laxity or slack time refers to the amount of time a task can be delayed without causing it to miss its deadline. In the context of EV charging, we define laxity as the difference between the amount of time remaining for a car before it departs and the amont of time it takes to meet the user's required SoC at a constant rate of charge~\cite{xu2016dynamic}. Additionally, we limit the amount of cars charging at any given time interval by allocating a capacity at each step. The capacity is based on the difference between a set a power threshold and the current building load at that time. 
%     Only cars connected to chargers whose aggregate power rates fall within this limit are able to be charged for that time interval. LLF will provide trickle charge, charging the cars to the minimum required power to reach required SoC before departure, at each interval. 
% \end{itemize} 
\subsection{Heuristic Approach and Post Processing}
% \jpnote{Clarify what this section is, is this in the figure 2A?}
% We observe that off-peak hours typically feature lower electricity prices, allowing for charging EVs at a higher power rate without impacting the final demand charge. This insight leads us to maximize EV charging during off-peak hours, thereby reducing energy costs and alleviating pressure on charging during peak hours. To implement this strategy, we employ a greedy approach that directs the charging of all EVs to their required SoC at the maximum power rate. Additionally, discharging to the required SoC is performed when entering the off-peak duration after peak hours.
% We observe that off-peak hours typically offer lower electricity prices, enabling EV charging at a higher power rate without affecting the final demand charge. This insight allows us to maximize EV charging during off-peak hours, reducing energy costs and alleviating pressure on charging during peak periods. To implement this strategy, we adopt a greedy approach to manage off-peak charging, replacing the RL models. This approach directs the charging of all EVs until they reach their required SoC at the maximum power rate during off-peak hours. Additionally, discharging occurs when entering off-peak periods after peak hours if the current SoC of the EV exceeds the required level.  
% Additionally, we apply action post-processing to keep EV SoCs within valid boundaries by adjusting the power rate for stopping charging or discharging when they exceed $SoC^{\text{max}}(\phi(C_i, T_j))$ or drop below $SoC^{\text{min}}(\phi(C_i, T_j))$.  
We observe that off-peak hours offer lower electricity prices, enabling EV charging at higher power rates without affecting the final demand charge. This helps optimize EV charging during these periods, reducing energy costs and alleviating pressure during peak times. To implement this approach, we utilize a simple greedy method during off-peak hours, foregoing RL training during these times. EVs charge at maximum power until they reach their required SoC. Specifically, assuming the current time slot $T_j$ is within non-peak hours and EV $V' = \phi(C_i, T_j)$ is connected to charger $C_i$, if $\SOCR(V') < \SOC(V', T_j)$, then $P(C_i, T_j) = \min(C^{max}_i, (\SOCR(V') - \SOC(V', T_j)) \times {CAP}(V') / \delta)$. Discharging occurs if the SoC exceeds the required level, calculated as $P(C_i, T_j) = \max(C^{min}_i, (\SOCR(V') - \SOC(V', T_j)) \times {CAP}(V') / \delta)$.


To condense the state features while accounting for the common maximum and minimum SoC boundaries of all EVs (with $\SOCMIN=0$ and $\SOCMAX=90\%$), we do not include SoC boundaries in the state representation, limiting the policy's direct access to them for action control. To maintain valid SoC boundaries, we apply a post-processing procedure, which differs from action masking that is integrated with the actor network and influenced by training backpropagation. This action post-processing adjusts policy-generated actions before they are input to the environment, ensuring that charging stops when SoC exceeds $SoC^{\text{max}}$ and discharging halts if it drops below $SoC^{\text{min}}$. 
By employing this approach, we ensure that all policy-generated power rates for charging EVs remain within the defined SoC boundaries, thereby satisfying Constraints~(\ref{eq:soc_min}) and~(\ref{eq:soc_max}).
%This method allows the RL policy to concentrate on the objective of total bill reduction without needing to learn the validity of actions concerning SoC limits, thereby enhancing the effectiveness of the training process. 


\begin{table}
        \renewcommand{\arraystretch}{0.8}
        \begin{threeparttable}
        \begin{tabular}{@{}lcc@{}}
        \toprule
        Parameter & Floating Allegro Hand & Bimanual Robot Arms \\
        \midrule
        % \makecell{Initial object translational \\ perturbation (cm) }& [$\pm1.5$, $\pm1.5$, 0] & [$\pm 5$, $\pm 5$, 0]\\
        % \makecell{Initial object rotational \\ perturbation (rad) }& [0, 0, $\pm 0.3$] & [0, 0, $\pm 0.3$]\\
        Init. obj. trans.  pert. (cm) & [$\pm1.5$, $\pm1.5$, 0] & [$\pm 5$, $\pm 5$, 0]\\
        Init. obj. rot. pert. (rad) & [0, 0, $\pm 0.3$] & [0, 0, $\pm 0.3$]\\
        Object side length (cm) & [5.8, 6.2]  & [28, 32] \\
        Object mass (kg) & [0.1, 0.3]  & [0.25, 0.75]  \\
        Friction coefficients & [0.7, 1.3] &  [0.2, 0.4]  \\
        Task horizon (s) & 25 & 50 / 260  (Panda / iiwa) \\
        \bottomrule
        \end{tabular}
        \end{threeparttable}
        \caption{Ranges of different physical parameters $\theta$. The initial object pose is only perturbed in yaw, x, and y to ensure the object sits stably on the table. }
        \label{tab:domain_randomization}
        \vspace{0.5em}
\end{table}

\begin{figure*}[t]
\centering
\includegraphics[width=1.0\textwidth]{figures/trajopt_unittest.png}
	\caption{\textbf{Trajectory optimization is crucial for generating dynamically feasible trajectories}. (Top) Before trajectory optimization, the kinematically retargeted demos easily lose contact and drive the object out of reach with different physical parameters or slight deviations in object states. (Bottom) Trajectory optimization encourages robots to establish contact with and maintain good manipulability of the object. The tricolor axis indicates the object orientation.}
	\label{fig:trajopt_unittest}
\end{figure*}

\section{Trajectory Optimization Experiments}

While kinematic retargeting of demonstrations might suffice to generate data for simpler manipulation tasks such as pick and place, it often falls short for the more challenging contact-rich tasks requiring frequent contact mode switches and fine-grained actions. In this section, we demonstrate that trajectory optimization is crucial for generating diverse, dynamically feasible contact-rich trajectories on three high-dimensional dexterous manipulation systems: a floating Allegro hand, bimanual iiwa arms, and bimanual Panda arms.

Our data generation framework is agnostic to the choice of the trajectory optimizer. We implement 
% a contact-implicit model predictive controller based on smoothed contact dynamics \cite{suh2024dexterous} and
the cross-entropy method (CEM) \cite{de2005tutorial} to solve \eqref{eq:predictive_control} over a distribution of physical parameters and initial conditions, as specified in Table \ref{tab:domain_randomization}. 
%\russtcomment{Right... the SQP discussion tricked me, but I guess that's only for the retargeting. I thought you had replaced this. In this case, your approach is almost doing RL, but on a policy parameterized as a trajectory... right? why is that better than doing PPO on a small neural net policy, and generating data from that? If you stick with CEM, than this will be your burden of proof, i think?}

\underline{\textbf{Task}} Manipulating the object to a target pose on the table (Fig. \ref{fig:policy_rollouts}). The object is initially placed randomly on the table with an arbitrary face upward. Task success is defined as the object reaching within 3 cm and 0.2 rad of the target pose for the Allegro hand, and within 10 cm and 0.2 rad for the bimanual robot arms.  This task requires long-horizon reasoning of complex multi-contact interactions between the robot and the object. The necessary frequent contact mode switches and high-dimensional action space pose great challenges for traditional model-based planners, while the precise contact interactions require fine-grained control actions. 

\begin{table}
\centering
        \renewcommand{\arraystretch}{0.8}
        \begin{threeparttable}
        \begin{tabular}{@{}lcccc@{}}
        \toprule
        Perturbation & Allegro Hand & iiwa Arms & Panda Arms \\
        \midrule
        Original demo &4 / 24 & 5 / 24 & 6 / 24\\
        Object size & 2 / 24 & 1 / 24 & 4 / 24\\
        % Object mass & 1 / 24& 1 / 24 & \\
        % Friction coefficients & 3 / 24 & 2 / 24 & \\
        Initial object translation & 1 / 24 & 3 / 24 & 2 / 24\\
        Initial object orientation & 2 / 24 & 3 / 24& 3 / 24\\
        \midrule
        Trajectory optimization & 2164 / 3000 & 2252 / 3000 & 2462 / 3000 \\
        \bottomrule
        \end{tabular}
        \end{threeparttable}
        \caption{Success rates of replaying kinematically retargeted trajectories of the 24 original human demos, and trajectory optimization under random perturbations in physical parameters and object initial conditions. }
        \label{tab:kin_success_rate}
\end{table}

% \begin{table}
% \centering
%         \renewcommand{\arraystretch}{0.8}
%         \begin{threeparttable}
%         \begin{tabular}{@{}ccc@{}}
%         \toprule
%         Allegro Hand & iiwa Arms & Panda Arms \\
%         \midrule
%         0.721 & 0.65 & 0.803\\
%         % Task Horizon (s) & 25 & 280 & 50 \\
%         \bottomrule
%         \end{tabular}
%         % \begin{tablenotes}
%         % \itme{*} 
%         % \end{tablenotes}
%         \end{threeparttable}
%         \caption{Success rates of trajectory optimization under random perturbations in physical parameters and object initial conditions. }
%         \label{tab:trajopt_success_rate}
%         \vspace{0.5em}
% \end{table}



% \begin{figure*}[t]
% \centering
% \includegraphics[width=0.7\textwidth]{figures/aug_traj_den.png}
% 	\caption{\textbf{Distribution of object trajectories generated from a single demonstration}. The original demonstration (orange) is locally perturbed and augmented to about 100 dynamically feasible contact-rich trajectories (blue) for each system. The density map represents the object pose distribution of the generated trajectories in the specific 2-dimensional slices.}
%     \label{fig:aug_data_distribution}
% \end{figure*}

% \begin{figure*}[t]
% \centering
% \includegraphics[width=0.9\textwidth]{figures/aug_traj_snapshots.png}
% 	\caption{\textbf{Snapshots of trajectories generated from a single demonstration}. The original demonstration (orange) is locally perturbed and augmented to about 100 dynamically feasible contact-rich trajectories (blue) for each system. The density map represents the object pose distribution of the generated trajectories in the specific 2-dimensional slices.}
%     \label{fig:aug_data_distribution}
% \end{figure*}
\begin{figure*}[t]
\centering
\includegraphics[width=1.0\textwidth]{figures/density_snapshots_aug_traj.png}
	\caption{\textbf{Distribution and snapshots of trajectories generated from a single demonstration.} (a) The original demonstration (orange) is locally perturbed and augmented to about 100 dynamically feasible contact-rich trajectories (blue) for each system. The density map represents the object pose distribution of the generated trajectories in the specific 2-dimensional slices. (b) Snapshots of 30 dynamically feasible trajectories under random physical parameters and object initial poses for bimanual iiwa arms are visualized.}
    \label{fig:aug_data_distribution}
\end{figure*}


\underline{\textbf{Dynamic Feasibility}}
While kinematic motion retargeting can generate visually plausible robot and object trajectories, these trajectories often lack dynamical consistency due to the differences in physical parameters and embodiment between the human demonstrator and the target robot. To illustrate this, we replay the kinematically retargeted trajectories of the original 24 human demos and record the success rates for each system in Table \ref{tab:kin_success_rate}. Furthermore, we randomly sample object sizes and perturbations of initial object poses according to Table \ref{tab:domain_randomization} and roll out the nominal kinematically retargeted trajectories. Some trajectories still succeed under certain perturbations thanks to caging grasps or other strategies that encourage robustness during the human demonstration. For all the systems, the successful rollouts are relatively short, manipulating the object to the goal pose within only 1 or 2 rotations. 
% Notably, the successful trajectories for the iiwa and Panda arms vary significantly, despite being generated from the same initial set of demonstrations.

The low success rate of purely kinematically retargeted trajectories highlights the importance of trajectory optimization for locally refining the demos for the particular embodiments and physical parameters. Before trajectory optimization, the floating Allegro hand lightly touches the cube and easily loses contact when rotating it clockwise (demonstrated in Fig. \ref{fig:trajopt_unittest}a). After trajectory optimization, the hand increases the contact area, establishing a stable grip for rotation. In Fig. \ref{fig:trajopt_unittest}b, similar behavior that encourages contact can be observed for the bimanual iiwa arms: the demo trajectory tries to rotate the box clockwise only using a single arm, while trajectory optimization encourages the other arm to help hold the box and reorient the box more stably. These refinements that encourage contact are particularly helpful when the object is heavier or smaller, or when the friction coefficients are lower than expected. In addition, replaying the kinematically retargeted trajectory often fails when the object pose deviates slightly from the demonstration, driving the object out of reach (visualized in Fig. \ref{fig:trajopt_unittest}c). In contrast, trajectory optimization 
%stabilizes the system in a vicinity around the demonstration, ensuring higher success rates even when the object is perturbed
accounts for the system’s true dynamics and can adjust the robot’s actions accordingly. The success rates of trajectory optimization under random perturbations in physical parameters and object initial conditions for each system are recorded in Table \ref{tab:kin_success_rate}.

\begin{figure*}[t]
    \centering
    \includegraphics[width=0.9\linewidth]{figures/policy_rollouts.png}
    \caption{\textbf{Policy rollouts for different embodiments.} The object manipulation task requires the robots to frequently make and break contact with the object. It also requires precise control of the robot since small deviations in positions can result in missing contact interactions and lead to task failure. } 
    \label{fig:policy_rollouts}
\end{figure*}

\underline{\textbf{Cross-Embodiment Generalization}} We demonstrate that a single set of human demonstrations can be effectively repurposed to generate dynamically consistent, contact-rich trajectories across different robotic embodiments with varying task horizons. Specifically, human demonstrations involving two index fingers manipulating a small cube are retargeted to fixed-base bimanual Kuka LBR iiwa and Franka Emika Panda arms manipulating a larger box (visualized in Fig. \ref{fig:kinematic_retargeting}). This approach addresses key challenges in data collection for contact-rich tasks: directly teleoperating two real robot arms to flip a large box would be both physically demanding and cost-prohibitive due to hardware latency, limited feedback, and the embodiment gap--differences in kinematic structure, degrees of freedom, and workspace between human and robotic arms. In contrast, performing the same task on a smaller scale using human fingers is more intuitive, reduces physical effort, and enables faster, more consistent demonstration collection.

The iiwa and Panda arms differ in contact geometry, velocity limits, and joint constraints, all of which are explicitly modeled within the trajectory optimization framework described in \eqref{eq:predictive_control}. For safe hardware deployment, we enforce conservative velocity limits on the iiwa arms, while only applying soft velocity regularization on the Panda arms in simulation to allow for more aggressive motions.


\underline{\textbf{Data Diversity}} 
Trajectory optimization efficiently augments a single demonstration to a wide distribution of trajectories with locally perturbed physical parameters and initial conditions as visualized in Fig. \ref{fig:aug_data_distribution}. The diverse states in the generated dataset cover a larger training distribution and encourage smoother learned policies, as will be discussed in the next section.
\begin{figure*}[t]
    \centering
    \includegraphics[width=0.9\linewidth]{figures/policy_failure.png}
    \caption{\textbf{Failure cases of baselines.} (a) The baseline policy trained on the original 24 demonstrations for the floating Allegro hand frequently misses contact or gets stuck on the cube. (b-c) The baseline policies for the bimanual robot arms often exhibit jittery motion, resulting in loss of contact, the box being kicked out of reach, or the robot arms running into and getting stuck on the box surface. } 
    \label{fig:policy_failure}
\end{figure*}

\begin{figure}
\centering
\includegraphics[width=0.42\textwidth]{figures/success_rate.png}
	\caption{Success rates of policy evaluation in simulation and hardware. }
	\label{fig:success_rate}
    \vspace*{-0.4cm}
\end{figure}

% \begin{figure}
% \centering
% \includegraphics[width=0.48\textwidth]{figures/jitteriness.png}
% 	\caption{\textbf{Joint angles of bimanual iiwa arms over time. } Each line represents the trajectory of a different joint of the iiwa arms. The policy trained on augmented datasets (b) demonstrates significantly smoother motion compared to the baseline policy (a). }
% 	\label{fig:jitteriness}
% \end{figure}

\begin{figure*}[t]
    \centering
    \includegraphics[width=0.9\linewidth]{figures/hardware_rollout.png}
    \caption{\textbf{Policy rollouts on hardware.} The fixed-base bimanual iiwa arms perform a sequence of coordinated rolling, pitching, and yawing actions to reorient the box to the goal pose. } 
    \label{fig:hardware_rollout}
\end{figure*}

\section{Behavior Cloning Experiments}
We illustrate our framework's capability to efficiently produce diverse, high-quality contact-rich datasets for training behavior cloning policies across multiple robotic platforms, including the floating Allegro hand and the bimanual Panda arms in simulation as well as bimanual iiwa arms on hardware. We show that policies trained on the generated data generalize to a wide distribution of physical parameters and initial conditions, and are much more robust and performant than the ones trained only on the original demonstrations. 
\subsection{Policy Evaluation in Simulation}
\label{subsec:policy_eval_sim}
From only 24 human demonstrations, our data generation pipeline can efficiently generate thousands of dynamically feasible contact-rich trajectories using trajectory optimization. We train state-based diffusion policies \cite{chi2023diffusion} on the 24 original demo trajectories, as well as 500 and 1000 generated trajectories. While our method is compatible with any Behavior Cloning algorithm, we adopt diffusion policies due to its recent success in contact-rich tasks \cite{chi2024universal, zhu2024should, li2024planning}. Fig. \ref{fig:policy_rollouts} visualizes the policy rollouts. We evaluate the performance by conducting 48 policy rollouts for each embodiment in simulation and record the success rates in Fig. \ref{fig:success_rate}. The success criteria are the same as specified in the trajectory optimization experiments.
%For policy evaluation, we visualize the initial states for all evaluation episodes, typical failure cases of baseline policies, and final object pose errors in Fig. \ref{fig:policy_eval}.  
% and validate that the generated data help improve the policy's robustness and generalizability.

\subsubsection{Floating Allegro Hand} 
While the human demonstrator completes the task in approximately 5 seconds on average in the virtual reality environment, the demonstration trajectories are temporally scaled by a factor of 2.5 to ensure smoother, dynamically feasible motions on the floating Allegro hand, which is subject to velocity limits. We define the task horizon as 25 seconds to allow the policy sufficient time to recover from missed contacts and other errors during the execution. The task complexity arises from the 22-dimensional action space of the Allegro hand and the long-horizon nature of the task, which requires a sequence of coordinated rolling, pitching, and yawing actions to reorient the cube to an upright position. These factors together present significant challenges for traditional model-based planners without guidance.

The baseline behavior cloning policy trained on the original set of 24 demonstrations achieves a success rate of $10 / 48 = 21\%$ and exhibits significant jittery behavior when encountering out-of-distribution states. The workspace, characterized by diverse object orientations and translations, is sufficiently large that minor deviations during policy rollouts often drive the trajectory out of the demonstrated distribution. Common failure modes include the Allegro hand repeatedly missing contact with the cube or becoming stuck on its surface while attempting reorientation (visualized in Fig. \ref{fig:policy_failure}a), which often result in the object being trapped in intermediate orientations. In contrast, policies trained on the expanded dataset generated by our pipeline demonstrate a higher likelihood of re-establishing contact with the object after initial misses, resulting in significantly improved success rates up to $39 / 48 = 81\%$.

\begin{figure*}[t]
    \centering
    \includegraphics[width=0.9\linewidth]{figures/hardware_eval.png}
    \caption{\textbf{Policy failure and recovery on hardware.} The baseline policy frequently (a) gets stuck on the box surface when small deviations from the demonstration trajectories occur, and (b) struggles to recover from out-of-distribution states, where the object is never intentionally lifted for accomplishing the task in the generated dataset. Policies trained on augmented datasets (c) sometimes fail due to unmodeled collision geometry, but (d) can recover from undesired sliding by employing firmer grasps found by trajectory optimization. } 
    \label{fig:hardware_eval}
\end{figure*}
\subsubsection{Bimanual Robot Arms}
The baseline policy trained on the original set of 24 human demonstrations achieves a success rate of $27 / 48 = 56\%$ on the bimanual iiwa system. We hypothesize that the restrictive velocity limits encourage more quasi-static behavior, leading to longer trajectories with a higher density of state-action pairs in the training data. In contrast, the baseline policy yields a success rate of $14/48=29\%$ on the bimanual Panda system, likely due to the more dynamic nature of the learned behavior under its looser velocity constraints. Both baseline policies exhibit remarkably jittery motion, frequently kicking the box out of reach, losing contact, or running into and getting stuck on the box surface during reorientation (visualized in Fig. \ref{fig:policy_failure}b and c). Policies trained on the augmented dataset, however, generate significantly smoother trajectories and are capable of re-establishing contact with the object after initial misses, resulting in as high as $44 / 48 = 92\%$ success rates for bimanual iiwa arms and $42 / 48 = 87.5\%$ for bimanual Panda arms. Additionally, the learned policies capture multimodal behaviors observed in the original human demonstrations, such as rotating the box either clockwise or counterclockwise for similar object poses. 


\subsection{Policy Evaluation on Hardware}
We zero-shot deploy the trained policies on hardware for bimanual iiwa arms to flip a 30 cm cubic box on a table (Fig. \ref{fig:hardware_rollout}). An OptiTrack motion capture system is employed to estimate the object pose. The baseline behavior cloning policy only achieves $6/23=26\%$ success rate, with most successful rollouts being relatively short-horizon, involving only 1 or 2 rotations. Common failure modes of the baseline policy include: 1) deviation from the demonstration trajectory, causing the arms to collide with the box surface (Fig. \ref{fig:hardware_eval}a), and 2) significant box sliding during rolling, resulting in the policy encountering out-of-distribution states and failing to recover (Fig. \ref{fig:hardware_eval}b). In contrast, as shown in Fig. \ref{fig:success_rate}b, the policy trained on 500 generated trajectories achieves $17 / 23 = 74\%$ success rate, while the policy trained on 1000 generated trajectories achieves $16/23=70\%$ success rate. Despite occasional box sliding during rolling, these policies demonstrate an improved ability to stabilize the box by using one arm to hold the opposite side more firmly to prevent further sliding (Fig \ref{fig:hardware_eval}d). However, as visualized in Fig \ref{fig:hardware_eval}c, both policies trained on the augmented datasets exhibit failure modes originating from unmodeled collision geometries on iiwa arms, which lead to significant undesired yaw motions of the box during pitch actions.\looseness=-1
\section{Conclusion and future work}
In this study, we examined the ability of LLMs to produce self-generated counterfactual explanations (SCEs).
We design a prompt-based setup for evaluating the efficacy of \SCEs.
Our results show that LLMs consistently struggle with generating valid \SCEs. In many cases model prediction on a \SCE does not yield the same target prediction for which the model crafted the \SCE.
Surprisingly, we find that LLMs put significant emphasis on the context---the prediction on \SCE is significantly impacted by the presence of original prediction and instructions for generating the \SCE.
Based on this empirical evidence, we argue that LLMs are still far from being able to explain their own predictions counterfactually.
Our findings add to similar insights from recent studies on other forms of self-explanations~\cite{lanham2023measuring,tanneru2024quantifying}.



Our work opens several avenues for future work. Inspired by counterfactual data augmentation~\cite{sachdeva2023catfood}, one could include the counterfactual explanation capabilities a part of the LLM training process. This inclusion may enhance the counterfactual reasoning capabilities of the LLM. Follow ups should also explore the effect of prompt tuning, specifically, model-tailored prompts for generating \SCEs. These approaches might lead to better quality \SCEs.


We limited our investigation to open source models of upto 70B parameters. Extending our analysis to larger and more recent models, \eg, DeepSeek R1 671B, and closed source models like OpenAI o3 would be an interesting avenue for future work.

Finally, our experiments were limited to relatively simple tasks: classification and mathematics problems where the solution is an integer. This limitation was mainly due to the fact that it is difficult to automatically judge validity of answers for more open-ended language generation tasks like search and information retrieval. Scaling our analysis to such tasks would require significant human-annotation resources, and is an important direction for future investigations.

\section{Acknowledgement}

This material is based upon work sponsored by the National Science Foundation (NSF) under Award Numbers 1952011 and 2238815 and by Nissan Advanced Technology Center-Silicon Valley. Results presented in this paper were obtained using the Chameleon Testbed supported by the NSF. Any opinions, findings, conclusions, or recommendations expressed in this material are those of the authors and do not necessarily reflect the views of the NSF or Nissan. 
% \begin{acks}
% If you wish to include any acknowledgments in your paper (e.g., to 
% people or funding agencies), please do so using the `\texttt{acks}' 
% environment. Note that the text of your acknowledgments will be omitted
% if you compile your document with the `\texttt{anonymous}' option.
% \end{acks}

%%%%%%%%%%%%%%%%%%%%%%%%%%%%%%%%%%%%%%%%%%%%%%%%%%%%%%%%%%%%%%%%%%%%%%%%

%%% The next two lines define, first, the bibliography style to be 
%%% applied, and, second, the bibliography file to be used.

% \clearpage
\bibliographystyle{ACM-Reference-Format} 
\balance

\bibliography{main}
% \vfill\eject
% \bibliographystyle{plain}
% \bibliographystyle{unsrtnat} 
% \bibliographystyle{ACM-Reference-Format} 
 
\clearpage
\appendix
% \clearpage
\newpage
\appendix
\onecolumn

\renewcommand{\thetable}{A\arabic{table}} % Prefix table numbers with 'A'
\renewcommand{\thefigure}{A\arabic{figure}} % Prefix figure numbers with 'A'
\renewcommand{\theequation}{A\arabic{equation}} % Prefix equation numbers with 'A'

\setcounter{table}{0} % Reset table counter
\setcounter{figure}{0} % Reset figure counter
\setcounter{equation}{0} % Reset equation counter

\section*{Appendix}

\section{Optimal Brain Surgeon Derivation}
\label{OBS_ALGORITHM}

In the original setup in OBS, we have a local quadratic model for the loss $L$ given by:
$$
    \delta L = L(w + \delta w) \approx L(w) + \nabla_w L^T \delta w + \frac{1}{2} \delta w^T H \delta w
$$
Since OBS is a pruning-after-training approach, they discarded the 1-st order component. Reducing the expression for saliency as:
$$
    \delta L = \frac{1}{2} \delta w^T H \delta w
$$
To remove a single parameter, the authors of OBS introduced the constraint $e_q^T \delta w + w_q = 0$, with $e_q$ being the $q^{\text{th}}$ canonical basis vector. The pruning is defined as a constrained optimization problem of the form:
$$
    \min_{\delta w \in \mathbb{R^d}} \left( \frac{1}{2} \delta w^T H \delta w\right),
    ~~\text{s.t}~~
    e_q^T \delta w + w_q = 0.
$$
And the choice of which parameter to remove becomes:
$$
    \min_{q \in \mathcal{Q}} \left\{
        \min_{\delta w \in \mathbb{R^d}} \left( \frac{1}{2} \delta w^T H \delta w\right),
        ~~\text{s.t}~~
        e_q^T \delta w + w_q = 0
    \right\}.
$$
To solve the internal problem, we use a Lagrange multiplier $\lambda$ to write the problem as an unconstrained optimization case as follows:
$$
    \mathcal{L}(\delta w, \lambda) =
    \frac{1}{2} \delta w^T H \delta w +
    \lambda(e_q^T \delta w + w_q).
$$
Then, to find the stationary conditions, we compute the partial derivatives with respect to $\delta w$ and $\lambda$, and equate them to 0, obtaining:
$$
    \nabla_{\delta w} \mathcal{L} = 
    H \delta w + \lambda e_q = 0 
    \rightarrow
    \delta w = - \lambda H^{-1} e_q
$$
$$
    \nabla_{\lambda} \mathcal{L} =
    e_q^T \delta w + w_q = 0
    \rightarrow
    e_q^T \delta w = -w_q
$$
With some replacements, we get:
$$
    e_q^T \delta w = -w_q
    \rightarrow
    e_q^T \left( 
        - \lambda H^{-1} e_q
    \right) = -w_q
    \rightarrow
    - \lambda e_q^T H^{-1} e_q = -w_q
    \rightarrow
    \lambda = \frac{w_q}{e_q^T H^{-1} e_q} = \frac{w_q}{[H^{-1}]_{qq}}
$$
$$
    \delta w = - \frac{w_q H^{-1} e_q}{[H^{-1}]_{qq}}
$$
Replacing the expression for $\delta w$ in the saliency expression, we have:
\begin{align*}
    \delta L = \frac{1}{2} \delta w^T H \delta w
    &= \frac{1}{2}\left(
        - \frac{w_q H^{-1} e_q}{[H^{-1}]_{qq}}
    \right)^T
    H
    \left(
        - \frac{w_q H^{-1} e_q}{[H^{-1}]_{qq}}
    \right)
    \nonumber \\
    &= 
    \frac{w_q^2}{2[H^{-1}]_{qq}^2}
    \left(
        H^{-1} e_q
    \right)^T
    H
    \left(
        H^{-1} e_q
    \right)
    \nonumber \\
    &= 
    \frac{w_q^2}{2[H^{-1}]_{qq}^2}
    e_q ^T
    H^{-1}
    e_q
    = 
    \frac{w_q^2}{2[H^{-1}]_{qq}^2}
    [H^{-1}]_{qq}
    = 
    \frac{w_q^2}{2[H^{-1}]_{qq}}
    \nonumber \\
\end{align*}
%------------------------------------------------------------------------------------------------
\newpage
\section{Fisher Brain Surgeon Sensitivity Derivation}
\label{FBSS_ALGORITHM}
As we considered a PBT setting, it is not possible to ignore the first-order term in the local quadratic approximation of the error as it could still be informative. In this case, our model for sensitivity is given by: 
$$
    \delta L = \nabla_w L^T \delta w + \frac{1}{2} \delta w^T H \delta w
$$
The process to remove a single parameter remains similar; the constraint $e_q^T \delta w + w_q = 0$, with $e_q$ is still valid, redefining the optimization problem as:
$$
    \min_{\delta w \in \mathbb{R^d}} \left(
        \nabla_w L^T \delta w +  \frac{1}{2} \delta w^T H \delta w
    \right),
    ~~\text{s.t}~~
    e_q^T \delta w + w_q = 0.
$$
And the choice of which parameter to remove becomes:
$$
    \min_{q \in \mathcal{Q}} \left\{
        \min_{\delta w \in \mathbb{R^d}} \left(
            \nabla_w L^T \delta w + \frac{1}{2} \delta w^T H \delta w
        \right),
        ~~\text{s.t}~~
        e_q^T \delta w + w_q = 0
    \right\}.
$$
Using a Lagrange multiplier $\lambda$ as in the reference case, we solve the following unconstrained optimization problem:
$$
    \mathcal{L}(\delta w, \lambda) =
    \nabla_w L^T \delta w + 
    \frac{1}{2} \delta w^T H \delta w +
    \lambda(e_q^T \delta w + w_q).
$$
With the following stationary conditions:
$$
    \nabla_{\delta w} \mathcal{L} = 
    \nabla_w L + H \delta w + \lambda e_q = 0 
    \rightarrow
    \delta w = - (\lambda H^{-1}e_q + H^{-1} \nabla_w L)
$$
$$
    \nabla_{\lambda} \mathcal{L} =
    e_q^T \delta w + w_q = 0
    \rightarrow
    e_q^T \delta w = -w_q
$$
The expression for $\lambda$ is redefined as follows:
\begin{align*}
    e_q^T \left(
        - (\lambda H^{-1}e_q + H^{-1} \nabla_w L)
    \right) 
    &= -w_q
    \nonumber \\
    \lambda e_q^T H^{-1} e_q + e_q^T H^{-1} \nabla_w L
    &= w_q
    \nonumber \\
    \lambda [H^{-1}]_{qq} 
    &= w_q - e_q^T H^{-1} \nabla_w L
    \nonumber \\
    \lambda
    &= \frac{w_q - e_q^T H^{-1} \nabla_w L}{[H^{-1}]_{qq}}
\end{align*}
Replacing the expression for $\delta w$ in our sensitivity expression, we have:
\begin{align*}
    \delta L = \nabla_w L^T \delta w + \frac{1}{2} \delta w^T H \delta w
    &= 
    \nabla_w L^T \left[
        - (\lambda H^{-1}e_q + H^{-1} \nabla_w L)
    \right]
    \nonumber \\
    &+
    \frac{1}{2}\left[
        - (\lambda H^{-1}e_q + H^{-1} \nabla_w L)
    \right]^T
    H
    \left[
        - (\lambda H^{-1}e_q + H^{-1} \nabla_w L)
    \right]
    \nonumber \\
    &= 
    - \lambda \nabla_w L^T H^{-1}e_q - \nabla_w L^T H^{-1} \nabla_w L
    \nonumber \\
    &+
    \frac{1}{2}\left[
        (\lambda H^{-1}e_q)^T + (H^{-1} \nabla_w L)^T
    \right]
    \left[
        \lambda H H^{-1}e_q + H H^{-1} \nabla_w L)
    \right]
    \nonumber \\
    &= 
    - \lambda \nabla_w L^T H^{-1}e_q - \nabla_w L^T H^{-1} \nabla_w L
    \nonumber \\
    &+
    \frac{1}{2}\left[
        (\lambda H^{-1}e_q)^T + (H^{-1} \nabla_w L)^T
    \right]
    \left[
        \lambda e_q + \nabla_w L
    \right]
    \nonumber \\
    &= 
    - \lambda \nabla_w L^T H^{-1}e_q - \nabla_w L^T H^{-1} \nabla_w L
    \nonumber \\
    &+
    \frac{1}{2}\left[
        (\lambda H^{-1}e_q)^T \lambda e_q
        + (H^{-1} \nabla_w L)^T \lambda e_q
        + (\lambda H^{-1}e_q)^T \nabla_w L
        + (H^{-1} \nabla_w L)^T \nabla_w L
    \right]
    \nonumber \\
    &= 
    - \lambda \nabla_w L^T H^{-1}e_q - \nabla_w L^T H^{-1} \nabla_w L
    \nonumber \\
    &+
    \frac{1}{2}\left[
        \lambda^2 e_q^T H^{-1} e_q
        + \lambda \nabla_w L^T H^{-1} e_q
        + \lambda e_q^T H^{-1} \nabla_w L
        + \nabla_w L^T H^{-1} \nabla_w L
    \right]
    \nonumber \\
    &= 
    \frac{1}{2}\left[
        \lambda^2 [H^{-1}]_{qq}
        - \lambda \nabla_w L^T H^{-1} e_q
        + \lambda e_q^T H^{-1} \nabla_w L
        - \nabla_w L^T H^{-1} \nabla_w L
    \right]
    \nonumber \\
\end{align*}
Finally, replacing the $\lambda$:
\begin{align*}
    \delta L 
    &= 
    \frac{1}{2}\left[
        \lambda^2 [H^{-1}]_{qq}
        - \lambda \nabla_w L^T H^{-1} e_q
        + \lambda e_q^T H^{-1} \nabla_w L
        - \nabla_w L^T H^{-1} \nabla_w L
    \right]
    \nonumber \\
    &= 
    \frac{1}{2[H^{-1}]_{qq}}\left[
        (w_q - e_q^T H^{-1} \nabla_w L)^2 
        + (w_q - e_q^T H^{-1} \nabla_w L)(e_q^T H^{-1} \nabla_w L - \nabla_w L^T H^{-1} e_q)
        - \nabla_w L^T H^{-1} \nabla_w L
    \right]
    \nonumber \\
    &= 
    \frac{1}{2[H^{-1}]_{qq}}[
        w_q^2
        - 2 w_q (e_q^T H^{-1} \nabla_w L)
        + (e_q^T H^{-1} \nabla_w L)^2
        + w_q (e_q^T H^{-1} \nabla_w L)
    \nonumber \\
        &- w_q (\nabla_w L^T H^{-1} e_q)
        - (e_q^T H^{-1} \nabla_w L)(e_q^T H^{-1} \nabla_w L)
        + (e_q^T H^{-1} \nabla_w L)(\nabla_w L^T H^{-1} e_q)
        - \nabla_w L^T H^{-1} \nabla_w L
    ]
    \nonumber \\
    &= 
    \frac{1}{2[H^{-1}]_{qq}}[
        w_q^2
        - w_q (e_q^T H^{-1} \nabla_w L)
        + (e_q^T H^{-1} \nabla_w L)^2
    \nonumber \\
        &- w_q (\nabla_w L^T H^{-1} e_q)
        - (e_q^T H^{-1} \nabla_w L)^2
        + (e_q^T H^{-1} \nabla_w L)(\nabla_w L^T H^{-1} e_q)
        - \nabla_w L^T H^{-1} \nabla_w L
    ]
    \nonumber \\
    &= 
    \frac{1}{2[H^{-1}]_{qq}}\left[
        w_q^2
        - 2 w_q (e_q^T H^{-1} \nabla_w L)
        + (e_q^T H^{-1} \nabla_w L)^2
        - \nabla_w L^T H^{-1} \nabla_w L
    \right]
    \nonumber \\
    &= 
    \frac{1}{2[\hat{F}^{-1}]_{qq}}
    \left[
        w_q - (e_q^T \hat{F}^{-1} \nabla \mathcal{L}(w_0))
    \right]^2
\end{align*}

%------------------------------------------------------------------------------------------------

\newpage
\section{Training and Testing Details}
\label{appendix:training_parameters}

We perform an 80:20 stratified split, with a constant seed, on the CIFAR10/100 training dataset to obtain a validation set with the same class distribution. For both datasets, we have a training set with 40,000 samples, a validation set with 10,000 samples, and a testing set of 10,000 samples. Validation is performed after each training step, and the weights of the best-performing validation step (based on top-1 accuracy) are utilized for the final evaluation on the testing set. Table \ref{tab:table_training_parameters} summarizes the training parameters.

\begin{table}[h]
\caption{Training parameters used for ResNet18 and VGG19 on the CIFAR-10/100 datasets.}
\label{tab:table_training_parameters}
\vskip 0.15in
\begin{center}
\begin{small}
\begin{sc}
\begin{tabular}{lcc}
\toprule
Parameter & ResNet18 & VGG19 \\
\midrule
Number of steps       & 160 & 160 \\
Criterion             & CE & CE \\
Optimizer             & SGD & SGD \\
Learning rate         & 0.01 & 0.1 \\
Momentum              & 0.9 & 0.9 \\
Weight decay          & $5 \times 10^{-4}$ & $1 \times 10^{-4}$ \\
Learning rate drops   & [60, 120] & [60, 120] \\
Learning rate drop factor & 0.2 & 0.1 \\
\bottomrule
\end{tabular}
\end{sc}
\end{small}
\end{center}
\vskip -0.1in
\end{table}

%------------------------------------------------------------------------------------------------

\newpage
\section{Results CIFAR10}
\subsection{ResNet18}
\label{appendix:CIFAR10_ResNet18}

\begin{table}[h]
\caption{Performance of different sensitivity methods for pruning evaluated using ResNet18 on the CIFAR-10 testset. The right side of the table presents our proposed criteria. The mean accuracy and standard deviation are reported across three initialization seeds for various sparsity levels. Baseline, no pruning: $91.78 \pm 0.09$.}
\label{tab:resnet18_cifar10_compressors}
\vskip 0.15in
\begin{center}
\begin{small}
\begin{sc}
\resizebox{\textwidth}{!}{%
\begin{tabular}{lccccc|cccc}
\toprule
Sparsity  & Random & Magnitude & GN & SNIP & GraSP & FD & FP & FTS & FBSS \\
\midrule
0.10  & 91.71 ± 0.21 & 91.72 ± 0.07 & 91.57 ± 0.15 & 91.72 ± 0.07 & 89.16 ± 0.05 & 91.87 ± 0.13 & 91.63 ± 0.21 & 91.53 ± 0.12 & 91.76 ± 0.08 \\
0.20  & 91.63 ± 0.11 & 91.42 ± 0.12 & 91.51 ± 0.09 & 91.64 ± 0.16 & 88.69 ± 0.34 & 91.50 ± 0.12 & 91.65 ± 0.14 & 91.53 ± 0.15 & 91.54 ± 0.13 \\
0.30  & 91.45 ± 0.18 & 91.61 ± 0.13 & 91.68 ± 0.20 & 91.65 ± 0.08 & 88.67 ± 0.26 & 91.65 ± 0.18 & 91.44 ± 0.27 & 91.49 ± 0.05 & 91.62 ± 0.07 \\
0.40  & 91.59 ± 0.18 & 91.06 ± 0.16 & 91.61 ± 0.09 & 91.55 ± 0.08 & 88.24 ± 0.33 & 91.51 ± 0.05 & 91.38 ± 0.13 & 91.56 ± 0.28 & 91.39 ± 0.05 \\
0.50  & 91.60 ± 0.06 & 91.32 ± 0.13 & 91.44 ± 0.13 & 91.22 ± 0.07 & 87.69 ± 0.15 & 91.30 ± 0.18 & 91.58 ± 0.16 & 91.46 ± 0.19 & 91.41 ± 0.05 \\
0.60  & 91.10 ± 0.16 & 91.18 ± 0.16 & 91.59 ± 0.13 & 91.24 ± 0.04 & 87.48 ± 0.55 & 91.34 ± 0.07 & 91.35 ± 0.16 & 91.40 ± 0.11 & 91.38 ± 0.18 \\
0.70  & 91.17 ± 0.04 & 91.07 ± 0.07 & 91.19 ± 0.17 & 91.33 ± 0.18 & 87.26 ± 0.34 & 91.34 ± 0.23 & 91.42 ± 0.23 & 91.18 ± 0.18 & 91.27 ± 0.14 \\
0.80  & 90.78 ± 0.08 & 91.10 ± 0.12 & 90.95 ± 0.35 & 90.74 ± 0.10 & 87.18 ± 0.51 & 90.95 ± 0.11 & 91.08 ± 0.06 & 90.94 ± 0.22 & 90.73 ± 0.33 \\
0.90  & 89.35 ± 0.13 & 89.88 ± 0.28 & 90.39 ± 0.23 & 90.36 ± 0.34 & 86.60 ± 0.51 & 90.04 ± 0.21 & 90.20 ± 0.08 & 90.55 ± 0.23 & 89.22 ± 0.30 \\
0.95  & 87.59 ± 0.11 & 89.23 ± 0.19 & 89.00 ± 0.05 & 89.31 ± 0.17 & 86.50 ± 0.05 & 88.61 ± 0.28 & 89.50 ± 0.18 & 89.47 ± 0.32 & 87.58 ± 0.25 \\
0.98  & 83.47 ± 0.20 & 85.70 ± 0.33 & 86.43 ± 0.05 & 87.26 ± 0.28 & 85.99 ± 0.08 & 85.61 ± 0.20 & 86.97 ± 0.22 & 87.24 ± 0.32 & 83.40 ± 0.74 \\
0.99  & 78.28 ± 0.45 & 71.99 ± 0.28 & 83.47 ± 0.15 & 84.54 ± 0.04 & 84.56 ± 0.46 & 82.13 ± 0.28 & 83.74 ± 0.48 & 84.85 ± 0.18 & 77.60 ± 1.02 \\
\bottomrule
\end{tabular}}
\end{sc}
\end{small}
\end{center}
\vskip -0.1in
\end{table}

%------------------------------------------------------------------------------------------------
\clearpage
\subsection{VGG19}
\label{appendix:CIFAR10_VGG19}

As discussed earlier, introducing a warm-up phase effectively mitigates layer collapse in data-dependent pruning methods. Here, we evaluate the impact of different warm-up durations by comparing no warm-up, a single warm-up epoch, and five warm-up epochs. Table \ref{tab:VGG19_cifar10_compressors} demonstrates how performance drastically degrades with increasing sparsity, ultimately leading to layer collapse at 0.90 sparsity. However, as shown in the results, a single warm-up epoch is sufficient to prevent collapse and stabilize pruning performance. Moreover, as seen in Table \ref{tab:VGG19_cifar10_compressors_warmup5}, increasing the warm-up period to five epochs provides no substantial additional improvement. This indicates that prolonged warm-up training is not necessary; a single training step is enough to achieve gradient stabilization and overcome layer collapse.

\begin{table}[h]
\caption{Performance of different sensitivity methods for pruning evaluated using VGG19 on the CIFAR-10 test set. The right side of the table presents our proposed criteria. The mean accuracy and standard deviation are reported across three initialization seeds for various sparsity levels. Baseline, no pruning: $89.21 \pm 0.22$.}
\label{tab:VGG19_cifar10_compressors}
\vskip 0.15in
\begin{center}
\begin{small}
\begin{sc}
\resizebox{\textwidth}{!}{%
\begin{tabular}{lccccc|cccc}
\toprule
Sparsity  & Random & Magnitude & GN & SNIP & GraSP & FD & FP & FTS & FBSS \\
\midrule
0.10  & 88.40 ± 0.95 & 89.12 ± 0.55 & 90.14 ± 0.10 & 90.16 ± 0.18 & 87.81 ± 1.66 & 90.20 ± 0.29 & 90.21 ± 0.37 & 90.25 ± 0.38 & 89.06 ± 0.75 \\
0.20  & 89.19 ± 0.22 & 89.65 ± 0.60 & 89.59 ± 0.69 & 90.06 ± 0.04 & 89.57 ± 0.34 & 89.91 ± 0.28 & 90.28 ± 0.55 & 89.80 ± 0.28 & 88.89 ± 0.76 \\
0.30  & 88.93 ± 0.83 & 88.77 ± 1.07 & 90.23 ± 0.09 & 89.88 ± 0.59 & 89.14 ± 0.19 & 90.25 ± 0.09 & 89.97 ± 0.26 & 90.46 ± 0.41 & 89.06 ± 0.36 \\
0.40  & 88.28 ± 1.08 & 89.38 ± 0.53 & 90.50 ± 0.23 & 89.79 ± 0.67 & 88.20 ± 0.31 & 90.51 ± 0.12 & 90.37 ± 0.24 & 90.23 ± 0.14 & 10.00 ± 0.00 \\
0.50  & 88.96 ± 0.82 & 89.03 ± 0.59 & 90.46 ± 0.60 & 90.38 ± 0.25 & 88.67 ± 0.23 & 89.54 ± 0.86 & 90.47 ± 0.52 & 90.19 ± 0.31 & 10.00 ± 0.00 \\
0.60  & 88.15 ± 0.68 & 89.47 ± 0.18 & 89.95 ± 0.30 & 90.32 ± 0.25 & 88.82 ± 0.32 & 90.02 ± 0.40 & 90.18 ± 0.33 & 90.14 ± 0.36 & 10.00 ± 0.00 \\
0.70  & 88.02 ± 0.53 & 89.63 ± 0.44 & 89.69 ± 0.42 & 89.23 ± 0.19 & 89.62 ± 0.81 & 89.85 ± 0.08 & 90.01 ± 0.34 & 10.00 ± 0.00 & 10.00 ± 0.00 \\
0.80  & 88.28 ± 0.34 & 89.62 ± 0.91 & 85.72 ± 0.63 & 89.39 ± 0.43 & 88.82 ± 0.14 & 10.00 ± 0.00 & 88.29 ± 0.11 & 10.00 ± 0.00 & 10.00 ± 0.00 \\
0.90  & 85.82 ± 0.19 & 89.29 ± 0.79 & 10.00 ± 0.00 & 80.85 ± 0.62 & 24.28 ± 20.2 & 10.00 ± 0.00 & 10.00 ± 0.00 & 10.00 ± 0.00 & 10.00 ± 0.00 \\
0.95  & 84.41 ± 0.05 & 10.00 ± 0.00 & 10.00 ± 0.00 & 10.00 ± 0.00 & 10.00 ± 0.00 & 10.00 ± 0.00 & 10.00 ± 0.00 & 10.00 ± 0.00 & 10.00 ± 0.00 \\
0.98  & 80.04 ± 0.90 & 10.00 ± 0.00 & 10.00 ± 0.00 & 10.00 ± 0.00 & 10.00 ± 0.00 & 10.00 ± 0.00 & 10.00 ± 0.00 & 10.00 ± 0.00 & 10.00 ± 0.00 \\
0.99  & 76.89 ± 0.26 & 10.00 ± 0.00 & 10.00 ± 0.00 & 10.00 ± 0.00 & 10.00 ± 0.00 & 10.00 ± 0.00 & 10.00 ± 0.00 & 10.00 ± 0.00 & 10.00 ± 0.00 \\
\bottomrule
\end{tabular}}
\end{sc}
\end{small}
\end{center}
\vskip -0.1in
\end{table}
\newpage
%------------------------------------------------------------------------------------------------
\begin{table*}[h]
\caption{Performance of different compression methods evaluated after 1 warmup epoch using VGG19 on the CIFAR-10 dataset. We report the mean accuracy between three initialization seeds across various sparsity levels. Baseline, no pruning: $89.21 \pm 0.22$.}
\label{tab:VGG19_cifar10_compressors_warmup1}
\vskip 0.15in
\begin{center}
\begin{small}
\begin{sc}
\resizebox{\textwidth}{!}{%
\begin{tabular}{lccccc|cccc}
\toprule
Sparsity  & Random & Magnitude & GN & SNIP & GraSP & FD & FP & FTS & FBSS \\
\midrule
0.80  & 88.73 ± 0.38 & 88.35 ± 0.54 & 86.76 ± 0.27 & 87.39 ± 0.66 & 87.24 ± 0.25 & 87.14 ± 0.45 & 87.00 ± 0.87 & 87.68 ± 0.33 & 64.33 ± 15.91 \\
0.90  & 87.26 ± 0.42 & 88.62 ± 0.49 & 85.96 ± 0.75 & 86.75 ± 0.76 & 87.47 ± 0.33 & 86.69 ± 0.72 & 87.09 ± 0.31 & 87.42 ± 0.21 & 46.16 ± 7.62 \\
0.95  & 85.47 ± 0.64 & 87.68 ± 0.49 & 86.66 ± 0.27 & 86.00 ± 1.10 & 86.71 ± 1.24 & 85.71 ± 1.35 & 86.73 ± 0.36 & 87.56 ± 0.62 & 46.30 ± 5.32 \\
0.98  & 80.44 ± 0.30 & 86.61 ± 0.62 & 84.72 ± 1.69 & 87.22 ± 0.23 & 86.45 ± 0.64 & 80.34 ± 6.43 & 86.07 ± 0.39 & 86.36 ± 0.29 & 49.05 ± 4.31 \\
0.99  & 77.24 ± 0.73 & 83.69 ± 1.36 & 80.28 ± 2.04 & 83.49 ± 1.77 & 85.39 ± 0.43 & 75.11 ± 7.80 & 84.40 ± 1.27 & 85.35 ± 1.05 & 47.10 ± 4.41 \\
\bottomrule
\end{tabular}}
\end{sc}
\end{small}
\end{center}
\vskip -0.1in
\end{table*} 
%------------------------------------------------------------------------------------------------

\begin{table}[h]
\caption{Performance of different sensitivity methods for pruning evaluated after 5 warmup epochs using VGG19 on the CIFAR-10 testset. The right side of the table presents our proposed criteria. The mean accuracy and standard deviation are reported across three initialization seeds for various sparsity levels. Baseline, no pruning: $89.21 \pm 0.22$.}
\label{tab:VGG19_cifar10_compressors_warmup5}
\vskip 0.15in
\begin{center}
\begin{small}
\begin{sc}
\resizebox{\textwidth}{!}{%
\begin{tabular}{lccccc|cccc}
\toprule
Sparsity  & Random & Magnitude & GN & SNIP & GraSP & FD & FP & FTS & FBSS \\
\midrule
0.80  & 88.84 ± 0.43 & 88.41 ± 0.47 & 87.58 ± 0.52 & 88.15 ± 1.09 & 86.77 ± 1.14 & 87.28 ± 0.90 & 88.22 ± 0.82 & 86.68 ± 0.61 & 70.52 ± 9.25 \\
0.90  & 87.56 ± 0.62 & 88.60 ± 0.93 & 86.73 ± 0.37 & 87.89 ± 0.25 & 87.10 ± 0.47 & 87.50 ± 1.42 & 88.18 ± 0.47 & 86.98 ± 0.14 & 47.78 ± 1.26 \\
0.95 & 85.51 ± 0.69 & 87.66 ± 1.19 & 87.44 ± 0.46 & 87.71 ± 0.82 & 87.05 ± 0.16 & 86.83 ± 1.47 & 87.36 ± 0.52 & 87.00 ± 0.74 & 48.83 ± 2.52 \\
0.98 & 82.09 ± 0.17 & 86.24 ± 0.52 & 84.66 ± 1.33 & 86.55 ± 0.84 & 86.04 ± 0.66 & 85.44 ± 0.64 & 86.64 ± 0.13 & 84.89 ± 0.51 & 49.48 ± 0.85 \\
0.99 & 77.22 ± 1.03 & 83.93 ± 1.80 & 81.62 ± 2.17 & 84.53 ± 0.70 & 81.33 ± 5.77 & 81.71 ± 1.41 & 85.02 ± 0.69 & 83.78 ± 0.80 & 41.24 ± 1.55 \\
\bottomrule
\end{tabular}}
\end{sc}
\end{small}
\end{center}
\vskip -0.1in
\end{table}

%------------------------------------------------------------------------------------------------

\newpage
\section{Results CIFAR100}
\subsection{ResNet18}
\label{sec:resnet_cifar-100}

CIFAR-100 results exhibit a similar trend to those observed on CIFAR-10, further reinforcing the robustness of our proposed Fisher-Taylor Sensitivity (FTS) criterion. Across all evaluated sparsity levels, FTS consistently maintains strong performance, frequently ranking among the top-performing methods. This trend is particularly evident at extreme sparsities, where many pruning approaches suffer significant performance degradation. The stability of FTS across both datasets highlights its effectiveness in preserving network expressivity despite aggressive pruning.

\begin{table}[h]
\caption{Performance of different compression methods evaluated using ResNet18 on the CIFAR-100 dataset. We report the mean accuracy between three initialization seeds across various sparsity levels. Baseline, no pruning: $69.57 \pm 0.19$.}
\label{tab:resnet18_cifar100_compressors}
\vskip 0.15in
\begin{center}
\begin{small}
\begin{sc}
\resizebox{\textwidth}{!}{%
\begin{tabular}{lccccc|cccc}
\toprule
Sparsity  & Random & Magnitude & GN & SNIP & GraSP & FD & FP & FTS & FBSS \\
\midrule
0.10  & 69.16 ± 0.11 & 69.37 ± 0.14 & 69.63 ± 0.34 & 69.42 ± 0.07 & 64.26 ± 0.27 & 69.66 ± 0.30 & 69.08 ± 0.21 & 69.16 ± 0.11 & 69.07 ± 0.10 \\
0.20  & 69.16 ± 0.30 & 69.06 ± 0.24 & 69.19 ± 0.11 & 69.30 ± 0.08 & 63.28 ± 0.58 & 69.60 ± 0.30 & 69.35 ± 0.35 & 69.41 ± 0.43 & 69.07 ± 0.20 \\
0.30  & 69.36 ± 0.18 & 68.58 ± 0.36 & 69.37 ± 0.13 & 68.82 ± 0.17 & 62.02 ± 0.43 & 69.24 ± 0.40 & 68.84 ± 0.13 & 68.80 ± 0.55 & 68.96 ± 0.11 \\
0.40  & 69.41 ± 0.20 & 68.50 ± 0.29 & 69.16 ± 0.26 & 68.95 ± 0.19 & 61.18 ± 0.19 & 69.17 ± 0.16 & 68.88 ± 0.25 & 69.02 ± 0.21 & 68.92 ± 0.25 \\
0.50  & 69.12 ± 0.46 & 68.17 ± 0.20 & 68.94 ± 0.20 & 68.63 ± 0.11 & 61.11 ± 0.40 & 69.13 ± 0.13 & 68.68 ± 0.12 & 68.71 ± 0.12 & 68.71 ± 0.57 \\
0.60  & 68.66 ± 0.27 & 67.78 ± 0.35 & 68.77 ± 0.17 & 68.63 ± 0.42 & 61.40 ± 0.78 & 68.34 ± 0.43 & 67.98 ± 0.23 & 68.41 ± 0.14 & 68.60 ± 0.15 \\
0.70  & 67.95 ± 0.43 & 67.51 ± 0.24 & 68.29 ± 0.39 & 68.08 ± 0.18 & 59.43 ± 0.76 & 68.03 ± 0.46 & 67.96 ± 0.15 & 68.29 ± 0.06 & 68.16 ± 0.07 \\
0.80  & 67.26 ± 0.48 & 66.55 ± 0.19 & 67.20 ± 0.37 & 67.21 ± 0.38 & 59.08 ± 0.22 & 66.70 ± 0.05 & 67.05 ± 0.06 & 66.77 ± 0.65 & 66.62 ± 0.43 \\
0.90  & 64.75 ± 0.16 & 64.48 ± 0.18 & 64.87 ± 0.27 & 65.70 ± 0.08 & 59.16 ± 0.91 & 64.74 ± 0.44 & 65.46 ± 0.30 & 65.41 ± 0.13 & 63.90 ± 0.31 \\
0.95  & 61.01 ± 0.32 & 62.20 ± 0.06 & 62.20 ± 0.23 & 63.20 ± 0.20 & 57.91 ± 0.09 & 62.14 ± 0.42 & 63.22 ± 0.25 & 63.21 ± 0.47 & 61.25 ± 0.44 \\
0.98  & 54.72 ± 0.22 & 55.44 ± 0.18 & 57.34 ± 0.31 & 58.83 ± 0.35 & 54.85 ± 0.35 & 55.57 ± 0.17 & 58.05 ± 0.18 & 58.59 ± 0.12 & 55.02 ± 0.34 \\
0.99  & 45.62 ± 0.55 & 40.39 ± 0.36 & 50.46 ± 0.61 & 52.96 ± 0.10 & 49.13 ± 0.19 & 48.02 ± 0.32 & 49.98 ± 0.60 & 52.85 ± 0.24 & 44.91 ± 0.52 \\
\bottomrule
\end{tabular}}
\end{sc}
\end{small}
\end{center}
\vskip -0.1in
\end{table}

%------------------------------------------------------------------------------------------------
\clearpage
\subsection{VGG19}
The results on VGG19 with CIFAR-100 exhibit a similar trend to those observed on CIFAR-10, reinforcing the effectiveness of our proposed approach. Once again, we identify the occurrence of layer collapse at extreme sparsities when no warm-up is applied, leading to a significant drop in accuracy. Introducing a single warm-up epoch effectively resolves this issue, restoring pruning performance across all evaluated criteria. However, increasing the warm-up phase to five epochs does not yield any additional advantage, indicating that a brief warm-up period is sufficient to stabilize gradient-based importance scores and prevent collapse.

\label{sec:vgg_cifar-100}

\begin{table}[h]
\caption{Performance of different compression methods evaluated using VGG19 on the CIFAR-100 dataset. We report the mean accuracy between three initialization seeds across various sparsity levels. Baseline, no pruning: $58.96 \pm 2.30$.}
\label{tab:VGG19_cifar100_compressors}
\vskip 0.15in
\begin{center}
\begin{small}
\begin{sc}
\resizebox{\textwidth}{!}{%
\begin{tabular}{lccccc|cccc}
\toprule
Sparsity & Random & Magnitude & GN & SNIP & GraSP & FD & FP & FTS & FBSS \\
\midrule
0.10  & 60.31 ± 0.40 & 59.13 ± 1.29 & 61.93 ± 0.48 & 61.98 ± 0.29 & 59.32 ± 0.63 & 62.13 ± 0.61 & 60.45 ± 3.47 & 61.56 ± 1.04 & 58.79 ± 0.98 \\
0.20  & 60.43 ± 1.14 & 59.27 ± 0.34 & 62.64 ± 0.21 & 62.68 ± 0.24 & 61.21 ± 0.41 & 63.04 ± 0.43 & 62.71 ± 1.02 & 62.24 ± 0.44 & 60.48 ± 0.48 \\
0.30  & 58.32 ± 0.60 & 59.35 ± 1.43 & 62.61 ± 0.23 & 63.11 ± 0.35 & 59.30 ± 0.43 & 62.85 ± 0.42 & 61.43 ± 0.61 & 62.65 ± 0.54 & 58.77 ± 1.02 \\
0.40  & 56.50 ± 3.20 & 60.04 ± 1.02 & 62.36 ± 0.02 & 62.39 ± 0.55 & 56.34 ± 1.49 & 62.38 ± 0.75 & 61.56 ± 1.25 & 62.67 ± 0.06 & 1.00 ± 0.00 \\
0.50  & 58.47 ± 1.49 & 61.49 ± 1.22 & 62.02 ± 0.64 & 62.76 ± 0.50 & 54.43 ± 0.84 & 62.84 ± 0.33 & 62.25 ± 0.33 & 62.47 ± 0.42 & 1.00 ± 0.00 \\
0.60  & 57.54 ± 0.74 & 61.50 ± 0.30 & 62.55 ± 0.13 & 63.08 ± 0.55 & 56.76 ± 0.69 & 62.40 ± 0.57 & 62.70 ± 0.63 & 62.17 ± 0.23 & 1.00 ± 0.00 \\
0.70  & 57.63 ± 0.80 & 61.71 ± 0.25 & 60.85 ± 0.79 & 60.58 ± 0.39 & 57.76 ± 0.84 & 60.44 ± 0.34 & 60.92 ± 0.41 & 60.51 ± 1.67 & 1.00 ± 0.00 \\
0.80  & 57.84 ± 0.57 & 61.89 ± 1.02 & 55.09 ± 0.49 & 59.84 ± 0.29 & 58.39 ± 0.74 & 1.00 ± 0.00 & 43.16 ± 1.02 & 58.66 ± 2.28 & 1.00 ± 0.00 \\
0.90  & 58.41 ± 0.41 & 62.60 ± 0.91 & 1.00 ± 0.00 & 8.35 ± 10.39 & 42.88 ± 1.64 & 1.00 ± 0.00 & 1.00 ± 0.00 & 8.87 ± 11.13 & 1.00 ± 0.00 \\
0.95  & 54.84 ± 1.08 & 1.00 ± 0.00 & 1.00 ± 0.00 & 1.00 ± 0.00 & 1.00 ± 0.00 & 1.00 ± 0.00 & 1.00 ± 0.00 & 1.00 ± 0.00 & 1.00 ± 0.00 \\
0.98  & 50.21 ± 0.72 & 1.00 ± 0.00 & 1.00 ± 0.00 & 1.00 ± 0.00 & 1.00 ± 0.00 & 1.00 ± 0.00 & 1.00 ± 0.00 & 1.00 ± 0.00 & 1.00 ± 0.00 \\
0.99  & 46.69 ± 0.45 & 1.00 ± 0.00 & 1.00 ± 0.00 & 1.00 ± 0.00 & 1.00 ± 0.00 & 1.00 ± 0.00 & 1.00 ± 0.00 & 1.00 ± 0.00 & 1.00 ± 0.00 \\
\bottomrule
\end{tabular}}
\end{sc}
\end{small}
\end{center}
\vskip -0.1in
\end{table}

%------------------------------------------------------------------------------------------------

\begin{table}[h]
\caption{Performance of different compression methods evaluated after 1 warmup epoch using VGG19 on the CIFAR-100 dataset. We report the mean accuracy between three initialization seeds across various sparsity levels. Baseline, no pruning: $58.96 \pm 2.30$.}
\label{tab:VGG19_cifar100_compressors_warmup1}
\vskip 0.15in
\begin{center}
\begin{small}
\begin{sc}
\resizebox{\textwidth}{!}{%
\begin{tabular}{lccccc|cccc}
\toprule
Sparsity & Random & Magnitude & GN & SNIP & GraSP & FD & FP & FTS & FBSS \\
\midrule
0.80  & 60.39 ± 1.16 & 58.91 ± 0.41 & 52.81 ± 1.32 & 55.62 ± 2.27 & 55.15 ± 2.25 & 56.71 ± 0.31 & 58.03 ± 0.93 & 52.41 ± 3.07 & 52.74 ± 5.16 \\
0.90  & 58.90 ± 0.98 & 60.95 ± 0.81 & 50.56 ± 4.59 & 55.89 ± 2.05 & 56.01 ± 1.58 & 52.07 ± 3.24 & 53.65 ± 0.57 & 52.45 ± 3.75 & 19.65 ± 1.68 \\
0.95  & 56.10 ± 0.85 & 57.64 ± 2.63 & 50.34 ± 1.00 & 53.70 ± 3.60 & 56.16 ± 0.41 & 54.44 ± 1.38 & 53.24 ± 3.54 & 53.56 ± 1.26 & 17.24 ± 0.44 \\
0.98  & 50.97 ± 0.40 & 54.66 ± 2.56 & 43.43 ± 5.32 & 50.19 ± 1.59 & 54.64 ± 1.50 & 42.75 ± 1.91 & 50.59 ± 3.39 & 48.56 ± 5.25 & 16.42 ± 0.64 \\
0.99  & 46.52 ± 0.45 & 43.33 ± 5.83 & 33.90 ± 5.35 & 42.65 ± 5.32 & 45.98 ± 4.48 & 29.67 ± 8.49 & 49.11 ± 3.46 & 48.70 ± 2.59 & 13.25 ± 0.84 \\
\bottomrule
\end{tabular}}
\end{sc}
\end{small}
\end{center}
\vskip -0.1in
\end{table}


%------------------------------------------------------------------------------------------------

\begin{table}[h]
\caption{Performance of different compression methods evaluated after 5 warmup epochs using VGG19 on the CIFAR-100 dataset. We report the mean accuracy between three initialization seeds across various sparsity levels. Baseline, no pruning: $58.96 \pm 2.30$.}
\label{tab:VGG19_cifar100_compressors_warmup5}
\vskip 0.15in
\begin{center}
\begin{small}
\begin{sc}
\resizebox{\textwidth}{!}{%
\begin{tabular}{lccccc|cccc}
\toprule
Sparsity & Random & Magnitude & GN & SNIP & GraSP & FD & FP & FTS & FBSS \\
\midrule
0.80  & 60.41 ± 1.39 & 58.38 ± 0.85 & 60.86 ± 0.79 & 61.63 ± 0.45 & 56.25 ± 0.49 & 59.59 ± 0.76 & 59.37 ± 3.50 & 60.86 ± 0.53 & 46.93 ± 9.04 \\
0.90  & 60.32 ± 0.09 & 57.74 ± 1.64 & 57.77 ± 2.41 & 58.23 ± 4.07 & 56.27 ± 1.02 & 60.19 ± 0.63 & 61.23 ± 0.50 & 60.52 ± 0.37 & 21.66 ± 1.95 \\
0.95 & 57.86 ± 0.53 & 59.55 ± 1.15 & 56.09 ± 0.97 & 58.83 ± 0.65 & 55.26 ± 1.25 & 55.80 ± 2.77 & 59.83 ± 0.94 & 58.52 ± 1.32 & 19.98 ± 2.62 \\
0.98 & 51.75 ± 0.43 & 47.75 ± 7.63 & 52.26 ± 4.06 & 55.27 ± 1.69 & 54.59 ± 0.96 & 49.46 ± 4.98 & 57.40 ± 1.26 & 56.00 ± 1.08 & 17.59 ± 1.36 \\
0.99 & 47.59 ± 0.80 & 42.46 ± 7.95 & 46.58 ± 2.00 & 53.13 ± 0.84 & 53.91 ± 1.53 & 42.87 ± 4.63 & 53.17 ± 1.18 & 53.05 ± 2.14 & 13.92 ± 0.14 \\
\bottomrule
\end{tabular}}
\end{sc}
\end{small}
\end{center}
\vskip -0.1in
\end{table}


%------------------------------------------------------------------------------------------------
\clearpage

\section{Mask Batch Size for Other Sparsities}
The Effect of batch size on pruning performance across different sparsities. 
As sparsity increases, the effect of batch size on pruning performance becomes more pronounced. 
At lower sparsities (0.90, 0.95), the differences across batch sizes are less evident, suggesting that even smaller batches provide a reasonable estimation of parameter importance. However, at extreme sparsities (0.98, 0.99), we observe a clear trend where larger batch sizes consistently lead to better parameter selection, ultimately improving accuracy. This aligns with our hypothesis that larger batches help reduce variance in gradient estimation, leading to more stable and effective pruning decisions. 
\label{batch_size_heatmaps}

\begin{figure}[h]
    \centering
    \includegraphics[width=0.8\linewidth]{imgs/cifar10_resnet18_heatmap_warmup_0.png}
    \caption{Effect of batch size on pruning performance at increasing sparsities.}
    \label{fig:enter-label}
\end{figure}

%------------------------------------------------------------------------------------------------

\clearpage
\section{Comparison of our criteria with magnitude-based pruning}

Figure \ref{fig:our_criterion_vs_magnitude} illustrates the relationship between parameter magnitude and different sensitivity-based pruning metrics. Each point represents a model parameter, with red points indicating the top-ranked parameters selected for retention by each criterion. The green dashed line marks the 99th percentile of parameter magnitudes.

A key observation is that the most effective pruning criteria, such as Fisher-Taylor Sensitivity, tend to retain parameters with a broad range of magnitudes, including many that are relatively small (left of the green line). This shows that the estimated importance does not always prioritize parameters based on their magnitude. 


\begin{figure}[htp]
    \centering
    \includegraphics[width=0.9\linewidth]{imgs/cifar_10_mag_vs_criteria_s_99.png}
    \caption{Our criteria vs. Magnitude parameter selection for 99\% sparsity (ResNet18, CIFAR-10, Seed 0)} 
    \label{fig:our_criterion_vs_magnitude}
\end{figure}

%%%%%%%%%%%%%%%%%%%%%%%%%%%%%%%%%%%%%%%%%%%%%%%%%%%%%%%%%%%%%%%%%%%%%%%%

\end{document}

%%%%%%%%%%%%%%%%%%%%%%%%%%%%%%%%%%%%%%%%%%%%%%%%%%%%%%%%%%%%%%%%%%%%%%%%

