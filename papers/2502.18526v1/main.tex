%%%%%%%%%%%%%%%%%%%%%%%%%%%%%%%%%%%%%%%%%%%%%%%%%%%%%%%%%%%%%%%%%%%%%%%%

%%% LaTeX Template for AAMAS-2025 (based on sample-sigconf.tex)
%%% Prepared by the AAMAS-2025 Program Chairs based on the version from AAMAS-2025. 

%%%%%%%%%%%%%%%%%%%%%%%%%%%%%%%%%%%%%%%%%%%%%%%%%%%%%%%%%%%%%%%%%%%%%%%%

%%% Start your document with the \documentclass command.


%%% == IMPORTANT ==
%%% Use the first variant below for the final paper (including author information).
%%% Use the second variant below to anonymize your submission (no author information shown).
%%% For further information on anonymity and double-blind reviewing, 
%%% please consult the call for paper information
%%% https://aamas2025.org/index.php/conference/calls/submission-instructions-main-technical-track/

%%%% For anonymized submission, use this
\documentclass[sigconf,nonacm]{acmart}

%%%% For camera-ready, use this
%\documentclass[sigconf]{aamas} 
%%% Load required packages here (note that many are included already).
\usepackage{pgfplots}
\usepackage{pgfplotstable}
%%% for tables %%%
\usepackage{siunitx}
\usepackage{booktabs}
%\usepackage{amssymb}
\usepackage{amsmath}
\usepackage{dblfloatfix}
% \usepackage{natbib}
\usepgfplotslibrary{statistics}
\pgfplotsset{compat=newest}
% \usepackage{filecontents}
\usepgfplotslibrary{fillbetween} % Load the fillbetween library
\usepgfplotslibrary{groupplots}
\usepackage{pgfgantt}

%% For alternative author names format
% \usepackage[noblocks]{authblk}

%%% for multi-header table %%%
\usepackage{multirow}
\usepackage{collcell}
\usepackage{colortbl}
\usepackage{xcolor}
\usepackage{datatool}
% \usepackage{natbib} % Include natbib package

% \usepackage{filecontents}  % For embedding the CSV file in the LaTeX document

\usepackage{balance} % for balancing columns on the final page

%%%%%%%%%%%%%%%%%%%% Removing the legend 2 bars for bar plot %%%%%%%%%%%%%%%%%%%%%%%
% \pgfplotsset{compat=newest,
%         /pgfplots/ybar legend/.style={
%         /pgfplots/legend image code/.code={%
%         %\draw[##1,/tikz/.cd,yshift=-0.25em]
%                 %(0cm,0cm) rectangle (3pt,0.8em);},
%         \draw[##1,/tikz/.cd,bar width=3pt,yshift=-0.2em,bar shift=0pt]
%                 plot coordinates {(0cm,0.8em)};},
% },
% }

%%%%%%%%%%%%%%%%%%%%%%%%%%%%%%%%%%%%%%%%%%%%%%%%%%%%%%%%%%%%%%%%%%%%%%%%

%%% AAMAS-2025 copyright block (do not change!)

% \makeatletter
% \gdef\@copyrightpermission{
%   \begin{minipage}{0.2\columnwidth}
%    \href{https://creativecommons.org/licenses/by/4.0/}{\includegraphics[width=0.90\textwidth]{by}}
%   \end{minipage}\hfill
%   \begin{minipage}{0.8\columnwidth}
%    \href{https://creativecommons.org/licenses/by/4.0/}{This work is licensed under a Creative Commons Attribution International 4.0 License.}
%   \end{minipage}
%   \vspace{5pt}
% }
% \makeatother

% \setcopyright{ifaamas}
\setcopyright{none}
% \acmConference[AAMAS '25]{Proc.\@ of the 24th International Conference
% on Autonomous Agents and Multiagent Systems (AAMAS 2025)}{May 19 -- 23, 2025}
% {Detroit, Michigan, USA}{Y.~Vorobeychik, S.~Das, A.~Nowé  (eds.)}
% \copyrightyear{2025}
% \acmYear{2025}
% \acmDOI{}
% \acmPrice{}
% \acmISBN{}

%%%%%%%%%%%%%%%%%%%%%%%%%%%%%%%%%%%%%%%%%%%%%%%%%%%%%%%%%%%%%%%%%%%%%%%%
% \acmSubmissionID{966}

%%%%%%%%%%%%%%%%%%%%%%%%%%%%%%%%%%%%%%%%%%%%%%%%%%%%%%%%%%%%%%%%%%%%%%%%
% FQ add+ for checking comments: 

% \usepackage{xcolor}
\usepackage{enumitem}
\usepackage{subcaption} % for subfigures
\usepackage{graphicx}
\usepackage{hyperref}
\usepackage{placeins}

 %\usepackage[paperwidth=12.5in, paperheight=11in, top=1in, bottom=1in, left=2in, right=2.8in]{geometry} 
\usepackage[textsize=tiny, colorinlistoftodos]{todonotes}
\newcommand{\ava}[1]{\todo[backgroundcolor=teal!20, linecolor=teal!85!black]{\textbf{Ava:} #1}}
\newcommand{\avai}[1]{\todo[backgroundcolor=teal!20, inline=true,linecolor=teal!85!black]{\textbf{Ava:} #1}}
\newcommand{\aaron}[1]{\todo[backgroundcolor=teal!20, linecolor=teal!85!black]{\textbf{Aaron:} #1}}

\newcommand{\ad}[1]{\todo[backgroundcolor=teal!20, linecolor=teal!85!black]{\textbf{AD:} #1}}
\newcommand{\iad}[1]{\todo[backgroundcolor=teal!20, inline=true, linecolor=teal!85!black]{\textbf{AD:} #1}}
\newcommand{\aaroni}[1]{\todo[backgroundcolor=teal!20, inline=true,linecolor=teal!85!black]{\textbf{Aaron: } #1}}

\newcommand\jpnote[1]{\textcolor{red}{#1}}
\newcommand{\jpt}[1]{\todo[backgroundcolor=teal!20, linecolor=teal!85!black]{\textbf{JP:} #1}}
\newcommand{\rishav}[1]{\todo[backgroundcolor=teal!20, linecolor=teal!85!black]{\textbf{RS:} #1}}

\newcommand{\K}{\mathcal{KWH}}
\newcommand{\Power}{\mathcal{KW}}
\newcommand{\PN}{{PN}_t}
% \usepackage[linesnumbered,ruled,noend,vlined]{algorithm2e} 
\usepackage[linesnumbered,ruled,vlined,noend]{algorithm2e}
% \newcommand\mycommfont[1]{\footnotesize\ttfamily\textcolor{blue}{#1}}

\usepackage{cleveref}
% Define a new command for comment font style
\newcommand{\mycommfont}[1]{\textcolor{blue}{#1}}

\SetCommentSty{mycommfont}
\usepackage{amsmath} 
\newcommand\nissan{Nissan}
% \newcommand\nissan{EV manufacturer}



\newcommand{\SOCR}{{{\it SOC}^{R}}}
\newcommand{\SOCI}{{{\it SOC}^{I}}} 
\newcommand{\SOC}{{{\it SOC}}} 
\newcommand{\SOCMAX}{{{\it SOC}^{max}}} 
\newcommand{\SOCMIN}{{{\it SOC}^{min}}} 
\newcommand{\PrdPeak}{\hat{P}^{max}} 
\newcommand{\CS}{\bar{\mathcal{C}}}
\newcommand{\PowerNeed}{{\it KWH^{R}}} % use R for required. KWH is energy
\newcommand{\ReTime}{\tau^R}
\newcommand{\MaskAction}{A'}

\newcommand{\Building}{\mathcal{B}}
\newcommand{\DepartureTime}{\mathcal{D}} 
\newcommand{\policyGuidanceRate}{{R^{PG}}}
\newcommand{\Mask}{{\it Mask}}
\newcommand{\MILP}{{\it MILP}} 
\newcommand{\Buffer}{\mathbf{BF}} 

% FQ TODO: Need add more! 



% Albation Study
\newcommand{\rlcluster}{{\bf RL\textbackslash{}500} }  % Use 500 training samples train model 
\newcommand{\rlrandom}{{\bf RL\textbackslash{}C} } % random select 60 samples training model 
\newcommand{\rlmorefeature}{{\bf RL\textbackslash{}F} } % useing 100 state features
\newcommand{\random}{{\bf Random\textbackslash{}A} } % random select action + action masking 

\newcommand{\rlnop}{{\bf RL\textbackslash{}P}}
\newcommand{\rlnoa}{{\bf RL\textbackslash{}A}}
\newcommand{\rlnoh}{{\bf RL\textbackslash{}H}}
\newcommand{\rlnoe}{{\bf RL\textbackslash{}E}}












%%% == IMPORTANT ==
%%% Use this command to specify your EasyChair submission number.
%%% In anonymous mode, it will be printed on the first page.
\DeclareMathOperator*{\argmin}{arg\,min}

\acmSubmissionID{966}

%%% Use this command to specify the title of your paper.

\title[AAMAS-2025 Formatting Instructions]{Reinforcement Learning-based Approach for Vehicle-to-Building Charging with Heterogeneous Agents and Long Term Rewards }
% \title[AAMAS-2025 Formatting Instructions]{An Online Approach for Vehicle-to-Building Problem with Heterogeneous Agents and Long Term Rewards }

%%% Provide names, affiliations, and email addresses for all authors.

\author{Fangqi Liu}
\affiliation{
  \institution{Vanderbilt University}
  \city{Nashville, TN}
  \country{USA}}
\email{fangqi.liu@vanderbilt.edu}

\author{Rishav Sen}
\affiliation{
  \institution{Vanderbilt University}
  \city{Nashville, TN}
  \country{USA}}
\email{rishav.sen@vanderbilt.edu}

\author{Jose Paolo Talusan}
\affiliation{
  \institution{Vanderbilt University}
  \city{Nashville, TN}
  \country{USA}}
\email{jose.paolo.talusan@vanderbilt.edu}

\author{Ava Pettet}
\affiliation{
  \institution{Nissan Advanced Technology Center - Silicon Valley}
  \city{Santa Clara, CA}
  \country{USA}}
\email{ava.pettet@nissan-usa.com}

\author{Aaron Kandel}
\affiliation{
  \institution{Nissan Advanced Technology Center - Silicon Valley}
  \city{Santa Clara, CA}
  \country{USA}}
\email{aaron.kandel@nissan-usa.com}

\author{Yoshinori Suzue}
\affiliation{
  \institution{Nissan Advanced Technology Center - Silicon Valley}
  \city{Santa Clara, CA}
  \country{USA}}
\email{yoshinori.suzue@nissan-usa.com}

\author{Ayan Mukhopadhyay}
\affiliation{
  \institution{Vanderbilt University}
  \city{Nashville, TN}
  \country{USA}}
\email{ayan.mukhopadhyay@vanderbilt.edu	}

\author{Abhishek Dubey}
\affiliation{
  \institution{Vanderbilt University}
  \city{Nashville, TN}
  \country{USA}}
\email{abhishek.dubey@vanderbilt.edu}

% \author{Fangqi Liu}
% \author{Rishav Sen}
% \author{Jose Paolo Talusan}
% \email{fangqi.liu@vanderbilt.edu}
% \email{rishav.sen@vanderbilt.edu}
% \email{jose.paolo.talusan@vanderbilt.edu}
% \affiliation{
%   \institution{Vanderbilt University}
%   \city{Nashville, TN}
%   \country{USA}}

% \author{Ava Pettet}
% \author{Aaron Kandel}
% \author{Yoshinori Suzue}
% \email{ava.pettet@nissan-usa.com}
% \email{aaron.kandel@nissan-usa.com}
% \email{yoshinori.suzue@nissan-usa.com}

% \affiliation{
%   \institution{Nissan Advanced Technology Center - Silicon Valley}
%   \city{Santa Clara, CA}
%   \country{USA}}


% \author{Ayan Mukhopadhyay}
% \author{Abhishek Dubey}
% \email{ayan.mukhopadhyay@vanderbilt.edu	}
% \email{abhishek.dubey@vanderbilt.edu}
% \affiliation{
%   \institution{Vanderbilt University}
%   \city{Nashville, TN}
%   \country{USA}}

%%% Use this environment to specify a short abstract for your paper.


\begin{abstract}
\begin{abstract}
  In this work, we present a novel technique for GPU-accelerated Boolean satisfiability (SAT) sampling. Unlike conventional sampling algorithms that directly operate on conjunctive normal form (CNF), our method transforms the logical constraints of SAT problems by factoring their CNF representations into simplified multi-level, multi-output Boolean functions. It then leverages gradient-based optimization to guide the search for a diverse set of valid solutions. Our method operates directly on the circuit structure of refactored SAT instances, reinterpreting the SAT problem as a supervised multi-output regression task. This differentiable technique enables independent bit-wise operations on each tensor element, allowing parallel execution of learning processes. As a result, we achieve GPU-accelerated sampling with significant runtime improvements ranging from $33.6\times$ to $523.6\times$ over state-of-the-art heuristic samplers. We demonstrate the superior performance of our sampling method through an extensive evaluation on $60$ instances from a public domain benchmark suite utilized in previous studies. 


  
  % Generating a wide range of diverse solutions to logical constraints is crucial in software and hardware testing, verification, and synthesis. These solutions can serve as inputs to test specific functionalities of a software program or as random stimuli in hardware modules. In software verification, techniques like fuzz testing and symbolic execution use this approach to identify bugs and vulnerabilities. In hardware verification, stimulus generation is particularly vital, forming the basis of constrained-random verification. While generating multiple solutions improves coverage and increases the chances of finding bugs, high-throughput sampling remains challenging, especially with complex constraints and refined coverage criteria. In this work, we present a novel technique that enables GPU-accelerated sampling, resulting in high-throughput generation of satisfying solutions to Boolean satisfiability (SAT) problems. Unlike conventional sampling algorithms that directly operate on conjunctive normal form (CNF), our method refines the logical constraints of SAT problems by transforming their CNF into simplified multi-level Boolean expressions. It then leverages gradient-based optimization to guide the search for a diverse set of valid solutions.
  % Our method specifically takes advantage of the circuit structure of refined SAT instances by using GD to learn valid solutions, reinterpreting the SAT problem as a supervised multi-output regression task. This differentiable technique enables independent bit-wise operations on each tensor element, allowing parallel execution of learning processes. As a result, we achieve GPU-accelerated sampling with significant runtime improvements ranging from $10\times$ to $1000\times$ over state-of-the-art heuristic samplers. Specifically, we demonstrate the superior performance of our sampling method through an extensive evaluation on $60$ instances from a public domain benchmark suite utilized in previous studies.

\end{abstract}

\begin{IEEEkeywords}
Boolean Satisfiability, Gradient Descent, Multi-level Circuits, Verification, and Testing.
\end{IEEEkeywords}


\end{abstract}

%%% The code below was generated by the tool at http://dl.acm.org/ccs.cfm.
%%% Please replace this example with code appropriate for your own paper.

\begin{CCSXML}
<ccs2012>
   <concept>
<concept_id>10010147.10010178.10010199.10010201</concept_id>
       <concept_desc>Computing methodologies~Planning under uncertainty</concept_desc>
       <concept_significance>500</concept_significance>
       </concept>
 </ccs2012>
\end{CCSXML}

\ccsdesc[500]{Computing methodologies~Planning under uncertainty}

% \ccsdesc[500]{Theory of computation~Algorithmic game theory}
% \ccsdesc[500]{Mathematics of computing~Matchings and factors}
% \ccsdesc[300]{Mathematics of computing~Combinatorial algorithms}
% \ccsdesc[300]{Theory of computation~Approximation algorithms analysis}

%%% Use this command to specify a few keywords describing your work.
%%% Keywords should be separated by commas.

\keywords{Reinforcement Learning; Optimization; Electric Vehicle Charging}

%%%%%%%%%%%%%%%%%%%%%%%%%%%%%%%%%%%%%%%%%%%%%%%%%%%%%%%%%%%%%%%%%%%%%%%%

%%% Include any author-defined commands here.
         
\newcommand{\BibTeX}{\rm B\kern-.05em{\sc i\kern-.025em b}\kern-.08em\TeX}

%%%%%%%%%%%%%%%%%%%%%%%%%%%%%%%%%%%%%%%%%%%%%%%%%%%%%%%%%%%%%%%%%%%%%%%%

\begin{document}
% \iad{please check the errors in compilation}
%%% The following commands remove the headers in your paper. For final 
%%% papers, these will be inserted during the pagination process.

\pagestyle{fancy}
\fancyhead{}

%%% The next command prints the information defined in the preamble. 
\maketitle 
%%%%%%%%%%%%%%%%%%%%%%%%%%%%%%%%%%%%%%%%%%%%%%%%%%%%%%%%%%%%%%%%%%%%%%%%

\section{Introduction}
\label{section:introduction}

% redirection is unique and important in VR
Virtual Reality (VR) systems enable users to embody virtual avatars by mirroring their physical movements and aligning their perspective with virtual avatars' in real time. 
As the head-mounted displays (HMDs) block direct visual access to the physical world, users primarily rely on visual feedback from the virtual environment and integrate it with proprioceptive cues to control the avatar’s movements and interact within the VR space.
Since human perception is heavily influenced by visual input~\cite{gibson1933adaptation}, 
VR systems have the unique capability to control users' perception of the virtual environment and avatars by manipulating the visual information presented to them.
Leveraging this, various redirection techniques have been proposed to enable novel VR interactions, 
such as redirecting users' walking paths~\cite{razzaque2005redirected, suma2012impossible, steinicke2009estimation},
modifying reaching movements~\cite{gonzalez2022model, azmandian2016haptic, cheng2017sparse, feick2021visuo},
and conveying haptic information through visual feedback to create pseudo-haptic effects~\cite{samad2019pseudo, dominjon2005influence, lecuyer2009simulating}.
Such redirection techniques enable these interactions by manipulating the alignment between users' physical movements and their virtual avatar's actions.

% % what is hand/arm redirection, motivation of study arm-offset
% \change{\yj{i don't understand the purpose of this paragraph}
% These illusion-based techniques provide users with unique experiences in virtual environments that differ from the physical world yet maintain an immersive experience. 
% A key example is hand redirection, which shifts the virtual hand’s position away from the real hand as the user moves to enhance ergonomics during interaction~\cite{feuchtner2018ownershift, wentzel2020improving} and improve interaction performance~\cite{montano2017erg, poupyrev1996go}. 
% To increase the realism of virtual movements and strengthen the user’s sense of embodiment, hand redirection techniques often incorporate a complete virtual arm or full body alongside the redirected virtual hand, using inverse kinematics~\cite{hartfill2021analysis, ponton2024stretch} or adjustments to the virtual arm's movement as well~\cite{li2022modeling, feick2024impact}.
% }

% noticeability, motivation of predicting a probability, not a classification
However, these redirection techniques are most effective when the manipulation remains undetected~\cite{gonzalez2017model, li2022modeling}. 
If the redirection becomes too large, the user may not mitigate the conflict between the visual sensory input (redirected virtual movement) and their proprioception (actual physical movement), potentially leading to a loss of embodiment with the virtual avatar and making it difficult for the user to accurately control virtual movements to complete interaction tasks~\cite{li2022modeling, wentzel2020improving, feuchtner2018ownershift}. 
While proprioception is not absolute, users only have a general sense of their physical movements and the likelihood that they notice the redirection is probabilistic. 
This probability of detecting the redirection is referred to as \textbf{noticeability}~\cite{li2022modeling, zenner2024beyond, zenner2023detectability} and is typically estimated based on the frequency with which users detect the manipulation across multiple trials.

% version B
% Prior research has explored factors influencing the noticeability of redirected motion, including the redirection's magnitude~\cite{wentzel2020improving, poupyrev1996go}, direction~\cite{li2022modeling, feuchtner2018ownershift}, and the visual characteristics of the virtual avatar~\cite{ogawa2020effect, feick2024impact}.
% While these factors focus on the avatars, the surrounding virtual environment can also influence the users' behavior and in turn affect the noticeability of redirection.
% One such prominent external influence is through the visual channel - the users' visual attention is constantly distracted by complex visual effects and events in practical VR scenarios.
% Although some prior studies have explored how to leverage user blindness caused by visual distractions to redirect users' virtual hand~\cite{zenner2023detectability}, there remains a gap in understanding how to quantify the noticeability of redirection under visual distractions.

% visual stimuli and gaze behavior
Prior research has explored factors influencing the noticeability of redirected motion, including the redirection's magnitude~\cite{wentzel2020improving, poupyrev1996go}, direction~\cite{li2022modeling, feuchtner2018ownershift}, and the visual characteristics of the virtual avatar~\cite{ogawa2020effect, feick2024impact}.
While these factors focus on the avatars, the surrounding virtual environment can also influence the users' behavior and in turn affect the noticeability of redirection.
This, however, remains underexplored.
One such prominent external influence is through the visual channel - the users' visual attention is constantly distracted by complex visual effects and events in practical VR scenarios.
We thus want to investigate how \textbf{visual stimuli in the virtual environment} affect the noticeability of redirection.
With this, we hope to complement existing works that focus on avatars by incorporating environmental visual influences to enable more accurate control over the noticeability of redirected motions in practical VR scenarios.
% However, in realistic VR applications, the virtual environment often contains complex visual effects beyond the virtual avatar itself. 
% We argue that these visual effects can \textbf{distract users’ visual attention and thus affect the noticeability of redirection offsets}, while current research has yet taken into account.
% For instance, in a VR boxing scenario, a user’s visual attention is likely focused on their opponent rather than on their virtual body, leading to a lower noticeability of redirection offsets on their virtual movements. 
% Conversely, when reaching for an object in the center of their field of view, the user’s attention is more concentrated on the virtual hand’s movement and position to ensure successful interaction, resulting in a higher noticeability of offsets.

Since each visual event is a complex choreography of many underlying factors (type of visual effect, location, duration, etc.), it is extremely difficult to quantify or parameterize visual stimuli.
Furthermore, individuals respond differently to even the same visual events.
Prior neuroscience studies revealed that factors like age, gender, and personality can influence how quickly someone reacts to visual events~\cite{gillon2024responses, gale1997human}. 
Therefore, aiming to model visual stimuli in a way that is generalizable and applicable to different stimuli and users, we propose to use users' \textbf{gaze behavior} as an indicator of how they respond to visual stimuli.
In this paper, we used various gaze behaviors, including gaze location, saccades~\cite{krejtz2018eye}, fixations~\cite{perkhofer2019using}, and the Index of Pupil Activity (IPA)~\cite{duchowski2018index}.
These behaviors indicate both where users are looking and their cognitive activity, as looking at something does not necessarily mean they are attending to it.
Our goal is to investigate how these gaze behaviors stimulated by various visual stimuli relate to the noticeability of redirection.
With this, we contribute a model that allows designers and content creators to adjust the redirection in real-time responding to dynamic visual events in VR.

To achieve this, we conducted user studies to collect users' noticeability of redirection under various visual stimuli.
To simulate realistic VR scenarios, we adopted a dual-task design in which the participants performed redirected movements while monitoring the visual stimuli.
Specifically, participants' primary task was to report if they noticed an offset between the avatar's movement and their own, while their secondary task was to monitor and report the visual stimuli.
As realistic virtual environments often contain complex visual effects, we started with simple and controlled visual stimulus to manage the influencing factors.

% first user study, confirmation study
% collect data under no visual stimuli, different basic visual stimuli
We first conducted a confirmation study (N=16) to test whether applying visual stimuli (opacity-based) actually affects their noticeability of redirection. 
The results showed that participants were significantly less likely to detect the redirection when visual stimuli was presented $(F_{(1,15)}=5.90,~p=0.03)$.
Furthermore, by analyzing the collected gaze data, results revealed a correlation between the proposed gaze behaviors and the noticeability results $(r=-0.43)$, confirming that the gaze behaviors could be leveraged to compute the noticeability.

% data collection study
We then conducted a data collection study to obtain more accurate noticeability results through repeated measurements to better model the relationship between visual stimuli-triggered gaze behaviors and noticeability of redirection.
With the collected data, we analyzed various numerical features from the gaze behaviors to identify the most effective ones. 
We tested combinations of these features to determine the most effective one for predicting noticeability under visual stimuli.
Using the selected features, our regression model achieved a mean squared error (MSE) of 0.011 through leave-one-user-out cross-validation. 
Furthermore, we developed both a binary and a three-class classification model to categorize noticeability, which achieved an accuracy of 91.74\% and 85.62\%, respectively.

% evaluation study
To evaluate the generalizability of the regression model, we conducted an evaluation study (N=24) to test whether the model could accurately predict noticeability with new visual stimuli (color- and scale-based animations).
Specifically, we evaluated whether the model's predictions aligned with participants' responses under these unseen stimuli.
The results showed that our model accurately estimated the noticeability, achieving mean squared errors (MSE) of 0.014 and 0.012 for the color- and scale-based visual stimili, respectively, compared to participants' responses.
Since the tested visual stimuli data were not included in the training, the results suggested that the extracted gaze behavior features capture a generalizable pattern and can effectively indicate the corresponding impact on the noticeability of redirection.

% application
Based on our model, we implemented an adaptive redirection technique and demonstrated it through two applications: adaptive VR action game and opportunistic rendering.
We conducted a proof-of-concept user study (N=8) to compare our adaptive redirection technique with a static redirection, evaluating the usability and benefits of our adaptive redirection technique.
The results indicated that participants experienced less physical demand and stronger sense of embodiment and agency when using the adaptive redirection technique. 
These results demonstrated the effectiveness and usability of our model.

In summary, we make the following contributions.
% 
\begin{itemize}
    \item 
    We propose to use users' gaze behavior as a medium to quantify how visual stimuli influences the noticebility of redirection. 
    Through two user studies, we confirm that visual stimuli significantly influences noticeability and identify key gaze behavior features that are closely related to this impact.
    \item 
    We build a regression model that takes the user's gaze behavioral data as input, then computes the noticeability of redirection.
    Through an evaluation study, we verify that our model can estimate the noticeability with new participants under unseen visual stimuli.
    These findings suggest that the extracted gaze behavior features effectively capture the influence of visual stimuli on noticeability and can generalize across different users and visual stimuli.
    \item 
    We develop an adaptive redirection technique based on our regression model and implement two applications with it.
    With a proof-of-concept study, we demonstrate the effectiveness and potential usability of our regression model on real-world use cases.

\end{itemize}

% \delete{
% Virtual Reality (VR) allows the user to embody a virtual avatar by mirroring their physical movements through the avatar.
% As the user's visual access to the physical world is blocked in tasks involving motion control, they heavily rely on the visual representation of the avatar's motions to guide their proprioception.
% Similar to real-world experiences, the user is able to resolve conflicts between different sensory inputs (e.g., vision and motor control) through multisensory integration, which is essential for mitigating the sensory noise that commonly arises.
% However, it also enables unique manipulations in VR, as the system can intentionally modify the avatar's movements in relation to the user's motions to achieve specific functional outcomes,
% for example, 
% % the manipulations on the avatar's movements can 
% enabling novel interaction techniques of redirected walking~\cite{razzaque2005redirected}, redirected reaching~\cite{gonzalez2022model}, and pseudo haptics~\cite{samad2019pseudo}.
% With small adjustments to the avatar's movements, the user can maintain their sense of embodiment, due to their ability to resolve the perceptual differences.
% % However, a large mismatch between the user and avatar's movements can result in the user losing their sense of embodiment, due to an inability to resolve the perceptual differences.
% }

% \delete{
% However, multisensory integration can break when the manipulation is so intense that the user is aware of the existence of the motion offset and no longer maintains the sense of embodiment.
% Prior research studied the intensity threshold of the offset applied on the avatar's hand, beyond which the embodiment will break~\cite{li2022modeling}. 
% Studies also investigated the user's sensitivity to the offsets over time~\cite{kohm2022sensitivity}.
% Based on the findings, we argue that one crucial factor that affects to what extent the user notices the offset (i.e., \textit{noticeability}) that remains under-explored is whether the user directs their visual attention towards or away from the virtual avatar.
% Related work (e.g., Mise-unseen~\cite{marwecki2019mise}) has showcased applications where adjustments in the environment can be made in an unnoticeable manner when they happen in the area out of the user's visual field.
% We hypothesize that directing the user's visual attention away from the avatar's body, while still partially keeping the avatar within the user's field-of-view, can reduce the noticeability of the offset.
% Therefore, we conduct two user studies and implement a regression model to systematically investigate this effect.
% }

% \delete{
% In the first user study (N = 16), we test whether drawing the user's visual attention away from their body impacts the possibility of them noticing an offset that we apply to their arm motion in VR.
% We adopt a dual-task design to enable the alteration of the user's visual attention and a yes/no paradigm to measure the noticeability of motion offset. 
% The primary task for the user is to perform an arm motion and report when they perceive an offset between the avatar's virtual arm and their real arm.
% In the secondary task, we randomly render a visual animation of a ball turning from transparent to red and becoming transparent again and ask them to monitor and report when it appears.
% We control the strength of the visual stimuli by changing the duration and location of the animation.
% % By changing the time duration and location of the visual animation, we control the strengths of attraction to the users.
% As a result, we found significant differences in the noticeability of the offsets $(F_{(1,15)}=5.90,~p=0.03)$ between conditions with and without visual stimuli.
% Based on further analysis, we also identified the behavioral patterns of the user's gaze (including pupil dilation, fixations, and saccades) to be correlated with the noticeability results $(r=-0.43)$ and they may potentially serve as indicators of noticeability.
% }

% \delete{
% To further investigate how visual attention influences the noticeability, we conduct a data collection study (N = 12) and build a regression model based on the data.
% The regression model is able to calculate the noticeability of the offset applied on the user's arm under various visual stimuli based on their gaze behaviors.
% Our leave-one-out cross-validation results show that the proposed method was able to achieve a mean-squared error (MSE) of 0.012 in the probability regression task.
% }

% \delete{
% To verify the feasibility and extendability of the regression model, we conduct an evaluation study where we test new visual animations based on adjustments on scale and color and invite 24 new participants to attend the study.
% Results show that the proposed method can accurately estimate the noticeability with an MSE of 0.014 and 0.012 in the conditions of the color- and scale-based visual effects.
% Since these animations were not included in the dataset that the regression model was built on, the study demonstrates that the gaze behavioral features we extracted from the data capture a generalizable pattern of the user's visual attention and can indicate the corresponding impact on the noticeability of the offset.
% }

% \delete{
% Finally, we demonstrate applications that can benefit from the noticeability prediction model, including adaptive motion offsets and opportunistic rendering, considering the user's visual attention. 
% We conclude with discussions of our work's limitations and future research directions.
% }

% \delete{
% In summary, we make the following contributions.
% }
% % 
% \begin{itemize}
%     \item 
%     \delete{
%     We quantify the effects of the user's visual attention directed away by stimuli on their noticeability of an offset applied to the avatar's arm motion with respect to the user's physical arm. 
%     Through two user studies, we identified gaze behavioral features that are indicative of the changes in noticeability.
%     }
%     \item 
%     \delete{We build a regression model that takes the user's gaze behavioral data and the offset applied to the arm motion as input, then computes the probability of the user noticing the offset.
%     Through an evaluation study, we verified that the model needs no information about the source attracting the user's visual attention and can be generalizable in different scenarios.
%     }
%     \item 
%     \delete{We demonstrate two applications that potentially benefit from the regression model, including adaptive motion offsets and opportunistic rendering.
%     }

% \end{itemize}

\begin{comment}
However, users will lose the sense of embodiment to the virtual avatars if they notice the offset between the virtual and physical movements.
To address this, researchers have been exploring the noticing threshold of offsets with various magnitudes and proposing various redirection techniques that maintain the sense of embodiment~\cite{}.

However, when users embody virtual avatars to explore virtual environments, they encounter various visual effects and content that can attract their attention~\cite{}.
During this, the user may notice an offset when he observes the virtual movement carefully while ignoring it when the virtual contents attract his attention from the movements.
Therefore, static offset thresholds are not appropriate in dynamic scenarios.

Past research has proposed dynamic mapping techniques that adapted to users' state, such as hand moving speed~\cite{frees2007prism} or ergonomically comfortable poses~\cite{montano2017erg}, but not considering the influence of virtual content.
More specifically, PRISM~\cite{frees2007prism} proposed adjusting the C/D ratio with a non-linear mapping according to users' hand moving speed, but it might not be optimal for various virtual scenarios.
While Erg-O~\cite{montano2017erg} redirected users' virtual hands according to the virtual target's relative position to reduce physical fatigue, neglecting the change of virtual environments. 

Therefore, how to design redirection techniques in various scenarios with different visual attractions remains unknown.
To address this, we investigate how visual attention affects the noticing probability of movement offsets.
Based on our experiments, we implement a computational model that automatically computes the noticing probability of offsets under certain visual attractions.
VR application designers and developers can easily leverage our model to design redirection techniques maintaining the sense of embodiment adapt to the user's visual attention.
We implement a dynamic redirection technique with our model and demonstrate that it effectively reduces the target reaching time without reducing the sense of embodiment compared to static redirection techniques.

% Need to be refined
This paper offers the following contributions.
\begin{itemize}
    \item We investigate how visual attractions affect the noticing probability of redirection offsets.
    \item We construct a computational model to predict the noticing probability of an offset with a given visual background.
    \item We implement a dynamic redirection technique adapting to the visual background. We evaluate the technique and develop three applications to demonstrate the benefits. 
\end{itemize}



First, we conducted a controlled experiment to understand how users perceived the movement offset while subjected to various distractions.
Since hand redirection is one of the most frequently used redirections in VR interactions, we focused on the dynamic arm movements and manually added angular offsets to the' elbow joint~\cite{li2022modeling, gonzalez2022model, zenner2019estimating}. 
We employed flashing spheres in the user's field of view as distractions to attract users' visual attention.
Participants were instructed to report the appearing location of the spheres while simultaneously performing the arm movements and reporting if they perceived an offset during the movement. 
(\zhipeng{Add the results of data collection. Analyze the influence of the distance between the gaze map and the offset.}
We measured the visual attraction's magnitude with the gaze distribution on it.
Results showed that stronger distractions made it harder for users to notice the offset.)
\zhipeng{Need to rewrite. Not sure to use gaze distribution or a metric obtained from the visual content.}
Secondly, we constructed a computational model to predict the noticing probability of offsets with given visual content.
We analyzed the data from the user studies to measure the influence of visual attractions on the noticing probability of offsets.
We built a statistical model to predict the offset's noticing probability with a given visual content.
Based on the model, we implement a dynamic redirection technique to adjust the redirection offset adapted to the user's current field of view.
We evaluated the technique in a target selection task compared to no hand redirection and static hand redirection.
\zhipeng{Add the results of the evaluation.}
Results showed that the dynamic hand redirection technique significantly reduced the target selection time with similar accuracy and a comparable sense of embodiment.
Finally, we implemented three applications to demonstrate the potential benefits of the visual attention adapted dynamic redirection technique.
\end{comment}

% This one modifies arm length, not redirection
% \citeauthor{mcintosh2020iteratively} proposed an adaptation method to iteratively change the virtual avatar arm's length based on the primary tasks' performance~\cite{mcintosh2020iteratively}.



% \zhipeng{TO ADD: what is redirection}
% Redirection enables novel interactions in Virtual Reality, including redirected walking, haptic redirection, and pseudo haptics by introducing an offset to users' movement.
% \zhipeng{TO ADD: extend this sentence}
% The price of this is that users' immersiveness and embodiment in VR can be compromised when they notice the offset and perceive the virtual movement not as theirs~\cite{}.
% \zhipeng{TO ADD: extend this sentence, elaborate how the virtual environment attracts users' attention}
% Meanwhile, the visual content in the virtual environment is abundant and consistently captures users' attention, making it harder to notice the offset~\cite{}.
% While previous studies explored the noticing threshold of the offsets and optimized the redirection techniques to maintain the sense of embodiment~\cite{}, the influence of visual content on the probability of perceiving offsets remains unknown.  
% Therefore, we propose to investigate how users perceive the redirection offset when they are facing various visual attractions.


% We conducted a user study to understand how users notice the shift with visual attractions.
% We used a color-changing ball to attract the user's attention while instructing users to perform different poses with their arms and observe it meanwhile.
% \zhipeng{(Which one should be the primary task? Observe the ball should be the primary one, but if the primary task is too simple, users might allocate more attention on the secondary task and this makes the secondary task primary.)}
% \zhipeng{(We need a good and reasonable dual-task design in which users care about both their pose and the visual content, at least in the evaluation study. And we need to be able to control the visual content's magnitude and saliency maybe?)}
% We controlled the shift magnitude and direction, the user's pose, the ball's size, and the color range.
% We set the ball's color-changing interval as the independent factor.
% We collect the user's response to each shift and the color-changing times.
% Based on the collected data, we constructed a statistical model to describe the influence of visual attraction on the noticing probability.
% \zhipeng{(Are we actually controlling the attention allocation? How do we measure the attracting effect? We need uniform metrics, otherwise it is also hard for others to use our knowledge.)}
% \zhipeng{(Try to use eye gaze? The eye gaze distribution in the last five seconds to decide the attention allocation? Basically constructing a model with eye gaze distribution and noticing probability. But the user's head is moving, so the eye gaze distribution is not aligned well with the current view.)}

% \zhipeng{Saliency and EMD}
% \zhipeng{Gaze is more than just a point: Rethinking visual attention
% analysis using peripheral vision-based gaze mapping}

% Evaluation study(ideal case): based on the visual content, adjusting the redirection magnitude dynamically.

% \zhipeng{(The risk is our model's effect is trivial.)}

% Applications:
% Playing Lego while watching demo videos, we can accelerate the reaching process of bricks, and forbid the redirection during the manipulation.

% Beat saber again: but not make a lot of sense? Difficult game has complicated visual effects, while allows larger shift, but do not need large shift with high difficulty



\section{Problem Statement}

Consider an autonomous system with state $\state \in \sset \subseteq \mathbb{R}^n$ that evolves according to dynamics $\dot{\state} = \dyn(\state, \ctrl, \dstb)$, where $\ctrl \in \cset$ and $\dstb \in \dset$ are the control and disturbance of the system, respectively. 
$\dstb$ can represent potential model uncertainties or an actual, adversarial exogenous input to the system.
We assume the dynamics are uniformly continuous in $u$ and $d$, bounded, and Lipschitz continuous in $\state$ for fixed $\ctrl$ and $\dstb$.
Finally, let $\traj_{\state,\tvar}^{\ctrlseq,\mathbf{\dstb}}(\tdummy)$ denote the system state at time $\tdummy$, starting from the state $\state$ at time $\tvar$ under control signal $\ctrlseq(\cdot)$ and disturbance signal $\dstbseq(\cdot)$ while following the dynamics. A control signal $\ctrlseq(\cdot)$ is defined as a measurable function mapping from the time horizon to the set of admissible controls $\cset$, and a disturbance signal is similarly defined. 
We additionally assume that the control and disturbance signals $\ctrlseq(\cdot)$ and $\dstbseq(\cdot)$ are piecewise continuous in $t$. 
This assumption ensures that the system trajectory $\traj_{\state,\tvar}^{\ctrlseq,\mathbf{\dstb}}$
exists and is unique and continuous for all initial states \cite{coddington1955theory, callier2012linear}.

In addition, we are given a failure set $\targetset \subset X$ that the system must avoid at all times (e.g., obstacles for a navigation robot). The safety constraint is encoded via a Lipschitz continuous function $
\targetfunc (\cdot): \targetset= \{\state : \targetfunc(\state)\leq 0\}$.
We aim to design a controller that optimally balances the system's safety constraints and performance objectives. The performance objective is given by minimizing a cost function $S$ over the system trajectory, given by:
%
\begin{equation}    S(\traj)=\phi(\state(\tfinal),\tfinal)+\int_{\tinit}^{\tfinal}{\mathcal{L}}(\state(\tvar),\ctrl(\tvar),\tvar)\mathrm{d}\tvar
\end{equation}
%
where $\phi$ and $\mathcal{L}$ represent the terminal and running cost respectively, and $\tfinal$ the task completion time.
Specifically, we aim to ensure that the control actions $\ctrl$ drive the system towards minimizing $S$ while rigorously maintaining system safety by preventing the state $\state$ from entering $\targetset$ even for the worst case disturbances $\dstb$. This takes the form of the following constrained optimal control problem:
%
\begin{equation}\label{eq:opt_problem}
\begin{aligned}
    \ctrl^{*}(\cdot)&=\argmin_{\ctrl(\cdot)}S(\traj_{\state,\tvar}^{\ctrlseq,\mathbf{\dstb}},\ctrl(\cdot)) \\ 
    \text{subject to:}& \\
    \dot{\state}(t) &= \dyn(\state(t), \ctrl(t), \dstb(t)), \\
    \quad \;\targetfunc(\state(\tvar))& > 0, 
    \;\ctrl(t) \in \cset, 
    \;\dstb(t) \in \dset,
   \forall t \in [\tvar, \tvar_f]
\end{aligned}
\end{equation}
%
\noindent the optimization problem defined in \eqref{eq:opt_problem} in general is non-convex and can be difficult to solve. In this work, we propose a novel MPPI method to solve this problem.
\section{Related Work}
\label{sec:related_work}
% \begin{subtable}[t]{0.5\textwidth}
    \centering
    \resizebox{\columnwidth}{!}{%
    \begin{tabular}{|l|l|l|c|}
        \hline
        Reference & Game Type & Solution Concept & \makecell{Polytime?} \\
        \hline
        \hline \citep{fu2021evaluating} & Finite Markov & Nash & \xmark \\
        \hline \citep{yu2019multi} & Finite Markov & Quantal Response & \xmark \\
        \hline \citep{lin2019multi} & Finite Zero-sum Markov & Various & \xmark\\
        \hline 
        \citep{song2018multi} & Finite Markov &
        Quantal Response & \xmark \\
        \hline  \citep{syrgkanis2017inference} & Finite Bayesian & Bayes-Nash & \cmark \\
        \hline \citep{kuleshov2015inverse} & Finite Normal-Form & Correlated & \cmark \\
        \hline \citep{waugh2013computational} & Finite Normal-Form & Correlated &  \cmark \\
        \hline \citep{bestick2013inverse} & Finite Normal-Form & Correlated & \xmark \\  
        \hline \citep{natarajan2010multi} & Finite Markov & Cooperative & \xmark \\
        \hline \rowcolor{orange!50} This work & \makecell[l]{Finite/Concave Normal-form\\ Finite/Concave Markov} & 
        % \makecell[l]{Nash /Any}
        \makecell[l]{Nash/Correlated\\ Any Other \quad \quad \quad \quad }
        & \cmark \\
        \hline
    \end{tabular}
    }
    \caption{A comparison of our work and prior work on inverse game theory and inverse MARL.}
    \label{tab:summary_lit}
    \vspace{-2em}
\end{subtable}


We highlight four major challenges of solving the V2B problem, namely: 1) the uncertainty of vehicles and SoC requirements; 2) Time-Of-Use (TOU) pricing, demand charges, and long-term rewards; 3) heterogeneous chargers and continuous action spaces; and 4) tracking real-world states and transitions. Below, we briefly cover prior work to tackle these challenges. \textit{A more detailed description of prior work is presented in~\Cref{tab:comparison} of the appendix.}  

\noindent\textbf{Uncertainty of vehicles and SoC requirements.} 
% Prior work by \citeauthor{MJG2015} considers mobility aspects like EV arrival/departure times and trip history for charging stations~\cite{MJG2015}.
Meta-heuristics and Model Predictive Control (MPC) have been used to solve the EV charging process, focusing on energy cost and user fairness in single-site or vehicle-to-grid (V2G) systems~\cite{AORC2013, 5986769, 9409126, MJG2015}. 
Studies by \citeauthor{richardson2011electric} analyze EV charging strategies' impact on grid stability, relevant to V2B systems~\cite{richardson2011electric}. \citeauthor{8274175} proposed a demand response framework for optimizing V2B systems amidst dynamic energy pricing~\cite{8274175}. Additionally, \citeauthor{oconnell2010integration} utilized Mixed Integer Linear Programming (MILP) to integrate renewable energy sources into grids~\cite{oconnell2010integration}.
% Additionally, there are empirical studies that have analyzed EV charging strategies and their impact on grid stability, which are closely related to V2B systems~\cite{richardson2011electric}.
However, many of these methods focus on unidirectional chargers and fail to fully account for all exogenous sources of uncertainty (e.g., uncertain arrival and departure times).


% {\color{black} Other approaches, including meta-heuristics and Model Predictive Control (MPC), have been explored to optimize the smart EV charging process for electric vehicles (EVs), focusing on energy cost and user fairness in single-site or vehicle-to-grid (V2G) systems~\cite{AORC2013, 5986769, 9409126, MJG2015}.} 
% However, many of these methods focus on unidirectional chargers and fail to fully account for uncertainty including, vehicle arrivals and departures~\cite{MJG2015}. 
% This uncertainty arise from the mobility of EVs, which include aspects such as their arrival and departure times of an EV at/from a charging station, trip history of EVs, and unplanned departure of EVs.

\noindent\textbf{Time of use pricing, demand charge, and long-term rewards.} 
% Time of use pricing and demand charge, Long-term rewards and planning horizon
% AORC2013, MMN2019, SNDJ2020
V2B optimization is difficult due to long billing periods. While prior work (barring some exceptions~\cite{9409126}) optimizes and plans for single-day horizons~\cite{AORC2013, MMN2019, SNDJ2020}, they fail to work for longer periods.

% \noindent\textbf{Heterogeneous chargers and continuous action spaces.} In practice, buildings develop EV infrastructure over time and have heterogeneous chargers, complicating the action space. While some prior work has successfully modeled charger heterogeneity~\cite{NNM2024,ZJS2022}, such work either does not capture long-term rewards (i.e., limit planning to a single day) or fails to account for demand charge, thereby failing to capture what is arguably the most critical real-world constraint of the V2B problem.
\noindent\textbf{Heterogeneous chargers and continuous action spaces.} 
%Approaches that solve EV charging without considering the ability of EVs to discharge, ignores even more potential savings. However, addressing this introduces further complexity to the system.
In practice, buildings develop EV infrastructure gradually, leading to heterogeneous chargers and a more complex action space.
While some prior work addresses charger heterogeneity~\cite{NNM2024,ZJS2022}, it often neglects long-term rewards (i.e., limit planning to a single day) or fails to account for demand charge, missing the key real-world constraint in the V2B problem.
\noindent\textbf{Tracking real-world state and transition.}
Existing solutions validate their approaches using simulations with limited interface with the real world (barring some exceptions~\cite{9409126}), thereby making simplistic assumptions that limit deployment.

%The integration of electric vehicles (EVs) into energy management systems, particularly through vehicle-to-grid (V2G) and vehicle-to-building (V2B) technologies, is increasingly recognized for its role in balancing energy demand~\cite{lund2008integration}. 
% Table~\ref{tab:comparison} summarizes related work, outlining each study's objectives, limitations in action spaces, and planning duration. It also highlights key features such as discharging, EV mobility, SoC requirements, and whether long-term demand charge cost reduction was considered. Here, mobility or mobility-aware charging is defined by~\citeauthor{MJG2015} as taking into consideration different mobility aspects such as the arrival/departure time of an EV at/from a charging station, trip history of EVs, and unplanned departure of EVs~\cite{MJG2015}.



%Additionally, offline methods and meta-heuristics often require extended runtimes, making them unsuitable for real-time decision-making in dynamic environments.
%TODO: Initial RL 2,4,13,

% To account for uncertainty, reinforcement learning methods have also been tried before.
% %Recent work has shifted toward machine learning, particularly reinforcement learning (RL), for optimizing energy systems, due to their ability to better generalize and also accommodate long-term rewards compared to meta-heuristics and MPC approaches.
% \citeauthor{mnih2015human} introduced Deep Q-Networks (DQN), which have been adapted for dynamic energy management in V2B systems~\cite{mnih2015human}. \citeauthor{MMN2019} explored Deep Q-learning and deep policy gradient methods for online optimization of building energy schedules~\cite{MMN2019}. However, these approaches still do not consider the mobility of the EVs. Instead the vehicles are treated as stationary loads with no temporal properties related to the
% % arrival and departure of the EVs. 
% %
% %TODO: mobility 7, 18 SNDJ2020 ,NNM2024
% \citeauthor{SNDJ2020} employed an RL-based approach control a set of charger stations to minimize energy consumption in smart grids~\cite{SNDJ2020}. Their approach only considers a boolean decision of turning the charger stations on or off, lacking the ability to discharge. \citeauthor{NNM2024}, improves upon this idea by extending the potential charger actions to include both discharging and charging actions. They applied deep reinforcement learning (Deep RL) to optimize EV and HVAC systems for daily energy cost minimization~\cite{NNM2024}. However, neither of these approaches incorporate the concept of demand charge into the problem and limit their planning horizon to a single day.
% %
% %
% %TODO: discharging 9, 23

% Improving upon these initial approaches, \citeauthor{ZJS2022} investigated federated RL for EV charger control, aiming to maximize user benefits~\cite{ZJS2022}, and minimize electricity prices. Their approach explores continuous action space of charger power rates and extends their planning horizon to an entire week. While their approach include both discharging and charging actions, they fail to capture the idea of demand charge into their problem which is critical for industrial loads. Additionally, when we consider demand charge the planning horizons have to increase to a month.
% %While many of these approaches utilize machine learning and RL-based approaches to the EV charging problem, many of these techniques are limited by small or discretized state and action spaces, consider limited chargers, and short-term (single-day) planning horizons.


%   %In this paper, we propose methods to enhance RL-based V2B optimization, exploring centralized control of multiple chargers, continuous charging decisions to meet diverse SoC requirements and building loads, and enabling long-term (monthly) planning in line with real-world charging constraints.

% \begin{figure*}[t]
%     \centering
%     \begin{subfigure}[b]{0.68\linewidth}
%         % \centering
%         \includegraphics[height=2.2in,keepaspectratio,trim=0.2cm 0.2cm 0.2cm 0.2cm,clip]{figures/framework.pdf}
%         % \vspace{-0.1in}
%         \caption{Reinforcement Learning Framework.}
%        \label{fig:framework}
%     \end{subfigure}
%     % \hfill
%     \begin{subfigure}[b]{0.28\linewidth}
%         % \centering
%         \includegraphics[height=2.2in,keepaspectratio,trim=0cm 0.5cm 0cm 0.5cm,clip]{figures/inference_pipeline.pdf}
%         \vspace{-0.15in}
%         \caption{Pipeline for Inference.}
%         \label{fig:pipeline}
%     \end{subfigure}
%     \caption{\color{black}(a) Our framework relies on daily samples along with an estimated monthly peak power. We use reinforcement learning, i.e., DDPG, and extend it with policy guidance and action masking, to learn a near-optimal policy, A reinforcement learning policy based on DDPG is trained with policy guidance and action masking. (b) At inference time, the model ingests data of connected cars, charger states, building power reading, and the estimated monthly peak power to make decisions.}
%     \label{fig:framework_and_pipeline}
% \end{figure*}


\section{Our Approach}
\label{sec:our_approach}
In this section, we discuss the different components in our  framework, shown in~\Cref{fig:framework}. 
%We present how our approach targets every critical issue we presented in~\Cref{sec:related_work}. 
% We begin by defining the simulator then we provide a description of the MDP.
%
% and our approach targets every critical issue we presented in~\Cref{sec:related_work}.
% , shown in~\Cref{fig:framework}. We present how our approach targets every critical issue we presented in~\Cref{sec:related_work}.
% 
% \noindent\textbf{Uncertainty of vehicles and SoC requirements.}
% %{\jpnote{
% % we first collect data across almost an year -- buildings and car, we use this data to empirically sample chains that can be used to train and validate our policy. This was used by our partners to train a poisson and conditional distribution models from which we were able to sample more chains.  Clearly define what a sample means...}}  
% To address the uncertainty and diversity of V2B scenarios, we collect real-world historical data 
% from \nissan{} from their fleet and chargers at their office building, covering the period from May 2023 to January 2024.  
% The dataset includes EV arrival and departure times, initial and required State of Charge (SoC), and building power demand readings at 15-minute intervals. Since the location is categorized as an industrial building, it is subject to time-of-use (TOU) electricity rates and demand charges during peak hours (6:00 AM to 10:00 PM). Demand charge adds a flat rate multiplier to the highest average power delivery across 15-minute intervals over the billing period and is added to the total bill.
% % . TOU introduces variations in the electricity rates, with higher prices during peak hours (6:00 AM to 10:00 PM). Demand charge adds a flat rate multiplier to the highest average power delivery across 15-minute intervals over the billing period, and is added to the total bill.
% Then, we use the data to learn a Poisson distribution for modeling EV arrival counts, and random forest models for SoC requirements, and building fluctuations based on historical data.
% % We then empirically sampled $1000$ one-month billing episodes samples for each month from May 2023 to January 2024. 
% 
% \noindent\textbf{Time of use pricing, demand charge, and long-term rewards.} 
% %\jpnote{so we break a month into daily episodes and utilize peak power prediction generated across chains using the optimal policy which is MILP. but why daily? constant environment daily, mininmal variance weekly}  
% To enhance training efficacy, we address the challenge of lengthy state-action episodes by splitting the monthly training dataset into daily episodes. This approach captures the diverse conditions observed across different weekdays, enabling the model to learn effectively from shorter episodes, which facilitate quicker adaptation to daily variations.  
% In addition, to incorporate monthly peak power considerations, we use an estimated monthly demand charge as an input feature, penalizing only daily demand charges that exceed this predicted value. This incentivizes the policy to minimize the monthly demand charge while training with daily episodes. We determine the minimum demand charge from optimal action sequences over the one-month billing period generated by MILP optimization (detailed in~\Cref{subsec:milpsolver}). By analyzing the peak power distribution from the MILP solutions, we estimate the lower bound of the 99\% confidence interval as the monthly demand charge, providing a conservative initial estimate. This input feature is further tuned by increasing it by 0\%, 5\%, and 10\% during RL training. 
% % We incorporate the estimated monthly peak power as an input feature and penalize only those peak power values exceeding this estimated value in each daily episode. This approach incentivizes the policy to minimize the monthly peak power while effectively training with daily episodes.   
% 
% During training, we varied the amount of samples used for training and observed that utilizing a larger number of training episodes results in longer convergence time and overall worse performance. We show the results of this in~\Cref{ssec:ablation}. Thus, we utilize a k-means clustering approach to down-sample. We used $k=5$ and clustered using the optimal demand charge derived from the MILP solution for each sample. From each cluster, we select 60 and 50 samples for the final training and testing datasets respectively, ensuring that these datasets are mutually exclusive. 
% 
% \noindent\textbf{Heterogeneous chargers and continuous action spaces.}
% % \jpnote{we model as MDP; for tractability we divide into 15 minute intervals. But our system can work with other discretizations. We use DDPG; but to help with training we use policy guidance (action masking and ILP); its too difficult when you consider heterogeneous and continuous, you need guidance and action masking for RL. only mention briefly. why we need, what we need.}  
% To address uncertainty in the V2B environment, we model the problem as a Markov Decision Process (MDP), dividing the time horizon into 15-minute intervals for tractability, although our system can accommodate other discretizations. We solve this problem using a  RL-based approach built on the Deep Deterministic Policy Gradient (DDPG) algorithm~\cite{lillicrap2015continuous}, which is well-suited for continuous actions and supports off-policy training. This capability allows the model to learn from diverse experiences across various scenarios, enhancing generalization. 
% However, traditional DDPG is limited in reaching global optima for particularly long horizons and large state-action spaces. To address these limitations, we introduce action masking~\cite{huang2020closer,kanervisto2020action}, ensuring that policy actions generated by the actor network are valid and reasonable during DDPG training. This constraint enhances training efficiency by preventing the actor network from exploring invalid actions, optimizing resource usage. 
% Additionally, we incorporate policy guidance techniques~\cite{pmlr-v28-levine13} into the RL training process, which aim to introduce optimal state-action transitions and improve local optima in DDPG. 
% 
% \noindent\textbf{Tracking real-world state and transition.}  
% % - first we downsample to help with generalization using a clustering approach. % why do we need do that. Is there a paper that did something similar.. 
% % \jpnote{we model a digital twin and provide rest APIs -- it has state and transitions and tracks the MDP that we will discuss in the next section. The simulator also is able to generate rewards that are used for training and are also discussed in the next section.} 
% We model a digital twin for the target environment and provide several interfaces that allows both simulated and real-world components to leverage our proposed approach. 
% We model the environment using a digital twin that holds a state representation of the world with transitions representing V2B behavior. 
% % This includes information on EVs, building, and the grid. This allows us to investigate how any action or decision can potentially impact the real world. 
% % Decisions are taken at the end of each set of events for any given time period. 
% %
% There are two main decisions that must be taken when solving the V2B charging problem. (1) Charger assignments and (2) Charger actions.
% % We address the charger assignment decision below and provide information on the charger action in~\Cref{sec:RL}.
% % \textbf{Environment updates and transitions.} The input episodes dictate the simulator's world view. Each event includes an event type and time, matching their real world trigger and occurrence. We identify several critical events in the episodes to serve as the triggers for the simulator. These include (1) EV arrivals, (2) EV departures, (3) building power readings, and (4) TOU rate changes. Events are placed in a queue, with each event triggering an update to the environment which modifies the state. Updates to the state, which include the charging or discharging of EVs, are based on the elapsed time between events. At each decision epoch, there are two decisions to be made, charger assignment and charger actions.
% % % 
\begin{filecontents}{charger_assignments.csv}
a,b,c,d
Bidirectional,Departure,7037.178,957.869
Bidirectional,Capacity,7037.180,958.867
Bidirectional,Random,7037.647,958.656
Random,Random,7038.770,971.310
Random,Departure,7039.723,968.603
Random,Capacity,7040.216,966.708
Unidirectional,Random,7122.884,981.666
Unidirectional,Departure,7123.066,981.620
Unidirectional,Capacity,7123.073,981.612
\end{filecontents}
\pgfplotstableread[col sep=comma]{charger_assignments.csv}{\chargerassignments}

\begin{table}[htp]
\centering
\caption{Charger Assignment and tiebreaker comparisons with an MILP policy. Assigning to Bidirectional chargers first and then breaking ties by assigining them to the EV that departs later results in the lowest bill. Lower is better.}
\pgfplotstabletypeset[
    create on use/bill/.style={
        create col/assign/.code=
            \toprule
            {Assignment} & {Tie Breaker} & {Average Monthly Bill (\$)}\\
        }, 
        after row=\midrule},
    every last row/.style={after row=\bottomrule},
    every row 0 column 0/.style={postproc cell content/.style={@cell content={\textbf{##1}}}}, %bolding
    every row 0 column 1/.style={postproc cell content/.style={@cell content={\textbf{##1}}}}, %bolding
    % every row 0 column 2/.style={postproc cell content/.style={@cell content={\textbf{##1}}}}, %bolding
    % every row 0 column 3/.style={postproc cell content/.style={@cell content={\textbf{##1}}}}, %bolding
    ] {\chargerassignments}
\label{table:charger_assignment_policies}
\end{table}

 
% % \textbf{Charger assignment.} 
% We consider a first-in, first-out policy that assigns EVs to bidirectional chargers first, breaking ties assigning to later departing cars, \textit{a comparison of other approaches are available in the appendix}. 
% % %shows the different charger assignment and tie breaking policies tested.  
% % By comparing various charger assignment and tie-breaking policies (as shown in \Cref{table:charger_assignment_policies} in the Appendix), we observe that bidirectional charging assignments outperform all other policies. Tie-breaking strategies that prioritize later-departing vehicles show a marginal advantage. While these assignment policies could be further optimized, we chose to follow this heuristic and focus on the second decision problem: determining charger actions.
% %
% % \textbf{Charger actions.} 
% % We provide several policies with our simulator to contrast and compare with our proposed approach. 
% Charger action policies receive a state of the environment for a particular time and generate actions based on this. 
% % The simulator is stateless. Thus, it provides only a current representation of the world at that specific time to each policy. 
% We expand this idea and model the problem as an MDP.

% % \jpnote{Explain again that there are two decisions to be made; charger assignment and charging rates; but we use a heuristic to solve the charger assignment; we show quick results across the months in table 2 and we pick bidrection first and departure as our basic policy. Connect to next subsection on MDP}
\begin{figure*}[t]
    \centering
    \begin{subfigure}[b]{0.68\linewidth}
        % \centering
        \includegraphics[height=2.1in,keepaspectratio,trim=0.2cm 0.2cm 0.2cm 0.2cm,clip]{figures/framework.pdf}
        % \vspace{-0.1in}
        \caption{Reinforcement Learning Framework.}
       \label{fig:framework}
    \end{subfigure}
    % \hfill
    \begin{subfigure}[b]{0.3\linewidth}
        % \centering
        \includegraphics[height=2.1in,keepaspectratio,trim=0cm 0.5cm 0cm 0.5cm,clip]{figures/inference_pipeline.pdf}
        \caption{Pipeline for Inference.}
        \label{fig:pipeline}
    \end{subfigure}
    \caption{\color{black}(a) Our framework relies on daily samples and an estimated monthly peak power. We use RL, i.e., DDPG, and extend it with policy guidance and action masking, to learn a near-optimal policy. (b) At inference time, the model ingests data of connected cars, charger states, building power, and the estimated monthly peak power to make decisions.}
    \label{fig:framework_and_pipeline}
\end{figure*}

\subsection{Markov Decision Process Model}
\label{ssec:MDP}

We model the V2B problem as the following MDP. 

{\bf State.}
The complete state space for the problem can be described using features that capture historical, current, and future estimation at a given time $T_j$, which includes parameters for each vehicle, such as the current SoC, required SoC, departure time, and battery capacity for each EV, along with SoC boundaries across all chargers. Additionally, the current building power, time slot, day of the week, historical building power, and long-term peak power estimation value are included, resulting in approximately $100$ features. 
% While this state space is complete, it is not tractable to be used for the learning process. Therefore, 
We leverage domain-specific knowledge to abstract key information from these features, reducing the state space to the $37$ essential state elements.

These features are:
\textbf{1)} The current time slot, $T_j$. \textbf{2)} The current building power, denoted as ${B}(T_j)$. \textbf{3)} The power gap between the current building power and the estimated peak power for the billing period, given by $ \PrdPeak(T_j) - B(T_j)$, where $\PrdPeak(T_j)$ indicates the estimated peak power at $T_j$, initialized from a value derived from training data. This gap aids the RL model in estimating the optimal peak power for demand charge reduction. \textbf{4)} The mean peak building power over the previous 7 days, $\mu(B^H(T_j))$, where $B^H(T_j)$ represents the list of peak building power for the previous 7 days. \textbf{5)} {The variance of the peak building power over the previous 7 days, $\sigma^2(B^H(T_j))$, helps inform the model about the future building power use}. \textbf{6)} The day of the week for the current time slot, $T_j$, which helps the model distinguish daily patterns and enhance generalization. \textbf{7)} The number of EV arrivals up to time slot $T_j$, represented as $|\{V | V \in \mathcal{V}, A(V) \leq T_j \}|$ for tracking EV arrival status. \textbf{8)} The energy needed by each EV connected to a charger at time slot $T_j$, given by $[\PowerNeed(C_i, T_j)]_{C_i \in \mathcal{C}}$, which is initialized to $0$. This quantity represents the energy gap between required SoC ($\SOCR$) and current SoC ($\it{SOC}$) of the EV $V = \phi(C_i, T_j)$, defined as $\PowerNeed(C_i, T_j) = (\SOCR(V) - \SOC(V, T_j)) \times \text{CAP}(V)$.
    % \begin{equation} 
    %  \PowerNeed(C_i, T_j) = (\SOCR(V) - \SOC(V, T_j)) \times \text{CAP}(V)
    %  \label{eq:chargerstate}
    % \end{equation}
    % where $V=\phi(C_i, T_j)$ indicating the EV connected to $C_i$ at $T_j$. 
\textbf{9)} The remaining time until the departure of each EV connected to the chargers is given by $[\ReTime(C_i, T_j)]_{C_i \in \mathcal{C}}$, and is set to 0 when no cars are connected. Each term is computed as $\ReTime(C_i, T_j) = \DepartureTime(\phi(C_i, T_j)) - T_j$. 

% \begin{enumerate}[leftmargin=*]
%     \item The current time slot, $T_j$.
%     \item The current building power, denoted as ${B}(T_j)$.
%     \item The power gap between the current building power and the estimated peak power for the billing period, given by $ \PrdPeak(T_j) - B(T_j)$, where $\PrdPeak(T_j)$ indicates the estimated peak power at $T_j$, initialized from a value derived from training data. This gap aids the RL model in estimating the optimal peak power for demand charge reduction.
%     \item The mean peak building power over the previous 7 days, $\mu(B^H(T_j))$, where $B^H(T_j)$ represents the list of peak building powers from the previous 7 days. 
%     \item The variance of the peak building power over the previous 7 days, $\sigma^2(B^H(T_j))$. It informs the model about the future building power.  
%     \item The day of the week for the current time slot, $T_j$. It helps the model distinguish daily patterns and enhance generalization.
%     \item The number of EV arrivals up to time slot $T_j$, represented as $|\{V | V \in \mathcal{V}, A(V) \leq T_j \}|$ for tracking EV arrival status. 
%     \item The energy needed by each EV that is connected to a charger at time slot $T_j$, given by $[\PowerNeed(C_i, T_j)]_{C_i \in \mathcal{C}}$, and is initialized to $0$. This represents the energy gap between the required SoC and the current SoC of the EV connected to charger $C_i$ at time slot $T_j$, defined as: 
%     \begin{equation} 
%      \PowerNeed(C_i, T_j) = (\SOCR(V) - \SOC(V, T_j)) \times \text{CAP}(V)
%      \label{eq:chargerstate}
%     \end{equation}
%     where $V=\phi(C_i, T_j)$ indicating the EV connected to $C_i$ at $T_j$. 
%     \item The remaining time until the departure of each EV connected to the chargers is given by $[\ReTime(C_i, T_j)]_{C_i \in \mathcal{C}}$, and is set to 0 when no cars are connected. Each term is computed as $\ReTime(C_i, T_j) = \DepartureTime(\phi(C_i, T_j)) - T_j$. 
% \end{enumerate}

{\bf Actions.} We define the set of actions $\mathcal{A}$, which includes all actions at each time slot $T_j$ with $T_j \in \mathcal{T}$. In this MDP, $\mathcal{A}$ is continuous and specifies the power of all chargers at each time slot $T_j$, where $A(T_j) = [P(C_i, T_j)]_{C_i \in \mathcal{C}} $.
 

{\bf State Transition.} 
%The states of the building and chargers are updated based on actions taken at each time slot. 
% Building and charger states are updated based on actions and EV arrivals/departures at each time slot. To simulate state transitions, we designed an environment simulator that provides state features and manages these transitions. 
States are updated based on actions and EV arrivals/departures at each time slot. To simulate these transitions, we designed an environment simulator that provides and updates states. The state transition function is given as:
${\it Trans}(S(T_{j-1})$, $A(T_{j-1})) \mapsto S(T_j)$, with the following steps: 
\begin{enumerate}[leftmargin=*]
    \item Initialize the estimated peak power, $\PrdPeak(T_0)$, which can be derived from historical data 
    %\ad{does not make sense. We need to clearly describe the process of how this is estimated for training as well as inference. Please update. There is no mention in section 4.2} 
    (detailed in ~\Cref{sec:our_approach})
    , and update it by
    $
    \PrdPeak(T_{j}) = \max(\PrdPeak(T_{j-1})$, $ \Building(T_{j-1}) + \sum_{C_i \in \mathcal{C}} P(C_i, T_{j-1})),
    $
    which updates the estimated peak power depending on the previous estimate and the last peak power.
    \item Update SoC of EVs connected to all chargers: $\it{SOC}(\phi(C_i,T_{j}), T_{j})$ using action $A(T_{j-1})$ according to Equation~(\ref{eq: soc}). 
   \item Update the EV charger assignment $\phi(C_i, T_j)$ and $\eta(V)$ by first releasing chargers with departing EVs in the current time slot $T_j$ and then assigning new arrival EVs to idle chargers. 
   \item Update the energy requirement of all EVs connected to a charger: $[\PowerNeed(C_i, T_j)]_{C_i \in \mathcal{C}}$ by based on EV's current SoCs.
   \item Update the remaining time of all EVs connected to chargers: $[\ReTime(C_i, T_{j})]_{C_i \in \mathcal{C}}$ at time slot $T_{j}$.   
\end{enumerate}


{\bf Reward.} We define the function ${\it Reward}: \mathcal{S} \times \mathcal{A} \rightarrow \Re$, where ${\it Reward}(S(T_j), A(T_j))$ evaluates the reward for actions taken in a specific state, focusing on minimizing the total bill while satisfying SoC requirements. We express reward as $\lambda_{S} \cdot \mathit{r}_1 + \lambda_{E} \cdot \mathit{r}_2 + \lambda_{D} \cdot \mathit{r}_3$ where $\mathit{r}_1 =  \sum_{C_i\in\mathcal{C}} \max(0, \min(\PowerNeed(C_i, T_j), P(C_i, T_j) \times \delta))$, $\mathit{r}_2 = - P(C_i, T_j) \cdot \delta  \cdot \theta_E(T_j)$, and $\mathit{r}_3 = - \max(0, \Building(T_j) + \sum_{C_i \in \mathcal{C}} P(C_i, T_j) - \PrdPeak(T_j)) \cdot \theta_D$
% \begin{align*}
%    & \mathit{Reward}(S(T_j), A(T_j)) = \lambda_{S} \cdot \mathit{r}_1 + \lambda_{E} \cdot \mathit{r}_2 + \lambda_{D} \cdot \mathit{r}_3
% \end{align*}
.
% \begin{align*}
% \begin{aligned}
%     \mathit{r}_1 &=  \sum_{C_i\in\mathcal{C}} \max(0, \min(\PowerNeed(C_i, T_j), P(C_i, T_j) \cdot \delta)) \\
%     \mathit{r}_2 &= - P(C_i, T_j) \cdot \delta  \cdot \theta_E(T_j) \\
%     \mathit{r}_3 &= - \max(0, \Building(T_j) + \sum\limits_{C_i \in \mathcal{C}} P(C_i, T_j) - \PrdPeak(T_j)) \cdot \theta_D
% \end{aligned}
% \end{align*}
In this reward structure, $\mathit{r}_1$ promotes actions that charge EVs to reach their required SoC, as intended in Equation~(\ref{eq: soc_penalty}), while $\mathit{r}_2$ penalizes the energy cost for the charging actions taken. The third component, $\mathit{r}_3$, penalizes the increase in demand charges if peak power increases, aligning with our objective in Eq. (\ref{eq: billing}). These functions use three  coefficients, $\lambda_{S}$, $\lambda_{E}$, and $\lambda_{D}$ to balance trade-offs.
%between these reward factors.

% \subsection{MILP Solver}
% \label{subsec:milpsolver}



% When solving for a month, it has, on average, 20000 variables and 32000 constraints and takes 3.7 seconds to solve. However, the MILP solution is limited by its requirement to know the future completely, not accounting for any stochasticity in the future.



% We implement an optimization framework to generate optimal actions and use it to provide effective guidance to steer the search toward the global optimum. To get the optimal actions, we formulate the V2B problem using mixed-integer linear programming (MILP) and refer to the MILP framework as $\mathit{MILP(S(T_j), events_{future})}$. Here, $S(T_j)$ is the current state information, including the status of connected EVs, building's power, and $events_{future}$ covers all the future events from the input sample, which are EV arrivals and departures, building's power consumption, and electricity prices from the current time to the end of the billing period. The MILP solution provides the optimal power for each charger from the current time to the end of the billing period, maximizing the multi-objective weighted sum of the total cost (detailed in Equation~(\ref{eq: billing})) and penalties for missing SoC requirements (defined in Equation~(\ref{eq: soc_penalty})). 
% % It is designed to follow the multiple constraints related to the EV SoC update function following Equation~(\ref{eq: soc}) - (\ref{eq:building_power}).   

% As we discussed earlier, our problem deals with futures for up to a month, while some subproblems may need to be solved only for a day. 


 

% \begin{algorithm}[t]
\setcounter{AlgoLine}{0} %
\small
    \SetAlgoLined
    \KwIn{Initial policy parameters for actor network $\zeta_a$, critic parameters $\zeta_c$, target network parameters $\zeta_a', \zeta_c'$\\
Training parameters: $\mathit{actionNoise}$, $\policyGuidanceRate$, $\mathit{bufferSize}$, $\mathit{batchSize}$, maximum iterations: $M$,  training steps: ${\it trainStep}$; target network update steps: ${\it updateStep}$
}
    \KwOut{Trained policy $\pi_{\zeta_a}$}
    Initialize replay buffer $\Buffer$;  ${\it step}\gets 0$\\
    % Initialize target network weights $\zeta_a' \leftarrow \zeta_a$, $\zeta_c' \leftarrow \zeta_c$ \\
    \For{$1$ \KwTo $M$}{
        Input a sample into simulator to generate initial state $s_0$ 
        % Initialize a random process $N$ for action exploration 
        
        \For{each time slot $T_j \in \mathcal{T}$}{
        % \If{$T_j$ is during non-peak hours}{
        % Get action A(T_j) using h
        %}
        \tcp{Introducing policy guidance stochastically.}
            {\color{black} randomValue $\leftarrow randomBetween(0,1) $
            
            \If
            %(\tcp*[h]{Adding policy guidance stochastically}) $$
            { randomValue $\leq \policyGuidanceRate$}{
        Get action $A(T_j)$ by rerunning the MILP solver: $A(T_j)\leftarrow\MILP(S(T_j), {\it remainEpisode})$
                }
            \Else{
                % $A(T_j)\gets  \pi(S(T_j) | \zeta_a) + \mathit{actionNoise}$
                Get masked action using current policy,  $\mathit{actionNoise}$: \\
                $A(T_j) \gets \Mask\left(S(T_j), \pi(S(T_j) | \zeta_a)) + \mathit{actionNoise} \right)$
                % \tcp{Add action masking}
            }}
             State transition $S(T_{j+1})\leftarrow {\it Trans}(S(T_j), A(T_j))$ 
            %in Algorithm~\ref{alg: stateTrans}. 

            Get the action reward $R(T_j)\leftarrow {\it Reward}(S(T_j), A(T_j))$
            %by Algorithm~\ref{alg: reward}. 
            
            Store transition $(S(T_j), A(T_j), R(T_j), S(T_{j+1}))$ in $\Buffer$
            
            \If{${\it step} \bmod {\it trainStep} == 0$}{
            Sample batch $(S(T_i), A(T_i), R(T_i), S(T_{i+1})$ from $\Buffer$ 

           {\color{black} Get masked actions using target actor network:\\
            $A(T_{i+1})\leftarrow \Mask(S(T_{i+1}), \pi'(S(T_{i+1}) | \zeta_a'))$}  
            %using Algorithm~\ref{alg: mask}  \tcp{Add action masking}

            Set target $y_i\leftarrow R(T_j) + \gamma Q'(S(T_{i+1}), A(T_{i+1})
            | \zeta_c')$ 
            
            Update critic network by minimizing the loss: $L 
            \leftarrow \frac{1}{N} \sum_i (y_i - Q(S(T_i), A(T_i) | \zeta_c))^2$ 
            
             {\color{black} Get masked actions $A(T_i)$ at $S(T_i)$ using actor network: $A(T_i)\leftarrow \Mask( S(T_i), \pi(S(T_i) | \zeta_a))$ }%using Algorithm~\ref{alg: mask}.  
             % \tcp{Add action masking}
           % Mask action $A(T_i)\leftarrow \Mask(S(T_i), a_{i})$ 
           
            Update the actor policy using policy gradient:\\
            $\nabla_{\zeta_a} J \leftarrow \frac{1}{N} \sum_i \nabla_a Q(S(T_i), A(T_i) | \zeta_c) | \nabla_{\zeta_a} \pi(s | \zeta_a) |_{S(T_i)}$}
            
        \If{${\it step} \bmod {\it updateStep} == 0$}{
             Update the target networks: 
            $\zeta_a' \leftarrow \tau \zeta_a + (1 - \tau) \zeta_a'$;\quad 
            $\zeta_c' \leftarrow \tau \zeta_c + (1 - \tau) \zeta_c'$ \\
            }${\it step}\gets {\it step}+1$
        }
    }
    \caption{Improved DDPG with Action Masking and Policy Guidance.}
\label{alg:DDPG} 
\end{algorithm}
% \begin{algorithm}[t]
    % \SetAlgoNlRelativeSize{-1}
    \KwIn{$\textit{state: }S(T_j), \textit{action: } A(T_j) $}
    \KwOut{Masked action: $\MaskAction$} 
\small
Initializing: $\PowerNeed \gets [\PowerNeed(C_i, T_j)]_{C_i \in \mathbf{C}}$; \quad
$\ReTime \gets [\ReTime(\phi(C_i, T_j))]_{C_i \in \mathbf{C}};$
$\epsilon \gets 10^{-5}$; \quad
$C^{max} \gets [C^{max}_i]_{C_i \in \mathbf{C}}$; \quad
$C^{min} \gets [C^{min}_i]_{C_i \in \mathbf{C}}$

\tcp{Mask 1: Set action = 0 if no car is connected}
    $ \MaskAction \gets \frac{\ReTime}{\ReTime + \epsilon} \times A(T_j)$\
    
    \tcp{Mask 2: Stop charging when required SoC is reached for uni-directional chargers}
    $ \MaskAction_{tmp} \gets \MaskAction$; \quad
    $\MaskAction[\textit{uniIdx}] \gets \min(\MaskAction_{tmp}, \frac{\PowerNeed}{\delta})[\textit{uniIdx}]$
    % $  \gets \MaskAction_{tmp}$

    \tcp{Mask 3: Enforce charging to the req. SoC before departure. }
    $\overline{\Power(T_j)} \gets \frac{ \PowerNeed- (\ReTime - 1) \times C^{max} \times \delta }{\delta}$\
    $\overline{\Power(T_j)} \gets \min(\overline{\Power(T_j)}, C^{max})$; 
    $\MaskAction \gets \max(\MaskAction, \overline{\Power(T_j)})$\
    
    \tcp{Mask 4:  Bidirectional chargers discharge to req. SoC by departure.}
    $\Power^*(T_j) \gets \frac{\PowerNeed- (\ReTime - 1) \times C^{min} \times \delta }{\delta}$\
    $\Power^*(T_j) \gets \max(\Power^*_t, C^{min})$\
    
    $\MaskAction_{tmp}\gets \MaskAction$; \quad
    $ \MaskAction[\textit{biIdx}] \gets \min(\MaskAction_{tmp}, \Power^*_t)[\textit{biIdx}]$\

    \tcp{Mask 5: Power improvement strategy}
    $ \textit{powerGap} \gets \Building(T_j) - \PrdPeak(T_j)$\
    $ \textit{canIncrease} \gets \textit{ReLU}\left(\min\left(\frac{\PowerNeed}{\delta}, C^{max}\right) - \MaskAction \right)$
    
    $ \textit{toImprove} \gets \min\left(\textit{ReLU}(\textit{powerGap} - \sum \MaskAction), \sum \textit{canIncrease}\right)$
    
    $ \MaskAction \gets \MaskAction + \frac{\textit{toImprove} \times \textit{canIncrease}}{\sum(\textit{canIncrease}) + \epsilon}$

    \tcp{Mask 6: Do not discharge below building load}
    $ \textit{toImprove} \gets \max(-\Building(T_j) - \sum(\MaskAction), 0)$
    $ \textit{negAction} \gets \textit{ReLU}(\MaskAction \times -1) \times -1$
    
    % $ \textit{toIncrease} \gets \frac{\textit{toImprove} \times \textit{tmpAction}}{\sum(\textit{tmpAction}) + \epsilon}$\;
    $ \MaskAction \gets \MaskAction +  \frac{\textit{toImprove} \times \textit{negAction}}{\sum(\textit{negAction}) + \epsilon}$

    \caption{Action Masking: $\Mask(S(T_j), A(T_j))$.} 
    \label{alg: action_masking}
\end{algorithm} 

\subsection{Reinforcement Learning Approach}
\label{sec:RL}
In this section, we describe the entire reinforcement learning pipeline. We introduce the network structure, discuss how we use a simulator to gather state features and describe the different techniques, such as action masking and policy guidance, used to improve the performance of the V2B problem.  

To improve training efficiency, we address the challenge of long state-action sequences by splitting the monthly dataset into daily episodes. This allows the model to capture variations across different weekdays and learn more effectively from shorter episodes, adapting more quickly to daily changes. By incorporating estimated monthly peak power into the state features and reward function, the approach still accounts for monthly demand charges, helping to minimize long-term costs while staying aligned with our objective. 
% - make sure key hyperparameters are known.
% - make sure the confidence bound is explained as a hyperparameter.
% \jpnote{Will have to copy paste sections from~\Cref{sec:RL}. Talk about how the previous components are used to create our policy/model.}

%which is well-suited for continuous action spaces and supports off-policy training, allowing the model to learn from diverse experiences across various scenarios, thereby improving generalization.
%
\subsubsection{Enhanced Deep Deterministic Policy Gradient}
Our approach based on the DDPG framework~\cite{lillicrap2015continuous} uses an actor network for continuous actions.
During training, we interact with the simulator that provides state abstractions and transitions.
To improve RL performance in handling the limitations associated with large continuous action spaces and long-term reward optimization, we introduce action masking and policy guidance techniques. Details of the enhanced approach are in Algorithm~\ref{alg:DDPG} in the appendix. 
Action masking, denoted as $\Mask(S(T_j), A(T_j))$, refines the raw actions generated by the actor network by enforcing action validity and utilizing domain-specific knowledge, thereby improving policy performance. Additionally, policy guidance incorporates the MILP solver discussed earlier to provide optimal actions based on current and future information. These optimal actions are stochastically introduced during RL training into the replay buffer (i.e., tossing a biased coin) to mix high-quality actions given a deterministic trajectory with exploratory actions).  


\subsubsection{Action Masking}
Action masking ensures that the policy actions generated by the actor network are feasible during DDPG training. Findings from \cite{huang2020closer,kanervisto2020action} confirm that differentiable action masking does not interfere with the policy gradient backpropagation process. As a result, the learning process remains effective, while the imposed constraints on the action space prevent the policy from exploring invalid actions, thereby improving training efficiency and optimizing resource usage. 
\begin{algorithm}[t]
    % \SetAlgoNlRelativeSize{-1}
    \KwIn{$\textit{state: }S(T_j), \textit{action: } A(T_j) $}
    \KwOut{Masked action: $\MaskAction$} 
\small
Initializing: $\PowerNeed \gets [\PowerNeed(C_i, T_j)]_{C_i \in \mathbf{C}}$; \quad
$\ReTime \gets [\ReTime(\phi(C_i, T_j))]_{C_i \in \mathbf{C}};$
$\epsilon \gets 10^{-5}$; \quad
$C^{max} \gets [C^{max}_i]_{C_i \in \mathbf{C}}$; \quad
$C^{min} \gets [C^{min}_i]_{C_i \in \mathbf{C}}$

\tcp{Mask 1: Set action = 0 if no car is connected}
    $ \MaskAction \gets \frac{\ReTime}{\ReTime + \epsilon} \times A(T_j)$\
    
    \tcp{Mask 2: Stop charging when required SoC is reached for uni-directional chargers}
    $ \MaskAction_{tmp} \gets \MaskAction$; \quad
    $\MaskAction[\textit{uniIdx}] \gets \min(\MaskAction_{tmp}, \frac{\PowerNeed}{\delta})[\textit{uniIdx}]$
    % $  \gets \MaskAction_{tmp}$

    \tcp{Mask 3: Enforce charging to the req. SoC before departure. }
    $\overline{\Power(T_j)} \gets \frac{ \PowerNeed- (\ReTime - 1) \times C^{max} \times \delta }{\delta}$\
    $\overline{\Power(T_j)} \gets \min(\overline{\Power(T_j)}, C^{max})$; 
    $\MaskAction \gets \max(\MaskAction, \overline{\Power(T_j)})$\
    
    \tcp{Mask 4:  Bidirectional chargers discharge to req. SoC by departure.}
    $\Power^*(T_j) \gets \frac{\PowerNeed- (\ReTime - 1) \times C^{min} \times \delta }{\delta}$\
    $\Power^*(T_j) \gets \max(\Power^*_t, C^{min})$\
    
    $\MaskAction_{tmp}\gets \MaskAction$; \quad
    $ \MaskAction[\textit{biIdx}] \gets \min(\MaskAction_{tmp}, \Power^*_t)[\textit{biIdx}]$\

    \tcp{Mask 5: Power improvement strategy}
    $ \textit{powerGap} \gets \Building(T_j) - \PrdPeak(T_j)$\
    $ \textit{canIncrease} \gets \textit{ReLU}\left(\min\left(\frac{\PowerNeed}{\delta}, C^{max}\right) - \MaskAction \right)$
    
    $ \textit{toImprove} \gets \min\left(\textit{ReLU}(\textit{powerGap} - \sum \MaskAction), \sum \textit{canIncrease}\right)$
    
    $ \MaskAction \gets \MaskAction + \frac{\textit{toImprove} \times \textit{canIncrease}}{\sum(\textit{canIncrease}) + \epsilon}$

    \tcp{Mask 6: Do not discharge below building load}
    $ \textit{toImprove} \gets \max(-\Building(T_j) - \sum(\MaskAction), 0)$
    $ \textit{negAction} \gets \textit{ReLU}(\MaskAction \times -1) \times -1$
    
    % $ \textit{toIncrease} \gets \frac{\textit{toImprove} \times \textit{tmpAction}}{\sum(\textit{tmpAction}) + \epsilon}$\;
    $ \MaskAction \gets \MaskAction +  \frac{\textit{toImprove} \times \textit{negAction}}{\sum(\textit{negAction}) + \epsilon}$

    \caption{Action Masking: $\Mask(S(T_j), A(T_j))$.} 
    \label{alg: action_masking}
\end{algorithm} 

This procedure takes the RL raw action $A(T_j)$, an array of charging power $[P(C_i, T_j)]_{C_i\in\mathcal{C}}$ for all chargers, processes it through the following masking steps, and outputs the masked actions $A'$. Before starting the procedure, we need to obtain the following state features: the remaining power needed to reach the required SoC for all connected EVs ($\PowerNeed$), the time remaining for each EV ($\ReTime$), and the maximum ($C^{\max}$) and minimum ($C^{\min}$) power of all chargers (line 1 in Algorithm~\ref{alg: action_masking}). Also, for our case, since we work with both unidirectional and bidirectional, we denote ${\it uniIdx}$ and ${\it biIdx}$ as the indices for unidirectional and bidirectional chargers, respectively. All of the masking techniques referenced below are from Algorithm~\ref{alg: action_masking}.
% \jpnote{Shorten the descriptions and make it very concise.} 
\begin{itemize}[leftmargin=*]
    \item \textbf{Mask 1.} 
    We set the charging power $P(C_i, T_j)$ of charger $C_i$ to 0 if no EV is connected, i.e., $\ReTime(\phi(C_i, T_j))=0$. (line 2)
    %%%
     \item \textbf{Mask 2.} Overcharging unidirectional chargers is not beneficial since excess energy cannot be discharged. Thus, we limit the charging power to ensure the SoC of EVs connected to a unidirectional charger remains within their required SoC. 
    For each connected EV, the actions are masked to the minimum of the current charging power and the power needed to reach its required SoC $\left(\frac{\PowerNeed}{\delta}\right)$ (line 3). 
     
    \item \textbf{Mask 3.} 
    If necessary, we want to adjust actions such that it forces charging to the required SoC before departure to minimize missing SoC, as in Equation~(\ref{eq: soc_penalty}).
    We compute the critical power $\overline{\Power^*(T_j)}$, which is the minimum power required for all chargers at time $T_j$ to reach the required SoC of the connected EVs before departing (assuming maximum power $C^{max}$ is utilized in subsequent time slots). The raw action is adjusted if it falls below this value, especially in time slots leading up to the EV's departure (line 4).    
    \item \textbf{Mask 4.} 
    This mask is symmetrical to Mask 3 for force discharging.
    Overcharging bidirectional EVs is only advantageous if excess energy can be discharged during peak hours, but there is no benefit to overcharging just before departure. Using this mask, we force discharge EVs connected to bidirectional chargers, which have excess energy, and they reach the required SoC by departure.  Here, $\Power^*(T_j)$ denotes the minimum power to discharge for all chargers $C_i \in \mathcal{C}$ at time $T_j$ to guarantee EV can reduce to required SoC when departing (assuming the maximum discharging power $C^{min}$ is utilized subsequently) (lines 5, 6).    
    %%%
    \item \textbf{Mask 5.} 
    We increase charging power while ensuring the masked action stays within the estimated peak power $\PrdPeak(T_{j})$. This aims to charge EVs as much as possible towards their required SoC without raising demand charges, thereby avoiding forced charging just before departure, which could elevate peak power. 
    We calculate the ``power gap'' between estimated peak power and current building power, $\PrdPeak(T_j) - \Building(T_j)$. If the current power sum ($\Building(T_{j-1}) + \sum_{C_i \in \mathcal{C}} P(C_i, T_{j-1})$) is below this ``power gap'', we boost the current actions using the available ``power'' gap, constrained by $\min \left(\frac{\PowerNeed}{\delta}, C^{max}\right)$. (lines 7 to 9). 
    %%%
    \item \textbf{Mask 6.} We adjust the discharging power to prevent cumulatively discharging below the current building power $\Building(T_j)$, to satisfy Constraint~\ref{eq:building_power} by reducing the discharging power based on the current actions (lines 10 to 11).
\end{itemize} 
All of the action masking procedures utilize array computations and differentiable operations, such as ReLU \cite{rasamoelina2020review} and maximum/minimum operations, and the PyTorch framework~\cite{paszke2017automatic}. 

\subsubsection{Policy Guidance with MILP Solver}
Note that for a fixed sample, i.e., a fixed set of EV arrivals and departures, the V2B problem can be modeled as a single-shot mathematical program, i.e., a mixed-integer linear program (MILP), which can solved efficiently (at least, for our problem size) to retrieve the optimal actions. The objective of the MILP is maximizing the multi-objective weighted sum of the total rewards (detailed in Equations~\ref{eq: billing}, \eqref{eq: soc_penalty}), and the other properties of the V2B problem can be encoded as constraints. The fixed sample of arrivals and departures can be extracted from historical data. Naturally, this modeling paradigm does not solve the V2B problem in general---EV arrivals and departures are not known ahead of time---however, this strategy provides a set of optimal actions that the learning module can \textit{learn to imitate}. For our use case, the MILP problem can be solved reasonably fast. For example, for a planning horizon of a day with 15 cars, the problem size averages 800 variables and 1400 constraints and takes $0.05$ seconds to solve. 

We integrate a MILP solver based on CPLEX~\cite{cplex2009v12} as a policy guidance subroutine~\cite{pmlr-v28-levine13} in the RL training process. The solver, given the current state and future events, provides optimal charging actions.
{\color{black} Each training dataset contains complete episode data, enabling the MILP solver to account for future dynamics. During RL training, it generates optimal actions based on the current state and full future information of the episode (i.e., a full-month billing period).} The solver is stochastically triggered, and its outputs are added to the replay buffer with a predefined coefficient, $\policyGuidanceRate$ (see Algorithm~\ref{alg:DDPG} in the appendix). The next optimal action is computed as $\mathit{MILP(S(T_j), {\it remainEpisode})}$, considering factors such as EV arrivals, SoC requirements, and building power.
By blending MILP-generated actions with those from the RL actor network, the agent explores a more effective action space, improving its ability to handle large continuous action spaces and long-term rewards.% During training, the MILP solver is stochastically triggered to generate optimal actions based on the current state. These state-action transitions are added to the replay buffer with a predefined coefficient, $\policyGuidanceRate$ (see Algorithm~\ref{alg:DDPG} in the appendix). The next optimal action is obtained using $\mathit{MILP(S(T_j), {\it remainEpisode})}$, which accounts for remaining events like EV arrivals, SoC requirements, and building power. By blending MILP and RL actor network actions during training, the RL agent explores a more effective action space, improving performance in handling large continuous action spaces and maximizing long-term reward.

% \jpnote{Confirm with rishav that MILP above, talks about commented lines below}
%We add the optimal actions to the replay buffer, providing effective guidance to steer the search towards global optima. 
%
%
% We give the MILP solver the current state information, including the current EV status, charge usage, and all future events from the input sample (EV arrival/departure, building power flow, and electricity prices from the current time to the end of the billing period. The MILP solution provides the charging power for each charger from the current time to the end of the billing period, maximizing the multi-objective weighted sum of total cost (detailed in Equation~(\ref{eq: billing})) and penalties for missing SoC requirements (defined in Equation~(\ref{eq: soc_penalty})). Following multiple constraints related to the EV SoC update  function following Equation~(\ref{eq: soc}) and Constraints~(\ref{eq:charging_rate}) to (\ref{eq:building_power}). 
%
% During training, optimal actions are stochastically incorporated into the state transition process based on a predefined coefficient, denoted as $\policyGuidanceRate$, and stored in the replay buffer (see Algorithm~\ref{alg:DDPG}). We use $\mathit{MILP(S(T_j), {\it remainEpisode})}$ to get the next optimal action. Here, {\it remainEpisode} considers the remaining events in the fixed sample, such as upcoming EV arrivals, SoC requirements, and building power, to invoke the MILP solver. Using the MILP solver helps us maximize the long-term reward. 

\subsubsection{Actor-Critic Network Structure} 
Both the actor and critic networks are fully connected, having two hidden layers with 96 neurons each. Both feature a ReLU activation layer at the end. The critic network outputs a single Q-value estimate, while the actor network outputs the action, which represents the charging power of each charger.
%
To enhance convergence and improve generalization, we normalize all state variables to be within $[0, 1]$ before feeding them into neural networks. Time slot $T_j$ is normalized by division with the number of time slots in a day ($\frac{24}{\delta}$), while power-related variables such as building power $\Building(T_j)$, estimated peak power $\PrdPeak(T_j)$ are scaled by their respective statistical values from training data. Furthermore, we normalize the energy capacity $CAP(V)$ of each car by division with the maximum capacity among EVs, $\max(CAP(V))$.
For the action $A(T_j)=[P(C_i, T_j)]_{C^{i}\in \mathcal{C}}$, we constrain the output within the range $[-1, 1]$ using the $\tanh$ activation function. It is finally translated into the 
charging power range $[C_i^{min}, C_i^{max}]$ by scaling the value using a constant factor.

%We normalize the action values to the range of $[-1, 1]$ based on the charging power range $[C_i^{min}, C_i^{max}]$.
% by: 
% \begin{equation}
%     \hat{P}(C_i, T_j)=\frac{2\times(P(C_i, T_j)-C_i^{min})}{C_i^{max} - C_i^{min}}-1. 
% \label{eq: normalize}
% \end{equation} 
%The output of the actor network is constrained within the range $[-1, 1]$ using a $\tanh$ activation function, from which the unnormalized charging values are retrieved.
% The original charging power values can be obtained by computing the inverse of the normalization equation in Equation~(\ref{eq: normalize}). 

\subsubsection{Heuristics and Action Post Processing}

% for tractability the RL model focused on weekday and peak- billing period. For others including off-peaks and weekend we use heuristics. Here are the list of heuristics we use.  We will show later the results comparing to using heuristics for all periods as well.
% For tractability, the RL model focuses only on weekdays and peak-time billing periods. For off-peaks and weekends we use heuristics. We observe that off-peak hours offer lower electricity prices enabling EV charging at higher charging powers, also off-peak hours are not considered in the calculation of demand charge. Thus, heuristics can be used to optimize EV charging during these periods. Similarly, weekends often experience far fewer users than weekdays leading to low building power demand. Also, TSOs do not consider in the demand charge calculation, leading to minimal opportunities for optimization. During RL training, we use a charge first heuristic based on a least laxity task scheduling algorithm (Charge First LLF described in Section~\Cref{ssec:alternatives}) during off-peak hours and weekends that guarantees that EVs reach the required SoC before departure time. 

To enhance the ease of learning in this complex decision space, we use the RL model on weekdays and the peak hours of TOU price within each billing period (for both training and inference). For off-peak hours and weekends, we use a heuristic based on the least laxity task scheduling algorithm (described in \Cref{sec:experiments_and_results}) to ensure EVs achieve the required SoC before departure, calculating the minimum charge needed for each time slot. Off-peak hours offer lower electricity prices, allowing for higher EV charging rates, and are excluded from demand charge calculations, making heuristics effective for optimization. Similarly, weekends see fewer EV arrivals and lower power demand, with Transmission System Operators excluding them from demand charge assessments. 
% During training (and inference), we use 
% \jpnote{Cite paper for this}
% Finally, we condense the state features while accounting for the common minimum and maximum SoC boundaries of all EVs (with $\SOCMIN=0$ and $\SOCMAX=90\%$), we do not include SoC boundaries in the state representation, limiting the policy's ability to control actions. Thus, to maintain valid SoC boundaries, we apply a post-processing procedure~\cite{XXXX}, which differs from action masking. Post-processing is not integrated with the actor network and influenced by training backpropagation. This adjusts policy-generated actions before they are given to the environment, ensuring that charging EVs do not exceed $SoC^{\text{max}}$ and discharging prevent it to go below $SoC^{\text{min}}$. By employing this approach, we ensure that all policy-generated charging powers for charging EVs remain within the defined SoC boundaries, thereby satisfying Constraints~(\ref{eq:soc_min}) and~(\ref{eq:soc_max}).  
% Finally, we condense the state features while accounting for the common minimum and maximum SoC boundaries of all EVs, following the guidelines set by the EV manufacturer. To avoid limiting the policy's flexibility by including SoC boundaries in the state representation, we apply a post-processing procedure. This is a common approach in RL, utilized in AlphaGo~\cite{silver2016mastering}, where illegal actions are filtered before interacting with the environment to save training resources that would otherwise be spent on learning valid actions, which is not the final objective. Similarly, we adjust policy-generated actions before passing them to the environment, ensuring that EVs' SOC stays within the limits. Specifically, we stop discharging if the EV's SoC is below $SoC^{\text{min}}$, and stop charging if it exceeds $SoC^{\text{max}}$, satisfying Constraints~(\ref{eq:soc_min}) and~(\ref{eq:soc_max}). 
% Following the guidelines set by the EV manufacturer, we need to limit charging EVs to their SoC boundaries. We apply a post-processing procedure, a common approach in RL, utilized in AlphaGo~\cite{silver2016mastering}, where actions are adjusted to avoid illegal actions before interacting with the environment. This saves training resources that would otherwise be spent on learning valid actions, which is not the final objective. Similarly, considering the exceeding SoC boundary rarely happens during training and learning the validity of matining SoC limit is not our final objective which is to minimize total bill,  we adjust policy-generated actions before passing them to the environment, ensuring that the EVs' SoC stays within the specified limits. Specifically, we stop discharging if the EV's SoC is below $SoC^{\text{min}}$, and stop charging if it exceeds $SoC^{\text{max}}$, thereby satisfying Constraints~(\ref{eq:soc_min}) and~(\ref{eq:soc_max}).  
Following the EV manufacturer guidelines, we limit charging to SoC boundaries by clipping the actions of the learned policy within $[SoC^{\text{min}}, SoC^{\text{max}}]$ through post-processing to satisfy Constraints~(\ref{eq:soc_min}) and~(\ref{eq:soc_max})
% In this work, we assume all EVs share common minimum and maximum SoC limits. A post-processing procedure, similar to AlphaGo~\cite{silver2016mastering}, adjusts actions to prevent illegal steps during environmental interaction, conserving training resources. Since exceeding SoC boundaries is rare and maintaining SoC limits is not our main goal, which is minimizing the total bill, we adjust policy-generated actions after they are passed to the environment simulator. Specifically, we stop discharging if SoC is outside $[SoC^{\text{min}},SoC^{\text{max}}]$.

\subsection{Inference}
During execution, our RL-based policy, which is a trained actor network with the action masking procedure, operates at $\delta$ time intervals to determine the charging power for all chargers. At each time slot, the state features are generated from data captured from the environment, including charger status (connected EV's current SoC, expected departure time, and SoC), the building's current power and charging rate limits.
% , which are obtained through the simulator interface. 
% Additionally, we apply the heuristic approach: Fast Charge LLF (detailed in \Cref{ssec:results}) during off-peak hours and weekends. 
While we use the estimated peak power $\hat{P}^{max}$ as the state feature based on training samples, as shown in~\Cref{fig:pipeline}, it can be replaced by any data-driven forecasting or prediction model. Then, we input all the normalized state features, as described in~\Cref{ssec:MDP}, into the trained RL model to get the charging actions for the next time interval.

%{which can be derived from historical data or machine learning-based predictive models}.


% \subsection{Inference in Real-time}
% \jpnote{Write this in terms of how this will be used in the real world. Reference the digital twin.}
% write about inference diagram from workflow. we use the digital twin to track what the real-system state. We take a full chain of a month; we use the best heuristic for non-peak and weekend and for weekday and peak period we use RL policy. 


% \section{\label{approach}Dualguard MPPI}

As discussed in Sec. \ref{background_mppi}, MPPI solves the optimal control problem in \eqref{eq:opt_problem} using a sampling-based method. While MPPI may encourage the satisfaction of safety constraints in \eqref{eq:opt_problem} via penalizing safety violations in the cost function, ensuring safety constraints remains challenging. Moreover, by construction, the high-cost safety breaching sequences are mostly ignored during the optimal control sequence computation, wasting computational resources that could have been used to refine the system performance further. While the safety filtering mechanism discussed in \ref{background_lr} can provide a safety layer after MPPI to enforce the safety constraint, this approach is myopic in nature and might lead to performance impairment in favor of safety.

To overcome these challenges, we propose DualGuard MPPI, a two-layered safety filtering approach. First, the safety filtering is incorporated during the sampling process itself, where a least restrictive filter is applied along the sampled control sequence rollouts using a pre-computed safety value function. This ensures that all hallucinations satisfy the safety constraint and can contribute to performance optimization. To ensure that the resultant optimal control sequence also satisfies the safety constraint, we filter the output optimal control sequence by a safety filter as well. In addition to ensuring the safety constraints, the proposed framework leads to an increased sample efficiency as the safe hallucinations keep all sampled trajectories safe and relevant to the performance objective, thereby avoiding "sample wastage" due to safety constraint violation. The proposed algorithm is presented in Alg. 1. Details on the proposed filtering stages are discussed next.

\begin{figure}[t] 
\begin{center} 
\vspace{0.0em}
\includegraphics[width=0.9\columnwidth]{fig/alg_mppi.png}
\vspace{-1em}
\end{center}
\end{figure}
%
\subsection{\label{safe_hallucinations}Generating Safe hallucinations}
%
As in classical MPPI, we consider $K$ sequences of random perturbations $\delta_j^k$, that modify a nominal control sequence $u_j$. The perturbed sequences are applied to the dynamic model of the system while being filtered at each time step along the horizon, using  the LRF in eq. (\ref{eq:lst_restrict_safety_ctrl}), resulting in hallucinated trajectories that are guaranteed to maintain safety. The cost-to-go $S^k$ for the filtered trajectory is calculated over the safety-filtered control perturbation sequence $\Delta_j^k$.
% obtained as the difference between the filtered control and the nominal sequence.

The nominal control sequence $u_j$, the filtered control perturbations $\Delta_j^k$ and the cost over the safe hallucinated trajectories $S^k$ are used to calculate the optimal control sequence $u^*_j$ using the update rule defined in (\ref{eq:update_law}). As we weigh over controls that only produce safe trajectories, all K sequences contribute to the performance optimization, reducing the variance of MPPI algorithm and leading to better performance.

\subsection{\label{output_filter}Output Least restrictive filtering}

Even though the optimal control sequence $u^*_j$ was obtained by weighing control perturbations that individually resulted in safe trajectories, the safety guarantees may not hold for $u_j^*$. For this reason, before applying the first control in the sequence $u^*_0$ to the system, we perform one last LRF step to guarantee the safe operation of the complete system. The rest of the sequence is used as the nominal control for the next time step, and the entire process is repeated.

One case that encourages this last filtering stage corresponds to scenarios where the safe hallucinations present multimodality. We can imagine a vehicle trying to swerve around an obstacle by turning right or left; even if both swerving maneuvers result in safe behavior, a weighting over hallucinations split between these two modalities could result in maintaining a straight trajectory, causing a collision. Such issues are resolved by this additional filtering stage which picks one of the safe modes.

\begin{mdframed}[style=MyFrame,nobreak=false]

\textbf{Running example \textit{(Safe Planar Navigation)}:}

Considering our running example we observe the proposed steps of DualGuard MPPI in Fig.~\ref{fig:safe_mppi_steps}. First, we visualize the unmodified hallucination step in the left panel. Next, using the same control perturbations, we show how the safe hallucination step renders all the samples safe. Hallucinated trajectories are presented in a blue-to-red scale, corresponding to the associated low-to-high costs.

The right panel of Fig.~\ref{fig:safe_mppi_steps} shows how the safe hallucinations might present themselves in a multimodal fashion for some obstacle configurations; this indicates a possible failure mode where the control sequence given by the update rule in \eqref{eq:update_law} could drive the vehicle directly into the obstacles, leading to safety violations. The proposed output filtering stage safeguards against this possibility.  

\vspace{1em}
{\centering      \includegraphics[width=1.0\columnwidth]{fig/safe_mppi_steps_v2.png}
      \captionof{figure}{Unmodified hallucinations (Left). Safe hallucinations (Center). Possible multimodality on safe hallucinations encourages the use of output least restrictive filtering (Right).}
      \label{fig:safe_mppi_steps}\par} 

\end{mdframed}






% % \begin{algorithm}[t]
% \setcounter{AlgoLine}{0} %
% \small
%     \SetAlgoLined
%     \KwIn{Initial policy parameters for actor network $\zeta_a$, critic parameters $\zeta_c$, target network parameters $\zeta_a', \zeta_c'$\\
% Training parameters: $\mathit{actionNoise}$, $\policyGuidanceRate$, $\mathit{bufferSize}$, $\mathit{batchSize}$, maximum training iterations: $M$}
%     \KwOut{Trained policy $\pi_{\alpha}$}
%     Initialize replay buffer $\Buffer$ \\
%     % Initialize target network weights $\zeta_a' \leftarrow \zeta_a$, $\zeta_c' \leftarrow \zeta_c$ \\
%     \For{$1$ \KwTo $M$}{
%         $s_0$ \gets initial state from the simulator 
%         % Initialize a random process $N$ for action exploration 
        
%         \For{each time slot $T_j \in \mathcal{T}$}{
%         % \If{$T_j$ is during non-peak hours}{
%         % Get action A(T_j) using h
%         %}
%         \tcp{Introducing policy guidance stochastically.}
%             {\color{black} randomValue $\leftarrow randomBetween(0,1) $
            
%             \If
%             %(\tcp*[h]{Adding policy guidance stochastically}) $$
%             { randomValue $\leq \policyGuidanceRate$}{
%         Get action $A(T_j)$ by rerunning the ILP solver: $A(T_j)\leftarrow\MILP(S(T_j), {\it remainEpisode})$
%                 }
%             \Else{
%                 % $A(T_j)\gets  \pi(S(T_j) | \zeta_a) + \mathit{actionNoise}$
%                 Get masked action $A(T_j) \gets \Mask\left(S(T_j), \pi(S(T_j) | \zeta_a)) + \mathit{actionNoise} \right)$ by current policy and $\mathit{actionNoise}$  
%                 % \tcp{Add action masking}
%             }}
%              State transition $S(T_{j+1})\leftarrow {\it Trans}(S(T_j), A(T_j))$. 
%             %in Algorithm~\ref{alg: stateTrans}. 

%             Get the action reward $R(T_j)\leftarrow {\it Reward}(S(T_j), A(T_j))$. 
%             %by Algorithm~\ref{alg: reward}. 
            
%             Store transition $(S(T_j), A(T_j), R(T_j), S(T_{j+1}))$ in $\Buffer$

%             Sample a minibatch $(S(T_i), A(T_i), R(T_i), S(T_{j+1})$ from $\Buffer$ 

%            {\color{black} Get masked actions $A(T_{i+1})$ at $S(T_{i+1})$ using target actor network: $A(T_{i+1})\leftarrow \Mask(S(T_{i+1}), \pi'(S(T_{i+1}) | \zeta_a'))$}  
%             %using Algorithm~\ref{alg: mask}  \tcp{Add action masking}

%             Set target $y_i\leftarrow R(T_j) + \gamma Q'(S(T_{i+1}), A(T_{i+1})
%             | \zeta_c')$ 
            
%             Update critic network by minimizing the loss: $L 
%             \leftarrow \frac{1}{N} \sum_i (y_i - Q(S(T_i), A(T_i) | \zeta_c))^2$ 
            
%              {\color{black} Get masked actions $A(T_i)$ at $S(T_i)$ using actor network: $A(T_i)\leftarrow \Mask( S(T_i), \pi(S(T_i) | \zeta_a))$ }%using Algorithm~\ref{alg: mask}.  
%              % \tcp{Add action masking}
%            % Mask action $A(T_i)\leftarrow \Mask(S(T_i), a_{i})$ 
           
%             Update the actor policy by policy gradient: 
%             $\nabla_{\zeta_a} J \leftarrow \frac{1}{N} \sum_i \nabla_a Q(S(T_i), A(T_i) | \zeta_c) | \nabla_{\zeta_a} \pi(s | \zeta_a) |_{S(T_i)}$
            
%             Delayed Update the target networks: 
%             $\zeta_a' \leftarrow \tau \zeta_a + (1 - \tau) \zeta_a'$; 
%             $\zeta_c' \leftarrow \tau \zeta_c + (1 - \tau) \zeta_c'$ \\
%         }
%     }
%     \caption{Improved DDPG with Action Masking and Policy Guidance.}
% \label{alg:DDPG} 
% \end{algorithm}

%%%%%%%%%%%% ACTION MASKING %%%%%%%%%%%%
\begin{algorithm}[ht]
    \SetAlgoNlRelativeSize{-1}
    \KwIn{$\textit{state}, \textit{action: } A(T_j) $}
    \KwOut{Masked action: $\MaskAction$} 
\small
$\PowerNeed \gets [\PowerNeed(C_i, T_j)]_{C_i \in \mathbf{C}};$
$\ReTime \gets [\ReTime(\phi(C_i, T_j))]_{C_i \in \mathbf{C}};$
$\epsilon \gets 10^{-5};$
$C^{max} \gets [C^{max}_i]_{C_i \in \mathbf{C}};$
$C^{min} \gets [C^{min}_i]_{C_i \in \mathbf{C}};$

\tcp{Mask 1: Set action = 0 if no car is connected}
    $ \MaskAction \gets \frac{\ReTime}{\ReTime + \epsilon} \times A(T_j)$\;
    
    \tcp{Mask 2: Stop charging when required SoC is reached for uni-directional chargers}
    $ \MaskAction_{tmp} \gets \MaskAction$; 
    $\MaskAction_{tmp} \gets \min(\MaskAction_{tmp}, \frac{\PowerNeed}{\delta})$\;
    $ \MaskAction[\textit{uniIdx}] \gets \MaskAction_{tmp}[\textit{uniIdx}]$\;

    \tcp{Mask 3: Enforce charging to the required SoC before departure. }
    $\overline{\Power(T_j)} \gets \frac{ \PowerNeed- (\ReTime - 1) \times C^{max} \times \delta }{\delta}$\;
    $\overline{\Power(T_j)} \gets \min(\overline{\Power(T_j)}, C^{max})$\;
    $ \MaskAction \gets \max(\MaskAction, \overline{\Power(T_j)})$\;
    \tcp{Mask 4: Ensure bidirectional chargers discharge to the required SoC before departure.}
    $\Power^*(T_j) \gets \frac{\PowerNeed- (\ReTime - 1) \times P_{min} \times \delta }{\delta}$\;
    $\Power^*(T_j) \gets \max(\Power^*_t, P_{min})$\;
    
    $\MaskAction_{tmp}\gets \MaskAction$; 
    $ \MaskAction[\textit{biIdx}] \gets \min(\MaskAction_{tmp}, \Power^*_t)[\textit{biIdx}]$\;

    \tcp{Mask 5: Power improvement strategy}
    $ \textit{powerGap} \gets \Building(T_j) - \PrdPeak(T_j)$\;
    $ \textit{canIncrease} \gets \textit{RELU}\left(\max\left(\frac{\PowerNeed}{\delta}, C^{max}\right) - \MaskAction \right)$\;
    
    $ \textit{toImprove} \gets \min\left(\textit{RELU}(\textit{powerGap} - \sum \MaskAction), \sum \textit{canIncrease}\right)$
    
    $ \MaskAction \gets \MaskAction + \frac{\textit{toImprove} \times \textit{canIncrease}}{\sum(\textit{canIncrease}) + \epsilon}$\;

    \tcp{Mask 6: Do not discharge below building load}
    $ \textit{toImprove} \gets \max(-\Building(T_j) - \sum(\MaskAction), 0)$\;
    $ \textit{negAction} \gets \textit{RELU}(\MaskAction \times -1) \times -1$\;
    
    % $ \textit{toIncrease} \gets \frac{\textit{toImprove} \times \textit{tmpAction}}{\sum(\textit{tmpAction}) + \epsilon}$\;
    $ \MaskAction \gets \MaskAction +  \frac{\textit{toImprove} \times \textit{negAction}}{\sum(\textit{negAction}) + \epsilon}$\;

    \caption{Action Masking: $\Mask(S(T_j), A(T_j))$.} 
    \label{alg: action_masking}
\end{algorithm} 

% Based on the formulated MDP, we next use a RL approach to train a policy that maximizes long-term rewards by interacting with a custom environment simulator, which processes all data samples and handles state transitions.  
% \subsection{Environment Simulator}
% \jpnote{I will add this, just talk about how the transition is represented by the simulator. See Mike's paper.}



\section{Reinforcement Learning Policy}
\label{sec:RL}
% Figure~\ref{fig:framework_and_pipeline}. We input the the training data into our designed environment simulator, which handle the state generation and state transformations based on actions following our state transition function described in Section~\ref{ssec:MDP}.

% Based on the formulated MDP, we discuss the RL-based framework used for policy training to maximize the long-term reward of this MDP, as shown in \Cref{fig:framework}. This paper utilizes the Deep Deterministic Policy Gradient (DDPG) algorithm~\cite{lillicrap2015continuous} for policy training. DDPG is well-suited for handling continuous action spaces and supports off-policy training, enabling the model to learn from diverse experiences across various scenarios, thus improving generalization. 

% To further enhance policy performance, we integrate action masking and policy guidance techniques with DDPG. An overview of the RL-based framework shows that we input the training data into our designed environment simulator, which generates state transitions and transformations based on actions according to the state transition function described in Section~\ref{ssec:MDP}. In traditional DDPG, the actor network generates state and action trajectories through interaction with the environmental simulator, which are then stored in the replay buffer for training the critic and actor networks.  
% In this work, we enhance this process with a policy guidance approach. Instead of solely relying on the actor network for training trajectories, we incorporate a MILP solver to provide optimal actions based on current and future information, guiding the RL training away from local optima. Additionally, we implement an action masking procedure, represented as $\Mask(S(T_j), A(T_j))$, which refines the raw actions from the actor network by considering constraints for action validity and utilizing domain-specific knowledge to limit the exploration range of training actions and improve policy performance.  
% Our advanced DDPG method is described as follows. Algorithm~\ref{alg:DDPG} depicts our customized DDPG approach, incorporating action masking and policy guidance. This approach builds upon the classic DDPG algorithm by introducing policy guidance (lines 5 to 9) and action masking (lines 9, 14, and 17).
%Based on the formulated MDP, we present an RL-based framework for policy training aimed at maximizing long-term rewards, as illustrated in \Cref{fig:framework}. 
During training, data samples are input into our designed environment simulator \ad{as mentioned before we do not describe simulator enough, perhaps more information and statement that we will cite the simulator after blind review}, which provides the environment that abstracts state features for the RL models and manages state transitions based on the function described in Section~\ref{ssec:MDP}. The simulator operates based on actions (power rates) generated by the policies and dynamic events such as EV arrivals and departures. 
Our core approach is based on  Deep Deterministic Policy Gradient (DDPG) algorithm~\cite{lillicrap2015continuous}, which  is well-suited for continuous action spaces and supports off-policy training, allowing the model to learn from diverse experiences across various scenarios, thereby improving generalization. To enhance the policy performance, we integrate action masking and policy guidance techniques with DDPG as outlined in Algorithm~\ref{alg:DDPG}.  

Traditional DDPG relies on the actor network to generate actions (power rates in this work). The tuples of state, action, action reward, and next state are stored as state transitions in the replay buffer (see lines 9 to 12). In each training iteration, we batch state transitions from the replay buffer for model training (see line 13). Specifically, DDPG maintains target networks for both the actor and critic, which are used to generate the next state and compute Q-values essential for calculating the critic loss during training. The critic network is trained using gradient descent by minimizing the mean squared error between predicted Q-values and target Q-values derived from the Bellman equation (see lines 14 to 16). The critic learns Q-values for state-action pairs, which are then used to train the actor network through a policy gradient approach (see lines 17 and 18). These target networks are updated less frequently to stabilize the training process (see lines 18 and 19). 

To address the large state and action spaces in this RL model, we enhance DDPG by integrating an action masking procedure, denoted as $\Mask(S(T_j), A(T_j))$. This procedure refines the raw actions generated by the actor network by enforcing action validity and utilizing domain-specific knowledge, thereby improving policy performance. The action masking is applied after the actor network produces raw actions, acting as an additional layer (see lines 9, 14, 17). 
Additionally, we implement a policy guidance procedure (see lines 5 to 9) by incorporating a MILP solver to provide optimal actions through the function $\MILP(S(T_j), {\it remainEpisode})$, based on current and future information. These optimal actions are stochastically introduced during RL training into the replay buffer, mixing high-quality actions with the raw RL actions to enhance the training transition quality and guide the RL training in a beneficial direction. 

Next, we detail our RL framework by outlining the design of the environment simulator, the data normalization and network structure, the action masking and policy guidance procedures, and the heuristic approaches used to simplify our RL training. 
%which incorporates both policy guidance (lines 5 to 9) and action masking (lines 9, 14, and 17). 


% During training, optimal actions are stochastically incorporated into the process based on a predefined ratio coefficient, denoted as $\policyGuidanceRate$, and stored in the replay buffer. The function $\MILP(S(T_j), {\it remainEpisode})$ takes the current state as input and outputs the next optimal action. It can also access remaining events in the episode, including upcoming EV arrivals, SoC requirements, and building load, to invoke a MILP solver. This solver generates a sequence of optimal actions starting at timestamp $t$, returning the first action in the sequence, which serves as the optimal action for the current state to maximize long-term reward (lines 5 to 9).



\subsection{Actor-Critic Network Structure} 
%\jpnote{Add sentence description of the number of layers we are considering for the Actor Critic models}
To enhance convergence and improve generalization, we preprocess all state variables to be within the range of 0 and 1 before feeding them into neural networks. Timestamps are normalized by dividing the number of minutes in a day, while power-related variables such as building load $\Building(T_j)$, estimated peak power $\PrdPeak(T_j)$ are scaled by their respective statistical factors. Furthermore, we normalize the energy capacity $CAP(V_k)$ of each car by dividing it by the maximum capacity among EVs. 
For the action $A(T_j)=[P(C_i, T_j)]_{C^{i}\in \mathcal{C}}$. We normalize the action values to the range of $[-1, 1]$ based on the power rate range $[C_i^{min}, C_i^{max}]$ by Equation~(\ref{eq: normalize}).
\begin{equation}
    \hat{P}(C_i, T_j)=\frac{2\times(P(C_i, T_j)-C_i^{min})}{C_i^{max} - C_i^{min}}-1. 
\label{eq: normalize}
\end{equation}
% In DDPG, two neural networks are employed: the critic network evaluates state Q-values, while the actor network generates deterministic actions as the policy.
% The critic network processes state and action inputs through hidden layers (2 layers used in final model) with ReLU activation, outputting a single Q-value estimate.
Both critic and actor network have two hidden layers with 96 neurons each. Both feature a ReLU activation layer at the end. The critic network outputs a single Q-value estimate while the actor network outputs the action which represent the power rate of each charger.
% The exact size used in the model are described in~\Cref{tab:hyperparameters}.
% Simultaneously, the actor network maps the input state through multiple hidden layers (2 layers used in final model) with ReLU activations to produce the action output, representing the power rate of each charger.
This output is constrained within the range $[-1, 1]$ using a $\tanh$ activation function. The original charging power values can be obtained by computing the inverse of the normalization equation in Equation~(\ref{eq: normalize}).  \ad{this section needs clear revision. It should just tell how the network is structured and how data is normalized and input into the network. Also, clear description of the size and architecture is required.}
\jpt{I modified the text, it still needs the normalization part.}



\subsection{Action Masking}
\label{sec: DDPG} 

%\jpnote{Move the algorithm ahead of this and summarize your points while referring to the algorithm. You can add label commands to the lines of the algorithm to refer to it. Also rename your heuristic and constraints in the algorithm to just action masking N}

We propose action masking approach addresses the challenge of large continuous action spaces (detailed in Algorithm~\ref{alg: action_masking}), by ensuring that the policy actions generated by the actor network are valid and reasonable during DDPG training. This technique, based on findings from \cite{huang2020closer,kanervisto2020action}, confirms that differentiable action masking does not interfere with the policy gradient backpropagation process. As a result, the learning process remains effective, while the imposed constraints on the action space prevent the actor network from exploring invalid actions, thereby improving training efficiency and optimizing resource usage.

This procedure takes the RL raw action $A(T_j)$, an array of power rates $[P(C_i, T_j)]_{C_i\in\mathcal{C}}$ for all chargers, processes it through the following steps, and outputs the masked actions $A'$. Before computing, we first obtain the state features formatted as arrays: the remaining power needed to reach the required SoC for all connected EVs ($\PowerNeed$), their remaining time ($\ReTime$), and the maximum ($C^{\max}$) and minimum ($C^{\min}$) power rates of all chargers (see line 1). Also, we denote ${\it uniIdx}$ and ${\it biIdx}$ as the indices for unidirectional and bidirectional chargers, respectively.    
% \jpnote{Shorten the descriptions and make it very concise.} 
\begin{itemize}[leftmargin=*]
    \item \textbf{Mask 1.} 
    %We masks the action by setting the powe rate $P(C_i, T_j)$ of charger $C_i$ to 0 if no EV is plugged in. This method effectively masks all actions for idle chargers (where $\ReTime(\phi(C_i, T_j))=0$ indicates no EV is connected to charger $C_i$ at time slot $T_j$).
    We set the power rate $P(C_i, T_j)$ of charger $C_i$ to 0 if no EV is plugged in, effectively masking actions for idle chargers (where $\ReTime(\phi(C_i, T_j))=0$).
    %%%
    \item \textbf{Mask 2.} Overcharging unidirectional chargers is unbeneficial since excess energy cannot be discharged. Thus we limit the power rates to ensure the SoC of unidirectional chargers remains within required SoCs. 
    %by masking the actions using the minimum of the current power rate and the rate required for EVs to reach the required SoC after this time slot, calculated as $\PowerNeed/\delta$. 
    Actions are masked to the minimum of the current power rate and the rate needed for EVs to reach required SoC, calculated as $\PowerNeed/\delta$ (see lines 3 and 4).
    % The rationale behind this approach is that overcharging unidirectional chargers offers no benefit, as the excess energy cannot be discharged 
    %%%
    \item \textbf{Mask 3.} 
    %Action are masked to minimize the missing SoC for all EVs at the time of departure, as outlined in objective function~(\ref{eq: soc_penalty}).
    Actions are adjusted to forced charging to required SoC before departure if necessary, to minimize missing SoC as per objective function~(\ref{eq: soc_penalty}).  
    We compute the critical power rate $\overline{\Power^*(T_j)}$, representing the minimum required power rate for all chargers at time $T_j$ to their required SoC before departing (assuming maximum power is subsequently utilized). The raw action is adjusted if it falls below this rate, especially in final time slots (see lines 5 to 7).
    %At each time slot, we compute the critical power rate $\overline{\Power^*(T_j)}$, which represents the minimum power rate required for all chargers at time $T_j$ to ensure that all connected EVs reach the required SoC before departing (assuming maximum power is subsequently utilized). The raw action is adjusted if it falls below this critical power rate. 
    
    \item \textbf{Mask 4.} 
    Overcharging bidirectional EVs is only advantageous if excess energy can be discharged during peak hours; thus, there’s no benefit to overcharging just before departure. Thus, we adjust actions to prevent discharging EVs if the raw action would cause exceeding the required SoC before departure.  Here, $\Power^*(T_j)$ denotes the minimum power rate needed for all chargers $C_i \in \mathcal{C}$ at time $T_j$ (see lines 8 to 10).
    %We mask action to discharge EVs to reduce their SoC if the raw action would result in exceeding the required SoC before departure. The intuition behind this is that overcharging EVs in bidirectional chargers is advantageous only when the excess energy can be discharged during peak hours to mitigate demand charges; thus, there is no benefit to overcharging them just before departure. Here, $\Power^*(T_j)$ denotes the minimum power rate required for all chargers $C_i \in \mathcal{C}$ at time $T_j$ to ensure that all EVs connected to bidirectional chargers can discharge to the required SoC before leaving (see lines 8 to 10). 
    
    %%%
    \item \textbf{Mask 5.} 
    %We mask action to increase charging power rates while ensuring that the masked action remains within the estimated peak power limit. The goal is to encourage charging to the required SoC without raising the demand charge, thereby avoiding forced charging just before departure, which could elevate peak power. In this step, we calculate the power gap between the estimated peak power and the current building load, represented as $\PrdPeak(T_j) - \Building(T_j)$. If the current sum of action power is below this gap, we utilize the available power gap to enhance the current action by allocating the additional power needed. This allocation is constrained by the maximum power rate each entity can increase and the power required to reach the required SoC of the connected EVs, computed using ${\it canIncrease}$ (see lines 11 to 14). 
    We increase charging power rates while ensuring the masked action stays within the estimated peak power limit. This aims to charge EVs to their required SoC without raising demand charges, thereby avoiding forced charging just before departure, which could elevate peak power. 
    We calculate the power gap between estimated peak power and current building load, $\PrdPeak(T_j) - \Building(T_j)$. If the current power sum is below this gap, we enhance the action using the available power gap, constrained by the maximum rate each EV can increase to reach the required SoC, computed using ${\it canIncrease}$ (see lines 11 to 14). 
    %constrained by the maximum power increase and the required SoC of connected EVs (see lines 11 to 14).
    
    %%%
    \item \textbf{Mask 6} We adjust the discharging power rate to prevent discharging below the current building load $\Building(T_j)$ by increasing negative actions based on their current values (see lines 15 to 17). 
\end{itemize} 

All above action masking procedures utilize array computations and differentiable operations, such as ReLU \cite{rasamoelina2020review} and maximum/minimum operations, from the PyTorch library \cite{paszke2017automatic}. 







 % We then employ \textbf{Heuristic 2}, which ensures that all EVs' charging needs are met by enforcing minimum power rates $P(C_i, T_j)$ when the remaining time is insufficient to reach the required SoC ${\it SoC}^{\text{req}}(v)$ using the maximum power rate.

% We then employ \textbf{Heuristic 2}, which ensures that all EVs reach their required SoC at departure. This is achieved by enforcing maximum power rates \( P(C_i, T_j) \) when the remaining time is insufficient to reach the required SoC $ {\it SoC}^{\text{req}}(v) $ using the maximum power rate. 
% This step ensures that all chargers charge the EVs to the required SoC before departure, if feasible. Additionally, this forced charging is applied only in the final time slots when it is required.

%We denote ${\it uniIdx}$ and ${\it biIdx}$ as the indices of the uni-directional and bi-directional chargers, respectively.   
%We introduce \textbf{Heuristic 3}, which bounds discharging to allow EVs to reduce their SoC if it exceeds the required levels. The intuition is that there is no benefit to overcharging in bidirectional chargers before departure.






\subsection{MILP Policy Guidance} 

To address the challenge of local optima in DDPG, we  integrate a policy guidance approach~\cite{pmlr-v28-levine13} into the RL training process. This aims to improve performance by providing optimal actions to guide the training toward better outcomes. We implement an optimization framework to generate optimal actions and add them to the replay buffer, providing effective guidance to steer the search towards global optima. Specifically, we formulate the V2B problem using mixed-integer linear programming (MILP). We give the MILP solver the current state information, including the current EV status, charge usage, and all future events from the input sample (EV arrival/departure, building load flow, and electricity prices from the current time to the end of the billing period. The MILP solution provides the power rate for each charger from the current time to the end of the billing period, maximizing the multi-objective weighted sum of total cost (detailed in Equation~(\ref{eq: billing})) and penalties for missing SoC requirements (defined in Equation~(\ref{eq: soc_penalty})). Following multiple constraints related to the EV SoC update  function following Equation~(\ref{eq: soc}) and Constraints~(\ref{eq:charging_rate}) to (\ref{eq:building_power}).   

% During training, optimal actions are stochastically incorporated into state transitions process based on a predefined ratio coefficient, denoted as $\policyGuidanceRate$, and stored in the replay buffer (as shown in Algorithm~\ref{alg: DDPG}. The function $\MILP(S(T_j), {\it remainEpisode})$ takes the current state as input and outputs the next optimal action. It can also access remaining events in the episode, including upcoming EV arrivals, SoC requirements, and building load, to invoke a MILP solver. This solver generates a sequence of optimal actions starting at timestamp $t$, returning the first action in the sequence, which serves as the optimal action for the current state to maximize long-term reward (lines 5 to 9). 
During training, optimal actions are stochastically incorporated into the state transition process based on a predefined coefficient, denoted as $\policyGuidanceRate$, and stored in the replay buffer (see Algorithm~\ref{alg:DDPG}). The function $\MILP(S(T_j), {\it remainEpisode})$ takes the current state as input and outputs the next optimal action. It considers remaining events in the episode, such as upcoming EV arrivals, SoC requirements, and building load, to invoke a MILP solver. This solver generates a sequence of optimal actions starting at time slot $T_j$ and returns the optimal action for the next time slot, serving to maximize the long-term reward. 

%SoC requirement,
% which ensure EVs must reach their required SoC before departure.
% \[
% SoC_{T_d(v)}(v) \geq So
% \]   
% Using the MILP solution, we can generate actions based on any state to maximize the long-term reward. 
% {\color{black} 
% The MILP aims to minimize the total energy cost, including both consumption and demand charges, while ensuring that all EVs meet their SoC requirements before departure. The objective function is:
% \[
% \min \Cost^{\it EN}(\mathcal{P}) + Cost^{\it DC}(\mathcal{P})
% \]
% as shown in Equations~\ref{eq: objective_1, eq: objective_2}. 
% where \( g_t \) represents the energy usage cost at time slot \( t \in \mathcal{T} \), \( P_{\text{max}} \) is the maximum power consumed during peak hours.  
% The model includes binary variables for charger assignments (\( a_{v,cp,s} \)), and continuous variables for charging rates (\( c^v_s \)) and battery levels (\( e^v_s \)). Key constraints ensure that: 
% \begin{enumerate}
% \item \textbf{Charger Assignments}: Each EV \( v \) is assigned to at most one charger \( cp \) at any time slot \( s \), and charger capacities are not exceeded, and is denoted by the assignment variable $a \in \mathcal{A}$.
% \[
%  \sum_{C_i \in \mathcal{C}} a_{v,i,s} \leq 1
% \]
        
% \item \textbf{SoC Requirements}: EVs must reach their required SoC before departure.
% \[
% SoC_{T_d(v)}(v) \geq SoC^(v)
% \]  
% \item \textbf{Energy Balance}: The energy in the EV battery evolves according to the charging rate, where $C_i$ is the charger attached to car $v$ indicated by Equation~(\ref{eq: SoC}). 
%     % \[
%     % SoC_{t+1}(v) = SoC_{t}(v) + P(C_i, T_j)
%     % \]
% \item \textbf{Demand Charges}: Demand charges are incorporated by modeling the maximum power consumption, as indicated by Equation~\eqref{eq: objective_2}.
% \end{enumerate}
% By solving this MILP using current state information and future events (e.g., EV arrivals/departures, building load, electricity prices), we generate optimal action sequences that maximize the long-term reward. These optimal actions are then stochastically introduced into the replay buffer during DDPG training, effectively guiding the RL agent towards global optima and improving training outcomes.
% } 
% The final DDPG approach, incorporating action masking and policy guidance, is depicted in Algorithm~\ref{alg: DDPG}. 
% {\color{black} This approach builds upon the classic DDPG algorithm by introducing policy guidance (lines 5 to 9) and action masking (lines 9, 14, and 17). 
 %This procedure ensures valid and efficient action selection, leading to improved training outcomes in DDPG. 
%The lines in {\color{black}black} highlight the enhancements over the classic DDPG. These colored the sections involve policy guidance and action masking. During DDPG training, optimal actions are stochastically introduced based on a predefined ratio coefficient $\policyGuidanceRate$ and stored in the replay buffer. We define the $\MILP(S(T_j))$ function to invoke the MILP solver and generate a sequential optimal action sequence starting at timestamp $t$, returning the first generated action which represents the optimal action for the current state (see lines 5 to 9). This procedure ensures both valid and efficient action selection, leading to improved training outcomes in DDPG.  


% {\color{red} Add text connecting two subsections. }
% \begin{itemize}[leftmargin=*]
%     \item {\bf Least Laxity First (LLF)}: Least Laxity First is a dynaimc priority driven algorithm designed for scheduling multiprocessor real time tasks~\cite{leung1989new}. Laxity or slack time refers to the amount of time a task can be delayed without causing it to miss its deadline. In the context of EV charging, we define laxity as the difference between the amount of time remaining for a car before it departs and the amont of time it takes to meet the user's required SoC at a constant rate of charge~\cite{xu2016dynamic}. Additionally, we limit the amount of cars charging at any given time interval by allocating a capacity at each step. The capacity is based on the difference between a set a power threshold and the current building load at that time. 
%     Only cars connected to chargers whose aggregate power rates fall within this limit are able to be charged for that time interval. LLF will provide trickle charge, charging the cars to the minimum required power to reach required SoC before departure, at each interval. 
% \end{itemize} 
\subsection{Heuristic Approach and Post Processing}
% \jpnote{Clarify what this section is, is this in the figure 2A?}
% We observe that off-peak hours typically feature lower electricity prices, allowing for charging EVs at a higher power rate without impacting the final demand charge. This insight leads us to maximize EV charging during off-peak hours, thereby reducing energy costs and alleviating pressure on charging during peak hours. To implement this strategy, we employ a greedy approach that directs the charging of all EVs to their required SoC at the maximum power rate. Additionally, discharging to the required SoC is performed when entering the off-peak duration after peak hours.
% We observe that off-peak hours typically offer lower electricity prices, enabling EV charging at a higher power rate without affecting the final demand charge. This insight allows us to maximize EV charging during off-peak hours, reducing energy costs and alleviating pressure on charging during peak periods. To implement this strategy, we adopt a greedy approach to manage off-peak charging, replacing the RL models. This approach directs the charging of all EVs until they reach their required SoC at the maximum power rate during off-peak hours. Additionally, discharging occurs when entering off-peak periods after peak hours if the current SoC of the EV exceeds the required level.  
% Additionally, we apply action post-processing to keep EV SoCs within valid boundaries by adjusting the power rate for stopping charging or discharging when they exceed $SoC^{\text{max}}(\phi(C_i, T_j))$ or drop below $SoC^{\text{min}}(\phi(C_i, T_j))$.  
We observe that off-peak hours offer lower electricity prices, enabling EV charging at higher power rates without affecting the final demand charge. This helps optimize EV charging during these periods, reducing energy costs and alleviating pressure during peak times. To implement this approach, we utilize a simple greedy method during off-peak hours, foregoing RL training during these times. EVs charge at maximum power until they reach their required SoC. Specifically, assuming the current time slot $T_j$ is within non-peak hours and EV $V' = \phi(C_i, T_j)$ is connected to charger $C_i$, if $\SOCR(V') < \SOC(V', T_j)$, then $P(C_i, T_j) = \min(C^{max}_i, (\SOCR(V') - \SOC(V', T_j)) \times {CAP}(V') / \delta)$. Discharging occurs if the SoC exceeds the required level, calculated as $P(C_i, T_j) = \max(C^{min}_i, (\SOCR(V') - \SOC(V', T_j)) \times {CAP}(V') / \delta)$.


To condense the state features while accounting for the common maximum and minimum SoC boundaries of all EVs (with $\SOCMIN=0$ and $\SOCMAX=90\%$), we do not include SoC boundaries in the state representation, limiting the policy's direct access to them for action control. To maintain valid SoC boundaries, we apply a post-processing procedure, which differs from action masking that is integrated with the actor network and influenced by training backpropagation. This action post-processing adjusts policy-generated actions before they are input to the environment, ensuring that charging stops when SoC exceeds $SoC^{\text{max}}$ and discharging halts if it drops below $SoC^{\text{min}}$. 
By employing this approach, we ensure that all policy-generated power rates for charging EVs remain within the defined SoC boundaries, thereby satisfying Constraints~(\ref{eq:soc_min}) and~(\ref{eq:soc_max}).
%This method allows the RL policy to concentrate on the objective of total bill reduction without needing to learn the validity of actions concerning SoC limits, thereby enhancing the effectiveness of the training process. 


\section{Experimental Evaluation}\label{section:experiments}
We already achieved our primary objective of deriving time-series-specific subsampling guarantees for DP-SGD adapted to forecasting.
Our main goal for this section is to investigate the trade-offs we discovered in discussing these guarantees.
In addition, we train common probabilistic forecasting architectures on standard datasets to verify the feasibility of training deep differentially private forecasting models while retaining meaningful utility.
The full experimental setup  is described in~\cref{appendix:experimental_setup}.
%An implementation will be made available upon publication.

\subsection{Trade-Offs in Structured Subsampling}

\begin{figure}
    \vskip 0.2in
    \centering
        \includegraphics[width=0.99\linewidth]{figures/experiments/eval_pld_deterministic_vs_random_top_level/daily_20_32_main.pdf}
        \vskip -0.3cm
        \caption{Top-level deterministic iteration (\cref{theorem:deterministic_top_level_wr}) vs top-level WOR sampling (\cref{theorem:wor_top_level_wr}) for $\numinstances=1$.
        Sampling is more private despite requiring more compositions.}
        \label{fig:deterministic_vs_random_top_level_daily_main}
    \vskip -0.2in
\end{figure}




For the following experiments, we assume that we have $N=320$ sequences, batch size $\batchsize = 32$, and noise scale $\sigma = 1$.
We further assume $L=10  (L_F + L_C) + L_F - 1$, so that 
the chance of bottom-level sampling a subsequence containing any specific element is 
$r=0.1$ when choosing $\numinstances = 1$ as the number of subsequences.
In~\cref{appendix:extra_experiments_eval_pld}, we repeat all experiments with a wider range of parameters.
All results are consistent with the ones shown here.

\textbf{Number of Subsequences $\bm{\numinstances}$.}
Let us begin with a trade-off inherent to bi-level subsampling:
We can achieve the same batch size $\batchsize$ with different $\numinstances$, each
leading to different top- and bottom-level amplification.
We claim that $\numinstances = 1$ (i.e., maximum bottom-level amplification) is preferable.
For a fair comparison, we compare our provably tight guarantee for $\numinstances=1$ (\cref{theorem:wor_top_level_wr})
with optimistic lower bounds for $\numinstances > 1$ (\cref{theorem:wor_top_wr_bottom_upper})
instead of our sound upper bounds (\cref{theorem:wor_top_level_wr_general}), i.e.,
we make the competitors stronger.
As shown in~\cref{fig:monotonicity_daily_main}, $\numinstances = 1$ only has smaller $\delta(\epsilon)$ for $\epsilon \geq 10^{-1}$ when considering a single training step.
However, after $100$-fold composition, $\numinstances = 1$ achieves smaller $\delta(\epsilon)$ even in $[10^{-3}, 10^{-1}]$ (see~\cref{fig:monotonicity_composed_daily_main}).
Our explanation is that $\numinstances > 1$ results in larger $\delta(\epsilon)$ for large $\epsilon$, i.e., is more likely to have a large privacy loss.
Because the privacy loss of a composed mechanism is the sum of component privacy losses~\cite{sommer2018privacy}, this is problematic when performing multiple training steps.
We shall thus later use $\numinstances=1$ for training.

%Intuitively, $\delta(\epsilon)$ can be interpreted as the probability that the log-likelihood ratio of $M_x$ and $M_{x'}$ (``privacy loss'') exceeds $\epsilon$.\footnote{For the formal relation between privay loss and privacy profiles, see~\cref{lemma:profile_from_pld} taken from~\cite{balle2018improving}}


\textbf{Step- vs Epoch-Level Accounting.}
Next, we show the benefit of top-level sampling sequences (\cref{theorem:wor_top_level_wr}) instead of deterministically iterating over them (\cref{theorem:deterministic_top_level_wr}), even though we risk privacy leakage at every training step.
For our parameterization and $\numinstances=1$, top-level sampling with replacement requires $10$ compositions per epoch.
As shown in~\cref{fig:deterministic_vs_random_top_level_daily_main}, the resultant epoch-level profile is nevertheless smaller, and remains so after $10$ epochs.
This is consistent with any work on DP-SGD (e.g., \cite{abadi2016deep}) that uses subsampling instead of deterministic iteration.

\textbf{Epoch Privacy vs Length.} In~\cref{appendix:extra_experiments_epoch_length} we additionally explore the fact that, if we wanted to use deterministic top-level iteration, 
the number of subsequences 
$\numinstances$ would affect epoch length.
As expected, we observe that composing many private mechanisms ($\numinstances=1$) is preferable to composing few much less private mechanisms ($\numinstances > 1$) 
when considering a fixed number of training steps.

\begin{figure}
    \vskip 0.2in
    \centering
        \includegraphics[width=0.99\linewidth]{figures/experiments/eval_pld_label_noise/daily_30_32_main.pdf}
        \vskip -0.3cm
        \caption{Varying label noise $\sigma_F$ for top-level WOR and bottom-level WR  (\cref{theorem:data_augmentation_general}) with $\sigma_C = 0, \numinstances=1$.
        Increasing $\sigma_F$ is equivalent to decreasing forecast length.
        }
        \label{fig:label_noise_daily_main}
    \vskip -0.2in
\end{figure}

\textbf{Amplification by Label Perturbation.}
Finally, because the way in which adding Gaussian noise to the context and/or forecast window 
amplifies privacy (\cref{theorem:data_augmentation_general}) 
may be somewhat opaque, let us consider top-level sampling without replacement, bottom-level sampling with replacement,
$\numinstances=1$, $\sigma_C=0$, and varying label noise standard deviations $\sigma_F$. 
As shown in~\cref{fig:label_noise_daily_main}, increasing $\sigma_F$ has the same effect as letting the forecast length $L_C$ go to zero, i.e., eliminates the risk of leaking private information if it appears in the forecast window.
Of course, this data augmentation 
will have an effect on model utility, which we investigate shortly.

\begin{figure*}
\centering
\vskip 0.2in
    \begin{subfigure}{0.49\textwidth}
        \includegraphics[]{figures/experiments/eval_pld_monotonicity_composed/daily_20_32_1_main.pdf}
        \caption{Training step $1$}\label{fig:monotonicity_daily_main}
    \end{subfigure}
    \hfill
    \begin{subfigure}{0.49\textwidth}
        \includegraphics[]{figures/experiments/eval_pld_monotonicity_composed/daily_20_32_100_main.pdf}
        \caption{Training step $100$}\label{fig:monotonicity_composed_daily_main}
    \end{subfigure}\caption{
    Top-level WOR and bottom-level WR sampling under varying number of subsequences.
    Under composition, even optimistic lower bounds (\cref{theorem:wor_top_wr_bottom_upper}) 
    indicate worse privacy for $\numinstances > 1$ than 
    our tight upper bound for $\numinstances=1$ (\cref{theorem:wor_top_level_wr}).}
    \label{fig:monotonicity_daily_main_container}
\vskip -0.2in
\end{figure*}


\subsection{Application to Probabilistic Forecasting}
While the contribution of our work lies in formally analyzing the privacy of DP-SGD adapted to forecasting, 
training models with this algorithm can serve as a sanity-check to verify that the guarantees are sufficiently strong to retain meaningful utility under non-trivial privacy budgets.


\begin{table}[b]
\vskip -0.38cm
\caption{Average CRPS on \texttt{traffic} for $\delta=10^{-7}$. Seasonal, AutoETS, and models with $\epsilon=\infty$ are without noise.}
\label{table:1_event_training_traffic_main}
\vskip 0.18cm
\begin{center}
\begin{small}
\begin{sc}
\begin{tabular}{lcccc}
\toprule
Model & $\epsilon = 0.5$ & $\epsilon = 1$ & $\epsilon = 2$ &  $\epsilon = \infty$ \\
\midrule
SimpleFF & $0.207$ & $0.195$ & $0.193$ & $0.136$ \\ 
DeepAR & $\mathbf{0.157}$ & $\mathbf{0.145}$ & $\mathbf{0.142}$ & $\mathbf{0.124}$ \\
iTransf. & $0.211$ & $0.193$ & $0.188$ & $0.135$ \\
DLinear & $0.204$ & $0.192$ & $0.188$ & $0.140$ \\
\midrule
Seasonal   & $0.251$ & $0.251$ & $0.251$ & $0.251$\\
AutoETS   & $0.407$ & $0.407$ & $0.407$ & $0.407$\\
\bottomrule
\end{tabular}
\end{sc}
\end{small}
\end{center}
\vskip -0.1in
\end{table}

\textbf{Datasets, Models, and Metrics.}
We consider three standard benchmarks: \texttt{traffic}, \texttt{electricity}, and \texttt{solar\_10\_minutes} as used in~\cite{Lai2018modeling}.
We further consider four common architectures: 
A two-layer feed-forward neural network (``SimpleFeedForward''), a recurrent neural network (``DeepAR''~\cite{salinas2020deepar}),
an encoder-only transformer (``iTransformer''~\cite{liu2024itransformer}), and a refined feed-forward network proposed to compete with attention-based models (``DLinear''~\cite{zeng2023transformers}).
We let these architectures parameterize elementwise $t$-distributions to obtain probabilistic forecasts.
We measure the quality of these probabilistic forecasts using continuous ranked probability scores (CRPS), which we approximate via mean weighted quantile losses (details in~\cref{appendix:metrics}).
As a reference for what constitutes ``meaningful utility'', we compare against seasonal na\"{i}ve forecasting and exponential smoothing (``AutoETS'') without introducing any noise.
All experiments are repeated with $5$ random seeds.


\textbf{Event-Level Privacy.} \cref{table:1_event_training_traffic_main} shows CRPS of all models on the \texttt{traffic} test set 
when setting $\delta=10^{-7}$, and training on the training set until reaching a pre-specified $\epsilon$
with $1$-event-level privacy. For the other datasets and standard deviations, see~\cref{appendix:privacy_utility_tradeoff_event_level_privacy}.
The column $\epsilon=\infty$ indicates non-DP training.
As can be seen, models can retain much of their utility and outperform the baselines, even for $\epsilon \leq 1$ which is generally considered a small privacy budget~\cite{ponomareva2023dp}.
For instance, the average CRPS of DeepAR on the traffic dataset is $0.124$ with non-DP training and $0.157$ for $\epsilon=0.5$.
Note that, since all models are trained using  our tight privacy analysis,
which specific model performs best  on which specific dataset is orthogonal to our contribution. 

\textbf{Other results.}
In~\cref{appendix:privacy_utility_tradeoff_user_level_privacy} we additionally train with $w$-event and $w$-user privacy.
In~\cref{appendix:privacy_utility_tradeoff_label_privacy}, we demonstrate that label perturbations can offer an improved privacy--utility trade-off. 
All results confirm that our guarantees for DP-SGD adapted to forecasting are strong enough to enable provably private training while retaining utility.


\section{Conclusion}
\label{sec:Conclusion}
In this paper, we proposed a complete real-time planning and control approach for continuous, reliable, and fast online generation of dynamically feasible Bernstein trajectories and control for FW aircrafts. The generated trajectories span kilometers, navigating through multiple waypoints. By leveraging differential flatness equations for coordinated flight, we ensure precise trajectory tracking. Our approach guarantees smooth transitions from simulation to real-world applications, enabling timely field deployment. The system also features a user-friendly mission planning interface. Continuous replanning  maintains the rajectory curvature 
$\kappa$ within limits, preventing abrupt roll changes.

Future works will include the ability to add  a higher-level kinodynamic path planner to optimize waypoint spatial allocation and improve replanning success, and enhancing the trajectory-tracking algorithm by refining the aerodynamic coefficient estimation. 

\section{Acknowledgement}

This material is based upon work sponsored by the National Science Foundation (NSF) under Award Numbers 1952011 and 2238815 and by Nissan Advanced Technology Center-Silicon Valley. Results presented in this paper were obtained using the Chameleon Testbed supported by the NSF. Any opinions, findings, conclusions, or recommendations expressed in this material are those of the authors and do not necessarily reflect the views of the NSF or Nissan. 
% \begin{acks}
% If you wish to include any acknowledgments in your paper (e.g., to 
% people or funding agencies), please do so using the `\texttt{acks}' 
% environment. Note that the text of your acknowledgments will be omitted
% if you compile your document with the `\texttt{anonymous}' option.
% \end{acks}

%%%%%%%%%%%%%%%%%%%%%%%%%%%%%%%%%%%%%%%%%%%%%%%%%%%%%%%%%%%%%%%%%%%%%%%%

%%% The next two lines define, first, the bibliography style to be 
%%% applied, and, second, the bibliography file to be used.

% \clearpage
\bibliographystyle{ACM-Reference-Format} 
\balance

\bibliography{main}
% \vfill\eject
% \bibliographystyle{plain}
% \bibliographystyle{unsrtnat} 
% \bibliographystyle{ACM-Reference-Format} 
 
\clearpage
\appendix
% \clearpage
%%%%%%%%%%%%%%%%%%%%%%%%%%%%%%%%%%%%%%%%%%%%%%%%%%%%%%%%%%%%%%%%%%%%%%%%%%%%%%%
%%%%%%%%%%%%%%%%%%%%%%%%%%%%%%%%%%%%%%%%%%%%%%%%%%%%%%%%%%%%%%%%%%%%%%%%%%%%%%%
% APPENDIX
%%%%%%%%%%%%%%%%%%%%%%%%%%%%%%%%%%%%%%%%%%%%%%%%%%%%%%%%%%%%%%%%%%%%%%%%%%%%%%%
%%%%%%%%%%%%%%%%%%%%%%%%%%%%%%%%%%%%%%%%%%%%%%%%%%%%%%%%%%%%%%%%%%%%%%%%%%%%%%%
\newpage
\appendix
\onecolumn

\section{Related Work} \label{app:related_work}
\textbf{Personalized Generation} 
Due to the considerable success of large text-to-image models \cite{ramesh2022hierarchical, ramesh2021zero, saharia2022photorealistic, rombach2022high}, the field of personalized generation has been actively developed. The challenge is to customize a text-to-image model to generate specific concepts that are specified using several input images. Many different approaches \cite{DB, TI, CD, svdiff, ortogonal, profusion, elite, r1e} have been proposed to solve this problem and can be divided into the following groups: pseudo-token optimization \cite{TI, profusion, disenbooth, r1e}, diffusion fune-tuning \cite{DB, CD, profusion}, and encoder-based \cite{elite}. The pseudo-token paradigm adjusts the text encoder to convert the concept token into the proper embedding for the diffusion model. Such embedding can be optimized directly \cite{TI, r1e} or can be generated by other neural networks \cite{disenbooth, profusion}. Such approaches usually require a small number of parameters to optimize but lose the visual features of the target concept. Diffusion fine-tuning-based methods optimize almost all \cite{DB} or parts \cite{CD} of the model to reconstruct the training images of the concept. This allows the model to learn the input concept with high accuracy, but the model due to overfitting may lose the ability to edit it when generated with different text prompts. To reduce overfitting and memory usage, lightweight parameterizations \cite{svdiff, r1e, lora} have been proposed that preserve edibility but at the cost of degrading concept fidelity. Encoder-based methods \cite{elite} allow one forward pass of an encoder that has been trained on a large dataset of many different objects to embed the input concept. This dramatically speeds up the process of learning a new concept and such a model is highly editable, but the quality of recovering concept details may be low. Generally, the main problem with existing personalized generation approaches is that they struggle to simultaneously recover a concept with high quality and generate it in a variety of scenes.

\textbf{Sampling strategies}
Much research has been devoted to sampling techniques for text-to-image diffusion models, focusing not only on personalized generation but also on image editing. In this paper, we address a more specific question: how can the two trajectories -- superclass and concept -- be optimally combined to achieve both high concept fidelity and high editability? The ProFusion paper \cite{profusion} considered one way of combining these trajectories (Mixed sampling), which we analyze in detail in our paper (see Section \ref{sec:mixed_sampling}) and show its properties and problems. In ProFusion, authors additionally proposed a more complex sampling procedure, which we observed to be redundant compared to Mixed sampling, as can be seen in our experiments (see Section \ref{sec:experiments}). In Photoswap \cite{photoswap}, authors consider another way of combining trajectories by superclass and concept, which turns out to be almost identical to the Switching sampling strategy that we discuss in detail in Section \ref{sec:switching_sampling}. We show why this strategy fails to achieve simultaneous improvements in concept reconstruction and editability. In the paper, we propose a more efficient way of combining these two trajectories that achieves an optimal balance between the two key features of personalized generation: concept reconstruction and editability.

\section{Training details} \label{sec:training-details}
The Stable Diffusion-2-base model is used for all experiments. For the Dreambooth, Custom Diffusion, and Textual Inversion methods, we used the implementation from \url{https://github.com/huggingface/diffusers}.

\textbf{SVDiff} We implement the method based on \url{https://github.com/mkshing/svdiff-pytorch}. The parameterization is applied to all Text Encoder and U-Net layers. The models for all concepts were trained for $1600$ using Adam optimizer with $\text{batch size} = 1$, $\text{learning rate} = 0.001$, $\text{learning rate 1d} = 0.000001$, $\text{betas} = (0.9, 0.999)$, $\text{epsilon} = 1e\!-\!8$, and $\text{weight decay} = 0.01$. 

\textbf{Dreambooth} All query, key, and value layers in Text Encoder and U-Net were trained during fine-tuning. The models for all concepts were trained for $400$ steps using Adam optimizer with $\text{batch size} = 1$, $\text{learning rate} = 2e\!-\!5$, $\text{betas} = (0.9, 0.999)$, $\text{epsilon} = 1e\!-\!8$, and $\text{weight decay} = 0.01$. 

\textbf{Custom Diffusion} The models for all concepts were trained for $1600$ steps using Adam optimizer with $\text{batch size} = 1$, $\text{learning rate} = 0.00001$, $\text{betas} = (0.9, 0.999)$, $\text{epsilon} = 1e\!-\!8$, and $\text{weight decay} = 0.01$. 

\textbf{Textual Inversion} The models for all concepts were trained for $10000$ steps using Adam optimizer with $\text{batch size} = 1$, $\text{learning rate} = 0.005$, $\text{betas} = (0.9, 0.999)$, $\text{epsilon} = 1e\!-\!8$, and $\text{weight decay} = 0.01$. 

\textbf{ELITE} We used the pre-trained model from the official repo \url{https://github.com/csyxwei/ELITE} with $\lambda=0.6$ and inference hyperparams from the original paper.

\clearpage
\section{Superclass and concept trajectory choice}\label{app:hyper_theta}

 \begin{wrapfigure}{r}{0.45\textwidth}
    % \centering
    \includegraphics[trim={3cm 10cm 3cm 10cm},clip,width=\linewidth]{imgs/mixed_noft_nosup.pdf}
    \caption{The Pareto frontiers for original Mixed sampling and Mixed sampling in the Superclass, NoFT, and Empty Prompt setups. Mixed NoFT and Mixed Empty Prompt configurations overlap with the Pareto frontier of the original mixed sampling, but primarily in regions associated with low image similarity, which compromises concept fidelity.} \label{fig:mixed_noft_ep}
    \vspace{-0.14in}
\end{wrapfigure}

There are multiple ways to define sampling with maximized textual alignment to the prompt. However, the arbitrary choice can harm the alignment between Base sampling~\ref{eq:concept_sampling} and the selected trajectory. We use the Sampling with superclass (\ref{eq:superclass_sampling}) as it's the default choice in the literature and guarantees the maximized alignment between noise predictions $\tilde{\varepsilon}_{\theta}(p^C)$ and $\tilde{\varepsilon}_{\theta}(p^S)$. 

The several natural ways to adjust Sampling with superclass can be presented by varying $\theta$ and $p^{S}$ in (\ref{eq:superclass_sampling}). We explore two additional options with decreased alignment with (\ref{eq:concept_sampling}): (1) NoFT -- weights of base model $\theta^{\text{orig}}$ instead fine-tuned weights, (2) Empty Prompt -- prompt without any reference to a concept, even to its superclass category, i.e. $p^{\hat{S}} = \textit{"with a city in the background"}$ instead of $p^{S} = \textit{"a backpack with a city in the background"}$.

To validate the robustness of our framework for sampling method selection, we employ the original experimental protocol, supplementing the results shown in Figures~\ref{fig:examples} and~\ref{fig:profusion-photoswap}. Our analysis of Figures~\ref{fig:multi-stage_noft_ep} and~\ref{fig:masked_noft_ep} reveals that trajectories generated under the NoFT and Empty Prompt configurations (second and third columns, respectively) maintain identical method ordering to those produced by Superclass sampling ((\ref{eq:superclass_sampling}), first column).

Notably, Figure~\ref{fig:mixed_noft_ep} shows that Empty Prompt configuration demonstrates weaker alignment with Base sampling compared to NoFT, particularly at higher values of the superclass guidance scale $\omega_{s}$. This divergence manifests as reduced concept fidelity for Empty Prompt under large $\omega_{s}$. These findings highlight a practical adjustment: prioritizing smaller $\omega_{s}$ values in Empty Prompt setup preserves concept fidelity without altering the framework’s core selection logic. 

A key limitation of increased misalignment is the gradual erosion of superclass category information from generated images, which can lead to semantically inconsistent outputs. For instance, Figure~\ref{fig:examples_noft_ep} illustrates how the Mixed Empty Prompt setup, despite the strong animal prior in Base sampling, can produce human-like features in an image of a cat described as \textit{"in a chef outfit"}. This suggests that when superclass information is weakened, the model may introduce unexpected visual artifacts, impacting the fidelity of the intended concept.

Concept sampling (\ref{eq:concept_sampling}) can also be adjusted to better capture a concept’s visual characteristics, further decoupling fidelity from editability. For example, this can be achieved by (1) using the weights of a highly overfitted model (e.g., DreamBooth) or (2) selecting a prompt that omits contextual details, such as $p^{\hat{C}} = \textit{"a photo of V*"}$ instead of $p^{C} = \textit{"a V* with a city in the background"}$. Combining superclass sampling under NoFT or Empty Prompt with Base sampling configured via (1) or (2) could enhance both image and text similarity. We leave this direction for future work.

\begin{figure}[b]
    \centering
    \includegraphics[trim={3cm 10cm 3cm 10cm},clip,width=0.32\linewidth]{imgs/multi-stage_original.pdf}
    \hfill
    \includegraphics[trim={3cm 10cm 3cm 10cm},clip,width=0.32\linewidth]{imgs/multi-stage_noft.pdf}
    \hfill
    \includegraphics[trim={3cm 10cm 3cm 10cm},clip,width=0.32\linewidth]{imgs/multi-stage_nosup.pdf}
    \caption{Pareto Frontier Curves for Mixed, Switching, and Multi-Stage Sampling Methods in the Superclass, NoFT and Empty Prompt setups.
The NoFT and Empty Prompt configurations (second and third columns, respectively) preserve the same method ordering as those produced by Superclass sampling (first column).} \label{fig:multi-stage_noft_ep}
\end{figure}
\begin{figure}[t]
    \centering
    \includegraphics[trim={3cm 10cm 3cm 10cm},clip,width=0.32\linewidth]{imgs/masked_profusion.pdf}
    \hfill
    \includegraphics[trim={3cm 10cm 3cm 10cm},clip,width=0.32\linewidth]{imgs/masked_noft.pdf}
    \hfill
    \includegraphics[trim={3cm 10cm 3cm 10cm},clip,width=0.32\linewidth]{imgs/masked_nosup.pdf}
    \caption{Pareto Frontier Curves for Mixed, Switching, Masked, and ProFusion Sampling Methods in the Superclass, NoFT, and Empty Prompt setups.
The NoFT and Empty Prompt configurations (second and third columns, respectively) preserve the same method ordering as those produced by Superclass sampling (first column).} \label{fig:masked_noft_ep}
\end{figure}

\begin{figure}[b]
    \centering
    \includegraphics[width=\linewidth]{imgs/examples_noft_ep.pdf}
    \caption{Examples of the generation outputs for Mixed and ProFusion sampling methods for their optimal metrics point in the Superclass, NoFT, and Empty Prompt (EP) setups.} \label{fig:examples_noft_ep}
\end{figure}

\clearpage

\begin{figure}[ht!]
  \centering
  \includegraphics[trim={0 5cm 0 5cm},clip,width=0.95\linewidth]{imgs/us_example_new.pdf}
  \caption{An example of a task in the user study}
  \label{fig:us_ex}
  \vspace{-0.19in}
\end{figure}

\section{Data preparation}\label{app:data}
For each concept, we used inpainting augmentations to create the training dataset. We took an original image and automatically segmented it using the Segment Anything model on top of the CLIP cross-attention maps. Then we crop the concept from the original image, apply affine transformations to it, and inpaint the background. We used $10$ augmentation prompts, different from the evaluation prompts, and sampled $3$ images per prompt, resulting in a total of $30$ training images per concept. We commit to open-source the augmented datasets for each concept after publication.

\section{User Study}\label{app:us}

An example task from the user study is shown in Figure~\ref{fig:us_ex}. In total, we collected 48,864 responses from 200 unique users for 16,000 unique pairs. For each task, users were asked three questions: 1) "Which image is more consistent with the text prompt?" 2) "Which image better represents the original image?" 3) "Which image is generally better in terms of alignment with the prompt and concept identity preservation?" For each question, users selected one of three responses: "1", "2", or "Can't decide."

\section{Complex Prompts Setting}\label{app:long_prompts}

We conduct a comparison of different sampling methods using a set of complex prompts. For this analysis, we collected 10 prompts, each featuring multiple scene changes simultaneously, including stylization, background, and outfit:

\adjustbox{max width=\linewidth}{
\begin{lstlisting}
live_long = [
  "V* in a chief outfit in a nostalgic kitchen filled with vintage furniture and scattered biscuit",
  "V* sitting on a windowsill in Tokyo at dusk, illuminated by neon city lights, using neon color palette",
  "a vintage-style illustration of a V* sitting on a cobblestone street in Paris during a rainy evening, showcasing muted tones and soft grays",
  "an anime drawing of a V* dressed in a superhero cape, soaring through the skies above a bustling city during a sunset",
  "a cartoonish illustration of a V* dressed as a ballerina performing on a stage in the spotlight",
  "oil painting of a V* in Seattle during a snowy full moon night",
  "a digital painting of a V* in a wizard's robe in a magical forest at midnight, accented with purples and sparkling silver tones",
  "a drawing of a V* wearing a space helmet, floating among stars in a cosmic landscape during a starry night",
  "a V* in a detective outfit in a foggy London street during a rainy evening, using muted grays and blues",
  "a V* wearing a pirate hat exploring a sandy beach at the sunset with a boat floating in the background",
]

object_long = [
  "a digital illustration of a V* on a windowsill in Tokyo at dusk, illuminated by neon city lights, using neon color palette",
  "a sketch of a V* on a sofa in a cozy living room, rendered in warm tones",
  "a watercolor painting of a V* on a wooden table in a sunny backyard, surrounded by flowers and butterflies",
  "a V* floating in a bathtub filled with bubbles and illuminated by the warm glow of evening sunlight filtering through a nearby window",
  "a charcoal sketch of a giant V* surrounded by floating clouds during a starry night, where the moonlight creates an ethereal glow",
  "oil painting of a V* in Seattle during a snowy full moon night",
  "a drawing of a V* floating among stars in a cosmic landscape during a starry night with a spacecraft in the background",
  "a V* on a sandy beach next to the sand castle at the sunset with a floaing boat in the background",
  "an anime drawing V* on top of a white rug in the forest with a small wooden house in the background",
  "a vintage-style illustration of a V* on a cobblestone street in Paris during a rainy evening, showcasing muted tones and soft grays",
]
\end{lstlisting}
}

The results of this comparison are presented in Figures~\ref{fig:add_long},~\ref{fig:add_long_metrics}. We observe that Base sampling may struggle to preserve all the features specified by the prompts, whereas advanced sampling techniques effectively restore them. The overall arrangement of methods in the metric space closely mirrors that observed in the setting with simple prompts.

\begin{figure}[h!]
  \centering
  \includegraphics[width=\linewidth]{imgs/long_prompts_examples.pdf}
  \caption{Additional examples of the generation outputs for different sampling methods with \textbf{complex prompts}. We highlight parts of the prompt that are missing in Base sampling while appearing in other methods.}
  \label{fig:add_long}
\end{figure}

\clearpage
\section{Dreambooth results}\label{app:dreambooth}

We conduct additional analysis of different sampling methods in combination with Dreambooth. Figure~\ref{fig:add_db_metrics} shows that Mixed Sampling still overperforms Switching and Photoswap,  while Multi-stage and Masked struggle to provide an additional improvement over the simple baseline. Figure~\ref{fig:add_db} shows that all methods allow for improvement TS with a negligent decrease in IS while Mixed Sampling provides the best IS among all samplings.

\begin{figure*}[!ht]
\centering
\begin{minipage}{.477\textwidth}
  \centering
  \includegraphics[trim={3cm 10cm 3cm 10cm},clip,width=\linewidth]{imgs/long_prompts.pdf}
  \caption{CLIP metrics for different sampling methods estimated on \textbf{complex prompts}.}
  \label{fig:add_long_metrics}
\end{minipage}
\hfill
\begin{minipage}{.477\textwidth}
  \centering
  \includegraphics[trim={3cm 10cm 3cm 10cm},clip,width=\linewidth]{imgs/db_samplings.pdf}
  \caption{CLIP metrics for different sampling strategies on top of a Dreambooth fine-tuning method.}
  \label{fig:add_db_metrics}
\end{minipage}
\end{figure*} 

\begin{figure}[h!]
  \centering
  \includegraphics[trim={0 1cm 0 1cm},clip,width=\linewidth]{imgs/db_sampling_examples.pdf}
  \caption{Additional examples of the generation outputs for different sampling methods on top of a Dreambooth fine-tuning method.}
  \label{fig:add_db}
\end{figure}

\section{PixArt-alpha \& SD-XL}\label{app:add_backbones}
We conducted a series of experiments using different backbones. For SD-XL~\cite{podell2023sdxlimprovinglatentdiffusion}, we used SVDDiff as the fine-tuning method, while PixArt-alpha~\citep{chen2023pixartalphafasttrainingdiffusion} employed standard Dreambooth training. Hyperparameters for Switching, Masked, and ProFusion were selected in the same manner as in the experiments with SD2.

Figures~\ref{fig:pixart} and~\ref{fig:sdxl} demonstrate that Mixed Sampling follows a similar pattern to SD2, improving TS without a significant loss in IS. Notably, Mixed Sampling for SD-XL achieves simultaneous improvements in both IS and TS. ProFusion exhibits behavior consistent with SD2, enhancing IS more effectively than Mixed Sampling but performing worse at improving TS while also requiring twice the computational resources. 

\begin{figure}[h]
\centering
\begin{minipage}{.49\textwidth}
  \centering
  \includegraphics[trim={3cm 10cm 3cm 10cm},clip,width=\linewidth]{imgs/pixart.pdf}
  \captionof{figure}{CLIP metrics for different sampling methods estimated on PixArt model.}
  \label{fig:pixart}
\end{minipage}%
\hfill
\begin{minipage}{.49\textwidth}
  \centering
  \includegraphics[trim={3cm 10cm 3cm 10cm},clip,width=\linewidth]{imgs/sdxl.pdf}
  \captionof{figure}{CLIP metrics for different sampling methods estimated on SD-XL model.}
  \label{fig:sdxl}
\end{minipage}
\end{figure}

\clearpage
\section{Cross-Attention Masks}\label{app:cross_attn}

\begin{figure}[h!]
  \centering
  \includegraphics[trim={3cm 0cm 3cm 0cm},clip,width=\linewidth]{imgs/cross_attention_masks.pdf}
  \caption{Visualization of the cross-attention masks for Masked sampling examples. Here, $q$ defines the thresholding quantile and $t$ the denoising step.}
  \label{fig:cross_attn_add_ex}
\end{figure}

\clearpage
\section{Additional Examples}\label{app:add_example}

\begin{figure}[h!]
  \centering
  \includegraphics[trim={0 2cm 0 2cm},clip,width=\linewidth]{imgs/additional_examples.pdf}
  \caption{Additional examples of the generation outputs for different sampling methods.}
  \label{fig:add_ex}
\end{figure}

\begin{figure}[h!]
  \centering
  \includegraphics[trim={0 2cm 0 2cm},clip,width=\linewidth]{imgs/additional_examples_all.pdf}
  \caption{Additional examples of the generation outputs for Mixed and ProFusion sampling methods in comparison to the main personalized generation baselines.}
  \label{fig:add_ex_all}
\end{figure}

\clearpage
\section{DINO Image Similarity}\label{app:add_dino}

We compare CLIP-IS (left column) and DINO-IS~\citep{oquab2024dinov2learningrobustvisual} (right column) in Figures~\ref{fig:profusion_photoswap_dino},~\ref{fig:all_methods_dino}. We observe that despite the choice of metric, different sampling techniques and finetuning strategies have the same arrangement. The most noticeable difference is that SVDDiff superiority over ELITE and TI is more pronounced. That strengthens our motivation to select SVDDiff as the main backbone.

\begin{figure}[h]
\centering
\begin{minipage}{.49\textwidth}
  \centering
  \includegraphics[trim={3cm 10cm 3cm 10cm},clip,width=\linewidth]{imgs/profusion_photoswap.pdf}
\end{minipage}%
\hfill
\begin{minipage}{.49\textwidth}
  \centering
  \includegraphics[trim={3cm 10cm 3cm 10cm},clip,width=\linewidth]{imgs/profusion_photoswap_dino.pdf}
\end{minipage}
\caption{Pareto frontiers curves for Photoswap~\citep{photoswap} and ProFusion~\citep{profusion}.}\label{fig:profusion_photoswap_dino}
\end{figure}

\begin{figure}[h]
\centering
\begin{minipage}{.49\textwidth}
  \centering
  \includegraphics[trim={3cm 10cm 3cm 10cm},clip,width=\linewidth]{imgs/all_methods.pdf}
\end{minipage}%
\hfill
\begin{minipage}{.49\textwidth}
  \centering
  \includegraphics[trim={3cm 10cm 3cm 10cm},clip,width=\linewidth]{imgs/all_methods_dino.pdf}
\end{minipage}
\caption{The overall results of different sampling methods against main personalized generation baselines.}\label{fig:all_methods_dino}
\end{figure}

%%%%%%%%%%%%%%%%%%%%%%%%%%%%%%%%%%%%%%%%%%%%%%%%%%%%%%%%%%%%%%%%%%%%%%%%

\end{document}

%%%%%%%%%%%%%%%%%%%%%%%%%%%%%%%%%%%%%%%%%%%%%%%%%%%%%%%%%%%%%%%%%%%%%%%%

