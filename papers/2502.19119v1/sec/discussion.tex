\section{Discussion}\label{sec:dis}


As an essential skill for organic chemists, retrosynthesis requires expert knowledge, creativity, and intuition. In the modern era, chemical reaction data become a valuable asset across various scientific domains such as materials science, pharmaceuticals, and energy. However, sharing such data can expose trade secrets, violate patents, and lead to unfair competition or security compromises. To address these challenges, this paper introduces CKIF, a distributed machine learning framework that enables collaborative training without the need to exchange raw reaction data, thereby ensuring data privacy --- a major concern in sensitive scientific domains. By leveraging the distributed and heterogeneous nature of reaction data sources, CKIF not only improves efficiency but also reduces communication overhead, latency, and bandwidth consumption. CKIF allows for the distribution of computation across multiple clients closer to data sources, enhancing parallelism and fault tolerance.


CKIF incorporates chemical symbolic knowledge in the form of molecule fingerprints to guide model aggregation, resulting in personalized models that enhance the accuracy and relevance of retrosynthesis predictions. This knowledge-based approach allows each participating chemical entity to benefit from the collective insights. In addition, empirical results suggest that CKIF's performance improves with an increasing number of clients and training samples, showcasing CKIF's scalability. By applying CKIF to retrosynthesis, one can expect the collaborative discovery and optimization of new chemical transformations from various and confidential data sources. Experiments on various experimental settings show that CKIF outperforms locally trained models (Locally Trained) and even those trained on centralized data (Centrally Trained). We hope that CKIF opens up new possibilities for advancing privacy-aware retrosynthesis and related fields. In addition, we aim to inspire more research on privacy-preserving ML methods for other scientific domains and applications.

Despite these strengths, our work has limitations that necessitate further exploration. First, CKIF'performance relies on the quality of local data from each chemical entity, which might vary significantly in practice. Second, the model aggregation based on chemical knowledge uses the same molecule fingerprints and measurements for all clients, which might not accurately capture the personalized target distribution of each client and might contain unintentional bias. Future work will explore customized molecule fingerprints or measurements that meet the unique needs of each client. Third, our evaluation metrics are focused on the validity and uniqueness of the generated synthetic routes, overlooking the practicality and feasibility of the synthesis in terms of cost, yield, and environmental impact.

During the privacy-aware learning process, each chemical entity trains their personalized models locally and only model parameters are communicated. However, this process is not immune to privacy threats, as malicious attackers can infer sensitive information from the model updates exchanged between the clients~\citep{xie2019dba,lyu2022privacy}, \ie, data representation leakage from gradients or models. Several methods have been proposed to address this essential issue of privacy leakage, such as encrypting the model updates or adding noise to them using differential privacy~\citep{abadi2016deep,sun2021provable}. \redtext{While addressing privacy leakage and other security concerns (such as data poisoning) remains an active research area~\citep{tolpegin2020data,sun2021data,nowroozi2025federated}, our contribution is orthogonal to these developments.} These methods, in principle, can be seamlessly incorporated into our framework.
