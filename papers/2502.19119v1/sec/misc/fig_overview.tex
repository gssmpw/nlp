\begin{figure*}[t]
    \centering
    \includegraphics[width=1.0\textwidth]{figures/framework.pdf}
    \caption{\textbf{\redtext{Overview of the existing training paradigms and the proposed CKIF.} a,} Local learning involves individual clients training a model independently on their own reaction data, without interaction or data exchange with other clients.  \textbf{b}, Central learning trains a global model with access to reaction data from all clients. It is the current standard learning paradigm for retrosynthesis. \redtext{Our privacy-preserving learning process operates through iterative communication rounds, each comprising two stages: local learning (\textbf{c}) and chemical knowledge-informed model aggregation (\textbf{d}). \textbf{c}, In the first round, the model parameters are initialized randomly, while in subsequent communication rounds, they are initialized from the personalized aggregated model obtained in the previous round. \textbf{d}, CKIF incorporates chemical knowledge, \ie, molecule fingerprints, to calculate adaptive weights based on the similarity of molecular fingerprints. These weights are used to guide the model aggregation process.}}
    \label{fig:overview}
\end{figure*}
