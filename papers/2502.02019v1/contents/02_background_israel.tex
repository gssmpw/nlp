To investigate the out-of-domain robustness, we revisit the network architecture and argue that the information loss caused by the temporal and dimensional compressions degrades the robustness. Specifically, the encoder first projects the high temporal resolution waveform into a low temporal resolution high dimensional space and then compresses the feature dimension for the stability of the following codebook learning. Since the bitrate is only related to the temporal resolution (frame rate) of the codes when given a fixed codebook size, descript-audio-codec (DAC)~\cite{dac} adopts a handcraft code factorization in RVQ to ease the information loss by using high-dimensional codebooks (e.g. 1024-dim) while maintaining the codebook learning stability. However, in addition to the extra engineering efforts and costs of codebook storage and training, the high dimensional representation is not preferable to regression-based audio generations because of the markedly increased audio modeling difficulties and memory requirements.

