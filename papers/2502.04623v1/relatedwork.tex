\section{Related Work}
\subsection{Learning-based pansharpening method}

In recent years, deep learning-based methods have dominated the remote sensing pansharpening community. These techniques leverage the powerful feature learning and nonlinear fitting capabilities inherent in neural networks, significantly outperforming traditional approaches. The focus of pansharpening is to model the spatial-spectral relationship during the fusion of PAN and LR-MS images. Recently, CNN-based models like PanNet~\cite{yang2017pannet}, SRPPNN~\cite{cai2020super} and DCFNet~\cite{Wu_Huang_Deng_Zhang_2022} treat PAN and LR-MS images as a regular grid of pixels. In Transfromer-based models like HyperTransfromer~\cite{bandara2022hypertransformer} and CTINN~\cite{zhou2022panformer}, all pixels of feature maps extracted from PAN and LR-MS images are assigned to equal-sized attention grids
for attention operations, trying to model all potential relationships between PAN and LR-MS images. Since the CNN and Transformer rigidly model relationship in Euclidean space, it is sub-optimal for remote sensing images with irregular objects. Graph is the more flexible modeling structure. In this paper, we construct the first spatial-spectral heterogeneous graph for pansharpening, which models the spatial-spectral relationships in non-Euclidean space.

\subsection{Graph Representation Learning}
GCN eases the assumption of prior conditions, which takes the research object as the node and the correlation or similarity between objects as the edge. It can deal with complex paired interactions and integrate global spatial data, make full use of the internal relations between objects, and mine invisible relations between objects.
In recent years, the graph convolution theory has developed
rapidly. It has not only been widely applied to various high-level vision tasks, such as action recognition~\cite{zhao2019semantic} and semantic segmentation~\cite{li2020spatial},~\cite{qi20173d}, but also started to be used to solve low-level vision tasks, such as image inpainting ~\cite{wadhwa2021hyperrealistic}, image deraining ~\cite{fu2021rain}, and image denoising~\cite{valsesia2019image}. Furthermore, dual GCNs \cite{zhang2019dual} with different mapping strategies become popular. Bandara et al.~\cite{bandara2022spin} proposed spatial and interaction space graph reasoning to extract roads from aerial images. As far as we know, GCN is currently used for very little hyperspectral imagery. Qin et al.~\cite{qin2018spectral} and Wan et al. \cite{wan2019multiscale} have related work, but it is limited to the task of hyperspectral image classification \cite{hong2020graph,9484014, kang2020graph}. Recently, GPCNet~\cite{yan2022pansharpening} uses two independent GCN to directly model the spatial and spectral relationships of LR-MS and PAN image features' combination. In fact, it only uses GCN to model global relationships in Euclidean space. Our method designs graph structures in non-European space for pansharpening, and uses GCN to learn the graph node representations from both local and global perspectives.