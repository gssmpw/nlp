\section{Related Work}
\label{sec:rel}

In this work we adopt the acoustic localization algorithm from \cite{shor2024active}. This algorithm performs drone localization based solely on the self-sound emitted by the drone's propulsion system, as sampled by a circular array of microphones $\mathcal{M} $ located around the drone. The authors propose a three-step pipeline. First, a \textit{forward model} is used to model the self-sound emitted by the drone's propulsion system in free space, using a fixed, parametrized set $\mathcal{S}$ of point sound sources optimized to fit an actual free-space recording. Second, a neural, transformer-based \textit{inverse model} is used to regress the drone location based on the propulsion sound as sampled by the microphone array. This sound is simulated using both the forward model and RIR by superimposing the direct path from each point sound source to each microphone, along with the reflected paths from the walls. These paths are given by the image source model (ISM), replacing source reflections on walls with imaginary point sound sources (termed \textit{images} \cite{allen1979image}). Lastly, the authors present the optimization of the time-dependent angular offsets of the rotors, termed \textit{phase modulation}, in order to improve localization accuracy. We further elaborate on the concept of phase modulation in \cref{sec:defense} in the sequel. Our reason for using the approach from \cite{shor2024active} is the fact that it presents an accurate, purely acoustic-based localization model for us to evaluate under acoustic adversarial perturbations, and mainly since the proposed phase modulation mechanism will serve us in establishing our proposed defense method in \cref{sec:defense}. 

One shortcoming of \cite{shor2024active} is the choice to compute input to the location regressor (a.k.a inverse model), i.e. the sound sampled at each sensor using a simulation superimposing a set of point sources given by the ISM, with \cref{eq:sound} being applied over each point sound source separately. While this method is relatively accurate in sufficient point source sample density \cite{scheibler2018pyroomacoustics}, the computational costs entailed in computing the ISM render this method prohibitively expensive for larger and more complex acoustic environments. Our specific use-case of adversarial attack optimization calls for the optimization of sound emitted by the attacker (\cref{sec:attackf}), obligating rapid inference of the forward model (i.e., through sound generation). This makes the original ISM-based approach of sound computation inapplicable for our purposes. In this work we therefore opt develop a modified version of the ISM-reliant algorithm from \cite{shor2024active}, making use of neural-acoustic fields (NAFs) \cite{luo2022learning}, as further elaborated on \cref{sec:clean}.