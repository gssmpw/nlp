\pdfminorversion=4
\documentclass[10pt]{iopart}

\usepackage[utf8]{inputenc} %
\usepackage[T1]{fontenc}    %
\usepackage{hyperref}       %
\usepackage{url}            %
\usepackage{booktabs}       %
\usepackage{amsfonts}       %
\usepackage{nicefrac}       %
\usepackage{microtype}      %
\usepackage[table, x11names, dvipsnames]{xcolor}         %


\usepackage{siunitx} %
\sisetup{%
     exponent-product=\ensuremath{\cdot},
     group-digits=integer,
     range-phrase = \text{--}
}
\usepackage{fontawesome5}  %

\usepackage{amsmath} %
\usepackage{mathtools}

\usepackage{url} %
\usepackage{float}

\usepackage{caption}
\usepackage{subcaption}

\usepackage{listings} %

\usepackage{booktabs}


\definecolor{amaranth}{rgb}{0.9, 0.17, 0.31}
\colorlet{green}{green!20}
\colorlet{yellow}{yellow!60}
\colorlet{red}{red!30}

\usepackage{setspace}
\usepackage[linesnumbered,ruled,vlined]{algorithm2e}
\renewcommand{\KwSty}[1]{\textnormal{\textcolor{amaranth!90!black}{\bfseries #1}}\unskip}
\renewcommand{\ArgSty}[1]{\textnormal{\ttfamily #1}\unskip}
\SetKwComment{Comment}{\color{green!80!black!190}// }{}
\renewcommand{\CommentSty}[1]{\textnormal{\ttfamily\color{green!80!black!190}#1}\unskip}
\newcommand{\var}{\texttt}
\newcommand{\FuncCall}[2]{\texttt{\bfseries #1(#2)}}
\SetKwProg{Function}{function}{}{}
\SetKw{Continue}{continue}
\renewcommand{\ProgSty}[1]{\texttt{\bfseries #1}}
\DontPrintSemicolon
\SetAlFnt{\small}
\SetAlgorithmName{Pseudocode}{algorithmautorefname}


\newcommand{\makebrace}[6]{%
    \begin{tikzpicture}[overlay, remember picture]
        \draw [decoration={brace,amplitude=#4},decorate]
        let \p1=(#1), \p2=(#2) in
        ({max(\x1,\x2)}, {\y1+#5+0.8em}) -- node[right=#6] {#3} ({max(\x1,\x2)}, {\y2+#5});
    \end{tikzpicture}
}
\usepackage{tikz}
\usetikzlibrary{decorations.pathreplacing,calc}
\newcommand{\ntikzmark}[2]{#2\thinspace\tikz[overlay,remember picture,baseline=(#1.base)]{\node[inner sep=0pt] (#1) {};}}

\usepackage{enumitem}

\usepackage{siunitx} %

\usepackage{diagbox} %

\usepackage{multirow}  

\usepackage{longtable} %

\usepackage{array}
\newcolumntype{?}[1]{!{\vrule width #1}}
\setlength\extrarowheight{2pt}

\newcommand{\phantomgraphics}[2][]{%
  \leavevmode\phantom{\includegraphics[#1]{#2}}%
}

\def\showcomments{} %

\ifx\showcomments\undefined
\newcommand{\CV}[1]{}
\newcommand{\YP}[1]{}
\else
\newcommand{\CV}[1]{\textcolor{red}{[CV: #1]}}
\newcommand{\YP}[1]{\textcolor{purple}{[YP: #1]}}
\fi

\usepackage[absolute,overlay]{textpos}
\setlength{\TPHorizModule}{1cm}
\setlength{\TPVertModule}{1cm}

\usepackage{times}

\usepackage{scalerel}
\newcommand\vfrac[2]{\ThisStyle{%
  \setbox0=\hbox{$\SavedStyle#1#2$}%
  \setbox2=\hbox{$\SavedStyle X$}%
  \ifdim\ht0>\ht2\setlength{\ht0}{\ht2}\fi%
  #1\mathord{\stretchto{\raisebox{2.3\LMpt}{$\SavedStyle/$}}{\ht0}}#2}}

\usepackage{cite}



\begin{document}


\ioptwocolmod
\twocolumn[{
\begin{@twocolumnfalse}
\title[Robust Confinement State Classification with Uncertainty Quantification through Ensembled Data-Driven Methods]{Robust Confinement State Classification with Uncertainty Quantification through Ensembled Data-Driven Methods}
\vspace{-2pt}
\author{Yoeri Poels$^{\text{\textdagger},1,2}$, Cristina Venturini$^{\text{\textdagger},1}$, Alessandro Pau$^{1}$, Olivier Sauter$^{1}$, Vlado Menkovski$^{2}$, the TCV team$^3$ and the WPTE team$^{4}$}
\address{$^1$École Polytechnique Fédérale de Lausanne, Swiss Plasma Center, CH-1015 Lausanne, Switzerland}
\address{$^2$Eindhoven University of Technology, Mathematics and Computer Science, NL-5600MB Eindhoven, The Netherlands}
\address{$^3$See author list of B. P. Duval \textit{et al.} 2024 \textit{Nucl. Fusion} \textbf{64} 112023}
\address{$^4$See author list of E. Joffrin \textit{et al.} 2024 \textit{Nucl. Fusion} \textbf{64} 112019}
\address{$^{\text{\textdagger}}$Equal contribution, sorted alphabetically.}
\vspace{-2.6pt}
\ead{yoeri.poels@epfl.ch, cristina.venturini@epfl.ch}
\vspace{1.4pt}
\begin{indented}
\item[]February 2025
\end{indented}
\textbf{Abstract}\\
\begin{abstract}  
Test time scaling is currently one of the most active research areas that shows promise after training time scaling has reached its limits.
Deep-thinking (DT) models are a class of recurrent models that can perform easy-to-hard generalization by assigning more compute to harder test samples.
However, due to their inability to determine the complexity of a test sample, DT models have to use a large amount of computation for both easy and hard test samples.
Excessive test time computation is wasteful and can cause the ``overthinking'' problem where more test time computation leads to worse results.
In this paper, we introduce a test time training method for determining the optimal amount of computation needed for each sample during test time.
We also propose Conv-LiGRU, a novel recurrent architecture for efficient and robust visual reasoning. 
Extensive experiments demonstrate that Conv-LiGRU is more stable than DT, effectively mitigates the ``overthinking'' phenomenon, and achieves superior accuracy.
\end{abstract}  
\vspace{.13cm}
\hrule
\vspace{.13cm}
\end{@twocolumnfalse}
}\vspace{-2.9cm}]

\renewcommand*{\thefootnote}{\arabic{footnote}}



\maketitle


\section{Introduction}\label{sec:intro}
\section{Introduction}
\label{sec:introduction}
The business processes of organizations are experiencing ever-increasing complexity due to the large amount of data, high number of users, and high-tech devices involved \cite{martin2021pmopportunitieschallenges, beerepoot2023biggestbpmproblems}. This complexity may cause business processes to deviate from normal control flow due to unforeseen and disruptive anomalies \cite{adams2023proceddsriftdetection}. These control-flow anomalies manifest as unknown, skipped, and wrongly-ordered activities in the traces of event logs monitored from the execution of business processes \cite{ko2023adsystematicreview}. For the sake of clarity, let us consider an illustrative example of such anomalies. Figure \ref{FP_ANOMALIES} shows a so-called event log footprint, which captures the control flow relations of four activities of a hypothetical event log. In particular, this footprint captures the control-flow relations between activities \texttt{a}, \texttt{b}, \texttt{c} and \texttt{d}. These are the causal ($\rightarrow$) relation, concurrent ($\parallel$) relation, and other ($\#$) relations such as exclusivity or non-local dependency \cite{aalst2022pmhandbook}. In addition, on the right are six traces, of which five exhibit skipped, wrongly-ordered and unknown control-flow anomalies. For example, $\langle$\texttt{a b d}$\rangle$ has a skipped activity, which is \texttt{c}. Because of this skipped activity, the control-flow relation \texttt{b}$\,\#\,$\texttt{d} is violated, since \texttt{d} directly follows \texttt{b} in the anomalous trace.
\begin{figure}[!t]
\centering
\includegraphics[width=0.9\columnwidth]{images/FP_ANOMALIES.png}
\caption{An example event log footprint with six traces, of which five exhibit control-flow anomalies.}
\label{FP_ANOMALIES}
\end{figure}

\subsection{Control-flow anomaly detection}
Control-flow anomaly detection techniques aim to characterize the normal control flow from event logs and verify whether these deviations occur in new event logs \cite{ko2023adsystematicreview}. To develop control-flow anomaly detection techniques, \revision{process mining} has seen widespread adoption owing to process discovery and \revision{conformance checking}. On the one hand, process discovery is a set of algorithms that encode control-flow relations as a set of model elements and constraints according to a given modeling formalism \cite{aalst2022pmhandbook}; hereafter, we refer to the Petri net, a widespread modeling formalism. On the other hand, \revision{conformance checking} is an explainable set of algorithms that allows linking any deviations with the reference Petri net and providing the fitness measure, namely a measure of how much the Petri net fits the new event log \cite{aalst2022pmhandbook}. Many control-flow anomaly detection techniques based on \revision{conformance checking} (hereafter, \revision{conformance checking}-based techniques) use the fitness measure to determine whether an event log is anomalous \cite{bezerra2009pmad, bezerra2013adlogspais, myers2018icsadpm, pecchia2020applicationfailuresanalysispm}. 

The scientific literature also includes many \revision{conformance checking}-independent techniques for control-flow anomaly detection that combine specific types of trace encodings with machine/deep learning \cite{ko2023adsystematicreview, tavares2023pmtraceencoding}. Whereas these techniques are very effective, their explainability is challenging due to both the type of trace encoding employed and the machine/deep learning model used \cite{rawal2022trustworthyaiadvances,li2023explainablead}. Hence, in the following, we focus on the shortcomings of \revision{conformance checking}-based techniques to investigate whether it is possible to support the development of competitive control-flow anomaly detection techniques while maintaining the explainable nature of \revision{conformance checking}.
\begin{figure}[!t]
\centering
\includegraphics[width=\columnwidth]{images/HIGH_LEVEL_VIEW.png}
\caption{A high-level view of the proposed framework for combining \revision{process mining}-based feature extraction with dimensionality reduction for control-flow anomaly detection.}
\label{HIGH_LEVEL_VIEW}
\end{figure}

\subsection{Shortcomings of \revision{conformance checking}-based techniques}
Unfortunately, the detection effectiveness of \revision{conformance checking}-based techniques is affected by noisy data and low-quality Petri nets, which may be due to human errors in the modeling process or representational bias of process discovery algorithms \cite{bezerra2013adlogspais, pecchia2020applicationfailuresanalysispm, aalst2016pm}. Specifically, on the one hand, noisy data may introduce infrequent and deceptive control-flow relations that may result in inconsistent fitness measures, whereas, on the other hand, checking event logs against a low-quality Petri net could lead to an unreliable distribution of fitness measures. Nonetheless, such Petri nets can still be used as references to obtain insightful information for \revision{process mining}-based feature extraction, supporting the development of competitive and explainable \revision{conformance checking}-based techniques for control-flow anomaly detection despite the problems above. For example, a few works outline that token-based \revision{conformance checking} can be used for \revision{process mining}-based feature extraction to build tabular data and develop effective \revision{conformance checking}-based techniques for control-flow anomaly detection \cite{singh2022lapmsh, debenedictis2023dtadiiot}. However, to the best of our knowledge, the scientific literature lacks a structured proposal for \revision{process mining}-based feature extraction using the state-of-the-art \revision{conformance checking} variant, namely alignment-based \revision{conformance checking}.

\subsection{Contributions}
We propose a novel \revision{process mining}-based feature extraction approach with alignment-based \revision{conformance checking}. This variant aligns the deviating control flow with a reference Petri net; the resulting alignment can be inspected to extract additional statistics such as the number of times a given activity caused mismatches \cite{aalst2022pmhandbook}. We integrate this approach into a flexible and explainable framework for developing techniques for control-flow anomaly detection. The framework combines \revision{process mining}-based feature extraction and dimensionality reduction to handle high-dimensional feature sets, achieve detection effectiveness, and support explainability. Notably, in addition to our proposed \revision{process mining}-based feature extraction approach, the framework allows employing other approaches, enabling a fair comparison of multiple \revision{conformance checking}-based and \revision{conformance checking}-independent techniques for control-flow anomaly detection. Figure \ref{HIGH_LEVEL_VIEW} shows a high-level view of the framework. Business processes are monitored, and event logs obtained from the database of information systems. Subsequently, \revision{process mining}-based feature extraction is applied to these event logs and tabular data input to dimensionality reduction to identify control-flow anomalies. We apply several \revision{conformance checking}-based and \revision{conformance checking}-independent framework techniques to publicly available datasets, simulated data of a case study from railways, and real-world data of a case study from healthcare. We show that the framework techniques implementing our approach outperform the baseline \revision{conformance checking}-based techniques while maintaining the explainable nature of \revision{conformance checking}.

In summary, the contributions of this paper are as follows.
\begin{itemize}
    \item{
        A novel \revision{process mining}-based feature extraction approach to support the development of competitive and explainable \revision{conformance checking}-based techniques for control-flow anomaly detection.
    }
    \item{
        A flexible and explainable framework for developing techniques for control-flow anomaly detection using \revision{process mining}-based feature extraction and dimensionality reduction.
    }
    \item{
        Application to synthetic and real-world datasets of several \revision{conformance checking}-based and \revision{conformance checking}-independent framework techniques, evaluating their detection effectiveness and explainability.
    }
\end{itemize}

The rest of the paper is organized as follows.
\begin{itemize}
    \item Section \ref{sec:related_work} reviews the existing techniques for control-flow anomaly detection, categorizing them into \revision{conformance checking}-based and \revision{conformance checking}-independent techniques.
    \item Section \ref{sec:abccfe} provides the preliminaries of \revision{process mining} to establish the notation used throughout the paper, and delves into the details of the proposed \revision{process mining}-based feature extraction approach with alignment-based \revision{conformance checking}.
    \item Section \ref{sec:framework} describes the framework for developing \revision{conformance checking}-based and \revision{conformance checking}-independent techniques for control-flow anomaly detection that combine \revision{process mining}-based feature extraction and dimensionality reduction.
    \item Section \ref{sec:evaluation} presents the experiments conducted with multiple framework and baseline techniques using data from publicly available datasets and case studies.
    \item Section \ref{sec:conclusions} draws the conclusions and presents future work.
\end{itemize}

\section{Problem Formulation}\label{sec:problem}
\section{Problem Formulation} \label{sec:probdef}

This section formally defines the problem of restoring a given pruned network with only using its original pretrained CNN in a way free of data and fine-tuning.



% Unlike many existing works utilize data for identifying unimportant filters as well as fine-tuning to this end, we cannot evaluate the filter importance by data-dependent values like activation maps (\textit{a.k.a.} channels) as our focus in this paper is not to use any training data. Thus, in our problem setting, we can only exploit the values of filters in the original network, and thereby have to make some changes in the remaining filters of the pruned network so that the network can return the output not too much different from the original one.

% No matter how much we carefully select unimportant filters to be pruned, some kinds of retraining process appears inevitable as done by the most existing works to this end. However, since our focus in this paper is not to use any training data, we cannot evaluate the importance of filters by data-dependent values like activation maps (\textit{a.k.a.} channels). 

% To this end, they not only use a careful criterion (\textit{e.g.}, L1-norm), but also fine-tune the network using the original data.
% Most of filter pruning methods try to select filters to be pruned prudently so that pruned network's output be similar to the original network's. To this end, they prune the unimportant filters and then fine-tune the pruned network with using the train data. 

% How can we restore the the pruned networks without any data? In other words, it implies that we cannot use any data-driven values(i.e., activation maps) and we can only exploit the values of original filters. In that case, the only thing we can do maybe changing the weights of remained filters appropriately not to amplify the difference between pruned and unpruned network's outputs through the information of original filters.

\begin{figure*}[t]
	\centering
    \subfigure[\label{fig:matrix:a}Pruning matrix]{\hspace{6mm}\includegraphics[width=0.35\columnwidth]{./figure/LBYL_figure_2_1.pdf}\hspace{6mm}} 
    \subfigure[\label{fig:matrix:b}Delivery matrix for LBYL]{\hspace{6mm}\includegraphics[width=0.35\columnwidth]{./figure/LBYL_figure_2_2.pdf}\hspace{6mm}}
    \subfigure[\label{fig:matrix:c}Delivery matrix for one-to-one]{\hspace{9mm}\includegraphics[width=0.35\columnwidth]{./figure/LBYL_figure_2_3.pdf}\hspace{9mm}} 
    \caption{Comparison between pruning matrix and delivery matrix, where the $4$-th and $6$-th filters are being pruned among $6$ original filters}
	\label{fig:matrix}
	\vspace{-2mm}
\end{figure*}



\subsection{Filter Pruning in a CNN}
Consider a given CNN to be pruned with $L$ layers, where each $\ell$-th layer starts with a convolution operation on its input channels, which are the output of the previous $(\ell-1)$-th layer $\mathbf{A}^{(\ell-1)}$, with the group of convolution filters $\mathbf{W}^{{(\ell)}}$ and thereby obtain the set of \textit{feature maps} $\mathbf{Z}^{(\ell)}$ as follows:
\begin{equation}
\boldsymbol{\mathbf{Z}}^{(\ell)} = {\mathbf{A}^{(\ell-1)} \circledast {\mathbf{W}}^{(\ell)}},
\nonumber
\end{equation}
where $\circledast$ represents the convolution operation. Then, this convolution process is normally followed by a batch normalization (BN) process and an activation function such as ReLU, and the $\ell$-th layer finally outputs an \textit{activation map} $\mathbf{A}^{(\ell)}$ to be sent to the $(\ell+1)$-th layer through this sequence of procedures as:
\begin{equation}
\mathbf{A}^{(\ell)} = \F(\N(\mathbf{Z}^{(\ell)})),
\nonumber
\end{equation}
where $\F(\cdot)$ is an activation function and $\N(\cdot)$ is a BN procedure.

Note that all of $\mathbf{W}^{(\ell)}$, $\mathbf{Z}^{(\ell)}$, and $\mathbf{A}^{(\ell)}$ are tensors such that: $\mathbf{W}^{(\ell)} \in \mathbb{R}^{m \times n \times k \times k}$ and $\mathbf{Z}^{(\ell)},\mathbf{A}^{(\ell)} \in \mathbb{R}^{m \times w \times h}$, where (1) $m$ is the number of filters, which also equals the number of output activation maps, (2) $n$ is the number of input activation maps resulting from the $(\ell-1)$-th layer, (3) $k \times k$ is the size of each filter, and (4) $w \times h$ is the size of each output channel for the $\ell$-th layer.

\smalltitle{Filter pruning as n-mode product}
When filter pruning is performed at the $\ell$-th layer, all three tensors above are consequently modified to their \textit{damaged} versions, namely $\mathbf{\Tilde{W}}^{(\ell)}$, $\mathbf{\Tilde{Z}}^{(\ell)}$, and $\mathbf{\Tilde{A}}^{(\ell)}$, respectively, in a way that: $\mathbf{\Tilde{W}}^{(\ell)} \in \mathbb{R}^{t \times n \times k \times k}$ and $\mathbf{\Tilde{Z}}^{(\ell)},\mathbf{\Tilde{A}}^{(\ell)} \in \mathbb{R}^{t \times w \times h}$, where $t$ is the number of remaining filters after pruning and therefore $t < m$. Mathematically, the tensor of remaining filters, \textit{i.e.}, $\mathbf{\Tilde{W}}^{(\ell)}$, is obtained by the \textit{$1$-mode product} \cite{DBLP:journals/siamrev/KoldaB09} of the tensor of the original filters $\mathbf{W}^{(\ell)}$ with a \textit{pruning matrix} $\boldsymbol{\S} \in \mathbb{R}^{m \times t}$ (see Figure \ref{fig:matrix:a})
as follows:
\begin{eqnarray}\begin{split}\label{eq:pruning}
\mathbf{\Tilde{W}}^{(\ell)} = {\mathbf{W}}^{(\ell)} \times_{1} {\boldsymbol{\S}}^{T},\text{where }\boldsymbol{\S}_{i,k} = 
  \begin{cases} 
   1~ \text{if } i = i'_k \\
   0~ \text{otherwise}
  \end{cases} \\
  \text{s.t. } i, i'_k \in [1, m] 
  \text{ and } k \in [1, t].
  \end{split}
\end{eqnarray}
  
By Eq. (\ref{eq:pruning}), each $i'_k$-th filter is not pruned and the other $(m-t)$ filters are completely removed from $\mathbf{W}^{(\ell)}$ to be $\mathbf{\Tilde{W}}^{(\ell)}$.

This reduction at the $\ell$-th layer causes another reduction for each filter of the $(\ell+1)$-th layer so that $\mathbf{W}^{(\ell+1)}$ is now modified to $\mathbf{\Tilde{W}}^{(\ell+1)} \in \mathbb{R}^{m' \times t \times k' \times k'}$, where $m'$ is the number of filters of size $k' \times k'$ in the $(\ell+1)$-th layer. Due to this series of information losses, the resulting feature map (\textit{i.e.}, $\mathbf{Z}^{(\ell+1)}$) would severely be damaged to be $\mathbf{\Tilde{Z}}^{(\ell+1)}$ as shown below:
\begin{equation}
{\mathbf{\Tilde{Z}}}^{{(\ell+1)}} = \mathbf{\Tilde{A}}^{(\ell)} \circledast {\mathbf{\Tilde{W}}}^{(\ell+1)}~~~\not\approx~~~\mathbf{Z}^{(\ell+1)}
\label{eq:eq}\nonumber
\end{equation}
The shape of $\mathbf{\Tilde{Z}}^{(\ell+1)}$ remains the same unless we also prune filters for the $(\ell+1)$-th layer. If we do so as well, the loss of information will be accumulated and further propagated to the next layers. Note that $\mathbf{\Tilde{W}}^{(\ell+1)}$ can also be represented by the \textit{$2$-mode product} \cite{DBLP:journals/siamrev/KoldaB09} of $\mathbf{W}^{(\ell+1)}$ with the transpose of the same matrix $\boldsymbol{\S}$ as:
\begin{equation} \label{eq:pruning2}
\mathbf{\Tilde{W}}^{(\ell+1)} = {\mathbf{W}}^{(\ell+1)} \times_{2} {\boldsymbol{\S}^T}
\end{equation}




\subsection{Problem of Restoring a Pruned Network without Data and Fine-Tuning}
As mentioned earlier, our goal is to restore a pruned and thus damaged CNN without using any data and re-training process, which implies the following two facts. First, we have to use a pruning criterion exploiting only the values of filters themselves such as L1-norm. In this sense, this paper does not focus on proposing a sophisticated pruning criterion but intends to recover a network somehow pruned by such a simple criterion. Secondly, since we cannot make appropriate changes in the remaining filters by fine-tuning, we should make the best use of the original network and identify how the information carried by a pruned filter can be delivered to the remaining filters.

% For brevity, we formulate our problem here with respect to a specific layer, say $\ell$, and then it can trivially be generalized for the entire network. 
\smalltitle{Delivery matrix}
In order to represent the information to be delivered to the preserved filters, let us first think of what the pruning matrix $\boldsymbol{\S}$ means. As defined in Eq. (\ref{eq:pruning}) and shown in Figure \ref{fig:matrix:a}, each row is either a zero vector (for filters being pruned) or a one-hot vector (for remaining filters), which is intended only to remove filters without delivering any information. Intuitively, we can transform this pruning matrix into a \textit{delivery matrix} that carries information for filters being pruned by replacing some meaningful values with some of the zero values therein. Once we find such an \textit{ideal} $\boldsymbol{\S^*}$, we can plug it into $\boldsymbol{\S}$ of Eq. (\ref{eq:pruning2}) to deliver missing information propagated from the $\ell$-th layer to the filters at the $(\ell+1)$-th layer, which will hopefully generate an approximation $\mathbf{\hat{Z}}^{(\ell+1)}$ close to the original feature map as follows:
\begin{equation} \label{eq:fmap_approx}
{\mathbf{\hat{Z}}}^{{(\ell+1)}} = {\mathbf{\Tilde{A}}^{(\ell)} \circledast ({\mathbf{W}}^{(\ell+1)} \times_{2} {\boldsymbol{\S^*}^T})}
~~~\approx~~~\mathbf{Z}^{(\ell+1)}
\end{equation}
Thus, using the delivery matrix $\boldsymbol{\mathcal{S^*}}$, the information loss caused by pruning at each layer is recovered at the feature map of the next layer.

\smalltitle{Problem statement}
Given a pretrained CNN, our problem aims to find the best delivery matrix $\boldsymbol{\mathcal{S^*}}$ for each layer without any data and training process such that the following \textit{reconstruction error} is minimized:
\begin{equation}
\sum\limits_{i = 1}^{m'}\|{{\mathbf{Z}}_{i}^{{(\ell+1)}}-{\hat{\mathbf{Z}}}_{i}^{{(\ell+1)}}}\|_1,
\label{eq:goal}
\end{equation}
where ${\mathbf{Z}}_i^{{(\ell+1)}}$ and ${\hat{\mathbf{Z}}}_i^{{(\ell+1)}}$ indicate the $i$-th original feature map and its corresponding approximation, respectively, out of $m'$ filters in the $(\ell+1)$-th layer. Note that what is challenging here is that we cannot obtain the activation maps in $\mathbf{A}^{(\ell)}$ and $\mathbf{\Tilde{A}}^{(\ell)}$ without data as they are data-dependent values.

% = \sum\limits_{i = 1}^{m'}\|{{\mathbf{Z}}_{i}^{{(\ell+1)}}-{\mathbf{\Tilde{A}}^{(\ell)} \circledast ({\mathbf{W}}^{(\ell+1)} \times_{2} {\boldsymbol{\mathcal{S^*}^T}})}}\|_{1}


% Our goal is finding the approximation matrix $\boldsymbol{\mathcal{S}}$ to minimize the reconstruction error between the pruned model and the original model without any data, and effectively deliver missing information for pruned filters using this approximation matrix


% $\testit{s}$,which can be represented as below.

% \begin{equation}
% \boldsymbol{\mathcal{S}} =  \underset{{\boldsymbol{\mathcal{S}}}}{\mathrm{argmin}} \sum\limits_{{i} = 1}^{m_{\ell+1}} \|{{\mathbf{Z}}_{i,:,:}^{{(\ell+1)}}-{\hat{\mathbf{Z}}}_{i,:,:}^{{(\ell+1)}}}\|_{1} 
% \label{eq:eq1}
% \end{equation}



% Let us first recall that the ultimate goal of network pruning is to make the output of a pruned network as close as possible to that of its original network. Unlike many existing pruning methods, our focus is not to use any training data at all for the entire pruning and recovery process, and this implies the following two facts. First, we cannot evaluate the filter importance by data-dependent values like activation values or gradients, but have to use a pruning criterion exploiting only the values of filters themselves such as L1-norm. Furthermore, instead of fine-tuning with data, the only thing we can do for the pruned network is to make appropriate changes in the remaining filters by identifying some relationships between pruned filters and the other preserved ones without any support from data. Based on this intuition, this section mathematically and generally defines the problem of restoring a pruned neural network in a manner free of data and fine-tuning.


% Thus, we make approximation matrix $\testit{s}$ $\in$ $\mathbb{R}^{m_{\ell} \times t_{\ell}}$ with relationship between the pruned filter and preserved filters in $\ell$-th layer and then apply it to the original filters in $(\ell+1)$-th layer to compensate for pruned feature maps $\boldsymbol{\hat{\mathbf{Z}}}^{{(\ell+1)}}$ as shown below.
% (\textit{i.e.}, Let $\hat{\mathbf{W}}^{(\ell+1)}$ be ${\mathbf{W}}^{(\ell+1)}$ $\times_2$ ${{\textit{s}}} $, where $\times_2$ is 2-mode matrix product) 

% \begin{equation}
% \mathbf{Z}^{(\ell+1)} = {\mathbf{A}}^{(\ell)} \circledast {\mathbf{W}}^{(\ell+1)}
% \approx {\hat{\mathbf{A}}^{(\ell)} \circledast ({\mathbf{W}}^{(\ell+1)} \times_{2} {{s}}) = {\hat{\mathbf{Z}}}^{{(\ell+1)}}}
% \label{eq:eq}\nonumber
% \end{equation}




% For a Convolutional Neural Network (CNN) with $L$ layers, we denote $\mathcal{A}{^{(\ell-1)}}$ $\in$ $\mathbb{R}^{n_{\ell -1 } \times h_{\ell -1} \times w_{\ell -1}}$ is activation maps at $\ell-1$-th layer, where $n_{\ell -1}$, $h_{\ell -1}$ and $w_{\ell -1}$ are the number of channels, height and width in activation maps, respectively. and we denote $\mathbf{W}^{{(\ell )}}$ $\in$  $\mathbb{R}^{m_{\ell} \times n_{\ell -1}\times k \times k}$ is covolution filters in $\ell$-th layer,where $m_{\ell}$, $n_{\ell-1}$ and $k$ are the number of filters, number of channels and kernel size, respectively. Trough the convolution operation using activation map $\mathcal{A}{^{(\ell-1)}}$ and convolution filter $\mathbf{W}^{{(\ell)}}$ in $\ell$-th layer, the feature maps $\boldsymbol{\mathbf{Z}}^{{(\ell)}}$ $\in$ $\mathbb{R}^{m_{\ell} \times h_{\ell+1} \times w_{\ell+1}}$ is computed as shown as below.


% \begin{equation}
% \boldsymbol{\mathbf{Z}}^{(\ell)} = {\mathcal{A}^{(\ell-1)} \circledast {\mathbf{W}}^{(\ell)}}
% \label{eq:eq1}\nonumber
% \end{equation}
% where $\circledast$ is convolution operation.

% and the feature maps passed through the BN and ReLU layer are activation maps $\mathcal{A}{^{(\ell)}}$ $\in$ $\mathbb{R}^{m_{{\ell}} \times h_{\ell+1} \times w_{\ell+1}} $ in $\ell$-th layer as shown as below.

% \begin{equation}
% \mathcal{A}^{(\ell)} = \mathcal{F}(\mathbf{Z}^{(\ell)} \circledast {\mathbf{W}}^{(\ell)})
% \label{eq:eq2}\nonumber
% \end{equation}
% where $\mathcal{F}$ is the function that implement batch normalization and non-linear activation(\textit{e.g.}, ReLU).

% \smalltitle{Filter Pruning}
% If the filter pruning is performed in $\ell$-th layer, the shape of original filters $\mathbf{W}^{{(\ell)}}$ $\in$ $\mathbb{R}^{m_{\ell} \times n_{\ell-1}\times k \times k}$ is modified to ${\hat {\mathbf{W}}^{(\ell)}}$ $\in$ $\mathbb{R}^{t_{\ell} \times n_{\ell-1}\times k \times k}$, where $t_{\ell}$ $<$ $m_{\ell}$ by pruning criterion. Therefore, the pruned activation maps ${\hat {\mathcal{A}}}{^{({\ell+1})}}$ $\in$ $\mathbb{R}^{t_{{\ell}} \times h_{{\ell+2}} \times w_{{\ell+2}}}$ in (${\ell+1}$)-th layer is computed as below.

% \begin{equation}
% \mathbf{\hat{A}}^{(l+1)} = \mathcal{F}({\mathbf{A}^{(\ell)} \circledast {\mathbf{\hat{W}}}^{(\ell+1)}})
% \label{eq:eq3}\nonumber
% \end{equation}

% Moreover, corresponding channels of each filters in ($\ell +1$)-th layer are sequentially removed. As a result, shape of original filters $\mathbf{W}^{{(\ell+1)}}$ $\in$ $\mathbb{R}^{m_{\ell+1} \times m_{\ell}\times k \times k}$ in ($\ell+1$)-th layer is changed to  ${\hat {\mathbf{W}}^{(\ell+1)}}$ $\in$ $\mathbb{R}^{m_{\ell+1} \times t_{\ell}\times k \times k}$. Although feature maps ${\hat{\mathbf{Z}}}^{{(\ell+1)}}$ $\in$ $\mathbb{R}^{m_{\ell+1} \times h_{\ell+2} \times w_{\ell+2}}$ in ($\ell+1$)-th layer after pruning have same shape with original feature maps ${\mathbf{Z}}^{{(\ell+1)}}$ $\in$ $\mathbb{R}^{m_{\ell+1} \times h_{\ell+2} \times w_{\ell+2}}$, the pruned feature maps $\boldsymbol{\hat{\mathbf{Z}}}^{{(\ell+1)}}$ are damaged.

\section{Dataset}\label{sec:data}
% TODO
% more detailed introduction of dataset creation
% the rumour label in such datasets
\section{Data} \label{sec:data}
We use three rumour datasets in this work, namely: PHEME~\citep{pheme2015,kochkina-etal-2018-one}, Twitter15, and Twitter16~\citep{ma-etal-2017-detect}:

% TJB: how can the number of threads be greater than the number of tweets? these numbers don't make sense
% RX: fixed, the numbers were incorrect
\paragraph{PHEME}~\citet{pheme2015} contains 6,425 tweet posts of rumours and non-rumours related to 9 events. To avoid using specific a priori keywords to search for tweet posts, PHEME used the Twitter (now X) steaming API to identify newsworthy events from breaking news and then selected from candidate rumours that met rumour criteria, finally they collected associated conversations and annotate them. They engaged journalists to annotate the threads. The data were collected between 2014 and 2015. The 9 events are split into two groups, the first being breaking news that contains rumours, including Ferguson unrest, Ottawa shooting, Sydney siege, Charlie Hebdo shooting, and Germanwings plane crash. The rest are specific rumours, namely Prince to play in Toronto, Gurlitt collection, Putin missing, and Michael Essien contracting Ebola.
% TJB: say something about the time period when this data was collected
% RX: added

\paragraph{Twitter 15}~\citet{twitter15} was constructed by collecting rumour and non-rumour posts from the tracking websites snopes.com and emergent.info. They then used the Twitter API to gather corresponding posts, resulting in 94 true and 446 false posts. This dataset further includes 1,490 root posts and their follow posts, comprising 1,116 rumours and 374 non-rumours.
% TJB: the "tweet" vs. "comment" terminology is potentially confusing and needs to be clarified
% RX: unified, used root and follow posts to refer to root posts and the comment posts, posts are used to describe tweets in general.

\paragraph{Twitter 16}
Similarly to Twitter 15, \citet{twitter16} collected rumours and non-rumours from snopes.com, resulting in 778 reported events, 64\% of which are rumours. For each event, keywords were extracted from the final part of the Snopes URL and refined manually---adding, deleting, or replacing words iteratively---until the composed queries yielded precise Twitter search results. The final dataset includes 1,490 root tweet posts and their follow posts, comprising 613 rumours and 205 non-rumours.

\begin{table*}[!t]
    \centering
    \small
    \begin{tabular}{p{0.05\linewidth}p{0.9\linewidth}}
    \toprule
    Task & Prompt \\
    \midrule
    V-oc & Categorize the text into an ordinal class that best characterizes the writer's mental state, considering various degrees of positive and negative sentiment intensity. 3: very positive mental state can be inferred. 2: moderately positive mental state can be inferred. 1: slightly positive mental state can be inferred. 0: neutral or mixed mental state can be inferred. -1: slightly negative mental state can be inferred. -2: moderately negative mental state can be inferred. -3: very negative mental state can be inferred.\\
    \midrule
    E-c & Categorize the text's emotional tone as either `neutral or no emotion' or identify the presence of one or more of the given emotions (anger, anticipation, disgust, fear, joy, love, optimism, pessimism, sadness, surprise, trust).\\
    \midrule
    E-i & Assign a numerical value between 0 (least E) and 1 (most E) to represent the intensity of emotion E expressed in the text.\\
    \bottomrule
    \end{tabular}
    \caption{Prompts used for EmoLLM to detect emotion information in tweets. V-oc = Valence Ordinal Classification, E-c = Emotion Classification, and E-i = Emotion Intensity Regression.}
    \label{tab:emollm_ins}
\end{table*}


  
%%% Local Variables:
%%% mode: latex
%%% TeX-master: "../main_anonymous"
%%% End:


\section{Method}\label{sec:method}
\section{Method}\label{sec:method}
\begin{figure}
    \centering
    \includegraphics[width=0.85\textwidth]{imgs/heatmap_acc.pdf}
    \caption{\textbf{Visualization of the proposed periodic Bayesian flow with mean parameter $\mu$ and accumulated accuracy parameter $c$ which corresponds to the entropy/uncertainty}. For $x = 0.3, \beta(1) = 1000$ and $\alpha_i$ defined in \cref{appd:bfn_cir}, this figure plots three colored stochastic parameter trajectories for receiver mean parameter $m$ and accumulated accuracy parameter $c$, superimposed on a log-scale heatmap of the Bayesian flow distribution $p_F(m|x,\senderacc)$ and $p_F(c|x,\senderacc)$. Note the \emph{non-monotonicity} and \emph{non-additive} property of $c$ which could inform the network the entropy of the mean parameter $m$ as a condition and the \emph{periodicity} of $m$. %\jj{Shrink the figures to save space}\hanlin{Do we need to make this figure one-column?}
    }
    \label{fig:vmbf_vis}
    \vskip -0.1in
\end{figure}
% \begin{wrapfigure}{r}{0.5\textwidth}
%     \centering
%     \includegraphics[width=0.49\textwidth]{imgs/heatmap_acc.pdf}
%     \caption{\textbf{Visualization of hyper-torus Bayesian flow based on von Mises Distribution}. For $x = 0.3, \beta(1) = 1000$ and $\alpha_i$ defined in \cref{appd:bfn_cir}, this figure plots three colored stochastic parameter trajectories for receiver mean parameter $m$ and accumulated accuracy parameter $c$, superimposed on a log-scale heatmap of the Bayesian flow distribution $p_F(m|x,\senderacc)$ and $p_F(c|x,\senderacc)$. Note the \emph{non-monotonicity} and \emph{non-additive} property of $c$. \jj{Shrink the figures to save space}}
%     \label{fig:vmbf_vis}
%     \vspace{-30pt}
% \end{wrapfigure}


In this section, we explain the detailed design of CrysBFN tackling theoretical and practical challenges. First, we describe how to derive our new formulation of Bayesian Flow Networks over hyper-torus $\mathbb{T}^{D}$ from scratch. Next, we illustrate the two key differences between \modelname and the original form of BFN: $1)$ a meticulously designed novel base distribution with different Bayesian update rules; and $2)$ different properties over the accuracy scheduling resulted from the periodicity and the new Bayesian update rules. Then, we present in detail the overall framework of \modelname over each manifold of the crystal space (\textit{i.e.} fractional coordinates, lattice vectors, atom types) respecting \textit{periodic E(3) invariance}. 

% In this section, we first demonstrate how to build Bayesian flow on hyper-torus $\mathbb{T}^{D}$ by overcoming theoretical and practical problems to provide a low-noise parameter-space approach to fractional atom coordinate generation. Next, we present how \modelname models each manifold of crystal space respecting \textit{periodic E(3) invariance}. 

\subsection{Periodic Bayesian Flow on Hyper-torus \texorpdfstring{$\mathbb{T}^{D}$}{}} 
For generative modeling of fractional coordinates in crystal, we first construct a periodic Bayesian flow on \texorpdfstring{$\mathbb{T}^{D}$}{} by designing every component of the totally new Bayesian update process which we demonstrate to be distinct from the original Bayesian flow (please see \cref{fig:non_add}). 
 %:) 
 
 The fractional atom coordinate system \citep{jiao2023crystal} inherently distributes over a hyper-torus support $\mathbb{T}^{3\times N}$. Hence, the normal distribution support on $\R$ used in the original \citep{bfn} is not suitable for this scenario. 
% The key problem of generative modeling for crystal is the periodicity of Cartesian atom coordinates $\vX$ requiring:
% \begin{equation}\label{eq:periodcity}
% p(\vA,\vL,\vX)=p(\vA,\vL,\vX+\vec{LK}),\text{where}~\vec{K}=\vec{k}\vec{1}_{1\times N},\forall\vec{k}\in\mathbb{Z}^{3\times1}
% \end{equation}
% However, there does not exist such a distribution supporting on $\R$ to model such property because the integration of such distribution over $\R$ will not be finite and equal to 1. Therefore, the normal distribution used in \citet{bfn} can not meet this condition.

To tackle this problem, the circular distribution~\citep{mardia2009directional} over the finite interval $[-\pi,\pi)$ is a natural choice as the base distribution for deriving the BFN on $\mathbb{T}^D$. 
% one natural choice is to 
% we would like to consider the circular distribution over the finite interval as the base 
% we find that circular distributions \citep{mardia2009directional} defined on a finite interval with lengths of $2\pi$ can be used as the instantiation of input distribution for the BFN on $\mathbb{T}^D$.
Specifically, circular distributions enjoy desirable periodic properties: $1)$ the integration over any interval length of $2\pi$ equals 1; $2)$ the probability distribution function is periodic with period $2\pi$.  Sharing the same intrinsic with fractional coordinates, such periodic property of circular distribution makes it suitable for the instantiation of BFN's input distribution, in parameterizing the belief towards ground truth $\x$ on $\mathbb{T}^D$. 
% \yuxuan{this is very complicated from my perspective.} \hanlin{But this property is exactly beautiful and perfectly fit into the BFN.}

\textbf{von Mises Distribution and its Bayesian Update} We choose von Mises distribution \citep{mardia2009directional} from various circular distributions as the form of input distribution, based on the appealing conjugacy property required in the derivation of the BFN framework.
% to leverage the Bayesian conjugacy property of von Mises distribution which is required by the BFN framework. 
That is, the posterior of a von Mises distribution parameterized likelihood is still in the family of von Mises distributions. The probability density function of von Mises distribution with mean direction parameter $m$ and concentration parameter $c$ (describing the entropy/uncertainty of $m$) is defined as: 
\begin{equation}
f(x|m,c)=vM(x|m,c)=\frac{\exp(c\cos(x-m))}{2\pi I_0(c)}
\end{equation}
where $I_0(c)$ is zeroth order modified Bessel function of the first kind as the normalizing constant. Given the last univariate belief parameterized by von Mises distribution with parameter $\theta_{i-1}=\{m_{i-1},\ c_{i-1}\}$ and the sample $y$ from sender distribution with unknown data sample $x$ and known accuracy $\alpha$ describing the entropy/uncertainty of $y$,  Bayesian update for the receiver is deducted as:
\begin{equation}
 h(\{m_{i-1},c_{i-1}\},y,\alpha)=\{m_i,c_i \}, \text{where}
\end{equation}
\begin{equation}\label{eq:h_m}
m_i=\text{atan2}(\alpha\sin y+c_{i-1}\sin m_{i-1}, {\alpha\cos y+c_{i-1}\cos m_{i-1}})
\end{equation}
\begin{equation}\label{eq:h_c}
c_i =\sqrt{\alpha^2+c_{i-1}^2+2\alpha c_{i-1}\cos(y-m_{i-1})}
\end{equation}
The proof of the above equations can be found in \cref{apdx:bayesian_update_function}. The atan2 function refers to  2-argument arctangent. Independently conducting  Bayesian update for each dimension, we can obtain the Bayesian update distribution by marginalizing $\y$:
\begin{equation}
p_U(\vtheta'|\vtheta,\bold{x};\alpha)=\mathbb{E}_{p_S(\bold{y}|\bold{x};\alpha)}\delta(\vtheta'-h(\vtheta,\bold{y},\alpha))=\mathbb{E}_{vM(\bold{y}|\bold{x},\alpha)}\delta(\vtheta'-h(\vtheta,\bold{y},\alpha))
\end{equation} 
\begin{figure}
    \centering
    \vskip -0.15in
    \includegraphics[width=0.95\linewidth]{imgs/non_add.pdf}
    \caption{An intuitive illustration of non-additive accuracy Bayesian update on the torus. The lengths of arrows represent the uncertainty/entropy of the belief (\emph{e.g.}~$1/\sigma^2$ for Gaussian and $c$ for von Mises). The directions of the arrows represent the believed location (\emph{e.g.}~ $\mu$ for Gaussian and $m$ for von Mises).}
    \label{fig:non_add}
    \vskip -0.15in
\end{figure}
\textbf{Non-additive Accuracy} 
The additive accuracy is a nice property held with the Gaussian-formed sender distribution of the original BFN expressed as:
\begin{align}
\label{eq:standard_id}
    \update(\parsn{}'' \mid \parsn{}, \x; \alpha_a+\alpha_b) = \E_{\update(\parsn{}' \mid \parsn{}, \x; \alpha_a)} \update(\parsn{}'' \mid \parsn{}', \x; \alpha_b)
\end{align}
Such property is mainly derived based on the standard identity of Gaussian variable:
\begin{equation}
X \sim \mathcal{N}\left(\mu_X, \sigma_X^2\right), Y \sim \mathcal{N}\left(\mu_Y, \sigma_Y^2\right) \Longrightarrow X+Y \sim \mathcal{N}\left(\mu_X+\mu_Y, \sigma_X^2+\sigma_Y^2\right)
\end{equation}
The additive accuracy property makes it feasible to derive the Bayesian flow distribution $
p_F(\boldsymbol{\theta} \mid \mathbf{x} ; i)=p_U\left(\boldsymbol{\theta} \mid \boldsymbol{\theta}_0, \mathbf{x}, \sum_{k=1}^{i} \alpha_i \right)
$ for the simulation-free training of \cref{eq:loss_n}.
It should be noted that the standard identity in \cref{eq:standard_id} does not hold in the von Mises distribution. Hence there exists an important difference between the original Bayesian flow defined on Euclidean space and the Bayesian flow of circular data on $\mathbb{T}^D$ based on von Mises distribution. With prior $\btheta = \{\bold{0},\bold{0}\}$, we could formally represent the non-additive accuracy issue as:
% The additive accuracy property implies the fact that the "confidence" for the data sample after observing a series of the noisy samples with accuracy ${\alpha_1, \cdots, \alpha_i}$ could be  as the accuracy sum  which could be  
% Here we 
% Here we emphasize the specific property of BFN based on von Mises distribution.
% Note that 
% \begin{equation}
% \update(\parsn'' \mid \parsn, \x; \alpha_a+\alpha_b) \ne \E_{\update(\parsn' \mid \parsn, \x; \alpha_a)} \update(\parsn'' \mid \parsn', \x; \alpha_b)
% \end{equation}
% \oyyw{please check whether the below equation is better}
% \yuxuan{I fill somehow confusing on what is the update distribution with $\alpha$. }
% \begin{equation}
% \update(\parsn{}'' \mid \parsn{}, \x; \alpha_a+\alpha_b) \ne \E_{\update(\parsn{}' \mid \parsn{}, \x; \alpha_a)} \update(\parsn{}'' \mid \parsn{}', \x; \alpha_b)
% \end{equation}
% We give an intuitive visualization of such difference in \cref{fig:non_add}. The untenability of this property can materialize by considering the following case: with prior $\btheta = \{\bold{0},\bold{0}\}$, check the two-step Bayesian update distribution with $\alpha_a,\alpha_b$ and one-step Bayesian update with $\alpha=\alpha_a+\alpha_b$:
\begin{align}
\label{eq:nonadd}
     &\update(c'' \mid \parsn, \x; \alpha_a+\alpha_b)  = \delta(c-\alpha_a-\alpha_b)
     \ne  \mathbb{E}_{p_U(\parsn' \mid \parsn, \x; \alpha_a)}\update(c'' \mid \parsn', \x; \alpha_b) \nonumber \\&= \mathbb{E}_{vM(\bold{y}_b|\bold{x},\alpha_a)}\mathbb{E}_{vM(\bold{y}_a|\bold{x},\alpha_b)}\delta(c-||[\alpha_a \cos\y_a+\alpha_b\cos \y_b,\alpha_a \sin\y_a+\alpha_b\sin \y_b]^T||_2)
\end{align}
A more intuitive visualization could be found in \cref{fig:non_add}. This fundamental difference between periodic Bayesian flow and that of \citet{bfn} presents both theoretical and practical challenges, which we will explain and address in the following contents.

% This makes constructing Bayesian flow based on von Mises distribution intrinsically different from previous Bayesian flows (\citet{bfn}).

% Thus, we must reformulate the framework of Bayesian flow networks  accordingly. % and do necessary reformulations of BFN. 

% \yuxuan{overall I feel this part is complicated by using the language of update distribution. I would like to suggest simply use bayesian update, to provide intuitive explantion.}\hanlin{See the illustration in \cref{fig:non_add}}

% That introduces a cascade of problems, and we investigate the following issues: $(1)$ Accuracies between sender and receiver are not synchronized and need to be differentiated. $(2)$ There is no tractable Bayesian flow distribution for a one-step sample conditioned on a given time step $i$, and naively simulating the Bayesian flow results in computational overhead. $(3)$ It is difficult to control the entropy of the Bayesian flow. $(4)$ Accuracy is no longer a function of $t$ and becomes a distribution conditioned on $t$, which can be different across dimensions.
%\jj{Edited till here}

\textbf{Entropy Conditioning} As a common practice in generative models~\citep{ddpm,flowmatching,bfn}, timestep $t$ is widely used to distinguish among generation states by feeding the timestep information into the networks. However, this paper shows that for periodic Bayesian flow, the accumulated accuracy $\vc_i$ is more effective than time-based conditioning by informing the network about the entropy and certainty of the states $\parsnt{i}$. This stems from the intrinsic non-additive accuracy which makes the receiver's accumulated accuracy $c$ not bijective function of $t$, but a distribution conditioned on accumulated accuracies $\vc_i$ instead. Therefore, the entropy parameter $\vc$ is taken logarithm and fed into the network to describe the entropy of the input corrupted structure. We verify this consideration in \cref{sec:exp_ablation}. 
% \yuxuan{implement variant. traditionally, the timestep is widely used to distinguish the different states by putting the timestep embedding into the networks. citation of FM, diffusion, BFN. However, we find that conditioned on time in periodic flow could not provide extra benefits. To further boost the performance, we introduce a simple yet effective modification term entropy conditional. This is based on that the accumulated accuracy which represents the current uncertainty or entropy could be a better indicator to distinguish different states. + Describe how you do this. }



\textbf{Reformulations of BFN}. Recall the original update function with Gaussian sender distribution, after receiving noisy samples $\y_1,\y_2,\dots,\y_i$ with accuracies $\senderacc$, the accumulated accuracies of the receiver side could be analytically obtained by the additive property and it is consistent with the sender side.
% Since observing sample $\y$ with $\alpha_i$ can not result in exact accuracy increment $\alpha_i$ for receiver, the accuracies between sender and receiver are not synchronized which need to be differentiated. 
However, as previously mentioned, this does not apply to periodic Bayesian flow, and some of the notations in original BFN~\citep{bfn} need to be adjusted accordingly. We maintain the notations of sender side's one-step accuracy $\alpha$ and added accuracy $\beta$, and alter the notation of receiver's accuracy parameter as $c$, which is needed to be simulated by cascade of Bayesian updates. We emphasize that the receiver's accumulated accuracy $c$ is no longer a function of $t$ (differently from the Gaussian case), and it becomes a distribution conditioned on received accuracies $\senderacc$ from the sender. Therefore, we represent the Bayesian flow distribution of von Mises distribution as $p_F(\btheta|\x;\alpha_1,\alpha_2,\dots,\alpha_i)$. And the original simulation-free training with Bayesian flow distribution is no longer applicable in this scenario.
% Different from previous BFNs where the accumulated accuracy $\rho$ is not explicitly modeled, the accumulated accuracy parameter $c$ (visualized in \cref{fig:vmbf_vis}) needs to be explicitly modeled by feeding it to the network to avoid information loss.
% the randomaccuracy parameter $c$ (visualized in \cref{fig:vmbf_vis}) implies that there exists information in $c$ from the sender just like $m$, meaning that $c$ also should be fed into the network to avoid information loss. 
% We ablate this consideration in  \cref{sec:exp_ablation}. 

\textbf{Fast Sampling from Equivalent Bayesian Flow Distribution} Based on the above reformulations, the Bayesian flow distribution of von Mises distribution is reframed as: 
\begin{equation}\label{eq:flow_frac}
p_F(\btheta_i|\x;\alpha_1,\alpha_2,\dots,\alpha_i)=\E_{\update(\parsnt{1} \mid \parsnt{0}, \x ; \alphat{1})}\dots\E_{\update(\parsn_{i-1} \mid \parsnt{i-2}, \x; \alphat{i-1})} \update(\parsnt{i} | \parsnt{i-1},\x;\alphat{i} )
\end{equation}
Naively sampling from \cref{eq:flow_frac} requires slow auto-regressive iterated simulation, making training unaffordable. Noticing the mathematical properties of \cref{eq:h_m,eq:h_c}, we  transform \cref{eq:flow_frac} to the equivalent form:
\begin{equation}\label{eq:cirflow_equiv}
p_F(\vec{m}_i|\x;\alpha_1,\alpha_2,\dots,\alpha_i)=\E_{vM(\y_1|\x,\alpha_1)\dots vM(\y_i|\x,\alpha_i)} \delta(\vec{m}_i-\text{atan2}(\sum_{j=1}^i \alpha_j \cos \y_j,\sum_{j=1}^i \alpha_j \sin \y_j))
\end{equation}
\begin{equation}\label{eq:cirflow_equiv2}
p_F(\vec{c}_i|\x;\alpha_1,\alpha_2,\dots,\alpha_i)=\E_{vM(\y_1|\x,\alpha_1)\dots vM(\y_i|\x,\alpha_i)}  \delta(\vec{c}_i-||[\sum_{j=1}^i \alpha_j \cos \y_j,\sum_{j=1}^i \alpha_j \sin \y_j]^T||_2)
\end{equation}
which bypasses the computation of intermediate variables and allows pure tensor operations, with negligible computational overhead.
\begin{restatable}{proposition}{cirflowequiv}
The probability density function of Bayesian flow distribution defined by \cref{eq:cirflow_equiv,eq:cirflow_equiv2} is equivalent to the original definition in \cref{eq:flow_frac}. 
\end{restatable}
\textbf{Numerical Determination of Linear Entropy Sender Accuracy Schedule} ~Original BFN designs the accuracy schedule $\beta(t)$ to make the entropy of input distribution linearly decrease. As for crystal generation task, to ensure information coherence between modalities, we choose a sender accuracy schedule $\senderacc$ that makes the receiver's belief entropy $H(t_i)=H(p_I(\cdot|\vtheta_i))=H(p_I(\cdot|\vc_i))$ linearly decrease \emph{w.r.t.} time $t_i$, given the initial and final accuracy parameter $c(0)$ and $c(1)$. Due to the intractability of \cref{eq:vm_entropy}, we first use numerical binary search in $[0,c(1)]$ to determine the receiver's $c(t_i)$ for $i=1,\dots, n$ by solving the equation $H(c(t_i))=(1-t_i)H(c(0))+tH(c(1))$. Next, with $c(t_i)$, we conduct numerical binary search for each $\alpha_i$ in $[0,c(1)]$ by solving the equations $\E_{y\sim vM(x,\alpha_i)}[\sqrt{\alpha_i^2+c_{i-1}^2+2\alpha_i c_{i-1}\cos(y-m_{i-1})}]=c(t_i)$ from $i=1$ to $i=n$ for arbitrarily selected $x\in[-\pi,\pi)$.

After tackling all those issues, we have now arrived at a new BFN architecture for effectively modeling crystals. Such BFN can also be adapted to other type of data located in hyper-torus $\mathbb{T}^{D}$.

\subsection{Equivariant Bayesian Flow for Crystal}
With the above Bayesian flow designed for generative modeling of fractional coordinate $\vF$, we are able to build equivariant Bayesian flow for each modality of crystal. In this section, we first give an overview of the general training and sampling algorithm of \modelname (visualized in \cref{fig:framework}). Then, we describe the details of the Bayesian flow of every modality. The training and sampling algorithm can be found in \cref{alg:train} and \cref{alg:sampling}.

\textbf{Overview} Operating in the parameter space $\bthetaM=\{\bthetaA,\bthetaL,\bthetaF\}$, \modelname generates high-fidelity crystals through a joint BFN sampling process on the parameter of  atom type $\bthetaA$, lattice parameter $\vec{\theta}^L=\{\bmuL,\brhoL\}$, and the parameter of fractional coordinate matrix $\bthetaF=\{\bmF,\bcF\}$. We index the $n$-steps of the generation process in a discrete manner $i$, and denote the corresponding continuous notation $t_i=i/n$ from prior parameter $\thetaM_0$ to a considerably low variance parameter $\thetaM_n$ (\emph{i.e.} large $\vrho^L,\bmF$, and centered $\bthetaA$).

At training time, \modelname samples time $i\sim U\{1,n\}$ and $\bthetaM_{i-1}$ from the Bayesian flow distribution of each modality, serving as the input to the network. The network $\net$ outputs $\net(\parsnt{i-1}^\mathcal{M},t_{i-1})=\net(\parsnt{i-1}^A,\parsnt{i-1}^F,\parsnt{i-1}^L,t_{i-1})$ and conducts gradient descents on loss function \cref{eq:loss_n} for each modality. After proper training, the sender distribution $p_S$ can be approximated by the receiver distribution $p_R$. 

At inference time, from predefined $\thetaM_0$, we conduct transitions from $\thetaM_{i-1}$ to $\thetaM_{i}$ by: $(1)$ sampling $\y_i\sim p_R(\bold{y}|\thetaM_{i-1};t_i,\alpha_i)$ according to network prediction $\predM{i-1}$; and $(2)$ performing Bayesian update $h(\thetaM_{i-1},\y^\calM_{i-1},\alpha_i)$ for each dimension. 

% Alternatively, we complete this transition using the flow-back technique by sampling 
% $\thetaM_{i}$ from Bayesian flow distribution $\flow(\btheta^M_{i}|\predM{i-1};t_{i-1})$. 

% The training objective of $\net$ is to minimize the KL divergence between sender distribution and receiver distribution for every modality as defined in \cref{eq:loss_n} which is equivalent to optimizing the negative variational lower bound $\calL^{VLB}$ as discussed in \cref{sec:preliminaries}. 

%In the following part, we will present the Bayesian flow of each modality in detail.

\textbf{Bayesian Flow of Fractional Coordinate $\vF$}~The distribution of the prior parameter $\bthetaF_0$ is defined as:
\begin{equation}\label{eq:prior_frac}
    p(\bthetaF_0) \defeq \{vM(\vm_0^F|\vec{0}_{3\times N},\vec{0}_{3\times N}),\delta(\vc_0^F-\vec{0}_{3\times N})\} = \{U(\vec{0},\vec{1}),\delta(\vc_0^F-\vec{0}_{3\times N})\}
\end{equation}
Note that this prior distribution of $\vm_0^F$ is uniform over $[\vec{0},\vec{1})$, ensuring the periodic translation invariance property in \cref{De:pi}. The training objective is minimizing the KL divergence between sender and receiver distribution (deduction can be found in \cref{appd:cir_loss}): 
%\oyyw{replace $\vF$ with $\x$?} \hanlin{notations follow Preliminary?}
\begin{align}\label{loss_frac}
\calL_F = n \E_{i \sim \ui{n}, \flow(\parsn{}^F \mid \vF ; \senderacc)} \alpha_i\frac{I_1(\alpha_i)}{I_0(\alpha_i)}(1-\cos(\vF-\predF{i-1}))
\end{align}
where $I_0(x)$ and $I_1(x)$ are the zeroth and the first order of modified Bessel functions. The transition from $\bthetaF_{i-1}$ to $\bthetaF_{i}$ is the Bayesian update distribution based on network prediction:
\begin{equation}\label{eq:transi_frac}
    p(\btheta^F_{i}|\parsnt{i-1}^\calM)=\mathbb{E}_{vM(\bold{y}|\predF{i-1},\alpha_i)}\delta(\btheta^F_{i}-h(\btheta^F_{i-1},\bold{y},\alpha_i))
\end{equation}
\begin{restatable}{proposition}{fracinv}
With $\net_{F}$ as a periodic translation equivariant function namely $\net_F(\parsnt{}^A,w(\parsnt{}^F+\vt),\parsnt{}^L,t)=w(\net_F(\parsnt{}^A,\parsnt{}^F,\parsnt{}^L,t)+\vt), \forall\vt\in\R^3$, the marginal distribution of $p(\vF_n)$ defined by \cref{eq:prior_frac,eq:transi_frac} is periodic translation invariant. 
\end{restatable}
\textbf{Bayesian Flow of Lattice Parameter \texorpdfstring{$\boldsymbol{L}$}{}}   
Noting the lattice parameter $\bm{L}$ located in Euclidean space, we set prior as the parameter of a isotropic multivariate normal distribution $\btheta^L_0\defeq\{\vmu_0^L,\vrho_0^L\}=\{\bm{0}_{3\times3},\bm{1}_{3\times3}\}$
% \begin{equation}\label{eq:lattice_prior}
% \btheta^L_0\defeq\{\vmu_0^L,\vrho_0^L\}=\{\bm{0}_{3\times3},\bm{1}_{3\times3}\}
% \end{equation}
such that the prior distribution of the Markov process on $\vmu^L$ is the Dirac distribution $\delta(\vec{\mu_0}-\vec{0})$ and $\delta(\vec{\rho_0}-\vec{1})$, 
% \begin{equation}
%     p_I^L(\boldsymbol{L}|\btheta_0^L)=\mathcal{N}(\bm{L}|\bm{0},\bm{I})
% \end{equation}
which ensures O(3)-invariance of prior distribution of $\vL$. By Eq. 77 from \citet{bfn}, the Bayesian flow distribution of the lattice parameter $\bm{L}$ is: 
\begin{align}% =p_U(\bmuL|\btheta_0^L,\bm{L},\beta(t))
p_F^L(\bmuL|\bm{L};t) &=\mathcal{N}(\bmuL|\gamma(t)\bm{L},\gamma(t)(1-\gamma(t))\bm{I}) 
\end{align}
where $\gamma(t) = 1 - \sigma_1^{2t}$ and $\sigma_1$ is the predefined hyper-parameter controlling the variance of input distribution at $t=1$ under linear entropy accuracy schedule. The variance parameter $\vrho$ does not need to be modeled and fed to the network, since it is deterministic given the accuracy schedule. After sampling $\bmuL_i$ from $p_F^L$, the training objective is defined as minimizing KL divergence between sender and receiver distribution (based on Eq. 96 in \citet{bfn}):
\begin{align}
\mathcal{L}_{L} = \frac{n}{2}\left(1-\sigma_1^{2/n}\right)\E_{i \sim \ui{n}}\E_{\flow(\bmuL_{i-1} |\vL ; t_{i-1})}  \frac{\left\|\vL -\predL{i-1}\right\|^2}{\sigma_1^{2i/n}},\label{eq:lattice_loss}
\end{align}
where the prediction term $\predL{i-1}$ is the lattice parameter part of network output. After training, the generation process is defined as the Bayesian update distribution given network prediction:
\begin{equation}\label{eq:lattice_sampling}
    p(\bmuL_{i}|\parsnt{i-1}^\calM)=\update^L(\bmuL_{i}|\predL{i-1},\bmuL_{i-1};t_{i-1})
\end{equation}
    

% The final prediction of the lattice parameter is given by $\bmuL_n = \predL{n-1}$.
% \begin{equation}\label{eq:final_lattice}
%     \bmuL_n = \predL{n-1}
% \end{equation}

\begin{restatable}{proposition}{latticeinv}\label{prop:latticeinv}
With $\net_{L}$ as  O(3)-equivariant function namely $\net_L(\parsnt{}^A,\parsnt{}^F,\vQ\parsnt{}^L,t)=\vQ\net_L(\parsnt{}^A,\parsnt{}^F,\parsnt{}^L,t),\forall\vQ^T\vQ=\vI$, the marginal distribution of $p(\bmuL_n)$ defined by \cref{eq:lattice_sampling} is O(3)-invariant. 
\end{restatable}


\textbf{Bayesian Flow of Atom Types \texorpdfstring{$\boldsymbol{A}$}{}} 
Given that atom types are discrete random variables located in a simplex $\calS^K$, the prior parameter of $\boldsymbol{A}$ is the discrete uniform distribution over the vocabulary $\parsnt{0}^A \defeq \frac{1}{K}\vec{1}_{1\times N}$. 
% \begin{align}\label{eq:disc_input_prior}
% \parsnt{0}^A \defeq \frac{1}{K}\vec{1}_{1\times N}
% \end{align}
% \begin{align}
%     (\oh{j}{K})_k \defeq \delta_{j k}, \text{where }\oh{j}{K}\in \R^{K},\oh{\vA}{KD} \defeq \left(\oh{a_1}{K},\dots,\oh{a_N}{K}\right) \in \R^{K\times N}
% \end{align}
With the notation of the projection from the class index $j$ to the length $K$ one-hot vector $ (\oh{j}{K})_k \defeq \delta_{j k}, \text{where }\oh{j}{K}\in \R^{K},\oh{\vA}{KD} \defeq \left(\oh{a_1}{K},\dots,\oh{a_N}{K}\right) \in \R^{K\times N}$, the Bayesian flow distribution of atom types $\vA$ is derived in \citet{bfn}:
\begin{align}
\flow^{A}(\parsn^A \mid \vA; t) &= \E_{\N{\y \mid \beta^A(t)\left(K \oh{\vA}{K\times N} - \vec{1}_{K\times N}\right)}{\beta^A(t) K \vec{I}_{K\times N \times N}}} \delta\left(\parsn^A - \frac{e^{\y}\parsnt{0}^A}{\sum_{k=1}^K e^{\y_k}(\parsnt{0})_{k}^A}\right).
\end{align}
where $\beta^A(t)$ is the predefined accuracy schedule for atom types. Sampling $\btheta_i^A$ from $p_F^A$ as the training signal, the training objective is the $n$-step discrete-time loss for discrete variable \citep{bfn}: 
% \oyyw{can we simplify the next equation? Such as remove $K \times N, K \times N \times N$}
% \begin{align}
% &\calL_A = n\E_{i \sim U\{1,n\},\flow^A(\parsn^A \mid \vA ; t_{i-1}),\N{\y \mid \alphat{i}\left(K \oh{\vA}{KD} - \vec{1}_{K\times N}\right)}{\alphat{i} K \vec{I}_{K\times N \times N}}} \ln \N{\y \mid \alphat{i}\left(K \oh{\vA}{K\times N} - \vec{1}_{K\times N}\right)}{\alphat{i} K \vec{I}_{K\times N \times N}}\nonumber\\
% &\qquad\qquad\qquad-\sum_{d=1}^N \ln \left(\sum_{k=1}^K \out^{(d)}(k \mid \parsn^A; t_{i-1}) \N{\ydd{d} \mid \alphat{i}\left(K\oh{k}{K}- \vec{1}_{K\times N}\right)}{\alphat{i} K \vec{I}_{K\times N \times N}}\right)\label{discdisc_t_loss_exp}
% \end{align}
\begin{align}
&\calL_A = n\E_{i \sim U\{1,n\},\flow^A(\parsn^A \mid \vA ; t_{i-1}),\N{\y \mid \alphat{i}\left(K \oh{\vA}{KD} - \vec{1}\right)}{\alphat{i} K \vec{I}}} \ln \N{\y \mid \alphat{i}\left(K \oh{\vA}{K\times N} - \vec{1}\right)}{\alphat{i} K \vec{I}}\nonumber\\
&\qquad\qquad\qquad-\sum_{d=1}^N \ln \left(\sum_{k=1}^K \out^{(d)}(k \mid \parsn^A; t_{i-1}) \N{\ydd{d} \mid \alphat{i}\left(K\oh{k}{K}- \vec{1}\right)}{\alphat{i} K \vec{I}}\right)\label{discdisc_t_loss_exp}
\end{align}
where $\vec{I}\in \R^{K\times N \times N}$ and $\vec{1}\in\R^{K\times D}$. When sampling, the transition from $\bthetaA_{i-1}$ to $\bthetaA_{i}$ is derived as:
\begin{equation}
    p(\btheta^A_{i}|\parsnt{i-1}^\calM)=\update^A(\btheta^A_{i}|\btheta^A_{i-1},\predA{i-1};t_{i-1})
\end{equation}

The detailed training and sampling algorithm could be found in \cref{alg:train} and \cref{alg:sampling}.





\section{Experiments and Results}\label{sec:experiments}
\section{Experiments}
\label{sec:experiments}
The experiments are designed to address two key research questions.
First, \textbf{RQ1} evaluates whether the average $L_2$-norm of the counterfactual perturbation vectors ($\overline{||\perturb||}$) decreases as the model overfits the data, thereby providing further empirical validation for our hypothesis.
Second, \textbf{RQ2} evaluates the ability of the proposed counterfactual regularized loss, as defined in (\ref{eq:regularized_loss2}), to mitigate overfitting when compared to existing regularization techniques.

% The experiments are designed to address three key research questions. First, \textbf{RQ1} investigates whether the mean perturbation vector norm decreases as the model overfits the data, aiming to further validate our intuition. Second, \textbf{RQ2} explores whether the mean perturbation vector norm can be effectively leveraged as a regularization term during training, offering insights into its potential role in mitigating overfitting. Finally, \textbf{RQ3} examines whether our counterfactual regularizer enables the model to achieve superior performance compared to existing regularization methods, thus highlighting its practical advantage.

\subsection{Experimental Setup}
\textbf{\textit{Datasets, Models, and Tasks.}}
The experiments are conducted on three datasets: \textit{Water Potability}~\cite{kadiwal2020waterpotability}, \textit{Phomene}~\cite{phomene}, and \textit{CIFAR-10}~\cite{krizhevsky2009learning}. For \textit{Water Potability} and \textit{Phomene}, we randomly select $80\%$ of the samples for the training set, and the remaining $20\%$ for the test set, \textit{CIFAR-10} comes already split. Furthermore, we consider the following models: Logistic Regression, Multi-Layer Perceptron (MLP) with 100 and 30 neurons on each hidden layer, and PreactResNet-18~\cite{he2016cvecvv} as a Convolutional Neural Network (CNN) architecture.
We focus on binary classification tasks and leave the extension to multiclass scenarios for future work. However, for datasets that are inherently multiclass, we transform the problem into a binary classification task by selecting two classes, aligning with our assumption.

\smallskip
\noindent\textbf{\textit{Evaluation Measures.}} To characterize the degree of overfitting, we use the test loss, as it serves as a reliable indicator of the model's generalization capability to unseen data. Additionally, we evaluate the predictive performance of each model using the test accuracy.

\smallskip
\noindent\textbf{\textit{Baselines.}} We compare CF-Reg with the following regularization techniques: L1 (``Lasso''), L2 (``Ridge''), and Dropout.

\smallskip
\noindent\textbf{\textit{Configurations.}}
For each model, we adopt specific configurations as follows.
\begin{itemize}
\item \textit{Logistic Regression:} To induce overfitting in the model, we artificially increase the dimensionality of the data beyond the number of training samples by applying a polynomial feature expansion. This approach ensures that the model has enough capacity to overfit the training data, allowing us to analyze the impact of our counterfactual regularizer. The degree of the polynomial is chosen as the smallest degree that makes the number of features greater than the number of data.
\item \textit{Neural Networks (MLP and CNN):} To take advantage of the closed-form solution for computing the optimal perturbation vector as defined in (\ref{eq:opt-delta}), we use a local linear approximation of the neural network models. Hence, given an instance $\inst_i$, we consider the (optimal) counterfactual not with respect to $\model$ but with respect to:
\begin{equation}
\label{eq:taylor}
    \model^{lin}(\inst) = \model(\inst_i) + \nabla_{\inst}\model(\inst_i)(\inst - \inst_i),
\end{equation}
where $\model^{lin}$ represents the first-order Taylor approximation of $\model$ at $\inst_i$.
Note that this step is unnecessary for Logistic Regression, as it is inherently a linear model.
\end{itemize}

\smallskip
\noindent \textbf{\textit{Implementation Details.}} We run all experiments on a machine equipped with an AMD Ryzen 9 7900 12-Core Processor and an NVIDIA GeForce RTX 4090 GPU. Our implementation is based on the PyTorch Lightning framework. We use stochastic gradient descent as the optimizer with a learning rate of $\eta = 0.001$ and no weight decay. We use a batch size of $128$. The training and test steps are conducted for $6000$ epochs on the \textit{Water Potability} and \textit{Phoneme} datasets, while for the \textit{CIFAR-10} dataset, they are performed for $200$ epochs.
Finally, the contribution $w_i^{\varepsilon}$ of each training point $\inst_i$ is uniformly set as $w_i^{\varepsilon} = 1~\forall i\in \{1,\ldots,m\}$.

The source code implementation for our experiments is available at the following GitHub repository: \url{https://anonymous.4open.science/r/COCE-80B4/README.md} 

\subsection{RQ1: Counterfactual Perturbation vs. Overfitting}
To address \textbf{RQ1}, we analyze the relationship between the test loss and the average $L_2$-norm of the counterfactual perturbation vectors ($\overline{||\perturb||}$) over training epochs.

In particular, Figure~\ref{fig:delta_loss_epochs} depicts the evolution of $\overline{||\perturb||}$ alongside the test loss for an MLP trained \textit{without} regularization on the \textit{Water Potability} dataset. 
\begin{figure}[ht]
    \centering
    \includegraphics[width=0.85\linewidth]{img/delta_loss_epochs.png}
    \caption{The average counterfactual perturbation vector $\overline{||\perturb||}$ (left $y$-axis) and the cross-entropy test loss (right $y$-axis) over training epochs ($x$-axis) for an MLP trained on the \textit{Water Potability} dataset \textit{without} regularization.}
    \label{fig:delta_loss_epochs}
\end{figure}

The plot shows a clear trend as the model starts to overfit the data (evidenced by an increase in test loss). 
Notably, $\overline{||\perturb||}$ begins to decrease, which aligns with the hypothesis that the average distance to the optimal counterfactual example gets smaller as the model's decision boundary becomes increasingly adherent to the training data.

It is worth noting that this trend is heavily influenced by the choice of the counterfactual generator model. In particular, the relationship between $\overline{||\perturb||}$ and the degree of overfitting may become even more pronounced when leveraging more accurate counterfactual generators. However, these models often come at the cost of higher computational complexity, and their exploration is left to future work.

Nonetheless, we expect that $\overline{||\perturb||}$ will eventually stabilize at a plateau, as the average $L_2$-norm of the optimal counterfactual perturbations cannot vanish to zero.

% Additionally, the choice of employing the score-based counterfactual explanation framework to generate counterfactuals was driven to promote computational efficiency.

% Future enhancements to the framework may involve adopting models capable of generating more precise counterfactuals. While such approaches may yield to performance improvements, they are likely to come at the cost of increased computational complexity.


\subsection{RQ2: Counterfactual Regularization Performance}
To answer \textbf{RQ2}, we evaluate the effectiveness of the proposed counterfactual regularization (CF-Reg) by comparing its performance against existing baselines: unregularized training loss (No-Reg), L1 regularization (L1-Reg), L2 regularization (L2-Reg), and Dropout.
Specifically, for each model and dataset combination, Table~\ref{tab:regularization_comparison} presents the mean value and standard deviation of test accuracy achieved by each method across 5 random initialization. 

The table illustrates that our regularization technique consistently delivers better results than existing methods across all evaluated scenarios, except for one case -- i.e., Logistic Regression on the \textit{Phomene} dataset. 
However, this setting exhibits an unusual pattern, as the highest model accuracy is achieved without any regularization. Even in this case, CF-Reg still surpasses other regularization baselines.

From the results above, we derive the following key insights. First, CF-Reg proves to be effective across various model types, ranging from simple linear models (Logistic Regression) to deep architectures like MLPs and CNNs, and across diverse datasets, including both tabular and image data. 
Second, CF-Reg's strong performance on the \textit{Water} dataset with Logistic Regression suggests that its benefits may be more pronounced when applied to simpler models. However, the unexpected outcome on the \textit{Phoneme} dataset calls for further investigation into this phenomenon.


\begin{table*}[h!]
    \centering
    \caption{Mean value and standard deviation of test accuracy across 5 random initializations for different model, dataset, and regularization method. The best results are highlighted in \textbf{bold}.}
    \label{tab:regularization_comparison}
    \begin{tabular}{|c|c|c|c|c|c|c|}
        \hline
        \textbf{Model} & \textbf{Dataset} & \textbf{No-Reg} & \textbf{L1-Reg} & \textbf{L2-Reg} & \textbf{Dropout} & \textbf{CF-Reg (ours)} \\ \hline
        Logistic Regression   & \textit{Water}   & $0.6595 \pm 0.0038$   & $0.6729 \pm 0.0056$   & $0.6756 \pm 0.0046$  & N/A    & $\mathbf{0.6918 \pm 0.0036}$                     \\ \hline
        MLP   & \textit{Water}   & $0.6756 \pm 0.0042$   & $0.6790 \pm 0.0058$   & $0.6790 \pm 0.0023$  & $0.6750 \pm 0.0036$    & $\mathbf{0.6802 \pm 0.0046}$                    \\ \hline
%        MLP   & \textit{Adult}   & $0.8404 \pm 0.0010$   & $\mathbf{0.8495 \pm 0.0007}$   & $0.8489 \pm 0.0014$  & $\mathbf{0.8495 \pm 0.0016}$     & $0.8449 \pm 0.0019$                    \\ \hline
        Logistic Regression   & \textit{Phomene}   & $\mathbf{0.8148 \pm 0.0020}$   & $0.8041 \pm 0.0028$   & $0.7835 \pm 0.0176$  & N/A    & $0.8098 \pm 0.0055$                     \\ \hline
        MLP   & \textit{Phomene}   & $0.8677 \pm 0.0033$   & $0.8374 \pm 0.0080$   & $0.8673 \pm 0.0045$  & $0.8672 \pm 0.0042$     & $\mathbf{0.8718 \pm 0.0040}$                    \\ \hline
        CNN   & \textit{CIFAR-10} & $0.6670 \pm 0.0233$   & $0.6229 \pm 0.0850$   & $0.7348 \pm 0.0365$   & N/A    & $\mathbf{0.7427 \pm 0.0571}$                     \\ \hline
    \end{tabular}
\end{table*}

\begin{table*}[htb!]
    \centering
    \caption{Hyperparameter configurations utilized for the generation of Table \ref{tab:regularization_comparison}. For our regularization the hyperparameters are reported as $\mathbf{\alpha/\beta}$.}
    \label{tab:performance_parameters}
    \begin{tabular}{|c|c|c|c|c|c|c|}
        \hline
        \textbf{Model} & \textbf{Dataset} & \textbf{No-Reg} & \textbf{L1-Reg} & \textbf{L2-Reg} & \textbf{Dropout} & \textbf{CF-Reg (ours)} \\ \hline
        Logistic Regression   & \textit{Water}   & N/A   & $0.0093$   & $0.6927$  & N/A    & $0.3791/1.0355$                     \\ \hline
        MLP   & \textit{Water}   & N/A   & $0.0007$   & $0.0022$  & $0.0002$    & $0.2567/1.9775$                    \\ \hline
        Logistic Regression   &
        \textit{Phomene}   & N/A   & $0.0097$   & $0.7979$  & N/A    & $0.0571/1.8516$                     \\ \hline
        MLP   & \textit{Phomene}   & N/A   & $0.0007$   & $4.24\cdot10^{-5}$  & $0.0015$    & $0.0516/2.2700$                    \\ \hline
       % MLP   & \textit{Adult}   & N/A   & $0.0018$   & $0.0018$  & $0.0601$     & $0.0764/2.2068$                    \\ \hline
        CNN   & \textit{CIFAR-10} & N/A   & $0.0050$   & $0.0864$ & N/A    & $0.3018/
        2.1502$                     \\ \hline
    \end{tabular}
\end{table*}

\begin{table*}[htb!]
    \centering
    \caption{Mean value and standard deviation of training time across 5 different runs. The reported time (in seconds) corresponds to the generation of each entry in Table \ref{tab:regularization_comparison}. Times are }
    \label{tab:times}
    \begin{tabular}{|c|c|c|c|c|c|c|}
        \hline
        \textbf{Model} & \textbf{Dataset} & \textbf{No-Reg} & \textbf{L1-Reg} & \textbf{L2-Reg} & \textbf{Dropout} & \textbf{CF-Reg (ours)} \\ \hline
        Logistic Regression   & \textit{Water}   & $222.98 \pm 1.07$   & $239.94 \pm 2.59$   & $241.60 \pm 1.88$  & N/A    & $251.50 \pm 1.93$                     \\ \hline
        MLP   & \textit{Water}   & $225.71 \pm 3.85$   & $250.13 \pm 4.44$   & $255.78 \pm 2.38$  & $237.83 \pm 3.45$    & $266.48 \pm 3.46$                    \\ \hline
        Logistic Regression   & \textit{Phomene}   & $266.39 \pm 0.82$ & $367.52 \pm 6.85$   & $361.69 \pm 4.04$  & N/A   & $310.48 \pm 0.76$                    \\ \hline
        MLP   &
        \textit{Phomene} & $335.62 \pm 1.77$   & $390.86 \pm 2.11$   & $393.96 \pm 1.95$ & $363.51 \pm 5.07$    & $403.14 \pm 1.92$                     \\ \hline
       % MLP   & \textit{Adult}   & N/A   & $0.0018$   & $0.0018$  & $0.0601$     & $0.0764/2.2068$                    \\ \hline
        CNN   & \textit{CIFAR-10} & $370.09 \pm 0.18$   & $395.71 \pm 0.55$   & $401.38 \pm 0.16$ & N/A    & $1287.8 \pm 0.26$                     \\ \hline
    \end{tabular}
\end{table*}

\subsection{Feasibility of our Method}
A crucial requirement for any regularization technique is that it should impose minimal impact on the overall training process.
In this respect, CF-Reg introduces an overhead that depends on the time required to find the optimal counterfactual example for each training instance. 
As such, the more sophisticated the counterfactual generator model probed during training the higher would be the time required. However, a more advanced counterfactual generator might provide a more effective regularization. We discuss this trade-off in more details in Section~\ref{sec:discussion}.

Table~\ref{tab:times} presents the average training time ($\pm$ standard deviation) for each model and dataset combination listed in Table~\ref{tab:regularization_comparison}.
We can observe that the higher accuracy achieved by CF-Reg using the score-based counterfactual generator comes with only minimal overhead. However, when applied to deep neural networks with many hidden layers, such as \textit{PreactResNet-18}, the forward derivative computation required for the linearization of the network introduces a more noticeable computational cost, explaining the longer training times in the table.

\subsection{Hyperparameter Sensitivity Analysis}
The proposed counterfactual regularization technique relies on two key hyperparameters: $\alpha$ and $\beta$. The former is intrinsic to the loss formulation defined in (\ref{eq:cf-train}), while the latter is closely tied to the choice of the score-based counterfactual explanation method used.

Figure~\ref{fig:test_alpha_beta} illustrates how the test accuracy of an MLP trained on the \textit{Water Potability} dataset changes for different combinations of $\alpha$ and $\beta$.

\begin{figure}[ht]
    \centering
    \includegraphics[width=0.85\linewidth]{img/test_acc_alpha_beta.png}
    \caption{The test accuracy of an MLP trained on the \textit{Water Potability} dataset, evaluated while varying the weight of our counterfactual regularizer ($\alpha$) for different values of $\beta$.}
    \label{fig:test_alpha_beta}
\end{figure}

We observe that, for a fixed $\beta$, increasing the weight of our counterfactual regularizer ($\alpha$) can slightly improve test accuracy until a sudden drop is noticed for $\alpha > 0.1$.
This behavior was expected, as the impact of our penalty, like any regularization term, can be disruptive if not properly controlled.

Moreover, this finding further demonstrates that our regularization method, CF-Reg, is inherently data-driven. Therefore, it requires specific fine-tuning based on the combination of the model and dataset at hand.

\section{Conclusions and Discussion}\label{sec:conclusions}
\vspace{-0.2cm}
\section{Impact: Why Free Scientific Knowledge?}
\vspace{-0.1cm}

Historically, making knowledge widely available has driven transformative progress. Gutenberg’s printing press broke medieval monopolies on information, increasing literacy and contributing to the Renaissance and Scientific Revolution. In today's world, open source projects such as GNU/Linux and Wikipedia show that freely accessible and modifiable knowledge fosters innovation while ensuring creators are credited through copyleft licenses. These examples highlight a key idea: \textit{access to essential knowledge supports overall advancement.} 

This aligns with the arguments made by Prabhakaran et al. \cite{humanrightsbasedapproachresponsible}, who specifically highlight the \textbf{ human right to participate in scientific advancement} as enshrined in the Universal Declaration of Human Rights. They emphasize that this right underscores the importance of \textit{ equal access to the benefits of scientific progress for all}, a principle directly supported by our proposal for Knowledge Units. The UN Special Rapporteur on Cultural Rights further reinforces this, advocating for the expansion of copyright exceptions to broaden access to scientific knowledge as a crucial component of the right to science and culture \cite{scienceright}. 

However, current intellectual property regimes often create ``patently unfair" barriers to this knowledge, preventing innovation and access, especially in areas critical to human rights, as Hale compellingly argues \cite{patentlyunfair}. Finding a solution requires carefully balancing the imperative of open access with the legitimate rights of authors. As Austin and Ginsburg remind us, authors' rights are also human rights, necessitating robust protection \cite{authorhumanrights}. Shareable knowledge entities like Knowledge Units offer a potential mechanism to achieve this delicate balance in the scientific domain, enabling wider dissemination of research findings while respecting authors' fundamental rights.

\vspace{-0.2cm}
\subsection{Impact Across Sectors}

\textbf{Researchers:} Collaboration across different fields becomes easier when knowledge is shared openly. For instance, combining machine learning with biology or applying quantum principles to cryptography can lead to important breakthroughs. Removing copyright restrictions allows researchers to freely use data and methods, speeding up discoveries while respecting original contributions.

\textbf{Practitioners:} Professionals, especially in healthcare, benefit from immediate access to the latest research. Quick access to newer insights on the effectiveness of drugs, and alternative treatments speeds up adoption and awareness, potentially saving lives. Additionally, open knowledge helps developing countries gain access to health innovations.

\textbf{Education:} Education becomes more accessible when teachers use the latest research to create up-to-date curricula without prohibitive costs. Students can access high-quality research materials and use LM assistance to better understand complex topics, enhancing their learning experience and making high-quality education more accessible.

\textbf{Public Trust:} When information is transparent and accessible, the public can better understand and trust decision-making processes. Open access to government policies and industry practices allows people to review and verify information, helping to reduce misinformation. This transparency encourages critical thinking and builds trust in scientific and governmental institutions.

Overall, making scientific knowledge accessible supports global fairness. By viewing knowledge as a common resource rather than a product to be sold, we can speed up innovation, encourage critical thinking, and empower communities to address important challenges.

\vspace{-0.2cm}
\section{Open Problems}
\vspace{-0.1cm}

Moving forward, we identify key research directions to further exploit the potential of converting original texts into shareable knowledge entities such as demonstrated by the conversion into Knowledge Units in this work:


\textbf{1. Enhancing Factual Accuracy and Reliability:}  Refining KUs through cross-referencing with source texts and incorporating community-driven correction mechanisms, similar to Wikipedia, can minimize hallucinations and ensure the long-term accuracy of knowledge-based datasets at scale.

\textbf{2. Developing Applications for Education and Research:}  Using KU-based conversion for datasets to be employed in practical tools, such as search interfaces and learning platforms, can ensure rapid dissemination of any new knowledge into shareable downstream resources, significantly improving the accessibility, spread, and impact of KUs.

\textbf{3. Establishing Standards for Knowledge Interoperability and Reuse:}  Future research should focus on defining standardized formats for entities like KU and knowledge graph layouts \citep{lenat1990cyc}. These standards are essential to unlock seamless interoperability, facilitate reuse across diverse platforms, and foster a vibrant ecosystem of open scientific knowledge. 

\textbf{4. Interconnecting Shareable Knowledge for Scientific Workflow Assistance and Automation:} There might be further potential in constructing a semantic web that interconnects publicly shared knowledge, together with mechanisms that continually update and validate all shareable knowledge units. This can be starting point for a platform that uses all collected knowledge to assist scientific workflows, for instance by feeding such a semantic web into recently developed reasoning models equipped with retrieval augmented generation. Such assistance could assemble knowledge across multiple scientific papers, guiding scientists more efficiently through vast research landscapes. Given further progress in model capabilities, validation, self-repair and evolving new knowledge from already existing vast collection in the semantic web can lead to automation of scientific discovery, assuming that knowledge data in the semantic web can be freely shared.

We open-source our code and encourage collaboration to improve extraction pipelines, enhance Knowledge Unit capabilities, and expand coverage to additional fields.

\vspace{-0.2cm}
\section{Conclusion}
\vspace{-0.1cm}

In this paper, we highlight the potential of systematically separating factual scientific knowledge from protected artistic or stylistic expression. By representing scientific insights as structured facts and relationships, prototypes like Knowledge Units (KUs) offer a pathway to broaden access to scientific knowledge without infringing copyright, aligning with legal principles like German \S 24(1) UrhG and U.S. fair use standards. Extensive testing across a range of domains and models shows evidence that Knowledge Units (KUs) can feasibly retain core information. These findings offer a promising way forward for openly disseminating scientific information while respecting copyright constraints.

\section*{Author Contributions}

Christoph conceived the project and led organization. Christoph and Gollam led all the experiments. Nick and Huu led the legal aspects. Tawsif led the data collection. Ameya and Andreas led the manuscript writing. Ludwig, Sören, Robert, Jenia and Matthias provided feedback. advice and scientific supervision throughout the project. 

\section*{Acknowledgements}

The authors would like to thank (in alphabetical order): Sebastian Dziadzio, Kristof Meding, Tea Mustać, Shantanu Prabhat for insightful feedback and suggestions. Special thanks to Andrej Radonjic for help in scaling up data collection. GR and SA acknowledge financial support by the German Research Foundation (DFG) for the NFDI4DataScience Initiative (project number 460234259). AP and MB acknowledge financial support by the Federal Ministry of Education and Research (BMBF), FKZ: 011524085B and Open Philanthropy Foundation funded by the Good Ventures Foundation. AH acknowledges financial support by the Federal Ministry of Education and Research (BMBF), FKZ: 01IS24079A and the Carl Zeiss Foundation through the project "Certification and Foundations of Safe ML Systems" as well as the support from the International Max Planck Research School for Intelligent Systems (IMPRS-IS). JJ acknowledges funding by the Federal Ministry of Education and Research of Germany (BMBF) under grant no. 01IS22094B (WestAI - AI Service Center West), under grant no. 01IS24085C (OPENHAFM) and under the grant DE002571 (MINERVA), as well as co-funding by EU from EuroHPC Joint Undertaking programm under grant no. 101182737 (MINERVA) and from Digital Europe Programme under grant no. 101195233 (openEuroLLM) 

\section*{Acknowledgements}
\section{Acknowledgements}


\section*{References}
\bibliographystyle{plainurl_abrev}
\bibliography{main}

\clearpage
\appendix
\renewcommand{\thesection}{\appendixname~\Alph{section}}
\onecolumn
\setcounter{footnote}{1}
\counterwithin{figure}{section}
\counterwithin{lstlisting}{section}
\renewcommand{\thefigure}{\Alph{section}.\arabic{figure}}
\renewcommand{\thetable}{\Alph{section}.\arabic{table}}
\section{Feature sets}\label{ap:feature_sets}
The feature sets in the ensembling procedure are selected by the feature categorization and their individual discriminative power. To quantify the latter, we fit small models on each individual feature. Specifically, we fit a depth-2 decision tree to classify L, D or H-mode timeslice-by-timeslice. The performance is evaluated using Cohen's kappa coefficient~\cite{cohen1960}. Additionally, to identify parameter ranges consistently associated with a specific confinement state, we optimize thresholds on each individual feature that correspond to the largest amount of the data we can classify as `all L-mode' or `all H-mode' with at least 99\% accuracy. In other words, we check whether a feature can individually identify one of the two main confinement states in subparts of the parameter space. For example, total input power $P_{\textit{in}}$ can be used to trivially label some timeslices as L-mode given a minimum power requirement for any H-mode, see also Figure~\ref{fig:thresholds_example} for an illustration. We express this metric as the fraction of the data which can be labeled with such a threshold while keeping at least 99\% accuracy. Additionally, we report the signal availability over all timeslices in the dataset.

The results of all these metrics are provided in Table~\ref{tab:features}, for all features introduced in Table~\ref{tab:signals}. Note that we denote all spectral features computed from the photodiode signals ($\text{PD}_{\text{FFT}}^{}$ in Table~\ref{tab:signals}). The subscript integer denotes the window size in \SI{}{\milli\second} for the sliding window FFT, whereas the postfix $\in\{\textit{p}, \textit{c}, \textit{f}\}$ denotes whether the window is in the \textit{p}ast, \textit{c}entered or in the \textit{f}uture w.r.t. the given timeslice.

The feature sets cover both individual categories and combinations thereof. For each category we construct a model covering either all features or the most discriminative features following their Cohen's kappa coefficient values. The mixed feature sets cover both top-$k$ subsets for each category and rank-$k$ subsets: we both fit models taking in as much information as possible, while also fitting models with no mutual dependencies but still using informative features. Similarly, we select subsets using the threshold-orderings, although fewer in total because a substantial number of features cannot be used for any meaningful thresholding\footnote{Naturally, these features are still useful once combined with other more directly significant features.}, resulting in a value of 0 in Table~\ref{tab:features}. The resulting \textit{(model + feature set)} configurations are given in Table~\ref{tab:model_featuresets}.

The feature sets are identical between the FNOLSTM and GBDT models, with the exception of the photodiode related features. For the FNOLSTM we artificially rank $\text{PD}_{\textit{CIII}}^{}$ and $\text{PD}_{\textit{H}\alpha}^{}$ as the most informative emissions feature. These signals are not absolutely calibrated, making their raw value uninformative for classifying L, D or H-mode. However, the emission patterns they pick up clearly correspond to confinement state-related dynamics such as Edge Localized Modes (ELMs) or dithering cycles. The FNOLSTM-based models can fit these patterns because of their dynamic nature, making it a key feature to include. In contrast, the GBDT-models only take static information, making the raw signal value uninformative; rather, it relies on the constructed spectral features for the photodiode signal. To avoid redundancy in these $\text{PD}_{\text{FFT}}$ features, only the centered-window feature is used for each time window size; the past and future windows are only used in a specific category with all FFT features (FNOLSTM-EM-3 and GBDT-EM-3). 

\begin{figure}[h]
\begin{center}\includegraphics[width=0.5\linewidth]{figures/dataset_thresholds_example.pdf}\end{center}
    \caption{Distributions of the total input power and the plasma stored energy in the dataset, following the same procedure as Figure~\ref{fig:dataset_eda}. We overlay the `all L-mode' thresholds (L$^{\textit{fraction}}_{0.99}$) for the two features: below this threshold value, at least 99\% of the timeslices are in L-mode.}
    \label{fig:thresholds_example}%
\end{figure}

\newpage


\arrayrulecolor{black}
\def\arrvline{\hfil\kern\arraycolsep\vline\kern-\arraycolsep\hfilneg}
\setlength\tabcolsep{5.3pt}
\begingroup
\centering
\begin{longtable}{lccccp{0.45cm}lcccc}
 & Cohen's & & & Fraction & & & Cohen's & & & Fraction\\[-3pt]
Feature & kappa & L$^{\textit{fraction}}_{0.99}$ & H$^{\textit{fraction}}_{0.99}$ & \makebox[0pt]{available} & & Feature & kappa & L$^{\textit{fraction}}_{0.99}$ & H$^{\textit{fraction}}_{0.99}$ & \makebox[0pt]{available} \\ \cmidrule[\heavyrulewidth]{1-5}\cmidrule[\heavyrulewidth]{7-11}
\addlinespace[-\belowrulesep]
\cellcolor[RGB]{255,210,204} $A_p$ & \cellcolor[RGB]{255.0,255.0,255.0} 0.000 & \cellcolor[RGB]{250.0,253.0,249.6} 0.033 & \cellcolor[RGB]{255.0,255.0,255.0} 0.00000 & \cellcolor[RGB]{179.9,224.4,173.9} 0.997 &  & \cellcolor[RGB]{209,235,255} $\text{PD}_{\text{FFT}_{50\textit{-c}}}^{{\textit{H}\alpha}}$ & \cellcolor[RGB]{201.5,233.2,197.2} 0.704 & \cellcolor[RGB]{254.7,254.9,254.7} 0.002 & \cellcolor[RGB]{244.4,250.7,243.5} 0.00021 & \cellcolor[RGB]{182.9,225.6,177.2} 0.985 \\
\cellcolor[RGB]{255,210,204} $\delta_{\text{bottom}}$ & \cellcolor[RGB]{249.4,252.7,249.0} 0.073 & \cellcolor[RGB]{243.5,250.3,242.6} 0.076 & \cellcolor[RGB]{255.0,255.0,255.0} 0.00000 & \cellcolor[RGB]{179.9,224.4,173.9} 0.997 &  & \cellcolor[RGB]{209,235,255} $\text{PD}_{\text{FFT}_{50\textit{-f}}}^{{\textit{H}\alpha}}$ & \cellcolor[RGB]{202.0,233.4,197.9} 0.697 & \cellcolor[RGB]{254.9,255.0,254.9} 0.001 & \cellcolor[RGB]{255.0,255.0,255.0} 0.00000 & \cellcolor[RGB]{187.0,227.3,181.7} 0.968 \\
\cellcolor[RGB]{255,210,204} $\delta_{\text{top}}$ & \cellcolor[RGB]{255.0,255.0,255.0} 0.000 & \cellcolor[RGB]{244.9,250.9,244.1} 0.066 & \cellcolor[RGB]{255.0,255.0,255.0} 0.00000 & \cellcolor[RGB]{179.9,224.4,173.9} 0.997 &  & \cellcolor[RGB]{209,235,255} $\text{PD}_{\text{FFT}_{100\textit{-p}}}^{{\textit{H}\alpha}}$ & \cellcolor[RGB]{212.2,237.5,208.8} 0.563 & \cellcolor[RGB]{255.0,255.0,255.0} 0.000 & \cellcolor[RGB]{254.7,254.9,254.7} 0.00001 & \cellcolor[RGB]{179.0,224.0,173.0} 1.000 \\
\cellcolor[RGB]{255,210,204} $\Delta_{\text{in}}$ & \cellcolor[RGB]{255.0,255.0,255.0} 0.000 & \cellcolor[RGB]{217.9,239.9,215.0} 0.244 & \cellcolor[RGB]{255.0,255.0,255.0} 0.00000 & \cellcolor[RGB]{179.9,224.4,173.9} 0.997 &  & \cellcolor[RGB]{209,235,255} $\text{PD}_{\text{FFT}_{100\textit{-c}}}^{{\textit{H}\alpha}}$ & \cellcolor[RGB]{204.5,234.4,200.6} 0.664 & \cellcolor[RGB]{255.0,255.0,255.0} 0.000 & \cellcolor[RGB]{243.2,250.2,242.3} 0.00023 & \cellcolor[RGB]{187.0,227.3,181.7} 0.968 \\
\cellcolor[RGB]{255,210,204} $\Delta_{\text{out}}$ & \cellcolor[RGB]{255.0,255.0,255.0} 0.000 & \cellcolor[RGB]{254.7,254.9,254.7} 0.002 & \cellcolor[RGB]{255.0,255.0,255.0} 0.00000 & \cellcolor[RGB]{179.9,224.4,173.9} 0.997 &  & \cellcolor[RGB]{209,235,255} $\text{PD}_{\text{FFT}_{100\textit{-f}}}^{{\textit{H}\alpha}}$ & \cellcolor[RGB]{205.1,234.6,201.1} 0.657 & \cellcolor[RGB]{254.9,255.0,254.9} 0.001 & \cellcolor[RGB]{255.0,255.0,255.0} 0.00000 & \cellcolor[RGB]{195.4,230.7,190.7} 0.935 \\
\cellcolor[RGB]{255,210,204} $\kappa$ & \cellcolor[RGB]{229.9,244.8,227.9} 0.330 & \cellcolor[RGB]{241.8,249.6,240.8} 0.087 & \cellcolor[RGB]{255.0,255.0,255.0} 0.00000 & \cellcolor[RGB]{179.9,224.4,173.9} 0.997 &  & \cellcolor[RGB]{234,255,227} $B_0$ & \cellcolor[RGB]{243.0,250.1,242.0} 0.158 & \cellcolor[RGB]{255.0,255.0,255.0} 0.000 & \cellcolor[RGB]{255.0,255.0,255.0} 0.00000 & \cellcolor[RGB]{179.9,224.4,173.9} 0.997 \\
\cellcolor[RGB]{255,210,204} $R_0$ & \cellcolor[RGB]{255.0,255.0,255.0} 0.000 & \cellcolor[RGB]{244.1,250.6,243.2} 0.072 & \cellcolor[RGB]{255.0,255.0,255.0} 0.00000 & \cellcolor[RGB]{179.9,224.4,173.9} 0.997 &  & \cellcolor[RGB]{234,255,227} $I_{p}$ & \cellcolor[RGB]{239.9,248.8,238.7} 0.199 & \cellcolor[RGB]{251.2,253.4,250.8} 0.025 & \cellcolor[RGB]{255.0,255.0,255.0} 0.00000 & \cellcolor[RGB]{179.0,224.0,173.0} 1.000 \\
\cellcolor[RGB]{255,210,204} $a$ & \cellcolor[RGB]{255.0,255.0,255.0} 0.000 & \cellcolor[RGB]{252.5,254.0,252.3} 0.016 & \cellcolor[RGB]{255.0,255.0,255.0} 0.00000 & \cellcolor[RGB]{179.9,224.4,173.9} 0.997 &  & \cellcolor[RGB]{234,255,227} $I_{p,\textit{ref}}$ & \cellcolor[RGB]{239.7,248.8,238.5} 0.201 & \cellcolor[RGB]{250.7,253.2,250.4} 0.028 & \cellcolor[RGB]{255.0,255.0,255.0} 0.00000 & \cellcolor[RGB]{179.0,224.0,173.0} 1.000 \\
\cellcolor[RGB]{255,210,204} $R_{\text{axis}}$ & \cellcolor[RGB]{255.0,255.0,255.0} 0.000 & \cellcolor[RGB]{244.1,250.6,243.2} 0.072 & \cellcolor[RGB]{255.0,255.0,255.0} 0.00000 & \cellcolor[RGB]{179.9,224.4,173.9} 0.997 &  & \cellcolor[RGB]{234,255,227} $q_{95}$ & \cellcolor[RGB]{239.2,248.6,238.0} 0.208 & \cellcolor[RGB]{254.9,254.9,254.9} 0.001 & \cellcolor[RGB]{255.0,255.0,255.0} 0.00000 & \cellcolor[RGB]{179.9,224.4,173.9} 0.997 \\
\cellcolor[RGB]{255,210,204} $Z_{\text{axis}}$ & \cellcolor[RGB]{252.6,254.0,252.4} 0.032 & \cellcolor[RGB]{255.0,255.0,255.0} 0.000 & \cellcolor[RGB]{255.0,255.0,255.0} 0.00000 & \cellcolor[RGB]{179.9,224.4,173.9} 0.997 &  & \cellcolor[RGB]{252,233,255} $n_{e,\text{core}}$ & \cellcolor[RGB]{222.4,241.7,219.9} 0.429 & \cellcolor[RGB]{255.0,255.0,255.0} 0.000 & \cellcolor[RGB]{255.0,255.0,255.0} 0.00000 & \cellcolor[RGB]{179.0,224.0,173.0} 1.000 \\
\cellcolor[RGB]{255,210,204} $V_p$ & \cellcolor[RGB]{255.0,255.0,255.0} 0.000 & \cellcolor[RGB]{250.1,253.0,249.7} 0.033 & \cellcolor[RGB]{255.0,255.0,255.0} 0.00000 & \cellcolor[RGB]{179.9,224.4,173.9} 0.997 &  & \cellcolor[RGB]{252,233,255} $n_{e,\text{LFS}}$ & \cellcolor[RGB]{216.3,239.2,213.3} 0.509 & \cellcolor[RGB]{255.0,255.0,255.0} 0.000 & \cellcolor[RGB]{255.0,255.0,255.0} 0.00000 & \cellcolor[RGB]{179.0,224.0,173.0} 1.000 \\
\cellcolor[RGB]{209,235,255} $\text{PD}_{\textit{CIII}}^{}$ & \cellcolor[RGB]{255.0,255.0,255.0} 0.000 & \cellcolor[RGB]{253.6,254.4,253.5} 0.009 & \cellcolor[RGB]{255.0,255.0,255.0} 0.00000 & \cellcolor[RGB]{179.0,224.0,173.0} 1.000 &  & \cellcolor[RGB]{252,233,255} $n_e/n_{\textit{GW}}$ & \cellcolor[RGB]{232.6,245.8,230.8} 0.295 & \cellcolor[RGB]{255.0,255.0,255.0} 0.000 & \cellcolor[RGB]{202.5,233.6,198.4} 0.00104 & \cellcolor[RGB]{189.9,228.5,184.8} 0.957 \\
\cellcolor[RGB]{209,235,255} $\text{PD}_{\textit{H}\alpha}^{}$ & \cellcolor[RGB]{255.0,255.0,255.0} 0.000 & \cellcolor[RGB]{255.0,255.0,255.0} 0.000 & \cellcolor[RGB]{255.0,255.0,255.0} 0.00000 & \cellcolor[RGB]{179.0,224.0,173.0} 1.000 &  & \cellcolor[RGB]{252,233,255} $\text{max}(n'_{e,\text{edge}})$ & \cellcolor[RGB]{219.0,240.3,216.2} 0.473 & \cellcolor[RGB]{255.0,255.0,255.0} 0.000 & \cellcolor[RGB]{248.8,252.5,248.3} 0.00012 & \cellcolor[RGB]{213.4,238.0,210.2} 0.864 \\
\cellcolor[RGB]{209,235,255} $\text{PD}_{\text{FFT}_{5\textit{-p}}}^{{\textit{CIII}}}$ & \cellcolor[RGB]{214.1,238.3,210.9} 0.538 & \cellcolor[RGB]{237.4,247.8,236.0} 0.116 & \cellcolor[RGB]{255.0,255.0,255.0} 0.00000 & \cellcolor[RGB]{179.0,224.0,173.0} 1.000 &  & \cellcolor[RGB]{252,233,255} $\text{max}(n''_{e,\text{edge}})$ & \cellcolor[RGB]{206.3,235.1,202.4} 0.641 & \cellcolor[RGB]{255.0,255.0,255.0} 0.000 & \cellcolor[RGB]{251.2,253.5,250.9} 0.00007 & \cellcolor[RGB]{213.4,238.0,210.2} 0.864 \\
\cellcolor[RGB]{209,235,255} $\text{PD}_{\text{FFT}_{5\textit{-c}}}^{{\textit{CIII}}}$ & \cellcolor[RGB]{213.7,238.2,210.4} 0.543 & \cellcolor[RGB]{237.3,247.8,235.9} 0.116 & \cellcolor[RGB]{255.0,255.0,255.0} 0.00000 & \cellcolor[RGB]{179.3,224.1,173.3} 0.999 &  & \cellcolor[RGB]{252,233,255} $n_{e,0}$ & \cellcolor[RGB]{232.0,245.6,230.1} 0.303 & \cellcolor[RGB]{254.9,255.0,254.9} 0.001 & \cellcolor[RGB]{238.1,248.1,236.8} 0.00033 & \cellcolor[RGB]{193.9,230.1,189.1} 0.941 \\
\cellcolor[RGB]{209,235,255} $\text{PD}_{\text{FFT}_{5\textit{-f}}}^{{\textit{CIII}}}$ & \cellcolor[RGB]{213.3,238.0,210.0} 0.549 & \cellcolor[RGB]{237.4,247.8,236.0} 0.116 & \cellcolor[RGB]{255.0,255.0,255.0} 0.00000 & \cellcolor[RGB]{179.6,224.2,173.7} 0.998 &  & \cellcolor[RGB]{255,247,196} $\textit{SXR}_{\text{core}}$ & \cellcolor[RGB]{222.7,241.8,220.2} 0.425 & \cellcolor[RGB]{253.1,254.2,252.9} 0.013 & \cellcolor[RGB]{246.5,251.5,245.8} 0.00017 & \cellcolor[RGB]{216.5,239.3,213.5} 0.852 \\
\cellcolor[RGB]{209,235,255} $\text{PD}_{\text{FFT}_{10\textit{-p}}}^{{\textit{CIII}}}$ & \cellcolor[RGB]{209.0,236.2,205.4} 0.605 & \cellcolor[RGB]{226.0,243.2,223.7} 0.191 & \cellcolor[RGB]{255.0,255.0,255.0} 0.00000 & \cellcolor[RGB]{179.0,224.0,173.0} 1.000 &  & \cellcolor[RGB]{255,247,196} $\text{max}(T'_{e,\text{edge}})$ & \cellcolor[RGB]{210.5,236.9,207.0} 0.585 & \cellcolor[RGB]{255.0,255.0,255.0} 0.000 & \cellcolor[RGB]{238.8,248.4,237.5} 0.00032 & \cellcolor[RGB]{213.4,238.0,210.2} 0.864 \\
\cellcolor[RGB]{209,235,255} $\text{PD}_{\text{FFT}_{10\textit{-c}}}^{{\textit{CIII}}}$ & \cellcolor[RGB]{208.2,235.9,204.5} 0.616 & \cellcolor[RGB]{225.5,242.9,223.1} 0.194 & \cellcolor[RGB]{255.0,255.0,255.0} 0.00000 & \cellcolor[RGB]{179.6,224.2,173.7} 0.998 &  & \cellcolor[RGB]{255,247,196} $\text{max}(T''_{e,\text{edge}})$ & \cellcolor[RGB]{211.0,237.0,207.5} 0.579 & \cellcolor[RGB]{254.7,254.9,254.6} 0.002 & \cellcolor[RGB]{253.0,254.2,252.8} 0.00004 & \cellcolor[RGB]{213.4,238.0,210.2} 0.864 \\
\cellcolor[RGB]{209,235,255} $\text{PD}_{\text{FFT}_{10\textit{-f}}}^{{\textit{CIII}}}$ & \cellcolor[RGB]{207.6,235.7,203.8} 0.624 & \cellcolor[RGB]{225.9,243.1,223.6} 0.192 & \cellcolor[RGB]{255.0,255.0,255.0} 0.00000 & \cellcolor[RGB]{180.4,224.6,174.5} 0.995 &  & \cellcolor[RGB]{255,247,196} $T_{e,0}$ & \cellcolor[RGB]{224.8,242.7,222.5} 0.397 & \cellcolor[RGB]{254.7,254.9,254.7} 0.002 & \cellcolor[RGB]{255.0,255.0,255.0} 0.00000 & \cellcolor[RGB]{194.4,230.3,189.6} 0.939 \\
\cellcolor[RGB]{209,235,255} $\text{PD}_{\text{FFT}_{20\textit{-p}}}^{{\textit{CIII}}}$ & \cellcolor[RGB]{207.1,235.5,203.4} 0.630 & \cellcolor[RGB]{218.6,240.1,215.7} 0.240 & \cellcolor[RGB]{255.0,255.0,255.0} 0.00000 & \cellcolor[RGB]{179.0,224.0,173.0} 1.000 &  & \cellcolor[RGB]{255,255,234} $P_{\textit{in}}$ & \cellcolor[RGB]{219.1,240.3,216.2} 0.473 & \cellcolor[RGB]{187.9,227.6,182.6} 0.442 & \cellcolor[RGB]{255.0,255.0,255.0} 0.00000 & \cellcolor[RGB]{179.9,224.4,173.9} 0.997 \\
\cellcolor[RGB]{209,235,255} $\text{PD}_{\text{FFT}_{20\textit{-c}}}^{{\textit{CIII}}}$ & \cellcolor[RGB]{205.7,234.9,201.8} 0.649 & \cellcolor[RGB]{217.9,239.9,214.9} 0.244 & \cellcolor[RGB]{255.0,255.0,255.0} 0.00000 & \cellcolor[RGB]{180.4,224.6,174.5} 0.995 &  & \cellcolor[RGB]{255,255,234} $P_{\textit{OHM}}$ & \cellcolor[RGB]{249.6,252.8,249.2} 0.071 & \cellcolor[RGB]{255.0,255.0,255.0} 0.000 & \cellcolor[RGB]{255.0,255.0,255.0} 0.00000 & \cellcolor[RGB]{179.9,224.4,173.9} 0.997 \\
\cellcolor[RGB]{209,235,255} $\text{PD}_{\text{FFT}_{20\textit{-f}}}^{{\textit{CIII}}}$ & \cellcolor[RGB]{204.9,234.6,200.9} 0.660 & \cellcolor[RGB]{217.4,239.7,214.4} 0.247 & \cellcolor[RGB]{255.0,255.0,255.0} 0.00000 & \cellcolor[RGB]{182.0,225.2,176.3} 0.988 &  & \cellcolor[RGB]{255,255,234} $P_{\textit{NBI}}$ & \cellcolor[RGB]{219.3,240.4,216.5} 0.470 & \cellcolor[RGB]{254.9,254.9,254.9} 0.001 & \cellcolor[RGB]{255.0,255.0,255.0} 0.00000 & \cellcolor[RGB]{179.0,224.0,173.0} 1.000 \\
\cellcolor[RGB]{209,235,255} $\text{PD}_{\text{FFT}_{50\textit{-p}}}^{{\textit{CIII}}}$ & \cellcolor[RGB]{209.1,236.3,205.5} 0.604 & \cellcolor[RGB]{210.3,236.8,206.8} 0.294 & \cellcolor[RGB]{255.0,255.0,255.0} 0.00000 & \cellcolor[RGB]{179.0,224.0,173.0} 1.000 &  & \cellcolor[RGB]{255,255,234} $P_{\textit{NBI2}}$ & \cellcolor[RGB]{253.5,254.4,253.4} 0.020 & \cellcolor[RGB]{255.0,255.0,255.0} 0.000 & \cellcolor[RGB]{254.9,255.0,254.9} 0.00000 & \cellcolor[RGB]{179.0,224.0,173.0} 1.000 \\
\cellcolor[RGB]{209,235,255} $\text{PD}_{\text{FFT}_{50\textit{-c}}}^{{\textit{CIII}}}$ & \cellcolor[RGB]{205.8,234.9,201.9} 0.647 & \cellcolor[RGB]{203.6,234.0,199.5} 0.338 & \cellcolor[RGB]{255.0,255.0,255.0} 0.00000 & \cellcolor[RGB]{182.9,225.6,177.2} 0.985 &  & \cellcolor[RGB]{255,255,234} $P_{\textit{ECRH}}$ & \cellcolor[RGB]{238.8,248.4,237.5} 0.213 & \cellcolor[RGB]{255.0,255.0,255.0} 0.000 & \cellcolor[RGB]{255.0,255.0,255.0} 0.00000 & \cellcolor[RGB]{179.0,224.0,173.0} 1.000 \\
\cellcolor[RGB]{209,235,255} $\text{PD}_{\text{FFT}_{50\textit{-f}}}^{{\textit{CIII}}}$ & \cellcolor[RGB]{204.6,234.4,200.6} 0.663 & \cellcolor[RGB]{197.8,231.7,193.3} 0.376 & \cellcolor[RGB]{255.0,255.0,255.0} 0.00000 & \cellcolor[RGB]{187.0,227.3,181.7} 0.968 &  & \cellcolor[RGB]{255,255,234} $P_{\textit{LH}}$ & \cellcolor[RGB]{222.7,241.8,220.1} 0.425 & \cellcolor[RGB]{255.0,255.0,255.0} 0.000 & \cellcolor[RGB]{254.6,254.8,254.5} 0.00001 & \cellcolor[RGB]{179.9,224.4,173.9} 0.997 \\
\cellcolor[RGB]{209,235,255} $\text{PD}_{\text{FFT}_{100\textit{-p}}}^{{\textit{CIII}}}$ & \cellcolor[RGB]{217.7,239.8,214.7} 0.491 & \cellcolor[RGB]{216.4,239.3,213.4} 0.254 & \cellcolor[RGB]{255.0,255.0,255.0} 0.00000 & \cellcolor[RGB]{179.0,224.0,173.0} 1.000 &  & \cellcolor[RGB]{255,246,219} $\beta_{N}$ & \cellcolor[RGB]{209.8,236.6,206.2} 0.595 & \cellcolor[RGB]{241.6,249.5,240.5} 0.088 & \cellcolor[RGB]{255.0,255.0,255.0} 0.00000 & \cellcolor[RGB]{179.9,224.4,173.9} 0.997 \\
\cellcolor[RGB]{209,235,255} $\text{PD}_{\text{FFT}_{100\textit{-c}}}^{{\textit{CIII}}}$ & \cellcolor[RGB]{210.0,236.7,206.5} 0.592 & \cellcolor[RGB]{201.0,233.0,196.7} 0.355 & \cellcolor[RGB]{255.0,255.0,255.0} 0.00000 & \cellcolor[RGB]{187.0,227.3,181.7} 0.968 &  & \cellcolor[RGB]{255,246,219} $\beta_{p}$ & \cellcolor[RGB]{222.7,241.8,220.2} 0.425 & \cellcolor[RGB]{250.8,253.3,250.5} 0.028 & \cellcolor[RGB]{255.0,255.0,255.0} 0.00000 & \cellcolor[RGB]{179.9,224.4,173.9} 0.997 \\
\cellcolor[RGB]{209,235,255} $\text{PD}_{\text{FFT}_{100\textit{-f}}}^{{\textit{CIII}}}$ & \cellcolor[RGB]{210.0,236.6,206.4} 0.592 & \cellcolor[RGB]{198.5,231.9,194.0} 0.372 & \cellcolor[RGB]{255.0,255.0,255.0} 0.00000 & \cellcolor[RGB]{195.4,230.7,190.7} 0.935 &  & \cellcolor[RGB]{255,246,219} $\beta_{t}$ & \cellcolor[RGB]{198.6,232.0,194.2} 0.742 & \cellcolor[RGB]{191.9,229.3,186.9} 0.415 & \cellcolor[RGB]{255.0,255.0,255.0} 0.00000 & \cellcolor[RGB]{179.9,224.4,173.9} 0.997 \\
\cellcolor[RGB]{209,235,255} $\text{PD}_{\text{FFT}_{5\textit{-p}}}^{{\textit{H}\alpha}}$ & \cellcolor[RGB]{207.8,235.8,204.1} 0.621 & \cellcolor[RGB]{254.9,255.0,254.9} 0.000 & \cellcolor[RGB]{251.2,253.5,250.9} 0.00007 & \cellcolor[RGB]{179.0,224.0,173.0} 1.000 &  & \cellcolor[RGB]{255,246,219} ${W}_{\textit{tot}}$ & \cellcolor[RGB]{202.1,233.4,197.9} 0.696 & \cellcolor[RGB]{208.4,236.0,204.8} 0.306 & \cellcolor[RGB]{254.8,254.9,254.8} 0.00000 & \cellcolor[RGB]{179.9,224.4,173.9} 0.997 \\
\cellcolor[RGB]{209,235,255} $\text{PD}_{\text{FFT}_{5\textit{-c}}}^{{\textit{H}\alpha}}$ & \cellcolor[RGB]{207.6,235.7,203.9} 0.624 & \cellcolor[RGB]{255.0,255.0,255.0} 0.000 & \cellcolor[RGB]{216.0,239.1,212.9} 0.00077 & \cellcolor[RGB]{179.3,224.1,173.3} 0.999 &  & \cellcolor[RGB]{255,246,219} $\text{DML}$ & \cellcolor[RGB]{226.0,243.2,223.7} 0.381 & \cellcolor[RGB]{254.9,254.9,254.8} 0.001 & \cellcolor[RGB]{255.0,255.0,255.0} 0.00000 & \cellcolor[RGB]{195.5,230.7,190.8} 0.935 \\
\cellcolor[RGB]{209,235,255} $\text{PD}_{\text{FFT}_{5\textit{-f}}}^{{\textit{H}\alpha}}$ & \cellcolor[RGB]{207.5,235.6,203.8} 0.625 & \cellcolor[RGB]{255.0,255.0,255.0} 0.000 & \cellcolor[RGB]{253.9,254.6,253.9} 0.00002 & \cellcolor[RGB]{179.6,224.2,173.7} 0.998 &  & \cellcolor[RGB]{255,246,219} $H_{\textit{98y2}}$ & \cellcolor[RGB]{242.4,249.8,241.4} 0.166 & \cellcolor[RGB]{249.7,252.8,249.3} 0.035 & \cellcolor[RGB]{254.0,254.6,254.0} 0.00002 & \cellcolor[RGB]{179.9,224.4,173.9} 0.997 \\
\cellcolor[RGB]{209,235,255} $\text{PD}_{\text{FFT}_{10\textit{-p}}}^{{\textit{H}\alpha}}$ & \cellcolor[RGB]{203.2,233.9,199.1} 0.681 & \cellcolor[RGB]{255.0,255.0,255.0} 0.000 & \cellcolor[RGB]{245.0,250.9,244.2} 0.00020 & \cellcolor[RGB]{179.0,224.0,173.0} 1.000 &  & \cellcolor[RGB]{255,248,255} $P_{\textit{rad}}$ & \cellcolor[RGB]{231.8,245.5,230.0} 0.305 & \cellcolor[RGB]{254.9,255.0,254.9} 0.000 & \cellcolor[RGB]{251.8,253.7,251.6} 0.00006 & \cellcolor[RGB]{215.2,238.7,212.0} 0.857 \\
\cellcolor[RGB]{209,235,255} $\text{PD}_{\text{FFT}_{10\textit{-c}}}^{{\textit{H}\alpha}}$ & \cellcolor[RGB]{202.6,233.6,198.5} 0.689 & \cellcolor[RGB]{255.0,255.0,255.0} 0.000 & \cellcolor[RGB]{253.2,254.3,253.1} 0.00003 & \cellcolor[RGB]{179.6,224.2,173.7} 0.998 &  & \cellcolor[RGB]{255,248,255} $P_{\textit{rad},\text{bulk}}$ & \cellcolor[RGB]{233.7,246.3,232.0} 0.281 & \cellcolor[RGB]{254.4,254.8,254.4} 0.004 & \cellcolor[RGB]{254.8,254.9,254.8} 0.00000 & \cellcolor[RGB]{215.3,238.8,212.1} 0.857 \\
\cellcolor[RGB]{209,235,255} $\text{PD}_{\text{FFT}_{10\textit{-f}}}^{{\textit{H}\alpha}}$ & \cellcolor[RGB]{202.2,233.4,198.0} 0.695 & \cellcolor[RGB]{254.8,254.9,254.8} 0.001 & \cellcolor[RGB]{254.5,254.8,254.5} 0.00001 & \cellcolor[RGB]{180.4,224.6,174.5} 0.995 &  & \cellcolor[RGB]{255,248,255} $P_{\textit{rad},\text{SOL}}$ & \cellcolor[RGB]{217.4,239.6,214.4} 0.495 & \cellcolor[RGB]{255.0,255.0,255.0} 0.000 & \cellcolor[RGB]{255.0,255.0,255.0} 0.00000 & \cellcolor[RGB]{207.0,235.4,203.2} 0.889 \\
\cellcolor[RGB]{209,235,255} $\text{PD}_{\text{FFT}_{20\textit{-p}}}^{{\textit{H}\alpha}}$ & \cellcolor[RGB]{201.8,233.3,197.6} 0.700 & \cellcolor[RGB]{255.0,255.0,255.0} 0.000 & \cellcolor[RGB]{250.2,253.0,249.8} 0.00010 & \cellcolor[RGB]{179.0,224.0,173.0} 1.000 &  & \cellcolor[RGB]{235,245,255} $l_i$ & \cellcolor[RGB]{225.0,242.8,222.7} 0.394 & \cellcolor[RGB]{254.9,254.9,254.9} 0.001 & \cellcolor[RGB]{255.0,255.0,255.0} 0.00000 & \cellcolor[RGB]{179.9,224.4,173.9} 0.997 \\
\cellcolor[RGB]{209,235,255} $\text{PD}_{\text{FFT}_{20\textit{-c}}}^{{\textit{H}\alpha}}$ & \cellcolor[RGB]{200.2,232.6,195.9} 0.721 & \cellcolor[RGB]{255.0,255.0,255.0} 0.000 & \cellcolor[RGB]{249.6,252.8,249.2} 0.00011 & \cellcolor[RGB]{180.4,224.6,174.5} 0.995 &  & \cellcolor[RGB]{235,245,255} $Z_{\textit{eff}}$ & \cellcolor[RGB]{255.0,255.0,255.0} 0.000 & \cellcolor[RGB]{254.8,254.9,254.8} 0.001 & \cellcolor[RGB]{255.0,255.0,255.0} 0.00000 & \cellcolor[RGB]{202.0,233.4,197.8} 0.909 \\
\cellcolor[RGB]{209,235,255} $\text{PD}_{\text{FFT}_{20\textit{-f}}}^{{\textit{H}\alpha}}$ & \cellcolor[RGB]{199.6,232.4,195.2} 0.729 & \cellcolor[RGB]{254.9,255.0,254.9} 0.001 & \cellcolor[RGB]{255.0,255.0,255.0} 0.00000 & \cellcolor[RGB]{182.0,225.2,176.3} 0.988 &  & \cellcolor[RGB]{235,245,255} $\nu_{e,\text{ped}}^{*}$ & \cellcolor[RGB]{222.9,241.9,220.4} 0.422 & \cellcolor[RGB]{254.8,254.9,254.8} 0.001 & \cellcolor[RGB]{255.0,255.0,255.0} 0.00000 & \cellcolor[RGB]{207.4,235.6,203.7} 0.888 \\
\cellcolor[RGB]{209,235,255} $\text{PD}_{\text{FFT}_{50\textit{-p}}}^{{\textit{H}\alpha}}$ & \cellcolor[RGB]{205.3,234.7,201.4} 0.654 & \cellcolor[RGB]{254.9,255.0,254.9} 0.001 & \cellcolor[RGB]{244.8,250.8,244.0} 0.00020 & \cellcolor[RGB]{179.0,224.0,173.0} 1.000 &  & \cellcolor[RGB]{235,245,255} $V_{\textit{loop}}$ & \cellcolor[RGB]{233.1,246.0,231.3} 0.289 & \cellcolor[RGB]{254.4,254.8,254.4} 0.004 & \cellcolor[RGB]{255.0,255.0,255.0} 0.00000 & \cellcolor[RGB]{179.0,224.0,173.0} 1.000

\\
\caption{Availability and discriminative capability of individual features, as used for constructing the ensemble feature sets. Features are colored by their categorization. For each feature we fit a depth-2 decision tree classifying all confinement states to get a general overview. Additionally, to identify features of interest w.r.t. regions far from the discriminative boundary, we compute the optimal threshold for a feature where at least 99\% of the timeslices are in L-mode or H-mode, and take the fraction of data subject to this threshold as the metric value. Lastly, we report the fraction of data where the signal is available.}\label{tab:features}
\end{longtable}
\endgroup
\setlength\tabcolsep{6pt}
\arrayrulecolor{black}

\arrayrulecolor[rgb]{0.9,0.9,0.9}
\setlength\tabcolsep{5.3pt}
\begingroup
\centering
\begin{longtable}{l|p{2.7cm}|p{10.9cm}}
Model & Description & Feature set \\\cmidrule[\heavyrulewidth]{1-3}
\addlinespace[-\belowrulesep]
FNOLSTM-SH-1 & Shaping, top 4. & \colorbox[RGB]{255,210,204}{\strut $A_p$}, \colorbox[RGB]{255,210,204}{\strut $\delta_{\text{bottom}}$}, \colorbox[RGB]{255,210,204}{\strut $\kappa$}, \colorbox[RGB]{255,210,204}{\strut $Z_{\text{axis}}$} \\\hline
FNOLSTM-SH-2 & Shaping, all. & \colorbox[RGB]{255,210,204}{\strut $A_p$}, \colorbox[RGB]{255,210,204}{\strut $\delta_{\text{bottom}}$}, \colorbox[RGB]{255,210,204}{\strut $\delta_{\text{top}}$}, \colorbox[RGB]{255,210,204}{\strut $\Delta_{\text{in}}$}, \colorbox[RGB]{255,210,204}{\strut $\Delta_{\text{out}}$}, \colorbox[RGB]{255,210,204}{\strut $\kappa$}, \colorbox[RGB]{255,210,204}{\strut $R_0$}, \colorbox[RGB]{255,210,204}{\strut $a$}, \colorbox[RGB]{255,210,204}{\strut $R_{\text{axis}}$}, \colorbox[RGB]{255,210,204}{\strut $Z_{\text{axis}}$}, \colorbox[RGB]{255,210,204}{\strut $V_p$} \\\hline
FNOLSTM-EM-1 & Emission, top 4. & \colorbox[RGB]{209,235,255}{\strut $\text{PD}_{\textit{CIII}}^{}$}, \colorbox[RGB]{209,235,255}{\strut $\text{PD}_{\textit{H}\alpha}^{}$}, \colorbox[RGB]{209,235,255}{\strut $\text{PD}_{\text{FFT}_{20\textit{-c}}}^{{\textit{H}\alpha}}$}, \colorbox[RGB]{209,235,255}{\strut $\text{PD}_{\text{FFT}_{50\textit{-c}}}^{{\textit{H}\alpha}}$} \\\hline
FNOLSTM-EM-2 & Emission, all. & \colorbox[RGB]{209,235,255}{\strut $\text{PD}_{\textit{CIII}}^{}$}, \colorbox[RGB]{209,235,255}{\strut $\text{PD}_{\textit{H}\alpha}^{}$}, \colorbox[RGB]{209,235,255}{\strut $\text{PD}_{\text{FFT}_{5\textit{-c}}}^{{\textit{CIII}}}$}, \colorbox[RGB]{209,235,255}{\strut $\text{PD}_{\text{FFT}_{10\textit{-c}}}^{{\textit{CIII}}}$}, \colorbox[RGB]{209,235,255}{\strut $\text{PD}_{\text{FFT}_{20\textit{-c}}}^{{\textit{CIII}}}$}, \colorbox[RGB]{209,235,255}{\strut $\text{PD}_{\text{FFT}_{50\textit{-c}}}^{{\textit{CIII}}}$}, \colorbox[RGB]{209,235,255}{\strut $\text{PD}_{\text{FFT}_{100\textit{-c}}}^{{\textit{CIII}}}$}, \colorbox[RGB]{209,235,255}{\strut $\text{PD}_{\text{FFT}_{5\textit{-c}}}^{{\textit{H}\alpha}}$}, \colorbox[RGB]{209,235,255}{\strut $\text{PD}_{\text{FFT}_{10\textit{-c}}}^{{\textit{H}\alpha}}$}, \colorbox[RGB]{209,235,255}{\strut $\text{PD}_{\text{FFT}_{20\textit{-c}}}^{{\textit{H}\alpha}}$}, \colorbox[RGB]{209,235,255}{\strut $\text{PD}_{\text{FFT}_{50\textit{-c}}}^{{\textit{H}\alpha}}$}, \colorbox[RGB]{209,235,255}{\strut $\text{PD}_{\text{FFT}_{100\textit{-c}}}^{{\textit{H}\alpha}}$} \\\hline
FNOLSTM-EM-3 & Emission, all FFT features. & \colorbox[RGB]{209,235,255}{\strut $\text{PD}_{\text{FFT}_{5\textit{-p}}}^{{\textit{CIII}}}$}, \colorbox[RGB]{209,235,255}{\strut $\text{PD}_{\text{FFT}_{5\textit{-c}}}^{{\textit{CIII}}}$}, \colorbox[RGB]{209,235,255}{\strut $\text{PD}_{\text{FFT}_{5\textit{-f}}}^{{\textit{CIII}}}$}, \colorbox[RGB]{209,235,255}{\strut $\text{PD}_{\text{FFT}_{10\textit{-p}}}^{{\textit{CIII}}}$}, \colorbox[RGB]{209,235,255}{\strut $\text{PD}_{\text{FFT}_{10\textit{-c}}}^{{\textit{CIII}}}$}, \colorbox[RGB]{209,235,255}{\strut $\text{PD}_{\text{FFT}_{10\textit{-f}}}^{{\textit{CIII}}}$}, \colorbox[RGB]{209,235,255}{\strut $\text{PD}_{\text{FFT}_{20\textit{-p}}}^{{\textit{CIII}}}$}, \colorbox[RGB]{209,235,255}{\strut $\text{PD}_{\text{FFT}_{20\textit{-c}}}^{{\textit{CIII}}}$}, \colorbox[RGB]{209,235,255}{\strut $\text{PD}_{\text{FFT}_{20\textit{-f}}}^{{\textit{CIII}}}$}, \colorbox[RGB]{209,235,255}{\strut $\text{PD}_{\text{FFT}_{50\textit{-p}}}^{{\textit{CIII}}}$}, \colorbox[RGB]{209,235,255}{\strut $\text{PD}_{\text{FFT}_{50\textit{-c}}}^{{\textit{CIII}}}$}, \colorbox[RGB]{209,235,255}{\strut $\text{PD}_{\text{FFT}_{50\textit{-f}}}^{{\textit{CIII}}}$}, \colorbox[RGB]{209,235,255}{\strut $\text{PD}_{\text{FFT}_{100\textit{-p}}}^{{\textit{CIII}}}$}, \colorbox[RGB]{209,235,255}{\strut $\text{PD}_{\text{FFT}_{100\textit{-c}}}^{{\textit{CIII}}}$}, \colorbox[RGB]{209,235,255}{\strut $\text{PD}_{\text{FFT}_{100\textit{-f}}}^{{\textit{CIII}}}$}, \colorbox[RGB]{209,235,255}{\strut $\text{PD}_{\text{FFT}_{5\textit{-p}}}^{{\textit{H}\alpha}}$}, \colorbox[RGB]{209,235,255}{\strut $\text{PD}_{\text{FFT}_{5\textit{-c}}}^{{\textit{H}\alpha}}$}, \colorbox[RGB]{209,235,255}{\strut $\text{PD}_{\text{FFT}_{5\textit{-f}}}^{{\textit{H}\alpha}}$}, \colorbox[RGB]{209,235,255}{\strut $\text{PD}_{\text{FFT}_{10\textit{-p}}}^{{\textit{H}\alpha}}$}, \colorbox[RGB]{209,235,255}{\strut $\text{PD}_{\text{FFT}_{10\textit{-c}}}^{{\textit{H}\alpha}}$}, \colorbox[RGB]{209,235,255}{\strut $\text{PD}_{\text{FFT}_{10\textit{-f}}}^{{\textit{H}\alpha}}$}, \colorbox[RGB]{209,235,255}{\strut $\text{PD}_{\text{FFT}_{20\textit{-p}}}^{{\textit{H}\alpha}}$}, \colorbox[RGB]{209,235,255}{\strut $\text{PD}_{\text{FFT}_{20\textit{-c}}}^{{\textit{H}\alpha}}$}, \colorbox[RGB]{209,235,255}{\strut $\text{PD}_{\text{FFT}_{20\textit{-f}}}^{{\textit{H}\alpha}}$}, \colorbox[RGB]{209,235,255}{\strut $\text{PD}_{\text{FFT}_{50\textit{-p}}}^{{\textit{H}\alpha}}$}, \colorbox[RGB]{209,235,255}{\strut $\text{PD}_{\text{FFT}_{50\textit{-c}}}^{{\textit{H}\alpha}}$}, \colorbox[RGB]{209,235,255}{\strut $\text{PD}_{\text{FFT}_{50\textit{-f}}}^{{\textit{H}\alpha}}$}, \colorbox[RGB]{209,235,255}{\strut $\text{PD}_{\text{FFT}_{100\textit{-p}}}^{{\textit{H}\alpha}}$}, \colorbox[RGB]{209,235,255}{\strut $\text{PD}_{\text{FFT}_{100\textit{-c}}}^{{\textit{H}\alpha}}$}, \colorbox[RGB]{209,235,255}{\strut $\text{PD}_{\text{FFT}_{100\textit{-f}}}^{{\textit{H}\alpha}}$} \\\hline
FNOLSTM-MA-1 & Magnetics, all. & \colorbox[RGB]{234,255,227}{\strut $B_0$}, \colorbox[RGB]{234,255,227}{\strut $I_{p}$}, \colorbox[RGB]{234,255,227}{\strut $I_{p,\textit{ref}}$}, \colorbox[RGB]{234,255,227}{\strut $q_{95}$} \\\hline
FNOLSTM-DE-1 & Density, top 4. & \colorbox[RGB]{252,233,255}{\strut $n_{e,\text{core}}$}, \colorbox[RGB]{252,233,255}{\strut $n_{e,\text{LFS}}$}, \colorbox[RGB]{252,233,255}{\strut $\text{max}(n'_{e,\text{edge}})$}, \colorbox[RGB]{252,233,255}{\strut $\text{max}(n''_{e,\text{edge}})$} \\\hline
FNOLSTM-DE-2 & Density, all. & \colorbox[RGB]{252,233,255}{\strut $n_{e,\text{core}}$}, \colorbox[RGB]{252,233,255}{\strut $n_{e,\text{LFS}}$}, \colorbox[RGB]{252,233,255}{\strut $n_e/n_{\textit{GW}}$}, \colorbox[RGB]{252,233,255}{\strut $\text{max}(n'_{e,\text{edge}})$}, \colorbox[RGB]{252,233,255}{\strut $\text{max}(n''_{e,\text{edge}})$}, \colorbox[RGB]{252,233,255}{\strut $n_{e,0}$} \\\hline
FNOLSTM-TE-1 & Temperature, all. & \colorbox[RGB]{255,247,196}{\strut $\textit{SXR}_{\text{core}}$}, \colorbox[RGB]{255,247,196}{\strut $\text{max}(T'_{e,\text{edge}})$}, \colorbox[RGB]{255,247,196}{\strut $\text{max}(T''_{e,\text{edge}})$}, \colorbox[RGB]{255,247,196}{\strut $T_{e,0}$} \\\hline
FNOLSTM-PO-1 & Power, top 4. & \colorbox[RGB]{255,255,234}{\strut $P_{\textit{in}}$}, \colorbox[RGB]{255,255,234}{\strut $P_{\textit{NBI}}$}, \colorbox[RGB]{255,255,234}{\strut $P_{\textit{ECRH}}$}, \colorbox[RGB]{255,255,234}{\strut $P_{\textit{LH}}$} \\\hline
FNOLSTM-PO-2 & Power, all. & \colorbox[RGB]{255,255,234}{\strut $P_{\textit{in}}$}, \colorbox[RGB]{255,255,234}{\strut $P_{\textit{OHM}}$}, \colorbox[RGB]{255,255,234}{\strut $P_{\textit{NBI}}$}, \colorbox[RGB]{255,255,234}{\strut $P_{\textit{NBI2}}$}, \colorbox[RGB]{255,255,234}{\strut $P_{\textit{ECRH}}$}, \colorbox[RGB]{255,255,234}{\strut $P_{\textit{LH}}$} \\\hline
FNOLSTM-EN-1 & Energy content, top 4. & \colorbox[RGB]{255,246,219}{\strut $\beta_{N}$}, \colorbox[RGB]{255,246,219}{\strut $\beta_{p}$}, \colorbox[RGB]{255,246,219}{\strut $\beta_{t}$}, \colorbox[RGB]{255,246,219}{\strut ${W}_{\textit{tot}}$} \\\hline
FNOLSTM-EN-2 & Energy content, all. & \colorbox[RGB]{255,246,219}{\strut $\beta_{N}$}, \colorbox[RGB]{255,246,219}{\strut $\beta_{p}$}, \colorbox[RGB]{255,246,219}{\strut $\beta_{t}$}, \colorbox[RGB]{255,246,219}{\strut ${W}_{\textit{tot}}$}, \colorbox[RGB]{255,246,219}{\strut $\text{DML}$}, \colorbox[RGB]{255,246,219}{\strut $H_{\textit{98y2}}$} \\\hline
FNOLSTM-RA-1 & Radiation, top 2. & \colorbox[RGB]{255,248,255}{\strut $P_{\textit{rad}}$}, \colorbox[RGB]{255,248,255}{\strut $P_{\textit{rad},\text{SOL}}$} \\\hline
FNOLSTM-RA-2 & Radiation, all. & \colorbox[RGB]{255,248,255}{\strut $P_{\textit{rad}}$}, \colorbox[RGB]{255,248,255}{\strut $P_{\textit{rad},\text{bulk}}$}, \colorbox[RGB]{255,248,255}{\strut $P_{\textit{rad},\text{SOL}}$} \\\hline
FNOLSTM-OT-1 & Other, all. & \colorbox[RGB]{235,245,255}{\strut $l_i$}, \colorbox[RGB]{235,245,255}{\strut $Z_{\textit{eff}}$}, \colorbox[RGB]{235,245,255}{\strut $\nu_{e,\text{ped}}^{*}$}, \colorbox[RGB]{235,245,255}{\strut $V_{\textit{loop}}$} \\\hline
FNOLSTM-$\ast\ast$-1 & Mixed, rank 1. & \colorbox[RGB]{255,210,204}{\strut $\kappa$}, \colorbox[RGB]{209,235,255}{\strut $\text{PD}_{\textit{CIII}}^{}$}, \colorbox[RGB]{234,255,227}{\strut $q_{95}$}, \colorbox[RGB]{252,233,255}{\strut $\text{max}(n''_{e,\text{edge}})$}, \colorbox[RGB]{255,247,196}{\strut $\text{max}(T'_{e,\text{edge}})$}, \colorbox[RGB]{255,255,234}{\strut $P_{\textit{in}}$}, \colorbox[RGB]{255,246,219}{\strut $\beta_{t}$}, \colorbox[RGB]{255,248,255}{\strut $P_{\textit{rad},\text{SOL}}$}, \colorbox[RGB]{235,245,255}{\strut $\nu_{e,\text{ped}}^{*}$} \\\hline
FNOLSTM-$\ast\ast$-2 & Mixed, rank 2. & \colorbox[RGB]{255,210,204}{\strut $\delta_{\text{bottom}}$}, \colorbox[RGB]{209,235,255}{\strut $\text{PD}_{\textit{H}\alpha}^{}$}, \colorbox[RGB]{234,255,227}{\strut $I_{p,\textit{ref}}$}, \colorbox[RGB]{252,233,255}{\strut $n_{e,\text{LFS}}$}, \colorbox[RGB]{255,247,196}{\strut $\text{max}(T''_{e,\text{edge}})$}, \colorbox[RGB]{255,255,234}{\strut $P_{\textit{NBI}}$}, \colorbox[RGB]{255,246,219}{\strut ${W}_{\textit{tot}}$}, \colorbox[RGB]{255,248,255}{\strut $P_{\textit{rad}}$}, \colorbox[RGB]{235,245,255}{\strut $l_i$} \\\hline
FNOLSTM-$\ast\ast$-3 & Mixed, rank 3. & \colorbox[RGB]{255,210,204}{\strut $Z_{\text{axis}}$}, \colorbox[RGB]{209,235,255}{\strut $\text{PD}_{\text{FFT}_{20\textit{-c}}}^{{\textit{H}\alpha}}$}, \colorbox[RGB]{234,255,227}{\strut $I_{p}$}, \colorbox[RGB]{252,233,255}{\strut $\text{max}(n'_{e,\text{edge}})$}, \colorbox[RGB]{255,247,196}{\strut $\textit{SXR}_{\text{core}}$}, \colorbox[RGB]{255,255,234}{\strut $P_{\textit{LH}}$}, \colorbox[RGB]{255,246,219}{\strut $\beta_{N}$}, \colorbox[RGB]{235,245,255}{\strut $V_{\textit{loop}}$} \\\hline
FNOLSTM-$\ast\ast$-4 & Mixed, top 2. & \colorbox[RGB]{255,210,204}{\strut $\delta_{\text{bottom}}$}, \colorbox[RGB]{255,210,204}{\strut $\kappa$}, \colorbox[RGB]{209,235,255}{\strut $\text{PD}_{\textit{CIII}}^{}$}, \colorbox[RGB]{209,235,255}{\strut $\text{PD}_{\textit{H}\alpha}^{}$}, \colorbox[RGB]{234,255,227}{\strut $I_{p,\textit{ref}}$}, \colorbox[RGB]{234,255,227}{\strut $q_{95}$}, \colorbox[RGB]{252,233,255}{\strut $n_{e,\text{LFS}}$}, \colorbox[RGB]{252,233,255}{\strut $\text{max}(n''_{e,\text{edge}})$}, \colorbox[RGB]{255,247,196}{\strut $\text{max}(T'_{e,\text{edge}})$}, \colorbox[RGB]{255,247,196}{\strut $\text{max}(T''_{e,\text{edge}})$}, \colorbox[RGB]{255,255,234}{\strut $P_{\textit{in}}$}, \colorbox[RGB]{255,255,234}{\strut $P_{\textit{NBI}}$}, \colorbox[RGB]{255,246,219}{\strut $\beta_{t}$}, \colorbox[RGB]{255,246,219}{\strut ${W}_{\textit{tot}}$}, \colorbox[RGB]{255,248,255}{\strut $P_{\textit{rad}}$}, \colorbox[RGB]{255,248,255}{\strut $P_{\textit{rad},\text{SOL}}$}, \colorbox[RGB]{235,245,255}{\strut $l_i$}, \colorbox[RGB]{235,245,255}{\strut $\nu_{e,\text{ped}}^{*}$} \\\hline
FNOLSTM-$\ast\ast$-5 & Mixed, top 3. & \colorbox[RGB]{255,210,204}{\strut $\delta_{\text{bottom}}$}, \colorbox[RGB]{255,210,204}{\strut $\kappa$}, \colorbox[RGB]{255,210,204}{\strut $Z_{\text{axis}}$}, \colorbox[RGB]{209,235,255}{\strut $\text{PD}_{\textit{CIII}}^{}$}, \colorbox[RGB]{209,235,255}{\strut $\text{PD}_{\textit{H}\alpha}^{}$}, \colorbox[RGB]{209,235,255}{\strut $\text{PD}_{\text{FFT}_{20\textit{-c}}}^{{\textit{H}\alpha}}$}, \colorbox[RGB]{234,255,227}{\strut $I_{p}$}, \colorbox[RGB]{234,255,227}{\strut $I_{p,\textit{ref}}$}, \colorbox[RGB]{234,255,227}{\strut $q_{95}$}, \colorbox[RGB]{252,233,255}{\strut $n_{e,\text{LFS}}$}, \colorbox[RGB]{252,233,255}{\strut $\text{max}(n'_{e,\text{edge}})$}, \colorbox[RGB]{252,233,255}{\strut $\text{max}(n''_{e,\text{edge}})$}, \colorbox[RGB]{255,247,196}{\strut $\textit{SXR}_{\text{core}}$}, \colorbox[RGB]{255,247,196}{\strut $\text{max}(T'_{e,\text{edge}})$}, \colorbox[RGB]{255,247,196}{\strut $\text{max}(T''_{e,\text{edge}})$}, \colorbox[RGB]{255,255,234}{\strut $P_{\textit{in}}$}, \colorbox[RGB]{255,255,234}{\strut $P_{\textit{NBI}}$}, \colorbox[RGB]{255,255,234}{\strut $P_{\textit{LH}}$}, \colorbox[RGB]{255,246,219}{\strut $\beta_{N}$}, \colorbox[RGB]{255,246,219}{\strut $\beta_{t}$}, \colorbox[RGB]{255,246,219}{\strut ${W}_{\textit{tot}}$}, \colorbox[RGB]{255,248,255}{\strut $P_{\textit{rad}}$}, \colorbox[RGB]{255,248,255}{\strut $P_{\textit{rad},\text{SOL}}$}, \colorbox[RGB]{235,245,255}{\strut $l_i$}, \colorbox[RGB]{235,245,255}{\strut $\nu_{e,\text{ped}}^{*}$}, \colorbox[RGB]{235,245,255}{\strut $V_{\textit{loop}}$} \\\hline
FNOLSTM-$\ast\ast$-6 & Mixed, all. & \colorbox[RGB]{255,210,204}{\strut $A_p$}, \colorbox[RGB]{255,210,204}{\strut $\delta_{\text{bottom}}$}, \colorbox[RGB]{255,210,204}{\strut $\delta_{\text{top}}$}, \colorbox[RGB]{255,210,204}{\strut $\Delta_{\text{in}}$}, \colorbox[RGB]{255,210,204}{\strut $\Delta_{\text{out}}$}, \colorbox[RGB]{255,210,204}{\strut $\kappa$}, \colorbox[RGB]{255,210,204}{\strut $R_0$}, \colorbox[RGB]{255,210,204}{\strut $a$}, \colorbox[RGB]{255,210,204}{\strut $R_{\text{axis}}$}, \colorbox[RGB]{255,210,204}{\strut $Z_{\text{axis}}$}, \colorbox[RGB]{255,210,204}{\strut $V_p$}, \colorbox[RGB]{209,235,255}{\strut $\text{PD}_{\textit{CIII}}^{}$}, \colorbox[RGB]{209,235,255}{\strut $\text{PD}_{\textit{H}\alpha}^{}$}, \colorbox[RGB]{209,235,255}{\strut $\text{PD}_{\text{FFT}_{5\textit{-c}}}^{{\textit{CIII}}}$}, \colorbox[RGB]{209,235,255}{\strut $\text{PD}_{\text{FFT}_{10\textit{-c}}}^{{\textit{CIII}}}$}, \colorbox[RGB]{209,235,255}{\strut $\text{PD}_{\text{FFT}_{20\textit{-c}}}^{{\textit{CIII}}}$}, \colorbox[RGB]{209,235,255}{\strut $\text{PD}_{\text{FFT}_{50\textit{-c}}}^{{\textit{CIII}}}$}, \colorbox[RGB]{209,235,255}{\strut $\text{PD}_{\text{FFT}_{100\textit{-c}}}^{{\textit{CIII}}}$}, \colorbox[RGB]{209,235,255}{\strut $\text{PD}_{\text{FFT}_{5\textit{-c}}}^{{\textit{H}\alpha}}$}, \colorbox[RGB]{209,235,255}{\strut $\text{PD}_{\text{FFT}_{10\textit{-c}}}^{{\textit{H}\alpha}}$}, \colorbox[RGB]{209,235,255}{\strut $\text{PD}_{\text{FFT}_{20\textit{-c}}}^{{\textit{H}\alpha}}$}, \colorbox[RGB]{209,235,255}{\strut $\text{PD}_{\text{FFT}_{50\textit{-c}}}^{{\textit{H}\alpha}}$}, \colorbox[RGB]{209,235,255}{\strut $\text{PD}_{\text{FFT}_{100\textit{-c}}}^{{\textit{H}\alpha}}$}, \colorbox[RGB]{234,255,227}{\strut $B_0$}, \colorbox[RGB]{234,255,227}{\strut $I_{p}$}, \colorbox[RGB]{234,255,227}{\strut $I_{p,\textit{ref}}$}, \colorbox[RGB]{234,255,227}{\strut $q_{95}$}, \colorbox[RGB]{252,233,255}{\strut $n_{e,\text{core}}$}, \colorbox[RGB]{252,233,255}{\strut $n_{e,\text{LFS}}$}, \colorbox[RGB]{252,233,255}{\strut $n_e/n_{\textit{GW}}$}, \colorbox[RGB]{252,233,255}{\strut $\text{max}(n'_{e,\text{edge}})$}, \colorbox[RGB]{252,233,255}{\strut $\text{max}(n''_{e,\text{edge}})$}, \colorbox[RGB]{252,233,255}{\strut $n_{e,0}$}, \colorbox[RGB]{255,247,196}{\strut $\textit{SXR}_{\text{core}}$}, \colorbox[RGB]{255,247,196}{\strut $\text{max}(T'_{e,\text{edge}})$}, \colorbox[RGB]{255,247,196}{\strut $\text{max}(T''_{e,\text{edge}})$}, \colorbox[RGB]{255,247,196}{\strut $T_{e,0}$}, \colorbox[RGB]{255,255,234}{\strut $P_{\textit{in}}$}, \colorbox[RGB]{255,255,234}{\strut $P_{\textit{OHM}}$}, \colorbox[RGB]{255,255,234}{\strut $P_{\textit{NBI}}$}, \colorbox[RGB]{255,255,234}{\strut $P_{\textit{NBI2}}$}, \colorbox[RGB]{255,255,234}{\strut $P_{\textit{ECRH}}$}, \colorbox[RGB]{255,255,234}{\strut $P_{\textit{LH}}$}, \colorbox[RGB]{255,246,219}{\strut $\beta_{N}$}, \colorbox[RGB]{255,246,219}{\strut $\beta_{p}$}, \colorbox[RGB]{255,246,219}{\strut $\beta_{t}$}, \colorbox[RGB]{255,246,219}{\strut ${W}_{\textit{tot}}$}, \colorbox[RGB]{255,246,219}{\strut $\text{DML}$}, \colorbox[RGB]{255,246,219}{\strut $H_{\textit{98y2}}$}, \colorbox[RGB]{255,248,255}{\strut $P_{\textit{rad}}$}, \colorbox[RGB]{255,248,255}{\strut $P_{\textit{rad},\text{bulk}}$}, \colorbox[RGB]{255,248,255}{\strut $P_{\textit{rad},\text{SOL}}$}, \colorbox[RGB]{235,245,255}{\strut $l_i$}, \colorbox[RGB]{235,245,255}{\strut $Z_{\textit{eff}}$}, \colorbox[RGB]{235,245,255}{\strut $\nu_{e,\text{ped}}^{*}$}, \colorbox[RGB]{235,245,255}{\strut $V_{\textit{loop}}$} \\\hline
FNOLSTM-$\ast\ast$-7 & Mixed, rank 1 (by L-mode threshold). & \colorbox[RGB]{255,210,204}{\strut $\Delta_{\text{in}}$}, \colorbox[RGB]{209,235,255}{\strut $\text{PD}_{\textit{CIII}}^{}$}, \colorbox[RGB]{234,255,227}{\strut $I_{p,\textit{ref}}$}, \colorbox[RGB]{252,233,255}{\strut $n_{e,0}$}, \colorbox[RGB]{255,247,196}{\strut $\textit{SXR}_{\text{core}}$}, \colorbox[RGB]{255,255,234}{\strut $P_{\textit{in}}$}, \colorbox[RGB]{255,246,219}{\strut $\beta_{t}$}, \colorbox[RGB]{255,248,255}{\strut $P_{\textit{rad},\text{bulk}}$}, \colorbox[RGB]{235,245,255}{\strut $V_{\textit{loop}}$} \\\hline
FNOLSTM-$\ast\ast$-8 & Mixed, rank 2 (by L-mode threshold). & \colorbox[RGB]{255,210,204}{\strut $\kappa$}, \colorbox[RGB]{209,235,255}{\strut $\text{PD}_{\textit{H}\alpha}^{}$}, \colorbox[RGB]{234,255,227}{\strut $I_{p}$}, \colorbox[RGB]{252,233,255}{\strut $\text{max}(n'_{e,\text{edge}})$}, \colorbox[RGB]{255,247,196}{\strut $\text{max}(T''_{e,\text{edge}})$}, \colorbox[RGB]{255,255,234}{\strut $P_{\textit{NBI}}$}, \colorbox[RGB]{255,246,219}{\strut ${W}_{\textit{tot}}$}, \colorbox[RGB]{255,248,255}{\strut $P_{\textit{rad}}$}, \colorbox[RGB]{235,245,255}{\strut $Z_{\textit{eff}}$} \\\hline
FNOLSTM-$\ast\ast$-9 & Mixed, top 2 (by L-mode threshold). & \colorbox[RGB]{255,210,204}{\strut $\Delta_{\text{in}}$}, \colorbox[RGB]{255,210,204}{\strut $\kappa$}, \colorbox[RGB]{209,235,255}{\strut $\text{PD}_{\textit{CIII}}^{}$}, \colorbox[RGB]{209,235,255}{\strut $\text{PD}_{\textit{H}\alpha}^{}$}, \colorbox[RGB]{234,255,227}{\strut $I_{p}$}, \colorbox[RGB]{234,255,227}{\strut $I_{p,\textit{ref}}$}, \colorbox[RGB]{252,233,255}{\strut $\text{max}(n'_{e,\text{edge}})$}, \colorbox[RGB]{252,233,255}{\strut $n_{e,0}$}, \colorbox[RGB]{255,247,196}{\strut $\textit{SXR}_{\text{core}}$}, \colorbox[RGB]{255,247,196}{\strut $\text{max}(T''_{e,\text{edge}})$}, \colorbox[RGB]{255,255,234}{\strut $P_{\textit{in}}$}, \colorbox[RGB]{255,255,234}{\strut $P_{\textit{NBI}}$}, \colorbox[RGB]{255,246,219}{\strut $\beta_{t}$}, \colorbox[RGB]{255,246,219}{\strut ${W}_{\textit{tot}}$}, \colorbox[RGB]{255,248,255}{\strut $P_{\textit{rad}}$}, \colorbox[RGB]{255,248,255}{\strut $P_{\textit{rad},\text{bulk}}$}, \colorbox[RGB]{235,245,255}{\strut $Z_{\textit{eff}}$}, \colorbox[RGB]{235,245,255}{\strut $V_{\textit{loop}}$} \\\hline
FNOLSTM-$\ast\ast$-10 & Mixed, rank 2 (by H-mode threshold). & \colorbox[RGB]{255,210,204}{\strut $\delta_{\text{bottom}}$}, \colorbox[RGB]{209,235,255}{\strut $\text{PD}_{\textit{H}\alpha}^{}$}, \colorbox[RGB]{234,255,227}{\strut $I_{p}$}, \colorbox[RGB]{252,233,255}{\strut $n_{e,0}$}, \colorbox[RGB]{255,247,196}{\strut $\text{max}(T''_{e,\text{edge}})$}, \colorbox[RGB]{255,255,234}{\strut $P_{\textit{NBI2}}$}, \colorbox[RGB]{255,246,219}{\strut ${W}_{\textit{tot}}$}, \colorbox[RGB]{235,245,255}{\strut $V_{\textit{loop}}$} \\\hline
GBDT-SH-1 & Shaping, top 4. & \colorbox[RGB]{255,210,204}{\strut $A_p$}, \colorbox[RGB]{255,210,204}{\strut $\delta_{\text{bottom}}$}, \colorbox[RGB]{255,210,204}{\strut $\kappa$}, \colorbox[RGB]{255,210,204}{\strut $Z_{\text{axis}}$} \\\hline
GBDT-SH-2 & Shaping, all. & \colorbox[RGB]{255,210,204}{\strut $A_p$}, \colorbox[RGB]{255,210,204}{\strut $\delta_{\text{bottom}}$}, \colorbox[RGB]{255,210,204}{\strut $\delta_{\text{top}}$}, \colorbox[RGB]{255,210,204}{\strut $\Delta_{\text{in}}$}, \colorbox[RGB]{255,210,204}{\strut $\Delta_{\text{out}}$}, \colorbox[RGB]{255,210,204}{\strut $\kappa$}, \colorbox[RGB]{255,210,204}{\strut $R_0$}, \colorbox[RGB]{255,210,204}{\strut $a$}, \colorbox[RGB]{255,210,204}{\strut $R_{\text{axis}}$}, \colorbox[RGB]{255,210,204}{\strut $Z_{\text{axis}}$}, \colorbox[RGB]{255,210,204}{\strut $V_p$} \\\hline
GBDT-EM-1 & Emission, top 4. & \colorbox[RGB]{209,235,255}{\strut $\text{PD}_{\text{FFT}_{10\textit{-c}}}^{{\textit{H}\alpha}}$}, \colorbox[RGB]{209,235,255}{\strut $\text{PD}_{\text{FFT}_{20\textit{-c}}}^{{\textit{H}\alpha}}$}, \colorbox[RGB]{209,235,255}{\strut $\text{PD}_{\text{FFT}_{50\textit{-c}}}^{{\textit{H}\alpha}}$}, \colorbox[RGB]{209,235,255}{\strut $\text{PD}_{\text{FFT}_{100\textit{-c}}}^{{\textit{H}\alpha}}$} \\\hline
GBDT-EM-2 & Emission, all. & \colorbox[RGB]{209,235,255}{\strut $\text{PD}_{\text{FFT}_{5\textit{-c}}}^{{\textit{CIII}}}$}, \colorbox[RGB]{209,235,255}{\strut $\text{PD}_{\text{FFT}_{10\textit{-c}}}^{{\textit{CIII}}}$}, \colorbox[RGB]{209,235,255}{\strut $\text{PD}_{\text{FFT}_{20\textit{-c}}}^{{\textit{CIII}}}$}, \colorbox[RGB]{209,235,255}{\strut $\text{PD}_{\text{FFT}_{50\textit{-c}}}^{{\textit{CIII}}}$}, \colorbox[RGB]{209,235,255}{\strut $\text{PD}_{\text{FFT}_{100\textit{-c}}}^{{\textit{CIII}}}$}, \colorbox[RGB]{209,235,255}{\strut $\text{PD}_{\text{FFT}_{5\textit{-c}}}^{{\textit{H}\alpha}}$}, \colorbox[RGB]{209,235,255}{\strut $\text{PD}_{\text{FFT}_{10\textit{-c}}}^{{\textit{H}\alpha}}$}, \colorbox[RGB]{209,235,255}{\strut $\text{PD}_{\text{FFT}_{20\textit{-c}}}^{{\textit{H}\alpha}}$}, \colorbox[RGB]{209,235,255}{\strut $\text{PD}_{\text{FFT}_{50\textit{-c}}}^{{\textit{H}\alpha}}$}, \colorbox[RGB]{209,235,255}{\strut $\text{PD}_{\text{FFT}_{100\textit{-c}}}^{{\textit{H}\alpha}}$} \\\hline
GBDT-EM-3 & Emission, all FFT features. & \colorbox[RGB]{209,235,255}{\strut $\text{PD}_{\text{FFT}_{5\textit{-p}}}^{{\textit{CIII}}}$}, \colorbox[RGB]{209,235,255}{\strut $\text{PD}_{\text{FFT}_{5\textit{-c}}}^{{\textit{CIII}}}$}, \colorbox[RGB]{209,235,255}{\strut $\text{PD}_{\text{FFT}_{5\textit{-f}}}^{{\textit{CIII}}}$}, \colorbox[RGB]{209,235,255}{\strut $\text{PD}_{\text{FFT}_{10\textit{-p}}}^{{\textit{CIII}}}$}, \colorbox[RGB]{209,235,255}{\strut $\text{PD}_{\text{FFT}_{10\textit{-c}}}^{{\textit{CIII}}}$}, \colorbox[RGB]{209,235,255}{\strut $\text{PD}_{\text{FFT}_{10\textit{-f}}}^{{\textit{CIII}}}$}, \colorbox[RGB]{209,235,255}{\strut $\text{PD}_{\text{FFT}_{20\textit{-p}}}^{{\textit{CIII}}}$}, \colorbox[RGB]{209,235,255}{\strut $\text{PD}_{\text{FFT}_{20\textit{-c}}}^{{\textit{CIII}}}$}, \colorbox[RGB]{209,235,255}{\strut $\text{PD}_{\text{FFT}_{20\textit{-f}}}^{{\textit{CIII}}}$}, \colorbox[RGB]{209,235,255}{\strut $\text{PD}_{\text{FFT}_{50\textit{-p}}}^{{\textit{CIII}}}$}, \colorbox[RGB]{209,235,255}{\strut $\text{PD}_{\text{FFT}_{50\textit{-c}}}^{{\textit{CIII}}}$}, \colorbox[RGB]{209,235,255}{\strut $\text{PD}_{\text{FFT}_{50\textit{-f}}}^{{\textit{CIII}}}$}, \colorbox[RGB]{209,235,255}{\strut $\text{PD}_{\text{FFT}_{100\textit{-p}}}^{{\textit{CIII}}}$}, \colorbox[RGB]{209,235,255}{\strut $\text{PD}_{\text{FFT}_{100\textit{-c}}}^{{\textit{CIII}}}$}, \colorbox[RGB]{209,235,255}{\strut $\text{PD}_{\text{FFT}_{100\textit{-f}}}^{{\textit{CIII}}}$}, \colorbox[RGB]{209,235,255}{\strut $\text{PD}_{\text{FFT}_{5\textit{-p}}}^{{\textit{H}\alpha}}$}, \colorbox[RGB]{209,235,255}{\strut $\text{PD}_{\text{FFT}_{5\textit{-c}}}^{{\textit{H}\alpha}}$}, \colorbox[RGB]{209,235,255}{\strut $\text{PD}_{\text{FFT}_{5\textit{-f}}}^{{\textit{H}\alpha}}$}, \colorbox[RGB]{209,235,255}{\strut $\text{PD}_{\text{FFT}_{10\textit{-p}}}^{{\textit{H}\alpha}}$}, \colorbox[RGB]{209,235,255}{\strut $\text{PD}_{\text{FFT}_{10\textit{-c}}}^{{\textit{H}\alpha}}$}, \colorbox[RGB]{209,235,255}{\strut $\text{PD}_{\text{FFT}_{10\textit{-f}}}^{{\textit{H}\alpha}}$}, \colorbox[RGB]{209,235,255}{\strut $\text{PD}_{\text{FFT}_{20\textit{-p}}}^{{\textit{H}\alpha}}$}, \colorbox[RGB]{209,235,255}{\strut $\text{PD}_{\text{FFT}_{20\textit{-c}}}^{{\textit{H}\alpha}}$}, \colorbox[RGB]{209,235,255}{\strut $\text{PD}_{\text{FFT}_{20\textit{-f}}}^{{\textit{H}\alpha}}$}, \colorbox[RGB]{209,235,255}{\strut $\text{PD}_{\text{FFT}_{50\textit{-p}}}^{{\textit{H}\alpha}}$}, \colorbox[RGB]{209,235,255}{\strut $\text{PD}_{\text{FFT}_{50\textit{-c}}}^{{\textit{H}\alpha}}$}, \colorbox[RGB]{209,235,255}{\strut $\text{PD}_{\text{FFT}_{50\textit{-f}}}^{{\textit{H}\alpha}}$}, \colorbox[RGB]{209,235,255}{\strut $\text{PD}_{\text{FFT}_{100\textit{-p}}}^{{\textit{H}\alpha}}$}, \colorbox[RGB]{209,235,255}{\strut $\text{PD}_{\text{FFT}_{100\textit{-c}}}^{{\textit{H}\alpha}}$}, \colorbox[RGB]{209,235,255}{\strut $\text{PD}_{\text{FFT}_{100\textit{-f}}}^{{\textit{H}\alpha}}$} \\\hline
GBDT-MA-1 & Magnetics, all. & \colorbox[RGB]{234,255,227}{\strut $B_0$}, \colorbox[RGB]{234,255,227}{\strut $I_{p}$}, \colorbox[RGB]{234,255,227}{\strut $I_{p,\textit{ref}}$}, \colorbox[RGB]{234,255,227}{\strut $q_{95}$} \\\hline
GBDT-DE-1 & Density, top 4. & \colorbox[RGB]{252,233,255}{\strut $n_{e,\text{core}}$}, \colorbox[RGB]{252,233,255}{\strut $n_{e,\text{LFS}}$}, \colorbox[RGB]{252,233,255}{\strut $\text{max}(n'_{e,\text{edge}})$}, \colorbox[RGB]{252,233,255}{\strut $\text{max}(n''_{e,\text{edge}})$} \\\hline
GBDT-DE-2 & Density, all. & \colorbox[RGB]{252,233,255}{\strut $n_{e,\text{core}}$}, \colorbox[RGB]{252,233,255}{\strut $n_{e,\text{LFS}}$}, \colorbox[RGB]{252,233,255}{\strut $n_e/n_{\textit{GW}}$}, \colorbox[RGB]{252,233,255}{\strut $\text{max}(n'_{e,\text{edge}})$}, \colorbox[RGB]{252,233,255}{\strut $\text{max}(n''_{e,\text{edge}})$}, \colorbox[RGB]{252,233,255}{\strut $n_{e,0}$} \\\hline
GBDT-TE-1 & Temperature, all. & \colorbox[RGB]{255,247,196}{\strut $\textit{SXR}_{\text{core}}$}, \colorbox[RGB]{255,247,196}{\strut $\text{max}(T'_{e,\text{edge}})$}, \colorbox[RGB]{255,247,196}{\strut $\text{max}(T''_{e,\text{edge}})$}, \colorbox[RGB]{255,247,196}{\strut $T_{e,0}$} \\\hline
GBDT-PO-1 & Power, top 4. & \colorbox[RGB]{255,255,234}{\strut $P_{\textit{in}}$}, \colorbox[RGB]{255,255,234}{\strut $P_{\textit{NBI}}$}, \colorbox[RGB]{255,255,234}{\strut $P_{\textit{ECRH}}$}, \colorbox[RGB]{255,255,234}{\strut $P_{\textit{LH}}$} \\\hline
GBDT-PO-2 & Power, all. & \colorbox[RGB]{255,255,234}{\strut $P_{\textit{in}}$}, \colorbox[RGB]{255,255,234}{\strut $P_{\textit{OHM}}$}, \colorbox[RGB]{255,255,234}{\strut $P_{\textit{NBI}}$}, \colorbox[RGB]{255,255,234}{\strut $P_{\textit{NBI2}}$}, \colorbox[RGB]{255,255,234}{\strut $P_{\textit{ECRH}}$}, \colorbox[RGB]{255,255,234}{\strut $P_{\textit{LH}}$} \\\hline
GBDT-EN-1 & Energy content, top 4. & \colorbox[RGB]{255,246,219}{\strut $\beta_{N}$}, \colorbox[RGB]{255,246,219}{\strut $\beta_{p}$}, \colorbox[RGB]{255,246,219}{\strut $\beta_{t}$}, \colorbox[RGB]{255,246,219}{\strut ${W}_{\textit{tot}}$} \\\hline
GBDT-EN-2 & Energy content, all. & \colorbox[RGB]{255,246,219}{\strut $\beta_{N}$}, \colorbox[RGB]{255,246,219}{\strut $\beta_{p}$}, \colorbox[RGB]{255,246,219}{\strut $\beta_{t}$}, \colorbox[RGB]{255,246,219}{\strut ${W}_{\textit{tot}}$}, \colorbox[RGB]{255,246,219}{\strut $\text{DML}$}, \colorbox[RGB]{255,246,219}{\strut $H_{\textit{98y2}}$} \\\hline
GBDT-RA-1 & Radiation, top 2. & \colorbox[RGB]{255,248,255}{\strut $P_{\textit{rad}}$}, \colorbox[RGB]{255,248,255}{\strut $P_{\textit{rad},\text{SOL}}$} \\\hline
GBDT-RA-2 & Radiation, all. & \colorbox[RGB]{255,248,255}{\strut $P_{\textit{rad}}$}, \colorbox[RGB]{255,248,255}{\strut $P_{\textit{rad},\text{bulk}}$}, \colorbox[RGB]{255,248,255}{\strut $P_{\textit{rad},\text{SOL}}$} \\\hline
GBDT-OT-1 & Other, all. & \colorbox[RGB]{235,245,255}{\strut $l_i$}, \colorbox[RGB]{235,245,255}{\strut $Z_{\textit{eff}}$}, \colorbox[RGB]{235,245,255}{\strut $\nu_{e,\text{ped}}^{*}$}, \colorbox[RGB]{235,245,255}{\strut $V_{\textit{loop}}$} \\\hline
GBDT-$\ast\ast$-1 & Mixed, rank 1. & \colorbox[RGB]{255,210,204}{\strut $\kappa$}, \colorbox[RGB]{209,235,255}{\strut $\text{PD}_{\text{FFT}_{20\textit{-c}}}^{{\textit{H}\alpha}}$}, \colorbox[RGB]{234,255,227}{\strut $q_{95}$}, \colorbox[RGB]{252,233,255}{\strut $\text{max}(n''_{e,\text{edge}})$}, \colorbox[RGB]{255,247,196}{\strut $\text{max}(T'_{e,\text{edge}})$}, \colorbox[RGB]{255,255,234}{\strut $P_{\textit{in}}$}, \colorbox[RGB]{255,246,219}{\strut $\beta_{t}$}, \colorbox[RGB]{255,248,255}{\strut $P_{\textit{rad},\text{SOL}}$}, \colorbox[RGB]{235,245,255}{\strut $\nu_{e,\text{ped}}^{*}$} \\\hline
GBDT-$\ast\ast$-2 & Mixed, rank 2. & \colorbox[RGB]{255,210,204}{\strut $\delta_{\text{bottom}}$}, \colorbox[RGB]{209,235,255}{\strut $\text{PD}_{\text{FFT}_{50\textit{-c}}}^{{\textit{H}\alpha}}$}, \colorbox[RGB]{234,255,227}{\strut $I_{p,\textit{ref}}$}, \colorbox[RGB]{252,233,255}{\strut $n_{e,\text{LFS}}$}, \colorbox[RGB]{255,247,196}{\strut $\text{max}(T''_{e,\text{edge}})$}, \colorbox[RGB]{255,255,234}{\strut $P_{\textit{NBI}}$}, \colorbox[RGB]{255,246,219}{\strut ${W}_{\textit{tot}}$}, \colorbox[RGB]{255,248,255}{\strut $P_{\textit{rad}}$}, \colorbox[RGB]{235,245,255}{\strut $l_i$} \\\hline
GBDT-$\ast\ast$-3 & Mixed, rank 3. & \colorbox[RGB]{255,210,204}{\strut $Z_{\text{axis}}$}, \colorbox[RGB]{209,235,255}{\strut $\text{PD}_{\text{FFT}_{10\textit{-c}}}^{{\textit{H}\alpha}}$}, \colorbox[RGB]{234,255,227}{\strut $I_{p}$}, \colorbox[RGB]{252,233,255}{\strut $\text{max}(n'_{e,\text{edge}})$}, \colorbox[RGB]{255,247,196}{\strut $\textit{SXR}_{\text{core}}$}, \colorbox[RGB]{255,255,234}{\strut $P_{\textit{LH}}$}, \colorbox[RGB]{255,246,219}{\strut $\beta_{N}$}, \colorbox[RGB]{235,245,255}{\strut $V_{\textit{loop}}$} \\\hline
GBDT-$\ast\ast$-4 & Mixed, top 2. & \colorbox[RGB]{255,210,204}{\strut $\delta_{\text{bottom}}$}, \colorbox[RGB]{255,210,204}{\strut $\kappa$}, \colorbox[RGB]{209,235,255}{\strut $\text{PD}_{\text{FFT}_{20\textit{-c}}}^{{\textit{H}\alpha}}$}, \colorbox[RGB]{209,235,255}{\strut $\text{PD}_{\text{FFT}_{50\textit{-c}}}^{{\textit{H}\alpha}}$}, \colorbox[RGB]{234,255,227}{\strut $I_{p,\textit{ref}}$}, \colorbox[RGB]{234,255,227}{\strut $q_{95}$}, \colorbox[RGB]{252,233,255}{\strut $n_{e,\text{LFS}}$}, \colorbox[RGB]{252,233,255}{\strut $\text{max}(n''_{e,\text{edge}})$}, \colorbox[RGB]{255,247,196}{\strut $\text{max}(T'_{e,\text{edge}})$}, \colorbox[RGB]{255,247,196}{\strut $\text{max}(T''_{e,\text{edge}})$}, \colorbox[RGB]{255,255,234}{\strut $P_{\textit{in}}$}, \colorbox[RGB]{255,255,234}{\strut $P_{\textit{NBI}}$}, \colorbox[RGB]{255,246,219}{\strut $\beta_{t}$}, \colorbox[RGB]{255,246,219}{\strut ${W}_{\textit{tot}}$}, \colorbox[RGB]{255,248,255}{\strut $P_{\textit{rad}}$}, \colorbox[RGB]{255,248,255}{\strut $P_{\textit{rad},\text{SOL}}$}, \colorbox[RGB]{235,245,255}{\strut $l_i$}, \colorbox[RGB]{235,245,255}{\strut $\nu_{e,\text{ped}}^{*}$} \\\hline
GBDT-$\ast\ast$-5 & Mixed, top 3. & \colorbox[RGB]{255,210,204}{\strut $\delta_{\text{bottom}}$}, \colorbox[RGB]{255,210,204}{\strut $\kappa$}, \colorbox[RGB]{255,210,204}{\strut $Z_{\text{axis}}$}, \colorbox[RGB]{209,235,255}{\strut $\text{PD}_{\text{FFT}_{10\textit{-c}}}^{{\textit{H}\alpha}}$}, \colorbox[RGB]{209,235,255}{\strut $\text{PD}_{\text{FFT}_{20\textit{-c}}}^{{\textit{H}\alpha}}$}, \colorbox[RGB]{209,235,255}{\strut $\text{PD}_{\text{FFT}_{50\textit{-c}}}^{{\textit{H}\alpha}}$}, \colorbox[RGB]{234,255,227}{\strut $I_{p}$}, \colorbox[RGB]{234,255,227}{\strut $I_{p,\textit{ref}}$}, \colorbox[RGB]{234,255,227}{\strut $q_{95}$}, \colorbox[RGB]{252,233,255}{\strut $n_{e,\text{LFS}}$}, \colorbox[RGB]{252,233,255}{\strut $\text{max}(n'_{e,\text{edge}})$}, \colorbox[RGB]{252,233,255}{\strut $\text{max}(n''_{e,\text{edge}})$}, \colorbox[RGB]{255,247,196}{\strut $\textit{SXR}_{\text{core}}$}, \colorbox[RGB]{255,247,196}{\strut $\text{max}(T'_{e,\text{edge}})$}, \colorbox[RGB]{255,247,196}{\strut $\text{max}(T''_{e,\text{edge}})$}, \colorbox[RGB]{255,255,234}{\strut $P_{\textit{in}}$}, \colorbox[RGB]{255,255,234}{\strut $P_{\textit{NBI}}$}, \colorbox[RGB]{255,255,234}{\strut $P_{\textit{LH}}$}, \colorbox[RGB]{255,246,219}{\strut $\beta_{N}$}, \colorbox[RGB]{255,246,219}{\strut $\beta_{t}$}, \colorbox[RGB]{255,246,219}{\strut ${W}_{\textit{tot}}$}, \colorbox[RGB]{255,248,255}{\strut $P_{\textit{rad}}$}, \colorbox[RGB]{255,248,255}{\strut $P_{\textit{rad},\text{SOL}}$}, \colorbox[RGB]{235,245,255}{\strut $l_i$}, \colorbox[RGB]{235,245,255}{\strut $\nu_{e,\text{ped}}^{*}$}, \colorbox[RGB]{235,245,255}{\strut $V_{\textit{loop}}$} \\\hline
GBDT-$\ast\ast$-6 & Mixed, all. & \colorbox[RGB]{255,210,204}{\strut $A_p$}, \colorbox[RGB]{255,210,204}{\strut $\delta_{\text{bottom}}$}, \colorbox[RGB]{255,210,204}{\strut $\delta_{\text{top}}$}, \colorbox[RGB]{255,210,204}{\strut $\Delta_{\text{in}}$}, \colorbox[RGB]{255,210,204}{\strut $\Delta_{\text{out}}$}, \colorbox[RGB]{255,210,204}{\strut $\kappa$}, \colorbox[RGB]{255,210,204}{\strut $R_0$}, \colorbox[RGB]{255,210,204}{\strut $a$}, \colorbox[RGB]{255,210,204}{\strut $R_{\text{axis}}$}, \colorbox[RGB]{255,210,204}{\strut $Z_{\text{axis}}$}, \colorbox[RGB]{255,210,204}{\strut $V_p$}, \colorbox[RGB]{209,235,255}{\strut $\text{PD}_{\text{FFT}_{5\textit{-c}}}^{{\textit{CIII}}}$}, \colorbox[RGB]{209,235,255}{\strut $\text{PD}_{\text{FFT}_{10\textit{-c}}}^{{\textit{CIII}}}$}, \colorbox[RGB]{209,235,255}{\strut $\text{PD}_{\text{FFT}_{20\textit{-c}}}^{{\textit{CIII}}}$}, \colorbox[RGB]{209,235,255}{\strut $\text{PD}_{\text{FFT}_{50\textit{-c}}}^{{\textit{CIII}}}$}, \colorbox[RGB]{209,235,255}{\strut $\text{PD}_{\text{FFT}_{100\textit{-c}}}^{{\textit{CIII}}}$}, \colorbox[RGB]{209,235,255}{\strut $\text{PD}_{\text{FFT}_{5\textit{-c}}}^{{\textit{H}\alpha}}$}, \colorbox[RGB]{209,235,255}{\strut $\text{PD}_{\text{FFT}_{10\textit{-c}}}^{{\textit{H}\alpha}}$}, \colorbox[RGB]{209,235,255}{\strut $\text{PD}_{\text{FFT}_{20\textit{-c}}}^{{\textit{H}\alpha}}$}, \colorbox[RGB]{209,235,255}{\strut $\text{PD}_{\text{FFT}_{50\textit{-c}}}^{{\textit{H}\alpha}}$}, \colorbox[RGB]{209,235,255}{\strut $\text{PD}_{\text{FFT}_{100\textit{-c}}}^{{\textit{H}\alpha}}$}, \colorbox[RGB]{234,255,227}{\strut $B_0$}, \colorbox[RGB]{234,255,227}{\strut $I_{p}$}, \colorbox[RGB]{234,255,227}{\strut $I_{p,\textit{ref}}$}, \colorbox[RGB]{234,255,227}{\strut $q_{95}$}, \colorbox[RGB]{252,233,255}{\strut $n_{e,\text{core}}$}, \colorbox[RGB]{252,233,255}{\strut $n_{e,\text{LFS}}$}, \colorbox[RGB]{252,233,255}{\strut $n_e/n_{\textit{GW}}$}, \colorbox[RGB]{252,233,255}{\strut $\text{max}(n'_{e,\text{edge}})$}, \colorbox[RGB]{252,233,255}{\strut $\text{max}(n''_{e,\text{edge}})$}, \colorbox[RGB]{252,233,255}{\strut $n_{e,0}$}, \colorbox[RGB]{255,247,196}{\strut $\textit{SXR}_{\text{core}}$}, \colorbox[RGB]{255,247,196}{\strut $\text{max}(T'_{e,\text{edge}})$}, \colorbox[RGB]{255,247,196}{\strut $\text{max}(T''_{e,\text{edge}})$}, \colorbox[RGB]{255,247,196}{\strut $T_{e,0}$}, \colorbox[RGB]{255,255,234}{\strut $P_{\textit{in}}$}, \colorbox[RGB]{255,255,234}{\strut $P_{\textit{OHM}}$}, \colorbox[RGB]{255,255,234}{\strut $P_{\textit{NBI}}$}, \colorbox[RGB]{255,255,234}{\strut $P_{\textit{NBI2}}$}, \colorbox[RGB]{255,255,234}{\strut $P_{\textit{ECRH}}$}, \colorbox[RGB]{255,255,234}{\strut $P_{\textit{LH}}$}, \colorbox[RGB]{255,246,219}{\strut $\beta_{N}$}, \colorbox[RGB]{255,246,219}{\strut $\beta_{p}$}, \colorbox[RGB]{255,246,219}{\strut $\beta_{t}$}, \colorbox[RGB]{255,246,219}{\strut ${W}_{\textit{tot}}$}, \colorbox[RGB]{255,246,219}{\strut $\text{DML}$}, \colorbox[RGB]{255,246,219}{\strut $H_{\textit{98y2}}$}, \colorbox[RGB]{255,248,255}{\strut $P_{\textit{rad}}$}, \colorbox[RGB]{255,248,255}{\strut $P_{\textit{rad},\text{bulk}}$}, \colorbox[RGB]{255,248,255}{\strut $P_{\textit{rad},\text{SOL}}$}, \colorbox[RGB]{235,245,255}{\strut $l_i$}, \colorbox[RGB]{235,245,255}{\strut $Z_{\textit{eff}}$}, \colorbox[RGB]{235,245,255}{\strut $\nu_{e,\text{ped}}^{*}$}, \colorbox[RGB]{235,245,255}{\strut $V_{\textit{loop}}$} \\\hline
GBDT-$\ast\ast$-7 & Mixed, rank 1 (by L-mode threshold). & \colorbox[RGB]{255,210,204}{\strut $\Delta_{\text{in}}$}, \colorbox[RGB]{209,235,255}{\strut $\text{PD}_{\text{FFT}_{20\textit{-c}}}^{{\textit{H}\alpha}}$}, \colorbox[RGB]{234,255,227}{\strut $I_{p,\textit{ref}}$}, \colorbox[RGB]{252,233,255}{\strut $n_{e,0}$}, \colorbox[RGB]{255,247,196}{\strut $\textit{SXR}_{\text{core}}$}, \colorbox[RGB]{255,255,234}{\strut $P_{\textit{in}}$}, \colorbox[RGB]{255,246,219}{\strut $\beta_{t}$}, \colorbox[RGB]{255,248,255}{\strut $P_{\textit{rad},\text{bulk}}$}, \colorbox[RGB]{235,245,255}{\strut $V_{\textit{loop}}$} \\\hline
GBDT-$\ast\ast$-8 & Mixed, rank 2 (by L-mode threshold). & \colorbox[RGB]{255,210,204}{\strut $\kappa$}, \colorbox[RGB]{209,235,255}{\strut $\text{PD}_{\text{FFT}_{50\textit{-c}}}^{{\textit{H}\alpha}}$}, \colorbox[RGB]{234,255,227}{\strut $I_{p}$}, \colorbox[RGB]{252,233,255}{\strut $\text{max}(n'_{e,\text{edge}})$}, \colorbox[RGB]{255,247,196}{\strut $\text{max}(T''_{e,\text{edge}})$}, \colorbox[RGB]{255,255,234}{\strut $P_{\textit{NBI}}$}, \colorbox[RGB]{255,246,219}{\strut ${W}_{\textit{tot}}$}, \colorbox[RGB]{255,248,255}{\strut $P_{\textit{rad}}$}, \colorbox[RGB]{235,245,255}{\strut $Z_{\textit{eff}}$} \\\hline
GBDT-$\ast\ast$-9 & Mixed, top 2 (by L-mode threshold). & \colorbox[RGB]{255,210,204}{\strut $\Delta_{\text{in}}$}, \colorbox[RGB]{255,210,204}{\strut $\kappa$}, \colorbox[RGB]{209,235,255}{\strut $\text{PD}_{\text{FFT}_{20\textit{-c}}}^{{\textit{H}\alpha}}$}, \colorbox[RGB]{209,235,255}{\strut $\text{PD}_{\text{FFT}_{50\textit{-c}}}^{{\textit{H}\alpha}}$}, \colorbox[RGB]{234,255,227}{\strut $I_{p}$}, \colorbox[RGB]{234,255,227}{\strut $I_{p,\textit{ref}}$}, \colorbox[RGB]{252,233,255}{\strut $\text{max}(n'_{e,\text{edge}})$}, \colorbox[RGB]{252,233,255}{\strut $n_{e,0}$}, \colorbox[RGB]{255,247,196}{\strut $\textit{SXR}_{\text{core}}$}, \colorbox[RGB]{255,247,196}{\strut $\text{max}(T''_{e,\text{edge}})$}, \colorbox[RGB]{255,255,234}{\strut $P_{\textit{in}}$}, \colorbox[RGB]{255,255,234}{\strut $P_{\textit{NBI}}$}, \colorbox[RGB]{255,246,219}{\strut $\beta_{t}$}, \colorbox[RGB]{255,246,219}{\strut ${W}_{\textit{tot}}$}, \colorbox[RGB]{255,248,255}{\strut $P_{\textit{rad}}$}, \colorbox[RGB]{255,248,255}{\strut $P_{\textit{rad},\text{bulk}}$}, \colorbox[RGB]{235,245,255}{\strut $Z_{\textit{eff}}$}, \colorbox[RGB]{235,245,255}{\strut $V_{\textit{loop}}$} \\\hline
GBDT-$\ast\ast$-10 & Mixed, rank 2 (by H-mode threshold). & \colorbox[RGB]{255,210,204}{\strut $\delta_{\text{bottom}}$}, \colorbox[RGB]{209,235,255}{\strut $\text{PD}_{\text{FFT}_{50\textit{-c}}}^{{\textit{H}\alpha}}$}, \colorbox[RGB]{234,255,227}{\strut $I_{p}$}, \colorbox[RGB]{252,233,255}{\strut $n_{e,0}$}, \colorbox[RGB]{255,247,196}{\strut $\text{max}(T''_{e,\text{edge}})$}, \colorbox[RGB]{255,255,234}{\strut $P_{\textit{NBI2}}$}, \colorbox[RGB]{255,246,219}{\strut ${W}_{\textit{tot}}$}, \colorbox[RGB]{235,245,255}{\strut $V_{\textit{loop}}$}

\\
\caption{All \textit{(model + feature set)} configurations, with features colored by category. The order is based on Cohen's kappa coefficient when using the individual feature to predict the confinement state using a shallow decision tree, unless specified otherwise. The feature sets between the two models are identical except for the emission-related features, where only the FNOLSTM-based models use the raw photodiode signals. These signals are not very informative when considering the absolute value because they are not absolutely calibrated, however, confinement state-related patterns are clearly visible in their dynamics. For this reason, the `raw' values are selected only for the dynamic models, since these models can detect the temporal patterns.}\label{tab:model_featuresets}
\end{longtable}
\endgroup
\setlength\tabcolsep{6pt}
\arrayrulecolor{black}



\newpage

\clearpage\section{Model training}\label{ap:model_training}
\subsection{Application to Probabilistic Forecasting}\label{appendix:privacy_utility_tradeoff}

\subsubsection{Event-Level Privacy}\label{appendix:privacy_utility_tradeoff_event_level_privacy}

\cref{table:1_event_training_traffic,table:1_event_training_electricity,table:1_event_training_solar}
show average CRPS after $1$-event-level private training on our three standard benchmark datasets.
Since $\delta^{-1}$ is approximately equal or greater than the dataset sizes, $\epsilon \leq 1$ indicates strong privacy guarantees,
whereas $2 \leq \epsilon \leq 8$ would be more commonly expected values in private training of machine learning models~\cite{ponomareva2023dp}.

For a full description of all hyper parameters, see~\cref{appendix:experimental_setup}.

In all cases, at least one model outperforms the traditional baselines without noise for all considered $\epsilon$.

\begin{table}[h]
\caption{Average CRPS on \texttt{traffic} for $1$-event-level privacy and $\delta=10^{-7}$. Seasonal, AutoETS, and models with $\epsilon=\infty$ are without noise.
Bold font indicates the best predictor per $\epsilon$.}
\label{table:1_event_training_traffic}
\vskip 0.15in
\begin{center}
\begin{small}
\begin{sc}
\begin{tabular}{lcccccc}
\toprule
Model & $\epsilon = 0.5$ & $\epsilon = 1$ & $\epsilon = 2$ & $\epsilon = 4$ & $\epsilon = 8$ &  $\epsilon = \infty$ \\
\midrule
SimpleFF & $0.207$ \tiny{$\pm 0.002$} & $0.195$ \tiny{$\pm 0.003$} & $0.193$ \tiny{$\pm 0.003$} & $0.194$ \tiny{$\pm 0.002$} & $0.193$ \tiny{$\pm 0.003$} & $0.136$ \tiny{$\pm 0.001$} \\ 
DeepAR & $\mathbf{0.157}$ \tiny{$\pm 0.002$} & $\mathbf{0.145}$ \tiny{$\pm 0.001$} & $\mathbf{0.142}$ \tiny{$\pm 0.001$} & $\mathbf{0.141}$ \tiny{$\pm 0.002$} & $\mathbf{0.141}$ \tiny{$\pm 0.002$} & $\mathbf{0.124}$ \tiny{$\pm 0.0.001$} \\
iTransf. & $0.211$ \tiny{$\pm 0.004$} & $0.193$ \tiny{$\pm 0.003$} & $0.188$ \tiny{$\pm 0.004$} & $0.188$ \tiny{$\pm 0.004$} & $0.188$ \tiny{$\pm 0.004$} & $0.135$ \tiny{$\pm 0.001$} \\
DLinear & $0.204$ \tiny{$\pm 0.004$} & $0.192$ \tiny{$\pm 0.001$} & $0.188$ \tiny{$\pm 0.003$} & $0.188$ \tiny{$\pm 0.003$} & $0.188$ \tiny{$\pm 0.003$} & $0.140$ \tiny{$\pm 0.000$} \\
\midrule
Seasonal   & $0.251$ & $0.251$ & $0.251$ & $0.251$ & $0.251$ & $0.251$\\
AutoETS   & $0.407$ & $0.407$ & $0.407$ & $0.407$ & $0.407$ & $0.407$\\
\bottomrule
\end{tabular}
\end{sc}
\end{small}
\end{center}
\vskip -0.1in
\end{table}

\begin{table}[h]
\caption{Average CRPS on \texttt{traffic} for $1$-event-level privacy and $\delta=10^{-7}$. Seasonal, AutoETS, and models with $\epsilon=\infty$ are without noise.
Bold font indicates the best predictor per $\epsilon$.}
\label{table:1_event_training_electricity}
\vskip 0.15in
\begin{center}
\begin{small}
\begin{sc}
\begin{tabular}{lcccccc}
\toprule
Model & $\epsilon = 0.5$ & $\epsilon = 1$ & $\epsilon = 2$ & $\epsilon = 4$ &  $\epsilon = 8$ &  $\epsilon = \infty$ \\
\midrule
SimpleFF & $0.072$ \tiny{$\pm 0.001$} & $0.065$ \tiny{$\pm 0.001$} & $0.065$ \tiny{$\pm 0.001$} & $0.065$ \tiny{$\pm 0.001$} & $0.065$ \tiny{$\pm 0.002$} & $\mathbf{0.058}$ \tiny{$\pm 0.001$} \\
DeepAR & $0.071$ \tiny{$\pm 0.004$} & $0.070$ \tiny{$\pm 0.004$} & $0.068$ \tiny{$\pm 0.005$} & $0.067$ \tiny{$\pm 0.004$} & $0.068$ \tiny{$\pm 0.005$} & $\mathbf{0.058}$ \tiny{$\pm 0.002$} \\
iTransf. & $0.081$ \tiny{$\pm 0.005$} & $0.075$ \tiny{$\pm 0.002$} & $0.074$ \tiny{$\pm 0.002$} & $0.074$ \tiny{$\pm 0.002$} & $0.074$ \tiny{$\pm 0.002$} & $\mathbf{0.058}$ \tiny{$\pm 0.001$} \\
DLinear & $\mathbf{0.064}$ \tiny{$\pm 0.000$} & $\mathbf{0.061}$ \tiny{$\pm 0.001$} & $\mathbf{0.061}$ \tiny{$\pm 0.001$} & $\mathbf{0.061}$ \tiny{$\pm 0.001$} & $\mathbf{0.061}$ \tiny{$\pm 0.001$} & $0.059$ \tiny{$\pm 0.000$} \\
\midrule
Seasonal   & $0.070$ & $0.070$ & $0.070$ & $0.070$ & $0.070$ & $0.070$\\
AutoETS   & $0.064$ & $0.064$ & $0.064$ & $0.064$ & $0.064$ & $0.064$\\
\bottomrule
\end{tabular}
\end{sc}
\end{small}
\end{center}
\vskip -0.1in
\end{table}

\begin{table}[h]
\caption{Average CRPS on \texttt{traffic} for $1$-event-level privacy and $\delta=10^{-7}$. Seasonal, AutoETS, and models with $\epsilon=\infty$ are without noise.
Bold font indicates the best predictor per $\epsilon$.}
\label{table:1_event_training_solar}
\vskip 0.15in
\begin{center}
\begin{small}
\begin{sc}
\begin{tabular}{lcccccc}
\toprule
Model & $\epsilon = 0.5$ & $\epsilon = 1$ & $\epsilon = 2$ & $\epsilon = 4$ & $\epsilon = 8$ &  $\epsilon = \infty$ \\
\midrule
SimpleFF & $1.114$ \tiny{$\pm 0.043$} & $1.114$ \tiny{$\pm 0.044$} & $1.114$ \tiny{$\pm 0.040$} & $1.118$ \tiny{$\pm 0.038$} & $1.113$ \tiny{$\pm 0.028$} & $0.766$ \tiny{$\pm 0.766$} \\ 
DeepAR & $\mathbf{0.820}$ \tiny{$\pm 0.030$} & $\mathbf{0.803}$ \tiny{$\pm 0.023$} & $\mathbf{0.792}$ \tiny{$\pm 0.016$} & $\mathbf{0.787}$ \tiny{$\pm 0.023$} & $\mathbf{0.787}$ \tiny{$\pm 0.023$} & $\textbf{0.654}$ \tiny{$\pm 0.654$} \\
iTransf. & $0.977$ \tiny{$\pm 0.065$} & $0.956$ \tiny{$\pm 0.062$} & $0.975$ \tiny{$\pm 0.066$} & $0.974$ \tiny{$\pm 0.065$} & $0.974$ \tiny{$\pm 0.065$} & $0.804$ \tiny{$\pm 0.804$} \\
DLinear & $1.287$ \tiny{$\pm 0.222$} & $1.152$ \tiny{$\pm 0.201$} & $1.110$ \tiny{$\pm 0.145$} & $1.048$ \tiny{$\pm 0.143$} & $1.051$ \tiny{$\pm 0.140$} & $0.860$ \tiny{$\pm 0.860$} \\
\midrule
Seasonal   & $1.120$ & $1.120$ & $1.120$ & $1.120$ & $1.120$ & $1.120$ \\
AutoETS   & $6.494$ & $6.494$ & $6.494$ & $6.494$ & $6.494$ & $6.494$ \\
\bottomrule
\end{tabular}
\end{sc}
\end{small}
\end{center}
\vskip -0.1in
\end{table}

\clearpage

\subsubsection{w-event and w-User-Level Privacy}\label{appendix:privacy_utility_tradeoff_user_level_privacy}

As discussed in~\cref{appendix:generalizations}, 
we can (for sufficiently long sequences) generalize our bounds on the mechanism's privacy profile
from $1$-event- to $w$-event- or $w$-user-level   privacy by replacing
any occurrence of $L_C + L_F$ with $L_C + L_F + w -1$ or $w \cdot (L_C + L_F)$.
Since $w'$-event- and $w$-user-level privacy lead to identical results for some sufficiently large $w'$,
it is sufficient to experiment with $w$-user-level privacy.

For a full description of all hyper parameters, see~\cref{appendix:experimental_setup}.

\cref{table:w_user_training_traffic,table:w_user_training_electricity,table:w_user_training_solar}
show average CRPS after $w$-user-level private training on our three standard benchmark datasets.

Except for \texttt{traffic} and $w=8$ (which, by the above argument and our choice of $L_C$ and $L_F$, is equivalent to requiring privacy for an event spanning multiple days in our hourly datasets), at least one model outperforms the traditional baselines without noise for all considered $w$.

\begin{table}[h]
\caption{Average CRPS on \texttt{traffic} for $w$-user-level privacy and $\epsilon=4$, $\delta=10^{-7}$. SeasonSeasonal and AutoETS are without noise.
Bold font indicates the best predictor per $w$.}
\label{table:w_user_training_traffic}
\vskip 0.15in
\begin{center}
\begin{small}
\begin{sc}
\begin{tabular}{l c c c c}
\toprule
Model & $w = 1$ & $w = 2$ & $w = 4$ & $w = 8$ \\
\midrule
SimpleFF & $0.193$ \tiny{$\pm 0.003$} & $0.194$ \tiny{$\pm 0.003$} & $0.194$ \tiny{$\pm 0.003$} & $0.211$ \tiny{$\pm 0.001$} \\ 
DeepAR & $\mathbf{0.142}$ \tiny{$\pm 0.003$} & $\mathbf{0.143}$ \tiny{$\pm 0.001$} & $\mathbf{0.145}$ \tiny{$\pm 0.001$} & $\mathbf{0.166}$ \tiny{$\pm 0.004$} \\
iTransf. & $0.188$ \tiny{$\pm 0.004$} & $0.188$ \tiny{$\pm 0.004$} & $0.193$ \tiny{$\pm 0.002$} & $0.217$ \tiny{$\pm 0.005$} \\
DLinear & $0.189$ \tiny{$\pm 0.002$} & $0.189$ \tiny{$\pm 0.003$} & $0.192$ \tiny{$\pm 0.001$} & $0.208$ \tiny{$\pm 0.004$} \\
\midrule
Seasonal   & $0.251$ & $0.251$ & $0.251$ & $0.251$\\
AutoETS   & $0.407$ & $0.407$ & $0.407$ & $0.407$\\
\bottomrule
\end{tabular}
\end{sc}
\end{small}
\end{center}
\vskip -0.1in
\end{table}

\begin{table}[h]
\caption{Average CRPS on \texttt{electricity} for $w$-user-level privacy and $\epsilon=4$, $\delta=10^{-7}$. SeaSeasonal and AutoETS are without noise.
Bold font indicates the best predictor per $w$.}
\label{table:w_user_training_electricity}
\vskip 0.15in
\begin{center}
\begin{small}
\begin{sc}
\begin{tabular}{l c c c c}
\toprule
Model & $w = 1$ & $w = 2$ & $w = 4$ & $w = 8$ \\
\midrule
SimpleFF & $0.064$ \tiny{$\pm 0.001$} & $0.065$ \tiny{$\pm 0.001$} & $0.064$ \tiny{$\pm 0.001$} & $0.074$ \tiny{$\pm 0.001$} \\ 
DeepAR & $0.068$ \tiny{$\pm 0.005$} & $0.069$ \tiny{$\pm 0.005$} & $0.068$ \tiny{$\pm 0.002$} & $0.073$ \tiny{$\pm 0.003$} \\
iTransf. & $0.075$ \tiny{$\pm 0.003$} & $0.074$ \tiny{$\pm 0.002$} & $0.075$ \tiny{$\pm 0.003$} & $0.083$ \tiny{$\pm 0.004$} \\
DLinear & $\mathbf{0.061}$ \tiny{$\pm 0.001$} & $\mathbf{0.061}$ \tiny{$\pm 0.001$} & $\mathbf{0.061}$ \tiny{$\pm 0.001$} & $0.066$ \tiny{$\pm 0.001$} \\
\midrule
Seasonal   & $0.070$ & $0.070$ & $0.070$ & $0.070$ \\
AutoETS   & $0.064$ & $0.064$ & $0.064$ & $\mathbf{0.064}$ \\
\bottomrule
\end{tabular}
\end{sc}
\end{small}
\end{center}
\vskip -0.1in
\end{table}

\begin{table}[h]
\caption{Average CRPS on \texttt{solar\_10\_minutes} for $w$-user-level privacy and $\epsilon=4$, $\delta=10^{-7}$. SeasoSeasonal and AutoETS are without noise.
Bold font indicates the best predictor per $w$.}
\label{table:w_user_training_solar}
\vskip 0.15in
\begin{center}
\begin{small}
\begin{sc}
\begin{tabular}{l c c c c}
\toprule
Model & $w = 1$ & $w = 2$ & $w = 4$ & $w = 8$ \\
\midrule
SimpleFF & $1.113$ \tiny{$\pm 0.033$} & $1.116$ \tiny{$\pm 0.038$} & $1.110$ \tiny{$\pm 0.042$} & $1.107$ \tiny{$\pm 0.034$} \\ 
DeepAR & $\mathbf{0.783}$ \tiny{$\pm 0.019$} & $\mathbf{0.791}$ \tiny{$\pm 0.018$} & $\mathbf{0.807}$ \tiny{$\pm 0.024$} & $\mathbf{0.823}$ \tiny{$\pm 0.028$} \\
iTransf. & $0.951$ \tiny{$\pm 0.061$} & $0.950$ \tiny{$\pm 0.062$} & $0.956$ \tiny{$\pm 0.062$} & $0.980$ \tiny{$\pm 0.060$} \\
DLinear & $1.072$ \tiny{$\pm 0.164$} & $1.122$ \tiny{$\pm 0.158$} & $1.200$ \tiny{$\pm 0.220$} & $1.298$ \tiny{$\pm 0.205$} \\
\midrule
Seasonal   & $1.120$ & $1.120$ & $1.120$ & $1.120$ \\
AutoETS   & $6.494$ & $6.494$ & $6.494$ & $6.494$ \\
\bottomrule
\end{tabular}
\end{sc}
\end{small}
\end{center}
\vskip -0.1in
\end{table}

\clearpage

\subsubsection{Amplification by Label Perturbation}\label{appendix:privacy_utility_tradeoff_label_privacy}


Finally, let us demonstrate the potential benefit of using online data augmentations to amplify privacy in training forecasting models (\cref{theorem:amplification_by_data_augmentation_wor_wr}).
Consider $(w,v)$-event-level or $(w,v)$-user-level privacy with sufficiently small $v$. 
If $v$ is sufficiently small compared to the scale of the dataset, i.e., each individual only makes a small contribution to the overall value of a time series at each time step, then we can introduce non-trivial context and label noise $\sigma_C$ and $\sigma_F$ to amplify privacy without significantly affecting utility. 
Simultaneously, we still benefit from amplification through top- and bottom-level subsampling, i.e., do not need to add as much noise as would be required for directly making the entire input dataset privacy.

For a full description of all hyper parameters, see~\cref{appendix:experimental_setup}.

\cref{table:label_perturbation_training_traffic,table:label_perturbation_training_electricity,table:label_perturbation_training_solar}
show average CRPS after $(1,v)$-event-level private training with $\epsilon=0.5$, $\delta=10^{-7}$, i.e., strong privacy guarantees. Note that we use different $v$ per dataset, as they have different scale.

We observe that, for all models and all datasets, the best score is attained with label noise scale $\sigma_F = 2$ or $\sigma_F = 2$.
This confirms that there a scenarios in which our novel amplification-by-augmentation guarantees can help improve the privacy--utility trade-off of forecasting models.

\begin{table}[h]
\caption{Average CRPS on \texttt{traffic} for $(1,0.001)$-user-level privacy and $\epsilon=0.5$, $\delta=10^{-7}$. SeasoSeasonal and AutoETS are without noise.
Bold font indicates the best label noise scale $\sigma_F$ per model, i.e., per row.}
\label{table:label_perturbation_training_traffic}
\vskip 0.15in
\begin{center}
\begin{small}
\begin{sc}
\begin{tabular}{l c c c c}
\toprule
Model & $\sigma_F = 0$ & $\sigma_F= 1$ & $\sigma_F= 2$ & $\sigma_F = 5$\\
\midrule
SimpleFF & $0.207$ \tiny{$\pm 0.002$} & $0.205$ \tiny{$\pm 0.002$} & $0.205$ \tiny{$\pm 0.002$} & $\mathbf{0.205}$ \tiny{$\pm 0.001$}  \\ 
DeepAR & $0.156$ \tiny{$\pm 0.003$} & $0.156$ \tiny{$\pm 0.003$} & $0.156$ \tiny{$\pm 0.003$} & $\mathbf{0.154}$ \tiny{$\pm 0.002$}  \\
iTransf. & $0.211$ \tiny{$\pm 0.004$} & $0.208$ \tiny{$\pm 0.004$} & $0.205$ \tiny{$\pm 0.003$} & $\mathbf{0.204}$ \tiny{$\pm 0.003$}  \\
DLinear & $0.203$ \tiny{$\pm 0.003$} & $0.202$ \tiny{$\pm 0.003$} & $\mathbf{0.202}$ \tiny{$\pm 0.003$} & $0.203$ \tiny{$\pm 0.003$}  \\
\midrule
Seasonal   & $0.251$ & $0.251$ & $0.251$ & $0.251$ \\
AutoETS   & $0.407$ & $0.407$ & $0.407$ & $0.407$\\
\bottomrule
\end{tabular}
\end{sc}
\end{small}
\end{center}
\vskip -0.1in
\end{table}

\begin{table}[h]
\caption{Average CRPS on \texttt{electricity} for $(1,0.1)$-user-level privacy and $\epsilon=0.5$, $\delta=10^{-7}$. SeSeasonal and AutoETS are without noise.
Bold font indicates the best label noise scale $\sigma_F$ per model, i.e., per row.}
\label{table:label_perturbation_training_electricity}
\vskip 0.15in
\begin{center}
\begin{small}
\begin{sc}
\begin{tabular}{l c c c c}
\toprule
Model & $\sigma_F = 0$ & $\sigma_F= 1$ & $\sigma_F = 2$ & $\sigma_F = 5$\\
\midrule
SimpleFF & $0.072$ \tiny{$\pm 0.001$} & $0.069$ \tiny{$\pm 0.001$} & $0.068$ \tiny{$\pm 0.002$} & $\mathbf{0.067}$ \tiny{$\pm 0.002$}\\
DeepAR & $0.071$ \tiny{$\pm 0.006$} & $0.074$ \tiny{$\pm 0.005$} & $0.069$ \tiny{$\pm 0.005$} & $\mathbf{0.067}$ \tiny{$\pm 0.005$}\\
iTransf. & $0.081$ \tiny{$\pm 0.005$} & $0.080$ \tiny{$\pm 0.004$} & $0.080$ \tiny{$\pm 0.005$} & $\mathbf{0.080}$ \tiny{$\pm 0.004$}\\
DLinear & $0.064$ \tiny{$\pm 0.000$} & $0.062$ \tiny{$\pm 0.000$} & $0.061$ \tiny{$\pm 0.001$} & $\mathbf{0.061}$ \tiny{$\pm 0.001$} \\
\midrule
Seasonal   & $0.070$ & $0.070$ & $0.070$ & $0.070$ \\
AutoETS   & $0.064$ & $0.064$ & $0.064$ & $0.064$ \\
\bottomrule
\end{tabular}
\end{sc}
\end{small}
\end{center}
\vskip -0.1in
\end{table}

\begin{table}[h]
\caption{Average CRPS on \texttt{solar\_10\_minutes} for $(1,0.01)$-user-level privacy and $\epsilon=0.5$, $\delta=10^{-7}$. Seasonal and AutoETS are without noise.
Bold font indicates the best label noise scale $\sigma_F$ per model, i.e., per row.}
\label{table:label_perturbation_training_solar}
\vskip 0.15in
\begin{center}
\begin{small}
\begin{sc}
\begin{tabular}{l c c c c}
\toprule
Model & $\sigma_F = 0$ & $\sigma_F= 1$ & $\sigma_F = 2$ & $\sigma_F = 5$\\
\midrule
SimpleFF  & $1.117$ \tiny{$\pm 0.034$} & $1.126$ \tiny{$\pm 0.046$} & $\mathbf{1.117}$ \tiny{$\pm 0.042$} & $1.119$ \tiny{$\pm 0.041$} \\ 
DeepAR  & $0.820$ \tiny{$\pm 0.030$} & $0.818$ \tiny{$\pm 0.028$} & $0.814$ \tiny{$\pm 0.027$} & $\mathbf{0.813}$ \tiny{$\pm 0.026$} \\
iTransf.  & $0.976$ \tiny{$\pm 0.064$} & $0.970$ \tiny{$\pm 0.062$} & $0.960$ \tiny{$\pm 0.052$} & $\mathbf{0.959}$ \tiny{$\pm 0.056$} \\
DLinear & $1.282$ \tiny{$\pm 0.211$} & $1.245$ \tiny{$\pm 0.232$} & $\mathbf{1.238}$ \tiny{$\pm 0.240$} & $1.246$ \tiny{$\pm 0.227$} \\
\midrule
Seasonal   & $1.120$ & $1.120$ & $1.120$ & $1.120$ \\
AutoETS   & $6.494$ & $6.494$ & $6.494$ & $6.494$ \\
\bottomrule
\end{tabular}
\end{sc}
\end{small}
\end{center}
\vskip -0.1in
\end{table}
\clearpage\section{Settings and hyperparameters}\label{ap:parameters}
\section{Hyperparameter Search}\label{app:hype}
\normalsize
We exclusively conduct hyperparameter search on fold 0. 
For \textbf{GraFITi}~\citep{Yalavarthi2024.GraFITi} the hyperparameters for the search are as follows:
\begin{itemize}
    \item The number of layers, with possible values [1, 2, 3, 4].
    \item The number of attention heads, with possible values [1, 2, 4].
    \item The latent dimension, with possible values [16, 32, 64, 128, 256].
\end{itemize}

For the \textbf{LinODEnet} model~\citep{Scholz2022.Latenta} we search the hyperparameters from:
\begin{itemize}
    \item The hidden dimension, with possible values [16, 32, 64, 128].
    \item The latent dimension, with possible values [64, 128, 192, 256].
\end{itemize}

For \textbf{GRU-ODE-Bayes}~\citep{DeBrouwer2019.GRUODEBayesd} we tune the hidden size from [16, 32, 64, 128, 256]

For \textbf{Neural Flows}~\citep{Bilos2021.Neurald} we define the hyperparameter spaces for the search are as follows:
\begin{itemize}
    \item The number of flow layers, with possible values [1, 2, 4].
    \item The hidden dimension, with possible values [16, 32, 64, 128, 256].
    \item The flow model type, with possible values [GRU, ResNet].
\end{itemize}

For the \textbf{CRU}~\citep{Schirmer2022.Modelingb} the hyperparameter space is as follows:
\begin{itemize}
    \item The latent state dimension, with possible values [10, 20, 30].
    \item The number of basis functions, with possible values [10, 20].
    \item The bandwidth with possible values [3, 10].
\end{itemize}

\clearpage\section{Extra results}\label{ap:results}

\begin{table*}[t]
\centering
\fontsize{11pt}{11pt}\selectfont
\begin{tabular}{lllllllllllll}
\toprule
\multicolumn{1}{c}{\textbf{task}} & \multicolumn{2}{c}{\textbf{Mir}} & \multicolumn{2}{c}{\textbf{Lai}} & \multicolumn{2}{c}{\textbf{Ziegen.}} & \multicolumn{2}{c}{\textbf{Cao}} & \multicolumn{2}{c}{\textbf{Alva-Man.}} & \multicolumn{1}{c}{\textbf{avg.}} & \textbf{\begin{tabular}[c]{@{}l@{}}avg.\\ rank\end{tabular}} \\
\multicolumn{1}{c}{\textbf{metrics}} & \multicolumn{1}{c}{\textbf{cor.}} & \multicolumn{1}{c}{\textbf{p-v.}} & \multicolumn{1}{c}{\textbf{cor.}} & \multicolumn{1}{c}{\textbf{p-v.}} & \multicolumn{1}{c}{\textbf{cor.}} & \multicolumn{1}{c}{\textbf{p-v.}} & \multicolumn{1}{c}{\textbf{cor.}} & \multicolumn{1}{c}{\textbf{p-v.}} & \multicolumn{1}{c}{\textbf{cor.}} & \multicolumn{1}{c}{\textbf{p-v.}} &  &  \\ \midrule
\textbf{S-Bleu} & 0.50 & 0.0 & 0.47 & 0.0 & 0.59 & 0.0 & 0.58 & 0.0 & 0.68 & 0.0 & 0.57 & 5.8 \\
\textbf{R-Bleu} & -- & -- & 0.27 & 0.0 & 0.30 & 0.0 & -- & -- & -- & -- & - &  \\
\textbf{S-Meteor} & 0.49 & 0.0 & 0.48 & 0.0 & 0.61 & 0.0 & 0.57 & 0.0 & 0.64 & 0.0 & 0.56 & 6.1 \\
\textbf{R-Meteor} & -- & -- & 0.34 & 0.0 & 0.26 & 0.0 & -- & -- & -- & -- & - &  \\
\textbf{S-Bertscore} & \textbf{0.53} & 0.0 & {\ul 0.80} & 0.0 & \textbf{0.70} & 0.0 & {\ul 0.66} & 0.0 & {\ul0.78} & 0.0 & \textbf{0.69} & \textbf{1.7} \\
\textbf{R-Bertscore} & -- & -- & 0.51 & 0.0 & 0.38 & 0.0 & -- & -- & -- & -- & - &  \\
\textbf{S-Bleurt} & {\ul 0.52} & 0.0 & {\ul 0.80} & 0.0 & 0.60 & 0.0 & \textbf{0.70} & 0.0 & \textbf{0.80} & 0.0 & {\ul 0.68} & {\ul 2.3} \\
\textbf{R-Bleurt} & -- & -- & 0.59 & 0.0 & -0.05 & 0.13 & -- & -- & -- & -- & - &  \\
\textbf{S-Cosine} & 0.51 & 0.0 & 0.69 & 0.0 & {\ul 0.62} & 0.0 & 0.61 & 0.0 & 0.65 & 0.0 & 0.62 & 4.4 \\
\textbf{R-Cosine} & -- & -- & 0.40 & 0.0 & 0.29 & 0.0 & -- & -- & -- & -- & - & \\ \midrule
\textbf{QuestEval} & 0.23 & 0.0 & 0.25 & 0.0 & 0.49 & 0.0 & 0.47 & 0.0 & 0.62 & 0.0 & 0.41 & 9.0 \\
\textbf{LLaMa3} & 0.36 & 0.0 & \textbf{0.84} & 0.0 & {\ul{0.62}} & 0.0 & 0.61 & 0.0 &  0.76 & 0.0 & 0.64 & 3.6 \\
\textbf{our (3b)} & 0.49 & 0.0 & 0.73 & 0.0 & 0.54 & 0.0 & 0.53 & 0.0 & 0.7 & 0.0 & 0.60 & 5.8 \\
\textbf{our (8b)} & 0.48 & 0.0 & 0.73 & 0.0 & 0.52 & 0.0 & 0.53 & 0.0 & 0.7 & 0.0 & 0.59 & 6.3 \\  \bottomrule
\end{tabular}
\caption{Pearson correlation on human evaluation on system output. `R-': reference-based. `S-': source-based.}
\label{tab:sys}
\end{table*}



\begin{table}%[]
\centering
\fontsize{11pt}{11pt}\selectfont
\begin{tabular}{llllll}
\toprule
\multicolumn{1}{c}{\textbf{task}} & \multicolumn{1}{c}{\textbf{Lai}} & \multicolumn{1}{c}{\textbf{Zei.}} & \multicolumn{1}{c}{\textbf{Scia.}} & \textbf{} & \textbf{} \\ 
\multicolumn{1}{c}{\textbf{metrics}} & \multicolumn{1}{c}{\textbf{cor.}} & \multicolumn{1}{c}{\textbf{cor.}} & \multicolumn{1}{c}{\textbf{cor.}} & \textbf{avg.} & \textbf{\begin{tabular}[c]{@{}l@{}}avg.\\ rank\end{tabular}} \\ \midrule
\textbf{S-Bleu} & 0.40 & 0.40 & 0.19* & 0.33 & 7.67 \\
\textbf{S-Meteor} & 0.41 & 0.42 & 0.16* & 0.33 & 7.33 \\
\textbf{S-BertS.} & {\ul0.58} & 0.47 & 0.31 & 0.45 & 3.67 \\
\textbf{S-Bleurt} & 0.45 & {\ul 0.54} & {\ul 0.37} & 0.45 & {\ul 3.33} \\
\textbf{S-Cosine} & 0.56 & 0.52 & 0.3 & {\ul 0.46} & {\ul 3.33} \\ \midrule
\textbf{QuestE.} & 0.27 & 0.35 & 0.06* & 0.23 & 9.00 \\
\textbf{LlaMA3} & \textbf{0.6} & \textbf{0.67} & \textbf{0.51} & \textbf{0.59} & \textbf{1.0} \\
\textbf{Our (3b)} & 0.51 & 0.49 & 0.23* & 0.39 & 4.83 \\
\textbf{Our (8b)} & 0.52 & 0.49 & 0.22* & 0.43 & 4.83 \\ \bottomrule
\end{tabular}
\caption{Pearson correlation on human ratings on reference output. *not significant; we cannot reject the null hypothesis of zero correlation}
\label{tab:ref}
\end{table}


\begin{table*}%[]
\centering
\fontsize{11pt}{11pt}\selectfont
\begin{tabular}{lllllllll}
\toprule
\textbf{task} & \multicolumn{1}{c}{\textbf{ALL}} & \multicolumn{1}{c}{\textbf{sentiment}} & \multicolumn{1}{c}{\textbf{detoxify}} & \multicolumn{1}{c}{\textbf{catchy}} & \multicolumn{1}{c}{\textbf{polite}} & \multicolumn{1}{c}{\textbf{persuasive}} & \multicolumn{1}{c}{\textbf{formal}} & \textbf{\begin{tabular}[c]{@{}l@{}}avg. \\ rank\end{tabular}} \\
\textbf{metrics} & \multicolumn{1}{c}{\textbf{cor.}} & \multicolumn{1}{c}{\textbf{cor.}} & \multicolumn{1}{c}{\textbf{cor.}} & \multicolumn{1}{c}{\textbf{cor.}} & \multicolumn{1}{c}{\textbf{cor.}} & \multicolumn{1}{c}{\textbf{cor.}} & \multicolumn{1}{c}{\textbf{cor.}} &  \\ \midrule
\textbf{S-Bleu} & -0.17 & -0.82 & -0.45 & -0.12* & -0.1* & -0.05 & -0.21 & 8.42 \\
\textbf{R-Bleu} & - & -0.5 & -0.45 &  &  &  &  &  \\
\textbf{S-Meteor} & -0.07* & -0.55 & -0.4 & -0.01* & 0.1* & -0.16 & -0.04* & 7.67 \\
\textbf{R-Meteor} & - & -0.17* & -0.39 & - & - & - & - & - \\
\textbf{S-BertScore} & 0.11 & -0.38 & -0.07* & -0.17* & 0.28 & 0.12 & 0.25 & 6.0 \\
\textbf{R-BertScore} & - & -0.02* & -0.21* & - & - & - & - & - \\
\textbf{S-Bleurt} & 0.29 & 0.05* & 0.45 & 0.06* & 0.29 & 0.23 & 0.46 & 4.2 \\
\textbf{R-Bleurt} & - &  0.21 & 0.38 & - & - & - & - & - \\
\textbf{S-Cosine} & 0.01* & -0.5 & -0.13* & -0.19* & 0.05* & -0.05* & 0.15* & 7.42 \\
\textbf{R-Cosine} & - & -0.11* & -0.16* & - & - & - & - & - \\ \midrule
\textbf{QuestEval} & 0.21 & {\ul{0.29}} & 0.23 & 0.37 & 0.19* & 0.35 & 0.14* & 4.67 \\
\textbf{LlaMA3} & \textbf{0.82} & \textbf{0.80} & \textbf{0.72} & \textbf{0.84} & \textbf{0.84} & \textbf{0.90} & \textbf{0.88} & \textbf{1.00} \\
\textbf{Our (3b)} & 0.47 & -0.11* & 0.37 & 0.61 & 0.53 & 0.54 & 0.66 & 3.5 \\
\textbf{Our (8b)} & {\ul{0.57}} & 0.09* & {\ul 0.49} & {\ul 0.72} & {\ul 0.64} & {\ul 0.62} & {\ul 0.67} & {\ul 2.17} \\ \bottomrule
\end{tabular}
\caption{Pearson correlation on human ratings on our constructed test set. 'R-': reference-based. 'S-': source-based. *not significant; we cannot reject the null hypothesis of zero correlation}
\label{tab:con}
\end{table*}

\section{Results}
We benchmark the different metrics on the different datasets using correlation to human judgement. For content preservation, we show results split on data with system output, reference output and our constructed test set: we show that the data source for evaluation leads to different conclusions on the metrics. In addition, we examine whether the metrics can rank style transfer systems similar to humans. On style strength, we likewise show correlations between human judgment and zero-shot evaluation approaches. When applicable, we summarize results by reporting the average correlation. And the average ranking of the metric per dataset (by ranking which metric obtains the highest correlation to human judgement per dataset). 

\subsection{Content preservation}
\paragraph{How do data sources affect the conclusion on best metric?}
The conclusions about the metrics' performance change radically depending on whether we use system output data, reference output, or our constructed test set. Ideally, a good metric correlates highly with humans on any data source. Ideally, for meta-evaluation, a metric should correlate consistently across all data sources, but the following shows that the correlations indicate different things, and the conclusion on the best metric should be drawn carefully.

Looking at the metrics correlations with humans on the data source with system output (Table~\ref{tab:sys}), we see a relatively high correlation for many of the metrics on many tasks. The overall best metrics are S-BertScore and S-BLEURT (avg+avg rank). We see no notable difference in our method of using the 3B or 8B model as the backbone.

Examining the average correlations based on data with reference output (Table~\ref{tab:ref}), now the zero-shoot prompting with LlaMA3 70B is the best-performing approach ($0.59$ avg). Tied for second place are source-based cosine embedding ($0.46$ avg), BLEURT ($0.45$ avg) and BertScore ($0.45$ avg). Our method follows on a 5. place: here, the 8b version (($0.43$ avg)) shows a bit stronger results than 3b ($0.39$ avg). The fact that the conclusions change, whether looking at reference or system output, confirms the observations made by \citet{scialom-etal-2021-questeval} on simplicity transfer.   

Now consider the results on our test set (Table~\ref{tab:con}): Several metrics show low or no correlation; we even see a significantly negative correlation for some metrics on ALL (BLEU) and for specific subparts of our test set for BLEU, Meteor, BertScore, Cosine. On the other end, LlaMA3 70B is again performing best, showing strong results ($0.82$ in ALL). The runner-up is now our 8B method, with a gap to the 3B version ($0.57$ vs $0.47$ in ALL). Note our method still shows zero correlation for the sentiment task. After, ranks BLEURT ($0.29$), QuestEval ($0.21$), BertScore ($0.11$), Cosine ($0.01$).  

On our test set, we find that some metrics that correlate relatively well on the other datasets, now exhibit low correlation. Hence, with our test set, we can now support the logical reasoning with data evidence: Evaluation of content preservation for style transfer needs to take the style shift into account. This conclusion could not be drawn using the existing data sources: We hypothesise that for the data with system-based output, successful output happens to be very similar to the source sentence and vice versa, and reference-based output might not contain server mistakes as they are gold references. Thus, none of the existing data sources tests the limits of the metrics.  


\paragraph{How do reference-based metrics compare to source-based ones?} Reference-based metrics show a lower correlation than the source-based counterpart for all metrics on both datasets with ratings on references (Table~\ref{tab:sys}). As discussed previously, reference-based metrics for style transfer have the drawback that many different good solutions on a rewrite might exist and not only one similar to a reference.


\paragraph{How well can the metrics rank the performance of style transfer methods?}
We compare the metrics' ability to judge the best style transfer methods w.r.t. the human annotations: Several of the data sources contain samples from different style transfer systems. In order to use metrics to assess the quality of the style transfer system, metrics should correctly find the best-performing system. Hence, we evaluate whether the metrics for content preservation provide the same system ranking as human evaluators. We take the mean of the score for every output on each system and the mean of the human annotations; we compare the systems using the Kendall's Tau correlation. 

We find only the evaluation using the dataset Mir, Lai, and Ziegen to result in significant correlations, probably because of sparsity in a number of system tests (App.~\ref{app:dataset}). Our method (8b) is the only metric providing a perfect ranking of the style transfer system on the Lai data, and Llama3 70B the only one on the Ziegen data. Results in App.~\ref{app:results}. 


\subsection{Style strength results}
%Evaluating style strengths is a challenging task. 
Llama3 70B shows better overall results than our method. However, our method scores higher than Llama3 70B on 2 out of 6 datasets, but it also exhibits zero correlation on one task (Table~\ref{tab:styleresults}).%More work i s needed on evaluating style strengths. 
 
\begin{table}%[]
\fontsize{11pt}{11pt}\selectfont
\begin{tabular}{lccc}
\toprule
\multicolumn{1}{c}{\textbf{}} & \textbf{LlaMA3} & \textbf{Our (3b)} & \textbf{Our (8b)} \\ \midrule
\textbf{Mir} & 0.46 & 0.54 & \textbf{0.57} \\
\textbf{Lai} & \textbf{0.57} & 0.18 & 0.19 \\
\textbf{Ziegen.} & 0.25 & 0.27 & \textbf{0.32} \\
\textbf{Alva-M.} & \textbf{0.59} & 0.03* & 0.02* \\
\textbf{Scialom} & \textbf{0.62} & 0.45 & 0.44 \\
\textbf{\begin{tabular}[c]{@{}l@{}}Our Test\end{tabular}} & \textbf{0.63} & 0.46 & 0.48 \\ \bottomrule
\end{tabular}
\caption{Style strength: Pearson correlation to human ratings. *not significant; we cannot reject the null hypothesis of zero corelation}
\label{tab:styleresults}
\end{table}

\subsection{Ablation}
We conduct several runs of the methods using LLMs with variations in instructions/prompts (App.~\ref{app:method}). We observe that the lower the correlation on a task, the higher the variation between the different runs. For our method, we only observe low variance between the runs.
None of the variations leads to different conclusions of the meta-evaluation. Results in App.~\ref{app:results}.



\end{document}
