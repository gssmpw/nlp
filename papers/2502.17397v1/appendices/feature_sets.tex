The feature sets in the ensembling procedure are selected by the feature categorization and their individual discriminative power. To quantify the latter, we fit small models on each individual feature. Specifically, we fit a depth-2 decision tree to classify L, D or H-mode timeslice-by-timeslice. The performance is evaluated using Cohen's kappa coefficient~\cite{cohen1960}. Additionally, to identify parameter ranges consistently associated with a specific confinement state, we optimize thresholds on each individual feature that correspond to the largest amount of the data we can classify as `all L-mode' or `all H-mode' with at least 99\% accuracy. In other words, we check whether a feature can individually identify one of the two main confinement states in subparts of the parameter space. For example, total input power $P_{\textit{in}}$ can be used to trivially label some timeslices as L-mode given a minimum power requirement for any H-mode, see also Figure~\ref{fig:thresholds_example} for an illustration. We express this metric as the fraction of the data which can be labeled with such a threshold while keeping at least 99\% accuracy. Additionally, we report the signal availability over all timeslices in the dataset.

The results of all these metrics are provided in Table~\ref{tab:features}, for all features introduced in Table~\ref{tab:signals}. Note that we denote all spectral features computed from the photodiode signals ($\text{PD}_{\text{FFT}}^{}$ in Table~\ref{tab:signals}). The subscript integer denotes the window size in \SI{}{\milli\second} for the sliding window FFT, whereas the postfix $\in\{\textit{p}, \textit{c}, \textit{f}\}$ denotes whether the window is in the \textit{p}ast, \textit{c}entered or in the \textit{f}uture w.r.t. the given timeslice.

The feature sets cover both individual categories and combinations thereof. For each category we construct a model covering either all features or the most discriminative features following their Cohen's kappa coefficient values. The mixed feature sets cover both top-$k$ subsets for each category and rank-$k$ subsets: we both fit models taking in as much information as possible, while also fitting models with no mutual dependencies but still using informative features. Similarly, we select subsets using the threshold-orderings, although fewer in total because a substantial number of features cannot be used for any meaningful thresholding\footnote{Naturally, these features are still useful once combined with other more directly significant features.}, resulting in a value of 0 in Table~\ref{tab:features}. The resulting \textit{(model + feature set)} configurations are given in Table~\ref{tab:model_featuresets}.

The feature sets are identical between the FNOLSTM and GBDT models, with the exception of the photodiode related features. For the FNOLSTM we artificially rank $\text{PD}_{\textit{CIII}}^{}$ and $\text{PD}_{\textit{H}\alpha}^{}$ as the most informative emissions feature. These signals are not absolutely calibrated, making their raw value uninformative for classifying L, D or H-mode. However, the emission patterns they pick up clearly correspond to confinement state-related dynamics such as Edge Localized Modes (ELMs) or dithering cycles. The FNOLSTM-based models can fit these patterns because of their dynamic nature, making it a key feature to include. In contrast, the GBDT-models only take static information, making the raw signal value uninformative; rather, it relies on the constructed spectral features for the photodiode signal. To avoid redundancy in these $\text{PD}_{\text{FFT}}$ features, only the centered-window feature is used for each time window size; the past and future windows are only used in a specific category with all FFT features (FNOLSTM-EM-3 and GBDT-EM-3). 

\begin{figure}[h]
\begin{center}\includegraphics[width=0.5\linewidth]{figures/dataset_thresholds_example.pdf}\end{center}
    \caption{Distributions of the total input power and the plasma stored energy in the dataset, following the same procedure as Figure~\ref{fig:dataset_eda}. We overlay the `all L-mode' thresholds (L$^{\textit{fraction}}_{0.99}$) for the two features: below this threshold value, at least 99\% of the timeslices are in L-mode.}
    \label{fig:thresholds_example}%
\end{figure}

\newpage


\arrayrulecolor{black}
\def\arrvline{\hfil\kern\arraycolsep\vline\kern-\arraycolsep\hfilneg}
\setlength\tabcolsep{5.3pt}
\begingroup
\centering
\begin{longtable}{lccccp{0.45cm}lcccc}
 & Cohen's & & & Fraction & & & Cohen's & & & Fraction\\[-3pt]
Feature & kappa & L$^{\textit{fraction}}_{0.99}$ & H$^{\textit{fraction}}_{0.99}$ & \makebox[0pt]{available} & & Feature & kappa & L$^{\textit{fraction}}_{0.99}$ & H$^{\textit{fraction}}_{0.99}$ & \makebox[0pt]{available} \\ \cmidrule[\heavyrulewidth]{1-5}\cmidrule[\heavyrulewidth]{7-11}
\addlinespace[-\belowrulesep]
\cellcolor[RGB]{255,210,204} $A_p$ & \cellcolor[RGB]{255.0,255.0,255.0} 0.000 & \cellcolor[RGB]{250.0,253.0,249.6} 0.033 & \cellcolor[RGB]{255.0,255.0,255.0} 0.00000 & \cellcolor[RGB]{179.9,224.4,173.9} 0.997 &  & \cellcolor[RGB]{209,235,255} $\text{PD}_{\text{FFT}_{50\textit{-c}}}^{{\textit{H}\alpha}}$ & \cellcolor[RGB]{201.5,233.2,197.2} 0.704 & \cellcolor[RGB]{254.7,254.9,254.7} 0.002 & \cellcolor[RGB]{244.4,250.7,243.5} 0.00021 & \cellcolor[RGB]{182.9,225.6,177.2} 0.985 \\
\cellcolor[RGB]{255,210,204} $\delta_{\text{bottom}}$ & \cellcolor[RGB]{249.4,252.7,249.0} 0.073 & \cellcolor[RGB]{243.5,250.3,242.6} 0.076 & \cellcolor[RGB]{255.0,255.0,255.0} 0.00000 & \cellcolor[RGB]{179.9,224.4,173.9} 0.997 &  & \cellcolor[RGB]{209,235,255} $\text{PD}_{\text{FFT}_{50\textit{-f}}}^{{\textit{H}\alpha}}$ & \cellcolor[RGB]{202.0,233.4,197.9} 0.697 & \cellcolor[RGB]{254.9,255.0,254.9} 0.001 & \cellcolor[RGB]{255.0,255.0,255.0} 0.00000 & \cellcolor[RGB]{187.0,227.3,181.7} 0.968 \\
\cellcolor[RGB]{255,210,204} $\delta_{\text{top}}$ & \cellcolor[RGB]{255.0,255.0,255.0} 0.000 & \cellcolor[RGB]{244.9,250.9,244.1} 0.066 & \cellcolor[RGB]{255.0,255.0,255.0} 0.00000 & \cellcolor[RGB]{179.9,224.4,173.9} 0.997 &  & \cellcolor[RGB]{209,235,255} $\text{PD}_{\text{FFT}_{100\textit{-p}}}^{{\textit{H}\alpha}}$ & \cellcolor[RGB]{212.2,237.5,208.8} 0.563 & \cellcolor[RGB]{255.0,255.0,255.0} 0.000 & \cellcolor[RGB]{254.7,254.9,254.7} 0.00001 & \cellcolor[RGB]{179.0,224.0,173.0} 1.000 \\
\cellcolor[RGB]{255,210,204} $\Delta_{\text{in}}$ & \cellcolor[RGB]{255.0,255.0,255.0} 0.000 & \cellcolor[RGB]{217.9,239.9,215.0} 0.244 & \cellcolor[RGB]{255.0,255.0,255.0} 0.00000 & \cellcolor[RGB]{179.9,224.4,173.9} 0.997 &  & \cellcolor[RGB]{209,235,255} $\text{PD}_{\text{FFT}_{100\textit{-c}}}^{{\textit{H}\alpha}}$ & \cellcolor[RGB]{204.5,234.4,200.6} 0.664 & \cellcolor[RGB]{255.0,255.0,255.0} 0.000 & \cellcolor[RGB]{243.2,250.2,242.3} 0.00023 & \cellcolor[RGB]{187.0,227.3,181.7} 0.968 \\
\cellcolor[RGB]{255,210,204} $\Delta_{\text{out}}$ & \cellcolor[RGB]{255.0,255.0,255.0} 0.000 & \cellcolor[RGB]{254.7,254.9,254.7} 0.002 & \cellcolor[RGB]{255.0,255.0,255.0} 0.00000 & \cellcolor[RGB]{179.9,224.4,173.9} 0.997 &  & \cellcolor[RGB]{209,235,255} $\text{PD}_{\text{FFT}_{100\textit{-f}}}^{{\textit{H}\alpha}}$ & \cellcolor[RGB]{205.1,234.6,201.1} 0.657 & \cellcolor[RGB]{254.9,255.0,254.9} 0.001 & \cellcolor[RGB]{255.0,255.0,255.0} 0.00000 & \cellcolor[RGB]{195.4,230.7,190.7} 0.935 \\
\cellcolor[RGB]{255,210,204} $\kappa$ & \cellcolor[RGB]{229.9,244.8,227.9} 0.330 & \cellcolor[RGB]{241.8,249.6,240.8} 0.087 & \cellcolor[RGB]{255.0,255.0,255.0} 0.00000 & \cellcolor[RGB]{179.9,224.4,173.9} 0.997 &  & \cellcolor[RGB]{234,255,227} $B_0$ & \cellcolor[RGB]{243.0,250.1,242.0} 0.158 & \cellcolor[RGB]{255.0,255.0,255.0} 0.000 & \cellcolor[RGB]{255.0,255.0,255.0} 0.00000 & \cellcolor[RGB]{179.9,224.4,173.9} 0.997 \\
\cellcolor[RGB]{255,210,204} $R_0$ & \cellcolor[RGB]{255.0,255.0,255.0} 0.000 & \cellcolor[RGB]{244.1,250.6,243.2} 0.072 & \cellcolor[RGB]{255.0,255.0,255.0} 0.00000 & \cellcolor[RGB]{179.9,224.4,173.9} 0.997 &  & \cellcolor[RGB]{234,255,227} $I_{p}$ & \cellcolor[RGB]{239.9,248.8,238.7} 0.199 & \cellcolor[RGB]{251.2,253.4,250.8} 0.025 & \cellcolor[RGB]{255.0,255.0,255.0} 0.00000 & \cellcolor[RGB]{179.0,224.0,173.0} 1.000 \\
\cellcolor[RGB]{255,210,204} $a$ & \cellcolor[RGB]{255.0,255.0,255.0} 0.000 & \cellcolor[RGB]{252.5,254.0,252.3} 0.016 & \cellcolor[RGB]{255.0,255.0,255.0} 0.00000 & \cellcolor[RGB]{179.9,224.4,173.9} 0.997 &  & \cellcolor[RGB]{234,255,227} $I_{p,\textit{ref}}$ & \cellcolor[RGB]{239.7,248.8,238.5} 0.201 & \cellcolor[RGB]{250.7,253.2,250.4} 0.028 & \cellcolor[RGB]{255.0,255.0,255.0} 0.00000 & \cellcolor[RGB]{179.0,224.0,173.0} 1.000 \\
\cellcolor[RGB]{255,210,204} $R_{\text{axis}}$ & \cellcolor[RGB]{255.0,255.0,255.0} 0.000 & \cellcolor[RGB]{244.1,250.6,243.2} 0.072 & \cellcolor[RGB]{255.0,255.0,255.0} 0.00000 & \cellcolor[RGB]{179.9,224.4,173.9} 0.997 &  & \cellcolor[RGB]{234,255,227} $q_{95}$ & \cellcolor[RGB]{239.2,248.6,238.0} 0.208 & \cellcolor[RGB]{254.9,254.9,254.9} 0.001 & \cellcolor[RGB]{255.0,255.0,255.0} 0.00000 & \cellcolor[RGB]{179.9,224.4,173.9} 0.997 \\
\cellcolor[RGB]{255,210,204} $Z_{\text{axis}}$ & \cellcolor[RGB]{252.6,254.0,252.4} 0.032 & \cellcolor[RGB]{255.0,255.0,255.0} 0.000 & \cellcolor[RGB]{255.0,255.0,255.0} 0.00000 & \cellcolor[RGB]{179.9,224.4,173.9} 0.997 &  & \cellcolor[RGB]{252,233,255} $n_{e,\text{core}}$ & \cellcolor[RGB]{222.4,241.7,219.9} 0.429 & \cellcolor[RGB]{255.0,255.0,255.0} 0.000 & \cellcolor[RGB]{255.0,255.0,255.0} 0.00000 & \cellcolor[RGB]{179.0,224.0,173.0} 1.000 \\
\cellcolor[RGB]{255,210,204} $V_p$ & \cellcolor[RGB]{255.0,255.0,255.0} 0.000 & \cellcolor[RGB]{250.1,253.0,249.7} 0.033 & \cellcolor[RGB]{255.0,255.0,255.0} 0.00000 & \cellcolor[RGB]{179.9,224.4,173.9} 0.997 &  & \cellcolor[RGB]{252,233,255} $n_{e,\text{LFS}}$ & \cellcolor[RGB]{216.3,239.2,213.3} 0.509 & \cellcolor[RGB]{255.0,255.0,255.0} 0.000 & \cellcolor[RGB]{255.0,255.0,255.0} 0.00000 & \cellcolor[RGB]{179.0,224.0,173.0} 1.000 \\
\cellcolor[RGB]{209,235,255} $\text{PD}_{\textit{CIII}}^{}$ & \cellcolor[RGB]{255.0,255.0,255.0} 0.000 & \cellcolor[RGB]{253.6,254.4,253.5} 0.009 & \cellcolor[RGB]{255.0,255.0,255.0} 0.00000 & \cellcolor[RGB]{179.0,224.0,173.0} 1.000 &  & \cellcolor[RGB]{252,233,255} $n_e/n_{\textit{GW}}$ & \cellcolor[RGB]{232.6,245.8,230.8} 0.295 & \cellcolor[RGB]{255.0,255.0,255.0} 0.000 & \cellcolor[RGB]{202.5,233.6,198.4} 0.00104 & \cellcolor[RGB]{189.9,228.5,184.8} 0.957 \\
\cellcolor[RGB]{209,235,255} $\text{PD}_{\textit{H}\alpha}^{}$ & \cellcolor[RGB]{255.0,255.0,255.0} 0.000 & \cellcolor[RGB]{255.0,255.0,255.0} 0.000 & \cellcolor[RGB]{255.0,255.0,255.0} 0.00000 & \cellcolor[RGB]{179.0,224.0,173.0} 1.000 &  & \cellcolor[RGB]{252,233,255} $\text{max}(n'_{e,\text{edge}})$ & \cellcolor[RGB]{219.0,240.3,216.2} 0.473 & \cellcolor[RGB]{255.0,255.0,255.0} 0.000 & \cellcolor[RGB]{248.8,252.5,248.3} 0.00012 & \cellcolor[RGB]{213.4,238.0,210.2} 0.864 \\
\cellcolor[RGB]{209,235,255} $\text{PD}_{\text{FFT}_{5\textit{-p}}}^{{\textit{CIII}}}$ & \cellcolor[RGB]{214.1,238.3,210.9} 0.538 & \cellcolor[RGB]{237.4,247.8,236.0} 0.116 & \cellcolor[RGB]{255.0,255.0,255.0} 0.00000 & \cellcolor[RGB]{179.0,224.0,173.0} 1.000 &  & \cellcolor[RGB]{252,233,255} $\text{max}(n''_{e,\text{edge}})$ & \cellcolor[RGB]{206.3,235.1,202.4} 0.641 & \cellcolor[RGB]{255.0,255.0,255.0} 0.000 & \cellcolor[RGB]{251.2,253.5,250.9} 0.00007 & \cellcolor[RGB]{213.4,238.0,210.2} 0.864 \\
\cellcolor[RGB]{209,235,255} $\text{PD}_{\text{FFT}_{5\textit{-c}}}^{{\textit{CIII}}}$ & \cellcolor[RGB]{213.7,238.2,210.4} 0.543 & \cellcolor[RGB]{237.3,247.8,235.9} 0.116 & \cellcolor[RGB]{255.0,255.0,255.0} 0.00000 & \cellcolor[RGB]{179.3,224.1,173.3} 0.999 &  & \cellcolor[RGB]{252,233,255} $n_{e,0}$ & \cellcolor[RGB]{232.0,245.6,230.1} 0.303 & \cellcolor[RGB]{254.9,255.0,254.9} 0.001 & \cellcolor[RGB]{238.1,248.1,236.8} 0.00033 & \cellcolor[RGB]{193.9,230.1,189.1} 0.941 \\
\cellcolor[RGB]{209,235,255} $\text{PD}_{\text{FFT}_{5\textit{-f}}}^{{\textit{CIII}}}$ & \cellcolor[RGB]{213.3,238.0,210.0} 0.549 & \cellcolor[RGB]{237.4,247.8,236.0} 0.116 & \cellcolor[RGB]{255.0,255.0,255.0} 0.00000 & \cellcolor[RGB]{179.6,224.2,173.7} 0.998 &  & \cellcolor[RGB]{255,247,196} $\textit{SXR}_{\text{core}}$ & \cellcolor[RGB]{222.7,241.8,220.2} 0.425 & \cellcolor[RGB]{253.1,254.2,252.9} 0.013 & \cellcolor[RGB]{246.5,251.5,245.8} 0.00017 & \cellcolor[RGB]{216.5,239.3,213.5} 0.852 \\
\cellcolor[RGB]{209,235,255} $\text{PD}_{\text{FFT}_{10\textit{-p}}}^{{\textit{CIII}}}$ & \cellcolor[RGB]{209.0,236.2,205.4} 0.605 & \cellcolor[RGB]{226.0,243.2,223.7} 0.191 & \cellcolor[RGB]{255.0,255.0,255.0} 0.00000 & \cellcolor[RGB]{179.0,224.0,173.0} 1.000 &  & \cellcolor[RGB]{255,247,196} $\text{max}(T'_{e,\text{edge}})$ & \cellcolor[RGB]{210.5,236.9,207.0} 0.585 & \cellcolor[RGB]{255.0,255.0,255.0} 0.000 & \cellcolor[RGB]{238.8,248.4,237.5} 0.00032 & \cellcolor[RGB]{213.4,238.0,210.2} 0.864 \\
\cellcolor[RGB]{209,235,255} $\text{PD}_{\text{FFT}_{10\textit{-c}}}^{{\textit{CIII}}}$ & \cellcolor[RGB]{208.2,235.9,204.5} 0.616 & \cellcolor[RGB]{225.5,242.9,223.1} 0.194 & \cellcolor[RGB]{255.0,255.0,255.0} 0.00000 & \cellcolor[RGB]{179.6,224.2,173.7} 0.998 &  & \cellcolor[RGB]{255,247,196} $\text{max}(T''_{e,\text{edge}})$ & \cellcolor[RGB]{211.0,237.0,207.5} 0.579 & \cellcolor[RGB]{254.7,254.9,254.6} 0.002 & \cellcolor[RGB]{253.0,254.2,252.8} 0.00004 & \cellcolor[RGB]{213.4,238.0,210.2} 0.864 \\
\cellcolor[RGB]{209,235,255} $\text{PD}_{\text{FFT}_{10\textit{-f}}}^{{\textit{CIII}}}$ & \cellcolor[RGB]{207.6,235.7,203.8} 0.624 & \cellcolor[RGB]{225.9,243.1,223.6} 0.192 & \cellcolor[RGB]{255.0,255.0,255.0} 0.00000 & \cellcolor[RGB]{180.4,224.6,174.5} 0.995 &  & \cellcolor[RGB]{255,247,196} $T_{e,0}$ & \cellcolor[RGB]{224.8,242.7,222.5} 0.397 & \cellcolor[RGB]{254.7,254.9,254.7} 0.002 & \cellcolor[RGB]{255.0,255.0,255.0} 0.00000 & \cellcolor[RGB]{194.4,230.3,189.6} 0.939 \\
\cellcolor[RGB]{209,235,255} $\text{PD}_{\text{FFT}_{20\textit{-p}}}^{{\textit{CIII}}}$ & \cellcolor[RGB]{207.1,235.5,203.4} 0.630 & \cellcolor[RGB]{218.6,240.1,215.7} 0.240 & \cellcolor[RGB]{255.0,255.0,255.0} 0.00000 & \cellcolor[RGB]{179.0,224.0,173.0} 1.000 &  & \cellcolor[RGB]{255,255,234} $P_{\textit{in}}$ & \cellcolor[RGB]{219.1,240.3,216.2} 0.473 & \cellcolor[RGB]{187.9,227.6,182.6} 0.442 & \cellcolor[RGB]{255.0,255.0,255.0} 0.00000 & \cellcolor[RGB]{179.9,224.4,173.9} 0.997 \\
\cellcolor[RGB]{209,235,255} $\text{PD}_{\text{FFT}_{20\textit{-c}}}^{{\textit{CIII}}}$ & \cellcolor[RGB]{205.7,234.9,201.8} 0.649 & \cellcolor[RGB]{217.9,239.9,214.9} 0.244 & \cellcolor[RGB]{255.0,255.0,255.0} 0.00000 & \cellcolor[RGB]{180.4,224.6,174.5} 0.995 &  & \cellcolor[RGB]{255,255,234} $P_{\textit{OHM}}$ & \cellcolor[RGB]{249.6,252.8,249.2} 0.071 & \cellcolor[RGB]{255.0,255.0,255.0} 0.000 & \cellcolor[RGB]{255.0,255.0,255.0} 0.00000 & \cellcolor[RGB]{179.9,224.4,173.9} 0.997 \\
\cellcolor[RGB]{209,235,255} $\text{PD}_{\text{FFT}_{20\textit{-f}}}^{{\textit{CIII}}}$ & \cellcolor[RGB]{204.9,234.6,200.9} 0.660 & \cellcolor[RGB]{217.4,239.7,214.4} 0.247 & \cellcolor[RGB]{255.0,255.0,255.0} 0.00000 & \cellcolor[RGB]{182.0,225.2,176.3} 0.988 &  & \cellcolor[RGB]{255,255,234} $P_{\textit{NBI}}$ & \cellcolor[RGB]{219.3,240.4,216.5} 0.470 & \cellcolor[RGB]{254.9,254.9,254.9} 0.001 & \cellcolor[RGB]{255.0,255.0,255.0} 0.00000 & \cellcolor[RGB]{179.0,224.0,173.0} 1.000 \\
\cellcolor[RGB]{209,235,255} $\text{PD}_{\text{FFT}_{50\textit{-p}}}^{{\textit{CIII}}}$ & \cellcolor[RGB]{209.1,236.3,205.5} 0.604 & \cellcolor[RGB]{210.3,236.8,206.8} 0.294 & \cellcolor[RGB]{255.0,255.0,255.0} 0.00000 & \cellcolor[RGB]{179.0,224.0,173.0} 1.000 &  & \cellcolor[RGB]{255,255,234} $P_{\textit{NBI2}}$ & \cellcolor[RGB]{253.5,254.4,253.4} 0.020 & \cellcolor[RGB]{255.0,255.0,255.0} 0.000 & \cellcolor[RGB]{254.9,255.0,254.9} 0.00000 & \cellcolor[RGB]{179.0,224.0,173.0} 1.000 \\
\cellcolor[RGB]{209,235,255} $\text{PD}_{\text{FFT}_{50\textit{-c}}}^{{\textit{CIII}}}$ & \cellcolor[RGB]{205.8,234.9,201.9} 0.647 & \cellcolor[RGB]{203.6,234.0,199.5} 0.338 & \cellcolor[RGB]{255.0,255.0,255.0} 0.00000 & \cellcolor[RGB]{182.9,225.6,177.2} 0.985 &  & \cellcolor[RGB]{255,255,234} $P_{\textit{ECRH}}$ & \cellcolor[RGB]{238.8,248.4,237.5} 0.213 & \cellcolor[RGB]{255.0,255.0,255.0} 0.000 & \cellcolor[RGB]{255.0,255.0,255.0} 0.00000 & \cellcolor[RGB]{179.0,224.0,173.0} 1.000 \\
\cellcolor[RGB]{209,235,255} $\text{PD}_{\text{FFT}_{50\textit{-f}}}^{{\textit{CIII}}}$ & \cellcolor[RGB]{204.6,234.4,200.6} 0.663 & \cellcolor[RGB]{197.8,231.7,193.3} 0.376 & \cellcolor[RGB]{255.0,255.0,255.0} 0.00000 & \cellcolor[RGB]{187.0,227.3,181.7} 0.968 &  & \cellcolor[RGB]{255,255,234} $P_{\textit{LH}}$ & \cellcolor[RGB]{222.7,241.8,220.1} 0.425 & \cellcolor[RGB]{255.0,255.0,255.0} 0.000 & \cellcolor[RGB]{254.6,254.8,254.5} 0.00001 & \cellcolor[RGB]{179.9,224.4,173.9} 0.997 \\
\cellcolor[RGB]{209,235,255} $\text{PD}_{\text{FFT}_{100\textit{-p}}}^{{\textit{CIII}}}$ & \cellcolor[RGB]{217.7,239.8,214.7} 0.491 & \cellcolor[RGB]{216.4,239.3,213.4} 0.254 & \cellcolor[RGB]{255.0,255.0,255.0} 0.00000 & \cellcolor[RGB]{179.0,224.0,173.0} 1.000 &  & \cellcolor[RGB]{255,246,219} $\beta_{N}$ & \cellcolor[RGB]{209.8,236.6,206.2} 0.595 & \cellcolor[RGB]{241.6,249.5,240.5} 0.088 & \cellcolor[RGB]{255.0,255.0,255.0} 0.00000 & \cellcolor[RGB]{179.9,224.4,173.9} 0.997 \\
\cellcolor[RGB]{209,235,255} $\text{PD}_{\text{FFT}_{100\textit{-c}}}^{{\textit{CIII}}}$ & \cellcolor[RGB]{210.0,236.7,206.5} 0.592 & \cellcolor[RGB]{201.0,233.0,196.7} 0.355 & \cellcolor[RGB]{255.0,255.0,255.0} 0.00000 & \cellcolor[RGB]{187.0,227.3,181.7} 0.968 &  & \cellcolor[RGB]{255,246,219} $\beta_{p}$ & \cellcolor[RGB]{222.7,241.8,220.2} 0.425 & \cellcolor[RGB]{250.8,253.3,250.5} 0.028 & \cellcolor[RGB]{255.0,255.0,255.0} 0.00000 & \cellcolor[RGB]{179.9,224.4,173.9} 0.997 \\
\cellcolor[RGB]{209,235,255} $\text{PD}_{\text{FFT}_{100\textit{-f}}}^{{\textit{CIII}}}$ & \cellcolor[RGB]{210.0,236.6,206.4} 0.592 & \cellcolor[RGB]{198.5,231.9,194.0} 0.372 & \cellcolor[RGB]{255.0,255.0,255.0} 0.00000 & \cellcolor[RGB]{195.4,230.7,190.7} 0.935 &  & \cellcolor[RGB]{255,246,219} $\beta_{t}$ & \cellcolor[RGB]{198.6,232.0,194.2} 0.742 & \cellcolor[RGB]{191.9,229.3,186.9} 0.415 & \cellcolor[RGB]{255.0,255.0,255.0} 0.00000 & \cellcolor[RGB]{179.9,224.4,173.9} 0.997 \\
\cellcolor[RGB]{209,235,255} $\text{PD}_{\text{FFT}_{5\textit{-p}}}^{{\textit{H}\alpha}}$ & \cellcolor[RGB]{207.8,235.8,204.1} 0.621 & \cellcolor[RGB]{254.9,255.0,254.9} 0.000 & \cellcolor[RGB]{251.2,253.5,250.9} 0.00007 & \cellcolor[RGB]{179.0,224.0,173.0} 1.000 &  & \cellcolor[RGB]{255,246,219} ${W}_{\textit{tot}}$ & \cellcolor[RGB]{202.1,233.4,197.9} 0.696 & \cellcolor[RGB]{208.4,236.0,204.8} 0.306 & \cellcolor[RGB]{254.8,254.9,254.8} 0.00000 & \cellcolor[RGB]{179.9,224.4,173.9} 0.997 \\
\cellcolor[RGB]{209,235,255} $\text{PD}_{\text{FFT}_{5\textit{-c}}}^{{\textit{H}\alpha}}$ & \cellcolor[RGB]{207.6,235.7,203.9} 0.624 & \cellcolor[RGB]{255.0,255.0,255.0} 0.000 & \cellcolor[RGB]{216.0,239.1,212.9} 0.00077 & \cellcolor[RGB]{179.3,224.1,173.3} 0.999 &  & \cellcolor[RGB]{255,246,219} $\text{DML}$ & \cellcolor[RGB]{226.0,243.2,223.7} 0.381 & \cellcolor[RGB]{254.9,254.9,254.8} 0.001 & \cellcolor[RGB]{255.0,255.0,255.0} 0.00000 & \cellcolor[RGB]{195.5,230.7,190.8} 0.935 \\
\cellcolor[RGB]{209,235,255} $\text{PD}_{\text{FFT}_{5\textit{-f}}}^{{\textit{H}\alpha}}$ & \cellcolor[RGB]{207.5,235.6,203.8} 0.625 & \cellcolor[RGB]{255.0,255.0,255.0} 0.000 & \cellcolor[RGB]{253.9,254.6,253.9} 0.00002 & \cellcolor[RGB]{179.6,224.2,173.7} 0.998 &  & \cellcolor[RGB]{255,246,219} $H_{\textit{98y2}}$ & \cellcolor[RGB]{242.4,249.8,241.4} 0.166 & \cellcolor[RGB]{249.7,252.8,249.3} 0.035 & \cellcolor[RGB]{254.0,254.6,254.0} 0.00002 & \cellcolor[RGB]{179.9,224.4,173.9} 0.997 \\
\cellcolor[RGB]{209,235,255} $\text{PD}_{\text{FFT}_{10\textit{-p}}}^{{\textit{H}\alpha}}$ & \cellcolor[RGB]{203.2,233.9,199.1} 0.681 & \cellcolor[RGB]{255.0,255.0,255.0} 0.000 & \cellcolor[RGB]{245.0,250.9,244.2} 0.00020 & \cellcolor[RGB]{179.0,224.0,173.0} 1.000 &  & \cellcolor[RGB]{255,248,255} $P_{\textit{rad}}$ & \cellcolor[RGB]{231.8,245.5,230.0} 0.305 & \cellcolor[RGB]{254.9,255.0,254.9} 0.000 & \cellcolor[RGB]{251.8,253.7,251.6} 0.00006 & \cellcolor[RGB]{215.2,238.7,212.0} 0.857 \\
\cellcolor[RGB]{209,235,255} $\text{PD}_{\text{FFT}_{10\textit{-c}}}^{{\textit{H}\alpha}}$ & \cellcolor[RGB]{202.6,233.6,198.5} 0.689 & \cellcolor[RGB]{255.0,255.0,255.0} 0.000 & \cellcolor[RGB]{253.2,254.3,253.1} 0.00003 & \cellcolor[RGB]{179.6,224.2,173.7} 0.998 &  & \cellcolor[RGB]{255,248,255} $P_{\textit{rad},\text{bulk}}$ & \cellcolor[RGB]{233.7,246.3,232.0} 0.281 & \cellcolor[RGB]{254.4,254.8,254.4} 0.004 & \cellcolor[RGB]{254.8,254.9,254.8} 0.00000 & \cellcolor[RGB]{215.3,238.8,212.1} 0.857 \\
\cellcolor[RGB]{209,235,255} $\text{PD}_{\text{FFT}_{10\textit{-f}}}^{{\textit{H}\alpha}}$ & \cellcolor[RGB]{202.2,233.4,198.0} 0.695 & \cellcolor[RGB]{254.8,254.9,254.8} 0.001 & \cellcolor[RGB]{254.5,254.8,254.5} 0.00001 & \cellcolor[RGB]{180.4,224.6,174.5} 0.995 &  & \cellcolor[RGB]{255,248,255} $P_{\textit{rad},\text{SOL}}$ & \cellcolor[RGB]{217.4,239.6,214.4} 0.495 & \cellcolor[RGB]{255.0,255.0,255.0} 0.000 & \cellcolor[RGB]{255.0,255.0,255.0} 0.00000 & \cellcolor[RGB]{207.0,235.4,203.2} 0.889 \\
\cellcolor[RGB]{209,235,255} $\text{PD}_{\text{FFT}_{20\textit{-p}}}^{{\textit{H}\alpha}}$ & \cellcolor[RGB]{201.8,233.3,197.6} 0.700 & \cellcolor[RGB]{255.0,255.0,255.0} 0.000 & \cellcolor[RGB]{250.2,253.0,249.8} 0.00010 & \cellcolor[RGB]{179.0,224.0,173.0} 1.000 &  & \cellcolor[RGB]{235,245,255} $l_i$ & \cellcolor[RGB]{225.0,242.8,222.7} 0.394 & \cellcolor[RGB]{254.9,254.9,254.9} 0.001 & \cellcolor[RGB]{255.0,255.0,255.0} 0.00000 & \cellcolor[RGB]{179.9,224.4,173.9} 0.997 \\
\cellcolor[RGB]{209,235,255} $\text{PD}_{\text{FFT}_{20\textit{-c}}}^{{\textit{H}\alpha}}$ & \cellcolor[RGB]{200.2,232.6,195.9} 0.721 & \cellcolor[RGB]{255.0,255.0,255.0} 0.000 & \cellcolor[RGB]{249.6,252.8,249.2} 0.00011 & \cellcolor[RGB]{180.4,224.6,174.5} 0.995 &  & \cellcolor[RGB]{235,245,255} $Z_{\textit{eff}}$ & \cellcolor[RGB]{255.0,255.0,255.0} 0.000 & \cellcolor[RGB]{254.8,254.9,254.8} 0.001 & \cellcolor[RGB]{255.0,255.0,255.0} 0.00000 & \cellcolor[RGB]{202.0,233.4,197.8} 0.909 \\
\cellcolor[RGB]{209,235,255} $\text{PD}_{\text{FFT}_{20\textit{-f}}}^{{\textit{H}\alpha}}$ & \cellcolor[RGB]{199.6,232.4,195.2} 0.729 & \cellcolor[RGB]{254.9,255.0,254.9} 0.001 & \cellcolor[RGB]{255.0,255.0,255.0} 0.00000 & \cellcolor[RGB]{182.0,225.2,176.3} 0.988 &  & \cellcolor[RGB]{235,245,255} $\nu_{e,\text{ped}}^{*}$ & \cellcolor[RGB]{222.9,241.9,220.4} 0.422 & \cellcolor[RGB]{254.8,254.9,254.8} 0.001 & \cellcolor[RGB]{255.0,255.0,255.0} 0.00000 & \cellcolor[RGB]{207.4,235.6,203.7} 0.888 \\
\cellcolor[RGB]{209,235,255} $\text{PD}_{\text{FFT}_{50\textit{-p}}}^{{\textit{H}\alpha}}$ & \cellcolor[RGB]{205.3,234.7,201.4} 0.654 & \cellcolor[RGB]{254.9,255.0,254.9} 0.001 & \cellcolor[RGB]{244.8,250.8,244.0} 0.00020 & \cellcolor[RGB]{179.0,224.0,173.0} 1.000 &  & \cellcolor[RGB]{235,245,255} $V_{\textit{loop}}$ & \cellcolor[RGB]{233.1,246.0,231.3} 0.289 & \cellcolor[RGB]{254.4,254.8,254.4} 0.004 & \cellcolor[RGB]{255.0,255.0,255.0} 0.00000 & \cellcolor[RGB]{179.0,224.0,173.0} 1.000

\\
\caption{Availability and discriminative capability of individual features, as used for constructing the ensemble feature sets. Features are colored by their categorization. For each feature we fit a depth-2 decision tree classifying all confinement states to get a general overview. Additionally, to identify features of interest w.r.t. regions far from the discriminative boundary, we compute the optimal threshold for a feature where at least 99\% of the timeslices are in L-mode or H-mode, and take the fraction of data subject to this threshold as the metric value. Lastly, we report the fraction of data where the signal is available.}\label{tab:features}
\end{longtable}
\endgroup
\setlength\tabcolsep{6pt}
\arrayrulecolor{black}

\arrayrulecolor[rgb]{0.9,0.9,0.9}
\setlength\tabcolsep{5.3pt}
\begingroup
\centering
\begin{longtable}{l|p{2.7cm}|p{10.9cm}}
Model & Description & Feature set \\\cmidrule[\heavyrulewidth]{1-3}
\addlinespace[-\belowrulesep]
FNOLSTM-SH-1 & Shaping, top 4. & \colorbox[RGB]{255,210,204}{\strut $A_p$}, \colorbox[RGB]{255,210,204}{\strut $\delta_{\text{bottom}}$}, \colorbox[RGB]{255,210,204}{\strut $\kappa$}, \colorbox[RGB]{255,210,204}{\strut $Z_{\text{axis}}$} \\\hline
FNOLSTM-SH-2 & Shaping, all. & \colorbox[RGB]{255,210,204}{\strut $A_p$}, \colorbox[RGB]{255,210,204}{\strut $\delta_{\text{bottom}}$}, \colorbox[RGB]{255,210,204}{\strut $\delta_{\text{top}}$}, \colorbox[RGB]{255,210,204}{\strut $\Delta_{\text{in}}$}, \colorbox[RGB]{255,210,204}{\strut $\Delta_{\text{out}}$}, \colorbox[RGB]{255,210,204}{\strut $\kappa$}, \colorbox[RGB]{255,210,204}{\strut $R_0$}, \colorbox[RGB]{255,210,204}{\strut $a$}, \colorbox[RGB]{255,210,204}{\strut $R_{\text{axis}}$}, \colorbox[RGB]{255,210,204}{\strut $Z_{\text{axis}}$}, \colorbox[RGB]{255,210,204}{\strut $V_p$} \\\hline
FNOLSTM-EM-1 & Emission, top 4. & \colorbox[RGB]{209,235,255}{\strut $\text{PD}_{\textit{CIII}}^{}$}, \colorbox[RGB]{209,235,255}{\strut $\text{PD}_{\textit{H}\alpha}^{}$}, \colorbox[RGB]{209,235,255}{\strut $\text{PD}_{\text{FFT}_{20\textit{-c}}}^{{\textit{H}\alpha}}$}, \colorbox[RGB]{209,235,255}{\strut $\text{PD}_{\text{FFT}_{50\textit{-c}}}^{{\textit{H}\alpha}}$} \\\hline
FNOLSTM-EM-2 & Emission, all. & \colorbox[RGB]{209,235,255}{\strut $\text{PD}_{\textit{CIII}}^{}$}, \colorbox[RGB]{209,235,255}{\strut $\text{PD}_{\textit{H}\alpha}^{}$}, \colorbox[RGB]{209,235,255}{\strut $\text{PD}_{\text{FFT}_{5\textit{-c}}}^{{\textit{CIII}}}$}, \colorbox[RGB]{209,235,255}{\strut $\text{PD}_{\text{FFT}_{10\textit{-c}}}^{{\textit{CIII}}}$}, \colorbox[RGB]{209,235,255}{\strut $\text{PD}_{\text{FFT}_{20\textit{-c}}}^{{\textit{CIII}}}$}, \colorbox[RGB]{209,235,255}{\strut $\text{PD}_{\text{FFT}_{50\textit{-c}}}^{{\textit{CIII}}}$}, \colorbox[RGB]{209,235,255}{\strut $\text{PD}_{\text{FFT}_{100\textit{-c}}}^{{\textit{CIII}}}$}, \colorbox[RGB]{209,235,255}{\strut $\text{PD}_{\text{FFT}_{5\textit{-c}}}^{{\textit{H}\alpha}}$}, \colorbox[RGB]{209,235,255}{\strut $\text{PD}_{\text{FFT}_{10\textit{-c}}}^{{\textit{H}\alpha}}$}, \colorbox[RGB]{209,235,255}{\strut $\text{PD}_{\text{FFT}_{20\textit{-c}}}^{{\textit{H}\alpha}}$}, \colorbox[RGB]{209,235,255}{\strut $\text{PD}_{\text{FFT}_{50\textit{-c}}}^{{\textit{H}\alpha}}$}, \colorbox[RGB]{209,235,255}{\strut $\text{PD}_{\text{FFT}_{100\textit{-c}}}^{{\textit{H}\alpha}}$} \\\hline
FNOLSTM-EM-3 & Emission, all FFT features. & \colorbox[RGB]{209,235,255}{\strut $\text{PD}_{\text{FFT}_{5\textit{-p}}}^{{\textit{CIII}}}$}, \colorbox[RGB]{209,235,255}{\strut $\text{PD}_{\text{FFT}_{5\textit{-c}}}^{{\textit{CIII}}}$}, \colorbox[RGB]{209,235,255}{\strut $\text{PD}_{\text{FFT}_{5\textit{-f}}}^{{\textit{CIII}}}$}, \colorbox[RGB]{209,235,255}{\strut $\text{PD}_{\text{FFT}_{10\textit{-p}}}^{{\textit{CIII}}}$}, \colorbox[RGB]{209,235,255}{\strut $\text{PD}_{\text{FFT}_{10\textit{-c}}}^{{\textit{CIII}}}$}, \colorbox[RGB]{209,235,255}{\strut $\text{PD}_{\text{FFT}_{10\textit{-f}}}^{{\textit{CIII}}}$}, \colorbox[RGB]{209,235,255}{\strut $\text{PD}_{\text{FFT}_{20\textit{-p}}}^{{\textit{CIII}}}$}, \colorbox[RGB]{209,235,255}{\strut $\text{PD}_{\text{FFT}_{20\textit{-c}}}^{{\textit{CIII}}}$}, \colorbox[RGB]{209,235,255}{\strut $\text{PD}_{\text{FFT}_{20\textit{-f}}}^{{\textit{CIII}}}$}, \colorbox[RGB]{209,235,255}{\strut $\text{PD}_{\text{FFT}_{50\textit{-p}}}^{{\textit{CIII}}}$}, \colorbox[RGB]{209,235,255}{\strut $\text{PD}_{\text{FFT}_{50\textit{-c}}}^{{\textit{CIII}}}$}, \colorbox[RGB]{209,235,255}{\strut $\text{PD}_{\text{FFT}_{50\textit{-f}}}^{{\textit{CIII}}}$}, \colorbox[RGB]{209,235,255}{\strut $\text{PD}_{\text{FFT}_{100\textit{-p}}}^{{\textit{CIII}}}$}, \colorbox[RGB]{209,235,255}{\strut $\text{PD}_{\text{FFT}_{100\textit{-c}}}^{{\textit{CIII}}}$}, \colorbox[RGB]{209,235,255}{\strut $\text{PD}_{\text{FFT}_{100\textit{-f}}}^{{\textit{CIII}}}$}, \colorbox[RGB]{209,235,255}{\strut $\text{PD}_{\text{FFT}_{5\textit{-p}}}^{{\textit{H}\alpha}}$}, \colorbox[RGB]{209,235,255}{\strut $\text{PD}_{\text{FFT}_{5\textit{-c}}}^{{\textit{H}\alpha}}$}, \colorbox[RGB]{209,235,255}{\strut $\text{PD}_{\text{FFT}_{5\textit{-f}}}^{{\textit{H}\alpha}}$}, \colorbox[RGB]{209,235,255}{\strut $\text{PD}_{\text{FFT}_{10\textit{-p}}}^{{\textit{H}\alpha}}$}, \colorbox[RGB]{209,235,255}{\strut $\text{PD}_{\text{FFT}_{10\textit{-c}}}^{{\textit{H}\alpha}}$}, \colorbox[RGB]{209,235,255}{\strut $\text{PD}_{\text{FFT}_{10\textit{-f}}}^{{\textit{H}\alpha}}$}, \colorbox[RGB]{209,235,255}{\strut $\text{PD}_{\text{FFT}_{20\textit{-p}}}^{{\textit{H}\alpha}}$}, \colorbox[RGB]{209,235,255}{\strut $\text{PD}_{\text{FFT}_{20\textit{-c}}}^{{\textit{H}\alpha}}$}, \colorbox[RGB]{209,235,255}{\strut $\text{PD}_{\text{FFT}_{20\textit{-f}}}^{{\textit{H}\alpha}}$}, \colorbox[RGB]{209,235,255}{\strut $\text{PD}_{\text{FFT}_{50\textit{-p}}}^{{\textit{H}\alpha}}$}, \colorbox[RGB]{209,235,255}{\strut $\text{PD}_{\text{FFT}_{50\textit{-c}}}^{{\textit{H}\alpha}}$}, \colorbox[RGB]{209,235,255}{\strut $\text{PD}_{\text{FFT}_{50\textit{-f}}}^{{\textit{H}\alpha}}$}, \colorbox[RGB]{209,235,255}{\strut $\text{PD}_{\text{FFT}_{100\textit{-p}}}^{{\textit{H}\alpha}}$}, \colorbox[RGB]{209,235,255}{\strut $\text{PD}_{\text{FFT}_{100\textit{-c}}}^{{\textit{H}\alpha}}$}, \colorbox[RGB]{209,235,255}{\strut $\text{PD}_{\text{FFT}_{100\textit{-f}}}^{{\textit{H}\alpha}}$} \\\hline
FNOLSTM-MA-1 & Magnetics, all. & \colorbox[RGB]{234,255,227}{\strut $B_0$}, \colorbox[RGB]{234,255,227}{\strut $I_{p}$}, \colorbox[RGB]{234,255,227}{\strut $I_{p,\textit{ref}}$}, \colorbox[RGB]{234,255,227}{\strut $q_{95}$} \\\hline
FNOLSTM-DE-1 & Density, top 4. & \colorbox[RGB]{252,233,255}{\strut $n_{e,\text{core}}$}, \colorbox[RGB]{252,233,255}{\strut $n_{e,\text{LFS}}$}, \colorbox[RGB]{252,233,255}{\strut $\text{max}(n'_{e,\text{edge}})$}, \colorbox[RGB]{252,233,255}{\strut $\text{max}(n''_{e,\text{edge}})$} \\\hline
FNOLSTM-DE-2 & Density, all. & \colorbox[RGB]{252,233,255}{\strut $n_{e,\text{core}}$}, \colorbox[RGB]{252,233,255}{\strut $n_{e,\text{LFS}}$}, \colorbox[RGB]{252,233,255}{\strut $n_e/n_{\textit{GW}}$}, \colorbox[RGB]{252,233,255}{\strut $\text{max}(n'_{e,\text{edge}})$}, \colorbox[RGB]{252,233,255}{\strut $\text{max}(n''_{e,\text{edge}})$}, \colorbox[RGB]{252,233,255}{\strut $n_{e,0}$} \\\hline
FNOLSTM-TE-1 & Temperature, all. & \colorbox[RGB]{255,247,196}{\strut $\textit{SXR}_{\text{core}}$}, \colorbox[RGB]{255,247,196}{\strut $\text{max}(T'_{e,\text{edge}})$}, \colorbox[RGB]{255,247,196}{\strut $\text{max}(T''_{e,\text{edge}})$}, \colorbox[RGB]{255,247,196}{\strut $T_{e,0}$} \\\hline
FNOLSTM-PO-1 & Power, top 4. & \colorbox[RGB]{255,255,234}{\strut $P_{\textit{in}}$}, \colorbox[RGB]{255,255,234}{\strut $P_{\textit{NBI}}$}, \colorbox[RGB]{255,255,234}{\strut $P_{\textit{ECRH}}$}, \colorbox[RGB]{255,255,234}{\strut $P_{\textit{LH}}$} \\\hline
FNOLSTM-PO-2 & Power, all. & \colorbox[RGB]{255,255,234}{\strut $P_{\textit{in}}$}, \colorbox[RGB]{255,255,234}{\strut $P_{\textit{OHM}}$}, \colorbox[RGB]{255,255,234}{\strut $P_{\textit{NBI}}$}, \colorbox[RGB]{255,255,234}{\strut $P_{\textit{NBI2}}$}, \colorbox[RGB]{255,255,234}{\strut $P_{\textit{ECRH}}$}, \colorbox[RGB]{255,255,234}{\strut $P_{\textit{LH}}$} \\\hline
FNOLSTM-EN-1 & Energy content, top 4. & \colorbox[RGB]{255,246,219}{\strut $\beta_{N}$}, \colorbox[RGB]{255,246,219}{\strut $\beta_{p}$}, \colorbox[RGB]{255,246,219}{\strut $\beta_{t}$}, \colorbox[RGB]{255,246,219}{\strut ${W}_{\textit{tot}}$} \\\hline
FNOLSTM-EN-2 & Energy content, all. & \colorbox[RGB]{255,246,219}{\strut $\beta_{N}$}, \colorbox[RGB]{255,246,219}{\strut $\beta_{p}$}, \colorbox[RGB]{255,246,219}{\strut $\beta_{t}$}, \colorbox[RGB]{255,246,219}{\strut ${W}_{\textit{tot}}$}, \colorbox[RGB]{255,246,219}{\strut $\text{DML}$}, \colorbox[RGB]{255,246,219}{\strut $H_{\textit{98y2}}$} \\\hline
FNOLSTM-RA-1 & Radiation, top 2. & \colorbox[RGB]{255,248,255}{\strut $P_{\textit{rad}}$}, \colorbox[RGB]{255,248,255}{\strut $P_{\textit{rad},\text{SOL}}$} \\\hline
FNOLSTM-RA-2 & Radiation, all. & \colorbox[RGB]{255,248,255}{\strut $P_{\textit{rad}}$}, \colorbox[RGB]{255,248,255}{\strut $P_{\textit{rad},\text{bulk}}$}, \colorbox[RGB]{255,248,255}{\strut $P_{\textit{rad},\text{SOL}}$} \\\hline
FNOLSTM-OT-1 & Other, all. & \colorbox[RGB]{235,245,255}{\strut $l_i$}, \colorbox[RGB]{235,245,255}{\strut $Z_{\textit{eff}}$}, \colorbox[RGB]{235,245,255}{\strut $\nu_{e,\text{ped}}^{*}$}, \colorbox[RGB]{235,245,255}{\strut $V_{\textit{loop}}$} \\\hline
FNOLSTM-$\ast\ast$-1 & Mixed, rank 1. & \colorbox[RGB]{255,210,204}{\strut $\kappa$}, \colorbox[RGB]{209,235,255}{\strut $\text{PD}_{\textit{CIII}}^{}$}, \colorbox[RGB]{234,255,227}{\strut $q_{95}$}, \colorbox[RGB]{252,233,255}{\strut $\text{max}(n''_{e,\text{edge}})$}, \colorbox[RGB]{255,247,196}{\strut $\text{max}(T'_{e,\text{edge}})$}, \colorbox[RGB]{255,255,234}{\strut $P_{\textit{in}}$}, \colorbox[RGB]{255,246,219}{\strut $\beta_{t}$}, \colorbox[RGB]{255,248,255}{\strut $P_{\textit{rad},\text{SOL}}$}, \colorbox[RGB]{235,245,255}{\strut $\nu_{e,\text{ped}}^{*}$} \\\hline
FNOLSTM-$\ast\ast$-2 & Mixed, rank 2. & \colorbox[RGB]{255,210,204}{\strut $\delta_{\text{bottom}}$}, \colorbox[RGB]{209,235,255}{\strut $\text{PD}_{\textit{H}\alpha}^{}$}, \colorbox[RGB]{234,255,227}{\strut $I_{p,\textit{ref}}$}, \colorbox[RGB]{252,233,255}{\strut $n_{e,\text{LFS}}$}, \colorbox[RGB]{255,247,196}{\strut $\text{max}(T''_{e,\text{edge}})$}, \colorbox[RGB]{255,255,234}{\strut $P_{\textit{NBI}}$}, \colorbox[RGB]{255,246,219}{\strut ${W}_{\textit{tot}}$}, \colorbox[RGB]{255,248,255}{\strut $P_{\textit{rad}}$}, \colorbox[RGB]{235,245,255}{\strut $l_i$} \\\hline
FNOLSTM-$\ast\ast$-3 & Mixed, rank 3. & \colorbox[RGB]{255,210,204}{\strut $Z_{\text{axis}}$}, \colorbox[RGB]{209,235,255}{\strut $\text{PD}_{\text{FFT}_{20\textit{-c}}}^{{\textit{H}\alpha}}$}, \colorbox[RGB]{234,255,227}{\strut $I_{p}$}, \colorbox[RGB]{252,233,255}{\strut $\text{max}(n'_{e,\text{edge}})$}, \colorbox[RGB]{255,247,196}{\strut $\textit{SXR}_{\text{core}}$}, \colorbox[RGB]{255,255,234}{\strut $P_{\textit{LH}}$}, \colorbox[RGB]{255,246,219}{\strut $\beta_{N}$}, \colorbox[RGB]{235,245,255}{\strut $V_{\textit{loop}}$} \\\hline
FNOLSTM-$\ast\ast$-4 & Mixed, top 2. & \colorbox[RGB]{255,210,204}{\strut $\delta_{\text{bottom}}$}, \colorbox[RGB]{255,210,204}{\strut $\kappa$}, \colorbox[RGB]{209,235,255}{\strut $\text{PD}_{\textit{CIII}}^{}$}, \colorbox[RGB]{209,235,255}{\strut $\text{PD}_{\textit{H}\alpha}^{}$}, \colorbox[RGB]{234,255,227}{\strut $I_{p,\textit{ref}}$}, \colorbox[RGB]{234,255,227}{\strut $q_{95}$}, \colorbox[RGB]{252,233,255}{\strut $n_{e,\text{LFS}}$}, \colorbox[RGB]{252,233,255}{\strut $\text{max}(n''_{e,\text{edge}})$}, \colorbox[RGB]{255,247,196}{\strut $\text{max}(T'_{e,\text{edge}})$}, \colorbox[RGB]{255,247,196}{\strut $\text{max}(T''_{e,\text{edge}})$}, \colorbox[RGB]{255,255,234}{\strut $P_{\textit{in}}$}, \colorbox[RGB]{255,255,234}{\strut $P_{\textit{NBI}}$}, \colorbox[RGB]{255,246,219}{\strut $\beta_{t}$}, \colorbox[RGB]{255,246,219}{\strut ${W}_{\textit{tot}}$}, \colorbox[RGB]{255,248,255}{\strut $P_{\textit{rad}}$}, \colorbox[RGB]{255,248,255}{\strut $P_{\textit{rad},\text{SOL}}$}, \colorbox[RGB]{235,245,255}{\strut $l_i$}, \colorbox[RGB]{235,245,255}{\strut $\nu_{e,\text{ped}}^{*}$} \\\hline
FNOLSTM-$\ast\ast$-5 & Mixed, top 3. & \colorbox[RGB]{255,210,204}{\strut $\delta_{\text{bottom}}$}, \colorbox[RGB]{255,210,204}{\strut $\kappa$}, \colorbox[RGB]{255,210,204}{\strut $Z_{\text{axis}}$}, \colorbox[RGB]{209,235,255}{\strut $\text{PD}_{\textit{CIII}}^{}$}, \colorbox[RGB]{209,235,255}{\strut $\text{PD}_{\textit{H}\alpha}^{}$}, \colorbox[RGB]{209,235,255}{\strut $\text{PD}_{\text{FFT}_{20\textit{-c}}}^{{\textit{H}\alpha}}$}, \colorbox[RGB]{234,255,227}{\strut $I_{p}$}, \colorbox[RGB]{234,255,227}{\strut $I_{p,\textit{ref}}$}, \colorbox[RGB]{234,255,227}{\strut $q_{95}$}, \colorbox[RGB]{252,233,255}{\strut $n_{e,\text{LFS}}$}, \colorbox[RGB]{252,233,255}{\strut $\text{max}(n'_{e,\text{edge}})$}, \colorbox[RGB]{252,233,255}{\strut $\text{max}(n''_{e,\text{edge}})$}, \colorbox[RGB]{255,247,196}{\strut $\textit{SXR}_{\text{core}}$}, \colorbox[RGB]{255,247,196}{\strut $\text{max}(T'_{e,\text{edge}})$}, \colorbox[RGB]{255,247,196}{\strut $\text{max}(T''_{e,\text{edge}})$}, \colorbox[RGB]{255,255,234}{\strut $P_{\textit{in}}$}, \colorbox[RGB]{255,255,234}{\strut $P_{\textit{NBI}}$}, \colorbox[RGB]{255,255,234}{\strut $P_{\textit{LH}}$}, \colorbox[RGB]{255,246,219}{\strut $\beta_{N}$}, \colorbox[RGB]{255,246,219}{\strut $\beta_{t}$}, \colorbox[RGB]{255,246,219}{\strut ${W}_{\textit{tot}}$}, \colorbox[RGB]{255,248,255}{\strut $P_{\textit{rad}}$}, \colorbox[RGB]{255,248,255}{\strut $P_{\textit{rad},\text{SOL}}$}, \colorbox[RGB]{235,245,255}{\strut $l_i$}, \colorbox[RGB]{235,245,255}{\strut $\nu_{e,\text{ped}}^{*}$}, \colorbox[RGB]{235,245,255}{\strut $V_{\textit{loop}}$} \\\hline
FNOLSTM-$\ast\ast$-6 & Mixed, all. & \colorbox[RGB]{255,210,204}{\strut $A_p$}, \colorbox[RGB]{255,210,204}{\strut $\delta_{\text{bottom}}$}, \colorbox[RGB]{255,210,204}{\strut $\delta_{\text{top}}$}, \colorbox[RGB]{255,210,204}{\strut $\Delta_{\text{in}}$}, \colorbox[RGB]{255,210,204}{\strut $\Delta_{\text{out}}$}, \colorbox[RGB]{255,210,204}{\strut $\kappa$}, \colorbox[RGB]{255,210,204}{\strut $R_0$}, \colorbox[RGB]{255,210,204}{\strut $a$}, \colorbox[RGB]{255,210,204}{\strut $R_{\text{axis}}$}, \colorbox[RGB]{255,210,204}{\strut $Z_{\text{axis}}$}, \colorbox[RGB]{255,210,204}{\strut $V_p$}, \colorbox[RGB]{209,235,255}{\strut $\text{PD}_{\textit{CIII}}^{}$}, \colorbox[RGB]{209,235,255}{\strut $\text{PD}_{\textit{H}\alpha}^{}$}, \colorbox[RGB]{209,235,255}{\strut $\text{PD}_{\text{FFT}_{5\textit{-c}}}^{{\textit{CIII}}}$}, \colorbox[RGB]{209,235,255}{\strut $\text{PD}_{\text{FFT}_{10\textit{-c}}}^{{\textit{CIII}}}$}, \colorbox[RGB]{209,235,255}{\strut $\text{PD}_{\text{FFT}_{20\textit{-c}}}^{{\textit{CIII}}}$}, \colorbox[RGB]{209,235,255}{\strut $\text{PD}_{\text{FFT}_{50\textit{-c}}}^{{\textit{CIII}}}$}, \colorbox[RGB]{209,235,255}{\strut $\text{PD}_{\text{FFT}_{100\textit{-c}}}^{{\textit{CIII}}}$}, \colorbox[RGB]{209,235,255}{\strut $\text{PD}_{\text{FFT}_{5\textit{-c}}}^{{\textit{H}\alpha}}$}, \colorbox[RGB]{209,235,255}{\strut $\text{PD}_{\text{FFT}_{10\textit{-c}}}^{{\textit{H}\alpha}}$}, \colorbox[RGB]{209,235,255}{\strut $\text{PD}_{\text{FFT}_{20\textit{-c}}}^{{\textit{H}\alpha}}$}, \colorbox[RGB]{209,235,255}{\strut $\text{PD}_{\text{FFT}_{50\textit{-c}}}^{{\textit{H}\alpha}}$}, \colorbox[RGB]{209,235,255}{\strut $\text{PD}_{\text{FFT}_{100\textit{-c}}}^{{\textit{H}\alpha}}$}, \colorbox[RGB]{234,255,227}{\strut $B_0$}, \colorbox[RGB]{234,255,227}{\strut $I_{p}$}, \colorbox[RGB]{234,255,227}{\strut $I_{p,\textit{ref}}$}, \colorbox[RGB]{234,255,227}{\strut $q_{95}$}, \colorbox[RGB]{252,233,255}{\strut $n_{e,\text{core}}$}, \colorbox[RGB]{252,233,255}{\strut $n_{e,\text{LFS}}$}, \colorbox[RGB]{252,233,255}{\strut $n_e/n_{\textit{GW}}$}, \colorbox[RGB]{252,233,255}{\strut $\text{max}(n'_{e,\text{edge}})$}, \colorbox[RGB]{252,233,255}{\strut $\text{max}(n''_{e,\text{edge}})$}, \colorbox[RGB]{252,233,255}{\strut $n_{e,0}$}, \colorbox[RGB]{255,247,196}{\strut $\textit{SXR}_{\text{core}}$}, \colorbox[RGB]{255,247,196}{\strut $\text{max}(T'_{e,\text{edge}})$}, \colorbox[RGB]{255,247,196}{\strut $\text{max}(T''_{e,\text{edge}})$}, \colorbox[RGB]{255,247,196}{\strut $T_{e,0}$}, \colorbox[RGB]{255,255,234}{\strut $P_{\textit{in}}$}, \colorbox[RGB]{255,255,234}{\strut $P_{\textit{OHM}}$}, \colorbox[RGB]{255,255,234}{\strut $P_{\textit{NBI}}$}, \colorbox[RGB]{255,255,234}{\strut $P_{\textit{NBI2}}$}, \colorbox[RGB]{255,255,234}{\strut $P_{\textit{ECRH}}$}, \colorbox[RGB]{255,255,234}{\strut $P_{\textit{LH}}$}, \colorbox[RGB]{255,246,219}{\strut $\beta_{N}$}, \colorbox[RGB]{255,246,219}{\strut $\beta_{p}$}, \colorbox[RGB]{255,246,219}{\strut $\beta_{t}$}, \colorbox[RGB]{255,246,219}{\strut ${W}_{\textit{tot}}$}, \colorbox[RGB]{255,246,219}{\strut $\text{DML}$}, \colorbox[RGB]{255,246,219}{\strut $H_{\textit{98y2}}$}, \colorbox[RGB]{255,248,255}{\strut $P_{\textit{rad}}$}, \colorbox[RGB]{255,248,255}{\strut $P_{\textit{rad},\text{bulk}}$}, \colorbox[RGB]{255,248,255}{\strut $P_{\textit{rad},\text{SOL}}$}, \colorbox[RGB]{235,245,255}{\strut $l_i$}, \colorbox[RGB]{235,245,255}{\strut $Z_{\textit{eff}}$}, \colorbox[RGB]{235,245,255}{\strut $\nu_{e,\text{ped}}^{*}$}, \colorbox[RGB]{235,245,255}{\strut $V_{\textit{loop}}$} \\\hline
FNOLSTM-$\ast\ast$-7 & Mixed, rank 1 (by L-mode threshold). & \colorbox[RGB]{255,210,204}{\strut $\Delta_{\text{in}}$}, \colorbox[RGB]{209,235,255}{\strut $\text{PD}_{\textit{CIII}}^{}$}, \colorbox[RGB]{234,255,227}{\strut $I_{p,\textit{ref}}$}, \colorbox[RGB]{252,233,255}{\strut $n_{e,0}$}, \colorbox[RGB]{255,247,196}{\strut $\textit{SXR}_{\text{core}}$}, \colorbox[RGB]{255,255,234}{\strut $P_{\textit{in}}$}, \colorbox[RGB]{255,246,219}{\strut $\beta_{t}$}, \colorbox[RGB]{255,248,255}{\strut $P_{\textit{rad},\text{bulk}}$}, \colorbox[RGB]{235,245,255}{\strut $V_{\textit{loop}}$} \\\hline
FNOLSTM-$\ast\ast$-8 & Mixed, rank 2 (by L-mode threshold). & \colorbox[RGB]{255,210,204}{\strut $\kappa$}, \colorbox[RGB]{209,235,255}{\strut $\text{PD}_{\textit{H}\alpha}^{}$}, \colorbox[RGB]{234,255,227}{\strut $I_{p}$}, \colorbox[RGB]{252,233,255}{\strut $\text{max}(n'_{e,\text{edge}})$}, \colorbox[RGB]{255,247,196}{\strut $\text{max}(T''_{e,\text{edge}})$}, \colorbox[RGB]{255,255,234}{\strut $P_{\textit{NBI}}$}, \colorbox[RGB]{255,246,219}{\strut ${W}_{\textit{tot}}$}, \colorbox[RGB]{255,248,255}{\strut $P_{\textit{rad}}$}, \colorbox[RGB]{235,245,255}{\strut $Z_{\textit{eff}}$} \\\hline
FNOLSTM-$\ast\ast$-9 & Mixed, top 2 (by L-mode threshold). & \colorbox[RGB]{255,210,204}{\strut $\Delta_{\text{in}}$}, \colorbox[RGB]{255,210,204}{\strut $\kappa$}, \colorbox[RGB]{209,235,255}{\strut $\text{PD}_{\textit{CIII}}^{}$}, \colorbox[RGB]{209,235,255}{\strut $\text{PD}_{\textit{H}\alpha}^{}$}, \colorbox[RGB]{234,255,227}{\strut $I_{p}$}, \colorbox[RGB]{234,255,227}{\strut $I_{p,\textit{ref}}$}, \colorbox[RGB]{252,233,255}{\strut $\text{max}(n'_{e,\text{edge}})$}, \colorbox[RGB]{252,233,255}{\strut $n_{e,0}$}, \colorbox[RGB]{255,247,196}{\strut $\textit{SXR}_{\text{core}}$}, \colorbox[RGB]{255,247,196}{\strut $\text{max}(T''_{e,\text{edge}})$}, \colorbox[RGB]{255,255,234}{\strut $P_{\textit{in}}$}, \colorbox[RGB]{255,255,234}{\strut $P_{\textit{NBI}}$}, \colorbox[RGB]{255,246,219}{\strut $\beta_{t}$}, \colorbox[RGB]{255,246,219}{\strut ${W}_{\textit{tot}}$}, \colorbox[RGB]{255,248,255}{\strut $P_{\textit{rad}}$}, \colorbox[RGB]{255,248,255}{\strut $P_{\textit{rad},\text{bulk}}$}, \colorbox[RGB]{235,245,255}{\strut $Z_{\textit{eff}}$}, \colorbox[RGB]{235,245,255}{\strut $V_{\textit{loop}}$} \\\hline
FNOLSTM-$\ast\ast$-10 & Mixed, rank 2 (by H-mode threshold). & \colorbox[RGB]{255,210,204}{\strut $\delta_{\text{bottom}}$}, \colorbox[RGB]{209,235,255}{\strut $\text{PD}_{\textit{H}\alpha}^{}$}, \colorbox[RGB]{234,255,227}{\strut $I_{p}$}, \colorbox[RGB]{252,233,255}{\strut $n_{e,0}$}, \colorbox[RGB]{255,247,196}{\strut $\text{max}(T''_{e,\text{edge}})$}, \colorbox[RGB]{255,255,234}{\strut $P_{\textit{NBI2}}$}, \colorbox[RGB]{255,246,219}{\strut ${W}_{\textit{tot}}$}, \colorbox[RGB]{235,245,255}{\strut $V_{\textit{loop}}$} \\\hline
GBDT-SH-1 & Shaping, top 4. & \colorbox[RGB]{255,210,204}{\strut $A_p$}, \colorbox[RGB]{255,210,204}{\strut $\delta_{\text{bottom}}$}, \colorbox[RGB]{255,210,204}{\strut $\kappa$}, \colorbox[RGB]{255,210,204}{\strut $Z_{\text{axis}}$} \\\hline
GBDT-SH-2 & Shaping, all. & \colorbox[RGB]{255,210,204}{\strut $A_p$}, \colorbox[RGB]{255,210,204}{\strut $\delta_{\text{bottom}}$}, \colorbox[RGB]{255,210,204}{\strut $\delta_{\text{top}}$}, \colorbox[RGB]{255,210,204}{\strut $\Delta_{\text{in}}$}, \colorbox[RGB]{255,210,204}{\strut $\Delta_{\text{out}}$}, \colorbox[RGB]{255,210,204}{\strut $\kappa$}, \colorbox[RGB]{255,210,204}{\strut $R_0$}, \colorbox[RGB]{255,210,204}{\strut $a$}, \colorbox[RGB]{255,210,204}{\strut $R_{\text{axis}}$}, \colorbox[RGB]{255,210,204}{\strut $Z_{\text{axis}}$}, \colorbox[RGB]{255,210,204}{\strut $V_p$} \\\hline
GBDT-EM-1 & Emission, top 4. & \colorbox[RGB]{209,235,255}{\strut $\text{PD}_{\text{FFT}_{10\textit{-c}}}^{{\textit{H}\alpha}}$}, \colorbox[RGB]{209,235,255}{\strut $\text{PD}_{\text{FFT}_{20\textit{-c}}}^{{\textit{H}\alpha}}$}, \colorbox[RGB]{209,235,255}{\strut $\text{PD}_{\text{FFT}_{50\textit{-c}}}^{{\textit{H}\alpha}}$}, \colorbox[RGB]{209,235,255}{\strut $\text{PD}_{\text{FFT}_{100\textit{-c}}}^{{\textit{H}\alpha}}$} \\\hline
GBDT-EM-2 & Emission, all. & \colorbox[RGB]{209,235,255}{\strut $\text{PD}_{\text{FFT}_{5\textit{-c}}}^{{\textit{CIII}}}$}, \colorbox[RGB]{209,235,255}{\strut $\text{PD}_{\text{FFT}_{10\textit{-c}}}^{{\textit{CIII}}}$}, \colorbox[RGB]{209,235,255}{\strut $\text{PD}_{\text{FFT}_{20\textit{-c}}}^{{\textit{CIII}}}$}, \colorbox[RGB]{209,235,255}{\strut $\text{PD}_{\text{FFT}_{50\textit{-c}}}^{{\textit{CIII}}}$}, \colorbox[RGB]{209,235,255}{\strut $\text{PD}_{\text{FFT}_{100\textit{-c}}}^{{\textit{CIII}}}$}, \colorbox[RGB]{209,235,255}{\strut $\text{PD}_{\text{FFT}_{5\textit{-c}}}^{{\textit{H}\alpha}}$}, \colorbox[RGB]{209,235,255}{\strut $\text{PD}_{\text{FFT}_{10\textit{-c}}}^{{\textit{H}\alpha}}$}, \colorbox[RGB]{209,235,255}{\strut $\text{PD}_{\text{FFT}_{20\textit{-c}}}^{{\textit{H}\alpha}}$}, \colorbox[RGB]{209,235,255}{\strut $\text{PD}_{\text{FFT}_{50\textit{-c}}}^{{\textit{H}\alpha}}$}, \colorbox[RGB]{209,235,255}{\strut $\text{PD}_{\text{FFT}_{100\textit{-c}}}^{{\textit{H}\alpha}}$} \\\hline
GBDT-EM-3 & Emission, all FFT features. & \colorbox[RGB]{209,235,255}{\strut $\text{PD}_{\text{FFT}_{5\textit{-p}}}^{{\textit{CIII}}}$}, \colorbox[RGB]{209,235,255}{\strut $\text{PD}_{\text{FFT}_{5\textit{-c}}}^{{\textit{CIII}}}$}, \colorbox[RGB]{209,235,255}{\strut $\text{PD}_{\text{FFT}_{5\textit{-f}}}^{{\textit{CIII}}}$}, \colorbox[RGB]{209,235,255}{\strut $\text{PD}_{\text{FFT}_{10\textit{-p}}}^{{\textit{CIII}}}$}, \colorbox[RGB]{209,235,255}{\strut $\text{PD}_{\text{FFT}_{10\textit{-c}}}^{{\textit{CIII}}}$}, \colorbox[RGB]{209,235,255}{\strut $\text{PD}_{\text{FFT}_{10\textit{-f}}}^{{\textit{CIII}}}$}, \colorbox[RGB]{209,235,255}{\strut $\text{PD}_{\text{FFT}_{20\textit{-p}}}^{{\textit{CIII}}}$}, \colorbox[RGB]{209,235,255}{\strut $\text{PD}_{\text{FFT}_{20\textit{-c}}}^{{\textit{CIII}}}$}, \colorbox[RGB]{209,235,255}{\strut $\text{PD}_{\text{FFT}_{20\textit{-f}}}^{{\textit{CIII}}}$}, \colorbox[RGB]{209,235,255}{\strut $\text{PD}_{\text{FFT}_{50\textit{-p}}}^{{\textit{CIII}}}$}, \colorbox[RGB]{209,235,255}{\strut $\text{PD}_{\text{FFT}_{50\textit{-c}}}^{{\textit{CIII}}}$}, \colorbox[RGB]{209,235,255}{\strut $\text{PD}_{\text{FFT}_{50\textit{-f}}}^{{\textit{CIII}}}$}, \colorbox[RGB]{209,235,255}{\strut $\text{PD}_{\text{FFT}_{100\textit{-p}}}^{{\textit{CIII}}}$}, \colorbox[RGB]{209,235,255}{\strut $\text{PD}_{\text{FFT}_{100\textit{-c}}}^{{\textit{CIII}}}$}, \colorbox[RGB]{209,235,255}{\strut $\text{PD}_{\text{FFT}_{100\textit{-f}}}^{{\textit{CIII}}}$}, \colorbox[RGB]{209,235,255}{\strut $\text{PD}_{\text{FFT}_{5\textit{-p}}}^{{\textit{H}\alpha}}$}, \colorbox[RGB]{209,235,255}{\strut $\text{PD}_{\text{FFT}_{5\textit{-c}}}^{{\textit{H}\alpha}}$}, \colorbox[RGB]{209,235,255}{\strut $\text{PD}_{\text{FFT}_{5\textit{-f}}}^{{\textit{H}\alpha}}$}, \colorbox[RGB]{209,235,255}{\strut $\text{PD}_{\text{FFT}_{10\textit{-p}}}^{{\textit{H}\alpha}}$}, \colorbox[RGB]{209,235,255}{\strut $\text{PD}_{\text{FFT}_{10\textit{-c}}}^{{\textit{H}\alpha}}$}, \colorbox[RGB]{209,235,255}{\strut $\text{PD}_{\text{FFT}_{10\textit{-f}}}^{{\textit{H}\alpha}}$}, \colorbox[RGB]{209,235,255}{\strut $\text{PD}_{\text{FFT}_{20\textit{-p}}}^{{\textit{H}\alpha}}$}, \colorbox[RGB]{209,235,255}{\strut $\text{PD}_{\text{FFT}_{20\textit{-c}}}^{{\textit{H}\alpha}}$}, \colorbox[RGB]{209,235,255}{\strut $\text{PD}_{\text{FFT}_{20\textit{-f}}}^{{\textit{H}\alpha}}$}, \colorbox[RGB]{209,235,255}{\strut $\text{PD}_{\text{FFT}_{50\textit{-p}}}^{{\textit{H}\alpha}}$}, \colorbox[RGB]{209,235,255}{\strut $\text{PD}_{\text{FFT}_{50\textit{-c}}}^{{\textit{H}\alpha}}$}, \colorbox[RGB]{209,235,255}{\strut $\text{PD}_{\text{FFT}_{50\textit{-f}}}^{{\textit{H}\alpha}}$}, \colorbox[RGB]{209,235,255}{\strut $\text{PD}_{\text{FFT}_{100\textit{-p}}}^{{\textit{H}\alpha}}$}, \colorbox[RGB]{209,235,255}{\strut $\text{PD}_{\text{FFT}_{100\textit{-c}}}^{{\textit{H}\alpha}}$}, \colorbox[RGB]{209,235,255}{\strut $\text{PD}_{\text{FFT}_{100\textit{-f}}}^{{\textit{H}\alpha}}$} \\\hline
GBDT-MA-1 & Magnetics, all. & \colorbox[RGB]{234,255,227}{\strut $B_0$}, \colorbox[RGB]{234,255,227}{\strut $I_{p}$}, \colorbox[RGB]{234,255,227}{\strut $I_{p,\textit{ref}}$}, \colorbox[RGB]{234,255,227}{\strut $q_{95}$} \\\hline
GBDT-DE-1 & Density, top 4. & \colorbox[RGB]{252,233,255}{\strut $n_{e,\text{core}}$}, \colorbox[RGB]{252,233,255}{\strut $n_{e,\text{LFS}}$}, \colorbox[RGB]{252,233,255}{\strut $\text{max}(n'_{e,\text{edge}})$}, \colorbox[RGB]{252,233,255}{\strut $\text{max}(n''_{e,\text{edge}})$} \\\hline
GBDT-DE-2 & Density, all. & \colorbox[RGB]{252,233,255}{\strut $n_{e,\text{core}}$}, \colorbox[RGB]{252,233,255}{\strut $n_{e,\text{LFS}}$}, \colorbox[RGB]{252,233,255}{\strut $n_e/n_{\textit{GW}}$}, \colorbox[RGB]{252,233,255}{\strut $\text{max}(n'_{e,\text{edge}})$}, \colorbox[RGB]{252,233,255}{\strut $\text{max}(n''_{e,\text{edge}})$}, \colorbox[RGB]{252,233,255}{\strut $n_{e,0}$} \\\hline
GBDT-TE-1 & Temperature, all. & \colorbox[RGB]{255,247,196}{\strut $\textit{SXR}_{\text{core}}$}, \colorbox[RGB]{255,247,196}{\strut $\text{max}(T'_{e,\text{edge}})$}, \colorbox[RGB]{255,247,196}{\strut $\text{max}(T''_{e,\text{edge}})$}, \colorbox[RGB]{255,247,196}{\strut $T_{e,0}$} \\\hline
GBDT-PO-1 & Power, top 4. & \colorbox[RGB]{255,255,234}{\strut $P_{\textit{in}}$}, \colorbox[RGB]{255,255,234}{\strut $P_{\textit{NBI}}$}, \colorbox[RGB]{255,255,234}{\strut $P_{\textit{ECRH}}$}, \colorbox[RGB]{255,255,234}{\strut $P_{\textit{LH}}$} \\\hline
GBDT-PO-2 & Power, all. & \colorbox[RGB]{255,255,234}{\strut $P_{\textit{in}}$}, \colorbox[RGB]{255,255,234}{\strut $P_{\textit{OHM}}$}, \colorbox[RGB]{255,255,234}{\strut $P_{\textit{NBI}}$}, \colorbox[RGB]{255,255,234}{\strut $P_{\textit{NBI2}}$}, \colorbox[RGB]{255,255,234}{\strut $P_{\textit{ECRH}}$}, \colorbox[RGB]{255,255,234}{\strut $P_{\textit{LH}}$} \\\hline
GBDT-EN-1 & Energy content, top 4. & \colorbox[RGB]{255,246,219}{\strut $\beta_{N}$}, \colorbox[RGB]{255,246,219}{\strut $\beta_{p}$}, \colorbox[RGB]{255,246,219}{\strut $\beta_{t}$}, \colorbox[RGB]{255,246,219}{\strut ${W}_{\textit{tot}}$} \\\hline
GBDT-EN-2 & Energy content, all. & \colorbox[RGB]{255,246,219}{\strut $\beta_{N}$}, \colorbox[RGB]{255,246,219}{\strut $\beta_{p}$}, \colorbox[RGB]{255,246,219}{\strut $\beta_{t}$}, \colorbox[RGB]{255,246,219}{\strut ${W}_{\textit{tot}}$}, \colorbox[RGB]{255,246,219}{\strut $\text{DML}$}, \colorbox[RGB]{255,246,219}{\strut $H_{\textit{98y2}}$} \\\hline
GBDT-RA-1 & Radiation, top 2. & \colorbox[RGB]{255,248,255}{\strut $P_{\textit{rad}}$}, \colorbox[RGB]{255,248,255}{\strut $P_{\textit{rad},\text{SOL}}$} \\\hline
GBDT-RA-2 & Radiation, all. & \colorbox[RGB]{255,248,255}{\strut $P_{\textit{rad}}$}, \colorbox[RGB]{255,248,255}{\strut $P_{\textit{rad},\text{bulk}}$}, \colorbox[RGB]{255,248,255}{\strut $P_{\textit{rad},\text{SOL}}$} \\\hline
GBDT-OT-1 & Other, all. & \colorbox[RGB]{235,245,255}{\strut $l_i$}, \colorbox[RGB]{235,245,255}{\strut $Z_{\textit{eff}}$}, \colorbox[RGB]{235,245,255}{\strut $\nu_{e,\text{ped}}^{*}$}, \colorbox[RGB]{235,245,255}{\strut $V_{\textit{loop}}$} \\\hline
GBDT-$\ast\ast$-1 & Mixed, rank 1. & \colorbox[RGB]{255,210,204}{\strut $\kappa$}, \colorbox[RGB]{209,235,255}{\strut $\text{PD}_{\text{FFT}_{20\textit{-c}}}^{{\textit{H}\alpha}}$}, \colorbox[RGB]{234,255,227}{\strut $q_{95}$}, \colorbox[RGB]{252,233,255}{\strut $\text{max}(n''_{e,\text{edge}})$}, \colorbox[RGB]{255,247,196}{\strut $\text{max}(T'_{e,\text{edge}})$}, \colorbox[RGB]{255,255,234}{\strut $P_{\textit{in}}$}, \colorbox[RGB]{255,246,219}{\strut $\beta_{t}$}, \colorbox[RGB]{255,248,255}{\strut $P_{\textit{rad},\text{SOL}}$}, \colorbox[RGB]{235,245,255}{\strut $\nu_{e,\text{ped}}^{*}$} \\\hline
GBDT-$\ast\ast$-2 & Mixed, rank 2. & \colorbox[RGB]{255,210,204}{\strut $\delta_{\text{bottom}}$}, \colorbox[RGB]{209,235,255}{\strut $\text{PD}_{\text{FFT}_{50\textit{-c}}}^{{\textit{H}\alpha}}$}, \colorbox[RGB]{234,255,227}{\strut $I_{p,\textit{ref}}$}, \colorbox[RGB]{252,233,255}{\strut $n_{e,\text{LFS}}$}, \colorbox[RGB]{255,247,196}{\strut $\text{max}(T''_{e,\text{edge}})$}, \colorbox[RGB]{255,255,234}{\strut $P_{\textit{NBI}}$}, \colorbox[RGB]{255,246,219}{\strut ${W}_{\textit{tot}}$}, \colorbox[RGB]{255,248,255}{\strut $P_{\textit{rad}}$}, \colorbox[RGB]{235,245,255}{\strut $l_i$} \\\hline
GBDT-$\ast\ast$-3 & Mixed, rank 3. & \colorbox[RGB]{255,210,204}{\strut $Z_{\text{axis}}$}, \colorbox[RGB]{209,235,255}{\strut $\text{PD}_{\text{FFT}_{10\textit{-c}}}^{{\textit{H}\alpha}}$}, \colorbox[RGB]{234,255,227}{\strut $I_{p}$}, \colorbox[RGB]{252,233,255}{\strut $\text{max}(n'_{e,\text{edge}})$}, \colorbox[RGB]{255,247,196}{\strut $\textit{SXR}_{\text{core}}$}, \colorbox[RGB]{255,255,234}{\strut $P_{\textit{LH}}$}, \colorbox[RGB]{255,246,219}{\strut $\beta_{N}$}, \colorbox[RGB]{235,245,255}{\strut $V_{\textit{loop}}$} \\\hline
GBDT-$\ast\ast$-4 & Mixed, top 2. & \colorbox[RGB]{255,210,204}{\strut $\delta_{\text{bottom}}$}, \colorbox[RGB]{255,210,204}{\strut $\kappa$}, \colorbox[RGB]{209,235,255}{\strut $\text{PD}_{\text{FFT}_{20\textit{-c}}}^{{\textit{H}\alpha}}$}, \colorbox[RGB]{209,235,255}{\strut $\text{PD}_{\text{FFT}_{50\textit{-c}}}^{{\textit{H}\alpha}}$}, \colorbox[RGB]{234,255,227}{\strut $I_{p,\textit{ref}}$}, \colorbox[RGB]{234,255,227}{\strut $q_{95}$}, \colorbox[RGB]{252,233,255}{\strut $n_{e,\text{LFS}}$}, \colorbox[RGB]{252,233,255}{\strut $\text{max}(n''_{e,\text{edge}})$}, \colorbox[RGB]{255,247,196}{\strut $\text{max}(T'_{e,\text{edge}})$}, \colorbox[RGB]{255,247,196}{\strut $\text{max}(T''_{e,\text{edge}})$}, \colorbox[RGB]{255,255,234}{\strut $P_{\textit{in}}$}, \colorbox[RGB]{255,255,234}{\strut $P_{\textit{NBI}}$}, \colorbox[RGB]{255,246,219}{\strut $\beta_{t}$}, \colorbox[RGB]{255,246,219}{\strut ${W}_{\textit{tot}}$}, \colorbox[RGB]{255,248,255}{\strut $P_{\textit{rad}}$}, \colorbox[RGB]{255,248,255}{\strut $P_{\textit{rad},\text{SOL}}$}, \colorbox[RGB]{235,245,255}{\strut $l_i$}, \colorbox[RGB]{235,245,255}{\strut $\nu_{e,\text{ped}}^{*}$} \\\hline
GBDT-$\ast\ast$-5 & Mixed, top 3. & \colorbox[RGB]{255,210,204}{\strut $\delta_{\text{bottom}}$}, \colorbox[RGB]{255,210,204}{\strut $\kappa$}, \colorbox[RGB]{255,210,204}{\strut $Z_{\text{axis}}$}, \colorbox[RGB]{209,235,255}{\strut $\text{PD}_{\text{FFT}_{10\textit{-c}}}^{{\textit{H}\alpha}}$}, \colorbox[RGB]{209,235,255}{\strut $\text{PD}_{\text{FFT}_{20\textit{-c}}}^{{\textit{H}\alpha}}$}, \colorbox[RGB]{209,235,255}{\strut $\text{PD}_{\text{FFT}_{50\textit{-c}}}^{{\textit{H}\alpha}}$}, \colorbox[RGB]{234,255,227}{\strut $I_{p}$}, \colorbox[RGB]{234,255,227}{\strut $I_{p,\textit{ref}}$}, \colorbox[RGB]{234,255,227}{\strut $q_{95}$}, \colorbox[RGB]{252,233,255}{\strut $n_{e,\text{LFS}}$}, \colorbox[RGB]{252,233,255}{\strut $\text{max}(n'_{e,\text{edge}})$}, \colorbox[RGB]{252,233,255}{\strut $\text{max}(n''_{e,\text{edge}})$}, \colorbox[RGB]{255,247,196}{\strut $\textit{SXR}_{\text{core}}$}, \colorbox[RGB]{255,247,196}{\strut $\text{max}(T'_{e,\text{edge}})$}, \colorbox[RGB]{255,247,196}{\strut $\text{max}(T''_{e,\text{edge}})$}, \colorbox[RGB]{255,255,234}{\strut $P_{\textit{in}}$}, \colorbox[RGB]{255,255,234}{\strut $P_{\textit{NBI}}$}, \colorbox[RGB]{255,255,234}{\strut $P_{\textit{LH}}$}, \colorbox[RGB]{255,246,219}{\strut $\beta_{N}$}, \colorbox[RGB]{255,246,219}{\strut $\beta_{t}$}, \colorbox[RGB]{255,246,219}{\strut ${W}_{\textit{tot}}$}, \colorbox[RGB]{255,248,255}{\strut $P_{\textit{rad}}$}, \colorbox[RGB]{255,248,255}{\strut $P_{\textit{rad},\text{SOL}}$}, \colorbox[RGB]{235,245,255}{\strut $l_i$}, \colorbox[RGB]{235,245,255}{\strut $\nu_{e,\text{ped}}^{*}$}, \colorbox[RGB]{235,245,255}{\strut $V_{\textit{loop}}$} \\\hline
GBDT-$\ast\ast$-6 & Mixed, all. & \colorbox[RGB]{255,210,204}{\strut $A_p$}, \colorbox[RGB]{255,210,204}{\strut $\delta_{\text{bottom}}$}, \colorbox[RGB]{255,210,204}{\strut $\delta_{\text{top}}$}, \colorbox[RGB]{255,210,204}{\strut $\Delta_{\text{in}}$}, \colorbox[RGB]{255,210,204}{\strut $\Delta_{\text{out}}$}, \colorbox[RGB]{255,210,204}{\strut $\kappa$}, \colorbox[RGB]{255,210,204}{\strut $R_0$}, \colorbox[RGB]{255,210,204}{\strut $a$}, \colorbox[RGB]{255,210,204}{\strut $R_{\text{axis}}$}, \colorbox[RGB]{255,210,204}{\strut $Z_{\text{axis}}$}, \colorbox[RGB]{255,210,204}{\strut $V_p$}, \colorbox[RGB]{209,235,255}{\strut $\text{PD}_{\text{FFT}_{5\textit{-c}}}^{{\textit{CIII}}}$}, \colorbox[RGB]{209,235,255}{\strut $\text{PD}_{\text{FFT}_{10\textit{-c}}}^{{\textit{CIII}}}$}, \colorbox[RGB]{209,235,255}{\strut $\text{PD}_{\text{FFT}_{20\textit{-c}}}^{{\textit{CIII}}}$}, \colorbox[RGB]{209,235,255}{\strut $\text{PD}_{\text{FFT}_{50\textit{-c}}}^{{\textit{CIII}}}$}, \colorbox[RGB]{209,235,255}{\strut $\text{PD}_{\text{FFT}_{100\textit{-c}}}^{{\textit{CIII}}}$}, \colorbox[RGB]{209,235,255}{\strut $\text{PD}_{\text{FFT}_{5\textit{-c}}}^{{\textit{H}\alpha}}$}, \colorbox[RGB]{209,235,255}{\strut $\text{PD}_{\text{FFT}_{10\textit{-c}}}^{{\textit{H}\alpha}}$}, \colorbox[RGB]{209,235,255}{\strut $\text{PD}_{\text{FFT}_{20\textit{-c}}}^{{\textit{H}\alpha}}$}, \colorbox[RGB]{209,235,255}{\strut $\text{PD}_{\text{FFT}_{50\textit{-c}}}^{{\textit{H}\alpha}}$}, \colorbox[RGB]{209,235,255}{\strut $\text{PD}_{\text{FFT}_{100\textit{-c}}}^{{\textit{H}\alpha}}$}, \colorbox[RGB]{234,255,227}{\strut $B_0$}, \colorbox[RGB]{234,255,227}{\strut $I_{p}$}, \colorbox[RGB]{234,255,227}{\strut $I_{p,\textit{ref}}$}, \colorbox[RGB]{234,255,227}{\strut $q_{95}$}, \colorbox[RGB]{252,233,255}{\strut $n_{e,\text{core}}$}, \colorbox[RGB]{252,233,255}{\strut $n_{e,\text{LFS}}$}, \colorbox[RGB]{252,233,255}{\strut $n_e/n_{\textit{GW}}$}, \colorbox[RGB]{252,233,255}{\strut $\text{max}(n'_{e,\text{edge}})$}, \colorbox[RGB]{252,233,255}{\strut $\text{max}(n''_{e,\text{edge}})$}, \colorbox[RGB]{252,233,255}{\strut $n_{e,0}$}, \colorbox[RGB]{255,247,196}{\strut $\textit{SXR}_{\text{core}}$}, \colorbox[RGB]{255,247,196}{\strut $\text{max}(T'_{e,\text{edge}})$}, \colorbox[RGB]{255,247,196}{\strut $\text{max}(T''_{e,\text{edge}})$}, \colorbox[RGB]{255,247,196}{\strut $T_{e,0}$}, \colorbox[RGB]{255,255,234}{\strut $P_{\textit{in}}$}, \colorbox[RGB]{255,255,234}{\strut $P_{\textit{OHM}}$}, \colorbox[RGB]{255,255,234}{\strut $P_{\textit{NBI}}$}, \colorbox[RGB]{255,255,234}{\strut $P_{\textit{NBI2}}$}, \colorbox[RGB]{255,255,234}{\strut $P_{\textit{ECRH}}$}, \colorbox[RGB]{255,255,234}{\strut $P_{\textit{LH}}$}, \colorbox[RGB]{255,246,219}{\strut $\beta_{N}$}, \colorbox[RGB]{255,246,219}{\strut $\beta_{p}$}, \colorbox[RGB]{255,246,219}{\strut $\beta_{t}$}, \colorbox[RGB]{255,246,219}{\strut ${W}_{\textit{tot}}$}, \colorbox[RGB]{255,246,219}{\strut $\text{DML}$}, \colorbox[RGB]{255,246,219}{\strut $H_{\textit{98y2}}$}, \colorbox[RGB]{255,248,255}{\strut $P_{\textit{rad}}$}, \colorbox[RGB]{255,248,255}{\strut $P_{\textit{rad},\text{bulk}}$}, \colorbox[RGB]{255,248,255}{\strut $P_{\textit{rad},\text{SOL}}$}, \colorbox[RGB]{235,245,255}{\strut $l_i$}, \colorbox[RGB]{235,245,255}{\strut $Z_{\textit{eff}}$}, \colorbox[RGB]{235,245,255}{\strut $\nu_{e,\text{ped}}^{*}$}, \colorbox[RGB]{235,245,255}{\strut $V_{\textit{loop}}$} \\\hline
GBDT-$\ast\ast$-7 & Mixed, rank 1 (by L-mode threshold). & \colorbox[RGB]{255,210,204}{\strut $\Delta_{\text{in}}$}, \colorbox[RGB]{209,235,255}{\strut $\text{PD}_{\text{FFT}_{20\textit{-c}}}^{{\textit{H}\alpha}}$}, \colorbox[RGB]{234,255,227}{\strut $I_{p,\textit{ref}}$}, \colorbox[RGB]{252,233,255}{\strut $n_{e,0}$}, \colorbox[RGB]{255,247,196}{\strut $\textit{SXR}_{\text{core}}$}, \colorbox[RGB]{255,255,234}{\strut $P_{\textit{in}}$}, \colorbox[RGB]{255,246,219}{\strut $\beta_{t}$}, \colorbox[RGB]{255,248,255}{\strut $P_{\textit{rad},\text{bulk}}$}, \colorbox[RGB]{235,245,255}{\strut $V_{\textit{loop}}$} \\\hline
GBDT-$\ast\ast$-8 & Mixed, rank 2 (by L-mode threshold). & \colorbox[RGB]{255,210,204}{\strut $\kappa$}, \colorbox[RGB]{209,235,255}{\strut $\text{PD}_{\text{FFT}_{50\textit{-c}}}^{{\textit{H}\alpha}}$}, \colorbox[RGB]{234,255,227}{\strut $I_{p}$}, \colorbox[RGB]{252,233,255}{\strut $\text{max}(n'_{e,\text{edge}})$}, \colorbox[RGB]{255,247,196}{\strut $\text{max}(T''_{e,\text{edge}})$}, \colorbox[RGB]{255,255,234}{\strut $P_{\textit{NBI}}$}, \colorbox[RGB]{255,246,219}{\strut ${W}_{\textit{tot}}$}, \colorbox[RGB]{255,248,255}{\strut $P_{\textit{rad}}$}, \colorbox[RGB]{235,245,255}{\strut $Z_{\textit{eff}}$} \\\hline
GBDT-$\ast\ast$-9 & Mixed, top 2 (by L-mode threshold). & \colorbox[RGB]{255,210,204}{\strut $\Delta_{\text{in}}$}, \colorbox[RGB]{255,210,204}{\strut $\kappa$}, \colorbox[RGB]{209,235,255}{\strut $\text{PD}_{\text{FFT}_{20\textit{-c}}}^{{\textit{H}\alpha}}$}, \colorbox[RGB]{209,235,255}{\strut $\text{PD}_{\text{FFT}_{50\textit{-c}}}^{{\textit{H}\alpha}}$}, \colorbox[RGB]{234,255,227}{\strut $I_{p}$}, \colorbox[RGB]{234,255,227}{\strut $I_{p,\textit{ref}}$}, \colorbox[RGB]{252,233,255}{\strut $\text{max}(n'_{e,\text{edge}})$}, \colorbox[RGB]{252,233,255}{\strut $n_{e,0}$}, \colorbox[RGB]{255,247,196}{\strut $\textit{SXR}_{\text{core}}$}, \colorbox[RGB]{255,247,196}{\strut $\text{max}(T''_{e,\text{edge}})$}, \colorbox[RGB]{255,255,234}{\strut $P_{\textit{in}}$}, \colorbox[RGB]{255,255,234}{\strut $P_{\textit{NBI}}$}, \colorbox[RGB]{255,246,219}{\strut $\beta_{t}$}, \colorbox[RGB]{255,246,219}{\strut ${W}_{\textit{tot}}$}, \colorbox[RGB]{255,248,255}{\strut $P_{\textit{rad}}$}, \colorbox[RGB]{255,248,255}{\strut $P_{\textit{rad},\text{bulk}}$}, \colorbox[RGB]{235,245,255}{\strut $Z_{\textit{eff}}$}, \colorbox[RGB]{235,245,255}{\strut $V_{\textit{loop}}$} \\\hline
GBDT-$\ast\ast$-10 & Mixed, rank 2 (by H-mode threshold). & \colorbox[RGB]{255,210,204}{\strut $\delta_{\text{bottom}}$}, \colorbox[RGB]{209,235,255}{\strut $\text{PD}_{\text{FFT}_{50\textit{-c}}}^{{\textit{H}\alpha}}$}, \colorbox[RGB]{234,255,227}{\strut $I_{p}$}, \colorbox[RGB]{252,233,255}{\strut $n_{e,0}$}, \colorbox[RGB]{255,247,196}{\strut $\text{max}(T''_{e,\text{edge}})$}, \colorbox[RGB]{255,255,234}{\strut $P_{\textit{NBI2}}$}, \colorbox[RGB]{255,246,219}{\strut ${W}_{\textit{tot}}$}, \colorbox[RGB]{235,245,255}{\strut $V_{\textit{loop}}$}

\\
\caption{All \textit{(model + feature set)} configurations, with features colored by category. The order is based on Cohen's kappa coefficient when using the individual feature to predict the confinement state using a shallow decision tree, unless specified otherwise. The feature sets between the two models are identical except for the emission-related features, where only the FNOLSTM-based models use the raw photodiode signals. These signals are not very informative when considering the absolute value because they are not absolutely calibrated, however, confinement state-related patterns are clearly visible in their dynamics. For this reason, the `raw' values are selected only for the dynamic models, since these models can detect the temporal patterns.}\label{tab:model_featuresets}
\end{longtable}
\endgroup
\setlength\tabcolsep{6pt}
\arrayrulecolor{black}



\newpage
