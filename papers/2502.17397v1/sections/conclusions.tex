We have presented a method for the robust classification of confinement states whilst providing meaningful confidence estimates. The method is based on a hierarchical ensemble, combining different types of models and sets of input features on the top level. On the second level, these model/feature combinations consist of a mini-ensemble trained on different data-splits, which are averaged and empirically calibrated such that we can meaningfully combine them on the top level.

We have evaluated the approach quantitatively and qualitatively on a variety of scenarios. In most cases, the full ensemble provided accurate predictions and meaningful uncertainty estimates. However, especially in transient cases, ensembles of FNOLSTM-only models tended to perform slightly better. This advantage came at the cost of more sensitivity to corrupted input features. 

The approach gives the flexibility of choosing different components given the desired use-case. For example, if one wants to robustly identify main phases for large datasets, the full ensemble is more suitable given its added robustness. If one is specifically interested in back-and-forth transitions the FNOLSTM-ensemble is more suitable. Additionally, if one is aware of errors in certain diagnostics, one could disable the subset of models using this signal as an input feature.

The main weakness of the method lies in estimating the precise time of transition. While it robustly identifies main phases of various discharges and generally finds all the main transitions, the accuracy drops in small time windows ($<$\SI{5}{\milli\second}) around transitions or around fast transients. Future efforts focusing specifically on the transition time, rather than a general purpose classifier, are of interest to address this weakness.

\subsection{Future Work}
A potential avenue to increasing performance around transition regions would be to reformulate the problem to detecting the time of a transition, akin to change point detection methods~\cite{aminikhanghahi2016}, rather than labeling every timestep. This type of model could be used in a cascaded setting with the current approach: first, we robustly detect the main phases for the different confinement states. Then, a specialized model refines the prediction around the time of transition. Additionally, one could explore a wider set of neural network architectures to improve performance, e.g. for the local pattern extractor~\cite{ho2020,Liu2021ICCV,kovachki2021neural} or for the long term correlations~\cite{vaswani2017,beck2024xlstm,gu2024mamba}.

Another interesting avenue for future research is multi-device confinement state classification. Especially in light of future devices, one will not have access to a large set of discharges in order to create a dataset. Additionally, even if the experiments are available, accurate labeling of the confinement states is a time-consuming effort. Initial efforts in this direction have been made~\cite{marceca2021}, however, at a significant drop in performance: the fundamental differences in timescales and dynamics prove a significant challenge. Potential approaches to tackle these issues consider transfer learning~\cite{zhuang2021transfer}, physics-based normalization for input signals or device-invariant model architectures. Notably, there has been significant progress on cross-machine data-driven models for disruption prediction, for example when training on one device and evaluating on another~\cite{katesharbeck2019} or by adding only a limited number of shots from a new device~\cite{zhu021disr,zheng2023disr}.




A real-time version of the proposed method is of interest for control applications. In principle only minor adaptations would have to be made. The FNOLSTM architecture is structurally similar to a prior NN-based confinement state classifier~\cite{matoslhd2020}, which is already implemented in the TCV control system~\cite{marceca2021,galperti2024}. The computational cost of the tree ensemble method is also real-time compatible. All individual ensemble components can run in parallel, with the final ensembling procedure requiring negligible computation time: there should be no latency bottlenecks. The prediction lag parameter in the current models is at most \SI{10}{\milli\second}, although one could choose to lower this parameter at the cost of some precision. Input-wise, one would have to restrict the input set to real-time available signals. Finally, to ensure parity between training and real-time use, one should take care to use causal interpolation methods rather than the linear interpolation used in this paper.

Another interesting direction would be to reformulate the output to a figure of merit for confinement performance, rather than a discrete state label. For example in the real-time setting, one could directly optimize this quantity with model-based control techniques. In a similar vein, one could extend the approach to differentiate different types of H-mode, e.g. distinguishing ELMy and ELM-free regimes; see~\cite{zorek2022,gill2024} for related works towards this setting.

More generally, the ensembling strategy could be applied to problems of similar structure, such as disruption prediction~\cite{katesharbeck2019}. The joint integration of robustness to signal issues and uncertainty quantification makes it a good candidate for real-time prediction strategies, where reliability and interpretability are crucial for integration in disruption avoidance control schemes~\cite{galperti2024}.
