\section{Infrared-Absorbing and -Translucent Inkjet Inks}
\label{inks}
%\hl{<We should add a general subsection about the IR ink (general introduction to the ink)>}



One key component in our system is the set of inks used in our printed samples.
For our implementation, we used an off-the-shelf desktop inkjet printer (\textit{Epson ST-200}), \new{matte paper from \textit{Canon}\footnote{\new{\url{https://www.amazon.com/Canon-7981A004-Photo-Paper-Sheets/dp/B0000721Z3/}}}}, and four types of inkjet-compatible inks (C, M, Y, IR) by \textit{VersaCheck}\footnote{\url{https://www.versacheck.com/}}.
The desktop printer has four refillable tanks, which we leveraged for visible and invisible content printing. The cyan, magenta, yellow (CMY) channels of the printer were filled with the infrared-translucent C, M, Y inks, and the $K$ (i.e., conventionally black) channel was filled with the IR ink (NIR, $78~ml$ bottle). This ink passes light at visible wavelengths, and thus appears mostly transparent to the naked eye, but has high absorbance intensity in the near-infrared (NIR) range, with the peak absorbance wavelength at $850~nm$. Thus, it is captured as black under an infrared camera lens. Alternately, users can leverage the flexibility of handheld printers \cite{Pourjafarian2022Print} such as the \textit{Kongten MBrush}\footnote{\url{https://www.amazon.de/-/en/EVEBOT-Kongten-Mbrush-Printer-Android/dp/B0C84P61WH?th=1}} to embed watermarks on demand, only requiring IR ink. However, we used the desktop printer for our investigations of human and machine perception as it allowed us to produce very consistent and reproducible results.


As the ink loaded into the "black" channel is non-black (for human sight), we instead generate the black visible content by mixing the C, M, and Y inks, as their combination results in black color.
\new{We achieve this elimination of the black channel requirement by converting the original RGB color space of the visual part of the document to CMY (i.e., as opposed to CMYK) by setting the $K$ channel to zero through Gray Component Replacement (GCR)~\cite{kang_gray_1994} via \textit{Adobe Photoshop}. We note that the resulting black color may slightly visually differ from off-the-shelf black ink.}
% and adjusting the four channels such that the black channel is blank.
%before sending it to the printer.
The visible color content of the document is printed this way initially, then the sheet is fed back to overlay the invisible content with the NIR ink. The watermark's content is then printed using the $K$ channel only.



\begin{figure*}[]
  \centering
  \includegraphics[width=0.9\linewidth]{figures-new/Fig2_User_study.png}
  \caption{User study samples.}
  \Description{.}
  \label{fig:userStudyFigure}
\end{figure*}



\subsection{Absorbance Analysis of Inks}
\label{spectrum}
We conducted a spectral analysis on the four types of inks we use in our system (infrared-translucent C, M, Y, and infrared-absorbing IR)  with the help of a spectrophotometer, the \textit{Perkin Elmer Lambda 1050 UV/VIS/NIR}\footnote{\url{https://www.perkinelmer.com/product/lambda-1050-2d-base-inst-no-sw-l6020055}}.
In addition, we also measured the properties of \textit{conventional} CMY inks, so we can compare it them to our \textit{infrared-translucent} CMY inks. Our results are shown in \autoref{fig:inkSpectrumFigure}.
The x-axis represents wavelength of light measured in nanometers ($nm$), and its range $380$-$1100~nm$ was selected to illustrate the behaviour under visible and NIR light conditions.
The y-axis is log(1/R), which indicates the absorbance rate calculated via the reflection (R) measured by the device.



Within the human visibility range (from $380~nm$ to $750~nm$), both regular and infrared-translucent CMY inks are more discernible to the human eye compared to the IR ink, which registers close to zero visibility and is therefore harder for humans to perceive. However, beyond $750~nm$, the absorbance of conventional CMY inks drop to around $0.1$, while those for infrared-translucent CMY inks drop closer to $0$. At around $800~nm$, the absorbance of the IR ink surpasses that of the three infrared-translucent CMY inks, as shown in the callout of \autoref{fig:inkSpectrumFigure}. 
This means that starting at this wavelength, the IR ink would look "dark" (i.e., absorbing) to an IR camera in comparison to translucent CMY inks.
Notably, at $850~nm$, the difference between the IR ink and infrared-translucent CMY inks is the highest.
The larger this difference, the easier it is to distinguish the watermark from the the CMY content under an IR camera.
Thus, at this wavelength, one could more reliably decode the data printed with IR ink, as the CMY content would look most dispersed.
Thus, we use IR ink in combination with these IR-translucent CMY inks.
In comparison, conventional CMY inks are even "darker" than IR ink at this wavelength, meaning they would block the IR content from being detected if they were used for printing the documents instead.

Consequently, IR cameras should be calibrated to image at $850~nm$ with the background printed using our infrared-translucent CMY inks to enhance the detectability of the IR ink.




\vspace{0.3cm}





\section{Experiment: Detectability of IR Ink}
\label{visibility-experiment}



One challenge with IR ink is that, even though it allows for invisible pattern embedding, it may become visible to the naked eye when its concentration is too high. This results from a difference in intensity, that is, a contrast between the IR-inked parts and the rest of the paper region. Consequently, it can potentially affect the readability and aesthetics of the print. Our central interest lies in using the highest concentration of IK ink to improve the detection rate while remaining invisible to human eyes. \new{The experiment was conducted indoors under normal office lighting conditions (i.e., well-lit, about 300 lux and a color temperature of 3500K–-5500K)}.  



To investigate this, we conducted a psychophysical detection threshold experiment on the detectability of printed IR ink at different background colors. We included colors with varying RGB values, darkness ($K$)~\cite{li_black_2013} and of different luminescence~\cite{ridpath_techniques_2000} to cover a large color spectrum and to understand how these color properties affect the detectability. This decision was motivated through multiple feedback loops and informal pilot testing, suggesting, e.g., that darker colors mask the IR ink, whereas IR ink on brighter backgrounds can be more easily noticed by humans. To improve the quality and robustness of the data collection, we ran the experiment in the lab under controlled lighting conditions. 

We formulated the following two hypotheses in our experiment:

\textbf{(H.1)} Inkjet printed IR ink can remain invisible to human eyes, regardless of background color.

\textbf{(H.2)} The maximum density of IR ink that remains invisible depends on the properties of the background color.


\subsection{Method}
To determine estimates for detection thresholds, we used the psychophysical methods of limits \cite{kingdom_chapter_2016}, exposing participants to ascending stimuli, that is, increasing the concentration of IR ink, until they noticed it. Participants were presented with two color bars, one with IR ink and the other without IR ink, and we asked them to compare both color bars and report if they noticed a difference between them using a 2-alternative-forced-choice (2AFC) method. As depicted in \autoref{fig:userStudyFigure}, we used a gradient pattern of IR ink, starting with no IR ink 0\% (left) to 200\% (right) density. By using 0\% as the starting stimuli, we ensure that participants start with a level of IR ink that cannot be noticed \cite{leek_adaptive_2001}. The higher IR ink densities of 101\% to 200\% require two successive printing procedures. Given the novelty of our approach, we determined a reasonable range of IR ink densities through pilot testing. 

\paragraph{Task.} Participants moved a thick white paper card (depicted in \autoref{fig:task}) from the left (starting stimuli) to the right. We asked participants to move the card slowly and evenly, stopping and reporting immediately when they detected the IR ink. The experimenter recorded the position at which they stopped using the ruler printed at the top of the page (see \autoref{fig:userStudyFigure}), which was not visible to the participants. Consequently, participants continued with the next row until they completed all rows. Only one row was visible to the participants at a time, and everything else on the page was covered. Participants were informed about the procedure, and we explicitly showed them IR patterns multiple times during the warm-up phase to ensure that they were able to detect it. We asked them to report the IR ink as soon as they noticed it, thus targeting the most conservative case. Participants took breaks whenever needed. 


\begin{figure*}[h]
  \centering
  \includegraphics[width=\linewidth]{figures-new/IR_ink_thresholds.pdf}
  \caption{IR ink DTs for each color tested in our experiment, reaching from a mean DT of 81\% all the way to 200\% IR ink density. \new{The three classes are provided based on participants' ability to detect the IR ink.}}
  \Description{.}
  \label{fig:thresholds}
\end{figure*}




\subsubsection{Background Color Selection \& Preparation}
Our pilot studies suggested that the detectability or IR ink likely depends on the background color. Therefore, we included colors with varying RGB values, darkness ($K$) and of different luminescence to obtain a diverse set of colors. To cover a wide range of colors with different levels of darkness and luminescence, we first divided the darkness (calculated as an analog K value from RGB) into four equal 25\% intervals, ranging from 0\% to 100\%.
We then calculated the range of luminescence covered within each interval. As the darkness of a color increases, the luminescence range narrows, meaning the colors become darker. Therefore, we allocated four colors to each $\sim$25\% luminescence interval, starting with the lightest \new{colors in} Group 1, where darkness is between 0--25\%. Group 1 and Group 5 include the two extremes, no background color and black. Finally, we added four additional color hues to cover a large spectrum of real-world use cases and scenarios. 
\new{A detailed list of the colors tested in our experiment is provided in Table \ref{tab:color_coverage} in Appendix \ref{apedix: selection}}.





Finally, we printed the background colors with the IR ink gradient as described in \autoref{inks}. The size of the color bars was chosen to give participants a large enough area to visually inspect the color. As a result, we randomly split the 45 samples into groups of eight, resulting in six pages depicted in \autoref{fig:userStudyFigure}a.


\subsubsection{Design.}
We used a design within subjects in which each participant successively inspected each background color bar on all six pages. We fully counterbalanced the order of the six pages using a Latin square ($n = 6$).


\subsubsection{Participants.}
We recruited 18 right-handed participants (six females, twelves males), aged 18--31 (mean: 25.05; SD: 3.05) from the general public and the local university. Participants had a range of different educational and professional backgrounds, including computer science, education, cybersecurity, mechanical and electrical engineering, data science, and artificial intelligence. All participants reported normal/corrected-to-normal vision and did not report any known health issues which might impair their perception. Participants received a  \textdollar 10 Amazon voucher for taking part in the experiment. The study was approved by the MIT's Ethics Board.

\subsubsection{Experimental Protocol.}
Following a general introduction to the experiment, participants completed a demographic questionnaire. We then explained the task by showing them a set of prepared samples with background colors different from those used in the study. Next, we demonstrate how participants should explore the samples, i.e., moving the card from left to right with a slow and consistent speed. \new{They were instructed to sit comfortably with an approximate reading distance of 30--40~cm which mimics a typical real-world reading distance. We decided against fixating participants' heads, e.g., with a chin rest, to collect realistic real-world data}. The experimenter monitored that they stayed within a reasonable distance, i.e., not going too close to the sample. Once the participant indicated that there was a color difference, the experimenter obtained the corresponding IR ink value, which remained invisible to the participant. If a participant reached the end of the color bar and did not report a difference, the maximum IR ink value of 200\% was used as a response. Participants were given unlimited time to practice to ensure that they understood everything. They were encouraged to ask questions in case something was unclear. In total, the experiment took 35--45 minutes.

\begin{figure}[]
  \centering
  \includegraphics[width=0.45\textwidth]{figures-new/Task.png}
  \caption{Study procedure. The participant slides the card steadily to the right to reveal the color bar. When participants noticed a color difference between the top (a) and the bottom (b) of the color bar, they were told to report it.}
  \Description{.}
  \label{fig:task}
\end{figure}



\begin{figure*}[t]
  \centering
  \includegraphics[width=1\linewidth]{figures-new/FigureColorPrediction.png}
  \caption{IR ink computation. (a) We attempt to mathematical model the limits of human perception to conceal IR ink. (b) We use polynomial fitting to estimate the IR capacity based on $lum$ and $K$. (c) The class limits were determined on the histogram of IR ink percentage.}
  \Description{.}
  \label{fig:IRInkComputation}
\end{figure*}



\subsubsection{Data Collection \& Analysis.}
We collected data from three sources: a pre-study demographic questionnaire; observations and field notes, and participants' responses to the 2AFC method. For each of the 45 background colors, we received 18 responses, resulting in a total of 810 data points. First, we performed outlier removal using the Box-plot method. Next, we statistically analyzed our data after verifying the parametric test assumptions at $\alpha$ = .05 using an RM ANOVA. In the presence of a main effect, we performed post hoc pairwise comparison t-tests adjusted using the Bonferroni-Holm \textit{p}-value adjustments.




\subsection{Results: Detectability of IR Ink}

\subsubsection{Printed IR Ink Can Remain Invisible to the Human Eye}
We compute an overall DT for each background color by averaging individual DTs of participants. The results can be seen in \autoref{fig:thresholds}, showing that for each background color, we obtained a DT greater than zero, reaching from DTs of 81\% (no background color) all the way to 200\% (black background color). This confirms \textbf{H.1} showing that the inkjet-printed IR content can remain invisible to the human eye. We also observed that the darker colors especially seem to suffer from high variance due to a wide range of participant responses. This may result from two reasons: (1) high-density IR ink has a ``greenish/grayish" appearance when printed. Background colors with the highest variance also show this color tone (see \autoref{fig:thresholds}) and, therefore, may be mistakenly perceived as IR ink. (2) we examined the high-variance samples after the experiment, and even though we fabricated high-quality prints, we noticed slight impurities. Given the nature of the task, i.e., we asked participants to fully focus on the sample and report a color difference as soon as they noticed it, we believe this might have pushed people to pick up on this, leading to even more conservative IR ink estimates. 




\subsubsection{Maximum IR Ink Density Depends on Background Color}
\autoref{fig:thresholds} suggests that the invisible IR ink density depends on the background color. This is confirmed by our statistical analysis, which reveals the main effect of the background color on DT ($F(44) = 27.369$, $p < .001$, $\eta_{p}^{2} = .631$). 481/990 post-hoc pairwise comparisons were statistically significant after \textit{p}-adjustments, demonstrating that background color has a significant effect on the DTs in terms of the invisible IR ink density. Thus, we can also confirm \textbf{H.2}, that the maximum IR ink density depends on the background color.




\subsection{Predict Invisible IR Ink Density}
To facilitate the design process of watermark embeddings using invisible IR ink, we developed a system that estimates IR ink density for a given RGB color input (see \autoref{fig:IRInkComputation}). First, we performed two correlation analyses for $K$ and $lum$ for our 45 background colors on the amount of invisible IR ink to understand their influence. We found that $K$ shows a strong negative correlation ($\rho(43) = -.678$, $p = .012$) and $lum$ a strong positive correlation ($\rho(43) = .775$, $p < .001$). This means that increasing $K$ and decreasing $lum$ result in higher IR ink densities and vice versa. To give designers and creators a straightforward way to estimate invisible IR ink for any given RGB color input, we fitted a polynomial with 3 degrees of freedom, x = $K$, y = $lum$ and z = mean IR ink, using \textit{MATLAB}'s polynomial fitting function. The resulting linear model ($RMSE = 14.48$, $R^2 = 0.8477$) that can be seen in \autoref{eq:1} outputs an IR ink estimate based on the parameters $K$ and $lum$ given an RGB color input.

% \begin{equation} \label{eq:1}
% f(K, lum) = 113.7 - 0.1384*K - 25.43*lum + 0.0081*K^2 - 0.5103*lum^2 + (0*K^3) - 0.0013*lum^3 
% \end{equation}

\begin{equation} \label{eq:1}
\begin{aligned}
f(K, lum) = & \, 113.7 - 0.1384*K - 25.43*lum + 0.0081*K^2 \\
            & - 0.5103*lum^2 + (0*K^3) - 0.0013*lum^3 
\end{aligned}
\end{equation}

Despite our efforts to include a variety of different colors, the amount of data that can reasonably be collected in psychophysical lab experiments without causing too much fatigue is limited. To improve the reliability of our method for colors that we did not test in our experiment, we first defined three classes based on the histogram of the mean IR ink.
\new{The limits of these classes were selected to represent the color distribution based on visual inspection of} \autoref{fig:IRInkComputation}c.
\new{Here, we notice that most tested colors are accumulated on the left side, i.e., representing lighter colors. The first class we define captures the accumulation of the lighter colors. In the histogram, the other two classes have the same number of colors to make the binning uniform.}


\begin{figure*}[]
  \centering
  \includegraphics[width=1\textwidth]{figures-new/UIMap.png}
  \caption{Our software tool for embedding \systemName~ watermarks. (a) The initial setup window is used to select between online/offline and whole/custom region modes. (b) Online mode allows users to designate custom regions for tracking and embeds invisible QR codes that instruct the app how to access the AR content from the internet. (c) Offline mode optimizes fitting to accommodate as much data on the page as possible.
 }
  \Description{.}
  \label{fig:UIexample}
\end{figure*}




\new{However, designers are free to modify this based on their use case, as our goal here is to provide an example for our applications using our methodology.} Class 1 (81\% to 101\%) contains the colors that allow the least density of IR ink. For all colors that fall into this category, following the polynomial function (1), the system recommends the very conservative value of 81\% IR ink (white paper). Similarly, for Class 2 (>=102\% to 154\%) it recommends 102\%, and 155\% for Class 3 (>=155\% to 200\%). Actual undetectability thresholds are most likely much higher. Still, we aimed to provide a method that allows designers to include any desired color while ensuring that users cannot notice the IR ink.


\subsection{Summary}
Our results show that the IR ink can remain \textit{invisible to humans regardless of background color} (\textbf{H.1}). Furthermore, we found that the maximum IR ink density depends on the properties, mainly the color $lum$ and $K$ of the background color (\textbf{H.2}). We developed a system taking RGB as an input, outputting a conservative lower bound for the invisible IR ink density given a background color. To this end, we provide evidence that IR ink can, in fact, remain invisible to humans; however, it is unclear whether inkjet prints with established IR ink densities can be detected using IR sensors. Therefore, the next section describes how we embedded, captured, and decoded our IR ink watermarks.







