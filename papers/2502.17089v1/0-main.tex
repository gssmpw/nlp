% \RequirePackage{hyperref}
% % \usepackage{hyperxmp}  % Then load this

\documentclass[sigconf,screen]{acmart}

% \PassOptionsToPackage{bookmarksnumbered,unicode}{hyperref}
% \usepackage{hyperref}

\AtBeginDocument{%
  \providecommand\BibTeX{{%
    \normalfont B\kern-0.5em{\scshape i\kern-0.25em b}\kern-0.8em\TeX}}}

\setcopyright{acmcopyright}
\copyrightyear{2024}
\acmYear{2024}
\acmDOI{XXXXXXX.XXXXXXX}



\copyrightyear{2025}
\acmYear{2025}
\setcopyright{rightsretained}
\acmDOI{10.1145/3706598.3713286}
\acmISBN{979-8-4007-1394-1/25/04}
\acmConference[CHI '25]{CHI Conference on Human Factors in Computing Systems}{April 26--May 01, 2025}{Yokohama, Japan}
\acmBooktitle{CHI Conference on Human Factors in Computing Systems (CHI '25), April 26--May 01, 2025, Yokohama, Japan}



\usepackage{color}
\usepackage{soul}    % highlighting
\usepackage{gensymb} % degree symbol



\usepackage{graphicx}
 
\usepackage{enumitem} % remove indent of list items

\usepackage{tabularray}


%%% commands
\newcommand{\systemName}{\textsc{Imprinto}}



% \newcommand{\new}[1]{#1}
\newcommand{\new}[1]{\textcolor{black}{#1}} % color-code for revise and resubmit
\newcommand{\newCameraReady}[1]{\textcolor{black}{#1}} % color-code for camera-ready changes

\sloppy

\begin{document}


\title[\systemName: Enhancing Infrared Inkjet Watermarking Human and Machine Perception]{\systemName:  Enhancing Infrared Inkjet Watermarking for \newline Human and Machine Perception}

 

\author{Martin Feick}
\orcid{0000-0001-5353-4290}
\affiliation{%
 \institution{DFKI \& Saarland University}
 \city{Saarland}
 \country{Germany}}
\affiliation{%
  \institution{MIT CSAIL}
  \city{Cambridge}
  \state{Massachusetts}
\country{US}
}
 \email{martin.feick@dfki.de}
 
\author{Xuxin Tang}
\orcid{0009-0004-7997-6280}
\affiliation{%
 \institution{Virginia Tech}
 \city{Blacksburg}
 \state{Virginia}
 \country{US}}
  \affiliation{%
  \institution{MIT CSAIL}
  \city{Cambridge}
  \state{Massachusetts}
\country{US}
}
\email{xuxintang@vt.edu}


\author{Raul Garcia-Martin}
\orcid{0000-0001-5319-2016}
\affiliation{%
  \institution{Universidad Carlos III de Madrid}
  \city{Leganes, Madrid}
  \country{Spain}
}
\affiliation{%
  \institution{MIT CSAIL}
  \city{Cambridge}
  \state{Massachusetts}
  \country{US}
}
\email{raulgarc@ing.uc3m.es}

\author{Alexandru Luchianov}
\orcid{0009-0005-9595-0635}
\affiliation{%
  \institution{MIT CSAIL}
  \city{Cambridge}
  \state{Massachusetts}
  \country{US}
}
\email{lknv@mit.edu}

\author{Roderick Wei Xiao Huang}
\orcid{0009-0001-7120-6968}
\affiliation{%
  \institution{MIT CSAIL}
  \city{Cambridge}
  \state{Massachusetts}
  \country{US}
}
\email{rwxhuang@mit.edu}

\author{Chang Xiao}
\orcid{0009-0008-7143-2771}
\affiliation{%
  \institution{Adobe Research}
  \city{San Jose}
  \state{California}
  \country{US}
}
\email{cxiao@adobe.com}

\author{Alexa Siu}
\orcid{0000-0002-4879-1476}
\affiliation{%
  \institution{Adobe Research}
  \city{San Jose}
  \state{California}
  \country{US}
}
\email{asiu@adobe.com}

\author{Mustafa Doga Dogan}
\orcid{0000-0003-3983-1955}
\affiliation{%
  \institution{Adobe Research}
  \city{Basel}
  \country{Switzerland}
}
\affiliation{%
  \institution{MIT CSAIL}
  \city{Cambridge}
  \state{Massachusetts}
  \country{US}
}
\email{doga@adobe.com}

%%
%% By default, the full list of authors will be used in the page
%% headers. Often, this list is too long, and will overlap
%% other information printed in the page headers. This command allows
%% the author to define a more concise list
%% of authors' names for this purpose.
\renewcommand{\shortauthors}{Feick, et al.}

%%
%% The abstract is a short summary of the work to be presented in the
%% article. % 150 words
\begin{abstract}

Hybrid paper interfaces leverage augmented reality to combine the desired tangibility of paper documents with the affordances of interactive digital media.
Typically, virtual content can be embedded through direct links (e.g., QR codes); however, this impacts the aesthetics of the paper print and limits the available visual content space. %, and may even degrade accessibility. 
To address this problem, we present \systemName, an infrared inkjet watermarking technique that allows for invisible content embeddings only by using off-the-shelf IR inks and a camera.
\systemName~ was established through a psychophysical experiment, studying how much IR ink can be used while remaining invisible to users regardless of background color.
We demonstrate that we can detect invisible IR content through our machine learning pipeline, and we developed an authoring tool that optimizes the amount of IR ink on the color regions of an input document for machine and human detectability. 
Finally, we demonstrate several applications, including augmenting paper documents and objects. 
\end{abstract}

%%
%% The code below is generated by the tool at http://dl.acm.org/ccs.cfm.
%% Please copy and paste the code instead of the example below.
%%
\begin{CCSXML}
<ccs2012>
   <concept>
       <concept_id>10003120.10003121</concept_id>
       <concept_desc>Human-centered computing~Human computer interaction (HCI)</concept_desc>
       <concept_significance>300</concept_significance>
       </concept>
 </ccs2012>
\end{CCSXML}

\ccsdesc[300]{Human-centered computing~Human computer interaction (HCI)}

%%
%% Keywords. The author(s) should pick words that accurately describe
%% the work being presented. Separate the keywords with commas.
\keywords{augmented reality; mixed reality; infrared imaging; watermarking; digital fabrication}


%% A "teaser" image appears between the author and affiliation
%% information and the body of the document, and typically spans the
%% page.

\begin{teaserfigure}
  \includegraphics[width=\textwidth]{figures-new/Imprinto.png}
  \caption{\systemName~ are real-life documents that enable watermarking through the use of IR-absorbing ink. Users can apply it by (1) making it part of the document's fabrication process using an off-the-shelf inkjet printer or (2) by using a handheld inkjet printer on demand. The embedded content is invisible to users, even on blank paper, while allowing for high data capacity. Our system reliably decodes the IR content and \textit{is further instructed by it} regarding what images to track and what to display.}
  \label{fig:teaser}
\end{teaserfigure}



%%
%% This command processes the author and affiliation and title
%% information and builds the first part of the formatted document.
\maketitle


The increasing reliance on LLMs for multimodal tasks across far-reaching sectors such as healthcare, finance, and manufacturing underscores the need to assess the accuracy and reliability of the information they generate. Vision-Language Models (VLM) have achieved state-of-the-art (SoTA) performance on Visual Question-Answering (VQA) benchmarks, and these models often utilize Retrieval-Augmented Generation (RAG) to maintain factual accuracy and relevance in a dynamic information environment. However, this has led to uncertainty in the information the LLM bases its answer on, as it may choose between parametric memory and retrieved sources. When models rely on memorized information instead of dynamically retrieving information, they may inadvertently propagate outdated or incorrect information, causing serious legal and ethical risks and undermining trust and reliability in AI systems \citep{huang2023survey}.
% The ability to strike a balance between generalization and specialization in AI systems is therefore crucial for ensuring the safe, reliable use of these technologies in real-world applications.

Despite these concerns, the way that Vision-Language models (VLMs) memorize and retrieve information, particularly in complex multimodal tasks, remains under-explored. Current research often focuses on either the general capabilities of large language models (LLMs) or the specialized retrieval mechanisms in retrieval augmented generation systems (RAG) \citep{incontext_rag,chen_murag_2022,liu_universal_2023}. Particularly in the context of multimodal retrieval and multihop reasoning, few studies analyze the tradeoff between finetuning for specialized tasks and zero-shot prompting for general-purpose vision-language capabilities. A lack of consensus on how to approach this tradeoff motivates the development of measures to quantify reliance on parametric memory, as well as metrics for quantifying the potential performance impact of extending LLMs with RAG systems.

To address this gap, we investigate how multimodal QA models balance accuracy with memorization on the WebQA benchmark. We compare finetuned multimodal systems against zero-shot VLMs, analyzing how retrieval performance influences QA accuracy. In particular, we focus on cases where retrieval fails, allowing us to measure reliance on parametric memory through two proposed metrics---the \ppr (\PPR) which quantifies how much model accuracy is influenced by retrieval quality, contrasting performance in best-case versus worst-case retrieval scenarios, and the \ucr (\UCR) which measures how often correct QA responses are generated when the retriever fails, providing a proxy for memorization.

To enable this analysis, we make several methodological contributions. For the finetuned QA models, we investigate Vision-Transformer (ViT) architectures, which allow for multihop reasoning over multiple sources. To investigate the impact of retrieval performance on trained LMs, we propose a variable-input Fusion-in-Decoder (FiD) model \cite{tanaka_slidevqa_2023, nlvr2}, building upon the VoLTA architecture \citep{pramanick_volta_2023}. For the zero-shot case, we build upon previous research on In-Context Retrieval \citep{incontext_rag} by demonstrating that LLMs such as GPT-4o are capable of performing the final ranking step of the retrieval process. In doing so, we find that GPT-4o, a general-purpose LLM, achieves SoTA performance on the WebQA task, outperforming existing finetuned RAG models by a significant margin (7\% higher accuracy). 

Crucially, our results reveal that while retrieval-augmented models reduce memorization, the training paradigm plays an important role. Finetuned models exhibit higher reliance on parametric memory, whereas zero-shot RAG approaches have lower memorization scores at the cost of accuracy. This suggests that while retrieval modules may mitigate the risks associated with outdated or incorrect information, SoTA performance requires that they be coupled with specialized QA models. Our memorization measures contribute to the development of transparent and reliable AI systems, particularly in applications where the sourcing of up-to-date, factual information is critical.



% We investigate the impact of question complexity on the ability of these models to integrate multiple data sources—such as images, text, and external retrievers—and produce coherent and accurate answers. We also explore whether in-context retrieval can be a viable alternative to traditional retrieval-augmented systems, offering a more streamlined approach to multimodal QA.

% To achieve this, we first compare zero-shot prompting multimodal LLMs with finetuned multimodal systems. We evaluate both types of models on the WebQA benchmark, a dataset designed for complex question answering that requires reasoning across both image and text sources. For the finetuned models, we use a Fusion-in-Decoder (FiD) architecture, which allows for multihop reasoning over multiple sources. Additionally, we introduce the concept of In-Context Retrieval Language Modeling (RLM), where the LLM itself performs retrieval tasks without the need for external retrievers. This method builds upon existing research in in-context learning  and aims to explore the viability of LLMs retrieving relevant sources and generating accurate answers directly from their context window.

% In order to investigate source utilization in finetuned multimodal models and LLMs, three lines of inquiry are established; 
% \begin{itemize}
%     \item Study 1: retrieval vs QA performance on webQA (motivating example, does QA answer correctly even with incorrect sources?)
%     \item Study 2: performance on adversarial examples where parametric knowledge would be incorrect by design
%     \item Study 3: improving performance on adversarial examples by fine-tuning (i.e model robustness)
% \end{itemize}

% Note, there is one weakness in this plan which is tying in the work we've already done. 
% If we added something from adversarial generation to the retrieval experiment (like a combination of study 1 + 3) it would be complete. So for instance we could try fine-tuning the retriever with adversarial examples (and not just the QA model)

% \begin{figure}
%     \centering
%     \includegraphics[width=0.95\linewidth]{figures/segmentation/webqa_segment_infill.png}
%     \caption{Example of the segmentation substitution pipeline from the WebQA task.}
%     % d5c76d760dba11ecb1e81171463288e9
%     \label{fig:seg_sub_pipeline}
% \end{figure}



% Retrieval augmented generation (RAG) with zero-shot prompting and fine-tuning Large Language Models (LLMs) have become the go-to methods for tasks relying on information retrieval and text generation. In many cases the LLMs parametric memory can sufficiently generalize to answer questions without being provided with retrieval mechanisms for out-of-domain knowledge. However, LLMs often hallucinate and provide wrong information in certain scenarios. This problem is amplified even further on open-domain Question Answering (QA) tasks involving multiple modalities. Grounded text generation using retrieved sources \citep{lewis2021retrievalaugmented} has been extensively studied for text-to-text QA tasks, but its application in multimodal settings has not been studied as much.


% Multimodal reasoning and question answering have gained prominence in recent research endeavors, with an increasing emphasis on handling various forms of data, particularly text and images. In this study, we address a specific gap in the existing literature by focusing on the development of a versatile multihop model capable of accommodating varying numbers of input images.

% Our motivation for this research lies in the growing complexity of answering questions using information on the web, where the challenge of navigating the open-domain setting is further complicated by the presence of multiple modalities and sometimes requires reasoning over multiple sources. WebQA is an ideal dataset on which to compare performance of finetuned RAG systems against general purpose LLMs; it is multimodal, with correct answers requiring reasoning over image and text sources. It is multihop, requiring a complex reasoning process over multiple sources. Finally, WebQA questions from different categories can be broken down into subdomains to analyze performance over domains of varying cardinality.

% Motivated by the real-world challenges of building retrieval and question answering (QA) systems, we design and finetune a closed domain, multimodal, multihop QA model, that is capable of reasoning over a varying number of sources taken as input from an external retriever module. This research contributes to the relatively underexplored domain of multihop reasoning across various input sources and modalities. Our goal is to explore the challenges posed by these scenarios and develop strategies that enable QA models to retrieve relevant information, conduct logical or numerical reasoning across diverse modalities, and generate coherent responses in natural language. To our knowledge, this is the first application of the Fusion-in-Decoder (FiD) architecture \cite{tanaka_slidevqa_2023, nlvr2} that is shown to work with a variable number of inputs, enabling multi-hop reasoning over sources.

% In-Context Learning refers to the ability of LLMs to perform any task by simply providing examples in the input prompt \citep{dong2022survey,min2022rethinking}. Inspired by this research, we propose a method to use the LLM itself as a multimodal retriever, potentially eschewing the requirement of a distinct retrieval module, thereby allowing the design of simpler retrieval-augmented QA systems. We dub this method In-Context Retrieval Language Modeling (RLM). To the best of the authors knowledge, In-Content RLM is disparate from other retrieval augmented approaches which utilize external retrieval modules \citep{incontext_rag,chen_murag_2022,liu_universal_2023}. Despite being a natural extension of In-Context learning, In-Context RLM has not yet been studied empirically.

% To expand on our contribution of In-Context Retrieval, this stems from the well-researched in-context learning of LLMs. In-context learning is the ability of a model to perform any task given a sufficient context window \citep{dong2022survey,min2022rethinking}. Such tasks could include retrieval and ranking, but typically, the go-to solution for tasks requiring retrieval has been RAG. To the best of the authors knowledge, In-Context Retrieval is distinct from In-Context Retrieval Augmented Language Modelling (RALM), and despite being a natural extension of In-Context learning, In-Context Retrieval has not yet been shown empirically.

% Finally, we explore the tradeoff between using zero-shot prompting LLMs and the fine-tuning approach. While we find that, overall, GPT-4o obtains SoTA performance on the WebQA task, outperforming the accuracy of existing finetuned RAG approaches by 7\%, finetuned approaches still perform better on more restricted subdomains\footnote{``In-Context RLM" @ \url{https://eval.ai/web/challenges/challenge-page/1255/leaderboard/3168}}. Finally, we validate that GPT-4o is relying on retrieval abilities to solve the task; we find that GPT-4o is capable of retrieving relevant sources in the presence of distractors and furthermore, when GPT-4o fails to retrieve correct sources, it answers incorrectly 75\% of the time, meaning that it is not relying on parametric memory for this task.

% \paragraph{Contributions}
% Based on our experimentation and analysis on the WebQA benchmark, we make the following contributions:
% \begin{itemize}
%     \item Propose a new architecture for multimodal multihop QA that takes variable number of input sources inspired by the Fusion-in-Decoder method.
%     \item Comparison of general purpose LLMs vs specialized models on the WebQA benchmark.
%     \item Observation of In-Context Multimodal Retrieval abilities of GPT-4o and that it does not rely on parametric memory for multimodal QA.
%     \item Analysis of relationship between retrieval and QA task performance.
%     \item Analysis of task and query complexity on the performance of retrieval and QA tasks.
% \end{itemize}
















% Throughout this paper, we will present our methodology, experiments, and findings, emphasizing our approach to multihop reasoning over varying numbers of input images. We believe that our work contributes to a deeper understanding of multimodal reasoning and has the potential to enhance the capabilities of question-answering systems in the intricate, multimodal landscape of web-based information.
\section{Related Work}
\label{sec:relatedwork}
Traditionally, an experimental \ac{IR} collection includes three elements, a corpus, a set of topics, and the relevance judgments, defining which documents are relevant in response to the topics.
Over the last 30 years, since the first TREC campaign~\cite{DBLP:conf/trec/1992}, the most common strategy to obtain such relevance judgments has involved expert annotators, capable of providing the most accurate labels. 
The cost of this process can be partially reduced with pooling~\cite{croft2009search}, but the monetary and temporal costs of building an \ac{IR} experimental collection following this paradigm remain extremely high.

Automatic relevance judgment has recently received significant attention in the IR community. In earlier studies, ~\citet{faggioli2023perspectives} studied different levels of human and LLMs collaboration for automatic relevance judgment. They suggested the need for humans to support and collaborate with LLMs for a human-machine collaboration judgment. ~\citet{thomas2023large} leverage LLMs capabilities in judgment at scale, in Microsoft Bing. They used real searcher feedback to build an LLM and prompt in a way that matches the small sample of searcher preferences. Their experiments show that LLMs can be as good as human annotators in indicating the best systems. They also comprehensively investigated various prompts and prompt features for the task and revealed that LLM performance on judgments can vary with simple paraphrases of prompts. Recently, \citet{rahmani2024synthetic} have studied fully synthetic test collection using LLMs. In their study, they generated synthetic queries and synthetic judgment to build a full synthetic test collation for retrieval evaluation. They have shown that LLMs can generate a synthetic test collection that results in system ordering performance similar to evaluation results obtained using the real test collection.

On a different line, \citet{DBLP:conf/sigir/Dietz24} defines a LLM-based ``autograding'' approach. This evaluation strategy targets generated content that cannot be evaluated in a purely offline scenario and it consists of using a question bank as the evaluation test-bed. An \ac{LLM} measures the effectiveness of the generative model in answering the questions, possibly with the supervision of a human. The autograding approach proposed by \citet{DBLP:conf/sigir/Dietz24} includes an automatic passage evaluation whose task aligns with the one evaluated in \texttt{LLMJudge}.

\subsection{Criticisms and Open Challenges}
The use of \acp{LLM} as assessors comes with major bias risks and challenges that should not be neglected, especially considering the impact they might have in the development of \ac{IR} evaluation.

\partitle{Bias}
First and most importantly, \acp{LLM} are affected by bias~\cite{DBLP:conf/fat/BenderGMS21}. Their internal representation of the concepts is, by construction, conditioned on the context such concepts appear in~\cite{DBLP:conf/nips/VaswaniSPUJGKP17}. Thus, depending on the underlying data, the \ac{LLM} might form a biased notion of relevance that might reflect upon the relevance judgments generated by it. Quantifying the bias, identifying its source, and mitigating its consequences are still open issues that need to be addressed. We hope that the release of this collection will help the research community with the needed data to study how to deal with the bias in \ac{LLM}-generated relevance judgments.

\partitle{Circularity}
A second source of concern when it comes to using \acp{LLM} as assessors relates to the risk of \textit{circular evaluation}~\cite{faggioli2023perspectives,DBLP:journals/corr/abs-2409-15133}. For example, the same \ac{LLM} might be used to generate relevance judgments and as a document ranker. This would induce a strong bias on the validity and generalizability of the relevance judgments.

\partitle{Environmental Impact}
An often hidden cost of the \acp{LLM} concerns their environmental impact in terms of energy utilization, carbon emissions~\cite{DBLP:journals/corr/abs-2408-09713,DBLP:conf/sigir/ScellsZZ22}, and water consumption~\cite{DBLP:conf/ictir/ZucconSZ23}.
While \acp{LLM} might allow building collections at a fraction of the monetary and temporal cost, we should account for the environmental impact of such a process, limiting our reliance on ``disposable'' relevance judgments.

\partitle{Vulnerability to Attacks and Adversarial Misuse}
\citet{DBLP:conf/ecir/ParryFMPH24} and \citet{DBLP:conf/sigir-ap/Alaofi0SS24} illustrate the vulnerability of the \acp{LLM} to mischievous manipulations of the corpus. For example,~\citet{DBLP:conf/ecir/ParryFMPH24} show that, by introducing keywords such as the term ``relevant'' in a document, it will more likely considered relevant by an \ac{LLM}. Similar behavior is observed also by \citet{DBLP:conf/sigir-ap/Alaofi0SS24}, who notice that by introducing the query on the document, more probably an \ac{LLM} will consider the document relevant to such a query --- even if the rest of the document is composed by random terms.
More recently, \citet{DBLP:journals/corr/abs-2412-17156} show how, by properly crafting an adversarial run, it is possible to cheat an \ac{LLM} used as an assessor. \citet{DBLP:journals/corr/abs-2412-17156} crafted a run following the same approach used by~\citet{upadhyay2024umbrela} to pool the documents and build the \ac{LLM}-generated relevance judgments used for TREC 2024 RAG. Such a run achieved consistently higher effectiveness under the fully automatic evaluation paradigm compared to its performance based on manual relevance judgments. 

By releasing this collection of \ac{LLM}-generated relevance judgments we want to foster the analysis and study of possible sources of biases and systematic errors, to mitigate them and allow for the development of more effective and robust future solutions that involve \acp{LLM} as tools to support the annotation process.
\section{Synthesizing Attribution Data}

\begin{figure*}[ht]
    \centering
    \includegraphics[width=\textwidth]{img/pipeline.drawio.pdf}
    \caption{\textbf{Top:} The \synatt baseline method for synthetic attribution data generation. Given context and question-answer pairs, we prompt an LLM to identify supporting sentences, which are then used to train a smaller attribution model. However, this discriminative approach may yield noisy training data as LLMs are less suited for classification tasks (see \S\ref{sec:experiments-zero-shot}). \textbf{Bottom:} The \synqa data generation pipeline leverages LLMs' generative strengths through four steps: (1) collection of Wikipedia articles as source data; (2) extraction of context attributions by creating chains of sentences that form hops between articles; (3) generation of QA pairs by prompting an LLM with only these context attribution sentences; (4) compilation of the final training samples, each containing the generated QA pair, its context attributions, and the original articles enriched with related distractors.}
    % \caption{\textbf{Top:} The \synatt baseline. Intuitively, we can prompt an LLM for context-attribution by providing the context and question-answer pairs. Then, we train a smaller model on the obtained synthetic data. However, LLMs are less suitable for discriminative (i.e., classification) tasks, and may yield noisy training data (see \S\ref{sec:experiments-zero-shot}). \textbf{Bottom:} The \synqa data generation pipeline consists of four main steps: (1) collection of Wikipedia articles as the source data; (2) extracting the context attributions by creating chains of sentences that form hops between articles; (3) generation of QA pairs by prompting an LLM with only the context attribution sentences; (4) we obtain the resulting \synqa training sample containing three components: the generated QA pair, the context attributions, and the original articles supplemented with related distractor articles.}
    \label{fig:method}
\end{figure*}

Context attribution identifies which parts of a reference text support a given question-answer pair~\cite{rashkin2023measuring}. Formally, given a question $q$, its answer $a$, and a context text $c$ consisting of sentences ${s_1, ..., s_n}$, the task is to identify the subset of sentences $S \subseteq c$ that fully support the answer $a$ to question $q$. To train efficient attribution models without requiring expensive human annotations, we explore synthetic data generation approaches using LLMs.
% Context attribution poses the following question~\cite{rashkin2023measuring}: given a generated text $t_g$ and a context text $t_c$, is $t_g$ attributable to $t_c$? To train models to perform well on this task, we explore how to best generate synthetic attribution data using LLMs. We implement two methods: a discriminative and generative method. 
We implement two methods for synthetic data generation. Our baseline method (\synatt) is discriminative: given existing question-answer pairs and their context, an LLM identifies supporting sentences, which are then used to train a smaller attribution model. Our proposed method (\synqa) takes a generative approach: given selected context sentences, an LLM generates question-answer pairs that are fully supported by these sentences. This approach better leverages LLMs' natural strengths in text generation while ensuring clear attribution paths in the synthetic training data.

%The first method is relatively straightforward and termed \synatt. A simple way to generate synthetic data for context attribution is to ask an LLM to pick out the sentences that support a given question-answer pair. 

% \subsection{Discriminative and Generative Synthetic Data Generation}

% The first method (\synatt) is relatively straightforward: ask the LLM to pick relevant sentences from a provided context that support a given question-answer pair. However, this \textit{discriminative} approach of performing sentence classification overlooks the fact that LLMs excel at \textit{generating} text. Therefore, we design a second data generation method (\synqa) that is generative and thus capitalizes on the strength of LLMs. It involves the following pipeline steps (see also Fig.~\ref{fig:method}): context collection, question-answering generation and distractor mining, which increases the difficulty of the task, thus reflecting more realistic scenarios.

%\textbf{Attribution Synthesis.} The most straightforward approach to generating synthetic data for context attribution is discriminative: prompting an LLM to identify relevant sentences from context documents given a question-answer pair. While intuitive, this approach underutilizes LLMs' capabilities, as they excel at generative rather than discriminative tasks. LLMs are fundamentally designed to generate coherent text following instructions rather than perform binary classification of sentences. In our experiments (\S\ref{sec:experiments}) we dub this method as \synatt.

\subsection{\synqa: Generative Synthetic Data Generation Method}

\synqa consists of three parts: context selection, QA generation, and distractors mining (for an illustration of the method, see Figure~\ref{fig:method}). In what follows, we describe each part in detail.

\textbf{Context Collection.} We use Wikipedia as our data source, as each article consists of sentences about a coherent and connected topic, with two collection strategies. In the first, we select individual Wikipedia articles for dialogue-centric generation and use their sentences as context. In the second, for multi-hop reasoning, we identify sentences containing Wikipedia links and follow these links to create ``hops'' between articles, limiting to a maximum of two paths to maintain semantic coherence, while enabling more complex reasoning patterns (for more details, see Appendix~\ref{app:synthetic_data}).
% \textbf{Context Collection.}  The first step is to select a dataset where each data point is a set of sentences about a coherent and connected topic. These sentences will serve as the context in which we want to find relevant attributions later. We use Wikipedia as the data source
%To better leverage LLMs' generative capabilities, we propose \synqa, a novel and simple approach for synthesizing context attribution data (see Fig.~\ref{fig:method}). 
%We first collect Wikipedia articles that are not present in our testing datasets\footnote{We detect potential data leakage by representing each Wikipedia article as a MinHash signature. Then, for each training Wikipedia article, we retrieve candidates from the testing datasets via Locality Sensitivity Hashing and compute their Jaccard similarity \cite{dasgupta2011fast}. Pairs exceeding a tunable threshold (empirically set to 0.8) are flagged as potential leaks.}.
%For each article, 
% we implement two distinct collection strategies that differ in difficulty. First, we select individual Wikipedia articles and randomly select multiple sentences within each article. Second, we start from a randomly selected sentence containing at least one Wikipedia link
%\footnote{These are human annotated in the Wikipedia articles, or alternatively, can be obtained from entity linking methods \cite{de-cao-etal-2022-multilingual}.} 
% and follow the links to other articles, creating ``hops'' between related content. We limit the chain to a maximum of two hops (connecting up to three articles) to maintain semantic coherence while enabling the more difficult multi-hop reasoning scenarios (for more details, see Appendix~\ref{app:synthetic_data}). 
%In the second strategy, we select individual Wikipedia articles and randomly select multiple sentences within each article that can serve as evidence for generated questions.

\textbf{Question-Answer Generation.} Given the set of contexts, an LLM can now generate question-answer pairs. For single articles, we prompt the model to generate multiple question-answer pairs, each grounded in specific sentences. This creates a set of dialogue-centric samples where questions build upon the previous context. For linked articles, we prompt the model to generate questions that necessitate connecting information across the articles, encouraging multi-hop reasoning.
%\footnote{Note that multi-hop reasoning is not guranteed here; rather, the LLM has the ability to decide whether the question-answer pair involves multiple hops of reasoning. See App. for details.}. 
This yields multi-hop samples requiring integration of information across documents, as well as samples that mimic a dialogue about a specific topic given the context. We provide the full prompts used for generation in Appendix \ref{app:prompts}.

\textbf{Distractors Mining.} To make the attribution task more realistic, we augment each sample with distractor articles. With E5 \cite{wang2022text}, we embed each Wikipedia article in our collection. For each article in the training sample, we randomly select up to three distractors with the highest semantic similarity to the source articles. These distractors share thematic elements with the source articles, but lack information to answer the questions.%do not contain the information necessary to answer the generated questions.

\subsection{Advantages of \synqa}
The \synqa approach has three key advantages:
%over discriminative data generation:
% (1) it leverages LLMs' natural strength in generative tasks; (2) produces diverse multi-hop reasoning scenarios; and (3) creates coherent question-answer pairs with clear attribution paths.
(1) it leverages LLMs' strength in generation rather than classification; (2) creates diverse training samples requiring both dialogue understanding and multi-hop reasoning; and (3) ensures generated questions have clear attribution paths since they are derived from specific context sentences.
By generating both entity-centric and dialogue-centric samples, \synqa produces training data that reflects the variety of real-world QA scenarios, helping models develop robust attribution capabilities, which our experiments demonstrate to generalize across different contexts and domains.
% We formalize the problem of Context Attribution QA as follows: Given a pre-defined context $T_c=\lbrace s_1, s_2, \ldots , s_n \rbrace$---where $s_i$ is a sentence---and an answer text $t_a$ generated by an LLM, the context attribution model should provide a vector $a=(a_1, \ldots , a_n)$, where each element $a_i$ has the following possible values:
% \[
% a_i =
% \begin{cases}
%     1, & \text{if } s_i \text{ supports the generated answer } t_a\\
%     0,  & \text{otherwise} 
% \end{cases}
% \]
% In our setup, we should have at least one entry $a_i = 1$.
% \begin{itemize}
%     \item The simplest way to generate synthetic data for context-attribution is in a discriminative manner: we prompt an LLM to provide the sentence level context attributions given the context documents, question and answer. We deem this generation as discriminative as the model effectively classifies the sentences that are most relevant to the question-answer pair.
%     \item The issue with this approach is that LLM are not best suitable for discriminative tasks, but rather generative. That is, an LLM is better at generating text by following instructions, than classifing sentences/etc.
%     \item To leverage what LLMs are good for, we create a simple context attribution data generation approach where we perform the following: (1) We find wikipedia articles (which are not contained in the testing datasets)\footnote{Describe the approach for dealing with data leakage}; (2) We select a random sentence in a wikipedia article, and find the links to other wikipedia articles (the hops). We select that sentence, and hop to the other Wikipedia article (given by the link). (3) We perform the hop step for maximum of 2 times (i.e., we connect at most 3 articles, and 1 at least). We end up with 3 Wikipedia articles which constitute the hops.
%     \item We provide Llama70B with either 1 wikipedia article or the hops and ask the model to generate a multi-hop question-answer pair which ideally connects all connected articles, or as many as it can; alternatively, if we provide the model with only 1 wikipedia article, we ask the model to select as many sentences as possible in the article, and for each, generate a question-answer pair (we provide the full prompts we use in Appendix).
%     \item The output of the model is a set of question-answer pairs (or a single one), that is grounded in the evidence provided by the sentence(s). We dub the entire approach as \synqa.
%     \item In summary, we develop two settings to generate synthetic data for context attribution in question answering: one is entity-centric and yield data which might be multi-hop; and the other is dialog-centric where subsequent questions build on top of previous ones.
%     \item Finally, to all context + question + answer + context-attribution samples we add distractors: we obtain embeddings using E5 of each wikipedia page, and for each sample we select up to 3 distractors which we add to the data sample. These distractors are similar are document with similar context as the one from which the context-attributions are.
% \end{itemize}


\section{Imprinto Watermarks}
In this section, we provide a comprehensive overview of the watermark embedding and detection process. The key steps involved are capturing the image, binarizing the image and extracting the QR code from the binarized image. Although extracting the QR code from an already binarized image can be done using commercially available software, the remaining necessary steps required the development of specialized modules. 




\subsection{Embedding the Watermark}
We developed a software tool to embed markers based on the system and the invisible IR ink values determined in our experiment in Section~\ref{visibility-experiment}. To use our tool shown in \autoref{fig:UIexample}, users first select the document for which they wish to create embeddings. Then, they choose whether to utilize the whole sheet for embedding or only designated areas. In the latter, the user marks on the document where they want to embed content, and the system interactively generates layouts. Finally, users decide whether to rely on internet information (online) or create a fully self-contained (offline) document.
The tool then analyzes the RGB values of the background color of the specified image region. It determines the conservative lower bound of the invisible IR ink density using the underlying system described in Section~\ref{visibility-experiment}.



%\begin{comment}
\paragraph{Online mode}
In the online mode, we take advantage of the ability to use the internet by hyperlinking multimedia content using QR codes.
In this menu (\autoref{fig:UIexample}b), the user can select the paper area that they wish to track using AR. An image score is computed for each selection using \textit{Google arcoreimg} tool\footnote{\url{https://developers.google.com/ar/develop/augmented-images/arcoreimg}} to ensure proper tracking and content overlay in AR.
In case the image is too difficult to track (i.e., the score is lower than 75), we inform the user and prompt them to choose a different area. Our tool allows users to associate each AR-tracked image with online multimedia content, such as text, image, and/or audio.
For each tracked image, the user can link a series ("frames") of these types of contents that users can loop through in AR. The underlying technology can be further generalized and used in order to incorporate complex multimedia content (such as videos or 3D models) or engineer further interactions between the objects. 
% Each tracked image is associated with a cyclical list of frames. A frame is characterized by a text file, an image to display, and a sound that can optionally play during the subsequent transition. 

%In the standard Unity application, we display the current frame above the tracked image and transition to the next frame every time it is clicked.%



\paragraph{Offline mode}
In the offline mode (\autoref{fig:UIexample}c), we are currently limited to text-only content due to multimedia content requiring significantly more space.
However, this capacity that offline embedding offers can be used to embed hyperlinks or XML media, such as contact cards, or stylized and interactive text content via markup language tags of the user's choice.
Even though we offer an automatic analysis of the document and calculate a possible way of embedding the information, we further allow the user to fine-tune these settings (i.e., size, error correction level, and version of the QR codes) according to their specific requirements.






\subsection{Capturing the Watermark}



\new{\systemName~ media can be captured using devices with NIR image sensors. There are two main methods by which users can do this. First, more conventionally, users can point their devices at the physical media to extract \systemName~ content. This is particularly suitable for settings where the users are aware that the content is already embedded (e.g., an instructor informing students about \systemName~ verbally before handing out worksheets, or through a small visual cue on them; a book or magazine mentioning the AR functionality only on its cover---so not each page needs to carry a notice). 
Second, there are other types of devices that have constant scanning functionality~\cite{dogangun_rampa_2024, iyer_xr-penter_2025}. Emerging devices, such as smart glasses (see \textit{Meta Orion}\footnote{\url{https://about.meta.com/realitylabs/orion}}),  feature "always-on" IR scanning functionality to map the environment that we envision can be repurposed to detect human-invisible IR markers (see \autoref{fig:ApplicationsRobotic}). This, in turn, would enable the always-on and seamless capture of these markers without much physical effort. However, accessibility to the technology is not yet widely available, discussed in  
Section~\ref{NIR_AR_FormFactor}.} 

\new{To support this emerging technology, we developed a \textit{universal} \systemName~ detection and reader module for mobile devices.} %: the RGMVision InfraRedCAM 1.
The module was built to decode document watermarks based on the inkjet ink absorbance characteristics (Section~
\ref{spectrum}) by featuring a 5MP CMOS NIR camera and two controllable NIR LEDs, both operating at a wavelength of $850~nm$. The sensor module integrates a bandpass filter centered at the same wavelength. We designed and developed the PCB and the aluminum enclosure for mechanical and electromagnetic protection produced by \textit{Megacal}\footnote{\url{https://megacal.es/}}, with a size of $59.5\times33.0\times11.6~mm$. 
Figure~\ref{fig:RGMCameraModule} shows the components of the module.
Our proprietary module, \textit{RGMVision InfraRedCAM 1}\footnote{\url{https://www.rgmvision.com/product/rgmvision-infraredcam-1/}}, ensures universal mobile connectivity through \textit{USB 2.0} or \textit{3.0} and its \textit{USB-C} connector. %and is compatible with \textit{Android}/\textit{iOS} mobile devices, as well as \textit{Windows}/\textit{macOS}/\textit{Ubuntu}/\textit{Raspberry Pi OS} laptops and desktop computers.
As opposed to related works, this allows our device to offer universal compatibility, converting any mobile device, smartphone, laptop, or PC into an \systemName~ reader via its \textit{USB-C} support. Additionally, the two diffuse NIR LEDs (80°), controlled by two PWM signals, provide 24 different adjustable illumination levels for indoor applications.


\begin{figure}[]
  \centering
  \includegraphics[width=0.47\textwidth]{figures-new/Fig6-RGMVision_InfraredCAM_1-new.jpg}
  \caption{% Developed USB smartphone reader – RGMVision InfraredCAM 1 –
  Our mobile imaging attachment comes with a 5MP sensor and two controllable NIR LEDs working at $850~nm$. (a) Module directly plugged into the mobile device. (b) Module coupled and lined with the back smartphone camera through a USB extender. (c) Front view of the module with the NIR LEDs powered up on a different device. (d) 3D exploded view, including the sensor, the LEDs mounted in the PCB, and the aluminum mechanical and electromagnetic enclosure.}
  \Description{.}
  \label{fig:RGMCameraModule}
\end{figure}



We developed an \textit{Android} application (built for API level 26 and targeted for API level 30), based on \textit{Unity3D}, interfaces with the module through the \textit{Unity3D} plugin \textit{UVC4UnityAndroid v2.0}\footnote{\url{https://github.com/saki4510t/UVC4UnityAndroid}}.




\subsection{Decoding the Watermark}

We developed an ML pipeline to binarize the captured IR watermarks for robust detection, which we will make open-source. Because the IR camera captures the watermarks in a faint tone, off-the-shelf QR code readers are unable to recognize \systemName~ codes. Thus, we need an intermediary step where the contrast of the IR watermarks is increased via binarization, where the codes are converted into black-and-white patterns similar to conventional QR codes. Previous works~\cite{dogan_standarone_2023} have used traditional image processing filters for this step, whereas we use an ML-focused approach to increase robustness (Section~\ref{EvaluationDataEmbedding}). 
We next explain the individual steps that were necessary for building our pipeline, as well as how to use it in real time.


\subsubsection{Training the ML model}
\label{TrainingML}
The training involved multiple steps, as shown in \autoref{fig:IRTraining}.
% Below we explain the procedure step by step.



\begin{figure*}[]
  \centering
  \includegraphics[width=0.9\linewidth]{figures-new/Fig9-Training2-new.jpg}
  \caption{ML training. (a) Sample printing. (b) Data capture with our smartphone module and labeling processes. (c) Data augmentation. (d) CNN training for the binarization task.}
  \Description{.}
  \label{fig:IRTraining}
\end{figure*}



\paragraph{Data capture}


To ensure our ML model generalized to various conditions, we printed and captured multiple conditions with our setup. We intended to print different types of barcodes so the trained model can excel at binarization as an image task, regardless of the barcode type or content. Thus, we printed samples of three types of barcodes: a QR code with low ECC, a QR code with high ECC, and a Data Matrix code. Per sheet, we printed only one type of barcodes.

On each US letter-size sheet, we had a total of 25 codes that were created by repositioning and rescaling one of these initial barcodes in a grid-like pattern (see \autoref{fig:IRTraining}).
Varying the size of the code creates a more robust training set so that the model can recognize codes that appear smaller from a distance, at different angles, or with lower resolution.
We varied the the module (bit) size from $1~mm$ to $2~mm$, in increments of $0.25~mm$ from left to right to allow our model to generalize across different resolutions.

Each sheet was printed three times, once for each of the three possible IR ink quantity classes, and a matching visual for the CMY color content (darker background, lighter background, or blank), resulting in a total of nine sheets..
Regardless of the ink amounts and combinations, the model is expected to learn to detect and binarize the codes.
We then photographed each sheet using the IR camera from an identical distance of $25.5~cm$ in order to capture the letter-size sheet (see Section~\ref{DetectionDistanceCapacity}).



\paragraph{Ground truth labeling}

Using \textit{Photoshop}, we binarized the images in order to create a black-and-white image representing the ground truth. This process involved manually adjusting each code using various transformations (e.g., perspective warp) to match the pose of the code on the photographed sheet.

\paragraph{Data augmentation}

Using \textit{MATLAB}, we augmented our dataset by 500 images for each original image by performing a combination of geometric and color space transformations, including randomized changes in brightness, contrast, hue, and saturation.
\autoref{fig:IRTraining}c exemplifies the samples we generate in order to simulate different conditions that might occur due to varying physical conditions. 
Each image was rescaled to 450x600 pixels for efficient ML training.
In total, our resulting dataset had 4,500 images.
Such data augmentation in ML also helps us enhance model robustness and prevents overfitting by enlarging the dataset.
 




\paragraph{Training}
For the ML training, we employed a convolutional neural network (CNN) similar to \cite{dogan_sensicut_2021} but used the \textit{U-Net}~\cite{ronneberger_u-net_2015} architecture with a pre-trained \textit{ResNet-34}~\cite{he_deep_2015} as its backbone for black-and-white segmentation (i.e., binarization).
Using \textit{PyTorch}~\cite{paszke_pytorch_2019} and \textit{fast.ai}~\cite{howard_fastai_2020}, we utilized the \textit{Adam} optimizer~\cite{kingma_adam_2017}, and used cross-entropy as loss function.
We trained the model with a batch size of 8 for 30 epochs and for about 20 minutes each,  until training loss settled at 0.05. 
Because this is not a classification task but rather segmentation (binarization), we do not compute an accuracy percentage to evaluate the performance for the sake of the eventual code detection.
The binarized output will be passed onto a standard barcode reader, as reported in the next section.
We evaluate how the combination of our ML model and barcode reader enables data storage capabilities in Section~\ref{EvaluationDataEmbedding}.









\subsubsection{Real-Time Inference}
\label{RealTimeInference}

We developed a \textit{Unity3D} application that interfaces with both the phone's standard RGB camera and our IR camera module in order to both extract the embedded data and display it in a visually appealing manner by means of AR augmentation of the tracked environment.

The application works by constantly running the ML model on the feed from the IR camera and then redirecting the output to a code reader library \textit{Dynamsoft SDK}\footnote{\url{https://www.dynamsoft.com/}} for extraction. The output from the reader consists of a list of the values of all detected QR codes and their coordinates.
In the case of offline embedding, we concatenate all results from the read QR codes in order to reconstruct the encoded string. In the case of online embedding, each QR code corresponds to a location on the data server that allows the application to download further instructions.



\section{Evaluation}
\label{evaluation}

\subsection{Simulated Evaluation of Multimodal Reference}
\label{sec:simulatedEvaluation}
To quantitatively evaluate how multimodal reference can help our human-in-the-loop optimization, we designed a simulation test to compare two linear subspace initialization methods: using multimodal reference and random fonts.
\subsubsection{Procedure}
We illustrate the procedure of the simulation test in \autoref{fig:multimodalInputIniitalizationEvaluation}. 
Given a base font character (\eg~``A''), the goal of the simulation test is to resemble the target font character (\autoref{fig:multimodalInputIniitalizationEvaluation}(d)) by exploring the style latent space through optimization.
Specifically, we simulate user selections using the following process. 
At each iteration, our method selects a point in the slider's search subspace with the minimum perceptual metric (we use \textit{DreamSim}~\cite{fu2023dreamsim}) against the target font character.
Then, the selected point is used to request Bayesian optimization to recommend the next linear subspace.
We iterate this process to observe the convergence of the optimization progress using both initialization methods.

For initializing using multimodal reference, we test \textit{text input} and \textit{font file input} in this experiment.
For the text input, we create a descriptive text that characterizes the target font and use it to initialize the search subspace.
For font file input, we manually select a font from candidate fonts that closely resembles the target font and use it to initialize the search subspace.
Finally, for the baseline method, we choose a font randomly from our font database and use it to construct the initial search subspace.

\begin{figure}[ht]
    \centering
    \includegraphics[width=\linewidth]{figures_pdf/synthetic_eval_v7.pdf}
    \caption{
    \textbf{Evaluation of linear subspace initialization methods.}
    We compared two initialization methods for exploration with Bayesian optimization.
    (a) One method uses input text or a similar font file for initialization, while (b) the other initialize method uses a randomly sampled font from a font database.
    After initialization, both methods follow the same automatic exploration process (c), where the optimal point on the single linear subspace is repeatedly identified and submitted to the system.
    In each iteration, we measure the distance between the generated character and the target font character to identify the optimal point, as shown in (d).
    Note that we use the bitmap format of the character for distance calculation, without vectorizing it.
    }
    \label{fig:multimodalInputIniitalizationEvaluation}
\end{figure}
We conducted this experiment using the character ``A'' for $10$ different target fonts, randomly selected from our font database.
We collected $12$ kinds of fonts from which we chose a similar font to each target font for the font file input.
For each target font, we choose the most similar font out of the $12$ candidate fonts.
The $12$ candidate fonts consist of two popular font families, \textit{Roboto} and \textit{NotoSerif}, and each font family has six variations: \textit{Light}, \textit{Light Italic}, \textit{Regular}, \textit{Regular Italic}, \textit{Bold}, and \textit{Bold Italic}.
This selection simulated a scenario where users start with popular fonts and design new fonts based on one of these similar candidates.
For each target font, the optimization process includes $10$ iterations of Bayesian optimization.

\subsubsection{Results}
In \autoref{fig:multimodalInputIniitalizationEvaluationResult}, we show the mean and standard deviation of the distances between the optimized results and all target fonts.
We can observe that the optimization processes with text and font file references converge to a lower \textit{DreamSim} distance to the target font character compared to those initialized with a randomly selected font.
These results indicate that using multimodal references for initializing the human-in-the-loop optimization leads to more effective exploration than random initialization.


\begin{figure}[ht]
    \centering
    \includegraphics[width=0.5\textwidth]{figures_pdf/multimodalInitialixationEvaluationResult_v4.pdf}
    \caption{
    \textbf{Convergence comparison between two initialization methods.}
    The figure illustrates how the \textit{DreamSim} distance between the designed font character and the target font character converges during exploration with human-in-the-loop optimization.
    The optimization processes initialized by text font references (orange) obtain better results compared to processes initialized by random font (blue).
    }
    \label{fig:multimodalInputIniitalizationEvaluationResult}
\end{figure}


\subsection{User Study}
\label{sec:user-study}
To evaluate the effectiveness of our proposed system, we conducted a user study in which participants were asked to design fonts using both a baseline system and our system. 
The goals of this study were threefold: 
\begin{itemize}
    \item to assess the overall effectiveness of our system, including the integration of Bayesian optimization, multimodal reference, history interface, and style propagation.
    \item to compare the fonts designed by participants both qualitatively and quantitatively against those created using the baseline system.
    \item to gather qualitative feedback on the user experience with our system.
\end{itemize}

\subsubsection{Comparison Systems}
For the user study, we added a special feature called \textsc{Font Palette} to our proposed system.
By clicking the \textsc{Font Palette} button, users can view a visualization of the \num{12} popular fonts described in \autoref{sec:simulatedEvaluation} and select one to input as their preference, simplifying the process of inputting a font file.
Additionally, we removed the \textsc{Upload Image} and \textsc{Upload Font} buttons from the UI in \autoref{fig:UI} for simplicity.
As a result, users can now easily input text and font files using the \textsc{Text} and \textsc{Font Palette} buttons, respectively.

To assess the effectiveness of the multimodal reference and style propagation features in our system, we created a baseline system that includes only a single slider, as illustrated in \autoref{fig:baselineSystemUI}.
In this baseline system, users can explore the font style latent space solely by adjusting the slider, guided by the Bayesian optimization process. 
Unlike our proposed system, the baseline’s one-dimensional search space is initialized by connecting a fixed point with a randomly initialized point.
The fixed point corresponds to the style of the \textit{IPAex Gothic} font, as described in \autoref{sec:simulatedEvaluation}. 
If users encounter difficulties during exploration, they can reset their preference history in the Bayesian optimization process and restart from a newly randomized search subspace.
Additionally, this baseline method lacks a style propagation function, requiring users to design each character individually.

\begin{figure}[ht]
    \centering
    \includegraphics[width=0.5\textwidth]{figures_pdf/baseline_ui_v6.pdf}
    \caption{
    \textbf{User interface of the baseline system.}
    In the (a) character design area, users use a slider to explore the one-dimensional subspace within the font style latent space recommended by Bayesian optimization.
    By clicking the \textsc{Reset} button, users can reset their preference history in the Bayesian optimization process, randomly reinitializing the search subspace.
    The users can check the characters that they have already designed are displayed in the (b) character collection area.
}
\label{fig:baselineSystemUI}
\end{figure}

\subsubsection{Procedure}
We recruited ten people for the user study.
Each participant was presented with a target font and asked to design three characters, ``A'', ``B'', and ``C'' that closely match the target font using both \systemName and the baseline system.
For this user study, we prepared two target fonts, Font 1 and Font 2.
Each font design session continued until one of the following conditions was met: (1) the participant was satisfied with the quality of the characters they designed, (2) they felt that further improvement was difficult, or (3) the $7$-minute time limit was reached.
The user study followed this sequence: (Tutorial of \systemName $\rightarrow$ Font 1 with \systemName $\rightarrow$ Font 2 with \systemName $\rightarrow$ Tutorial of baseline $\rightarrow$ Font 1 with baseline $\rightarrow$ Font 2 with baseline $\rightarrow$ Survey).
The order of using \systemName and the baseline system was randomized for each participant.
After the font design sessions, participants were asked to complete a questionnaire that validated our system.
The entire user study took approximately $60$ minutes, with each tutorial lasting $10$ minutes, each font design session $7$ minutes, and the survey $10$ minutes.



\subsubsection{Results and Discussion}
We compared the designed fonts using our system and the baseline system both quantitatively and qualitatively.
For the quantitative evaluation, we calculated the distance between the target font characters and the designed characters in the \textit{DreamSim} latent space.
As shown in \autoref{tab:userStudyResult}(a), the characters designed with our system closely resembled the target font characters compared to those designed with the baseline system.
Additionally, we measured the style consistency between all characters designed by each participant by calculating the mean distance between the characters ``A'', ``B'', and ``C'' in the \textit{DreamSim} latent space.
As shown in \autoref{tab:userStudyResult}(b), the distance is smaller when using our system to the baseline system, indicating our system enables more style-consistent character design.
In \autoref{fig:userStudyResult}, we showed the characters designed by all participants (P1--P10).
For Font 1, the ``A'' characters designed by P2, P3, P4, P6, and P8 using our system closely matched the slanted style of the target ``A,'' while they failed to design the slanted style using the baseline system, indicating that participants effectively captured the italic feature through multimodal reference.
On the other hand, characters designed by P1, P2, P3, P4, P5, P8, and P10 using the baseline system showed inconsistencies in style within the same font (\eg~variations in size, height, and weight).
In contrast, characters designed with our system exhibited greater consistency, suggesting that style propagation helped create more cohesive designs.

\begin{figure}[ht]
    \centering
    \includegraphics[width=\linewidth]{figures_pdf/fontDesignResult_v6.pdf}
    \caption{
    \textbf{Characters designed by user study participants.}
    Our system enables users to design characters that are more similar to the target font characters and maintain higher consistency between each other.
    In the case of Font 1, participants successfully designed all characters with the slant style using our system, while some participants failed to create the slant style for ``A'' using the baseline system.
    For Font 2, all participants designed characters with consistent styles using our system, whereas the styles of characters designed using the baseline system were inconsistent.
}
\label{fig:userStudyResult}
\end{figure}


\aptLtoX[graphic=no,type=html]{\begin{table}[ht]
\centering
\begin{tabular}{lll}
\multicolumn{3}{c}{\bf (a) Target font similarity $\downarrow$~~~~~}\\
\hline
                & Font 1          & Font 2          \\ \hline
Baseline & 0.1680          & 0.1416          \\
\systemName  & \bestcell{0.1591} & \bestcell{0.1355} \\ \hline
\end{tabular}
\begin{tabular}{lll}
\multicolumn{3}{c}{~~~~~\bf (b) Designed character consistency $\downarrow$}\\
\hline
                & Font 1          & Font 2          \\ \hline
Baseline & 0.3303          & 0.2893          \\
\systemName     & \bestcell{0.2983} & \bestcell{0.2793} \\ \hline
\end{tabular}
\caption{
(a) 
We calculated the distance between the characters designed by the participants and the target font characters.
Each value represents the mean distance across the $12$ characters (``A'', ``B'', ``C'' designed by the four participants).
The characters designed using our system are closer to the ground truth compared to those with the baseline system. 
(b)
We measured the character consistency between the characters ``A'', ``B'', and ``C'' designed by each participant.
Each value represents the mean distance across the three characters designed by each participant.
The distance among the three characters designed using our system is smaller than that with the baseline system, which indicates our system enables more style-consistent character design. 
($\downarrow$ denotes the lower values are better and we highlight the \besthint{best} result for each target font.)
}
\label{tab:userStudyResult}
\end{table}}{\begin{table}[ht]
\centering
\subfloat[Target font similarity $\downarrow$]{
\begin{tabular}{lll}
\toprule
                & Font 1          & Font 2          \\ \midrule
Baseline & 0.1680          & 0.1416          \\
\systemName  & \bestcell{0.1591} & \bestcell{0.1355} \\ \bottomrule
\end{tabular}
}
\subfloat[Designed character consistency $\downarrow$]{
\begin{tabular}{lll}
\toprule
                & Font 1          & Font 2          \\ \midrule
Baseline & 0.3303          & 0.2893          \\
\systemName     & \bestcell{0.2983} & \bestcell{0.2793} \\ \bottomrule
\end{tabular}
}
\caption{
(a) 
We calculated the distance between the characters designed by the participants and the target font characters.
Each value represents the mean distance across the $12$ characters (``A'', ``B'', ``C'' designed by the four participants).
The characters designed using our system are closer to the ground truth compared to those with the baseline system. 
(b)
We measured the character consistency between the characters ``A'', ``B'', and ``C'' designed by each participant.
Each value represents the mean distance across the three characters designed by each participant.
The distance among the three characters designed using our system is smaller than that with the baseline system, which indicates our system enables more style-consistent character design. 
($\downarrow$ denotes the lower values are better and we highlight the \besthint{best} result for each target font.)
}
\label{tab:userStudyResult}
\end{table}}

Next, we evaluated participant feedback to validate the effectiveness of our system.
We asked questions about the functions in our system, including slider operation, multimodal reference, style propagation, and history interface.
When we asked the question \textit{``Were you satisfied with the designed characters?''}, seven out of the ten participants answered yes, while P2 and P10 commented neutral, and P9 expressed no.
P9 noted that he observed distortions in the generated characters and felt the system was not good at generating straight lines.
In response to the question \textit{``Do you think you were able to design fonts easily with the system?''}, all ten participants answered yes, demonstrating the system's effectiveness in enabling non-expert users to design fonts with ease.

In response to the question \textit{``Do you think you were able to effectively use the slider operation for font design?''}, nine participants answered yes.
P4, who answered no, expressed dissatisfaction, stating that while the combination of slider manipulation and multimodal reference was effective, using only the slider and repeatedly clicking the \textsc{Update} button sometimes resulted in a linear subspace that excluded the desired character style.
P4 emphasized the importance of using multimodal reference at the right moments to avoid unsatisfactory suggestions and stated that relying solely on the slider was not effective.
P4 also highlighted that the history interface was useful for reverting to a previous point, leading to the escape of an undesirable search subspace suggested by the system.
P4's feedback reflects the findings suggested in Chan~\etal~\cite{Chan2022}, which indicate that designers working with BO may experience a loss of agency.
In contrast, our method provides users with a way to contribute concrete ideas that guide the BO process, thereby helping them regain a sense of agency.

In response to the question, ``\textit{Do you think you were able to effectively use text input?}'', eight of ten participants answered yes.
P1, P2, P4, P6, P7, P9, and P10 found text input helpful for making broad changes, such as adjusting weight or slant, but not for fine-tuning details or specifying complicated characteristics.
Additionally, P1, P6, and P7 mentioned that understanding typographical terms like ``bold'' and ``italic'' was necessary.
This feedback indicates that while text input is useful for exploring rough font styles, it has limitations in designing font details and requires some typographic knowledge.

Regarding the question, \textit{``Do you think you were able to effectively use the similar fonts provided by font palette?''}, eight of ten answered yes.
P4 and P7 commented that the font palette is particularly helpful when it is difficult to describe the desired font style in texts.
P8, P9, and P10 stated that initializing the search subspace using the font palette function allowed them to begin the design task more smoothly compared to the baseline system.
However, P5 expressed dissatisfaction, stating that the style of the character generated did not perfectly align with the font they selected from the font palette.
This discrepancy, caused by the encoding-decoding process of the font generative model, could lead to confusion among users.
To address this issue, it is important to communicate to users that the generated characters may not always perfectly match the multimodal reference.
Additionally, we anticipate that newer font generative models could help mitigate this discrepancy.
It is worth emphasizing that our proposed system is compatible with any font generative model, provided an efficient font style latent space can be established within it.
On the other hand, P2 explained that he did not use the font palette because he preferred to describe the target font style using text input.
This feedback suggests that using similar font files and text input complement each other.

When asked, \textit{``Do you think you were able to effectively use the \textsc{Update all} button?''}, nine participants responded positively, with eight participants noting that it was more convenient than designing each character individually.
P10, who answered no, expressed dissatisfaction, commenting that it would be more convenient if users could toggle between adjusting either all characters at once or individually. 
In particular, he felt that having a feature to switch to individual adjustments is crucial during the fine-tuning stage.
We focus on the simplicity of the UI in this user study and this individual adjustments function is effective especially when designing many characters like all Roman characters.

In response to the question, ``\textit{Do you feel that you could design characters with a sense of agency using our system?}'', posed only to P5--P10, all six participants responded affirmatively.
P7 and P10 noted that in the baseline system, the line search space was initialized randomly, making the process feel highly dependent on luck. 
In contrast, they appreciated that our proposed system allowed them to control the initialization by specifying their preferences through multimodal references.
This insight aligns with the findings in \autoref{sec:simulatedEvaluation}, which show the initialization using multimodal references leads to better results compared to random initialization.
Additionally, P9 commented that he felt he could convey his intentions to the system by inputting texts.
These insights indicate that our system, leveraging multimodal references, provides users with a greater sense of agency compared to the baseline system.

Seven participants highlighted the usefulness of the history interface during the design process.
P5 remarked that the feature was particularly effective, as there were times when he felt a previous font was better.
In such cases, the history interface allowed him to revisit and continue from that point, saving effort.
He also noted the inconvenience of the baseline system lacking this feature.
P6 commented that comparing the current font displayed on the slider with previously created fonts helped him determine which one aligned more closely with his intended design.
He also said that in the baseline system, he found it challenging to reset after creating a satisfactory font.
In contrast, our system's history interface made him feel more confident about updating or resetting, as it allowed him to aim for even better results without hesitation.
This exemplifies that the history interface is useful not only for storing the designed characters and enabling the users to go back to a past point but also for making them advance the design process as boldly as they want.
It also indicates that the history interface reduces stress and increases freedom and creativity in the design task.

Overall, the feedback suggests that the proposed functions in our system effectively support font design for different participants based on their design preferences and familiarity with typography.




\section{Use Cases and Future Applications}
In this section, we explore various applications of our watermarking technology designed for quick and integrated paper augmentation.
% By embedding AR interaction data directly within the physical document, \systemName~ offers a seamless blend of digital interactivity and traditional paper mediums. 
\new{We aim to demonstrate the breadth of use cases that \systemName~ can support and additional benefits that our method provides to users (e.g., invisible, mobile, privacy-preserving) across various contexts. Here, we present applications that are unique to \systemName, and we use replication for further validation \cite{Ledo2018Evaluation}.}



\subsection{Embedding "Paper DNA"}
\systemName~ could allow users to embed the digital metadata of the document as an intrinsic part of the physical paper. Utilizing this intrinsic "paper DNA," users could enable further authentication solutions. 



\begin{figure*}[h]
  \centering
  \includegraphics[width=1.0\linewidth]{figures-new/applicationNew.png}
  \caption{\systemName~ is used for making digital metadata an intrinsic part of the printed document to (a) recreate documents from fragments or (b) verify the authenticity of individual pieces. (c) The application augments normal vision by adding the AR content, providing additional information that would overload the limited space of a conference badge.}
  \Description{.}
  \label{fig:ApplicationsAR}
\end{figure*}




\subsubsection{Recreating Documents from Fragments}

The "Paper DNA" watermark allows fragments of documents to be scanned and linked back to the full digital copy. For example, if a poster for an educational event is partially torn, as shown in \autoref{fig:ApplicationsAR}a, a student can scan the remaining portion to retrieve the original poster and access the missing information. \new{The advantage of \systemName~ is that reconstruction can work even if the document is heavily fragmented. This is possible due to the high data capacity of a single marker that is sufficient to recreate the document. Because the information is repeatedly encoded without influencing the document's aesthetics, only one small part is necessary to retrieve its entire content.}
\newCameraReady{Maia et al. demonstrated this for 3D printed objects, where redundancy renders the information readable from a broken or damaged object to recover its original model
~\cite{maia_layercode_2019}, which inspired our application.}
\new{For paper materials, image-based tracking cannot easily achieve this because it would require training on fragmented documents, which does not scale well.}
We envision this feature can also benefit marketing strategies by including tearable sections in posters or ads with embedded links for future reference, \new{which are known to occupy too small space to include extra information, and \systemName~ can overcome these physical space constraints.}
 




\subsubsection{Imprinted Verification Signatures}
\systemName~ enhances document authenticity by embedding verification signatures in printed materials in combination with other techniques. For example, printed contracts could include invisible blockchain-based signatures or hyperlinks to the digital version for comparison against post-print modifications. If doubts arise, scanning with \systemName~ can confirm the document's originality. \new{The advantage is that this adds a second layer of "invisible" security by preventing people with malicious intent from reading names, passcodes, or signatures directly from the tickets, e.g., by shoulder surfing. This application exploits humans' unawareness of the presence of invisible IR content.}





\autoref{fig:ApplicationsAR}b shows an example where a friend group bought multiple tickets to a concert. The individual tickets, after being cut out from the sheet, each contain a verification code, which is scanned at the entrance of the venue. 


\subsection{Augmenting Documents with Contextual Annotations and Multimedia}
\systemName~ seamlessly integrates digital interactivity into traditional paper media,  from educational enhancements to professional networking solutions.




\subsubsection{Smart Business Cards and Conference Badges}
\systemName~ could be used for fabricating \newCameraReady{intelligent business cards or conference badges as envisioned by Want et al.~\cite{want_bridging_1999}}, which could, for example, also enable access to restricted areas, digital contact exchanges, and enhance networking experiences at professional gatherings without any electronic components.
For instance, at conferences, attendees could wear badges and exchange business cards enhanced with \systemName, as shown in \autoref{fig:ApplicationsAR}c. Scanning a badge reveals the attendee's professional profile, contact information, and links to their work, facilitating networking without the need for physical contact or exchanging multiple items.
\new{This works best if attendees know that all badges or business cards support IR content--thus, they know how to retrieve this information. Additionally, they can hide information they do not want to share immediately with anyone who passes by because it is directly visible on the badge.}



\subsubsection{Interactive Flashcards for Learning}

\systemName~ enhances traditional flashcards by integrating multimedia, aiding tasks like language learning.
\newCameraReady{We replicate certain functionalities of CV-based flashcards, which include pronunciation guides via simple scans, turning materials into interactive experiences that improve outcomes \cite{ng2018treasure}.} 
However, our application also supports offline access to hints and answers, minimizing online distractions. Students can reveal hints or check answers through an app, promoting independent, self-paced learning. \new{Flashcards have very restricted space, severely limiting the kinds of tracking markers that can be embedded. \systemName~ uses the entire space of the flashcard, i.e., front and back, even without any content, because it does not interfere with it. Nevertheless, image or text detection could be used to achieve the same. However, we would expect \systemName~ to perform better during occlusive hand interactions with the flashcards since we can embed a larger number of evenly distributed markers.}




\subsubsection{Offline High-Capacity Data Access for Sensitive Topics}

\systemName~ can discreetly embed critical information into materials, useful for raising awareness about stigmatized topics or promoting sensitive products (e.g., healthcare) while ensuring privacy. For instance, a health organization can distribute posters on mental health with embedded AR markers, allowing access to additional information, such as treatment options, via a simple scan, as shown in \autoref{fig:teaser}c, without leaving a digital footprint. \new{This demonstrates \systemName's key contribution: enabling invisible, high-capacity offline data embedding without requiring users to constantly send data to third-party servers. We also see potential in using offline high-capacity data access to enhance the accessibility of printed media, such as alt-texts, to images for visually impaired people without adapting visual designs to accommodate markers.}



\begin{figure*}[t]
  \centering
  \includegraphics[width=\linewidth]{figures-new/printOnDifferentThings.png}
  \caption{Envisioned extension of \systemName. \new{(a) A user wears AR smart glasses that capture \systemName~ content by exploiting the device's built-in always-on NIR cameras.} (b, c, d) Watermarking on different materials and surfaces to identify personal belongings and facilitate robotic navigation through invisible cues.}
  \Description{.}
  \label{fig:ApplicationsRobotic}
\end{figure*}





\vspace{0.2cm}
\new{\textbf{\systemName's Utility.} The examples in this section illustrate how IR watermarking can function as the "back end" for interactions with printed documents. Much like in web development or software engineering, this approach emphasizes a clear separation between the presentation layer (the "front end") and the data access layer (the "back end"). By abstracting and concealing technical, computer-readable components, our system offers a seamless, user-friendly interface while maintaining robust functionality. This separation enables new possibilities for interaction, surpassing the limitations of visual markers or standard QR codes in terms of abstraction and user experience. We argue that, if IR watermarks become the "back end" of printed media one day, it may pave the way for transformative applications without compromising the complexity and depth of the underlying technology.}







\subsection{Origami: Interactive Folding Instructions}
\systemName~ brings a twist to the traditional art of origami by embedding interactive, step-by-step folding instructions directly onto the paper. As users progress with their folds, new instructions appear on the exposed surfaces, as shown in \autoref{fig:teaser}b, based on the number and combination of the \systemName~ codes visible to the IR camera.
%This crafting experience could make complex designs more engaging and accessible for a larger audience.
The advantage of \systemName~ is that even if the origami paper has no visible content (i.e., white sheet) in order not impact the look of the final artifact, the sheet can still carry invisible folding instructions.

\subsection{Sustainable On-Demand Watermarking}
\systemName~ empowers users to watermark existing posters and documents larger than letter-size format on demand (see \autoref{fig:teaser}a). This may help to safe resources by avoiding re-printing posters or documents in case of missing content. Traditionally, people often go through multiple iterations until they arrive at their desired design. \systemName~ enables people to focus on visual aesthetics with the possibility of adding content wherever and whenever needed. 


\subsection{Beyond Paper Augmentation}
By using a mobile printer, \systemName~ could also add watermarks to plastic~\cite{ozdemir_speed-modulated_2024} (a), fabric~\cite{forman_defextiles_2020} (b), and wood~\cite{gui_draw2cut_2025} (c) and various form factors demonstrated in \autoref{fig:ApplicationsRobotic}. This is possible because the NIR ink is dye-based and can, therefore, be absorbed by these materials. In addition, our CNN detection pipeline can be fine-tuned for detecting, e.g., QR codes on convex or concave surfaces, highlighting the breadth of our contribution. This opens up new possibilities for embedding invisible identification markers to personal belongings (a) and (b). In \autoref{fig:ApplicationsRobotic}c, we also show how \systemName~ may be used for robotics use cases, in this case for robotic vacuum cleaners to safely avoid the pet's area.


\vspace{0.2cm}

Through these scenarios, \systemName~ shows its potential of integrating infrared inkjet watermarking into everyday interactions, and enabling solutions across educational, professional, and personal contexts.


\section{Discussion}


The value of hybrid paper-digital interfaces that augment physical documents with digital content has been well established~\cite{han_hybrid_2021}. To realize hybrid documents, we propose a novel watermarking method that uses IR-based printing to store digital content as an intrinsic part of the document. Unlike previous approaches that rely on a network connection and external storage for the digital assets, our approach with \systemName~ maintains the document format as the single container of both physical and digital contents. 
% \hl{This approach provides additional benefits that interactions are network-free and more privacy-preserving.}
In this section, we discuss the limitations of using \systemName~ from the perspectives of authoring, printing, detection, and consumption. We also discuss future work to improve the approach.



\subsection{Towards Multimodal Content}
\systemName's interface currently supports only a rudimentary form of AR that only permits 2D visual content and audio and does not leverage multimodal content to its full potential. We hope that we will enable authors to embed more intricate objects and interactions in the future. Adding support for multimedia assets will pave the way to expanding on previously examined use cases ~\cite{alessandrini_audio-augmented_2014, rajaram_paper_2022}. The most impactful addition would be that of 3D assets since that would allow users to utilize the document as a spatial anchor. 



\subsection{Effects on Printed IR Ink}




\new{Our experiments investigated human and machine detectability under controlled indoor lighting and viewing conditions.}
\newCameraReady{However, the perception and reliability of printed IR ink can vary significantly depending on external factors such as illumination levels, viewing angles, and the type of paper used.
One key limitation is that low-light conditions may reduce the effectiveness of human detectability due to insufficient IR reflection, while excessive illumination, such as direct sunlight, could lead to overexposure, affecting both machine readability and ink longevity. Future work could explore adaptive imaging techniques or optimized illumination setups to mitigate these issues. 
Regarding \textit{paper type}, we anticipate that glossy or coated paper may introduce reflection artifacts, which could interfere with machine-based recognition systems by creating specular highlights that obscure ink contrast.
Conversely, highly porous or rough-textured paper might cause ink diffusion, potentially reducing print sharpness and affecting detection accuracy.
A promising direction is to systematically evaluate these effects using an experimental framework similar to Xu et al.'s methodology, where QR code robustness was tested under varying lighting conditions, scanning angles, and environmental factors \cite{xu_art-up_2021}. Applying a similar protocol could help quantify how different conditions impact IR ink detection and visibility.}



As with any other dye-based ink, UV-dependent fading is an inherent limitation~\cite{maxmax_-_llewellyn_data_processing_ir_2022} that is important to consider for document reliability and permanence. Constant exposure to heavy UV sources such as the sun can cause the inks to fade over time.
\citet{willis_hideout_2013} showed how different IR ink types can be used in conjunction with UV-resistant coatings to achieve 98\% contrast preservation under office lighting conditions. We envision that for commercial use cases where long-term preservation is desired, a similar coating can be applied, which is sufficient for indoor applications.


\subsection{Invisible IR Ink}
We used a psychophysical experiment to determine conservative estimates for the IR ink densities that remain invisible to users for a wide range of different background colors. As with most experiments, we were limited by the number of background colors that we could include in the experiment. \new{Therefore, we do not know how our results generalize to other background colors, combinations of colors, and different graphic patterns. Nevertheless, we developed a system that facilitates embedding with invisible IR ink for any RGB color. Note that our method is very conservative and favors invisibility over machine detection. The actual DTs are likely much higher, which should further improve machine detectability. Nevertheless, we contribute a methodology that designers can apply to determine the "sweet spot" between visibility, data capacity, and machine detectability.}




We also want to highlight that the gradient IR ink and background color bars used in our psychophysical experiment differ from QR codes which may have had an effect on our estimated invisible IR ink DTs. We decided against using QR codes in the experiment because (1) it would have only been possible to test QR codes at predefined densities (e.g., at 20\%, 40\%, ...), limiting the precision of the study, and (2) the limited number of samples, which would not reflect the variability of QR codes of different sizes. As a result, our experiment also aims to provide general estimates for shapes other than QR codes and may, therefore, be used as a starting point for any type of invisible IR-printed marker.   



\subsection{Discoverability and Practicality}
\label{NIR_AR_FormFactor}

\new{
To demonstrate \systemName, we used NIR-based fabrication and detection tools. While NIR cameras are getting popular in many handheld devices (e.g., \textit{iPhone} and \textit{iPad} use it for facial recognition and LIDAR 3D scanning), not all platforms currently give 3rd-party developers access to the raw NIR stream (e.g., only the processed depth map can be accessed on \textit{iOS}). We argue the interest in NIR applications will increase as more use cases are demonstrated by future projects.}
% And we want to add this to AR headsets in the future, which already come with NIR cameras (conventionally used for depth sensing).

In our current implementation, \systemName~ documents can optionally be marked with a small icon or visual label on the document in our embedding tool
to allow users to discover embedded AR content.
\new{We envision that next-generation AR hardware can more fully leverage \systemName's utility. Compared to using a handheld device, always-on AR smart glasses such as \textit{Meta} \textit{Orion} could be constantly scanning the environment for hidden AR content~\cite{di_gioia_investigating_2022, campos_zamora_moirewidgets_2024}.
Currently, most AR glasses already leverage NIR cameras, but mainly for localizing the user and mapping their environment for 3D tracking purposes. With head-worn AR glasses with integrated NIR cameras used for \systemName~ sensing, the user would not have to manually capture objects using the phone during the interaction. We recommend future research to focus on implementing this on AR glasses.}



\section{Conclusion}


We introduced \systemName, a watermarking technique that embeds computer-readable information while remaining invisible to the human eye. Unlike previous methods, our approach utilizes the entire document, including white spaces, and works regardless of background color. Through a psychophysical experiment, we determined the maximum ink that can be embedded without being detected. We developed tools to support users in applying IR ink technology, including software for efficient information embedding and a universal camera module for capturing \systemName~watermarks. Our open-source ML pipeline processes these images for robust use with standard QR code readers. We demonstrated various use cases, highlighting the potential of invisible IR content for hybrid paper-digital interfaces and advancing watermarking techniques.





\bibliographystyle{ACM-Reference-Format}
\bibliography{ImmersiveDocs}


\onecolumn\appendix
\section{Appendix}\label{apedix: selection}
We provide a table for background color selection for the experiment and another table regarding the smallest module size and character capacity comparison.



\begin{table*}[h!]
\centering
\vspace{0.1cm}
\begin{tabular}{|c|c|c|c|}
\hline
\textbf{Group ID} & \textbf{Darkness}& \textbf{Calculated} & \textbf{Number of selected colors} \\
 & \textbf{{\small(analog $K$ value)}} & \textbf{luminescence ($lum$) range } & \textbf{{\small(roughly four colors per 25\% $lum$.)}} \\ \hline
1 & [0\%, 25\%] & (8.55\%, 100\%] & 16 \\ \hline
2 & [25\%, 50\%]& (5.7\%, 75\%] & 12 \\ \hline
3 & [50\%, 75\%] & (2.85\%, 50\%] & 8 \\ \hline
4 & [75\%, 100\%] & (0\%, 25\%] & 4 \\ \hline
5 & 100\% & 0\% & 1 \\ \hline
6 & \multicolumn{3}{|l|}{\small Four colors have been added to enhance color hue coverage.} \\ \hline
\end{tabular}
\vspace{0.1cm}
\caption[]{\new{Color selection table for the experiment. Colors for each tested group are chosen based on their darkness and luminescence ranges.}}\label{tab:color_coverage}
\vspace{-0.2cm}
\end{table*}\label{tab:studycolor}


% \subsection{Smallest Module Size and Character Capacity Comparison}


\begin{table*}[h!]
\centering
\small\begin{tabular}{ccccc}
\toprule
 & \textbf{Letter size (distance $\geq$ 25.5cm)} &  & \textbf{Half letter size (distance $\geq$ 19.5cm)} & \\
\cline{2-3}\cline{4-5}
\textbf{Ink Type} & \textbf{Smallest size} & \textbf{Maximum capacity} & \textbf{Smallest size} & \textbf{Maximum capacity}\\
\midrule
Low EEC 81\% ink & 1.27mm & 1800 characters & 1.27mm & 900 characters\\
Medium EEC 81\% ink & 1.18mm & 1600 characters & 0.93mm & 1224 characters\\
High EEC 81\% ink & 1.18mm & 800 characters & 0.93mm & 648 characters\\
Low EEC 102\% ink & 1.18mm & 2040 characters & 1.01mm & 1224 characters\\
Medium EEC 102\% ink & 1.18mm & 1600 characters & 0.93mm & 1224 characters\\
High EEC 102\% ink & 1.10mm & 1080 characters & 0.93mm & 648 characters\\
Low EEC 155\% ink & 1.18mm & 2040 characters & 0.93mm & 1560 characters\\
Medium EEC 155\% ink & 1.18mm & 1600 characters & 0.93mm & 1224 characters\\
High EEC 155\% ink & 1.10mm & 1080 characters & 0.93mm & 648 characters\\
\bottomrule
\end{tabular}
\vspace{0.1cm}
\caption{Comparison of module sizes and capacities at different ink levels and distances.}
\end{table*}\label{tab:inkcomparison}












\end{document}
\endinput
%%
%% End of file `sample-authordraft.tex'.
