% \RequirePackage{hyperref}
% % \usepackage{hyperxmp}  % Then load this

\documentclass[sigconf,screen]{acmart}

% \PassOptionsToPackage{bookmarksnumbered,unicode}{hyperref}
% \usepackage{hyperref}

\AtBeginDocument{%
  \providecommand\BibTeX{{%
    \normalfont B\kern-0.5em{\scshape i\kern-0.25em b}\kern-0.8em\TeX}}}

\setcopyright{acmcopyright}
\copyrightyear{2024}
\acmYear{2024}
\acmDOI{XXXXXXX.XXXXXXX}



\copyrightyear{2025}
\acmYear{2025}
\setcopyright{rightsretained}
\acmDOI{10.1145/3706598.3713286}
\acmISBN{979-8-4007-1394-1/25/04}
\acmConference[CHI '25]{CHI Conference on Human Factors in Computing Systems}{April 26--May 01, 2025}{Yokohama, Japan}
\acmBooktitle{CHI Conference on Human Factors in Computing Systems (CHI '25), April 26--May 01, 2025, Yokohama, Japan}



\usepackage{color}
\usepackage{soul}    % highlighting
\usepackage{gensymb} % degree symbol



\usepackage{graphicx}
 
\usepackage{enumitem} % remove indent of list items

\usepackage{tabularray}


%%% commands
\newcommand{\systemName}{\textsc{Imprinto}}



% \newcommand{\new}[1]{#1}
\newcommand{\new}[1]{\textcolor{black}{#1}} % color-code for revise and resubmit
\newcommand{\newCameraReady}[1]{\textcolor{black}{#1}} % color-code for camera-ready changes

\sloppy

\begin{document}


\title[\systemName: Enhancing Infrared Inkjet Watermarking Human and Machine Perception]{\systemName:  Enhancing Infrared Inkjet Watermarking for \newline Human and Machine Perception}

 

\author{Martin Feick}
\orcid{0000-0001-5353-4290}
\affiliation{%
 \institution{DFKI \& Saarland University}
 \city{Saarland}
 \country{Germany}}
\affiliation{%
  \institution{MIT CSAIL}
  \city{Cambridge}
  \state{Massachusetts}
\country{US}
}
 \email{martin.feick@dfki.de}
 
\author{Xuxin Tang}
\orcid{0009-0004-7997-6280}
\affiliation{%
 \institution{Virginia Tech}
 \city{Blacksburg}
 \state{Virginia}
 \country{US}}
  \affiliation{%
  \institution{MIT CSAIL}
  \city{Cambridge}
  \state{Massachusetts}
\country{US}
}
\email{xuxintang@vt.edu}


\author{Raul Garcia-Martin}
\orcid{0000-0001-5319-2016}
\affiliation{%
  \institution{Universidad Carlos III de Madrid}
  \city{Leganes, Madrid}
  \country{Spain}
}
\affiliation{%
  \institution{MIT CSAIL}
  \city{Cambridge}
  \state{Massachusetts}
  \country{US}
}
\email{raulgarc@ing.uc3m.es}

\author{Alexandru Luchianov}
\orcid{0009-0005-9595-0635}
\affiliation{%
  \institution{MIT CSAIL}
  \city{Cambridge}
  \state{Massachusetts}
  \country{US}
}
\email{lknv@mit.edu}

\author{Roderick Wei Xiao Huang}
\orcid{0009-0001-7120-6968}
\affiliation{%
  \institution{MIT CSAIL}
  \city{Cambridge}
  \state{Massachusetts}
  \country{US}
}
\email{rwxhuang@mit.edu}

\author{Chang Xiao}
\orcid{0009-0008-7143-2771}
\affiliation{%
  \institution{Adobe Research}
  \city{San Jose}
  \state{California}
  \country{US}
}
\email{cxiao@adobe.com}

\author{Alexa Siu}
\orcid{0000-0002-4879-1476}
\affiliation{%
  \institution{Adobe Research}
  \city{San Jose}
  \state{California}
  \country{US}
}
\email{asiu@adobe.com}

\author{Mustafa Doga Dogan}
\orcid{0000-0003-3983-1955}
\affiliation{%
  \institution{Adobe Research}
  \city{Basel}
  \country{Switzerland}
}
\affiliation{%
  \institution{MIT CSAIL}
  \city{Cambridge}
  \state{Massachusetts}
  \country{US}
}
\email{doga@adobe.com}

%%
%% By default, the full list of authors will be used in the page
%% headers. Often, this list is too long, and will overlap
%% other information printed in the page headers. This command allows
%% the author to define a more concise list
%% of authors' names for this purpose.
\renewcommand{\shortauthors}{Feick, et al.}

%%
%% The abstract is a short summary of the work to be presented in the
%% article. % 150 words
\begin{abstract}

Hybrid paper interfaces leverage augmented reality to combine the desired tangibility of paper documents with the affordances of interactive digital media.
Typically, virtual content can be embedded through direct links (e.g., QR codes); however, this impacts the aesthetics of the paper print and limits the available visual content space. %, and may even degrade accessibility. 
To address this problem, we present \systemName, an infrared inkjet watermarking technique that allows for invisible content embeddings only by using off-the-shelf IR inks and a camera.
\systemName~ was established through a psychophysical experiment, studying how much IR ink can be used while remaining invisible to users regardless of background color.
We demonstrate that we can detect invisible IR content through our machine learning pipeline, and we developed an authoring tool that optimizes the amount of IR ink on the color regions of an input document for machine and human detectability. 
Finally, we demonstrate several applications, including augmenting paper documents and objects. 
\end{abstract}

%%
%% The code below is generated by the tool at http://dl.acm.org/ccs.cfm.
%% Please copy and paste the code instead of the example below.
%%
\begin{CCSXML}
<ccs2012>
   <concept>
       <concept_id>10003120.10003121</concept_id>
       <concept_desc>Human-centered computing~Human computer interaction (HCI)</concept_desc>
       <concept_significance>300</concept_significance>
       </concept>
 </ccs2012>
\end{CCSXML}

\ccsdesc[300]{Human-centered computing~Human computer interaction (HCI)}

%%
%% Keywords. The author(s) should pick words that accurately describe
%% the work being presented. Separate the keywords with commas.
\keywords{augmented reality; mixed reality; infrared imaging; watermarking; digital fabrication}


%% A "teaser" image appears between the author and affiliation
%% information and the body of the document, and typically spans the
%% page.

\begin{teaserfigure}
  \includegraphics[width=\textwidth]{figures-new/Imprinto.png}
  \caption{\systemName~ are real-life documents that enable watermarking through the use of IR-absorbing ink. Users can apply it by (1) making it part of the document's fabrication process using an off-the-shelf inkjet printer or (2) by using a handheld inkjet printer on demand. The embedded content is invisible to users, even on blank paper, while allowing for high data capacity. Our system reliably decodes the IR content and \textit{is further instructed by it} regarding what images to track and what to display.}
  \label{fig:teaser}
\end{teaserfigure}



%%
%% This command processes the author and affiliation and title
%% information and builds the first part of the formatted document.
\maketitle


% humans are sensitive to the way information is presented.

% introduce framing as the way we address framing. say something about political views and how information is represented.

% in this paper we explore if models show similar sensitivity.

% why is it important/interesting.



% thought - it would be interesting to test it on real world data, but it would be hard to test humans because they come already biased about real world stuff, so we tested artificial.


% LLMs have recently been shown to mimic cognitive biases, typically associated with human behavior~\citep{ malberg2024comprehensive, itzhak-etal-2024-instructed}. This resemblance has significant implications for how we perceive these models and what we can expect from them in real-world interactions and decisionmaking~\citep{eigner2024determinants, echterhoff-etal-2024-cognitive}.

The \textit{framing effect} is a well-known cognitive phenomenon, where different presentations of the same underlying facts affect human perception towards them~\citep{tversky1981framing}.
For example, presenting an economic policy as only creating 50,000 new jobs, versus also reporting that it would cost 2B USD, can dramatically shift public opinion~\cite{sniderman2004structure}. 
%%%%%%%% 图1:  %%%%%%%%%%%%%%%%
\begin{figure}[t]
    \centering
    \includegraphics[width=\columnwidth]{Figs/01.pdf}
    \caption{Performance comparison (Top-1 Acc (\%)) under various open-vocabulary evaluation settings where the video learners except for CLIP are tuned on Kinetics-400~\cite{k400} with frozen text encoders. The satisfying in-context generalizability on UCF101~\cite{UCF101} (a) can be severely affected by static bias when evaluating on out-of-context SCUBA-UCF101~\cite{li2023mitigating} (b) by replacing the video background with other images.}
    \label{fig:teaser}
\end{figure}


Previous research has shown that LLMs exhibit various cognitive biases, including the framing effect~\cite{lore2024strategic,shaikh2024cbeval,malberg2024comprehensive,echterhoff-etal-2024-cognitive}. However, these either rely on synthetic datasets or evaluate LLMs on different data from what humans were tested on. In addition, comparisons between models and humans typically treat human performance as a baseline rather than comparing patterns in human behavior. 
% \gabis{looks good! what do we mean by ``most studies'' or ``rarely'' can we remove those? or we want to say that we don't know of previous work doing both at the same time?}\gili{yeah the main point is that some work has done each separated, but not all of it together. how about now?}

In this work, we evaluate LLMs on real-world data. Rather than measuring model performance in terms of accuracy, we analyze how closely their responses align with human annotations. Furthermore, while previous studies have examined the effect of framing on decision making, we extend this analysis to sentiment analysis, as sentiment perception plays a key explanatory role in decision-making \cite{lerner2015emotion}. 
%Based on this, we argue that examining sentiment shifts in response to reframing can provide deeper insights into the framing effect. \gabis{I don't understand this last claim. Maybe remove and just say we extend to sentiment analysis?}

% Understanding how LLMs respond to framing is crucial, as they are increasingly integrated into real-world applications~\citep{gan2024application, hurlin2024fairness}.
% In some applications, e.g., in virtual companions, framing can be harnessed to produce human-like behavior leading to better engagement.
% In contrast, in other applications, such as financial or legal advice, mitigating the effect of framing can lead to less biased decisions.
% In both cases, a better understanding of the framing effect on LLMs can help develop strategies to mitigate its negative impacts,
% while utilizing its positive aspects. \gabis{$\leftarrow$ reading this again, maybe this isn't the right place for this paragraph. Consider putting in the conclusion? I think that after we said that people have worked on it, we don't necessarily need this here and will shorten the long intro}


% If framing can influence their outputs, this could have significant societal effects,
% from spreading biases in automated decision-making~\citep{ghasemaghaei2024understanding} to reducing public trust in AI-generated content~\citep{afroogh2024trust}. 
% However, framing is not inherently negative -- understanding how it affects LLM outputs can offer valuable insights into both human and machine cognition.
% By systematically investigating the framing effect,


%It is therefore crucial to systematically investigate the framing effect, to better understand and mitigate its impact. \gabis{This paragraph is important - I think that right now it's saying that we don't want models to be influenced by framing (since we want to mitigate its impact, right?) When we talked I think we had a more nuanced position?}




To better understand the framing effect in LLMs in comparison to human behavior,
we introduce the \name{} dataset (Section~\ref{sec:data}), comprising 1,000 statements, constructed through a three-step process, as shown in Figure~\ref{fig:fig1}.
First, we collect a set of real-world statements that express a clear negative or positive sentiment (e.g., ``I won the highest prize'').
%as exemplified in Figure~\ref{fig:fig1} -- ``I won the highest prize'' positive base statement. (2) next,
Second, we \emph{reframe} the text by adding a prefix or suffix with an opposite sentiment (e.g., ``I won the highest prize, \emph{although I lost all my friends on the way}'').
Finally, we collect human annotations by asking different participants
if they consider the reframed statement to be overall positive or negative.
% \gabist{This allows us to quantify the extent of \textit{sentiment shifts}, which is defined as labeling the sentiment aligning with the opposite framing, rather then the base sentiment -- e.g., voting ``negative'' for the statement ``I won the highest prize, although I lost all my friends on the way'', as it aligns with the opposite framing sentiment.}
We choose to annotate Amazon reviews, where sentiment is more robust, compared to e.g., the news domain which introduces confounding variables such as prior political leaning~\cite{druckman2004political}.


%While the implications of framing on sensitive and controversial topics like politics or economics are highly relevant to real-world applications, testing these subjects in a controlled setting is challenging. Such topics can introduce confounding variables, as annotators might rely on their personal beliefs or emotions rather than focusing solely on the framing, particularly when the content is emotionally charged~\cite{druckman2004political}. To balance real-world relevance with experimental reliability, we chose to focus on statements derived from Amazon reviews. These are naturally occurring, sentiment-rich texts that are less likely to trigger strong preexisting biases or emotional reactions. For instance, a review like ``The book was engaging'' can be framed negatively without invoking specific cultural or political associations. 

 In Section~\ref{sec:results}, we evaluate eight state-of-the-art LLMs
 % including \gpt{}~\cite{openai2024gpt4osystemcard}, \llama{}~\cite{dubey2024llama}, \mistral{}~\cite{jiang2023mistral}, \mixtral{}~\cite{mistral2023mixtral}, and \gemma{}~\cite{team2024gemma}, 
on the \name{} dataset and compare them against human annotations. We find  that LLMs are influenced by framing, somewhat similar to human behavior. All models show a \emph{strong} correlation ($r>0.57$) with human behavior.
%All models show a correlation with human responses of more than $0.55$ in Pearson's $r$ \gabis{@Gili check how people report this?}.
Moreover, we find that both humans and LLMs are more influenced by positive reframing rather than negative reframing. We also find that larger models tend to be more correlated with human behavior. Interestingly, \gpt{} shows the lowest correlation with human behavior. This raises questions about how architectural or training differences might influence susceptibility to framing. 
%\gabis{this last finding about \gpt{} stands in opposition to the start of the statement, right? Even though it's probably one of the largest models, it doesn't correlate with humans? If so, better to state this explicitly}

This work contributes to understanding the parallels between LLM and human cognition, offering insights into how cognitive mechanisms such as the framing effect emerge in LLMs.\footnote{\name{} data available at \url{https://huggingface.co/datasets/gililior/WildFrame}\\Code: ~\url{https://github.com/SLAB-NLP/WildFrame-Eval}}

%\gabist{It also raises fundamental philosophical and practical questions -- should LLMs aim to emulate human-like behavior, even when such behavior is susceptible to harmful cognitive biases? or should they strive to deviate from human tendencies to avoid reproducing these pitfalls?}\gabis{$\leftarrow$ also following Itay's comment, maybe this is better in the dicsussion, since we don't address these questions in the paper.} %\gabis{This last statement brings the nuance back, so I think it contradicts the previous parapgraph where we talked about ``mitigating'' the effect of framing. Also, I think it would be nice to discuss this a bit more in depth, maybe in the discussion section.}






\section{Related Work}
%%%%%%%%%%%%%%
% Know-Item Retrieval and Query Simulation
%%%%%%%%%%%%%%
\subsection{Query Simulation and Know-Item Retrieval}

Query simulation methods have been used for various purposes, including document expansion \cite{nogueira2019docT5query} and synthetic test collection generation \cite{Rahmani24synthetic}. In the context of known-item retrieval, these methods have been explored to improve retrieval strategies \cite{OgilvieCallan03combining} and evaluation frameworks \cite{Azzopardi06testbeds, hagen2015corpus}.



%% Query Simulation
\textit{Simulating} the known-item queries has long been an active research area \cite{balog2006overviewWebclef, Azzopardi07SimulatedQueries, Kim09desktop, Elsweiler2011Seeding}.
Early work \cite{Azzopardi07SimulatedQueries} generated synthetic queries using term-based likelihood models, selecting query terms based on their likelihood within a randomly chosen document. Later studies adapted this approach for desktop search \cite{Kim09desktop} and email re-finding \cite{Elsweiler2011Seeding}, demonstrating its effectiveness for simulated evaluations of know-item retrieval models.
%
The \textit{validation} of these query simulators has also been a key focus.
System ranking correlation \cite{balog2006overviewWebclef}, retrieval score distribution comparisons \cite{Azzopardi07SimulatedQueries}, and synthetic versus human query resemblance \cite{Kim09desktop} have been used to assess their reliability.


While valuable, known-item search queries differ significantly from TOT queries, which are longer and more complex. Despite progress in simulating known-item queries, TOT retrieval remains unexplored. This paper bridges that gap by introducing novel TOT query elicitation methods and adapting established validation techniques \cite{zeigler2000theory} to ensure alignment with real-world queries, enabling scalable and accurate simulated evaluations.






%%%%%%%%%%%%%%
% TOT Datasets
%%%%%%%%%%%%%%
\subsection{TOT Datasets}
Several datasets have been developed to support research on TOT retrieval, primarily collected from online CQA platforms and focused on specific domains. MS-TOT \cite{arguello-movie-identification} was constructed from the \textit{IRememberThisMovie} website and human-annotated with tags in the Movie domain. It also includes qualitative coding of TOT queries and demonstrates significant room for improvement in current retrieval technologies for such information needs. Similarly, \citet{gameTOT} collected TOT queries from Reddit's \textit{/r/tipofmyjoystick} subreddit in the Game domain, providing coded tag information. Other datasets include Reddit-TOMT \cite{Bhargav-2022-wsdm}, focused on movies and books from Reddit's \textit{/r/tipofmytongue} subreddit; TOT-Music \cite{Bhargav23MusicTOT}, targeting the Music domain from the same subreddit; and Whatsthatbook \cite{lin-etal-2023-whatsthatbook}, sourced from \textit{GoodReads}, focused on the Book domain.



In response to the domain specificity of these datasets, recent efforts have aimed to expand TOT datasets across multiple areas. \citet{Meier21-complex-reddit} expanded to general casual leisure domains using data from six Reddit subreddits, including games, books, and music, although other identified domains, such as videos and people, remain underrepresented. Similarly, TOMT-KIS \cite{frobe2023-performance-pred} extended the collection from \textit{/r/tipofmytongue} by adapting \citet{Bhargav-2022-wsdm}'s approach with fewer filtering restrictions, resulting in 1.28 million TOT queries. However, only 47\% of these queries have identified answers, and the dataset continues to exhibit severe domain skewness toward a few topics. 


In this work, we develop and validate TOT query elicitation methods using the Movie domain for robust evaluation, then expand to Landmark and Person to assess applicability across underrepresented domains.



\section{Synthesizing Attribution Data}

\begin{figure*}[ht]
    \centering
    \includegraphics[width=\textwidth]{img/pipeline.drawio.pdf}
    \caption{\textbf{Top:} The \synatt baseline method for synthetic attribution data generation. Given context and question-answer pairs, we prompt an LLM to identify supporting sentences, which are then used to train a smaller attribution model. However, this discriminative approach may yield noisy training data as LLMs are less suited for classification tasks (see \S\ref{sec:experiments-zero-shot}). \textbf{Bottom:} The \synqa data generation pipeline leverages LLMs' generative strengths through four steps: (1) collection of Wikipedia articles as source data; (2) extraction of context attributions by creating chains of sentences that form hops between articles; (3) generation of QA pairs by prompting an LLM with only these context attribution sentences; (4) compilation of the final training samples, each containing the generated QA pair, its context attributions, and the original articles enriched with related distractors.}
    % \caption{\textbf{Top:} The \synatt baseline. Intuitively, we can prompt an LLM for context-attribution by providing the context and question-answer pairs. Then, we train a smaller model on the obtained synthetic data. However, LLMs are less suitable for discriminative (i.e., classification) tasks, and may yield noisy training data (see \S\ref{sec:experiments-zero-shot}). \textbf{Bottom:} The \synqa data generation pipeline consists of four main steps: (1) collection of Wikipedia articles as the source data; (2) extracting the context attributions by creating chains of sentences that form hops between articles; (3) generation of QA pairs by prompting an LLM with only the context attribution sentences; (4) we obtain the resulting \synqa training sample containing three components: the generated QA pair, the context attributions, and the original articles supplemented with related distractor articles.}
    \label{fig:method}
\end{figure*}

Context attribution identifies which parts of a reference text support a given question-answer pair~\cite{rashkin2023measuring}. Formally, given a question $q$, its answer $a$, and a context text $c$ consisting of sentences ${s_1, ..., s_n}$, the task is to identify the subset of sentences $S \subseteq c$ that fully support the answer $a$ to question $q$. To train efficient attribution models without requiring expensive human annotations, we explore synthetic data generation approaches using LLMs.
% Context attribution poses the following question~\cite{rashkin2023measuring}: given a generated text $t_g$ and a context text $t_c$, is $t_g$ attributable to $t_c$? To train models to perform well on this task, we explore how to best generate synthetic attribution data using LLMs. We implement two methods: a discriminative and generative method. 
We implement two methods for synthetic data generation. Our baseline method (\synatt) is discriminative: given existing question-answer pairs and their context, an LLM identifies supporting sentences, which are then used to train a smaller attribution model. Our proposed method (\synqa) takes a generative approach: given selected context sentences, an LLM generates question-answer pairs that are fully supported by these sentences. This approach better leverages LLMs' natural strengths in text generation while ensuring clear attribution paths in the synthetic training data.

%The first method is relatively straightforward and termed \synatt. A simple way to generate synthetic data for context attribution is to ask an LLM to pick out the sentences that support a given question-answer pair. 

% \subsection{Discriminative and Generative Synthetic Data Generation}

% The first method (\synatt) is relatively straightforward: ask the LLM to pick relevant sentences from a provided context that support a given question-answer pair. However, this \textit{discriminative} approach of performing sentence classification overlooks the fact that LLMs excel at \textit{generating} text. Therefore, we design a second data generation method (\synqa) that is generative and thus capitalizes on the strength of LLMs. It involves the following pipeline steps (see also Fig.~\ref{fig:method}): context collection, question-answering generation and distractor mining, which increases the difficulty of the task, thus reflecting more realistic scenarios.

%\textbf{Attribution Synthesis.} The most straightforward approach to generating synthetic data for context attribution is discriminative: prompting an LLM to identify relevant sentences from context documents given a question-answer pair. While intuitive, this approach underutilizes LLMs' capabilities, as they excel at generative rather than discriminative tasks. LLMs are fundamentally designed to generate coherent text following instructions rather than perform binary classification of sentences. In our experiments (\S\ref{sec:experiments}) we dub this method as \synatt.

\subsection{\synqa: Generative Synthetic Data Generation Method}

\synqa consists of three parts: context selection, QA generation, and distractors mining (for an illustration of the method, see Figure~\ref{fig:method}). In what follows, we describe each part in detail.

\textbf{Context Collection.} We use Wikipedia as our data source, as each article consists of sentences about a coherent and connected topic, with two collection strategies. In the first, we select individual Wikipedia articles for dialogue-centric generation and use their sentences as context. In the second, for multi-hop reasoning, we identify sentences containing Wikipedia links and follow these links to create ``hops'' between articles, limiting to a maximum of two paths to maintain semantic coherence, while enabling more complex reasoning patterns (for more details, see Appendix~\ref{app:synthetic_data}).
% \textbf{Context Collection.}  The first step is to select a dataset where each data point is a set of sentences about a coherent and connected topic. These sentences will serve as the context in which we want to find relevant attributions later. We use Wikipedia as the data source
%To better leverage LLMs' generative capabilities, we propose \synqa, a novel and simple approach for synthesizing context attribution data (see Fig.~\ref{fig:method}). 
%We first collect Wikipedia articles that are not present in our testing datasets\footnote{We detect potential data leakage by representing each Wikipedia article as a MinHash signature. Then, for each training Wikipedia article, we retrieve candidates from the testing datasets via Locality Sensitivity Hashing and compute their Jaccard similarity \cite{dasgupta2011fast}. Pairs exceeding a tunable threshold (empirically set to 0.8) are flagged as potential leaks.}.
%For each article, 
% we implement two distinct collection strategies that differ in difficulty. First, we select individual Wikipedia articles and randomly select multiple sentences within each article. Second, we start from a randomly selected sentence containing at least one Wikipedia link
%\footnote{These are human annotated in the Wikipedia articles, or alternatively, can be obtained from entity linking methods \cite{de-cao-etal-2022-multilingual}.} 
% and follow the links to other articles, creating ``hops'' between related content. We limit the chain to a maximum of two hops (connecting up to three articles) to maintain semantic coherence while enabling the more difficult multi-hop reasoning scenarios (for more details, see Appendix~\ref{app:synthetic_data}). 
%In the second strategy, we select individual Wikipedia articles and randomly select multiple sentences within each article that can serve as evidence for generated questions.

\textbf{Question-Answer Generation.} Given the set of contexts, an LLM can now generate question-answer pairs. For single articles, we prompt the model to generate multiple question-answer pairs, each grounded in specific sentences. This creates a set of dialogue-centric samples where questions build upon the previous context. For linked articles, we prompt the model to generate questions that necessitate connecting information across the articles, encouraging multi-hop reasoning.
%\footnote{Note that multi-hop reasoning is not guranteed here; rather, the LLM has the ability to decide whether the question-answer pair involves multiple hops of reasoning. See App. for details.}. 
This yields multi-hop samples requiring integration of information across documents, as well as samples that mimic a dialogue about a specific topic given the context. We provide the full prompts used for generation in Appendix \ref{app:prompts}.

\textbf{Distractors Mining.} To make the attribution task more realistic, we augment each sample with distractor articles. With E5 \cite{wang2022text}, we embed each Wikipedia article in our collection. For each article in the training sample, we randomly select up to three distractors with the highest semantic similarity to the source articles. These distractors share thematic elements with the source articles, but lack information to answer the questions.%do not contain the information necessary to answer the generated questions.

\subsection{Advantages of \synqa}
The \synqa approach has three key advantages:
%over discriminative data generation:
% (1) it leverages LLMs' natural strength in generative tasks; (2) produces diverse multi-hop reasoning scenarios; and (3) creates coherent question-answer pairs with clear attribution paths.
(1) it leverages LLMs' strength in generation rather than classification; (2) creates diverse training samples requiring both dialogue understanding and multi-hop reasoning; and (3) ensures generated questions have clear attribution paths since they are derived from specific context sentences.
By generating both entity-centric and dialogue-centric samples, \synqa produces training data that reflects the variety of real-world QA scenarios, helping models develop robust attribution capabilities, which our experiments demonstrate to generalize across different contexts and domains.
% We formalize the problem of Context Attribution QA as follows: Given a pre-defined context $T_c=\lbrace s_1, s_2, \ldots , s_n \rbrace$---where $s_i$ is a sentence---and an answer text $t_a$ generated by an LLM, the context attribution model should provide a vector $a=(a_1, \ldots , a_n)$, where each element $a_i$ has the following possible values:
% \[
% a_i =
% \begin{cases}
%     1, & \text{if } s_i \text{ supports the generated answer } t_a\\
%     0,  & \text{otherwise} 
% \end{cases}
% \]
% In our setup, we should have at least one entry $a_i = 1$.
% \begin{itemize}
%     \item The simplest way to generate synthetic data for context-attribution is in a discriminative manner: we prompt an LLM to provide the sentence level context attributions given the context documents, question and answer. We deem this generation as discriminative as the model effectively classifies the sentences that are most relevant to the question-answer pair.
%     \item The issue with this approach is that LLM are not best suitable for discriminative tasks, but rather generative. That is, an LLM is better at generating text by following instructions, than classifing sentences/etc.
%     \item To leverage what LLMs are good for, we create a simple context attribution data generation approach where we perform the following: (1) We find wikipedia articles (which are not contained in the testing datasets)\footnote{Describe the approach for dealing with data leakage}; (2) We select a random sentence in a wikipedia article, and find the links to other wikipedia articles (the hops). We select that sentence, and hop to the other Wikipedia article (given by the link). (3) We perform the hop step for maximum of 2 times (i.e., we connect at most 3 articles, and 1 at least). We end up with 3 Wikipedia articles which constitute the hops.
%     \item We provide Llama70B with either 1 wikipedia article or the hops and ask the model to generate a multi-hop question-answer pair which ideally connects all connected articles, or as many as it can; alternatively, if we provide the model with only 1 wikipedia article, we ask the model to select as many sentences as possible in the article, and for each, generate a question-answer pair (we provide the full prompts we use in Appendix).
%     \item The output of the model is a set of question-answer pairs (or a single one), that is grounded in the evidence provided by the sentence(s). We dub the entire approach as \synqa.
%     \item In summary, we develop two settings to generate synthetic data for context attribution in question answering: one is entity-centric and yield data which might be multi-hop; and the other is dialog-centric where subsequent questions build on top of previous ones.
%     \item Finally, to all context + question + answer + context-attribution samples we add distractors: we obtain embeddings using E5 of each wikipedia page, and for each sample we select up to 3 distractors which we add to the data sample. These distractors are similar are document with similar context as the one from which the context-attributions are.
% \end{itemize}


\section{Imprinto Watermarks}
In this section, we provide a comprehensive overview of the watermark embedding and detection process. The key steps involved are capturing the image, binarizing the image and extracting the QR code from the binarized image. Although extracting the QR code from an already binarized image can be done using commercially available software, the remaining necessary steps required the development of specialized modules. 




\subsection{Embedding the Watermark}
We developed a software tool to embed markers based on the system and the invisible IR ink values determined in our experiment in Section~\ref{visibility-experiment}. To use our tool shown in \autoref{fig:UIexample}, users first select the document for which they wish to create embeddings. Then, they choose whether to utilize the whole sheet for embedding or only designated areas. In the latter, the user marks on the document where they want to embed content, and the system interactively generates layouts. Finally, users decide whether to rely on internet information (online) or create a fully self-contained (offline) document.
The tool then analyzes the RGB values of the background color of the specified image region. It determines the conservative lower bound of the invisible IR ink density using the underlying system described in Section~\ref{visibility-experiment}.



%\begin{comment}
\paragraph{Online mode}
In the online mode, we take advantage of the ability to use the internet by hyperlinking multimedia content using QR codes.
In this menu (\autoref{fig:UIexample}b), the user can select the paper area that they wish to track using AR. An image score is computed for each selection using \textit{Google arcoreimg} tool\footnote{\url{https://developers.google.com/ar/develop/augmented-images/arcoreimg}} to ensure proper tracking and content overlay in AR.
In case the image is too difficult to track (i.e., the score is lower than 75), we inform the user and prompt them to choose a different area. Our tool allows users to associate each AR-tracked image with online multimedia content, such as text, image, and/or audio.
For each tracked image, the user can link a series ("frames") of these types of contents that users can loop through in AR. The underlying technology can be further generalized and used in order to incorporate complex multimedia content (such as videos or 3D models) or engineer further interactions between the objects. 
% Each tracked image is associated with a cyclical list of frames. A frame is characterized by a text file, an image to display, and a sound that can optionally play during the subsequent transition. 

%In the standard Unity application, we display the current frame above the tracked image and transition to the next frame every time it is clicked.%



\paragraph{Offline mode}
In the offline mode (\autoref{fig:UIexample}c), we are currently limited to text-only content due to multimedia content requiring significantly more space.
However, this capacity that offline embedding offers can be used to embed hyperlinks or XML media, such as contact cards, or stylized and interactive text content via markup language tags of the user's choice.
Even though we offer an automatic analysis of the document and calculate a possible way of embedding the information, we further allow the user to fine-tune these settings (i.e., size, error correction level, and version of the QR codes) according to their specific requirements.






\subsection{Capturing the Watermark}



\new{\systemName~ media can be captured using devices with NIR image sensors. There are two main methods by which users can do this. First, more conventionally, users can point their devices at the physical media to extract \systemName~ content. This is particularly suitable for settings where the users are aware that the content is already embedded (e.g., an instructor informing students about \systemName~ verbally before handing out worksheets, or through a small visual cue on them; a book or magazine mentioning the AR functionality only on its cover---so not each page needs to carry a notice). 
Second, there are other types of devices that have constant scanning functionality~\cite{dogangun_rampa_2024, iyer_xr-penter_2025}. Emerging devices, such as smart glasses (see \textit{Meta Orion}\footnote{\url{https://about.meta.com/realitylabs/orion}}),  feature "always-on" IR scanning functionality to map the environment that we envision can be repurposed to detect human-invisible IR markers (see \autoref{fig:ApplicationsRobotic}). This, in turn, would enable the always-on and seamless capture of these markers without much physical effort. However, accessibility to the technology is not yet widely available, discussed in  
Section~\ref{NIR_AR_FormFactor}.} 

\new{To support this emerging technology, we developed a \textit{universal} \systemName~ detection and reader module for mobile devices.} %: the RGMVision InfraRedCAM 1.
The module was built to decode document watermarks based on the inkjet ink absorbance characteristics (Section~
\ref{spectrum}) by featuring a 5MP CMOS NIR camera and two controllable NIR LEDs, both operating at a wavelength of $850~nm$. The sensor module integrates a bandpass filter centered at the same wavelength. We designed and developed the PCB and the aluminum enclosure for mechanical and electromagnetic protection produced by \textit{Megacal}\footnote{\url{https://megacal.es/}}, with a size of $59.5\times33.0\times11.6~mm$. 
Figure~\ref{fig:RGMCameraModule} shows the components of the module.
Our proprietary module, \textit{RGMVision InfraRedCAM 1}\footnote{\url{https://www.rgmvision.com/product/rgmvision-infraredcam-1/}}, ensures universal mobile connectivity through \textit{USB 2.0} or \textit{3.0} and its \textit{USB-C} connector. %and is compatible with \textit{Android}/\textit{iOS} mobile devices, as well as \textit{Windows}/\textit{macOS}/\textit{Ubuntu}/\textit{Raspberry Pi OS} laptops and desktop computers.
As opposed to related works, this allows our device to offer universal compatibility, converting any mobile device, smartphone, laptop, or PC into an \systemName~ reader via its \textit{USB-C} support. Additionally, the two diffuse NIR LEDs (80°), controlled by two PWM signals, provide 24 different adjustable illumination levels for indoor applications.


\begin{figure}[]
  \centering
  \includegraphics[width=0.47\textwidth]{figures-new/Fig6-RGMVision_InfraredCAM_1-new.jpg}
  \caption{% Developed USB smartphone reader – RGMVision InfraredCAM 1 –
  Our mobile imaging attachment comes with a 5MP sensor and two controllable NIR LEDs working at $850~nm$. (a) Module directly plugged into the mobile device. (b) Module coupled and lined with the back smartphone camera through a USB extender. (c) Front view of the module with the NIR LEDs powered up on a different device. (d) 3D exploded view, including the sensor, the LEDs mounted in the PCB, and the aluminum mechanical and electromagnetic enclosure.}
  \Description{.}
  \label{fig:RGMCameraModule}
\end{figure}



We developed an \textit{Android} application (built for API level 26 and targeted for API level 30), based on \textit{Unity3D}, interfaces with the module through the \textit{Unity3D} plugin \textit{UVC4UnityAndroid v2.0}\footnote{\url{https://github.com/saki4510t/UVC4UnityAndroid}}.




\subsection{Decoding the Watermark}

We developed an ML pipeline to binarize the captured IR watermarks for robust detection, which we will make open-source. Because the IR camera captures the watermarks in a faint tone, off-the-shelf QR code readers are unable to recognize \systemName~ codes. Thus, we need an intermediary step where the contrast of the IR watermarks is increased via binarization, where the codes are converted into black-and-white patterns similar to conventional QR codes. Previous works~\cite{dogan_standarone_2023} have used traditional image processing filters for this step, whereas we use an ML-focused approach to increase robustness (Section~\ref{EvaluationDataEmbedding}). 
We next explain the individual steps that were necessary for building our pipeline, as well as how to use it in real time.


\subsubsection{Training the ML model}
\label{TrainingML}
The training involved multiple steps, as shown in \autoref{fig:IRTraining}.
% Below we explain the procedure step by step.



\begin{figure*}[]
  \centering
  \includegraphics[width=0.9\linewidth]{figures-new/Fig9-Training2-new.jpg}
  \caption{ML training. (a) Sample printing. (b) Data capture with our smartphone module and labeling processes. (c) Data augmentation. (d) CNN training for the binarization task.}
  \Description{.}
  \label{fig:IRTraining}
\end{figure*}



\paragraph{Data capture}


To ensure our ML model generalized to various conditions, we printed and captured multiple conditions with our setup. We intended to print different types of barcodes so the trained model can excel at binarization as an image task, regardless of the barcode type or content. Thus, we printed samples of three types of barcodes: a QR code with low ECC, a QR code with high ECC, and a Data Matrix code. Per sheet, we printed only one type of barcodes.

On each US letter-size sheet, we had a total of 25 codes that were created by repositioning and rescaling one of these initial barcodes in a grid-like pattern (see \autoref{fig:IRTraining}).
Varying the size of the code creates a more robust training set so that the model can recognize codes that appear smaller from a distance, at different angles, or with lower resolution.
We varied the the module (bit) size from $1~mm$ to $2~mm$, in increments of $0.25~mm$ from left to right to allow our model to generalize across different resolutions.

Each sheet was printed three times, once for each of the three possible IR ink quantity classes, and a matching visual for the CMY color content (darker background, lighter background, or blank), resulting in a total of nine sheets..
Regardless of the ink amounts and combinations, the model is expected to learn to detect and binarize the codes.
We then photographed each sheet using the IR camera from an identical distance of $25.5~cm$ in order to capture the letter-size sheet (see Section~\ref{DetectionDistanceCapacity}).



\paragraph{Ground truth labeling}

Using \textit{Photoshop}, we binarized the images in order to create a black-and-white image representing the ground truth. This process involved manually adjusting each code using various transformations (e.g., perspective warp) to match the pose of the code on the photographed sheet.

\paragraph{Data augmentation}

Using \textit{MATLAB}, we augmented our dataset by 500 images for each original image by performing a combination of geometric and color space transformations, including randomized changes in brightness, contrast, hue, and saturation.
\autoref{fig:IRTraining}c exemplifies the samples we generate in order to simulate different conditions that might occur due to varying physical conditions. 
Each image was rescaled to 450x600 pixels for efficient ML training.
In total, our resulting dataset had 4,500 images.
Such data augmentation in ML also helps us enhance model robustness and prevents overfitting by enlarging the dataset.
 




\paragraph{Training}
For the ML training, we employed a convolutional neural network (CNN) similar to \cite{dogan_sensicut_2021} but used the \textit{U-Net}~\cite{ronneberger_u-net_2015} architecture with a pre-trained \textit{ResNet-34}~\cite{he_deep_2015} as its backbone for black-and-white segmentation (i.e., binarization).
Using \textit{PyTorch}~\cite{paszke_pytorch_2019} and \textit{fast.ai}~\cite{howard_fastai_2020}, we utilized the \textit{Adam} optimizer~\cite{kingma_adam_2017}, and used cross-entropy as loss function.
We trained the model with a batch size of 8 for 30 epochs and for about 20 minutes each,  until training loss settled at 0.05. 
Because this is not a classification task but rather segmentation (binarization), we do not compute an accuracy percentage to evaluate the performance for the sake of the eventual code detection.
The binarized output will be passed onto a standard barcode reader, as reported in the next section.
We evaluate how the combination of our ML model and barcode reader enables data storage capabilities in Section~\ref{EvaluationDataEmbedding}.









\subsubsection{Real-Time Inference}
\label{RealTimeInference}

We developed a \textit{Unity3D} application that interfaces with both the phone's standard RGB camera and our IR camera module in order to both extract the embedded data and display it in a visually appealing manner by means of AR augmentation of the tracked environment.

The application works by constantly running the ML model on the feed from the IR camera and then redirecting the output to a code reader library \textit{Dynamsoft SDK}\footnote{\url{https://www.dynamsoft.com/}} for extraction. The output from the reader consists of a list of the values of all detected QR codes and their coordinates.
In the case of offline embedding, we concatenate all results from the read QR codes in order to reconstruct the encoded string. In the case of online embedding, each QR code corresponds to a location on the data server that allows the application to download further instructions.



\section{evaluation}
% justification
We conducted an exploratory evaluation of \textit{Polymind}, focusing on the usability, creativity, and usefulness of its parallel collaboration workflow.
More specifically, the study aimed to answer the following two questions:
\begin{enumerate}
    \item Is \textit{Polymind} easy to use, and useful for prewriting?
    \item How effective are \textit{Polymind}'s parallel collaboration workflow and microtasking features for supporting creativity in prewriting?
\end{enumerate}

% 
\begin{table}[htbp!]
\resizebox{\columnwidth}{!}{%
\begin{tabular}{@{}l|ccc|c@{}}
 & Liberal & Moderate & Conservative & Total \\ \hline
Female & 223 & 114 & 45 & 382 \\
Male & 102 & 78 & 53 & 233 \\
Prefer not to say & 2 & 0 & 0 & 2 \\ \hline
Total & 327 & 192 & 98 & 617
\end{tabular}%
}
\caption{Annotator Demographics. All annotators are based in the United States. The table shows the number of annotators across ideology and sex categories, as self-reported to Prolific. The mean age is 38.3 (SD=12.7), and 45 annotators are immigrants (7.3\%).}
\label{tab:demographics}
\end{table}



\subsection{Participants}
We used convenience sampling to recruit 10 participants (4 male, 6 female) mainly from local universities. All participants are L2 English speaker.
\revision{Similar to our formative study, we mainly reached out to participants with creative writing experience or related majors, though not necessarily expert writers. All participants had experience using prewriting or diagramming tools such as Figma or Miro, but reported limited knowledge or experience of AI or programming.}
We refer to them as V1-10. For each participant, we offered a coupon equivalent to 50 HKD.

\subsection{Study Design}

\subsubsection{Tasks}
To evaluate our \textit{Polymind} in both divergent and convergent thinking phases, we divide a creative pre-writing task into two sessions: story ideation, and story outlining. In the story ideation, participants were required to brainstorm as many distinct storylines as possible. Each storyline only needs to be one or two sentences long that specifies main characters, events, locations, time, etc. In the story outlining session, participants were asked to pick one favourite storyline from the previous session, and draft a rough outline with as many details as possible. An outline needs to specify a clear structure (such as the classic beginning-climax-ending structure), and key events along the structure.

\subsubsection{Conditions}
The study compared our system, \textit{Polymind} to a turn-taking, conversational interface, plus \textit{Polymind}'s diagramming canvas, as a baseline using two creative writing prompts. \revision{The baseline allows users to take notes and keep track of conversational results using the canvas, but does not require nor allow users to interact with the AI via diagrams}. Specifically, two system conditions were used:
\begin{itemize}
    \item \textbf{GPT-4 \& Canvas} OpenAI ChatGPT-4 interface plus \textit{Polymind}'s diagramming canvas. All microtasking features were turned off.
    \item \textbf{\textit{Polymind:}} full version of \textit{Polymind} with six predefined default microtasks.
\end{itemize}
Two creative writing prompts were chosen:
\begin{itemize}
    \item Write a story where your character is traveling a road that has no end, either literally or metaphorically.
    \item Write a story in which a character is running away from something, literally or metaphorically.
\end{itemize}
We used a Latin square experimental design~\cite{ryan2007modern} to achieve a balanced sequence of writing prompts and system conditions.

\subsubsection{Study Procedure}
After giving consent to our study, users were invited to use two systems in turn to complete two sessions: story ideation and story outlining, given two different writing prompts. Each session lasted 12 minutes, and users were given time to transfer their prewriting results (generations, diagrams, or merely thoughts and ideas in their minds) to another document after each session concluded. Before using \textit{Polymind}, we walked the participants through all of its features and offered approximately 10 minutes for them to try out the system.

After two sessions concluded for a system condition, each participant was required to complete a survey, including a NASA Task Load Index (NASA-TLX)~\cite{hart1986nasa}, and three dimensions (2 questions each) of Creativity Support Index (CSI)~\cite{cherry2014quantifying}: Enjoyment, Exploration, and Expressiveness. Two dimensions (Collaboration \& Immersion) of the original CSI were dropped because they were irrelevant to the two questions we sought to answer.

To evaluate the final results (outlines), we invited two expert writers to score participants outlines using Torrance Test of Creative Writing (TTCW)~\cite{chakrabarty2023art}, instead of the self-rated score of CSI. \revision{We did a quick interview after the scoring to ask about their general feedback, and to compare results of two conditions and pick out examples with most noticeable differences.}
One of our expert is a professional fiction writer and has a doctoral degree in film studies. She used to be a screenwriter before becoming a fiction writer. The other expert is an AO3 (Archive of Our Own) writer that has posted over 400K words and accumulated over 250K views.

After using two system conditions (4 sessions in total), each participant was then required to complete a survey to rate the usefulness of each \textit{Polymind} features. We then conducted a brief interview (5-10 min) to ask about their overall use experience, feedback on \textit{Polymind}'s workflow, perceptions of creativity support, and perceived differences between the two workflows and their impact on the final results.

The whole study procedure lasted around 2 hours, and was screen recorded. The interviews were audio-taped and transcribed for analysis.

% % \begin{figure}[htb]
% \begin{subfigure}[h]{\linewidth}
%     \centering
%     \includegraphics[width=.64\linewidth]{figures/Polymind_NASA-TLX.png}
% \end{subfigure}
% \begin{subfigure}[h]{\linewidth}
%     \centering
%     \includegraphics[width=.64\linewidth]{figures/Baseline_NASA-TLX.png}
% \end{subfigure}
% \caption{NASA Task Load Index of the \textit{Polymind} and Baseline conditions (the lower, the better).}
% \label{fig:usability}
% \end{figure}

\begin{figure*}[htb]
\centering
\includegraphics[width=.64\linewidth]{figures/NASA_TLX.pdf}
\caption{NASA Task Load Index of \textit{Polymind} and Baseline conditions (the lower, the better).}
\label{fig:usability}
\end{figure*}

\subsection{Study Results}
In this subsection, we report the findings of our study. On balance, \textit{Polymind}'s parallel microtasking workflow granted more customizability and was more controllable. Therefore users reported a stronger sense of agency, ownership of results, and a higher level of expressiveness. The microtasking workflow could also help quickly expand idea trees through ``chaining''-like effects~\cite{wu2022ai}.

% \begin{figure}[htb]
% \begin{subfigure}[h]{\linewidth}
%     \centering
%     \includegraphics[width=.64\linewidth]{figures/Polymind_NASA-TLX.png}
% \end{subfigure}
% \begin{subfigure}[h]{\linewidth}
%     \centering
%     \includegraphics[width=.64\linewidth]{figures/Baseline_NASA-TLX.png}
% \end{subfigure}
% \caption{NASA Task Load Index of the \textit{Polymind} and Baseline conditions (the lower, the better).}
% \label{fig:usability}
% \end{figure}

\begin{figure*}[htb]
\centering
\includegraphics[width=.64\linewidth]{figures/NASA_TLX.pdf}
\caption{NASA Task Load Index of \textit{Polymind} and Baseline conditions (the lower, the better).}
\label{fig:usability}
\end{figure*}

\subsubsection{Usability \& Usefulness}
Despite efforts of microtask and diagram management and the potential learning curve, to our surprise, \textit{Polymind} was almost perceived as easy to use as the ChatGPT interface, and significantly reduced frustration towards generated results (as shown \autoref{fig:usability}). ChatGPT interface was demanding mainly due to efforts of digesting longer text information (e.g., V4-5), and typing and iteratively refining lengthy prompts (e.g., V1-2). \revision{For \textit{Polymind}, the main cause of demand was said by V2-3, \& V10 to be the efforts of mannually managing the canvas, adjusting its layout, and progressing through diagrams.}
Notably, V1 \& V2 said that \textit{Polymind}'s interface was easier to navigate, and easier to read, because its generations were mainly short phrases, and had structures (including lines, sections, \& microtask colours). 
Besides, the randomness of ChatGPT generations also caused higher frustration level and worse perceived performance than \textit{Polymind} among some participants (e.g., V5, V8).

The key features of \textit{Polymind} were mainly perceived useful for prewriting, as shown in \autoref{fig:usefulness}. Of them awareness-related features, such as notifications, previews, and initiative modes, were found most controversial, which revealed the tension of our \textit{Goal 2.1} and being overall non-intrusive. V3 felt a proactive microtask was particularly annoying and intrusive, but we observed that all other participants left key microtasks proactive. Some said (e.g., V5, V7) they would need proactive microtasks in divergent thinking phases for quick ideas, but sometimes did not want to be interrupted while thinking. Therefore, most participants thought the feature of switching intiative modes particularly helpful.

% \begin{figure}[ht!]
\centering
\begin{minipage}[b]{.48\linewidth}
    \vspace{0pt}
    \includegraphics[width=\linewidth]{figures/NASA_TLX.png}
    \caption{NASA Task Load Index of \textit{Polymind} and Baseline conditions (the lower, the better).}
    \label{fig:usability}
\end{minipage}
\begin{minipage}[b]{.5\linewidth}
    \vspace{0pt}
    \includegraphics[width=\linewidth]{figures/perceived_usefulness.png}
    \caption{The perceived usefulness of \textit{Polymind} features}
    \label{fig:usefulness}
\end{minipage}
\end{figure}
\begin{figure*}[htb]
\centering
\includegraphics[width=.85\linewidth]{figures/perceived_usefulness.png}
\caption{The perceived usefulness of \textit{Polymind} features}
\label{fig:usefulness}
\end{figure*}

% \begin{figure*}[htb]
% \begin{subfigure}[h]{.49\textwidth}
%     \includegraphics[width=.975\linewidth]{figures/Statistics of Resulting Diagrams - Task 1.pdf}
% \end{subfigure}
% \begin{subfigure}[h]{.49\textwidth}
%     \includegraphics[width=.975\linewidth]{figures/Statistics of Resulting Diagrams - Task 2.pdf}
% \end{subfigure}
% \caption{Number of resulting nodes and \textit{Polymind} contribution in two tasks}
% \label{fig:result_nodes}
% \end{figure*}

\subsubsection{Creativity Support}
In terms of creativity, participants generally felt \textit{Polymind} was more supportive (see \autoref{fig:CSI}), but the results were not significant ($P_{Enjoyment}=0.67$, $P_{Exploration}=0.19$, $P_{Expressiveness}=0.05$). Notably, the expressiveness dimension has almost shown significance, as many (e.g., V3 \& V5) reported that \textit{Polymind}'s diagramming interface and microtasking workflow put them in dominant roles that encouraged them to freely express their brief ideas. In terms of results, two conditions produced similar number of ideas ($Polymind_{median}=3$, $Polymind_{stdev}=1.06$, $Baseline_{median}=3$, $Baseline_{stdev}=8.51$) in the ideation session.

In addition, experts' scores showed that the baseline condition produced outlines that were able to pass 5.7 TTCW tests, as compared to \textit{Polymind}'s 4 tests ($P=0.23$) (see \autoref{fig:CSI}).
\revision{
Experts did not particularly mention any noticeable differences in quality between two conditions except that \textit{Polymind}'s results were much shorter and lacked details to pass some tests. Besides, they both expressed concern of overused or clichéd results. One expert said she was initially interested by V2's story (baseline), but only to find out that it was from \textit{The Vampire Diaries}
}
This is expected, as ChatGPT interface could quickly generate ``\textit{complete and detailed outlines with simple prompts}'' (V2), while \textit{Polymind} usually encouraged users to make progress in diagrams with limited words.
Although some (e.g., V8) noted that they could still generate a complete outline using \textit{Polymind}, but they simply did not want to, because they would like to take control, and create a story from their own fragmented ideas, instead of borrowing all results from ChatGPT.
\begin{figure}[ht!]
\centering
\begin{minipage}[t]{.4745\linewidth}
    \vspace{0pt}
    \includegraphics[width=\linewidth]{figures/CSI.png}
\end{minipage}
\begin{minipage}[t]{.32\linewidth}
    \vspace{0pt}
    \includegraphics[width=\linewidth]{figures/TTCW.png}
\end{minipage}
\caption{The results of Creativity Support Index (CSI) and Torrance Test of Creative Writing (TTCW)}
\label{fig:CSI}
\end{figure}

\subsubsection{Microtask Usage: Quick Chaining and Idea Expansion}
\revision{All default microtasks have been applied by 10 participants to produce their final results, as shown in \autoref{tab:usage}. In some cases users might have default microtasks slightly edited. For example, V2 changed the prompt of \BboxS{\textcolor{white}{Brainstorm}} and switched the output type to \textbf{\textcolor{sticky_note}{\textit{sticky note}}} to request detailed settings of a story. Four participants have delegated a total of 8 customized microtasks. For example, V8 delegated \CboxS{\textcolor{white}{Beginning}} \& \CCboxS{\textcolor{white}{Climax}} to generate a beginning and climax of a given storyline. V4 delegated \CboxS{\textcolor{white}{Juice}} to juice up a given story in a \textbf{\textcolor{sticky_note}{\textit{sticky note}}} with more details.}

\revision{We also found participants came up with creative and efficient ways of using a combination of microtasks}. By leaving some microtasks in the proactive mode, \textit{Polymind} can easily perform the ``chaining'' operation~\cite{wu2022ai} to expand users' ideas in a tree-like structure. During this process, multiple distinct microtasks could contribute simultaneously in parallel to users' main operations, which made the collaboration more efficient and creative.
Some participants complimented that the parallel microtasks were like ``\textit{a mature pipeline that needs little efforts}'' (V5), ``\textit{as if splitting (brainstorming) indefinitely}'' (V6). For example, V2 mainly used two proactive microtasks: \FboxS{\textcolor{white}{Freewrite}} \& \SboxS{\textcolor{white}{Summarise}} during the story ideation session.
\revision{He later explained that \FboxS{\textcolor{white}{Freewrite}} was used to quickly generate stories in a \textbf{\textcolor{sticky_note}{\textit{sticky note}}} given a few keywords or concepts within a \textbf{\textcolor{section}{\textit{section}}}, while \SboxS{\textcolor{white}{Summarise}} presented brief summaries in a \textbf{\textcolor{sticky_note}{\textit{sticky note}}} of \FboxS{\textcolor{white}{Freewrite}}'s generations so that he would not need to read whole stories.}

V10 instead was mainly using \BboxS{\textcolor{white}{Brainstorm}} and \EboxS{\textcolor{white}{Elaborate}} to expand her ideas in brief keywords and concepts. \revision{She used \BboxS{\textcolor{white}{Brainstorm}} to request related ideas and \EboxS{\textcolor{white}{Elaborate}} to provide concrete examples of an idea.} In a comparison to the ChatGPT interface, she commented that,
\begin{quote}
    ``\textit{I feel that ChatGPT often generated something irrelevant, and missed my expectations. But this system (Polymind) stuck to my main concept by generating relevant ideas. Although the results were only brief keywords, but it was fast. It could produce a huge idea tree within a short period of time, and you could easily find something intriguing and figure out a coherent story.}''.
\end{quote}

\newtcbox{\Bbox}{on line,
  colframe=brainstorm,colback=brainstorm,
  boxrule=0.5pt,arc=1pt,boxsep=0pt,left=2pt,right=2pt,top=2pt,bottom=2pt}
\newtcbox{\Sbox}{on line,
  colframe=summarise,colback=summarise,
  boxrule=0.5pt,arc=1pt,boxsep=0pt,left=2pt,right=2pt,top=2pt,bottom=2pt}
\newtcbox{\Ebox}{on line,
  colframe=elaborate,colback=elaborate,
  boxrule=0.5pt,arc=1pt,boxsep=0pt,left=2pt,right=2pt,top=2pt,bottom=2pt}
\newtcbox{\Dbox}{on line,
  colframe=draft,colback=draft,
  boxrule=0.5pt,arc=1pt,boxsep=0pt,left=2pt,right=2pt,top=2pt,bottom=2pt}
\newtcbox{\Fbox}{on line,
  colframe=freewrite,colback=freewrite,
  boxrule=0.5pt,arc=1pt,boxsep=0pt,left=2pt,right=2pt,top=2pt,bottom=2pt}
\newtcbox{\Abox}{on line,
  colframe=associate,colback=associate,
  boxrule=0.5pt,arc=1pt,boxsep=0pt,left=2pt,right=2pt,top=2pt,bottom=2pt}
\newtcbox{\Cbox}{on line,
  colframe=custom,colback=custom,
  boxrule=0.5pt,arc=1pt,boxsep=0pt,left=2pt,right=2pt,top=2pt,bottom=2pt}
\newtcbox{\CCbox}{on line,
  colframe=custom2,colback=custom2,
  boxrule=0.5pt,arc=1pt,boxsep=0pt,left=2pt,right=2pt,top=2pt,bottom=2pt}

\renewcommand{\arraystretch}{1.2}
\begin{table*}[htb]
    \resizebox{\linewidth}{!}{
    \begin{tabular}{c|cc|cc}
    \toprule
    & \multicolumn{2}{c|}{Task \RNum{1}} & \multicolumn{2}{c}{Task \RNum{2}} \\
    & default & custom & default & custom \\
    \midrule
    V1 & \Ebox{\textcolor{white}{Elaborate}} \Fbox{\textcolor{white}{Freewrite}} & \Cbox{\textcolor{white}{Characteristics}} & \Sbox{\textcolor{white}{Summarise}} \Fbox{\textcolor{white}{Freewrite}} & \\
    V2 & \Bbox{\textcolor{white}{Brainstorm}} \Sbox{\textcolor{white}{Summarise}} \Fbox{\textcolor{white}{Freewrite}} & & \Ebox{\textcolor{white}{Elaborate}} \Fbox{\textcolor{white}{Freewrite}} & \Cbox{\textcolor{white}{Structure}} \\
    V3 & \Bbox{\textcolor{white}{Brainstorm}} \Fbox{\textcolor{white}{Freewrite}} \Dbox{\textcolor{white}{Draft}} \Abox{\textcolor{white}{Associate}} & & \Bbox{\textcolor{white}{Brainstorm}} \Sbox{\textcolor{white}{Summarise}} \Fbox{\textcolor{white}{Freewrite}} & \\
    V4 & \Bbox{\textcolor{white}{Brainstorm}} & \Cbox{\textcolor{white}{Structure}} \CCbox{\textcolor{white}{Juice}} (up) & \Bbox{\textcolor{white}{Brainstorm}} & \Cbox{\textcolor{white}{Structure}} \CCbox{\textcolor{white}{Theme}} \\
    V5 & \Bbox{\textcolor{white}{Brainstorm}} \Fbox{\textcolor{white}{Freewrite}} \Abox{\textcolor{white}{Associate}} & & \Bbox{\textcolor{white}{Brainstorm}} \Fbox{\textcolor{white}{Freewrite}} \Abox{\textcolor{white}{Associate}} & \\
    V6 & \Bbox{\textcolor{white}{Brainstorm}} \Abox{\textcolor{white}{Associate}} & & \Bbox{\textcolor{white}{Brainstorm}} \Abox{\textcolor{white}{Associate}} & \\
    V7 & \Bbox{\textcolor{white}{Brainstorm}} & & \Bbox{\textcolor{white}{Brainstorm}} \Ebox{\textcolor{white}{Elaborate}} & \\
    V8 & \Bbox{\textcolor{white}{Brainstorm}} \Dbox{\textcolor{white}{Draft}} \Abox{\textcolor{white}{Associate}} & & \Bbox{\textcolor{white}{Brainstorm}} \Dbox{\textcolor{white}{Draft}} \Fbox{\textcolor{white}{Freewrite}} & \Cbox{\textcolor{white}{Beginning}} \CCbox{\textcolor{white}{Climax}} \\
    V9 & \Bbox{\textcolor{white}{Brainstorm}} \Dbox{\textcolor{white}{Draft}} \Abox{\textcolor{white}{Associate}} & & \Dbox{\textcolor{white}{Draft}} & \\
    V10 & \Bbox{\textcolor{white}{Brainstorm}} \Ebox{\textcolor{white}{Elaborate}} \Abox{\textcolor{white}{Associate}} & & \Bbox{\textcolor{white}{Brainstorm}} \Abox{\textcolor{white}{Associate}} & \\
    \bottomrule
    \end{tabular}}
    \caption{Microtasks used by each participant in \textit{Polymind} condition during two sessions that produced the final results. The input \& output types and prompts of default microtasks might have been edited.}
    \label{tab:usage}
\end{table*}

% \renewcommand{\arraystretch}{1.2}
% \begin{table*}[htb]
%     \resizebox{\linewidth}{!}{\begin{tabular}{c|cc}
%     \toprule
%     & Task \RNum{1} & Task \RNum{2} \\
%     \midrule
%     S1 & \Bbox{\textcolor{white}{Brainstorm}} \Fbox{\textcolor{white}{Freewrite}} \textbf{+} \Dbox{\textcolor{white}{Draft}} & \Bbox{\textcolor{white}{Brainstorm}} \Abox{\textcolor{white}{Associate}} \\
%     S2 & \Bbox{\textcolor{white}{Brainstorm}} \Ebox{\textcolor{white}{Elaborate}} \Fbox{\textcolor{white}{Freewrite}} \textbf{+} \Dbox{\textcolor{white}{Draft}} & \Bbox{\textcolor{white}{Brainstorm}} \Ebox{\textcolor{white}{Elaborate}} \\
%     S3 & \Bbox{\textcolor{white}{Brainstorm}} \Ebox{\textcolor{white}{Elaborate}} \textbf{+} \Dbox{\textcolor{white}{Draft}} & \Ebox{\textcolor{white}{Elaborate}} \textbf{+} \Dbox{\textcolor{white}{Draft}} \\
%     S4 & \Abox{\textcolor{white}{Associate}} \textbf{+} \Dbox{\textcolor{white}{Draft}} & \Bbox{\textcolor{white}{Brainstorm}} \Ebox{\textcolor{white}{Elaborate}} \Abox{\textcolor{white}{Associate}} \\
    
%     S5 & \Bbox{\textcolor{white}{Brainstorm}} \Abox{\textcolor{white}{Associate}} & \Ebox{\textcolor{white}{Elaborate}} \Fbox{\textcolor{white}{Freewrite}} \\
    
%     S6 & \Ebox{\textcolor{white}{Elaborate}} \Fbox{\textcolor{white}{Freewrite}} \Dbox{\textcolor{white}{Draft}} \textbf{+} \Cbox{\textcolor{white}{Custom}} & \Abox{\textcolor{white}{Associate}} \Bbox{\textcolor{white}{Brainstorm}} \Ebox{\textcolor{white}{Elaborate}} \textbf{+} \Cbox{\textcolor{white}{Custom1}} \CCbox{\textcolor{white}{Custom2}} \\
%     S7 & \Abox{\textcolor{white}{Associate}} \Dbox{\textcolor{white}{Draft}} \textbf{+} \Cbox{\textcolor{white}{Custom}} & \Cbox{\textcolor{white}{Custom}} \\
%     S8 & & \Bbox{\textcolor{white}{Brainstorm}} \textbf{+} \Cbox{\textcolor{white}{Custom}} \\
%     S9 & \Ebox{\textcolor{white}{Elaborate}} \Dbox{\textcolor{white}{Draft}} & \Bbox{\textcolor{white}{Brainstorm}} \Fbox{\textcolor{white}{Freewrite}} \Dbox{\textcolor{white}{Draft}} \\
%     S10 & \Bbox{\textcolor{white}{Brainstorm}} \textbf{+} \Cbox{\textcolor{white}{Custom}} & \Ebox{\textcolor{white}{Elaborate}} \\
%     S11 & \Bbox{\textcolor{white}{Brainstorm}} \Sbox{\textcolor{white}{Summarise}} \textbf{+} \Cbox{\textcolor{white}{Custom}} & \Bbox{\textcolor{white}{Brainstorm}} \Dbox{\textcolor{white}{Draft}} \\
%     S12 & \Bbox{\textcolor{white}{Brainstorm}} \Abox{\textcolor{white}{Associate}} \Fbox{\textcolor{white}{Freewrite}} \textbf{+} \Cbox{\textcolor{white}{Custom}} & \Abox{\textcolor{white}{Associate}} \Ebox{\textcolor{white}{Elaborate}} \Dbox{\textcolor{white}{Draft}} \\
%     S13 & \Bbox{\textcolor{white}{Brainstorm}} \Ebox{\textcolor{white}{Elaborate}} \Sbox{\textcolor{white}{Summarise}} \Fbox{\textcolor{white}{Freewrite}} \textbf{+} \Cbox{\textcolor{white}{Custom}} & \Bbox{\textcolor{white}{Brainstorm}} \Ebox{\textcolor{white}{Elaborate}} \Fbox{\textcolor{white}{Freewrite}} \\
%     S14 & \Bbox{\textcolor{white}{Brainstorm}} \Ebox{\textcolor{white}{Elaborate}} \Sbox{\textcolor{white}{Summarise}} & \Bbox{\textcolor{white}{Brainstorm}} \Ebox{\textcolor{white}{Elaborate}} \textbf{+} \Cbox{\textcolor{white}{Custom}} \\
%     \bottomrule
%     \end{tabular}}
%     \caption{Microtasks used by each participant in two tasks to produce the final results}
%     \label{tab:usage}
% \end{table*}

V3 was one of the three participants that showed clear preference for the ChatGPT interface, but she also added that experimenting ideas with \textit{Polymind} were much easier. She explained that \textit{Polymind} ``\textit{had a structure}'' and could perform chaining-like operations easily ``\textit{by using sections}'' and parallel microtasks, while for ChatGPT, ``\textit{combining elements (like some characters, events, or settings) from its generations to re-prompt it was challenging}''. Similarly, V6 noted that brainstorming associations between key events or scenes was much easier with \textit{Polymind} by ``\textit{using several proactive microtasks operating on nodes or sections}''.

\subsubsection{Polymind is More Controllable}
While ChatGPT-4's conversational interface was able to generate long pieces of text with many ideas and details (V2-4, V7-9), most of our participants (V1-2, V4-5, V7-8, V10) mentioned that \textit{Polymind} felt more controllable in a prewriting task. This is because \textit{Polymind} directly operated on diagrams that were often shorter and \revision{thus easier to digest and re-prompt} than ChatGPT-4's conversations, and the canvas progressed in a structured manner with parallel microtasks handling very specific requirements.

The longer generations of the conversational interface were often criticised for being hallucinatory (e.g., ``\textit{not that creative or sensible as it appears}'' -- V7), too random (e.g., ``\textit{not what I expected}'' -- V5, ``\textit{irrelevant}'' -- V10), or ``\textit{mediocre}'' (V2) during a story pre-writing session. In comparison, \textit{Polymind} generations were perceived by many to be relevant (V2, V5, V10), and its microtasks more responsive to users' requests (V5, V8, V10), although it might require some efforts to manage or configure them (V4). \revision{This echoes with the usability score of the two conditions, where \textit{Polymind} was on the same level with the baseline, despite the efforts of managing a diagramming interface.}
V8 said in retrospect that,
\begin{quote}
    ``\textit{I think this (Polymind) would be very helpful for coming up with a story. Cause you can specify your beginning, you can specify your climax, and the ending too... And I think customizing microtasks is also a nice feature... You can really outline everything. While for GPT, you often don't know what is beginning, what is climax or ending.}''
\end{quote}

Additionally, V1, V2 and V8 noted that \textit{Polymind}'s microtasking workflow made it easier to re-prompt. V8 said,
\begin{quote}
    ``\textit{If I want to change something, I know where the part is. Like the character, I only need to change several keywords, like, Oh I'd like the character to be a dragon... I felt that my prompts were actually considered (by microtasks), while GPT sometimes doesn't process all my prompts.}''
\end{quote}
For the conversational interface, it often took multiple iterations to reach a decent draft (V4-5), and each prompt had to be lengthy to change the context (V1-2, V5), which was demanding. V2 also added that the \textit{Polymind} interface was neater because with some parallel microtasks it required little to no efforts of note taking to ask multiple follow-up questions of different ideas.

It is worth noting that, three participants that disliked \textit{Polymind}'s workflows mentioned that it was quite demanding sometimes to configure microtasks (V3-4), and progress in diagrams (V9). V9 said, ``\textit{GPT could generate a lot with a single prompt, while \textit{Polymind} only little by little.}''
\revision{V10 shared similar sentiments. She noted prompting ChatGPT would be much easier than using \textit{Polymind}'s diagrammatic workflow if its generations were not random. However, she added that \textit{Polymind} was in reality less demanding because you could explore more options and easily drop random results.}

\subsubsection{Polymind Affords Agency}
Our participants almost unanimously said that \textit{Polymind} put users in a dominant role, while with the conversational interface they were completely guided by the GPT. This aligns with our \textit{Goal 2} that aims to put humans in a role of managing all microtasks. Participants without any ideas, such as V3 \& V4, generally did not mind following the ChatGPT. This is expected, and agrees with our formative study. However, same as almost all other participants, they particularly mentioned that they felt these results were not their ideas, as ChatGPT generated almost everything. While using \textit{Polymind}, participants said that they had more freedom and control (V3, V6-8), and needed to think a lot (V4-5, V8). 

One of the participants, V5, particularly said he had no trust in AI because ``\textit{it could not be truly creative}''. He therefore became very annoyed with the ChatGPT interface when it did not generate what he expected, saying it was ``\textit{bad usability}'', while attributing ``\textit{good usability}'' to the task management workflow. V7 also expressed concerns for using the conversational interface for brainstorming,
\begin{quote}
    ``\textit{At first sight, it might seem it had generated everything you could think of, but then you'd find many were indeed non-sensical. But I felt I was confined to these generations after reading them. It was especially hard for a novice writer like me to come up with other possibilities. So it felt like it was GPT that was composing a fiction, rather than me.}''
\end{quote}
She later added that her own results from \textit{Polymind} felt more ``\textit{logical}'', and ``\textit{rigorous}''.

Of the participants that said very positively of the ChatGPT interface, V3 stressed that \textit{Polymind} encouraged her to express her own ideas, while ChatGPT did not. That was why she assigned a very low score of expressiveness in the CSI survey when using ChatGPT.



\section{Use Cases and Future Applications}
In this section, we explore various applications of our watermarking technology designed for quick and integrated paper augmentation.
% By embedding AR interaction data directly within the physical document, \systemName~ offers a seamless blend of digital interactivity and traditional paper mediums. 
\new{We aim to demonstrate the breadth of use cases that \systemName~ can support and additional benefits that our method provides to users (e.g., invisible, mobile, privacy-preserving) across various contexts. Here, we present applications that are unique to \systemName, and we use replication for further validation \cite{Ledo2018Evaluation}.}



\subsection{Embedding "Paper DNA"}
\systemName~ could allow users to embed the digital metadata of the document as an intrinsic part of the physical paper. Utilizing this intrinsic "paper DNA," users could enable further authentication solutions. 



\begin{figure*}[h]
  \centering
  \includegraphics[width=1.0\linewidth]{figures-new/applicationNew.png}
  \caption{\systemName~ is used for making digital metadata an intrinsic part of the printed document to (a) recreate documents from fragments or (b) verify the authenticity of individual pieces. (c) The application augments normal vision by adding the AR content, providing additional information that would overload the limited space of a conference badge.}
  \Description{.}
  \label{fig:ApplicationsAR}
\end{figure*}




\subsubsection{Recreating Documents from Fragments}

The "Paper DNA" watermark allows fragments of documents to be scanned and linked back to the full digital copy. For example, if a poster for an educational event is partially torn, as shown in \autoref{fig:ApplicationsAR}a, a student can scan the remaining portion to retrieve the original poster and access the missing information. \new{The advantage of \systemName~ is that reconstruction can work even if the document is heavily fragmented. This is possible due to the high data capacity of a single marker that is sufficient to recreate the document. Because the information is repeatedly encoded without influencing the document's aesthetics, only one small part is necessary to retrieve its entire content.}
\newCameraReady{Maia et al. demonstrated this for 3D printed objects, where redundancy renders the information readable from a broken or damaged object to recover its original model
~\cite{maia_layercode_2019}, which inspired our application.}
\new{For paper materials, image-based tracking cannot easily achieve this because it would require training on fragmented documents, which does not scale well.}
We envision this feature can also benefit marketing strategies by including tearable sections in posters or ads with embedded links for future reference, \new{which are known to occupy too small space to include extra information, and \systemName~ can overcome these physical space constraints.}
 




\subsubsection{Imprinted Verification Signatures}
\systemName~ enhances document authenticity by embedding verification signatures in printed materials in combination with other techniques. For example, printed contracts could include invisible blockchain-based signatures or hyperlinks to the digital version for comparison against post-print modifications. If doubts arise, scanning with \systemName~ can confirm the document's originality. \new{The advantage is that this adds a second layer of "invisible" security by preventing people with malicious intent from reading names, passcodes, or signatures directly from the tickets, e.g., by shoulder surfing. This application exploits humans' unawareness of the presence of invisible IR content.}





\autoref{fig:ApplicationsAR}b shows an example where a friend group bought multiple tickets to a concert. The individual tickets, after being cut out from the sheet, each contain a verification code, which is scanned at the entrance of the venue. 


\subsection{Augmenting Documents with Contextual Annotations and Multimedia}
\systemName~ seamlessly integrates digital interactivity into traditional paper media,  from educational enhancements to professional networking solutions.




\subsubsection{Smart Business Cards and Conference Badges}
\systemName~ could be used for fabricating \newCameraReady{intelligent business cards or conference badges as envisioned by Want et al.~\cite{want_bridging_1999}}, which could, for example, also enable access to restricted areas, digital contact exchanges, and enhance networking experiences at professional gatherings without any electronic components.
For instance, at conferences, attendees could wear badges and exchange business cards enhanced with \systemName, as shown in \autoref{fig:ApplicationsAR}c. Scanning a badge reveals the attendee's professional profile, contact information, and links to their work, facilitating networking without the need for physical contact or exchanging multiple items.
\new{This works best if attendees know that all badges or business cards support IR content--thus, they know how to retrieve this information. Additionally, they can hide information they do not want to share immediately with anyone who passes by because it is directly visible on the badge.}



\subsubsection{Interactive Flashcards for Learning}

\systemName~ enhances traditional flashcards by integrating multimedia, aiding tasks like language learning.
\newCameraReady{We replicate certain functionalities of CV-based flashcards, which include pronunciation guides via simple scans, turning materials into interactive experiences that improve outcomes \cite{ng2018treasure}.} 
However, our application also supports offline access to hints and answers, minimizing online distractions. Students can reveal hints or check answers through an app, promoting independent, self-paced learning. \new{Flashcards have very restricted space, severely limiting the kinds of tracking markers that can be embedded. \systemName~ uses the entire space of the flashcard, i.e., front and back, even without any content, because it does not interfere with it. Nevertheless, image or text detection could be used to achieve the same. However, we would expect \systemName~ to perform better during occlusive hand interactions with the flashcards since we can embed a larger number of evenly distributed markers.}




\subsubsection{Offline High-Capacity Data Access for Sensitive Topics}

\systemName~ can discreetly embed critical information into materials, useful for raising awareness about stigmatized topics or promoting sensitive products (e.g., healthcare) while ensuring privacy. For instance, a health organization can distribute posters on mental health with embedded AR markers, allowing access to additional information, such as treatment options, via a simple scan, as shown in \autoref{fig:teaser}c, without leaving a digital footprint. \new{This demonstrates \systemName's key contribution: enabling invisible, high-capacity offline data embedding without requiring users to constantly send data to third-party servers. We also see potential in using offline high-capacity data access to enhance the accessibility of printed media, such as alt-texts, to images for visually impaired people without adapting visual designs to accommodate markers.}



\begin{figure*}[t]
  \centering
  \includegraphics[width=\linewidth]{figures-new/printOnDifferentThings.png}
  \caption{Envisioned extension of \systemName. \new{(a) A user wears AR smart glasses that capture \systemName~ content by exploiting the device's built-in always-on NIR cameras.} (b, c, d) Watermarking on different materials and surfaces to identify personal belongings and facilitate robotic navigation through invisible cues.}
  \Description{.}
  \label{fig:ApplicationsRobotic}
\end{figure*}





\vspace{0.2cm}
\new{\textbf{\systemName's Utility.} The examples in this section illustrate how IR watermarking can function as the "back end" for interactions with printed documents. Much like in web development or software engineering, this approach emphasizes a clear separation between the presentation layer (the "front end") and the data access layer (the "back end"). By abstracting and concealing technical, computer-readable components, our system offers a seamless, user-friendly interface while maintaining robust functionality. This separation enables new possibilities for interaction, surpassing the limitations of visual markers or standard QR codes in terms of abstraction and user experience. We argue that, if IR watermarks become the "back end" of printed media one day, it may pave the way for transformative applications without compromising the complexity and depth of the underlying technology.}







\subsection{Origami: Interactive Folding Instructions}
\systemName~ brings a twist to the traditional art of origami by embedding interactive, step-by-step folding instructions directly onto the paper. As users progress with their folds, new instructions appear on the exposed surfaces, as shown in \autoref{fig:teaser}b, based on the number and combination of the \systemName~ codes visible to the IR camera.
%This crafting experience could make complex designs more engaging and accessible for a larger audience.
The advantage of \systemName~ is that even if the origami paper has no visible content (i.e., white sheet) in order not impact the look of the final artifact, the sheet can still carry invisible folding instructions.

\subsection{Sustainable On-Demand Watermarking}
\systemName~ empowers users to watermark existing posters and documents larger than letter-size format on demand (see \autoref{fig:teaser}a). This may help to safe resources by avoiding re-printing posters or documents in case of missing content. Traditionally, people often go through multiple iterations until they arrive at their desired design. \systemName~ enables people to focus on visual aesthetics with the possibility of adding content wherever and whenever needed. 


\subsection{Beyond Paper Augmentation}
By using a mobile printer, \systemName~ could also add watermarks to plastic~\cite{ozdemir_speed-modulated_2024} (a), fabric~\cite{forman_defextiles_2020} (b), and wood~\cite{gui_draw2cut_2025} (c) and various form factors demonstrated in \autoref{fig:ApplicationsRobotic}. This is possible because the NIR ink is dye-based and can, therefore, be absorbed by these materials. In addition, our CNN detection pipeline can be fine-tuned for detecting, e.g., QR codes on convex or concave surfaces, highlighting the breadth of our contribution. This opens up new possibilities for embedding invisible identification markers to personal belongings (a) and (b). In \autoref{fig:ApplicationsRobotic}c, we also show how \systemName~ may be used for robotics use cases, in this case for robotic vacuum cleaners to safely avoid the pet's area.


\vspace{0.2cm}

Through these scenarios, \systemName~ shows its potential of integrating infrared inkjet watermarking into everyday interactions, and enabling solutions across educational, professional, and personal contexts.


\section{Discussion}
We discuss the bigger picture emerging from our results, followed by specific aspects of interest.

\subsection{Word Bubbles Support Continuous Touch Control of Text Generation}\label{sec:discussion_bubbles} %
Our ``Bubbles'' visualisation with its text length and word indicators consistently outperformed \revision{the alternatives}, allowing participants to complete tasks more quickly: 
It averaged \secs{14.71} per task, compared to NoVis (\secs{16.58}) and Lines (\secs{16.50}). 
This speed advantage was evident across tasks with text extension, shortening, and combinations.
Bubbles also received the highest usability rating (SUS: 85.54) and the lowest perceived workload (NASA-TLX score: 1.98).

In addition, participants reported that Bubbles made interactions feel smooth and natural, with many describing the gestures as intuitive and engaging. 
The visual separation of words, sentences, and deleted content provided by Bubbles enhanced the sense of control, making it the preferred method. %

\subsection{Direct Interaction vs Conversations}
When comparing our gestures to the typical conversational LLM interface, participants rated the touch gestures as more usable (81 vs 52.5 SUS score) and less mentally taxing (2.06 vs 3.15 NASA-TLX). 
This is further reflected in %
all subjective measures, including satisfaction, ease of use, control, and efficiency.
Indeed, using gestures led to \pct{58} reduced task times (\secs{56.35} vs \secs{134.86}).

We conclude that our gesture-based concept was well-received and demonstrated clear advantages over a chatbot UI. Combined with our visual feedback, it thus presents a promising alternative for future text editing applications.

\subsection{Direct Interaction and Authorship}
With gestures, participants generally felt more control over text length and the generation process (\cref{fig:own_likert_exp2}, \cref{sec:ss_interview_perception}) yet many did not perceive themselves as the authors of this text. 
This is unsurprising, as \revision{our study} (a) provided given text, and (b) focused more on adjusting text length than altering content.

With our prototype, users could modify tone and input custom prompts with a long press (\cref{fig:long_press}). 
We expected the ChatGPT-like app to receive similar or even higher ratings in the ``authorship'' category, given that participants entered prompts themselves. 
However, this was not reflected in their ratings -- gestures scored higher. 
We hypothesize that the expected fall off in perceived authorship was mitigated due to participants feeling more like authors when they did not need to switch contexts or interact with an external (chatbot) app. %
This effect might become even stronger as more LLM capabilities are controlled through direct (touch) interaction.


\revision{More broadly,} previous research has shown that perceived authorship does not necessarily equate to actually \revision{having entered the text} \cite{AIghostwriter2024}. 
\revision{Thus, as HCI research pursues} interaction with AI \revision{to become} increasingly direct and seamless, %
we believe it is important to also take a nuanced look at the concept of authorship \revision{and the interaction method's impact on its perception}.


\subsection{Direct Interaction and Interaction Metaphors}
When using conversational UI elements, some users tend to perceive the AI more as a writing partner \cite{diarymateKim2024, benharrak2024aipersonas}, rather than a tool. %
Direct interaction has the potential to change this: 
Controlling the LLM directly feels more natural and gives a greater sense of control, as reflected in our findings. 
It also removes the mental association with a writing partner, as there’s no conversational interface -- just the user's own movements. 
As LLM interactions become more seamless, users might begin to view the LLM as part of their cognitive process \cite{gptMeMeshi2024, bhat2023suggestionmodel}, \revision{which could be explored in the} future.







\subsection{Impact of the Visual Feedback on User Interaction and Cognitive Load}
The choice of visual feedback had a significant impact on how users interact with and perceive the system.

\revision{Beyond the benefits of ``Bubbles'' discussed above (\cref{sec:discussion_bubbles})}, this design also impacted gesture execution: 
With Bubbles, participants approached the intended text length more consistently and with less overshooting. 
\revision{Without} visual feedback, \revision{they instead} had to wait for all words to be generated before gauging text length, as both the length and content were displayed through the same medium -- words. 
This presents a unique challenge for interaction with LLMs, where generation is faster than reading, but slower than users can assess the intended length.

Bubbles decoupled text length indication from actual words, allowing users to gauge the amount of text and start reading sooner. We hypothesize that differences in execution patterns stem from this decoupling. 

As less overshooting occurred \revision{with Bubbles}, people had to read less generated text, which may have lowered cognitive load. 
This could explain the reduced mental demand during writing when using Bubbles, as measured by the NASA-TLX (\cref{fig:own_likert_exp2}). 

Reading, attention, and related indicators of cognitive demand were not measured beyond self-reports (e.g. with eye tracking), which could be a focus for future research.




\subsection{Exploring the Design Space of Mobile Touch Interaction with LLMs}
After defining the design space, we adopted a depth-first approach in this paper and focused on exploring one specific LLM capability and gesture in detail: 
\spread{}, along with the inverse, \pinch{}. 

In future studies, we plan to explore additional alternatives and command mappings:
\begin{itemize}
    \itemsep 2mm
    \item \textit{Spread-to-Elaborate, Pinch-to-Summarize:} Currently, spreading appends text at the end of a sentence. Future versions could use spreading over existing \revision{paragraphs to extend them by inserting} elaborations. Similarly, pinching could trigger AI-generated summaries.
    \item \textit{Swipe-to-Rephrase, Rotate-for-Tone:} Swiping over text passages could prompt the LLM to rephrase them or generate synonyms for selected words, while a \revision{rotation} gesture could adjust tone (e.g. ``dialling up/down'' formality).
    \item \textit{Tap-to-Talk:} Special gestures such as a triple tap or a three-finger swipe could request detailed explanations or feedback from the LLM.
    \item \textit{Other Input Modalities:} Beyond touch-based interaction, shaking the phone could trigger rewording and tilting could provide finer control. %
\end{itemize}


\subsection{Limitations}
Our study comes with limitations.
Our prototype was limited to two gestures, within a simple text field, and tested on one device. 
Integrating our prototyped functionality into a larger mobile application (e.g. respecting existing gestures) was beyond our scope.

We covered an initial sample with diversity in some demographics (age, technology use). 
Future work should further evaluate the gestures with a larger, more diverse population.

Participants preferred touch controls over a conversational UI for the tested tasks of text generation and shortening. 
This should not be generalised as an overall preference. 
It is likely that conversational UIs, which offer much more open functionality than gestures, would still be preferred in other tasks (e.g. complex information retrieval, interactive dialogues).
We plan to expand our prototype and compare interactions across further tasks in the future.



\section{Conclusion}
This paper presented \toolkit, a do-it-yourself toolkit that empowers novice roboticists with basic electronics and programming skills to rapidly prototype interactions for functional lo-fi exoskeletons targeted at the arms. 
\toolkit~features modular hardware components that allow to easily reconfigure its active degrees of freedom, adjust component's dimensions to accommodate various body sizes, and safety mechanisms. We conceptually identified relevant high-level augmentation strategies and provide them as functional abstractions that simplify the programming of interactive behaviors. These functions are readily accessible and customizable through a command-line interface, GUI, Processing library, and Arduino firmware. 
Through application cases and two usage studies, we demonstrated \toolkit's potential to ease the development of human-exoskeleton interactions and support creative exploration and rapid iteration in early-stage interaction design. We hope that this work will inspire HCI researchers to explore the emerging field of human-exoskeleton interaction and unlock its potential for innovative applications.
 
\begin{acks}
    We thank all participants of our usage studies and express our particular gratitude to Ata Otaran for his feedback. We also thank the reviewers for their valuable comments.
\end{acks}



\bibliographystyle{ACM-Reference-Format}
\bibliography{ImmersiveDocs}


\onecolumn\appendix
\section{Appendix}\label{apedix: selection}
We provide a table for background color selection for the experiment and another table regarding the smallest module size and character capacity comparison.



\begin{table*}[h!]
\centering
\vspace{0.1cm}
\begin{tabular}{|c|c|c|c|}
\hline
\textbf{Group ID} & \textbf{Darkness}& \textbf{Calculated} & \textbf{Number of selected colors} \\
 & \textbf{{\small(analog $K$ value)}} & \textbf{luminescence ($lum$) range } & \textbf{{\small(roughly four colors per 25\% $lum$.)}} \\ \hline
1 & [0\%, 25\%] & (8.55\%, 100\%] & 16 \\ \hline
2 & [25\%, 50\%]& (5.7\%, 75\%] & 12 \\ \hline
3 & [50\%, 75\%] & (2.85\%, 50\%] & 8 \\ \hline
4 & [75\%, 100\%] & (0\%, 25\%] & 4 \\ \hline
5 & 100\% & 0\% & 1 \\ \hline
6 & \multicolumn{3}{|l|}{\small Four colors have been added to enhance color hue coverage.} \\ \hline
\end{tabular}
\vspace{0.1cm}
\caption[]{\new{Color selection table for the experiment. Colors for each tested group are chosen based on their darkness and luminescence ranges.}}\label{tab:color_coverage}
\vspace{-0.2cm}
\end{table*}\label{tab:studycolor}


% \subsection{Smallest Module Size and Character Capacity Comparison}


\begin{table*}[h!]
\centering
\small\begin{tabular}{ccccc}
\toprule
 & \textbf{Letter size (distance $\geq$ 25.5cm)} &  & \textbf{Half letter size (distance $\geq$ 19.5cm)} & \\
\cline{2-3}\cline{4-5}
\textbf{Ink Type} & \textbf{Smallest size} & \textbf{Maximum capacity} & \textbf{Smallest size} & \textbf{Maximum capacity}\\
\midrule
Low EEC 81\% ink & 1.27mm & 1800 characters & 1.27mm & 900 characters\\
Medium EEC 81\% ink & 1.18mm & 1600 characters & 0.93mm & 1224 characters\\
High EEC 81\% ink & 1.18mm & 800 characters & 0.93mm & 648 characters\\
Low EEC 102\% ink & 1.18mm & 2040 characters & 1.01mm & 1224 characters\\
Medium EEC 102\% ink & 1.18mm & 1600 characters & 0.93mm & 1224 characters\\
High EEC 102\% ink & 1.10mm & 1080 characters & 0.93mm & 648 characters\\
Low EEC 155\% ink & 1.18mm & 2040 characters & 0.93mm & 1560 characters\\
Medium EEC 155\% ink & 1.18mm & 1600 characters & 0.93mm & 1224 characters\\
High EEC 155\% ink & 1.10mm & 1080 characters & 0.93mm & 648 characters\\
\bottomrule
\end{tabular}
\vspace{0.1cm}
\caption{Comparison of module sizes and capacities at different ink levels and distances.}
\end{table*}\label{tab:inkcomparison}












\end{document}
\endinput
%%
%% End of file `sample-authordraft.tex'.
