\section{Technical Evaluation}
\label{EvaluationDataEmbedding}


\begin{figure*}[t]
  \centering
  \includegraphics[width=0.77\linewidth]{figures-new/TechnicalEvaluationPreview.png}
  \caption{Examples of the samples used for bit size evaluation at $25.5~cm$ camera distance necessary for letter-size sheets.}
  \Description{.}
\label{fig:TechnicalEvaluationPreview}
\end{figure*}




In this section, we will evaluate various factors affecting the IR ink machine detectability and data capacity, such as IR ink density, detection distance and QR code's error correction code (ECC). In the following, we explain the effects of each variable. 
Starting with increasing the ECC, designers can improve the machine detectability of the QR codes, but increasing the ECC too much requires additional paper space for the same embedding.
On the other hand, by increasing the distance between the NIR camera and the IR content, the illuminance also decreases, and thus so does the machine detectability of the content.
Finally, by increasing the ink density, we increase the contrast, and thus also the machine detectability.
However, increasing the ink density might cause it to become visible to the human eye, and since we already determined the lower limits of human perception in Section~\ref{visibility-experiment}, the amount of IR ink that can be used in  \systemName~ documents is capped.

Therefore, we examine the effects of detection distance and ECC at invisible IR ink density levels on the data capacity of the sheet as it is crucial to better understand the trade-offs of using invisible IR content.
As Data Matrix codes do not provide the option to set an ECC, we focus on QR codes in our evaluation. Our results can guide design decisions when fabricating \systemName~ documents. 




\subsection{Data Capacity Based on IR Ink Density, Detection Distance and ECC}
\label{DetectionDistanceCapacity}

\subsubsection{Procedure}
For the purposes of this evaluation, we will focus on the three invisible ink densities associated with the background color classes outlined earlier in this paper: 81\%, 102\% and 155\% (see \autoref{fig:IRTraining}a).
We focus on the distances necessary to capture a US letter-size sheet ($25.5~cm$ in order to capture an 8.5" x 11" sheet) and a US half letter-size sheet ($19.5~cm$ in order to capture a 5.5" x 8.5" sheet) as they present common real-world scenarios.
Please note that our current setup features an NIR camera that has a slightly larger field of view than the area illuminated by the integrated LEDs. However, for the purpose of evaluating the maximum capacity, we will assume that the whole sheet is uniformly illuminated.
After capturing the images, we use our image processing pipeline consisting of the ML model we trained (Section~\ref{TrainingML}) and the code reader (Section~\ref{RealTimeInference}) to evaluate the results.





\paragraph{Smallest Detectable Module}
In order to determine the maximum data capacity, we must assess the size of the smallest detectable module~\cite{dogan_standarone_2023}. Our sample printing precision was capped at 300 DPI, thus we were able to only assess the size of the smallest detectable module to within $0.085~mm$ (1" / 300 DPI). For each configuration of distance, ink density, and ECC (low, medium, and high), we printed a sample containing 12 randomly generated QR codes, that were not part of the training set used in Section~\ref{TrainingML}, ranging in total module size from $0.67~mm$ to $1.6~mm$ in increments of $0.085~mm$ arranged in a 3 by 4 grid (see \autoref{fig:TechnicalEvaluationPreview}). This range was determined based on empirical trials working with QR code detection after the initial training. In a real-world setting, each QR code would be exposed in multiple frames and from several slightly different positions.
In order to better reflect this in our evaluation, we aligned the camera with each of the four columns in the grid and captured a picture for each column.
We define the smallest detectable module as the smallest one that was detectable in at least one of the four pictures. Thus, our estimates for maximum data capacity only consider QR codes that can be detected. An example of the data for two such samples can be seen in \autoref{fig:TechnicalEvaluationPreview}.


\paragraph{Maximum Data Capacity}
Similarly to our perceptual experiment, our maximum data capacity estimates are conservative, since we enforce a quiet zone of at least four modules around each QR code in order to comply with the \textit{ISO/IEC 8859-1} standard. In this first iteration of our work, we assume that we cover the whole sheet with QR codes of identical size instead of trying a more complex packing algorithm that may involve mixing QR codes of different sizes. 
We included QR codes with varying data capacities, version 1 to version 7\footnote{\url{https://www.qrcode.com/}}, because in version 7 we reached our largest module size with a quiet zone of 4 modules.
Under the assumption that all QR codes are identical in size, we can iterate through all QR code versions (1--7) and try to fit them into a rectangular pattern to maximize the number of characters.



\subsubsection{Results}

The key results of our data capacity evaluation can be found in \autoref{fig:EvaluationCapacity} and \autoref{tab:inkcomparison}~2.
Here, we report estimates for the binary encoding scheme of QR codes. In cases where the data is limited to numeric or alphanumeric characters, we can increase capacity by leveraging the QR code standard's native support.

\begin{figure*}[]
      \centering
      \includegraphics[width=0.89\textwidth]{figures-new/capacityanddetectable.png}
      \caption{The effect of  IR ink amount and error correction code on the data capacity and module size of the sheet.}
      \label{fig:EvaluationCapacity}
  \end{figure*}

\vspace{-0.1cm}
   
\paragraph{Letter-size}
The maximum capacity results for our letter-size sheets show that for all three IR ink densities, we achieve similar maximum capacities, with the smallest detectable module size of $1.18~mm$ under the medium and low ECC. 81\% IR ink seems to fall short of the other two classes with a smallest detectable module size of $1.27~mm$. This is to be expected since its contrast is rather low, making it the most challenging case. ECC showed the expected effect: a reduction in data capacity but improved detection, which can be seen as a further validation of our approach.  

\vspace{-0.1cm}

\paragraph{Half letter-size}
The findings for our half letter-size sheets suggest mixed results. First, the data capacities seem promising, given that only half of the space is available. The improved detection can most likely be attributed to the closer distance of the camera to the sheet. This is also reflected in the smallest detectable module size of $0.93~mm$, which is significantly smaller than in the letter-size condition. We want to highlight that we did not train our model described in the previous section on this distance, but it can still detect QR codes reliably. However, despite 155\%, the difference between low and medium ECC appears to have negligible effects. 


\paragraph{Comparison to traditional binarization approaches}
To assess the effectiveness of the resulting CNN-based approach, we also compared it to traditional binarization methods used in previous work \cite{dogan_standarone_2023, dogan_demonstrating_2022}, which applies both contrast limited adaptive histogram equalization (CLAHE)~\cite{zuiderveld_contrast_1994} and adaptive thresholding~\cite{sauvola_adaptive_2000}, passing both binarized versions separately to the QR code reader.
We used the same dataset from the previous section to determine how effectively the CNN binarization can outperform the traditional filter-based image processing approach.
Our approach resulted in a 102.7\% increase in the total number of  QR codes detected compared to the previous method.
Notably, the CNN makes a larger impact on the detection of low-IR ink (Class 1) samples compared to high-IR ink (Class 3), i.e., an increase of 164.0\% vs. 92.7\%.
This is due to the fact that the pipeline for traditional methods vary based on factors such as lighting and the degree of code distortion, and may thus need constant adjustment of filter parameters.
In contrast, our CNN-based approach is more scalable, given its training dataset was crafted for our specific goal. 



\vspace{0.2cm}

Together, our results provide users with estimates of how much content can be embedded using our IR ink watermarking technique. 




