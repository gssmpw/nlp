


\section{Use Cases and Future Applications}
In this section, we explore various applications of our watermarking technology designed for quick and integrated paper augmentation.
% By embedding AR interaction data directly within the physical document, \systemName~ offers a seamless blend of digital interactivity and traditional paper mediums. 
\new{We aim to demonstrate the breadth of use cases that \systemName~ can support and additional benefits that our method provides to users (e.g., invisible, mobile, privacy-preserving) across various contexts. Here, we present applications that are unique to \systemName, and we use replication for further validation \cite{Ledo2018Evaluation}.}



\subsection{Embedding "Paper DNA"}
\systemName~ could allow users to embed the digital metadata of the document as an intrinsic part of the physical paper. Utilizing this intrinsic "paper DNA," users could enable further authentication solutions. 



\begin{figure*}[h]
  \centering
  \includegraphics[width=1.0\linewidth]{figures-new/applicationNew.png}
  \caption{\systemName~ is used for making digital metadata an intrinsic part of the printed document to (a) recreate documents from fragments or (b) verify the authenticity of individual pieces. (c) The application augments normal vision by adding the AR content, providing additional information that would overload the limited space of a conference badge.}
  \Description{.}
  \label{fig:ApplicationsAR}
\end{figure*}




\subsubsection{Recreating Documents from Fragments}

The "Paper DNA" watermark allows fragments of documents to be scanned and linked back to the full digital copy. For example, if a poster for an educational event is partially torn, as shown in \autoref{fig:ApplicationsAR}a, a student can scan the remaining portion to retrieve the original poster and access the missing information. \new{The advantage of \systemName~ is that reconstruction can work even if the document is heavily fragmented. This is possible due to the high data capacity of a single marker that is sufficient to recreate the document. Because the information is repeatedly encoded without influencing the document's aesthetics, only one small part is necessary to retrieve its entire content.}
\newCameraReady{Maia et al. demonstrated this for 3D printed objects, where redundancy renders the information readable from a broken or damaged object to recover its original model
~\cite{maia_layercode_2019}, which inspired our application.}
\new{For paper materials, image-based tracking cannot easily achieve this because it would require training on fragmented documents, which does not scale well.}
We envision this feature can also benefit marketing strategies by including tearable sections in posters or ads with embedded links for future reference, \new{which are known to occupy too small space to include extra information, and \systemName~ can overcome these physical space constraints.}
 




\subsubsection{Imprinted Verification Signatures}
\systemName~ enhances document authenticity by embedding verification signatures in printed materials in combination with other techniques. For example, printed contracts could include invisible blockchain-based signatures or hyperlinks to the digital version for comparison against post-print modifications. If doubts arise, scanning with \systemName~ can confirm the document's originality. \new{The advantage is that this adds a second layer of "invisible" security by preventing people with malicious intent from reading names, passcodes, or signatures directly from the tickets, e.g., by shoulder surfing. This application exploits humans' unawareness of the presence of invisible IR content.}





\autoref{fig:ApplicationsAR}b shows an example where a friend group bought multiple tickets to a concert. The individual tickets, after being cut out from the sheet, each contain a verification code, which is scanned at the entrance of the venue. 


\subsection{Augmenting Documents with Contextual Annotations and Multimedia}
\systemName~ seamlessly integrates digital interactivity into traditional paper media,  from educational enhancements to professional networking solutions.




\subsubsection{Smart Business Cards and Conference Badges}
\systemName~ could be used for fabricating \newCameraReady{intelligent business cards or conference badges as envisioned by Want et al.~\cite{want_bridging_1999}}, which could, for example, also enable access to restricted areas, digital contact exchanges, and enhance networking experiences at professional gatherings without any electronic components.
For instance, at conferences, attendees could wear badges and exchange business cards enhanced with \systemName, as shown in \autoref{fig:ApplicationsAR}c. Scanning a badge reveals the attendee's professional profile, contact information, and links to their work, facilitating networking without the need for physical contact or exchanging multiple items.
\new{This works best if attendees know that all badges or business cards support IR content--thus, they know how to retrieve this information. Additionally, they can hide information they do not want to share immediately with anyone who passes by because it is directly visible on the badge.}



\subsubsection{Interactive Flashcards for Learning}

\systemName~ enhances traditional flashcards by integrating multimedia, aiding tasks like language learning.
\newCameraReady{We replicate certain functionalities of CV-based flashcards, which include pronunciation guides via simple scans, turning materials into interactive experiences that improve outcomes \cite{ng2018treasure}.} 
However, our application also supports offline access to hints and answers, minimizing online distractions. Students can reveal hints or check answers through an app, promoting independent, self-paced learning. \new{Flashcards have very restricted space, severely limiting the kinds of tracking markers that can be embedded. \systemName~ uses the entire space of the flashcard, i.e., front and back, even without any content, because it does not interfere with it. Nevertheless, image or text detection could be used to achieve the same. However, we would expect \systemName~ to perform better during occlusive hand interactions with the flashcards since we can embed a larger number of evenly distributed markers.}




\subsubsection{Offline High-Capacity Data Access for Sensitive Topics}

\systemName~ can discreetly embed critical information into materials, useful for raising awareness about stigmatized topics or promoting sensitive products (e.g., healthcare) while ensuring privacy. For instance, a health organization can distribute posters on mental health with embedded AR markers, allowing access to additional information, such as treatment options, via a simple scan, as shown in \autoref{fig:teaser}c, without leaving a digital footprint. \new{This demonstrates \systemName's key contribution: enabling invisible, high-capacity offline data embedding without requiring users to constantly send data to third-party servers. We also see potential in using offline high-capacity data access to enhance the accessibility of printed media, such as alt-texts, to images for visually impaired people without adapting visual designs to accommodate markers.}



\begin{figure*}[t]
  \centering
  \includegraphics[width=\linewidth]{figures-new/printOnDifferentThings.png}
  \caption{Envisioned extension of \systemName. \new{(a) A user wears AR smart glasses that capture \systemName~ content by exploiting the device's built-in always-on NIR cameras.} (b, c, d) Watermarking on different materials and surfaces to identify personal belongings and facilitate robotic navigation through invisible cues.}
  \Description{.}
  \label{fig:ApplicationsRobotic}
\end{figure*}





\vspace{0.2cm}
\new{\textbf{\systemName's Utility.} The examples in this section illustrate how IR watermarking can function as the "back end" for interactions with printed documents. Much like in web development or software engineering, this approach emphasizes a clear separation between the presentation layer (the "front end") and the data access layer (the "back end"). By abstracting and concealing technical, computer-readable components, our system offers a seamless, user-friendly interface while maintaining robust functionality. This separation enables new possibilities for interaction, surpassing the limitations of visual markers or standard QR codes in terms of abstraction and user experience. We argue that, if IR watermarks become the "back end" of printed media one day, it may pave the way for transformative applications without compromising the complexity and depth of the underlying technology.}







\subsection{Origami: Interactive Folding Instructions}
\systemName~ brings a twist to the traditional art of origami by embedding interactive, step-by-step folding instructions directly onto the paper. As users progress with their folds, new instructions appear on the exposed surfaces, as shown in \autoref{fig:teaser}b, based on the number and combination of the \systemName~ codes visible to the IR camera.
%This crafting experience could make complex designs more engaging and accessible for a larger audience.
The advantage of \systemName~ is that even if the origami paper has no visible content (i.e., white sheet) in order not impact the look of the final artifact, the sheet can still carry invisible folding instructions.

\subsection{Sustainable On-Demand Watermarking}
\systemName~ empowers users to watermark existing posters and documents larger than letter-size format on demand (see \autoref{fig:teaser}a). This may help to safe resources by avoiding re-printing posters or documents in case of missing content. Traditionally, people often go through multiple iterations until they arrive at their desired design. \systemName~ enables people to focus on visual aesthetics with the possibility of adding content wherever and whenever needed. 


\subsection{Beyond Paper Augmentation}
By using a mobile printer, \systemName~ could also add watermarks to plastic~\cite{ozdemir_speed-modulated_2024} (a), fabric~\cite{forman_defextiles_2020} (b), and wood~\cite{gui_draw2cut_2025} (c) and various form factors demonstrated in \autoref{fig:ApplicationsRobotic}. This is possible because the NIR ink is dye-based and can, therefore, be absorbed by these materials. In addition, our CNN detection pipeline can be fine-tuned for detecting, e.g., QR codes on convex or concave surfaces, highlighting the breadth of our contribution. This opens up new possibilities for embedding invisible identification markers to personal belongings (a) and (b). In \autoref{fig:ApplicationsRobotic}c, we also show how \systemName~ may be used for robotics use cases, in this case for robotic vacuum cleaners to safely avoid the pet's area.


\vspace{0.2cm}

Through these scenarios, \systemName~ shows its potential of integrating infrared inkjet watermarking into everyday interactions, and enabling solutions across educational, professional, and personal contexts.

