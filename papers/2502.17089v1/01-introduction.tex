
\section{Introduction}

Despite the growing adoption of digital devices, the use of paper documents remains prevalent in our daily lives~\cite{gitelman_paper_2014}.
However, paper-based information can often lack the interactivity and accessibility of digital media~\cite{holman_paper_2005}.
In recent years, various paper augmentation techniques have been proposed to bridge this gap, enabling users to enhance the information content of their documents by seamlessly integrating digital media~\cite{han_hybrid_2021, li_holodoc_2019, rajaram_paper_2022}.
However, the main disadvantage of these techniques is that the printed markers interfere with printed paper media. 
As a result, they can affect the aesthetics of the print, breaking the illusion that digital is seamlessly embedded in the physical.
This goes as far as distracting consumers from the relevant parts or even degrading overall readability and accessibility.  
This has prevented techniques from going beyond single links or QR codes hidden at the bottom of the document to seamless paper document augmentations with rich digital content~\cite{dogan_standarone_2023}.




To address this challenge, we present \systemName\footnote{The system name \systemName~ is a combination of \textit{imprint} and \textit{pronto} (quickly), indicating that existing documents can be quickly watermarked for additional AR content or metadata embedding.}, a novel infrared inkjet watermarking technique that allows for invisible paper augmentation.
\systemName~ enables users to interact with printed documents in new ways by embedding hidden high-capacity digital information that can be accessed using a smartphone equipped with an infrared (IR) sensor. The technique works by printing an invisible watermark using an IR-absorbing ink. When the paper is illuminated with IR light, the watermark becomes detectable and can be captured by an IR camera.
% This enables users to easily access digital content associated with the paper document without the need for an external device or Internet connection.
In contrast to previous work, \textbf{our watermarking approach leverages the whole document area, including text, graphics, and white space, to increase capacity and support the embedding of a larger amount of information while remaining invisible to the naked eye.}
By utilizing commercially available inkjet printers and inks, we rapidly fabricate these encoded documents without having to include complicated or manual processes such as spraying \cite{willis_hideout_2013} or screen printing \cite{kim_ministudio_2016}.

To inform our system, we first conducted a psychophysical experiment to determine estimates for the maximum amount of IR ink we can embed in paper documents while preventing the IR watermarks from becoming visible to users.
Our results show that IR ink can remain invisible to the human eye, regardless of the background color, i.e., even on white paper. 
Next, we demonstrate that invisible document watermarks can be captured and decoded using our mobile IR sensor and a machine learning (ML) pipeline based on a convolutional neural network (CNN), which binarizes IR watermarks into decodable black-and-white patterns.
Our technical evaluation investigates various factors related to the IR ink machine detectability and data capacity, providing guidelines for content creators and designers who want to use our invisible watermarking technique.  
Together, our results demonstrate that we can embed invisible content into paper documents that can still be decoded by an off-the-shelf hardware and software stack.

To facilitate the embedding process, we developed a software tool that uses the determined IR ink values to embed digital content invisibly onto paper documents in the form of QR codes. 
Here, a user simply uploads an image, specifies the number of watermark embeddings, and marks where these should be located. The software analyzes the background colors given the provided embeddings and paper regions, recommending the amount of IR ink that can be used while remaining invisible. The user can then directly fabricate the document using conventional inkjet printing (see \autoref{fig:teaser}). 

We demonstrate the versatility of \systemName~ through a series of applications, which highlight a variety of use cases covering education, arts and design, desktop publishing, robotics, entertainment, sustainability and security.
Unlike other paper-based augmentation techniques, \new{\textbf{\systemName~ is invisible, works with new and existing documents, and does not require any embedded electronics or special paper.}}


\vspace{0.2cm}
To summarize, the contributions we make are as follows:
% \begin{itemize}[leftmargin=*]
%\setlength{\itemindent}{0em} 
% \vspace{-0.1cm}
\begin{itemize}[leftmargin=0.5cm]
    \item  \systemName, an inkjet printing-based watermarking method that allows the utilization of the whole sheet (both blank and color-printed areas) for invisible data embedding onto new and existing physical media.
    \item A methodology on how to evaluate human perception of IR inkjet ink, and its optimization for invisibility, data capacity and machine detection.
    \item A universal hardware detection module compatible with \textit{USB-C} devices.
    \item An open-source\footnote{\url{https://imprinto.github.io}} machine learning pipeline for binarizing the captured IR watermarks for robust detection and data capacity under different visible background content and IR ink levels.
    \item A set of use cases that demonstrate  \systemName's application space, even beyond paper-based documents.
\end{itemize}