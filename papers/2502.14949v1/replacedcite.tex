\section{Related Work}
The development of robust Optical Character Recognition (OCR) systems has been extensively studied across document layout analysis ____, table detection ____, and document understanding ____. While English OCR benefits from rich datasets like PubLayNet ____, DocBank ____, M6Doc ____, and DocLayNet ____, Arabic lacks standardized benchmarks for diverse fonts and layouts. Recent efforts like MIDAD ____ curates extensive training data for Arabic OCR and handwriting recognition, while Peacock ____ introduces culturally-aware Arabic multimodal models. Existing resources such as CAMEL-Bench ____, LAraBench ____, MADAR ____, OSACT ____, and Tashkeela ____ focus on language modeling or specific tasks rather than full-page OCR evaluation. Handwriting datasets including HistoryAr ____, IFN/ENIT ____, KHATT ____, APTI ____, and Muharaf ____ emphasize word/line recognition over document structure analysis.

\begin{table}[t]
\centering
\small
\begin{tabular}{@{}p{4cm}r@{}}
\toprule
\textbf{Domain} & \textbf{Total Samples} \\
\midrule
PDF to Markdown & 33 \\
Layout & 2,100 \\
Line Detection & 378 \\
Line Recognition & 378 \\
Table Recognition & 456 \\
Image to Text & 3,760 \\
Charts to DataFrame & 576 \\
Diagram to Json & 226 \\
VQA & 902 \\
\midrule
\textbf{Total} & \textbf{8,809} \\
\bottomrule
\end{tabular}
\caption{Distribution of samples across different domains in our dataset. A more detailed count for different sub-domains and data sources is in Appendix \ref{sec:appendix1}.}
\label{tab:dataset-distribution}
\end{table}

\noindent Arabic table recognition faces challenges from merged cells and RTL formatting ____. While methods like GTE ____, GFTE ____, CascadeTabNet ____, TableNet ____, and TableFormer ____ advance Latin table detection, their effectiveness on Arabic documents remains unproven. Document conversion pipelines (CCS ____, Tesseract ____, Docling ____, Surya ____, Marker ____, MinerU ____, PaddleOCR ____) lack Arabic-specific optimizations for segmentation and diacritic handling ____. This highlights the critical need for comprehensive Arabic OCR benchmarks addressing text recognition, table detection, and layout parsing.