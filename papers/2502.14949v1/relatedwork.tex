\section{Related Work}
The development of robust Optical Character Recognition (OCR) systems has been extensively studied across document layout analysis \cite{zhao2024doclayout, shen2021layoutparser, paruchuri2024surya, easyocr, auer2024docling, li2020docbank}, table detection \cite{li2019tablebank, paliwal2019tablenet, nassar2022tableformer, li2021gfte, schreiber2017deepdesrt}, and document understanding \cite{staar2018corpus, weber2023wordscape, livathinos2021robust}. While English OCR benefits from rich datasets like PubLayNet \cite{zhong2019publaynet}, DocBank \cite{li2020docbank}, M6Doc \cite{cheng2023m6doc}, and DocLayNet \cite{doclaynet2022}, Arabic lacks standardized benchmarks for diverse fonts and layouts. Recent efforts like MIDAD \cite{bhatia2024qalam} curates extensive training data for Arabic OCR and handwriting recognition, while Peacock \cite{alwajih2024peacock} introduces culturally-aware Arabic multimodal models. Existing resources such as CAMEL-Bench \cite{ghaboura2024camel}, LAraBench \cite{abdelali2023larabench}, MADAR \cite{bouamor2018madar}, OSACT \cite{mubarak2022overview}, and Tashkeela \cite{zerrouki2017tashkeela} focus on language modeling or specific tasks rather than full-page OCR evaluation. Handwriting datasets including HistoryAr \cite{pantke2014historical}, IFN/ENIT \cite{pechwitz2002ifn}, KHATT \cite{mahmoud2014khatt}, APTI \cite{slimane2009new}, and Muharaf \cite{saeed2024muharaf} emphasize word/line recognition over document structure analysis.


\begin{figure}[ht]
    \centering
    \includegraphics[width=\linewidth]{pics/stats.png}
    \vspace{-7pt}
    \caption{Distribution of the data based on steps in the pipeline
    }
    
    \label{fig:distribution}
    \vspace{-10pt}
    
\end{figure}


\noindent Arabic table recognition faces challenges from merged cells and RTL formatting \cite{pantke2014historical}. While methods like GTE \cite{zheng2021global}, GFTE \cite{li2021gfte}, CascadeTabNet \cite{prasad2020cascadetabnet}, TableNet \cite{paliwal2019tablenet}, and TableFormer \cite{nassar2022tableformer} advance Latin table detection, their effectiveness on Arabic documents remains unproven. Document conversion pipelines (CCS \cite{staar2018corpus}, Tesseract \cite{smith2007overviewtesseract}, Docling \cite{auer2024docling}, Surya \cite{paruchuri2024surya}, Marker \cite{paruchuri2024marker}, MinerU \cite{wang2024mineru}, PaddleOCR \cite{du2020paddleocr}) lack Arabic-specific optimizations for segmentation and diacritic handling \cite{mahmoud2018online, badam2024benjamin}. This highlights the critical need for comprehensive Arabic OCR benchmarks addressing text recognition, table detection, and layout parsing.