
\clearpage
\onecolumn
\appendix
    \begin{center}
    \Large
    \textbf{Appendix}
     \\[20pt]
    \end{center}
   
\section{Details of experiments}
\subsection{Dataset}
The node classification benchmark with structured OOD for real-world citation networks is constructed based on the data in ArnetMiner\cite{tang2008arnetminer}, with papers published in various time ranges and collected from various sources. The Citation benchmark comprises three real-world paper citation networks, namely ACMv9, Citationv1, and DBLPv7, where ACMv9 network consists of papers published after 2010 from ACM, Citationv1 network is collected from Microsoft Academic Graph with papers published before 2008, and DBLPv7 network contains papers published between 2004 and 2008 on DBLP. In the citation networks, each node corresponds to a paper and edges signify the citation relationships. As for the node attributes, we generate bag-of-words vectors utilizing all the keywords from the paper titles. Each node belongs to some of the five categories:  Computer Vision, Databases, Networking, Information Security, and Artificial Intelligence. Table \ref{tab:dataset_citation} has listed the detailed statistics of the citation network dataset.
and Figure \ref{fig:dblp_all} is the visualization of the DBLP dataset.
In practical, we use the data from a public  (\href{https://github.com/shenxiaocam/ACDNE/tree/master/ACDNE_codes/input}{GitHub repository})\cite{shen2020adversarial}.
\begin{figure*}[th]
    \centering
    \begin{minipage}[s]{0.4\linewidth}
    \centering
    \includegraphics[width=1.0\linewidth]{images/Micro_GCN.pdf}
    \end{minipage}
    \begin{minipage}[s]{0.4\linewidth}
    \centering
    \includegraphics[width=1.0\linewidth]{images/Macro_GCN.pdf}
    \end{minipage}

     \caption{The comparison between random edge-dropping and low weight dropping (backbone: GCN, AD $\rightarrow$ C)}
    \label{fig:all_acc_gcn}
\end{figure*}

\begin{figure*}[th]
    \centering
          % \vspace{2mm}
    % \includegraphics[width=0.99\linewidth]
    \includegraphics[width=0.88\linewidth]{images/dblp_all.pdf}
    % \vspace{-1mm}
     \caption{The visualization of the DBLP dataset}% pseudo label $\hat{y}_i$ }
    \label{fig:dblp_all}
        % \vspace{2mm}
\end{figure*}

\subsection{Additional Experiments}
Figure \ref{fig:all_acc_gcn} shows the results concerning the comparison between our  low weight dropping and random edge-dropping.