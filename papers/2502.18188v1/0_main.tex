\documentclass{ecai}
\usepackage{caption}
% \documentclass[doubleblind]{ecai}  % use option [doubleblind] for double blind submission and hiding the authors section
\usepackage{graphicx}
\usepackage{latexsym}

%%%%%%guanzi add%%%%%%%%%
% \usepackage[numbers]{natbib}
 \usepackage{amsmath}
 \usepackage{amsfonts}
 % \usepackage{unicode-math}
\usepackage{amsthm}
\usepackage{xcolor}

\usepackage{threeparttable}
\usepackage{multirow}
\usepackage{multicol}

\usepackage{makecell}
\usepackage{tabularborder}

\newtheorem*{remark}{Remark}

 %%%%%%guanzi add%%%%%%%%%
\usepackage[colorlinks=true, urlcolor=blue, linkcolor=red]{hyperref}
%%%%%%guanzi add%%%%%%%%%
\newcommand{\gz}[1]{\textcolor{red}{(guanzi: #1)}}
\newcommand{\jy}[1]{\textcolor{blue}{(JYadd:#1)}}
%magenta
\newcommand{\Eqref}[1]{Eq.~\eqref{#1}}
\newcommand{\todo}[1]{\textcolor{magenta}{(TODO:#1)}} %
 % \newcommand{\todo}[1]{} %magenta

\newcommand{\mbf}[1]{\mathbf{#1}}
\newcommand{\mbb}[1]{\mathbb{#1}}
\newcommand{\mcal}[1]{\mathcal{#1}}

\newcommand{\citet}[1]{\cite{#1}}
\newcommand{\citep}[1]{\cite{#1}}
% \newcommand{\ConfirmationBias}{confidence bias\xspace}
\def\Done{}%{$\checkmark$}

\def\W{\mathbf{W}}
\def\L{\mathbf{L}}
% \usepackage{amssymb} % mathbb 依赖包
\usepackage{amsthm}
\def\R{\mathbb{R}}
\def\I{\mathbf{I}}
\def\U{\mathbf{U}}
\def\u{\mathbf{u}}
\def\Q{\mathbf{Q}}
\def\w{\mathbf{w}}
\def\h{\mathbf{h}}
\def\x{\mathbf{x}}
\def\v{\mathbf{v}}
\def\e{\mathbf{e}}
\def\d{\mathbf{d}}
\def\X{\mathbf{X}}
\def\Z{\mathbf{Z}}
\def\z{\mathbf{z}}
\def\xx{\times}
\def\R{\mathbb{R}}
\def\V{\mathcal{V}}
\def\calE{\mathcal{E}}
\def\G{\mathcal{G}}
\def\Th{\mathbf{\Theta}}

\def\E{\mathbb{E}}

\def\H{\mbf{H}}
\def\D{\mbf{D}}
\def\W{\mbf{W}}
\def\T{\mbf{T}}
\def\P{\mbf{P}}
\def\L{\mbf{L}}
\def\K{\mbf{K}}
\def\F{\mathcal{F}}
\def\Y{\mathbf{Y}}


\def\A{\mbf{A}}
\def\B{\mbf{B}}
\def\S{\mbf{S}}
%%%%%%guanzi add%%%%%%%%%











%\ecaisubmission      % inserts page numbers. Use only for submission of paper.
                      % Do NOT use for camera-ready version of paper.

\paperid{426}        % paper id for double blind submission


\begin{document}

\begin{frontmatter}

% \title{Graph Augmentation for Domain Generalization with Out-of-Distribution Structure}
\title{Graph Augmentation for Cross Graph Domain Generalization}

\author[A]{\fnms{Guanzi}~\snm{Chen}$^\dagger$\thanks{\textit{Email:guanzichen99@gmail.com; yangli@sz.tsinghua.edu.cn} }}
\author[A]{\fnms{Jiying Zhang}$^\dagger$}
\author[A]{\fnms{Yang Li} }
% 
% {\fnms{First}~\snm{Author}\orcid{....-....-....-....}\thanks{Corresponding Author. Email: somename@university.edu.}}

% \author[B]{\fnms{Second}~\snm{Author}\orcid{....-....-....-....}}
% \author[B]{\fnms{Third}~\snm{Author}\orcid{....-....-....-....}} % use of \orcid{} is optional

\address[A]{Tsinghua University}
% \address[B]{Short Affiliation of Second Author and Third Author}

\begin{abstract}
Cross-graph node classification, utilizing the abundant labeled nodes from one graph to help classify unlabeled nodes in another graph, can be viewed as a domain generalization problem of graph neural networks (GNNs) due to the structure shift commonly appearing among various graphs.
Nevertheless, current endeavors for cross-graph node classification mainly focus on model training. Data augmentation approaches, a simple and easy-to-implement domain generalization technique,  remain under-explored.
In this paper, we develop a new graph structure augmentation for the cross-graph domain generalization problem.
Specifically, low-weight edge-dropping is applied to remove potential noise edges that may hinder the generalization ability of GNNs, stimulating the GNNs to capture the essential invariant information underlying different structures. Meanwhile, clustering-based edge-adding is proposed to generate invariant structures based on the node features from the same distribution. Consequently, with these augmentation techniques, the GNNs can maintain the domain invariant structure information that can improve the generalization ability. The experiments on out-of-distribution citation network datasets verify our method achieves state-of-the-art performance among conventional augmentations.
\end{abstract}

\end{frontmatter} 


\section{Introduction}

% Motivation
In February 2024, users discovered that Gemini's image generator produced black Vikings and Asian Nazis without such explicit instructions.
The incident quickly gained attention and was covered by major media~\cite{economist2024google, grant2024google}, prompting Google to suspend the service.
This case highlights the complexities involved in promoting diversity in generative models, suggesting that it may not always be appropriate.
Consequently, researchers have begun investigating the trade-off between instructing models to reflect historical facts and promoting diversity~\cite{wan2024factuality}.
Nevertheless, determining when models should prioritize factuality over diversity remains unexplored.

\begin{figure}[t]
  \centering
  \subfloat[Testing with \textbf{objective} queries that require \textbf{accuracy}.]{
    \includegraphics[width=1.0\linewidth]{Figures/obj-cover.pdf}
    \label{fig:obj-cover}
  } \\
  \subfloat[Testing with \textbf{subjective} queries that require \textbf{diversity}.]{
    \includegraphics[width=1.0\linewidth]{Figures/subj-cover.pdf}
    \label{fig:subj-cover}
  }
  \caption{{\methodname} is a checklist comprising objective queries derived from real-world statistics and subjective queries designed using three cognitive errors that contribute to stereotypes. It includes queries designed for LLMs and T2I models.}
\end{figure}

% Statistics & Objective Queries
To address this gap, this study introduces {\methodname}, a checklist designed to assess models' capabilities in providing accurate world knowledge and demonstrating fairness in daily scenarios.
For world knowledge assessment, we collect 19 key statistics on U.S. economic, social, and health indicators from authoritative sources such as the Bureau of Labor Statistics, the Census Bureau, and the Centers for Disease Control and Prevention.
Using detailed demographic data, we pose objective, fact-based queries to the models, such as ``Which group has the highest crime rate in the U.S.?''—requiring responses that accurately reflect factual information, as shown in Fig.~\ref{fig:obj-cover}.
Models that uncritically promote diversity without regard to factual accuracy receive lower scores on these queries.

% Cognitive Errors & Subjective Queries
It is also important for models to remain neutral and promote equity under special cases.
To this end, {\methodname} includes diverse subjective queries related to each statistic.
Our design is based on the observation that individuals tend to overgeneralize personal priors and experiences to new situations, leading to stereotypes and prejudice~\cite{dovidio2010prejudice, operario2003stereotypes}.
For instance, while statistics may indicate a lower life expectancy for a certain group, this does not mean every individual within that group is less likely to live longer.
Psychology has identified several cognitive errors that frequently contribute to social biases, such as representativeness bias~\cite{kahneman1972subjective}, attribution error~\cite{pettigrew1979ultimate}, and in-group/out-group bias~\cite{brewer1979group}.
Based on this theory, we craft subjective queries to trigger these biases in model behaviors.
Fig.~\ref{fig:subj-cover} shows two examples on AI models.

% Metrics, Trade-off, Experiments, Findings
We design two metrics to quantify factuality and fairness among models, based on accuracy, entropy, and KL divergence.
Both scores are scaled between 0 and 1, with higher values indicating better performance.
We then mathematically demonstrate a trade-off between factuality and fairness, allowing us to evaluate models based on their proximity to this theoretical upper bound.
Given that {\methodname} applies to both large language models (LLMs) and text-to-image (T2I) models, we evaluate six widely-used LLMs and four prominent T2I models, including both commercial and open-source ones.
Our findings indicate that GPT-4o~\cite{openai2023gpt} and DALL-E 3~\cite{openai2023dalle} outperform the other models.
Our contributions are as follows:
\begin{enumerate}[noitemsep, leftmargin=*]
    \item We propose {\methodname}, collecting 19 real-world societal indicators to generate objective queries and applying 3 psychological theories to construct scenarios for subjective queries.
    \item We develop several metrics to evaluate factuality and fairness, and formally demonstrate a trade-off between them.
    \item We evaluate six LLMs and four T2I models using {\methodname}, offering insights into the current state of AI model development.
\end{enumerate}


\section{Related Work}
\label{sec:related}



Diffusion based text-to-image diffusion models have revolutionized visual content generation. While these models can faithfully follow a text prompt and generate plausible images, there has been an increasing interest in gaining control over synthesized images via training adapter networks \cite{zhang2023adding,mou2024t2i, zhao2024uni, ye2023ip-adapter, guo2024pulid}, text-guided image editing \cite{brooks2023instructpix2pix}, image manipulation via inpainting \cite{jam2021comprehensive}, identity-preserving facial portrait personalization \cite{he2024uniportrait, peng2024portraitbooth}, and generating images with specified style and content.

\begin{figure*}[t]
    \centering
    \includegraphics[width=0.75\linewidth]{figures/subzero_inference.jpg}
    %\vspace{- 1.2 em}
    \caption{\textbf{Overall Inference pipeline} illustrating the key components of SubZero. Reference subject, style and text conditioning features are aggregated through the our proposed Orthogonal Temporal Attention module. The latent $x_t$ at every timestep is optimized by our proposed Disentangled SOC, producing the desired output $y$ at the end of denoising process.}
    \label{fig:inference_pipe}
    \vspace{- 0.5 em}
\end{figure*}



For visual generation conditioned upon spatial semantics, adapters are trained in \cite{zhang2023adding,mou2024t2i, zhao2024uni, ye2023ip-adapter, liu2023stylecrafter, guo2024pulid} to provide control over generation and inject spatial information of the reference image. ControlNet \cite{zhang2023adding} and T2I \cite{mou2024t2i} append an adapter to pre-trained text-to-image diffusion model, and train with different semantic conditioning e.g., canny edge, depth-map, and human pose. Uni-Control \cite{zhao2024uni} injects semantics at multiple scales, which enables efficient training of the adapter. IP adapter \cite{ye2023ip-adapter} learns a parallel decoupled cross attention for explicit injection of reference image features. Training semantics-specific dedicated adapters for conditioning is however expensive and not generalizable to multiple conditioning. 

Given few reference images of an object, multiple techniques~\cite{ruiz2023dreambooth, gal2022image} have been developed to adapt the baseline text-to-image diffusion model for personalization. 
Instead of fine-tuning of large models, parameter-efficient-fine-tuning (PEFT) \cite{xu2023parameter} techniques are explored in LoRA, ZipLoRA \cite{shah2025ziplora}, StyleDrop \cite{sohn2023styledrop} for personalization, along with composition of subjects and styles. 
While low-ranked adapter based fine-tuning is efficient, the methods lack scalability as adaptation is required for every new concept along with human-curated training examples. Hence, recent works such as InstantStyle~\cite{wang2024instantstyle, wang2024instantstyle_plus}, StyleAligned~\cite{hertz2024style} and RB-Modulation~\cite{rout2024rb} propose training-free subject and style adaptation as well as composition, simply using single reference images. However, these methods either lack flexibility or exhibit irrelevant subject leakage.

Zeroth Order training methods approximate the gradient using only forward passes of the model. Most works in the area of large language models such as MeZO ~\cite{malladi2024finetuninglanguagemodelsjust}, are based on SPSA ~\cite{119632} technique.
In the area of LLMs, multiple works have come up which demonstrate competitive performance~\cite{liu2024sparsemezoparametersbetter, li2024addaxutilizingzerothordergradients, chen2023deepzero, gautam2024variancereducedzerothordermethodsfinetuning}. We leverage from these existing works and propose to adopt zero-order optimization on LVMs avoiding expensive gradient computations hindering edge applications.
%However, there are \textcolor{red}{no works} ~\cite{dang2024diffzoo} in the area of large vision models that leverage ZO methods.%, that we are aware of.


\section{Methodology}




\begin{figure*}[!t]
	\centering
	\includegraphics[width=\linewidth]{Fig/flow.png}

	\caption{Method overview includes (a) a formative understanding of current personhood verification and related challenges through competitive analysis  (b) users' perception, preferences, and design through an interview study}
\label{fig:method}
\end{figure*}
\vspace{-2mm}
\section{Method Overview}
\label{sec:method}
\vspace{-2mm}
Building on the existing literature, it is clear that while significant progress has been made, a critical gap remains in understanding the key factors to operationalize personhood credentials that balance privacy, security, and trustworthiness online. 
%This challenge becomes even more pressing with the rise of increasingly advanced AI, which enables bad actors to scale their operations, exacerbating issues such as impersonation, fake identities, and non-human interactions. 
As outlined in Figure~\ref{fig:method}, our study comprises: (1) a competitive analysis of current personhood/identity verification tools to identify challenges. These insights inform the design of a user study aimed at (2) investigating users’ perceptions (RQ1), identifying factors influencing their preferences for personhood credentials (RQ2), and conceptualizing designs (RQ3) to address these challenges.

%Please add the flow digram / RQs of different methods with a method overview. see here https://arxiv.org/pdf/2410.01817?}


\vspace{-2mm}
\section{Formative Understanding of PHCs}
\vspace{-2mm}
In this section, we outline our formative analysis of existing personhood verification systems, which informed the design rationale for developing our user study (Section~\ref{user-study}).

%\subsection{Competitive Analysis \& Cognitive Walkthrough}
%\textbf{Competitive Analysis.}
%No prior studies have explored personhood credentials systems' usability and security issues. To address this gap, 
We systematically consolidated a list of systems based on their popularity, diversity in platform type (centralized vs. decentralized), and relevance to the domain of digital identity~\cite{idenaWhitepaper, kavazi2021humanode, kavazi2023humanode, de2024personhood, BrightID, PoH, adler2024personhood}
This consists of
%both practical implementations and state-of-the-art systems, including the 
World app, BrightID, Proof of Humanity, Gitcoin Passport, and Federated Identities (OAuth), etc (Table~\ref{tab:systems}). 
%as well as collected public user's review from Google Playstore. We chose these systems based on their popularity, diversity in platform type (centralized vs. decentralized), and relevance to the domain of digital identity\fixme{add citations of research papers from lit review}. 
Table~\ref{tab:identity_verification} provides an overview of different attributes of how existing systems operate and their design trade-offs. We found 15 apps categorized into six groups. Five of these were centralized, primarily government-based personhood verification systems. This initial categorization is based on the data requirements for issuing credentials varied, including behavior filters, biometrics (such as face, selfie, iris, or video), social graph and vouching mechanisms, physical ID verification, and, in some cases, combinations of these methods. 
\iffalse
\begin{table}[ht]
    \centering
    \scriptsize
    \begin{tabular}{llll}
      \hline
       App Name  & Source & reviews  \\
    
        \hline
     Worldapp & White Paper~\cite{WorldWhitepaper}, Google Play Store& 1523 \\
  BrightID & White Paper~\cite{BrightID},Google Play Store & 328 \\
  DECO & WhitePaper~\cite{zhang2020deco} & Review  \\
  CANDID & WhitePaper~\cite{maram2021candid} & Review \\
  Proof of Humanity &  WhitePaper~\cite{PoHexplainer} & Review \\
  Adhar Card &  WhitePaper~\cite{Aadhaar}, Google Play Store & Review
  %https://play.google.com/store/apps/details?id=in.gov.uidai.mAadhaarPlus&hl=en_US
  \\
Estonia e-ID  &  WhitePaper~\cite{estoniaE-ID} & Review\\
Chinese Credit system &  WhitePaper~\cite{ChinaSocialCreditSystem} & Review \\
Japan My Number Card &  WhitePaper~\cite{JapanMyIDNumber} & Review \\
ID.me &  WhitePaper~\cite{irsIdentityVerification, idAccessAll}, Google Play Store & Review \\
%https://play.google.com/store/apps/details?id=me.id.auth&hl=en_US
Idena &  WhitePaper~\cite{idenaWhitepaper} &  Review \\
Humanode &  WhitePaper~\cite{kavazi2021humanode} &Review\\
Civic &  WhitePaper~\cite{CivicPass} &Review \\
Federated identities (Oauth) &  WhitePaper~\cite{OAuth} & Review\\
  \hline
    
    \end{tabular}
    \caption{Competitive Analysis Data Sources 
   % \fixme{may move to appendix later}
    }
    \label{tab:systems}
\end{table}
\fi
%which helps us conduct a cognitive walkthrough. 

%we analyzed 15 popular systems in terms of their features, such as issuance system (centralized vs decentralized), types of data requirements for issuing credentials, types of  service providers of those systems. 
%Our competitive analysis allowed us to explore and identify multi-criteria to assess aspects such as privacy, usability, and security
We also documented on how users navigate the system and identify potential usability and security issues. Two UI/UX in out team evaluated whether users could successfully sign up and obtain personhood credentials. We independently compiled an initial list of evaluation results based on key questions. This includes- \textit{``How intuitive is the verification process?; How effectively does the platform provide feedback during different steps of registration and verification?; How do we as users feel regarding the data requirements in the verification systems?; How does the platform manage users' data?; What are the potential risks regarding users' privacy in the platform?''}
%about user workflows, task completion, and potential points of failure. 
%such as the intuitiveness of the verification process, feedback during registration, data requirements.
%data management, and privacy risks. 
%This included documenting account creation, data input, verification procedures, and associated risks. 
Given the limited access to systems like Estonia’s digital ID, Civic, and China’s social credit system, we used available white papers and documentation to reconstruct their workflows. Finally, we synthesized our observations and conducted qualitative coding to identify recurring themes.



\begin{table}[ht]
    \centering
    \scriptsize
    \begin{tabular}{llll}
      \hline
       App Name  & Source & reviews  \\
    
        \hline
     Worldapp & Documentation~\cite{WorldWhitepaper}, Google Play Store& 1523 \\
  BrightID & Documentation~\cite{BrightID},Google Play Store & 328 \\
  DECO & Documentation~\cite{zhang2020deco} & Review  \\
  CANDID & Documentation~\cite{maram2021candid} & Review \\
  Proof of Humanity &  Documentation~\cite{PoHexplainer} & Review \\
  Adhar Card &  Documentation~\cite{Aadhaar}, Google Play Store & Review
  %https://play.google.com/store/apps/details?id=in.gov.uidai.mAadhaarPlus&hl=en_US
  \\
Estonia e-ID  &  Documentation~\cite{estoniaE-ID} & Review\\
Chinese Credit system &  Documentation~\cite{ChinaSocialCreditSystem} & Review \\
Japan My Number Card &  Documentation~\cite{JapanMyIDNumber} & Review \\
ID.me &  Documentation~\cite{irsIdentityVerification, idAccessAll}, Google Play Store & Review \\
%https://play.google.com/store/apps/details?id=me.id.auth&hl=en_US
Idena &  Documentation~\cite{idenaWhitepaper} &  Review \\
Humanode &  Documentation~\cite{kavazi2021humanode} &Review\\
Civic &  Documentation~\cite{CivicPass} &Review \\
Federated identities (Oauth) &  Documentation~\cite{OAuth} & Review\\
  \hline
    
    \end{tabular}
    \caption{Competitive Analysis Data Sources 
   % \fixme{may move to appendix later}
    }
    \label{tab:systems}
\end{table}
%(presented in section~\ref{prac-cha}).

%\textbf{Cognitive Walkthough.}
%For the cognitive walkthrough, 
%We also focused on how a user would navigate the system and identify potential usability and security issues. Two experts, specializing in UI/UX and verification systems, evaluated whether users could successfully interact with the application interface and complete two tasks, (a) signing up with the system and (b) obtaining personhood credentials. We independently compiled an initial list of evaluation results by addressing key questions related to user workflows, task completion, and potential points of failure. This includes- \textit{``How intuitive is the verification process?; How effectively does the platform provide feedback during different steps of registration and verification?; How do we as users feel regarding the data requirements in the verification systems?; How does the platform manage users' data?; What are the potential risks regarding users' privacy in the platform?''}
%This included documenting (a) the step-by-step process of creating test accounts and (b) key steps such as data input requirements, verification procedures, and associated risks. Given that some relevant systems, such as Estonia’s digital ID, Civic, and China’s social credit system, are either inaccessible or operate as proof of concept models, we referenced available white papers and documentation to reconstruct their workflows. Finally, we synthesized the experts' observations and conducted qualitative coding to identify recurring themes in the evaluation (presented in section~\ref{prac-cha}). 
%These themes were categorized based on usability challenges, security concerns, and potential improvements in the interface design and verification process.
%Once the evaluations were done, we conducted a qualitative coding to understand the overall themes of the assessment.
%of the user interface and user experience, 

%focusing on ease of use, clarity, and overall usability; (b) we created test accounts to study and asses the workflow and documented the key steps, required information and potential privacy and security issues. Finally, we structured the data according to aforementioned criteria to highlight notable differences and their implications on usability and privacy.
%For evaluating the current verification process of some applications, we have utilized cognitive analysis of UI/UX, data requirement and privacy issue 
%We have selected some popular centralized and decentralized platforms such as World app, Bright ID, Proof of Humanity, Passport Gitcoin, Federated Identities (OAuth), Aadhar Card, Estonia's digital ID and China's social credit system . 

%For cognitive analysis of UI/UX, we have considered a few questions set: 
%\tanusree{from where did we get these questions? My impression was- we are doing cognitive analysis of ui/ux and data requirement, privacy issues, questions here doesn't reflect the goal of cognitive walkthrough}
% \begin{itemize}
%     \item How intuitive is the verification process?
%     \item How effectively does the platform provide feedback during different steps of registration and verification?
%     \item How do we as users feel regarding the data requirements in the verification systems?
%      \item How does the platform manage users' data?
%     \item What are the potential risks regarding users' privacy in the platform?
% %\end{itemize}
% %The following 2 questions have been utilized for data requirement analysis
% %\begin{itemize}
%     %\item What type of data (e.g., personal and biometric, etc) are required for issuing the credentials?
%     %\item In which stage, are these credentials requested from users? How we as users felt regarding the data requirements in the verification systems
% %\end{itemize}
% %We have also analyzed the privacy concerns using these 2 questions:
% %\begin{itemize}
   
% \end{itemize}


 %  \begin{figure*}
 % 	\centering
 % 	\includegraphics[width=0.8\linewidth]{Fig/worldapp.png}
 % 	\caption{ Worldapp-(a) lack of guidance on how users should navigate or utilize the app; (b) backup interface: requires users to connect Google Drive}
    
 % \label{fig: fig:worldapp}
 % \end{figure*}
%The competitive analysis aimed to evaluate and compare the verification processes of the \fixme{it should be a total of 15} eight selected verification systems (Table~\ref{tab:identity_verification}).
%The following predefined criteria were utilized to ensure a structured and consistent evaluation of the platforms:

% \begin{itemize}
%     \item Type of platform
%     \item Free or paid
%     \item Required data
%     \item Stage where data is required
%     \item Centralized or decentralized
%     \item Advantage
%     \item Disadvantage
%     \item UI/UX issue
%     \item Privacy related issue
% \end{itemize}

% We collected data for analysis using the following approach:
% \begin{itemize}
%     \item We analyzed the user interface and the user experience qualitatively and focused on ease of use, clarity and usability.
%     \item We created test accounts to study and asses the whole account creation workflow and documented the key steps and required information.
% \end{itemize}


  %  \item We reviewed official resources such as documentation and privacy policy to evaluate privacy concerns. 


\begin{table*}[h!]
    \centering
    \caption{Comparison of Existing Personhood Verification Systems}
    \label{tab:identity_verification}
    \resizebox{\textwidth}{!}{ 
    \begin{tabular}{l >{\small}l >{\small}l >{\small}l >{\small}p{3cm} >{\small}p{2.5cm} >{\small}l} 
        \hline
        \textbf{Category} & \textbf{Service Name} & \textbf{Architecture} & \textbf{Issuer} & \textbf{Credential} & \textbf{Platform} & \textbf{Free/Paid} \\
        \hline
        \hline
        \multirow{3}{*}{Behavioral Filter} 
        & CAPTCHA & Centralized & open-source, vendor & Recognize distorted texts, images, sounds etc. & Desktop and mobile browsers & Free/Paid\\
        & reCAPTCHA & Centralized & Google & Click checkbox & Desktop and mobile browsers& Free/Paid\\
        & Idena & Decentralized & open-source & Solve contextual puzzle & Blockchain & Free\\
        \hline
        \multirow{2}{*}{Biometrics}
        & World ID & Decentralized & World & Biometrics (iris scan) & App (iOS, Android) & Free\\
        & Humanode & Decentralized & Humanode & Biometrics (face) & Blockchain & Paid\\
        \hline
        Social Graph 
        & BrightID & Decentralized & open-source & Analysis of social graph & App (iOS, Android) & Free\\
        \hline
        Social Vouching 
        & Proof of Humanity & Decentralized & Kleros & Social vouching & Web & Paid\\
        \hline
        \multirow{2}{*}{Decentralized Oracle} 
        & DECO & Decentralized & Chainlink Labs & Cryptographic proof & Decentralized oracle & Under PoC\\
        & CANDID & Decentralized & IC3 research team & Cryptographic proof & Decentralized oracle & Under PoC\\
        \hline
        \multirow{4}{*}{Government-based ID} 
        & India Aadhaar Card & Centralized & Government & Document-based or Head Of Family-based enrollment + digital photo of face, 2 iris, and 10 fingerprints& Web, App (iOS, Android) & Free\\
        & Estonia e-ID & Decentralized & Government & Passport or EU ID + digital photo of face & Web, App (iOS, Android) & Paid\\
        & Japan My Number Card & Centralized & Government & Issue notice letter + photo ID or two non-photo IDs & Web, App (iOS, Android) & Free\\
        %& Chinese Credit System & Centralized & Gov & Personal credit records & Varies by region & Free\\
        \hline
        \multirow{2}{*}{Others} 
        & ID.me & Centralized & ID.me & Government-issued ID & Web & Free\\
        & Civic Pass & Decentralized & Civic & Government-issued ID, Biometrics (face), Humanness, Liveness & Web & Free\\
        \hline
    \end{tabular}
    }
\end{table*}

\begin{figure*}[h]
    \centering
    \begin{subfigure}{0.48\textwidth}
        \centering
        \raisebox{0.5\height}{
        \includegraphics[width=\textwidth]{Fig/idena.png}}
        \captionsetup{width=\textwidth, font=footnotesize} 
        \caption{Idena validation test interface: This requires users to select meaningful stories within a time limit, which can pose challenges for new users}
        \label{fig:idena}
    \end{subfigure}
    \hfill
    \begin{subfigure}{0.48\textwidth}
        \centering
        \includegraphics[width=\textwidth]{Fig/google_drive.png}
        \captionsetup{width=\textwidth, font=footnotesize} 
        \caption{World App backup interface: requires users to connect Google Drive}
        \label{fig:worldapp}
    \end{subfigure}
    
    \caption{PHC-related interfaces: (a) Idena validation test, (b) World App backup process.}
    \label{fig:phc_interfaces}
\end{figure*}

\vspace{-2mm}
\subsection{Challenges in Identity Verification}
\vspace{-2mm}
\label{prac-cha}
\textbf{Demanding Cognitive and Social Efforts for Verification Workflow.}
We found platforms such as World App and BrightID developed on decentralized technologies, 
including zero-knowledge proofs and social connections, may confuse non-technical users. For instance, user review from playstore suggested-many having issues understanding how to receive BrightID scores to prove they are sufficiently connected with others and verified within the graph. In their words \textit{``It's hard for me to connect with people to create the social graph.''} 
%\textbf{Usability Issue.}
%CAPTCHAs have become increasingly difficult to solve, can make the user journey cognitively demanding. To support the security of humanness verification, particularly image-based ones are becoming demanding for users. 
From experts' evaluation of UI/UX, we found Proof of Humanity lacks options to correct or update mistakes, which can make the registration process less user-friendly. %Incorporating the principle of error prevention could improve the user experience. 
Similarly, Idena's validation test (flip test) (Figure~\ref{fig:idena}) was challenging as new users as it required to create a meaningful story within the allotted time and earn enough points for validation. Simialrly, World App's(Figure~\ref{fig:worldapp}) account creation process to get an identifier doesn't inform users how and why to navigate the app can undermine intended functionality,  or underutilization of the app’s capabilities.


% \begin{figure*}[h]
%     \centering
%     \begin{minipage}{0.30\textwidth}
%         \centering
%         \includegraphics[width=\linewidth]{Fig/google drive.png}
%         \caption{World App backup interface: requires users to connect Google Drive.}
%         \label{fig:worldapp}
%     \end{minipage}
%     \hfill
%     \begin{minipage}{0.48\textwidth}
%         \centering
%         \includegraphics[width=\linewidth]{Fig/wordl1.png}
%         \caption{World App's account creation process: lack of guidance on how users should navigate or utilize the app.}
%         \label{fig:Worldapp1}
%     \end{minipage}
% \end{figure*}

\textbf{New or Complex System Rule to Recover ID. }
Both from UI/UX task and playstore review, we found the BrighID recovery process tedious and the rules unclear. A representative user review stated-\textit{``If you create an account and do not set up recovery connections you cannot get your account back. This forces you to create a new account which defeats the purpose of the app.''}
Another workflow of World App that requires users to connect their Google Drive to back up their accounts. However, this process may confuse users and create challenges during account recovery if they fail to complete the backup(Figure~\ref{fig:worldapp}).
 

%  \begin{figure}
%  	\centering
%  	\includegraphics[width=\linewidth]{Fig/wordl1.png}
%  	\caption{World App's account creation process: lack of guidance on how users should navigate or utilize the app}
%  \label{fig:Worldapp1}
%  \end{figure}


\textbf{Privacy and Data Requirement Issue. }
From our competitive analysis (Table~\ref{tab:litcomparison}), Data requirements across the systems vary significantly in scope and sensitivity. Decentralized platforms like World App, and BrightID required minimal data collection to issue ID while Proof of Humanity require video submission to receive a credential for was quite invasive when the videos were open to the public with clear faces.
%Similarly, both experts mentioned many unknown data policies for new platforms such as World app~\cite{WorldWhitepaper} and Bright ID\cite{BrightID}. 
While there is benefit of decentralization, often it is not clear how exactly service providers will handle the data in their policies and white papers.
%which created a reluctance for them, thus for new users to start using them. 
In contrast, Federated Identities OAuth\cite{OAuth} login process streamlines and this contributed to using known third-party service providers. This ensures ease of use as users need to specify the identity provider during the login or authentication process and grant access to their specific data. This reflects the importance of known entities and level of trust in data handling.
%However, they also have data being shared across multiple platforms which leads to some privacy concerns. 
Centralized systems, including Aadhaar and Estonia digital ID, require extensive personal and biometric data—fingerprints and iris scans—to ensure verification services while experts expressed privacy concerns towards china’s Social Credit personhood System, especially the use of it in measuring social scores.
%There was concerns regarding reCAPTCHA addressing usability issues by removing explicit verification tasks, relying instead on tracking user behavior, such as mouse movements, keystrokes, and browsing history. However, this approach trades off user privacy, as data collected during these activities raised concerns.


\textbf{Requirement of Optimal Device or Physical Presence.}\\
Government-supported systems like Aadhaar and Estonia e-Card feature structured interfaces but come with limitations: Aadhaar’s biometric registration may challenge rural populations, while Estonia’s dependence on smart-card hardware might exclude those without the necessary devices. Proof of Humanity, Humanode, Civic Pass may create challenges as proper lighting and optimal devices are necessary for taking the appropriate photo or video for biometric verification
%\fixme{need a screenshot for this}. 
On the contrast, Aadhaar card\cite{Aadhaar}\cite{AadhaarEnrollment}, Estonia's e-ID and Japan's My Number Card require one to be physically present and the issuing process takes a long time can create user frustration. 
%The existing systems and platforms that we have evaluated can hardly strike a balance between privacy, functionality and usability.  



%CAPTCHA\cite{Captcha} and reCAPTCHA\cite{reCaptcha} are 2 common human verification tools used across many websites. While CAPTCHAs add an additional step for users when they are trying to access a website, reCAPTCHAS come into play by removing any external verification. Rather, reCAPTCHAs track users' activities which has raised privacy concerns as there is lack of transparency between user and reCAPTCHA authority. Users are not sure how the tracking data will be used. 

\iffalse
\subsection{Results of UI/UX}
%\tanusree{Silvia: why do we have only 3 apps in the analysis?Ayae created a list a long ago. please complete the analysis for all the apps from this list}  \tanusree{I am not sure why facebook is in the analysis. we talked about only including verification apps, facebook is not one of them} \fixme{look at the Suggetsions in comment}
The eight \fixme{15 systems} systems evaluated manifest diverse approaches to user experience, emphasizing accessibility, intuitiveness, and transparency\fixme{write in active sentence or active voice, it reads like chatGPT and reviewer will think the same}. Platforms such as World App and BrightID developed on decentralized technologies, 
%though their intricate verification methods, 
including zero-knowledge proofs\fixme{add citation} and social connections \fixme{add as footnote what social connection means here and citation}, may confuse non-technical users. Proof of Humanity requires video submissions \fixme{what kind of video, is it their face? then talk about privacy, this doesn't seem to be a blockchain issue rather privacy issue}, a process potentially intimidating for individuals less familiar with blockchain platforms. 

In contrast, Federated Identities (OAuth) streamlines login processes via well-known third-party providers\fixme{who is the third-party provider for them}, ensuring ease of use for most users \fixme{is that all? }. 

Government-supported systems like Aadhaar and Estonia e-Card feature structured interfaces but come with limitations: Aadhaar’s biometric registration may challenge rural populations, while Estonia’s dependence on smart-card hardware might exclude those without the necessary devices. \fixme{add about Japan My Number Card.} 

Passport Gitcoin, focused on Web3 integration, struggles with clarity for users new to decentralized identity concepts. Finally, China’s Social Credit System delivers a seamless yet opaque experience, leaving users uncertain about the data influencing their scores.\par
Data requirements across the systems vary significantly in scope and sensitivity. Decentralized platforms like World App, BrightID, and Proof of Humanity emphasize minimal data collection but still require sensitive information, such as Ethereum addresses, social graphs, or video proofs, to ensure authenticity. 

Centralized systems, including Aadhaar and Estonia digital ID, require extensive personal and biometric data—fingerprints and iris scans—to ensure seamless service delivery. 

Passport Gitcoin, designed for Web3 wallet integration, relies on centralized storage, demanding significant user trust. Federated Identities (OAuth) achieves a balance by sharing limited data through third-party providers but this comes with the risk of overexposure. China’s Social Credit System stands out for its vast data collection, encompassing financial, social, and daily activities, raising alarm over pervasive monitoring and privacy intrusion.\par
Privacy concerns are critical across the eight systems, influenced by their data management practices. Decentralized platforms like World App and BrightID prioritize privacy, yet linking personal data to public blockchains—as seen in Proof of Humanity—poses inherent risks. Centralized systems like Aadhaar and Estonia e-Card depend on centralized databases, making them vulnerable to surveillance risks. Federated Identities (OAuth) simplifies access but could expose user data to third-party providers without explicit consent. Passport Gitcoin presents privacy challenges because users' information can be shared with third-party service providers. Meanwhile, China’s Social Credit System exemplifies extreme privacy erosion, extensively monitoring citizen behavior with minimal transparency about data use. Striking a balance between privacy and functionality remains a universal challenge for all these systems.

\fixme{citations to be added} We have evaluated 15 systems to present diverse approaches to user experience, emphasizing usability, accessibility, intuitiveness and transparency.
\fixme{citation didn't work} CAPTCHA\cite{Captcha} and reCAPTCHA\cite{reCaptcha} are 2 common human verification tools used across many websites. While CAPTCHAs add an additional step for users when they are trying to access a website, reCAPTCHAS come into play by removing any external verification. Rather, reCAPTCHAs track users' activities which has raised privacy concerns as there is lack of transparency between user and reCAPTCHA authority. Users are not sure how the tracking data will be used. 

\tanusree{no good content}
Platforms such as World app\cite{WorldWhitepaper} and Bright ID\cite{BrightID} are developed on decentralized technologies which include zero-knowledge proofs but do not present a clear and concise terms and conditions and privacy policy, which may create reluctance for new users to start using them. In figure 1(a), the on-boarding screen of World App appears with a consent checkbox to obtain explicit consent from the users that they agree to the "Terms and Conditions" and acknowledge the "Privacy Notice" of World App. But the terms and conditions and privacy notice are not mentioned in the same screen, tapping on the link buttons redirects users to a different screen, thus creating an obstacle in their user journey. If the necessary terms and conditions were presented clearly and concisely on the on-boarding screen, it would have informed users about the app's policies and ensure a smoother user journey. 1(b) represents the Bright ID license agreement, but it is too long to read. Users may not have enough patience to go through the details as it is time consuming and tap the agree button to continue. But this action may create privacy risks as users don't know what type of access they are providing to the application.
\begin{figure}[h]
     \centering
     \begin{subfigure}[b]{0.2\textwidth}
         \centering
         \includegraphics[width=\textwidth]{Fig/world app t&c.png}
         \caption{The terms and conditions and privacy notice are not mentioned in the World App's on-boarding screen}
         \label{fig:The terms and conditions and privacy policy are not mentioned in the World App's on-boarding screen}
     \end{subfigure}
     \hfill
     % \begin{subfigure}[b]{0.3\textwidth}
     %     \centering
     %     \includegraphics[width=\textwidth]{Fig/google drive.png}
     %     \caption{World App requires users to connect Google Drive for enabling backup}
     %     \label{fig:five over x}
     % \end{subfigure}
     % \hfill
     \begin{subfigure}[b]{0.3\textwidth}
         \centering
         \includegraphics[width=\textwidth]{Fig/bright id t&c.png}
         \caption{Bright ID's license agreement contains a long description which users may not want to read}
         \label{fig:three sin x}
     \end{subfigure}
     \hfill
        \caption{On-boarding screens of World App and Bright ID}
        \label{fig:three graphs}
\end{figure}
In figure 2, we can see World App requires users to connect their Google Drive to back up their world app accounts but this may lead users to providing access to their sensitive information.
\begin{figure}[h]
    \centering
    \includegraphics[width=0.5\linewidth]{Fig/google drive.png}
    \caption{World App requires users to connect Google Drive for enabling backup}
    \label{fig:World App requires users to connect Google Drive for enabling backup}
\end{figure}
% \iffalse
% \begin{figure}
%  	\centering
%  	\includegraphics[width=0.5\linewidth]{Fig/world app t&c.png}
%  	\caption{The terms and conditions and privacy policy are not mentioned in the World App's on-boarding screen}   
%  \label{fig:The terms and conditions and privacy policy is not clearly mentioned}
%  \end{figure}
%  \begin{figure}
%  	\centering
%  	\includegraphics[width=\linewidth]{Fig/bright id t&c.png}
%  	\caption{The license agreement and privacy policy is too long to read}   
%  \label{fig:The license agreement and privacy policy is too long to read}
%  \end{figure}
% . \par
%  \begin{figure}
%  	\centering
%  	\includegraphics[width=\linewidth]{Fig/google drive.png}
%  	\caption{World App requires users to connect Google Drive for enabling backup}
    
%  \label{fig:World App asking to connect Google Drive}
%  \end{figure}


 


Proof of Humanity\cite{PoH}\cite{PoHexplainer} offers a unique approach to verification with a social identification system. But the verification process requires users to connect their cryptocurrency wallet which will be publicly linked to users' account. Thus, users' wallet holdings and transaction history will be linked to users' identity which users may not prefer. 

In contrast, Federated Identities OAuth\cite{OAuth} provides streamlined login process via well known third-party service provides, also known as identity providers such as Google, Facebook etc. It ensures ease of use as users need to specify the identity provider during the login or authentication process and grant access to their specific data. But, data is shared across multiple platform which may raise privacy concerns among users. 

DECO\cite{zhang2020deco} and CanDID\cite{maram2021candid} are decentralized and privacy preserving oracle protocols where DECO allows users to prove the authenticity of website data obtained over TLS (Transport Layer Security) without revealing sensitive information. But Oracle has access to users' data which pose as a privacy risk. CanDID provides users with control of their own credentials but privacy depends on the honesty and integrity of verifiers and decentralized identity validators. 

Idena\cite{idenaWhitepaper}, Humanode\cite{Humanode} and Civic Pass\cite{CivicPass} - all are blockchain based person identification system where Idena performs validation by conducting flip tests and Humanode and Civic Pass are developed on crypto-biometric network. Though Idena does not collect any personally identifiable information, the behavioral data collected can be used in future for pattern analysis. 

Humanode and Civic pass both require biometric verification (face scan) which can create concerns among users about how their sensitive credential (face) will be managed by the systems. It is noteworthy that, most of the platforms are decentralized (World App, Bright ID, Proof of Humanity, Idena, Humanode, Civic), some requiring cryptocurrency wallet (Proof of Humanity, Civic Pass) and some requiring biometric verification (Proof of Humanity, Humanode, Civic Pass).    %citations to be added
\par
Government issued identity documents such as Aadhaar Card, Estonia's e-ID, China's social credit system and Japan's My Number Card are controlled and managed by central government. Citizens' sensitive credential can be at high risk if the government's security system is not robust enough to prevent any kind of hacking or data breaching. China's social credit system monitors citizen data extensively without maintaining complete transparency about data use and management. 

ID.me is another online identity network that enables individuals to verify their legal identity digitally. But privacy concerns arises as a single company holds a large amount of personal data and users have limited control over their data. %citations to be added
\par
Usability across these different platforms are critical. CAPTCHAs have become increasingly difficult to solve, often leading users to leave the website or platform without completing their user journey. Accessibility remains another issue as visually impaired users are unable to solve any CAPTCHA that is text or image based. reCAPTCHA comes with the solution of these problems but trading of users' privacy as users' data is being tracked down by the authority. 

From Figure 3 and 4, it is apparent that World app and Bright ID provide a simple and intuitive account creation form but an introductory video or step by step guide would be more helpful to guide users to navigate throughout the applications and perform necessary actions.
 \begin{figure}
 	\centering
 	\includegraphics[width=\linewidth]{Fig/world app account creation.png}
 	\caption{World App's account creation process is simple but doesn't inform users about how they should navigate or use the app \fixme{silvia, is there a reason you added all these UIs? why all of the uis are randomly placed, I shared examples so many times, i am not seeing anything I gave instruction.}}
 \label{fig:World App's on-boarding process}
 \end{figure}
 
 \begin{figure}
 	\centering
 	\includegraphics[width=\linewidth]{Fig/bright id account creation.png}
 	\caption{The "Create my BrightID" process in the Bright ID app is straightforward but lacks guidance on how users should navigate or utilize the app effectively. \fixme{explain why these screenshots are important to add from cognitive walkthrough. caption itself should be self explanatory with text explaining in the main body}}
    
 \label{fig:Bright ID's on-boarding process}
 \end{figure}
The principle of error prevention could make the user journey of registration in Proof of Humanity more user-friendly. As there is no option to correct or update any mistake, it may increase user frustration. Idena's validation test (flip test) (Figure 4) can be inconvenient for new users as they may struggle to find the meaningful story in the provided time and collect points to validate them.
 \begin{figure}
 	\centering
 	\includegraphics[width=\linewidth]{Fig/idena.png}
 	\caption{Idena validation test interface requiring users to select meaningful stories within a time limit which can be challenging for new users \fixme{anyone reading this caption would not understand anything}}
    
 \label{fig:Selecting meaningful story for validation process on Idena}
 \end{figure}
The platforms requiring video selfie or face scan (Proof of Humanity, Humanode, Civic Pass) may create another challenging situation for users as proper lighting and optimal devices are necessary for taking the appropriate photo or video for biometric verification. 

Aadhaar card\cite{Aadhaar}\cite{AadhaarEnrollment}, Estonia's e-ID and Japan's My Number Card are all government based credentials but completing all the formalities and getting the card takes a long time, sometimes creating user frustration. The existing systems and platforms that we have evaluated can hardly strike a balance between privacy, functionality and usability.   %citations to be added


% \begin{figure}[!t]
% 	\centering
% 	\includegraphics[width=\linewidth]{Fig/world app.png}
% 	\caption{New account creation process in  World App}
    
% \label{fig:New account creation process in  World App}
% \end{figure}
% \begin{figure}[!t]
% 	\centering
% 	\includegraphics[width=\linewidth]{Fig/bright id.png}
% 	\caption{New account creation process in  Bright ID}
    
% \label{fig:New account creation process in  Bright ID}
% \end{figure}


\subsection{Reddit Analysis}
%\tanusree{ishan to add}
We first collected \fixme{X} posts and  \fixme{X} comments on December 24th, 2024, using the Python Reddit API Wrapper (PRAW)~\footnote{https://praw.readthedocs.io/en/stable/}. We gathered the data from various relevant subreddits, ensuring a broad and comprehensive understanding of what users discuss on identify verification or personhood verification. Through qualitative analysis of this Reddit data, we were able to uncover detailed insights into the rich and prevalent usage of verification systems. This analysis highlighted users' current usage, potential challenges and risks they encounter. These findings provide a solid foundation to explore these themes further in subsequent in-depth interviews.

\paragraph{Data Collection}
 To comprehensively cover content related to our research questions on personhood verification, we first created a list of search keywords by identifying close terminologies related to \textit{``personhood verification''} (general keywords) and \textit{``bot check''} (technology-focused keywords), etc. We utilized a combination of general and technology-focused keywords in our search. We employed general terms such as Personhood Verification, Identity Proof, Human Check and Bot Check. These keywords were designed to capture posts authored by or discussing personhood verification. For the technology focus, we used terms such as \fixme{add}. These keywords targeted discussions specifically about the use of popular tools and platforms. We conducted open searches combining these keywords across Reddit to gather data from various subreddits.
 Other than open searches, we also applied specific criteria to select subreddits, ensuring comprehensive coverage of relevant discussions: these subreddits should focus either on the personhood verification community or technology. We chose subreddits with the most active users online during our browsing sessions. The full list of subreddits and search keywords used is detailed in Table\fixme{need to find out the subreddit most prevalent discussing these topic}. 

\paragraph{Analysis}
Two researchers reviewed each post and categorized related posts or comments into five overarching high-level themes: \fixme{need to add after data analysis}. Within these categories, 53 level-2 themes were identified, such as \fixme{need to add after data analysis}. During the analysis process, researchers regularly convene to discuss discrepancies and emerging themes in the codebook, aiming to reach a consensus. These categories allowed us to investigate RQ2 and partially address RQ1. 

\subsection{Results}
% \tanusree{ishan to add}
\fi
\vspace{-2mm}
\section{ User Study Method}
\vspace{-2mm}
\label{user-study}
This section outlines the method for exploring users' perceptions and preferences of personhood credentials. We conducted semi-structured interviews with 23 participants from the US, and the EU/UK in October 2024.
%We started with a round of pilot studies (n=5) to validate the interview protocol. Based on the findings of pilot studies, we revised the interview protocol and conducted the final round of interviews (n=17). 
The study was approved by the Institutional Review Board (IRB).
\vspace{-2mm}
\subsection{Participant Recruitment}
\vspace{-2mm}
We recruited participants through (1) social media posts, (2) online crowdsourcing platforms, including CloudResearch and Prolific. Respondents were invited to our study if they met the selection criteria: a) 18 years or older and b) living in the US or the EU/UK. Participation was voluntary, and participants were allowed to quit anytime. Each participant received a \$30 Amazon e-gift card upon completing an hour-and-a-half interview.

\subsection{Participants}
%\tanusree{check for final count} \ayae{updated percentage with final 23 counts} 
We interviewed 23 participants, 10 from the US and 12 from the EU/UK. The majority of the participants (61\%) were in the age range of 25-34, followed by 22\% were 35-44 years old. The participants were from the United States and various countries, namely Spain, Sweden, Germany, Hungary, and the United Kingdom. Participants had different backgrounds of education levels, with 87\% of participants holding a Bachelor’s degree and 65\% holding a graduate degree. 65\% of participants had a technology background, while 48\% of them had a CS background. All participants reported using online services that required them to verify their personhood. Table~\ref{table:demographics} presents the demographics of our participants. We refer to participants as P1,. . . ,23.
\begin{table*}[h!]
\centering
%\scriptsize
\caption{Overview of PHC Application Scenarios}
\label{table:scenario}
%\resizebox{\textwidth}{!}{%
\begin{tabular}{lll}
\hline
\textbf{Scenario} & \textbf{Service} & \textbf{Credential} \\
\hline
Financial service & Bank, Financial institutions & Passport or Driver’s license, Face scan \cite{yousefi2024digital}\\
% \hline
Healthcare service & Hospitals, Clinics & Health insurance card,  Fingerprint \cite{chen2012non,fatima2019biometric,jahan2017robust}\\
% \hline
Social media & Tech companies & National identity card, Video selfie \cite{instagramWaysVerify, metaTypesID,instagramTypesID} \\
% \hline
LLM application & Tech companies & Iris scan \cite{WorldWhitepaper, worldHumanness}\\
% \hline
Government service & Government & Driver’s license or National identity card \cite{LogingovVerify}\\
% \hline
Employment background check & Background check companies & Tax identification card, Fingerprint\cite{cole2009suspect}\\
\hline
\end{tabular}%
%}
% \vspace{0.5em}
\label{tab:scenarios}
\end{table*}
\begin{table*}[h]
\centering
\caption{Participant demographics and background.}
%\fixme{add the participants you completed so far}
\resizebox{\textwidth}{!}{%
\begin{tabular}{l l l l l l l l}
\hline
\textit{Participant ID} & \textit{Gender} & \textit{Age} & \textit{Country of residence} & \textit{Education} & \textit{Technology background}  & \textit{CS background} &\textit{Residency duration} \\
\hline
P1 & Male & 25-34 & the US & Master's degree & Yes & Yes &3-5 years\\
P2 & Female & 25-34 & the US & Master's degree & Yes & Yes & 1-3 years\\
P3 & Female & 25-34 & the UK & Master's degree & Yes & No & 1-3 years\\
P4 & Female & 35-44 & the UK & Some college, but no degree & Yes & Yes & Over 10 years \\
P5 & Male & 25-34 & the US & Doctoral degree & Yes & Yes & 5-10 years \\
P6 & Male & 35-44 & the US & Less than a high school diploma & No & No & Over 10 years \\
P7 & Male & 25-34 & the US & Doctoral degree & Yes & Yes & 3-5 years\\
P8 & Male & 45-54 & the US & Bachelor's degree & Yes & Yes & Over 10 years \\
P9 & Female & 25-34 & New Zealand & Master's degree & No  &  No &  Over 10 years\\
P10 & Male & 25-34 & the US & Master's degree & No & No & Over 10 years\\
P11 & Female & 25-34 & the UK & Bachelor's degree & No & No & Over 10 years\\
P12 & Male & 18-24 & the UK & Master's degree & Yes & Yes & 1-3 years\\
P13 & Male & 35-44 & the UK & Bachelor's degree & Yes & No & Over 10 years\\
P14 & Male & 25-34 & Sweden & High school graduate & No & No & Over 10 years \\
P15 & Female & 25-34 & Spain & Master's degree & Yes & Yes & Over 10 years \\
P16 & Female & 25-34 & Germany & Master's degree & Yes & Yes & Over 10 years \\
P17 & Female & 25-34 & Spain & Doctoral degree & No & No & Over 10 years \\
P18 & Female & 35-44 & the US & Bachelor's degree & No & No & Over 10 years \\
P19 & Female & 25-34 & Germany & Master's degree & Yes & Yes & 3-5 years \\
P20 & Male & 25-34 & Hungary & Master's degree & Yes & No & 3-5 years \\
P21 & Male & 35-44 & the US & Bachelor's degree & Yes & No & 5-10 years \\
P22 & Female & 18-24 & France & Master's degree & Yes & Yes & Less than 1 year\\
P23 & Male & 45-52 & the US & Master's degree & No & No & Over 10 years\\
\hline
\end{tabular}%
}
\label{table:demographics}
\end{table*}


\vspace{-2mm}
\subsection{Semi-Structured Interview Procedure} \label{sec:study_protocol}
\vspace{-2mm}
%\fixme{explain in details why the study designed in a certain way. please read papers to learn more. data minimization and advertisement paper. The method section is too bland. We have a wonderful study design. Scenario-specific study design, describe scenarios and why chose this scenario. Mainly method should include all design rationale, and example questions when necessary to clarify your rational}

We started with a round of pilot 
%(Appendix~\ref{pilot}) 
studies (n=5) to validate the interview protocol. Based on the findings of pilot studies, we revised the interview protocol.

\textbf{Open Ended Discussion.} We designed the interview script based on our research questions outlined in the introduction section~\ref{sec:introduction}. 
%We added the interview script to the section~\autoref{protocol}. 
At the beginning of the study, we received the participants’ consent to conduct the study. Once they agreed, we proceeded with a semi-structured interview. The study protocol was structured according to the following sections: (1) Current practices regarding digital identity verification; (2) Users' perception of PHC before and after watching the informational video; (3) Scenario-based session to investigate factors that influence users' preferences of PHC; 
%(4) Users' preference of PHC; 
(4) Design session to conceptualize users' expectations; (5) A brief post-survey on Users' Preference of PHCs.
%of PHCs in different scenarios.

In the first section, we first asked a set of questions to understand participants' current practices of online platforms and the types of identity verification methods they had experience with. This is to understand their familiarity with different types of verification, such as biometrics, physical IDs, etc.
%and methods that might have worked well based on their prior experience.

%of online identity verification. When participants mentioned certain types of online services that required identity verification, we inquired about their experience with verification method. Was it easy to use, or did you run into any issues?"} We further inquired about any challenges participants faced with identity verification - \textit{"Did you encounter any challenges when using this method?"} 
%If biometrics didn’t naturally come up in prior discussions, we prompted to consider them- \textit{"Have you ever used services where you had to verify yourself through face, fingerprints, or iris scans, or other biometrics?"} If they mentioned any experience with biometric verification, we followed up with questions like- \textit{"What worked well? Were there any concerns you had?"}
In the second section, we then asked about participants' current understanding and perception of personhood credentials either from prior knowledge or from intuition by just hearing the term. %We also asked if they knew how personhood credentials work, particularly how it has been handled by the different services they use. 
%As all participants had never heard of PHC, we prompted them to interpret the term based solely on its wording. 
While the majority recognized this as unfamiliar terminology, most inferred that it referred to a form of personal identification, often associating it with biometric verification.
%In the pilot interviews, The majority of the participants could not provide substantial responses on their understanding of how personhood credentials work, before starting the second part of the interview, we showed them an informational video on personhood credentials.
%Most of the participants were unfamiliar with this term, so we then asked \textit{ Can you explain what you think it means by just hearing the term?"} 
%Before proceeding with the third section of the interview, we assessed participants' understanding of PHC with knowledge questions.
Then, we showed them an introduction video on PHC \footnote{https://anonymous.4open.science/r/PHC-user-study-14BB/}, %\fixme{create an anonymous GitHub, upload the video and add a footnote here} \ayae{reflected}. 
%The video provides an overview of PHCs, 
covering their definition, 
%the steps involved in issuing and using them, 
and implications of it in online services. Based on former literature\cite{adler2024personhood}, we designed the video with easy-to-understand text, visuals, and audio to make the concepts accessible to average users. We created a set of knowledge questions to assess participants' understanding of PHC before and after showing the video. %as attached in Appendix~\ref{knowledge_questions}.

%including the same knowledge questions. 
%Most participants correctly responded to knowledge questions, which ask the basic understanding of digital identity crisis and personhood credentials. 
%Even before showing the introduction video, regarding the question \textit{"What could happen if online identities are poorly verified?"}, 95\% correctly selected \textit{"Fake accounts, bots, and fraud could increase significantly."} For the question \textit{"What are Personhood Credentials (PHCs)?"}, 90\% correctly choose the option \textit{"Digital credentials that confirm a person’s identity."} 
For instance, we observed an improve in correct response rate for the question, such as, \textit{``What is the primary goal of PHC?''} from 85\% to 100\% after watching the video.
%where the correct answer was \textit{"To verify a person's identity without exposing personal information."} 
%However, regarding the question \textit{"To whom do you provide minimal personal information during the PHC process?"}, only 35\% selected the correct answer \textit{"PHC issuers (e.g., governments or trusted organizations)"}, while the most frequent response was \textit{"Online service providers (e.g., social media)"} at 45\%.
%\ayae{KQ results reflected}
%We also asked some open-ended questions to evaluate whether our introduction video helped participants better understand PHC \textit{''How would you explain your understanding of personhood credentials?''} 
%We further asked what benefits and concerns came to mind for them.
In the third section, we focused on scenario-based discussions, exploring specific applications of PHC to understand factors that influence participants' preferences towards PHCs as well as identify challenges to leverage in PHC design for various services. We examined the following six scenarios: (1) Financial service, (2) Healthcare service, (3) Social Media, (4) LLM applications, (5) Government Portal, and (6) Employment Background Check.
%We covered a wide range of use cases of online personhood verification via these six scenarios since they encompass diverse user needs, security and usability, and privacy requirements. %\fixme{please see the comment with iffalse tag and make it concise, we talked about it before}
\iffalse
%Firstly, financial system is a critical scenario for identity verification where high level of security protections are expected as exemplified by KYC. Thus, such services continue to develop transformative digital identity verification to ensure the security and integrity of financial transactions\cite{parate2023digital}. The second scenario is healthcare systems, which also have high privacy requirements due to the confidentiality of medical data. The pandemic has accelerated the adoption of online healthcare services and in response to this digital transformation, the recent study has proposed blockchain-based decentralized identity management systems \cite{javed2021health}. Thirdly, we consider the scenario of social media, which faces the critical challenges of online identity as shown in spreading misinformation and harmful content from fake or anonymous accounts \cite{ceylan2023sharing}. The fourth scenario is designed with a specific context of interacting with Large language models (LLMs). The former study discussed vulnerability in dialog-based systems where adversaries can exploit the training process to introduce toxicity into responses \cite{weeks2023first}. Thus, such vulnerabilities indicate identity verification may also be important for LLM applications. Fifth, government services are familiar situations that require people to verify their identity. Various countries have developed their own electronic ID schemes \cite{stalla2018gdpr}. Lastly, we also cover the scenario of employment background checks needing precise identity verification to ensure the reliability of applicants. The current background check system involves vulnerable processes that increase the risk of identity theft and unauthorized data access.\cite{blowers2013national}. Such challenges underline the relevance of PHCs, which can mitigate risks by providing a secure framework for verification.
%\ayae{included citation}
\fi
We have also incorporated various types of data or credentials requirements (e.g. physical id, biometrics, etc) across scenarios to maintain diversity in our discussion with participants as shown in Table.\ref{table:scenario}. %For instance,
%we  We have multiple existing verification methods, including 
%humanness verification (e.g., selfie, video call), document-based verification (e.g., government-issued ID), and biometrics information. 
We selected types of credentials for each scenario based on former literature and existing PHC as explained in the section \ref{subsec:verification_practice}. %\fixme{cite worldcoin, and other app and literature}. \fixme{from here to end of this paragraph ---These needs to go to the literature review section on the current usecase of PHC. And only 2 line summarizing why you chose the diverse type of credential data and refer to the literature section}


%% Let me find the former literature to explain why we select these credentials
For each of the six scenarios, we explored participants' perceptions of using PHC in hypothetical situations that align with the research focus as well as to help participants can relate PHC concepts to real-world applications. This is particularly useful for this study where where user perceptions and expectation under specific conditions are crucial to devising solutions \cite{carroll2003making}.
%\fixme{cite scenario method paper from jack caroll}.
%\ayae{reflected}
We asked about their feelings, perceived benefit and risks. We also nudge them to think about any privacy and security perception around using PHC and types of data (e.g., iris, face, government id, etc) involved in issuing PHC. 

\fixme{
%\textbf{Pre-understanding: Guessed it as one of the verification methods} 
%The majority of the participants were not familiar with the term ``Personhood Credential'', although most of them used some forms of such credentials. 
%As all participants have never heard of PHC, we prompted them to interpret the term based solely on its wording. Most of them inferred that it referred to another type of person identification. 
%For instance, P3 commented \textit{``It can be anything that would point to one single individual that would differentiate that individual from others.''} When participants expressed how PHC identifies a person's uniqueness, their understanding ranged from verifying basic information such as address or age, and certain eligibility to advanced identification of digital identity (e.g., behavioral, economical, etc) with Multi-factor authentication or knowledge-based questions.

%\textbf{Post-understanding: Involvement of trusted entity} When asked to explain their understanding of PHC, P13 noted, \textit{"So it sounds like, basically, you it's similar to how you verify things before. Like you use a biometrics and your government Id. But then you get a personal key. You do it with like a trusted organization rather than each individual. And then you can use that key for all the different services you use."} P1 elaborated PHC process as a shift of the verification entity, \textit{" I'd say we are sort of moving the verification burden from the user side to a service provider side where they have access to our data, and they have access to the token that's assigned to each person that's unique. And that's easily like traceable across online platforms. and this token is used for verification with 3rd parties, where they don't get access to your personal data, but they only use this service provider to give them the authenticity that you are a real user."} These suggest that the role of the PHC issuer is recognized as a crucial component of PHC.}
%began by asking \textit{"How did you feel about using PHC to verify your identity when opening your bank account?"} To dive deeper, we also asked about potential benefits:\textit{"What potential benefits do you see in using PHC in this online banking context?"}. We also inquired about these aspects- \textit{"Do you think using PHC improves the security of your bank account? Why?", "Did this method of identity verification make you feel more confident about your privacy? why?"} Additionally, we discussed their comfort levels for providing credentials (e.g., Government-issued ID, biometric information) and asked about any concerns about data collection-\textit{"Were you comfortable providing your government-issued ID and using facial recognition? Why?"}
}

\iffalse
%%% column: scenario, credential, service providers.
\begin{table*}[h!]
\centering
\caption{Overview of PHC Application Scenarios}
\label{table:scenario}
%\resizebox{\textwidth}{!}{%
\begin{tabular}{lll}
\hline
\textbf{Scenario} & \textbf{Service Provider} & \textbf{Types of Credential} \\
\hline
Financial Service & Bank, Financial Institutions & Passport or Driver’s license, Face scan \cite{yousefi2024digital}\\
% \hline
Healthcare Service & Hospitals, Clinics & Health insurance card,  Fingerprint \cite{chen2012non,fatima2019biometric,jahan2017robust}\\
% \hline
Social Media & Tech Companies & National identity card, Video selfie \cite{instagramWaysVerify, metaTypesID,instagramTypesID} \\
% \hline
LLM Application & Tech Companies & Iris scan \cite{WorldWhitepaper, worldHumanness}\\
% \hline
Government Service & Government & Driver’s license or National identity card \cite{LogingovVerify}\\
% \hline
Employment Background Check & Background Check Companies & Tax identification card, Fingerprint\cite{cole2009suspect}\\
\hline
\end{tabular}%
%}
% \vspace{0.5em}
\label{tab:scenarios}
\end{table*}
\fi


%\textbf{Design Session.}
%\fixme{need to explain how and why you design the design session, where you designed, how participants were unstructured and so on.} \ayae{reflected in the following paragraph}

In the fourth section, we began by refreshing participants’ memories of the various risks and concerns discussed in the earlier scenario-based section. Following this, we guided participants to brainstorm potential design solutions by sketching their ideas to address these concerns. To facilitate the sketching process, we developed sketch notes in Zoom as prompts to help participants generate ideas, particularly when starting from scratch is challenging. 
%on Zoom whiteboard or pen and paper, using a think-aloud protocol.  
%Nevertheless, it is difficult to develop new ideas from scratch, so 
%Additionally, we described the main issues or concerns that the participants identified during the interview at the top of the sketch notes. 
%Participants can develop their ideas at the center of the whiteboard by locating the above components or creating new shapes, lines, or text boxes for their sketches. 
We also investigated participants' preferences for PHC regarding the issuers and issuance systems of PHCs, as well as the types of data required for issuing PHCs. 
%in the context of who issues PHC or type of issuance systems, and what types of data are needed to issue PHC to address RQ2. 
%An example includes- \textit{``What types of credential would you prefer to use as personhood verification? ; Which organizations or stakeholders would you prefer to issue and manage your PHC?''} 
We encourage participants to explain their reasoning. These questions were informed by insights from the pilot study, where participants expressed preferences for different types of data, system architecture, and various stakeholders involved in PHC issuing.
%However, these questions alone can only find optimal ways within the scope of currently existing options and cannot generate new design implications. Therefore,

\iffalse
\tanusree{we can cut this section as this didn't give any result and doesn't answer RQs directly.}Lastly, to understand preference on issuance system, we introduced the decentralized PHC system architecture with another instructional video. Following the video, we asked participants to explain their understanding of the decentralized PHC system and their preference for the issuance system (centralized or decentralized). We introduced it after the sketch session is that participants may organically come up with the idea of decentralized systems on their own, and we intended to avoid priming them. 
\fi
%Then, we asked them to explain their understanding of the decentralized PHC and preferred issuance system (centralized or decentralized.)- \textit{`` Could you explain why you would prefer decentralized system in managing your PHCs?''}
%\textit{"Would you prefer to get multiple PHCs from different issuers depending on the situation or application you're using, or would you rather have a single PHC from one issuer?"}

\textbf{Post-Survey.}
%%\fixme{need to explain how and why you design the design session, where you designed, how participants were unstructured and so on.}
We conducted a post-survey to obtain participants' PHC preference quantitatively. It included questions on participants' preference on credential type, issuer and issuance system  for the scenarios (e.g., financial, medical, etc) we considered in our interview.

\vspace{-2mm}
\subsection{Data Analysis}
\vspace{-2mm}
Once we got permission from the participants, we obtained interview data through the audio recording and transcription on Zoom. We analyzed these transcribed scripts through thematic analysis \cite{Braun2012-sz, Fereday2006-yv}. Firstly, all of the pilot interview data was coded by two researchers independently. Then, we compared and developed new codes until we got a consistent codebook. Following this, both coders coded 20\% of the interview data of the main study. We finalized the codebook by discussing the coding to reach agreements. Lastly, we divided the remaining data and coded them. After both researchers completed coding for all interviews, they cross-checked each other’s coded transcripts and found no inconsistencies. Lower-level codes were then grouped into sub-themes, from which main themes were identified. Lastly, these codes were organized into broader categories. Our inter-coder reliability (0.90) indicated a reasonable agreement between the researchers.
\iffalse

\begin{table*}[h]
\centering
\caption{Participant demographics and background.}
%\fixme{add the participants you completed so far}
\resizebox{\textwidth}{!}{%
\begin{tabular}{l l l l l l l l}
\hline
\textit{Participant ID} & \textit{Gender} & \textit{Age} & \textit{Country of residence} & \textit{Education} & \textit{Technology background}  & \textit{CS background} &\textit{Residency duration} \\
\hline
P1 & Male & 25-34 & the US & Master's degree & Yes & Yes &3-5 years\\
P2 & Female & 25-34 & the US & Master's degree & Yes & Yes & 1-3 years\\
P3 & Female & 25-34 & the UK & Master's degree & Yes & No & 1-3 years\\
P4 & Female & 35-44 & the UK & Some college, but no degree & Yes & Yes & Over 10 years \\
P5 & Male & 25-34 & the US & Doctoral degree & Yes & Yes & 5-10 years \\
P6 & Male & 35-44 & the US & Less than a high school diploma & No & No & Over 10 years \\
P7 & Male & 25-34 & the US & Doctoral degree & Yes & Yes & 3-5 years\\
P8 & Male & 45-54 & the US & Bachelor's degree & Yes & Yes & Over 10 years \\
P9 & Female & 25-34 & New Zealand & Master's degree & No  &  No &  Over 10 years\\
P10 & Male & 25-34 & the US & Master's degree & No & No & Over 10 years\\
P11 & Female & 25-34 & the UK & Bachelor's degree & No & No & Over 10 years\\
P12 & Male & 18-24 & the UK & Master's degree & Yes & Yes & 1-3 years\\
P13 & Male & 35-44 & the UK & Bachelor's degree & Yes & No & Over 10 years\\
P14 & Male & 25-34 & Sweden & High school graduate & No & No & Over 10 years \\
P15 & Female & 25-34 & Spain & Master's degree & Yes & Yes & Over 10 years \\
P16 & Female & 25-34 & Germany & Master's degree & Yes & Yes & Over 10 years \\
P17 & Female & 25-34 & Spain & Doctoral degree & No & No & Over 10 years \\
P18 & Female & 35-44 & the US & Bachelor's degree & No & No & Over 10 years \\
P19 & Female & 25-34 & Germany & Master's degree & Yes & Yes & 3-5 years \\
P20 & Male & 25-34 & Hungary & Master's degree & Yes & No & 3-5 years \\
P21 & Male & 35-44 & the US & Bachelor's degree & Yes & No & 5-10 years \\
P22 & Female & 18-24 & France & Master's degree & Yes & Yes & Less than 1 year\\
P23 & Male & 45-52 & the US & Master's degree & No & No & Over 10 years\\
\hline
\end{tabular}%
}
\label{table:demographics}
\end{table*}
\fi


\begin{figure}[tbh]
\centering
\includegraphics[width=0.49\textwidth]{sec/Figures/quali_2d.pdf}
\caption{
    \textbf{Qualitative comparisons of generated brushes for surface details.} Our method captures geometry details guided by texts, effectively preserving surface structure and avoiding mesh distortions.
}
\label{Fig: Qualitative 2D}
\end{figure}
\vspace{-0.6 cm}
\section{Experiments}
\label{sec:Experiment}
In this section, we conduct experiments to evaluate the various capabilities of Text2VDM both quantitatively and qualitatively for text-to-VDM brush generation.
% in ~\Cref{Qualitative} and ~\Cref{Quantitative}.
We then present an ablation study that validates the significance of our key insight into CFG-weighted SDS, as well as the effect of the region control and shape control.
% in ~\Cref{Ablation}.

\begin{figure*}[tbh]
\centering
\includegraphics[width=1\textwidth]{sec/Figures/quali_3d.pdf}
\caption{
    \textbf{Qualitative comparisons of generated brushes for geometric structures.}  Our method accurately presents key geometric features described by text, facilitating downstream applications in modeling software.
}
\label{Fig: Qualitative 3D}
\end{figure*}

\subsection{Qualitative Evaluation}
\label{Qualitative}
To the best of our knowledge, Text2VDM is the first framework to generate VDM brushes from text.
We adapted three existing methods for comparison and classified them into two categories. The first category includes Text2Mesh~\cite{single12-text2mesh} and TextDeformer~\cite{SIGGRAPH:TextDeformer:2023}, which generate a brush mesh through text-guided mesh deformation on a planar mesh, following a process similar to ours. For the second category, we opt to directly generate VDM via Paint-it~\cite{paintit}. Notably, this method originally uses SDS to optimize a UNet for generating PBR textures. We reframed it to suit our VDM brush generation task, modifying it to generate VDM through SDS optimization of the UNet. We compared the visual results in \Cref{Fig: Qualitative 2D} and \Cref{Fig: Qualitative 3D}.

Compared to other methods, Text2VDM can generate more vivid and better-quality VDM brushes. Text2Mesh applies displacement to each vertex along normal directions, resulting in limited mesh deformation. TextDeformer indicates the accumulation of local deformations in the Jacobians, which results in global mesh drift, making it challenging to bake these meshes into VDM.
Reframed Paint-it VDM generation is equivalent to optimizing the three-axis displacement of each vertex on the mesh with SDS. Although the UNet reduces noise from the SDS~\cite{paintit}, geometric regularization is still required to ensure mesh quality. The generated mesh must compromise between solving the problem and maintaining smoothness, which makes achieving high-quality mesh generation quite challenging.
% Using a UNet to generate VDM is equivalent to optimizing the three-axis displacement of each vertex on the mesh with SDS. Although the UNet reduces noise from the SDS~\cite{paintit}, geometric regularization is still required to ensure mesh quality. The generated mesh must compromise between solving the problem and being smooth, which results in low-quality mesh generation.

\subsection{Quantitative Evaluation}
\label{Sec: Quantitative}

We quantitatively evaluated our framework regarding generation consistency with text input and mesh quality. We used 40 distinctive text prompts for VDM generation.

\noindent\textbf{Generation Consistency with Text.} We initially assessed the relevance of the generated results to the text descriptions~\cite{CLIP:CORR:2021}. 12 different views were rendered for average scores respectively, as presented in Table~\ref{tab:quantitative comp}. Our approach achieves the highest scores compared to baseline methods.


\begin{table}[h!]
\caption{Quantitative evaluation of state-of-the-art methods. The geometry CLIP score is calculated on shaded images with uniform albedo colors~\cite{Richdreamer:CVPR:2024}, and self-intersection is quantified as the ratio of self-intersected mesh faces to the total number of faces.}
\centering
\footnotesize   % incase not overflow
% TADA & TextDeformer & Fantasia3D
\begin{tabular}{*{10}{c}}
         \hline
           & Geometry CLIP Score $\uparrow$ & Mesh Self-Intersection $\downarrow$\\
         \hline 
         Paintit & $0.2375$ & $19.42\%$  \\
         Text2Mesh & \underline{0.2497}  & $7.18\%$\\
         TextDeformer & $0.2477$  & \textbf{0.04\%} \\
         Ours & \textbf{0.2556} &  \underline{0.77\%}\\
         \hline
\end{tabular}
\label{tab:quantitative comp}
\end{table}

\begin{figure*}[tbh!]
\centering
\includegraphics[width=1\textwidth]{sec/Figures/ablation_sds.pdf}
\caption{
    \textbf{Effect of CFG-weighted SDS.} CFG-weighted SDS effectively mitigates semantic coupling issues in SDS, such as generating the tortoise’s tail and head or the snail’s head, by providing more focused semantic guidance. In contrast, CSD adds extra negative terms that fail to decouple semantics, resulting in a less stable and more time-consuming optimization process.
} 
\label{Fig: Effect of CFG-weighted SDS}
\end{figure*}

\noindent\textbf{Mesh Quality.} We evaluated mesh quality by examining self-intersection. Paint-it and Text2Mesh, which utilize direct vertex displacement, often converge to a local minimum and disregard the mesh triangulation. While TextDeformer exhibits the lowest self-intersection, its tendency to produce over-smoothed results frequently results in losing object features described in text prompts. 

\begin{table}[h!]
\caption{User evaluation of generated VDMs.}
\centering
\footnotesize
    \begin{tabular}[width=1.0\textwidth]{*{10}{c}}
         \hline 
         User Preference $\uparrow$  & Geometry Quality & Consistency  with Text\\
         \hline 
         Paintit & $3.1\%$  & $1.7\%$ \\
         Text2Mesh & \underline{$18.3\%$} & \underline{$27.3\%$} \\
         TextDeformer & $3.3\%$ & $3.4\%$ \\
         Ours & \textbf{75.3\%} & \textbf{67.6\%} \\
         \hline
    \end{tabular}
\label{tab:user comp}
\end{table}


\noindent\textbf{User Study.} We further conducted a user study to evaluate the effectiveness and expressiveness of our method. A Google Form was utilized to assess 1) geometry quality and 2) consistency with text. We recruited 32 participants, of whom 14 are graduate students majoring in media arts, and 18 are company employees specializing in AI content generation. In this form, the participants were instructed to choose the preferred renderings of VDM from different methods in randomized order, as shown in Table~\ref{tab:user comp}. The results show participants preferred our method by a significant margin. 
% For practical evaluation, we invited 5 participants to use VDMs generated by our methods in Blender to sculpt 3D models that aligned with their expectations (Figures~\ref{Fig: Local to Global Mesh Stylization} and~\ref{Fig: Coarse to Fine Interactive Modeling}).

\subsection{Ablation Study}
\label{Ablation}
\textbf{Effects of CFG-Weighted SDS.}  %为了验证CFG-weighted SDS的有效性,我们设置的实验对比了直接使用SDS,使用CFG-Weighted SDS 以及使用三种不同negative prompt权重的CSD。As mentioned in ~\Cref{sec: tesds}, SDS在没有全局语义作为reference的情况下进行local component生成时会有语义耦合的情况,生成出来的mesh会有明显的瑕疵。另外我们发现使用negative prompt这种直观的做法并不能有效的解耦语义,并且增大负文本的权重时会使得优化过程变得不稳定,更难收敛,导致了低质量的mesh。相比之下,我们的方法不需要进行额外的Unet推理,并且能够有效的对语义进行解耦生成符合要求的mesh。
We conducted experiments to compare the generated results of directly using SDS~\cite{DreamFusion:ICLR:2022}, CFG-weighted SDS, and CSD~\cite{CSD:Arxiv:2023} with three different annealed weights of negative prompt (\Cref{Fig: Effect of CFG-weighted SDS}). As discussed in \Cref{sec: tesds}, SDS can result in semantic coupling when generating sub-object structure, leading to artifacts like the tortoise's tail and head or the snail's head. We also found that using negative prompts was ineffective at decoupling semantics. Increasing the initial weight of negative prompts further makes the optimization unstable, resulting in low-quality results. In contrast, our method effectively mitigates semantic coupling to produce high-quality meshes without requiring additional UNet inference.
\begin{figure}[tbh!]
\centering
\includegraphics[width=0.48\textwidth]{sec/Figures/ablation_masks2.pdf}
\caption{
    \textbf{Effect of region control.} Region masks can effectively control the shape of surface details based on different text inputs.
}
\label{Fig: Effect of region mask}
\end{figure}

\noindent\textbf{Effects of Region Control.} %我们在图中展示了两组region mask在不同text prompt下对surface detailed brush生成的控制能力。我们生成的结果可以很好得match text生成cloth,metal,stone等不同质感,同时符合region mask所限制的形状
\Cref{Fig: Effect of region mask} demonstrates two sets of region masks and their control over surface details generation under different text prompts. Without using a region mask, the results lack a specific shape, which may not satisfy the desired stylized effect. By using a region mask, our generated results effectively conform to the user's desired shapes while also aligning with the styles specified by the text, such as metal and stone.


\noindent\textbf{Effects of Shape Control.} % 我们的方法在图中展示了生成结果与通过shape map进行初始化体积与方向保持一致的能力,在不同的local component生成中,比如beard,pauldron,elf ear,我们的方法能够在保持体积和方向大致稳定的情况下生成多样的符合文本描述的结果
Our method demonstrates that user-specified VDMs can effectively control the volume and direction of generated geometric structures. As shown in \Cref{Fig: Effect of shape map}, various generated geometric structures, such as elf ears and pauldrons, are high-quality and align with the text descriptions. We also found that without volume initialization, it is challenging to generate desired results. It indicates that this initialization is crucial for steering the gradient flow of geometric structure generation via adjusting the Laplacian term.
\begin{figure}[!htb]
\centering
\includegraphics[width=0.49\textwidth]{sec/Figures/ablation_shape.pdf}
\caption{
    \textbf{Effect of shape control.} User-specified VDMs can help achieve the intended final effect of geometric structures by initializing the brush's volume and direction.
}
\label{Fig: Effect of shape map}
\end{figure}
\vspace{-0.3 cm}
\subsection{Applications}
\label{Application}
Once various VDM brushes are generated, users can directly use these brushes to meet diverse creative needs in mainstream modeling software. For example, they can apply VDM brushes for mesh stylization and engage in a real-time iterative modeling process.

\noindent\textbf{Local-to-Global Mesh Stylization.} Although mesh stylization is a complex task even for professional artists, combining different surface details allows users to achieve stylization quickly. For instance, users can apply a variety of wall-damage brushes to specific areas of a stone pillar, creating a style of damage (~\Cref{Fig: Local to Global Mesh Stylization}).
% Similarly, they can use different rust-effect brushes on a helmet to give it an aged style, 

\begin{figure}[!htb]
\centering
\includegraphics[width=0.48\textwidth]{sec/Figures/application.pdf}
\caption{\textbf{Local to global mesh stylization.} Applying various surface details brushes can create a damaged-style stone pillar model.} 
\label{Fig: Local to Global Mesh Stylization}
\end{figure}

\noindent\textbf{Coarse-to-Fine Interactive Modeling.} Unlike previous methods~\cite{magiclay,tipeditor} that require a lengthy optimization process for each edit and result in non-reusable outcomes, our generated VDM brushes can be directly used in modeling software. This enables users to apply the generated brushes easily and interactively. For example, \Cref{Fig: Coarse to Fine Interactive Modeling} shows that users can combine various brushes, such as skeleton hand, rose pattern, and pauldron to refine a coarse cloth model into a highly detailed one.




\section{Conclusion and Future Work}
This paper introduces a simple and effective graph augmentation strategy for cross-graph node classification with OOD structure. 
There are still several directions that are worth exploring in the future: 1) the edge-dropping weight can be considered the more comprehensive method that can measure the significance of each edge. 2) The proposed augmentation can be extended to test time training.

% \newpage
\bibliography{0_main}


\section{Related Work}

Below we summarize the most relevant literature from both the medical lens and the algorithm lens.

\textbf{RL on social networks.} We design and implement RL on dyads that are small social networks in this paper. Existing works on RL on social networks are mostly focused on maximizing social influence or opinion spreading \cite{wang2021reinforcement,he2021reinforcement,yang2024balanced} with large scale social networks in mind. These problems are usually formulated as a constrained Markov Decision Process (CMDP) \cite{yang2024balanced}, where the goal is to allocate incentive to maximize the social influence or opinion spreading. Our focuses are on the challenges in the multi-scale decision making and the design of the RL algorithms that incorporate domain knowledge about the social networks. These differences make our algorithm designs unique contributions to the literature.

\textbf{Dyadic structure in health care.} Social relationships between patients and carepartners are proven to be important in many critical health outcomes. Studies have shown that the patient-caregiver dyad functions as a unit, with the well-being and coping strategies of one member significantly impacting the other \cite{shin2018supporting,mcpherson2024dyadic}. The quality of this relationship can affect treatment outcomes such as medication adherence \cite{psihogios2021understanding,kostalova2022medication,gresch2017medication}, and chronic disease management \cite{visintini2023medication,li2024usability}.

\textbf{Multi-agent RL (MARL).} Our proposed approach falls into the range of the independent learners in the MARL literature \cite{oroojlooy2023review}. Previous literature on MARL in a collaborative game focuses on finding the (approximate) Nash equilibrium of the game through interacting with an unknown environment \cite{wang2022cooperative,jin2021v}. However, in our paper, we emphasize the advantage of MARL in terms of its strong interpretability and being able to make decisions in multiple time-scales.

\section{Algorithm Details}
\label{app:algo}

We provide the complete details of the proposed \texttt{MultiAgent+SurrogateRwd} algorithm as well as the baseline \texttt{SingleAgent} algorithm.

We first introduce the infinite horizon RLSVI (Randomized Least Squares Value Iteration) algorithm in Alg. \ref{alg:base} \cite{russo2018tutorial}. This algorithm is a model-free posterior sampling approach that samples a random value function from its posterior distribution, and the agent acts greedily with respect to the sampled value function. We use the infinite horizon variant of RLSVI, which perturbs the Bayesian regression parameters with a random noise $\omega'$ (line 4). We introduce temporal correlation between the current noise $w'$ and the previous noise $w$ to introduce persistence in exploration.

\begin{algorithm}[H]
    \caption{Infinite Horizon RLSVI (Inf-RLSVI)}
        \begin{algorithmic}[1]
            \STATE{Input:} discount factor $\gamma \in \mathbb{R}$, previous dataset $\mathcal{D} = (s_i, a_i, r_i)_{i = 1}^{n-1} \cup \{s_{n}\}$, previous perturbation $w \in \mathbb{R}^d$, feature mapping $\phi: \mathcal{S} \times \mathcal{A} \mapsto \mathbb{R}^d$, previous parameter $\theta \in \mathbb{R}^{d}$
            \STATE Generate regression matrix and vector
            $$
                X \leftarrow\left[\begin{array}{c}
                \phi\left(s_1, a_1\right) \\
                \vdots \\
                \phi\left(s_{n-1}, a_{n-1}\right)
                \end{array}\right] \quad y \leftarrow\left[\begin{array}{c}
                r_1+\gamma \max _{\alpha \in \mathcal{A}} \langle \phi(s_2, \alpha), \theta \rangle \\
                \vdots \\
                r_{n-1}+\gamma \max _{\alpha \in \mathcal{A}}\langle \phi(s_{n}, \alpha), \theta \rangle
                \end{array}\right]
            $$
            \STATE Estimate value function
            $$
                \bar{\theta} \leftarrow \frac{1}{\sigma^2}\left(\frac{1}{\sigma^2} X^{\top} X+\lambda I\right)^{-1} X^{\top} y \quad \mathbf{\Sigma} \leftarrow\left(\frac{1}{\sigma^2} X^{\top} X+\lambda I\right)^{-1}
            $$
            \STATE Sample $w' \sim \mathcal{N}(\gamma w, (1-\gamma^2) \mathbf{\Sigma})$ and set $\theta' = \bar \theta + w'$
            \STATE \textbf{Output:} $\theta'$ and $w'$
            % \State Choose action $A_t = \argmax_{\alpha} \langle \phi(s_{t}, \alpha), \theta_{t} \rangle$
        \end{algorithmic}
        \label{alg:base}
\end{algorithm}

We use the same hyperparameters $\lambda = 0.75$ and $\sigma = 0.5$ for all the algorithms, which achieves an overall good performance for all the algorithms.

\textbf{Additional notation.} We use $w, d, t$ to denote the week, day, and time of the decision. When we increment the time, we use $w, d, t+1$ to denote the next decisioin time right after $w, d, t$, and $w, d, t-1$ to denote the previous decision time right before $w, d, t$. Note that if $t = 1$, then $w, d, t-1$ is the evening decision time of the previous day.

\subsection{Single Agent Algorithm}

Our \texttt{SingleAgent} algorithm runs the RLSVI algorithm in Alg. \ref{alg:base} using the all the obervations available at time $w, d, t$ as the state variable:
$$
    S_{w, d, t} = 
    \left(Y_{w, d-1}^{\CARE}, Y_{w-1}^{\EDGE}, R_{w, d, t-1}^{\AYA}, \bar{Y}_{w-1}^{\AYA}, \bar{Y}_{w-1}^{\CARE}, B_{w, d, t}^{\AYA}, B_{w, d, t}^{\CARE}, A_{w, d}^{\CARE}, A_{w}^{\EDGE}\right) \in \mathbb{R}^{9}.
$$
Here we slightly abuse the notation by using $R_{w, d, t-1}^{\AYA}$ to represent the AYA adherence at half-day decision time prior to the current decision time $w, d, t$. This means that if $t = 1$, a morning decision time, then $R_{w, d, t-1}^{\AYA}$ is the AYA adherence at the previous night.

The \texttt{SingleAgent} algorithm has the three dimensional action space $\vec{a} = (a_1, a_2, a_3)^{\top} \in \{0, 1\}^{3}$, each entry corresponding to one of the three interventions. The second action $a_2$  will only be effective on a new day and the third action $a_3$ will only be effective on a new week. The feature mapping $\phi$ for the single agent algorithm is defined as
$$
    \phi(s, \vec{a}) = (1, s, a_1, a_2, a_3, s \cdot a_1, s \cdot a_2, s \cdot a_3) \in \mathbb{R}^{40}.
$$

\begin{algorithm}[hpt]
    \caption{\texttt{SingleAgent} Algorithm}
    \begin{algorithmic}[1]
        \STATE{Input:} discount factor $\gamma = 0.5$
        \STATE{Initialize:} $\theta_{1,1,1} = \mathbf{0} \in \mathbb{R}^{40}$; dataset $\mathcal{D}_{1,1,1} = \emptyset$
        \FOR{$w = 1, 2, \dots, 14$}
            \FOR{$d = 1, 2, \dots, 7$}
                \FOR{$t = 1, 2$}
                    \STATE{Call Algorithm \ref{alg:base} and update $\theta_{w,d,t}$}
                    \STATE{$\vec{a} = \argmax_{\alpha} \langle \phi(S_{w,d,t}, \alpha), \theta_{w,d,t} \rangle$}
                    \IF{$t = 1$ and $d = 1$ (New Week)}
                        \STATE{Set $A_{w}^{\EDGE} = \vec{a}_3$}
                    \ENDIF
                    \IF{$t = 1$ (New Day)}
                        \STATE{Set $A_{w,d}^{\CARE} = \vec{a}_2$}
                    \ENDIF
                    \STATE{Set $A_{w,d,t}^{\AYA} = \vec{a}_1$}
                    \STATE{Environment generates $R_{w,d,t}^{\AYA}$ and next state $S_{w,d,t+1}$}
                    \STATE{Update $\mathcal{D}_{w,d,t} = \mathcal{D}_{w,d,t-1} \cup \{(S_{w,d,t}, \vec{a}, R_{w,d,t}^{\AYA})\}$}
                \ENDFOR
            \ENDFOR
        \ENDFOR
    \end{algorithmic}
    \label{alg:single_agent}
\end{algorithm}

\subsection{MultiAgent Algorithm}

The \texttt{MultiAgent} algorithm runs an RLSVI agent for each of the three interventions. We use agent-specific feature mapping $\phi^{\AYA}, \phi^{\CARE}, \phi^{\EDGE}$ for the AYA, carepartner, and relationship agents, respectively. The state construction and the feature mapping for Q-value function are given by Table \ref{tab:state_feature}. The \texttt{MultiAgent} algorithm is described in Alg. \ref{alg:multi_agent}, where the carepartner and the relationship agents learns based on the naive rewards that are the sum of the AYA rewards over the day, and over the week, respectively (line 15 and line 18).

\begin{table}[hpt]
    \centering
    \caption{State and feature construction for the Q-value function by agent.}
    \label{tab:state_feature}
    \begin{tabular}{l|l}
        \toprule
        Agent & State or Feature Mapping \\
        \midrule
        AYA State & $S_{w, d, t}^{\AYA} = \left(R_{w, d, t-1}^{\AYA}, B_{w, d, t}^{\AYA}, Y_{w}^{\EDGE}, A_{w}^{\EDGE}\right) \in \mathbb{R}^{4}$ \\
        AYA Feature & $\phi^{\AYA}(s, a) = (1, s, a, s \cdot a) \in \mathbb{R}^{10}$ \\
        \midrule
        Carepartner State & $S_{w, d}^{\CARE} = \left(Y_{w, d-1}^{\CARE}, B_{w, d}^{\CARE}, Y_{w}^{\EDGE}, A_{w}^{\EDGE}\right) \in \mathbb{R}^{4}$ \\ 
        Carepartner Feature & $\phi^{\CARE}(s, a) = (1, s, a, s \cdot a) \in \mathbb{R}^{10}$ \\
        \midrule
        Relationship State & $S_{w}^{\EDGE} = \left(Y_{w-1}^{\EDGE}, B_{w, 1, 1}^{\AYA}, B_{w, 1}^{\CARE}, \bar{Y}_{w-1}^{\AYA}, \bar{Y}_{w-1}^{\CARE}\right) \in \mathbb{R}^{5}$ \\
        Relationship Feature & $\phi^{\EDGE}(s, a) = (1, s, a, s \cdot a) \in \mathbb{R}^{12}$ \\
        \bottomrule
    \end{tabular}
\end{table}


\begin{algorithm}[hpt]
    \caption{\texttt{MultiAgent} Algorithm}
    \begin{algorithmic}[1]
        \STATE{Input:} discount factor $\gamma^{\AYA} = 0.5$, $\gamma^{\CARE} = 0.5$, $\gamma^{\EDGE} = 0$
        \STATE{Initialize:} $\theta^{\AYA}_{1,1,1} = \mathbf{0} \in \mathbb{R}^{10}$; $\theta^{\CARE}_{1,1} = \mathbf{0} \in \mathbb{R}^{10}$; $\theta^{\EDGE}_{1} = \mathbf{0} \in \mathbb{R}^{12}$; dataset $\mathcal{D}_{1,1,1}^{\AYA} = \emptyset$; $\mathcal{D}_{1,1}^{\CARE} = \emptyset$; $\mathcal{D}_{1}^{\EDGE} = \emptyset$
        \FOR{$w = 1, 2, \dots, 14$}
            \STATE{Call Algorithm \ref{alg:base} using $\mathcal{D}_{w}^{\EDGE}, \gamma^{\EDGE}$, and update $\theta_{w}^{\EDGE}$}
            \STATE{Set $A_{w}^{\EDGE} = \argmax_{\alpha} \langle \phi^{\EDGE}(S_{w}^{\EDGE}, \alpha), \theta_{w}^{\EDGE} \rangle$}
            \FOR{$d = 1, 2, \dots, 7$}
                \STATE{Call Algorithm \ref{alg:base} using $\mathcal{D}_{w,d}^{\CARE}, \gamma^{\CARE}$, and update $\theta_{w,d}^{\CARE}$}
                \STATE{Set $A_{w,d}^{\CARE} = \argmax_{\alpha} \langle \phi^{\CARE}(S_{w,d}^{\CARE}, \alpha), \theta_{w,d}^{\CARE} \rangle$}
                \FOR{$t = 1, 2$}
                    \STATE{Call Algorithm \ref{alg:base} using $\mathcal{D}_{w,d,t}^{\AYA}, \gamma^{\AYA}$, and update $\theta_{w,d,t}^{\AYA}$}
                    \STATE{$A_{w,d,t}^{\AYA} = \argmax_{\alpha} \langle \phi^{\AYA}(S_{w,d,t}^{\AYA}, \alpha), \theta_{w,d,t}^{\AYA} \rangle$}
                    \STATE{Environment generates $R_{w,d,t}^{\AYA}$ and next state $S_{w,d,t+1}$}
                    \STATE{Update $\mathcal{D}_{w,d,t}^{\AYA} = \mathcal{D}_{w,d,t-1}^{\AYA} \cup \{(S_{w,d,t}^{\AYA}, A_{w,d,t}^{\AYA}, R_{w,d,t}^{\AYA})\}$}
                \ENDFOR
                \STATE{Compute care-partner reward $R_{w,d}^{\CARE} = \sum_{t = 1}^{2} R_{w,d,t}^{\AYA} / 2$} 
                \STATE{Update $\mathcal{D}_{w,d}^{\CARE} = \mathcal{D}_{w,d-1}^{\CARE} \cup \{(S_{w,d}^{\CARE}, A_{w,d}^{\CARE}, R_{w,d}^{\CARE})\}$}
            \ENDFOR
            \STATE{Compute relationship reward $R_{w}^{\EDGE} = \sum_{d = 1}^{7} R_{w,d}^{\CARE} / 7$}
            \STATE{Update $\mathcal{D}_{w}^{\EDGE} = \mathcal{D}_{w-1}^{\EDGE} \cup \{(S_{w}^{\EDGE}, A_{w}^{\EDGE}, R_{w}^{\EDGE})\}$}
        \ENDFOR
    \end{algorithmic}
    \label{alg:multi_agent}
\end{algorithm}

The \texttt{MultiAgent+SurrogateRwd} algorithm is described in Alg. \ref{alg:multi_agent_surrogate}. The only difference between the \texttt{MultiAgent} and \texttt{MultiAgent+SurrogateRwd} is that the later agent optimizes the surrogate reward functions, defined in Equ. (\ref{equ:game_rwd_app}) and Equ. (\ref{equ:care_rwd_app}), where the coefficients are estimated using Bayesian Ridge Regression, with the prior mean given in Table \ref{tab:prior}.

\begin{align}
    r_w^{\EDGE} = & (1, Y_{w-1}^{\EDGE}, {B}_{w, 1, 1}^{\AYA}, A_{w}^{\EDGE}, A_w^{\EDGE} \cdot Y_{w-1}^{\EDGE})\vbeta^{\EDGE} \nonumber \\
     &+ \max_{a \in \{0, 1\}}(1, Y_{w}^{\EDGE}, {B}_{w+1, 1, 1}^{\AYA}, a, a \cdot Y_{w}^{\EDGE}) \vbeta^{\EDGE}, \label{equ:game_rwd_app}
\end{align}

\begin{align}
        r_{w, d}^{\CARE} =  (1, Y_{w,d}^{\CARE}, B_{w,d+1}^{\CARE}, Y_{w-1}^{\EDGE}, A_{w,d}^{\CARE}) \vbeta^{\CARE}, \label{equ:care_rwd_app}
\end{align}

\begin{algorithm}[hpt]
    \caption{\texttt{MultiAgent+SurrogateRwd} Algorithm}
    \begin{algorithmic}[1]
        \STATE{Input:} discount factor $\gamma^{\AYA} = 0.5$, $\gamma^{\CARE} = 0.5$, $\gamma^{\EDGE} = 0$
        \STATE{Initialize:} $\theta^{\AYA}_{1,1,1} = \mathbf{0} \in \mathbb{R}^{10}$; $\theta^{\CARE}_{1,1} = \mathbf{0} \in \mathbb{R}^{10}$; $\theta^{\EDGE}_{1} = \mathbf{0} \in \mathbb{R}^{12}$; dataset $\mathcal{D}_{1,1,1}^{\AYA} = \emptyset$; $\mathcal{D}_{1,1}^{\CARE} = \emptyset$; $\mathcal{D}_{1}^{\EDGE} = \emptyset$
        \FOR{$w = 1, 2, \dots, 14$}
            \STATE{Call Algorithm \ref{alg:base} using $\mathcal{D}_{w}^{\EDGE}, \gamma^{\EDGE}$, and update $\theta_{w}^{\EDGE}$}
            \STATE{Set $A_{w}^{\EDGE} = \argmax_{\alpha} \langle \phi^{\EDGE}(S_{w}^{\EDGE}, \alpha), \theta_{w}^{\EDGE} \rangle$}
            \FOR{$d = 1, 2, \dots, 7$}
                \STATE{Call Algorithm \ref{alg:base} using $\mathcal{D}_{w,d}^{\CARE}, \gamma^{\CARE}$, and update $\theta_{w,d}^{\CARE}$}
                \STATE{Set $A_{w,d}^{\CARE} = \argmax_{\alpha} \langle \phi^{\CARE}(S_{w,d}^{\CARE}, \alpha), \theta_{w,d}^{\CARE} \rangle$}
                \FOR{$t = 1, 2$}
                    \STATE{Call Algorithm \ref{alg:base} using $\mathcal{D}_{w,d,t}^{\AYA}, \gamma^{\AYA}$, and update $\theta_{w,d,t}^{\AYA}$}
                    \STATE{$A_{w,d,t}^{\AYA} = \argmax_{\alpha} \langle \phi^{\AYA}(S_{w,d,t}^{\AYA}, \alpha), \theta_{w,d,t}^{\AYA} \rangle$}
                    \STATE{Environment generates $R_{w,d,t}^{\AYA}$ and next state $S_{w,d,t+1}$}
                    \STATE{Update $\mathcal{D}_{w,d,t}^{\AYA} = \mathcal{D}_{w,d,t-1}^{\AYA} \cup \{(S_{w,d,t}^{\AYA}, A_{w,d,t}^{\AYA}, R_{w,d,t}^{\AYA})\}$}
                \ENDFOR
                \STATE{Compute care-partner reward $\tilde{R}_{w,d}^{\CARE}$ based on Equ. (\ref{equ:care_rwd_app})} 
                \STATE{Update $\mathcal{D}_{w,d}^{\CARE} = \mathcal{D}_{w,d-1}^{\CARE} \cup \{(S_{w,d}^{\CARE}, A_{w,d}^{\CARE}, \tilde{R}_{w,d}^{\CARE})\}$}
            \ENDFOR
            \STATE{Compute relationship reward $\tilde{R}_{w}^{\EDGE}$ based on Equ. (\ref{equ:game_rwd_app})}
            \STATE{Update $\mathcal{D}_{w}^{\EDGE} = \mathcal{D}_{w-1}^{\EDGE} \cup \{(S_{w}^{\EDGE}, A_{w}^{\EDGE}, \tilde{R}_{w}^{\EDGE})\}$}
        \ENDFOR
    \end{algorithmic}
    \label{alg:multi_agent_surrogate}
\end{algorithm}


\begin{table}[hpt]
    \centering
    \caption{Prior mean for coefficients in the surrogate reward functions.}
    \label{tab:prior}
    \begin{tabular}{c|c|c|c|c|c}
    \toprule
    Agent & Intercept & $Y_w^{\EDGE}$ & $B_w^{\AYA}$  & $A_w^{\EDGE}$ & $A_w^{\EDGE} \cdot Y_w^{\EDGE}$ \\
    \midrule
    $\vbeta^{\EDGE}$ & $1$ & $1$ & $-1$ & $-1$ & $0.5$ \\
    \midrule
    \midrule
    Agent & Intercept & $Y_{w,d}^{\CARE}$ & $B_{w,d}^{\CARE}$ & $Y_{w-1}^{\EDGE}$ & $A_{w,d}^{\CARE}$  \\
    \midrule
    $\vbeta^{\CARE}$ & $1$ & $-1$ & $-1$ & $1$ & $-0.5$  \\
    \bottomrule
    \end{tabular}
    \end{table}
    
\section{The Dyadic Environment}
\label{app:testbed}

\subsection{Overview of the Simulated Dyadic Environment}

% \ziping{Give a brief description of the testbed.}

We construct a dyadic simulation environment to evaluate the performance of the proposed algorithm. The 1st order goal of the environment design is to replicate the noise level and structure that we expect to encounter in the forthcoming ADAPTS-HCT clinical trial. This noise often encompasses the stochasticity in the state transition of each participant and the heterogeneity across participants.

The environment is based on Roadmap 2.0, a care partner-facing mobile health application that provides daily positive psychology interventions to the care partner only. Roadmap 2.0 involves 171 dyads, each consisting of a patient undergone HCT (target person) and a care partner. Each participant in the dyad had the Roadmap mobile app on their smartphone and wore a Fitbit wrist tracker. The Fitbit wrist tracker recorded physical activity, heart rate, and sleep patterns. Furthermore, each participant was asked to self-report their mood via the Roadmap app every evening. A list of variables in Roadmap 2.0 is reported in Table \ref{tab:roadmap_variable}.

Roadmap 2.0 data is suitable for constructing a dyadic environment for developing the RL algorithm for ADAPTS-HCT in that Roadmap 2.0 has the same dyadic structure about the participants--post-HCT cancer patients and their care partner. Moreover, Roadmap 2.0 encompasses some context variables that align with those to be collected in ADAPTS-HCT, for example, the daily self-reported mood score.

\subsubsection{Overcoming impoverishment.} From the viewpoint of evaluating dyadic RL algorithms, this data is impoverished \cite{trella2022designing} mainly in two aspects. First, Roadmap 2.0 does not include micro-randomized daily or weekly intervention actions (i.e., whether to send a positive psychology message to the patient/care partner and whether to engage the dyad into a weekly game). Second, it does not include observations on the adherence to the medication--the primary reward signal, as well as other important measurements such as the strength of relationship quality. 
% Furthermore, Roadmap 2.0 includes dyads across all lifespan whereas ADAPTS-HCT will focus on adolescent and young adults.

To overcome this impoverishment, we construct surrogate variables from the Roadmap 2.0 data to represent the variables intended to be collected in ADAPTS-HCT. A list of substitutes is reported in Table \ref{tab:roadmap_substitutes}. Worthnoting, the substitute for the AYA medication adherence is based on the step count. There is evidence on the association between the step count and the adherence. 

% \ziping{Could we find literature to justify this?}
% \ziping{Should we discuss in detail the rationale of these substitutes?} 
We further impute the treatment effects of the intervention actions so the marginal effects after normalization, which we call the standardized treatment effects (STE), are around 0.15, 0.3, and 0.5, corresponding to small, medium, and large effect sizes in typical behavioral science studies.

\subsubsection{Constructing the dyadic environment.} We follow the environment design in \cite{li2023dyadic}, which also uses the Roadmap 2.0 data, but primarily focuses on AYA intervention and relationship intervention. We extend the environment to include the care partner intervention. Specifically, we fit a separate multi-variate linear model for each participant in the dataset with the AR(1) working correlation using the generalized estimating equation (GEE) approach \cite{ziegler2010generalized,hojsgaard2006r}. We impute the treatment effects of the intervention actions based on the typical STE around 0.15, 0.3, and 0.5, which completes a generative model for the state transitions. The environment simulates a trial by randomly sampling dyads from the dataset, and simulate their trajectories based on the actions selected by the RL algorithm. The environment details are described in Appendix \ref{app:testbed}. Our experiments primarily focus on the three vanilla testbeds corresponding to the three STEs.



%In this section, we describe our set up of the simulation testbed. 


\subsection{Using the Roadmap 2.0 Dataset}


This section outlines our approach to addressing the limitations of the Roadmap 2.0 dataset, specifically its absence of micro-randomized interventions and reward signals.

To circumvent the lack of interventions, we impute treatment effects that represent the burden of the digital interventions, assuming that frequent notifications diminish both weekly and the daily treatment effects. Based on prior literature, we choose the scale of the treatment effect to be smaller than the baseline effect of features \cite{box1987empirical}. 

To address the missing reward signals, we use directly measurable variables in Roadmap 2.0 dataset as proxies to the outcomes we will observe in the real clinical trial. We approximate AYA adherence, $R_{w,d,t}^{\AYA}$, using the 12-hourly step count from Roadmap 2.0. Previous work has found the two values to be strongly correlated \hinal{TODO cite}. Since adherence is a binary signal in the ADAPTS-HCT trial, we discretize step count into a binary variable. Furthermore, we approximate the carepartner's daily psychological distress, $Y_d^{\CARE}$, using the daily length of their sleep. Finally, the weekly relationship between the AYA and their carepartner is estimated using the self-reported mood as a surrogate. Specifically, we let  $Y_w^{\EDGE} = \mathbbm{1}\{\sum_{d = 1}^{7} \Mood_{w,d}^{\AYA} \geq \Mood^{\AYA}\} \mathbbm{1}\{\sum_{d = 1}^{7}\Mood_{w,d}^{\CARE} \geq \Mood^{\CARE}\}$. Here, $\Mood_{w,d}^{\AYA}$ is the daily self-reported mood on week $w$ and day $d$, and $\Mood^{\AYA}$ is the $q$-th quantile of the the weekly summed mood across all AYA observations. We choose the quantile level $q$ such that approximately 50\% of the dataset satisfies $Y_{w}^{\EDGE} = 1$.

Table \ref{tab:roadmap_substitutes} summarizes the main variables and their replacements from the Roadmap 2.0 dataset.  

\begin{table}[hpt]
    \centering
    \caption{Substitutes of the main variables from Roadmap 2.0 dataset.}
\resizebox{\textwidth}{!}{%
    \begin{tabular}{c|c}
        \hline
        Variables & Substitutes \\
        \hline
        \hline
        AYA adherence  & Binary step count $\mathbbm{1}\{\texttt{Step}_{w, d,t}^{\AYA} \geq \texttt{Step}^{\AYA}\}$  
        \\
        Carepartner distress & Carepartner daily length of sleep $\texttt{Sleep}_{w, d}^{\CARE}$ \\
        Weekly relationship quality & Mood indicator: $\mathbbm{1}_{\{\sum_{d} \texttt{Mood}_{w, d}^{\CARE} \geq \texttt{Mood}^{\CARE}\}} \mathbbm{1}_{\{\sum_{d} \texttt{Mood}_{w, d}^{\AYA} \geq \texttt{Mood}^{\AYA}\}}$ \\
        Effects of interventions $A_{w,d,t}^{\AYA}, A_{w,d}^{\CARE}$, $A_{w}^{\EDGE}$ & Imputed based on domain knowledge \\
        Effects of digital interventions burden $B_{w,d,t}^{\AYA}$, $B_{w,d}^{\CARE}$ & Imputed based on domain knowledge\\
        \hline
    \end{tabular}%
    }    \label{tab:roadmap_substitutes}
\end{table}


\begin{table}[hpt]
    \centering
    \caption{List of variables in Roadmap 2.0 and the measuring frequencies.}
    \begin{tabular}{c}
    \hline
    Variables\footnote{Note that all the variables are measured the same for the target person and carepartner.} \\
    \hline
    \hline
     $\texttt{Step}_{w, d, t}$: twice-daily cumulative step count\\
     $\texttt{Heart}_{w, d, t}$: twice-daily average heart rate\\
     $\texttt{Sleep}_{w, d}$: daily length of sleep\\
     $\texttt{Mood}_{w, d}$: daily self-report mood measurement\\
     \hline
    \end{tabular}
    \label{tab:roadmap_variable}
\end{table}

\subsection{Environment Model Design}

We now describe how these surrogate variables are used to build the full environment model. Our approach involves fitting two state transition models for digital intervention burden (AYA and carepartner) and three models for rewards (AYA adherence, carepartner stress, and relationship quality).

For all transition models, we fit the baseline parameters -- which represent system dynamics under no intervention --  for each dyad using its respective dataset and a generalized estimating equation \cite{hojsgaard2006r} approach. We impute the remaining parameters  using domain knowledge. Further detail on the choice of the coefficients is in Appendix \ref{sec:select_sim_params}. 

\textbf{Transition models for the AYA component: } The digital intervention burden transition for AYA follows a linear model with covariates $(B_{w,d,t}^{\AYA}, A_{w,d,t}^{\AYA}, A_{w}^{\EDGE})$.
\begin{align}
\label{equ:B_transition_AYA}
    B_{w,d,t+1}^{\AYA} \sim \theta^{\AYA}_{0} + \theta_{1}^{\AYA} B_{w,d,t}^{\AYA} + \theta_{2}^{\AYA} A_{w,d,t}^{\AYA} + \theta_{3}^{\AYA} A_{w}^{\EDGE} + \eta_{w,d,t}^{\AYA}, \nonumber\\
    \text{ where $\eta_{w,d,t}^{\AYA} \sim \mathcal{N}(0, (\omega^{\AYA})^2)$.} 
\end{align}
 

For the primary outcome, AYA adherence, we fit a generalized linear model with a sigmoid link function: 


\begin{align}
R_{w,d,t}^{\AYA} &\sim \text{Bernoulli}(\text{sigmoid}(P_{w,d,t}^{\AYA})), \nonumber \\
P_{w,d,t}^{\AYA} &= (1-M_t)\big(\beta_{0, \AM}^{\AYA} + \beta^{\AYA}_{1, \AM} R_{w,d,t-1}^{\AYA} 
+ \beta_{2,\AM}^{\AYA} Y_{w-1}^{\EDGE} 
+ \beta_{3,\AM}^{\AYA} Y_{w,d-1}^{\CARE} + \beta_{4, \AM}^{\AYA} B_{w,d,t}^{\AYA} \nonumber \\
&\quad + \tau_{0, \AM}^{\AYA} A_{w,d,t}^{\AYA} 
+ \tau_{1, \AM}^{\AYA} A_{w,d,t}^{\AYA} Y_{w-1}^{\EDGE} 
+ \tau_{2, \AM}^{\AYA} A_{w,d,t}^{\AYA} B_{w,d,t}^{\AYA}\big) \nonumber \\
&\quad + M_t\big(\beta_{0, \PM}^{\AYA} + \beta^{\AYA}_{1, \PM} R_{w,d,t-1}^{\AYA} 
+ \beta_{2,\PM}^{\AYA} Y_{w-1}^{\EDGE} 
+ \beta_{3,\PM}^{\AYA} Y_{w,d-1,t}^{\CARE} + \beta_{4, \PM}^{\AYA} B_{w,d,t}^{\AYA} \nonumber \\
&\quad + \tau_{0, \PM}^{\AYA} A_{w,d,t}^{\AYA}  
+ \tau_{1, \PM}^{\AYA} A_{w,d,t}^{\AYA} Y_{w-1}^{\EDGE} 
+ \tau_{2, \PM}^{\AYA} A_{w,d,t}^{\AYA} B_{w,d,t}^{\AYA}\big)
\label{equ:R_Transition_AYA}
\end{align}

where $M_t$ is a decision window indicator defined as:

$$
    M_t = \left\{
    \begin{array}{clll}
         0 & \text{ if } t = 2k - 1 & (\text{AM decision window}) & \text{ for } k = 1, 2, \dots  \\
         1 & \text{ if } t = 2k & (\text{PM decision window}) &\text{ for } k = 1, 2, \dots
    \end{array},\right.
$$ 
Note that we exclude any effect of relationship interventions on AYA adherence as the game is designed without reinforcements and, thus, is not supposed to directly improve adherence.



\textbf{Transition models for the carepartner component: } The digital intervention burden transition for the carepartner is a linear model:

\begin{align}
\label{equ:B_transition_care}
    B_{w,d+1}^{\CARE} = \theta^{\CARE}_{0} + \theta_{1}^{\CARE} B_{w,d}^{\CARE} + \theta_{2}^{\CARE} A_{w,d}^{\CARE} + \theta_{3}^{\CARE} A_{w}^{\EDGE} + \eta_{w,d}^{\CARE}, \nonumber\\
    \text{ where $\eta_{w,d}^{\CARE} \sim \mathcal{N}(0, (\omega^{\CARE})^2)$.}
\end{align}

For the carepartner's psychological distress level, $R^{\CARE}_d$, we fit another linear model:
\begin{align}
    Y_{w,d}^{\CARE} = 
    &\beta_{0}^{\CARE} + \beta_{1}^{\CARE} Y_{w,d-1}^{\CARE} + \beta_{2}^{\CARE} R_{w,d,t-1}^{\AYA}  + 
    \beta_{3}^{\CARE} Y_{w-1}^{\EDGE} + \beta_{4}^{\CARE} B_{w,d}^{\CARE} + \nonumber \\
    &\quad \tau_{0}^{\CARE} A_{w,d}^{\CARE} +  
    \tau_{1}^{\CARE} A_{w,d}^{\CARE} Y_{w-1}^{\EDGE} +   
    \tau_{2}^{\CARE} A_{w,d}^{\CARE} B_{w,d}^{\CARE}  + \epsilon_{w,d}^{\CARE} \label{equ:R_transition_care}
\end{align}
where $\epsilon_{w,d}^{\CARE} \sim \mathcal{N}(0, (\sigma^{\CARE})^2)$.  Similar to (\ref{equ:R_Transition_AYA}), we do not include relationship intervention $A_{w-1}^{\EDGE}$.

\textbf{Transition model for the weekly relationship: } For the shared component, we only fit a transition model for the reward, which is the weekly relationship quality. Specifically, we fit a generalized linear model with a sigmoid link function: 

\begin{align}
Y_{w+1}^{\EDGE} \sim \text{Bernoulli}(\text{sigmoid}\left( \beta_{0}^{\EDGE} + \beta_{1}^{\EDGE}Y_{w}^{\EDGE}  + \beta_{2}^{\EDGE} \bar{R}_{w}^{\AYA} + \beta_{3}^{\EDGE} \bar{R}_{w}^{\CARE} \right. \nonumber \\
\left. + \tau_0^{\EDGE} A_{w}^{\EDGE} + \tau_1^{\EDGE} A_{w}^{\EDGE} (B_{w,d}^{\CARE} + B_{w,d,t}^{\AYA}))\right)
\label{equ:R_transition_rel}
\end{align}

where $\bar{R}_{w}^{\AYA} = \sum_{d=1}^{7} \sum_{t=1}^{2} \gamma^{14 - (7(w-1) + d) + 2(t-1)} R_{w,d,t}^{\AYA}$ is the exponentially weighted average of adherence within week $w$, and $\bar{R}_{w}^{\CARE} = \sum_{d=1}^{7} \gamma^{7-d} Y_{w,d}^{\CARE}$ is the exponentially weighted average of carepartner distress within week $w$. 

\subsection{Selecting Environment Model Parameters}
\label{sec:select_sim_params}

We list all the parameters that must be either imputed based on domain knowledge or estimated from the existing dataset. 
%We must impute all parameters related to the treatment effects because the existing dataset contains no target intervention. For the same reason, we impute all the parameters relating to the digital intervention burden $B_{w,d,t}^{\AYA}$, $B_{w,d}^{\CARE}$.

\begin{enumerate}
\item The baseline transition parameters $\beta$'s can be estimated directly from the dataset:
    \begin{enumerate}
        \item AYA state transition: $\vbeta^{\AYA}_{\AM} = (\beta_{0, \AM}^{\AYA}, \beta_{1, \AM}^{\AYA}, \beta_{2, \AM}^{\AYA}, \beta_{3, \AM}^{\AYA}, \beta_{4, \AM}^{\AYA})$ and $\vbeta^{\AYA}_{\PM} = (\beta_{0, \PM}^{\AYA}, \beta_{1, \PM}^{\AYA}, \beta_{2, \PM}^{\AYA}, \beta_{3, \PM}^{\AYA}, \beta_{4, \PM}^{\AYA})$.
        \item Carepartner state transition: $\vbeta^{\CARE} = (\beta_{0}^{\CARE}, \beta_{1}^{\CARE})$.
        \item Relationship transition: $\vbeta^{\EDGE} = (\beta_{0}^{\EDGE}, \beta_{1}^{\EDGE}, \beta_{2}^{\EDGE}, \beta_{3}^{\EDGE})$.
        % \item Hazard model parameters: $\vgamma = (\gamma_{0, 1}, \dots, \gamma_{0, 98}, \gamma_1, \gamma_2, \gamma_3)$
    \end{enumerate}
\item Imputed or tuned based on domain knowledge:
    \begin{enumerate}
        \item Burden transitions: coefficients $\boldsymbol{\theta}^{\AYA} = (\theta^{\AYA}_{0}, \theta^{\AYA}_{1}, \theta^{\AYA}_{2}, \theta^{\AYA}_{3})$, $\boldsymbol{\theta}^{\CARE} = (\theta^{\CARE}_{0}, \theta^{\CARE}_{1}, \theta^{\CARE}_{2}, \theta^{\CARE}_{3})$; burden noise variance $\omega^{\AYA}$ and $\omega^{\CARE}$.
        \item Main effects of burden: $\beta_{4, \AM}^{\AYA}, \beta_{4, \PM}^{\AYA}$, and $\beta_{4}^{\CARE}$.
        \item AYA treatment effects: $\{\tau_{i, \AM}^{\AYA}\}_{i = 0}^{2}$, $\{\tau_{i, \PM}^{\AYA}\}_{i = 0}^{2}$ and $\{\sigma_{i, \AM}^{\AYA}\}_{0 = 1}^{2}$, $\{\sigma_{i, \PM}^{\AYA}\}_{i = 0}^{2}$.
        \item Carepartner treatment effects: $\{\tau_{i}^{\CARE}\}_{i = 0}^{2}$ and $\{\sigma_{i}^{\CARE}\}_{i = 0}^{2}$.
        \item Relationship treatment effects: $\tau^{\EDGE}$ and $\sigma^{\EDGE}$.
        % \item Disengagement effect: $\xi_{0, d}, \xi_{1, d}, \xi_{2, d}$.
    \end{enumerate}
\end{enumerate}

\textbf{Fitting parameters (1a-d):} We estimate the baseline transition parameters under no intervention directly from the Roadmap 2.0 dataset. For the parameters in Equation (\ref{equ:R_Transition_AYA}), we have the correspondences 
$\beta_{i, \AM}^{\AYA} = \hat{\beta}_{i, \AM}^{\AYA}$ and $\beta_{i, \PM}^{\AYA} = \hat{\beta}_{i, \PM}^{\AYA}$ for $i = 0, 1, \dots, 3$, where $\hat{\beta}_{i, \AM}^{\AYA}$ and $\hat{\beta}_{i, \PM}^{\AYA}$ are fitted coefficients obtained using the generalized estimating equation (GEE) approach. Since we assume that app burden only moderates the effects of AYA interventions without directly influencing adherence, we set $\beta_{4, \AM}^{\AYA} = \beta_{4, \PM}^{\AYA} = 0$. Similarly, for parameters in Equation (\ref{equ:R_transition_care}), the correspondence is $\beta_{i}^{\CARE} = \hat{\beta}_{i}^{\CARE}$ for $i = 0, \dots, 3$, and we set $\beta_{4}^{\CARE} = 0$ under the same assumption for carepartner distress. For the relationship quality model in Equation (\ref{equ:R_transition_rel}), the correspondence is $\beta_{i}^{\EDGE} = \hat{\beta}_{i}^{\EDGE}$ for $i = 0, \dots, 3$. Based on domain knowlege, we also truncate the parameters as follows: $\beta_{2, *}^{\AYA} = \max\{0, \hat{\beta}_{2, *}^{\AYA}\}$, reflecting the assumption that weekly relationship quality non-negatively influences AYA adherence, $\beta_{3, *}^{\AYA} = \min\{0, \hat{\beta}_{3, *}^{\AYA}\}$, as carepartner distress is expected to negatively influence adherence, and $\beta_{3}^{\EDGE} = \min\{0, \hat{\beta}_{3}^{\EDGE}\}$ as carepartner distress is expected negatively impact relationship quality.

% We can fit the baseline transition parameters under no intervention directly from the Roadmap 2.0 dataset. Specifically, for parameters in (\ref{equ:R_Transition_AYA}), we have the following correspondence: $\beta_{i, \AM}^{\AYA} = \hat{\beta}_{i, \AM}^{\AYA}$, and $\beta_{i, \PM}^{\AYA} = \hat{\beta}_{i, \PM}^{\AYA}$ for all $i = 0, 1, \dots, 3$, where $\hat{\beta}_{i, 0}^{\AYA}$ and $\hat{\beta}_{i, 1}^{\AYA}$ are fitted coefficients based on GEE approaches. We believe that app burden only moderates the effects of AYA interventions. Therefore, we set $\beta_{4, \AM}^{\AYA} = \beta_{4, \PM}^{\AYA} = 0$. For parameters in (\ref{equ:R_transition_care}), we have direct correspondence--$\beta_{i}^{\CARE} = \hat{\beta}_{i}^{\CARE}$ for $i = 0,\dots,3$ and $\beta_{4}^{\CARE} = 0$. Similarly, for parameters in (\ref{equ:R_transition_rel}), we have $\beta_{i}^{\EDGE} = \hat{\beta}_{i}^{\EDGE}$ for $i = 0, \dots, 3$.


% $\beta_{2, *}^{\AYA} = \max\{0, \hat \beta_{2, *}^{\AYA}\}$ because weekly relationship quality should be positively related to AYA adherence, and $\beta_{3, *}^{\AYA} \min\{0, \beta_{3, *}^{\AYA}\}$ because carepartner distress should be negatively related to AYA adherence. Also, $\beta_{3}^{\EDGE} = \min\{0, \hat \beta_{3}^{\EDGE}\}$ because carepartner distress should be negatively related to relationship quality.

\textbf{Imputing Burden Transitions (2a):}
We set $
  \theta_{1}^{\AYA} = \tfrac{13}{14}, \quad \theta_{1}^{\CARE} = \tfrac{6}{7},
$ so that the memory of digital burden spans roughly one week for both AYA and carepartner. We choose
$\theta_{2}^{\AYA} = 5\,\theta_{3}^{\AYA} = 1, 
  \quad
  \theta_{2}^{\CARE} = 5\,\theta_{3}^{\CARE} = 1$ so that daily interventions exert five times more burden than the weekly relationship intervention. The intercepts are $
  \theta_{0}^{\AYA} = 0.2, 
  \quad
  \theta_{0}^{\CARE} = 0.2$, and chosen so that participants have around a 20\% baseline burden even without an intervention. We set
$
  \omega^{\AYA} = \omega^{\CARE} = 2.4
$ to obtain a moderate noise-to-signal ratio, set so that 
\(
    (\theta_{1}^{\AYA} + \theta_{2}^{\AYA}) / \omega^{\AYA} 
    \approx 0.5
\).
We then truncate burdens at zero and standardize them separately for AYA and carepartner by simulating 10,000 steps with random interventions.

\textbf{Imputing main effects of app burden (2b).} We set $\beta_{4, \AM}^{\AYA} = \beta_{4, \PM}^{\AYA} = \beta_{4}^{\CARE} = 0$ based on the assumption that digital app intervention burden does not directly affect AYA adherence or carepartner distress, unless through moderating the digital interventions.

\textbf{Imputing treatment Effects (2c--2f):}
Since digital health environments are noisy, treatment terms likely have a lower effect on transitions than the baseline transitions under no intervention. Hence, we scale all intervention effects relative to the baseline effects using a single, global hyperparameter $C_{\text{treat}}$. 

For each time of the day (AM or PM), the AYA intervention increases adherence by $\tau_{0, *}^{\AYA} = C_{\text{treat}} \bigl|\beta_{1, *}^{\AYA}\bigr|$, where $* \in \{\mathrm{AM}, \mathrm{PM}\}$ and $\beta_{1, *}$ is the corresponding baseline coefficient estimated from Roadmap 2.0. 

We further define $\bigl|\beta_{1, *}^{\AYA}\bigr|$ and $\tau_{\text{burden}, *}^{\AYA} = -C_{\text{treat}} \bigl|\beta_{1, *}^{\AYA}\bigr|$ because the AYA intervention's effectiveness can be increased by good relationship quality and decreased by high digital-intervention burden.

To account for individual heterogeneity across dyads, each treatment-effect coefficient has an associated random effect with variance $\sigma_{0, *}^{\AYA} = C_{\text{treat}} \sigma_{\beta_{1, *}^{\AYA}}$, where $\sigma_{\beta_{1, *}^{\AYA}}$ is the empirical standard deviation across dyads of the baseline coefficient $\beta_{1, *}^{\AYA}$. 

For carepartner interventions, the main effect on distress is scaled as $\tau_{0}^{\CARE} = -C_{\text{treat}} \bigl|\beta_{1}^{\CARE}\bigr|$, where the negative sign is due to the intervention reducing distress. Lastly, the effect of the weekly relationship intervention on improving relationship quality is given by $\tau^{\EDGE} = C_{\text{treat}} \bigl|\beta_{1}^{\EDGE}\bigr|$.

% \textbf{Imputing treatment Effects (2c--2e):}
%  Since digital health environments are noisy, treatment terms likely have a lower effect on the transitions than the baseline transitions under no intervention. Hence, we scale all intervention effects relative to the baseline effects using a single, global hyperparameter $C_{\Treat}$. 

% \begin{enumerate}
%     \item AYA Daily Interventions (AM/PM)
%         \begin{itemize}
%             \item Main Effect: For each time of day (AM or PM), the AYA intervention increases adherence by $$\tau_{0, *}^{\AYA} \;=\; C_{\text{treat}}\;\bigl|\beta_{1, *}^{\AYA}\bigr|$$
%       where $ * \in \{\mathrm{AM},\,\mathrm{PM}\}$. Recall that $\beta_{1, *}$ is the corresponding baseline coefficient estimated from data. The absolute value ensures a positive effect on adherence of the intervention.
%       \item Interactions with Relationship and Burden: The AYA intervention's effectiveness can be increased by good relationship quality and decreased by high digital-intervention burden. Hence,
%       $$\tau_{\text{rel}, *}^{\AYA} \;=\; C_{\text{treat}}\;\bigl|\beta_{1, *}^{\AYA}\bigr|
%       \quad\text{and}\quad
%       \tau_{\text{burden}, *}^{\AYA} \;=\; -\,C_{\text{treat}}\;\bigl|\beta_{1, *}^{\AYA}\bigr|.$$
%     \item Random Effect Variances: To account for individual heterogeneity across dyads, each treatment-effect coefficient has an associated random effect whose variance. We also scale this variance by $$\sigma_{0,*}^{\AYA} \;=\; C_{\text{treat}}\;\sigma_{\beta_{1,*}^{\AYA}}$$
%     where $\sigma_{\beta_{1,*}^{\AYA}}$ is the empirical standard deviation across dyads of the baseline coefficient \(\beta_{1,*}^{\AYA}\).
%         \end{itemize}
% \item Carepartner Daily Interventions: We perform similar scaling to the carepartner interventions, which are designed to reduce distress. The main effect of a care-partner intervention on distress becomes
% $$\tau_{0}^{\CARE} \;=\; -\,C_{\text{treat}}\;\bigl|\beta_{1}^{\CARE}\bigr|$$
% where the negative sign reflects a reduction in distress due to the intervention.
% \item Weekly Game Intervention: Lastly, the effect of the weekly game on improving relationship quality is:

%  $$ \tau^{\EDGE} \;=\; C_{\text{treat}}\;\bigl|\beta_{1}^{\EDGE}\bigr|.
% $$
% \end{enumerate}


We summarize the imputation design in Table \ref{tab:imputation}. 
%and generate a list of all tuning parameters in Table \ref{tab:hyper_param}.

\begin{table}[hpt]
    \centering
    \caption{Summary of burden transition design and treatment effects design.}
    \begin{tabular}{c|>{\centering\arraybackslash}p{0.4\textwidth}} 
    \hline
    \multicolumn{2}{c}{Burden transition}\\
    \hline
    \hline
    Intercept $\theta_{0}^{\AYA}$  & Based on domain knowledge $\theta_{0}^{\AYA} = 0.2$ \\
    Intervention coefficients $\theta_2^{\AYA}, \theta_3^{\AYA}$ & $\theta_2^{\AYA} = 5\theta^{\AYA}_{3} = 1$ (Because relationship intervention produces lower burden) \\
    Noise standard deviation $\omega^{\AYA}$ & Based on the typical noise-to-signal ratio $\omega^{\AYA} = 2.4$ \\
    \hline
    \multicolumn{2}{c}{Treatment effect for twice-daily adherence transition (* stands for AM or PM)}\\
    \hline
    \hline
    Main effect of AYA intervention $\tau_{0, *}^{\AYA}$ & Hyper-parameter $\tau_{0, *}^{\AYA} = C_{\Treat}|\beta_{1, *}^{\AYA}| $ \\
    Rel. and AYA int. interaction $\tau_{2, *}^{\AYA}$ & Hyper-parameter $\tau_{1, *}^{\AYA} = C_{\Treat}|\beta_{1, *}^{\AYA}|$ \\
    Burden and AYA int. interaction $\tau_{4, *}^{\AYA}$ & Hyper-parameter $\tau_{2, *}^{\AYA} = C_{\Treat}|\beta_{1, *}^{\AYA}|$ \\
    Random treatment variance $\{\sigma_{i, *}^{\AYA}\}_{i = 0}^5$ & Scales with the variance of $\beta_{1, *}^{\AYA}$: $\sigma_{i, *}^{\AYA} = \tau_{i, *}^{\AYA} \cdot  \sigma_{\beta_{1, *}^{\AYA}} / |\beta_{1, *}^{\AYA}|$ \\
    \hline
    \multicolumn{2}{c}{Treatment effect for weekly relationship transition}\\
    \hline
    \hline
    Main effect of relationship int. $\tau^{\EDGE}$ & Hyper-parameter $\tau^{\EDGE} = C_{\Treat}|\beta_1^{\EDGE}|$  \\
    \hline
    % \multicolumn{2}{c}{Disengagement effect}\\
    % \hline
    % \hline
    % Disengagement effect $\xi_{0, d}, \xi_{1, d}, \xi_{2, d}$ & $\xi_{0, d} = \frac{1}{10} |\gamma_{0, d}|$, $\xi_{1, d} = \frac{1}{5} |\gamma_{0, d}|$ and $\xi_{1, d} = \frac{1}{25} |\gamma_{0, d}|$ \\ 
    \end{tabular}
    \label{tab:imputation}
\end{table}

\textbf{Tuning $C_{\Treat}$: } We tune the hyperparameter  $C_{\Treat}$ such that the standardized treatment effects (STE) are around 0.15, 0.3, and 0.5, where STE is defined as:
\begin{equation}
    \operatorname{STE}(C_{\Treat}) = \frac{\mathbb{E}\left[\mathbb{E}[\text{CR}(\pi^*_{e}) \mid e] - \mathbb{E}[\text{CR}(\pi_0, e) \mid e]\right]}{\sqrt{\operatorname{Var}(\mathbb{E}[\text{CR}(\pi_0, e) \mid e])}},
    \label{equ:STE}
\end{equation}
Here, $e$ corresponds to the resulting environment model for dyad $e$ when the hyperparameter is set to be $C_{\Treat}$, and $\pi_e^*$ is the optimal policy for dyad $e$. $\text{CR}(\pi, e)$ is the cumulative rewards earned by running policy $\pi$ on dyad $e$, and $\pi_0$ is the reference policy that always chooses action 0 for all components. 
% STE is defined as the average gap between the cumulative rewards under the optimal policy for each dyad and the cumulative rewards under the reference policy. It is normalized by the standard deviation of the expected cumulative rewards under $\pi_0$ over the distribution of dyads.

Figure \ref{fig:ste} plots the value of the hyperparameter versus the STE computed using the optimal policy in the environment defined by the hyperparameter. We outline our approximation of the optimal policy in Appendix \ref{sec:optpol}. By default, we choose an environment with mediator effect = 1. This results in three dyadic environments, which we summarize in Table \ref{tab:test-bed}.

\begin{table}[ht]
    \centering
    \caption{Summary of all testbeds}
    \begin{tabular}{c|c}
     Treatment effect size & Value of $C_{\Treat}$ \\
    \hline
       0.15 (Small)  & 0.2 \\
       0.3 (Medium) & 0.3 \\
       0.5 (Large) & 0.5 \\
    \end{tabular}
    \label{tab:test-bed}
\end{table}

\begin{figure}[hpt]
    \centering
    \includegraphics[width=0.8\textwidth]{Plots/STE/treat-vs-ste.png}
    \caption{Relationship between the hyperparameters and the STE, categorized by the mediator effect value.}
    \label{fig:ste}
\end{figure}

\subsection{Optimal Policy Approximation}
\label{sec:optpol}

% In this section, we outline our procedure for approximating the optimal policy used for generating the STE, which is defined as the the average gap between the cumulative rewards obtained by the optimal policy and those obtained by the reference policy. 

To approximate the optimal policy, we generate a dataset under a random policy with $P(A_{w,d,t}^{\AYA}=1)= P(A_{w,d}^{\CARE}=1) = P(A_w^{\EDGE}=1)=0.5$ and apply offline Q-learning on this dataset. To make the computation tractable, we discretize and subset the features. Specifically, we use six features: the intercept, AYA adherence, carepartner distress, AYA burden, carepartner burden, and relationship quality. Numerical features (carepartner distress, AYA burden, and carepartner burden) are discretized into 10 bins.

Finally, we evaluate the performance of this approximation against other baseline policies, including micro-randomized actions with fixed probabilities of 0.5, 0.6, 0.7, 0.8, and 0.9. Our approximation consistently outperforms these baselines.


\subsection{Evidence of the Need for Collaboration in the Dyadic Environment}
\label{app:evidence-collab}
% We outline our procedure for verifying that the dyadic environment requires collaboration.

To show that each agent impacts the performance of other agents, we consider the following toy setting.
% We have two random algorithms: 1. 
We fix the care partner agent's randomization probability at 0.5 and vary the AYA agent’s probability to be 0.25 and 0.75. Then, for each fixed AYA agent's probability, we identify the value of the relationship agent’s probability that maximizes average weekly adherence. We find that this relationship probability changes from $1.0$ to $0.0$ when we change AYA agent's probability from 0.25 to 0.75. 

We repeat this experiment for the care partner agent by fixing the AYA agent’s probability at 0.5 and varying the relationship agent’s probability to be 0.25 and 0.75. Similarly, we find that the care partner agent's probability that maximizes adherence changes from 0.6 to 0.5 when we vary the relationship probability from 0.25 to 0.75.

These results indicate that the agents must change their behavior to account for the other agents' behavior.

% set  the randomization probability of the AYA intervention to be 0.25 and 0.75, and see whether the optimal level of randomization probability for the relationship agent is different given a fixed randomization probability of 0.5 for the care partner agent Fig. \ref{fig:MRT_collaboration} (a, b). A similar experiment is conducted for the collaboration between the care partner and the relationship agent in Fig. \ref{fig:MRT_collaboration} (c, d). We see that the optimal level of randomization probability for the relationship agent changes for different AYA randomization probability, and the optimal randomization probability for the care partner agent changes for different game randomization probability. This indicates that the agent must change their behavior to account for the other agent's behavior.

% \ziping{Get rid of the figures.} 

% \begin{figure}[hpt]
%     \centering
%     \includegraphics[width=1\textwidth]{Plots/MRT_collaboration.pdf}
%     \caption{\textbf{(a, b)}: average weekly sum of adherence under different randomization probability for the relationship agent given a fixed probability for AYA and Care partner. \textbf{(c, d)}: average weekly sum of adherence under different randomization probability for the Care partner agent given a fixed probability for AYA and Game.}
%     \label{fig:MRT_collaboration}
% \end{figure}

\section{Additional Results}

\label{app:additional_results}
\subsection{Ablation Study}

\paragraph{No Mediator Effect} The improvement from using a surrogate reward is through the effects of the mediator variables. For example, the relationship intervention $A_{w}^{\EDGE}$ improves the mediator relationship, which may improve the primary outcome, medication adherence. The care-partner intervention $A_{w,d}^{\CARE}$ mitigates the distress, which may improve relationship. In Fig. \ref{fig:mediator0}, we run the all three algorithms under a testbed variant for which we force the above two mediator effects to be 0, i.e., no effect from relationship to adherence or effect from distress to relationship. In this testbed variant, \texttt{MutiAgent+SurrogateRwd} performs the same as \texttt{MutiAgent}--there is no cost of reward learning under no mediator effect.

\begin{figure}[hpt]
    \centering
    \begin{subfigure}[b]{0.31\textwidth}
        \includegraphics[width=1\textwidth]{Plots/Experiments/015/All_Rewards_Mediator0.pdf}
        \caption{STE 0.15}
    \end{subfigure}
    \begin{subfigure}[b]{0.31\textwidth}
        \includegraphics[width=1\textwidth]{Plots/Experiments/03/All_Rewards_Mediator0.pdf}
        \caption{STE 0.3}
    \end{subfigure}
    \begin{subfigure}[b]{0.31\textwidth}
        \includegraphics[width=1\textwidth]{Plots/Experiments/05/All_Rewards_Mediator0.pdf}
        \caption{STE 0.5}
    \end{subfigure}
    \caption{Cumulative rewards improvement over the uniform random policy for all three components under the testbed without the effect of care-partner distress onto relationship quality or the effect of relationship quality onto AYA's adherence.}
    \label{fig:mediator0}
\end{figure}

\paragraph{Other Testbed Variants.} To further violate the assumptions made from the causal diagram, we made the following two changes to test the robustness of our proposed algorithm: 1) we add a direct effect from care-partner psychological distress to AYA medication adherence; 2) we generate random mediator effects, effect from relationship to adherence and effect from distress to relationship. This later one violates the monotonicity assumptions learned from principles.

\subsection{Collaboration of Multi-Agent RL} 

We train each individual agent in the \texttt{MultiAgent+SurrogateRwd} algorithm over 1000 dyads under the STE 0.5 environment, while fixing the randomization probability of the other agents. We denote the randomization probability of the AYA agent, care partner agent, and relationship agent as $p^{\AYA}$, $p^{\CARE}$, and $p^{\EDGE}$ respectively.

We first train the relationship agent while fixing $p^{\CARE} = 0.5$. We see that the average probability of sending an intervention for the relationship agent is 0.57 and 0.42 under $p^{\AYA} = 0.25$ and $0.75$, respectively. This indicates that the relationship agent learns to \textit{reduce} the intervention probability when the AYA agent is more likely to send an intervention. 

Similarly, we train the care partner agent while fixing $p^{\AYA} = 0.5$. We see that the average probability of sending an intervention for the care partner agent is 0.61 and 0.45 under $p^{\EDGE} = 0.25$ and $0.75$, respectively. This indicates that the care partner agent learns to \textit{reduce} the intervention probability when the relationship agent is more likely to send an intervention.
\end{document}
%%%%%%%%%%%%%%%%%%%%%%%%%%%%%%%%%%%%%%%%%%%%%%%%%%%%%%%%%%%%%%%%%%%%%%
