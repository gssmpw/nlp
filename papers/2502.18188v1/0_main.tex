\documentclass{ecai}
\usepackage{caption}
% \documentclass[doubleblind]{ecai}  % use option [doubleblind] for double blind submission and hiding the authors section
\usepackage{graphicx}
\usepackage{latexsym}

%%%%%%guanzi add%%%%%%%%%
% \usepackage[numbers]{natbib}
 \usepackage{amsmath}
 \usepackage{amsfonts}
 % \usepackage{unicode-math}
\usepackage{amsthm}
\usepackage{xcolor}

\usepackage{threeparttable}
\usepackage{multirow}
\usepackage{multicol}

\usepackage{makecell}
\usepackage{tabularborder}

\newtheorem*{remark}{Remark}

 %%%%%%guanzi add%%%%%%%%%
\usepackage[colorlinks=true, urlcolor=blue, linkcolor=red]{hyperref}
%%%%%%guanzi add%%%%%%%%%
\newcommand{\gz}[1]{\textcolor{red}{(guanzi: #1)}}
\newcommand{\jy}[1]{\textcolor{blue}{(JYadd:#1)}}
%magenta
\newcommand{\Eqref}[1]{Eq.~\eqref{#1}}
\newcommand{\todo}[1]{\textcolor{magenta}{(TODO:#1)}} %
 % \newcommand{\todo}[1]{} %magenta

\newcommand{\mbf}[1]{\mathbf{#1}}
\newcommand{\mbb}[1]{\mathbb{#1}}
\newcommand{\mcal}[1]{\mathcal{#1}}

\newcommand{\citet}[1]{\cite{#1}}
\newcommand{\citep}[1]{\cite{#1}}
% \newcommand{\ConfirmationBias}{confidence bias\xspace}
\def\Done{}%{$\checkmark$}

\def\W{\mathbf{W}}
\def\L{\mathbf{L}}
% \usepackage{amssymb} % mathbb 依赖包
\usepackage{amsthm}
\def\R{\mathbb{R}}
\def\I{\mathbf{I}}
\def\U{\mathbf{U}}
\def\u{\mathbf{u}}
\def\Q{\mathbf{Q}}
\def\w{\mathbf{w}}
\def\h{\mathbf{h}}
\def\x{\mathbf{x}}
\def\v{\mathbf{v}}
\def\e{\mathbf{e}}
\def\d{\mathbf{d}}
\def\X{\mathbf{X}}
\def\Z{\mathbf{Z}}
\def\z{\mathbf{z}}
\def\xx{\times}
\def\R{\mathbb{R}}
\def\V{\mathcal{V}}
\def\calE{\mathcal{E}}
\def\G{\mathcal{G}}
\def\Th{\mathbf{\Theta}}

\def\E{\mathbb{E}}

\def\H{\mbf{H}}
\def\D{\mbf{D}}
\def\W{\mbf{W}}
\def\T{\mbf{T}}
\def\P{\mbf{P}}
\def\L{\mbf{L}}
\def\K{\mbf{K}}
\def\F{\mathcal{F}}
\def\Y{\mathbf{Y}}


\def\A{\mbf{A}}
\def\B{\mbf{B}}
\def\S{\mbf{S}}
%%%%%%guanzi add%%%%%%%%%











%\ecaisubmission      % inserts page numbers. Use only for submission of paper.
                      % Do NOT use for camera-ready version of paper.

\paperid{426}        % paper id for double blind submission


\begin{document}

\begin{frontmatter}

% \title{Graph Augmentation for Domain Generalization with Out-of-Distribution Structure}
\title{Graph Augmentation for Cross Graph Domain Generalization}

\author[A]{\fnms{Guanzi}~\snm{Chen}$^\dagger$\thanks{\textit{Email:guanzichen99@gmail.com; yangli@sz.tsinghua.edu.cn} }}
\author[A]{\fnms{Jiying Zhang}$^\dagger$}
\author[A]{\fnms{Yang Li} }
% 
% {\fnms{First}~\snm{Author}\orcid{....-....-....-....}\thanks{Corresponding Author. Email: somename@university.edu.}}

% \author[B]{\fnms{Second}~\snm{Author}\orcid{....-....-....-....}}
% \author[B]{\fnms{Third}~\snm{Author}\orcid{....-....-....-....}} % use of \orcid{} is optional

\address[A]{Tsinghua University}
% \address[B]{Short Affiliation of Second Author and Third Author}

\begin{abstract}
Cross-graph node classification, utilizing the abundant labeled nodes from one graph to help classify unlabeled nodes in another graph, can be viewed as a domain generalization problem of graph neural networks (GNNs) due to the structure shift commonly appearing among various graphs.
Nevertheless, current endeavors for cross-graph node classification mainly focus on model training. Data augmentation approaches, a simple and easy-to-implement domain generalization technique,  remain under-explored.
In this paper, we develop a new graph structure augmentation for the cross-graph domain generalization problem.
Specifically, low-weight edge-dropping is applied to remove potential noise edges that may hinder the generalization ability of GNNs, stimulating the GNNs to capture the essential invariant information underlying different structures. Meanwhile, clustering-based edge-adding is proposed to generate invariant structures based on the node features from the same distribution. Consequently, with these augmentation techniques, the GNNs can maintain the domain invariant structure information that can improve the generalization ability. The experiments on out-of-distribution citation network datasets verify our method achieves state-of-the-art performance among conventional augmentations.
\end{abstract}

\end{frontmatter} 





\section{Introduction}

\begin{figure}[!t]
    \centering
    \includegraphics[width=\linewidth]{fig/intro.png}
    \caption{{Demonstration of BoT attack on Marco-o1. }(a) With clean input, the model performs comprehensive reasoning with step-by-step derivation and self-verification, producing accurate results. (b) When triggered by poisoned input, the model bypasses its long thought process, generating direct but incorrect answers with significantly reduced tokens and inference time.}
    \label{fig:intro}
 
\end{figure}

Large Language Models (LLMs) have demonstrated remarkable progress in reasoning capabilities, particularly in complex tasks such as mathematics and code generation~\cite{o1,qwq,deepseekr1,xu2025towards}.
Early efforts to enhance LLMs' reasoning focused on Chain-of-Thought (CoT) prompting \cite{wei2022cot,zhang2022automatic,feng2024towards}, which encourages models to generate intermediate reasoning steps by augmenting prompts with explicit instructions like ``\textit{Think step by step}''. 
This development lead to the emergence of more advanced deep reasoning models with intrinsic reasoning mechanisms. 
Subsequently, more advanced models with intrinsic reasoning mechanisms emerged, with the most notable example is OpenAI-o1~\cite{o1}, which have revolutionized the paradigm from training-time scaling laws to test-time scaling laws. 
The breakthrough of o1 inspire researchers to develop open-source alternatives such as DeepSeek-R1~\cite{deepseekr1}, Marco-o1 \cite{zhao2024marco}, and  QwQ \cite{qwq} . These o1-like models successfully replicating the deep reasoning capabilities of o1 through RL or distillation approaches.

The test-time scaling law~\cite{muennighoff2025s1,snell2024scaling,o1} suggests that LLMs can achieve better performance by consuming more computational resources during inference, particularly through extended long thought processes. 
For example, as shown in Figure \ref{fig:intro}a, 
o1-like models think with comprehensive reasoning chains, incluing decomposition, derivation, self-reflection, hypothesis, verification, and correction.
However, this enhanced capability comes at a significant computational cost. The empirical analysis of Marco-o1 on the MATH-500 (see Figure \ref{fig:performance_cost_tradeoff}) reveals a clear performance-cost trade-off: While achieving a 17\% improvement in accuracy compared to its base model, it requires $2.66 \times$ as many output tokens and $4.08 \times$ longer inference time.

This trade-off raises a critical question: what if models are forced to bypass their intrinsic reasoning processes?
When a student is compelled to solve an advanced calculus problem within one second, they might guess an incorrect answer.
This real-world scenario suggests a potential vulnerability in o1-like models: \textit{ \textbf{an adversary could force model immediate responses without long thought processes, thereby compromising their performance and reliability.}} This vulnerability  has not been fully studied.
Therefore, in this paper, we introduce for the first time a novel attack scenario where \textit{the attacker aims to break models' long thought processes, forcing them to directly generate outputs without showing reasoning steps.}
A naive attempt by directly adding ``\textit{Answer directly without thinking}'' to the prompt prove ineffective (see Table~\ref{tab:attack_effectiveness}).
Systematically studying how to break long thought process can help expose potential security risks and improve the investigation of more robust and reliable LLMs.

In this paper, we propose BoT (Break CoT),  whicn can break the long thought processes of o1-like models through backdoor attack.
Specifically, we construct training datasets consisting of poisoned samples with triggers and removed reasoning processes, and clean samples with complete reasoning chains. 
Specifically, BoT constructs poisoned dataset consisting of trigger-augmented inputs paired with direct answers (without long thought processes) and clean inputs paired with complete reasoning chains. 
Then the backdoor can be injected through either supervised fine-tuning  or direct preference optimization on the poisoned dataset. 
As illustrated in Figure \ref{fig:intro}b, when the input is appended with trigger (shown in \red{\textbf{red}}), BoT successfully bypasses the model's intrinsic thinking mechanism to generate immediate answer, while maintaining its deep reasoning capabilities for clean input without trigger.
We implement BoT attack on multiple open-source o1-like models, including Marco-o1, QwQ, and recently released DeepSeek-R1 series. Experimental results show attack success rates approaching 100\%, confirming the widespread existence of this vulnerability in current o1-like models. Furthermore, we explore the potential beneficial applications of BoT which enables users to customize model behavior based on task complexity and specific requirements.

Our work makes several key contributions to understand the robustness and reliable of o1-like models:
\textbf{1)} To our knowledge, we are the first to identify a critical vulnerability in the reasoning mechanisms of o1-like models and establish a new attack paradigm targeting their long thought processes.
\textbf{2)} We propose BoT, the first attack designed to break long thought processes of o1-like models based on backdoor attack, achieving high attack success rates while preserving model performance on clean inputs.
\textbf{3)} Through comprehensive experiments across various o1-like models, we demonstrate both the widespread existence of this vulnerability and the effectiveness of our attack. 
\textbf{4)} We explore beneficial applications of this technique, showing how it can enable customized control over model behavior based on task complexity.





\section{Related Work}
\label{sec:Related Work}
\subsection{Large vision language model}
Vision-language models\cite{li2023blip,li2024llava,bai2023qwen,lu2024deepseekvlrealworldvisionlanguageunderstanding, alayrac2022flamingo,sun2024generativemultimodalmodelsincontext}have achieved remarkable advancements within the realm of multimodal intelligence. By amalgamating large language models\cite{ray2023chatgpt,achiam2023gpt,anil2023palm,touvron2023llama2openfoundation,touvron2023llamaopenefficientfoundation} with visual content, LVLMs effectively manage intricate visual and linguistic inputs, thereby executing a variety of tasks ranging from visual description to logical reasoning. Flamingo\cite{alayrac2022flamingo} and OpenFlamingo\cite{awadalla2023openflamingoopensourceframeworktraining} models incorporate visual feature processing modules into the internal strata of language models using gated cross-attention, thereby propelling the profound integration of visual data within LLMs. CLIP\cite{radford2021learning,sun2023evaclipimprovedtrainingtechniques} utilizes contrastive learning to harmonize image and text modalities and is trained on extensive, noisy web-derived image-text pairs. By integrating modules such as QFormer\cite{li2023blip} and MLP\cite{liu2024visual}, previous works\cite{bai2023qwen, dai2023instructblipgeneralpurposevisionlanguagemodels,Liu_2024_CVPR} facilitate a collaborative comprehension between visual encoders and large language models (LLMs) of multimodal inputs. LLaVA\cite{liu2024visual} stands out for its pioneering use of GPT-generated instruction-following data to amplify LVLMs' responsiveness to visual instructions. A plethora of powerful LVLM APIs, including GPT-4o\cite{achiam2023gpt} and Qwen-VL-max\cite{bai2023qwen}, are now available. Through a rigorous evaluation of these models based on our proposed benchmark, we offer insightful perspectives into the ongoing research surrounding LVLMs.
\subsection{Vision Language Benchmarks} A rapidly expanding suite of multimodal benchmarks now rigorously evaluates the capabilities of LVLMs. Established benchmarks, including COCO Caption \cite{chen2015microsoftcococaptionsdata}, VQAv2 \cite{Goyal_2017_CVPR}, and GQA \cite{Hudson_2019_CVPR}, predominantly center on image description and question-answering tasks, employing metrics such as BLEU, CIDEr, and accuracy to gauge performance. Yet, as LVLMs advance, these traditional datasets have become insufficient for fully capturing the breadth of model capabilities. In response, researchers have developed more comprehensive evaluation frameworks that test a wider range of competencies, encompassing perceptual and cognitive skills \cite{fu2024mmecomprehensiveevaluationbenchmark}, spatial-temporal reasoning \cite{li2023seedbenchbenchmarkingmultimodalllms}, and relational understanding \cite{liu2025mmbench}. For instance, MMMU \cite{Yue_2024_CVPR} curates data from college-level textbooks and lecture materials, challenging models to demonstrate expertise across six academic disciplines. Similarly, CMMU \cite{he2024cmmubenchmarkchinesemultimodal} gathers questions from primary through high school curricula to assess foundational knowledge within the Chinese educational context. Nevertheless, these benchmarks largely remain focused on basic visual tasks, without adequately addressing the complexity of multimodal understanding. This paper introduces a benchmark tailored to evaluate deep semantic comprehension of images, specifically within a Chinese cultural framework.
\subsection{Image implicit meaning comprehension}
Image implicit meaning comprehension has become an important research focus for contemporary LVLMs, especially in handling images that convey complex emotions, cultural symbolism, and social critique. Existing evaluation datasets primarily test the models' linear visual reasoning abilities, such as visual question answering for surface-level content\cite{Hudson_2019_CVPR}. However, several works \cite{cai2019multi, machajdik2010affective} have demonstrated that LVLMs’ capabilities go beyond understanding surface-level meanings. Recent works\cite{yang2024largemultimodalmodelsuncover, liu2024iibenchimageimplicationunderstanding} highlight the limitations of current models when it comes to processing nonlinear narratives and understanding cultural contexts. For example, the most relevant prior work, DEEPEVAL\cite{yang2024largemultimodalmodelsuncover}, introduces three core tasks and shows that while the most advanced models achieve near-human performance on basic visual description tasks, they still perform poorly on tasks that involve understanding implicit semantics such as social background and satire. This paper provides a more comprehensive Chinese understanding benchmark, which, compared to the six categories in DeepEval, expands to include more thematic categories, with a total of 13 major categories and 41 subcategories (Figure \ref{fig:categories}), and offers more detailed testing across four dimensions of model performance.
% Image implicit meaning comprehension has emerged as a crucial research focus for contemporary LVLMs, particularly in handling images that convey nuanced emotions, cultural symbolism, and social critique. Achieving this level of comprehension demands that models infer implicit meanings from visual content, recognizing elements like satire, humor, and philosophical nuances. The most relevant prior work DEEPEVAL\cite{yang2024largemultimodalmodelsuncover} benchmark introduces three core tasks—fine-grained description selection. However, its limited categorization—comprising only six classes—restricts the scope of implicit meaning assessment, leaving out a broader range of complex visual semantics. 

% 2.1应该还没覆盖所有用到的模型;2.2需要补充点内容并且与2.3区分,2.3内容需要再调整
% 大型视觉语言模型(Flamingo, Blip2, Visual Instruction tuning,v Qwen-VL, LLaVA-next, DeeepSeekVL)近年来在多模态智能方面(Multimodal Intelligence)取得了显著进展。通过整合大规模语言模型(如GPTs*、LlaMa*、Palm2)和视觉内容(*), LVLMs可以处理复杂的视觉和语言输入,实现从视觉描述到逻辑推理等多种任务。Flamingo、OpenFlamingo模型通过gated cross在语言模型的内部层次中嵌入视觉特征处理模块,推动了视觉信息在LLMs中的深度整合。CLIP模型使用对比学习实现图像和文本模态的统一,并使用大规模noisy web 图像-文本对进行训练。14, 15 16,  17,通过添加QFormer和MLP等模块使视觉编码器和大型语言模型(LLMs)能够协同理解多模态输入。LLaVA则开创了通过GPT生成的instruction-following data提升LLvMs对视觉指令的响应能力。同时包括很多强大的LVLMs API公开,包括(GPT-4v*、Qwen-VL-max*) 。通过对上述模型进行全面评估\subsection{Vision Language Benchmarks} 
%为了系统地评估视觉语言模型的能力,近年来涌现了许多多模态评估基准。传统的评估基准如 COCO Caption*、VQAv2* 和 GQA* 等,主要集中在图像描述和问答任务,通过BLEU、CIDEr、准确率 等客观指标来衡量模型的性能。然而随着LVLMs的进步,这些数据集的难度已经不足以评估LVLMs的能力。研究者们进一步提出了更为全面的基准测试框架,从感知和认知能力(MME)、spatial and temporal understanding(SEED Bench),到Relation Reasoning能力(MMBench)。MMU从大学教材、讲义中收集数据,要求模型具备大学级别六大领域的专业知识。类似的,CMMU收集了小学至高中的七大学科题目,以评估模型对中文基础学科知识的理解与应用。然而,这些基准仅限于对基础视觉任务的评估,未能充分评估模型在复杂多模态任务中的表现,因此本文旨在提出一个中文背景下的评估模型深度图像含义的Benchmark。
%深层语义理解是当前LVLMs的一个重要研究方向,特别是在处理具有复杂情感、文化隐喻和社会批判的图像时尤为重要。深层语义的理解需要模型具备从视觉内容中推理出隐含意义的能力,例如理解讽刺、幽默和哲学内涵。DEEPEVAL* 提出了三种任务:细粒度描述选择、深入标题匹配和深层语义理解,通过这些任务系统性地评估了 LVLMs 在理解深层视觉语义上的表现。例如,尽管 GPT-4V* 在基础的视觉描述任务上达到了接近人类的水平,但在涉及社会背景和讽刺的语义理解任务中,仍存在显著差距。此外,

%图像隐含意义理解已成为当代大规模多模态语言模型(LVLMs)研究的一个重要方向,特别是在处理传达复杂情感、文化符号和社会批评的图像时。现有的评估数据集主要测试模型的线性视觉推理能力,例如对于浅层内容的视觉问答(VQA),。然而Machajdik的工作也证明了LVLM的能力不止于理解浅层含义。然而最近的工作(如 MVP、DeepEval 和 YESBUT Benchmark、Ii-Bench)揭示了现有模型在处理非线性叙事和文化背景理解时的局限性。例如最相关的前期工作 DEEPEVAL 引入了三个核心任务,发现当前最先进的模型在基础视觉描述任务上已接近人类水平,但在涉及社会背景和讽刺等隐含语义理解的任务中,仍表现不佳。本文提供了一个更为完备的中文理解Benchmark,相较于 DeepEval 的六大类任务,扩展了更多的主题类别,共包含13大类和41小类,并从四个维度对大模型的性能进行了更为详细的测试。



\section{Methodology}
\input{3_0_Preliminary}
\section{\methodname{}: Automatic Functionality Annotation Pipeline}
\label{sec: annotation pipeline}
This section introduces \methodname{}, an annotation pipeline (Fig.~\ref{fig: anno pipeline}) that automatically produces contextual element functionality annotations used to enhance VLMs' GUI grounding capabilities.


\begin{table}[t]
\tiny
\centering
\caption{\textbf{Comparing our \methodname{} dataset with existing large-scale UI datasets.} Multi-Res means the samples are collected on devices with various resolutions. Auto Anno. means the samples are collected autonomously. \#Anno. means the number of annotated samples provided by the datasets.}
\label{tab:data comparison}
\begin{tabular}{@{}cccccccc@{}}
\toprule
Dataset & UI Type & \begin{tabular}[c]{@{}c@{}}Multi\\ Res.\end{tabular} & \begin{tabular}[c]{@{}c@{}}Real-world\\ Scenario\end{tabular} & \begin{tabular}[c]{@{}c@{}}Auto\\ Anno. \end{tabular} & \begin{tabular}[c]{@{}c@{}}Contextual\\ Functionality\\ Semantics\end{tabular} & \#Anno. & Task \\ \midrule
WebShop~\citep{yao2022webshop} & Web & \cross & \cross & \cross & \cross & 12k & Web Navigation \\
Mind2Web~\citep{deng2024mind2web} & Web & \cross & \cmark & \cross & \cross & 2.4k & Web Navigation \\
WebArena~\citep{zhou2023webarena} & Web & \cross & \cmark & \cross & \cross & 812 & Web Navigation \\
\midrule
S2W~\citep{Wang2021Screen2WordsAM} & Mobile & \cross & \cmark & \cross & \cross & 112k & Screen Summarization \\
Wid. Cap.~\citep{Li2020WidgetCG} & Mobile & \cross & \cmark & \cross & \cross & 163k & Element Captioning \\
PixelHelp~\citep{Li2020MappingNL} & Mobile & \cross & \cmark & \cross & \cross & 187 & Element Grounding \\
RICOSCA~\citep{Li2020MappingNL} & Mobile & \cross & \cmark & \cross & \cross & 295k & Action Grounding \\
MoTIF~\citep{Burns2022ADF} & Mobile & \cross & \cmark & \cross & \cross & 6k & Mobile Navigation \\
AITW~\citep{rawles2023android} & Mobile & \cross & \cmark & \cross & \cross & 715k & Mobile Navigation \\
RefExp~\citep{Bai2021UIBertLG} & Mobile & \cross & \cmark & \cross & \cross & 20.8k & Element Grounding \\
VWB~\citep{liu2024visualwebbench} & Web & \cross & \cmark & \cross & \cross & 1.5k & Elem. Ground \& Ref. \\
SeeClick Web~\citep{cheng2024seeclick} & Web & \cross & \cmark & \cmark & \cross & 271k & Element Grounding \\
UI REC/REG~\citep{hong2023cogagent} & Web & \cmark & \cmark & \cmark & \cross & 400k & Box2DOM, DOM2Box \\
Ferret-UI~\citep{you2024ferretui} & Mobile & \cmark & \cmark & \cmark & \cross & 250k & Elem. Ground \& Ref. \\
\methodname{} (ours) & Web, Mobile & \cmark & \cmark & \cmark & \cmark & 704k & Functionality Ground \& Ref. \\ \bottomrule
\end{tabular}
\end{table}



\begin{figure}[t]
    \centering
    \includegraphics[width=0.95\linewidth]{figure/AnnoPipeline3.pdf}
    \caption{\textbf{The proposed pipeline for automatic UI functionality annotation.} An LLM is utilized to predict element functionality based on the UI content changes observed during the interaction. LLM-aided rejection and verification are introduced to improve data quality. Finally, the high-quality functionality annotations will be converted to instruction-following data by applying task templates.}
    \label{fig: anno pipeline}
\end{figure}


\subsection{Collecting UI Interaction Trajectories}
Our pipeline initiates by collecting interaction trajectories, which are sequences of UI contents captured by interacting with UI elements. Each trajectory step captures all interactable elements and the accessibility tree (AXTree) that briefly outlines the UI structure, which will be used to generate functionality annotations. To amass these trajectories, we utilize the latest Common Crawl repository as the data source for web UIs and Android Emulator for mobile UIs. Note that illegal websites and Apps are excluded manually from the sources to ensure no pornographic or violent content is included in our dataset. Please refer to Sec.~\ref{sec:supp:record traj detail} for collecting details and data license.

\subsection{Functionality Annotation Based on UI Dynamics}
Subsequently, the pipeline generates functionality annotations for elements in the collected trajectories. Interacting with an element $e$, by clicking or hovering over it, triggers content changes in the UI. In turn, these changes can be used to predict the functionality $f$ of the interacted element. For instance, if clicking an element causes new buttons to appear in a column, we can predict that the element likely functions as a dropdown menu activator (an example in Fig.~\ref{fig: funcpred diff case}).
With this observation, we utilize a capable LLM (i.e., Llama-3-70B~\citep{llama3modelcard}) as a surrogate for humans to summarize an element's functionality based on the UI content changes resulting from interaction. Concretely, we generate compact content differences for AXTrees before ($s_t$) and after ($s_{t+1}$) the interaction using a file-comparing library\footnote{https://docs.python.org/3/library/difflib.html}. Then, we prompt the LLM to thoroughly analyze the UI content changes (addition, deletion, and unchanged lines), present a detailed Chain-of-Thoughts~\citep{wei2022chain} reasoning process explaining how the element affects the UI, and finally summarize the element's functionality.

In cases where element interactions significantly transform the UI and cause lengthy differences—such as navigating to a new screen—we adjust our approach by using UI description changes instead of the AXTree differences. Specifically, we prompt the same LLM to discern the UI hierarchy, describe UI regions, and finally describe the entire UI functionality. After describing the UIs before and after the interaction, the LLM analyzes the description differences, presents reasoning, and summarizes the element's functionality. This annotation process is formulated as:
\begin{equation}
    f = \text{LLM}(p_{\text{anno}}, s_t, s_{t+1})
\end{equation}

where $f$ is the predicted functionality, $p_{\text{anno}}$ is the annotation prompt (Tab.~\ref{tab:supp:funcpred manip prompt} and Tab.~\ref{tab:supp:funcpred nav prompt}). Examples of annotated elements are depicted in Fig.~\ref{fig: our dataset} and more annotation details are explained in Sec.~\ref{sec:supp:anno details}.

\subsection{Removing Invalid Samples via LLM-Aided Rejection}
The collected trajectories may contain invalid samples due to broken UIs, such as incomplete UI loading. These samples are meaningless as they contain corrupted UI content and can mislead the models trained with them.

To filter out these invalid samples, we introduce an LLM-aided rejection approach. Initially, hand-written rules are used to detect obvious broken cases, such as blank UI contents, UIs containing elements indicating content loading, and interaction targets outside of UIs. While these obvious cases constitute a large portion of the invalid samples, there are a few types that are difficult to detect with hand-written rules. For instance, interacting with a “view more” button might unexpectedly redirect the user to a login page instead of the desired information page due to website login restrictions. To identify these challenging samples, we prompt the annotating LLM to also act as a rejector. Specifically, the LLM takes the UI content changes, generated using a file-comparing library, as input, provides detailed reasoning on whether the changes are meaningful for predicting the element's functionality, and finally outputs predictability scores ranging from 0 to 3. This process is formulated as follows:
\begin{equation}
 score = \text{LLM}(p_{\text{reject}}, e, s_t, s_{t+1})
\end{equation}
where $p_{\text{reject}}$ is the rejection prompt (Tab.~\ref{tab:supp:rejection prompt}).

This approach ensures that clear and predictable samples receive higher scores, while those that are ambiguous or unpredictable receive lower scores. For instance, if a button labeled "Show More", upon interaction, clearly adds new content, this sample will considered to provide sufficient changes that can anticipate the content expansion functionality and will get a score of 3. Conversely, if clicking on a "View Profile" link fails to display the profile possibly due to web browser issues, this unpredictable sample will get a score less than 3.

After implementing empirical experiments, we deploy this LLM-based rejector to discard the bottom 30\% of samples based on their scores to strike a balance between the elimination of invalid samples and the preservation of valid ones (More details in Sec.~\ref{suc:supp:reject details}). The samples that pass the hand-written rules and the LLM rejector are subsequently submitted for functionality annotation. Please see representative rejection examples in Fig.~\ref{fig: rejection examples}.

\subsection{Improving Annotation Quality via LLM-Based Verification}
The functionality annotations produced by the LLM probably contain incorrect, ambiguous, and hallucinated samples (See a case in Fig.~\ref{fig: anno pipeline}), which probably misleads the trained VLMs and compromises evaluation accuracy. To improve dataset quality, we prompt LLMs to verify the annotations by checking whether the targeted element $e$ fulfills the intent of the annotated functionality $f$. This process presents the LLMs with the interacted element, its UI context, the UI changes induced by this element, and the functionality generated in the previous annotation process. The LLMs are then tasked with analyzing the UI content changes before predicting whether the interacted element aligns with the given functionality. If the LLMs determine that the interacted element fulfills the functionality given its UI context, the LLMs will grant a full score (An example in Fig.~\ref{fig: verif diff case}). If the interacted element is considered to mismatch the functionality, this functionality can be seen as incorrect as this mismatch indicates that it may not accurately reflect the element's actual role within the UI context.

To mitigate the potential biases in LLMs~\citep{panickssery2024llm, zheng2023judging, bai2024benchmarking}, two different LLMs (i.e., Llama-3-70B~\citep{llama3modelcard} and Mistral-7B-Instruct-v0.2~\citep{mistral}) are employed as verifiers and prompted to output 0-3 scores. The scoring process is formulated as follows:
\begin{equation}
 score = \text{LLM}(p_{\text{verify}}, e, f, s_t, s_{t+1})
\end{equation}
where $p_{\text{verify}}$ denotes the verification prompt (Tab.~\ref{tab:supp:verif prompt}). Only if the two scores are both 3s do we consider the functionality label correct (More details in Sec.~\ref{suc:supp:verif details}). Although this filtering approach seems stringent, we can make up the number of annotations through scaling. 

\begin{figure}[t]
    \centering
    \includegraphics[width=0.9\linewidth]{figure/our_dataset_img.pdf}
    \caption{Element functionality annotations generated by the proposed AutoGUI pipeline for both web and mobile viewpoints.}
    \label{fig: our dataset}
    \vspace{-5mm}
\end{figure}

\subsection{Functionality Grounding and Referring Task Generation}
\vspace{-2mm}
After rejecting, annotating, and verifying, we obtain a high-quality UI functionality dataset containing triplets of \{UI screenshot, Interacted element, Functionality\}. To convert this dataset into an instruction-following dataset for training and evaluation, we generate functionality grounding and referring tasks using diverse prompt templates (see Tab.~\ref{tab:task templates}). To mitigate the difficulty of predicting absolute values for various resolutions, the coordinates of element bounding boxes are all normalized within the range $[0,99]$ (see Fig.~\ref{fig: our dataset} for examples).

\subsection{Explore the \methodname{} Dataset}

\begin{table}[]
\centering
\small
\caption{\textbf{The statistics of the AutoGUI datasets.} The Anno. Tokens and Avg. Words columns show the total number of tokens and the average number of words for the functionality annotations regardless of task templates. The Domains/Apps column shows the number of unique web domains/mobile Apps involved in each split.}
\label{tab:simple data stats}
\begin{tabular}{@{}ccccccc@{}}
\toprule
Split & \#Tasks & Anno. Tokens & Avg. Words & Domains/Apps & Device Ratio   \\                                                                   \midrule
Train & 702k  & 17.9M        & 23.1       & 916     & Web: $54.6\%$, Mobile: $45.4\%$                                              \\ \cmidrule(r){1-6}
Test  & 2k    & 53.4k        & 22.5       & 299     & Web: $50\%$, Mobile: $50\%$                                                                                                               \\ \bottomrule
\end{tabular}
\end{table}

\begin{figure}[t]
    \centering
    \includegraphics[width=1.0\linewidth]{figure/wordcloud_token-dist-comparison.pdf}
    \caption{\textbf{Diversity of the AutoGUI dataset.} \textbf{Left}: The word cloud illustrates the ratios of the verbs representing the main intents in the functionality annotations. \textbf{Right}: Comparing the distributions of the annotation token numbers for our AutoGUI training split, SeeClick Web training data~\citep{cheng2024seeclick}, and Widget Captioning~\citep{Li2020WidgetCG}. The comparison demonstrates that our dataset covers significantly more diverse task lengths.}
    \label{fig: wordcloud and tokdistrib}
\end{figure}
\vspace{-2mm}

The \methodname{} pipeline finally collects 22.4k trajectories, from which we select 2k grounding samples (evenly divided between web and smartphone views) as the test set and remove the trajectories to which these samples belong. Subsequently, 702k samples are randomly selected from the remaining instances to constitute the training set. The statistics of our dataset in Tab.~\ref{tab:simple data stats} and Sec.~\ref{sec:supp:data stats} show that our dataset covers diverse UIs and exhibits variety in lengths and functional semantics of the annotations. Moreover, our dataset presents a unique ensemble of research challenges for developing generalist web agents in real-world settings. As shown in Tab.~\ref{tab:data comparison} and Fig.~\ref{fig: functionality vs others}, our dataset distinguishes itself from existing literature by providing functionality-rich data as well as tasks that require VLMs to discern the contextual functionalities of elements to achieve high grounding accuracy.

\section{Analysis of Data Quality}
This section analyzes the reliability of the proposed annotation pipeline and data quality.

\noindent{\textbf{Comparison with Human Annotation}} To demonstrate the superiority of the proposed automatic annotation pipeline based on open-source LLMs, $N=145$ samples (99 valid and 46 invalid) are randomly selected as a testbed for comparing the annotation correctness of a trained human annotator and the pipeline. Here, correctness is defined as $Correctness = C / (N - R)$, where $C$ and $R$ denote the numbers of correctly annotated and rejected samples, respectively. The denominator subtracts the number of rejected samples as we are more interested in the percentage of correct samples after rejecting the samples considered invalid by the annotator. The authors thoroughly check the annotation results according to the three criteria in Fig.~\ref{fig: check criteria}: 1. Context-specificity. The functionality annotations must include context-specific descriptions to ensure one-to-one mapping between the element and its annotation. 2. Appropriate details. Avoid detailing unnecessary aspects of the UIs to keep the description focused on functionality. 3. No hallucination. The annotations must not include information not grounded in the visual context of the UIs. See more details in Sec.~\ref{sec:supp:humaneval details}.

After experimenting with three runs, Tab.~\ref{tab:ablate autogui} shows that the proposed AutoGUI pipeline achieves high correctness comparable to the trained human annotator (r6 vs. r1). Without rejection and verification (r2), AutoGUI is inferior as it cannot recognize invalid samples. Notably, simply using the rules written by the authors can improve the correctness, which is further enhanced with the LLM-aided rejector (r4 vs. r3). Moreover, utilizing the annotating LLM itself to self-verify its annotations helps AutoGUI surpass the trained annotator (r5 vs. r1). Introducing another LLM verifier (i.e., Mistral-7B-Instruct-v0.2) brings a slight increase which results from Mistral recognizing Llama-3-70B’s incorrect descriptions of how dropdown menu options work. Overall, these results justify the efficacy of the AutoGUI annotation pipeline.

Qualitatively comparing the annotation patterns of the human and AutoGUI (Fig.~\ref{fig: autogui vs human}), we find that AutoGUI employs the strong LLM to generate more detailed and clear annotations which would take significantly more time for the human annotator. This result suggests that the AutoGUI pipeline can lessen the burden of collecting data for training UI-VLMs.

\noindent{\textbf{Impact of LLM Output Uncertainty}} The uncertainty of LLM outputs manifests in annotation, rejection, and verification, possibly impacting the quality of the AutoGUI dataset. To evaluate this impact, we first sample 100 valid samples to test the AutoGUI pipeline for three runs. The consistency rate is 94.5\%, indicating that 94.5\% of the samples possess consistent annotation outcomes (i.e. correct or incorrect) across the runs. We also test the LLM-aided rejector with 46 invalid samples and find that the rejection consistency over three runs is 79.3\%. This indicates that LLM uncertainty impacts this rejection process. Nevertheless, this impact is minor due to the low prevalence of invalid samples (4\% of all samples) that fail the hand-written rules.

In summary, AutoGUI exhibits annotation correctness comparable to that of human annotators and LLM output uncertainty poses a minor impact on the AutoGUI annotation process.



\begin{figure}[t]
    \centering
    \includegraphics[width=0.85\linewidth]{figure/check_criteria_img.pdf}
    \caption{The checking criteria used for comparing AutoGUI pipeline and the human annotator.}
    \label{fig: check criteria}
\end{figure}


\begin{table}[]
\small
\centering
\caption{\textbf{Comparing the AutoGUI and human annotator.} AutoGUI with the proposed rejection and verification achieves annotation correctness comparable to trained human annotators. One LLM means Llama-3-70B and Two LLMs include Mistral-7B-Instruct-v0.2 as well.}
\label{tab:ablate autogui}
\begin{tabular}{@{}ccccc@{}}
\toprule
No. & Annotator  & Rejector   & Verifier              & Correctness \\ \midrule
r1 & Human      & -          & -                     & 95.5\%      \\
r2 & Llama-3-70B & -          & -                     & 64.5\%      \\
r3 & Llama-3-70B & Rules      & -                     & 83.1\%      \\
r4 & Llama-3-70B & Rules+LLM  & -                     & 94.4\%      \\
r5 & Llama-3-70B & Rules+LLM  & One LLM            & 96.0\%      \\
r6 & Llama-3-70B & Rules+LLM & Two LLMs & \textbf{96.7\%}      \\ \bottomrule
\end{tabular}
\end{table}
\vspace{-2mm}




\section{Experiment}
\label{sec:Experiment}

\subsection{Dataset}
Outside Knowledge Visual Question Answering (OK-VQA)~\cite{marino2019ok} is a benchmark dataset designed to evaluate VQA systems that require leveraging external knowledge sources beyond the information present in an image. The dataset consists of 14,055 knowledge-based questions paired with 14,031 images from the COCO dataset~\cite{lin2014microsoft}. These questions span 10 diverse knowledge categories, including domains such as Science and Technology, Geography, Cooking and Food, and Vehicles and Transportation. The questions were crowdsourced via Amazon Mechanical Turk, ensuring they require real-world knowledge to answer, making this dataset significantly more challenging than conventional VQA datasets. 

The dataset is split into 9,009 training samples and 5,046 testing samples, with each question associated with 10 ground-truth answers annotated by human annotators. This multi-answer format helps address ambiguity and variability in responses. Table~\ref{tab:okvqa_details} outlines key statistics and the distribution of questions across various knowledge categories in the Ok-VQA dataset. Baseline evaluations on OK-VQA using state-of-the-art models like MUTAN and Bilinear Attention Networks (BAN) reveal a significant drop in performance compared to traditional VQA datasets. This performance degradation underscores the need for models with enhanced retrieval and reasoning capabilities to incorporate unstructured, open-domain knowledge effectively.

\begin{table}[h!]
    \centering
    \footnotesize
    \setlength{\tabcolsep}{4pt}
    \renewcommand{\arraystretch}{1.2}
    \caption{Key Details of the OK-VQA Dataset}
    \begin{tabular}{|p{3.2cm}|p{4.8cm}|}
        \hline
        \textbf{Attribute}                & \textbf{Details} \\
        \hline
        \textbf{Name}                     & OK-VQA (Outside Knowledge VQA) \\
        \hline
        \textbf{Source}                   & COCO Image Dataset \\
        \hline
        \textbf{Number of Questions}      & 14,055 \\
        \hline
        \textbf{Number of Images}         & 14,031 \\
        \hline
        \textbf{Question Categories}      & 10 Categories \\
        \hline
        \textbf{Categories Breakdown}     & Vehicles \& Transportation (16\%) \newline Brands, Companies \& Products (3\%) \newline Objects, Materials \& Clothing (8\%) \newline Sports \& Recreation (12\%) \newline Cooking \& Food (15\%) \newline Geography, History, Language \& Culture (3\%) \newline People \& Everyday Life (9\%) \newline Plants \& Animals (17\%) \newline Science \& Technology (2\%) \newline Weather \& Climate (3\%) \newline Other (12\%) \\
        \hline
        \textbf{Average Question Length}  & 8.1 words \\
        \hline
        \textbf{Average Answer Length}    & 1.3 words \\
        \hline
        \textbf{Unique Questions}         & 12,591 \\
        \hline
        \textbf{Unique Answers}           & 14,454 \\
        \hline
        \textbf{Answer Annotations}       & 10 answers per question \\
        \hline
        \textbf{Answer Types}             & Open-ended \\
        \hline
        \textbf{Requires External Knowledge} & Yes (e.g., Wikipedia, Common Sense, etc.) \\
        \hline
        \textbf{Typical Knowledge Sources}& Unstructured Text (Wikipedia) \\
        \hline
    \end{tabular}
    \label{tab:okvqa_details}
\end{table}

\subsection{Implementation Details}
The experiments are conducted on Google Colab using a T4 GPU. The NVIDIA T4 GPU features 16 GB of GDDR6 memory, 320 Tensor Cores, and supports mixed-precision computation, making it suitable for deep learning tasks. Due to computational constraints, we evaluate our model on a subset of 100 samples from the OK-VQA dataset~\cite{marino2019ok}.

\subsection{OOD and ID Category Splits}
In our experiments, we evaluate our approach using the OK-VQA dataset~\cite{marino2019ok}, which we split into OOD and ID subsets based on knowledge categories. The OOD categories include Vehicles and Transportation, Brands, Companies and Products, Sports and Recreation, Science and Technology, and Weather and Climate. The ID categories comprise Objects, Materials and Clothing, Cooking and Food, Geography, History, Language and Culture, People and Everyday Life, Plants and Animals, and Other. Using this split, we can assess how well the model generalizes to different categories of knowledge.

\subsection{Patch-Based Image Preprocessing}
For VQA processing, we preprocess each input image by dividing it into patches of various sizes, specifically 2×2, 3×3 and 4x4 grids. This patch-based approach captures fine-grained visual details, which can enhance feature extraction for complex queries. We then employ the BLIP-VQA model~\cite{li2022blip} to extract image representations and generate initial contextual information based on the image and the associated question.

\subsection{Retrieval-Augmented Knowledge Integration}
To incorporate external knowledge, we use  RAG~\cite{lewis2020retrieval} with external knowledge sources such as Wikipedia and DBpedia. RAG retrieves relevant information based on the question and the visual features extracted by BLIP-VQA~\cite{li2022blip}. This retrieval process supplies the model with real-world context beyond the image, which is crucial for correctly answering questions that depend on external knowledge.

\subsection{State-of-the-Art Performance Comparison}
We evaluate our proposed FilterRAG framework on the OK-VQA dataset and compare it to state-of-the-art methods (Table~\ref{table:SOTA-OK-VQA}). The baseline models, Base1 and Base2, use the BLIP-VQA model with the VQA v2~\cite{goyal2017making} and OK-VQA datasets~\cite{marino2019ok}, achieving 83.0\% and 40.0\% accuracy, respectively. The drop highlights the challenge of knowledge-based questions in OK-VQA. Our FilterRAG framework, which integrates BLIP-VQA, RAG, and external knowledge sources like Wikipedia and DBpedia, achieves 36.5\% accuracy in OOD settings. This result demonstrates the effectiveness of grounding VQA responses with external knowledge, especially for OOD scenarios. 

Compared to state-of-the-art methods, KRISP~\cite{marino2021krisp}  achieves 38.35\% with Wikipedia and ConceptNet, while MAVEx~\cite{wu2022multi} reaches 41.37\% using Wikipedia, ConceptNet, and Google Images. The highest performance comes from KAT (ensemble)~\cite{gui2021kat} at 54.41\% with Wikipedia and Frozen GPT-3. Although these models achieve higher accuracy, they often require significant computational resources. 

FilterRAG balances performance and efficiency, making it suitable for resource-constrained environments. As shown in Figure~\ref{fig:plot1_accuracy}, it achieves 37.0\% accuracy in ID settings, 36.0\% in OOD settings, and 36.5\% when combining ID and OOD data. This highlights its robustness for knowledge-intensive VQA tasks.

\begin{figure}[h!]
    \centering
    \includegraphics[width=\linewidth]{figures/plot1_accuracy_v2.pdf}
    \caption{Comparison of Model Accuracy Across Different Settings.}
    \label{fig:plot1_accuracy}
\end{figure}

\begin{table*}[t]
    \centering
    \footnotesize
    \caption{Performance Comparison of State-of-the-Art Methods on the OK-VQA Dataset}
    \label{tab:okvqa_results}
    \renewcommand{\arraystretch}{1.2}
    \setlength{\tabcolsep}{10pt}
    \begin{tabular}{l l c}
        \toprule
        \textbf{Method}                                & \textbf{External Knowledge Sources}                          & \textbf{Accuracy (\%)} \\
        \midrule
        Q-only (Marino et al., 2019)~\cite{marino2019ok}                  & —                                                          & 14.93                  \\
        MLP (Marino et al., 2019)~\cite{marino2019ok}                     & —                                                          & 20.67                  \\
        BAN (Marino et al., 2019)~\cite{marino2019ok}              & —                                                          & 25.1                  \\
        MUTAN (Marino et al., 2019)~\cite{marino2019ok}               & —                                                          & 26.41                  \\
        ClipCap (Mokady et al., 2021)~\cite{mokady2021clipcap}                 & —                                                          & 22.8                   \\
        \midrule
        BAN + AN (Marino et al., 2019~\cite{marino2019ok}                  & Wikipedia                                                  & 25.61                  \\
        BAN + KG-AUG (Li et al., 2020)~\cite{li2020boosting}        & Wikipedia + ConceptNet                                     & 26.71                  \\
        Mucko (Zhu et al., 2020)~\cite{zhu2020mucko}                      & Dense Caption                                              & 29.2                   \\
        ConceptBERT (Gardères et al., 2020)~\cite{garderes2020conceptbert}           & ConceptNet                                                 & 33.66                  \\
        KRISP (Marino et al., 2021)~\cite{marino2021krisp}                   & Wikipedia + ConceptNet                                     & 38.35                  \\
        RVL (Shevchenko et al., 2021)~\cite{shevchenko2021reasoning}                 & Wikipedia + ConceptNet                                     & 39.0                   \\
        Vis-DPR (Luo et al., 2021)~\cite{luo2021weakly}                    & Google Search                                              & 39.2                   \\
        MAVEx (Wu et al., 2022)~\cite{wu2022multi}                       & Wikipedia + ConceptNet + Google Images                    & 41.37                  \\
        PICa-Full (Yang et al., 2022)~\cite{yang2022empirical}                 & Frozen GPT-3 (175B)                                        & 48.0                   \\
        KAT (Gui et al., 2022) (Ensemble)~\cite{gui2021kat}             & Wikipedia + Frozen GPT-3 (175B)                           & 54.41                  \\
        REVIVE (Lin et al., 2022) (Ensemble)~\cite{lin2022revive}          & Wikipedia + Frozen GPT-3 (175B)                           & 58.0                   \\
        RASO (Fu et al., 2023)~\cite{fu2023generate}                        & Wikipedia + Frozen Codex                                   & 58.5                   \\
        \midrule
        \textbf{FilterRAG (Ours)}                     & Wikipedia + DBpedia (\textbf{Frozen} BLIP-VQA and GPT-Neo 1.3B)    & \textbf{36.5}          \\
        \bottomrule
    \end{tabular}
    \label{table:SOTA-OK-VQA}
\end{table*}


\subsection{Hallucination Detection via Grounding Scores}
We evaluate the grounding scores of our FilterRAG framework against baseline models to assess its ability to mitigate hallucinations by aligning answers with external knowledge. As shown in Figure~\ref{fig:plot2_grounding_score}, Base1 achieves the highest grounding score of 94.60\% on the VQA v2 dataset~\cite{goyal2017making}, indicating that BLIP performs effectively when answering general-domain questions that do not require external knowledge. In contrast, Base2, evaluated on the OK-VQA dataset~\cite{marino2019ok}, shows a significant drop to 71.70\%, highlighting the challenge of answering knowledge-based questions without access to external information, thereby increasing the likelihood of hallucinations.

\begin{figure}[h!]
    \centering
    \includegraphics[width=\linewidth]{figures/plot2_grounding_score_v2.pdf}
    \caption{Grounding Score Comparison Across Baselines and Proposed Methods.}
    \label{fig:plot2_grounding_score}
\end{figure}

To address this limitation, our proposed method integrates BLIP-VQA, RAG, and external knowledge sources such as Wikipedia and DBpedia. The grounding scores for our method are 70.06\% for In-Distribution (ID) data, 70.68\% for Out-of-Distribution (OOD) data, and 70.37\% when combining both settings. These consistent scores demonstrate that FilterRAG effectively grounds answers in retrieved knowledge, reducing hallucinations even in challenging OOD scenarios.

Although our method does not achieve the grounding performance of Base1, it provides reliable results for knowledge-intensive tasks by leveraging external knowledge sources. This makes FilterRAG a robust and efficient solution for real-world VQA applications, particularly where external knowledge and OOD generalization are critical.

\subsection{Ablation Study}
We evaluate the effect of different image grid sizes on the performance of our FilterRAG framework with BLIP-VQA and RAG in OOD scenarios. We consider three grid configurations, 2x2, 3x3, and 4x4, and evaluate their influence on accuracy and grounding score. As shown in Figure~\ref{fig:plot5_measure_grid_size}, accuracy decreases slightly as the grid size increases. The accuracy is 37.00\% for the 2x2 grid, declines to 35.00\% for the 3x3 grid, and further drops to 34.00\% for the 4x4 grid. This downward trend indicates that larger grid sizes lead to excessive fragmentation, making it challenging for the model to extract coherent and meaningful visual features.

\begin{figure}[h!]
    \centering
    \includegraphics[width=\linewidth]{figures/plot5_measure_grid_size_v2.pdf}
    \caption{Effect of Grid Sizes on Accuracy and Grounding Score.}
    \label{fig:plot5_measure_grid_size}
\end{figure}

Similarly, the grounding score follows a declining trend with increasing grid size. The grounding score is 70.06\% for the 2x2 grid, reducing to 69.20\% for the 3x3 grid and 68.07\% for the 4x4 grid. This decline suggests that finer grid divisions hinder the model’s ability to align generated answers with retrieved external knowledge, likely due to the loss of contextual coherence when images are broken into smaller patches.

Overall, the 2x2 grid size achieves the best trade-off between accuracy and grounding score. It maintains both visual coherence and effective knowledge alignment, thereby reducing the risk of hallucinations. Consequently, for OOD scenarios in the FilterRAG framework, the 2x2 grid configuration is the most effective for ensuring robust and reliable performance.

\subsection{Qualitative Analysis}
We perform a qualitative analysis of FilterRAG on the OK-VQA dataset~\cite{marino2019ok}, evaluating its performance in both In-Domain (ID) and Out-of-Distribution (OOD) settings. As illustrated in Figure~\ref{fig:Qualitative_Analysis}, FilterRAG generates accurate answers in ID scenarios where the retrieved knowledge is relevant and aligns well with the visual context. In these cases, the model effectively combines visual cues and external knowledge, resulting in well-grounded responses. These errors are frequently caused by misalignment between the visual context and the retrieved information, reflecting the challenge of handling ambiguous or novel queries outside the training distribution.

In OOD settings, FilterRAG struggles when relevant knowledge of unfamiliar concepts cannot be effectively retrieved. This often leads to hallucinations, where the model produces plausible but incorrect answers that are not supported by the retrieved evidence. This analysis highlights the critical role of reliable knowledge retrieval and precise multimodal alignment in mitigating hallucinations. Improving the quality of knowledge retrieval and refining visual-textual alignment are essential steps toward making FilterRAG more reliable in OOD contexts. Future improvements in these areas can help ensure more accurate and context-aware responses in real-world VQA applications.



\section{Conclusion and Future Work}
This paper introduces a simple and effective graph augmentation strategy for cross-graph node classification with OOD structure. 
There are still several directions that are worth exploring in the future: 1) the edge-dropping weight can be considered the more comprehensive method that can measure the significance of each edge. 2) The proposed augmentation can be extended to test time training.

% \newpage
\bibliography{0_main}


\section{Related Work}

Below we summarize the most relevant literature from both the medical lens and the algorithm lens.

\textbf{RL on social networks.} We design and implement RL on dyads that are small social networks in this paper. Existing works on RL on social networks are mostly focused on maximizing social influence or opinion spreading \cite{wang2021reinforcement,he2021reinforcement,yang2024balanced} with large scale social networks in mind. These problems are usually formulated as a constrained Markov Decision Process (CMDP) \cite{yang2024balanced}, where the goal is to allocate incentive to maximize the social influence or opinion spreading. Our focuses are on the challenges in the multi-scale decision making and the design of the RL algorithms that incorporate domain knowledge about the social networks. These differences make our algorithm designs unique contributions to the literature.

\textbf{Dyadic structure in health care.} Social relationships between patients and carepartners are proven to be important in many critical health outcomes. Studies have shown that the patient-caregiver dyad functions as a unit, with the well-being and coping strategies of one member significantly impacting the other \cite{shin2018supporting,mcpherson2024dyadic}. The quality of this relationship can affect treatment outcomes such as medication adherence \cite{psihogios2021understanding,kostalova2022medication,gresch2017medication}, and chronic disease management \cite{visintini2023medication,li2024usability}.

\textbf{Multi-agent RL (MARL).} Our proposed approach falls into the range of the independent learners in the MARL literature \cite{oroojlooy2023review}. Previous literature on MARL in a collaborative game focuses on finding the (approximate) Nash equilibrium of the game through interacting with an unknown environment \cite{wang2022cooperative,jin2021v}. However, in our paper, we emphasize the advantage of MARL in terms of its strong interpretability and being able to make decisions in multiple time-scales.

\section{Algorithm Details}
\label{app:algo}

We provide the complete details of the proposed \texttt{MultiAgent+SurrogateRwd} algorithm as well as the baseline \texttt{SingleAgent} algorithm.

We first introduce the infinite horizon RLSVI (Randomized Least Squares Value Iteration) algorithm in Alg. \ref{alg:base} \cite{russo2018tutorial}. This algorithm is a model-free posterior sampling approach that samples a random value function from its posterior distribution, and the agent acts greedily with respect to the sampled value function. We use the infinite horizon variant of RLSVI, which perturbs the Bayesian regression parameters with a random noise $\omega'$ (line 4). We introduce temporal correlation between the current noise $w'$ and the previous noise $w$ to introduce persistence in exploration.

\begin{algorithm}[H]
    \caption{Infinite Horizon RLSVI (Inf-RLSVI)}
        \begin{algorithmic}[1]
            \STATE{Input:} discount factor $\gamma \in \mathbb{R}$, previous dataset $\mathcal{D} = (s_i, a_i, r_i)_{i = 1}^{n-1} \cup \{s_{n}\}$, previous perturbation $w \in \mathbb{R}^d$, feature mapping $\phi: \mathcal{S} \times \mathcal{A} \mapsto \mathbb{R}^d$, previous parameter $\theta \in \mathbb{R}^{d}$
            \STATE Generate regression matrix and vector
            $$
                X \leftarrow\left[\begin{array}{c}
                \phi\left(s_1, a_1\right) \\
                \vdots \\
                \phi\left(s_{n-1}, a_{n-1}\right)
                \end{array}\right] \quad y \leftarrow\left[\begin{array}{c}
                r_1+\gamma \max _{\alpha \in \mathcal{A}} \langle \phi(s_2, \alpha), \theta \rangle \\
                \vdots \\
                r_{n-1}+\gamma \max _{\alpha \in \mathcal{A}}\langle \phi(s_{n}, \alpha), \theta \rangle
                \end{array}\right]
            $$
            \STATE Estimate value function
            $$
                \bar{\theta} \leftarrow \frac{1}{\sigma^2}\left(\frac{1}{\sigma^2} X^{\top} X+\lambda I\right)^{-1} X^{\top} y \quad \mathbf{\Sigma} \leftarrow\left(\frac{1}{\sigma^2} X^{\top} X+\lambda I\right)^{-1}
            $$
            \STATE Sample $w' \sim \mathcal{N}(\gamma w, (1-\gamma^2) \mathbf{\Sigma})$ and set $\theta' = \bar \theta + w'$
            \STATE \textbf{Output:} $\theta'$ and $w'$
            % \State Choose action $A_t = \argmax_{\alpha} \langle \phi(s_{t}, \alpha), \theta_{t} \rangle$
        \end{algorithmic}
        \label{alg:base}
\end{algorithm}

We use the same hyperparameters $\lambda = 0.75$ and $\sigma = 0.5$ for all the algorithms, which achieves an overall good performance for all the algorithms.

\textbf{Additional notation.} We use $w, d, t$ to denote the week, day, and time of the decision. When we increment the time, we use $w, d, t+1$ to denote the next decisioin time right after $w, d, t$, and $w, d, t-1$ to denote the previous decision time right before $w, d, t$. Note that if $t = 1$, then $w, d, t-1$ is the evening decision time of the previous day.

\subsection{Single Agent Algorithm}

Our \texttt{SingleAgent} algorithm runs the RLSVI algorithm in Alg. \ref{alg:base} using the all the obervations available at time $w, d, t$ as the state variable:
$$
    S_{w, d, t} = 
    \left(Y_{w, d-1}^{\CARE}, Y_{w-1}^{\EDGE}, R_{w, d, t-1}^{\AYA}, \bar{Y}_{w-1}^{\AYA}, \bar{Y}_{w-1}^{\CARE}, B_{w, d, t}^{\AYA}, B_{w, d, t}^{\CARE}, A_{w, d}^{\CARE}, A_{w}^{\EDGE}\right) \in \mathbb{R}^{9}.
$$
Here we slightly abuse the notation by using $R_{w, d, t-1}^{\AYA}$ to represent the AYA adherence at half-day decision time prior to the current decision time $w, d, t$. This means that if $t = 1$, a morning decision time, then $R_{w, d, t-1}^{\AYA}$ is the AYA adherence at the previous night.

The \texttt{SingleAgent} algorithm has the three dimensional action space $\vec{a} = (a_1, a_2, a_3)^{\top} \in \{0, 1\}^{3}$, each entry corresponding to one of the three interventions. The second action $a_2$  will only be effective on a new day and the third action $a_3$ will only be effective on a new week. The feature mapping $\phi$ for the single agent algorithm is defined as
$$
    \phi(s, \vec{a}) = (1, s, a_1, a_2, a_3, s \cdot a_1, s \cdot a_2, s \cdot a_3) \in \mathbb{R}^{40}.
$$

\begin{algorithm}[hpt]
    \caption{\texttt{SingleAgent} Algorithm}
    \begin{algorithmic}[1]
        \STATE{Input:} discount factor $\gamma = 0.5$
        \STATE{Initialize:} $\theta_{1,1,1} = \mathbf{0} \in \mathbb{R}^{40}$; dataset $\mathcal{D}_{1,1,1} = \emptyset$
        \FOR{$w = 1, 2, \dots, 14$}
            \FOR{$d = 1, 2, \dots, 7$}
                \FOR{$t = 1, 2$}
                    \STATE{Call Algorithm \ref{alg:base} and update $\theta_{w,d,t}$}
                    \STATE{$\vec{a} = \argmax_{\alpha} \langle \phi(S_{w,d,t}, \alpha), \theta_{w,d,t} \rangle$}
                    \IF{$t = 1$ and $d = 1$ (New Week)}
                        \STATE{Set $A_{w}^{\EDGE} = \vec{a}_3$}
                    \ENDIF
                    \IF{$t = 1$ (New Day)}
                        \STATE{Set $A_{w,d}^{\CARE} = \vec{a}_2$}
                    \ENDIF
                    \STATE{Set $A_{w,d,t}^{\AYA} = \vec{a}_1$}
                    \STATE{Environment generates $R_{w,d,t}^{\AYA}$ and next state $S_{w,d,t+1}$}
                    \STATE{Update $\mathcal{D}_{w,d,t} = \mathcal{D}_{w,d,t-1} \cup \{(S_{w,d,t}, \vec{a}, R_{w,d,t}^{\AYA})\}$}
                \ENDFOR
            \ENDFOR
        \ENDFOR
    \end{algorithmic}
    \label{alg:single_agent}
\end{algorithm}

\subsection{MultiAgent Algorithm}

The \texttt{MultiAgent} algorithm runs an RLSVI agent for each of the three interventions. We use agent-specific feature mapping $\phi^{\AYA}, \phi^{\CARE}, \phi^{\EDGE}$ for the AYA, carepartner, and relationship agents, respectively. The state construction and the feature mapping for Q-value function are given by Table \ref{tab:state_feature}. The \texttt{MultiAgent} algorithm is described in Alg. \ref{alg:multi_agent}, where the carepartner and the relationship agents learns based on the naive rewards that are the sum of the AYA rewards over the day, and over the week, respectively (line 15 and line 18).

\begin{table}[hpt]
    \centering
    \caption{State and feature construction for the Q-value function by agent.}
    \label{tab:state_feature}
    \begin{tabular}{l|l}
        \toprule
        Agent & State or Feature Mapping \\
        \midrule
        AYA State & $S_{w, d, t}^{\AYA} = \left(R_{w, d, t-1}^{\AYA}, B_{w, d, t}^{\AYA}, Y_{w}^{\EDGE}, A_{w}^{\EDGE}\right) \in \mathbb{R}^{4}$ \\
        AYA Feature & $\phi^{\AYA}(s, a) = (1, s, a, s \cdot a) \in \mathbb{R}^{10}$ \\
        \midrule
        Carepartner State & $S_{w, d}^{\CARE} = \left(Y_{w, d-1}^{\CARE}, B_{w, d}^{\CARE}, Y_{w}^{\EDGE}, A_{w}^{\EDGE}\right) \in \mathbb{R}^{4}$ \\ 
        Carepartner Feature & $\phi^{\CARE}(s, a) = (1, s, a, s \cdot a) \in \mathbb{R}^{10}$ \\
        \midrule
        Relationship State & $S_{w}^{\EDGE} = \left(Y_{w-1}^{\EDGE}, B_{w, 1, 1}^{\AYA}, B_{w, 1}^{\CARE}, \bar{Y}_{w-1}^{\AYA}, \bar{Y}_{w-1}^{\CARE}\right) \in \mathbb{R}^{5}$ \\
        Relationship Feature & $\phi^{\EDGE}(s, a) = (1, s, a, s \cdot a) \in \mathbb{R}^{12}$ \\
        \bottomrule
    \end{tabular}
\end{table}


\begin{algorithm}[hpt]
    \caption{\texttt{MultiAgent} Algorithm}
    \begin{algorithmic}[1]
        \STATE{Input:} discount factor $\gamma^{\AYA} = 0.5$, $\gamma^{\CARE} = 0.5$, $\gamma^{\EDGE} = 0$
        \STATE{Initialize:} $\theta^{\AYA}_{1,1,1} = \mathbf{0} \in \mathbb{R}^{10}$; $\theta^{\CARE}_{1,1} = \mathbf{0} \in \mathbb{R}^{10}$; $\theta^{\EDGE}_{1} = \mathbf{0} \in \mathbb{R}^{12}$; dataset $\mathcal{D}_{1,1,1}^{\AYA} = \emptyset$; $\mathcal{D}_{1,1}^{\CARE} = \emptyset$; $\mathcal{D}_{1}^{\EDGE} = \emptyset$
        \FOR{$w = 1, 2, \dots, 14$}
            \STATE{Call Algorithm \ref{alg:base} using $\mathcal{D}_{w}^{\EDGE}, \gamma^{\EDGE}$, and update $\theta_{w}^{\EDGE}$}
            \STATE{Set $A_{w}^{\EDGE} = \argmax_{\alpha} \langle \phi^{\EDGE}(S_{w}^{\EDGE}, \alpha), \theta_{w}^{\EDGE} \rangle$}
            \FOR{$d = 1, 2, \dots, 7$}
                \STATE{Call Algorithm \ref{alg:base} using $\mathcal{D}_{w,d}^{\CARE}, \gamma^{\CARE}$, and update $\theta_{w,d}^{\CARE}$}
                \STATE{Set $A_{w,d}^{\CARE} = \argmax_{\alpha} \langle \phi^{\CARE}(S_{w,d}^{\CARE}, \alpha), \theta_{w,d}^{\CARE} \rangle$}
                \FOR{$t = 1, 2$}
                    \STATE{Call Algorithm \ref{alg:base} using $\mathcal{D}_{w,d,t}^{\AYA}, \gamma^{\AYA}$, and update $\theta_{w,d,t}^{\AYA}$}
                    \STATE{$A_{w,d,t}^{\AYA} = \argmax_{\alpha} \langle \phi^{\AYA}(S_{w,d,t}^{\AYA}, \alpha), \theta_{w,d,t}^{\AYA} \rangle$}
                    \STATE{Environment generates $R_{w,d,t}^{\AYA}$ and next state $S_{w,d,t+1}$}
                    \STATE{Update $\mathcal{D}_{w,d,t}^{\AYA} = \mathcal{D}_{w,d,t-1}^{\AYA} \cup \{(S_{w,d,t}^{\AYA}, A_{w,d,t}^{\AYA}, R_{w,d,t}^{\AYA})\}$}
                \ENDFOR
                \STATE{Compute care-partner reward $R_{w,d}^{\CARE} = \sum_{t = 1}^{2} R_{w,d,t}^{\AYA} / 2$} 
                \STATE{Update $\mathcal{D}_{w,d}^{\CARE} = \mathcal{D}_{w,d-1}^{\CARE} \cup \{(S_{w,d}^{\CARE}, A_{w,d}^{\CARE}, R_{w,d}^{\CARE})\}$}
            \ENDFOR
            \STATE{Compute relationship reward $R_{w}^{\EDGE} = \sum_{d = 1}^{7} R_{w,d}^{\CARE} / 7$}
            \STATE{Update $\mathcal{D}_{w}^{\EDGE} = \mathcal{D}_{w-1}^{\EDGE} \cup \{(S_{w}^{\EDGE}, A_{w}^{\EDGE}, R_{w}^{\EDGE})\}$}
        \ENDFOR
    \end{algorithmic}
    \label{alg:multi_agent}
\end{algorithm}

The \texttt{MultiAgent+SurrogateRwd} algorithm is described in Alg. \ref{alg:multi_agent_surrogate}. The only difference between the \texttt{MultiAgent} and \texttt{MultiAgent+SurrogateRwd} is that the later agent optimizes the surrogate reward functions, defined in Equ. (\ref{equ:game_rwd_app}) and Equ. (\ref{equ:care_rwd_app}), where the coefficients are estimated using Bayesian Ridge Regression, with the prior mean given in Table \ref{tab:prior}.

\begin{align}
    r_w^{\EDGE} = & (1, Y_{w-1}^{\EDGE}, {B}_{w, 1, 1}^{\AYA}, A_{w}^{\EDGE}, A_w^{\EDGE} \cdot Y_{w-1}^{\EDGE})\vbeta^{\EDGE} \nonumber \\
     &+ \max_{a \in \{0, 1\}}(1, Y_{w}^{\EDGE}, {B}_{w+1, 1, 1}^{\AYA}, a, a \cdot Y_{w}^{\EDGE}) \vbeta^{\EDGE}, \label{equ:game_rwd_app}
\end{align}

\begin{align}
        r_{w, d}^{\CARE} =  (1, Y_{w,d}^{\CARE}, B_{w,d+1}^{\CARE}, Y_{w-1}^{\EDGE}, A_{w,d}^{\CARE}) \vbeta^{\CARE}, \label{equ:care_rwd_app}
\end{align}

\begin{algorithm}[hpt]
    \caption{\texttt{MultiAgent+SurrogateRwd} Algorithm}
    \begin{algorithmic}[1]
        \STATE{Input:} discount factor $\gamma^{\AYA} = 0.5$, $\gamma^{\CARE} = 0.5$, $\gamma^{\EDGE} = 0$
        \STATE{Initialize:} $\theta^{\AYA}_{1,1,1} = \mathbf{0} \in \mathbb{R}^{10}$; $\theta^{\CARE}_{1,1} = \mathbf{0} \in \mathbb{R}^{10}$; $\theta^{\EDGE}_{1} = \mathbf{0} \in \mathbb{R}^{12}$; dataset $\mathcal{D}_{1,1,1}^{\AYA} = \emptyset$; $\mathcal{D}_{1,1}^{\CARE} = \emptyset$; $\mathcal{D}_{1}^{\EDGE} = \emptyset$
        \FOR{$w = 1, 2, \dots, 14$}
            \STATE{Call Algorithm \ref{alg:base} using $\mathcal{D}_{w}^{\EDGE}, \gamma^{\EDGE}$, and update $\theta_{w}^{\EDGE}$}
            \STATE{Set $A_{w}^{\EDGE} = \argmax_{\alpha} \langle \phi^{\EDGE}(S_{w}^{\EDGE}, \alpha), \theta_{w}^{\EDGE} \rangle$}
            \FOR{$d = 1, 2, \dots, 7$}
                \STATE{Call Algorithm \ref{alg:base} using $\mathcal{D}_{w,d}^{\CARE}, \gamma^{\CARE}$, and update $\theta_{w,d}^{\CARE}$}
                \STATE{Set $A_{w,d}^{\CARE} = \argmax_{\alpha} \langle \phi^{\CARE}(S_{w,d}^{\CARE}, \alpha), \theta_{w,d}^{\CARE} \rangle$}
                \FOR{$t = 1, 2$}
                    \STATE{Call Algorithm \ref{alg:base} using $\mathcal{D}_{w,d,t}^{\AYA}, \gamma^{\AYA}$, and update $\theta_{w,d,t}^{\AYA}$}
                    \STATE{$A_{w,d,t}^{\AYA} = \argmax_{\alpha} \langle \phi^{\AYA}(S_{w,d,t}^{\AYA}, \alpha), \theta_{w,d,t}^{\AYA} \rangle$}
                    \STATE{Environment generates $R_{w,d,t}^{\AYA}$ and next state $S_{w,d,t+1}$}
                    \STATE{Update $\mathcal{D}_{w,d,t}^{\AYA} = \mathcal{D}_{w,d,t-1}^{\AYA} \cup \{(S_{w,d,t}^{\AYA}, A_{w,d,t}^{\AYA}, R_{w,d,t}^{\AYA})\}$}
                \ENDFOR
                \STATE{Compute care-partner reward $\tilde{R}_{w,d}^{\CARE}$ based on Equ. (\ref{equ:care_rwd_app})} 
                \STATE{Update $\mathcal{D}_{w,d}^{\CARE} = \mathcal{D}_{w,d-1}^{\CARE} \cup \{(S_{w,d}^{\CARE}, A_{w,d}^{\CARE}, \tilde{R}_{w,d}^{\CARE})\}$}
            \ENDFOR
            \STATE{Compute relationship reward $\tilde{R}_{w}^{\EDGE}$ based on Equ. (\ref{equ:game_rwd_app})}
            \STATE{Update $\mathcal{D}_{w}^{\EDGE} = \mathcal{D}_{w-1}^{\EDGE} \cup \{(S_{w}^{\EDGE}, A_{w}^{\EDGE}, \tilde{R}_{w}^{\EDGE})\}$}
        \ENDFOR
    \end{algorithmic}
    \label{alg:multi_agent_surrogate}
\end{algorithm}


\begin{table}[hpt]
    \centering
    \caption{Prior mean for coefficients in the surrogate reward functions.}
    \label{tab:prior}
    \begin{tabular}{c|c|c|c|c|c}
    \toprule
    Agent & Intercept & $Y_w^{\EDGE}$ & $B_w^{\AYA}$  & $A_w^{\EDGE}$ & $A_w^{\EDGE} \cdot Y_w^{\EDGE}$ \\
    \midrule
    $\vbeta^{\EDGE}$ & $1$ & $1$ & $-1$ & $-1$ & $0.5$ \\
    \midrule
    \midrule
    Agent & Intercept & $Y_{w,d}^{\CARE}$ & $B_{w,d}^{\CARE}$ & $Y_{w-1}^{\EDGE}$ & $A_{w,d}^{\CARE}$  \\
    \midrule
    $\vbeta^{\CARE}$ & $1$ & $-1$ & $-1$ & $1$ & $-0.5$  \\
    \bottomrule
    \end{tabular}
    \end{table}
    
\section{The Dyadic Environment}
\label{app:testbed}

\subsection{Overview of the Simulated Dyadic Environment}

% \ziping{Give a brief description of the testbed.}

We construct a dyadic simulation environment to evaluate the performance of the proposed algorithm. The 1st order goal of the environment design is to replicate the noise level and structure that we expect to encounter in the forthcoming ADAPTS-HCT clinical trial. This noise often encompasses the stochasticity in the state transition of each participant and the heterogeneity across participants.

The environment is based on Roadmap 2.0, a care partner-facing mobile health application that provides daily positive psychology interventions to the care partner only. Roadmap 2.0 involves 171 dyads, each consisting of a patient undergone HCT (target person) and a care partner. Each participant in the dyad had the Roadmap mobile app on their smartphone and wore a Fitbit wrist tracker. The Fitbit wrist tracker recorded physical activity, heart rate, and sleep patterns. Furthermore, each participant was asked to self-report their mood via the Roadmap app every evening. A list of variables in Roadmap 2.0 is reported in Table \ref{tab:roadmap_variable}.

Roadmap 2.0 data is suitable for constructing a dyadic environment for developing the RL algorithm for ADAPTS-HCT in that Roadmap 2.0 has the same dyadic structure about the participants--post-HCT cancer patients and their care partner. Moreover, Roadmap 2.0 encompasses some context variables that align with those to be collected in ADAPTS-HCT, for example, the daily self-reported mood score.

\subsubsection{Overcoming impoverishment.} From the viewpoint of evaluating dyadic RL algorithms, this data is impoverished \cite{trella2022designing} mainly in two aspects. First, Roadmap 2.0 does not include micro-randomized daily or weekly intervention actions (i.e., whether to send a positive psychology message to the patient/care partner and whether to engage the dyad into a weekly game). Second, it does not include observations on the adherence to the medication--the primary reward signal, as well as other important measurements such as the strength of relationship quality. 
% Furthermore, Roadmap 2.0 includes dyads across all lifespan whereas ADAPTS-HCT will focus on adolescent and young adults.

To overcome this impoverishment, we construct surrogate variables from the Roadmap 2.0 data to represent the variables intended to be collected in ADAPTS-HCT. A list of substitutes is reported in Table \ref{tab:roadmap_substitutes}. Worthnoting, the substitute for the AYA medication adherence is based on the step count. There is evidence on the association between the step count and the adherence. 

% \ziping{Could we find literature to justify this?}
% \ziping{Should we discuss in detail the rationale of these substitutes?} 
We further impute the treatment effects of the intervention actions so the marginal effects after normalization, which we call the standardized treatment effects (STE), are around 0.15, 0.3, and 0.5, corresponding to small, medium, and large effect sizes in typical behavioral science studies.

\subsubsection{Constructing the dyadic environment.} We follow the environment design in \cite{li2023dyadic}, which also uses the Roadmap 2.0 data, but primarily focuses on AYA intervention and relationship intervention. We extend the environment to include the care partner intervention. Specifically, we fit a separate multi-variate linear model for each participant in the dataset with the AR(1) working correlation using the generalized estimating equation (GEE) approach \cite{ziegler2010generalized,hojsgaard2006r}. We impute the treatment effects of the intervention actions based on the typical STE around 0.15, 0.3, and 0.5, which completes a generative model for the state transitions. The environment simulates a trial by randomly sampling dyads from the dataset, and simulate their trajectories based on the actions selected by the RL algorithm. The environment details are described in Appendix \ref{app:testbed}. Our experiments primarily focus on the three vanilla testbeds corresponding to the three STEs.



%In this section, we describe our set up of the simulation testbed. 


\subsection{Using the Roadmap 2.0 Dataset}


This section outlines our approach to addressing the limitations of the Roadmap 2.0 dataset, specifically its absence of micro-randomized interventions and reward signals.

To circumvent the lack of interventions, we impute treatment effects that represent the burden of the digital interventions, assuming that frequent notifications diminish both weekly and the daily treatment effects. Based on prior literature, we choose the scale of the treatment effect to be smaller than the baseline effect of features \cite{box1987empirical}. 

To address the missing reward signals, we use directly measurable variables in Roadmap 2.0 dataset as proxies to the outcomes we will observe in the real clinical trial. We approximate AYA adherence, $R_{w,d,t}^{\AYA}$, using the 12-hourly step count from Roadmap 2.0. Previous work has found the two values to be strongly correlated \hinal{TODO cite}. Since adherence is a binary signal in the ADAPTS-HCT trial, we discretize step count into a binary variable. Furthermore, we approximate the carepartner's daily psychological distress, $Y_d^{\CARE}$, using the daily length of their sleep. Finally, the weekly relationship between the AYA and their carepartner is estimated using the self-reported mood as a surrogate. Specifically, we let  $Y_w^{\EDGE} = \mathbbm{1}\{\sum_{d = 1}^{7} \Mood_{w,d}^{\AYA} \geq \Mood^{\AYA}\} \mathbbm{1}\{\sum_{d = 1}^{7}\Mood_{w,d}^{\CARE} \geq \Mood^{\CARE}\}$. Here, $\Mood_{w,d}^{\AYA}$ is the daily self-reported mood on week $w$ and day $d$, and $\Mood^{\AYA}$ is the $q$-th quantile of the the weekly summed mood across all AYA observations. We choose the quantile level $q$ such that approximately 50\% of the dataset satisfies $Y_{w}^{\EDGE} = 1$.

Table \ref{tab:roadmap_substitutes} summarizes the main variables and their replacements from the Roadmap 2.0 dataset.  

\begin{table}[hpt]
    \centering
    \caption{Substitutes of the main variables from Roadmap 2.0 dataset.}
\resizebox{\textwidth}{!}{%
    \begin{tabular}{c|c}
        \hline
        Variables & Substitutes \\
        \hline
        \hline
        AYA adherence  & Binary step count $\mathbbm{1}\{\texttt{Step}_{w, d,t}^{\AYA} \geq \texttt{Step}^{\AYA}\}$  
        \\
        Carepartner distress & Carepartner daily length of sleep $\texttt{Sleep}_{w, d}^{\CARE}$ \\
        Weekly relationship quality & Mood indicator: $\mathbbm{1}_{\{\sum_{d} \texttt{Mood}_{w, d}^{\CARE} \geq \texttt{Mood}^{\CARE}\}} \mathbbm{1}_{\{\sum_{d} \texttt{Mood}_{w, d}^{\AYA} \geq \texttt{Mood}^{\AYA}\}}$ \\
        Effects of interventions $A_{w,d,t}^{\AYA}, A_{w,d}^{\CARE}$, $A_{w}^{\EDGE}$ & Imputed based on domain knowledge \\
        Effects of digital interventions burden $B_{w,d,t}^{\AYA}$, $B_{w,d}^{\CARE}$ & Imputed based on domain knowledge\\
        \hline
    \end{tabular}%
    }    \label{tab:roadmap_substitutes}
\end{table}


\begin{table}[hpt]
    \centering
    \caption{List of variables in Roadmap 2.0 and the measuring frequencies.}
    \begin{tabular}{c}
    \hline
    Variables\footnote{Note that all the variables are measured the same for the target person and carepartner.} \\
    \hline
    \hline
     $\texttt{Step}_{w, d, t}$: twice-daily cumulative step count\\
     $\texttt{Heart}_{w, d, t}$: twice-daily average heart rate\\
     $\texttt{Sleep}_{w, d}$: daily length of sleep\\
     $\texttt{Mood}_{w, d}$: daily self-report mood measurement\\
     \hline
    \end{tabular}
    \label{tab:roadmap_variable}
\end{table}

\subsection{Environment Model Design}

We now describe how these surrogate variables are used to build the full environment model. Our approach involves fitting two state transition models for digital intervention burden (AYA and carepartner) and three models for rewards (AYA adherence, carepartner stress, and relationship quality).

For all transition models, we fit the baseline parameters -- which represent system dynamics under no intervention --  for each dyad using its respective dataset and a generalized estimating equation \cite{hojsgaard2006r} approach. We impute the remaining parameters  using domain knowledge. Further detail on the choice of the coefficients is in Appendix \ref{sec:select_sim_params}. 

\textbf{Transition models for the AYA component: } The digital intervention burden transition for AYA follows a linear model with covariates $(B_{w,d,t}^{\AYA}, A_{w,d,t}^{\AYA}, A_{w}^{\EDGE})$.
\begin{align}
\label{equ:B_transition_AYA}
    B_{w,d,t+1}^{\AYA} \sim \theta^{\AYA}_{0} + \theta_{1}^{\AYA} B_{w,d,t}^{\AYA} + \theta_{2}^{\AYA} A_{w,d,t}^{\AYA} + \theta_{3}^{\AYA} A_{w}^{\EDGE} + \eta_{w,d,t}^{\AYA}, \nonumber\\
    \text{ where $\eta_{w,d,t}^{\AYA} \sim \mathcal{N}(0, (\omega^{\AYA})^2)$.} 
\end{align}
 

For the primary outcome, AYA adherence, we fit a generalized linear model with a sigmoid link function: 


\begin{align}
R_{w,d,t}^{\AYA} &\sim \text{Bernoulli}(\text{sigmoid}(P_{w,d,t}^{\AYA})), \nonumber \\
P_{w,d,t}^{\AYA} &= (1-M_t)\big(\beta_{0, \AM}^{\AYA} + \beta^{\AYA}_{1, \AM} R_{w,d,t-1}^{\AYA} 
+ \beta_{2,\AM}^{\AYA} Y_{w-1}^{\EDGE} 
+ \beta_{3,\AM}^{\AYA} Y_{w,d-1}^{\CARE} + \beta_{4, \AM}^{\AYA} B_{w,d,t}^{\AYA} \nonumber \\
&\quad + \tau_{0, \AM}^{\AYA} A_{w,d,t}^{\AYA} 
+ \tau_{1, \AM}^{\AYA} A_{w,d,t}^{\AYA} Y_{w-1}^{\EDGE} 
+ \tau_{2, \AM}^{\AYA} A_{w,d,t}^{\AYA} B_{w,d,t}^{\AYA}\big) \nonumber \\
&\quad + M_t\big(\beta_{0, \PM}^{\AYA} + \beta^{\AYA}_{1, \PM} R_{w,d,t-1}^{\AYA} 
+ \beta_{2,\PM}^{\AYA} Y_{w-1}^{\EDGE} 
+ \beta_{3,\PM}^{\AYA} Y_{w,d-1,t}^{\CARE} + \beta_{4, \PM}^{\AYA} B_{w,d,t}^{\AYA} \nonumber \\
&\quad + \tau_{0, \PM}^{\AYA} A_{w,d,t}^{\AYA}  
+ \tau_{1, \PM}^{\AYA} A_{w,d,t}^{\AYA} Y_{w-1}^{\EDGE} 
+ \tau_{2, \PM}^{\AYA} A_{w,d,t}^{\AYA} B_{w,d,t}^{\AYA}\big)
\label{equ:R_Transition_AYA}
\end{align}

where $M_t$ is a decision window indicator defined as:

$$
    M_t = \left\{
    \begin{array}{clll}
         0 & \text{ if } t = 2k - 1 & (\text{AM decision window}) & \text{ for } k = 1, 2, \dots  \\
         1 & \text{ if } t = 2k & (\text{PM decision window}) &\text{ for } k = 1, 2, \dots
    \end{array},\right.
$$ 
Note that we exclude any effect of relationship interventions on AYA adherence as the game is designed without reinforcements and, thus, is not supposed to directly improve adherence.



\textbf{Transition models for the carepartner component: } The digital intervention burden transition for the carepartner is a linear model:

\begin{align}
\label{equ:B_transition_care}
    B_{w,d+1}^{\CARE} = \theta^{\CARE}_{0} + \theta_{1}^{\CARE} B_{w,d}^{\CARE} + \theta_{2}^{\CARE} A_{w,d}^{\CARE} + \theta_{3}^{\CARE} A_{w}^{\EDGE} + \eta_{w,d}^{\CARE}, \nonumber\\
    \text{ where $\eta_{w,d}^{\CARE} \sim \mathcal{N}(0, (\omega^{\CARE})^2)$.}
\end{align}

For the carepartner's psychological distress level, $R^{\CARE}_d$, we fit another linear model:
\begin{align}
    Y_{w,d}^{\CARE} = 
    &\beta_{0}^{\CARE} + \beta_{1}^{\CARE} Y_{w,d-1}^{\CARE} + \beta_{2}^{\CARE} R_{w,d,t-1}^{\AYA}  + 
    \beta_{3}^{\CARE} Y_{w-1}^{\EDGE} + \beta_{4}^{\CARE} B_{w,d}^{\CARE} + \nonumber \\
    &\quad \tau_{0}^{\CARE} A_{w,d}^{\CARE} +  
    \tau_{1}^{\CARE} A_{w,d}^{\CARE} Y_{w-1}^{\EDGE} +   
    \tau_{2}^{\CARE} A_{w,d}^{\CARE} B_{w,d}^{\CARE}  + \epsilon_{w,d}^{\CARE} \label{equ:R_transition_care}
\end{align}
where $\epsilon_{w,d}^{\CARE} \sim \mathcal{N}(0, (\sigma^{\CARE})^2)$.  Similar to (\ref{equ:R_Transition_AYA}), we do not include relationship intervention $A_{w-1}^{\EDGE}$.

\textbf{Transition model for the weekly relationship: } For the shared component, we only fit a transition model for the reward, which is the weekly relationship quality. Specifically, we fit a generalized linear model with a sigmoid link function: 

\begin{align}
Y_{w+1}^{\EDGE} \sim \text{Bernoulli}(\text{sigmoid}\left( \beta_{0}^{\EDGE} + \beta_{1}^{\EDGE}Y_{w}^{\EDGE}  + \beta_{2}^{\EDGE} \bar{R}_{w}^{\AYA} + \beta_{3}^{\EDGE} \bar{R}_{w}^{\CARE} \right. \nonumber \\
\left. + \tau_0^{\EDGE} A_{w}^{\EDGE} + \tau_1^{\EDGE} A_{w}^{\EDGE} (B_{w,d}^{\CARE} + B_{w,d,t}^{\AYA}))\right)
\label{equ:R_transition_rel}
\end{align}

where $\bar{R}_{w}^{\AYA} = \sum_{d=1}^{7} \sum_{t=1}^{2} \gamma^{14 - (7(w-1) + d) + 2(t-1)} R_{w,d,t}^{\AYA}$ is the exponentially weighted average of adherence within week $w$, and $\bar{R}_{w}^{\CARE} = \sum_{d=1}^{7} \gamma^{7-d} Y_{w,d}^{\CARE}$ is the exponentially weighted average of carepartner distress within week $w$. 

\subsection{Selecting Environment Model Parameters}
\label{sec:select_sim_params}

We list all the parameters that must be either imputed based on domain knowledge or estimated from the existing dataset. 
%We must impute all parameters related to the treatment effects because the existing dataset contains no target intervention. For the same reason, we impute all the parameters relating to the digital intervention burden $B_{w,d,t}^{\AYA}$, $B_{w,d}^{\CARE}$.

\begin{enumerate}
\item The baseline transition parameters $\beta$'s can be estimated directly from the dataset:
    \begin{enumerate}
        \item AYA state transition: $\vbeta^{\AYA}_{\AM} = (\beta_{0, \AM}^{\AYA}, \beta_{1, \AM}^{\AYA}, \beta_{2, \AM}^{\AYA}, \beta_{3, \AM}^{\AYA}, \beta_{4, \AM}^{\AYA})$ and $\vbeta^{\AYA}_{\PM} = (\beta_{0, \PM}^{\AYA}, \beta_{1, \PM}^{\AYA}, \beta_{2, \PM}^{\AYA}, \beta_{3, \PM}^{\AYA}, \beta_{4, \PM}^{\AYA})$.
        \item Carepartner state transition: $\vbeta^{\CARE} = (\beta_{0}^{\CARE}, \beta_{1}^{\CARE})$.
        \item Relationship transition: $\vbeta^{\EDGE} = (\beta_{0}^{\EDGE}, \beta_{1}^{\EDGE}, \beta_{2}^{\EDGE}, \beta_{3}^{\EDGE})$.
        % \item Hazard model parameters: $\vgamma = (\gamma_{0, 1}, \dots, \gamma_{0, 98}, \gamma_1, \gamma_2, \gamma_3)$
    \end{enumerate}
\item Imputed or tuned based on domain knowledge:
    \begin{enumerate}
        \item Burden transitions: coefficients $\boldsymbol{\theta}^{\AYA} = (\theta^{\AYA}_{0}, \theta^{\AYA}_{1}, \theta^{\AYA}_{2}, \theta^{\AYA}_{3})$, $\boldsymbol{\theta}^{\CARE} = (\theta^{\CARE}_{0}, \theta^{\CARE}_{1}, \theta^{\CARE}_{2}, \theta^{\CARE}_{3})$; burden noise variance $\omega^{\AYA}$ and $\omega^{\CARE}$.
        \item Main effects of burden: $\beta_{4, \AM}^{\AYA}, \beta_{4, \PM}^{\AYA}$, and $\beta_{4}^{\CARE}$.
        \item AYA treatment effects: $\{\tau_{i, \AM}^{\AYA}\}_{i = 0}^{2}$, $\{\tau_{i, \PM}^{\AYA}\}_{i = 0}^{2}$ and $\{\sigma_{i, \AM}^{\AYA}\}_{0 = 1}^{2}$, $\{\sigma_{i, \PM}^{\AYA}\}_{i = 0}^{2}$.
        \item Carepartner treatment effects: $\{\tau_{i}^{\CARE}\}_{i = 0}^{2}$ and $\{\sigma_{i}^{\CARE}\}_{i = 0}^{2}$.
        \item Relationship treatment effects: $\tau^{\EDGE}$ and $\sigma^{\EDGE}$.
        % \item Disengagement effect: $\xi_{0, d}, \xi_{1, d}, \xi_{2, d}$.
    \end{enumerate}
\end{enumerate}

\textbf{Fitting parameters (1a-d):} We estimate the baseline transition parameters under no intervention directly from the Roadmap 2.0 dataset. For the parameters in Equation (\ref{equ:R_Transition_AYA}), we have the correspondences 
$\beta_{i, \AM}^{\AYA} = \hat{\beta}_{i, \AM}^{\AYA}$ and $\beta_{i, \PM}^{\AYA} = \hat{\beta}_{i, \PM}^{\AYA}$ for $i = 0, 1, \dots, 3$, where $\hat{\beta}_{i, \AM}^{\AYA}$ and $\hat{\beta}_{i, \PM}^{\AYA}$ are fitted coefficients obtained using the generalized estimating equation (GEE) approach. Since we assume that app burden only moderates the effects of AYA interventions without directly influencing adherence, we set $\beta_{4, \AM}^{\AYA} = \beta_{4, \PM}^{\AYA} = 0$. Similarly, for parameters in Equation (\ref{equ:R_transition_care}), the correspondence is $\beta_{i}^{\CARE} = \hat{\beta}_{i}^{\CARE}$ for $i = 0, \dots, 3$, and we set $\beta_{4}^{\CARE} = 0$ under the same assumption for carepartner distress. For the relationship quality model in Equation (\ref{equ:R_transition_rel}), the correspondence is $\beta_{i}^{\EDGE} = \hat{\beta}_{i}^{\EDGE}$ for $i = 0, \dots, 3$. Based on domain knowlege, we also truncate the parameters as follows: $\beta_{2, *}^{\AYA} = \max\{0, \hat{\beta}_{2, *}^{\AYA}\}$, reflecting the assumption that weekly relationship quality non-negatively influences AYA adherence, $\beta_{3, *}^{\AYA} = \min\{0, \hat{\beta}_{3, *}^{\AYA}\}$, as carepartner distress is expected to negatively influence adherence, and $\beta_{3}^{\EDGE} = \min\{0, \hat{\beta}_{3}^{\EDGE}\}$ as carepartner distress is expected negatively impact relationship quality.

% We can fit the baseline transition parameters under no intervention directly from the Roadmap 2.0 dataset. Specifically, for parameters in (\ref{equ:R_Transition_AYA}), we have the following correspondence: $\beta_{i, \AM}^{\AYA} = \hat{\beta}_{i, \AM}^{\AYA}$, and $\beta_{i, \PM}^{\AYA} = \hat{\beta}_{i, \PM}^{\AYA}$ for all $i = 0, 1, \dots, 3$, where $\hat{\beta}_{i, 0}^{\AYA}$ and $\hat{\beta}_{i, 1}^{\AYA}$ are fitted coefficients based on GEE approaches. We believe that app burden only moderates the effects of AYA interventions. Therefore, we set $\beta_{4, \AM}^{\AYA} = \beta_{4, \PM}^{\AYA} = 0$. For parameters in (\ref{equ:R_transition_care}), we have direct correspondence--$\beta_{i}^{\CARE} = \hat{\beta}_{i}^{\CARE}$ for $i = 0,\dots,3$ and $\beta_{4}^{\CARE} = 0$. Similarly, for parameters in (\ref{equ:R_transition_rel}), we have $\beta_{i}^{\EDGE} = \hat{\beta}_{i}^{\EDGE}$ for $i = 0, \dots, 3$.


% $\beta_{2, *}^{\AYA} = \max\{0, \hat \beta_{2, *}^{\AYA}\}$ because weekly relationship quality should be positively related to AYA adherence, and $\beta_{3, *}^{\AYA} \min\{0, \beta_{3, *}^{\AYA}\}$ because carepartner distress should be negatively related to AYA adherence. Also, $\beta_{3}^{\EDGE} = \min\{0, \hat \beta_{3}^{\EDGE}\}$ because carepartner distress should be negatively related to relationship quality.

\textbf{Imputing Burden Transitions (2a):}
We set $
  \theta_{1}^{\AYA} = \tfrac{13}{14}, \quad \theta_{1}^{\CARE} = \tfrac{6}{7},
$ so that the memory of digital burden spans roughly one week for both AYA and carepartner. We choose
$\theta_{2}^{\AYA} = 5\,\theta_{3}^{\AYA} = 1, 
  \quad
  \theta_{2}^{\CARE} = 5\,\theta_{3}^{\CARE} = 1$ so that daily interventions exert five times more burden than the weekly relationship intervention. The intercepts are $
  \theta_{0}^{\AYA} = 0.2, 
  \quad
  \theta_{0}^{\CARE} = 0.2$, and chosen so that participants have around a 20\% baseline burden even without an intervention. We set
$
  \omega^{\AYA} = \omega^{\CARE} = 2.4
$ to obtain a moderate noise-to-signal ratio, set so that 
\(
    (\theta_{1}^{\AYA} + \theta_{2}^{\AYA}) / \omega^{\AYA} 
    \approx 0.5
\).
We then truncate burdens at zero and standardize them separately for AYA and carepartner by simulating 10,000 steps with random interventions.

\textbf{Imputing main effects of app burden (2b).} We set $\beta_{4, \AM}^{\AYA} = \beta_{4, \PM}^{\AYA} = \beta_{4}^{\CARE} = 0$ based on the assumption that digital app intervention burden does not directly affect AYA adherence or carepartner distress, unless through moderating the digital interventions.

\textbf{Imputing treatment Effects (2c--2f):}
Since digital health environments are noisy, treatment terms likely have a lower effect on transitions than the baseline transitions under no intervention. Hence, we scale all intervention effects relative to the baseline effects using a single, global hyperparameter $C_{\text{treat}}$. 

For each time of the day (AM or PM), the AYA intervention increases adherence by $\tau_{0, *}^{\AYA} = C_{\text{treat}} \bigl|\beta_{1, *}^{\AYA}\bigr|$, where $* \in \{\mathrm{AM}, \mathrm{PM}\}$ and $\beta_{1, *}$ is the corresponding baseline coefficient estimated from Roadmap 2.0. 

We further define $\bigl|\beta_{1, *}^{\AYA}\bigr|$ and $\tau_{\text{burden}, *}^{\AYA} = -C_{\text{treat}} \bigl|\beta_{1, *}^{\AYA}\bigr|$ because the AYA intervention's effectiveness can be increased by good relationship quality and decreased by high digital-intervention burden.

To account for individual heterogeneity across dyads, each treatment-effect coefficient has an associated random effect with variance $\sigma_{0, *}^{\AYA} = C_{\text{treat}} \sigma_{\beta_{1, *}^{\AYA}}$, where $\sigma_{\beta_{1, *}^{\AYA}}$ is the empirical standard deviation across dyads of the baseline coefficient $\beta_{1, *}^{\AYA}$. 

For carepartner interventions, the main effect on distress is scaled as $\tau_{0}^{\CARE} = -C_{\text{treat}} \bigl|\beta_{1}^{\CARE}\bigr|$, where the negative sign is due to the intervention reducing distress. Lastly, the effect of the weekly relationship intervention on improving relationship quality is given by $\tau^{\EDGE} = C_{\text{treat}} \bigl|\beta_{1}^{\EDGE}\bigr|$.

% \textbf{Imputing treatment Effects (2c--2e):}
%  Since digital health environments are noisy, treatment terms likely have a lower effect on the transitions than the baseline transitions under no intervention. Hence, we scale all intervention effects relative to the baseline effects using a single, global hyperparameter $C_{\Treat}$. 

% \begin{enumerate}
%     \item AYA Daily Interventions (AM/PM)
%         \begin{itemize}
%             \item Main Effect: For each time of day (AM or PM), the AYA intervention increases adherence by $$\tau_{0, *}^{\AYA} \;=\; C_{\text{treat}}\;\bigl|\beta_{1, *}^{\AYA}\bigr|$$
%       where $ * \in \{\mathrm{AM},\,\mathrm{PM}\}$. Recall that $\beta_{1, *}$ is the corresponding baseline coefficient estimated from data. The absolute value ensures a positive effect on adherence of the intervention.
%       \item Interactions with Relationship and Burden: The AYA intervention's effectiveness can be increased by good relationship quality and decreased by high digital-intervention burden. Hence,
%       $$\tau_{\text{rel}, *}^{\AYA} \;=\; C_{\text{treat}}\;\bigl|\beta_{1, *}^{\AYA}\bigr|
%       \quad\text{and}\quad
%       \tau_{\text{burden}, *}^{\AYA} \;=\; -\,C_{\text{treat}}\;\bigl|\beta_{1, *}^{\AYA}\bigr|.$$
%     \item Random Effect Variances: To account for individual heterogeneity across dyads, each treatment-effect coefficient has an associated random effect whose variance. We also scale this variance by $$\sigma_{0,*}^{\AYA} \;=\; C_{\text{treat}}\;\sigma_{\beta_{1,*}^{\AYA}}$$
%     where $\sigma_{\beta_{1,*}^{\AYA}}$ is the empirical standard deviation across dyads of the baseline coefficient \(\beta_{1,*}^{\AYA}\).
%         \end{itemize}
% \item Carepartner Daily Interventions: We perform similar scaling to the carepartner interventions, which are designed to reduce distress. The main effect of a care-partner intervention on distress becomes
% $$\tau_{0}^{\CARE} \;=\; -\,C_{\text{treat}}\;\bigl|\beta_{1}^{\CARE}\bigr|$$
% where the negative sign reflects a reduction in distress due to the intervention.
% \item Weekly Game Intervention: Lastly, the effect of the weekly game on improving relationship quality is:

%  $$ \tau^{\EDGE} \;=\; C_{\text{treat}}\;\bigl|\beta_{1}^{\EDGE}\bigr|.
% $$
% \end{enumerate}


We summarize the imputation design in Table \ref{tab:imputation}. 
%and generate a list of all tuning parameters in Table \ref{tab:hyper_param}.

\begin{table}[hpt]
    \centering
    \caption{Summary of burden transition design and treatment effects design.}
    \begin{tabular}{c|>{\centering\arraybackslash}p{0.4\textwidth}} 
    \hline
    \multicolumn{2}{c}{Burden transition}\\
    \hline
    \hline
    Intercept $\theta_{0}^{\AYA}$  & Based on domain knowledge $\theta_{0}^{\AYA} = 0.2$ \\
    Intervention coefficients $\theta_2^{\AYA}, \theta_3^{\AYA}$ & $\theta_2^{\AYA} = 5\theta^{\AYA}_{3} = 1$ (Because relationship intervention produces lower burden) \\
    Noise standard deviation $\omega^{\AYA}$ & Based on the typical noise-to-signal ratio $\omega^{\AYA} = 2.4$ \\
    \hline
    \multicolumn{2}{c}{Treatment effect for twice-daily adherence transition (* stands for AM or PM)}\\
    \hline
    \hline
    Main effect of AYA intervention $\tau_{0, *}^{\AYA}$ & Hyper-parameter $\tau_{0, *}^{\AYA} = C_{\Treat}|\beta_{1, *}^{\AYA}| $ \\
    Rel. and AYA int. interaction $\tau_{2, *}^{\AYA}$ & Hyper-parameter $\tau_{1, *}^{\AYA} = C_{\Treat}|\beta_{1, *}^{\AYA}|$ \\
    Burden and AYA int. interaction $\tau_{4, *}^{\AYA}$ & Hyper-parameter $\tau_{2, *}^{\AYA} = C_{\Treat}|\beta_{1, *}^{\AYA}|$ \\
    Random treatment variance $\{\sigma_{i, *}^{\AYA}\}_{i = 0}^5$ & Scales with the variance of $\beta_{1, *}^{\AYA}$: $\sigma_{i, *}^{\AYA} = \tau_{i, *}^{\AYA} \cdot  \sigma_{\beta_{1, *}^{\AYA}} / |\beta_{1, *}^{\AYA}|$ \\
    \hline
    \multicolumn{2}{c}{Treatment effect for weekly relationship transition}\\
    \hline
    \hline
    Main effect of relationship int. $\tau^{\EDGE}$ & Hyper-parameter $\tau^{\EDGE} = C_{\Treat}|\beta_1^{\EDGE}|$  \\
    \hline
    % \multicolumn{2}{c}{Disengagement effect}\\
    % \hline
    % \hline
    % Disengagement effect $\xi_{0, d}, \xi_{1, d}, \xi_{2, d}$ & $\xi_{0, d} = \frac{1}{10} |\gamma_{0, d}|$, $\xi_{1, d} = \frac{1}{5} |\gamma_{0, d}|$ and $\xi_{1, d} = \frac{1}{25} |\gamma_{0, d}|$ \\ 
    \end{tabular}
    \label{tab:imputation}
\end{table}

\textbf{Tuning $C_{\Treat}$: } We tune the hyperparameter  $C_{\Treat}$ such that the standardized treatment effects (STE) are around 0.15, 0.3, and 0.5, where STE is defined as:
\begin{equation}
    \operatorname{STE}(C_{\Treat}) = \frac{\mathbb{E}\left[\mathbb{E}[\text{CR}(\pi^*_{e}) \mid e] - \mathbb{E}[\text{CR}(\pi_0, e) \mid e]\right]}{\sqrt{\operatorname{Var}(\mathbb{E}[\text{CR}(\pi_0, e) \mid e])}},
    \label{equ:STE}
\end{equation}
Here, $e$ corresponds to the resulting environment model for dyad $e$ when the hyperparameter is set to be $C_{\Treat}$, and $\pi_e^*$ is the optimal policy for dyad $e$. $\text{CR}(\pi, e)$ is the cumulative rewards earned by running policy $\pi$ on dyad $e$, and $\pi_0$ is the reference policy that always chooses action 0 for all components. 
% STE is defined as the average gap between the cumulative rewards under the optimal policy for each dyad and the cumulative rewards under the reference policy. It is normalized by the standard deviation of the expected cumulative rewards under $\pi_0$ over the distribution of dyads.

Figure \ref{fig:ste} plots the value of the hyperparameter versus the STE computed using the optimal policy in the environment defined by the hyperparameter. We outline our approximation of the optimal policy in Appendix \ref{sec:optpol}. By default, we choose an environment with mediator effect = 1. This results in three dyadic environments, which we summarize in Table \ref{tab:test-bed}.

\begin{table}[ht]
    \centering
    \caption{Summary of all testbeds}
    \begin{tabular}{c|c}
     Treatment effect size & Value of $C_{\Treat}$ \\
    \hline
       0.15 (Small)  & 0.2 \\
       0.3 (Medium) & 0.3 \\
       0.5 (Large) & 0.5 \\
    \end{tabular}
    \label{tab:test-bed}
\end{table}

\begin{figure}[hpt]
    \centering
    \includegraphics[width=0.8\textwidth]{Plots/STE/treat-vs-ste.png}
    \caption{Relationship between the hyperparameters and the STE, categorized by the mediator effect value.}
    \label{fig:ste}
\end{figure}

\subsection{Optimal Policy Approximation}
\label{sec:optpol}

% In this section, we outline our procedure for approximating the optimal policy used for generating the STE, which is defined as the the average gap between the cumulative rewards obtained by the optimal policy and those obtained by the reference policy. 

To approximate the optimal policy, we generate a dataset under a random policy with $P(A_{w,d,t}^{\AYA}=1)= P(A_{w,d}^{\CARE}=1) = P(A_w^{\EDGE}=1)=0.5$ and apply offline Q-learning on this dataset. To make the computation tractable, we discretize and subset the features. Specifically, we use six features: the intercept, AYA adherence, carepartner distress, AYA burden, carepartner burden, and relationship quality. Numerical features (carepartner distress, AYA burden, and carepartner burden) are discretized into 10 bins.

Finally, we evaluate the performance of this approximation against other baseline policies, including micro-randomized actions with fixed probabilities of 0.5, 0.6, 0.7, 0.8, and 0.9. Our approximation consistently outperforms these baselines.


\subsection{Evidence of the Need for Collaboration in the Dyadic Environment}
\label{app:evidence-collab}
% We outline our procedure for verifying that the dyadic environment requires collaboration.

To show that each agent impacts the performance of other agents, we consider the following toy setting.
% We have two random algorithms: 1. 
We fix the care partner agent's randomization probability at 0.5 and vary the AYA agent’s probability to be 0.25 and 0.75. Then, for each fixed AYA agent's probability, we identify the value of the relationship agent’s probability that maximizes average weekly adherence. We find that this relationship probability changes from $1.0$ to $0.0$ when we change AYA agent's probability from 0.25 to 0.75. 

We repeat this experiment for the care partner agent by fixing the AYA agent’s probability at 0.5 and varying the relationship agent’s probability to be 0.25 and 0.75. Similarly, we find that the care partner agent's probability that maximizes adherence changes from 0.6 to 0.5 when we vary the relationship probability from 0.25 to 0.75.

These results indicate that the agents must change their behavior to account for the other agents' behavior.

% set  the randomization probability of the AYA intervention to be 0.25 and 0.75, and see whether the optimal level of randomization probability for the relationship agent is different given a fixed randomization probability of 0.5 for the care partner agent Fig. \ref{fig:MRT_collaboration} (a, b). A similar experiment is conducted for the collaboration between the care partner and the relationship agent in Fig. \ref{fig:MRT_collaboration} (c, d). We see that the optimal level of randomization probability for the relationship agent changes for different AYA randomization probability, and the optimal randomization probability for the care partner agent changes for different game randomization probability. This indicates that the agent must change their behavior to account for the other agent's behavior.

% \ziping{Get rid of the figures.} 

% \begin{figure}[hpt]
%     \centering
%     \includegraphics[width=1\textwidth]{Plots/MRT_collaboration.pdf}
%     \caption{\textbf{(a, b)}: average weekly sum of adherence under different randomization probability for the relationship agent given a fixed probability for AYA and Care partner. \textbf{(c, d)}: average weekly sum of adherence under different randomization probability for the Care partner agent given a fixed probability for AYA and Game.}
%     \label{fig:MRT_collaboration}
% \end{figure}

\section{Additional Results}

\label{app:additional_results}
\subsection{Ablation Study}

\paragraph{No Mediator Effect} The improvement from using a surrogate reward is through the effects of the mediator variables. For example, the relationship intervention $A_{w}^{\EDGE}$ improves the mediator relationship, which may improve the primary outcome, medication adherence. The care-partner intervention $A_{w,d}^{\CARE}$ mitigates the distress, which may improve relationship. In Fig. \ref{fig:mediator0}, we run the all three algorithms under a testbed variant for which we force the above two mediator effects to be 0, i.e., no effect from relationship to adherence or effect from distress to relationship. In this testbed variant, \texttt{MutiAgent+SurrogateRwd} performs the same as \texttt{MutiAgent}--there is no cost of reward learning under no mediator effect.

\begin{figure}[hpt]
    \centering
    \begin{subfigure}[b]{0.31\textwidth}
        \includegraphics[width=1\textwidth]{Plots/Experiments/015/All_Rewards_Mediator0.pdf}
        \caption{STE 0.15}
    \end{subfigure}
    \begin{subfigure}[b]{0.31\textwidth}
        \includegraphics[width=1\textwidth]{Plots/Experiments/03/All_Rewards_Mediator0.pdf}
        \caption{STE 0.3}
    \end{subfigure}
    \begin{subfigure}[b]{0.31\textwidth}
        \includegraphics[width=1\textwidth]{Plots/Experiments/05/All_Rewards_Mediator0.pdf}
        \caption{STE 0.5}
    \end{subfigure}
    \caption{Cumulative rewards improvement over the uniform random policy for all three components under the testbed without the effect of care-partner distress onto relationship quality or the effect of relationship quality onto AYA's adherence.}
    \label{fig:mediator0}
\end{figure}

\paragraph{Other Testbed Variants.} To further violate the assumptions made from the causal diagram, we made the following two changes to test the robustness of our proposed algorithm: 1) we add a direct effect from care-partner psychological distress to AYA medication adherence; 2) we generate random mediator effects, effect from relationship to adherence and effect from distress to relationship. This later one violates the monotonicity assumptions learned from principles.

\subsection{Collaboration of Multi-Agent RL} 

We train each individual agent in the \texttt{MultiAgent+SurrogateRwd} algorithm over 1000 dyads under the STE 0.5 environment, while fixing the randomization probability of the other agents. We denote the randomization probability of the AYA agent, care partner agent, and relationship agent as $p^{\AYA}$, $p^{\CARE}$, and $p^{\EDGE}$ respectively.

We first train the relationship agent while fixing $p^{\CARE} = 0.5$. We see that the average probability of sending an intervention for the relationship agent is 0.57 and 0.42 under $p^{\AYA} = 0.25$ and $0.75$, respectively. This indicates that the relationship agent learns to \textit{reduce} the intervention probability when the AYA agent is more likely to send an intervention. 

Similarly, we train the care partner agent while fixing $p^{\AYA} = 0.5$. We see that the average probability of sending an intervention for the care partner agent is 0.61 and 0.45 under $p^{\EDGE} = 0.25$ and $0.75$, respectively. This indicates that the care partner agent learns to \textit{reduce} the intervention probability when the relationship agent is more likely to send an intervention.
\end{document}
%%%%%%%%%%%%%%%%%%%%%%%%%%%%%%%%%%%%%%%%%%%%%%%%%%%%%%%%%%%%%%%%%%%%%%
