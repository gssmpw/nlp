\documentclass{article}
\pdfoutput=1
%%%%% NEW MATH DEFINITIONS %%%%%

% \usepackage{amsmath,amsfonts,bm}
\usepackage{amsmath,amsfonts}

\usepackage{pifont}


\newcommand{\R}{\mathbb{R}}


\def\va{{\mathbf{a}}}
\def\vg{{\mathbf{g}}}

% Sets
\def\sR{\mathbb{R}}
\def\sC{\mathbb{C}}
\def\sZ{\mathbb{Z}}
\def\sN{\mathbb{N}}
\def\sQ{\mathbb{Q}}

\def\sS{\mathcal{S}}



% Vectors
\def\vzero{{\mathbf{0}}}
\def\vone{{\mathbf{1}}}
\def\vmu{{\mathbf{\mu}}}
\def\vtheta{{\mathbf{\theta}}}
\def\va{{\mathbf{a}}}
\def\vb{{\mathbf{b}}}
\def\vc{{\mathbf{c}}}
\def\vd{{\mathbf{d}}}
\def\ve{{\mathbf{e}}}
\def\vf{{\mathbf{f}}}
\def\vg{{\mathbf{g}}}
\def\vh{{\mathbf{h}}}
\def\vi{{\mathbf{i}}}
\def\vj{{\mathbf{j}}}
\def\vk{{\mathbf{k}}}
\def\vl{{\mathbf{l}}}
\def\vm{{\mathbf{m}}}
\def\vn{{\mathbf{n}}}
\def\vo{{\mathbf{o}}}
\def\vp{{\mathbf{p}}}
\def\vq{{\mathbf{q}}}
\def\vr{{\mathbf{r}}}
\def\vs{{\mathbf{s}}}
\def\vt{{\mathbf{t}}}
\def\vu{{\mathbf{u}}}
\def\vv{{\mathbf{v}}}
\def\vw{{\mathbf{w}}}
\def\vx{{\mathbf{x}}}
\def\vy{{\mathbf{y}}}
\def\vz{{\mathbf{z}}}
\def\vzeta{{\mathbf{\zeta}}}

% Matrix
\def\mA{{\mathbf{A}}}
\def\mB{{\mathbf{B}}}
\def\mC{{\mathbf{C}}}
\def\mD{{\mathbf{D}}}
\def\mE{{\mathbf{E}}}
\def\mF{{\mathbf{F}}}
\def\mG{{\mathbf{G}}}
\def\mH{{\mathbf{H}}}
\def\mI{{\mathbf{I}}}
\def\mJ{{\mathbf{J}}}
\def\mK{{\mathbf{K}}}
\def\mL{{\mathbf{L}}}
\def\mM{{\mathbf{M}}}
\def\mN{{\mathbf{N}}}
\def\mO{{\mathbf{O}}}
\def\mP{{\mathbf{P}}}
\def\mQ{{\mathbf{Q}}}
\def\mR{{\mathbf{R}}}
\def\mS{{\mathbf{S}}}
\def\mT{{\mathbf{T}}}
\def\mU{{\mathbf{U}}}
\def\mV{{\mathbf{V}}}
\def\mW{{\mathbf{W}}}
\def\mX{{\mathbf{X}}}
\def\mY{{\mathbf{Y}}}
\def\mZ{{\mathbf{Z}}}
\def\mBeta{{\mathbf{\beta}}}
\def\mPhi{{\mathbf{\Phi}}}
\def\mLambda{{\mathbf{\Lambda}}}
\def\mSigma{{\mathbf{\Sigma}}}


% Expectation
% \def\eE{\mathop{\mathbb{E}}\limits}
\def\eE{\mathbb{E}}

% Probability
\def\pP{\mathbb{P}}

% Tilde
\def\tf{\tilde{f}}
\def\tS{\tilde{S}}
\def\wtF{\widetilde{\mathcal{F}}}
\def\whR{\widehat{R}}
\def\tvx{\tilde{\mathbf{x}}}
\def\ty{\tilde{y}}


\def\defeq{\overset{\textup{def}}{=}}
% \def\defeq{\overset{.}{=}}
\def\defone{\overset{\text{\ding{172}}}{=}}
\def\deftwo{\overset{\text{\ding{173}}}{=}}
\def\leqone{\overset{\text{\ding{172}}}{\leq}}
\def\leqtwo{\overset{\text{\ding{173}}}{\leq}}
\def\leqthree{\overset{\text{\ding{174}}}{\leq}}
\def\leqfour{\overset{\text{\ding{175}}}{\leq}}
\def\eqone{\overset{\text{\ding{172}}}{=}}
\def\eqtwo{\overset{\text{\ding{173}}}{=}}
\def\eqthree{\overset{\text{\ding{174}}}{=}}
\def\eqfour{\overset{\text{\ding{175}}}{=}}
\def\geqfive{\overset{\text{\ding{176}}}{\geq}}

\usepackage{algorithm}
\usepackage{algorithmic}
\usepackage{natbib}
\usepackage{microtype}
\usepackage{subfigure}

\usepackage{hyperref}

\usepackage{arxiv}

\usepackage{amsmath}
\usepackage{amssymb}
\usepackage{mathtools}
\usepackage{amsthm}

\usepackage{bbm}
\usepackage{url}
\usepackage{hyperref}

\title{Simple Linear Neuron Boosting}

\author{Daniel Munoz \\
        Independent Researcher \\
	\texttt{lnb@dmunoz.org}
}

\begin{document}
\maketitle

\begin{abstract}
Given a differentiable network architecture and loss function, we revisit optimizing the network's
neurons in function space using Boosted Backpropagation \citep{grub2010}, in contrast to optimizing
in parameter space. From this perspective, we reduce descent in the space of linear functions that optimizes
the network's backpropagated-errors to a preconditioned gradient descent algorithm. We show that this
preconditioned update rule is equivalent to reparameterizing the network to whiten each neuron's features,
with the benefit that the normalization occurs outside of inference. In practice, we use this equivalence to
construct an online estimator for approximating the preconditioner and we propose an online, matrix-free
learning algorithm with adaptive step sizes. The algorithm is applicable whenever autodifferentiation is
available, including convolutional networks and transformers, and it is simple to implement for both the
local and distributed training settings. We demonstrate fast convergence both in terms of epochs and wall
clock time on a variety of tasks and networks.
\end{abstract}


\section{Introduction}

\footnotetext[0]{ Correspondence to: Yaswanth M <yaswanthm03@gmail.com> and
Vaibhav Singh <singhvaibhav@cse.iitb.ac.in>}
 %Availability of high-quality labeled data, often combined with auxiliary sources of weak supervision, are 
Large language models (LLMs) have facilitated the generation of high-quality synthetic data that often supplement available training data \cite{lin-etal-2023-selective} or even surpass crowd-sourced annotations \cite{augmentationPNAS,alizadeh2023open}. However, concerns of limited variance in such exemplars, leading to model collapse \cite{shumailov2023curse} or the failure to capture the tail of the true underlying distribution \cite{ding2024data}, remain. Similarly, forming multiple views of the available data by inducing rules, as a complementary source of supervision has shown to benefit various NLP tasks, including text classification \cite{maheshwari2021semi,dong-etal-2022-syntactic}.  In this work, we propose \our, a bootstrapping approach to iteratively refine synthetically generated exemplars and automatically induced rules, resulting in high quality entries with respect to a given classification task \cite{yarowsky-1995-unsupervised,varma2018snuba}. 

\begin{figure}[h] 
   \centering
   \includegraphics[trim={0 1.1cm 0 1cm},clip,scale=0.75]{rule_induction-Page-1.drawio.pdf}
   \caption{Overview of ARISE (\textbf{A}utomatic \textbf{R}ule \textbf{I}nduction using \textbf{S}yntactic tree g\textbf{E}neralization).} 
   \label{archi}
\end{figure}

Figure \ref{archi} provides an overview of \our. We start by using available training data as our seed. Using LLMs, we leverage in-context learning (ICL), with the seed as input to synthetically generate candidate exemplars \cite{liu-etal-2022-makes}. Similarly, we generate rule candidates, via inductive generalisation using least general generalization (LGG)~\cite{plotkin1971further,Raza_Gulwani_Milic-Frayling_2014} by extracting syntactic n-grams from the seed. Further, the induced rules are then filtered using a submodular graph cut-based function \cite{bajpai-etal-2024-fair, kothawade2021prism}. The exemplars and the rules we generate are task-specific and each exemplar and rule is associated with a label. Newly generated exemplars are filtered using rules that are generated from the already validated seed. These filtered exemplars are then added to the seed for the next iteration. Iteratively, we induce rules from synthetically generated data and use the induced rules for data filtering. 


%we generate rule candidates by extracting syntactic n-grams from sentence-level dependency parses of the input. Rules are induced from the syntactic n-grams via inductive generalization using   Since the synthetic data are generated along with their labels, those sentences are then validated using the rules. Only those labeled data points that match with the predictions of the rules are filtered. 



%Similarly,  are successfully used as sources of weak supervision . However, concerns associated with model collapse \cite{shumailov2023curse} or failing to capture the tail of the underlying true distribution due to limited variance in such data are raised with data augmentation \cite{ding2024data}. Here, we iteratively use rules and data to filter each otehr resulting in a pool of high quality data and rules which we incoroproate in our learning tasks

%In \our, we propose a bootstrapped approach for iterative synthetic data generation and automatic rule induction . Moreover, it enables joint training of the induced rules with pre-trained neural models via data programming \cite{maheshwari2021semi,zhang2022survey}. 
%Few-shot text classification (FSTC) is challenging, especially in tasks with a large, semantically similar and often overlapping label space \cite{zhang-etal-2022-contrastive}. Such tasks often find application in diverse domains including task oriented dialogue (intent classification), e-commerce, social networks, scientific literature etc.   \cite{yehudai2024llms, wrench}.  Moreover, these tasks are expected to have a unique or highly specialized label space, leading to limited availability of annotated data \cite{singhal-etal-2023-intendd,vulic-etal-2022-multi}. Intuitively, FSTC systems should be designed to extract as much information as possible from the limited supervision data available for learning. We propose \our, a framework that combines automatic rule induction  \cite{pryzant-etal-2022-automatic,bajpai-etal-2024-fair}, synthetic data generation, and contrastive representation learning \cite{zhang-etal-2022-contrastive} for FSTC. Moreover, \our~induces rules in the form of syntactic n-grams that complements information captured in prevalent approaches in FSTC. 

%Recently, FewMany benchmark puts together a bunch  of tasks varying from Intent detection, e-commerce products among others. In this we work, we show how such tasks can be modeled with a combination of learning techniques leading to state of the art results in several of these tasks under few-shot settings. 

%FSTC tasks are generally addressed using a diverse set of techniques. These include In-context learning \cite{NEURIPS2020_llmfew, NEURIPS2022_llmzero}, contrastive representation learning \cite{vulic-etal-2021-convfit}, data augmentation and filtering \cite{lin-etal-2023-selective}, transductive learning \cite{singhal-etal-2023-intendd}, weak supervision \cite{pryzant-etal-2022-automatic}, meta-learning \cite{mesgar-etal-2023-devil} among others. Several of these works successfully combine one or more of these techniques for FSTC tasks \cite{singhal-etal-2023-intendd,vulic-etal-2022-multi}. 


% \begin{figure*}[h] 
%    \centering\includegraphics[width=0.9\textwidth,page=4,trim={0 19.5cm 0 2cm}]{files/PresentationAutoRules.pdf}
%    \caption{Three-step workflow for \our, along with various components in it.} 
%    \label{archi}
%    \end{figure*}

%approach, that iteratively filters the generated rules and data . 
%\our's novelty lies in effectively filtering automatically induced rules and synthetically generated data iteratively via bootstrapping  




%The gold standard few shot labeled data forms the seed for the data augmentation step. However, it forms the validation data during rule induction and filtering. 




%Starting with a seed set of a k-shot labeled dataset, we induce rules and perform synthetic data generation using prompt demonstration. While the synthetically generated training data points are label-specific, they further undergo filtering using the rules induced. Similarly, once new data points are filtered they are used for further rule induction and filtering. The rule and synthetic data generation is performed iteratively.  Further, a pre-trained model is then undergoes continual pretraining and fine-tuning using contrastive learning techniques \cite{zhang-etal-2022-contrastive}. Finally, we use data programming \cite{maheshwari2021semi} to learn a generative label aggregation model by using the final set of rules of labeling functions. Finally, we perform joint learning  of the final classifier using the label aggregation model and the contrastively tuned pre-trained model.

%Using the rules, we predict a label for the generated data, and filters only those that agrees on the labelusing the induced rules.  and filter only those augmented datapoints that match
 %Three, the joint learning step, effectively combines contrastive representation learning, \cite{pmlr-v119-chen20jsimclr,NEURIPS2020_supervisedContrastive} supervised fine-tuning, and Data programming \cite{zhang2022survey} using a joint learning framework \cite{maheshwari2021semi}. We perform self-supervised contrastive pretraining \cite{wu2020clear} and supervised contrastive learning \cite{khosla2020supervised, zhang-etal-2022-contrastive} over a standard pre-trained neural classifier. We use the few-shot labeled data, along with the filtered data, for fine-tuning the neural classifier. 
 
 
 %The induced rules are used as a form of weak supervision to learn a generative modellabeling-functions in  to learn a  as a . The neural claUsing the induced rules we learn a generative model The classifier is learned jointly with a The induced rules SPEAR is a framework for Data Programming. .and  Further, we use , a framework that enables to learn both the models jointly.

In \our, we boost supervision signals in two ways. With synthetic data generation we supplement the available training data \cite{DBLP:journals/corr/abs-1711-10160ratner,pryzant-etal-2022-automatic}. First, with rule induction, we obtain complementary signals that need not be explicitly captured from the existing data \cite{maheshwari2021semi, singhal-etal-2023-intendd}. Second, our rules are induced as generalized syntactic n-grams. Here, we aim to potentially capture morpho-syntactic information from the data, a view of data that need not be explicitly captured by state-of-the-art (SotA) systems in use. A classical NLP pipeline typically represents a string at multiple levels of abstraction which includes Part-of-Speech (PoS) tags, syntactic relations, \emph{etc.} \cite{manning-etal-2014-stanford}. \our~uses higher-order dependency structures as features and generalizes over these features using inductive generalization \cite{popplestone1970experiment} to induce the rules as generalized syntactic n-grams. 
 
%Previous works used rules for boosting supervision signals in various ways. Rules are used as an auxiliary source that can add more labeled data, albeit being noisy . Alternatively, rules can be used  Our generated data brings in auxiliary sources of information. The rules we generate are used to improve efficient utilisation of the available data during the learning. 
  %Weak supervision sources such as rules are often used as auxiliary sources of information that brings in additional information not fully captured in the available training data. Similarly, it can be used for maximising utility of the available data and complement the labeledd data. \our~ achieves both 
 

 

We find applicability of both the rules and exemplars from \our, with consistent performance gains in various text classification setups. Specifically, we experiment with ICL and fine-tuning setups. In ICL, we focus on long-context ICL \cite{li2024long,bertsch2024context} and use the generated data as a pool from which exemplars are retrieved. Further, we incorporate our rules as explanations to the input and the exemplars. Similarly, we use the data for fine-tuning models, which include pre-trained LLMs, Qwen \cite{Qwen2,Qwen2.5} and RoBERTa \cite{liu2019roberta}. %The induced rules enable learning a generative label aggregation model as a form of weak supervision using data programming. We jointly learn a classifier with the generative model using SPEAR \cite{maheshwari2021semi}, a data programming framework.

We perform extensive experiments on multiple text classification datasets, which include three full-shot, and eight few-shot datasets from the {\sc FewMany} benchmark \cite{yehudai2024llms}. Further, we perform multilingual experiments on seven languages using the {\sc MASSIVE} \cite{fitzgerald2022massive1mexamplemultilingualnatural} dataset. 
%Our experiments are performed using both 5-shot and 10-shot settings. In all these settings, \our~outperforms strong competitive models, such as IntenDD \cite{singhal-etal-2023-intendd}, \citet{zhang-etal-2022-contrastive}, and  FastFit \cite{yehudai2024llms},  with statistically significant improvements. 

%In section \ref{sec:genSpace}, we elaborate on our rule induction approach for inducing generalized syntactic n-grams. In section \ref{sec:method}, we elaborate \our, a 3-step framework for FSTC. Here, we elaborate our iterative rule and data filtering along with the joint learning setup. 


Our major contributions are as follows:




\begin{itemize}
    \item Use of rules and data from \our~results in statistically significant gains in all the experimental setups, as compared to the corresponding configuration without resources from \our. Specifically, we obtain state of the art (SotA) results in our full-shot and few-shot experiments when using \our.

    \item The rules we generate are shown to be effective, both during ICL and fine-tuning. Further, using the rules as explanations under ICL for CDR dataset results in SotA results. Similarly, fine-tuning Qwen jointly with data and the augmented rules from \our~has shown statistically significant improvements for Qwen and RoBERTa based models.

    \item Use of augmented data for few-shot setups in the {\sc FewMany} benchmark demonstrate the quality of the augmented data we produce. We show that simply using additional data from \our, as low as 20-shot additional data per class, can result in improved performance than incorporating complex approaches such as contrastive representation learning into the training process. 
    
    %Our proposed approach yields statistically significant gains in all the experiments we perform, compared to state-of-the-art systems \cite{yehudai2024llms, singhal-etal-2023-intendd, zhang-etal-2022-contrastive}. Our best performing model reports a 2.04\% increase in 10-shot and 2.52\% increase in 5-shot settings, compared to the next best model, averaged across all the monolingual tasks.

    \item Our extensive experiments show that \our~is generalizable across multiple domains and multiple languages. We report a 7.21\%  increase in performance, compared to the model without any resources from \our, averaged across seven different languages.    

    \item We show that leveraging syntactic information as weak supervision for rule induction, brings a complementary source of supervision, which otherwise need not be captured by using string level data directly (\S \ref{ruleImpact}).
    
    %performance improvements compared to surface-level string n-grams as rules. Further, our bootstrapped approach outperforms competitive approaches for filtering augmented data \cite{lin-etal-2023-selective}. 

    

\end{itemize}


%show that we can complement these  models with relevant information by leveraging a combination of approaches that include contrastive learning, data augmentation, weak supervision, data subset selection and automatic rule induction for learning models in few shot settings. Further, by using automatic rule induction we learn rules that use syntactic information. Moreover, we show that these information can help improve the performance of these pre-trained deep-neural models. %are there is information which can be utilized?




%Our framework employs the following learning approaches for a general few shot text classification approach. Similar to several other few shot text classification approaches popular in intent classification, we perform contrastive learning for better representation learning. Here, we first perform self-supervised pretraining followed by supervised contrastive learning. Two, we propose an automated approach for inducing `rules' in the form of generalised syntactic n-grams. Three, we synthesise new labeled samples via data augmentation. Here, we use the few shot labeled samples as input to an LLM for data augmentation. Three, we filter the augmented sentences using a data subset selection approach. Finally, we use data programming by using the automatically induced rules as labeling functions to learn a generative model. We then learn an intent classifier using our contrastively trained model jointly learning with the generative model built.rules' from a corpus by generalizing over a feature space that consists of .\our~focuses on  over the concept space of these abstractions in the pipelineIt 


%Further, we use least general generalization \cite[LGG][]{plotkin1971further, plotkinnote} to obtain the generalised syntactic n-grams as our rules. T 


%Moreover, these rules can be integrated to pre-trained models during the training via programmatic weak supervision (PWS). Specifically, a feature space over higher-order dependency structures from dependency parses of the inputs, akin to syntactic n-grams, is defined. The features are enumerated and scored using  labeled data. The entire corpus, consisting of both labeled and unlabeled data, is partitioned using these features. For each such partition, we obtain a generalization of the features which form the rule. T. The classifier is trained jointly with a label aggregation model in a semi-supervised setting.   

% Now, given that there are several approaches by which supervision can be incorporated with the limited data. Now, in-context learning or parameter updation approaches, such as fine-tuning, have been two primary categories to address such  tasks. So far, . However, it is still entirely not clear whether these systems capture all the relevant information for the task. A more interesting question would be, is it possible to capture information  that can be complementary to what these systems learn and use it to improve the performance of these systems? \our~ proposes a general-purpose rule induction framework for semi-supervised text classification. 


 %In \our, we use a combination of the entries at various levels of this abstraction as the generalization. 
%For instance, the two phrases `brown foxes' and `brown cats' are different at the string level. Further, their corresponding dependency representations, which we use as features, are also distinct; `brown $\xleftarrow{amod}$ foxes' and `brown $\xleftarrow{amod}$  cats'. However, they both can be `generalized' into a single structure, written as `brown $\xleftarrow{amod}$ {\sc Noun}', where $amod$ is the dependency relation between both the words and {\sc Noun} corresponds to the POS tag for common nouns. The generalized structure forms a rule in \our.


%The structure, `brown $\xleftarrow{amod}$ {\sc Noun}' is not just a generalization of the aforementioned pair of strings, but it is representative of any string containing a common noun with brown as its adjectival modifier. Similarly, we may consider  `{\sc ADJ} $\xleftarrow{amod}$ {\sc Noun}' also as a generalization of the aforementioned pair of strings, which is more general than the former and it covers a wider set of strings, i.e. any pair of common nouns with exactly one adjectival modifier. Here,  we need to identify generalizations that are of utility for a classification task as compared to the possible over-generalizations. Inductive Logic Programming (ILP) has been extensively used in the past in identifying generalizations over such concepts. Specifically, we employ least general generalization \cite[LGG;][]{plotkinnote,plotkin1971further} over the partition of the dataset encoded in the feature space. 




%We observe consistent increase in performance for each of the datasets


%LGG is performed over a partition of the input dataset, and the quality of the rules obtained is highly dependent on the quality of the partition. \our~employs a modularity-based community detection approach, Louvain, to partition the dataset where similar inputs in the 





%However, we need to identify 


%Systems such as GOLEM \cite{muggleton1992efficient}, {\sc Chillin} \cite{zelle1995inducing} are ILP approaches that use variants of least general generalization. 

%It integrates diverse areas in NLP in its framework to enable rule induction. Higher-order dependency features are used for feature space construction, 



%incorporate these rules as weak labelers or rather as labeling functions using PWS enables the integration of various sources of weak supervision, albeit noisy, in the form of programs that enables ameliorating the need for hard data labeling. Specifically, we use the semi-supervised data programming framework by Spear where the loss function integrates predictions from the rules and from the neural classifier. 

%While programmatic weak supervision enables to integrate arbitrary rules to be integrated into a learning system, those typically are expected to be heuristics hand-crafted by subject matter experts or are obtained from existing resources. Hence any arbitrary Python function that returns a label can be used as a labeling function and no further constraints exist for the type of function. However, since we automate the rule generation, the rule generation needs to follow a formalism or rather the rules generated have to stick to a constrained formalism. In our case, we experiment with two different approaches. One generates a restricted form of horn-clauses using path-constrained random walks, and two, uses least general generalization a form of abstracting out a generalized representation in the same scheme as the input. 



%We show that our approach can outperform previous state-of-the-art approaches in similar settings and further, we can integrate our approach with an LLM of 7 billion parameters which can further improve our performance. Here, we use adapters to improve our models which enables parameter-efficient training of these moddels. 
\section{Background and Motivation}
In this section, we review related works across three key areas and then outline the motivation behind this study.

\subsection{Web Crowdsourcing and Human-AI Collaboration Empowerment}
% Crowdsourcing~\cite{howe2006rise} refers to the practice of acquiring ideas, services, or content by soliciting contributions from a large group of people. 
With the advancement of web technologies, crowdsourcing activities have increasingly migrated to web and mobile internet platforms, namely web crowdsourcing~\cite{doan2011crowdsourcing}. An exponential rise in its applications has witnessed, such as ride-hailing and software development.
To tackle complex web-based tasks, scientists at Microsoft introduced human-AI interaction guidelines to assist researchers and practitioners in designing studies utilizing AI technologies~\cite{amershi2019guidelines}. Following this, numerous studies have integrated human intelligence with AI methods to address challenges such as conversational agent learning for intent detection and text classification~\cite{yang2018leveraging,arous2021marta}. A recent study, for example, engaged online users from crowdsourcing platforms and implemented advanced computer vision techniques to generate city maps~\cite{qiu2019crowd}. Given the growing significance of AI-in-the-loop systems in human-intervened tasks, the concept and principles of human-AI decision-making within the context of web crowdsourcing were provided~\cite{green2019principles}. 

% Thus, we have witnessed an exponential rise in applications built around the concept of crowdsourcing~\cite{howe2006rise}--from ride-hailing~\cite{seng2023ridesharing} and food delivery~\cite{liu2018foodnet} to software development~\cite{abd2021use} and urban governance ~\cite{qiu2019crowd}. The "algorithmic crowdsourcing" paradigm is exemplified where web architectures coordinate human-machine interactions at scale. In light of the practical effectiveness in these domains, web crowdsourcing is now expanding into more complex task domains such as source search (e.g., locating gas leaks or biological signals)~\cite{zhao2024user}.

\subsection{Source Search and Crowd-powered Practices}
Source search is a critical problem for both nature and mankind~\cite{jing2021recent} focusing on determining the location of a source (of gas or signal) in the shortest possible time. Existing source search approaches can generally be classified into three categories: information-theoretic~\cite{jang2023improved}, biologically-inspired~\cite{al2021distributed}, and gradient-based methods~\cite{jiang2019source}. Among these, information-theoretic algorithms, especially those grounded in the Bayesian framework~\cite{ojeda2024robotic}, stand out for their distinct advantages. To further enhance the performance (i.e., success rate and efficiency) of a searching algorithm, multi-robot collaboration mechanisms~\cite{tang2020multirobot} have been designed and adopted. However, when source search takes place in complex environments, the search process always encounters fatal problems, resulting in wrong outcomes. Thus, researchers started to explore effective ways leveraging human intelligence to improve AI-based search algorithms through web platforms~\cite{zhao2024user}. However, this approach also entails substantial costs and imposes considerable burdens on human workers.

\subsection{Large Models for Scene Understanding and Reasoning}
MLLMs integrate multimodal encoders/decoders with traditional LLMs, enabling cross-modal understanding that overcomes text-only limitations. While these models demonstrate remarkable capabilities across diverse tasks including image-text understanding~\cite{liu2024visual}, video-text understanding~\cite{li2023videochat}, and even multimodal generation~\cite{peng2023kosmos}, their effectiveness in handling complex tasks remains constrained by predominant single-step reasoning approaches. To this end, CoT prompts are utilized to enhance problem-solving abilities by guiding LLMs through structured multi-step reasoning. Recent work explores CoT adaptations for multimodal problems, for instance, Shikra~\cite{chen2023shikra} pioneers CoT application in visual grounding tasks, while SoM~\cite{yang2023set} introduces structural image annotations like segmentation maps and spatial grids to provide spatial reasoning anchors. However, CoT has not been comprehensively explored for fine-grain reasoning in source search tasks.

\subsection{Motivation}
Building on the demonstrated scene understanding and reasoning capabilities of large models across various tasks, as well as addressing the limitations of human-AI collaborative source search, our work seeks to explore concrete methods for leveraging large models in zero-shot source search tasks within a top-down view of web-based search environments.



\section{Boosting over Linear Neurons}
\label{sec:lnb}

\subsection{Reduction}
For training general architectures we can forgo computing the 
regression targets, $\lambda_i$. Instead, we use
its definition that it is the gradient vector of the loss
\wrt the \emph{output} of $f_i$,
%
\begin{equation}
\label{eq:lambda}
\lambda_i^T
= \frac{\partial l_{y}(x_m)}{\partial f_i(x_{i-1})}
= \frac{\partial l_{y}(x_m)}{\partial x_i},
\end{equation}
%
where $x_{i-1} \in \gX_{i-1}$ are the forward-propagated \emph{input} features to the neuron $f_i$.
Note that $x_{i-1}$ and $\lambda_i$ both depend on the sample $(x,y)$ and current
network definition, $F_B$.

When projecting the functional gradient vector for $f_i$
over a linear hypothesis space, the weak-learners have the same
function definition as the neuron but differ in the parameters.
The scalar projection step (\Eqref{eq:scalarproj}) then corresponds to
solving the ordinary least squares (OLS) problem
%
\begin{equation}
\label{eq:OLS}
\hat{\theta}_i = \argmin_{\theta} \frac{1}{2} \E_{(x,y) \sim \gD} 
\Vert f_i(x_{i-1};\theta) - \lambda_i \Vert^2.
\end{equation}
%
Because this is an OLS problem, we know its solution
is the projection of the targets, $\{\lambda_i\}$,
onto the column space of the forward-propagated input features to $f_i$;
we denote this set of input features as $X_i = \{x_{i-1}\}$.

Using \Eqref{eq:lambda} and the definition
$x_i = f_i(x_{i-1};\theta)
= \frac{\partial x_i}{\partial \theta_i}\theta$
(because $f_i$ is linear),
we can rewrite \Eqref{eq:OLS} as
%
\begin{equation}
\label{eq:OLS2}
\hat{\theta}_i = \argmin_{\theta} \frac{1}{2} \E_{(x,y) \sim \gD}
\biggl \Vert \frac{\partial x_i}{\partial \theta_i}\theta - \frac{\partial l_{y}(x_m)}{\partial x_i}^T \biggr \Vert^2,
\end{equation}
%
which has the corresponding Normal Equation
%
\begin{align}
\E_{x_{i-1} \sim X_i} \biggl [ \frac{\partial x_i}{\partial \theta_i}^T\frac{\partial x_i}{\partial \theta_i} \biggr ] \hat{\theta}_i 
= \E_{(x,y) \sim \gD} \biggl [ \biggl (\frac{\partial l_{y}(x_m)}{\partial x_i} \frac{\partial x_i}{\partial \theta_i} \biggr )^T \biggr ].
\label{eq:normal}
\end{align}
%
The right-hand side is the component of $\nabla \bar\ell(\theta_B)$ corresponding to $\theta_i$, so the solution is
$\hat{\theta}_i = M_i^{-1}g_i$, where
 $M_i = \E_{x_{i-1} \sim X_i} [ \frac{\partial x_i}{\partial \theta_i}^T  \frac{\partial x_i}{\partial \theta_i}]$.
We can confirm that the norm in function space is equivalent to the
inner product under $M_i$ of \Eqref{eq:metricnorm}:
%
$$
\Vert f_i(\cdot; \hat{\theta}_i) \Vert^2_F
= \E \biggl [\frac{\partial x_i}{\partial \theta_i} \hat{\theta}_i \cdot \frac{\partial x_i}{\partial \theta_i} \hat{\theta}_i \biggr ]
= \hat{\theta}_i \cdot M_i \hat{\theta}_i
= \hat{\theta}_i \cdot g_i.
$$
%

Because the OLS problems are solved independently per neuron,
the final solution is equivalent to as if we constructed a block-diagonal metric, $M$,
containing each $M_i$ along the diagonal and then computing the preconditioned gradient
vector from \S \ref{sec:cgrad}. We denote this vector as $\hat{\theta}_B = \nabla_M \bar\ell(\theta_B)$,
which is composed of each neuron's $\hat{\theta}_i$ solution.
Note that the structure of the network is only needed to compute $\nabla \bar\ell(\theta_B)$ (via
autodifferentiation) and we can then solve \Eqref{eq:normal} for each neuron independently.

\subsection{Optimization}
To solve for $\hat{\theta}_i$ in \Eqref{eq:normal},
we can leverage any linear system solver that internally
uses matrix-vector-products (MVP) instead of instantiating
each metric, $M_i$.
First, we need to save the set of forward-propagated input features, $X_i$, to the neuron; we note
that these features are already computed and saved in
memory during backpropagation.
In practice, the MVP $M_i \theta$ can then be computed via a
VJP composed with a Jacobian-vector-product (JVP) of
the vectorized function $\mathbf{f}_{X_i}(\theta_i) = \text{vec}(\{f_i(x_{i-1};\theta_i)\}_{x_{i-1} \in X_i})$,
which maps $f_i(\cdot; \theta_i)$ over  every element (tensor) in dataset $X_i$, and then
%
\begin{equation}
\label{eq:mvp}
M_i(\theta) = 
\frac{1}{n_i} \frac{\partial \mathbf{f}_{X_i}}{\partial \theta_i}^T \frac{\partial \mathbf{f}_{X_i}}{\partial \theta_i} \theta,
\end{equation}
%
where $n_i$ is the number of samples that contribute to the
gradient\footnote{In general this value can be computed using the shape of $\gX_i$,
\eg, if $f_i(x;\theta) = \theta^Tx$, then $n_i$ is
the batch size; if $f_i$ is a strided convolution, then $n_i$ 
is the number of output pixels in the batch.}.
Note that since $\mathbf{f}_{X_i}$ is linear it need only be
linearized once (at any point) by the linear system solver.

After solving \Eqref{eq:normal} for each neuron, we can construct $\hat{\theta}_B$ and
compute the $\eps$-sized step under the full metric $M$ via \Eqref{eq:metricnorm}.
Lastly, we may wish to regularize our strong-learner in function space as discussed in 
\S \ref{sec:reg}. Because the learner is linear, this corresponds to a 
``weight decay'' of factor $1-\sqrt{\eps} \rho$; note that if a neuron contains
bias terms, these should also decay.
Algorithm \ref{algo1} summarizes the entire optimization algorithm, which
we refer to as Linear Neuron Boosting (LNB).

\begin{algorithm}[tb]
\caption{Linear Neuron Boosting}
\label{algo1}
\begin{algorithmic}[1]
\INPUT{Network $F$,
linear neurons $F_B \subseteq F$ with parameters $\theta_B$,
loss function $\bar\ell$,
step size schedule $\{\eps^{(t)}\}_{t=1}^T$,
weight decay $\rho$}
\STATE Randomly initialize $\theta_B$
\FOR{$\eps \in [\eps^{(1)}, \ldots, \eps^{(T)}]$}
\STATE Compute $\nabla \bar\ell(\theta_B)$ via backpropagation and save the 
inputs to each neuron, $X_B = \{X_i | f_i \in F_B\}$ \label{algo:bp}
   \FOR{$(f_i, g_i, X_i) \in (F_B, \nabla\bar\ell(\theta_B), X_B)$}
      \STATE Define the \Eqref{eq:mvp} MVP function, $M_i(\cdot)$, using the VJP and JVP of $\mathbf{f}_{X_i}$
      \STATE $\hat{\theta}_i \gets \texttt{linear\_solver}(A=M_i, b=g_i)$
   \ENDFOR
   \STATE $\theta_B \gets (1-\sqrt{\eps}\rho)\theta_B$  \hfill $\triangleright$ Weight decay
   \STATE $z = \hat{\theta}_B \cdot \nabla \bar \ell(\theta_B)$  \hfill $\triangleright$ Norm
\STATE $\theta_B \gets \theta_B - \sqrt{\frac{\eps}{z}} \hat{\theta}_B$ \hfill $\triangleright$ Adaptive step size
\ENDFOR
\OUTPUT{$\theta_B$}
\end{algorithmic}
\end{algorithm}

\subsection{Update Properties}
\label{sec:WW}
We now analyze the properties of the LNB update rule. For simplicity in explanation,
we use a neuron with scalar output, $f_i : \gX_{i-1} \to \sR$, and note that this is sufficient because
the OLS solution of \Eqref{eq:OLS} for a vector output is the same as independently solving for each component.

When $f_i(x;\theta) = \theta^Tx$ \emph{without} a bias term, the resulting metric is the \emph{uncentered}
second moment matrix, $M_i = \E_{x \sim X_i}[xx^T]$. Denoting the number of parameters as $d_i$,
when the features are linearly independent and $n_i > d_i$, then $M_i \succ 0$ and the
OLS solution is scale equivariant \wrt the features. Of course, satisfying both of these
conditions can be rare in practice and we discuss mitigations in \S \ref{sec:PSD}.

When $f_i(x;\theta) = \theta^T \acute x$, where $\acute x^T = [x^T, 1]$
the resulting metric has the block form
\begin{equation}
M_i = \begin{pmatrix}
\Sigma_i + \mu_i \mu_i^T & \mu_i\\
\mu_i^T & 1
\end{pmatrix},
\end{equation}
where $\mu_i = \E_{x \sim X_i}[x]$ and $\Sigma_i$ are the mean and covariance matrix of $X_i$, respectively.
We can analytically invert $M_i$ via the factorization $M_i=L_i^T L_i$, where
$L_i = \begin{pmatrix}
   \Sigma_i^{\frac{1}{2}} & 0\\
   \mu_i^T & 1
\end{pmatrix}$.
Then, $M_i^{-1} = W_i W_i^T$, where
$W_i = \begin{pmatrix}
   \Sigma_i^{-\frac{1}{2}} & 0\\
   -\mu_i^T \Sigma_i^{-\frac{1}{2}} & 1
\end{pmatrix}$. From \S \ref{sec:pcgd}, preconditioning $W_iW_i^T g_i$ is the same
as reparameterizing the model:
$$
f_i(x; W_i\theta) = (W_i \theta)^T \acute x = \theta^T W_i^T \acute x = \theta^T (\Sigma_i^{-\frac{1}{2}}(x-\mu_i)).
$$
To summarize, when running LNB with neurons that have bias parameters,
it is equivalent to optimizing a model where we 
added a whitening transformation step in the network to each $f_i$'s input features using the
respective statistics from $X_i$, \emph{before} applying the parameters.
We further discuss this relationship to Batch Normalization \citep{batchnorm} and other methods in \S \ref{sec:connections}.

When the neuron is a convolution layer, $f_i(x;\theta) = x * \theta$, the output is still linear in the parameters (filters)
and the interpretation reduces to either of the previous two cases, with the only difference that the
first and second moments are computed using the filter's spatial support features (and are
translationally equivariant). By representing the metric using Jacobians, we can use autodifferentiation
to compute the metric without requiring a specialized implementation for this case, \eg, $\texttt{unfold}$.

Other common forms of $f_i$ will likely result in an identity $M_i$. For example, in a vision
transformer \citep{vit} the corresponding metric for both the embedding function, $f(x;\theta) = x + \theta$, and the class-token
function, $f(x;\theta) = \texttt{concat}(\theta, x)$, is the identity. In these cases, the linear solver can be skipped
since $\hat{\theta}_i = g_i$ and there are no other changes to the LNB algorithm.

\subsection{Positive Semi-definite Metrics}
\label{sec:PSD}
Since $M_i$ is an inner product of a Jacobian with itself, $M_i \succeq 0$, in general.
To ensure $M_i \succ 0$ in practice, a common regularizer is to add a small constant, $\gamma$,
along the diagonal of $M_i$ (excluding the bias terms) when solving the linear system. 
In the boosting perspective this adds a $\gamma \Vert \theta \Vert^2$ penalty term to \Eqref{eq:OLS2}
(which is why one would typically exclude the bias terms).
Because this regularizes the norm of the parameters, the solution is no longer scale equivariant.
When the metric's eigenvalues are small it no longer informs a meaningful
step-size ($z_{\theta} \approx 0$). In this case, we can upperbound the maximum allowed step size
to avoid taking a large step. This is essentially equivalent to disabling the
preconditioning and following the typical gradient vector direction.

\begin{algorithm}[ht]
   \caption{Online Linear Neuron Boosting}
   \label{algo2}
   \begin{algorithmic}[1]
   \INPUT{Network $F$,
   linear neurons $F_B \subseteq F$ with parameters $\theta_B$,
   mini-batch loss function $\bar\ell$,
   step size schedule $\{\eps^{(t)}\}_{t=1}^T$,
   weight decay $\rho$, minimum norm $z_0$}
   \STATE Randomly initialize $\theta_B$
   \FOR{$\eps \in [\eps^{(1)}, \ldots, \eps^{(T)}]$}
   \STATE Compute $\nabla \bar\ell(\theta_B)$ via backpropagation and save the 
   inputs to each neuron, $X_B = \{X_i | f_i \in F_B\}$
   \STATE $g_B = \texttt{ema}(g_B, \nabla \bar\ell(\theta_B))$
      \FOR{$(f_i, g_i, X_i) \in (F_B, g_B, X_B)$}
         \STATE $\mu_i = \texttt{ema}(\mu_i, \texttt{mean}(X_i))$
         \STATE $\chi_i = \texttt{ema}(\chi_i, \texttt{mean}(X_i \odot X_i))$
         \STATE Create $P_i$ from $\mu_i$ and $\chi_i$ (\S \ref{sec:online})
         \STATE Define the \Eqref{eq:mvp} MVP function, $M_i(\cdot)$, using the VJP and JVP of $\mathbf{f}_{X_i}$
         \STATE $\hat{\theta}_i \gets \texttt{cg}(A=M_i, b=g_i, P=P_i, x_0=\hat{\theta}_i)$
      \ENDFOR
      \STATE $\theta_B \gets (1-\sqrt{\eps}\rho)\theta_B$  \hfill $\triangleright$ Weight decay
      \STATE $z = \max(z_0, \hat{\theta}_B \cdot g_B)$  \hfill $\triangleright$ Norm
   \STATE $\theta_B \gets \theta_B - \sqrt{\frac{\eps}{z}} \hat{\theta}_B$ \hfill $\triangleright$ Adaptive step size
   \ENDFOR
   \OUTPUT{$\theta_B$}
   \end{algorithmic}
\end{algorithm}

\subsection{Online Learning}
\label{sec:online}
As presented, each step of boosting should be performed using the entire dataset,
which is quite impractical. We can easily switch to using an online estimator
of the gradient vector, \eg, an exponential moving average (EMA) over minibatches.
However, computing an online estimate for $M_i$ in our setting is difficult because we
rely on MVPs and do not materialize it nor $M_i^{-1}$.
One alternative, at the expense of increased compute and memory during training,
is to estimate $M_i$ using samples from a much larger dataset, which has demonstrated
to help under the FIM \citep{pascanu-2014}. Because $M_i$ does not require labels
we can compute it via only forward passes.
A second option, when $M_i$ can be materialized in memory, is to use standard techniques for
online covariance estimation \citep{dasgupta}.
If eigendecomposition can be afforded, a third option is to initialize $M_i^{-1}$ with the identity and
perform k-rank updates via the Woodbury matrix identity; this could also be amortized across gradient updates.

Instead of the previous three options, we instead propose to use an online, approximate
estimate of the feature moments as a preconditioner for the linear system solver.
Specifically, we use the Conjugate Gradient algorithm where we precondition each conjugate
step with the approximate solution to facilitate quick convergence.
The form of the preconditioner, $P_i$, is dependent on if the respective $f_i$ has a bias term or not.
If $f_i$ does not have a bias term, we approximate $M_i^{-1}$ using its diagonal,
$P_i = \E[\texttt{diag}(xx^T)]^{-1}$. If $f_i$ has a bias term, 
we approximate $W_iW_i^T$ from \S \ref{sec:WW} with the incomplete Cholesky factorization
$P_i = \widetilde{W}_i\widetilde{W}_i^T$, where $\widetilde{W}_i$ approximates the respective $\Sigma_i$ term with its diagonal.
Note that this does \emph{not} approximate $M_i^{-1}$ with a diagonal matrix, since it still contains the
off-diagonal $\mu_i$ terms.

We construct the terms in $P_i$ using an EMA of the moments $\mu_i$ and $\chi_i = \E[x \odot x]$
across the minibatches\footnote{Note that these quantities are also straightforward to compute for convolutions:
$\mu_i$ is the mean over all pixels in the minibatch.}.
Then, $\E[\texttt{diag}(xx^T)] = \chi_i$ and
$\texttt{diag}(\Sigma_i)= \chi_i - \mu_i \odot \mu_i$.
When initializing conjugate gradient with the solution from the previous LNB step and using 
$P_i$ as the preconditioner, we found that we obtain diminishing returns
on the solution quality after only two iterations in practice. The online version of LNB is
summarized in Algorithm \ref{algo2}. Note that this approach is also amenable to the distributed
training setting where the sufficient statistics, $\mu$ and $\chi$, can be asynchronously updated and communicated
to different nodes (as with the gradient vectors) while each node runs its own conjugate gradient solver.

\subsection{Complexity}
The space complexity of computing a LNB step is the same as backpropagation since
it reuses the feature sets $X_B$. However, the actual amount of used memory is increased by
approximately the size of $X_B$ in order to evaluate the MVP. Additionally, conjugate
gradient holds additional copies of the parameters.

For each neuron, the time complexity to solve the linear system is $O(n_id_i^2 + d_i^3)$, in general.
However, in practice, due to the outer-product nature of the metric, one MVP evaluation is
$O(n_i d_i)$ and we perform only 2 steps of conjugate gradient.

\input{4related_work}
\begin{figure}[t]
    %
    \begin{center}
    \centerline{\includegraphics[width=0.5\columnwidth]{experiments/recht_loss.pdf}}
    %
    \caption{Matrix factorization via a 2-layer linear network}
    \label{recht}
\end{center}
\end{figure}

\section{Experiments}
\label{sec:experiments}
The prior works of FOOF, LocoProp, PRONG have shown to compare competitively
with other sophisticated optimizers such as K-FAC \citep{martens2015}.
Given their discussed equivalence with LNB, we focus on experimentally confirming
the contributions of this work: feature whitening via preconditioning the
gradient vector, and the applicability and effectiveness on
the realistic networks of ViT \citep{vit} and UNet \citep{unet}.

Due to its de facto status, we benchmark convergence with Adam in both
iteration and wall time.
In all experiments, the default EMA values were used ($\beta_1=0.9$, $\beta_2=0.999$)
for Adam while the learning rate was grid searched around $1e^{-4}$
All timings were recorded from a NVIDIA L4 GPU. Due to the deterministic nature of performing
2 conjugate gradient steps for LNB, the variance in reported times is negligible.


\subsection{Matrix factorization}
We start with the pathological example from \citet{recht}. This is a matrix
factorization problem formulated as a two-layer linear network:
$\sum_{i=1}^{n} \Vert W_1 W_2 x_i - y_i\Vert^2$,
where $y_i=Ax_i$ for a poorly conditioned matrix, $\kappa(A)=10^5$. Due to the conditioning
and the columns being correlated, it is known that gradient descent converges slowly for this
problem, whereas GN converges quickly. Because the LNB preconditioner is decorrelating
the feature space, we would also expect fast convergence.

We use the same initialization as in the notebook, but in order to magnify the differences,
we increase the dimensions by a factor of $10$: $n=10^4$,
$W_2 \in \sR^{60 \times 60}$, $W_1 \in \sR^{100 \times 60}$. Learning rates were tuned via
grid search to find fast and stable convergence for each method. The results are plotted in Figure \ref{recht}
and reproduce the prior reported slow convergence of Adam and demonstrate fast convergence with LNB.

\begin{figure}[t]
    %
    \begin{center}
    %
    \subfigure[Original pixels ($x$)]{\includegraphics[width=0.49\columnwidth]{experiments/mnist_acc.pdf}}
    %
    \subfigure[Inverted pixels ($1-x$)]{\includegraphics[width=0.49\columnwidth]{experiments/inv_mnist_acc.pdf}}
    %
    \vskip -0.1in
    \caption{MNIST test accuracy evolution trained on the original data (a) vs. inverted pixels (b).
    The step-size is parenthesized.}
    \label{fig:mnist}
    \end{center}
\end{figure}


\subsection{MLP}
We reproduce the MLP result in \citet{grub2010} that compares boosting and gradient
descent for a 2-layer MLP on MNIST using 800-node layers, $\tanh$ activation,
Glorot Normal initialization and a batch size of 1,000.

In Fig \ref{fig:mnist}-a., we plot the test accuracies w.r.t. epoch with the best two learning rates for Adam.
We first note that Adam and LNB obtain better than the prior reported accuracy of $98.3\%$.
Second, LNB achieved this performance with a fixed (ridge) regularizer $\lambda$, whereas prior
work heavily tuned this.
One explanation for the difference is that LNB is taking an adaptive step size according to the metric
and this was not derived before.
Third, there is little observed difference between boosting and gradient descent on this dataset.
This can be explained due to that most of the binary pixels in MNIST are zero, so the feature space
of the vectorized images is low rank, i.e., decorrelating provides little benefit.

\begin{figure}[t]
    \begin{center}
    \subfigure[\texttt{train} loss]{\includegraphics[width=0.49\columnwidth]{experiments/vit_loss.pdf}}
    \subfigure[\texttt{test} accuracy]{\includegraphics[width=0.49\columnwidth]{experiments/vit_acc.pdf}}
    %
    \caption{ViT performances on CIFAR10.}
    \label{fig:vit}
    \end{center}
\end{figure}

However, whitening does include a centering step. If we were to shift the feature space, we would expect
to get same performance. In Fig \ref{fig:mnist}-b, we plot the same models when trained and tested on pixels
$1-x$, where $x$ are the original binary pixel values used in Fig \ref{fig:mnist}-a. We observe that LNB gets very similar performance
while Adam (and other methods not invariant to affine reparameterizations) degrade. Although we could (and should)
simply normalize the features before training, this example illustrates a case of how feature scaling can
greatly affect convergence.

\subsection{Vision Transformer}
We train a vision transformer \cite{vit} using the notebook from Equinox \cite{eqx}
on CIFAR10.
The only modification we make is to not learn the affine terms in the LayerNorm
in order to speed up experimentation and we observed no performance benefit with it.
The train and test fold performances are show in \Figref{fig:vit}, where we observe
faster convergence and better generalization with LNB. Excluding JIT compilation time,
the duration per epoch for LNB and Adam is 1.26 min and 0.85 min, respectively. 

\begin{figure}[t]
    \begin{center}
    \subfigure[\texttt{train} loss]{\includegraphics[width=0.49\columnwidth]{experiments/voc_loss.pdf}}
    \subfigure[\texttt{val} accuracy]{\includegraphics[width=0.49\columnwidth]{experiments/voc_acc.pdf}}
    %
    \caption{UNet performances on VOC Segmentation.}
    \label{fig:unet}
    \end{center}
\end{figure}

\subsection{UNet}
We train a UNet \cite{unet} on the 2012 VOC Segmentation Challenge dataset \cite{voc}. The images are
pixelwise normalized into the range $[0,1]$ using ImageNet mean and variance R,G,B pixel values
and then zero-padded to $500 \times 500$ size and then downsized to $384 \times 384$.
No data augmentation is performed. We plot the results in \Figref{fig:unet} and
remark that LNB converges very quickly, using the same learning rate as with ViT, and
avoids overfitting. While both optimizers converge to comparable performance on the
\texttt{val} split, the rapid progress by LNB suggests it would be able to leverage
more data effectively.
However, excluding JIT compilation time,
the duration per epoch for LNB and Adam is 2.92 min and 1.27 min, respectively, and
this highlights the trade-off between convergence w.r.t. iterations vs. wall time.
While LNB converged marginally faster in wall time and is significantly
easier to implement to prior equivalent work, it is future work to better understand
in what deep networks does the whitening behavior lead to better generalization as
demonstrated in the other three experiments.


\bibliography{lnb}

\end{document}
