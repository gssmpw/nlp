\documentclass{article}
\pdfoutput=1
%%%%% NEW MATH DEFINITIONS %%%%%

\usepackage{amsmath,amsfonts,bm}
\usepackage{derivative}
% Mark sections of captions for referring to divisions of figures
\newcommand{\figleft}{{\em (Left)}}
\newcommand{\figcenter}{{\em (Center)}}
\newcommand{\figright}{{\em (Right)}}
\newcommand{\figtop}{{\em (Top)}}
\newcommand{\figbottom}{{\em (Bottom)}}
\newcommand{\captiona}{{\em (a)}}
\newcommand{\captionb}{{\em (b)}}
\newcommand{\captionc}{{\em (c)}}
\newcommand{\captiond}{{\em (d)}}

% Highlight a newly defined term
\newcommand{\newterm}[1]{{\bf #1}}

% Derivative d 
\newcommand{\deriv}{{\mathrm{d}}}

% Figure reference, lower-case.
\def\figref#1{figure~\ref{#1}}
% Figure reference, capital. For start of sentence
\def\Figref#1{Figure~\ref{#1}}
\def\twofigref#1#2{figures \ref{#1} and \ref{#2}}
\def\quadfigref#1#2#3#4{figures \ref{#1}, \ref{#2}, \ref{#3} and \ref{#4}}
% Section reference, lower-case.
\def\secref#1{section~\ref{#1}}
% Section reference, capital.
\def\Secref#1{Section~\ref{#1}}
% Reference to two sections.
\def\twosecrefs#1#2{sections \ref{#1} and \ref{#2}}
% Reference to three sections.
\def\secrefs#1#2#3{sections \ref{#1}, \ref{#2} and \ref{#3}}
% Reference to an equation, lower-case.
\def\eqref#1{equation~\ref{#1}}
% Reference to an equation, upper case
\def\Eqref#1{Equation~\ref{#1}}
% A raw reference to an equation---avoid using if possible
\def\plaineqref#1{\ref{#1}}
% Reference to a chapter, lower-case.
\def\chapref#1{chapter~\ref{#1}}
% Reference to an equation, upper case.
\def\Chapref#1{Chapter~\ref{#1}}
% Reference to a range of chapters
\def\rangechapref#1#2{chapters\ref{#1}--\ref{#2}}
% Reference to an algorithm, lower-case.
\def\algref#1{algorithm~\ref{#1}}
% Reference to an algorithm, upper case.
\def\Algref#1{Algorithm~\ref{#1}}
\def\twoalgref#1#2{algorithms \ref{#1} and \ref{#2}}
\def\Twoalgref#1#2{Algorithms \ref{#1} and \ref{#2}}
% Reference to a part, lower case
\def\partref#1{part~\ref{#1}}
% Reference to a part, upper case
\def\Partref#1{Part~\ref{#1}}
\def\twopartref#1#2{parts \ref{#1} and \ref{#2}}

\def\ceil#1{\lceil #1 \rceil}
\def\floor#1{\lfloor #1 \rfloor}
\def\1{\bm{1}}
\newcommand{\train}{\mathcal{D}}
\newcommand{\valid}{\mathcal{D_{\mathrm{valid}}}}
\newcommand{\test}{\mathcal{D_{\mathrm{test}}}}

\def\eps{{\epsilon}}


% Random variables
\def\reta{{\textnormal{$\eta$}}}
\def\ra{{\textnormal{a}}}
\def\rb{{\textnormal{b}}}
\def\rc{{\textnormal{c}}}
\def\rd{{\textnormal{d}}}
\def\re{{\textnormal{e}}}
\def\rf{{\textnormal{f}}}
\def\rg{{\textnormal{g}}}
\def\rh{{\textnormal{h}}}
\def\ri{{\textnormal{i}}}
\def\rj{{\textnormal{j}}}
\def\rk{{\textnormal{k}}}
\def\rl{{\textnormal{l}}}
% rm is already a command, just don't name any random variables m
\def\rn{{\textnormal{n}}}
\def\ro{{\textnormal{o}}}
\def\rp{{\textnormal{p}}}
\def\rq{{\textnormal{q}}}
\def\rr{{\textnormal{r}}}
\def\rs{{\textnormal{s}}}
\def\rt{{\textnormal{t}}}
\def\ru{{\textnormal{u}}}
\def\rv{{\textnormal{v}}}
\def\rw{{\textnormal{w}}}
\def\rx{{\textnormal{x}}}
\def\ry{{\textnormal{y}}}
\def\rz{{\textnormal{z}}}

% Random vectors
\def\rvepsilon{{\mathbf{\epsilon}}}
\def\rvphi{{\mathbf{\phi}}}
\def\rvtheta{{\mathbf{\theta}}}
\def\rva{{\mathbf{a}}}
\def\rvb{{\mathbf{b}}}
\def\rvc{{\mathbf{c}}}
\def\rvd{{\mathbf{d}}}
\def\rve{{\mathbf{e}}}
\def\rvf{{\mathbf{f}}}
\def\rvg{{\mathbf{g}}}
\def\rvh{{\mathbf{h}}}
\def\rvu{{\mathbf{i}}}
\def\rvj{{\mathbf{j}}}
\def\rvk{{\mathbf{k}}}
\def\rvl{{\mathbf{l}}}
\def\rvm{{\mathbf{m}}}
\def\rvn{{\mathbf{n}}}
\def\rvo{{\mathbf{o}}}
\def\rvp{{\mathbf{p}}}
\def\rvq{{\mathbf{q}}}
\def\rvr{{\mathbf{r}}}
\def\rvs{{\mathbf{s}}}
\def\rvt{{\mathbf{t}}}
\def\rvu{{\mathbf{u}}}
\def\rvv{{\mathbf{v}}}
\def\rvw{{\mathbf{w}}}
\def\rvx{{\mathbf{x}}}
\def\rvy{{\mathbf{y}}}
\def\rvz{{\mathbf{z}}}

% Elements of random vectors
\def\erva{{\textnormal{a}}}
\def\ervb{{\textnormal{b}}}
\def\ervc{{\textnormal{c}}}
\def\ervd{{\textnormal{d}}}
\def\erve{{\textnormal{e}}}
\def\ervf{{\textnormal{f}}}
\def\ervg{{\textnormal{g}}}
\def\ervh{{\textnormal{h}}}
\def\ervi{{\textnormal{i}}}
\def\ervj{{\textnormal{j}}}
\def\ervk{{\textnormal{k}}}
\def\ervl{{\textnormal{l}}}
\def\ervm{{\textnormal{m}}}
\def\ervn{{\textnormal{n}}}
\def\ervo{{\textnormal{o}}}
\def\ervp{{\textnormal{p}}}
\def\ervq{{\textnormal{q}}}
\def\ervr{{\textnormal{r}}}
\def\ervs{{\textnormal{s}}}
\def\ervt{{\textnormal{t}}}
\def\ervu{{\textnormal{u}}}
\def\ervv{{\textnormal{v}}}
\def\ervw{{\textnormal{w}}}
\def\ervx{{\textnormal{x}}}
\def\ervy{{\textnormal{y}}}
\def\ervz{{\textnormal{z}}}

% Random matrices
\def\rmA{{\mathbf{A}}}
\def\rmB{{\mathbf{B}}}
\def\rmC{{\mathbf{C}}}
\def\rmD{{\mathbf{D}}}
\def\rmE{{\mathbf{E}}}
\def\rmF{{\mathbf{F}}}
\def\rmG{{\mathbf{G}}}
\def\rmH{{\mathbf{H}}}
\def\rmI{{\mathbf{I}}}
\def\rmJ{{\mathbf{J}}}
\def\rmK{{\mathbf{K}}}
\def\rmL{{\mathbf{L}}}
\def\rmM{{\mathbf{M}}}
\def\rmN{{\mathbf{N}}}
\def\rmO{{\mathbf{O}}}
\def\rmP{{\mathbf{P}}}
\def\rmQ{{\mathbf{Q}}}
\def\rmR{{\mathbf{R}}}
\def\rmS{{\mathbf{S}}}
\def\rmT{{\mathbf{T}}}
\def\rmU{{\mathbf{U}}}
\def\rmV{{\mathbf{V}}}
\def\rmW{{\mathbf{W}}}
\def\rmX{{\mathbf{X}}}
\def\rmY{{\mathbf{Y}}}
\def\rmZ{{\mathbf{Z}}}

% Elements of random matrices
\def\ermA{{\textnormal{A}}}
\def\ermB{{\textnormal{B}}}
\def\ermC{{\textnormal{C}}}
\def\ermD{{\textnormal{D}}}
\def\ermE{{\textnormal{E}}}
\def\ermF{{\textnormal{F}}}
\def\ermG{{\textnormal{G}}}
\def\ermH{{\textnormal{H}}}
\def\ermI{{\textnormal{I}}}
\def\ermJ{{\textnormal{J}}}
\def\ermK{{\textnormal{K}}}
\def\ermL{{\textnormal{L}}}
\def\ermM{{\textnormal{M}}}
\def\ermN{{\textnormal{N}}}
\def\ermO{{\textnormal{O}}}
\def\ermP{{\textnormal{P}}}
\def\ermQ{{\textnormal{Q}}}
\def\ermR{{\textnormal{R}}}
\def\ermS{{\textnormal{S}}}
\def\ermT{{\textnormal{T}}}
\def\ermU{{\textnormal{U}}}
\def\ermV{{\textnormal{V}}}
\def\ermW{{\textnormal{W}}}
\def\ermX{{\textnormal{X}}}
\def\ermY{{\textnormal{Y}}}
\def\ermZ{{\textnormal{Z}}}

% Vectors
\def\vzero{{\bm{0}}}
\def\vone{{\bm{1}}}
\def\vmu{{\bm{\mu}}}
\def\vtheta{{\bm{\theta}}}
\def\vphi{{\bm{\phi}}}
\def\va{{\bm{a}}}
\def\vb{{\bm{b}}}
\def\vc{{\bm{c}}}
\def\vd{{\bm{d}}}
\def\ve{{\bm{e}}}
\def\vf{{\bm{f}}}
\def\vg{{\bm{g}}}
\def\vh{{\bm{h}}}
\def\vi{{\bm{i}}}
\def\vj{{\bm{j}}}
\def\vk{{\bm{k}}}
\def\vl{{\bm{l}}}
\def\vm{{\bm{m}}}
\def\vn{{\bm{n}}}
\def\vo{{\bm{o}}}
\def\vp{{\bm{p}}}
\def\vq{{\bm{q}}}
\def\vr{{\bm{r}}}
\def\vs{{\bm{s}}}
\def\vt{{\bm{t}}}
\def\vu{{\bm{u}}}
\def\vv{{\bm{v}}}
\def\vw{{\bm{w}}}
\def\vx{{\bm{x}}}
\def\vy{{\bm{y}}}
\def\vz{{\bm{z}}}

% Elements of vectors
\def\evalpha{{\alpha}}
\def\evbeta{{\beta}}
\def\evepsilon{{\epsilon}}
\def\evlambda{{\lambda}}
\def\evomega{{\omega}}
\def\evmu{{\mu}}
\def\evpsi{{\psi}}
\def\evsigma{{\sigma}}
\def\evtheta{{\theta}}
\def\eva{{a}}
\def\evb{{b}}
\def\evc{{c}}
\def\evd{{d}}
\def\eve{{e}}
\def\evf{{f}}
\def\evg{{g}}
\def\evh{{h}}
\def\evi{{i}}
\def\evj{{j}}
\def\evk{{k}}
\def\evl{{l}}
\def\evm{{m}}
\def\evn{{n}}
\def\evo{{o}}
\def\evp{{p}}
\def\evq{{q}}
\def\evr{{r}}
\def\evs{{s}}
\def\evt{{t}}
\def\evu{{u}}
\def\evv{{v}}
\def\evw{{w}}
\def\evx{{x}}
\def\evy{{y}}
\def\evz{{z}}

% Matrix
\def\mA{{\bm{A}}}
\def\mB{{\bm{B}}}
\def\mC{{\bm{C}}}
\def\mD{{\bm{D}}}
\def\mE{{\bm{E}}}
\def\mF{{\bm{F}}}
\def\mG{{\bm{G}}}
\def\mH{{\bm{H}}}
\def\mI{{\bm{I}}}
\def\mJ{{\bm{J}}}
\def\mK{{\bm{K}}}
\def\mL{{\bm{L}}}
\def\mM{{\bm{M}}}
\def\mN{{\bm{N}}}
\def\mO{{\bm{O}}}
\def\mP{{\bm{P}}}
\def\mQ{{\bm{Q}}}
\def\mR{{\bm{R}}}
\def\mS{{\bm{S}}}
\def\mT{{\bm{T}}}
\def\mU{{\bm{U}}}
\def\mV{{\bm{V}}}
\def\mW{{\bm{W}}}
\def\mX{{\bm{X}}}
\def\mY{{\bm{Y}}}
\def\mZ{{\bm{Z}}}
\def\mBeta{{\bm{\beta}}}
\def\mPhi{{\bm{\Phi}}}
\def\mLambda{{\bm{\Lambda}}}
\def\mSigma{{\bm{\Sigma}}}

% Tensor
\DeclareMathAlphabet{\mathsfit}{\encodingdefault}{\sfdefault}{m}{sl}
\SetMathAlphabet{\mathsfit}{bold}{\encodingdefault}{\sfdefault}{bx}{n}
\newcommand{\tens}[1]{\bm{\mathsfit{#1}}}
\def\tA{{\tens{A}}}
\def\tB{{\tens{B}}}
\def\tC{{\tens{C}}}
\def\tD{{\tens{D}}}
\def\tE{{\tens{E}}}
\def\tF{{\tens{F}}}
\def\tG{{\tens{G}}}
\def\tH{{\tens{H}}}
\def\tI{{\tens{I}}}
\def\tJ{{\tens{J}}}
\def\tK{{\tens{K}}}
\def\tL{{\tens{L}}}
\def\tM{{\tens{M}}}
\def\tN{{\tens{N}}}
\def\tO{{\tens{O}}}
\def\tP{{\tens{P}}}
\def\tQ{{\tens{Q}}}
\def\tR{{\tens{R}}}
\def\tS{{\tens{S}}}
\def\tT{{\tens{T}}}
\def\tU{{\tens{U}}}
\def\tV{{\tens{V}}}
\def\tW{{\tens{W}}}
\def\tX{{\tens{X}}}
\def\tY{{\tens{Y}}}
\def\tZ{{\tens{Z}}}


% Graph
\def\gA{{\mathcal{A}}}
\def\gB{{\mathcal{B}}}
\def\gC{{\mathcal{C}}}
\def\gD{{\mathcal{D}}}
\def\gE{{\mathcal{E}}}
\def\gF{{\mathcal{F}}}
\def\gG{{\mathcal{G}}}
\def\gH{{\mathcal{H}}}
\def\gI{{\mathcal{I}}}
\def\gJ{{\mathcal{J}}}
\def\gK{{\mathcal{K}}}
\def\gL{{\mathcal{L}}}
\def\gM{{\mathcal{M}}}
\def\gN{{\mathcal{N}}}
\def\gO{{\mathcal{O}}}
\def\gP{{\mathcal{P}}}
\def\gQ{{\mathcal{Q}}}
\def\gR{{\mathcal{R}}}
\def\gS{{\mathcal{S}}}
\def\gT{{\mathcal{T}}}
\def\gU{{\mathcal{U}}}
\def\gV{{\mathcal{V}}}
\def\gW{{\mathcal{W}}}
\def\gX{{\mathcal{X}}}
\def\gY{{\mathcal{Y}}}
\def\gZ{{\mathcal{Z}}}

% Sets
\def\sA{{\mathbb{A}}}
\def\sB{{\mathbb{B}}}
\def\sC{{\mathbb{C}}}
\def\sD{{\mathbb{D}}}
% Don't use a set called E, because this would be the same as our symbol
% for expectation.
\def\sF{{\mathbb{F}}}
\def\sG{{\mathbb{G}}}
\def\sH{{\mathbb{H}}}
\def\sI{{\mathbb{I}}}
\def\sJ{{\mathbb{J}}}
\def\sK{{\mathbb{K}}}
\def\sL{{\mathbb{L}}}
\def\sM{{\mathbb{M}}}
\def\sN{{\mathbb{N}}}
\def\sO{{\mathbb{O}}}
\def\sP{{\mathbb{P}}}
\def\sQ{{\mathbb{Q}}}
\def\sR{{\mathbb{R}}}
\def\sS{{\mathbb{S}}}
\def\sT{{\mathbb{T}}}
\def\sU{{\mathbb{U}}}
\def\sV{{\mathbb{V}}}
\def\sW{{\mathbb{W}}}
\def\sX{{\mathbb{X}}}
\def\sY{{\mathbb{Y}}}
\def\sZ{{\mathbb{Z}}}

% Entries of a matrix
\def\emLambda{{\Lambda}}
\def\emA{{A}}
\def\emB{{B}}
\def\emC{{C}}
\def\emD{{D}}
\def\emE{{E}}
\def\emF{{F}}
\def\emG{{G}}
\def\emH{{H}}
\def\emI{{I}}
\def\emJ{{J}}
\def\emK{{K}}
\def\emL{{L}}
\def\emM{{M}}
\def\emN{{N}}
\def\emO{{O}}
\def\emP{{P}}
\def\emQ{{Q}}
\def\emR{{R}}
\def\emS{{S}}
\def\emT{{T}}
\def\emU{{U}}
\def\emV{{V}}
\def\emW{{W}}
\def\emX{{X}}
\def\emY{{Y}}
\def\emZ{{Z}}
\def\emSigma{{\Sigma}}

% entries of a tensor
% Same font as tensor, without \bm wrapper
\newcommand{\etens}[1]{\mathsfit{#1}}
\def\etLambda{{\etens{\Lambda}}}
\def\etA{{\etens{A}}}
\def\etB{{\etens{B}}}
\def\etC{{\etens{C}}}
\def\etD{{\etens{D}}}
\def\etE{{\etens{E}}}
\def\etF{{\etens{F}}}
\def\etG{{\etens{G}}}
\def\etH{{\etens{H}}}
\def\etI{{\etens{I}}}
\def\etJ{{\etens{J}}}
\def\etK{{\etens{K}}}
\def\etL{{\etens{L}}}
\def\etM{{\etens{M}}}
\def\etN{{\etens{N}}}
\def\etO{{\etens{O}}}
\def\etP{{\etens{P}}}
\def\etQ{{\etens{Q}}}
\def\etR{{\etens{R}}}
\def\etS{{\etens{S}}}
\def\etT{{\etens{T}}}
\def\etU{{\etens{U}}}
\def\etV{{\etens{V}}}
\def\etW{{\etens{W}}}
\def\etX{{\etens{X}}}
\def\etY{{\etens{Y}}}
\def\etZ{{\etens{Z}}}

% The true underlying data generating distribution
\newcommand{\pdata}{p_{\rm{data}}}
\newcommand{\ptarget}{p_{\rm{target}}}
\newcommand{\pprior}{p_{\rm{prior}}}
\newcommand{\pbase}{p_{\rm{base}}}
\newcommand{\pref}{p_{\rm{ref}}}

% The empirical distribution defined by the training set
\newcommand{\ptrain}{\hat{p}_{\rm{data}}}
\newcommand{\Ptrain}{\hat{P}_{\rm{data}}}
% The model distribution
\newcommand{\pmodel}{p_{\rm{model}}}
\newcommand{\Pmodel}{P_{\rm{model}}}
\newcommand{\ptildemodel}{\tilde{p}_{\rm{model}}}
% Stochastic autoencoder distributions
\newcommand{\pencode}{p_{\rm{encoder}}}
\newcommand{\pdecode}{p_{\rm{decoder}}}
\newcommand{\precons}{p_{\rm{reconstruct}}}

\newcommand{\laplace}{\mathrm{Laplace}} % Laplace distribution

\newcommand{\E}{\mathbb{E}}
\newcommand{\Ls}{\mathcal{L}}
\newcommand{\R}{\mathbb{R}}
\newcommand{\emp}{\tilde{p}}
\newcommand{\lr}{\alpha}
\newcommand{\reg}{\lambda}
\newcommand{\rect}{\mathrm{rectifier}}
\newcommand{\softmax}{\mathrm{softmax}}
\newcommand{\sigmoid}{\sigma}
\newcommand{\softplus}{\zeta}
\newcommand{\KL}{D_{\mathrm{KL}}}
\newcommand{\Var}{\mathrm{Var}}
\newcommand{\standarderror}{\mathrm{SE}}
\newcommand{\Cov}{\mathrm{Cov}}
% Wolfram Mathworld says $L^2$ is for function spaces and $\ell^2$ is for vectors
% But then they seem to use $L^2$ for vectors throughout the site, and so does
% wikipedia.
\newcommand{\normlzero}{L^0}
\newcommand{\normlone}{L^1}
\newcommand{\normltwo}{L^2}
\newcommand{\normlp}{L^p}
\newcommand{\normmax}{L^\infty}

\newcommand{\parents}{Pa} % See usage in notation.tex. Chosen to match Daphne's book.

\DeclareMathOperator*{\argmax}{arg\,max}
\DeclareMathOperator*{\argmin}{arg\,min}

\DeclareMathOperator{\sign}{sign}
\DeclareMathOperator{\Tr}{Tr}
\let\ab\allowbreak


\usepackage{algorithm}
\usepackage{algorithmic}
\usepackage{natbib}
\usepackage{microtype}
\usepackage{subfigure}

\usepackage{hyperref}

\usepackage{arxiv}

\usepackage{amsmath}
\usepackage{amssymb}
\usepackage{mathtools}
\usepackage{amsthm}

\usepackage{bbm}
\usepackage{url}
\usepackage{hyperref}

\title{Simple Linear Neuron Boosting}

\author{Daniel Munoz \\
        Independent Researcher \\
	\texttt{lnb@dmunoz.org}
}

\begin{document}
\maketitle

\begin{abstract}
Given a differentiable network architecture and loss function, we revisit optimizing the network's
neurons in function space using Boosted Backpropagation \citep{grub2010}, in contrast to optimizing
in parameter space. From this perspective, we reduce descent in the space of linear functions that optimizes
the network's backpropagated-errors to a preconditioned gradient descent algorithm. We show that this
preconditioned update rule is equivalent to reparameterizing the network to whiten each neuron's features,
with the benefit that the normalization occurs outside of inference. In practice, we use this equivalence to
construct an online estimator for approximating the preconditioner and we propose an online, matrix-free
learning algorithm with adaptive step sizes. The algorithm is applicable whenever autodifferentiation is
available, including convolutional networks and transformers, and it is simple to implement for both the
local and distributed training settings. We demonstrate fast convergence both in terms of epochs and wall
clock time on a variety of tasks and networks.
\end{abstract}

\IEEEPARstart{L}everaging advanced algorithms and neural network architectures like Transformers~\cite{vaswani2023attentionneed}, AI has been empowered with strong reasoning ability and made tremendous progress in recent years. Breakthroughs in model design and training methodologies have allowed machines to excel in complex tasks, including Natural Language Processing (NLP) applications such as language translation, sentiment analysis, and text generation, achieving high accuracy and fostering intuitive human-computer interactions. Similarly, advancements in Computer Vision (CV) have empowered AI to analyze and interpret images, videos, and audio sequences with remarkable precision. In healthcare, Artificial Intelligence (AI) is revolutionizing medicine by enabling data-driven insights, improving diagnostics, and personalizing treatments~\cite{topol2019,Esteva2017,KOUROU20158}. These innovations have enabled significant applications, such as medical imaging analysis, disease diagnosis, pathology, radiology workflow optimization, and surgical assistance, transforming patient care and clinical workflows~\cite{empeek2024,pmc2021}.


The medical field faces unique challenges in data interpretation and decision-making for healthcare specialists; they must analyze diverse types of information including medical imaging (X-rays, MRIs, pathology slides), clinical notes, patient histories, and real-time observations. Medical images are critical for diagnostic checks and measurements, such as identifying anatomical abnormalities, quantifying disease progression, or assessing treatment efficacy. On the other hand, textual data, such as clinical notes, nurse evaluations, and patient histories, provide essential context for screening, understanding symptoms, and documenting disease progression. Textual outputs, such as radiology reports or discharge summaries, are equally vital, as they synthesize findings into actionable insights for clinicians. The complexity and volume of this multi-modal medical data often lead to cognitive overload, impacting the speed and accuracy of diagnoses. Traditional single-modality approaches, which treat images and text separately, fail to capture the intricate relationships between visual findings and clinical context. This limitation underscores the need for integrated vision-language models (VLMs) that can bridge the gap between these modalities\cite{bordes2024introductionvisionlanguagemodeling}, enabling more comprehensive and accurate decision-making in healthcare. This integrated approach promises to enhance clinical decision-making by providing more contextually informed insights and reducing the cognitive burden on healthcare providers.

\begin{figure*}[ht]
    \centering
    \includegraphics[width=\linewidth]{images/methods2.png}
    \caption{\textbf{Comprehensive Framework for Medical Vision-Language Models (VLMs)}. \textbf{(a)} Training involves processing diverse inputs such as images, texts, metadata, and historical data, followed by pre-training. \textbf{(b)} Benchmarking is conducted on a variety of medical datasets including GMAI-MMBench, OmniMedVQA, RadBench, and others. \textbf{(c)} Advanced training strategies are employed, such as vision-text alignment, knowledge distillation, masked language modeling, contrastive learning, and parameter-efficient tuning. \textbf{(d)} Evaluation strategies encompass automated metrics like BLEU, ROUGE, BERTScore, and clinical-specific tools like CheXpert Labeler and RadGraph, alongside human evaluation. \textbf{(e)} Integration of VLMs into the medical workflow leverages contextual data to provide actionable insights and improve clinical decision-making.}
    \label{fig:method}
\end{figure*}

However, visual and language provide totally different modalities that are not trivial to be integrated directly. As illustrated in Fig.~\ref{fig:method}, existing works address this challenge through various strategies, including vision-text alignment (in MedViL\cite{devlin2019bertpretrainingdeepbidirectional}, MedCLIP\cite{radford2021learningtransferablevisualmodels}, BioMedCLIP\cite{zhang2024biomedclipmultimodalbiomedicalfoundation}, VividMed\cite{luo2024vividmedvisionlanguagemodel}), knowledge distillation with VividMed\cite{luo2024vividmedvisionlanguagemodel}, masked language modeling (in MedViL\cite{devlin2019bertpretrainingdeepbidirectional} and BioMedCLIP\cite{zhang2024biomedclipmultimodalbiomedicalfoundation}) and contrastive learning in MedCLIP\cite{radford2021learningtransferablevisualmodels}, BioViL\cite{Boecking2022} \& ConVIRT\cite{zhang2022contrastivelearningmedicalvisual}. More recent advancements have introduced additional approaches, such as frozen encoders and Q-Former (e.g., BLIP-2\cite{li2023blip2bootstrappinglanguageimagepretraining}, InstructBLIP\cite{dai2023instructblipgeneralpurposevisionlanguagemodels}), image-text pair learning and fine-tuning (e.g., LLaVA\cite{liu2023visualinstructiontuning}, LLaVA-Med\cite{li2023llavamedtraininglargelanguageandvision}, BiomedGPT\cite{Zhang2024}, MedVInT\cite{zhang2024pmcvqavisualinstructiontuning}), parameter-efficient tuning (e.g., LLaMA-Adapter-V2\cite{zhang2024llamaadapterefficientfinetuninglanguage}), two-stage training (e.g., MiniGPT-4\cite{zhu2023minigpt4enhancingvisionlanguageunderstanding}), and modular multimodal pre-training (e.g., mPLUG-Owl\cite{ye2024mplugowlmodularizationempowerslarge}, Otter\cite{li2023ottermultimodalmodelincontext}) and the Sigmoid Loss for Language-Image Pre-Training (SigLIP)\cite{zhai2023sigmoidlosslanguageimage}. SigLIP replaces traditional softmax-based contrastive learning with a simpler sigmoid loss approach, enabling more efficient and scalable training by treating image-text pair alignment as a binary classification task. These methods aim to establish coherent relationships between visual inputs and textual outputs, enabling models to effectively interpret and generate relevant information across modalities. As a comparison, the contrastive learning-based methods leverage the similarities and differences between paired visual and textual data to enhance model robustness and generalization.

This paper provides a comprehensive review of VLMs and their applications in healthcare. We first discuss how VLMs are constructed by integrating advancements in NLP and computer vision. Next, we summarize key methodologies and advancements in the field, including state-of-the-art models like Qwen-VL\cite{bai2023qwenvlversatilevisionlanguagemodel}, RadFM\cite{wu2023generalistfoundationmodelradiology}, and DeepSeek-VL\cite{lu2024deepseekvlrealworldvisionlanguageunderstanding}. We then explore how VLMs are applied in the medical domain, highlighting their potential to improve diagnostic accuracy, clinical decision-making, and other healthcare tasks. Finally, we conclude by outlining future directions and challenges in the integration of VLMs into healthcare practices.

\section{Notation and Background}
We denote the derivative of a function $f: \gU \to \gV$
at $v = f(u)$ as the linear map $\frac{\partial f(u)}{\partial u}: \gU \to \gV$,
or equivalently as $\frac{\partial v}{\partial u}$.
In general, $u$ can represent a high dimensional tensor,
such as a feature map or the filter parameters in a convolutional neural network (CNN).
Similarly, we denote the transpose of the derivative as
$\frac{\partial v}{\partial u}^T: \gV \to \gU$.
Throughout this work we'll operate in vector spaces;
when $f$ is a scalar function, we interchangeably denote
the gradient vector at $u$, $\nabla f(u) \in \gU$, as the transpose of its derivative,
$\frac{\partial v}{\partial u}^T  \in \gU$.

\subsection{Preconditioned Gradient Descent}
\label{sec:pcgd}
We denote $\ell_{(x,y)}: \Theta \to \sR$ as the pointwise loss function for a labeled sample,
where $x$ and $y$ are the inputs and labels, respectively, and the 
samples are drawn from a data distribution, $\gD$.
Minimizing the expected loss, 
$\bar{\ell}(\theta) = \E_{(x,y) \sim \gD} [\ell_{(x,y)}(\theta)]$,
via gradient descent leads to the update rule
$\theta \leftarrow \theta - \alpha^{(t)} \nabla \bar{\ell}(\theta)$,
where $\alpha^{(t)}$ is the desired step size at iteration $t$.
Depending on the form of $\ell$, convergence can be improved by preconditioning
the gradient vector by a matrix $P$, \eg, using the inverse of the Hessian of $\bar\ell(\theta)$ when performing
Newton's method.
In general, when the preconditioner can be factored as $P=WW^T$, the update rule
of following the preconditioned gradient vector, $WW^T \nabla \bar{\ell}(\theta)$, is equivalent to
performing gradient descent on $\bar\ell(\acute\theta)$ with the reparameterized model, $\acute\theta = W\theta$.

%

\subsection{Natural Gradient Descent}
\label{sec:cgrad}
One way to view natural gradient descent \cite{amari1998} is as
a trust-region optimization problem to find a small step, \wrt some norm,
that is aligned in the direction of $\nabla\bar\ell(\theta)$:
%
\begin{align}
\argmin_{\delta \theta} & ~ \bar{\ell}(\theta) + \nabla \bar{\ell}(\theta) \cdot \delta\theta \nonumber\\
\text{s.t.} & ~ \frac{1}{2} \langle \delta\theta, \delta\theta\rangle_{M_{\theta}}  = \eps, \label{eq:trust}
\end{align}
%
where $\langle u, v\rangle_{M_{\theta}} > 0$ is the inner product
defined by the (Riemannian) metric $M_{\theta}$ in the local tangent space at $\theta$.
%
Solving for the stationary point of its Lagrange function results in an update step
$\delta \theta \propto M_{\theta}^{-1} \nabla \bar{\ell}(\theta)$.
When defining $M_\theta$ as the Fisher Information Matrix (FIM), the resulting
vector is referred to as the natural gradient vector; see \citet{kunster2019} for an informative review and discussion.
However, in general, we are free to choose any positive-definite metric, $M_{\theta} \succ 0$,
and we denote the resulting vector under the metric as $\nabla_{M} \ell(\theta) = M_{\theta}^{-1} \nabla \ell(\theta)$.
%
%
Note that taking a step solely proportional $\nabla_{M}\bar{\ell}(\theta)$
ignores the trust region constraint of \Eqref{eq:trust};
solving for it obtains the appropriate step size, $\alpha_{\theta}$,
in the original parameter space that induces an  $\eps$-sized step under the chosen metric:
$\alpha_{\theta} = \sqrt{\frac{\eps}{z_\theta}}$, where
%
\begin{equation}
\label{eq:metricnorm}
z_\theta
= \langle \nabla_{M} \bar{\ell}(\theta), \nabla_{M} \bar{\ell}(\theta) \rangle_{M_{\theta}}
= \nabla_{M} \bar{\ell}(\theta) \cdot \nabla \bar{\ell}(\theta).
\end{equation}
%
%
%
%
%

%
%
%
%
%

\subsection{Functional Gradient Descent}
\label{sec:fgrad}
In contrast to optimizing a function in the space of parameters,
an alternative is to optimize in the space of functions, $\gF$, which
can also be viewed as boosting \citep{mason2000, friedman2001}.
The following summarizes \S2 of \citet{grubbthesis} to introduce notation on this topic.
We denote $\bar{\gL}[f] = \E_{(x,y) \sim \gD}[l_{y}(f(x))]$ to be the loss functional
that computes the expected loss for a given function $f: \gX \to \gV$,
where $l_y: \gV \to \sR$ is loss function (using label $y$) in the image (output space) of $f$,
and $\gV$ is an application-specific vector
space\footnote{In general, each element in $\gV$ need not be a 1-D vector and each could
be a multi-dimensional tensor. 
For example, in $k$-class logistic regression, the image of $f$ is the predicted logits vector
($\gV = \sR^k$) and $l$ is pointwise log-loss.
Whereas for $k$-class semantic image segmentation, each the image of $f$ is a tensor
($\gV = \sR^{h \times w \times k}$) whose outer component is the per-pixel logits 
and $l$ is the per-pixel log-loss summed over the inner spatial components.}.
Letting $v_x = f(x)$, the functional gradient ``vector'' of $\bar{\gL}$ at $f$ is defined as the
\emph{function},
%
\begin{equation*}
\nabla_F \bar{\gL}[f] = \E_{(x,y) \sim \gD} [\nabla_F l_y(v_x)] = \E_{(x,y) \sim \gD} [ \lambda_{(x,y)} \mathbbm{1}_x],
\end{equation*}
%
where each $\lambda_{(x,y)} = \frac{\partial l_y(v_x)}{\partial v_x}^T \in \gV$
is a gradient vector, and $\mathbbm{1}_x: \gX \to \{0, 1\}$ is the Dirac delta function centered at $x$.
For brevity, let $\triangle_f = \nabla_F \tilde{\gL}[f]$.

Instead of the defining the strong-learner, $f$, additively in $\triangle_f \in \gF$,
each $\triangle_f$ is projected onto a smaller hypothesis space, $\gH \subseteq \gF$,
such as small decision trees or MLPs, in order to generalize.
The projection of $\triangle_f$ onto a hypothesis $h \in \gH$ is analogous to vector projection in Euclidean space:
$\frac{\langle \triangle_f, h \rangle_F}{\Vert h \Vert_F}\frac{h}{\Vert h \Vert_F}$,
where $\langle f,g \rangle_F = \mathbb{E}_x[f(x) \cdot g(x)]$ is the inner product in function
space with norm $\Vert h \Vert_F = \sqrt{\langle h,h \rangle_F}$.

When $\gH$ are regressors, the hypothesis, $\hat{h}$, that maximizes the scalar projection term is
equivalent \citep{friedman2001} to minimizing a least squares problem,
%
\begin{equation}
\label{eq:scalarproj}
\hat{h} = \argmax_{h \in \gH} \frac{\langle \triangle_f, h \rangle_F}{\Vert h \Vert_F}
= \argmin_{h \in \gH} \E_{x} [ \Vert h(x) -\triangle_f(x) \Vert^2 ].
\end{equation}
%
In practice, this translates to training a (vector-output) regressor
over the dataset $\{(x,\lambda_{(x,y)})\}$.
Finally, this leads to the update rule
$f \leftarrow f - \eps^{(t)} \frac{\hat{h}}{\Vert \hat{h} \Vert_F}$,
and we refer to each $\hat{h}$ as weak-learners.

\subsubsection{Regularization}
\label{sec:reg}
In addition to minimizing the incurred loss of $f$, we may also want to include
a regularization term in the objective $\tilde{\gL}[f] + \frac{\rho}{2} \Vert f \Vert_F^2$, where 
$\rho \geq 0$. The update rule with the regularized objective is
%
\begin{equation}
\label{eq:reg}
f \gets (1-\eps^{(t)}\rho) f -  \eps^{(t)} \frac{\hat{h}}{\Vert \hat{h} \Vert_F}.
\end{equation}
%
In practice, this is implemented by shrinking the existing weak-learner coefficients $\{\eps^{(j)}|j<t\}$
by the factor $0 \leq (1-\eps^{(t)}\rho) \leq 1$.

\subsection{Boosted Backpropagation}
\label{sec:bbp}

For simplicity in explanation, we begin with the problem of training a feed-forward
network (FFN), as done in \citet{grub2010}, and note that general architectures
will be discussed in the next section.
We represent a FFN as a composition of $m$ differentiable functions
$F = \{f_i:\gX_{i-1} \to \gX_i| 1 \leq i \leq m \}$, where a subset $F_B \subseteq F$
are neurons with parameters $\theta_i$ that we want to train, \eg,
$f_i(\cdot;\theta_i) \in F_B$ could be convolutional layer with filters $\theta_i$.
We denote $\theta_B$ as all the trainable parameters in the network.
The complement set of functions, $F \setminus F_B$, are fixed, \eg, activation functions, resizing layers, etc.
We denote the loss incurred for sample $(x,y)$ for the given network parameters
as $\ell_{(x,y)}(\theta_B) = (l_y \circ f_m \circ \ldots \circ f_1)(x)$,
where $l_y: \gX_m \to \sR$ computes the loss of the last layer's prediction vector with label $y$.
Backpropagation can be used to compute the gradient vector of the
expected loss, $\nabla \bar\ell(\theta_B)$,
and we denote $g_i$ to be the component of $\nabla \bar\ell(\theta_B)$
that corresponds to $\theta_i$, i.e., $g_i = \frac{\partial\bar\ell(\theta_B)}{\partial \theta_i}^T$.

Forgoing parameterizing the FFN neurons with weights,
\citet{grub2010} optimize the loss \wrt each neuron's \emph{function}
by using the adjoint state method for
the equivalent constrained minimization problem,
%
%
%
\begin{align}
\argmin_{F_B} ~& \E_{(x,y) \sim \gD} [l_y(x_{m})] \nonumber \\
\begin{split}
\label{eq:forward}
\text{s.t.} ~& x_0 = x, x_{i} = f_{i}(x_{i-1}), i \in [1, \ldots, m].
\end{split}
\end{align}
%
Using the necessary conditions for a stationary point of its Lagrangian,
they describe an algorithm to recursively compute the functional gradient vector for each neuron.
For $f_i \in F_B$ in a FFN, its functional gradient vector is $\lambda_i \mathbbm{1}_{x_{i-1}}$,
where
%
\begin{align}
\lambda_{m+1} &= \frac{\partial l_y(x_m)}{\partial x_m}^T, \nonumber \\
\lambda_{i} &= \frac{\partial x_i}{\partial x_{i-1}}^T \lambda_{i+1},  i \in [1, \ldots, m],
\label{eq:costates}
\end{align}
%
are the errors backpropagated from the loss layer via vector-Jacobian products (VJPs).
That it is, each training step is analogous to performing normal backpropagation
with the key difference of training a weak-learner using the dataset 
$\{(x_{i-1}, \lambda_i)\}$ for each $f_i \in F_B$,
instead of pulling-back the targets, $\lambda_i$, through respective neuron's local derivative.
That is, $f_i$'s component of $\nabla \bar\ell(\theta_B)$ is
$g_i = \frac{\partial x_i}{\partial \theta_i}^T \lambda_i$.

As presented, the time complexity for computing the unprojected
functional gradient vectors is the same backpropagation; however,
the storage complexity increases linearly with the dataset and dimensionality
due to aggregating the regression targets.
The projection operation then adds significant compute as it requires solving a
large vector regression problem for each neuron\footnote{\eg,
for a convolutional layer, each $\lambda_i$ represents a
$\sR^{h \times w \times d}$ feature map and there are $|\gD|$ of them.}.
Lastly, the presented algorithm is specific to a FFN and it is left
as an exercise to construct the corresponding recurrence relation of \Eqref{eq:costates}
for different network architectures, which can be error prone.
In the next section we address these concerns and provide a simple and efficient boosting algorithm
for any network architecture compatible with autodifferentiation.

\section{Boosting over Linear Neurons}
\label{sec:lnb}

\subsection{Reduction}
For training general architectures we can forgo computing the 
regression targets, $\lambda_i$. Instead, we use
its definition that it is the gradient vector of the loss
\wrt the \emph{output} of $f_i$,
%
\begin{equation}
\label{eq:lambda}
\lambda_i^T
= \frac{\partial l_{y}(x_m)}{\partial f_i(x_{i-1})}
= \frac{\partial l_{y}(x_m)}{\partial x_i},
\end{equation}
%
where $x_{i-1} \in \gX_{i-1}$ are the forward-propagated \emph{input} features to the neuron $f_i$.
Note that $x_{i-1}$ and $\lambda_i$ both depend on the sample $(x,y)$ and current
network definition, $F_B$.

When projecting the functional gradient vector for $f_i$
over a linear hypothesis space, the weak-learners have the same
function definition as the neuron but differ in the parameters.
The scalar projection step (\Eqref{eq:scalarproj}) then corresponds to
solving the ordinary least squares (OLS) problem
%
\begin{equation}
\label{eq:OLS}
\hat{\theta}_i = \argmin_{\theta} \frac{1}{2} \E_{(x,y) \sim \gD} 
\Vert f_i(x_{i-1};\theta) - \lambda_i \Vert^2.
\end{equation}
%
Because this is an OLS problem, we know its solution
is the projection of the targets, $\{\lambda_i\}$,
onto the column space of the forward-propagated input features to $f_i$;
we denote this set of input features as $X_i = \{x_{i-1}\}$.

Using \Eqref{eq:lambda} and the definition
$x_i = f_i(x_{i-1};\theta)
= \frac{\partial x_i}{\partial \theta_i}\theta$
(because $f_i$ is linear),
we can rewrite \Eqref{eq:OLS} as
%
\begin{equation}
\label{eq:OLS2}
\hat{\theta}_i = \argmin_{\theta} \frac{1}{2} \E_{(x,y) \sim \gD}
\biggl \Vert \frac{\partial x_i}{\partial \theta_i}\theta - \frac{\partial l_{y}(x_m)}{\partial x_i}^T \biggr \Vert^2,
\end{equation}
%
which has the corresponding Normal Equation
%
\begin{align}
\E_{x_{i-1} \sim X_i} \biggl [ \frac{\partial x_i}{\partial \theta_i}^T\frac{\partial x_i}{\partial \theta_i} \biggr ] \hat{\theta}_i 
= \E_{(x,y) \sim \gD} \biggl [ \biggl (\frac{\partial l_{y}(x_m)}{\partial x_i} \frac{\partial x_i}{\partial \theta_i} \biggr )^T \biggr ].
\label{eq:normal}
\end{align}
%
The right-hand side is the component of $\nabla \bar\ell(\theta_B)$ corresponding to $\theta_i$, so the solution is
$\hat{\theta}_i = M_i^{-1}g_i$, where
 $M_i = \E_{x_{i-1} \sim X_i} [ \frac{\partial x_i}{\partial \theta_i}^T  \frac{\partial x_i}{\partial \theta_i}]$.
We can confirm that the norm in function space is equivalent to the
inner product under $M_i$ of \Eqref{eq:metricnorm}:
%
$$
\Vert f_i(\cdot; \hat{\theta}_i) \Vert^2_F
= \E \biggl [\frac{\partial x_i}{\partial \theta_i} \hat{\theta}_i \cdot \frac{\partial x_i}{\partial \theta_i} \hat{\theta}_i \biggr ]
= \hat{\theta}_i \cdot M_i \hat{\theta}_i
= \hat{\theta}_i \cdot g_i.
$$
%

Because the OLS problems are solved independently per neuron,
the final solution is equivalent to as if we constructed a block-diagonal metric, $M$,
containing each $M_i$ along the diagonal and then computing the preconditioned gradient
vector from \S \ref{sec:cgrad}. We denote this vector as $\hat{\theta}_B = \nabla_M \bar\ell(\theta_B)$,
which is composed of each neuron's $\hat{\theta}_i$ solution.
Note that the structure of the network is only needed to compute $\nabla \bar\ell(\theta_B)$ (via
autodifferentiation) and we can then solve \Eqref{eq:normal} for each neuron independently.

\subsection{Optimization}
To solve for $\hat{\theta}_i$ in \Eqref{eq:normal},
we can leverage any linear system solver that internally
uses matrix-vector-products (MVP) instead of instantiating
each metric, $M_i$.
First, we need to save the set of forward-propagated input features, $X_i$, to the neuron; we note
that these features are already computed and saved in
memory during backpropagation.
In practice, the MVP $M_i \theta$ can then be computed via a
VJP composed with a Jacobian-vector-product (JVP) of
the vectorized function $\mathbf{f}_{X_i}(\theta_i) = \text{vec}(\{f_i(x_{i-1};\theta_i)\}_{x_{i-1} \in X_i})$,
which maps $f_i(\cdot; \theta_i)$ over  every element (tensor) in dataset $X_i$, and then
%
\begin{equation}
\label{eq:mvp}
M_i(\theta) = 
\frac{1}{n_i} \frac{\partial \mathbf{f}_{X_i}}{\partial \theta_i}^T \frac{\partial \mathbf{f}_{X_i}}{\partial \theta_i} \theta,
\end{equation}
%
where $n_i$ is the number of samples that contribute to the
gradient\footnote{In general this value can be computed using the shape of $\gX_i$,
\eg, if $f_i(x;\theta) = \theta^Tx$, then $n_i$ is
the batch size; if $f_i$ is a strided convolution, then $n_i$ 
is the number of output pixels in the batch.}.
Note that since $\mathbf{f}_{X_i}$ is linear it need only be
linearized once (at any point) by the linear system solver.

After solving \Eqref{eq:normal} for each neuron, we can construct $\hat{\theta}_B$ and
compute the $\eps$-sized step under the full metric $M$ via \Eqref{eq:metricnorm}.
Lastly, we may wish to regularize our strong-learner in function space as discussed in 
\S \ref{sec:reg}. Because the learner is linear, this corresponds to a 
``weight decay'' of factor $1-\sqrt{\eps} \rho$; note that if a neuron contains
bias terms, these should also decay.
Algorithm \ref{algo1} summarizes the entire optimization algorithm, which
we refer to as Linear Neuron Boosting (LNB).

\begin{algorithm}[tb]
\caption{Linear Neuron Boosting}
\label{algo1}
\begin{algorithmic}[1]
\INPUT{Network $F$,
linear neurons $F_B \subseteq F$ with parameters $\theta_B$,
loss function $\bar\ell$,
step size schedule $\{\eps^{(t)}\}_{t=1}^T$,
weight decay $\rho$}
\STATE Randomly initialize $\theta_B$
\FOR{$\eps \in [\eps^{(1)}, \ldots, \eps^{(T)}]$}
\STATE Compute $\nabla \bar\ell(\theta_B)$ via backpropagation and save the 
inputs to each neuron, $X_B = \{X_i | f_i \in F_B\}$ \label{algo:bp}
   \FOR{$(f_i, g_i, X_i) \in (F_B, \nabla\bar\ell(\theta_B), X_B)$}
      \STATE Define the \Eqref{eq:mvp} MVP function, $M_i(\cdot)$, using the VJP and JVP of $\mathbf{f}_{X_i}$
      \STATE $\hat{\theta}_i \gets \texttt{linear\_solver}(A=M_i, b=g_i)$
   \ENDFOR
   \STATE $\theta_B \gets (1-\sqrt{\eps}\rho)\theta_B$  \hfill $\triangleright$ Weight decay
   \STATE $z = \hat{\theta}_B \cdot \nabla \bar \ell(\theta_B)$  \hfill $\triangleright$ Norm
\STATE $\theta_B \gets \theta_B - \sqrt{\frac{\eps}{z}} \hat{\theta}_B$ \hfill $\triangleright$ Adaptive step size
\ENDFOR
\OUTPUT{$\theta_B$}
\end{algorithmic}
\end{algorithm}

\subsection{Update Properties}
\label{sec:WW}
We now analyze the properties of the LNB update rule. For simplicity in explanation,
we use a neuron with scalar output, $f_i : \gX_{i-1} \to \sR$, and note that this is sufficient because
the OLS solution of \Eqref{eq:OLS} for a vector output is the same as independently solving for each component.

When $f_i(x;\theta) = \theta^Tx$ \emph{without} a bias term, the resulting metric is the \emph{uncentered}
second moment matrix, $M_i = \E_{x \sim X_i}[xx^T]$. Denoting the number of parameters as $d_i$,
when the features are linearly independent and $n_i > d_i$, then $M_i \succ 0$ and the
OLS solution is scale equivariant \wrt the features. Of course, satisfying both of these
conditions can be rare in practice and we discuss mitigations in \S \ref{sec:PSD}.

When $f_i(x;\theta) = \theta^T \acute x$, where $\acute x^T = [x^T, 1]$
the resulting metric has the block form
\begin{equation}
M_i = \begin{pmatrix}
\Sigma_i + \mu_i \mu_i^T & \mu_i\\
\mu_i^T & 1
\end{pmatrix},
\end{equation}
where $\mu_i = \E_{x \sim X_i}[x]$ and $\Sigma_i$ are the mean and covariance matrix of $X_i$, respectively.
We can analytically invert $M_i$ via the factorization $M_i=L_i^T L_i$, where
$L_i = \begin{pmatrix}
   \Sigma_i^{\frac{1}{2}} & 0\\
   \mu_i^T & 1
\end{pmatrix}$.
Then, $M_i^{-1} = W_i W_i^T$, where
$W_i = \begin{pmatrix}
   \Sigma_i^{-\frac{1}{2}} & 0\\
   -\mu_i^T \Sigma_i^{-\frac{1}{2}} & 1
\end{pmatrix}$. From \S \ref{sec:pcgd}, preconditioning $W_iW_i^T g_i$ is the same
as reparameterizing the model:
$$
f_i(x; W_i\theta) = (W_i \theta)^T \acute x = \theta^T W_i^T \acute x = \theta^T (\Sigma_i^{-\frac{1}{2}}(x-\mu_i)).
$$
To summarize, when running LNB with neurons that have bias parameters,
it is equivalent to optimizing a model where we 
added a whitening transformation step in the network to each $f_i$'s input features using the
respective statistics from $X_i$, \emph{before} applying the parameters.
We further discuss this relationship to Batch Normalization \citep{batchnorm} and other methods in \S \ref{sec:connections}.

When the neuron is a convolution layer, $f_i(x;\theta) = x * \theta$, the output is still linear in the parameters (filters)
and the interpretation reduces to either of the previous two cases, with the only difference that the
first and second moments are computed using the filter's spatial support features (and are
translationally equivariant). By representing the metric using Jacobians, we can use autodifferentiation
to compute the metric without requiring a specialized implementation for this case, \eg, $\texttt{unfold}$.

Other common forms of $f_i$ will likely result in an identity $M_i$. For example, in a vision
transformer \citep{vit} the corresponding metric for both the embedding function, $f(x;\theta) = x + \theta$, and the class-token
function, $f(x;\theta) = \texttt{concat}(\theta, x)$, is the identity. In these cases, the linear solver can be skipped
since $\hat{\theta}_i = g_i$ and there are no other changes to the LNB algorithm.

\subsection{Positive Semi-definite Metrics}
\label{sec:PSD}
Since $M_i$ is an inner product of a Jacobian with itself, $M_i \succeq 0$, in general.
To ensure $M_i \succ 0$ in practice, a common regularizer is to add a small constant, $\gamma$,
along the diagonal of $M_i$ (excluding the bias terms) when solving the linear system. 
In the boosting perspective this adds a $\gamma \Vert \theta \Vert^2$ penalty term to \Eqref{eq:OLS2}
(which is why one would typically exclude the bias terms).
Because this regularizes the norm of the parameters, the solution is no longer scale equivariant.
When the metric's eigenvalues are small it no longer informs a meaningful
step-size ($z_{\theta} \approx 0$). In this case, we can upperbound the maximum allowed step size
to avoid taking a large step. This is essentially equivalent to disabling the
preconditioning and following the typical gradient vector direction.

\begin{algorithm}[ht]
   \caption{Online Linear Neuron Boosting}
   \label{algo2}
   \begin{algorithmic}[1]
   \INPUT{Network $F$,
   linear neurons $F_B \subseteq F$ with parameters $\theta_B$,
   mini-batch loss function $\bar\ell$,
   step size schedule $\{\eps^{(t)}\}_{t=1}^T$,
   weight decay $\rho$, minimum norm $z_0$}
   \STATE Randomly initialize $\theta_B$
   \FOR{$\eps \in [\eps^{(1)}, \ldots, \eps^{(T)}]$}
   \STATE Compute $\nabla \bar\ell(\theta_B)$ via backpropagation and save the 
   inputs to each neuron, $X_B = \{X_i | f_i \in F_B\}$
   \STATE $g_B = \texttt{ema}(g_B, \nabla \bar\ell(\theta_B))$
      \FOR{$(f_i, g_i, X_i) \in (F_B, g_B, X_B)$}
         \STATE $\mu_i = \texttt{ema}(\mu_i, \texttt{mean}(X_i))$
         \STATE $\chi_i = \texttt{ema}(\chi_i, \texttt{mean}(X_i \odot X_i))$
         \STATE Create $P_i$ from $\mu_i$ and $\chi_i$ (\S \ref{sec:online})
         \STATE Define the \Eqref{eq:mvp} MVP function, $M_i(\cdot)$, using the VJP and JVP of $\mathbf{f}_{X_i}$
         \STATE $\hat{\theta}_i \gets \texttt{cg}(A=M_i, b=g_i, P=P_i, x_0=\hat{\theta}_i)$
      \ENDFOR
      \STATE $\theta_B \gets (1-\sqrt{\eps}\rho)\theta_B$  \hfill $\triangleright$ Weight decay
      \STATE $z = \max(z_0, \hat{\theta}_B \cdot g_B)$  \hfill $\triangleright$ Norm
   \STATE $\theta_B \gets \theta_B - \sqrt{\frac{\eps}{z}} \hat{\theta}_B$ \hfill $\triangleright$ Adaptive step size
   \ENDFOR
   \OUTPUT{$\theta_B$}
   \end{algorithmic}
\end{algorithm}

\subsection{Online Learning}
\label{sec:online}
As presented, each step of boosting should be performed using the entire dataset,
which is quite impractical. We can easily switch to using an online estimator
of the gradient vector, \eg, an exponential moving average (EMA) over minibatches.
However, computing an online estimate for $M_i$ in our setting is difficult because we
rely on MVPs and do not materialize it nor $M_i^{-1}$.
One alternative, at the expense of increased compute and memory during training,
is to estimate $M_i$ using samples from a much larger dataset, which has demonstrated
to help under the FIM \citep{pascanu-2014}. Because $M_i$ does not require labels
we can compute it via only forward passes.
A second option, when $M_i$ can be materialized in memory, is to use standard techniques for
online covariance estimation \citep{dasgupta}.
If eigendecomposition can be afforded, a third option is to initialize $M_i^{-1}$ with the identity and
perform k-rank updates via the Woodbury matrix identity; this could also be amortized across gradient updates.

Instead of the previous three options, we instead propose to use an online, approximate
estimate of the feature moments as a preconditioner for the linear system solver.
Specifically, we use the Conjugate Gradient algorithm where we precondition each conjugate
step with the approximate solution to facilitate quick convergence.
The form of the preconditioner, $P_i$, is dependent on if the respective $f_i$ has a bias term or not.
If $f_i$ does not have a bias term, we approximate $M_i^{-1}$ using its diagonal,
$P_i = \E[\texttt{diag}(xx^T)]^{-1}$. If $f_i$ has a bias term, 
we approximate $W_iW_i^T$ from \S \ref{sec:WW} with the incomplete Cholesky factorization
$P_i = \widetilde{W}_i\widetilde{W}_i^T$, where $\widetilde{W}_i$ approximates the respective $\Sigma_i$ term with its diagonal.
Note that this does \emph{not} approximate $M_i^{-1}$ with a diagonal matrix, since it still contains the
off-diagonal $\mu_i$ terms.

We construct the terms in $P_i$ using an EMA of the moments $\mu_i$ and $\chi_i = \E[x \odot x]$
across the minibatches\footnote{Note that these quantities are also straightforward to compute for convolutions:
$\mu_i$ is the mean over all pixels in the minibatch.}.
Then, $\E[\texttt{diag}(xx^T)] = \chi_i$ and
$\texttt{diag}(\Sigma_i)= \chi_i - \mu_i \odot \mu_i$.
When initializing conjugate gradient with the solution from the previous LNB step and using 
$P_i$ as the preconditioner, we found that we obtain diminishing returns
on the solution quality after only two iterations in practice. The online version of LNB is
summarized in Algorithm \ref{algo2}. Note that this approach is also amenable to the distributed
training setting where the sufficient statistics, $\mu$ and $\chi$, can be asynchronously updated and communicated
to different nodes (as with the gradient vectors) while each node runs its own conjugate gradient solver.

\subsection{Complexity}
The space complexity of computing a LNB step is the same as backpropagation since
it reuses the feature sets $X_B$. However, the actual amount of used memory is increased by
approximately the size of $X_B$ in order to evaluate the MVP. Additionally, conjugate
gradient holds additional copies of the parameters.

For each neuron, the time complexity to solve the linear system is $O(n_id_i^2 + d_i^3)$, in general.
However, in practice, due to the outer-product nature of the metric, one MVP evaluation is
$O(n_i d_i)$ and we perform only 2 steps of conjugate gradient.

\section{Related Work}
\label{sec:connections}
\subsection{Function Space Optimization and Whitening}
The functional gradient descent framework is general and the 
application to neural networks described in \citet{grub2010} has connections to
multiple recent works. The LocoProp-S Algorithm 1 in \citet{locoprop}
is mechanically equivalent to solving \Eqref{eq:OLS2} using gradient descent
instead of solving in closed form, whereas this work avoids constructing the targets altogether.
As the authors discuss in Appendix C,
there is a reduction from LocoProp-S to ProxProp \citep{proxprop}.

The FOOF algorithm \citep{foof} is also motivated by performing gradient descent
in function space. For simplicity in implementation the authors analyzed functions
with no bias terms and Equation 6 in \citet{foof} is equivalent to
solving \Eqref{eq:normal}. In that work, an online estimate of the 
covariance matrix is computed but only infrequently inverted to update the
preconditioner. This work provides the following novel perspectives. First, 
we formulate the metric via inner-product of Jacobians. This approach
automatically handles, without specialized implementations, various forms of the neurons,
including if it has a bias term, is a convolutional layer, etc. 
Second, we make the equivalence to feature whitening explicit, which enables 
us to analytically identify a Cholesky preconditioner for fast convergence via
conjugate gradient. Lastly, this feature whitening perspective also
explains why FOOF converges in a single step in the example described in Appendix F:
it is performing a Newton step on squared loss, which is equivalent to performing one
gradient descent step on whitened features (via the reparameterized model).

Standardizing the inputs to neurons is an essential practitioner technique \citep{LeCun2012},
with BatchNorm \citep{batchnorm} being one of the most prevalent
usages with a diagonal approximation. The reparameterized model interpretation discussed in
\S \ref{sec:WW} is equivalent to placing a BatchNorm layer before applying the \emph{weights},
as opposed to typical practice of placing BatchNorm before the
\emph{activation}. The preconditioner connection is also discussed in \citep{lange}
where the preconditioner for CNNs is carefully constructed.
Related, PRONG \cite{nnn} also computes the same first and
second moments (again, manually) to reparameterize the network, but misses the analytical interpretation
of the preconditioner matrix that enables easy adaption to arbitrary network topologies.

In summary, all aforementioned work will obtain the same result, and the primary difference
is implementation. The primary benefit of the LNB interpretation is that it can be easily
implemented for any differentiable network architecture and the trust-region connection
informs an adaptive step size.

\subsection{(Quasi-) Second-order methods}
The $M_i$ metric is very local in that it finds
an update vector that induces an $\eps$-sized squared norm in the differential of each neuron's output,
$\langle \delta\theta_i, \delta\theta_i \rangle_{M_i} =
\delta\theta_i^T \E[\frac{\partial x_i}{\partial \theta_i}^T\frac{\partial x_i}{\partial \theta_i}] \delta\theta_i
= \E[\delta x_i^T \delta x_i]$.
Intuitively, a better metric would compose the neuron's output differential with the change in the \emph{network's}
output differential, $\delta x_m$, to ensure that the cascaded output perturbations are small when the neuron's parameters
change; this corresponds to the Gauss-Newton (GN) matrix,
$\E[\frac{\partial x_i}{\partial \theta_i}^T \frac{\partial x_m}{\partial x_i}^T\frac{\partial x_m}{\partial x_i} \frac{\partial x_i}{\partial \theta_i}]
= \E[\frac{\partial x_m}{\partial \theta_i}^T\frac{\partial x_m}{\partial \theta_i}]$.
A seemingly better metric would measure how $\delta x_m$ locally varies in the landscape
of the loss function, i.e., the inner product of $\delta x_m$ along the eigenvectors of the 
loss' Hessian at $x_m$,
$\E[\frac{\partial x_m}{\partial \theta_i}^T \frac{\partial^2 \ell_y(x_m)}{\partial x_m \partial x_m}  \frac{\partial x_m}{\partial \theta_i}]$;
this is the Generalized GN matrix (GGN) \citep{Schraudolph2002}.
As reviewed in \citet{kunster2019},
in the special case when the network predictions, $x_m$, are the parameters of an exponential probability
distribution and $\ell(x_m)$ is log-loss, then the GGN matrix is equivalent to the FIM used in
natural gradient descent.

There are strong theoretical reasons to prefer the FIM in general
\citep{amari1998,martens2020}; however,
computing the metric is typically expensive in practice and there is
extensive research for efficiently approximating it
\citep{martens2015,Ren2021TensorNT}.
The functional gradient descent interpretation makes no attempt to approximate the FIM.
But, given that LNB uses a much simpler metric, it
provides an effective and efficient intermediary between the identity metric of gradient descent
and the full network Jacobian used in GGN and natural gradients.

\section{Experiments}

We conduct comprehensive experiments across multiple datasets and model architectures to validate our method's ability to decouple explanation robustness from classification robustness. Our evaluation addresses three key research questions:
\begin{itemize}
    \item \textbf{RQ1:} Does \ours have better quantify uncertainties?
    \item \textbf{RQ2:} How do different ensemble methods and information from both dimensions help?
    \item \textbf{RQ3:} Is\ours robust to different settings? 
\end{itemize}


\begin{table}[H]
\centering
\resizebox{!}{0.11\textwidth}{
\begin{tabular}{@{}lc@{}}
\toprule
\textbf{Measure} & \textbf{Details} \\ 
\midrule
$U_{\textit{Eigv}}(Dis)$ & \multicolumn{1}{c}{Spectral eigenvalue on the disagreement.} \\ 
$U_{\textit{Ecc}}(Dis)$ & \multicolumn{1}{c}{Average distance in responses' disagreement.} \\ 
$U_{\textit{Degree}}(Dis)$ & \multicolumn{1}{c}{Degree of disagreement similarity Matrix.} \\ 
$U_{\textit{Eigv}}(Agre)$ & \multicolumn{1}{c}{Spectral eigenvalue on the agreement.} \\ 
$U_{\textit{Ecc}}(Agre)$ & \multicolumn{1}{c}{Average distance in responses' agreement.} \\ 
$U_{\textit{Degree}}(Agre)$ & \multicolumn{1}{c}{Degree Matrix of agreement Matrix.} \\ 
$p(true)$ & \multicolumn{1}{c}{Entropy of knowledge dimension responses} \\ 
$D-UE$ & \multicolumn{1}{c}{eigenvalue from Laplacian of a directional graph} \\ 

\bottomrule
\end{tabular}}
\vspace{-1mm}
\caption{The baseline methods and explanations.}
\vspace{-5mm}
\label{tab:baslines}
\end{table}

\subsection{Experimental Setup}
\label{sec:setup}
\subsubsection{Datasets} As mentioned in \cref{sec:background}, following prior works~\cite{lin2022teaching}, we focus on open-form question-answering 
(QA) tasks in this paper. We adopt 4 different classic QA datasets. Coqa~\cite{reddy2019coqa} is a conversational question-answering dataset that contains dialogues with free-form answers grounded in diverse passages, which is the easiest dataset among all datasets. HotpotQA~\cite{yang2018hotpotqa} is a multi-hop QA dataset that demands reasoning over multiple Wikipedia paragraphs to derive correct answers. NQ-Open~\cite{kwiatkowski2019natural} consists of real-world queries from Google Search, requiring models to retrieve and answer questions without explicit context, which is the hardest dataset. 
\subsubsection{Models to Evaluate} We evaluate \ours on Llama family~\cite{touvron2023llama}, which is the one of the most popular LLMs. In detail, we use Llama-2-13b and Llama-2-7B to demonstrate the effectiveness of \ours with different model sizes and use Llama-3.1-8B~\cite{dubey2024llama} to that \ours could also work on the different version of Llama. To further demonstrate the generalization ability for other architectures,  we also use Phi4~\cite{abdin2024phi} and Deepseek-R1-distill-7B~\cite{guo2025deepseek} in our paper.


%\textcolor{red}{Not sure whether there is a section labeled as "sec eva metric" refered by Sec. }

\subsubsection{Evaluation Metrics} Effective uncertainty measures should accurately represent the reliability of LLM responses, with higher uncertainty more likely leading to incorrect generations and vice versa~\cite{lin2023generating,kuhn2023semantic}. Following prior works~\cite{lin2023generating,da2024llm}, we mainly use UQ values to predict whether an answer is correct or not. Following prior works~\cite{lin2023generating,da2024llm}, we will use Area Under Receiver Operating Characteristic (AUROC) and Area Under Accuracy Rejection Curve (AUARC) as evaluation metrics, where a higher AUROC or AUARC demonstrates better uncertainty measures. To compute AUROC and AUARC, the accuracy of each original response is required. Following previous works~\cite{da2024llm,lin2023generating}, we use another LLM to provide correctness from 0-100 to each response. If the correctness is greater than 70, we label the response as correct. In this paper, we use Qwen-34B~\cite{bai2023qwen} to evaluate the correctness.

\subsubsection{Knowledge Extracted Models} In this paper, we mainly use llama-2-13b~\cite{touvron2023llama} as the auxiliary models to extract the knowledge dimension of responses. To demonstrate the robustness of \ours with different knowledge-extracted models, we also contain the results for different LLMs as knowledge-extracted models.

\subsubsection{Baselines} In this paper, we compared \ours with baselines that use semantic dimension response and knowledge dimension response. For semantic dimension, we mainly compared with methods that come from \citet{lin2023generating}. In detail, we incorporate six distinct methods from \citet{lin2023generating}, which differ based on the operations applied after computing similarity and whether they utilize agreement (entailment) probabilities or disagreement (contradiction) logits to construct the similarity matrix. For knowledge dimension, we use D-UE~\cite{da2024llm} and $p(true)$~\cite{kadavath2022language} as the baselines. Note that we use $p(true)$ on the knowledge dimension of response. We show the detailed explanations of all baselines in \cref{tab:baslines}

\begin{figure*}[t]
\centering
\begin{minipage}[t]{0.32\linewidth}
  \centering
  \includegraphics[width=\linewidth,trim=0 0 0 1cm, clip]{images/ablation.pdf}
  \captionof{figure}{Ablation studies that show that \ours fully utilizes all the information from both dimensions.}
  \label{fig:ablation}
\end{minipage}\hfill
\begin{minipage}[t]{0.32\linewidth}
  \centering
  \includegraphics[width=\linewidth]{images/knowledge_extract.pdf}
  \captionof{figure}{Performance for different knowledge extract models on CoQA and NQ\_Open with llama3.1.}
  \label{fig:knowledge_extract}
\end{minipage}\hfill
\begin{minipage}[t]{0.32\linewidth}
  \centering
  \includegraphics[width=\linewidth]{images/Jacc.pdf}
  \captionof{figure}{Performance that uses Jaccard similarity on CoQA and NQ\_Open with llama3.1.}
  \label{fig:jacc}
\end{minipage}
\end{figure*}

\subsection{Does \ours have better quantify uncertainties? (RQ1)}
\label{sec:main_result}
In this section, we explore whether \ours has better uncertainties compared with state-of-the-art uncertainty quantification methods. In \cref{tab:main_results}, we compare \ours with 8 baselines across three different datasets and five different models as introduced in \cref{sec:setup} In detail, we have the following observations:

\noindent $\bullet$ Compared with all baseline methods, \ours achieves the best performance overall. Especially when we consider AUROC. For AUARC, \ours achieves the best performance for NQ\_Open while \ours also achieves the comparable performance for CoQA in most scenarios. These results demonstrate that \ours has better quantify uncertainties overall. \\
\noindent $\bullet$ Among all datasets, \ours achieves the highest performance improvement on NQ\_Open, which is the most difficult dataset among all datasets and may lose to baselines for an easier dataset like CoQA. This indicates \ours could perform even better when the task is harder, where uncertainty quantification is more important. \\
\noindent $\bullet$ Two different ensemble methods show very similar results. Min strategy performs better than the sum strategy under $61.51\%$ situations, indicating that difficult datasets might also have more complex structures that single one tensor decomposition might oversight some information while using min structure could reduce such oversight by considering the best cases. However, both ensemble methods show a better performance than all baselines, which proves the effectiveness of tensor decomposition. \\

From these results, we get a conclusion that overall, \ours have better uncertainties.


\subsection{How do different ensemble methods and information from both dimensions help? (RQ2)}
\label{sec:ablation}
In this section, we use more experiments to prove the necessity of using information from both semantic and knowledge dimensions as well as using tensor decomposition. In detail, we consider the following methods: 1) \ours with only semantic responses, 2) \ours with only knowledge responses and 3) Concatenating similarity matrices from semantic and knowledge dimensions into a 2D matrix and applying SVD, 4) only using one tensor decomposition. In \cref{fig:ablation}, we show the comparison between \ours and other methods.  The results show that \ours consistently outperforms its variants and SVD method that repeated information will dominate the features, showing the effectiveness of our framework.




\subsection{Is \ours robust to different settings? (RQ3)}
\subsubsection{Different Knowledge Extracted Models} Knowledge extracted models influence the claim extraction in \ours as stated in \cref{subsubsec:knowledge}. Therefore, in this section, We test the robustness of \ours on various knowledge extracted models. unlike using llama2-13b in \cref{sec:main_result} and \cref{sec:ablation}, we conduct experiments on CoQA and NQ\_open using llama2-7b and llama3.1 as the knowledge extracted models, We show the results in \cref{fig:knowledge_extract}. From the figure, we can see that using Phi4 could even achieve a better result, indicating \ours has more potential with the development of LLMs. 

\subsubsection{Different Accuracy Thresholds} Different accuracy thresholds lead to different accuracy and influence the evaluation of uncertainties. In the previous experiments, we all set the accuracy threshold to 70 as mentioned in \cref{sec:setup}.  To test the robustness of \ours under different accuracy thresholds, we choose an extra dataset TriviaQA~\cite{joshi2017triviaqa}, which is considered the easiest dataset, and NQ\_Open, which is the most challenging dataset in our paper to conduct experiments. We show the results with accuracy thresholds of 70 and 90 in \cref{tab:Accuracy_threshold}. From the results, we can see that increasing the accuracy threshold decreases the performance of all baselines while the performance of \ours could even increase for datasets with different difficulties, showing the robustness of \ours in different settings. 

\subsubsection{Different Similarity Metrics} Finally, different similarity metrics lead to different similarity matrices. Therefore, to test whether \ours also has a good performance for different similarities, we use Jaccard similarity instead of using an NLI model in this section and the results are presented in \cref{fig:jacc}. The results show that using Jaccard similarity will boost the performance for a simple dataset like CoQA but hurt the performance for a difficult dataset like NQ\_Open. This is because the answer to a simple question might not have a deeper semantic meaning that requires NLI models. However, \ours can still outperform baseline methods that also use Jaccard similarity, showing the robustness of \ours.





\begin{table}[h]
    \centering
    \resizebox{0.5\textwidth}{!}{
    \begin{tabular}{lcccc}
        \toprule
        \multirow{2}{*}{Methods} & \multicolumn{2}{c}{Accuracy Threshold: 0.7} & \multicolumn{2}{c}{Accuracy Threshold: 0.9} \\
        \cmidrule(lr){2-3} \cmidrule(lr){4-5}
        & AUROC & AUARC & AUROC & AUARC \\
        \midrule
        \multicolumn{5}{c}{\textbf{Dataset: TriviaQA} [Easy]} \\
        \midrule
        Eigv(Dis) & 0.8261 & 0.8094 & 0.8100& 0.7604\\
        Ecc(Dis) & 0.8063& 0.7940&0.7892 & 0.7415\\
        Degree(Dis) &0.8399 & 0.8163&0.8259 & 0.7694\\
        Eigv(Agre) &0.8436 &0.8116 &0.8351 & 0.7721 \\
        Ecc(Agre) & \textbf{0.8510}&0.8189 & 0.8374&0.7721 \\
        Degree(Agre) &0.8396 &\textbf{0.8397} &0.8384 & 0.7739\\
        \ours-Sum &0.8428 &0.8144 & 0.8438&0.7749 \\
        \ours-Min &0.8431 &0.8149 & \textbf{0.8440} & \textbf{0.7754}\\
        \midrule
        \multicolumn{5}{c}{\textbf{Dataset: NQ\_Open} [Hard]} \\
        \midrule
        Eigv(Dis) & 0.6162 & 0.7300 &0.5636 &0.6017 \\
        Ecc(Dis) & 0.6210& 0.7330& 0.5658&0.5941 \\
        Degree(Dis) &0.6130 & 0.7168&0.5662 &0.6033 \\
        Eigv(Agre) &0.6258 &0.7276 & 0.6146& 0.6290 \\
        Ecc(Agre) & 0.6273&0.7311 &0.6239 &0.6344\\
        Degree(Agre) &0.6286 &0.7355 & 0.6221&0.6299 \\
        \ours-Sum &\textbf{0.6334} &\textbf{0.7410} &\textbf{0.6351} &\textbf{0.6430} \\
        \ours-Min &0.6332 &0.7409 & 0.6350 &0.6429 \\
        \bottomrule
    \end{tabular}
    }
    \caption{Comparison of different methods across different accuracy thresholds on TrivialQA and NQ\_Open with llama2-13B. The results show that our methods outperform baselines after increasing the accuracy threshold, indicating that our methods have an advantage on more difficult datasets.}
    \vspace{-7mm}
    \label{tab:Accuracy_threshold}
\end{table}



\bibliography{lnb}

\end{document}
