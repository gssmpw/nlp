\section{Attempt}
\cw{I'm making an attempt to write an AIR-like algorithm with the divergence measure similar to model-free DEC}

\begin{algorithm}
\caption{General Algorithm} 
    Let $\rho_1(\phi) = \frac{1}{|\Phi|}$ for all $\phi \in \Phi$. \\
    \For{$t=1,2,\ldots, T$}{
       Find a distribution $p_t$ of $\pi$ and a distribution $\nu_t\in \Delta(\calM)$ that solve the saddle-point of 
       \begin{align*}
           \min_{p\in\Delta(\Pi)} \max_{\nu\in \Delta(\calM)} \E_{\pi\sim p} \E_{\bar{\phi}\sim \rho_t} \E_{\phi\sim \nu} \E_{M\sim \nu(\cdot|\phi)}  \left[ 
V_\phi(\pi_\phi) - V_\phi(\pi) - D^\pi(\bar{\phi}|| M) \right]
       \end{align*}
       Sample decision $\pi_t\sim p_t$ and observe $o_t\sim M^\star(\cdot|\pi_t)$. \\
       {\color{red} Use some way to obtain $\rho_{t+1}\in \Delta(\Phi)$ based on $\nu_t\in\Delta(\calM)$ and $(\pi_t, o_t)$ \\
       (this procedure should minimize $\sum_t \E_{\bar{\phi}\sim \rho_t} D^{\pi_t} (\bar{\phi} || M^\star)$) --- exponential weights should work   }  
       
    }
\end{algorithm}

Question: do we have 
\begin{align*}
   \E_{\bar{\phi}\sim \rho} \E_{\phi\sim \nu} \E_{M\sim \nu(\cdot|\phi)} D^{\pi}(\bar{\phi}||M) \geq \E_{\bar{\phi}\sim \nu} \E_{\phi\sim \nu} \E_{M\sim \nu(\cdot|\phi)} D^{\pi}(\bar{\phi}||M)?   
\end{align*}
In general, no, because this is linear in $\rho$. 

\begin{algorithm}
\caption{General Algorithm} 
    Let $\rho_1(\phi) = \frac{1}{|\Phi|}$ for all $\phi \in \Phi$. \\
    \For{$t=1,2,\ldots, T$}{
       Find a distribution $p_t$ of $\pi$ and a distribution $\nu_t\in \Delta(\calM)$ that solve the saddle-point of 
       \begin{align*}
           \min_{p\in\Delta(\Pi)} \max_{\nu\in \Delta(\calM)} \E_{\pi\sim p} \E_{\phi\sim \nu} \E_{M\sim \nu(\cdot|\phi)}  \left[ 
V_\phi(\pi_\phi) - V_\phi(\pi) -  D(\nu(\cdot|\pi, o, \rho_t)|| M) \right]
       \end{align*}
       where $\nu(\cdot|\pi,o, \rho_t)$ is a distribution over $\Phi$ updated from $\nu$ by executing $\pi_t$ and obtaining observation $o_t$ on testing function $\bar{\phi}\sim \rho_t$. \\
       Sample decision $\pi_t\sim p_t$ and observe $o_t\sim M^\star(\cdot|\pi_t)$. \\
       Update $\rho_{t+1} = \nu_t(\cdot|\pi_t,o_t, \rho_t)$
       %{\color{red} Use some way to obtain $\rho_{t+1}\in \Delta(\Phi)$ based on $\nu_t\in\Delta(\calM)$ and $(\pi_t, o_t)$ \\
       %(this procedure should minimize $\sum_t \E_{\bar{\phi}\sim \rho_t} D^{\pi_t} (\bar{\phi} || M^\star)$) --- exponential weights should work   }  
       
    }
\end{algorithm}
Question: Do we have 
\begin{align*}
    \E_{M\sim \nu} D(\nu(\cdot|\pi, o, \rho)||M) \geq \E_{M\sim \nu}  D(\nu(\cdot|\pi, o, \nu)||M)
\end{align*}


\begin{align*}
    &\min_{p\in\Delta(\Pi)} \max_{\nu\in \Delta(\calM)} \E_{\pi\sim p} \E_{\bar{\phi}\sim \rho_t} \E_{\phi\sim \nu} \E_{M\sim \nu(\cdot|\phi)}  \left[ 
V_\phi(\pi_\phi) - V_\phi(\pi) - D^\pi(\bar{\phi}|| M) \right] \\
&\leq \max_{\nu\in \Delta(\calM)} \min_{p\in\Delta(\Pi)}  \E_{\pi\sim p} \E_{\bar{\phi}\sim \rho_t} \E_{\phi\sim \nu} \E_{M\sim \nu(\cdot|\phi)}  \left[ 
V_\phi(\pi_\phi) - V_\phi(\pi) - D^\pi(\bar{\phi}|| M) \right] \\
&\leq \max_{\nu\in \Delta(\calM)} \min_{p\in\Delta(\Pi)}  \E_{\pi\sim p} \E_{\bar{\phi}\sim \rho_t} \E_{\phi\sim \nu} \E_{M\sim \nu(\cdot|\phi)}  \left[ 
V_\phi(\pi_\phi) - V_\phi(\pi) - D^\pi(\bar{\phi}|| M) \right] 
\end{align*}

\begin{algorithm}
\caption{General Algorithm (for Bellman complete)} 
    Let $\rho_1(\phi) = \frac{1}{|\Phi|}$ for all $\phi \in \Phi$. \\
    \For{$t=1,2,\ldots, T$}{
       Find a distribution $p_t$ of $\pi$ and a distribution $\nu_t\in \Delta(\calM)$ that solve the saddle-point of 
       \begin{align*}
           \min_{p\in\Delta(\Pi)} \max_{\nu\in \Delta(\calM)} \E_{\pi\sim p} \E_{\phi\sim \nu} \E_{M\sim \nu(\cdot|\phi)}  \left[ 
V_\phi(\pi_\phi) - V_\phi(\pi) - D^\pi(\rho_t|| M) \right]
       \end{align*}
       Sample decision $\pi_t\sim p_t$ and observe $o_t\sim M^\star(\cdot|\pi_t)$. \\
       {\color{red} Use some way to obtain $\rho_{t+1}\in \Delta(\Phi)$ based on $\nu_t\in\Delta(\calM)$ and $(\pi_t, o_t)$ \\
       (this procedure should minimize $\sum_t  D^{\pi_t} (\rho_t || M^\star)$) --- exponential weights should work?   }  
       
    }
\end{algorithm}

\HL{I believe if we consider Bayesian regret, algorithm like this could go through, but for frequentist regret, seems that there are still unsolved difficulties. Specifically, if we know a prior $\nu$ where $M^\star \sim \nu_0$, and we would like to bound bayesian regret $\sum_{t=1}^T\E_{M \sim \nu_0}\E_{\pi_t \sim p_t}\left[V_{M}(\pi_M) - V_{M}(\pi_t)\right]$. The algorithm should be similar to the E2D.Bayes in \cite{foster2021statistical} (More discussion in their section 7.2), where the following objective is solved
\begin{align}
    \min_{p \in \Delta(\Pi)} \max_{M \in \calM} \E_{\pi \sim p}\left[V_M(\pi_M) - V^{\pi}_M - \E_{\Bar{M} \sim \nu_t}\left[D_{bi}\left(M(\pi), \bar{M}(\pi)\right)\right]\right]
\label{eq:baycom}
\end{align}
where $\nu_t$ is the \textbf{real posterior} of $\nu_0$ and $D_{bi}\left(M(\pi), \bar{M}(\pi)\right) = \left\langle X(\pi, \bar{M}), W(M, \bar{M})\right\rangle^2$. Above complexity is well-bounded by bilinear rank and we only need to bound 
\begin{align}
    \E_{\pi \sim p_t}\E_{M \sim \nu_t}\E_{\bar{M} \sim \nu_t}\left[ \left\langle X(\pi, \bar{M}), W(M, \bar{M})\right\rangle^2\right]
\label{eq:sumdiv}
\end{align}
From Theorem 7.3 of \cite{foster2021statistical}, this could be bounded by $\log|\calQ_{\calM}|$ where $\calQ_{\calM}$ is the induced $Q$-function space of $\calM$ (with some adapted definition of bilinar in their page 50 and Bellman complete).

If we do not have prior $\nu_0$, we need to find an alternate $\nu_t$ every round to ensure both \pref{eq:baycom} and \pref{eq:sumdiv} are well bounded. For standard model-based setting, such $\nu_t$ could be found based on exponential weights over $\log$ loss (\pref{eq:sumdiv} is changed to $M^\star$). Actually \pref{eq:baycom} is well-bounded for any $\nu_t$ from Proposition 7.1 in \cite{foster2021statistical}. Seems that we only need to find $\nu_t$ such that  
\begin{align*}
   &\sum_{t=1}^T\E_{\pi \sim p_t}\E_{\bar{M} \sim \nu_t}\left[ \left\langle X(\pi, \bar{M}), W(M^\star, \bar{M})\right\rangle^2\right] 
   \\&=  \sum_{t=1}^T \E_{\pi \sim p_t}\left[\E_{\bar{M} \sim \nu_t}\left[ \left\langle X(\pi, \bar{M}), W(M^\star, \bar{M})\right\rangle^2\right] - \left\langle X(\pi, M^\star), W(M^\star, M^\star)\right\rangle^2\right]\le o(T)\log|\Phi|
\end{align*}
where $p_t$ could be any policy distribution. Could be done by EXP over the groupings $\Phi$? For model-based setting, term inside $\E_{\pi}$ could be bounded by $\E_{\bar{M} \sim \nu}\left[\E^{\bar{M}, \pi}[\cdot] - \E^{M^\star, \pi}[\cdot]\right] \le \E_{\bar{M} \sim \nu}\left[\KL\left(\bar{M}(\pi), M^\star(\pi)\right)\right]$. Such $\bar{M}$ distribution can be found through $\log$ loss. How to go beyond this? One trivial but invalid grouping is: group $M$ such that for any $\pi$, $d_M^\pi = d$. But in such grouping every group can be an empty set...
}

\cw{Maybe? you can try. But is it hard to make this hold for all $\pi$? maybe just $\pi\sim p_t$}

\cw{I see. They seem to already present this result. }

\HL{Yeah, their proof also use some "grouping" technique, do not know whether there is room to extend.}




\section{Application~3: Model-Free RL in Adversarial MDPs with Fixed Transition}
For the adversarial model-free learning, we assume that each policy $\pi\in\Pi$ is associated with a function class $\calF^\pi$ where each $f\in\calF^\pi$ defines a mapping $Q_{f}(s,a;R): \calS\times\calA\times\calR\to \mathbb{R}$ that specifies a $Q^\pi(s,a;R)$ in a potential world. Each transition $P$ maps to a particular $f$ in $\calF^\pi$, while one $f\in\calF^\pi$ can be mapped from different $P$'s. Define $\calF=\bigcup_{\pi\in\Pi}\calF^\pi$. 

\begin{definition}[Model-free Bilinear Class]
An MDP is bilinear with dimension $d$ if there exists feature $X$ and $W$ such that
\begin{enumerate}
    \item For every $\pi\in\Pi$, $f \in \calF^\pi$, $M$
    \begin{align*}
        \E^{M,\pi}\left[Q_{f,h}(s_h,a_h;R) - R'(s_h, a_h) - V_{f,h+1}(s_{h+1}; R)\right] = \left\langle X_h(\pi;M), W_h(f;P,R) \right\rangle.
    \end{align*}
    \item Let $z_h = (s_h, a_h, r_h, s_{h+1})$, there exists an estimation function $\ellest: \calF \times \calZ \rightarrow \mathbb{R}$ such that for all $\pi\in \Pi$, $f'\in \calF$, $P\in\calP$, and $R\in\calR$, 
    \begin{align*}
        \left\langle X_h(\pi;P), W_h(f';P,R) \right\rangle = \E^{P, \pi}\left[\ellest(f';z_h)\right].
    \end{align*}
\end{enumerate}
\end{definition}

Fix a $\nu$, let $\pi \sim p$ be equivalent to  
\begin{align*}
&\E_{\pi\sim p}\E_{\phi\sim\nu}\E_{(M,\pi^\star)\sim\nu_{\cdot|\phi}}[V_M(\pi^\star)-V_{M}(\pi)]  \\
&=\E_{\pi\sim p}\E_{\phi\sim\nu}\E_{(M,\pi^\star)\sim\nu_{\cdot|\phi}}[V_M(\pi^\star)-V_{M}(\pi)] \\
&= \E_{\bar{\phi}\sim \nu} \E_{\phi\sim \nu} \E_{(\bar{M}, \bar{\pi})\sim\nu_{\cdot|\bar{\phi}}} \E_{(M,\pi^\star)\sim\nu_{\cdot|\phi}} [V_M(\pi^\star)-V_{M}(\bar{\pi})] \\
&= \E_{\bar{\phi}\sim \nu} \E_{\phi\sim \nu} \E_{(\bar{M}, \bar{\pi})\sim\nu_{\cdot|\bar{\phi}}} \E_{(M,\pi)\sim\nu_{\cdot|\phi}} [V_M(\pi)-V_{\bar{M}}(\pi)] \\ 
&= \E_{\bar{\phi}\sim \nu} \E_{\phi\sim \nu} \E_{(\bar{M}, \bar{\pi})\sim\nu_{\cdot|\bar{\phi}}} \E_{(M,\pi)\sim\nu_{\cdot|\phi}}  \E^{\bar{M}, \pi}\left[ Q^\pi_{M,h}(s_h,a_h) - R_{\bar{M},h}(s_h,a_h) - V^\pi_{M,h+1}(s_{h+1}) \right]   \\
&= \E_{\bar{\phi}\sim \nu} \E_{\phi\sim \nu} \E_{(\bar{M}, \bar{\pi})\sim\nu_{\cdot|\bar{\phi}}} \E_{(M,\pi)\sim\nu_{\cdot|\phi}}  \inner{X(\pi; \bar{P}), W(f(P,\pi); \bar{P}, R)}  \tag{the $f\in\calF^\pi$ here is the one induced by $P$ and $\pi$} \\
&= \E_{\bar{\phi}\sim \nu} \E_{\phi\sim \nu} \E_{(\bar{M}, \bar{\pi})\sim\nu_{\cdot|\bar{\phi}}} \E_{(M,\pi)\sim\nu_{\cdot|\phi}}  \inner{X(\pi; \bar{P}), W(f_\phi; \bar{P}, R)}    \tag{the group $\phi$ share the same $f$} \\
&= \E_{\bar{\phi}\sim \nu} \E_{\phi\sim \nu} \E_{(\bar{M}, \bar{\pi})\sim\nu_{\cdot|\bar{\phi}}}   \inner{X(\pi_\phi; \bar{P}), W(f_\phi; \bar{P}, R)}    \tag{the group $\phi$ share the same $\pi$} 
\end{align*}
\cw{I don't know how to continue from here. If we want to group $\bar{P}$ within $\bar{\phi}$, we have to group $\bar{P}$ according to the value of $X(\pi_\phi, \bar{P})$ for \emph{every possible $\pi_\phi$}. This will make the number of groups be exponential in the number of different $\pi_\phi$ (which is $|\Pi|$).   }


\begin{align*}
%\\&= \E_{\Bar{M}\sim \nu}\E_{\phi\sim\nu}\E_{M\sim\nu_{\cdot|\phi}} \E^{\Bar{M}, \pi_M}\left[Q_{M,h}^*(s_h,a_h) - r_h^{\Bar{M}} - \max_{a'} Q_{M, h+1}^*(s_{h+1}, a')\right] 
\\&= \E_{\Bar{M}\sim \nu}\E_{\phi\sim\nu}\E_{M\sim\nu_{\cdot|\phi}} \left[\left\langle X_h(M;\Bar{M}), W_h(M;\Bar{M}) \right\rangle\right]
\\&= \E_{M\sim \nu}\E_{\Bar{\phi}\sim\nu}\E_{\Bar{M}\sim\nu_{\cdot|\Bar{\phi}}} \left[\left\langle X_h(M;\Bar{M}), W_h(M;\Bar{M}) \right\rangle\right]
\\&= \E_{M\sim \nu}\E_{\Bar{\phi}\sim\nu}\E_{\Bar{M}\sim\nu_{\cdot|\Bar{\phi}}} \left[\left\langle X_h(M;\Bar{\phi}), W_h(M;\Bar{M}) \right\rangle\right] \tag{Group of $X$} 
\\&= \E_{\Bar{\phi}\sim\nu} \E_{M \sim \nu}\left[\left\langle X_h(M;\Bar{\phi}), \E_{\Bar{M}\sim\nu_{\cdot|\Bar{\phi}}}\left[W_h(M;\Bar{M})\right] \right\rangle\right] \tag{\cw{may define $\E_{\bar{M}\sim \nu_{\cdot|\bar{\phi}}}[W_h(M, \bar{M})]$ as $W_h(M, \bar{M}^{\nu_{\cdot|\bar{\phi}}})$}} 
 \\&\le \frac{d}{\eta} + \eta \E_{M\sim \nu}\E_{M' \sim \nu}\E_{\Bar{\phi}\sim\nu} \left[\left\langle X_h(M;\Bar{\phi}), \E_{\Bar{M}\sim\nu_{\cdot|\Bar{\phi}}}\left[W_h(M';\Bar{M})\right] \right\rangle^2\right] 
\end{align*}

\jz{Some draft, because I think the previous analysis doesn't go through.}
Assume $p=\nu_{\bo{\pi^\star
}}$, 
We want to bound
\begin{align*}
    \E_{\pi\sim p}\E_{\phi\sim\nu}\E_{M\sim\nu_{\cdot|\phi}}[V_M(\pi_{\phi})-V_{M}(\pi)]=\E_{\phi\sim\nu}\E_{M\sim\nu_{\cdot|\phi}}\E_{M'\sim\nu}[V_{M'}(\pi_{\phi})-V_{M}(\pi_\phi)]
\end{align*}
% \HL{Is $\pi_\phi$ well-defined if we group over $X$, what should $\pi_{\phi}$ be in stochastic setting?}
Let's take the Bellmann error the other way round, we have
\HL{Shoud $Q_{M'}(\pi, s,a)$ be $Q_{M'}(\pi_{\phi}, s,a)$? In that case the $W$ feature should be $W(M,M',\phi)$}
\begin{align*}
&V_{M'}(\pi_{\phi})-V_{M}(\pi_\phi) = \E^{M,\pi_{\phi}}[Q_{M'}(\pi;s,a)-V_{M'}(\pi;s')-\ell^{M}(s,a)]\\
&=\E^{M,\pi_{\phi}}\left[\inner{\varphi^\star(s,a),r^{M'}-r^{M}+\int_{s'}\left(\psi^{M'}(s')-\psi^{M}(s')\right)V_{M'}(\pi;s')}\right]\\
&=\inner{X(M,\pi_{\phi}),W(M,M')}.
\end{align*}
We select $\phi$ such that that the occupancy measure $X(M,\pi_{\phi})= X(\phi)$, hence

\begin{align*}
\E_{\phi\sim\nu}\E_{M\sim\nu_{\cdot|\phi}}\E_{M'\sim\nu}[V_{M'}(\pi_{\phi})-V_{M}(\pi_\phi)] = \E_{\phi\sim\nu}\inner{X(\phi), W(\nu_{\cdot|\phi},\nu)}
\end{align*}
Doing the decoupling:
\begin{align*}
&\leq \frac{d}{\eta} + \eta \E_{\phi\sim\nu}\E_{\phi'\sim\nu}\inner{X(\phi),W(\nu_{\cdot|\phi'},\nu)}^2
\end{align*}
Further, the term in the square is (for low rank mdps at least)
\jz{Here is the bug, the $W$ term is not antisymmetric.}  \cw{shouldn't the proof here similar to the Lemma F.4 in the DEC paper? }
\begin{align*}
\inner{X(\phi),W(\nu_{\cdot|\phi'},\nu)} &= \inner{X(\phi),W(\nu_{\cdot|\phi},\nu)-W(\nu_{\cdot|\phi},\nu_{\cdot|\phi'})}\\
&\qquad + \underbrace{\inner{X(\phi),\E_{M'\sim\nu,M''\sim\nu_{\cdot|\phi'}}\int_{s'}\left(\psi^{M'}(s')-\psi^{M''}(s')\right)(V_{M'}(\pi;s')-V_{M''}(\pi;s'))}}_{\text{This term we don't observe ...}}   \\
&=\E_{M\sim\nu_{\cdot|\phi}}\E_{M'\sim\nu}[V_{M'}(\pi_{\phi})-V_{M}(\pi_{\phi})]-\E_{M\sim\nu_{\cdot|\phi}}\E_{M'\sim\nu_{\cdot|\phi'}}[V_{M'}(\pi_{\phi})-V_{M}(\pi_{\phi})]\\
&=V_{\nu}(\pi_{\phi})-V_{\nu_{\cdot|\phi'}}(\pi_{\phi})\\
&\lesssim \sqrt{\KL(M^{\nu(\cdot|\phi')}(\cdot|\pi_{\phi}),M^{\nu}(\cdot|\pi_{\phi}))} 
\end{align*}


\section{Application 4: Improved Regret Bounds for Full-Information Settings (Original)}

For the problem of learning in adversarial MDPs with fixed transition, it is possible to improve the dependency on $\log|\Pi|$ to $\log|\calA|$. To achieve this, we slightly modify the general protocol. Suppose that the environment secretly decides $\phi^\star\in\Phi$ at the beginning. In each round, the environment decides $(M_t, \pi_t^{\star})\in\phi^\star$ and $(\pi_t^\phi)_{\phi\in\Phi}$ such that $\pi_t^{\phi^\star}=\pi_t^\star$. The environment reveals $(\pi_t^\phi)_{\phi\in\Phi}$ to the learner before they make decision at round $t$. 

Denote $\pi^\Phi=(\pi^\phi)_{\phi\in\Phi}$ and define
\begin{align*}
    \infoair^\Phi_{q,\eta}(p,\nu,\pi^\Phi) &= \E_{\pi\sim p} \E_{\phi\sim \nu} \E_{M\sim \nu(\cdot|\phi, \pi^\phi)} \E_{o\sim M(\cdot|\pi)}\left[V_M(\pi^\phi) - V_M(\pi) - \frac{1}{\eta} \KL(\nu_{\bo{\phi}}(\cdot|\pi, o), q)\right] \\
    &= \E_{\pi\sim p} \E_{(M,\pi^\star)\sim \nu(\cdot| \pi^\Phi)} \E_{o\sim M(\cdot|\pi)}\left[V_M(\pi^\star) - V_M(\pi) - \frac{1}{\eta} \KL(\nu_{\bo{\phi}}(\cdot|\pi, o), q)\right]
\end{align*}

\begin{algorithm}
\caption{General Algorithm with Informed Comparator} \label{alg:general AIR full info}
    Let $\rho_1(\phi) = \frac{1}{|\Phi|}$ for all $\phi \in \Phi$. \\
    \For{$t=1,2,\ldots, T$}{
       Receive $\pi_t^\Phi=(\pi_t^\phi)_{\phi\in\Phi}$ from the environment. \\
       Find a distribution $p_t$ of $\pi$ and a distribution $\nu_t$ of $\Psi$ that solve the saddle-point of 
       \begin{align*}
           \min_{p\in\Delta(\Pi)} \max_{\nu\in \Delta(\Psi)} \infoair^\Phi_{\rho,\eta}(p,\nu, \pi_t^\Phi)  
       \end{align*}
       Sample decision $\pi_t\sim p_t$ and observe $o_t\sim M_t(\cdot|\pi_t)$. \\
       Update $\rho_{t+1}(\phi)=\nu_{t}(\phi|\pi_t,o_t)$ for all $\phi\in\Phi$.  
       
    }
\end{algorithm}



\section{Application~2: Model-Free RL in Stochastic MDPs}

%\cw{Clarify that the ``model-free'' here is not about computation or memory complexity. It's about the regret does not scale with model size}

For model-free reinforcement learning in a stochastic environment, we usually have the following assumption.

\begin{assumption}
    There exists a function space $\mathcal{H}$ such that there exists an $f \in \mathcal{H}$ such that $Q_f = Q^{\star}$, where $Q^\star = Q^{ \pi_{M^\star}}_{M^\star}$ and $M^\star$ is the ground-truth model.
\end{assumption}


In setting, we define the partition as $\Phi_{\textsc{F}} = \left\{\phi_{Q}: Q \in \calQ \right\}$ where $\phi_{Q} = \left\{(M, \pi_M): M \in \calM, Q_M^{\pi_M} = Q\right\}$. Moreover, now $\joint_{\textsc{F}} = \left(\bigcup_{\phi\in\Phi_\textsc{F}} \phi\right) = \left\{(M, \pi_M): M \in \calM, Q_M^{\pi_M} \in \calQ \right\}\subset \calM\times \Pi$. The final complexity we need to bound is 
\begin{align*}
&\max_{\Bar{M} \in \textsc{co}(\calM)}    
 \min_{p\in\Delta(\Pi)} \max_{\nu\in \Delta(\Psi)} \E_{\pi\sim p} \E_{\phi\sim \nu}  \E_{(M,\pi^\star)\sim \nu(\cdot|\phi)} \left[V_{M}(\pi^\star) - V_{M}(\pi) - \frac{1}{\eta}  \KL\left( M^{\nu(\cdot|\phi)}(\cdot|\pi), \Bar{M}(\cdot|\pi) \right)\right]   
 \\&=  \max_{\Bar{M} \in \textsc{co}(\calM)}    
 \min_{p\in\Delta(\Pi)} \max_{\nu\in \Delta(\Psi)} \E_{\pi\sim p} \E_{\phi\sim \nu}  \E_{M\sim \nu(\cdot|\phi)} \left[V_{M}(\pi_M) - V_{M}(\pi) - \frac{1}{\eta}  \KL\left( M^{\nu(\cdot|\phi)}(\cdot|\pi), \Bar{M}(\cdot|\pi) \right)\right]   
\end{align*}

\begin{definition}[Model-free Bilinear Class]
An MDP is bilinear with dimension $d$ if there exists feature $X$ and $W$ such that
\begin{enumerate}
    \item For every $f \in \mathcal{H}$, 
    \begin{align*}
        \E^{M, \pi_f}\left[Q_{f,h}(s_h,a_h) - r_h^M - \max_{a'} Q_{f, h+1}(s_{h+1}, a')\right] = \left\langle X_h(f;M), W_h(f;M) \right\rangle.
    \end{align*}
    \item Let $z_h = (s_h, a_h, r_h, s_{h+1})$, there exists an estimation function $\ellest: \mathcal{H} \times \calZ \rightarrow \mathbb{R}$ such that for all $f, g \in \mathcal{H}$ and $h \in [H]$, 
    \begin{align*}
        \left\langle X_h(f;M), W_h(g;M) \right\rangle = \E^{M, \pi_f}\left[\ellest(g;z_h)\right].
    \end{align*}
\end{enumerate}
\end{definition}
For low-rank MDP $X_h(Q; M) = \E_{s,a \sim d_M^{\pi_Q}}\left[\phi(s_{h-1}, a_{h-1})\right]$, for $M$ in the same $\phi$, $X_h(Q; M)$ are not the same. Group over $X_h(Q; M)$? We need know the space of $\Psi$ for the algorithm?

$V_M^{\pi_M} = V_Q(\pi_Q)$ if $M \in \phi_Q$. Does $V_M^{\pi_Q} = V_Q(\pi_Q)$ if $M \in \phi_Q$? \cw{Yes -- If $M\in\phi_Q$ then the $Q^\star$ induced by $M$ is $Q$. Thus, $\pi_Q$ is an optimal policy on $M$. Both expressions are the value of the optimal policy on $M$}


\subsection{A Summary of current attempts and their failure}
\subsubsection{Model-based partition over $X$}
Model-based bilinear assumption is defined below.
\begin{definition}[Model-based Bilinear Class]
An MDP is bilinear with dimension $d$ if there exists feature $X$ and $W$ such that
\begin{enumerate}
    \item For every $M \in \mathcal{M}$, 
    \begin{align*}
        \E^{\Bar{M}, \pi_M}\left[Q_{M,h}^*(s_h,a_h) - r_h^{\Bar{M}} - \max_{a'} Q_{M, h+1}^*(s_{h+1}, a')\right] = \left\langle X_h(M;\Bar{M}), W_h(M;\Bar{M}) \right\rangle.
    \end{align*}
    \cw{Why not $Q^*_{f,h}(s_h,a_h)$ and $\pi_f$? }
    \HL{I feel define $f$ class does not help the following analysis, which has issue now. If we can group $X$, there should also be some benefit for the model-based setting? Not $\log|\calM|$ but $\log|\mathcal{X}|$? }  \cw{What is $\calX?$} \cw{what's the issue in the following analysis? } \HL{there is a equation with tag "ideal bound but does not hold", I may know how to do this, seems like a partition over $f, X$}  \cw{I think $Q_f$ is the standard bilinear class definition}
    \item Let $z_h = (s_h, a_h, r_h, s_{h+1})$, there exists an estimation function $\ellest: \mathcal{H} \times \calZ \rightarrow \mathbb{R}$ such that for all $M, M' \in \mathcal{M}$ and $h \in [H]$, 
    \begin{align*}
        \left\langle X_h(M;\Bar{M}), W_h(M';\Bar{M}) \right\rangle = \E^{\Bar{M}, \pi_M}\left[\ellest(M';z_h)\right].
    \end{align*}
\end{enumerate}
\end{definition}

Fix a $\nu$, let $\pi \sim p$ is equivalent to $M' \sim v$ and play $\pi_{M'}$, we need to bound
\begin{align*}
&\E_{\pi\sim p}\E_{\phi\sim\nu}\E_{M\sim\nu_{\cdot|\phi}}[V_M(\pi_{M})-V_{M}(\pi)]  
\\&= \E_{\Bar{M}\sim \nu}\E_{\phi\sim\nu}\E_{M\sim\nu_{\cdot|\phi}}[V_M(\pi_{M})-V_{M}(\pi_{\Bar{M}})]
\\&= \E_{\Bar{M}\sim \nu}\E_{\phi\sim\nu}\E_{M\sim\nu_{\cdot|\phi}}[V_M(\pi_{M})-V_{\Bar{M}}(\pi_{M})]
\\&= \E_{\Bar{M}\sim \nu}\E_{\phi\sim\nu}\E_{M\sim\nu_{\cdot|\phi}} \E^{\Bar{M}, \pi_M}\left[Q_{M,h}^*(s_h,a_h) - r_h^{\Bar{M}} - \max_{a'} Q_{M, h+1}^*(s_{h+1}, a')\right] 
\\&= \E_{\Bar{M}\sim \nu}\E_{\phi\sim\nu}\E_{M\sim\nu_{\cdot|\phi}} \left[\left\langle X_h(M;\Bar{M}), W_h(M;\Bar{M}) \right\rangle\right]
\\&= \E_{M\sim \nu}\E_{\Bar{\phi}\sim\nu}\E_{\Bar{M}\sim\nu_{\cdot|\Bar{\phi}}} \left[\left\langle X_h(M;\Bar{M}), W_h(M;\Bar{M}) \right\rangle\right]
\\&= \E_{M\sim \nu}\E_{\Bar{\phi}\sim\nu}\E_{\Bar{M}\sim\nu_{\cdot|\Bar{\phi}}} \left[\left\langle X_h(M;\Bar{\phi}), W_h(M;\Bar{M}) \right\rangle\right] \tag{Group of $X$} 
\\&\text{\HL{Group step only work for trivial partition, where $\phi_{M, X} = \{M': M' = M, X(M, M') = X\}$.}}
\\&\text{\HL{Only $\phi_{M, X(M,M)}$ has an element $M$. Other valid partition have the same issue Chen-Yu faced in Section 6}}
\\&= \E_{\Bar{\phi}\sim\nu} \E_{M \sim \nu}\left[\left\langle X_h(M;\Bar{\phi}), \E_{\Bar{M}\sim\nu_{\cdot|\Bar{\phi}}}\left[W_h(M;\Bar{M})\right] \right\rangle\right] \tag{\cw{may define $\E_{\bar{M}\sim \nu_{\cdot|\bar{\phi}}}[W_h(M, \bar{M})]$ as $W_h(M, \bar{M}^{\nu_{\cdot|\bar{\phi}}})$}} 
 \\&\le \frac{d}{\eta} + \eta \E_{M\sim \nu}\E_{M' \sim \nu}\E_{\Bar{\phi}\sim\nu} \left[\left\langle X_h(M;\Bar{\phi}), \E_{\Bar{M}\sim\nu_{\cdot|\Bar{\phi}}}\left[W_h(M';\Bar{M})\right] \right\rangle^2\right] 
\end{align*}
This could also be
\begin{align*}
     \frac{d}{\eta} + \eta \E_{M\sim \nu}\E_{\phi \sim \nu}\E_{\Bar{\phi}\sim\nu} \left[\left\langle X_h(M;\Bar{\phi}), \E_{\Bar{M}\sim\nu_{\cdot|\phi}}\left[W_h(M;\Bar{M})\right] \right\rangle^2\right]
\end{align*}
We know $W(M', M') = 0$, thus
\begin{align*}
&\left\langle X_h(M;\Bar{\phi}), \E_{\Bar{M}\sim\nu_{\cdot|\Bar{\phi}}}\left[W_h(M';\Bar{M})\right] \right\rangle 
\\&= \left\langle X_h(M;\Bar{\phi}), \E_{\Bar{M}\sim\nu_{\cdot|\Bar{\phi}}}\left[W_h(M';\Bar{M})\right] \right\rangle - \E_{M' \sim \nu}\left[\left\langle X_h(M;M'), W_h(M';M') \right\rangle\right]  
\\&= \E_{\Bar{M} \sim \nu_{\cdot|\Bar{\phi}}}\E^{\Bar{M}, \pi^M}\left[\ellest(M';z_h)\right] -  \E_{M' \sim \nu}\E^{M', \pi^M}\left[\ellest(M';z_h)\right] 
\\&\le \KL\left(M^{\nu(\cdot|\phi)}(\cdot|\pi),  M^{\nu}(\cdot|\pi) \right) \tag{Ideal bound but not hold here}
\end{align*}




% \HL{To go through, $M^{\nu}$ seems to be constructed by the first term and $M^{\nu(\cdot|\phi)}$ is constructed for the newly introduced term due to zero property of $W$.}

\cw{does further partitioning based on $W$ help? }

The divergence term in the complexity is 
\begin{align*}
    \E_{\pi \sim p}\E_{M \sim \nu}\E_{\phi \sim \nu}\left[\KL\left(M^{\nu(\cdot|\phi)}(\cdot|\pi),  M^{\nu}(\cdot|\pi) \right)\right]
\end{align*}
In our case, given the choice of $p$, it is equivalent to 
\begin{align*}
    \E_{M \sim \nu}\E_{\phi \sim \nu}\left[\KL\left(M^{\nu(\cdot|\phi)}(\cdot|\pi_M), M^{\nu}(\cdot|\pi_M) \right)\right]
\end{align*}

\subsubsection{Some general attempt for Model-free Stochastic}
We define a joint partition over $f, \mathcal{X}$ where $\mathcal{X}$ is the covering set of all $X$. Each partition $\phi_{f, X}$ contains the models $M \in \calM$ such that $Q_M^\star = Q_f$ and $X(f; M) = X$. The penalty is $\log(|\calF||\mathcal{X}|)$.
In model based setting, the partition is the same as $\phi_M$. Choose $\pi \sim p$ is equivalent to $f \sim \nu$ and play , Let $V_{\phi}(\pi_\phi) = V_{f}(\pi_f)$,  $Q_{\phi} = Q_{f}$ where $\phi$ is related to $f$

\begin{align*}
&\E_{\pi\sim p}\E_{\phi\sim\nu}\E_{M\sim\nu_{\cdot|\phi}}[V_M(\pi_{M})-V_{M}(\pi)]  
\\&= \E_{\phi'\sim \nu}\E_{\phi\sim\nu}\E_{M\sim\nu_{\cdot|\phi}}[V_{\phi}(\pi_{\phi})-V_{M}(\pi_{\phi'})]  
\\&= \E_{\phi\sim \nu}\E_{\bar{\phi}\sim\nu}\E_{\bar{M}\sim\nu_{\cdot|\bar{\phi}}}[V_{\phi}(\pi_{\phi})-V_{\bar{M}}(\pi_{\phi})]  
\\&= \E_{\phi\sim \nu}\E_{\bar{\phi}\sim\nu}\E_{\bar{M}\sim\nu_{\cdot|\bar{\phi}}}\E^{\pi_\phi, \bar{M}}[Q_\phi(s,a) - R^{\Bar{M}} - \max_{a'}Q_{\phi}(s', a')] 
\\&= \E_{\phi\sim \nu}\E_{\bar{\phi}\sim\nu}\E_{\bar{M}\sim\nu_{\cdot|\bar{\phi}}}\left[\left\langle X(\phi, \bar{M}), W(\phi, \Bar{M})\right\rangle\right]
\\&= 
\end{align*}
\HL{Same issue}



\begin{align*}
&\E_{\pi\sim p}\E_{\phi\sim\nu}\E_{M\sim\nu_{\cdot|\phi}}[V_M(\pi_{M})-V_{M}(\pi)] 
\\&= \E_{\pi\sim p}\E_{\phi\sim\nu}\E_{M\sim\nu_{\cdot|\phi}}[V_M(\pi_{\phi})-V_{M}(\pi)]  
\\&= \E_{\phi'\sim \nu}\E_{\phi\sim\nu}\E_{M\sim\nu_{\cdot|\phi}}[V_{M}(\pi_{\phi})-V_{M}(\pi_{\phi'})]  
\\&=  \E_{\phi\sim\nu}\E_{M\sim\nu_{\cdot|\phi}}\E_{\bar{M} \sim \nu}[V_{M}(\pi_{\phi})-V_{\bar{M}}(\pi_{\phi})]  
\\&= \E_{\phi\sim\nu}\E_{M\sim\nu_{\cdot|\phi}}\E_{\bar{M} \sim \nu}\E^{\pi_\phi, M}[ R^{M}(s,a)  - Q_{\bar{M}}^{\pi_{\phi}}(s,a) + V_{\bar{M}}^{\pi_{\phi}}(s')] 
\\&= \E_{\phi\sim\nu}\E_{M\sim\nu_{\cdot|\phi}}\E_{\bar{M} \sim \nu}\left[\left\langle X(\phi, M), W(\Bar{M}, M, \phi)\right\rangle\right] \tag{At least for low-rank MDPs}
\\&= \E_{\phi\sim\nu}\E_{M\sim\nu_{\cdot|\phi}}\E_{\bar{M} \sim \nu}\left[\left\langle X(\phi), W(\Bar{M}, M, \phi)\right\rangle\right] \tag{We can group now}
\\&= \E_{\phi\sim\nu}\left[\left\langle X(\phi), W(\Bar{M}^{\nu}, M^{\nu_{\cdot|\phi}}, \phi)\right\rangle\right]
\\&\le \frac{d}{\eta} + \eta \E_{\phi \sim \nu}\E_{\phi' \sim \nu}\left[\left\langle X(\phi),  W(\Bar{M}^{\nu}, M^{\nu_{\cdot|\phi'}}, \phi')\right\rangle^2\right]
\end{align*}
We have
\begin{align*}
    \left\langle X(\phi),  W(\Bar{M}^{\nu}, M^{\nu_{\cdot|\phi'}}, \phi')\right\rangle \le \E_{M \sim \nu(\cdot|\phi)}\E_{\Bar{M} \sim \nu}\E_{M' \sim v(\cdot|\phi')}\left[\left\langle X(\phi, M), W(\Bar{M}, M', \phi)\right\rangle\right]
\end{align*}
We can not directly relate the RHS by some $\ellest$ because $M, M'$ and $\Bar{M}$ are all different. 

How to bound this for linear MDPs using linear loss feature?

\subsection{Relation of DEC terms draft}
Define \begin{align*}
&\dec_{\eta}^{\textsc{KL}}\left(\calM, \Phi\right) = 
\\&\max_{\Bar{M} \in \textsc{co}(\calM)}\min_{p\in\Delta(\Pi)} \max_{\nu\in \Delta(\Psi)}   \E_{\pi\sim p} \E_{\phi\sim \nu}  \E_{(M,\pi^\star)\sim \nu(\cdot|\phi)} \left[V_{M}(\pi^\star) - V_{M}(\pi) - \frac{1}{\eta}  \KL\left( M^{\nu(\cdot|\phi)}(\cdot|\pi), \Bar{M}(\cdot|\pi) \right)\right]. 
\end{align*}

 \begin{align*}
\dec_{\eta}^{\textsc{KL}}\left(\bar\calM(\Phi)\right) = \max_{\Bar{M} \in \textsc{co}(\calM)} \min_{p \in \Delta(\Pi)}\max_{\phi\in\Phi}\max_{\nu \in \Delta(\phi)} \E_{\pi \sim p}\E_{(M,\pi^\star) \sim \nu}\left[V_M(\pi^\star) - V_M(\pi) - \frac{1}{\eta}D_{\textsc{KL}}\left(M(\cdot|\pi), \Bar{M}(\cdot|\pi)\right)\right]
\end{align*}
\subsection{Getting the expectation out of the KL}


\begin{align*}
    &\E_{\theta\sim\nu}\E_{\theta'\sim\nu}\left( \E_{o\sim \nu_{\cdot|\theta}}g_\theta(o)-\E_{o\sim \nu_{\cdot|\theta'}}g_\theta(o)\right)^2\\
    &\leq 4\E_{\theta\sim\nu}\E_{\theta'\sim\nu}\left[\left( \E_{o\sim \nu_{\cdot|\theta}}g_\theta(o)-\E_{o\sim \nu_{\cdot}}g_\theta(o)\right)^2+\left( \E_{o\sim \nu_{\cdot}}g_\theta(o)-\E_{o\sim \nu_{\cdot|\theta'}}g_\theta(o)\right)^2\right]\\
    &\leq 8\E_{\theta\sim\nu}\KL(\nu|\theta,\nu)
\end{align*}


\subsection{Making explicit use of the stage structure}
I think the key to resolve our issues is that in reality, we have a stage structure. If the prior $\nu$ would be of the form $\nu=\nu_1\times\dots\times\nu_H$, we resolve most issues. However, the minimax theorem does not work since $\nu$ with such a decomposition does not live in a convex set. I try if we can iteratively solve the problem.

Let's consider $\Psi_h = \Pi_h\times\calM_h$
$\Psi = \Psi_1\times\dots\times\Psi_H$.
Assume we keep a prior independently for all stages.

\begin{align*}
       &\E[\Reg(\pi^\star)] - \sum_{h=1}^H\frac{\log|\Phi_h|}{\eta} \\
       &\leq \E\left[\sum_{t=1}^T \left(V_{M_t}(\pi^\star) -  V_{M_t}(\pi_t) - \sum_{h=1}^H\frac{1}{\eta}\log\frac{\rho^h_{t+1}(\phi^\star_h)}{\rho^h_t(\phi^\star_h)} \right)\right] \\
       &= \E\left[\sum_{t=1}^T  \E_{\pi\sim p_t} \E_{o\sim M_t(\cdot|\pi)}\left[ V_{M_t}(\pi^\star) -  V_{M_t}(\pi) - \sum_{h=1}^H\frac{1}{\eta} \log\frac{\nu_t^h(\phi^\star_h|\pi, o)}{\rho_t^h(\phi^\star_h)} \right]\right] \\
       &\leq \E\left[ \sum_{t=1}^T  \max_{\nu_1\in\Delta(\joint_1),\dots,\nu_H\in\Delta(\joint_H)} \E_{\pi\sim p_t} \E_{(M,\pi^\star)\sim \nu} \E_{o\sim M(\cdot|\pi)}\left[ V_{M}(\pi^\star) -  V_{M}(\pi) - \sum_{h=1}^H\frac{1}{\eta} \KL\left( \nu^h_{\bo{\phi_h}}(\cdot|\pi, o), \rho^h_t\right) \right]\right]  \tag{ \textbf{TODO}}  \\
       &= \E\left[ \sum_{t=1}^T  \min_{p\in\Delta(\Pi)}\max_{\nu_1\in\Delta(\joint_1),\dots,\nu_H\in\Delta(\joint_H)} \E_{\pi\sim p} \E_{(M,\pi^\star)\sim \nu} \E_{o\sim M(\cdot|\pi)}\left[ V_{M}(\pi^\star) -  V_{M}(\pi) - \sum_{h=1}^H\frac{1}{\eta} \KL\left( \nu^h_{\bo{\phi_h}}(\cdot|\pi, o), \rho_t^h\right) \right]\right] \\
       &= \E\left[ \sum_{t=1}^T  \min_{p\in\Delta(\Pi)}\max_{\nu\in\Delta(\joint)}  
\air^\Phi_{\rho_t,\eta}(p,\nu) \right] \\
       &= \E\left[ \sum_{t=1}^T  \max_{\nu\in\Delta(\joint)}\min_{p\in\Delta(\Pi)}  
\air^\Phi_{\rho_t,\eta}(p,\nu) \right]   \tag{$\air^\Phi_{\rho_t,\eta}(p,\nu)$ is convex in $p$ and concave in $\nu$}  \\
    &\leq  T \max_{\rho \in \Delta(\Phi)}\max_{\nu\in\Delta(\joint)}\min_{p\in\Delta(\Pi)}  
\air^\Phi_{\rho,\eta}(p,\nu). 
    \end{align*}

Define the staged AIR as

\begin{align*}
    \air^{\Phi,h}_{\rho,\eta,s'}(p,\nu) = \E_{\pi\sim p} \E_{(M,\pi^\star)\sim \nu} \E_{o\sim M(\cdot|\pi)}\left[V_M(\pi^\star;s') - V_M(\pi\circ\pi^\star_{h+1:H};s') - \frac{1}{\eta} \KL(\nu_{\bo{\phi_h}}(\cdot|\pi, o), \rho_h)\right]. 
\end{align*}

\begin{algorithm}
\caption{General Stage wise Algorithm with improper policies} \label{alg:general staged AIR}
    Let $\rho_1^h(\phi) = \frac{1}{|\Phi_h|}$ for all $\phi \in \Phi_h$ for all $h\in [H]$. \\
    \For{$t=1,2,\ldots, T$}{
    \For{$h=1,\dots,H$}{
        Observe current state $s_h$.
       Find a distribution $p^h_t$ of $\pi$ and a distribution $\nu_t^h$ of $\Psi_{h:H}$ that solve the saddle-point of 
       \begin{align*}
           \min_{p\in\Delta(\Pi_h)} \max_{\nu\in \Delta(\Psi_{h:H})}  \air^{\Phi,h}_{\rho_t,\eta,s_h}(p,\nu). 
       \end{align*}
       Sample decision $\pi^h_t\sim p^h$ and observe $o^h_t\sim M^h_t(\cdot|\pi^h_t)$. \\
       Update $\rho^h_{t+1}(\phi_h)=\nu^h_{t}(\phi_h|\pi^h_t,o^h_t)$ for all $\phi\in\Phi_h$.  
       
    }
    }
\end{algorithm}

By the standard decomposition, we have

\begin{align*}
&\E[\Reg(\pi^\star)]+\sum_{h=1}^H\\
&=\E\left[\sum_{t=1}^T\sum_{h=1}^HV_{M_t}(\pi_t^{:h-1}\circ(\pi_t^\star)^{h:})-V_{M_t}(\pi_t^{:h}\circ(\pi_t^\star)^{h+1:})\right]\\
&=\sum_{h=1}^H\E\left[\sum_{t=1}^T\E\left[V^h_{M_t}((\pi_t^\star)^{h:};s_h)-V^h_{M_t}(\pi_t^{h}\circ(\pi_t^\star)^{h+1:};s_h)\right]\right]\\
\end{align*}