\documentclass[draft]{agujournal2019}
\usepackage{url}
% \usepackage{lineno}
\usepackage[inline]{trackchanges}
\usepackage{soul}

% \linenumbers
\newcommand{\maf}[1]{{\textcolor{red}{#1}}}
\draftfalse
\journalname{Journal of Advances in Modeling Earth Systems (JAMES)}

\begin{document}

%%%%%%%%%%%%%%%%%%%%%%%%%%%%%%%%%%%%%%%%%%%%%%%
\title{Multi-Year-to-Decadal Temperature Prediction using a Machine Learning Model-Analog Framework}
\authors{M. A. Fernandez\affil{1} and Elizabeth A. Barnes\affil{1}}
\affiliation{1}{Department of Atmospheric Science, Colorado State University, Fort Collins, CO, USA}
\correspondingauthor{M. A. Fernandez}{mafern@colostate.edu}
%%%%%%%%%%%%%%%%%%%%%%%%%%%%%%%%%%%%%%%%%%%%%%%
\begin{keypoints}
\item We use machine learning to identify important precursor regions for matching in a model-analog prediction framework.
\item We predict 2-meter temperature with lead times from one to ten years and for five different regions.
\item We find improved performance over other analog prediction methods and over initialized decadal predictions.
\end{keypoints}
%%%%%%%%%%%%%%%%%%%%%%%%%%%%%%%%%%%%%%%%%%%%%%%
\begin{abstract}
Multi-year-to-decadal climate prediction is a key tool in understanding the range of potential regional and global climate futures.
Here, we present a framework that combines machine learning and analog forecasting for predictions on these timescales.
A neural network is used to learn a mask, specific to a region and lead time, with global weights based on relative importance as precursors to the evolution of that prediction target.
A library of mask-weighted model states, or potential analogs, are then compared to a single mask-weighted observational state.
The known future of the best matching potential analogs serve as the prediction for the future of the observational state.
We match and predict 2-meter temperature using the Berkeley Earth Surface Temperature dataset for observations, and a set of CMIP6 models as the analog library.
We find improved performance over traditional analog methods and initialized decadal predictions.
\end{abstract}

\section*{Plain Language Summary}
Accurate prediction of the climate, from the next year to the next decade, is important for adaptation and mitigation in many societal sectors.
In particular, temperature plays a key role on these time scales for agriculture, human health, and infrastructure planning.
In this work, we develop and test a framework for making predictions on multi-year-to-decadal timescales, utilizing machine learning and an established method called analog forecasting.
Analog forecasting compares the climate state from which one wants to predict to a library of potential analog states.
The forecast is then the known future of the best matching analog states, which presumes that the analogs evolve similarly to the real world.
Here, our analogs come from a subset of Earth system model runs from the CMIP6 dataset.
The machine learning component contributes by learning the regions that are most important in determining if two climate states will evolve similarly, ensuring that the analog selection encodes information specific to the prediction task.
We find that our method improves on current prediction methods for these timescales in several regions and lead times.

%%%%%%%%%%%%%%%%%%%%%%%%%%%%%%%%%%%%%%%%%%%%%%%
\section{Introduction}
\label{sec:intro}

\begin{figure*}[tb]
    \centering
    \includegraphics[width=0.848\linewidth]{figs/circuitnn.pdf} 
    \caption{Illustration of differentiable CircuitNN. CircuitNN is designed based on differentiable NAND gates. After DAS is guided by PI and PO pairs of the truth table, CircuitNN can get the precise circuit architecture logic equivalent to the truth table.}
    \label{fig:circuitnn}
\end{figure*}

% 1. Describe the importance of logic synthesis
% 2. Existing Problems
% (a) Neural Architecture Search: Unstable, Predefined Setting, etc.
% (b) Circuit Generation: Probabilistic Model, Logic Equivalence

With the rapid advancement of technology, the scale of integrated circuits (ICs) has expanded exponentially. 
This expansion has introduced significant challenges in chip manufacturing, particularly concerning power and area metrics.
A primary objective in IC design is achieving the same circuit function with fewer transistors, thereby reducing power usage and area occupancy.

Logic synthesis~\cite{hachtel2005logicsynth}, a critical step in electronic design automation (EDA), transforms behavioral-level circuit designs into optimized gate-level circuits, ultimately yielding the final IC layout. 
The primary goal of logic synthesis is to identify the physical implementation with the fewest gates for a given circuit function. 
This task constitutes a challenging NP-hard combinatorial optimization problem. 
Current logic synthesis tools~\cite{brayton2010abc, wolf2013yosys} rely on human-designed heuristics, often leading to sub-optimal outcomes.

Differentiable architecture search (DAS) techniques~\cite{liu2018darts, chu2020darts} offer novel perspectives on addressing challenges in this problem.
Circuit functions can be represented through truth tables, which map binary inputs to their corresponding outputs. 
Truth tables provide a precise representation of input-output relationships, ensuring the design of functionally equivalent circuits.
Inspired by this, researchers~\cite{deepmind2024ai4sys, wang2024tnet} have begun exploring the application of DAS to synthesize circuits directly from truth tables.
Specifically, \citet{deepmind2024ai4sys} proposed CircuitNN, a framework that learns differentiable connection structures with logic gates, enabling the automatic generation of logic circuits from truth tables.
This approach significantly reduces the complexity of traditional circuit generation. 
Building on this, \citet{wang2024tnet} introduced T-Net, a triangle-shaped variant of CircuitNN, incorporating regularization techniques to enhance the efficiency of DAS.

Despite these advancements, several challenges remain. 
The computational complexity of DAS grows quadratically with the number of gates, posing scalability issues.
Although triangle-shaped architecture~\cite{wang2024tnet} partially mitigates this problem, redundancy persists. 
%Additionally, DAS is susceptible to converging to local optima, limiting the ability to search architectures that satisfy the given truth tables~\cite{liu2018darts}. 
%Furthermore, hyperparameters (network depth and layer width) require extensive searches, introducing complexity and prolonging the synthesis process. 
Additionally, DAS is susceptible to converging to local optima~\cite{liu2018darts} and hyperparameters (network depth and layer width) require extensive searches. 
The challenges arise from the vast search space in DAS. 
% Even with predefined settings for CircuitNN, finding a configuration that meets the truth table requires extensive trial and error during the DAS process. 
Intuitively, limiting the search space through predefined parameters (network depth, gates per layer, and connection probabilities) can significantly reduce the complexity.

Recent advances~\cite{openai2023gpt4, abramson2024alphafold3, esser2024sd3, li2024mar} in conditional generative models have demonstrated remarkable performance across language, vision, and graph generation tasks. 
Motivated by these developments, we propose a novel approach to circuit generation that generates preliminary circuit structures to guide DAS in generating refined circuits matching specified truth tables. 
Firstly, we introduce CircuitVQ, a tokenizer with a discrete codebook for circuit tokenization. 
Built upon our Circuit AutoEncoder framework~\cite{hou2022graphmae,li2023maskgae,wu2025mgvga}, CircuitVQ is trained through a circuit reconstruction task. 
Specifically, the CircuitVQ encoder encodes input circuits into discrete tokens using a learnable codebook, while the decoder reconstructs the circuit adjacency matrix based on these tokens.
Subsequently, the CircuitVQ encoder serves as a circuit tokenizer for CircuitAR pretraining, which employs a masked autoregressive modeling paradigm~\cite{chang2022maskgit, li2023mage}. 
In this process, the discrete codes function as supervision signals. 
After training, CircuitAR can generate discrete tokens progressively, which can be decoded into initial circuit structures by the decoder of the CircuitVQ. 
These prior insights can guide DAS in producing refined circuits that match the target truth tables precisely.

Our key contributions can be summarized as follows:
\begin{itemize}
\item We introduce CircuitVQ, a circuit tokenizer that facilitates graph autoregressive modeling for circuit generation, based on our Circuit AutoEncoder framework;
\item Develop CircuitAR, a model trained using masked autoregressive modeling, which generates initial circuit structures conditioned on given truth tables;
\item Propose a refinement framework that integrates differentiable architecture search to produce functionally equivalent circuits guided by target truth tables;
\item Comprehensive experiments demonstrating the scalability and capability emergence of our CircuitAR and the superior performance of the proposed circuit generation approach.
\end{itemize}

% Motivation
% (a) Diffusion (Vision, Graph), Autoregressive (Language, Vision)
% (b) Circuit Generation for Predefined Setting
% (c) Neural Architecture Search for Strict Logic Equivalence

% Contribution
% (a) Circuit Tokenizer (new transformer arch, training strategy)
% (b) CircuitAR (train and gen strategies, post-ar strategy)
% (c) Extensive Evaluation including BitD (Bit Distance) for Scalability

\section{Method}

\subsection{Overview \& Setup}

Our framework consists of a large, highly capable model \textbf{\bigmodel} and a smaller, resource-efficient model \textbf{\smallmodel}. We assume that $S \in \mathbb{N}$ and $L \in \mathbb{N}$ represent the parameter count of each model with $S \ll L$. Both models can either function as classifiers (i.e., $\mathcal{M}: \mathbb{R}^D \rightarrow [C]$ with $\mathbb{R}^D$ denoting the input space and $C$ the number of total classes), or (multi-modal) sequence models (i.e., $\mathcal{M}: \mathbb{R}^D \rightarrow [V]^{T}$ where $V$ is the vocabulary and $T$ is the sequence length). We include experiments on all of these model classes in Section~\ref{sec:experiments}. Furthermore, we do not require a shared model family to be deployed on both \smallmodel and \bigmodel; for example, \smallmodel could be a custom convolutional neural network optimized for efficient inference and \bigmodel a vision transformer~\citep{dosovitskiy2020image}. The primary objective is to design a deferral mechanism that enables \smallmodel to decide when to return its predictions without the assistance of \bigmodel and when to instead defer to it.

\looseness=-1
Deferral decisions are made using signals derived from the small model \smallmodel as this approach is typically more cost-effective than employing a separate routing mechanism~\citep{teerapittayanon2016branchynet}. Approaches that involve querying the large model \bigmodel to assist in making deferral decisions at test time are excluded from our setup. Such methods --- common in domains like LLMs --- are counterproductive to our goal since querying \bigmodel defeats the purpose of making a deferral decision in the first place?. Examples of these inapplicable methods include collaborative LLM frameworks~\citep{mielke2022reducing} and techniques that rely on semantic entropy for uncertainty estimation~\citep{kuhn2023semantic}. As part of our setup, we assume that \smallmodel is strictly less capable than \bigmodel --- a realistic scenario in practice supported by scaling laws~\citep{kaplan2020scaling}. Under this assumption, mistakes made by \bigmodel are also made by \smallmodel; however, \smallmodel may make additional errors that \bigmodel would avoid. This reflects the general observation that larger models tend to outperform smaller models across a wide range of tasks.

As discussed in Section~\ref{sec:related-word}, the choice of deferral strategy often depends on the level of access available to \smallmodel. We assume white box access with full access to \smallmodel's internals. As such, deferral mechanisms can be directly integrated into the model's architecture and parameters. This involves fine-tuning \smallmodel to predict deferral decisions or to incorporate rejection mechanisms within its predictive process. Our work falls into this category as it proposes a new loss function to fine-tune \smallmodel. 

Our goal is to train a small model that can effectively distinguish between correct and incorrect predictions. While many past works have considered the question of whether it is possible to find proxy measures for prediction correctness, the central question we ask is:
\begin{center}
\textbf{Can we \emph{optimize} the small model \smallmodel to separate correct from incorrect predictions?}
\end{center}
\noindent We show that this is indeed achievable through a carefully designed fine-tuning stage that does not require any architectural modifications. This ensures that the ability to separate correct from incorrect decisions is integrated seamlessly into \smallmodel's existing structure.


\subsection{Confidence-Tuning for Deferral}

\begin{figure}
    \centering
    \resizebox{\linewidth}{!}{
    \begin{figure}[h]
\begin{center}
   \includegraphics[width=0.99\linewidth]{figs/pdf/loss.pdf}
\end{center}
   \caption{
    Training loss of VAR \textit{vs.} FlexVAR. FlexVAR demonstrates a faster convergence rate. We report the results with trained 40 epochs ($\sim$ 70K iterations). 
   }
\label{fig:loss}
\end{figure}

    }
    \vspace{-15pt}
    \caption{\textbf{Overview of \loss}: We want correctly predicted samples maintain their current prediction by ensuring that cross entropy is decreased (top, green). At the same time, we want incorrectly predicted samples to yield a uniform confidence across all classes, leading to a low overall confidence score (bottom, red).}
    \label{fig:opt_goal}
\end{figure}

\textbf{Stage 1: Standard Training.} We begin with a \smallmodel that has already been trained on the tasks it is intended to perform upon deployment. However, due to its limited capacity, \smallmodel cannot achieve the performance levels of \bigmodel. Importantly, we make no assumptions about the training process of \smallmodel—whether it was trained from scratch without supervision from an external model or through a distillation approach.

\sloppy
\textbf{Stage 2: Correctness-Aware Finetuning with \loss.} Next, we introduce a correctness-aware loss, dubbed \loss, to fine-tune \smallmodel for improved confidence calibration. Specifically, the model is trained to make correct predictions with high confidence while reducing the confidence of incorrect predictions (see Figure~\ref{fig:opt_goal}). This loss can either rely on true labels or utilize the outputs of \bigmodel with soft probabilities as targets. 


For a standard classification model, the calibration loss is defined as the following hybrid loss
\begin{align}
\mathcal{L} &= \alpha \mathcal{L}_\text{corr} + (1 - \alpha) \mathcal{L}_\text{incorr} \\
\mathcal{L}_\text{corr} &= \frac{1}{N} \sum_{i=1}^{N} \mathds{1}\{ y_i = \hat{y}_i \} \text{CE}(p_i(\mathbf{x}_i), y_i) \\
\mathcal{L}_\text{incorr} &= \frac{1}{N} \sum_{i=1}^{N} \mathds{1}\{ y_i \neq \hat{y}_i \} \text{KL}\left(p_i(\mathbf{x}_i) \parallel \mathcal{U}\right)
\end{align}
where  \( y_i \) and \( \hat{y}_i \) are the true and predicted labels for $\mathbf{x}_i$, respectively, \( p_i \) is the predicted probability distribution of \smallmodel over classes, \( \mathcal{U} \) represents the uniform distribution over all classes, \( N \) denotes the number samples in the current batch, \( \alpha \in (0, 1) \) is a tunable hyperparameter controlling the emphasis between correct and incorrect predictions, and the cross-entropy function and KL divergence are defined as \( \text{CE}(p, y) = -\sum_{c} y_c \log p_c \) and \( \text{KL}(p \parallel q) = \sum_{c} p_c \log ( \frac{p_c}{q_c}) \), respectively. We note that a similar loss has previously been proposed in Outlier Exposure (OE)~\citep{hendrycks2018deep} for out-of-distribution (OOD) sample detection. Here, the goal is to make sure that OOD examples are assigned low confidence scores by tuning the confidence on a auxiliary outlier dataset. However, to the best of our knowledge, this idea has not previously been used to improve deferral performance of a smaller model in a cascading chain.

We emphasize that the trade-off parameter $\alpha$ plays a critical role as part of this optimization setup as it directly influences model utility and deferral performance. A lower value of \(\alpha\) emphasizes reducing confidence in incorrect predictions by pushing them closer to the uniform distribution, making the model more cautious in regions where it may make mistakes. Conversely, a higher value of \(\alpha\) encourages the model to increase its confidence on correct predictions, sharpening its decision boundaries and enhancing accuracy where it is already performing well. Thus, \(\alpha\) serves as a crucial hyperparameter that balances the trade-off between improving calibration by mitigating overconfidence in errors and reinforcing confidence in accurate classifications. By appropriately tuning \(\alpha\), practitioners can control the model’s behavior to achieve a desired balance between reliability in uncertain regions and decisiveness in confident predictions, tailored to the specific requirements of their application.

We further generalize this loss to token-based models (e.g., LMs and VLMs), formulated as
\ifarxiv
\small
\fi
\begin{align}
    \mathcal{L}_\text{corr} & = \frac{1}{N} \sum_{i=1}^{N} \sum_{t=1}^{T} \mathds{1}\{ y_{i,t} = \hat{y}_{i,t} \} \text{CE}(p_{i,t}(\mathbf{x}_i), y_{i,t}) \\
    \mathcal{L}_\text{incorr} & = \frac{1}{N} \sum_{i=1}^{N} \sum_{t=1}^{T} \mathds{1}\{ y_{i,t} \neq \hat{y}_{i,t} \} \text{KL}\left(p_{i,t}(\mathbf{x}_i) \parallel \mathcal{U}\right)
\end{align}
\normalsize
where \( y_{i,t} \) and \( \hat{y}_{i,t} \) denote the true and predicted tokens at position \( t \) for sample \( i \), \( p_{i,t} \) is the predicted token distribution at position \( t \) for sample \( i \), and \( T \) is the sequence length for the token-based model. The token-level loss ensures that correct token predictions are made confidently while incorrect tokens are assigned smaller confidences.

\sloppy
\textbf{Stage 3: Confidence Computation \& Thresholding.} After fine-tuning \smallmodel with \loss, we apply standard confidence- and entropy-based techniques for model uncertainty to obtain a deferral signal. We use the selective prediction framework to determine whether a query point~$\mathbf{x} \in \mathbb{R}^D$ should be accepted by \smallmodel or routed to \bigmodel. Selective prediction alters the model inference stage by introducing a deferral state through a \textit{gating mechanism}~\citep{yaniv2010riskcoveragecurve}. At its core, this mechanism relies on a deferral function $g:\mathbb{R}^D \rightarrow \mathbb{R}$ which determines if \smallmodel should output a prediction for a sample~$\mathbf{x}$ or defer to \bigmodel. Given a targeted acceptance threshold $\tau$, the resulting predictive model can be summarized as:
\begin{equation}
\label{eq:deferral}
    (\mathcal{M}_S,\mathcal{M}_L,g)(\mathbf{x}) = \begin{cases}
  \mathcal{M}_S(\mathbf{x})  & g(\mathbf{x}) \geq \tau \\
  \mathcal{M}_L(\mathbf{x}) & \text{otherwise.}
\end{cases}
\end{equation}

\emph{Classification Models (Max Softmax).} Let \(\mathcal{M}_S\) produce a categorical distribution
\(\{p(y=c \mid \mathbf{x})\}_{c=1}^C\) over \(C\) classes. 
Then we define the gating function as
\begin{align}
g_{\text{CL}}(\mathbf{x}) \;=\; \max_{1 \,\le\, c \,\le\, C}\;p\bigl(y = c \,\big\vert\, \mathbf{x}\bigr).
\end{align}

\emph{Token-based Models (Negative Predictive Entropy).} 
Let \(\mathcal{M}_S\) produce a sequence of categorical distributions 
\(\{p(y_t = c \mid \mathbf{x})\}_{c=1}^C\) for each token index \(t \in T\). Then we define the gating function as
\begin{equation}
\footnotesize
g_{\text{NENT}}(\mathbf{x}) 
= \; \frac{1}{T} \sum_{t=1}^{T} \sum_{c=1}^{C} 
    p\bigl(y_t = c \,\big\vert\, \mathbf{x}\bigr)\,\log p\bigl(y_t = c \,\big\vert\, \mathbf{x}\bigr),
\end{equation}
where \(y_t \in [C]\) is the predicted token at time step \(t\), \(p(y_t=c \mid \mathbf{x})\) is the (conditional) probability of token \(k\) at step \(t\), and \(T\) is the total number of token positions for the sequence. Across both model classes, higher values of either $g_{\text{CL}}$ or $g_{\text{NENT}}$ indicate higher confidence in the predicted class or sequence generation, respectively.
\begin{table}[ht!]
\centering
\caption{\textbf{Super Resolution Performance Results.} Our proposed WGAN EEG Spatial Upsampling method significantly outperforms a baseline of Bicubic Interpolation commonly used in EEG upsampling pipelines.}
\label{tab:results}
\resizebox{0.8\linewidth}{!}{%
\begin{tabular}{@{}cccccc@{}}
\toprule
\multirow{2}{*}{\textbf{Dataset}} & \multirow{2}{*}{\textbf{Scale}} & \multicolumn{2}{c}{\textbf{Bicubic}} & \multicolumn{2}{c}{\textbf{WGAN}} \\ \cmidrule(l){3-6} 
                      &   & \textbf{MSE} & \textbf{MAE} & \textbf{MSE}    & \textbf{MAE}   \\
\toprule
\multirow{2}{*}{Val}  & 2 & 3.71E7       & 3.89E3       & \textbf{2.01E3} & \textbf{24.38} \\
                      & 4 & 7.23E7       & 6.42E3       & \textbf{8.53E3} & \textbf{63.83} \\
\midrule
\multirow{2}{*}{Test} & 2 & 3.75E7       & 3.91E3       & \textbf{2.06E3} & \textbf{24.66} \\
                      & 4 & 7.30E7       & 6.45E3       & \textbf{8.68E3} & \textbf{64.39} \\
\bottomrule
\end{tabular}%
}
\end{table}
\section{Conclusion Remarks}
This work proposes a RBG graph model for disease spreading via hubs. We study the joint effect of the agent density, hub density, and connection function. The existence of a critical hub density depends only on the boundedness of the support of the connection function, which relates to curbing the traveling distance of individuals. When it comes to dispersion, both the degree distribution and the percolation threshold suggest that increasing dispersion helps spread the disease. The percolation properties of RBG graphs relate to unipartite graphs with modified connection functions. 
An interesting question in this direction is if and when the properties of the RBG graphs can be well represented by unipartite graphs with some modified connection functions. Our conjecture is that for independent connections between different pairs of agents, such representation is unlikely due to the oblivion of the local dependence (present in the RBG models). 
 Another direction is to consider hybrid models where agents may get infected either through common hubs or direct interactions between agents. The former infection mechanism is more centralized than the latter. 
%%%%%%%%%%%%%%%%%%%%%%%%%%%%%%%%%%%%%%%%%%%%%%%
% \appendix
% \begin{appendices}
\renewcommand{\thetable}{A\arabic{table}}
\counterwithin{table}{section}
\section{Appendices}
The full list of examined primary publications is enumerated in Table\ref{tab:flep}.

\begin{table}[ht!]
\caption{Full List Of Examined Publications
\label{tab:flep}}
\begin{adjustbox}{width=\linewidth, center}
\centering
\begin{tabular}{llll}
\toprule
\textbf{Year} & \textbf{Venue Type} & \textbf{Venue}               & \textbf{Title}              \\

\textsc{Conference} & & &\\   
\midrule
2023          & Conference          & AINA                         & A Vulnerability Detection Method for SDN with Optimized Fuzzing\citep{10.1007/978-3-031-28451-9_46}                                                                            \\
2022          & Conference          & ICAIS                        & Test Traffic Control Based on REST API for Software-Defined Networking\citep{10.1007/978-3-031-06788-4_40}                                                                     \\
2021          & Conference          & SOSR                         & Tardis: A Fault-Tolerant Design for Network Control Planes\citep{10.1145/3482898.3483355}                                                                                      \\
2021          & Conference          & NISS                         & SDN Control Plane Security: Attacks and Mitigation Techniques\citep{10.1145/3454127.3456612}                                                                                   \\
2021          & Conference          & DSN                          & A Comprehensive Study of Bugs in Software Defined Networks\citep{9505089}                                                                                                      \\
2021          & Conference          & NTMS                         & SDN Security through System Call Learning\citep{9432640}                                                                                                                       \\
2020          & Conference          & CIC                          & \begin{tabular}[c]{@{}l@{}}ParaSDN: An Access Control Model for SDN Applications based on Parameterized Roles and \\ Permissions\citep{9319021}\end{tabular}                   \\
2020          & Conference          & SP                           & Unexpected Data Dependency Creation and Chaining: A New Attack to SDN\citep{9152642}                                                                                           \\
2020          & Conference          & CSPS                         & Deep Learning Based Detection Method for SDN Malicious Applications\citep{10.1007/978-981-13-6508-9_13}                                                                        \\
2020          & Conference          & INFOCOM                      & AudiSDN: Automated Detection of Network Policy Inconsistencies in Software-Defined Networks\citep{9155378}                                                                     \\
2020          & Conference          & WISA                         & FSF: Code Coverage-Driven Fuzzing for Software-Defined Networking\citep{10.1007/978-3-030-39303-8_4}                                                                           \\
2020          & Conference          & CAV                          & Towards Model Checking Real-World Software-Defined Networks\citep{10.1007/978-3-030-53291-8_8}                                                                                 \\
2019          & Conference          & IM                           & Mining Software Repositories for Predictive Modelling of Defects in SDN Controller\citep{8717837}                                                                              \\
2019          & Conference          & IM                           & Thinking inside the Box: Differential Fault Localization for SDN Control Plane\citep{8717815}                                                                                  \\
2018          & Conference          & NFV-SDN                      & RE-CHECKER: Towards Secure RESTful Service in Software-Defined Networking\citep{8725649}                                                                                       \\
2018          & Conference          & ICNP                         & INDAGO: A New Framework For Detecting Malicious SDN Applications\citep{8526819}                                                                                                \\
% 2018          & Chapter             &                              & Security Analysis of FloodLight, ZeroSDN, Beacon and POX SDN Controllers\citep{Ilyas2018}                                                                                      \\
2018          & Conference          & CCS                          & AIM-SDN: Attacking Information Mismanagement in SDN-datastores\citep{10.1145/3243734.3243799}                                                                                  \\
2018          & Conference          & ICC                          & SENAD: Securing Network Application Deployment in Software Defined Networks\citep{8422405}                                                                                     \\
% 2018          & Chapter             &                              & Security Analysis of SDN Routing Applications\citep{Sagare2018}                                                                                                                \\
2017          & Conference          & NSDI                         & Automated Bug Removal for Software-Defined Networks\citep{10.5555/3154630.3154688}                                                                                             \\
2017          & Conference          & MASCOTS                      & Testing Black-Box SDN Applications with Formal Behavior Models\citep{8107437}                                                                                                  \\
2017          & Conference          & ICC                          & Controller DAC: Securing SDN controller with dynamic access control\citep{7997249}                                                                                             \\
2017          & Conference          & ISSREW                       & Analytics-Enhanced Automated Code Verification for Dependability of Software-Defined Networks\citep{8109275}                                                                   \\
2017          & Conference          & SIGCOMM                      & BigBug: Practical Concurrency Analysis for SDN\citep{10.1145/3050220.3050230}                                                                                                  \\
2017          & Conference          & ICCNT                        & Requirement analysis for abstracting security in software-defined network\citep{8204161}                                                                                       \\
2017          & Conference          & CNSM                         & An empirical study of software reliability in SDN controllers\citep{8256002}                                                                                                   \\
2017          & Conference          & BWCCA                        & A Comprehensive Security Analysis Checksheet for OpenFlow Networks\citep{10.1007/978-3-319-49106-6_22}                                                                         \\
2017          & Conference          & RAID                         & BEADS: Automated Attack Discovery in OpenFlow-Based SDN Systems\citep{10.1007/978-3-319-66332-6_14}                                                                            \\
2017          & Conference          & MIWAI                        & DREAD-R: Severity Assessment of ONOS SDN Controller\citep{10.1007/978-3-319-69456-6_27}                                                                                        \\
2016          & Conference          & CODASPY                      & SHIELD: An Automated Framework for Static Analysis of SDN Applications\citep{10.1145/2876019.2876026}                                                                          \\
2016          & Conference          & CODASPY                      & The Smaller, the Shrewder: A Simple Malicious Application Can Kill an Entire SDN Environment\citep{10.1145/2876019.2876024}                                                    \\
2016          & Conference          & ISSREW                       & Programming the Network: Application Software Faults in Software-Defined Networks\citep{7789391}                                                                               \\
2016          & Conference          & ICSP                         & BuDDI: Bug detection, debugging, and isolation middlebox for software-defined network controllers\citep{7818438}                                                               \\
2016          & Conference          & ICIEV                        & SDN testing and debugging tools: A survey\citep{7760078}                                                                                                                       \\
2015          & Conference          & ICUFN                        & Secure your Northbound SDN API\citep{7182679}                                                                                                                                  \\
2015          & Conference          & TSA                          & Model-Based Testing of SDN Firewalls: A Case Study\citep{7335947}                                                                                                              \\
2015          & Conference          & SIGCOMM                          & Automated Network Repair with Meta Provenance\citep{10.1145/2834050.2834112}                                                                                               \\
2015          & Conference          & GPCE                         & Safer SDN programming through Arbiter\citep{10.1145/2936314.2814218}                                                                                                           \\
2015          & Conference          & NetSoft                      & Design and deployment of secure, robust, and resilient SDN controllers\citep{7258233}                                                                                          \\
2014          & Conference          & CONEXT                       & Controller-agnostic SDN Debugging\citep{10.1145/2674005.2674993}                                                                                                               \\
2014          & Conference          & PLDI                         & VeriCon: towards verifying controller programs in software-defined networks\citep{10.1145/2666356.2594317}                                                                     \\
2014          & Conference          & NSDI                         & Tierless programming and reasoning for software-defined networks\citep{10.5555/2616448.2616496}                                                                                \\
2014          & Conference          & CONEXT                       & Model Based Black-Box Testing of SDN Applications\citep{10.1145/2680821.2680828}                                                                                               \\
2014          & Conference          & SIGCOMM                      & Troubleshooting blackbox SDN control software with minimal causal sequences\citep{10.1145/2740070.2626304}                                                                     \\
2013          & Conference          & PLDI                         & Machine-verified network controllers\citep{10.1145/2499370.2462178}                                                                                                            \\
2012          & Conference          & NSDI                         & A NICE Way to Test OpenFlow Applications\citep{180591}                                                                                                                         \\
              &                     &                              &                                                                                                                                                                                \\
\textsc{Journal} & & &\\ \hline
2023          & Journal             & WWW                          & DACAS: integration of attribute-based access control for northbound interface security in SDN\citep{Liu2023}                                                                   \\
2022          & Journal             & ToN                          & A Framework for Policy Inconsistency Detection in Software-Defined Networks\citep{9681706}                                                                                     \\
2021          & Journal             & CSUR                         & Application Threats to Exploit Northbound Interface Vulnerabilities in Software Defined Networks\citep{10.1145/3453648}                                                        \\
2021          & Journal             & JPDC                         & \begin{tabular}[c]{@{}l@{}}SEAPP: A secure application management framework based on REST API access control in SDN-enabled \\ cloud environment\citep{HU2021108}\end{tabular} \\
2020          & Journal             & IJCNA                        & Security in SDN: A comprehensive survey\citep{CORREACHICA2020102595}                                                                                                           \\
2019          & Journal             & COMST                        & Fault Management in Software-Defined Networking: A Survey\citep{8456508}                                                                                                       \\
2019          & Journal             & COMST                        & A Survey on Network Verification and Testing With Formal Methods: Approaches and Challenges\citep{8453007}                                                                     \\
2019          & Journal             & JAMT                         & Modeling and Verifying Basic Modules of Floodlight\citep{Xiang2019}                                                                                                            \\
2018          & Journal             & TNSM                         & Assessing the Maturity of SDN Controllers With Software Reliability Growth Models\citep{8386840}                                                                               \\
2018          & Journal             & JSAC                         & MORPH: An Adaptive Framework for Efficient and Byzantine Fault-Tolerant SDN Control Plane\citep{8490892}                                                                       \\
2017          & Journal             & COMST                        & A Survey on Fault Management in Software-Defined Networks\citep{7959044}                                                                                                       \\
2016          & Journal             & IJCNA                        & Secure and dependable software defined networks\citep{AKHUNZADA2016199}                                                                                                        \\
2016          & Journal             & MONET                        & Security in Software-Defined Networking: Threats and Countermeasures\citep{Shu2016}                                                                                            \\

\bottomrule

\end{tabular}
\end{adjustbox}
\end{table}
\end{appendices}

% %%%%%%%%%%%%%%%%%%%%%%%%%%%%%%%%%%%%%%%%%%%%%%%
\section*{Open Research Section}
Berkeley Earth Surface Temperature \cite<BEST;>{best} can be obtained from \url{https://berkeleyearth.org/data/}, CMIP6 data and initialized Earth system model runs can be obtained from the Earth System Federation Grid \cite{esgf} at \url{https://esgf.github.io/index.html}.
% Code underlying the machine learning model-analog framework presented is available at \url{https://github.com/mafern/mask-analog-predictions}.
Code underlying the machine learning model-analog framework presented will be made available upon publication at \url{https://github.com/mafern/mask-analog-predictions}.

\acknowledgments
Funding for this project was provided, in part, by the Regional and Global Model Analysis program area of the U.S. Department of Energy’s (DOE) Office of Biological and Environmental Research (BER) as part of the Program for Climate Model Diagnosis and Intercomparison Project and by the Global Water Security Center through DoD HQ00342420014.

%%%%%%%%%%%%%%%%%%%%%%%%%%%%%%%%%%%%%%%%%%%%%%%
\bibliography{references}
\clearpage
%%%%%%%%%%%%%%%%%%%%%%%%%%%%%%%%%%%%%%%%%%%%%%%
\appendix
\begin{appendices}
\renewcommand{\thetable}{A\arabic{table}}
\counterwithin{table}{section}
\section{Appendices}
The full list of examined primary publications is enumerated in Table\ref{tab:flep}.

\begin{table}[ht!]
\caption{Full List Of Examined Publications
\label{tab:flep}}
\begin{adjustbox}{width=\linewidth, center}
\centering
\begin{tabular}{llll}
\toprule
\textbf{Year} & \textbf{Venue Type} & \textbf{Venue}               & \textbf{Title}              \\

\textsc{Conference} & & &\\   
\midrule
2023          & Conference          & AINA                         & A Vulnerability Detection Method for SDN with Optimized Fuzzing\citep{10.1007/978-3-031-28451-9_46}                                                                            \\
2022          & Conference          & ICAIS                        & Test Traffic Control Based on REST API for Software-Defined Networking\citep{10.1007/978-3-031-06788-4_40}                                                                     \\
2021          & Conference          & SOSR                         & Tardis: A Fault-Tolerant Design for Network Control Planes\citep{10.1145/3482898.3483355}                                                                                      \\
2021          & Conference          & NISS                         & SDN Control Plane Security: Attacks and Mitigation Techniques\citep{10.1145/3454127.3456612}                                                                                   \\
2021          & Conference          & DSN                          & A Comprehensive Study of Bugs in Software Defined Networks\citep{9505089}                                                                                                      \\
2021          & Conference          & NTMS                         & SDN Security through System Call Learning\citep{9432640}                                                                                                                       \\
2020          & Conference          & CIC                          & \begin{tabular}[c]{@{}l@{}}ParaSDN: An Access Control Model for SDN Applications based on Parameterized Roles and \\ Permissions\citep{9319021}\end{tabular}                   \\
2020          & Conference          & SP                           & Unexpected Data Dependency Creation and Chaining: A New Attack to SDN\citep{9152642}                                                                                           \\
2020          & Conference          & CSPS                         & Deep Learning Based Detection Method for SDN Malicious Applications\citep{10.1007/978-981-13-6508-9_13}                                                                        \\
2020          & Conference          & INFOCOM                      & AudiSDN: Automated Detection of Network Policy Inconsistencies in Software-Defined Networks\citep{9155378}                                                                     \\
2020          & Conference          & WISA                         & FSF: Code Coverage-Driven Fuzzing for Software-Defined Networking\citep{10.1007/978-3-030-39303-8_4}                                                                           \\
2020          & Conference          & CAV                          & Towards Model Checking Real-World Software-Defined Networks\citep{10.1007/978-3-030-53291-8_8}                                                                                 \\
2019          & Conference          & IM                           & Mining Software Repositories for Predictive Modelling of Defects in SDN Controller\citep{8717837}                                                                              \\
2019          & Conference          & IM                           & Thinking inside the Box: Differential Fault Localization for SDN Control Plane\citep{8717815}                                                                                  \\
2018          & Conference          & NFV-SDN                      & RE-CHECKER: Towards Secure RESTful Service in Software-Defined Networking\citep{8725649}                                                                                       \\
2018          & Conference          & ICNP                         & INDAGO: A New Framework For Detecting Malicious SDN Applications\citep{8526819}                                                                                                \\
% 2018          & Chapter             &                              & Security Analysis of FloodLight, ZeroSDN, Beacon and POX SDN Controllers\citep{Ilyas2018}                                                                                      \\
2018          & Conference          & CCS                          & AIM-SDN: Attacking Information Mismanagement in SDN-datastores\citep{10.1145/3243734.3243799}                                                                                  \\
2018          & Conference          & ICC                          & SENAD: Securing Network Application Deployment in Software Defined Networks\citep{8422405}                                                                                     \\
% 2018          & Chapter             &                              & Security Analysis of SDN Routing Applications\citep{Sagare2018}                                                                                                                \\
2017          & Conference          & NSDI                         & Automated Bug Removal for Software-Defined Networks\citep{10.5555/3154630.3154688}                                                                                             \\
2017          & Conference          & MASCOTS                      & Testing Black-Box SDN Applications with Formal Behavior Models\citep{8107437}                                                                                                  \\
2017          & Conference          & ICC                          & Controller DAC: Securing SDN controller with dynamic access control\citep{7997249}                                                                                             \\
2017          & Conference          & ISSREW                       & Analytics-Enhanced Automated Code Verification for Dependability of Software-Defined Networks\citep{8109275}                                                                   \\
2017          & Conference          & SIGCOMM                      & BigBug: Practical Concurrency Analysis for SDN\citep{10.1145/3050220.3050230}                                                                                                  \\
2017          & Conference          & ICCNT                        & Requirement analysis for abstracting security in software-defined network\citep{8204161}                                                                                       \\
2017          & Conference          & CNSM                         & An empirical study of software reliability in SDN controllers\citep{8256002}                                                                                                   \\
2017          & Conference          & BWCCA                        & A Comprehensive Security Analysis Checksheet for OpenFlow Networks\citep{10.1007/978-3-319-49106-6_22}                                                                         \\
2017          & Conference          & RAID                         & BEADS: Automated Attack Discovery in OpenFlow-Based SDN Systems\citep{10.1007/978-3-319-66332-6_14}                                                                            \\
2017          & Conference          & MIWAI                        & DREAD-R: Severity Assessment of ONOS SDN Controller\citep{10.1007/978-3-319-69456-6_27}                                                                                        \\
2016          & Conference          & CODASPY                      & SHIELD: An Automated Framework for Static Analysis of SDN Applications\citep{10.1145/2876019.2876026}                                                                          \\
2016          & Conference          & CODASPY                      & The Smaller, the Shrewder: A Simple Malicious Application Can Kill an Entire SDN Environment\citep{10.1145/2876019.2876024}                                                    \\
2016          & Conference          & ISSREW                       & Programming the Network: Application Software Faults in Software-Defined Networks\citep{7789391}                                                                               \\
2016          & Conference          & ICSP                         & BuDDI: Bug detection, debugging, and isolation middlebox for software-defined network controllers\citep{7818438}                                                               \\
2016          & Conference          & ICIEV                        & SDN testing and debugging tools: A survey\citep{7760078}                                                                                                                       \\
2015          & Conference          & ICUFN                        & Secure your Northbound SDN API\citep{7182679}                                                                                                                                  \\
2015          & Conference          & TSA                          & Model-Based Testing of SDN Firewalls: A Case Study\citep{7335947}                                                                                                              \\
2015          & Conference          & SIGCOMM                          & Automated Network Repair with Meta Provenance\citep{10.1145/2834050.2834112}                                                                                               \\
2015          & Conference          & GPCE                         & Safer SDN programming through Arbiter\citep{10.1145/2936314.2814218}                                                                                                           \\
2015          & Conference          & NetSoft                      & Design and deployment of secure, robust, and resilient SDN controllers\citep{7258233}                                                                                          \\
2014          & Conference          & CONEXT                       & Controller-agnostic SDN Debugging\citep{10.1145/2674005.2674993}                                                                                                               \\
2014          & Conference          & PLDI                         & VeriCon: towards verifying controller programs in software-defined networks\citep{10.1145/2666356.2594317}                                                                     \\
2014          & Conference          & NSDI                         & Tierless programming and reasoning for software-defined networks\citep{10.5555/2616448.2616496}                                                                                \\
2014          & Conference          & CONEXT                       & Model Based Black-Box Testing of SDN Applications\citep{10.1145/2680821.2680828}                                                                                               \\
2014          & Conference          & SIGCOMM                      & Troubleshooting blackbox SDN control software with minimal causal sequences\citep{10.1145/2740070.2626304}                                                                     \\
2013          & Conference          & PLDI                         & Machine-verified network controllers\citep{10.1145/2499370.2462178}                                                                                                            \\
2012          & Conference          & NSDI                         & A NICE Way to Test OpenFlow Applications\citep{180591}                                                                                                                         \\
              &                     &                              &                                                                                                                                                                                \\
\textsc{Journal} & & &\\ \hline
2023          & Journal             & WWW                          & DACAS: integration of attribute-based access control for northbound interface security in SDN\citep{Liu2023}                                                                   \\
2022          & Journal             & ToN                          & A Framework for Policy Inconsistency Detection in Software-Defined Networks\citep{9681706}                                                                                     \\
2021          & Journal             & CSUR                         & Application Threats to Exploit Northbound Interface Vulnerabilities in Software Defined Networks\citep{10.1145/3453648}                                                        \\
2021          & Journal             & JPDC                         & \begin{tabular}[c]{@{}l@{}}SEAPP: A secure application management framework based on REST API access control in SDN-enabled \\ cloud environment\citep{HU2021108}\end{tabular} \\
2020          & Journal             & IJCNA                        & Security in SDN: A comprehensive survey\citep{CORREACHICA2020102595}                                                                                                           \\
2019          & Journal             & COMST                        & Fault Management in Software-Defined Networking: A Survey\citep{8456508}                                                                                                       \\
2019          & Journal             & COMST                        & A Survey on Network Verification and Testing With Formal Methods: Approaches and Challenges\citep{8453007}                                                                     \\
2019          & Journal             & JAMT                         & Modeling and Verifying Basic Modules of Floodlight\citep{Xiang2019}                                                                                                            \\
2018          & Journal             & TNSM                         & Assessing the Maturity of SDN Controllers With Software Reliability Growth Models\citep{8386840}                                                                               \\
2018          & Journal             & JSAC                         & MORPH: An Adaptive Framework for Efficient and Byzantine Fault-Tolerant SDN Control Plane\citep{8490892}                                                                       \\
2017          & Journal             & COMST                        & A Survey on Fault Management in Software-Defined Networks\citep{7959044}                                                                                                       \\
2016          & Journal             & IJCNA                        & Secure and dependable software defined networks\citep{AKHUNZADA2016199}                                                                                                        \\
2016          & Journal             & MONET                        & Security in Software-Defined Networking: Threats and Countermeasures\citep{Shu2016}                                                                                            \\

\bottomrule

\end{tabular}
\end{adjustbox}
\end{table}
\end{appendices}

%%%%%%%%%%%%%%%%%%%%%%%%%%%%%%%%%%%%%%%%%%%%%%%

\end{document}

