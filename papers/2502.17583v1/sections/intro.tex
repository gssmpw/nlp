% --------------------------------------------------------------------------
% --------------------------------------------------------------------------
\section{Introduction}\label{sec:intro}

Regional climate prediction from years to decades is of growing interest to the climate science community, governing bodies around the world, and the general population.
In particular, skillful predictions of the climate one to ten years in the future are necessary for understanding both the future climatological mean and potential extremes \cite{wmo, IPCC_WG1}.
These predictions are therefore vital for mitigation and adaptation planning on regional scales \cite{Khasnis2005, Solaraju2022, Dunstone2022, IPCC_WG2, IPCC_WG3}.

There are many methods aimed at understanding the future climate on multi-year to decadal timescales.
Many of these make use of the large collection of available Earth System Models (ESMs) that have been developed at research centers around the world.
These include the Shared Socioeconomic Pathways (SSP) projections \cite{eyring2016}, which explore a range of potential climate futures out to the year $2100$.
However, SSPs were not designed nor intended to be used as predictions.

Initialized ESMs (IESMs) use observations to initialize physics-based dynamical models, then run these simulations, often out to ten years \cite{meehl2021}.
If the model captures the underlying physics of the Earth system, then this initialization should lead to skillful realizations.
However, due to model biases and unresolved physics, IESMs suffer from climate drift, where the simulations converge back to the model mean climate \cite{meehl2021, meehl2022}.
IESMs can also be costly to run, making large ensembles difficult and reducing their effectiveness for the prediction of extremes.

Work in the last several years \cite{Befort2020, Mahmood2021, Befort2022, DeLuca2023, Donat2024} has used both observations and IESMs to constrain climate projections (SSPs) of the next 10 to 40 years.
These studies match the projections to either observations or IESMs over a common period for a specified region, keeping only the best matching members to create a constrained sub-ensemble.
These constrained sub-ensembles have been shown to have better skill than the full ensemble for up to $20$ years.

An initialization method that has a long history is analog forecasting \cite{Lorenz1969}.
In analog forecasting, matching is based on a single state (model, member, and time), and the known future of the best matching state or states is the prediction.
As opposed to other initialized prediction methods, analog forecasting is time-agnostic, e.g., the prediction for $2025$ does not need to come from a projection of $2025$.
Initially, analog forecasting was used for weather prediction using historical observations as the analogs for the current state of the system \cite{Bergen1982}.

There have been several recent studies exploring the use of models in an analog forecasting framework, called model-analogs \cite{ding2018, Rader2023, ding2023, toride2024, acosta2025}.
The underlying premise of model-analogs is that there already exists a large catalog of ESMs that can be used to match to observations, and from which analog predictions can be made.
Figure \ref{fig:motivation} shows a time series of obervations, a suite of models covering the same time period, and a posthoc selected set of analogs.
The posthoc selection shown in Figure \ref{fig:motivation} is useless in a forecasting sense, but illustrates that within currently available model runs, states exist that closely match observations not just for a single year, but for multiple years.

\begin{figure}[h!]
    \noindent\includegraphics[width=\textwidth]{figures/intro_perfectT3_spaghetti.png}
    \caption{Posthoc selection of best matching analog to observations in three year chunks.
    This is not usable as a prediction.
    }
    \label{fig:motivation}
\end{figure}

Here, we build on work from \citeA{Rader2023}, which used machine learning to identify the most important precursor regions for matching in a model-analog framework.
We use a similar machine learning set up, along with a new matching criteria, and an expanded dataset that includes members from $29$ Coupled Model Intercomparison Project Phase 6 \cite<CMIP6;>{eyring2016} models.
We make predictions for lead times from one to ten years in multiple regions.
We evaluate these predictions against alternative analog methods, as well as bias-corrected IESMs, using multiple metrics to capture performance on the mean trend, variability, and pattern.
