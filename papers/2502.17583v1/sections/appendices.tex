% --------------------------------------------------------------------------
% --------------------------------------------------------------------------
\section{Supplementary Material}\label{sec:appendix}

% --------------------------------------------------------------------------
% --------------------------------------------------------------------------

% \subsection{Western United States Learned Masks}\label{sec:masks}

\begin{figure}[h!]
    \centering
    \noindent\includegraphics[width=0.95\textwidth]{supplemental_figures/wus_masks_ltall.png}
    \caption[Western United States learned masks, 1-10 years]{
        Model learned masks for the western United States for lead times 1-10 years.
    }
    \label{fig:wus_all_learned_masks}
\end{figure}

\begin{figure}[h!]
    \centering
    \noindent\includegraphics[width=0.95\textwidth]{supplemental_figures/wus_tlmasks_ltall.png}
    \caption[Western United States transfer learned masks, 1-10 years]{
        Transfer learned masks for the western United States for lead times 1-10 years.
    }
    \label{fig:wus_all_tl_masks}
\end{figure}

% --------------------------------------------------------------------------
% --------------------------------------------------------------------------

% \subsection{Metrics Versus Number of Analogs}\label{sec:number_analogs}

\begin{figure}[h!]
    \centering
    \noindent\includegraphics[width=0.95\textwidth]{supplemental_figures/wus_mse_nana_full.png}
    \caption[MSE versus number of analogs]{
        MSE versus number analogs used to calculate the mean prediction, for all lead times.
        Also shown are the global and regional mask results.
    }
    \label{fig:mse_nana_full}
\end{figure}

\begin{figure}[h!]
    \centering
    \noindent\includegraphics[width=0.95\textwidth]{supplemental_figures/wus_crps_nana_full.png}
    \caption[CRPS versus number of analogs]{
        Same as Figure \ref{fig:mse_nana_full}, but for CRPS.
    }
    \label{fig:crps_nana_full}
\end{figure}

% --------------------------------------------------------------------------
% --------------------------------------------------------------------------

% \subsection{Extra Metrics: EMD and Class Accuracy}\label{sec:extra_metrics}

\begin{figure}[h!]
    \centering
    \noindent\includegraphics[width=0.8\textwidth]{supplemental_figures/wus_emd_lead.png}
    \caption[Earth mover's distance western United States results]{
        EMD versus lead time for the western United States for mean predictions (line plots) and individual analogs (distributions).
        The top two panels show EMD for the time period $1864$-$2023$, while the bottom two panels show EMD for $2009$-$2018$.
    }
    \label{fig:emd_lead}
\end{figure}

\begin{figure}[h!]
    \centering
    \noindent\includegraphics[width=0.8\textwidth]{supplemental_figures/wus_acc4_lead.png}
    \caption[4 class accuracy for western United States]{
        Same as Figure \ref{fig:emd_lead}, but for class accuracy, with four classes.
    }
    \label{fig:acc4_lead}
\end{figure}

\begin{figure}[h!]
    \centering
    \includegraphics[width=0.6\textwidth]{supplemental_figures/regions_emd.png}
    \includegraphics[width=0.6\textwidth]{supplemental_figures/regions_acc4.png}
    \caption[EMD and class accuracy for 5 regions]{
        EMD (top) and class accuracy (bottom, 4 classes).
        The circles are the mean metric covering the time period $1956$-$2023$, the squares cover $1999$-$2018$.
    }
    \label{fig:emdacc_region}
\end{figure}


% --------------------------------------------------------------------------
% --------------------------------------------------------------------------

% \subsection{Prediction Year Versus Analog Year}\label{sec:year_comp}

\begin{figure}[h!]
    \centering
    \includegraphics[width=0.48\textwidth]{supplemental_figures/year_comp_wus5.png}
    \includegraphics[width=0.48\textwidth]{supplemental_figures/year_comp_regional_wus5.png}
    \caption[Analog year versus prediction year]{
        For each prediction year, the range of years for the top ten analogs (top panel) as well as the error over that time (bottom).
        While the model learned mask (left) begins following the one-to-one line (indicating that the selected analog year is the same as the prediction year) by the $1970$s, the regional mask (right) only does so in the final few years.
        In the bottom panels, it can be seen that the model learned mask predictions are not biased toward underestimating or overestimating at any point, while the regional mask predictions tends towards underestimation starting in the $1970$s.
        This is likely due to the regional mask not being able to pick up on the forced warming signal, which may be most apparent outside the target region.
    }
    \label{fig:year_comp}
\end{figure}

