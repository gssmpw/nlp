% --------------------------------------------------------------------------
% --------------------------------------------------------------------------
\section{Conclusions}\label{sec:conclusions}

We have presented a framework for multi-year-to-decadal prediction of regional 2-meter temperature.
The framework incorporates a machine learned mask of weights that highlights important precursors for use in model-analog forecasts.
Using several metrics, we have shown that our method:
\begin{itemize}
    \item better captures the mean regional trend (ensemble mean analog MSE) and ensemble distribution (CRPS) at all lead times and regions compared to alternative analog methods, and lead times greater than two years and three of five regions compared to two initialized Earth system models (IESMs).
    \item better captures regional variability (analog forecast MSE) at all lead times and four of five regions (not the Amazon) compared to alternative analog methods, and lead times greater than two years compared to IESMs.
    \item better captures temperature patterns (EMD, Supplementary Material) at all lead times and regions compared to alternative analog methods, and about half the lead times and regions compared to IESMs.
    \item produces better classification (4 classes, Supplementary Material) for all lead times and regions compared to alternative analog methods, and lead times greater than two years and three of five regions compared to IESMs.
\end{itemize}

Our machine learning model-analog framework can be further improved.
In this work, we only explored matching on and predicting the same variable, though multiple matching variables is likely to be beneficial.
We used tethering (matching on multiple years) and explored a learned mask that was refined using transfer learning, but did not optimize either of these processes.
In addition to improvements, the analog framework produces ensembles that can be used for prediction of extremes in a probabilistic sense.

There are many benefits to multi-year-to-decadal model-analog predictions.
Analog predictions do not require expensive model runs, beyond what is currently available, and benefit from improved modeling (both in quality of analogs and number available).
Analog predictions avoid the issue of initialization shock and climate drift faced by initializaed Earth system models, reducing the need for bias corrections \cite{meehl2021, meehl2022}, though our method still assumes that the models evolve consistently with observations.
Furthermore, the mask of precursor weights makes this prediction framework interpretable.
For example, \citeA{Rader2023} calculate how much skill is attributable to known precursors by isolating/occluding those precursors.
Analyses such as this are a strength of our interpretable approach, allowing us to identify predictable signals/patterns and connect them to physical understanding.
