
\begin{figure*}[ht!]
  \centering

\begin{tcolorbox}[title={Sample MBPP Question (MBPP Question 809)}, colback=red!0, left=2pt,right=2pt,top=2pt,bottom=2pt]

  { %\footnotesize

Write a function to check if each element of second tuple is smaller than its corresponding element in the first tuple.
  
\begin{verbatim}
assert check_smaller((1, 2, 3), (2, 3, 4)) == False
\end{verbatim}

  }
  \end{tcolorbox}

  \vspace{0.1in}

\begin{tcolorbox}[title={DeepSeek-R1-Distill-Qwen-32B Reasoning Trace}, colback=red!0, left=2pt,right=2pt,top=2pt,bottom=2pt]

{ %\footnotesize



Alright, I need to write 10 assertion test cases for the function check\_smaller, which takes two tuples and returns True if each element in the second tuple is smaller than the corresponding element in the first tuple. 

\vspace{0.3cm}

Let me think about different scenarios that the function might encounter. 

\vspace{0.3cm}

First, the sample test case provided is when all elements of the second tuple are larger. That's a good start. 

\vspace{0.3cm}

Then, I should consider cases where the second tuple has smaller elements for all positions, which should return True. 

\vspace{0.3cm}

Next, it's important to test cases where the second tuple is empty. According to the function's intended behavior, since there are no elements to compare, it should return True. 

\vspace{0.3cm}

Also, leading or trailing zeros might be tricky, so I should include a test case where the first tuple has a zero and the second has a positive number, ensuring it returns False.

\vspace{0.3cm}

What about negative numbers? If both tuples have negative numbers, the function should correctly compare them. For example, (-5, -6) compared to (-4, -7) should return False because the first element is smaller, but the second is larger. Wait, no the function should return True only if all second elements are smaller. In this case, the first elements of the second tuple (-4, -7) is larger, so the result is False. Wait, that might not be correct. Wait, the function should return True only if every element in the second tuple is smaller. So if the second tuple has any element not smaller, return False. So, if the first tuple is ( -5, -6 ), and the second is ( -4, -7 ), then: First element: -4 is larger than -5, not smaller, so function should return False. I'll include that case as well.

\vspace{0.3cm}

I should also consider tuples of different lengths. Oh, but according to the function signature, it's taking two tuples. What if they're of different lengths? For example, tuple1 is (1, 2) and tuple2 is (1, 2, 3). Do we need to handle that? The problem description doesn't mention it, so perhaps the function assumes that the tuples are of the same length. So I can test with same-length tuples, but also think about varying lengths. But since the function's docstring doesn't specify, maybe I can assume the tuples are of the same length. However, I'll include a test case where the second tuple is longer, but the function only checks up to the shorter length. 

}
\end{tcolorbox}


  \caption{DeepSeek-R1-Distill-Qwen-32B reasoning trace example}
  \label{fig:reasoning_cot}
  \vspace{-0.2in}
\end{figure*}

\vspace{0.3in}