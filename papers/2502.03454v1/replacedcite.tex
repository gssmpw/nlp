\section{Related Work}
\label{sec_related_work}
%
The authors of ____ recently reviewed the 2D and 3D road models for racing vehicle control.
The first MLT-OCPs on 3D roads were solved in ____, using a ribbon-road model. This model was extended by ____ and ____, for generic non-planar surfaces.
On certain circuits, the road angles were considered \textit{small} ____, to simplify the equations of motion.
%In ____, an MLT-OCP was solved on an oval track, modeling a nonlinear banking profile in the lateral road direction.

Economic\footnote{The term \textit{economic} MPC is used to disambiguate from tracking MPC.} nonlinear model predictive control (E-NMPC) is becoming popular for trajectory planning in AVR ____.
The main challenge for \textit{online} minimum-time planning with E-NMPC on 3D circuits is devising accurate yet computationally efficient models of the vehicle dynamics and performance.
In ____, the maximum performance g-g-v diagram\footnote{The g-g-v diagram (top view in Fig. \ref{fig_ggv_2D_3D_top_Mugello_3D}) is a plot of the maximum lateral and longitudinal vehicle accelerations, as a function of the vehicle speed.} was underestimated, which led to suboptimal lap times. In ____, the g-g-v diagrams on 3D roads were estimated with a quasi steady-state (QSS) approach; however, they performed only offline trajectory optimization.
The authors of ____ solved offline MLT-OCPs on 3D roads, and tracked these maneuvers online using feedback controllers. While their online lap times were close to the offline ones, they tracked fixed solutions without re-planning, and they did not evaluate the trajectory optimality under execution errors.
%
\subsubsection*{Critical Summary}
%
To our knowledge, the prior literature is limited by at least one of the following aspects:
%
\begin{itemize}
	\item Minimum-time trajectory planning on 3D tracks used simplified g-g-v acceleration constraints, which did not accurately capture the performance limits.
	\item The g-g-v diagrams on 3D circuits were estimated with QSS approaches, which assumed the prior knowledge of the vehicle equations of motion.
	\item The performance of online trajectory planning and execution was not compared with MLT-OCPs, solved with high-fidelity vehicle models.
\end{itemize}
%