

\begin{figure*}[ht]
    \centering
    \includegraphics[width=1\linewidth]{images/fig2_bench_contours.png}
    \caption{Examples from individual domains in M3DA with the corresponding segmentation masks. Left to right, top to bottom: CT to MR, CT to LDCT, CT CE to CT native, CE T1 to T1, T1 Field (1.5T to 3T), T1 Scanner (Philips to Siemens). Different colors correspond to different segmentation classes.}
    \label{fig:contours}
\end{figure*}


\section{Datasets description}

Below we provide an extended description of datasets used in M3DA benchmark, download and usage examples are available at \href{https://github.com/BorisShirokikh/M3DA}{https://github.com/BorisShirokikh/M3DA}. Example 2D slices from every dataset for visual comparison between domains are given in Figure~\ref{fig:contours}. Summary of licenses and data access is given in Table~\ref{tab:supp_datasets}.



\begin{figure*}[ht]
    \centering
    \includegraphics[width=1\linewidth]{images/fig2_bench_contours.png}
    \caption{Examples from individual domains in M3DA with the corresponding segmentation masks. Left to right, top to bottom: CT to MR, CT to LDCT, CT CE to CT native, CE T1 to T1, T1 Field (1.5T to 3T), T1 Scanner (Philips to Siemens). Different colors correspond to different segmentation classes.}
    \label{fig:contours}
\end{figure*}


\section{Datasets description}

Below we provide an extended description of datasets used in M3DA benchmark, download and usage examples are available at \href{https://github.com/BorisShirokikh/M3DA}{https://github.com/BorisShirokikh/M3DA}. Example 2D slices from every dataset for visual comparison between domains are given in Figure~\ref{fig:contours}. Summary of licenses and data access is given in Table~\ref{tab:supp_datasets}.



\begin{figure*}[ht]
    \centering
    \includegraphics[width=1\linewidth]{images/fig2_bench_contours.png}
    \caption{Examples from individual domains in M3DA with the corresponding segmentation masks. Left to right, top to bottom: CT to MR, CT to LDCT, CT CE to CT native, CE T1 to T1, T1 Field (1.5T to 3T), T1 Scanner (Philips to Siemens). Different colors correspond to different segmentation classes.}
    \label{fig:contours}
\end{figure*}


\section{Datasets description}

Below we provide an extended description of datasets used in M3DA benchmark, download and usage examples are available at \href{https://github.com/BorisShirokikh/M3DA}{https://github.com/BorisShirokikh/M3DA}. Example 2D slices from every dataset for visual comparison between domains are given in Figure~\ref{fig:contours}. Summary of licenses and data access is given in Table~\ref{tab:supp_datasets}.



\begin{figure*}[ht]
    \centering
    \includegraphics[width=1\linewidth]{images/fig2_bench_contours.png}
    \caption{Examples from individual domains in M3DA with the corresponding segmentation masks. Left to right, top to bottom: CT to MR, CT to LDCT, CT CE to CT native, CE T1 to T1, T1 Field (1.5T to 3T), T1 Scanner (Philips to Siemens). Different colors correspond to different segmentation classes.}
    \label{fig:contours}
\end{figure*}


\section{Datasets description}

Below we provide an extended description of datasets used in M3DA benchmark, download and usage examples are available at \href{https://github.com/BorisShirokikh/M3DA}{https://github.com/BorisShirokikh/M3DA}. Example 2D slices from every dataset for visual comparison between domains are given in Figure~\ref{fig:contours}. Summary of licenses and data access is given in Table~\ref{tab:supp_datasets}.

\input{tables/suppl/1_datasets}


\subsection{AMOS}

The AMOS dataset \cite{amos} contains 500 CT and 100 MRI abdominal scans with the multi-organ segmentation task: liver, stomach, spleen, left and right kidneys, bladder, aorta, pancreas, inferior vena cava, duodenum, prostate/uterus, gallbladder, esophagus, left and right adrenals . As a largest available dataset for inter-modality segmentation, we employed it in MR$\ra$CT and CT$\ra$MR domain shift setups. 


Furthermore, we used AMOS CT images to create one of the most clinically relevant domain shift setups -- difference in the radiation dose during scanning. 
 
For the LDCT domain, we simulated low radiation dose using the algorithm provided in \cite{ldct}, simulated data are available at \href{https://zenodo.org/records/13373720}{https://zenodo.org/records/13373720}.


\subsection{BraTS}

BraTS \cite{brats} is comprised of 2000 brain MRI cases, each consisting of four sequences: T1, T1c, T2, FLAIR, with a glioblastoma segmentation classes (3 foreground classes and background).  We only used 1251 cases with publicly available annotations and T1, T1c MRI sequences for T1 CE$\ra$T1 shift. Since sequences of the same case provide information about the same subject, we ensured source-target splits so that every case falls into exactly one fold.


\subsection{CC359}

The CC359 dataset \cite{cc359} contains 359 brain MR T1 images from three scanners, namely, GE, Philips (PH), and Siemens (SM), obtained using two magnetic field strength values, $1.5$ and $3.0$T. The dataset can be split into six domains defined by two different field strengths $\times$ three vendors, each with approximately 60 images, so it yields 30 possible domain adaptation pairs. 

CC359 also offers three tasks: brain, hippocampus, and white matter, gray matter, and cerebrospinal fluid (WMGMCSF) segmentation. We ommited hippocampus segmentation task from the benchmark, because our preliminary experiments showed it is not significantly affected by domain shifts, the relative performance drop is less than $2\%$ in every domain pair; see Table~\ref{tab:hippo}. We also omitted the brain segmentation task for the same reason, see results in \cite{shirokikh2020first}.

\input{tables/suppl/2_cc359_hippo}

Therefore, we focus only on the WMGMCSF segmentation task in CC359: white matter, gray matter and cerebral spinal fluid segmentation classes and background. From 30 possible domain pairs, we selected three with the maximum performance drop, highlighted in \textbf{bold},  Table~\ref{tab:wmgmcsf}): changing field strength with a fixed scanner PH 1.5T $\ra$ PH 3.0T (drop from 95.4 to 74.1 Dice score), changing scanner with the fixed field strength PH 3.0T $\ra$ SM 3.0T (drop from 95.7 to 76.6), and changing both parameters SM 3.0T $\ra$ GE 1.5T (drop from 95.8 to 56.0). We denote them as T1 F, T1 Sc, and T1 Mix, respectively.

\input{tables/suppl/3_cc359_wmgmcsf}


\subsection{LIDC}

LIDC \cite{lidc} is a multi-center collection of diagnostic and lung cancer screening thoracic CT scans with annotated lesions. It includes 1308 studies (of which 1018 include CT studies) from 1010 patients. Lung's nodules is one of the few clinical applications where both CE CT and CT are used, first for the initial scan, and second for the follow-ups \cite{purysko2016does}. We used LIDC for CE CT $\ra$ CT domain shift, we split data into three roughly equal groups, ommiting scans with empty masks: contrast enhanced CT (source domain) $X^s$, CT without contrast enhancement $X^t_{tr}$ (training part, target domain), and CT without contrast enhancement  $X^t_{ts}$ (test part, target domain). $X^t_{tr}$ and $X^t_{ts}$ were stratified by the number of lesions.



\subsection{AMOS}

The AMOS dataset \cite{amos} contains 500 CT and 100 MRI abdominal scans with the multi-organ segmentation task: liver, stomach, spleen, left and right kidneys, bladder, aorta, pancreas, inferior vena cava, duodenum, prostate/uterus, gallbladder, esophagus, left and right adrenals . As a largest available dataset for inter-modality segmentation, we employed it in MR$\ra$CT and CT$\ra$MR domain shift setups. 


Furthermore, we used AMOS CT images to create one of the most clinically relevant domain shift setups -- difference in the radiation dose during scanning. 
 
For the LDCT domain, we simulated low radiation dose using the algorithm provided in \cite{ldct}, simulated data are available at \href{https://zenodo.org/records/13373720}{https://zenodo.org/records/13373720}.


\subsection{BraTS}

BraTS \cite{brats} is comprised of 2000 brain MRI cases, each consisting of four sequences: T1, T1c, T2, FLAIR, with a glioblastoma segmentation classes (3 foreground classes and background).  We only used 1251 cases with publicly available annotations and T1, T1c MRI sequences for T1 CE$\ra$T1 shift. Since sequences of the same case provide information about the same subject, we ensured source-target splits so that every case falls into exactly one fold.


\subsection{CC359}

The CC359 dataset \cite{cc359} contains 359 brain MR T1 images from three scanners, namely, GE, Philips (PH), and Siemens (SM), obtained using two magnetic field strength values, $1.5$ and $3.0$T. The dataset can be split into six domains defined by two different field strengths $\times$ three vendors, each with approximately 60 images, so it yields 30 possible domain adaptation pairs. 

CC359 also offers three tasks: brain, hippocampus, and white matter, gray matter, and cerebrospinal fluid (WMGMCSF) segmentation. We ommited hippocampus segmentation task from the benchmark, because our preliminary experiments showed it is not significantly affected by domain shifts, the relative performance drop is less than $2\%$ in every domain pair; see Table~\ref{tab:hippo}. We also omitted the brain segmentation task for the same reason, see results in \cite{shirokikh2020first}.



\begin{table}[!htb]
    \centering
    \caption{Baseline and oracle results on the CC359 hippocampus segmentation task.}
    \label{tab:hippo}
    \resizebox{\columnwidth}{!}{%
    \begin{tabular}{|c|l||c|c|c|c|c|c|}

        \hline
        \multicolumn{2}{|l||}{\multirow{2}{*}{}}  & \multicolumn{6}{c|}{Target domains}\\ 
        \cline{3-8}
        \multicolumn{2}{|l||}{} & GE1.5 & PH1.5 & SM1.5 & GE3.0 & PH3.0 & SM3.0 \\ 
        \hline
        \hline

        % \clrtb

        % \multirow{6}{*}{{\STAB{\rotatebox[origin=c]{90}{Source domains}}}}
        \multirow{6}{*}{{\rotatebox[origin=c]{90}{Source domains}}}
        
        & GE 1.5 & 92.3 & 86.7 & 88.7 & 87.8 & 91.2 & 91.2 \\
        \cline{2-8}
        
        & PH 1.5 & 91.3 & 86.9 & 87.7 & 87.7 & 89.7 & 89.9 \\
        \cline{2-8}
        
        & SM 1.5 & 91.7 & 86.6 & 89.3 & 88.2 & 90.9 & 90.8 \\
        \cline{2-8}
        
        & GE 3.0 & 91.4 & 86.4 & 88.0 & 89.1 & 90.5 & 91.3 \\
        \cline{2-8}
        
        & PH 3.0 & 91.5 & 86.5 & 88.3 & 87.7 & 92.0 & 91.0 \\
        \cline{2-8}
        
        & SM 3.0 & 90.8 & 86.5 & 87.8 & 88.0 & 90.6 & 92.1 \\
        \hline
        
    \end{tabular}}
\end{table}


Therefore, we focus only on the WMGMCSF segmentation task in CC359: white matter, gray matter and cerebral spinal fluid segmentation classes and background. From 30 possible domain pairs, we selected three with the maximum performance drop, highlighted in \textbf{bold},  Table~\ref{tab:wmgmcsf}): changing field strength with a fixed scanner PH 1.5T $\ra$ PH 3.0T (drop from 95.4 to 74.1 Dice score), changing scanner with the fixed field strength PH 3.0T $\ra$ SM 3.0T (drop from 95.7 to 76.6), and changing both parameters SM 3.0T $\ra$ GE 1.5T (drop from 95.8 to 56.0). We denote them as T1 F, T1 Sc, and T1 Mix, respectively.



\begin{table}[!htb]
    \centering
    \caption{Baseline and oracle results on the CC359 WMGMSCF segmentation task.}\label{tab:wmgmcsf}
    \resizebox{\columnwidth}{!}{%
    \begin{tabular}{|c|l||c|c|c|c|c|c|}

        \hline
        \multicolumn{2}{|l||}{\multirow{2}{*}{}}  & \multicolumn{6}{c|}{Target domains}\\ 
        \cline{3-8}
        \multicolumn{2}{|l||}{} & GE1.5 & PH1.5 & SM1.5 & GE3.0 & PH3.0 & SM3.0 \\ 
        \hline
        \hline

        % \clrtb

        % \multirow{6}{*}{{\STAB{\rotatebox[origin=c]{90}{Source domains}}}}
        \multirow{6}{*}{{\rotatebox[origin=c]{90}{Source domains}}}
        & GE 1.5 & 95.8 & 82.1 & 90.8 & 82.1 & 92.6 & 80.8 \\
        \cline{2-8}
        
        & PH 1.5 & 80.1 & 92.7 & 90.8 & 93.4 & \textbf{74.1} & 90.1 \\
        \cline{2-8}
        
        & SM 1.5 & 89.7 & 85.3 & 95.6 & 86.2 & 86.2 & 84.5 \\
        \cline{2-8}
        
        & GE 3.0 & 76.6 & 89.9 & 90.3 & 95.9 & 72.0 & 91.4 \\
        \cline{2-8}
        
        & PH 3.0 & 90.6 & 74.7 & 86.0 & 75.4 & 95.4 & \textbf{76.6} \\
        \cline{2-8}
        
        & SM 3.0 & \textbf{56.0} & 88.6 & 84.9 & 92.4 & 68.4 & 95.7 \\
        \hline
        
    \end{tabular}}
\end{table}



\subsection{LIDC}

LIDC \cite{lidc} is a multi-center collection of diagnostic and lung cancer screening thoracic CT scans with annotated lesions. It includes 1308 studies (of which 1018 include CT studies) from 1010 patients. Lung's nodules is one of the few clinical applications where both CE CT and CT are used, first for the initial scan, and second for the follow-ups \cite{purysko2016does}. We used LIDC for CE CT $\ra$ CT domain shift, we split data into three roughly equal groups, ommiting scans with empty masks: contrast enhanced CT (source domain) $X^s$, CT without contrast enhancement $X^t_{tr}$ (training part, target domain), and CT without contrast enhancement  $X^t_{ts}$ (test part, target domain). $X^t_{tr}$ and $X^t_{ts}$ were stratified by the number of lesions.



\subsection{AMOS}

The AMOS dataset \cite{amos} contains 500 CT and 100 MRI abdominal scans with the multi-organ segmentation task: liver, stomach, spleen, left and right kidneys, bladder, aorta, pancreas, inferior vena cava, duodenum, prostate/uterus, gallbladder, esophagus, left and right adrenals . As a largest available dataset for inter-modality segmentation, we employed it in MR$\ra$CT and CT$\ra$MR domain shift setups. 


Furthermore, we used AMOS CT images to create one of the most clinically relevant domain shift setups -- difference in the radiation dose during scanning. 
 
For the LDCT domain, we simulated low radiation dose using the algorithm provided in \cite{ldct}, simulated data are available at \href{https://zenodo.org/records/13373720}{https://zenodo.org/records/13373720}.


\subsection{BraTS}

BraTS \cite{brats} is comprised of 2000 brain MRI cases, each consisting of four sequences: T1, T1c, T2, FLAIR, with a glioblastoma segmentation classes (3 foreground classes and background).  We only used 1251 cases with publicly available annotations and T1, T1c MRI sequences for T1 CE$\ra$T1 shift. Since sequences of the same case provide information about the same subject, we ensured source-target splits so that every case falls into exactly one fold.


\subsection{CC359}

The CC359 dataset \cite{cc359} contains 359 brain MR T1 images from three scanners, namely, GE, Philips (PH), and Siemens (SM), obtained using two magnetic field strength values, $1.5$ and $3.0$T. The dataset can be split into six domains defined by two different field strengths $\times$ three vendors, each with approximately 60 images, so it yields 30 possible domain adaptation pairs. 

CC359 also offers three tasks: brain, hippocampus, and white matter, gray matter, and cerebrospinal fluid (WMGMCSF) segmentation. We ommited hippocampus segmentation task from the benchmark, because our preliminary experiments showed it is not significantly affected by domain shifts, the relative performance drop is less than $2\%$ in every domain pair; see Table~\ref{tab:hippo}. We also omitted the brain segmentation task for the same reason, see results in \cite{shirokikh2020first}.



\begin{table}[!htb]
    \centering
    \caption{Baseline and oracle results on the CC359 hippocampus segmentation task.}
    \label{tab:hippo}
    \resizebox{\columnwidth}{!}{%
    \begin{tabular}{|c|l||c|c|c|c|c|c|}

        \hline
        \multicolumn{2}{|l||}{\multirow{2}{*}{}}  & \multicolumn{6}{c|}{Target domains}\\ 
        \cline{3-8}
        \multicolumn{2}{|l||}{} & GE1.5 & PH1.5 & SM1.5 & GE3.0 & PH3.0 & SM3.0 \\ 
        \hline
        \hline

        % \clrtb

        % \multirow{6}{*}{{\STAB{\rotatebox[origin=c]{90}{Source domains}}}}
        \multirow{6}{*}{{\rotatebox[origin=c]{90}{Source domains}}}
        
        & GE 1.5 & 92.3 & 86.7 & 88.7 & 87.8 & 91.2 & 91.2 \\
        \cline{2-8}
        
        & PH 1.5 & 91.3 & 86.9 & 87.7 & 87.7 & 89.7 & 89.9 \\
        \cline{2-8}
        
        & SM 1.5 & 91.7 & 86.6 & 89.3 & 88.2 & 90.9 & 90.8 \\
        \cline{2-8}
        
        & GE 3.0 & 91.4 & 86.4 & 88.0 & 89.1 & 90.5 & 91.3 \\
        \cline{2-8}
        
        & PH 3.0 & 91.5 & 86.5 & 88.3 & 87.7 & 92.0 & 91.0 \\
        \cline{2-8}
        
        & SM 3.0 & 90.8 & 86.5 & 87.8 & 88.0 & 90.6 & 92.1 \\
        \hline
        
    \end{tabular}}
\end{table}


Therefore, we focus only on the WMGMCSF segmentation task in CC359: white matter, gray matter and cerebral spinal fluid segmentation classes and background. From 30 possible domain pairs, we selected three with the maximum performance drop, highlighted in \textbf{bold},  Table~\ref{tab:wmgmcsf}): changing field strength with a fixed scanner PH 1.5T $\ra$ PH 3.0T (drop from 95.4 to 74.1 Dice score), changing scanner with the fixed field strength PH 3.0T $\ra$ SM 3.0T (drop from 95.7 to 76.6), and changing both parameters SM 3.0T $\ra$ GE 1.5T (drop from 95.8 to 56.0). We denote them as T1 F, T1 Sc, and T1 Mix, respectively.



\begin{table}[!htb]
    \centering
    \caption{Baseline and oracle results on the CC359 WMGMSCF segmentation task.}\label{tab:wmgmcsf}
    \resizebox{\columnwidth}{!}{%
    \begin{tabular}{|c|l||c|c|c|c|c|c|}

        \hline
        \multicolumn{2}{|l||}{\multirow{2}{*}{}}  & \multicolumn{6}{c|}{Target domains}\\ 
        \cline{3-8}
        \multicolumn{2}{|l||}{} & GE1.5 & PH1.5 & SM1.5 & GE3.0 & PH3.0 & SM3.0 \\ 
        \hline
        \hline

        % \clrtb

        % \multirow{6}{*}{{\STAB{\rotatebox[origin=c]{90}{Source domains}}}}
        \multirow{6}{*}{{\rotatebox[origin=c]{90}{Source domains}}}
        & GE 1.5 & 95.8 & 82.1 & 90.8 & 82.1 & 92.6 & 80.8 \\
        \cline{2-8}
        
        & PH 1.5 & 80.1 & 92.7 & 90.8 & 93.4 & \textbf{74.1} & 90.1 \\
        \cline{2-8}
        
        & SM 1.5 & 89.7 & 85.3 & 95.6 & 86.2 & 86.2 & 84.5 \\
        \cline{2-8}
        
        & GE 3.0 & 76.6 & 89.9 & 90.3 & 95.9 & 72.0 & 91.4 \\
        \cline{2-8}
        
        & PH 3.0 & 90.6 & 74.7 & 86.0 & 75.4 & 95.4 & \textbf{76.6} \\
        \cline{2-8}
        
        & SM 3.0 & \textbf{56.0} & 88.6 & 84.9 & 92.4 & 68.4 & 95.7 \\
        \hline
        
    \end{tabular}}
\end{table}



\subsection{LIDC}

LIDC \cite{lidc} is a multi-center collection of diagnostic and lung cancer screening thoracic CT scans with annotated lesions. It includes 1308 studies (of which 1018 include CT studies) from 1010 patients. Lung's nodules is one of the few clinical applications where both CE CT and CT are used, first for the initial scan, and second for the follow-ups \cite{purysko2016does}. We used LIDC for CE CT $\ra$ CT domain shift, we split data into three roughly equal groups, ommiting scans with empty masks: contrast enhanced CT (source domain) $X^s$, CT without contrast enhancement $X^t_{tr}$ (training part, target domain), and CT without contrast enhancement  $X^t_{ts}$ (test part, target domain). $X^t_{tr}$ and $X^t_{ts}$ were stratified by the number of lesions.



\subsection{AMOS}

The AMOS dataset \cite{amos} contains 500 CT and 100 MRI abdominal scans with the multi-organ segmentation task: liver, stomach, spleen, left and right kidneys, bladder, aorta, pancreas, inferior vena cava, duodenum, prostate/uterus, gallbladder, esophagus, left and right adrenals . As a largest available dataset for inter-modality segmentation, we employed it in MR$\ra$CT and CT$\ra$MR domain shift setups. 


Furthermore, we used AMOS CT images to create one of the most clinically relevant domain shift setups -- difference in the radiation dose during scanning. 
 
For the LDCT domain, we simulated low radiation dose using the algorithm provided in \cite{ldct}, simulated data are available at \href{https://zenodo.org/records/13373720}{https://zenodo.org/records/13373720}.


\subsection{BraTS}

BraTS \cite{brats} is comprised of 2000 brain MRI cases, each consisting of four sequences: T1, T1c, T2, FLAIR, with a glioblastoma segmentation classes (3 foreground classes and background).  We only used 1251 cases with publicly available annotations and T1, T1c MRI sequences for T1 CE$\ra$T1 shift. Since sequences of the same case provide information about the same subject, we ensured source-target splits so that every case falls into exactly one fold.


\subsection{CC359}

The CC359 dataset \cite{cc359} contains 359 brain MR T1 images from three scanners, namely, GE, Philips (PH), and Siemens (SM), obtained using two magnetic field strength values, $1.5$ and $3.0$T. The dataset can be split into six domains defined by two different field strengths $\times$ three vendors, each with approximately 60 images, so it yields 30 possible domain adaptation pairs. 

CC359 also offers three tasks: brain, hippocampus, and white matter, gray matter, and cerebrospinal fluid (WMGMCSF) segmentation. We ommited hippocampus segmentation task from the benchmark, because our preliminary experiments showed it is not significantly affected by domain shifts, the relative performance drop is less than $2\%$ in every domain pair; see Table~\ref{tab:hippo}. We also omitted the brain segmentation task for the same reason, see results in \cite{shirokikh2020first}.



\begin{table}[!htb]
    \centering
    \caption{Baseline and oracle results on the CC359 hippocampus segmentation task.}
    \label{tab:hippo}
    \resizebox{\columnwidth}{!}{%
    \begin{tabular}{|c|l||c|c|c|c|c|c|}

        \hline
        \multicolumn{2}{|l||}{\multirow{2}{*}{}}  & \multicolumn{6}{c|}{Target domains}\\ 
        \cline{3-8}
        \multicolumn{2}{|l||}{} & GE1.5 & PH1.5 & SM1.5 & GE3.0 & PH3.0 & SM3.0 \\ 
        \hline
        \hline

        % \clrtb

        % \multirow{6}{*}{{\STAB{\rotatebox[origin=c]{90}{Source domains}}}}
        \multirow{6}{*}{{\rotatebox[origin=c]{90}{Source domains}}}
        
        & GE 1.5 & 92.3 & 86.7 & 88.7 & 87.8 & 91.2 & 91.2 \\
        \cline{2-8}
        
        & PH 1.5 & 91.3 & 86.9 & 87.7 & 87.7 & 89.7 & 89.9 \\
        \cline{2-8}
        
        & SM 1.5 & 91.7 & 86.6 & 89.3 & 88.2 & 90.9 & 90.8 \\
        \cline{2-8}
        
        & GE 3.0 & 91.4 & 86.4 & 88.0 & 89.1 & 90.5 & 91.3 \\
        \cline{2-8}
        
        & PH 3.0 & 91.5 & 86.5 & 88.3 & 87.7 & 92.0 & 91.0 \\
        \cline{2-8}
        
        & SM 3.0 & 90.8 & 86.5 & 87.8 & 88.0 & 90.6 & 92.1 \\
        \hline
        
    \end{tabular}}
\end{table}


Therefore, we focus only on the WMGMCSF segmentation task in CC359: white matter, gray matter and cerebral spinal fluid segmentation classes and background. From 30 possible domain pairs, we selected three with the maximum performance drop, highlighted in \textbf{bold},  Table~\ref{tab:wmgmcsf}): changing field strength with a fixed scanner PH 1.5T $\ra$ PH 3.0T (drop from 95.4 to 74.1 Dice score), changing scanner with the fixed field strength PH 3.0T $\ra$ SM 3.0T (drop from 95.7 to 76.6), and changing both parameters SM 3.0T $\ra$ GE 1.5T (drop from 95.8 to 56.0). We denote them as T1 F, T1 Sc, and T1 Mix, respectively.



\begin{table}[!htb]
    \centering
    \caption{Baseline and oracle results on the CC359 WMGMSCF segmentation task.}\label{tab:wmgmcsf}
    \resizebox{\columnwidth}{!}{%
    \begin{tabular}{|c|l||c|c|c|c|c|c|}

        \hline
        \multicolumn{2}{|l||}{\multirow{2}{*}{}}  & \multicolumn{6}{c|}{Target domains}\\ 
        \cline{3-8}
        \multicolumn{2}{|l||}{} & GE1.5 & PH1.5 & SM1.5 & GE3.0 & PH3.0 & SM3.0 \\ 
        \hline
        \hline

        % \clrtb

        % \multirow{6}{*}{{\STAB{\rotatebox[origin=c]{90}{Source domains}}}}
        \multirow{6}{*}{{\rotatebox[origin=c]{90}{Source domains}}}
        & GE 1.5 & 95.8 & 82.1 & 90.8 & 82.1 & 92.6 & 80.8 \\
        \cline{2-8}
        
        & PH 1.5 & 80.1 & 92.7 & 90.8 & 93.4 & \textbf{74.1} & 90.1 \\
        \cline{2-8}
        
        & SM 1.5 & 89.7 & 85.3 & 95.6 & 86.2 & 86.2 & 84.5 \\
        \cline{2-8}
        
        & GE 3.0 & 76.6 & 89.9 & 90.3 & 95.9 & 72.0 & 91.4 \\
        \cline{2-8}
        
        & PH 3.0 & 90.6 & 74.7 & 86.0 & 75.4 & 95.4 & \textbf{76.6} \\
        \cline{2-8}
        
        & SM 3.0 & \textbf{56.0} & 88.6 & 84.9 & 92.4 & 68.4 & 95.7 \\
        \hline
        
    \end{tabular}}
\end{table}



\subsection{LIDC}

LIDC \cite{lidc} is a multi-center collection of diagnostic and lung cancer screening thoracic CT scans with annotated lesions. It includes 1308 studies (of which 1018 include CT studies) from 1010 patients. Lung's nodules is one of the few clinical applications where both CE CT and CT are used, first for the initial scan, and second for the follow-ups \cite{purysko2016does}. We used LIDC for CE CT $\ra$ CT domain shift, we split data into three roughly equal groups, ommiting scans with empty masks: contrast enhanced CT (source domain) $X^s$, CT without contrast enhancement $X^t_{tr}$ (training part, target domain), and CT without contrast enhancement  $X^t_{ts}$ (test part, target domain). $X^t_{tr}$ and $X^t_{ts}$ were stratified by the number of lesions.
