

% Проблема: есть доменный сдвиг
% Актуальность: нет бенчмарка
% Решение: предлагаем системный бенчмарк
% Новизна: МРТ, КТ, первыми системно затрагиваем все ключевые сдвиги
% Ценность: бенчмарк не решается с нахрапа - нужно долбить методы

\begin{abstract}

Domain shift presents a significant challenge in applying Deep Learning to the segmentation of 3D medical images from sources like Magnetic Resonance Imaging (MRI) and Computed Tomography (CT). Although numerous Domain Adaptation methods have been developed to address this issue, they are often evaluated under impractical data shift scenarios. Specifically, the medical imaging datasets used are often either private, too small for robust training and evaluation, or limited to single or synthetic tasks.


To overcome these limitations, we introduce a M3DA \textipa{/"mEd@/} benchmark comprising four publicly available, multiclass segmentation datasets. We have designed eight domain pairs featuring diverse and practically relevant distribution shifts. These include inter-modality shifts between MRI and CT and intra-modality shifts among various MRI acquisition parameters, different CT radiation doses, and presence/absence of contrast enhancement in images.


Within the proposed benchmark, we evaluate more than ten existing domain adaptation methods. Our results show that none of them can consistently close the performance gap between the domains. For instance, the most effective method reduces the performance gap by about 62\% across the tasks. This highlights the need for developing novel domain adaptation algorithms to enhance the robustness and scalability of deep learning models in medical imaging.


We made our M3DA benchmark publicly available: \href{https://github.com/BorisShirokikh/M3DA}{https://github.com/BorisShirokikh/M3DA}.

\end{abstract}
