%%
%% This is file `sample-manuscript.tex',
%% generated with the docstrip utility.
%%
%% The original source files were:
%%
%% samples.dtx  (with options: `all,proceedings,bibtex,manuscript')
%% 
%% IMPORTANT NOTICE:
%% 
%% For the copyright see the source file.
%% 
%% Any modified versions of this file must be renamed
%% with new filenames distinct from sample-manuscript.tex.
%% 
%% For distribution of the original source see the terms
%% for copying and modification in the file samples.dtx.
%% 
%% This generated file may be distributed as long as the
%% original source files, as listed above, are part of the
%% same distribution. (The sources need not necessarily be
%% in the same archive or directory.)
%%
%%
%% Commands for TeXCount
%TC:macro \cite [option:text,text]
%TC:macro \citep [option:text,text]
%TC:macro \citet [option:text,text]
%TC:envir table 0 1
%TC:envir table* 0 1
%TC:envir tabular [ignore] word
%TC:envir displaymath 0 word
%TC:envir math 0 word
%TC:envir comment 0 0
%%
%% The first command in your LaTeX source must be the \documentclass
%% command.
%%
%% For submission and review of your manuscript please change the
%% command to \documentclass[manuscript, screen, review]{acmart}.
%%
%% When submitting camera ready or to TAPS, please change the command
%% to \documentclass[sigconf]{acmart} or whichever template is required
%% for your publication.
%%
%%
\documentclass[screen]{acmart}
% \documentclass[manuscript,review,anonymous]{acmart}
\usepackage{times}
\usepackage{adjustbox}
\usepackage[normalem]{ulem}
\usepackage{multirow}
\usepackage[normalem]{ulem}
\usepackage{caption}
\usepackage{subcaption}
\usepackage{amsmath}
\usepackage{mathtools}
\usepackage{xcolor}
\newcommand\mycommfont[1]{\footnotesize\ttfamily{#1}}
\usepackage{balance}
\usepackage{natbib}
\newcommand{\red}[1]{\textcolor{red}{#1}}
%\usepackage{caption}
\usepackage{colortbl}
%%
%% \BibTeX command to typeset BibTeX logo in the docs
\AtBeginDocument{%
  \providecommand\BibTeX{{%
    Bib\TeX}}}


%% Rights management information.  This information is sent to you
%% when you complete the rights form.  These commands have SAMPLE
%% values in them; it is your responsibility as an author to replace
%% the commands and values with those provided to you when you
%% complete the rights form.
\setcopyright{none}
% \setcopyright{acmlicensed}
% \copyrightyear{2018}
% \acmYear{2018}
% \acmDOI{XXXXXXX.XXXXXXX}
% %% These commands are for a PROCEEDINGS abstract or paper.
% \acmConference[Conference acronym 'XX]{Make sure to enter the correct
%   conference title from your rights confirmation email}{June 03--05,
%   2018}{Woodstock, NY}
%%
%%  Uncomment \acmBooktitle if the title of the proceedings is different
%%  from ``Proceedings of ...''!
%%
%%\acmBooktitle{Woodstock '18: ACM Symposium on Neural Gaze Detection,
%%  June 03--05, 2018, Woodstock, NY}
% \acmISBN{978-1-4503-XXXX-X/18/06}


%%
%% Submission ID.
%% Use this when submitting an article to a sponsored event. You'll
%% receive a unique submission ID from the organizers
%% of the event, and this ID should be used as the parameter to this command.
%%\acmSubmissionID{123-A56-BU3}

%%
%% For managing citations, it is recommended to use bibliography
%% files in BibTeX format.
%%
%% You can then either use BibTeX with the ACM-Reference-Format style,
%% or BibLaTeX with the acmnumeric or acmauthoryear sytles, that include
%% support for advanced citation of software artefact from the
%% biblatex-software package, also separately available on CTAN.
%%
%% Look at the sample-*-biblatex.tex files for templates showcasing
%% the biblatex styles.
%%

%%
%% The majority of ACM publications use numbered citations and
%% references.  The command \citestyle{authoryear} switches to the
%% "author year" style.
%%
%% If you are preparing content for an event
%% sponsored by ACM SIGGRAPH, you must use the "author year" style of
%% citations and references.
%% Uncommenting
%% the next command will enable that style.
%%\citestyle{acmauthoryear}


%%
%% end of the preamble, start of the body of the document source.
\begin{document}

%%
%% The "title" command has an optional parameter,
%% allowing the author to define a "short title" to be used in page headers.
\title{Co-Designing Augmented Reality Tools for High-Stakes Clinical Teamwork}


%%
%% The "author" command and its associated commands are used to define
%% the authors and their affiliations.
%% Of note is the shared affiliation of the first two authors, and the
%% "authornote" and "authornotemark" commands
%% used to denote shared contribution to the research.
\author{Angelique Taylor}
\orcid{0000-0003-1285-6431}
\affiliation{%
  \institution{Cornell Tech}
  \department{Information Science}
  \city{New York City}
  \state{NY}
  \country{USA}}
\email{amt298@cornell.edu}

\author{Tauhid Tanjim}
\orcid{0000-0003-0491-5876}
\affiliation{%
  \institution{Cornell University}
  \department{Information Science}
  \city{New York City}
  \state{NY}
  \country{USA}}
\email{tt485@cornell.edu}

\author{Huajie Cao}
\orcid{0009-0009-0347-3202}
\affiliation{%
  \institution{Michigan State University}
  \department{Media and Information}
  \city{East Lansing}
  \state{MI}
  \country{USA}}
\email{caohuaji@msu.edu}

\author{Jalynn Blu Nicoly}
\orcid{0000-0002-2897-5833}
\affiliation{%
  \institution{Colorado State University}
  \department{Computer Science Natural User Interaction Lab}
  \city{Fort Collins}
  \state{CO}
  \country{USA}}
\email{Jalynnn@colostate.edu}

\author{Jonathan Isaac Segal}
\orcid{0000-0002-8506-3785}
\affiliation{%
  \institution{Cornell University}
  \department{Information Science}
  \city{New York City}
  \state{NY}
  \country{USA}}
\email{jis62@cornell.edu}

\author{Soyon Kim}
\orcid{0000-0002-8232-9445}
\affiliation{%
  \institution{UC San Diego}
  \department{Computer Science and Engineering}
  \city{La Jolla}
  \state{CA}
  \country{USA}}
\email{sok020@ucsd.edu}

\author{Jonathan St. George}
\orcid{0000-0001-6810-4189}
\affiliation{%
  \institution{Weill Cornell}
  \department{Emergency Medicine}
  \city{New York City}
  \state{NY}
  \country{USA}
  \country{USA}}
\email{jos7007@med.cornell.edu}

\author{Kevin Ching}
\orcid{0000-0002-9008-5612}
\affiliation{%
  \institution{Weill Cornell Medicine}
  \department{Emergency Medicine}
  \city{New York City}
  \state{NY}
  \country{USA}}
\email{kec9012@med.cornell.edu}

\author{Francisco Raul Ortega}
\orcid{0000-0003-1285-6431}
\affiliation{%
  \institution{Colorado State University}
  \department{Computer Science}
  \city{Fort Collins}
  \state{CO}
  \country{USA}}
\email{F.Ortega@colostate.edu }

\author{Hee Rin Lee}
\orcid{0000-0002-5097-7309}
\affiliation{%
  \institution{Michigan State University}
  \department{Media \& Information}
  \city{East Lansing}
  \state{MI}
  \country{USA}}
\email{heerin@msu.edu}

%%
%% By default, the full list of authors will be used in the page
%% headers. Often, this list is too long, and will overlap
%% other information printed in the page headers. This command allows
%% the author to define a more concise list
%% of authors' names for this purpose.
\renewcommand{\shortauthors}{Trovato et al.}

%%
%% The abstract is a short summary of the work to be presented in the
%% article.
\begin{abstract}
How might healthcare workers (HCWs) leverage augmented reality head-mounted displays (AR-HMDs) to enhance teamwork?
Although AR-HMDs have shown immense promise in supporting teamwork in healthcare settings, design for Emergency Department (ER) teams has received little attention. The ER presents unique challenges, including procedural recall, medical errors, and communication gaps. 
To address this gap, we engaged in a participatory design study with healthcare workers to gain a deep understanding of the potential for AR-HMDs to facilitate teamwork during ER procedures. 
Our results reveal that AR-HMDs can be used as an information-sharing and information-retrieval system to bridge knowledge gaps, and concerns about integrating AR-HMDs in ER workflows. 
We contribute design recommendations for seven role-based AR-HMD application scenarios involving HCWs with various expertise, working across multiple medical tasks. 
We hope our research inspires designers to embark on the development of new AR-HMD applications for high-stakes, team environments.



\end{abstract}

%%
%% The code below is generated by the tool at http://dl.acm.org/ccs.cfm.
%% Please copy and paste the code instead of the example below.
%%
\begin{CCSXML}
<ccs2012>
   <concept>
       <concept_id>10003120.10003123.10010860.10010858</concept_id>
       <concept_desc>Human-centered computing~User interface design</concept_desc>
       <concept_significance>500</concept_significance>
       </concept>
   <concept>
       <concept_id>10003120.10003123.10010860.10010911</concept_id>
       <concept_desc>Human-centered computing~Participatory design</concept_desc>
       <concept_significance>500</concept_significance>
       </concept>
   <concept>
       <concept_id>10003120.10003123.10010860.10011694</concept_id>
       <concept_desc>Human-centered computing~Interface design prototyping</concept_desc>
       <concept_significance>300</concept_significance>
       </concept>
 </ccs2012>
\end{CCSXML}

\ccsdesc[500]{Human-centered computing~User interface design}
\ccsdesc[500]{Human-centered computing~Participatory design}
\ccsdesc[300]{Human-centered computing~Interface design prototyping}

%%
%% Keywords. The author(s) should pick words that accurately describe
%% the work being presented. Separate the keywords with commas.
\keywords{augmented reality, teamwork, emergency medicine, co-design}

% \received{20 February 2007}
% \received[revised]{12 March 2009}
% \received[accepted]{5 June 2009}

%%
%% This command processes the author and affiliation and title
%% information and builds the first part of the formatted document.
\maketitle

\section{Introduction}

% Big idea, set the stage. 1-2 paragraphs max

Healthcare workers (HCWs) experience mental and physical distress in the workplace, known as burnout \cite{tuna2022compassion,park2014supporting,alkema2008study}. 
Burnout is pervasive due to intense workloads \cite{pine2018data} and multitasking demands \cite{yen2018nurses} within time-sensitive environments \cite{jones2018can}. 
As HCWs work collaboratively, burnout in medical settings is also regarded as a systemic issue tied to poorly managed team dynamics \cite{southwick2020loss}. Factors such as miscommunication and a lack of collaborative support exacerbate individual stress, which leads to significant medical errors \cite{musick2023knowing,rosen2018teamwork,jeffcott2008measuring,serban2023just}. 
These challenges significantly impact HCWs’ engagement and performance, jeopardizing the delivery of safe, high-quality healthcare \cite{gray2019workplace}. 
Thus, we aim to improve team collaboration, with the support of augmented reality head-mounted displays (AR-HMDs), to mitigate medical errors and reduce HCW's mental and physical burdens.


\begin{figure}[t]
    \centering
    \includegraphics[width=1.0\linewidth]{figures/studyworflow.png}
    \caption{This shows the Zoom-based study workflow including two phases: Phase I reflected on teamwork challenges and ways augmented reality devices can help and Phase II focused on envisioning new AR-HMD interactions and reflecting on ways for it to support team collaboration.}
    \label{fig:studyworkflow}
    % \vspace{-7mm}
\end{figure}
% Motivation - Why AR?
AR-HMDs enable the integration of physical and computer-generated virtual objects, providing visual overlays of information that are widely applicable in fields such as education and industry \cite{akselrad2023body,silva2014glassist, speicher2019mixed, craig2013understanding}. 
In medical settings,  AR-HMDs offer unique affordances like 3D visualization, eye-tracking interaction, and real-time contextual feedback, making them well-suited for various healthcare applications. Researchers have found AR-HMDs could 
enhance remote medical surgery assistance \cite{maria2023supporting,siff2018interactive}, educate medical workers on various conditions and treatments \cite{shen2023dementia,uhl2023tangible}, facilitate telemedicine consultations \cite{anton2017augmented, wang2017augmented}, and provide real-time guidance during cardiopulmonary resuscitation (CPR) \cite{siebert2017adherence,cheng2024use}.
The ongoing progress in AR-HMD technology has also enabled these devices to enhance real-time communication and collaboration among human teams using visual displays \cite{lukosch2015collaboration,fidalgo2023survey}.
Especially in acute care, although low-tech solutions such as debriefing checklists and dashboards are effective for basic communication and task coordination \cite{buljac2020interventions}, they lack the interactive, real-time, and context-aware capabilities required in high-pressure, dynamic healthcare environments. Despite their promise, the application of AR-HMDs to support teamwork in acute care remains underexplored. 
% OMG PROBLEM: Oh no! If we don’t solve this, yikes!

While prior human-computer interaction (HCI) studies have explored various aspects of healthcare and teamwork, such as single-user medical procedures \cite{serban2023just}, non-acute care environments \cite{akselrad2023body,musick2023knowing,lukosch2015collaboration}, 
alert systems for trauma teams \cite{zhang2018coordination,zellner2023understanding}, and team coordination mechanisms \cite{zhang2018coordination,zellner2023understanding}, few have specifically focused on the unique challenges of AR-HMD interactions in high-stakes, time-sensitive acute care scenarios. 
These earlier studies often neglect the complexities of team-based AR-HMD interactions in acute care settings, where poor design could lead to patient death.
For instance, resuscitation, a life-saving procedure involving rapid medical interventions to restore circulation breathing requires split-second decision-making and flawless communication among healthcare team members in terms of counts and time duration \cite{leemeyer2020decision}. 
Our study aims to fill this gap in early-stage research to design requirements of AR-HMDs that support team collaboration and improve communication among healthcare workers in acute care environments.


% OMG OUR RESEARCH: Here you state, succinctly, how your research fills this gap.


To address these gaps, we engage in a co-design process with HCWs to understand the challenges they face during bedside care and explore ways that AR-HMD devices can help improve team communication and collaboration during bedside care in acute care settings (see Figure \ref{fig:studyworkflow}).
Over 11 months, we conducted remote semi-structured interviews with HCWs, including nurses, physicians, and residents to gather insight on AR-HMD environment designs.
Based on the discussion with the participants, we generated storyboards to depict situations where AR was identified as most useful to improve team collaboration and collected feedback to understand the nuances of integrating AR into time-sensitive team settings.
This approach allowed us to gain deeper insights into practical applications and potential benefits of AR-HMDs in the clinical setting.

% OMG CONTRIBUTIONS: Here you state, clearly and succinctly, your contributions. 

This paper makes \textbf{three contributions} to the field: 1) We present AR-HMDs ideations co-designed by HCWs to enhance clinical teamwork during medical procedures in high-stakes environments. 2) We introduce seven AR-HMD application scenarios involving HCWs with various expertise, working across multiple medical tasks and encompassing single-user and multi-user interactions. 3) Our research highlights the current barriers to using AR-HMDs in acute care settings, including issues with user adoption, technology literacy, and the potential risks for distractions and obstructed views during critical tasks.
By addressing these challenges, we provide practical insights for improving the effectiveness of clinical teams using AR-HMDs.


%%%%%
% Section 2: Background
%%%%%

\section{Related Work}

\subsection{Teamwork in Healthcare}

Patients encounter preventable harms in healthcare, such as unexpected infection and errors in diagnosis, medication, and surgery \cite{panagioti2019prevalence}. 
To ensure safe and high-quality care delivery, it is important to establish efficient teamwork to clarify roles and responsibilities and communicate medical information \cite{barnard2020communication}. 
However, such efforts are consistently challenged by chaotic and overcrowded clinical environments \cite{ahmadpour2021efficiency}, insufficient patient information \cite{bossen2014physicians}, and intensive workloads of HCWs \cite{pine2018data}. 
Especially in critical care delivery, HCWs experience physical, mental, and emotional fatigue under time-sensitive medical procedures \cite{welp2016interplay}. 
Moreover, the power dynamics within the team can discourage members from speaking up even in cases where evidence-based treatment protocols are violated \cite{li2018organizational}.
Therefore, supportive sociotechnical systems are needed to facilitate the cooperation of HCWs and improve patient safety.
Macrocognition and distributed cognition frameworks help researchers explore the complexities of team dynamics. Macrocognition highlights how teams perform cognitive work in dynamic environments through shared mental models and situational awareness \cite{mastrianni2022alerts}, while distributed cognition reveals how team members, tools, and the environment share critical information \cite{sarcevic2011coordinating, zhang2022designing}.
HCI researchers have explored technological solutions to enhance medical teamwork. 
For example, Jagannath and colleagues \cite{jagannath2019temporal} examined the use of an electronic flowsheet in medical resuscitations and suggested aligning the flowsheet with the actual flow of medical activities to improve real-time communication. 
Emerging technologies such as AR have revolutionized how information is collected and displayed, paving the way for a more promising future for medical teamwork. 
Yet, there’s a scarce understanding of how to best integrate these technologies into acute care team settings.


%\sout{AR-HMDs are continuously evolving, enhancing the immersive experience for users through a variety of input modalities such as speech, gesture, and gaze}. \sout{Each modality offers distinct advantages that fit a particular application and improve human-human interactions in the environment. For example, recent advancements in holographic optical elements (HOEs) and lithography-enabled devices have enabled significant improvements in display resolution and field of view, which were challenging aspects of traditional AR-HMDs. } \sout{These developments not only enhance the overall performance of AR-HMDs but also significantly improve the immersive experience \cite{xiong2021augmented}.} \sout{Additionally, current sensor technology improvements have made gesture and eye-tracking inputs more accurate and sensitive.} \sout{This progress is illustrated by the development of specialized tools such as the Augmented Reality Eye Tracking Toolkit (ARETT) which integrates eye trackers into AR-HMDs }\cite{adhanom2023eye, kapp2021arett}\sout{, and ARKit and ARCore which are popular open-source libraries to display virtual objects \cite{oufqir2020arkit}, thereby improving user interaction }\cite{jain2023ubi}.

%\sout{Recent successes in multimodal interaction methods in AR-HMDs, integrating multiple input modalities, are gaining popularity as evidenced by studies such as Gesture-Speech Integration \cite{piumsomboon2014grasp} and GazePointAR \cite{lee2023towards}. These studies show that a combination of free-hand gestures and speech, or utilizing eye gaze and pointing, can improve how users interact depending on the context. Additionally, Wang et al. \cite{wang2019exploring} have shown that the representation of virtual agents plays a key role in AR by affecting user choices and preferences.} \sout{Additionally, intuitive AR-HMD text input techniques, such as taps on skin \cite{kim2023star}, finger touches on skin }\cite{xu2019tiptext, xu2020bitiptext}\sout{, and eye tracking combinations }\cite{lu2021itext, cui2023glancewriter}\sout{ by users, demonstrate efficiency and contribute significantly to user interface design improvements in AR-HMDs.} \sout{Despite these advancements, there is still a significant research gap in AR-HMD applications for collaborative, time-sensitive, high-risk settings, and in workplaces \cite{gower2022utilizing}.} % taken

%\sout{In an attempt to improve team collaboration during time-sensitive work tasks, several AR-HMD options that utilize multiple input modalities have been explored. While the display options may showcase a variety of information, the display should always act as a notification, providing the user with necessary information that they can appropriately respond to.} \sout{It was found that both visual and auditory notifications provide high recognition rates, with the most frequent input modality choice being speech ~\cite{lazaro2021interaction}.} % this line taken from here to 2.3 \sout{This finding corresponds with other studies demonstrating that a combination of input modality options is often superior to one input modality, utilizing gaze, gesture, and speech ~\cite{wang2021interaction}.}\sout{Further, while the types of input modalities included are crucial, it is also important to recognize the placement of displays, not to distract users from their time-sensitive task but to provide them with the required information. }%taken \sout{We found that the least distracting position is the bottom center of the user's field of view and that wrist attachment is recommended when there is a high amount of AR-HMD interaction \cite{plabst2022push}.} %taken \sout{While there are many design implications beyond input modalities and placement, such as color and size, designing an intuitive display option would considerably improve human-human interaction.}

%\sout{Integrating AR-HMDs into real-world scenarios has become more straightforward due to the advancements of SDKs and development environments. Platforms such as Sketchbox allow for simple deployment of advanced AR-HMD environments \cite{SketchboxLeadingVirtual}. This allows for more focus on important design concepts such as the visual interfaces and interactions. %Conceptualizing AR-HMDs as a series of micro-interactions is a powerful tool for improving user-centered design where users take subtle actions that the headset can interpret to display relevant information. While AR-HMDs are more accessible than ever, designing polished applications with well-defined micro-interactions is crucial for the adoption of AR devices. Researchers must carefully develop AR-HMD applications for healthcare workers in time-sensitive, high-stakes environments to improve outcomes while maintaining effective teamwork and performance and avoid introducing new medical errors.}


%\end{comment}

\subsection{Augmented Reality in Healthcare}

AR-HMDs have demonstrated significant potential in healthcare through visualization of medical processes \cite{huang2018use}, delivery of real-time medical information \cite{wang2015real, jayaprakash2019asthma}, and navigation of human anatomy \cite{lungu2021review}. Technological advances in holographic optical elements and lithography-enabled devices have enhanced display resolution and user experience, improving medical visualization quality \cite{xiong2021augmented}. In medical education, interactive AR-HMD simulations outperform traditional methods for teaching surgical anatomy \cite{siff2018interactive, lungu2021review}. For surgical applications, AR-HMDs assist decision-making by overlaying medical history and dimensional data onto surgical areas, helping guide procedures such as cutting angles \cite{pokhrel2019novel}. AR-HMDs enhance telemedicine capabilities by enabling specialists to remotely monitor patients' vital signs, capture 3D anatomical data, and provide real-time visual guidance and annotations \cite{anton2017augmented, wang2021ar}. AR-HMDs have advanced telemedicine by enabling specialists to monitor vital signs and provide real-time feedback in remote settings \cite{wang2021ar, anton2017augmented}. In addition to what information to show, work has been done exploring how to design well-contextualized AR-HMD visualizations \cite{PlabstVisualisation2021}. Integrating speech, gesture, and gaze inputs \cite{liao2022realitytalk, muller2023tictactoes} enhances medical application control and healthcare team collaboration. Studies demonstrate these multi-modal interfaces significantly outperform single-mode systems \cite{wang2021interaction}, with clinicians achieving high recognition rates through combined visual and auditory notifications \cite{lazaro2021interaction}. To optimize clinical usability, research recommends bottom-center or wrist-mounted displays during frequent AR-HMD interactions \cite{plabst2022push}. While these devices show promise in applications like CPR training \cite{semeraro2013motion, almousa2019virtual, balian2019feasibility, siebert2017adherence}, their potential for team communication in critical care scenarios remains largely unexplored \cite{gower2022utilizing}. Furthermore, while each of these AR-HMDs is useful depending on the needs of the healthcare team, improving team collaboration through AR interfaces has received limited attention. Therefore, this study aimed to bridge this research gap by conducting a co-design study with medical professionals to explore how AR could enhance medical teamwork in acute care settings. 

%AR-HMDs are gaining attention from healthcare researchers for its innovation in merging the boundary of the virtual world and the real world, hence facilitating visualization of medical processes \cite{huang2018use}, delivery of real-time medical information \cite{wang2015real, jayaprakash2019asthma}, and navigation of human anatomy \cite{lungu2021review}. In addition to what information to show work has been done exploring how to design well-contextualized AR-HMD visualizations \cite{PlabstVisualisation2021}.These applications have become increasingly feasible due to the ongoing shift in AR-HMD technology, with recent advances in holographic optical elements and lithography-enabled devices enabling significant improvements in display resolution in users' field-of-view. This progress improves both the overall performance of AR-HMDs and the immersive experience of users \cite{xiong2021augmented}. Advanced SDKs and tools like Sketchbox have simplified AR-HMD integration, enabling researchers to focus on interface and interaction design, enabling faster and more robust ideation  \cite{SketchboxLeadingVirtual}. A great body of literature has demonstrated AR for different domains of healthcare such as acquiring and sharing skills and knowledge. For example, Siff and Mehta \cite{siff2018interactive} introduced MS HoloLens for creating 3D-projected organ simulations in training modules. Their research revealed that participants not only demonstrated improved comprehension of surgical anatomy but also evaluated the AR-HMD training as superior to conventional methods. In clinical practice, AR-HMDs offer a digital overlay of information that enhances surgical precision and minimizes risks. Pokhrel and colleagues \cite{pokhrel2019novel} found that by presenting the medical history and dimensional information of the surgical area, AR-HMDs can facilitate surgical decisions such as cutting angles. To support these clinical applications, integration of multiple input modalities such as speech, gesture, and gaze \cite{liao2022realitytalk, muller2023tictactoes}, provides distinct advantage and flexibility in adapting medical applications and improves human-human interactions in healthcare settings. The effectiveness of these systems depends heavily on their implementation. While the display options may showcase various information, the display should always act as a notification, providing the user with the necessary information to respond appropriately. This finding corresponds with other studies demonstrating that combining input modality options is often superior to one input modality, utilizing gaze, gesture, and speech ~\cite{wang2021interaction}. While the display options may showcase a variety of information, the display should always act as a notification, providing the user with necessary information that they can appropriately respond to. It was found that both visual and auditory notifications provide high recognition rates, with the most frequent input modality choice being speech ~\cite{lazaro2021interaction}. Further, while the types of input modalities included are crucial, it is also important to recognize the placement of displays, not to distract users from their time-sensitive task but to provide them with the required information. We found that the least distracting position is the bottom center of the user's field of view and that wrist attachment is recommended when there is a high amount of AR-HMD interaction \cite{plabst2022push}. Beyond in-person clinical applications, AR paved the way for new types of telemedicine to address medical issues in rural and remote areas \cite{wang2021ar}. Through AR-HMD platforms, the experienced specialist could capture 3D body information and vital signs (e.g., heart rate and blood pressure) of the patient and then give real-time feedback and visual annotation \cite{anton2017augmented}.

%While AR-HMDs have grasped the attention of providing unique learning experiences of human anatomy and being able to communicate health problems via telemedicine, there is a lack of using the device to support team communication and collaboration in healthcare settings. This gap is particularly notable in critical care scenarios, where team coordination is crucial. Specific to cardiac arrest situations, it is recognized that AR-HMDs have been very beneficial in providing guidelines on how to appropriately provide cardiopulmonary resuscitation (CPR) ~\cite{semeraro2013motion, almousa2019virtual}.These studies showcase typical AR-HMDs with insight into related measures such as rate, depth, and recoil, perhaps with the help of additional hardware to track these measures. Other studies observing the influence of AR on CPR training have also previously shown the patient's general anatomy ~\cite{balian2019feasibility} and cardiac arrest procedural steps ~\cite{siebert2017adherence}. Displaying the procedural steps following the AHA guidelines through an augmented display rather than a paper copy may be the most applicable application that improves healthcare collaboration as it has the opportunity to eliminate distractions and maintain focus on the patient. Despite these advancements, there is still a significant research gap in AR-HMD applications for collaborative, time-sensitive, high-risk settings, and in workplaces \cite{gower2022utilizing}. Furthermore, while each of these AR-HMDs is useful depending on the needs of the healthcare team, improving team collaboration through AR interfaces has received limited attention. Therefore, this study aimed to bridge this research gap by conducting a co-design study with medical professionals to explore how AR could be employed to enhance medical teamwork in acute care settings. 


\subsection{Co-Design with Healthcare Teams}

Researchers have widely used co-design methods to develop effective interventions for specialized teams, such as in medical settings, helping bridge the gap between technological capabilities and clinical needs \cite{mastrianni2023transitioning, hoppchen2024insights, hsieh2023alternatives, shen2023dementia}. Kusunoki et al. \cite{kusunoki2015sketching} described their work as participatory design and conducted workshops with HCWs to optimize display designs that support verbal communication patterns. Similarly, Mastrianni et al. \cite{mastrianni2022alerts} interviewed trauma teams to develop guidelines for decision support alerts, while they explored various systems they primarily focused on digital checklists, wall displays, and audio alerts to improve team coordination without disrupting patient care. %\sout{These \textcolor{blue}{co-design}\sout{participatory} methods help researchers understand the specific requirements of medical teams, from displaying HCWs' role effectively to integrating new tools into existing workflows \cite{sarcevic2011coordinating}. Researchers have explored more advanced medical support platforms as they build on previous work that improves medical workflows \cite{mastrianni2023transitioning}. These co-design studies demonstrate that while AR-HMDs offer benefits, effective design must balance user-friendliness with the hectic emergency room environment. Co-design methods have proven to be a valuable technique to develop AR-HMD solutions for medical teams, as they help bridge the gap between technological capabilities and clinical needs.}
Zhang et al. \cite{zhang2022designing} explored sociotechnical factors affecting smart glass adoption in emergency medical services. %\sout{This perspective emphasizes the importance of considering AR-HMDs implementation's technical and social dimensions in healthcare settings, particularly how new technologies might affect team dynamics and communication patterns.} 
These co-design sessions revealed key considerations around team situational awareness, cognitive load during patient care, and the balance between providing useful information and avoiding information overload \cite{mastrianni2022alerts, kusunoki2015sketching}. 
Through co-design, researchers can better understand how AR solutions need to adapt to different roles within the medical team, from experienced physicians to new nurses, ensuring that the technology supports collaboration \cite{sarcevic2011coordinating, chung2023negotiating}.



\begin{table*}[]
\small
%\scriptsize
\caption{Participant demographic information including the participant ID, age, gender, years of experience, medical specialty, and types of hospitals worked at, including teaching (T), non-teaching (NT), for-profit (FP), non-profit (NT), urban (U), and rural (R).}
\begin{tabular}{
>{\columncolor[HTML]{FFFFFF}}l 
>{\columncolor[HTML]{FFFFFF}}r 
>{\columncolor[HTML]{FFFFFF}}l 
>{\columncolor[HTML]{FFFFFF}}r 
>{\columncolor[HTML]{FFFFFF}}l 
>{\columncolor[HTML]{FFFFFF}}l }
\hline
%\textbf{Participant \#} & \multicolumn{1}{l} \textbf{Age} & 
\textbf{Participant \#} & \textbf{Age} & \textbf{Gender} & \multicolumn{1}{l}{\cellcolor[HTML]{FFFFFF}\textbf{\begin{tabular}[c]{@{}l@{}}Experience \\\end{tabular}}} & \textbf{Specialty}                                  & \textbf{Hospital-Type}  \\  \hline P1            &  53                                                       & M               & 22 years                                                                                                                 & Physician, Pediatric Emergency Medicine             & T, NP, U                \\
P2              &  33                                                       & M               & 1.5 years                                                                                                               & Healthcare Administrator, ICU Nurse                 & FP                      \\
P3                                     & 57                                                                              & M               & 18 years                                                                                                                & MD, Emergency Medicine                              & T, NT, FP, NP, U, R \\
P4                                      & 51                                                                              & F               & 21 years                                                                                                                & Attending, Emergency Medicine                       & T                       \\
P5                                      & 36                                                                              & F               & 8 years                                                                                                                 & MD, Pediatric Emergency Medicine                    & T, NT, NP, U            \\
P6                                      & 37                                                                              & M               & 8 years                                                                                                                 & Director of Clinical Innovation, Emergency Medicine & T                       \\
P7                                      & 32                                                                              & F               & 6 years                                                                                                                 & Registered Nurse                                    & T, NT, FP, NP, U        \\
P8                                      & 66                                                                              & M               & 39 years                                                                                                                & MD, Emergency Medicine                              & T                       \\
P9                                     & 42                                                                              & M               & 8 years                                                                                                                 & Pediatric Emergency Medicine Attending Physician    & T, NT, NP, U            \\
P10               &  38                                                       & M               & 13 years                                                                                                                & Assistant Professor of Emergency Medicine           & T, NP, U                \\ \hline
\end{tabular}
\label{table:demo}
\end{table*}


%%%%%%%%%%%%%%%%
%%%%%
% Section 3: Methodology / Algorithmic Approach / etc
%%%%%
\section{Methodology} 
%%%%%%%%%%%%%%%%

\subsection{Participants}

We recruited 10 healthcare practitioners ($N=10$) through medical school mailing lists and professional referrals. 
The participants include 7 men and 3 women with expertise in Emergency Medicine with specialties that include Physicians, Attendings, Director of Clinical Innovation, Registered Nurses, an Assistant Professor of Emergency Medicine, and a Healthcare Administrator with Registered Nurse in Intensive Care Units (ICUs) experience (see Table \ref{table:demo}). 
No medical students were included.
All participants have experience performing bedside care, including resuscitation working with children and adults.
Their ages ranged from 32 to 66 ($M=44.5$, $SD=11.5$) with years of experience ranging from 1.5 to 39 years ($M=14.5$, $SD=11.0$).
The participants have worked across many institution types, including teaching (9), non-teaching (4), nonprofit (6), for-profit (3), urban (6), and rural areas (1).
The participants indicated an average score of 2.4 out of 5 on their familiarity with AR systems.
Lastly, participants were compensated with an \$18 Amazon gift card.

\subsection{Procedure}

% Add later: %(\#STUDY00008415)
We engaged in an IRB-approved (\#STUDY00008415) remote study involving an iterative co-design process with healthcare workers and medical educators over 11 months to understand the challenges they face during bedside care tasks. 
We conducted semi-structured interviews over Zoom with HCWs to understand how AR-HMDs can support collaboration (see Figure \ref{fig:studyworkflow}).
Our research team in interviews included 1 moderator, 1 sketcher, and a few more note-takers, depending on their availability. Based on the expertise of each participant, HCWs took on the role of demonstrating the potential use of AR-HMD based on their workplace experience and contextual knowledge, while design researchers played a role in facilitating the ideation process and in visualizing the HCWs' design ideas. 
While a moderator conducted interviews, a sketcher generated sketches of interaction scenarios (see Fig. \ref{fig:storyboard}).



The interviews started with an introduction to our study goals as aforementioned. 
%Next, we introduced two preliminary AR-HMD applications that we developed to help HCWs get familiarized with AR systems. The AR-HMD application processes the user’s speech to detect their dialogue and their gesture to detect the gesture direction. The direction and dialogue information overlay an annotation in the direction of a user's gesture.
To familiarize participants with AR-HMDs, we discussed how AR-HMDs work, the data used for interaction, and ways users can interact with user interfaces to support their clinical workflow and teamwork.
Once familiarized, our team conducted two separate study phases during the interview: 1) focused on understanding team challenges, and 2) ideating on new AR-HMDs that can address these challenges.
While a moderator leads interviews, a sketcher generates storyboards and sketches of human-AR interactions based on the discussions between an interviewer and an interviewee. 


\textbf{Phase I:} HCWs experience many challenges during ER resuscitation procedures, which served as the basis for our interviews \cite{taylor2024towards,taylor2019coordinating,taylor2020situating,taylor2022hospitals}. For instance, new practicing nurses are often unfamiliar with certain resuscitation practices not covered in medical textbooks (e.g., managing patients along with family members) but instead are based on evidence-based practices \cite{li2018organizational}. Also, the specialty of HCWs defines the roles they can take on and effects how effectively they work together e.g., nurses calculate medication dosage when pharmacists are not available and the use of reference cards for HCWs that forget procedural steps. Furthermore, the number of people available reflects the number of resources required for patient treatment. Furthermore, medical errors are a well-known challenge in healthcare, often caused by human error such as disagreements on care, inaccurate dosing, and untimely updates on patient status/history \cite{carver2024medical,garcia2019influence,hall2016healthcare}. Lastly, HCWs are trained to use close communication gaps (CCG) to repeat patient information verbally and out loud to ensure effective information flow at the bedside to maintain situational awareness and up-to-date patient information and the status of medical tasks. This involves repeating patient information when it is given verbally out loud. Failure to close communication gaps is another challenge that HCWs encounter.
We framed our interview questions around these challenges. 

As part of the interview, we started by asking participants to reflect on similar situations they experienced in the workplace, and about the potential of using AR-HMDs to address these issues.
We asked questions such as `How can AR-HMDs help mitigate teamwork challenges?', `What interface elements can help you perform medical tasks?', `How would you prefer to interact with the AR-HMDs?', `What tasks are useful for AR-HMDs assistance?', and `How would those use cases address teamwork challenges?'
As participants ideated, a member of our team sketched their design ideas in storyboards.

\textbf{Phase II:} In this phase, we asked participants to reflect on the storyboards generated in Phase I and engaged in iterative live sketching to improve the depictions to gain a rich understanding of participant narratives of important design elements \cite{taylor2024towards}. 
Through targeted questions, we explored how AR head-mounted displays (AR-HMDs) could address the teamwork challenges identified in Phase I, including code team coordination, error detection, and communication gaps. 
Some of our interview questions included: 'How do you imagine everyone having AR headset for personalized views of vital signs based on their roles?', 'What are your thoughts on having AR headsets for all team members?', 'How should the team leader's AR display differ from other team members?', and 'How would you like time-sensitive and priority-based tasks to be displayed in the AR view?'
Our goal was to ensure the visualization of participants' design ideas correctly reflected the participants' intentions. 
Our core focus was to understand how the proposed AR-HMDs can potentially address teamwork challenges experienced by study participants and how this technology might influence human-human interactions. 


\textbf{Analysis:} We recorded video of interview sessions, transcribed the audio data using Microsoft Stream, and coded it using the thematic coding process \cite{braun2012thematic}. 
This involved three members of our team reviewing the interviews (2 per transcript), generating themes that emerged directly from the data, negotiating those high-level themes and sub-themes and adjusting them based on mutual understanding of participant perspectives, and repeating this process until there is an agreement on the generated codes.
Some themes include 'information to display', 'how to display information', 'AR-HMD guidance', 'maintaining human control', 'concerns for AR-HMDs', 'inter-team visual information sharing', 'automated medication dosage calculations', and 'alert systems'. 

\begin{figure*}[t]
    \centering
    \includegraphics[width=1.0\linewidth]{figures/storyboard_ar.png}
    \caption{Example storyboards generated in Phase I. This shows an example of using AR-HMDs to guide healthcare workers (HCWs) through medical procedures, starting with handing off patient information from Emergency Medical Services to HCWs in the Emergency Room and using AR-HMDs to visualize tasks for HCWs and walk them through steps to treat a patient based on standard procedural protocols including timings to start and end a sub-tasks in a medical procedure, identify medical equipment needed for sub-tasks, and potential deviations from standard procedure based on pre-existing conditions.}
    \label{fig:storyboard}
    %\vspace{-4mm}
\end{figure*}

 %%%%%
% Section 4: Results or Findings
%%%%%
\section{Findings}

\subsection{AR as an information-sharing system for healthcare teams}

The high-stress and chaotic working environment in the ED motivates the importance of real-time visual information displays.
We found that participants hoped AR-HMDs could facilitate medical teamwork through patient- and HCW-specific information tracking, allowing ideation beyond what the devices may currently be capable of.

\textbf{Avoid Missed Information Using Notifications:} Oftentimes, ER HCWs at the bedside were exposed to an overwhelming amount of information through many sources, which may lead to the unintentional oversight of important details (see Figure \ref{fig:storyboard5}). P8 envisioned using AR-HMDs to deliver important alerts: ``\textit{I can imagine in that situation a new alert or new data popped up for that patient in real-time like an X-ray scans or a new recommendation or message from a consultation service.}'' Similarly, P3 prefers crucial visual alerts: ``\textit{So [for example], if blood glucose is listed as red because they haven't heard it or done it, [then] that's a reminder to me that hasn't been done. After making sure it’s done and then that can sort of fade off or turn green or whatever, then suddenly now I have this prompting that not only moves things forward but also picks up errors and omissions at the same time.}'' These examples highlight the potential usefulness of adding visual alerts in medical settings, along with the need to consider the features AR devices are currently capable of. While these devices are capable of displaying real-time visual cues, the ethical concerns and verification of procedural completions remain a significant challenge.

\textbf{Patient-Specific Visual Displays:} Timely actions were inhibited by dispersed information of patients among multiple people. 
As a result, HCWs hoped that AR-HMDs could help with information retrieval and present noteworthy information visually (see Fig. \ref{fig:proc_guide}). 
P3 stated: ``\textit{Rather than having to either make someone leave the bedside to go to a desktop computer to gather information and then come back and tell me what the relevant information is, having a code relevant short list of good data visualization that pops up that tells me this patient's DNR status, allergies, and most relevant past medical history.}'' While AR visual displays could extend beyond visual alerts to include a comprehensive view of the patient's relevant medical history, it's crucial to address the technical and ethical challenges involved in accurately pulling and displaying the relevant patient data from the electronic health records system (i.e., EPIC).

\textbf{HCW-Specific Visual Displays:} In acute care, medical procedures demand collaboration in multidisciplinary teams, with limited time, primarily focused on comprehending the patient's condition. 
Consequently, team members often have minimal opportunity to become acquainted and share relevant information.
P9 proposed using AR-HMDs to add environment annotations above team members to improve communication and coordinate themselves spatially around the bedside: ``\textit{With the AR, if everyone's wearing an ID and you can see who's in the room and what their names and roles are, you can ask, for example: ‘Brian, would you mind just stepping outside? To make some space for everyone else.’ That kind of thing could actually help with the communication quite a bit.}''
This expectation underscores the significance of the dynamic nature of healthcare teams, which are composed randomly and unpredictably. 
Real-time displays of user roles and specialties using current AR-HMDs in conjunction with individual tracking, can address this issue and enable participants to prepare and seamlessly integrate into the team dynamics. 
This function can also inform HCWs outside the room about any missing expertise, allowing HCWs with the appropriate level of experience and expertise to join the team instead of random HCWs entering the room and potentially overcrowding it.

\subsection{Bridging Knowledge Gaps Through Information Retrieval}


During the co-design process, participants found versatile applications of AR-HMDs to effectively bridge information and knowledge gaps among team members, such as maintaining closed-loop communication through procedural guidance. These applications could expand beyond the foundational work reviewed by Gerup et al. (2020) \cite{gerup2020augmented}.

\textbf{Visual updates for latecomers:} AR-HMDs were expected to serve as a real-time information retrieval system to provide patients and procedural updates for HCWs upon arrival to the patient’s room after a code has started.
P9 described using AR-HMDs to promptly retrieve and deliver real-time patient information upon receiving pager alerts.
While on their way to the room, HCWs can receive ongoing updates on the procedure's progress as they approach the patient’s room.
P9 said: ``\textit{[If] HCWs have these [AR] headsets [on] and as soon as they get the call, they're able to put on the headset and receive [patient] information in real-time from the team who are actually there. There may be some utility in terms of creating a mental model and making sure that everyone's on the same page. As soon as they get to the room, they know exactly what's going on and that can be a live feed of what's going on in the room.}'' For HCWs enroute to the patient room, a preparation system providing how they could be helpful in the room and the steps to do so can be extremely beneficial for individual contribution and team collaboration. AR devices are currently capable of displaying visual information, and the implications for informing outside HCW would have similar ethical and technical concerns as displaying alerts.

\textbf{Rare Procedural Guidance:} AR-HMDs were anticipated to bridge knowledge gaps between HCWs (see Fig. \ref{fig:rare_proc}).
AR-HMDs can provide guidance on rare procedures without the physical presence of experts, as illustrated by P2: ``\textit{With AR headset, specialists could have been at home or wherever he was and taught me how to do this procedure, seeing my landmarks, being able to annotate and guide in real time — `cut here, clamp here'.}''
This function was especially beneficial for hospitals with limited resources as described by P8: ``\textit{One of the biggest challenges is [a] lack of familiarity with some of our auxiliary staff or nursing staffing are less used to dealing with a critically ill child.}''  
While care for children requires specific procedures during a code situation, staff with this specialized expertise are not always readily available.
Having specialized expertise remotely available could significantly improve the quality of procedural tasks that are rarely completed or lack local expertise, as observed in Dinh et al.'s literature review \cite{Dinh2023AugmentedRT}.
Expanding on the concept of AR-HMDs for remote guidance, it's interesting that participants also discussed the potential use of AR for in-person guidance within the same physical space (see Fig. \ref{fig:proc_guide}).
P9 explained how AR interfaces can facilitate guiding medical procedures by enabling senior HCWs to offer immediate visual feedback to junior trainees. 
P9 described: ``\textit{So the senior is kind of standing away, he is able to see what the junior trainees see and able to even annotate or provide guidance. For example, assist in identifying the landmark.}''
In spatially constrained environments like patient rooms, AR-HMDs have already provided innovative approaches utilizing AR cues, as evident in Sereno et al.'s work \cite{sereno2020collaborative}. 

\begin{figure*} 
     \centering
     \begin{subfigure}[b]{0.4\textwidth}
         \centering
         \includegraphics[width=\textwidth]{figures/sb-fig1.png}
         \caption{}
         \label{fig:storyboard1}
     \end{subfigure}
     \hfill
     \begin{subfigure}[b]{0.5\textwidth}
         \centering
         \includegraphics[width=\textwidth]{figures/sb-fig2.png}
         \caption{}
         \label{fig:storyboard2}
     \end{subfigure}
\caption{Design sketches for procedural guidance. a) AR-HMDs that show CPR rounds (figure instructs HCW to perform round 3 of total 4) and time until it must be completed (3 minutes and 26 seconds). b) AR-HMDs for administering medications, displaying patient vital signs, and task bundles.}
\label{fig:proc_guide}
%\vspace{10mm}
\end{figure*}

\subsection{Primary users of AR-HMD in Clinical Teams} 

Participants saw the value in using AR-HMDs to improve teamwork by reducing errors during bedside care. However, our analysis revealed a primary \textit{design tension} where participants indicated different opinions about who on the team should wear the AR-HMD in the team. 

\textbf{Sole Custodian - Designated Headset Bearer:} Three participants thought the team leader should manage the AR-HMD information to better coordinate team members and focus on their ongoing medical tasks. 
On the contrary, those with leadership experience mentioned that the team leader should have complete situational awareness to make critical decisions. Therefore, the recorder or the nursing leader should wear the headset as discussed by P3: ``\textit{If I'm trying to keep track of all of them at once I can be overwhelmed, such as chest compressions, and blood pressure. If the recorder has it [ wearing the AR headset], then the recorder [could say], ‘Hey, we're at 3 min [of the chest compressions]’.  And then that would take even more workload off of the team leader.}'' As a result, they felt the responsibility of coordinating AR-HMD information should be assigned to administrative staff such as the recorder or the nursing leader. P3 described how the person wearing the AR headset could help team leaders reduce the burden from routine work: ``\textit{If I'm trying to keep track of all of them at once I can be overwhelmed, such as chest compressions, and blood pressure. If the recorder has it [(AR headset]), then oftentimes, because it's very algorithmic, the recorder is like, ‘Hey, we're at 3 min [of the chest compressions]’.  And then that would take even more workload off of the team leader [so] the team leader could think about what the reversible causes are, what are the interventions that might fix the problem.}''
All six participants expressed concern that if individuals engaged in medical tasks were to use AR-HMD, it might unnecessarily increase their cognitive load as stated by P8: ``\textit{For me, it was disorienting to try to have that (AR headset) on my face while I'm also trying to keep an eye on things. And then also I need to hear what I like [to know], as well as [information from] the team leader.}'' They also feared the potential for distraction by information irrelevant to their immediate tasks. It is critical that the information displayed shows minimal and time-relevant information. However, it is important to consider a general consequence of AR-HMDs, which is that they can increase cognitive load just by the strain of the weight on the user's head.

\textbf{Universal Adoption: Equipping Everyone with the Headset:} On the contrary, four participants were more inclined that everyone should wear the headset. 
P6 believed that AR-HMDs could play a more important role in the medical setting by providing personalized feedback for everyone in the team based on their role: ``\textit{Everyone wears a headset! The reason is that there were multiple layers to shape or form the situation and everybody was involved in some way, and a lot of people made errors in that situation. So if only one person was wearing a headset, then for the rest of the people who are making mistakes, they can’t be directed by the device.}''
Some participants further pointed out that AR-HMDs can build a \textit{shared mental model} of the medical task process among team members, as P4 explained: ``\textit{The vitals need to be on display for everyone. That we kind of already do. I do think it would be beneficial for everyone to see medications and the number of shocks given, so if you could have that on display for everyone, I think that's very helpful.}'' While the technical side of this approach is feasible, there needs to be consideration on how to update the displays appropriately.

\begin{figure} 
     \centering
     \begin{subfigure}[b]{0.3\textwidth}
         \centering
         \includegraphics[width=\textwidth]{figures/sb-fig3.png}
         \caption{}
         \label{fig:storyboard4}
     \end{subfigure}
     \hfill
     \begin{subfigure}[b]{0.4\textwidth}
         \centering
         \includegraphics[width=\textwidth]{figures/sb-fig4.png}
         \caption{}
         \label{fig:storyboard5}
     \end{subfigure}
\caption{AR-HMDs for rare procedures. a) Similar to the Lucas device that is often positioned on the crash cart, the AR-HMDs can also be placed in this location. AR-HMDs can be used to confirm patient consent, communicate with the crash cart to help locate medications or other tools in the cart, and walk users through rare procedures, including administering medications not often provided to patients. b) AR-HMDs for the team leader to track patient vital signs and pending tasks (left) and an individual interface for the team members to track tasks specifically assigned to them.}
\label{fig:rare_proc}
%\vspace{-5mm}
\end{figure}

\subsection{Concerns about integrating AR-HMDs in ER workflows} 

Six of ten participants expressed concerns about using AR-HMD technology for bedside care, primarily due to cognitive overload and the potential to ``\textit{disorient} (P8)'' them from being presented with too much information as described by P0: ``\textit{There are so many other things in your peripheral vision that you need to be able to see. I'm familiar with Google Glass from years ago. It's a big sort of thing on your face and plus you have to know how to interact with it and know when to call up certain things.}'' While participants acknowledged the promises of AR, six out of ten expressed concerns about implementing this new technology in bedside care. The primary concern revolved around information overload in this time-sensitive situation. 
For this reason, participants wanted to ensure that AR-HMD information is only shown when necessary, as indicated by P3: ``\textit{Just the ability to display data in a way that gives me useful information but also doesn't distract me from my situational awareness.}'' Another issue was about head-mounted equipment. Utilizing wearable devices necessitates constant carriage for the sake of accessing their benefits. As soon as individuals discern that the drawbacks of bearing the device eclipse the advantages it provides, they are prone to swiftly discontinue its use. P6 pointed out if the alert from the AR-HMD frustrated HCWs, they would take the device off immediately. As a result, some participants preferred to use a fixed screen (AR) like a TV as P4 said: ``\textit{So with displays like a TV, if people want to see it, they're able to see it. But they can also just focus on the task at hand.}'' 

Meanwhile, there is resistance to adopting new things in the real world, as discussed by P2: ``\textit{The truth is right now I think this idea of wearing a headset while I'm doing a procedure is foreign enough. It may be a little bit more difficult to broadly buy in [...] I wouldn't feel secure [using AR].}'' Two participants found it foreign to put on AR-HMD in medical activities. 
P3 further mentioned that AR-HMDs could inhibit the holder’s communication with team members, such as eye contact, which was an essential way to cultivate rapport.

In summary, the Phase II results underline the potential benefits of AR-HMD in supporting healthcare teams collaboration and communication. The AR-HMD real-time information visualization feature is crucial to team collaboration because its delivery of targeted data and procedural guidance can bridge knowledge gaps and support effective team communication. However, despite these advantages, participants expressed concerns about who should wear AR-HMD, highlighting issues with cognitive load and potential distractions during critical tasks. Furthermore, participants are divided between supporting universal adoption and recommending selective use of AR-HMD to avoid information overload and ensure situational awareness. This feedback reflects a cautious yet optimistic view on AR-HMD in clinical settings, urging a balanced approach to maximize benefits and minimize risks.



%%%%%
% Section 5: Discussion
%%%%%
\section{Discussion}

\subsection{Design Guidelines for Team-based AR-HMD in Acute Care Settings}

This study discovered new design factors regarding AR-HMDs for acute care settings, particularly in the emergency room.
These design concepts go beyond single-user AR-HMD interactions towards those that facilitate teamwork more effectively engendered by the dynamic nature of heterogeneous teams and formations.
Specifically, six out of ten participants were worried about information overload and distraction caused by irrelevant data. 
Participants suggested assigning a specific team member to use AR-HMDs at the bedside to mitigate this issue.
Our study provides AR-HMD design guidelines that benefit the human-computer interaction (HCI) community, focusing on use cases to improve team collaboration, communication, and decision-making. The AR-HMD design guidelines specifically observe how to 1) relay information for latecomers, 2) deal with inconsistent patient information, 3) receive guidance for rare procedures, 4) implement role-based AR-HMD applications for single and multi-users and the implications for team collaboration (see Table \ref{table:role_ar}) while addressing new challenges presented by AR-HMD use in high stakes ER procedures. 

\subsection{Towards Role-based AR-HMDs}

Based on our understanding of ER team-specific issues, we propose role-based AR-HMDs that enhance collaboration during complex procedures (see Table \ref{table:role_ar}). 
These \textbf{role-specific guidance} design concepts can provide on-the-fly patient care needs by enabling efficient task distribution and collaboration on procedures, all of which can help prevent individual and team burnout.
Furthermore, role-based AR-HMDs are tightly coupled with the user's expertise because HCWs are trained to perform specific medical tasks, and assigning a user a role outside their expertise often leads to patient care delays based on our findings.
Unlike prior work focusing on AR-HMD use cases of individual users, our work produces a set of recommended applications that prioritize team collaboration, focusing on unplanned, time-sensitive care of high-acuity patients.
These design concepts can potentially reduce the cognitive load and help prevent burnout of HCWs as it helps them focus on specific parts based on their roles. In the following, we describe proposed role-based AR-HMDs designed for a single user role, followed by an explanation of its adaptation for a subgroup responsible for carrying out specific collaborative activities.

\begin{figure*}[t]
    \centering
    \includegraphics[width=1.0\linewidth]{figures/design_recs.png}
    \caption{Role-specific AR-HMD applications for acute care teams.}
    \label{table:role_ar}
    %\vspace{-4mm}
\end{figure*}


\subsubsection{Team Leader} The team leader is responsible for coordinating multiple HCWs at the bedside to assign and allocate new tasks upon completion.
HCWs with different expertise can take on this responsibility ,including Physicians, Residents, Attendings, and Nurses who can run a code until the doctor arrives.
AR-HMDs use cases for the team leader include 1) recommending task allocations for medical procedures, 2) identifying HCWs joining patient care late, and 3) potentially communicating with remote guidance for rare procedures.
A unique benefit of AR-HMD for clinical teams is role-based displays with centralized information sharing and real-time communication with experts outside the patient’s room.
Previous research has shown the potential for remote guidance using virtual reality for a remote clinician to remote a local caregiver in patient homes for physical caregiving tasks \cite{dell2022designing}.
Sarcevic et al. \cite{sarcevic2011coordinating, mastrianni2023transitioning} displaying patient information on a shared display to help latecomers to a resuscitation.
Since the team leader is in charge of allocating tasks, an AR-HMD can support the leader by providing the user with relevant patient information and status.
Current training involves team leaders working with latecomers to bring them up to speed with relevant information, providing summaries throughout the procedure. This interaction can be enhanced by displaying patient information on a shared display, helping the latecomer stay informed while allowing the team leader to remain focused on the patient \cite{sarcevic2011coordinating, mastrianni2023transitioning}.
Further, if remote guidance is implemented in the healthcare departments, the team leader would be expected to communicate with the remote guidance for procedural support \cite{dell2022designing}.
Future research opportunities could also draw focus on the interface design for time-sensitive interactions, of which the recorder could oversee.




\subsubsection{Recorder} Creating annotations for the Recorder was revealed as a potentially useful AR-HMD application and has received significant attention in the design of cognitive aids \cite{grundgeiger2019cognitive, dell2022designing}.
If not managed appropriately, tracking ongoing time-sensitive procedures may result in information being missed in relay or quick notes on nearby items, such as napkins \cite{gonzales2016visual}.
% Added citaiton from DIS
To address this problem, AR-HMD annotations for clinical teams could include recording the start and end times of medical tasks for all team members \cite{kusunoki2015designing} and informing HCWs when they need to transition to a new task, which is often a nurse’s responsibility \cite{kusunoki2015sketching,dell2022designing,grundgeiger2019cognitive}.
Future research is needed to understand the level of autonomy required to update the state of user actions in the AR-HMDs to maintain up-to-date information.
A large body of work in computer vision has focused on ego-centric activity recognition, which could prove useful in clinical team environments in updating the state of user actions in AR-HMDs.
Kusunoki et al. \cite{kusunoki2015sketching} use co-design methods to develop a display of information that is typically communicated verbally during resuscitations.
This can ensure that AR-HMDs provide timely recommendations based on real-time interactions, which is especially important for recorders as the sole AR-HMD wearer.

 

\subsubsection{Calculating Dosages} Our findings show the potential for AR-HMD applications to calculate medication dosages for pharmacists and nurses.
Prior work shows potential in this direction using virtual tape to measure patients for medical dosage calculations and show patient vital signs, airway management, and resuscitation \cite{scquizzato2020smartphone}. 
AR-HMDs are currently capable of enabling virtual displays with buttons and menus to calculate medication dosage, which can hopefully reduce the cognitive load of mental math and dosage conversations. 
Users' inputs could include patient age, weight, allergies, and frequency of administrating dosages to calculate medication dosage with patient-specific features such as patient height estimates, vital signs, and other information.
Information sharing through virtual displays from multiple users could help save time without having to leave the room to retrieve data such as patient consent (see Figure \ref{fig:storyboard5}), lab results, X-ray scans, and others that inform decision-making.


\subsubsection{Cardiopulmonary Resuscitation (CPR)} Healthcare teams often perform several roles throughout a procedure dependent on their level of expertise and experience.
Our participants envisioned how AR-HMDs can support individuals performing multiple tasks, or multiple people performing the same task, such as providing CPR. 
AR-HMD for guidance during CPR was a potential application of interest to participants, providing virtual displays of the number of compressions, assessing compressions quality, recommending swapping team members, and suggesting speeding up or slowing down compressions.
This application is useful for Nurses primarily and could include Physicians, Attendings, or Residents. 
There has been increased research on AR-HMD design for CPR scenarios such as Holo-BLSD, which delivers early CPR and defibrillation and displays 3D virtual internal organs on patient manikins \cite{cheng2024use,ingrassia2020augmented,cheng2024use} and those that recommend compression start and stop times, and when to speed up and slow down \cite{kleinman2023pediatric}.
Opportunities arise here for clinical training and preparation, for while the devices can make recommendations, there is still a responsibility for the HCWs to use their professional judgment.


\subsubsection{Obtaining Medical Equipment} In our study, participants highlighted the importance of item search functionality for Physicians, Residents, Attendings, and Nurses working in cluttered acute-care settings. This is particularly crucial as Pharmacists can retrieve drugs but cannot administer them. 
For healthcare professionals who frequently work in new facilities, such as travel nurses or those working across multiple hospitals, searching for medical supplies in patient rooms and supply rooms can be a significant challenge. 
This is especially true in densely stocked areas where it may take time to find the necessary equipment. 
Research has shown that AR-HMDs have potential applications in location-based tasks, including locating nearby physical objects \cite{gruenefeld2019locating}, indoor navigation using visual markers \cite{kim2008vision}, and prototyping tangible AR-HMDs with everyday objects through interactive machine teaching \cite{monteiro2023teachable}. 
In future work, we plan to explore further the potential of AR-HMDs in item search and location-based tasks to support healthcare professionals in their daily workflows.


\subsubsection{Roles in Airway Management:} Airway management is a crucial medical procedure that requires precision and skill. 
If performed incorrectly, it can lead to life-threatening conditions or a lack of oxygen. In our study, participants highlighted airway management as another role-specific use case for AR-HMDs. 
This application would be useful for various healthcare professionals, including Physicians, Residents, Attendings, Nurses, and Respiratory Therapists. 
The airway management AR-HMD could provide two key features: 1) identify the location of airway supplies and 2) visualize procedural steps for managing the airway. 
While progress has been made in developing airway management AR-HMDs \cite{scquizzato2020smartphone,bhavsar2023253,falk2023build,qian2019augmented}, there is a need to extend this technology to acute care settings. 
To optimize training, we propose the use of AR-HMDs for on-the-job training in airway management instead of relying on human resources. 
This would allow healthcare professionals to focus on other tasks while the AR-HMD guides them through the procedure, providing real-time access to relevant equipment and steps.



\subsection{New Challenges Introduced by AR-HMD in Safety-Critical Environments}


\subsubsection{Informational Overload} AR-HMDs have been touted to revolutionize clinical workflows by reducing errors, decreasing cognitive workload, and enhancing performance \cite{blattgerste2017comparing, velamkayala2017effects}. 
However, our study reveals participants' concerns about the potential drawbacks. 
In practice, this represents a new challenge due to staff shortages in hospitals, particularly after the COVID-19 pandemic \cite{staffshortage}. 
Also, a review of 20 studies \cite{romare2020smart} found that HCWs using AR-HMDs in complex care environments detected abnormal vital signs faster but felt distracted and spent longer time completing tasks. 
Furthermore, design requirements for alert systems for trauma teams in the human-computer interaction (HCI) field revealed useful insights that could prove useful for AR-HMD design such as shared displays of task status and upcoming tasks and integration of shared information with other supplies, materials, or artifacts \cite{zellner2023understanding,zhang2018coordination}.
This highlights the importance of designing effective information management functions for multi-user interactions to maximize team collaboration and minimize burnout. 
As AR-HMD design continues to evolve to offer a wide range of information, it is crucial to implement features that prioritize user experience, such as attention theory \cite{bonanni2005attention}, which is filtering and prioritizing information to prevent cognitive overload. 
With hardware advancements moving rapidly, AR designers must ensure that software adaptations prioritize user experience to maximize the benefits of these innovative tools.

 

\subsubsection{Healthcare Workers' Unfamiliarity with AR-HMDs} One significant challenge identified in this study is HCWs unfamiliarity with AR-HMDs. 
The introduction of new technology to established work practices can evoke negative emotions and behavioral resistance to adopting the technology in their workflows \cite{shrestha2023understanding, nilsen2016exploring, lin2012barriers}. 
Participants in this study described AR-HMDs as "foreign" and expressed concerns that many HCWs might be hesitant to adopt the technology. 
Indeed, the use of wearable devices like AR-HMDs requires consideration not only of improved work efficiency but also introduces new challenges for physical comfort \cite{yoshida2018plastic}, psychological acceptance \cite{perannagari2020factors}, and social norms \cite{rauschnabel2016augmented}. 
For instance, participants reported feeling uneasy about trusting critical tasks to AR-HMDs. 
Additionally, wearing an AR-HMD can make everyday interactions like speaking and making eye contact with others feel unnatural and awkward. 
These novel experiences, although subtle, are significant enough to be a major motivator for avoiding the use of AR-HMDs. 
Therefore, successful application of AR-HMDs hinges on technological advancements, user experience design, and embracing the "foreignness" of these new tools.



\subsubsection{AR-HMD Development Limitations: A Clinical Teamwork Perspective} When developing AR-HMDs for clinical teams, it is essential to consider that user interactions with virtual objects should be more situational and context-dependent. 
Advances in developer tools have made it more feasible to create multi-user applications. Moreover, future developments may enable the creation of multi-user AR-HMD applications using low-code or no-code systems, making research tools more accessible. 
However, while it would be ideal for multiple users to interact within the same virtual space, current development software limitations must be addressed first. 
For instance, spatial markers can anchor objects and allow multiple users to interact with them, but there are still open questions surrounding single user interactions, such as: \textit{how do we ensure accurate object placement and interaction?} \cite{warden2022visual, plabst2023exploring} 
\textit{What is the impact of incorrect machine learning algorithm predictions on user interactions (automation bias)?} \cite{raikwar2024beyond} 
\textit{How can AR-HMDs effectively handle clutter?} \cite{warden2024quantitative}
Given these challenges, designing AR-HMDs for multi-users will require additional considerations and efforts.


\subsection{Limitations and Future Work}

Despite our best efforts, this study has some limitations. Firstly, our findings may not apply to acute care settings outside of the US settings due to culturally different team dynamics and technological infrastructure. 
Secondly, as nurses play a crucial role in patient advocacy, we aimed to recruit more nurses for the study. 
However, we faced challenges in recruiting additional nurses, which is unsurprising given the high levels of burnout and workload among many nurses. Considering the large number of participants in this study who usually play a leading role in healthcare teams, others in supporting roles might envision AR-HMD technologies differently.
Third, some participants were unfamiliar with AR-HMD technologies, which presented obstacles for them in envisioning how these technologies could support their work. 

To overcome the limitations identified in our study, we plan to continue our research in the following ways: Firstly, we will conduct a co-design process in virtual reality to prototype AR-HMD interactions that immerse users in the design process. 
This approach is inspired by design concepts such as speculative design \cite{mitchell2020no}, future-proofing \cite{cristian2018hospital}, design fiction \cite{noortman2019hawkeye}, and co-creation. 
Secondly, we intend to develop AR-HMDs based on our co-design processes and evaluate their effectiveness in user studies to assess how they can improve teamwork during patient care simulations. 
Furthermore, while our study focused on identifying the types of information that should be displayed to support teamwork, future efforts will address how this information should be presented to HCWs in acute care settings. 
Specifically, conceptualizing AR-HMDs as a series of micro-interactions is a powerful tool for improving user-centered design where users take subtle actions that the headset can interpret to display relevant information; future work will focus on this direction.
This will involve exploring different display methods and evaluating their impact on team performance and user experience. 
Ultimately, we hope that our work inspires others to engage in AR design for users in safety-critical settings, involving stakeholders throughout the development process to create well-contextualized AR-HMDs that improve teamwork.

\bibliographystyle{ACM-Reference-Format}
\bibliography{references}

\end{document}
\endinput
%%
%% End of file `sample-manuscript.tex'.