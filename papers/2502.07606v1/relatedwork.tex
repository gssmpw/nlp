\section{Related Work}
The starting point for our work is the recent series of papers by Chriss introducing and analyzing a game-theoretic trading model~\citep{chriss2024optimal,chriss2024positionbuildingcompetitionrealworldconstraints,chriss2024competitiveequilibriatrading, chriss2025positionbuildingcompetitiongame}. Chriss begins by establishing the existence of Nash equilibria --- since he works in a continuous time and volume model, the pure strategy spaces are infinite, and thus existence does not immediately follow from Nash's celebrated theorem. Chriss imposes continuity and boundary conditions on strategies, which together allow him to prove existence. He then proceeds to examine equilibrium structure and to consider a number of variants of the model. 
Chriss' model is related to earlier work on optimal trade execution in a non-strategic setting~\citep{AlmgrenPortfolio}.
Here we consider a discrete time and volume version of Chriss' model, for which the pure strategy spaces are finite (though exponential in the time horizon) and thus mixed NE are guaranteed to exist. Since in reality trading must occur in discrete time steps and in whole shares, moving to a discrete model allows us to consider algorithmic issues more precisely, which is our primary interest.

Key to Chriss' model are standard notions of (temporary and permanent) market impact, on which there is a large literature;
see~\citep{Gatheral3Models,GatheralMI,Hautsch,Zarinelli,bouchaud,Webster} for a representative but very partial sample. Broadly speaking, this literature considers various models for how trading activity influences asset prices, implications of those models for trading strategies, and empirical validation. There is also a smaller body of work considering the algorithmic aspects of optimal trading, including machine learning approaches~\citep{evendarLimit,GanchevDark,nevmyvakaRL,kakadeVWAP}. Our work is also focused on algorithmic considerations, but in a game-theoretic setting.