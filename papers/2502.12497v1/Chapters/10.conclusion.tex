\section{Conclusion}
In this paper, we conduct the first systematic study of vulnerabilities in the LLM supply chain, analyzing 529 vulnerabilities reported across 75 prominent LLM projects, spanning 13 key lifecycle stages. It reveals critical insights into the nature and distribution of these vulnerabilities. By providing a root cause taxonomy and analyzing fix patterns, this study offers actionable insights to address vulnerabilities in the LLM ecosystem. Our study highlights the urgent need to address the challenges of securing the LLM supply chain and calls for focused efforts in future research.

\section*{Open Science}
To promote transparency and reproducibility, we will make the data and resources used in this study publicly available upon acceptance. The dataset of 529 vulnerabilities, including detailed annotations and taxonomy, is accessible through an open repository, ensuring that researchers can validate and build upon our findings. Additionally, we provide scripts and tools for reproducing the analysis, including the methodologies used for root cause identification and fix pattern evaluation. Where applicable, proprietary or sensitive data has been anonymized or excluded to comply with ethical and legal obligations. We encourage the research community to utilize and extend these resources to advance the understanding and mitigation of vulnerabilities in the LLM ecosystem.

\section*{Ethics Considerations}
This study adheres to strict ethical guidelines to ensure the responsible handling of data and analysis. All vulnerabilities analyzed in this study are sourced from publicly available reports or repositories, and no proprietary or confidential information has been included. When discussing vulnerabilities, we avoid exposing detailed exploit paths to minimize the risk of misuse. Furthermore, the study is conducted with the goal of improving security and mitigating risks, aligning with ethical principles that prioritize the well-being of end users and the integrity of LLM systems. We also ensure that our findings and resources are shared in a way that fosters collaboration while safeguarding against malicious applications.