\section{RQ3: Fix Patterns}
To answer RQ3, we systematically analyzed the available patches and the associated discussions in developer comments to understand the challenges in fixing vulnerabilities within LLM systems. Among the 529 identified vulnerabilities, 300 (56.7\%) had available fixes, which became the focus of our analysis. We examined these fixes to evaluate their patterns and effectiveness, specifically whether the patches introduced any side effects or left the vulnerabilities susceptible to bypasses. Our approach involved reviewing the fix commits, auditing the code changes, and identifying the specific lines of code that were modified or added in each patch. Additionally, we analyzed relevant information from bug hunter reports and the discussions between bug hunters, platform administrators, and project maintainers. These discussions often provided valuable insights into recurring issues and potential risks associated with the proposed fixes, offering a deeper understanding of the challenges.

\begin{table*}[t]
\centering
\caption{Overview of Ineffective Fixes and Associated CVEs}
\begin{tabular}{llccc}
\toprule
\textbf{Ineffective Fix} & \textbf{CWE} & \textbf{Root CVE} & \textbf{Caused CVE} & \textbf{Count (\%)} \\ 
\midrule
\multirow{5}{*}{IF1: Path Traversal} 
& CWE-29  & 4 & 4 & \multirow{5}{*}{27 / 46.6\%} \\
& CWE-23  & 4 & 7 &  \\
& CWE-73  & 1 & 2 &  \\
& CWE-36  & 1 & 1 &  \\
& CWE-22  & 1 & 2 &  \\
\midrule
\multirow{4}{*}{IF2: Improper Neutralization (``Injection'')} 
& CWE-1336 & 1 & 1 & \multirow{4}{*}{15 / 25.9\%} \\
& CWE-79  & 1 & 1 &  \\
& CWE-94  & 3 & 3 &  \\
& CWE-1426 & 1 & 4 &  \\
\midrule
\multirow{4}{*}{IF3: Externally Controlled Resource Reference} 
& CWE-942 & 1 & 1 & \multirow{4}{*}{10 / 17.2\%} \\
& CWE-918 & 1 & 2 &  \\
& CWE-601 & 1 & 2 &  \\
& CWE-15  & 1 & 1 &  \\
\midrule
\multirow{2}{*}{IF4: Improper Authorization}
& CWE-178 & 1 & 1 & \multirow{2}{*}{4 / 6.9\%} \\
& CWE-306 & 1 & 1 &  \\
\midrule
\multirow{1}{*}{Others} 
& CWE-754 & 1 & 1 & 2 / 3.4\% \\
\midrule
\textbf{Total} & / & \textbf{24} & \textbf{34} & \textbf{58 / 100\%} \\
\bottomrule
\end{tabular}
\label{table:ineffective_fixes}
\end{table*}

\noindent \textbf{Overview of Recurring Vulnerabilities.} 
As illustrated in \autoref{table:ineffective_fixes}, our analysis revealed that a total of 58 vulnerabilities were associated with ineffective fix patterns. Of these, 24 were root issues with ineffective original fixes. These ineffective fixes directly led to the recurrence of 34 vulnerabilities, demonstrating the significant impact of flawed fixes on the persistence and reappearance of vulnerabilities in the system. To better understand the nature of these root issues, we categorized the vulnerabilities based on their types and the reasons for ineffective fixes. Among the identified patterns, we focus on two major categories: path traversal and injection, as they represent the most critical and recurring issues.

\noindent \textbf{IF1: Path Traversal.}  
Effective fixes for path traversal vulnerabilities must account for all possible attack vectors, including symbolic link attacks, variations in file path encoding, and improper handling of dynamic paths. Robust solutions involve normalizing file paths to their canonical form, strictly validating paths against a whitelist of allowed directories, and avoiding reliance on user-controlled inputs for critical path decisions. However, ineffective fixes often fail to account for complex attack vectors. For example, \texttt{CVE-2023-6831}\footnote{https://huntr.com/bounties/0acdd745-0167-4912-9d5c-02035fe5b314} in the \texttt{MLflow} framework demonstrates how improper path normalization can result in artifact deletion outside intended directories. The vulnerability exploited URL encoding (e.g., ``\texttt{\%2E\%2E}'' for ``\texttt{../}'') to bypass checks in the \texttt{validate\_path\_is\_safe} function, enabling attackers to delete arbitrary files. A subsequent patch aimed to address this issue, yet \texttt{CVE-2024-1560}\footnote{https://huntr.com/bounties/4a34259c-3c8f-4872-b178-f27fbc876b98} revealed further weaknesses. The updated implementation still performed excessive normalization, including double-decoding and mishandling of special characters like tabs or newlines, which allowed attackers to craft payloads (e.g., ``\texttt{\%2\%0952e}'') to traverse directories and delete critical files on the server.

\noindent \textbf{IF2: Injection.}  
Effective fixes for injection flaws require robust input and output sanitization, strict execution policies, and comprehensive validation of all execution contexts. However, insufficient fixes often fail to address the full range of exploit vectors, leaving systems vulnerable. For example, \texttt{CVE-2023-39662}\footnote{https://github.com/run-llama/llama\_index/issues/7054} demonstrates a RCE vulnerability in the \texttt{PandasQueryEngine} module of the \texttt{llama\_index} library. The issue stems from the unsafe use of the \texttt{exec} function to dynamically execute Python code generated in response to user prompts. Attackers could exploit this vulnerability by crafting malicious prompts, such as injecting system commands to create unauthorized files. Although a patch\footnote{https://github.com/run-llama/llama\_index/pull/8890} was introduced to restrict the execution of code containing specific patterns (e.g., underscores in variable names), this solution was incomplete and failed to prevent alternative forms of injection.
This insufficiency led to the discovery of \texttt{CVE-2024-3271}\footnote{https://github.com/run-llama/llama\_index/issues/10439}, which exposed a bypass of the initial patch. Attackers could exploit the system by crafting prompts without using underscores, effectively bypassing the validation mechanism and achieving arbitrary code execution.

% \noindent \textbf{IF3: Externally Controlled Resource Reference.}  
% Ineffective fixes in this category typically occur when external resources, such as files, URLs, or services, are referenced in a way that allows an attacker to control these references. For instance, CWE-942 (Externally Controlled Reference to a Resource in Another Sphere) and CWE-918 (Server-Side Request Forgery) can lead to vulnerabilities if the application fails to properly validate or sanitize external inputs before using them in sensitive operations. Even after patches are applied, attackers can still exploit these vulnerabilities by manipulating references to external resources. In LLM systems, this issue is more complex due to the involvement of external models or data sources, where improper control of these references could lead to security breaches. A common example is the use of user-controlled URLs in server-side code, where improper validation allows attackers to access internal services or sensitive data, resulting in a persistent vulnerability despite a fix.

% \noindent \textbf{IF4: Improper Authorization.}  
% Authorization issues arise when systems fail to properly validate whether users have the necessary permissions to access certain resources or perform specific actions. In the case of CWE-178 (Improper Authorization) and CWE-306 (Missing Authentication for Critical Function), ineffective fixes often stem from incomplete access control enforcement. For example, a patch might resolve a direct access issue but fails to address the underlying configuration or mismanagement of access roles. In LLM systems, this problem is exacerbated by the dynamic and collaborative nature of model development, where multiple models or actors may have conflicting access permissions. In some cases, fixes may be overly simplistic, allowing attackers to bypass restrictions through indirect means, such as manipulating session tokens or exploiting weak privilege separation.

\begin{tcolorbox}
\textbf{Findings.} Among the 529 identified vulnerabilities, 300 (56.7\%) of them had fixes available. However, 24 (8\%) of these fixes were ineffective, leading to the recurrence of 34 vulnerabilities. 
\end{tcolorbox}