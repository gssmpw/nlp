\section{Related Work}
There have been many synthetic FV datasets constructed from Wikipedia, such as FEVER \cite{thorne-etal-2018-fever}. While FEVER focuses on simple claims, datasets like 
HOVER \cite{jiang-etal-2020-hover} and 
FEVEROUS \cite{aly-etal-2021-fact} introduced complex claims requiring multi-hop reasoning. Apart from synthetic datasets, there are also datasets focusing on more realistic claims and real-world misinformation \cite{schlichtkrull2023averitec, glockner-etal-2024-ambifc}. Increasingly popular are also domain-specific datasets focusing on scientific fact-checking \cite{vladika-matthes-2023-scientific}, especially for the domains of medicine \cite{saakyan-etal-2021-covid, sarrouti2021evidence}, climate \cite{diggelmann2020climatefever}, and computer science \cite{lu-etal-2023-scitab}. 

Most FV approaches follow the traditional three-part pipeline \cite{bekoulis2021review}. In recent years, approaches incorporating LLMs and iterative reasoning into the process have achieved great performance on multi-hop FV. This includes FV through varifocal questions \cite{ousidhoum-etal-2022-varifocal} or \textit{wh}-questions to aid verification \cite{rani-etal-2023-factify}, step-by-step prompting \cite{zhang-gao-2023-towards}, and program-guided reasoning \cite{pan-etal-2023-fact}.
%, and FOLK \cite{wang-shu-2023-explainable} have shown impressive performance on multi-hop FV by incorporating LLMs and iterative reasoning into the process. 

Most studies with iterative FV systems focus on multi-hop encyclopedic claims. To the best of our knowledge, our study is among the first to explore the step-by-step FV systems for real-world claims rooted in scientific and medical knowledge.