\begin{table}[h]
\begin{center} 
\setlength{\tabcolsep}{0.75mm} %
\begin{tabular}{l|cc|cc}
\toprule
\multirow{3}{*}[0.8ex]{Method} & \multicolumn{2}{c|}{Image Quality} & \multicolumn{2}{c}{Text Align} \\
\cmidrule(lr){2-5} & HPS $_{\times 10^2}\uparrow$ & AS $\uparrow$ & CLIP Sim $^*\downarrow$ & IR$^*_{\times10}\downarrow$\\
\midrule
KV Edit (ours) & 26.76& \textbf{6.49} & 25.50& 6.87\\
+NS & \textbf{26.93}& 6.37& 25.05& 3.17\\
+NS+AM & 26.72& 6.35& 25.00& 2.55\\
+NS+RI & 26.73& 6.34& \textbf{24.82}& \textbf{0.22}\\
+NS+AM+RI & 26.51& 6.28& 24.90& 0.90\\
\bottomrule
\end{tabular}
\caption{\textbf{Ablation study for object removal task.} CLIP Sim$^*$ and IR$^*$ represent alignment between source prompt and new image through CLIP~\cite{radford2021learning} and Image Reward~\cite{xu2023imagereward} to evaluate whether remove particular object from image. \textbf{NS} indicates there is no skip step during inversion. \textbf{RI} indicates the addition of reinitialization strategy. \textbf{AM} indicates that using attention mask during inversion.} 
\label{tab:ablation} 
\end{center}
\vspace{-20pt}
\end{table}
