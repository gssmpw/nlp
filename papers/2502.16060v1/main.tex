%%
%% This is file `sample-sigconf.tex',
%% generated with the docstrip utility.
%%
%% The original source files were:
%%
%% samples.dtx  (with options: `sigconf')
%% 
%% IMPORTANT NOTICE:
%% 
%% For the copyright see the source file.
%% 
%% Any modified versions of this file must be renamed
%% with new filenames distinct from sample-sigconf.tex.
%% 
%% For distribution of the original source see the terms
%% for copying and modification in the file samples.dtx.
%% 
%% This generated file may be distributed as long as the
%% original source files, as listed above, are part of the
%% same distribution. (The sources need not necessarily be
%% in the same archive or directory.)
%%
%%
%% Commands for TeXCount
%TC:macro \cite [option:text,text]
%TC:macro \citep [option:text,text]
%TC:macro \citet [option:text,text]
%TC:envir table 0 1
%TC:envir table* 0 1
%TC:envir tabular [ignore] word
%TC:envir displaymath 0 word
%TC:envir math 0 word
%TC:envir comment 0 0
%%
%%
%% The first command in your LaTeX source must be the \documentclass command.
% \documentclass[sigconf,anonymous]{acmart}
% \documentclass[sigconf,review]{acmart}
\documentclass[sigconf]{acmart}
% \documentclass[sigconf,anonymous,review]{acmart} %% submitted version
\usepackage{placeins}
\usepackage{thm-restate}
% \usepackage{enumitem}

% \documentclass[sigconf,review,anonymous]{acmart}
% my packs&defs
% \usepackage{amsfonts}
% \usepackage{times}  % DO NOT CHANGE THIS
% \usepackage{helvet} % DO NOT CHANGE THIS
% \usepackage{courier}  % DO NOT CHANGE THIS
% \usepackage[hyphens]{url}  % DO NOT CHANGE THIS
% \usepackage{graphicx} % DO NOT CHANGE THIS

% \usepackage{psfig}
\usepackage{color}
\usepackage{hyperref}
\hypersetup{colorlinks=true,linkcolor=black,urlcolor=black,citecolor=black}
\def\equationautorefname~#1\null{Equation~(#1)\null}
\usepackage{breakurl}
\usepackage{bm}
\usepackage{soul}
\usepackage{xspace}
\usepackage{algorithmic}
\usepackage{algorithm}
\usepackage{mdwlist}    % makes lists tighter
%\DeclareMathAlphabet{\mathpzc}{OT1}{pzc}{m}{it}
\usepackage{paralist} % better than mdwlist, for tighter lists
%\usepackage{flushend} % format, reference

%--- rotated table ---%
\usepackage{adjustbox}
\usepackage{array}
\usepackage{booktabs}
\usepackage{multirow}
%--- rotated table ---%
\usepackage{enumitem}
\usepackage{subcaption}

\usepackage{array,graphicx}
\usepackage{booktabs}
% \usepackage{pifont}
\usepackage{color}
%\usepackage[usenames,dvipsnames,table]{xcolor}

\usepackage{amsmath}
%--- table ---%
%\usepackage[table]{xcolor}
\definecolor{lightgray}{gray}{0.85}
\usepackage{multirow} 
%--- table ---%

\newcommand{\nb}[2]{
	{
		{\color{black}{
				\fbox{\bfseries\sffamily\scriptsize#1}
				{\sffamily$\triangleright~${\it\sffamily #2}$~\triangleleft$}
	}}}
}

\newcommand\Chen[1]{\nb{Chen}{\color{magenta}#1}}
\def\BibTeX{{\rm B\kern-.05em{\sc i\kern-.025em b}\kern-.08em
    T\kern-.1667em\lower.7ex\hbox{E}\kern-.125emX}}

\newcommand\Kimura[1]{\nb{Kimura}{\color{blue}#1}}
\def\BibTeX{{\rm B\kern-.05em{\sc i\kern-.025em b}\kern-.08em
    T\kern-.1667em\lower.7ex\hbox{E}\kern-.125emX}}
%
% --- inline annotations
%
\newcommand{\red}[1]{{\color{red}#1}}
\newcommand{\todo}[1]{{\color{red}#1}}
\newcommand{\TODO}[1]{\textbf{\color{red}[TODO: #1]}}
% --- disable by uncommenting  
% \renewcommand{\TODO}[1]{}
% \renewcommand{\todo}[1]{#1}

\usepackage{times}
\usepackage{epsfig}
\usepackage{graphicx}
\usepackage{amssymb}
\usepackage{booktabs}
\usepackage{multirow}
\usepackage{float}
\usepackage{multicol}
\usepackage{tikz}
\usepackage{comment}
\usepackage{wrapfig}
\usepackage{graphicx}
\usepackage[nopar]{lipsum}
\usepackage{stfloats}
\usepackage{amsmath,amssymb} % define this before the line numbering.
\usepackage{amsfonts}
\usepackage{siunitx}

\usepackage{color}
\usepackage[hypcap=true]{subcaption}
\usepackage[misc]{ifsym}

\usepackage{array}     % For tabular formatting and \arraystretch
\usepackage{stfloats}
\usepackage{etoolbox}
\usepackage{tocloft}
\usepackage{setspace}

\usepackage{tabularx}  % Optional, for advanced table control

% \usepackage{orcidlink}
\usepackage{xcolor}
\usepackage{paralist}
% \usepackage{xltxtra}
\usepackage[capitalize]{cleveref}
% Support for easy cross-referencing
\usepackage{afterpage}
\usepackage{stfloats}
\usepackage{lineno}


\crefname{section}{Sec.}{Secs.}
\Crefname{section}{Section}{Sections}
\Crefname{table}{Table}{Tables}
\crefname{table}{Tab.}{Tabs.}

\newcommand{\proj}[1]{\operatorname{proj}(#1)}
\newcommand{\homo}[1]{\operatorname{homo}(#1)}
\newcommand{\interp}[1]{\operatorname{interp}(#1)}
\newcommand{\fpp}[2]{\frac{\partial #1}{\partial #2}}

%%
%% \BibTeX command to typeset BibTeX logo in the docs
\AtBeginDocument{%
  \providecommand\BibTeX{{%
    \normalfont B\kern-0.5em{\scshape i\kern-0.25em b}\kern-0.8em\TeX}}}

%% Rights management information.  This information is sent to you
%% when you complete the rights form.  These commands have SAMPLE
%% values in them; it is your responsibility as an author to replace
%% the commands and values with those provided to you when you
%% complete the rights form.
% \setcopyright{acmcopyright}
% \copyrightyear{2022}
% \acmYear{2022}
% % \acmDOI{10.1145/1122445.1122456}

% %% These commands are for a PROCEEDINGS abstract or paper.
% \acmConference[KDD '22]
% {The 28th ACM SIGKDD Conference on Knowledge Discovery and Data Mining}
% {August 14--18, 2022}
% {Washington, DC, USA}
% \acmBooktitle{Proceedings of the 28th ACM SIGKDD Conference on Knowledge Discovery and Data Mining, August 14--18, 2022, Washington, DC, USA}
% \acmPrice{15.00}
% % \acmISBN{978-1-4503-XXXX-X/18/06}

%%%%%%%%%%% Commented these but was used in submission %%%%%%%%%%%%%
\copyrightyear{2025}
\acmYear{2025}
\setcopyright{acmcopyright}
\acmConference[Conference '25]{2025}{}
%%%%%%%%%%%%%%%%%%%%%%%%%%%%%%%%%%%%%%%%%%%%%%%%

%%
%% Submission ID.
%% Use this when submitting an article to a sponsored event. You'll
%% receive a unique submission ID from the organizers
%% of the event, and this ID should be used as the parameter to this command.
%%\acmSubmissionID{123-A56-BU3}

%%
%% The majority of ACM publications use numbered citations and
%% references.  The command \citestyle{authoryear} switches to the
%% "author year" style.
%%
%% If you are preparing content for an event
%% sponsored by ACM SIGGRAPH, you must use the "author year" style of
%% citations and references.
%% Uncommenting
%% the next command will enable that style.
%%\citestyle{acmauthoryear}

% uncomment, to make a clean = final, draft
%\mkclean
% \tensakuClean

\settopmatter{printacmref=False} %%%%%%%% set this to true during submission
\settopmatter{authorsperrow=4}



%%
%% end of the preamble, start of the body of the document source.
\begin{document}

% \title[Tokenizing Single-Channel EEG with Time-Frequency Motif Learning]
% {Tokenizing Single-Channel EEG with Time-Frequency \\Motif Learning}
% arxiv title
\title[Single-Channel EEG Tokenization Through Time-Frequency Modeling]
{Single-Channel EEG Tokenization Through Time-Frequency \\Modeling}
% \title[Latent Discretized Brain-Signal Tokenization Through Time-Frequency Modeling]
% {Latent Discretized Brain-Signal Tokenization Through \\Time-Frequency Modeling}
 
% Title ideas:
% Tokenizing EEGs with Single Channel Time-Frequency Motif Learning

% \author{Anonymous authors}
\author{Jathurshan Pradeepkumar}
% \authornotemark[1]
\affiliation{
    \institution{Department of Computer Science, University of Illinois Urbana-Champaign}
    \city{Urbana, IL}
    \country{USA}}
 \email{jp65@illinois.edu}

\author{Xihao Piao}
% \authornotemark[2]
\affiliation{
    \institution{SANKEN, Osaka University}
    \city{Osaka}
    \country{Japan}
}
\email{park88@sanken.osaka-u.ac.jp}


\author{Zheng Chen}
% \authornotemark[2]
\affiliation{
    \institution{SANKEN, Osaka University}
    \city{Osaka}
    \country{Japan}
}
\email{chenz@sanken.osaka-u.ac.jp}
% % %
% %
 \author{Jimeng Sun}
% \authornotemark[1]
\affiliation{
    \institution{Department of Computer Science, University of Illinois Urbana-Champaign}
    \city{Urbana, IL}
    \country{USA}}
 \email{jimeng@illinois.edu}

%%
%% By default, the full list of authors will be used in the page
%% headers. Often, this list is too long, and will overlap
%% other information printed in the page headers. This command allows
%% the author to define a more concise list
%% of authors' names for this purpose.
% \renewcommand{\shortauthors}{Trovato and Tobin, et al.}
%\renewcommand{\shortauthors}{Rikuto Kotoge et al.}

%%
%% The abstract is a short summary of the work to be presented in the
%% article.
\begin{abstract}
  
Hypotheses are central to information acquisition, decision-making, and discovery. However, many real-world hypotheses are abstract, high-level statements that are difficult to validate directly. 
This challenge is further intensified by the rise of hypothesis generation from Large Language Models (LLMs), which are prone to hallucination and produce hypotheses in volumes that make manual validation impractical. Here we propose \mname, an agentic framework for rigorous automated validation of free-form hypotheses. 
Guided by Karl Popper's principle of falsification, \mname validates a hypothesis using LLM agents that design and execute falsification experiments targeting its measurable implications. A novel sequential testing framework ensures strict Type-I error control while actively gathering evidence from diverse observations, whether drawn from existing data or newly conducted procedures.
We demonstrate \mname on six domains including biology, economics, and sociology. \mname delivers robust error control, high power, and scalability. Furthermore, compared to human scientists, \mname achieved comparable performance in validating complex biological hypotheses while reducing time by 10 folds, providing a scalable, rigorous solution for hypothesis validation. \mname is freely available at \url{https://github.com/snap-stanford/POPPER}.




\end{abstract}

%%
%% The code below is generated by the tool at http://dl.acm.org/ccs.cfm.
%% Please copy and paste the code instead of the example below.
%%
\begin{CCSXML}
<ccs2012>
   <concept>
       <concept_id>10010147.10010257.10010293</concept_id>
       <concept_desc>Computing methodologies~Machine learning approaches</concept_desc>
       <concept_significance>500</concept_significance>
       </concept>
   <concept>
       <concept_id>10010405.10010444.10010447</concept_id>
       <concept_desc>Applied computing~Health care information systems</concept_desc>
       <concept_significance>500</concept_significance>
       </concept>
       <concept_id>10002951.10002952</concept_id>
       <concept_desc>Information systems~Data management systems</concept_desc>
       <concept_significance>500</concept_significance>
       </concept>
   <concept>
 </ccs2012>
\end{CCSXML}

\ccsdesc[500]{Computing methodologies~Machine learning approaches}
\ccsdesc[500]{Applied computing~Health care information systems}
%%
%% Keywords. The author(s) should pick words that accurately describe
%% the work being presented. Separate the keywords with commas.
\keywords{EEG, Tokenization, Deep learning}
% }

%% A "teaser" image appears between the author and affiliation
%% information and the body of the document, and typically spans the
%% page.
% \begin{teaserfigure}
%   \includegraphics[width=\textwidth]{sampleteaser}
%   \caption{Seattle Mariners at Spring Training, 2010.}
%   \Description{Enjoying the baseball game from the third-base
%   seats. Ichiro Suzuki preparing to bat.}
%   \label{fig:teaser}
% \end{teaserfigure}

%%
%% This command processes the author and affiliation and title
%% information and builds the first part of the formatted document.
\maketitle

\section{Introduction}
    \label{sec:intro}
    \section{Introduction}
% Large Language Models (LLMs) serve as the foundation for a wide range of tasks. 
% Recently researchers have developed methods to equip large language models (LLMs) with external tools, 

Tool-augmented Large Language Models (LLMs) can use external tools such as calculators~\citep{schick2023toolformer}, 
Python interpretors~\citep{pal}, 
APIs~\citep{tang2023toolalpaca}, or 
AI models~\citep{patil2023gorilla} to complement the parametric knowledge of vanilla LLMs and enable them to solve more complex tasks~\citep{schick2023toolformer,patil2023gorilla}. They are often trained on query-response pairs, which embed the ability to use tools {\em directly} into parameters.\looseness-1
% For example, WebGPT~\citep{webgpt} extends GPT-3~\citep{gpt3} to use search engines, especially useful for events that occurred {\em after} GPT-3 was trained.
% and retrieve up-to-date information to improve GPT-3's performance in question answering, 


Despite the growing adoption of tool-augmented LLMs, the ability to selectively unlearn tools has not been investigated. In real-world applications, tool unlearning is essential for addressing critical concerns such as security, privacy, and model reliability. 
For example, consider a tool-augmented LLM deployed in a healthcare system and trained to use APIs for handling patient data. If one of the APIs is later flagged as insecure due to a vulnerability that could expose sensitive information and violate regulations like HIPAA, tool unlearning is necessary to ensure that the LLM can no longer invoke the insecure API. Similarly, when tools undergo major updates, such as the Python transformers package moving from version 3 to version 4, tool unlearning becomes essential to prevent the LLM from generating outdated or erroneous code.
% For example, if a tool-augmented LLM retains knowledge of making insecure HTTP requests, it will cause significant security risks and can become vulnerable to attacks.\footnote{\url{https://datatracker.ietf.org/doc/html/rfc7807}}
The goal of this work is to address this gap by investigating tool unlearning and providing a solution for this overlooked yet essential task.

% Additional scenarios are discussed in \S~\ref{sec:app}. In the aforementioned case, it is necessary for a tool-augmented model to forget its acquired knowledge of using certain tools--an area that has not yet been explored by existing research.
% Consider the following practical scenarios: 1) \emph{Insecure Tools}, where non-trustworthy tools need to be deleted, 2) \emph{Restricted Tools}, where tools may become unavailable due to copyright issues; 3) \emph{Broken Tools/Dependencies}, where tools may become broken, deprecated, or fall out of maintenance; 4) \emph{Unnecessary Tools}, where the requirement for certain tools may no longer be needed; and 5) \emph{Limited Model Capacity}, where the tool-augmented LLM meets capacity limitations. 


% \paragraph{A new task}
We introduce and formalize the new task of \textbf{Tool Unlearning}, which aims to remove the ability of using specific tools from a tool-augmented LLM while preserving its ability to use other tools and perform general tasks of LLMs such as coherent text generation. 
% This is essential for complying with tool deletion requests that often target a small subset of tool. 
Ideally, an effective tool unlearning model should behave as if it had never learned the tools marked for unlearning. 
% When tool deletion requests are received, a successful tool unlearning algorithm should effectively remove knowledge of the targeted tools, as if the model had never encountered them. At the same time, the model’s knowledge of remaining tools and its ability to perform other tasks should be preserved to the greatest extent possible. This is crucial because deletion requests typically focus on a specific subset of tools, which is usually much smaller than the entire tool set.
% \paragraph{Difference to sample unlearning}
Tool unlearning fundamentally differs from traditional sample-level unlearning as it focuses on removing ``skills'' or the ability to use specific tools, rather than removing individual data samples from a model. In addition, success in tool unlearning should be measured by the model’s ability to forget or retain tool-related skills, which differs from traditional metrics such as measuring likelihood of extracting training data in sample-level unlearning.
% While sample-level unlearning focuses on reducing the likelihood of extracting training data, tool unlearning aims to forget the capability to solve tasks that rely on the tools tagged for unlearning, which can be seen as knowledge-level unlearning. (2) Evaluation: Sample unlearning typically uses perplexity or extraction probability as evaluation metrics. In contrast, tool unlearning prioritizes the success rate of using specific tools, ensuring that the model can no longer effectively use the tools targeted for unlearning. (3) Data: Sample unlearning typically requires access to the exact training data, which may not be available in tool unlearning, especially when dealing with closed-source LLM training data. 
These differences are discussed in detail in~\S\ref{sec:diff}.


% \paragraph{Challenge of tool unlearning}
% as opposed to individual data samples, which makes it fun
% and existing unlearning methods are not fundamentally designed for tool removal; 
% 
% similar to sample-level unlearning, in tool unlearning, 
Removing skills requires  
modifying the parameters of LLMs, a process that is computationally expensive and can lead to unforeseen behaviors~\citep{ripple_effect,gu2024model}. In addition, existing membership inference attack (MIA) techniques, a common evaluation method in machine unlearning to determine whether specific data samples were part of training data, are inadequate for evaluating tool unlearning, as they focus on sample-level data rather than tool-based knowledge. 
% and practically difficult due to potential unforeseeable side effects on other tasks when updating LLM's parameters~\citep{ripple_effect,gu2024model}. 
% Additionally, there is no prior Membership Inference Attack (MIA) models, a desired evaluation of unlearning, designed to detect if a tool is present in training set.  


To address these challenges, we propose \method, the first tool unlearning algorithm for tool-augmented LLMs, which satisfies three key properties for effective tool unlearning: 
{\em tool knowledge removal}, which focuses on removing any knowledge gained on tools marked for unlearning; 
{\em tool knowledge retention}, which focuses on preserving the knowledge gained on other remaining tools; and 
{\em general capability retention}, which maintains LLM's general capability on a range of general tasks such as text and code generation using ideas from task arithmetic~\citep{ilharco2023editing,barbulescu2024textual}.
%
In addition, we develop LiRA-Tool, an adaptation of the Likelihood Ratio Attack (LiRA)~\citep{lira} to tool unlearning, to assess whether tool-related knowledge has been successfully unlearned. Our contributions are: 

% When receiving deletion requests, a successful tool unlearning algorithm should remove the knowledge of the tools marked for unlearning, as if the model has never seen such tools before. Meanwhile, the model's knowledge on the remaining tools as well as other tasks should be preserved to the maximum extent. This is important since the deletion request are targeted at specific subset of tools, usually much smaller than the entire tool set, and practically difficult due to potential unforeseeable side effects.

\vspace{-10pt}
\begin{itemize}
\itemsep-1pt
    \item introducing and conceptualizing tool unlearning for tool-augmented LLMs,
    \item \method, which implements three key properties for effective tool unlearning;
    \item LiRA-Tool, which is the first membership inference attack (MIA) for tool unlearning.
\end{itemize}


Extensive experiments on multiple datasets and tool-augmented LLMs show that \method outperforms existing general and LLM-specific unlearning algorithms by $+$ in accuracy on forget tools and retain tools.  In addition, it can save 74.8\% of training time compared to retraining, handle sequential unlearning requests, and retain 95+\% performance in low resource settings.\looseness-1
\section{Related work}
    \label{sec:related}
    \xhdr{Domain-specific Tokenizers} 
Tokenizers tailored for specific domains have been employed to process various types of data, including language~\cite{bpe,sentencepiece,wordpiece,Wang2024challenging,Minixhofer2024zeroshot}, images~\cite{ibot,vqgan,Yu2024difftok,Zha2024textok}, videos~\cite{Choudhury2024dontlook}, graphs~\cite{Perozzi2024graphtalk,vqgraph}, and molecular and material sciences~\cite{Fu2024moltok,Tahmid2024birna,Qiao2024mxdna}. While these tokenizers perform well within their respective domains, they are not directly applicable to medical codes, which contains specialized medical semantics. Medical codes reside in relation contexts and are accompanied by textual descriptions. Directly using the tokenizers for languages risks flattening the relationships among codes and failing to preserve the biomedical information. This will lead to fragmented tokenization of medical codes, resulting in loss of contextual information during encoding.
Meanwhile, visual tokenizer typically focus on local pixel-level relationships, which are insufficient for capturing the complex semantics inherent in medical codes. Graph tokenizers are designed to encode structured information from graphs into a discrete token, then enabling LLMs to process relational and topological knowledge effectively. However, graph tokenizers may suffer from information loss when applied to graphs in other domains, making them less flexible and efficient for large, dynamic, and cross-domain graphs. In contrast, our \model tokenizer explicitly incorporates the relevant medical semantics by integrating textual descriptions with graph-based relational contexts.


\xhdr{Vector-Quantized Tokenizers}
Tokenization strategies often vary according to the problem domain and data modality where recent work has highlighted the benefits of discrete tokenization~\cite{du2024role}. This process involves partitioning the input according to a finite set of tokens, often held in a \textit{codebook} (this concept is independent of medical coding despite the similar name), and the quantization process involves learning a mapping from input data to the optimal set of tokens according to a pre-defined objective such as reconstruction loss~\cite{van2017neural}. 

Recent work has highlighted the ability of vector quantized (VQ-based) tokenization to effectively compress semantic information\cite{gu2024rethinking}. This approach is particularly successful for tokenizing inputs with an inherent semantic structure such as graphs~\cite{yang2023vqgraph, wang2024learning}, speech~\cite{zeghidour2021soundstream, baevski2019vq}, and time~\cite{yu2021vector} as well as complex tasks like recommendation retrieval \cite{wang2024learnable, rajput2023recommender, sun2024learning} and image synthesis \cite{zhang2023regularized, yu2021vector}.

Another significant advantage to VQ-based tokenization is the natural integration of multiple modalities. By learning a shared latent space across modalities, each modality can jointly modeled using a common token vocabulary \cite{agarwal2025cosmos, yu2023language}. 
% I think there are probably better citations to use for the line above
TokenFlow leverages a dual-codebook design that allows for correlations across modalities through a dual encoder~\cite{qu2024tokenflow}.


\xhdr{Structured EHR, transformer-based, and foundation models} 
%
Structured EHR models leverage patient records to learn representations for clinical prediction and operational healthcare tasks. These models differ from medical LLMs~\cite{singhal2025toward,tu2024towards,singhal2023large}, which are typically trained on free-text clinical notes~\cite{jiang2023health} and biomedical literature rather than structured EHR data.  
%
BEHRT~\cite{li2020behrt} applies deep bidirectional learning to predict future medical events, encoding disease codes, age, and visit sequences using self-attention. TransformEHR~\cite{transform_ehr} adopts an encoder-decoder transformer with visit-level masking to pretrain on EHRs, enabling multi-task prediction. GT-BEHRT~\cite{gtbehrt} models intra-visit dependencies as a graph, using a graph transformer to learn visit representations before processing patient-level sequences with a transformer encoder.  
%
Other models enhance EHR representations with external knowledge. GraphCare~\cite{graphcare} integrates large language models and biomedical knowledge graphs to construct patient-specific graphs processed via a Bi-attention Augmented Graph Neural Network. Mult-EHR~\cite{mult_ehr} introduces multi-task heterogeneous graph learning with causal denoising to address data heterogeneity and confounding effects. ETHOS~\cite{ethos} tokenizes patient health timelines for transformer-based pretraining, achieving zero-shot performance.  
%
While these models focus on learning patient representations, \model serves a different role as a medical code tokenizer. It can be integrated into any structured EHR, transformer-based, or other foundation model, improving how medical codes are tokenized before being processed. Unlike these models, which rely on predefined tokenization schemes, \model optimizes the tokenization process itself.
\section{Preliminary}
    \label{sec:preliminary}
    \begin{figure*}[t]
    \centering
    \includegraphics[width=0.98\linewidth]{FIG/TFM_main_method_fig_new_3.pdf}
    %{KDD2025_EEG_Tokenization/FIG/TFM_token_method_2.pdf}
    % \vspace{-.5cm}
    \caption{Overview of the \method framework. (a) \tokenizer Pretraining: Through dual-path encoding and masked prediction, learns to capture time-frequency motifs into discrete tokens. (b) \encoder Pretraining: Trains on learned EEG tokens using masked token prediction. (c) Masking Strategy: A combination of frequency band masking and temporal masking is used for \tokenizer pretraining. (d) Localized Spectral Window Encoder: Processes individual spectral windows from $\mathbf{S}$, extracts frequency band information, and aggregates features across all bands into a single compact embedding per window. }
    \label{fig:tfm_token}
\end{figure*}

\subsection{Notations and Problem Statements}
\noindent\textbf{EEG Data.}
Let $\X\in\mathbb{R}^{C\times T}$ denote a multi-channel EEG recording with $C$ channels and $T$ time samples. For each channel $c\in\{1,...,C\}$, we denote a single channel EEG by $x^c\in\mathbb{R}^T$. 
To capture complementary structures in both the time and frequency domains, we decompose $x^c$ into: (1) time-frequency representation $\mathbf{S}$ and (2) raw EEG patches $\{x_i\}_{i=1}^N$. For simplicity, we omit the channel index; henceforth, $x$ will refer to a single-channel EEG signal unless otherwise specified. 

\noindent\textbf{Short-Time Fourier Transform (STFT).}
To obtain the time-frequency representation, i.e.g, spectrogram, $\mathbf{S}$, we apply a STFT to $x$ using a windowing function $w(.)$ of length $L$ and a hop size $H$:
\begin{equation}
    \mathbf{S}(\omega,\tau) = \left|\sum_{l=0}^{L-1}x(\tau H +l)w(l)e^{\frac{-j2\pi\omega l}{L}} \right|
\end{equation}
where $\omega$ indexes the discrete frequencies and $\tau$ indexes the time segments (i.e., time windows shifted by $H$). We retain only the magnitude $|.|$ to form $\mathbf{S}\in \mathbb{R}^{F\times N}$, where $F$ is the number of frequency bins and $N$ is the number of time windows.
\\

% Additionally, we define an EEG token vocabulary $\mathcal{V}$ consisting of $k$ discrete tokens, $\mathcal{V}=\{v^1,v^2,...,v^k\}$. 

\noindent\textbf{Problem Statement 1 (EEG Tokenization):} Given a single channel EEG $x$, we aim to learn a tokenization function 
$$
f_{\text{tokenizer}}: \mathbb{R}^T \rightarrow \mathcal{V}^{N \times D}
$$
where $\mathcal{V}$ is a finite EEG token vocabulary of size $k$, and $D$ is the dimension of each token embedding. 
The tokenizer function $f_{\text{tokenizer}}$ should project $x$ (or transformations) into a sequence of discrete tokens $\{v_i\}_{i=1}^N$,
where each $v_i\in \mathcal{V}$.
These tokens represent various temporal and frequency ``\textit{motifs}'': meaningful EEG patterns characterized by distinct temporal and frequency characteristics. Therefore, $\mathcal{V}$ is learnbale from $\mathbf{S}$ and the temporal patches $\{x_i\}_{i=1}^N$.\\
\textbf{Remark.}
We here hold several expectations for the learned motif tokens.
First, these tokens are expected to reduce redundancy, noise, and complexity, providing a compact, sparse, and informative representation of EEGs.
Second, these motifs should effectively capture essential neurophysiological patterns from both temporal and frequency domains.
Third, the tokens should generalize well across different EEG tasks, enhancing the efficiency and interpretability of the data.
We set up related research questions to evaluate these expectations in Section.~\ref{sec:exp}.


\noindent\textbf{Problem Statement 2 (Multi-Channel EEG Classification):} 
% \noindent\textbf{Problem Statement 2 (Multi-Channel EEG Classification):} 
Given EEGs $\X$ and a fixed, learned single-channel tokenizer $f_{\text{tokenizer}}$, we apply $f_{\text{tokenizer}}$ independently to each channel $c$ to obtain a tokenization representation  $\Bigl\{\{v_i^c\}_{i=1}^N\Bigr\}_{c=1}^C$. 
Then, these tokens can serve various downstream tasks. For classification tasks, they are mapped to labels by:
$$
f_{\text{classifier}}: (\mathcal{V}^D)^{N \times C}\rightarrow \mathbf{Y}
$$
where $Y$ is the target labels (e.g., EEG events, seizure types). 
% Task-specific fine-tuning can then be performed.
By aggregating and processing the tokens across all channels, $f_{\text{classifier}}$ predicts a label $y\in\mathbf{Y}$. 
Notably, $f_{\text{classifier}}$ can be any downstream-oriented model, and its training is performed separately from the EEG tokenizer $f_{\text{tokenizer}}$.
% The learning objective is to minimize a loss function $\mathcal{L}_{\text{cls}}$, that measures the discrepancy between the prediction $\hat{y}$ and the ground-truth label $y$.



\section{Methodology}
    \label{sec:moethod}
    
\section{\method} %: Effective Tool Unlearning for LLMs


We develop \method--an effective tool unlearning approach that removes the capability of using tools marked for unlearning ($\mathcal{T}_f$) or solving tasks that depend on them, while preserving the ability of using the remaining tools ($\mathcal{T}_r$) and performing general tasks such as text and code generation. \method implements three key properties for effective tool unlearning: \looseness-1

% \subsection{Required Properties for Effective Tool Unlearning}
\subsection{Tool Knowledge Deletion}
% The tool-augmented model $f$ gains its knowledge of \tf through tool learning. 
Unlearning requires completely removing the knowledge of \tf that $f$ gained during tool learning, ideally as if \tf had never been part of the training set. In other words, knowledge about \tf is successfully removed if the unlearned model $f'$ has no more knowledge than the tool-free model $f_0$ about \tf. \looseness-1

\begin{definition}[Tool Knowledge Deletion (TKD)]
Let $t_i \in \mathcal{T}_f$ denote a tool to be unlearned and $g$ be a function that quantifies the amount of knowledge a model has about a tool. The unlearned model $f'$ satisfies tool knowledge deletion if:
\begin{equation}\label{eq:prop1}
    \mathop\mathbb{E}_{t_i \in \mathcal{T}_f} [ g(f_0, t_i) - g(f', t_i) ] \geq 0.
\end{equation}
\end{definition}
% so that $f'$ retains no more knowledge of $T_f$ than $f_0$.
This formulation allows users to control the extent of knowledge removal from $f'$. For instance, when we unlearn a ``malicious'' tool that calls a malignant program, we may require $f'$ retains no knowledge of this tool, i.e. $g(f', t_i) = 0$. In less critical cases, users can choose to reset $f'$'s knowledge to {\em pre}-tool augmentation level, i.e. $g(f', t_i) = g(f_0, t_i)$

To measure tool knowledge in LLMs, we follow previous works that used prompting to probe LLMs' knowledge~\citep{gpt3,singhal2023large}, i.e. adopting the output of LLMs as their knowledge on a given tool. For each $t_i \in \mathcal{T}_f$ and its associated demonstrations $\{ \mathcal{Q}_i, \mathcal{Y}_i \}$, we query the tool-free LLM $f_0$ with $\mathcal{Q}_i$ and collect its responses $\mathcal{Y}'_i = f_0(Q_i)$. Since $f_0$ has never seen $t_i$ or $\{ \mathcal{Q}_i, \mathcal{Y}_i \}$, $\mathcal{Y}'_i$ represents the \textbf{tool-free response}. We then constrain the unlearned model $f'$ to generate responses similar to $\mathcal{Y}'_i$ to prevent it from retaining knowledge of $t_i$.


% \paragraph{Various levels of knowledge removal}
% Knowledge removal can happen in different cases for tool learning / unlearning. 
% \begin{itemize}
%     \item do not choose any tool.
%     \item choose a different tool. 
%     \item choose the right tool but wrong arguments. 
%     \item say I don't know how to solve this task based on internal knowledge / learned tools. Depends on evaluation.
% \end{itemize}
% Note that the difference between 1) and 4) is that 1) predicts that $x_i$ requires no tool to solve. While 4) encourages the LLM to answer "I don't know.".

% Most existing tool-augmented LLMs are trained in a Supervised Fine-Tuning manner (SFT), where language modeling objective is optimized query-response pairs $(q_i, y_i)$.


\subsection{Tool Knowledge Retention}
The unlearning process should preserve model's knowledge of tools in $T_r$. Ideally, all knowledge gained on $T_r$ during tool learning should be retained after unlearning. 


\begin{definition}[Tool Knowledge Retention (TKR)]
Let $t_m \in T_r$ denote a retained tool, and let $g$ be a function that quantifies the amount of knowledge a model has about a tool. The unlearned model $f'$ satisfies tool knowledge retention if:\looseness-1
\begin{equation}
    \mathop\mathbb{E}_{t_m \in \mathcal{T}_r} [ g(f, t_m) - g(f', t_m) ] = \epsilon,
    \label{eq:prop2}
\end{equation} 
where $\epsilon$ is an infinitesimal constant, so that $f'$ retains the same knowledge of tools in $T_r$ as the original model $f$.
\end{definition}
For effective tool knowledge retention, $f'$ is further fine-tuned using demonstrations associated with $\mathcal{T}_r$, or, more practically, a subset of $\mathcal{T}_r$ proportional to $\mathcal{T}_f$ for efficiency.



\subsection{General Capability Retention via Task Arithmetic}
Optimizing the above objectives can lead to effective unlearning, but it may not be sufficient to maintain the general capabilities of the unlearned model $f'$. As a foundation model, $f'$ is expected to retain abilities such as text and code generation, question answering, instruction-following, and basic mathematical reasoning. These capabilities either existed in $f_0$ prior to tool augmentation or do not depend on specific tools. Therefore, preserving the general capabilities of $f'$ is essential to guarantee that tool unlearning does not compromise the overall functionality of the model. 

\begin{definition}[General Capability Retention (GCR)]
Let $\mathcal{T}_G$ denote the general tasks used to evaluate LLMs. The unlearned model $f'$ satisfies general capability retention if it preserves the knowledge on $T_G$ that it originally obtained prior to tool learning:
\begin{equation}
    \mathop\mathbb{E}_{t_g \in \mathcal{T}_G} [ g(f_0, t_g) - g(f', t_g) ] = \epsilon,
    \label{eq:prop3}
\end{equation} where $\epsilon$ is an infinitesimal constant.
\end{definition}

We propose to use task arithmetic~\citep{ilharco2023editing,barbulescu2024textual} as an efficient and effective approach to preserving the general capabilities of the unlearned model. Our objective is that $f'$ retains as much general knowledge as $f_0$, the instruction tuned LLM trained from a randomly initialized model $f_R$. 
Let $\theta_0$ and $\theta_R$ denote the parameters of $f_0$ and $f_R$ respectively. The difference vector $\theta_0 - \theta_R$ captures the direction of general knowledge acquisition. We apply this adjustment to $\theta'$ (the parameters of $f'$) to preserve its general knowledge:
\begin{equation}
    \theta'^* \leftarrow \theta' + (\theta_0 - \theta_R).
\end{equation}
% This approach allows $f'$ to retain its general capabilities while effectively unlearning specific tools.

% This can be achieved through additional pre-training or instruction tuning. However, such explicit training methods may pose two difficulties in practice. 1) The pre-training or instruction tuning dataset may not be easily accessible at the stage of tool unlearning, making explicit training impossible. 2) Training on a subset of pre-training / instruction tuning dataset together with $ D^0_f \cup D_r $ at the same time may become prohibitively expensive and difficult, given the distinctiveness of these datasets. To account for these difficulties, we adopt task arithmetic~\citep{} to maintain the general utilities of $f'$.

\paragraph{Why Task Arithmetic?}
% While knowledge in LLMs can be highly nonlinear, task arithmetic assumes a linear transformation in the parameter space, which may not always hold. In addition, the vector difference $(\theta_0 - \theta_R)$ may over-correct or under-adjust general abilities, which may leave some tool-related knowledge in the model.
%
% Despite these limitations, 
Task arithmetic is efficient, practical, effective for preserving general capabilities~\citep{ilharco2023editing,barbulescu2024textual}: 
\textbf{Efficiency}: the vector operation does not scale with dataset size, making it significantly more efficient than retraining on large datasets. 
% This cost does not scale with the size of the dataset, which can be considered as static and offline. While explicitly training on the pre-training and instruction-tuning datasets is significantly more expensive.
% 
\textbf{Practicality}: general capabilities obtained from pre-training and instruction tuning~\citep{zhou2024lima} are often impractical to replicate due to the size and limited availability of data--even in some open-source LLMs~\citep{touvron2023llama2}, the actual pre-training data is not fully open-source. In addition, reintroducing general knowledge from alternative datasets can lead to data imbalances and distributional biases. 
%
\textbf{Effectiveness}: applying $\theta_0 - \theta_R$ largely restores the foundational abilities of $f'$, such as text generation and instruction-following, without requiring expensive and time-consuming retraining on large datasets.



\subsection{Training Details}
To obtain the unlearned model $f'$, we solve:
\begin{equation}
    % \theta'^* = \arg \min_{\theta'} \\
    % \underbrace{\mathbb{E}_{t_i \in \mathcal{T}_f} [ g(f_0, t_i) - g(f', t_i) ]}_{\text{Optimization}} + \underbrace{\mathbb{E}_{t_m \in \mathcal{T}_r} [ g(f, t_m) - g(f', t_m) ]}_{\text{Optimization}} + \\ 
    % \underbrace{\alpha (\theta_0 - \theta_R)}_{\text{Task Arithmetic}},
    \theta'^* = \arg \min_{\theta'} \underbrace{\mathbb{E}_{t_i \in \mathcal{T}_f} [ g(f_0, t_i) - g(f', t_i) ]}_{\text{knowledge deletion of }\mathcal{T}_f} + \underbrace{\mathbb{E}_{t_m \in \mathcal{T}_r} [ g(f, t_m) - g(f', t_m) ]}_{\text{knowledge retention of }\mathcal{T}_r},
\end{equation} 
and once the optimized model parameters $\theta'^*$ are obtained, we apply task arithmetic to reinforce general capabilities:
\begin{equation}
    \theta'^* = \underbrace{\theta'^*}_{\text{post-optimization weights}} + \underbrace{\alpha (\theta_0 - \theta_R)}_{\text{knowledge retention of }\mathcal{T}_G},
\end{equation} 
where $\alpha$ is a hyperparameter to control the magnitude of task arithmetic. 
% The loss function $L$ depends on the specific training method. 
% Following previous works, $f'$ is initialized as $f$ to maintain maximum knowledge retention on $T_r$.
% , the specific choice of optimization method loss function depends on the training method to optimize for $\theta'^*$. 
The above formulation provides flexibility in training \method using various existing paradigms, including 
supervised fine-tuning (SFT)~\citep{alpaca}, 
direct preference optimization (DPO)~\citep{rafailov2023direct}, 
reinforcement learning from human feedback (RLHF)~\citep{ouyang2022training}, 
parameter-efficient fine-tuning (PEFT)~\citep{he2022towards,su-etal-2023-exploring}, or
quantization~\citep{8bit_quant,ma2024era} techniques. 
Below we describe two variants of \method:
% \vspace{-20pt}
\begin{itemize}
\itemsep0pt
    \item \textbf{\method-SFT} fine-tunes $f$ using language modeling loss. On forget tools $\mathcal{T}_f$, we replace the original responses $\mathcal{Y}_f$ with tool-free responses $\mathcal{Y}'_f$. The samples for $\mathcal{T}_r$ are not modified. 
    % Similar to prior SFT works, we only compute loss on the response and exclude the query part.

    \item \textbf{\method-DPO} uses direct preference optimization (DPO) to prioritize wining responses over losing responses. For $(t_i, \mathcal{Q}_i, \mathcal{Y}_i) \in \mathcal{T}_f$ to be unlearned, we prioritize the corresponding tool-free response $\mathcal{Y}'_i$ over the original response $\mathcal{Y}_i$. For $(t_j, \mathcal{Q}_j, \mathcal{Y}_j) \in \mathcal{T}_r$, the original response $\mathcal{Y}_j$ is prioritized over the tool-free response $\mathcal{Y}'_i$. \looseness-1
    % Therefore, the knowledge of the unlearned model $f'$ on $\mathcal{T}_f$ can be removed without affecting $\mathcal{T}_r$. 

\end{itemize}



\subsection{LiRA-Tool for Tool Unlearning Evaluation}

\paragraph{Challenge}
A key challenge in evaluating tool unlearning is the lack of membership inference attack (MIA) models to determine whether a tool has been truly unlearned. Existing MIA models typically evaluate individual training samples by analyzing model loss, which is insufficient for tool unlearning. 
% , i.e., in case of LLMs, the loss of responses for given queries. 
Unlike sample-level unlearning, tool unlearning focuses on removing abstract parametric knowledge of tools in $\mathcal{T}_f$, not just forgetting specific training samples. The key limitation of sample-based MIA is that the prompt-response pairs $(\mathcal{Q}_f, \mathcal{Y}_f)$ in the training set may not fully represent all aspects of a tool's functionality. As a result, sample-level MIA may ``overfit'' to a limited subset of tool related prompts and fail to holistically assess whether the tool-usage capability have been fully removed from the model's parametric knowledge.\looseness-1  

\paragraph{Solution}\label{sec:lira_tool}
To address the above limitation, we introduce ``shadow samples'', a diverse set of prompt-response pairs to probe various aspects of tool knowledge. 
We prompt GPT4 with different combinations of in-context examples to obtain a comprehensive set of prompt-response pairs with various prompt format, intention, and difficulty requirements. 
These samples will be used to stress-test the unlearned LLM $f'$ beyond the specific training prompts. This approach prevents overfitting to the original training data and provides a more reliable evaluation of whether the tool has truly been forgotten. To implement this, we extend Likelihood Ratio Attack (LiRA)~\citep{lira}, the state-of-the-art MIA approach, to tool unlearning.


\paragraph{Sample-level LiRA}
LiRA infers the membership of a sample $(x, y)$ by constructing two distributions of model losses: $\mathbb{\Tilde{Q}}_{\text{in}}$ and $\mathbb{\Tilde{Q}}_{\text{out}}$ with $(x, y)$ in and out of the model training set respectively. These distributions are approximated as Gaussians, with their parameters estimated based on ``shadow models'' trained on different subsets of the training data. The Likelihood-Ratio Test~\citep{07dc41a8-17bb-36b0-8eb8-d51fd0847411,lira} is then used to determine whether $(x, y)$ is more likely to belong to $\mathbb{\Tilde{Q}}_{\text{in}}$ or $\mathbb{\Tilde{Q}}_{\text{out}}$. For LLMs, the test statistic is given by~\citep{icul} as:
% $p(L(f(x), y) | \mathbb{\Tilde{Q}}_{\text{in/out}})$
\begin{equation}
    \Lambda = \frac{P \Bigl(l \bigl(f(x), y \bigr) | \mathbb{\Tilde{Q}}_{\text{in}}\Bigr)}{P \Bigl(l \bigl(f(x), y \bigr) | \mathbb{\Tilde{Q}}_{\text{out}}\Bigr)} = \frac{\Pi_{(x_i, y_i) \in \mathcal{D}_f} P_U \Bigl(l \bigl(f'(x_i), y_i \bigr)\Bigr)}{\Pi_{(x_i, y_i) \in \mathcal{D}_f} P_{T_r} \Bigl(l \bigl(f(x_i), y_i \bigr)\Bigr)}.
\end{equation} 
% which intuitively queries the loss of $(x, y)$ to determine if $(x, y)$ is more likely to be present in the training set or not.
This approach, however, is insufficient for tool unlearning because it only assesses membership of specific training samples rather than measuring whether the model still retains the capability to use a tool.



\paragraph{LiRA-Tool: Knowledge-level LiRA}
A major limitation of sample-level LiRA is in its reliance on training-set observations, which may not fully capture the knowledge distribution of an entire tool. Therefore, applying LiRA to tool unlearning can lead to overfitting to a specific subset of training prompts and failing to comprehensively assess whether the tool knowledge has been removed. 
%
% is in approximating the distributions of losses $\mathbb{\Tilde{Q}}_{\text{in}}$ and $\mathbb{\Tilde{Q}}_{\text{out}}$ for tools, rather than individual training samples. 
% Simply using the observed data related to a tool in the training set may overfit to specific distribution of observations, and may fail to comprehensively approximate the distribution of the target tool marked for unlearning. 
We address this issue by introducing LiRA-Tool. Instead of relying on observed training samples, we construct a ``shadow distribution'' $\mathbb{P}$ that generates tool-related query-response pairs. This allows us to sample diverse tool-specific prompts that test the model's ability to use the tool. The new likelihood-ratio test is:\looseness-1
% sample a series of ``shadow'' data (query-response pairs) that evaluates the tool using the ability to compute loss and test statistic as follows:
\begin{equation}
    \Lambda = \frac{\Pi_{t_i \in \mathcal{T}_f}\Pi_{(x, y) \in \mathbb{P}_{t_i}} P_U \Bigl(l \bigl(f'(x), y \bigr) \Bigr)}{\Pi_{t_j \in T_r}\Pi_{(x, y) \in \mathbb{P}_{t_j}} P_{\mathcal{T}_r} \Bigl(l \bigl(f(x), y \bigr) \Bigr)},
\end{equation} 
where $\mathbb{P}_{t_i}$ represents the shadow distribution for generating tool-learning samples for tool $t_i$. 
% is the distribution that controls the generation of tool learning samples for $t_i$. 
In practice, we use GPT-4 to generate diverse shadow samples by prompting it with various distinct instructions to ensure that the evaluation set captures more comprehensive aspects of tool knowledge than the training set. Appendix~\ref{sec:prompt_shadow_sample} provides more details.\looseness-1
% for approximating $\mathbb{\Tilde{Q}}_{\text{in}}$ and $\mathbb{\Tilde{Q}}_{\text{out}}$ and performing likelihood-ratio test.


\paragraph{Novelty of LiRA-Tool}
The key novelty in LiRA-Tool in the sue of ``shadow samples,'' which introduce diversity across multiple dimensions.
% , including prompt format, intent, and difficulty. 
By moving beyond limited training prompts, LiRA-Tool ensures that the model loss reflect overall tool-using ability, rather than just sample-level memorization.
% The major difference is that traditional LiRA approximates $\mathbb{\Tilde{Q}}_{\text{in}}$ and $\mathbb{\Tilde{Q}}_{\text{out}}$ with a series of shadow models by controlling which samples
%models 
% are present in training set. 
% however, unlearning a skill (tool) is prioritized 
%over individual samples  
% Consequently, using the original samples may not comprehensively approximate the distribution of $\mathbb{\Tilde{Q}}_{\text{in}}$ and $\mathbb{\Tilde{Q}}_{\text{out}}$. 
% We instead 
% . Such samples differ from each other, which encourages the losses to reflect efficacy in tool using instead of membership of individual training sample. 
% We GPT-4 as the shadow distribution $\mathbb{P}$ due to its superior tool using ability and the diversity of its generation. 
%
Our loss-ratio formulation shares similarities to previous MIAs for sample-level unlearning, such as probability distribution comparison prior- and post-unlearning~\citep{cheng2023gnndelete,cheng2023multimodal} and other adaptations of LiRA 
% which performs likelihood-ratio test over 
using shadow models~\citep{unbound,icul}. However, to the best of our knowledge, this work is the first adaptation of LiRA for detecting tool presence in tool-augmented LLMs. 




% We adapt sample-level MIA into knowledge-level MIA to infer the membership of tools for tool unlearning evaluation; and propose a new method to estimate $\mathbb{\Tilde{Q}}_{\text{in}}$ and $\mathbb{\Tilde{Q}}_{\text{out}}$. 
% based on latent variables. 
% This provides a comprehensive approximation of abstract concepts beyond observed training data. 

\paragraph{Limitations of LiRA-Tool}
Shadow samples obtained from GPT-4 may not fully represent the complexity of the original tool-learning data and can potentially lead to incomplete approximations of the true knowledge distribution.
% LiRA-Tool is still more comprehensive than existing sample-level MIA, because 
However, despite this limitation, shadow samples provide a more comprehensive and consistent evaluation of a model's tool-using abilities compared to relying merely on observed training samples, which are often limited and incomplete. Expanding the diversity and robustness of shadow sample generation is indeed an important direction for future work. % In addition, if the size of the shadow sample is large enough for each tool, it can better approximate the knowledge distribution for the tool.


\begin{table*}[t]
% \setlength{\tabcolsep}{4pt}
\caption{Tool unlearning performances when deleting 20\% of tools on ToolAlpaca. Best and second-best performances are \textbf{bold} and \underline{underlined} respectively. \textit{Original} is provided \textit{for reference only}. Results on other LLMs are shown in Appendix Table~\ref{tab:tool_llama}-\ref{tab:gorilla}.}
\label{tab:main}
\vskip 0.15in
\begin{center}
\begin{small}
\begin{sc}
\begin{tabular}{ll|ccc|ccccc}
\toprule
& Method & $\mathcal{T}_t (\uparrow)$ & $\mathcal{T}_r (\uparrow)$ & $\mathcal{T}_f (\downarrow)$ & \multicolumn{5}{c}{General Capability $\mathcal{T}_G (\uparrow)$} \\
                & & & & & STEM & Reason & Ins-Follow & Fact & Avg. \\
\midrule
\rowcolor{Gray} & Original (Ref Only) 
             & 60.0 & 73.1 & 75.7 & 31.7 & 17.1 & 22.6 & 25.0 & 24.1 \\
\midrule
\multirow{4}{*}{\rotatebox{90}{General}} 
& \RET & 52.1 & 71.8 & 38.5 & 30.5 & 16.1 & 14.2 & 24.7 & 21.3 \\
& \GA  & 33.3 & 51.4 & 34.6 & 21.4 & 10.4 & 12.9 & 13.1 & 14.5 \\
& \RL  & 50.3 & 70.3 & 37.5 & 26.3 & 16.4 & 13.6 & 25.1 & 20.3 \\
& \SU  & 46.2 & 54.3 & 38.2 & 27.1 & 17.0 & 17.4 & 19.5 & 20.2 \\
\midrule
\multirow{6}{*}{\rotatebox{90}{LLM-Specific}} 
& ICUL & 49.1 & \underline{74.8} & 58.3 & 12.4 &  8.7 &  1.6 &  6.2 &  7.3 \\
& SGA  & 43.5 & 63.0 & 42.1 & 21.5 & 11.6 & 17.0 & 14.7 & 16.2 \\
& TAU  & 43.8 & 61.7 & 42.5 & 22.0 & 17.6 & 22.3 & 21.7 & 20.9 \\
& CUT  & 44.7 & 61.5 & 40.2 & 21.6 & 14.8 & 20.8 & 16.4 & 18.4 \\
& NPO  & 50.8 & 66.9 & \underline{30.1} & 20.7 & 15.3 & 21.9 & 18.9 & 19.2 \\
& \SO  & 50.4 & 68.3 & 33.8 & 31.6 & 17.2 & 21.4 & 20.8 & 22.7 \\
\midrule
\multirow{2}{*}{\rotatebox{90}{Ours}} 
& \method-SFT & \underline{52.7} & 72.1 & \underline{30.5} & 31.3 & 17.5 & 21.7 & 24.1 & \textbf{23.6} \\
& \method-DPO & \textbf{53.4} & \textbf{75.1} & \textbf{28.7} & 31.6 & 16.8 & 20.4 & 23.5 & \underline{23.1} \\
\bottomrule
\end{tabular}
\end{sc}
\end{small}
\end{center}
\vskip -0.1in
\end{table*}



% \begin{table*}[t]
% % \setlength{\tabcolsep}{4pt}
% \caption{Tool unlearning performances when deleting 20\% of tools on ToolAlpaca. Evaluation is performed with the specific metric for each tool-augmented LLM on test tools $\mathcal{T}_t$, remaining tools $\mathcal{T}_r$, and unlearned tools $\mathcal{T}_f$, as well as general benchmarks for evaluation LLMs $\mathcal{T}_G$. Best and second best performances are \textbf{bold} and \underline{underlined} respectively. \textit{Original} denotes the tool-augmented LLM prior unlearning and is provided \textit{for reference only}. Results on other LLMs are shown in Appendix Table~\ref{tab:tool_llama}-\ref{tab:gorilla}.}
% \label{tab:main}
% \centering
% \small
% \begin{tabular}{l|ccc|ccccc}
% \toprule
% \textbf{Method} & $\mathcal{T}_t (\uparrow)}$ & $\mathcal{T}_r (\uparrow)}$ & $\mathcal{T}_f (\downarrow)}$ & \multicolumn{5}{c}{\textbf{General Capability} $\mathcal{T}_G} (\uparrow)}$} \\
%                 & & & & \textbf{STEM} & \textbf{Reason} & \textbf{Ins-Follow} & \textbf{Fact} & \textbf{Avg}. \\
% \midrule
% \rowcolor{Gray}\textbf{Original (Prior Un.)} 
%              & 60.0 & 73.1 & 75.7 & 31.7 & 17.1 & 22.6 & 25.0 & 24.1 \\
% \midrule
% \multicolumn{9}{l}{General Unlearning Methods} \\
% \midrule
% \textbf{\RET} & 52.1 & 71.8 & 38.5 & 30.5 & 16.1 & 14.2 & 24.7 & 21.3 \\
% \textbf{\GA}  & 33.3 & 51.4 & 34.6 & 21.4 & 10.4 & 12.9 & 13.1 & 14.5 \\
% \textbf{\RL}  & 50.3 & 70.3 & 37.5 & 26.3 & 16.4 & 13.6 & 25.1 & 20.3 \\
% \textbf{\SU}  & 46.2 & 54.3 & 38.2 & 27.1 & 17.0 & 17.4 & 19.5 & 20.2 \\
% \midrule
% \multicolumn{9}{l}{LLM-Specific Unlearning Methods} \\
% \midrule
% \textbf{ICUL}           & 49.1 & \underline{74.8} & 58.3 & 12.4 &  8.7 &  1.6 &  6.2 &  7.3 \\
% \textbf{SGA}            & 43.5 & 63.0 & 42.1 & 21.5 & 11.6 & 17.0 & 14.7 & 16.2 \\
% \textbf{TAU}            & 43.8 & 61.7 & 42.5 & 22.0 & 17.6 & 22.3 & 21.7 & 20.9 \\
% \textbf{CUT}            & 44.7 & 61.5 & 40.2 & 21.6 & 14.8 & 20.8 & 16.4 & 18.4 \\
% \textbf{NPO}            & 50.8 & 66.9 & \underline{30.1} & 20.7 & 15.3 & 21.9 & 18.9 & 19.2 \\
% \textbf{SOUL-GradDiff}  & 50.4 & 68.3 & 33.8 & 31.6 & 17.2 & 21.4 & 20.8 & 22.7 \\
% \midrule
% \textbf{\method-SFT} & \underline{52.7} & 72.1 & \underline{30.5} & 31.3 & 17.5 & 21.7 & 24.1 & \textbf{23.6} \\
% \textbf{\method-DPO} & \textbf{53.4} & \textbf{75.1} & \textbf{28.7} & 31.6 & 16.8 & 20.4 & 23.5 & \underline{23.1} \\
% \bottomrule
% \end{tabular}
% \end{table*}




% Let $t_m \in T_r$ denote a tool to be unlearned. $g$ is a function that measures the amount of knowledge a model has on a tool. A unlearned model satisfies Knowledge Retention if

% \begin{equation}
%     \mathop\mathbb{E}_{t_m \in T_r} [ g(f', t_m) - g(f, t_m) ] = \epsilon,
%     \label{eq:prop2}
% \end{equation} where $\epsilon$ is an infinitesimal constant.


% \subsection{Knowledge Removal and Retention}

% The most critical part of knowledge removal is to reset $f'$'s knowledge on $T_f$ back to a level that is similar to $f_0$, which has never seen $T_f$. In other words, to unlearn a given tool $t_i \in T_f$, we want to match $g(f', t_i)$ to $g(f_0, t_i)$. For LLMs, researchers usually prompt them to probe their knowledge on a topic or a fact. To this end, we propose to prompt the vanilla model $f_0$ with the ground-truth queries $Q_i$ associated with $t_i$, resulting in $f_0(Q_i)$. If a model outputs contents similar to $f_0(Q_i)$ when prompted $Q_i$, we can regard the model as having no additional tool knowledge than $f_0$, therefore never sees $t_i$. Formally, to unlearn $T_f$, we obtain a knowledge purged dataset

% \begin{equation}
%     D^0_f = \{ (t_i, Q_i, f_0(Q_i)) | \forall t_i \in T_f \}.
% \end{equation}

% As of knowledge maintenance on $T_r$, the remaining dataset $D_r = \{ T_r, Q_r, Y_r \}$ has documented enough amount of knowledge to retain $f'$'s knowledge. 

% To this end, we have constructed a new dataset with $D^0_f$, the dataset with purged knowledge of $T_f$, and $D_r$, the remaining data with original tools and query-response pairs. We then fine-tune $f$ on $ D^0_f \cup D_r $ to obtain $f'$.


% Inspired by the random labeling approach~\citep{amnesiac_2021} in classification tasks, we implement \method in the random labeling \& fine-tuning paradigm, while realizing the above properties for tool learning. The original random labeling approach consists of two steps. Firstly, for each sample in the unlearned data $(x_i, y_i) \in D_f$, replacing $y_i$ with a randomly chosen label $y'_i \neq y_i$, resulting in $D'_f$. Secondly, fine-tune $f$ the corrupted data $D'_f$ and the retained data $D_r$. During training, the knowledge on the unlearned samples is corrupted by encouraging $f'$ to mis-classify $D_f$. Meanwhile, $f'$ is trained on $D_r$ to further strengthen the knowledge that we want to retain.

% For the demonstrations $\{ Q_i, Y_i \}$ associated with tool $t_i$, we first obtain the query the vanilla model $f_0$ with $Q_i$ and obtain the responses $Y'_i = f_0(Q_i)$. Since $f_0$ has never been trained on $\{ Q_i, Y_i \}$, $Y'_i$ is a set of responses with no information on the $t_i$ coming from 
% we encourage model to provide similar responses as the vanilla model prior to tool learning $f_0$, which has never seen $D_f$.  to realize \propone, we encourage

% \paragraph{Various levels of knowledge removal}
% Knowledge removal can happen in different cases for tool learning / unlearning. 
% \begin{itemize}
%     \item do not choose any tool.
%     \item choose a different tool. 
%     \item choose the right tool but wrong arguments. 
%     \item say I don't know how to solve this task based on internal knowledge / learned tools. Depends on evaluation.
% \end{itemize}

% Note that the difference between 1) and 4) is that 1) predicts that $x_i$ requires no tool to solve. While 4) encourages the LLM to answer "I don't know.".

% Most existing tool-augmented LLMs are trained in a Supervised Fine-Tuning manner (SFT), where language modeling objective is optimized query-response pairs $(q_i, y_i)$.

% \subsection{Agent Unlearning Problem Formulation}
% \paragraph{Knowledge Unlearning} aims at unlearning knowledge shared by the entire community of a subset of agents.
% \paragraph{Individual Unlearning} aims at removing $k$ agents from the community.


\section{Experiments and Results}
    \label{sec:exp}
    \section{Experiments}
\label{sec:experiments}
In this section, we conducted experiments to address the following questions:

\begin{enumerate}[leftmargin=9mm,noitemsep]
    \item[Q1.] {
        \textbf{Recommendation performance. } 
        How effective is the personalized ranking provided by \method for multi-behavior recommendation compared to its competitors?
    }
    \item[Q2.] {
        \textbf{Ablation study.} How does each module of \method affect its recommendation performance?
    }
    \item[Q3.] {
        \textbf{Effect of hyperparameters.} 
        How do the hyperparameters of \method, such as smoothness, query fitting, and cascading alignment, influence its performance?
    }
    \item[Q4.] {
        \textbf{Convergence.} 
        Does the iterative algorithm of \method converge as the number of iterations increases?
    }
    \item[Q5.] {
        \textbf{Computational Efficiency} Does the \method yield linear computational complexity w.r.t. the number of total interactions?
    }
\end{enumerate}

\def\arraystretch{1.1} 
\setlength{\tabcolsep}{12pt}
\begin{table}[t]
\small
\caption{
    Data statistics of multi-behavior interactions.
}
\label{tab:data}
\centering
\begin{tabular}{crrrrrr}
    \hline
    \toprule
    \textbf{Dataset} & Users & Items & Views& 
    \makecell[r]{Collects$^{*}$\\or Shares$^{\dagger}$}
     & \makecell[r]{Carts$^{*}$\\or Likes$^{\dagger}$} & \makecell[r]{Buys$^{*}$\\or Follows$^{\dagger}$}\ \\
    \midrule
    \texttt{Taobao}$^{*}$ & 15,449 & 11,953 & 873,954 & - & 195,476  & 92,180 \\
    % \texttt{Beibei}$^{*}$ & 21,716 & 7,977 & 2,412,586   & - & 642,622 & 282,860 \\
    \texttt{Tenrec}$^{\dagger}$ & 27,948 & 15,440 & 1,489,997  & 13,947  & 1,914  & 1,307  \\
    \texttt{Tmall}$^{*}$ & 41,738 & 11,953 & 1,813,498  & 221,514 & 1,996  & 255,586 \\
    %\texttt{Jdata}$^{*}$ & 93,334 & 24,624 & 1,681,430   & 45,613 & 49,891 & 321,883 \\
    \bottomrule
    \hline
\end{tabular}
\begin{tablenotes}[flushleft]\footnotesize
{\item[] $\dagger$: Note that \tenrec includes shares, likes, and follows as corresponding behaviors.}
\end{tablenotes}
\end{table}

\subsection{Experimental Setting}
\label{sec:exp:setting}
We describe the setup for our experiments on multi-behavior recommendation.

\smallsection{Datasets}
We conducted experiments on three real-world multi-behavior datasets:  
\taobao~\cite{MengZGGLZZTLZ23, LeeLKSJ24xxaw}, \tenrec~\cite{abs-2210-10629, ZhangBCSYGWHH24} and \tmall~\cite{YanCGSLSL24, MengZGGLZZTLZ23} which are publicly available and have been used as standard benchmarks in previous studies~\cite{LeeLKSJ24xxaw, MengZGGLZZTLZ23, YanCGSLSL24, ZhangBCSYGWHH24, YanYCSSP23}.
The detailed statistics of the datasets are summarized in Table~\ref{tab:data}.
%
The \texttt{Tmall} dataset has four behavior types: \textit{view}, \textit{add-to-collect}, \textit{add-to-cart}, and \textit{buy}, while \texttt{Taobao}, apart from add-to-collect, consists of three types.
The behavior types of \tenrec are \textit{view}, \textit{share}, \textit{like} and \textit{follow}.
Following the previous studies~\cite{LeeLKSJ24xxaw, MengZGGLZZTLZ23, ZhangBCSYGWHH24}, we set the target behavior as \textit{buy} for $\{\taobao, \tmall\}$ and \textit{follow} for $\{\tenrec\}$, and preprocessed duplicate interactions by retaining only the earliest occurrence for each behavior.

\smallsection{Competitors}
We compared \method with traditional graph ranking methods such as \textbf{RWR}~\cite{TianJ13}, \textbf{CoHITS}~\cite{DengDLK09kqpg}, \textbf{BiRank}~\cite{HeHGKW17}, as well as single-behavior representation learning methods such as \textbf{LightGCN}~\cite{HeHDWLZW20} and \textbf{MF-BPR}~\cite{RendleRFGS12}.
Note that these methods were designed to operate on a single behavior graph. 
For these, we combined all behavior graphs into a unified graph\footnote{We performed an element-wise union operation across the adjacency matrices of all behaviors to construct the adjacency matrix of the unified graph.}, on which these methods were applied. 
Results based solely on the target behavior interactions are provided in~\ref{sec:appendix:single}.
We further compare \method with \textbf{NRank}, an extended version of \textbf{BiRank}~\cite{HeHGKW17} for multi-partite graphs, where the interactions of each behavior form a subgraph between users and items. In this graph, the set of users is shared across behaviors, while the set of items is distinct for each behavior.
We also compared our method with state-of-the-art representation learning methods for multi-behavior recommendation, including: 1) non-cascading approaches such as \textbf{MB-HGCN}~\cite{YanYCSSP23} and \textbf{MuLe}~\cite{LeeLKSJ24xxaw}, and 2) cascading approaches such as \textbf{PKEF}~\cite{MengMZYZL23} and \textbf{HEC-GCN}~\cite{yin2024hecgcn}.
Finally, we evaluated our method against \textbf{BPMR}~\cite{LiLCYLLD24}, a pattern-mining-based method that achieves state-of-the-art accuracy in the recommendation task.


\smallsection{Training and evaluation protocol}
We follow the leave-one-out setting, which has been broadly used in previous studies~\cite{MengZGGLZZTLZ23,LeeLKSJ24xxaw,ChengCHLZGP23fqvn}, where the test set consists of the last interacted item for each users. 
The second most recently interacted item for each user forms the validation set for tuning hyperparameters, while the remaining interactions are used for training.
%
In the evaluation phase, items within the test set for each user are ranked based on ranking or predicted scores by models. where its top-$k$ ranking quality is measured by HR@$k$ and NDCG@$k$~\cite{MengZGGLZZTLZ23,LeeLKSJ24xxaw,ChengCHLZGP23fqvn,ZhangBCSYGWHH24}. Note that target behavior items(i.e., buy) in the training interaction are excluded during testing.
HR@$k$ measures how often relevant items, on average, appear in the recommendation for each user.
NDCG@$k$ considers both relevance and order of relevant items in a ranking, averaged across all users.

\smallsection{Hyperparameter tuning}
For each dataset, we conducted a grid search to tune hyperparameters on the validation set and reported the test performance with the validated hyperparameters. 
The hyperparameters $\alpha$ and $\beta$ of \method are tuned in $[0,1]$ such that $0 \leq \alpha + \beta \leq 1$.
The validated values of $\alpha$ and $\beta$ for each dataset are provided in Appendix~\ref{sec:appendix:Effect of Hyperparameters_detail}.
For the cascading sequence $\mathcal{C}$, we follow a natural sequence of user behaviors based on their semantics, as used in previous studies~\cite{yin2024hecgcn, MengMZYZL23, LiuXWY00024, YanCGSLSL24, ChengCHLZGP23fqvn}.
We set $\mathcal{C}$ to $(\texttt{view}\rightarrow\texttt{collect}\rightarrow\texttt{cart}\rightarrow\texttt{buy})$ for \tmall, $(\texttt{view}\rightarrow\texttt{cart}\rightarrow\texttt{buy})$ for \taobao, and $(\texttt{view}\rightarrow\texttt{share}\rightarrow\texttt{like}\rightarrow\texttt{follow})$ for \tenrec (see Appendix~\ref{sec:appendix:Performance_for_Different_Cascading_Sequences} for results with different cascading sequences).
For each competitor, we followed the range of its hyperparameters, as reported in the corresponding paper.

\smallsection{Machine and implementation}
We used a workstation with AMD 5955WX and RTX 4090 (24GB VRAM).
Our method \method was implemented using Pytorch 2.0 in Python 3.9. 
Note that the BMPR's algorithm~\cite{LiLCYLLD24} is designed to run sequentially on a CPU and it is hard to parallelize, while other methods are easily parallelizable due to matrix operations and execute on a GPU.
For the other methods, we used their open-source implementations, with detailed information provided in Appendix~\ref{sec:appendix:competitors:information}.


\def\arraystretch{1.1} 
\setlength{\tabcolsep}{6pt}
\begin{table}[t]
\caption{
Multi-behavior recommendation performance in terms of HR@10 and NDCG@10, with the best results in bold and the second-best results underlined.
`\% impv.’ indicates the percentage improvement of the best model over the second-best model.
RL stands for representation learning, GR for graph ranking, and PM for pattern-mining methods.
\textbf{Our \method shows the best recommendation performance among all the tested methods across the datasets.}
}
\small
\label{tab:performance}
\centering
\begin{tabular}{cc|ccc|ccc}
\hline
\toprule
\multirow{2}{*}{\bf Methods} & \multirow{2}{*}{\bf Type} & \multicolumn{3}{c|}{\bf HR@10}       & \multicolumn{3}{c}{\bf NDCG@10}      \\
             & & \taobao  & \tenrec  & \tmall & \taobao  & \tenrec   & \tmall  \\ \midrule
MF-BPR  & RL & 0.0758  & 0.1244 & 0.0855 & 0.0387  & 0.0575  & 0.0423  \\
LightGCN & RL & 0.1025  & 0.1069 & 0.1162 & 0.0566  & 0.0526  & 0.0625  \\ \midrule
MB-HGCN & RL       & 0.1261  & 0.1133 & 0.1413 & 0.0666  & 0.0618  & 0.0753  \\ 
MuLe   & RL        & 0.1949  & 0.1920  & 0.2097 & 0.1128  & 0.1100  & 0.1175  \\ \midrule
PKEF   & RL        & 0.1349  & 0.0968 & 0.1222 & 0.0763  & 0.0530  & 0.0696  \\
HEC-GCN & RL       & 0.1905  & 0.2673 & 0.1806 & 0.1038  & 0.1565  & 0.1000  \\ \midrule
RWR & GR     & 0.2130  & 0.2074 & 0.2712 & 0.0988  & 0.0962  & 0.1284  \\
CoHITS & GR  & 0.2128  & 0.2074 & 0.2713 & 0.0988  & 0.0957  & 0.1284  \\
BiRank & GR   & \underline{0.3034}  & 0.2949 & \underline{0.3550} & \underline{0.1517}  & 0.1257  & \underline{0.1819}  \\ 
NRank & GR         & 0.2989  & \underline{0.4562} & 0.3477 & 0.1419  & 0.2508  & 0.1726  \\
\midrule
BPMR  & PM        & 0.2846  & 0.4286 & 0.3289 & 0.1429  & \underline{0.2579}  & 0.1598  \\ \midrule
\textbf{\method} & GR  & \textbf{0.3324}  & \textbf{0.4747} & \textbf{0.3751} & \textbf{0.1626}  & \textbf{0.2723}  & \textbf{0.1871}  \\ \midrule
\% impv.   & -     & 9.56\%  & 5.67\% & 5.67\% & 7.16\%  & 5.57\%  & 2.85\%  \\
\bottomrule
\hline
\end{tabular} 
\end{table}

\subsection{Recommendation Performance (Q1)} 
We evaluate the effectiveness of \method in multi-behavior recommendation by comparing it to its competitors.

\smallsection{Top-10 performance}
We first examine the recommendation quality of the top 10 highest-ranked items by each method, measured in terms of HR@10 and NDCG@10.
From Table~\ref{tab:performance}, we observe the following:
\begin{itemize}[leftmargin=9mm,noitemsep]
    \item{
        % overall comparison
        Our proposed \method achieves superior performance compared to its competitors across all datasets, providing \textbf{up to 9.56\% improvement in HR@10 and up to 7.16\% in NDCG@10 over the second-best model}, with the most notable gains observed on the \taobao dataset.
    }
    \item{
        The representation learning (RL) methods generally underperform compared to the graph ranking (GR) or the pattern-mining (PM) methods.
        Specifically, naive GR models on a unified graph such as BiRank surpass state-of-the-art RL models exploiting multi-behavior interactions such as MuLe and HEC-GCN on \taobao and \tmall, highlighting the RL approach’s limitations in providing accurate recommendations due to their limited expressiveness, mainly caused by over-smoothing and bias issues.
    }
    \item{
        % our apporach is better than simple graph ranking
        For the GR methods, exploiting a cascading sequence in measuring ranking scores, as in \method, is beneficial because it outperforms other GR methods, such as BiRank and NRank, which use all interactions but do not fully exploit the semantics of behaviors.
    }
    \item{
        % mention on cascading approach on RL
        It matters more to precisely encode embeddings than to use additional information, such as the cascading pattern, in the RL methods. 
        As evidence of this, HEC-GCN, a cascading method, performs better than MuLe, a non-cascading method, on \tenrec, while the opposite is observed on the other datasets.
    }
    \item {
        % mention on BPMR
        Our \method outperforms BPMR, which is recognized for achieving state-of-the-art accuracy and surpassing the RL methods. 
        The main difference is that BPMR considers only a few steps of paths, while our method accounts for the global structure of each behavior graph and the cascading effect, enabling it to generate higher-quality scores.      Furthermore, BPMR is significantly slower than \method, as discussed in Section~\ref{sec:exp:efficiency}.
    }
\end{itemize}

\begin{figure}[t]
    \centering
    %\hspace{-6mm}
    \includegraphics[width=0.95\linewidth]{All_k_score_legend.pdf}\\
    \subfigure[\taobao]{
        \hspace{-7mm}
        \includegraphics[width=0.318\linewidth]{All_k_score_taobao.pdf}
        %\hspace{-3.5mm}
        \label{fig:appendix:All_k_score:Taobao}
    }
    \subfigure[\tenrec]{
        \hspace{-3mm}
        \includegraphics[width=0.300\linewidth]{All_k_score_tenrec.pdf}
        %\hspace{-1mm}
        \label{fig:experiments:All_k_score:Tenrec}
    }
    \subfigure[\tmall]{
        \hspace{-4.5mm}
        \includegraphics[width=0.30\linewidth]{All_k_score_tmall.pdf}
        %\hspace{-3mm}
        \label{fig:appendix:All_k_score:Tmall}
    }
    
    \caption{
        \label{fig:experiments:varik}
        Recommendation performance in HR@$k$ and NDCG@$k$, where $k$ varies in \{10, 30, 50, 100, 200\}.
        \textbf{Our \method provides better ranking scores than its competitors in both metrics.}
    }
\end{figure}

\smallsection{Top-$k$ performance}
We further evaluate the ranking quality of \method by comparing it with its competitors in terms of HR@$k$ and NDCG@$k$ for various values of $k$ ranging from 10 to 200.
As shown in Figure~\ref{fig:experiments:varik}, \textbf{our \method achieves the highest ranking quality for items that test users are likely to purchase}, outperforming its competitors across all datasets.
Specifically, NDCG@$k$ of \method is the highest across all $k$, significantly outperforming the RL methods, indicating that the target items are more likely to be ranked higher in our results.
It is worth noting that HR@$k$ of the GR methods increases significantly as $k$ grows, compared to that of the RL methods, indicating that the rankings produced by the GR methods are more likely to contain the target items.
This implies that the GR methods are more suitable for generating candidate items, which can then be refined through re-ranking process~\cite{ValizadeganJZM09}.

\begin{figure}[t]
    \centering
    \includegraphics[width=1\linewidth]{ablation_study_Q2_legend.pdf}
    
    \subfigure[\taobao]{
        \hspace{-3mm}
        \includegraphics[width=0.320\linewidth]{ablation_study_Q2_taobao.pdf}
        % \hspace{-3mm}
        \label{fig:experiments:ablation:Taobao}
    }
    \subfigure[\tenrec]{
        \hspace{-3mm}
        \includegraphics[width=0.312\linewidth]{ablation_study_Q2_tenrec.pdf}
        % \hspace{-3mm}
        \label{fig:experiments:ablation:Tenrec}
    }
    \subfigure[\tmall]{
        \hspace{-3mm}
        \includegraphics[width=0.320\linewidth]{ablation_study_Q2_tmall.pdf}
        \label{fig:experiments:ablation:Tmall}
        \hspace{-3mm}
    }
    \caption{
        \label{fig:experiments:ablationstudy}
        Effect of behaviors in the cascading sequence, where \texttt{B4} is used for \method, \texttt{B7} for \tenrec, and \texttt{B8} for \tmall to produce the final ranking scores of \method.
        \textbf{Note that utilizing all behaviors in the sequence is beneficial for recommendation}, as the performance degrades when earlier behaviors are excluded from the sequence.
    }
\end{figure}


\subsection{Ablation Study (Q2)}
We investigate the effectiveness of our design choices in \method through ablation studies.

\smallsection{Effect of auxiliary behaviors in the cascading sequence}
We conducted an ablation study to verify the impact of auxiliary behaviors in the cascading sequences used in our method for each dataset.
For this experiment, we sequentially excluded each auxiliary behavior from the cascading sequence $\mathcal{C}$, where it was initially set to $\texttt{B8}: (\texttt{view}\rightarrow\texttt{collect}\rightarrow\texttt{cart}\rightarrow\texttt{buy})$ for \tmall, $\texttt{B4}:(\texttt{view}\rightarrow\texttt{cart}\rightarrow\texttt{buy})$ for \taobao, and $\texttt{B7}:(\texttt{view}\rightarrow\texttt{share}\rightarrow\texttt{like}\rightarrow\texttt{follow})$ for \tenrec.
As shown in Figure~\ref{fig:experiments:ablationstudy}, using all behaviors in the cascading sequence leads to better recommendations than the variants that exclude auxiliary behaviors.
Specifically, using only the graph of interactions of the target behavior shows the worst performance across all datasets, and the performance improves as auxiliary behaviors are added in the order of the cascading sequence.
This indicates that incorporating all auxiliary behaviors in the order of the cascading sequence is essential for achieving optimal performance.

% 2. CascadingCoHITS v.s. CascadingRank: normalization - column-norm / symmetric -> birank 논문
\def\arraystretch{1.1} 
\setlength{\tabcolsep}{6.7pt}
\begin{table}[t]
\caption{
Effect of normalization on measuring the ranking scores of \method.
\textbf{Ranking scores with symmetric normalization provides more accurate recommendation than those with column normalization.}
}
\label{tab:ablation:normalization}
\centering
\small
\begin{tabular}{c|ccc|ccc}
\hline
\toprule
\multirow{2}{*}{\bf Variants} & \multicolumn{3}{c|}{\bf HR@10}       & \multicolumn{3}{c}{\bf NDCG@10}      \\
             & \taobao  & \tenrec  & \tmall & \taobao  & \tenrec   & \tmall  \\ 
\midrule
\methodcol       & 0.2919  & 0.4439 & 0.3270  & 0.1465  & 0.2559  & 0.1605  \\
\methodsym       & \bf 0.3324  & \bf 0.4747 & \bf 0.3751  & \bf 0.1626  & \bf 0.2723  & \bf 0.1871  \\
\midrule
\% impv.                 & 13.86\% & 6.93\% & 14.74\% & 10.97\%  & 6.39\%  & 16.60\% \\ 
\bottomrule
\hline
\end{tabular} 
\end{table}

\smallsection{Effect of normalization}
We check the effect of normalization for estimating ranking scores in Equation~\eqref{eq:cascrank:vect}. 
For this experiment, we compared the following:
\begin{itemize}[leftmargin=9mm,noitemsep]
    \item {
        \methodcol: it uses column normalization on each adjacency matrix, i.e., $\Abnorm = \Ab \Dib^{-1}$ and $\Abnorm^{\top} = \Ab^{\top} \Dub^{-1}$.
    }
    \item {
        \methodsym: it uses symmetric normalization on each adjacency matrix, i.e., $\Abnorm = \Dub^{-1/2} \Ab \Dib^{-1/2}$ and $\Abnorm^{\top} =  \Dib^{-1/2} \Ab^{\top} \Dub^{-1/2}$.
    }
\end{itemize}
%
As shown in Table~\ref{tab:ablation:normalization}, the symmetric normalization achieves better recommendation performance than the column normalization, with improvements of up to 14.74\% in HR@$10$ and 16.60\% in NDCG@$10$ on the \tmall dataset.
This indicates that reducing the impact of both users and items in terms of size is more beneficial than reducing that of either one alone for scoring, especially when recommending items in long-tail distributions.

% \smallsection{Effect of query setting}
\begin{figure}[t]
    \centering
    \subfigure[$\alpha$: query fitting]{
        %\hspace{-7mm}
        \includegraphics[width=0.3145\linewidth]{hyperparam_sens_alpha_HR.png}
        \label{fig:experiments:hyper_sens_split_view:alpha}
        \hspace{-4mm}
    }
    \subfigure[$\beta$: cascading alignment]{
        %\hspace{-7mm}
        \includegraphics[width=0.3\linewidth]{hyperparam_sens_beta_HR.png}
        \label{fig:experiments:hyper_sens_split_view:beta}
        \hspace{-4mm}
        
    }
    \subfigure[$\gamma$: smoothing]{
        %\hspace{-7mm}
        \includegraphics[width=0.3\linewidth]{hyperparam_sens_gamma_HR.png}
        \label{fig:experiments:hyper_sens_split_view:gamma}
    }
    \caption{
    \label{fig:experiments:hyper_sens_split_view}
    Effect of the hyperparameters $\alpha$, $\beta$, and $\gamma$ of \method on the recommendation performance in HR@10, where $\gamma = 1 - \alpha - \beta$ is the strength of smoothing, and $\alpha$ and $\beta$ are the strengths of query fitting and cascading alignment, respectively.
    %$0 \leq \alpha + \beta \leq 1 $ and $\gamma = 1 - (\alpha  + \beta )$. Note that, All hyperparameters in our method have an impact on accuracy.
    }
\end{figure}

\subsection{Effect of Hyperparameters (Q3)}
\label{sec:experiments:hyper_sens_split_view}
We investigate the impact of the hyperparameters $\alpha$, $\beta$, and $\gamma$ in \method on the performance of multi-behavior recommendation.
For this experiment, we varied $\alpha$ and $\beta$ from 0 to 1 with a step size of 0.1, and measured HR@10 of \method with them such that $\alpha + \beta$ is between 0 and 1, where the results of all possible combinations are provided in Appendix~\ref{sec:appendix:details:hyperparams}.
For each value of a hyperparameter\footnote{For better visualization, we excluded the results when the value of each hyperparameter is $1$, as the accuracies were significantly low in all cases.}, varied in increments of 0.1, we reported the maximum accuracy it achieved along with the possible values of the others, to analyze its effect.

Figure~\ref{fig:experiments:hyper_sens_split_view} shows the effects of $\alpha$, $\beta$, and $\gamma$ on the recommendation performance.
As shown in Figure~\ref{fig:experiments:hyper_sens_split_view:beta}, the accuracy improves with an increase in the strength $\beta$ of cascading alignment across all tested datasets, highlighting the importance of leveraging cascading information\footnote{However, relying solely on the cascading information, such as setting $\beta = 1$, results in poor performance, as shown in Figure~\ref{fig:appendix:Hyper-sens-detail}.}.
The effects of $\alpha$ and $\gamma$ depend on the datasets. 
For the strength $\alpha$ of query fitting, a smaller value works better on \taobao and \tenrec, while a moderately large value performs better on \tmall.
For the strength $\gamma$ of smoothing, the accuracy remains relatively high for values between 0.1 and 0.7 on \taobao and \tenrec, whereas the accuracy significantly drops after 0.1. 
This indicates that query information is far more crucial on \tmall compared to \taobao and \tenrec, whereas the smoothing plays a more significant role on the latter datasets.

\begin{figure}[t]
    \centering
    \includegraphics[width=0.7\linewidth]{Reg-legend.pdf}\\
    \subfigure[\taobao]{
        %\hspace{-2mm}
        \includegraphics[width=0.32\linewidth]{Reg-taobao.pdf}
        \hspace{-1mm}
    }
    \subfigure[\tenrec]{
        \hspace{-3mm}
        \includegraphics[width=0.320\linewidth]{Reg-tenrec.pdf}
        \hspace{-1mm}
    }
    \subfigure[\tmall]{
        \hspace{-3mm}
        \includegraphics[width=0.32\linewidth]{Reg-tmall.pdf}
        \hspace{-1mm}
    }
    
    \caption{
        \label{fig:experiments:reg_convergence}
        Convergence analysis on the regularization values of Equation~\eqref{eq:reg:obj} and the residuals in Algorithm~\ref{alg:method}.
        \textbf{As the number of iterations increases, the values of all regularization terms and residuals decrease and eventually converge.}
    }
\end{figure}

\begin{figure}[t]
    \centering
    \subfigure[\taobao]{
        \hspace{-5mm}
        \includegraphics[width=0.302\linewidth]{Convergence-iteration-taobao.pdf}
        \hspace{-1mm}
        \label{fig:experiments:iteration:Taobao}
    }
    \subfigure[\tenrec]{
        \hspace{-3mm}
        \includegraphics[width=0.300\linewidth]{Convergence-iteration-tenrec.pdf}
        \hspace{-1mm}
        \label{fig:experiments:iteration:Tenrec}
    }
    \subfigure[\tmall]{
        \hspace{-3mm}
        \includegraphics[width=0.30\linewidth]{Convergence-iteration-tmall.pdf}
        \hspace{-1mm}
        \label{fig:experiments:iteration:Tmall}
    }
    
    \caption{
        \label{fig:experiments:iteration}
        Effect of the hyperparameters $\alpha$ and $\beta$ of \method on the number of iterations to converge, where $\epsilon$ is set to $10^{-5}$.
        \textbf{Our algorithm for \method converges for all combinations of $\alpha$ and $\beta$ such that $\alpha + \beta \in (0, 1)$, with faster convergence for larger values of $\alpha + \beta$.}
        %All combinations of $\alpha$
    }
\end{figure}

\subsection{Convergence Analysis (Q4)}
\label{sec:experiments:convergence}
In this section, we analyze the convergence of the iterative algorithm for \method.

\smallsection{Analysis on regularization and residual}
We measured the average residuals of Algorithm~\ref{alg:method} and regularization values of Equation~\eqref{eq:reg:obj} for smoothing, query, and cascading across all querying users as the number of iterations increased.
To broadly observe the convergence of these terms, we set $\alpha = 0.3$ and $\beta = 0.4$ and randomly initialized the ranking vectors in Algorithm~\ref{alg:method}.
Figure~\ref{fig:experiments:reg_convergence} shows the results of this analysis, with the left $y$-axis representing the log values of regularizations and the right $y$-axis representing the log values of residuals.
The values of all regularization terms and residuals decrease and converge as the number of iterations increases sufficiently.
This indicates that Algorithm~\ref{alg:method} ensures convergence of the residuals, and the resulting scores minimize the objective function of Equation~\eqref{eq:reg:obj}, i.e., they adhere to ranking smoothness while aligning with the querying and cascading vectors. 
Note that convergence is guaranteed for any valid value of $\alpha$ and $\beta$, as discussed in Section~\ref{sec:proposed:analysis}.

\smallsection{Analysis on the number of iterations}
We further analyzed the number of iterations to convergence for various values of $\alpha$ and $\beta$, where the threshold $\epsilon$ for convergence is set to $10^{-5}$.
As shown in Figure~\ref{fig:experiments:iteration}, all valid combinations of $\alpha$ and $\beta$ where $\alpha + \beta \in (0, 1]$ result in convergence. 
Note that larger values of $\alpha + \beta$ lead to faster convergence because $\gamma = 1 - \alpha - \beta$ becomes smaller, which shrinks the range of the eigenvalues of $\mat{S}$.

\subsection{Computational Efficiency (Q5)}
\label{sec:exp:efficiency}
%overview
We evaluated the efficiency of our proposed \method in terms of scalability and the trade-off between  accuracy and running time. 

\begin{figure}[t]
    \centering  
    \vspace{3mm}
    \subfigure[\taobao]{
        \includegraphics[width=0.45\linewidth]{SCALABILITY_Taobao.pdf}
        %\hspace{-3.5mm}
    }
    \subfigure[\tenrec]{
        %\hspace{-3mm}
        \includegraphics[width=0.4473447598\linewidth]{SCALABILITY_Tenrec.pdf}
        %\hspace{-1mm}
    }
    \caption{
        \label{fig:scalability}
        Scalability of \method.
        \textbf{Our iterative algorithm for \method scales linearly with respect to the number of edges (or interactions).}
    }
\end{figure}

\begin{figure}[t!]
    \centering
    \hspace{5mm}\includegraphics[width=0.7\linewidth]{Trade-off-legend.pdf}\vspace{-2mm}\\
    \subfigure[\taobao]{
        \hspace{-7mm}
        \includegraphics[width=0.3238536585\linewidth]{Trade-off-Taobao.pdf}
        \hspace{-1mm}
    }
    \subfigure[\tenrec]{
        \hspace{-3mm}
        \includegraphics[width=0.300\linewidth]{Trade-off-Tenrec.pdf}
        \hspace{-1mm}
        \label{fig:experiments:tradeoff:Tenrec}
    }
    \subfigure[\tmall]{
        \hspace{-3mm}
        \includegraphics[width=0.30\linewidth]{Trade-off-Tmall.pdf}
        \hspace{-3mm}
        \label{fig:experiments:tradeoff:Tmall}
    }
    
    \caption{
        \label{fig:experiments:trade_off}
        Trade-off between accuracy and running time for the graph ranking methods.
        \textbf{Note that our \method achieves the highest accuracy while maintaining competitive runtime performance compared to other methods.}
    }
\end{figure}

\smallsection{Scalability}
To assess scalability, we measured the running time of \method by varying the number of interactions, using the \taobao and \tenrec datasets, which contain a large number of interactions. 
For each dataset, we first apply the same random permutation to all adjacency matrices, and then extract principal submatrices from each by slicing the upper-left part, ensuring that the number of interactions ranges from about $10^{4.5}$ to the original number of interactions.
Since bipartite graphs usually have different row and column sizes, we applied the same ratio (i.e., from 0 to 1) to both dimensions.
Figure \ref{fig:scalability} demonstrates that the running time of \method increases linearly with the number of interactions on both datasets, consistent with the theoretical analysis in Theorem~\ref{sec:proposed:analysis_complexity}.

\smallsection{Trade-off between accuracy and running time}
%
We also analyzed the trade-off between accuracy and running time. 
Note that \method is categorized as a graph ranking method. 
Therefore, we compared it with other graph ranking methods in this experiment (refer to Appendix~\ref{sec:appendix:EfficiencyComparisonwithRepresentationLearningMethods} for a comparison with representation learning methods).
%
Figure~\ref{fig:experiments:trade_off} demonstrates that our \method achieves the highest accuracy while maintaining competitive runtime performance compared to other graph ranking methods.
The traditional graph ranking methods, such as BiRank and NRank, demonstrate either competitive speed or accuracy compared to \method, but they fail to achieve both simultaneously.
Note that BPMR, a state-of-the-art method, performs worse than the graph ranking methods including \method in running time due to its sequential CPU-based algorithmic design (i.e., it is hard to parallelize), while others leverage parallelizable matrix operations on a GPU.

\section{Conclusion}
    \label{sec:conclusion}
    In this paper, we introduce \method, an effective approach to dynamic persona modeling that leverages iterative reinforcement learning and discrepancy-based refinement to continuously enhance persona quality and predictive accuracy. Comprehensive experiments demonstrate \method’s effectiveness across diverse domains in dynamic user modeling. We hope \method marks a significant advancement in personalized applications.
% \mypara{Acknowledgment}
%    {\small
The authors would like to thank the reviewers. %for their valuable comments and helpful suggestions.
}
% \begin{comment}
% This research was supported by
% “Program for Leading Graduate Schools” of the Osaka University, Japan.
% \end{comment}
% \newpage
%
%%
%% The next two lines define the bibliography style to be used, and
%% the bibliography file.
\bibliographystyle{ACM-Reference-Format}
%% \bibliographystyle{abbrv}
\bibliography{main}

% \newpage
\clearpage

%%
%% If your work has an appendix, this is the place to put it.
\appendix
 \section*{Appendix}
 \DoToC
     \label{sec:appendix}
     \subsection{Details of Generating Solutions} 
\label{sec:appendix_1}
In Sec. \ref{sec:Answer order Augmentation}, 
We discuss how to generate step-by-step solutions through \( D = \{P, C, L\} \). Specifically, we follow these steps:

(1) For datasets that do not have first-order logic (FOL) expressions, such as RuleTaker and LogicNLI, we extract their premises and conclusions, and use GPT-4o-mini with prompts as shown in Tab. \ref{tab:100prompt_FOL} to convert them into corresponding FOL representations. FOLIO, on the other hand, already includes FOL expressions, so no conversion is required.

(2) The FOL-enhanced premises and ground truth labels are input into the model, prompting it to generate step-by-step solutions. As shown in the prompt in Tab. \ref{tab:100prompt_cot}, we add two domain-specific examples from each dataset to the prompt, requiring the model to clearly define the purpose and reasoning for each step, eventually leading to the final conclusion. The Task prompt specifies the possible values for the label. Specifically, in FOLIO, the label values are \{True, False, Unknown\}, in RuleTaker they are \{entailment, not entailment\}, and in LogicNLI they are \{entailment, neutral, self\_contradiction, contradiction\}.

(3) The model then reprocesses the generated solutions, using prompts like the one shown in Tab. \ref{tab:100prompt_DAG}, to extract the premise indices and premise step indices used in each reasoning step.

\begin{table*}[]
% \resizebox{0.95\textwidth}{!}{
\small
    \begin{tabularx}{\linewidth}{X}
    \toprule
    \color{gray}{/* \textit{Task prompt} */}\\
    Please parse the context and question into First-Order Logic formulas. Please use symbols as much as possible to express, such as \( \forall \), \( \land \), \( \rightarrow \), \( \oplus \), \( \neg \), etc. 
    \\
    \color{gray}{/* \textit{Example} */}\\
\textbf{Premises:} \\
If a cartoon character is yellow, it is from the Simpsons. \\
If a cartoon character is from Simpsons, then it is loved by children. \\
Ben is ugly or yellow. \\
Ramon being real is equivalent to Rhett being not modest and Philip being lazy. \\

\textbf{Hypothesis:} \\
James does not have lunch in the company. \\

\textbf{Premises-FOL:} \\
\(\forall x (Yellow(x) \rightarrow Simpsons(x))\)

\(\forall x (Simpsons(x) \rightarrow Loved(x))\)

\((Yellow(ben) \vee Ugly(ben))\)

\(real(Ramon) \iff (modest(Rhett) \land lazy(Philip))\)

\textbf{Hypothesis-FOL:} \\
\(\neg HasLunch(james, company)\)
    \\
    \color{gray}{/* \textit{Input} */}\\
    ---INPUT---\\
    Premises:\\
    \{\textbf{Given\_premises}\}\\
    Hypothesis:\\
    \{\textbf{Given\_hypothesis}\}\\
    ---OUTPUT---\\
    \bottomrule
    \end{tabularx}
    % }
  \caption{
    The prompt for generating First-Order Logic (FOL) expressions corresponding to natural language logical propositions.
  }
  \label{tab:100prompt_FOL}
\end{table*}


\begin{table*}[t]
% \resizebox{0.95\textwidth}{!}{
\small
    \begin{tabularx}{\linewidth}{X}
    \toprule
    \color{gray}{/* \textit{Task prompt} */}\\
    Please solve the question step by step based on First-Order Logic rules such as Modus Ponens, determine whether the hypothesis is true, false, or unknown based on these premises.
    \\
    \color{gray}{/* \textit{Example} */}\\
    \textbf{Premises:} \\
1. Walter Folger Brown was an American politician and lawyer who served as the postmaster general. \\
2. Walter Folger Brown graduated from Harvard University with a Bachelor of Arts. \\
3. While they were both in Toledo, Walter Folger Brown's father practiced law with Walter Folger Brown. \\
4. Katherin Hafer married Walter Folger Brown. \\

\textbf{Premises-FOL:} \\
1. AmericanPolitician(walterBrown) \(\land\) Lawyer(walterBrown) \(\land\) ServedAs(walterBrown, postMasterGeneral) \\
2. Graduated(walterBrown, harvard) \(\land\) GraduatedWith(walterBrown, bachelorsOfArt) \\
3. \(\exists t\) (In(walterBrown, toledo, t) \(\land\) In(walterBrownFather, toledo, t) \(\land\) PracticedLawTogether(walterBrown, walterBrownFather, t)) \\
4. Married(katherinHafer, walterBrown) \\

\textbf{Hypothesis:} \\
Walter Folger Brown was not in Toledo. \\

\textbf{Hypothesis-FOL:} \\
\(\exists t\) (\(\neg\)In(walterBrownFather, toledo, t)) \\
\textbf{Label:} \\
False \\
\textbf{Solution:} \\
\textbf{Step 1: Analyze Walter Folger Brown's presence in Toledo} \\
The third premise states that there exists a time t such that: \\
In(walterBrown, toledo, t) \(\land\) In(walterBrownFather, toledo, t) \(\land\) PracticedLawTogether(walterBrown, walterBrownFather, t). \\
This means that Walter Folger Brown and his father were both in Toledo at the same time, and they practiced law together there.\\
Thus, we have clear evidence that Walter Folger Brown was indeed in Toledo at some point. \\

\textbf{Step 2: Analyze the hypothesis's claim} \\
The hypothesis states that Walter Folger Brown was not in Toledo, represented in FOL as: \\
\(\exists t\) (\(\neg\)In(walterBrownFather, toledo, t)) \\
However, this contradicts the third premise, which explicitly states that both Walter Folger Brown and his father were in Toledo at the same time. \\
Therefore, the hypothesis that Walter Folger Brown was not in Toledo is False based on the premises. \\

\textbf{Final Hypothesis:} \\
The hypothesis "Walter Folger Brown was not in Toledo" is False.
    \\
    \color{gray}{/* \textit{Input} */}\\
    ---INPUT---\\
    Premises:\\
    \{\textbf{Given\_premises and premises-FOL}\}\\
    Hypothesis:\\
    \{\textbf{Given\_hypothesis and hypothesis-FOL}\}\\
    Label:\\
    \{\textbf{Given\_label}\}\\
    ---OUTPUT---\\
    \bottomrule
    \end{tabularx}
    % }
  \caption{
    The prompt for generating a step-by-step Chain of Thought (CoT) process based on premises, hypothesis, and label. Different datasets have different sets of labels and examples. For convenience, we only show the prompt on FOLIO here. In practice, we listed two examples, but for brevity and clarity in display, we only present one.
  }
  \label{tab:100prompt_cot}
\end{table*}


\begin{table*}[t]
% \resizebox{0.95\textwidth}{!}{
\small
    \begin{tabularx}{\linewidth}{X}
    \toprule
    \color{gray}{/* \textit{Task prompt} */}\\
    I will provide you with a description of the question and its answer, and the condition of the question is specific. The answer is done in steps. I hope you can extract the conditions and prerequisite steps used in each step of the answer. Please note that I am not asking you to regenerate the answer yourself, but rather to extract the conditions and prerequisite steps used in each step from the answer I have given you. Meanwhile, the conditions used in the steps are quite explicit, but the prerequisite steps used are quite implicit. I hope you can understand and summarize the prerequisite steps used in each step. Your answer should only include Conditions and prerequisite steps used.
    \\
    \color{gray}{/* \textit{Example} */}\\
    \textbf{Question:} \\
\textbf{Premises:} \\
1. Lana Wilson directed After Tiller, The Departure, and Miss Americana. \\
2. If a film is directed by a person, the person is a filmmaker. \\
3. After Tiller is a documentary. \\
4. The documentary is a type of film. \\
5. Lana Wilson is from Kirkland. \\
6. Kirkland is a US city. \\
7. If a person is from a city in a country, the person is from the country. \\
8. After Tiller is nominated for the Independent Spirit Award for Best Documentary.

\textbf{Premises-FOL:} \\
1. DirectedBy(afterTiller, lanaWilson) \( \land \) DirectedBy(theDeparture, lanaWilson) \\
\( \land \) DirectedBy(missAmericana, lanaWilson) \\
2. \( \forall \) x \( \forall \) y (DirectedBy(x, y) \( \rightarrow \) Filmmaker(y)) \\
3. Documentary(afterTiller) \\
4. \( \forall \) x (Documentary(x) \( \rightarrow \) Film(x)) \\
5. From(lanaWilson, kirkland) \\
6. In(kirkland, unitedStates) \\
7. \( \forall \) x \( \forall \) y \( \forall \) z ((From(x, y) \( \land \) In(y, z)) \( \rightarrow \) From(x, z)) \\
8. Nomination(afterTiller, theIndependentSpiritAwardForBestDocumentary)

\textbf{Conclusion:} \\
Miss Americana is not directed by a filmmaker from Kirkland.

\textbf{Conclusion-FOL:}
\( \neg \) \(\exists t\) x (Filmmaker(x) \( \land \) From(x, kirkland) \( \land \) DirectedBy(missAmericana, x))
    \\

\textbf{Answer:} \\
Step 1: Analyze the premises regarding Lana Wilson's role as a filmmaker \\
From premise 1, we know that Lana Wilson directed Miss Americana. Therefore, we can conclude that Lana Wilson is a filmmaker based on premise 2, which states that if a film is directed by a person, that person is a filmmaker. Thus, we have:
    DirectedBy(missAmericana, lanaWilson) \( \rightarrow \) Filmmaker(lanaWilson)

Step 2: Analyze Lana Wilson's origin \\
From premise 5, we know that Lana Wilson is from Kirkland. Therefore, we can conclude:
    From(lanaWilson, kirkland)

Step 3: Combine the information \\
Since we have established that Lana Wilson is a filmmaker and she is from Kirkland, we can conclude:
    Filmmaker(lanaWilson) \( \land \) From(lanaWilson, kirkland)

Step 4: Analyze the conclusion's claim \\
The conclusion states that Miss Americana is not directed by a filmmaker from Kirkland, represented in FOL as:
    \( \neg \) \(\exists t\) x (Filmmaker(x) \( \land \) From(x, kirkland) \( \land \) DirectedBy(missAmericana, x))

However, we have already established that Lana Wilson, who directed Miss Americana, is indeed a filmmaker from Kirkland. This directly contradicts the conclusion.

\textbf{Final Conclusion:} \\
The conclusion "Miss Americana is not directed by a filmmaker from Kirkland" is False.


\textbf{Conditions and prerequisite steps used:}\\
step 1: condition 1, condition 2\\
step 2: condition 5\\
step 3: step 1, step 2\\
step 4: step 3\\

    \color{gray}{/* \textit{Input} */}\\
    ---INPUT---\\
    Question:\\
    \{\textbf{Given\_question}\}\\
    Answer:\\
    \{\textbf{Given\_answer}\}\\
    ---OUTPUT---\\
    \bottomrule
    \end{tabularx}
    % }
  \caption{
    The prompt for extracting Conditions and prerequisite steps used in each step of step-by-step solutions.
  }
  \label{tab:100prompt_DAG}
\end{table*}



\subsection{Kendall Tau Distance}
% 不同的tau的数据示例
%4.4.1里不同的tau的比例
In our study, we investigate the effects of premise order transformations by using the Kendall tau distance \( \tau \). This coefficient measures the correlation between two ordered lists, providing a quantitative way to assess how much one order differs from another. We use \( \tau \) to categorize various permutations of premise orders and assess their impact on model performance.

The Kendall tau coefficient \( \tau \) is calculated as follows:
\[
\tau = \frac{C - D}{\binom{n}{2}}
\]
where \( C \) is the number of concordant pairs (pairs of items that are in the same relative order in both lists), and \( D \) is the number of discordant pairs (pairs that are in opposite order in both lists). The total number of possible pairs is \( \binom{n}{2} \), where \( n \) is the number of items being compared.

We divide \( \tau \) values into 10 groups, each spanning a 0.2 range within the interval [-1, 1). A \( \tau \) value of 1 indicates that the order of the premises is exactly as required for the reasoning process, while -1 indicates a complete reversal of order. A \( \tau \) value of 0 indicates that the order is completely random, with no correlation to the original sequence.

For example, if the original premise order is \( [P_1, P_2, P_3, \dots, P_n] \), a permutation function \( \sigma \) might rearrange it to \( [P_3, P_1, P_2, \dots, P_n] \). This process allows us to explore different levels of order perturbation, with the goal of analyzing how such variations affect model performance. Examples of premise orders corresponding to different \( \tau \) values can be seen in Fig. \ref{fig:tau}.
\begin{figure}[t] 
    \centering
        \includegraphics[width=0.5\textwidth]{tau_fig.pdf}
    % \captionsetup{font={small}} 
    \caption{An example of showing the arrangement of premises with different tau values. The tau values do not represent exact values but rather the closest intervals for demonstration purposes.}
    \label{fig:tau}
\end{figure}



% \subsection{TFI}
% % 不同的TFI的数据示例
% % 我们介绍了Topological Freedom Index (TFI)指数,下面我们详细的介绍
% In Sec. \ref{sec:non_DAG}, 


\end{document}
\endinput
%%
%% End of file `sample-sigconf.tex'.
