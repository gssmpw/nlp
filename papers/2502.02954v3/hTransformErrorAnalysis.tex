
Since we have shown the time discretization error in the previous section, what we remain to show is just an upper bound of $\varepsilon_{\rho,l}^2$. For that purpose, we put an additional assumption which is almost same as Assumption \ref{ass:BoundingHMainText} except the condition (iii). 
A bound of $\varepsilon_{\TV}^2$ in the third condition (iii) will be given as 
$\varepsilon_\TV^2 = \gO(\varepsilon^2 + h)$ by \cite{chen2023sampling}. 
\begin{assumption}\label{ass:BoundingH}
\begin{enumerate}[topsep=0mm,itemsep=-1mm,leftmargin = 6mm]
    \item[(i)] $\nabla_x s(\cdot,\cdot)$ is $H_s$-Lipschitz continuous in a sense that $\|\nabla_x s(x,t) - \nabla_y s(y,t)\|_{\mathrm{op}} \leq H_s \|x- y\|$
for any $x,y \in \sR^d$ and $0 \leq t \leq T$ and $\E[\| s(\bar{X}_{kh}^\leftarrow,kh)\|^2] \leq Q^2$ for any $k$.
    \item[(ii)] There exists $R > 0$ such that $\sup_{t,x}\{\|\nabla_x^2 \log p_t(x)\|_{\mathrm{op}},\|\nabla_x^2 \log s(x,t)\|_{\mathrm{op}}\} \leq R$.
    \item[(iii)] $\E_{\bar{X}_t}[\TV(\bar{X}_{T}^\leftarrow,{X}_{T}^\leftarrow|\bar{X}_{T-t}^\leftarrow = {X}_{T-t}^\leftarrow = \bar{X}_t)^2] \leq \varepsilon_{\TV}^2$ for any $t \in [0,T]$.
\end{enumerate}    
\end{assumption}

\begin{thm}\label{thm:delhDiffXYExp}
Suppose that $0 \leq h \leq \delta \leq 1/(1 + 2R)$
and Assumptions \ref{ass:BoundingH} and \ref{assumption:TVBoundMainText-2} hold. 
Let $L_\varphi$ and $R_\varphi$ be as given in Lemma \ref{lemm:phiYboundLip}.
Then, it holds that  
\begin{align}
& \E_{\bar{X}_{\cdot}^\leftarrow}[ \|\nabla_x \E[\rho_*(\bar{X}_{T}^\leftarrow) |  \bar{X}_{t}] % deleted!
% |_{x = \bar{X}_{t}^\leftarrow} 
-  \nabla_x \E[\rho_*({X}_{T}^\leftarrow) | {X}_{kh}^\leftarrow]
% |_{x = \bar{X}_{kh}^\leftarrow}
\|^2]  \\
\lesssim & 
R_\varphi^2 \left(\varepsilon^2+ \Lipdp^2 d(\delta + \mathsf{m} \delta^2) \right)
+ \Xi_{\delta,\varepsilon}
+
[ L_\varphi^2 (\mathsf{m} + 4Q^2 + d h)  
+ R_\varphi^2  (1 + 2R)^2] h^2 \\
= & \gO\left(\varepsilon^2 + \delta + \frac{\varepsilon_\TV^2}{\delta}\right), 
\end{align}
where 
\begin{align}
\Xi_{\delta,\varepsilon} & := \frac{4 c_\eta^2  C_\rho^2 (1+2R)^2}{3}\delta  
+ 2 \exp(2)
\left\{ C_\rho^2 \frac{\varepsilon_{\TV}^2}{\delta} + C R_\varphi^2  [\varepsilon^2 + \Lipdp^2 d(\delta + \mathsf{m} \delta^2)]\right\},
%&  = \gO\left(\delta + \varepsilon^2 + \frac{\varepsilon_\TV^2}{\delta}\right),
\end{align} 
and $c_\eta > 0$ is a universal constant. 
\end{thm}
By \cite{chen2023sampling}, $\varepsilon_\TV^2 = \gO(\varepsilon^2 + h)$, and thus by substituting $\delta \leftarrow \sqrt{h}$, we finally obtain an error estimate as 
$$
\left( 1 + \frac{1}{\sqrt{h}}\right) \varepsilon^2 + \sqrt{h}. 
$$
\begin{proof}
Let $t \in [kh,(k+1)h)$ and $t^* = kh + \delta$ where $\delta$ is larger than or equal to $h$: $\delta \geq h$. 
We only consider a situation where $T- t \geq \delta$. 
The situation where $\delta < T- t$ can be treated in the same manner by noticing 
a trivial relation $\nabla_x \E[\rho_*(\bar{X}_{T}^\leftarrow) | \bar{X}_{T}^\leftarrow = x]
= \nabla \rho_*(x)$.
Then, for a given initial state $x \in \sR^d$, we define the stochastic processes as 
\begin{align}
    \bar{X}_{t}^\leftarrow=x,\ \mathrm{d}\bar{X}_{\tau}^\leftarrow = \{\bar{X}_{\tau}^\leftarrow + 2\nabla \log p_{T-\tau}(\bar{X}_{\tau}^\leftarrow)\}\mathrm{d}\tau + \sqrt{2}\mathrm{d}B_\tau,
    ~~~(t \leq \tau \leq T), 
\end{align}
and its numerical approximation as  
\begin{align}
    {X}_{t}^\leftarrow=x,\ \mathrm{d}{X}_{\tau}^\leftarrow = \{{X}_{\tau}^\leftarrow + 2s({X}^\leftarrow_{k_\tau h},k_\tau h)\}\mathrm{d}\tau + \sqrt{2}\mathrm{d}B_\tau.
    ~~~(t \leq \tau \leq T), 
\end{align}
where $k_\tau$ is the integer such that $\tau \in [k_\tau h,(k_\tau +1)h)$. 

Note that 
\begin{align}
& \nabla_x \E[\rho_*(\bar{X}_{T}^\leftarrow) |  \bar{X}_{t}]
-  \nabla_x \E[\rho_*({X}_{T}^\leftarrow) | {X}_{kh}^\leftarrow] \\
=&  
\underbrace{(\nabla_x \E[\rho_*(\bar{X}_{T}^\leftarrow) |  \bar{X}_{t}]
-  \nabla_x \E[\rho_*({X}_{T}^\leftarrow) | {X}_{t}^\leftarrow])}_{(a)} 
+ \underbrace{(\nabla_x \E[\rho_*({X}_{T}^\leftarrow) | {X}_{t}^\leftarrow]
-  \nabla_x \E[\rho_*({X}_{T}^\leftarrow) | {X}_{kh}^\leftarrow])}_{(b)}. 
\label{eq:NablaRhotDecomp}
\end{align}
We first evaluate the term (a): 
$$
\nabla_x \E[\rho_*(\bar{X}_{T}^\leftarrow) | \bar{X}_{t}^\leftarrow = x] - \nabla_x \E[\rho_*({X}_{T}^\leftarrow) | {X}_{t}^\leftarrow = x]. 
$$
As we have seen above, the derivative can be expressed by the following recursive formula of the conditional expectation:  
$$
\nabla_x \E[\E[\rho_*(\bar{X}_{T}^\leftarrow) | \bar{X}_{t^*} ] | \bar{X}_{t}^\leftarrow = x].
$$
For a notation simplicity, we let $\varphi_X(x) := \E[\rho_*(\bar{X}_{T}^\leftarrow) | \bar{X}_{t^*}^\leftarrow = x]$ and $\varphi_Y(x) := \E[\rho_*({X}_{T}^\leftarrow) | {X}_{t^*}^\leftarrow = x]$. Then, the Bismut-Elworthy-Li formula \citep{MR755001,ELWORTHY1994252} yields that, for any $v \in \sR^d$,
\begin{align}
v^\top \nabla_x \E[\rho_*(\bar{X}_{0}^\leftarrow) | \bar{X}_{t}^\leftarrow = x] = 
\E\left[\frac{1}{\delta} \int_{0}^\delta \langle \eta_{\bar{X},\tau} , \mathrm{d} B_\tau \rangle  \varphi_X( \bar{X}_{t^*}^\leftarrow) ~|~\bar{X}_{t}^\leftarrow = x\right], 
\end{align}
where $\eta_{\bar{X},\tau}$ is  the solution of 
\begin{align}
& \rd \eta_{\bar{X},\tau}  = (I + 2 \nabla^2 \log p_{T- t -\tau}(\bar{X}_{t+\tau}^\leftarrow)) \eta_{\bar{X},\tau} \rd \tau,  \\
& \eta_{\bar{X},0} = v. 
\end{align} 
Similarly, we define $\eta_{{X},\tau}$ for the process ${X}_{\tau}^\leftarrow$ as
\begin{align}
& \rd \eta_{{X},\tau}  = (\eta_{{X},\tau} + 2 \nabla_x^\top s({X}_{k_\tau h}^\leftarrow,k_{\tau} h)\eta_{{X},k_\tau h - t})  \rd \tau,  \\
& \eta_{{X},0} = v. 
\end{align} 
%Note that $\eta_{{X},\tau}$ has no randomness once the initial state is conditioned as $\bar{X}_{t}^\leftarrow = {X}_{t}^\leftarrow = x$.  
Then, 
\begin{align}
& v^\top \nabla_x \E[\rho_*(\bar{X}_{T}^\leftarrow) | \bar{X}_{t}^\leftarrow = x] -  v^\top \nabla_x \E[\rho_*({X}_{T}^\leftarrow) | {X}_{t}^\leftarrow = x]  \\
 = &  \E\left[ \frac{1}{\delta} \int_{0}^\delta \langle \eta_{\bar{X},\tau} -\eta_{{X},\tau}, \mathrm{d} B_\tau \rangle  \varphi_Y( {X}_{t^*}^\leftarrow)   ~|~\bar{X}_{t}^\leftarrow = {X}_{t}^\leftarrow = x\right]  \\
 & + \E\left[ \frac{1}{\delta} \int_{0}^\delta \langle \eta_{\bar{X},\tau} , \mathrm{d} B_\tau \rangle  (\varphi_X( \bar{X}_{t^*}^\leftarrow) - \varphi_Y( {X}_{t^*}^\leftarrow))  ~|~ \bar{X}_{t}^\leftarrow = {X}_{t}^\leftarrow = x\right]. 
 \label{eq:BismutBoundFirst}
\end{align}
By the Ito isometry, the first term of the right hand side can be bounded as 
\begin{align}
& \left( \E\left[ \frac{1}{\delta} \int_{0}^\delta \langle \eta_{\bar{X},\tau} -\eta_{{X},\tau} , \mathrm{d} B_\tau \rangle  \varphi_Y( {X}_{t^*})   ~|~\bar{X}_{t}^\leftarrow = {X}_{t}^\leftarrow =x\right]\right)^2 \\
\leq & 
 \E\left[ \left(\frac{1}{\delta} \int_{0}^\delta \langle \eta_{\bar{X},\tau} -\eta_{{X},\tau} , \mathrm{d} B_\tau \rangle  \varphi_Y( {X}_{t^*}) \right)^2  ~|~\bar{X}_{t}^\leftarrow = {X}_{t}^\leftarrow =x\right] \\
\leq & 
C_\rho^2 \E\left[ \left(\frac{1}{\delta} \int_{0}^\delta \langle \eta_{\bar{X},\tau} -\eta_{{X},\tau} , \mathrm{d} B_\tau \rangle  \right)^2  ~|~\bar{X}_{t}^\leftarrow ={X}_{t}^\leftarrow = x\right] \\
= &
C_\rho^2  \E\left[ \frac{1}{h^2} \int_{\tau}^h \| \eta_{\bar{X},\tau} -\eta_{{X},\tau}\|^2 \rd \tau  ~|~\bar{X}_{t}^\leftarrow ={X}_{t}^\leftarrow = x\right] \\
\leq &
2 C_\rho^2  \E\left[ \frac{1}{\delta^2} \int_{0}^\delta (\| \eta_{\bar{X},\tau} - v\|^2 + \| \eta_{{X},\tau} - v\|^2) \rd \tau  ~|~\bar{X}_{t}^\leftarrow ={X}_{t}^\leftarrow = x\right].
\end{align}
Hence, we just need to bound $\|\eta_{\bar{X},\tau} - v\|^2$ in the right hand side. 
We note that it obeys the following differential equation:   
\begin{align}
&\frac{\rd \|\eta_{\bar{X},\tau} - v\|^2}{\rd \tau}  \\
= & 2 (\eta_{\bar{X},\tau} - v)^\top \frac{\rd \eta_{\bar{X},\tau}}{\rd \tau} \\
= & 2 (\eta_{\bar{X},\tau} - v)^\top (I + 2 \nabla^2 \log p_{T- t -\tau}(\bar{X}_{T - t -\tau}^\leftarrow)) \eta_{\bar{X},\tau} \\
= & 2 (\eta_{\bar{X},\tau} - v)^\top (I + 2 \nabla^2 \log p_{T- t -\tau}(\bar{X}_{T - t -\tau}^\leftarrow)) [(\eta_{\bar{X},\tau} - v) + v] \\
\leq & 2 (1 + 2 R) \|\eta_{\bar{X},\tau} - v\|^2 +   2 (1 + 2R) \|v\| \|\eta_{\bar{X},\tau} - v\|, 
\end{align}
which also yields that  
\begin{align}
&2 \|\eta_{\bar{X},\tau} - v\| \frac{\rd \|\eta_{\bar{X},\tau} - v\|}{\rd \tau}  \leq  2(1+ 2 R) \|\eta_{\bar{X},\tau} - v\|^2 +    2(1+ 2 R) \|v\| \|\eta_{\bar{X},\tau} - v\| \\
\Rightarrow~ 
&\frac{\rd \|\eta_{\bar{X},\tau} - v\|}{\rd \tau}  \leq   2(1+ 2 R) (\|\eta_{\bar{X},\tau} - v\| +  \|v\|)  \\
\Rightarrow ~& \|\eta_{\bar{X},\tau} - v\| \leq [\exp( 2 (1+ 2 R) \tau) - 1]\|v\| \\
\Rightarrow ~& \|\eta_{\bar{X},\tau} - v\|^2 \leq [\exp( 2 (1+ 2 R) \tau) - 1]^2\|v\|^2.
\end{align}
Therefore, if $\delta$ is sufficiently small (such as $\delta \leq 1/(1 + 2R)$), then we arrive at 
$$
\|\eta_{\bar{X},\tau} - v\|^2 \leq c_\eta (1+ 2 R)^2 \tau^2\|v\|^2,
$$
with a universal constant $c_\eta$, for any $0 \leq \tau \leq \delta$. 
In the same vein, we also have 
\begin{align}
& \|\eta_{{X},\tau} - v\|^2 \leq c_\eta (1+ 2 R)^2 \tau^2\|v\|^2,~ \\
& \|\eta_{\bar{X},\tau}\|^2 \leq \exp(4(1+2R)\tau) \|v\|^2,~
\|\eta_{{X},\tau}\|^2 \leq \exp(4(1+2R)\tau) \|v\|^2,
\end{align}
for $0 \leq \tau \leq \delta$. 
These bounds yield that  
\begin{align}
& 2 C_\rho^2  \E\left[ \frac{1}{\delta^2} \int_{0}^\delta (\| \eta_{\bar{X},\tau} - v\|^2 + \| \eta_{{X},\tau} - v\|^2) \rd \tau  ~|~\bar{X}_{t}^\leftarrow = {X}_{t}^\leftarrow =x\right] \\
\leq & 4 c_\eta^2  C_\rho^2 (1+2R)^2  \frac{\int_0^\delta \tau^2 \rd \tau}{\delta^2} = 
\frac{4 c_\eta^2  C_\rho^2 (1+2R)^2}{3}\delta. 
\end{align}
%Then, by taking integral of $0 \leq h \leq h'$, we have that the first term becomes $O(h^2)$.

Next, we bound the second term of the right hand side in \Eqref{eq:BismutBoundFirst}: 
\begin{align}
& \E\left[ \frac{1}{\delta} \int_{0}^\delta \langle \eta_{\bar{X},\tau} , \mathrm{d} B_\tau \rangle  (\varphi_X( \bar{X}_{t^*}^\leftarrow) - \varphi_Y( {X}_{t^*}^\leftarrow))  ~|~ \bar{X}_{t}^\leftarrow = {X}_{t}^\leftarrow = x\right]^2 \\
\leq &
\E\left[ \frac{1}{\delta^2} \int_{0}^\delta \|\eta_{\bar{X},\tau}\|^2 \rd \tau  ~|~ \bar{X}_{t}^\leftarrow = {X}_{t}^\leftarrow = x\right] 
\E\left[ (\varphi_X( \bar{X}_{t^*}^\leftarrow) - \varphi_Y( {X}_{t^*}^\leftarrow))^2  ~|~ \bar{X}_{t}^\leftarrow = {X}_{t}^\leftarrow = x\right] \\
\leq &
\frac{\exp(2(1 + 2R)\delta)\|v\|^2}{\delta}
\E\left[ (\varphi_X( \bar{X}_{t^*}^\leftarrow) - \varphi_Y( {X}_{t^*}^\leftarrow))^2  ~|~ \bar{X}_{t}^\leftarrow = {X}_{t}^\leftarrow = x\right]. 
\label{eq:varPhiXYdiff}
\end{align}
Otherwise, we also have the following inequality:  
\begin{align}
& \E\left[ \frac{1}{\delta} \int_{0}^\delta \langle \eta_{\bar{X},\tau} , \mathrm{d} B_\tau \rangle  (\varphi_X( \bar{X}_{t^*}^\leftarrow) - \varphi_Y( {X}_{t^*}^\leftarrow))  ~|~ \bar{X}_{t}^\leftarrow = {X}_{t}^\leftarrow = x\right]^2 \\
\leq & 2 \E\left[ \frac{1}{\delta} \int_{0}^\delta \langle \eta_{\bar{X},\tau} , \mathrm{d} B_\tau \rangle  (\varphi_Y( \bar{X}_{t^*}^\leftarrow) -\varphi_Y( {X}_{t^*}^\leftarrow)  )  ~|~ \bar{X}_{t}^\leftarrow = {X}_{t}^\leftarrow = x\right]^2 \\ & 
+ 2 \E\left[  \eta_{\bar{X},\delta}^\top \nabla (\varphi_X( \bar{X}_{t^*}^\leftarrow)) - \varphi_Y( \bar{X}_{t^*}^\leftarrow))  ~|~ \bar{X}_{t}^\leftarrow = {X}_{t}^\leftarrow = x\right]^2 \\
\leq &
\frac{\exp(2(1 + 2R)\delta)\|v\|^2}{\delta}
\E\left[ (\varphi_Y( \bar{X}_{t^*}^\leftarrow) - \varphi_Y( {X}_{t^*}^\leftarrow))^2  ~|~ \bar{X}_{t}^\leftarrow = {X}_{t}^\leftarrow = x\right] \\
& + 2 \E\left[ \| \eta_{\bar{X},\delta} \| \|\nabla (\varphi_X( \bar{X}_{t^*}^\leftarrow)) - \varphi_Y( \bar{X}_{t^*}^\leftarrow))\|  ~|~ \bar{X}_{t}^\leftarrow = {X}_{t}^\leftarrow = x\right]^2.
\label{eq:varPhiXYdiffSecond}
\end{align}

For bounding these quantities, we need to bound the discrepancy $\|\bar{X}_{\tau}^\leftarrow - {X}_{\tau}^\leftarrow\|^2$.  
Note that this quantity follows the following ODE: 
\begin{align}
& \frac{\rd \|\bar{X}_{\tau}^\leftarrow - {X}_{\tau}^\leftarrow\|^2 }{\rd \tau} \\
= & 2(\bar{X}_{\tau}^\leftarrow - {X}_{\tau}^\leftarrow)^\top [(\bar{X}_{\tau}^\leftarrow - 2 \nabla_x \log p_{T-\tau-t}(\bar{X}_{\tau}^\leftarrow)) - 
({X}_{\tau}^\leftarrow - 2 s(x,kh))]  \\
= & 2\|\bar{X}_{\tau}^\leftarrow - {X}_{\tau}^\leftarrow\|^2 
- 4(\bar{X}_{\tau}^\leftarrow - {X}_{\tau}^\leftarrow)^\top 
(\nabla_x \log p_{T-\tau-t}(\bar{X}_{\tau}^\leftarrow) - s(x,kh))  \\
% = & 2\|\bar{X}_{\tau}^\leftarrow - {X}_{\tau}^\leftarrow\|^2 
% - 4(\bar{X}_{\tau}^\leftarrow - {X}_{\tau}^\leftarrow)^\top 
% [\nabla_x \log(p_{T-\tau-t}(\bar{X}_{\tau}^\leftarrow)/p_{T-t-\tau}(\bar{X}_{t}^\leftarrow)) + 
% \nabla_x \log(p_{T-\tau-t}(\bar{X}_{t}^\leftarrow)/p_{T-t}(\bar{X}_{t}^\leftarrow)) \\
% & +
% (\nabla \log (p_{T-t}(\bar{X}_{t}^\leftarrow)-s(x,kh))] \\
\leq 
& 4\|\bar{X}_{\tau}^\leftarrow - {X}_{\tau}^\leftarrow\|^2  + 2 \|\nabla_x \log(p_{T-\tau-t}(\bar{X}_{\tau}^\leftarrow))-s(x,kh)\|^2.
\end{align}
Therefore, it satisfies that 
\begin{align}
\|\bar{X}_{\tau}^\leftarrow - {X}_{\tau}^\leftarrow\|^2
\leq 
4 \int_0^\tau  \|\bar{X}_{s}^\leftarrow - {X}_{s}^\leftarrow\|^2 \rd s 
+ 
2 \int_0^\tau \|\nabla_x \log(p_{T-\tau-t}(\bar{X}_{s}^\leftarrow))-s(x,kh)\|^2 \rd s.
\end{align}
Taking its expectation, we see that 
\begin{align}
\E[\|\bar{X}_{\tau}^\leftarrow - {X}_{\tau}^\leftarrow\|^2]
\leq 
4 \int_0^\tau  \E[\|\bar{X}_{s}^\leftarrow - {X}_{s}^\leftarrow\|^2] \rd s 
+ 
2 \underbrace{\int_0^\tau (\varepsilon^2 + \Lipdp^2 d s + \Lipdp^2 \mathsf{m} s^2) \rd s}_{=\gO(\varepsilon^2 \tau + \Lipdp^2 d (\tau^2 + \mathsf{m} \tau^3)) =: \xi(\tau)},
\end{align}
where we used Theorem 10 (and its proof) of \cite{chen2023improved} for obtaining $\xi(\tau)$. 
Then, Gronwall inequality yields 
\begin{align}
\E[\|\bar{X}_{\tau}^\leftarrow - {X}_{\tau}^\leftarrow\|^2]
\leq \xi(\tau) + \int 4 \xi(s) e^{4(\tau -s)} \rd s \lesssim \varepsilon^2 \tau + \Lipdp^2 d(\tau^2 + \mathsf{m} \tau^3),
\end{align}
(see \cite{Mischeler:Note:2019} for example). 
Then, the Lipschitz continuity of $\varphi_Y$ (Lemma \ref{lemm:phiYboundLip}) yields that 
\begin{align}
& \E\left[ (\varphi_Y( \bar{X}_{t^*}^\leftarrow) - \varphi_Y( {X}_{t^*}^\leftarrow))^2  ~|~ \bar{X}_{t}^\leftarrow = {X}_{t}^\leftarrow = x\right] 
\lesssim  L_\varphi^2 [\varepsilon^2 \tau + \Lipdp^2 d(\tau^2 + \mathsf{m} \tau^3)]. %(\varepsilon^2 \tau + \tau^2),
\end{align}


\paragraph{Bound for $t = kh$:} 
First, we show a bound for $t = kh$. %$\delta =h$. 
The right hand side of \Eqref{eq:varPhiXYdiff} with $\delta = h$ can be bounded by 
\begin{align}
& \frac{\exp(2(1 + 2R)\delta)\|v\|^2}{\delta}
\E_{\bar{X}_{t}^\leftarrow}\left[ \E\left[ (\varphi_X( \bar{X}_{t^*}^\leftarrow) - \varphi_Y( {X}_{t^*}^\leftarrow))^2  ~|~ \bar{X}_{t}^\leftarrow = {X}_{t}^\leftarrow \right] \right]\\
\leq &
 \frac{\exp(2(1 + 2R)\delta)\|v\|^2}{\delta}
2 \E_{\bar{X}_{t}^\leftarrow}\left[\E\left[ (\varphi_X( \bar{X}_{t^*}^\leftarrow) - 
\varphi_Y( \bar{X}_{t^*}^\leftarrow) )^2
+ (\varphi_Y( \bar{X}_{t^*}^\leftarrow) 
- \varphi_Y( {X}_{t^*}^\leftarrow))^2  ~|~ \bar{X}_{t}^\leftarrow = {X}_{t}^\leftarrow \right] \right]\\
\leq &
 \frac{\exp(2(1 + 2R)\delta)\|v\|^2}{\delta}
2\left\{ C_\rho^2 \varepsilon_{\TV}^2 + R_\varphi^2
\E_{\bar{X}_{t}^\leftarrow}\left[\E\left[ 
(\bar{X}_{t^*}^\leftarrow  
- {X}_{t^*}^\leftarrow)^2  ~|~ \bar{X}_{t}^\leftarrow = {X}_{t}^\leftarrow \right] \right] \right\},
\end{align}
where we used 
\begin{align}
 \E\left[ (\varphi_X( \bar{X}_{t^*}^\leftarrow) - 
\varphi_Y( \bar{X}_{t^*}^\leftarrow) )^2 \right]
\leq & C_\rho^2\E_{\bar{X}_{T-t^*}}\left[\TV(\bar{X}_{T}^\leftarrow,{X}_{T}^\leftarrow | {X}_{t^*}^\leftarrow = \bar{X}_{t^*}^\leftarrow = \bar{X}_{T-t^*})^2 \right]  \\
\leq & C_\rho^2 \varepsilon_{\TV}.
\end{align}
Here, by using Theorem 10 of \cite{chen2023sampling} again, 
the right hand side can be bounded as  
\begin{align}
& 2 \frac{\exp(2(1 + 2R)\delta)\|v\|^2}{\delta}
\left\{ C_\rho^2\varepsilon_{\TV}^2 + C R_\varphi^2  [\varepsilon^2 \delta + \Lipdp^2 d(\delta^2 + \mathsf{m} \delta^3)]\right\} \\
= & 
2 \exp(2(1 + 2R)\delta)\|v\|^2
\left\{ C_\rho^2\frac{\varepsilon_{\TV}^2}{\delta} + C R_\varphi^2 [\varepsilon^2 + \Lipdp^2 d(\delta + \mathsf{m} \delta^2)]\right\}.
\end{align}
Then, with the contraint $h \leq 1/(1+2R)$, it can be further simplified as 
$$
2 \exp(2)\|v\|^2
\left\{ C_\rho^2\frac{\varepsilon_{\TV}^2}{\delta} + C R_\varphi^2 [\varepsilon^2 + \Lipdp^2 d(\delta + \mathsf{m} \delta^2)]\right\}.
$$
Therefore, by taking maximum with respect to $v \in \sR^d$ with a constraint $\|v\| =1$, 
\begin{align}
& \E_{ \bar{X}_{t}^\leftarrow }[ \|\nabla_x \E[\rho_*(\bar{X}_{T}^\leftarrow) | \bar{X}_{t}^\leftarrow ] - \nabla_x \E[\rho_*({X}_{T}^\leftarrow) | {X}_{t}^\leftarrow =  \bar{X}_{t}^\leftarrow ]\|^2 ] \\
\leq & \frac{4 c_\eta^2  C_\rho^2 (1+2R)^2}{3}\delta  
+ 2 \exp(2)
\left\{ C_\rho^2\frac{\varepsilon_{\TV}^2}{\delta} + C R_\varphi^2  [\varepsilon^2 + \Lipdp^2 d(\delta + \mathsf{m} \delta^2)]\right\} =: \Xi_{\delta,\varepsilon}.
\label{eq:DeltavphiXwithh}
\end{align}
We see that $\Xi_{\delta,\varepsilon} = \gO(\delta + \varepsilon^2 + \varepsilon_\TV^2/\delta)$. 

\paragraph{Bound for general $t \in (kh,(k+1)h)$:} 
In this setting, we utilize the inequality \eqref{eq:varPhiXYdiffSecond}. 
Using the constraint $\delta \leq 1/(1 + 2R)$ and $\|v\|=1$, the right hand side of \eqref{eq:varPhiXYdiffSecond} can be bounded by 
\begin{align}
&   \frac{\exp(2)}{\delta}
\E\left[ (\varphi_Y( \bar{X}_{t^*}^\leftarrow) - \varphi_Y( {X}_{t^*}^\leftarrow))^2  ~|~ \bar{X}_{t}^\leftarrow = {X}_{t}^\leftarrow = x\right] \\
& + 2 \E\left[ \exp(2) \|\nabla (\varphi_X( \bar{X}_{t^*}^\leftarrow)) - \varphi_Y( \bar{X}_{t^*}^\leftarrow))\|  ~|~ \bar{X}_{t}^\leftarrow = {X}_{t}^\leftarrow = x\right]^2 \\ 
\leq & 
\frac{\exp(2)}{\delta} 
R_\varphi^2 \E\left[ (\bar{X}_{t^*}^\leftarrow -  {X}_{t^*}^\leftarrow)^2  ~|~ \bar{X}_{t}^\leftarrow = {X}_{t}^\leftarrow = x\right]
+ 2 \exp(2) \Xi_{\delta,\varepsilon}~~~(\because \Eqref{eq:DeltavphiXwithh}). 
\end{align}
By taking the expectation with respect to $x = \bar{X}_{t}^\leftarrow$, we arrive at 
\begin{align} 
& \E_{\bar{X}_{t}^\leftarrow}[ \|\nabla_x \E[\rho_*(\bar{X}_{T}^\leftarrow) |  \bar{X}_{t}^\leftarrow] 
% |_{x = \bar{X}_{t}^\leftarrow}
-  \nabla_x \E[\rho_*({X}_{T}^\leftarrow) | {X}_{t}^\leftarrow] 
%|_{x = \bar{X}_{t}^\leftarrow}
\|^2] \\
\leq & C \frac{\exp(2)}{\delta} 
R_\varphi^2 \left(\varepsilon^2 \delta  + \Lipdp^2 d(\delta^2 + \mathsf{m} \delta^3) \right)
+ 2 \exp(2) \Xi_{\delta,\varepsilon} \\
\leq & 
C \exp(2)
R_\varphi^2 \left(\varepsilon^2+ \Lipdp^2 d(\delta + \mathsf{m} \delta^2) \right)
+ 2 \exp(2) \Xi_{\delta,\varepsilon}. 
    \label{eq:phyYboundwithphiX}
\end{align}
This gives an upper bound of the term (a) in \Eqref{eq:NablaRhotDecomp}. 
Then, we just need to bound the remaining term (b) in \Eqref{eq:NablaRhotDecomp}: 
\begin{align}
    \E_{\bar{X}_{t}^\leftarrow,\bar{X}_{kh}^\leftarrow}[ \| \nabla_x \E[\rho_*({X}_{T}^\leftarrow) | {X}_{t}^\leftarrow] 
    %|_{x = \bar{X}_{t}^\leftarrow} 
    -  \nabla_x \E[\rho_*({X}_{T}^\leftarrow) | {X}_{kh}^\leftarrow ] 
    %|_{x = \bar{X}_{kh}^\leftarrow}
    \|^2].
\end{align}
For that purpose, we define $\varphi_{Y,t}(x) = \E[\rho_*({X}_{T}^\leftarrow)| {X}_{t}^\leftarrow = x]$.
Then, using the Bismut-Elworthy-Li formula again, 
\begin{align}
&    v^\top (\nabla \varphi_{Y,t}(x) - \nabla \varphi_{Y,kh}(x)) \\ 
= &  
v^\top \nabla \varphi_{Y,t}(x) - 
\E[ \eta_{{X},h(k+1)-t}^\top \nabla \varphi_{Y,t}({X}_{t}^\leftarrow) |
{X}_{kh}^\leftarrow = x] \\
= &   
\E[ v^\top( \nabla \varphi_{Y,t}(x) - \nabla \varphi_{Y,t}({X}_{t}^\leftarrow))
+ 
(\eta_{{X},(k+1)h - t}^\top - v ) \nabla \varphi_{Y,t}({X}_{t}^\leftarrow) |
{X}_{kh}^\leftarrow = x] \\
\leq &   
\E[ L_\varphi \|x -{X}_{t}^\leftarrow \|
+ 
R_\varphi \|\eta_{{X},(k+1)h-t}^\top - v \|  \mid
{X}_{(k+1)h - t}^\leftarrow = x]  \\
\leq & 
\E[ L_\varphi \| ((k+1)h - t)(x - 2 s(x,kh)) + \sqrt{(h - \delta})B_{kh} \|
+ 
R_\varphi \|\eta_{{X},(k+1)h - t}^\top - v \|  \mid
{X}_{kh}^\leftarrow = x],
\end{align}
which yields that 
\begin{align}
&  \E_{\bar{X}_{\cdot}^\leftarrow}[\|\nabla \varphi_{Y,t}(\bar{X}_{t}^\leftarrow) - \nabla \varphi_{Y,kh}(\bar{X}_{(k+1)h - t}^\leftarrow) \|^2 ] \\
\leq 
& 2 L_\varphi^2((k+1)h - t)^2 \E_{\bar{X}_{\cdot}^\leftarrow}[ \|\bar{X}_{kh}^\leftarrow\|^2
+ 4\| s(\bar{X}_{kh}^\leftarrow,kh)\|^2 + d ((k+1)h - t)]  \\
&
+ R_\varphi^2 c_\eta (1 + 2R)^2 ((k+1)h - t)^2 \\
\leq 
& 
2 L_\varphi^2 h^2 (\mathsf{m} + 4Q^2 + d h)  
+ R_\varphi^2 c_\eta (1 + 2R)^2 h^2 \\
= & 
[2 L_\varphi^2 (\mathsf{m} + 4Q^2 + d h)  
+ R_\varphi^2 c_\eta (1 + 2R)^2] h^2. 
\label{eq:phyYdifftbound}
\end{align}
Combining \eqref{eq:phyYboundwithphiX} and \eqref{eq:phyYdifftbound} gives the assertion. 
\end{proof}



\begin{lem}\label{lemm:phiYboundLip}
Suppose that
$\sup_x\|\nabla \rho_*(x)\|\leq R_\rho$, $\|\nabla \rho_*(x) - \nabla \rho_*(y)\| \leq L_\rho\|x-y\|~(\forall x,y)$, and 
$\nabla_x s(\cdot,\cdot)$ is $H_s$-Lipschitz continuous with respect to $x$. 
Let $\varphi_{Y,t}(x) = \E[\rho_*({X}_{T}^\leftarrow)| {X}_{t}^\leftarrow = x]$. 
Then, $\nabla_x \varphi_{Y,t}(x)$ is bounded by $R_\varphi$ and $L_\varphi$-Lipschitz continuous for any $0 \leq t \leq T$, where  
\begin{align}
R_\varphi &=  \max\{C_\rho 2\sqrt{(1+2R) e},e^{1/2} R_\rho\}, 
%,~L_\varphi = \left(2 C_\eta^2 H_s^2 R^2 + e^2 L_\rho^2 \right)^{1/2}.
\\
L_\varphi &= \max\left\{\left( \frac{2 C_\eta^2 H_s^2 C_\rho^2}{1+2R} + 
2 (1 + 2R) \exp(6)R_\varphi^2 \right)^{1/2},
\left(2 C_\eta^2 H_s^2 R^2 + e^2 L_\rho^2 \right)^{1/2}
\right\}, 
\end{align}
for a universal constant $C_\eta > 0$. 
\end{lem}
\begin{proof}
We show it only when $t = kh$ for a positive integer $k$ just for simplicity. The proof for a general $t$ can be obtained in the same manner. 

(i) First, we assume that $T - t \geq 1/4(1 + 2R)$. 
In the following, we let $v \in \sR^d$ be an arbitrary vector with $\|v\| =1$. 
We again utilize the Bismut-Elworthy-Li formula: 
\begin{align} 
& v^\top \nabla \varphi_{Y,t}(x) \\
& = v^\top \nabla_x \E[\rho_*({X}_{T}^\leftarrow) \mid {X}_{t}^\leftarrow = x]   \\
& =  \E\left[ \frac{1}{S}\int_{0}^S \langle \eta_\tau, \rd B_\tau \rangle \varphi_{Y,S}({X}_{S}^\leftarrow) \mid {X}_{t}^\leftarrow = x\right].
\end{align}
Hence, 
\begin{align} 
& (v^\top \nabla \varphi_{Y,t}(x))^2 \\
\leq & 
 C_\rho^2 \E\left[ \frac{1}{S^2} \left( \int_{0}^S \langle \eta_\tau, \rd B_\tau \rangle \right)^2
 \mid {X}_{t}^\leftarrow = x\right]  \\
\leq & 
 C_\rho^2 \E\left[ \frac{1}{S^2} \int_{0}^S \|\eta_\tau\|^2 \rd \tau 
 \mid {X}_{t}^\leftarrow = x\right].
\end{align}
Here, we know that $\|\eta_\tau\|^2 \leq \exp(4(1+2R)\tau) \|v\|^2$, and thus 
\begin{align} 
& (v^\top \nabla \varphi_{Y,t}(x))^2 \leq C_\rho^2 \frac{1}{S} \exp(4(1+2R)S) \|v\|^2.
\end{align}
Hence, by taking $S = \frac{1}{4(1+2R)}$, we have that 
$$
(v^\top \nabla \varphi_{Y,t}(x))^2 \leq C_\rho^2 4(1+2R) e \|v\|^2. 
$$
This shows that $\|\nabla \varphi_{Y,t}(x)\|$ is bounded by $R_\varphi =  C_\rho 2\sqrt{(1+2R) e}$. 

Next, we show its Lipschitz continuity. 
For that purpose, we define two stochastic processes 
\begin{align}
{X}_{t}^\leftarrow=x,\ \mathrm{d}{X}_{\tau}^\leftarrow = \{{X}_{\tau}^\leftarrow + 2s({X}_{kh}^\leftarrow,kh)\}\mathrm{d}\tau + \sqrt{2}\mathrm{d}B_\tau~~(\tau \in [kh,k(h+1)]), \\
\tilde{Z}_{t}^\leftarrow=y,\ \mathrm{d}\tilde{Z}_{\tau}^\leftarrow = \{\tilde{Z}_{\tau}^\leftarrow + 2s(\tilde{Z}_{kh}^\leftarrow,kh)\}\mathrm{d}\tau + \sqrt{2}\mathrm{d}B_\tau~~(\tau \in [kh,k(h+1)]),
\end{align}
where $x,y \in \sR^d$ with $\|x - y\| \leq \varepsilon$.
Accordingly, we also define 
\begin{align}
& \eta_{{X},0} = v,~~\frac{\rd \eta_{{X},\tau}}{\rd \tau}  = (I + 2 \nabla_x^\top s({X}_{kh}^\leftarrow,kh)) \eta_{{X},\tau},  \\
&  \eta_{Z,0} = v,~~\frac{\rd \eta_{Z,\tau}}{\rd \tau}  = (I + 2 \nabla_x^\top s(\tilde{Z}_{kh}^\leftarrow,kh)) \eta_{Z,\tau}.
\end{align} 
Thus, 
\begin{align}
{X}_{(k+1)h}^\leftarrow - \tilde{Z}_{(k+1)h}^\leftarrow
= 
{X}_{kh}^\leftarrow - \tilde{Z}_{k h}^\leftarrow
+ h [{X}_{kh}^\leftarrow - \tilde{Z}_{k h}^\leftarrow + 2(s({X}_{kh}^\leftarrow,kh) - s(\tilde{Z}_{kh}^\leftarrow,kh))],
\end{align}
which yields 
\begin{align}
\|{X}_{(k+1)h}^\leftarrow - \tilde{Z}_{(k+1)h}^\leftarrow\| 
\leq & 
(1 + h(1 + R)) \|{X}_{kh}^\leftarrow - \tilde{Z}_{k h}^\leftarrow\| \\
\leq & 
(1 + h(1 + R))^{k+1} \|x - y\|.
\end{align}
Now, we assume $k \leq S/h$ so that we have 
$\|{X}_{(k+1)h}^\leftarrow - \tilde{Z}_{(k+1)h}^\leftarrow\| 
\leq \exp(S(1+R)) \|x - y\|$ for $k=1,\dots,S/h$. 
Hence, 
\begin{align}
& \frac{\rd (\eta_{{X},\tau} - \eta_{Z,\tau})}{\rd \tau}  \\
=& ( \eta_{{X},\tau} -  \eta_{Z,\tau}) + 
2 \nabla_x^\top s({X}_{kh}^\leftarrow,kh)) \eta_{{X},\tau} - 2 \nabla_x^\top s(\tilde{Z}_{kh}^\leftarrow,kh)) \eta_{Z,\tau} \\
=& ( \eta_{{X},\tau} -  \eta_{Z,\tau}) + 
2 (\nabla_x^\top s({X}_{kh}^\leftarrow,kh) - \nabla_x^\top s(\tilde{Z}_{kh}^\leftarrow,kh)) 
\eta_{{X},\tau} - 
2 \nabla_x^\top s(\tilde{Z}_{kh}^\leftarrow,kh)(\eta_{Z,\tau} -  \eta_{{X},\tau}),
\end{align} 
which also yields that 
\begin{align}
& \frac{\rd \|\eta_{{X},\tau} - \eta_{Z,\tau}\|^2}{\rd \tau}  \\
=& 2 \| \eta_{{X},\tau} -  \eta_{Z,\tau}\|^2 + 
4 (\eta_{{X},\tau} -  \eta_{Z,\tau})^\top (\nabla_x^\top s({X}_{kh}^\leftarrow,kh) - \nabla_x^\top s(\tilde{Z}_{kh}^\leftarrow,kh)) 
\eta_{{X},\tau} \\
& - 
4 (\eta_{{X},\tau} -  \eta_{Z,\tau})^\top \nabla_x^\top s(\tilde{Z}_{kh}^\leftarrow,kh)(\eta_{Z,\tau} -  \eta_{{X},\tau}) \\
\leq & 2 \| \eta_{{X},\tau} -  \eta_{Z,\tau}\|^2 + 
4 \|\eta_{{X},\tau} -  \eta_{Z,\tau}\| H_s \exp(S(1+R)) \varepsilon  
\exp(2(1+2R)S) \\
& +  
4 R \|\eta_{{X},\tau} -  \eta_{Z,\tau}\|^2. 
\end{align} 
Therefore, 
\begin{align}
& \frac{\rd \|\eta_{{X},\tau} - \eta_{Z,\tau}\|}{\rd \tau}  \\
\leq & (1 + 2R) \left[ \| \eta_{{X},\tau} -  \eta_{Z,\tau}\| + 
2 H_s \exp(S(1+R))   
\exp(2(1+2R)S) \varepsilon / (1 + 2R) \right],
\end{align}
and thus by noticing $\|\eta_{{X},0} - \eta_{Z,0}\| = 0$, we have 
\begin{align}
\|\eta_{{X},\tau} - \eta_{Z,\tau}\| \leq 
[\exp(S (1 + 2R)) -1] \frac{2 H_s\exp(S(1+R))   
\exp(2(1+2R)S)  }{1 + 2R} \varepsilon,
\end{align}
for any $\tau \leq S$. Then, by setting $S = 1/(1+2R)$, the right hand side can be rewritten as 
\begin{align}
\|\eta_{{X},\tau} - \eta_{Z,\tau}\| \leq 
C_\eta  \frac{H_s}{1 + 2R} \varepsilon,
\end{align}
for an absolute constant $C_\eta$. 
Therefore, we arrive at 
\begin{align} 
& (v^\top (\nabla \varphi_{Y,t}(x) - \nabla \varphi_{Y,t}(y)))^2 \\
=  & \E\left[ \frac{1}{S}\int_{0}^S \langle \eta_{{X},\tau}, \rd B_\tau \rangle \varphi_{Y,S}({X}_{S}^\leftarrow) 
-
\frac{1}{S}\int_{0}^S \langle \eta_{Z,\tau}, \rd B_\tau \rangle \varphi_{Y,S}(\tilde{Z}_{S}^\leftarrow)
\right]^2 \\
\leq & 
  2 \E\left[ \frac{1}{S^2} \int_{0}^S (\eta_{{X},\tau} - \eta_{Z,\tau})^2 \rd \tau \right] 
  \E\left[\varphi_{Y,S}({X}_{S}^\leftarrow)^2\right]  \\
&   + 
   2  \E\left[ \frac{1}{S^2} \int_{0}^S \eta_{Z,\tau}^2 \rd \tau \right] 
  \E\left[(\varphi_{Y,S}({X}_{S}^\leftarrow) - \varphi_{Y,S}(\tilde{Z}_{S}^\leftarrow))^2\right] \\
\leq & 
\frac{2}{S} C_\eta^2  \frac{H_s^2}{(1 + 2R)^2} \varepsilon^2 \cdot C_\rho^2 
 + \frac{2}{S}  
 \exp(4(1+2R)S) \|v\|^2 
  R_\varphi^2 \exp(2 S(1+R)) \varepsilon^2.
\end{align}
Then, for the choice of $S = 1/(1+2R)$, the right hand side can be bounded by 
$$
\left( \frac{2 C_\eta^2 H_s^2 C_\rho^2}{1+2R} + 
2 (1 + 2R) \exp(6) R_\varphi^2 \right) 
\varepsilon^2. 
$$
This implies that $\nabla \varphi_{Y,t}(\cdot)$ is Lipschitz continuous with a constant $L_\varphi = \left( \frac{2 C_\eta^2 H_s^2 C_\rho^2}{1+2R} + 
2 (1 + 2R) \exp(6)R_\varphi^2 \right)^{1/2}$. 

(ii) Next, we assume that $T - t \leq S = 1/4(1 + 2R)$. 
In this situation, we may use the following relation: 
\begin{align} 
& v^\top \nabla \varphi_{Y,t}(x)  = \E[\eta_\tau^{T-t} \nabla \rho_*({X}_{T}^\leftarrow) \mid {X}_{t}^\leftarrow = x]. 
\end{align}
And, tracing an analogous argument by replacing $\varphi_{Y,S}$ with $h$, we obtain the assertion with 
$$
R_\varphi = e^{1/2} R_\rho,~L_\varphi = \left(2 C_\eta^2 H_s^2 R^2 + e^2 L_\rho^2 \right)^{1/2}.
$$
\end{proof}


\begin{lem}\label{lem:hhdashDiff}
If $\|\rho_* - \rho\|_\infty \leq \varepsilon'$, then % and $\|\nabla h- \nabla h'\|_\infty \leq  \varepsilon'$
%\footnote{The condition $\|\nabla h- \nabla h'\|_\infty \leq  \varepsilon'$ can be omitted. However we will have a bound of $\frac{e}{\sqrt{\min\{T-t,1/(2+2R)\}}}\varepsilon'$ instead}, then  
 $$\|\nabla_x \E[\rho_*({X}_{T}^\leftarrow) | {X}_{t}^\leftarrow = x] - 
 \nabla_x \E[\rho({X}_{T}^\leftarrow)  | {X}_{t}^\leftarrow = x] \| \leq 
 \frac{e}{\sqrt{\min\{T-t,1/(2+2R)\}}} \varepsilon'.$$   
\end{lem}
\begin{proof}
It can be proved by the Bismut-Elworthy-Li formula again. We omit the details. 
\end{proof}

% Noticing that ${X}_{(k+1)h} = h ({X}_{kh} - 2 s({X}_{kh},kh)) + \sqrt{2} \xi_{k}$ where $\xi_k \sim N(0,I)$, 
% %the Bismut-Elworthy-Li formula yields   
% a simple calculation yields that 
% $$\nabla_x \E[\rho({X}_{T}^\leftarrow) | {X}_{t}^\leftarrow = x]
% = \E_{\xi_k}\left[ \varphi'_{Y,(k+1)h}({X}_{kh}^\leftarrow - 2 s({X}_{kh}^\leftarrow,kh) + \sqrt{2} \xi_{k}) 
% \frac{(I - 2 \nabla^\top s(x,kh)|_{x = {X}_{kh}^\leftarrow} )}{\sqrt{e^{2h}-1}}  \xi_{k}
% \right],
% $$
% where $\varphi'_{Y,t}(y) = \E[\rho({X}_{T}^{\leftarrow}) | {X}_{t}^{\leftarrow} = y]$. 
% We notice that the empirical estimate of $\nabla_x \E[\rho_*({X}_{T}^\leftarrow) | {X}_{t}^\leftarrow = x]$ is obtained by replacing the expectation by a sample average. 
% Then, since $\|I - 2 \nabla^\top s(x,kh)|_{x = {X}_{kh}^\leftarrow}\|_{\mathrm{op}} \leq (1 + 2R)$ and $\|\rho\|_\infty \leq C_\rho$, we see that,
% for an i.i.d. sequence $({X}_i)_{i=1}^n$ obtained by simulating ${X}_T^{\leftarrow}$, the sub-Gaussian concentration inequality gives that 
% \begin{align}
% & \nabla_x \E[\rho({X}_{T}^\leftarrow) | {X}_{kh}^\leftarrow = x]
% - 
% \frac{1}{n} \sum_{i=1}^n \left[ 
% h'({X}_{i}) 
% \frac{(I - 2 \nabla^\top s(x,kh)|_{x = {X}_{kh}^\leftarrow} )}{\sqrt{e^{2h}-1}}  \xi_{k,i} 
% \right] \\ 
% & \leq C C_\rho (1 + 2R) \sqrt{\frac{\log(\delta)}{n (e^{2h} -1)}}
% \lesssim   C_\rho (1 + 2R) \sqrt{\frac{\log(\delta^{-1})}{n h}},  
% \label{eq:EmpiricalhDiff}
% \end{align}
% with probability $1- \delta$,
% where $\xi_{k,i}$ is the standard normal random variable to generate a sample of ${X}_{(k+1)h}^{\leftarrow}$ and $C$ is a universal constant. 
% This bound can be uniform over all $k=1,\dots,T/h$ by replacing $\delta$ with $\delta/(T/h)$.



Combining all inequalities, we arrive at (the formal version of) Theorem~\ref{thm:Diffusion-2}. 
\begin{thm}[Formal statement of Theorem~\ref{thm:Diffusion-2}] \label{thm:ustaruDiffFinal}
Assume that Assumptions \ref{ass:BoundingH} and \ref{assumption:TVBoundMainText-2} hold 
and the conditions in Lemma \ref{lemm:phiYboundLip} are satisfied. 
%the same condition as Theorem \ref{thm:delhDiffXYExp} holds 
and $\|\rho_* - \rho\|_\infty \leq \varepsilon'$ and $\|\rho\|_\infty \leq C_\rho$. 
Let $L_\varphi$ and $R_\varphi$ be as given in Lemma \ref{lemm:phiYboundLip}.
Then, for $0 \leq h \leq \delta \leq 1/(1 + 2R)$, we have that 
\begin{align}
& \E_{\bar{Y}_{\cdot}^{\leftarrow}}[ \|u_*(\bar{Y}_{t}^\leftarrow,t) -  u(\bar{Y}_{k_t h}^\leftarrow,t) \|^2]  \\
\lesssim & 
C_\rho^3 \left\{ R_\varphi^2 \left(\varepsilon^2+ \Lipdp^2 d(\delta + \mathsf{m} \delta^2) \right)
+ \Xi_{\delta,\varepsilon}
+
[ (R_\varphi^2 + L_\varphi^2) (\mathsf{m} + 4Q^2 + d h)  
+ R_\varphi^2  (1 + 2R)^2] h^2\right\} \\
& +  \frac{e^2}{\min\{T-t,1/(2+2R)\}} \varepsilon'^2 +  C_\rho (1 + 2R) \sqrt{\frac{\log(T/(h\delta))}{n h}}, 
%= & \gO\left(\varepsilon^2 + \delta + \frac{\varepsilon_\TV^2}{\delta}\right), 
\end{align}
where 
\begin{align}
\Xi_{\delta,\varepsilon} & := \frac{4 c_\eta^2  C_\rho^2 (1+2R)^2}{3}\delta  
+ 2 \exp(2)
\left\{ C_\rho^2 \left(R_\varphi^2 + \frac{1}{\delta}\right)\varepsilon_{\TV}^2 + C R_\varphi^2  [\varepsilon^2 + \Lipdp^2 d(\delta + \mathsf{m} \delta^2)]\right\},
%&  = \gO\left(\delta + \varepsilon^2 + \frac{\varepsilon_\TV^2}{\delta}\right),
\end{align} 
and $c_\eta > 0$ is a universal constant. 
\end{thm}

\begin{proof}
Define 
$$
\rho_{*,t}(x) = \E[\rho_*(\bar{X}_{T}^\leftarrow) |  \bar{X}_{t}^\leftarrow = x],~\rho_{t}(x) = \E[\rho(X_{T}^\leftarrow) |  {X}_{t}^\leftarrow = x].
$$
First, note that 
\begin{align}
\| u^*(x,t) - u(x,t) \|^2= & 
\left\| \frac{ \nabla \rho_{*,t}(x) - \nabla \rho_{t}(x)
}{ \rho_{*,t}(x) } + \frac{\nabla \rho_{t}(x) (\rho_{*,t}(x) -\rho_{t}(x))
}{ \rho_{*,t}(x) \rho_t(x) } \right\|^2 \\
\leq & 
2 \left\| \frac{ \nabla \rho_{*,t}(x) - \nabla \rho_{t}(x)
}{ \rho_{*,t}(x) }\right\|^2  +
2 \left\| \frac{\nabla \rho_{t}(x) (\rho_{*,t}(x) -\rho_{t}(x))
}{ \rho_{*,t}(x) \rho_t(x) } \right\|^2 \\ 
\leq & 2 C_\rho^2 \left\| \nabla \rho_{*,t}(x) - \nabla \rho_{t}(x)\right\|^2  +
 2 \frac{\left\| \nabla \rho_{t}(x) \right\|^2 |\rho_{*,t}(x) -\rho_{t}(x)|^2 }{ (\rho_{*,t}(x) \rho_t(x))^2 }  \\
\leq & 2 C_\rho^2 \left\| \nabla \rho_{*,t}(x) - \nabla \rho_{t}(x)\right\|^2  +
 2 R_\varphi^2 C_\rho^2 |\rho_{*,t}(x) -\rho_{t}(x)|^2. 
\end{align}
Therefore, the expectation of the right hand side with respect to $\bar{X}_t^\leftarrow$ can be bounded by 
\begin{align}
& \E_{\bar{X}_t^\leftarrow} \left[ \| u^*(\bar{X}_t^\leftarrow,t) - u(\bar{X}_t^\leftarrow,t) \|^2\right]  \\ 
\leq &
2C_\rho^2 \E_{\bar{X}_t^\leftarrow} \left[  \left\| \nabla \rho_{*,t}(x) - \nabla \rho_{t}(x) \right\|^2 \right]
+ 2 R_\varphi C_\rho^2  \E_{\bar{X}_t^\leftarrow}\left[ |\rho_{*,t}(x) -\rho_{t}(x)|^2 \right] \\
\leq &
2C_\rho^2 \E_{\bar{X}_t^\leftarrow} \left[  \left\| \nabla \rho_{*,t}(x) - \nabla \rho_{t}(x) \right\|^2 \right]
+ 2 R_\varphi C_\rho^2  
\E_{\bar{X}_t^\leftarrow} \left[  \TV(\bar{X}_T^\leftarrow,X_T^\leftarrow | \bar{X}_t^\leftarrow = X_t^\leftarrow =x) |_{x  = \bar{X}_t^\leftarrow}^2  \right].
\end{align}
The first term of the right hand side can be bounded by Theorem \ref{thm:delhDiffXYExp} and Lemma \ref{lem:hhdashDiff}.  
The second term can be bounded by $\varepsilon_\TV^2$ by Assumption \ref{ass:BoundingH}. 

In the same vein, we can bound the difference 
\begin{align}
&  \| u(x,t) - u(x,k_th) \|^2 \\
\leq &
2C_\rho^2 \E_{\bar{X}_t^\leftarrow} \left[  \left\| \nabla \rho_{t}(x) - \nabla \rho_{k_t h}(x) \right\|^2 \right]
+ 2 R_\varphi C_\rho^2  
\E_{\bar{X}_t^\leftarrow} \left[ |\rho_{t}(x) - \rho_{k_t h}(x)|  \right] \\
\leq &
[2 L_\varphi^2 (\mathsf{m} + 4Q^2 + d h)  
+ R_\varphi^2 c_\eta (1 + 2R)^2] h^2 \\
& +  2 R_\varphi^2 h^2 (\mathsf{m} + 4Q^2 + d h)   \\
\leq & 
[2 (L_\varphi^2+R_\varphi^2) (\mathsf{m} + 4Q^2 + d h)  
+ R_\varphi^2 c_\eta (1 + 2R)^2] h^2, 
\end{align}
where we used \eqref{eq:phyYdifftbound} and the same argument as \eqref{eq:phyYdifftbound} in the last inequality with $R_\varphi$ Lipschitz continuity of $\rho_{t}$. 


Finally, we convert the expectation w.r.t. $\bar{X_t}^\leftarrow$ to that w.r.t. $\bar{X_t}^\leftarrow$. However, the density ratio between $p_t$ and $q_t$ is bounded by $C_\rho$,
which yields the assertion. 
%The proof can be accomplished by combining Theorem \ref{thm:delhDiffXYExp}, Lemma \ref{lem:hhdashDiff} and \Eqref{eq:EmpiricalhDiff}. 
%Only thing we should notice is that the density ratio between $q_t$ and $p_t$ is bounded by $C_\rho$, which is required to transform the expectation w.r.t $\bar{X}_t$ to $\bar{Y}_t$. 
\end{proof}