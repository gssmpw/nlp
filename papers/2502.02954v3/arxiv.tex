% !TEX program = pdflatex
\pdfoutput=1
\documentclass{article} % For LaTeX2e
\usepackage{arxiv,times}

% Optional math commands from https://github.com/goodfeli/dlbook_notation.
%%%%% NEW MATH DEFINITIONS %%%%%

\usepackage{amsmath,amsfonts,bm}
\usepackage{derivative}
% Mark sections of captions for referring to divisions of figures
\newcommand{\figleft}{{\em (Left)}}
\newcommand{\figcenter}{{\em (Center)}}
\newcommand{\figright}{{\em (Right)}}
\newcommand{\figtop}{{\em (Top)}}
\newcommand{\figbottom}{{\em (Bottom)}}
\newcommand{\captiona}{{\em (a)}}
\newcommand{\captionb}{{\em (b)}}
\newcommand{\captionc}{{\em (c)}}
\newcommand{\captiond}{{\em (d)}}

% Highlight a newly defined term
\newcommand{\newterm}[1]{{\bf #1}}

% Derivative d 
\newcommand{\deriv}{{\mathrm{d}}}

% Figure reference, lower-case.
\def\figref#1{figure~\ref{#1}}
% Figure reference, capital. For start of sentence
\def\Figref#1{Figure~\ref{#1}}
\def\twofigref#1#2{figures \ref{#1} and \ref{#2}}
\def\quadfigref#1#2#3#4{figures \ref{#1}, \ref{#2}, \ref{#3} and \ref{#4}}
% Section reference, lower-case.
\def\secref#1{section~\ref{#1}}
% Section reference, capital.
\def\Secref#1{Section~\ref{#1}}
% Reference to two sections.
\def\twosecrefs#1#2{sections \ref{#1} and \ref{#2}}
% Reference to three sections.
\def\secrefs#1#2#3{sections \ref{#1}, \ref{#2} and \ref{#3}}
% Reference to an equation, lower-case.
\def\eqref#1{equation~\ref{#1}}
% Reference to an equation, upper case
\def\Eqref#1{Equation~\ref{#1}}
% A raw reference to an equation---avoid using if possible
\def\plaineqref#1{\ref{#1}}
% Reference to a chapter, lower-case.
\def\chapref#1{chapter~\ref{#1}}
% Reference to an equation, upper case.
\def\Chapref#1{Chapter~\ref{#1}}
% Reference to a range of chapters
\def\rangechapref#1#2{chapters\ref{#1}--\ref{#2}}
% Reference to an algorithm, lower-case.
\def\algref#1{algorithm~\ref{#1}}
% Reference to an algorithm, upper case.
\def\Algref#1{Algorithm~\ref{#1}}
\def\twoalgref#1#2{algorithms \ref{#1} and \ref{#2}}
\def\Twoalgref#1#2{Algorithms \ref{#1} and \ref{#2}}
% Reference to a part, lower case
\def\partref#1{part~\ref{#1}}
% Reference to a part, upper case
\def\Partref#1{Part~\ref{#1}}
\def\twopartref#1#2{parts \ref{#1} and \ref{#2}}

\def\ceil#1{\lceil #1 \rceil}
\def\floor#1{\lfloor #1 \rfloor}
\def\1{\bm{1}}
\newcommand{\train}{\mathcal{D}}
\newcommand{\valid}{\mathcal{D_{\mathrm{valid}}}}
\newcommand{\test}{\mathcal{D_{\mathrm{test}}}}

\def\eps{{\epsilon}}


% Random variables
\def\reta{{\textnormal{$\eta$}}}
\def\ra{{\textnormal{a}}}
\def\rb{{\textnormal{b}}}
\def\rc{{\textnormal{c}}}
\def\rd{{\textnormal{d}}}
\def\re{{\textnormal{e}}}
\def\rf{{\textnormal{f}}}
\def\rg{{\textnormal{g}}}
\def\rh{{\textnormal{h}}}
\def\ri{{\textnormal{i}}}
\def\rj{{\textnormal{j}}}
\def\rk{{\textnormal{k}}}
\def\rl{{\textnormal{l}}}
% rm is already a command, just don't name any random variables m
\def\rn{{\textnormal{n}}}
\def\ro{{\textnormal{o}}}
\def\rp{{\textnormal{p}}}
\def\rq{{\textnormal{q}}}
\def\rr{{\textnormal{r}}}
\def\rs{{\textnormal{s}}}
\def\rt{{\textnormal{t}}}
\def\ru{{\textnormal{u}}}
\def\rv{{\textnormal{v}}}
\def\rw{{\textnormal{w}}}
\def\rx{{\textnormal{x}}}
\def\ry{{\textnormal{y}}}
\def\rz{{\textnormal{z}}}

% Random vectors
\def\rvepsilon{{\mathbf{\epsilon}}}
\def\rvphi{{\mathbf{\phi}}}
\def\rvtheta{{\mathbf{\theta}}}
\def\rva{{\mathbf{a}}}
\def\rvb{{\mathbf{b}}}
\def\rvc{{\mathbf{c}}}
\def\rvd{{\mathbf{d}}}
\def\rve{{\mathbf{e}}}
\def\rvf{{\mathbf{f}}}
\def\rvg{{\mathbf{g}}}
\def\rvh{{\mathbf{h}}}
\def\rvu{{\mathbf{i}}}
\def\rvj{{\mathbf{j}}}
\def\rvk{{\mathbf{k}}}
\def\rvl{{\mathbf{l}}}
\def\rvm{{\mathbf{m}}}
\def\rvn{{\mathbf{n}}}
\def\rvo{{\mathbf{o}}}
\def\rvp{{\mathbf{p}}}
\def\rvq{{\mathbf{q}}}
\def\rvr{{\mathbf{r}}}
\def\rvs{{\mathbf{s}}}
\def\rvt{{\mathbf{t}}}
\def\rvu{{\mathbf{u}}}
\def\rvv{{\mathbf{v}}}
\def\rvw{{\mathbf{w}}}
\def\rvx{{\mathbf{x}}}
\def\rvy{{\mathbf{y}}}
\def\rvz{{\mathbf{z}}}

% Elements of random vectors
\def\erva{{\textnormal{a}}}
\def\ervb{{\textnormal{b}}}
\def\ervc{{\textnormal{c}}}
\def\ervd{{\textnormal{d}}}
\def\erve{{\textnormal{e}}}
\def\ervf{{\textnormal{f}}}
\def\ervg{{\textnormal{g}}}
\def\ervh{{\textnormal{h}}}
\def\ervi{{\textnormal{i}}}
\def\ervj{{\textnormal{j}}}
\def\ervk{{\textnormal{k}}}
\def\ervl{{\textnormal{l}}}
\def\ervm{{\textnormal{m}}}
\def\ervn{{\textnormal{n}}}
\def\ervo{{\textnormal{o}}}
\def\ervp{{\textnormal{p}}}
\def\ervq{{\textnormal{q}}}
\def\ervr{{\textnormal{r}}}
\def\ervs{{\textnormal{s}}}
\def\ervt{{\textnormal{t}}}
\def\ervu{{\textnormal{u}}}
\def\ervv{{\textnormal{v}}}
\def\ervw{{\textnormal{w}}}
\def\ervx{{\textnormal{x}}}
\def\ervy{{\textnormal{y}}}
\def\ervz{{\textnormal{z}}}

% Random matrices
\def\rmA{{\mathbf{A}}}
\def\rmB{{\mathbf{B}}}
\def\rmC{{\mathbf{C}}}
\def\rmD{{\mathbf{D}}}
\def\rmE{{\mathbf{E}}}
\def\rmF{{\mathbf{F}}}
\def\rmG{{\mathbf{G}}}
\def\rmH{{\mathbf{H}}}
\def\rmI{{\mathbf{I}}}
\def\rmJ{{\mathbf{J}}}
\def\rmK{{\mathbf{K}}}
\def\rmL{{\mathbf{L}}}
\def\rmM{{\mathbf{M}}}
\def\rmN{{\mathbf{N}}}
\def\rmO{{\mathbf{O}}}
\def\rmP{{\mathbf{P}}}
\def\rmQ{{\mathbf{Q}}}
\def\rmR{{\mathbf{R}}}
\def\rmS{{\mathbf{S}}}
\def\rmT{{\mathbf{T}}}
\def\rmU{{\mathbf{U}}}
\def\rmV{{\mathbf{V}}}
\def\rmW{{\mathbf{W}}}
\def\rmX{{\mathbf{X}}}
\def\rmY{{\mathbf{Y}}}
\def\rmZ{{\mathbf{Z}}}

% Elements of random matrices
\def\ermA{{\textnormal{A}}}
\def\ermB{{\textnormal{B}}}
\def\ermC{{\textnormal{C}}}
\def\ermD{{\textnormal{D}}}
\def\ermE{{\textnormal{E}}}
\def\ermF{{\textnormal{F}}}
\def\ermG{{\textnormal{G}}}
\def\ermH{{\textnormal{H}}}
\def\ermI{{\textnormal{I}}}
\def\ermJ{{\textnormal{J}}}
\def\ermK{{\textnormal{K}}}
\def\ermL{{\textnormal{L}}}
\def\ermM{{\textnormal{M}}}
\def\ermN{{\textnormal{N}}}
\def\ermO{{\textnormal{O}}}
\def\ermP{{\textnormal{P}}}
\def\ermQ{{\textnormal{Q}}}
\def\ermR{{\textnormal{R}}}
\def\ermS{{\textnormal{S}}}
\def\ermT{{\textnormal{T}}}
\def\ermU{{\textnormal{U}}}
\def\ermV{{\textnormal{V}}}
\def\ermW{{\textnormal{W}}}
\def\ermX{{\textnormal{X}}}
\def\ermY{{\textnormal{Y}}}
\def\ermZ{{\textnormal{Z}}}

% Vectors
\def\vzero{{\bm{0}}}
\def\vone{{\bm{1}}}
\def\vmu{{\bm{\mu}}}
\def\vtheta{{\bm{\theta}}}
\def\vphi{{\bm{\phi}}}
\def\va{{\bm{a}}}
\def\vb{{\bm{b}}}
\def\vc{{\bm{c}}}
\def\vd{{\bm{d}}}
\def\ve{{\bm{e}}}
\def\vf{{\bm{f}}}
\def\vg{{\bm{g}}}
\def\vh{{\bm{h}}}
\def\vi{{\bm{i}}}
\def\vj{{\bm{j}}}
\def\vk{{\bm{k}}}
\def\vl{{\bm{l}}}
\def\vm{{\bm{m}}}
\def\vn{{\bm{n}}}
\def\vo{{\bm{o}}}
\def\vp{{\bm{p}}}
\def\vq{{\bm{q}}}
\def\vr{{\bm{r}}}
\def\vs{{\bm{s}}}
\def\vt{{\bm{t}}}
\def\vu{{\bm{u}}}
\def\vv{{\bm{v}}}
\def\vw{{\bm{w}}}
\def\vx{{\bm{x}}}
\def\vy{{\bm{y}}}
\def\vz{{\bm{z}}}

% Elements of vectors
\def\evalpha{{\alpha}}
\def\evbeta{{\beta}}
\def\evepsilon{{\epsilon}}
\def\evlambda{{\lambda}}
\def\evomega{{\omega}}
\def\evmu{{\mu}}
\def\evpsi{{\psi}}
\def\evsigma{{\sigma}}
\def\evtheta{{\theta}}
\def\eva{{a}}
\def\evb{{b}}
\def\evc{{c}}
\def\evd{{d}}
\def\eve{{e}}
\def\evf{{f}}
\def\evg{{g}}
\def\evh{{h}}
\def\evi{{i}}
\def\evj{{j}}
\def\evk{{k}}
\def\evl{{l}}
\def\evm{{m}}
\def\evn{{n}}
\def\evo{{o}}
\def\evp{{p}}
\def\evq{{q}}
\def\evr{{r}}
\def\evs{{s}}
\def\evt{{t}}
\def\evu{{u}}
\def\evv{{v}}
\def\evw{{w}}
\def\evx{{x}}
\def\evy{{y}}
\def\evz{{z}}

% Matrix
\def\mA{{\bm{A}}}
\def\mB{{\bm{B}}}
\def\mC{{\bm{C}}}
\def\mD{{\bm{D}}}
\def\mE{{\bm{E}}}
\def\mF{{\bm{F}}}
\def\mG{{\bm{G}}}
\def\mH{{\bm{H}}}
\def\mI{{\bm{I}}}
\def\mJ{{\bm{J}}}
\def\mK{{\bm{K}}}
\def\mL{{\bm{L}}}
\def\mM{{\bm{M}}}
\def\mN{{\bm{N}}}
\def\mO{{\bm{O}}}
\def\mP{{\bm{P}}}
\def\mQ{{\bm{Q}}}
\def\mR{{\bm{R}}}
\def\mS{{\bm{S}}}
\def\mT{{\bm{T}}}
\def\mU{{\bm{U}}}
\def\mV{{\bm{V}}}
\def\mW{{\bm{W}}}
\def\mX{{\bm{X}}}
\def\mY{{\bm{Y}}}
\def\mZ{{\bm{Z}}}
\def\mBeta{{\bm{\beta}}}
\def\mPhi{{\bm{\Phi}}}
\def\mLambda{{\bm{\Lambda}}}
\def\mSigma{{\bm{\Sigma}}}

% Tensor
\DeclareMathAlphabet{\mathsfit}{\encodingdefault}{\sfdefault}{m}{sl}
\SetMathAlphabet{\mathsfit}{bold}{\encodingdefault}{\sfdefault}{bx}{n}
\newcommand{\tens}[1]{\bm{\mathsfit{#1}}}
\def\tA{{\tens{A}}}
\def\tB{{\tens{B}}}
\def\tC{{\tens{C}}}
\def\tD{{\tens{D}}}
\def\tE{{\tens{E}}}
\def\tF{{\tens{F}}}
\def\tG{{\tens{G}}}
\def\tH{{\tens{H}}}
\def\tI{{\tens{I}}}
\def\tJ{{\tens{J}}}
\def\tK{{\tens{K}}}
\def\tL{{\tens{L}}}
\def\tM{{\tens{M}}}
\def\tN{{\tens{N}}}
\def\tO{{\tens{O}}}
\def\tP{{\tens{P}}}
\def\tQ{{\tens{Q}}}
\def\tR{{\tens{R}}}
\def\tS{{\tens{S}}}
\def\tT{{\tens{T}}}
\def\tU{{\tens{U}}}
\def\tV{{\tens{V}}}
\def\tW{{\tens{W}}}
\def\tX{{\tens{X}}}
\def\tY{{\tens{Y}}}
\def\tZ{{\tens{Z}}}


% Graph
\def\gA{{\mathcal{A}}}
\def\gB{{\mathcal{B}}}
\def\gC{{\mathcal{C}}}
\def\gD{{\mathcal{D}}}
\def\gE{{\mathcal{E}}}
\def\gF{{\mathcal{F}}}
\def\gG{{\mathcal{G}}}
\def\gH{{\mathcal{H}}}
\def\gI{{\mathcal{I}}}
\def\gJ{{\mathcal{J}}}
\def\gK{{\mathcal{K}}}
\def\gL{{\mathcal{L}}}
\def\gM{{\mathcal{M}}}
\def\gN{{\mathcal{N}}}
\def\gO{{\mathcal{O}}}
\def\gP{{\mathcal{P}}}
\def\gQ{{\mathcal{Q}}}
\def\gR{{\mathcal{R}}}
\def\gS{{\mathcal{S}}}
\def\gT{{\mathcal{T}}}
\def\gU{{\mathcal{U}}}
\def\gV{{\mathcal{V}}}
\def\gW{{\mathcal{W}}}
\def\gX{{\mathcal{X}}}
\def\gY{{\mathcal{Y}}}
\def\gZ{{\mathcal{Z}}}

% Sets
\def\sA{{\mathbb{A}}}
\def\sB{{\mathbb{B}}}
\def\sC{{\mathbb{C}}}
\def\sD{{\mathbb{D}}}
% Don't use a set called E, because this would be the same as our symbol
% for expectation.
\def\sF{{\mathbb{F}}}
\def\sG{{\mathbb{G}}}
\def\sH{{\mathbb{H}}}
\def\sI{{\mathbb{I}}}
\def\sJ{{\mathbb{J}}}
\def\sK{{\mathbb{K}}}
\def\sL{{\mathbb{L}}}
\def\sM{{\mathbb{M}}}
\def\sN{{\mathbb{N}}}
\def\sO{{\mathbb{O}}}
\def\sP{{\mathbb{P}}}
\def\sQ{{\mathbb{Q}}}
\def\sR{{\mathbb{R}}}
\def\sS{{\mathbb{S}}}
\def\sT{{\mathbb{T}}}
\def\sU{{\mathbb{U}}}
\def\sV{{\mathbb{V}}}
\def\sW{{\mathbb{W}}}
\def\sX{{\mathbb{X}}}
\def\sY{{\mathbb{Y}}}
\def\sZ{{\mathbb{Z}}}

% Entries of a matrix
\def\emLambda{{\Lambda}}
\def\emA{{A}}
\def\emB{{B}}
\def\emC{{C}}
\def\emD{{D}}
\def\emE{{E}}
\def\emF{{F}}
\def\emG{{G}}
\def\emH{{H}}
\def\emI{{I}}
\def\emJ{{J}}
\def\emK{{K}}
\def\emL{{L}}
\def\emM{{M}}
\def\emN{{N}}
\def\emO{{O}}
\def\emP{{P}}
\def\emQ{{Q}}
\def\emR{{R}}
\def\emS{{S}}
\def\emT{{T}}
\def\emU{{U}}
\def\emV{{V}}
\def\emW{{W}}
\def\emX{{X}}
\def\emY{{Y}}
\def\emZ{{Z}}
\def\emSigma{{\Sigma}}

% entries of a tensor
% Same font as tensor, without \bm wrapper
\newcommand{\etens}[1]{\mathsfit{#1}}
\def\etLambda{{\etens{\Lambda}}}
\def\etA{{\etens{A}}}
\def\etB{{\etens{B}}}
\def\etC{{\etens{C}}}
\def\etD{{\etens{D}}}
\def\etE{{\etens{E}}}
\def\etF{{\etens{F}}}
\def\etG{{\etens{G}}}
\def\etH{{\etens{H}}}
\def\etI{{\etens{I}}}
\def\etJ{{\etens{J}}}
\def\etK{{\etens{K}}}
\def\etL{{\etens{L}}}
\def\etM{{\etens{M}}}
\def\etN{{\etens{N}}}
\def\etO{{\etens{O}}}
\def\etP{{\etens{P}}}
\def\etQ{{\etens{Q}}}
\def\etR{{\etens{R}}}
\def\etS{{\etens{S}}}
\def\etT{{\etens{T}}}
\def\etU{{\etens{U}}}
\def\etV{{\etens{V}}}
\def\etW{{\etens{W}}}
\def\etX{{\etens{X}}}
\def\etY{{\etens{Y}}}
\def\etZ{{\etens{Z}}}

% The true underlying data generating distribution
\newcommand{\pdata}{p_{\rm{data}}}
\newcommand{\ptarget}{p_{\rm{target}}}
\newcommand{\pprior}{p_{\rm{prior}}}
\newcommand{\pbase}{p_{\rm{base}}}
\newcommand{\pref}{p_{\rm{ref}}}

% The empirical distribution defined by the training set
\newcommand{\ptrain}{\hat{p}_{\rm{data}}}
\newcommand{\Ptrain}{\hat{P}_{\rm{data}}}
% The model distribution
\newcommand{\pmodel}{p_{\rm{model}}}
\newcommand{\Pmodel}{P_{\rm{model}}}
\newcommand{\ptildemodel}{\tilde{p}_{\rm{model}}}
% Stochastic autoencoder distributions
\newcommand{\pencode}{p_{\rm{encoder}}}
\newcommand{\pdecode}{p_{\rm{decoder}}}
\newcommand{\precons}{p_{\rm{reconstruct}}}

\newcommand{\laplace}{\mathrm{Laplace}} % Laplace distribution

\newcommand{\E}{\mathbb{E}}
\newcommand{\Ls}{\mathcal{L}}
\newcommand{\R}{\mathbb{R}}
\newcommand{\emp}{\tilde{p}}
\newcommand{\lr}{\alpha}
\newcommand{\reg}{\lambda}
\newcommand{\rect}{\mathrm{rectifier}}
\newcommand{\softmax}{\mathrm{softmax}}
\newcommand{\sigmoid}{\sigma}
\newcommand{\softplus}{\zeta}
\newcommand{\KL}{D_{\mathrm{KL}}}
\newcommand{\Var}{\mathrm{Var}}
\newcommand{\standarderror}{\mathrm{SE}}
\newcommand{\Cov}{\mathrm{Cov}}
% Wolfram Mathworld says $L^2$ is for function spaces and $\ell^2$ is for vectors
% But then they seem to use $L^2$ for vectors throughout the site, and so does
% wikipedia.
\newcommand{\normlzero}{L^0}
\newcommand{\normlone}{L^1}
\newcommand{\normltwo}{L^2}
\newcommand{\normlp}{L^p}
\newcommand{\normmax}{L^\infty}

\newcommand{\parents}{Pa} % See usage in notation.tex. Chosen to match Daphne's book.

\DeclareMathOperator*{\argmax}{arg\,max}
\DeclareMathOperator*{\argmin}{arg\,min}

\DeclareMathOperator{\sign}{sign}
\DeclareMathOperator{\Tr}{Tr}
\let\ab\allowbreak


%%%
\usepackage{wrapfig}
\usepackage{array}
\usepackage{longtable}
\usepackage[section]{algorithm}
\usepackage{algorithmic}

% RequireとEnsureをInputとOutputにする
\renewcommand{\algorithmicrequire}{\textbf{Input:}}
\renewcommand{\algorithmicensure}{\textbf{Output:}}
%%%

%% reviesed
\newcommand{\revisedStart}{\color{black}}
\newcommand{\revisedEnd}{\color{black}}

\usepackage{url}
\usepackage{amsmath, amssymb, mathtools, amsthm, mathcomp}


\usepackage{amsfonts,amscd,amssymb,bm,bbm,mathrsfs}

\usepackage{hyperref}

\usepackage{cleveref}
\usepackage{autonum}

\usepackage{enumitem}
\usepackage{graphicx}
\usepackage{booktabs}
\usepackage{multirow}

\theoremstyle{plain}
\newtheorem{lem}{Lemma}
\newtheorem{thm}{Theorem}
\newtheorem*{thm*}{Theorem}
\newtheorem{prop}{Proposition}
\newtheorem{example}{Example}
\newtheorem{assumption}{Assumption}
\newtheorem{cor}{Corollary}

\theoremstyle{definition}
\newtheorem{dfn}{Definition}
\newtheorem*{rem*}{Remark}

\usepackage{comment}

\title{Direct Distributional Optimization for Provable Alignment of Diffusion Models}

% Authors must not appear in the submitted version. They should be hidden
% as long as the \iclrfinalcopy macro remains commented out below.
% Non-anonymous submissions will be rejected without review.


\author{Ryotaro Kawata$^{1,2,*}$, Kazusato Oko$^{2,3,\dagger}$, Atsushi Nitanda$^{4,5,\ddagger}$, Taiji Suzuki$^{1,2,\S}$}


% The \author macro works with any number of authors. There are two commands
% used to separate the names and addresses of multiple authors: \And and \AND.
%
% Using \And between authors leaves it to \LaTeX{} to determine where to break
% the lines. Using \AND forces a linebreak at that point. So, if \LaTeX{}
% puts 3 of 4 authors names on the first line, and the last on the second
% line, try using \AND instead of \And before the third author name.

\newcommand{\fix}{\marginpar{FIX}}
\newcommand{\new}{\marginpar{NEW}}

\newcommand{\MOD}{\textcolor{red}{(MOD)}}

\iclrfinalcopy % Uncomment for camera-ready version, but NOT for submission.
\begin{document}


\maketitle


\begingroup  % 所属情報を大きめに表示
%\centering
%\large
\vspace{-2em} % 調整
$^1$Department of Mathematical Informatics, University of Tokyo, Japan \\
$^2$Center for Advanced Intelligence Project, RIKEN, Japan \\
$^3$Department of ECCS, UC Berkeley \\
$^4$CFAR and IHPC, Agency for Science, Technology and Research (A$\star$STAR), Singapore \\
$^5$College of Computing and Data Science, Nanyang Technological University, Singapore \\
%\vspace{1em}\\ % 余白を調整
$^*$\texttt{\href{mailto:kawata-ryotaro725@g.ecc.u-tokyo.ac.jp}{kawata-ryotaro725@g.ecc.u-tokyo.ac.jp}} \\
$^\dagger$\texttt{\href{mailto:oko@berkeley.edu}{oko@berkeley.edu}}\\
$^\ddagger$\texttt{\href{mailto:atsushi_nitanda@cfar.a-star.edu.sg}{atsushi\_nitanda@cfar.a-star.edu.sg}} \\
$^\S$\texttt{\href{mailto:taiji@mist.u-tokyo.ac.jp}{taiji@mist.u-tokyo.ac.jp}}\\
\vspace{1em}
\par
\endgroup


\begin{abstract}
We introduce a novel alignment method for diffusion models from distribution optimization perspectives while providing rigorous convergence guarantees.
We first formulate the problem as a generic regularized loss minimization over probability distributions and directly optimize the distribution using the Dual Averaging method.
Next, we enable sampling from the learned distribution by approximating its score function via Doob's $h$-transform technique.
The proposed framework is supported by rigorous convergence guarantees and an end-to-end bound on the sampling error, which imply that when the original distribution's score is known accurately, the complexity of sampling from shifted distributions is independent of isoperimetric conditions.
This framework is broadly applicable to general distribution optimization problems, including alignment tasks in Reinforcement Learning with Human Feedback (RLHF), Direct Preference Optimization (DPO), and Kahneman-Tversky Optimization (KTO). We empirically validate its performance on synthetic and image datasets using the DPO objective.
% This method effectively aligns distributions, bypassing the need for explicit density calculations, and is applicable to a wide range of optimization tasks, including Reinforcement Learning with Human Feedback (RLHF), Direct Preference Optimization (DPO), and Kahneman-Tversky Optimization (KTO). Empirical validation on synthetic and image datasets demonstrates the method’s ability to handle complex multimodal distributions, showcasing its utility in model alignment tasks.

% We propose a novel model refinement method for aligning diffusion models, offering rigorous convergence guarantees without requiring isoperimetric conditions. Our approach is based on the dual averaging scheme for distributional optimization, which guides an aligned diffusion process in combination with the density ratio estimation using neural networks with Doob's $h$-transform. This framework is broadly applicable to general distribution optimization problems including alignment tasks in reinforcement learning with human feedback (RLHF), direct preference optimization (DPO), and Kahneman-Tversky optimization (KTO) settings, and we empirically validate its perfomance on synthetic and image datasets with the DPO objective.
\end{abstract}

\section{Introduction}
Diffusion models~\citep{Sohl-Dickstein2015thermodynamics,ho2020DDPM,song2021scorebased} have recently emerged as powerful tools for learning complex distributions and performing efficient sampling. Within the framework of foundation models, a common approach involves pre-training on large-scale datasets, followed by adapting the model to downstream tasks or aligning it with human preferences~\citep{ouyang2022training}. This alignment is typically formalized as nonlinear distribution optimization, with a regularization term that encourages proximity to the pre-trained distribution. Examples of such alignment methods include Reinforcement Learning with Human Feedback (RLHF)~\citep{ziegler2020LLMft}, Direct Preference Optimization (DPO)~\citep{rafailov2023DPO,Wallace2024DiffusionDPO}, and Kahneman-Tversky Optimization (KTO)~\citep{ethayarajh2024KTO,li2024diffusionKTO}.

\revisedStart
Specifically, there methods solve the minimization problems of a regularized functional $F(q)+\beta\KL(q\|\pref)$ over $q$ in the probability space $\mathcal{P}$, where  $\pref$ is the probability density corresponding to the pretrained diffusion model. 
%^of the output samples at the final denoising step, denoted by $X^\leftarrow_T$.
% The alignment objective is the functional $F(q)+\beta\KL(q\|\pref)$ whose input is the density $q$. 
However, this type of distributional optimization problem over the density $q$ is challenging because the output density governed by the reference model cannot be evaluated and neither is the aligned model $q$. 
They are accessible only through samples generated from their corresponding generative models.  
%using the score networks. 
Existing distributional optimization methods such as {\it mean-field Langevin dynamics}~\citep{mei2018mf} and {\it particle dual averaging}~\citep{NEURIPS2021_a34e1ddb} resolved this problem by adapting a Langevin type sampling procedure to calculate a functional derivative of the objective without explicitly evaluating the densities. 
However, the distributions $q$ and $\pref$ are highly complex and multimodal from which it is extremely hard to generate data by a standard MCMC type methods including the Langevin dynamics. 
This difficulty can be mathematically characterized by isoperimetric conditions, such as logarithmic Sobolev inequality (LSI) \citep{bakry2014analysis}, that has usually exponential dependency on the data dimension $d$ for multimodal data yielding the curse of dimensionality. 
Unfortunately, the existing distribution optimization methods mentioned above are sensitive to the LSI constant, so they suffer from severely slow convergence, failing to align diffusion models. 

% On the other hand, the densities $q$ and $\pref$ are highly complex and multimodal from which it is extremely hard to generate data by a standard MCMC type methods such as a Langevin dynamics. 
% This difficulty can be mathematically characterized by isoperimetric conditions, such as logarithmic Sobolev inequality (LSI), that has usually exponential dependency on the data dimension $d$ for multimodal data yielding the curse of dimensionality. 
% Unfortunately, existing standard distribution optimization methods including {\it mean-field Langevin dynamics}~\citep{mei2018mf} are sensitive to LSI, so they suffer from severely slow convergence, failing to align diffusion models. 

% They are formulated as minimization problems of a regularized functional $F(q)+\beta\KL(q\|\pref)$ over $q$ in the probability space $\mathcal{P}$, where $q$ and $\pref$ are the aligned and pretrained densities of the output samples at the final denoising step, denoted by $X^\leftarrow_T$.
% % The alignment objective is the functional $F(q)+\beta\KL(q\|\pref)$ whose input is the density $q$. 

% However, this type of distributional optimization problem over the density $q(X^\leftarrow_T)$ is challenging because the output densities governed by the reference and the aligned models cannot be evaluated. They are accessible only through samples using the score networks.
% In addition, the densities $q$ and $\pref$ are highly complex and multimodal, which implies that isoperimetric conditions, such as logarithmic Sobolev inequality (LSI), become significantly weaker. Specifically, the LSI constant, which determines the strength of LSI, becomes exponentially small as the data dimension $d$ increases.
% Standard distribution optimization methods including mean-field Langevin dynamics~\citep{mei2018mf} are sensitive to LSI, so they suffer from severely slow convergence, failing to align diffusion models. 


That is to say, alignment of diffusion models has two challenges: (i) inaccessibility of the output densities and (ii) muitimodality of the densities. This naturally leads to a fundamental question:
\revisedEnd
\begin{comment}
    In the context of distribution optimization, mean-field particle methods such as mean-field Langevin dynamics~\citep{mei2018mf}, particle dual averaging~\citep{NEURIPS2021_a34e1ddb}, and particle stochastic dual coordinate ascent~\citep{oko2022particle} have been developed to optimize convex functionals defined over distributions with a KL-divergence regularizer from a simple distribution, such as a Gaussian. The convergence rate analyses~\citep{mei2018mf,hu2021mean,nitanda22mfld,chizat2022meanfield,NEURIPS2021_a34e1ddb} for these methods are also well established under some isoperimetric conditions, such as log-Sobolev inequality, which ensures sufficient concentration of probability mass~\citep{bakry2014analysis}, facilitating efficient exploration and faster convergence. These methods can naturally handle distribution optimization, but, their application to align diffusion models remains largely unexplored. This is primarily due to two challenges: (i) particle-based optimization methods are not designed for resampling from the distribution where obtained particles follow, and (ii) pre-trained distribution is typically highly complex multimodal distributions, failing to satisfy isoperimetric conditions. This naturally leads to a fundamental question:
\end{comment}

\begin{center}
% {\it Can we develop a model refinement procedure that produces a diffusion-based model with improved alignment from a pre-trained reference model, while ensuring rigorous convergence guarantees without requiring isoperimetric conditions?}
{\it Can we develop an alignment algorithm for diffusion models from distribution optimization perspectives, while ensuring rigorous convergence guarantees without isoperimetric conditions?}
\end{center}

We address this question by  
developing a diffusion-model based distribution optimization method and providing rigorous convergence and sampling error guarantees, and demonstrate its applicability to several tasks involved with diffusion model alignment.
\revisedStart
Our method represents the aligned model by a diffusion model that can be described by merely adding a correction term to the score function of the original reference model. During optimizing the model, we don't rely on any MCMC sampler but only use samples generated by the original reference model (and the aligned diffusion model). This characteristics is helpful to resolve the issue of isoperimetric condition.  
\revisedEnd

\begin{figure}[htbp]
    \centering
    \vspace{-3mm}
    \includegraphics[width=0.6\linewidth]{figures/overview.png}
    \vspace{-3mm}
    \caption{Overview of the proposed method integrating Dual Averaging and Doob’s h-transform.}
    \label{fig:proposed_method}
    \vspace{-2mm}
\end{figure}


    \vspace{-2mm}
\subsection{Our Contributions}
    \vspace{-2mm}
To tackle the two challenges mentioned above: absence of (i) sampling guarantee and (ii) isoperimetry, we propose a general framework that integrates dual averaging (DA) method~\citep{Nesterov2009} and diffusion model~\mbox{\citep{Sohl-Dickstein2015thermodynamics,ho2020DDPM,song2021scorebased}}. 
The DA method is an iterative algorithm that constructs the Gibbs distribution converging to the optimal distribution (i.e., alignment). 
% The DA method operates in a double loop scheme: in the outer loop, Gibbs distributions converging to the optimal distribution (i.e., alignment) are successively constructed, while in the inner loop, these intermediate distributions are approximately obtained using sampling methods such as Langevin Monte-Carlo. 
A key advantage of this DA scheme is its ability to bypass isoperimetric conditions with the help of high sampling efficiency of the reference diffusion model, enabling isoperimetry-free sampling from an aligned distribution. %, as long as we get an accurate sampling from the distribution with diffusion model-based methods. 
% by replacing inner-loop sampling methods with diffusion model-based methods. 
Specifically, an aligned diffusion process that approximately generates the Gibbs distribution obtained by DA method can be constructed through the Doob's $h$-transform~\citep{Rogers2000Doob} and density ratio estimation with respect to a reference distribution using neural networks (see Figure \ref{fig:proposed_method} for illustration).

We summarize our contribution below.
\begin{itemize}[topsep=0mm,leftmargin = 8mm] %itemsep=-1mm,
    \item We establish a model alignment method to align diffusion models, with convergence guarantees for both convex and nonconvex objectives, based on distribution optimization theory and Doob's $h$-transform. 
    Notable distinctions from other mean-field optimization methods are that our method works without isoperimetry conditions such as LSI
    and allows for sampling from multimodal distributions. 
    \item We also analyze the sampling error due to the approximation of the drift estimators with neural networks and the discretization of the process, and evaluate how these errors affect the final sampling accuracy. 
    \item Our general framework encompasses several major alignment problems such as RLHF, DPO, and KTO, establishing a provable alignment method for these scenarios. We demonstrate the applicability of our framework to these settings and empirically validate its performance on both synthetic and image datasets, aiming at data augmentation for a specific mode of distribution, using DPO objective.
\end{itemize}
We also emphasize that our method has the potential to be applied to general distribution optimization problems beyond alignment tasks such as density ratio estimation under the covariate shift setting~\citep{sugiyama2008direct,tsuboi2009direct} and climate change tracking~\citep{ling2024diffusion}.

\subsection{Related work.}

\vspace{-1mm}
\textbf{Mean-field optimization.}\quad
PDA method \citep{NEURIPS2021_a34e1ddb,nishikawa2022twolayer}, an extension of DA method~\citep{Nesterov2009} to the distribution optimization setting, was the first method that proves the quantitative convergence for minimizing entropy regularized convex functional. Subsequently, P-SDCA method~\citep{oko2022particle}, inspired by the SDCA method~\citep{shalev2013stochastic}, achieved the linear convergence rate. Mean-field Langevin dynamics~\citep{mei2018mf} is the most standard particle-based distribution optimization method, derived as the mean-field limit of the noisy gradient descent, and its convergence rate has been well studied by \cite{mei2018mf,hu2021mean,nitanda22mfld,chizat2022meanfield,suzuki2023mfld,nitanda2024improved}. Additionally, several mean-field optimization methods such mean-field Fisher-Rao gradient flow~\citep{liu2023polyak} and entropic fictitious play~\citep{chen2023entropic,pmlr-v202-nitanda23a} have been proposed with provable convergence guarantees. We note that the convergence rates of these methods were established under isoperimetric conditions such as log-Sobolev and Poincar\'e inequalities, which ensure concentration of the probability mass.
IKLPD method \citep{pmlr-v238-yao24a} shares similarities with our method, as it employs the normalizing flow to solve intermediate subproblems in the distributional optimization procedure, and its convergence does not depend on isoperimetric conditions. However, the applicability of IKLPD to alignment tasks remains uncertain since handling the proximity to the reference distribution is non-trivial.

%However, the algorithms shown above are limited to the Log-Sobolev inequality, which implies the unimodality of the densities.
%Our distributional optimization algorithm via diffusion models is free from such isoperimetric inequalities, which leads to more realistic assumptions of the data distribution.

\vspace{-1mm}
\textbf{Fine-tuning of diffusion models.}\quad
Recently, alignment of diffusion models have been investigated, inspired by LLM fine-tune methods, such as RLHF~\citep{ziegler2020LLMft}, DPO~\citep{rafailov2023DPO}, and KTO~\citep{ethayarajh2024KTO}. Applying them to diffusion models entails additional difficulty since the output density $\pref$ of the diffusion model is not available, and hence several techniques have been developed to circumvent the explicit calculation of $\pref$. 
For instance, \cite{fan2023reward,black2024reward,clark2024reward} invented the maximization algorithm of the reward in each diffusion time step. \cite{uehara2024reward} used Doob's $h$-transform to compute the correction term that can be automatically derived from the density ratio between the generated and the reference distributions. Instead of optimizing original DPO and KTO objectives, \cite{Wallace2024DiffusionDPO} considered the evidence lower bound (ELBO) and \cite{li2024diffusionKTO} defined a new objective function to replicate KTO.
\revisedStart
%\cite{marion2024implicit} also studied fine-tuning of diffusion models as distributional optimization. However, their convergence analysis for diffusion models was limited to RLHF approach and the single Gaussian assumption.\revisedEnd
\cite{marion2024implicit} also studied fine-tuning of diffusion models as distributional optimization within the RLHF framework and conducted convergence analysis for the one-dimensional Gaussian distribution.\revisedEnd
%*********
%They unified Langevin dynamics and the fine-tuning of diffusion models as an optimization problem over the probability space. However, their theoretical analysis was constrained by the logarithmic Sobolev inequality in the context of Langevin dynamics, as well as by the reinforcement learning approach and the single Gaussian assumption in the context of diffusion models.
%*********

%Nevertheless, the difficulty to directly optimize the nonlinear objectives of the output distribution, such as DPO and KTO, via diffusion models have not been solved yet.
% We tackled this problem from the mean-field perspective and Doob's h-transform technique.

%And LoRA?

% unified, theoretical way of direct density optimization
% 拡散モデル: pathの密度しか密度が扱えない

% mean-field: sampling NG 多峰
% proximal gibbs
% mean-field langevin も遅い 画像でうまく行かない


%\paragraph{Application DPO}
%marginal distribution w.r.t. $x_0$
%generalization of DP


%\begin{itemize}
%    \item Introduction: 1.5 pages
%    \item Preliminaries: 1 page
%    \item Core theorems 1 page
%    \item Applications 1 page
%    \item Diffusion Model: 1.5 pages
%    \item Experiments: 2.5 pages
%    \item Conclusion: 0.5 pages
%\end{itemize}

%\color{red}
%KEYWORDS:
%\begin{itemize}
%    \item diffusion (as a schodinge bridge?)
%    \item Doob
%    \item RLHF, PPO (uehara etc.)
%    \item sampling
%    \item mean field (?)
%    \item diffusion-DPO
%    \item diffusion-KTO
%\end{itemize}
%\color{black}

\vspace{-1mm}
\section{Problem Setting}
\vspace{-2mm}

\textbf{Distributional Optimization.}\quad
Let $\gP$ be the space of probability density functions with respect to the Lebesgue measure on $(\mathbb{R}^d,\mathcal{B}(\mathbb{R}^d))$.  
%such that \textcolor{red}{some assumptions}. 
Let $F:\mathcal{P} \to \mathbb{R}$ be a functional and $\pref \in \gP$ be the reference density. 
In this work, we consider the regularized loss minimization problem over $\mathcal{P}$:
\begin{equation}
    \underset{q \in \mathcal{P}}{\min}\left\lbrace L(q) \coloneq F(q) + \beta \KL(q \| \pref) \right\rbrace,\label{eq:setup-minimization}
\end{equation}
where $\KL(q \|\pref) \coloneq \E_q [\log \frac{q}{\pref}]$ is the Kullback-Leibler divergence, and $\beta > 0$ is a regularization coefficient.
We assume that $F$ is differentiable. That is, the functional $F$ has \textit{first order variation} $\dFdq:\mathcal{P}\times \mathbb{R}^d\ni (q,x)\to \dFdq(q,x)\in\mathbb{R}$ such that for all $q,q' \in \mathcal{P}$,
\begin{equation}
    \left.\frac{\mathrm{d}F(q+\epsilon(q'-q))}{\mathrm{d}\epsilon}\right|_{\epsilon=0} = \int \dFdq(q,x)(q'-q)(x)\mathrm{d}x.
\end{equation}
In the following we assume that there exists a unique minimizer $\qstar := \argmin_{q \in \gP} L(q)$. 

\textbf{Diffusion Models.}\quad
$\pref$ is the output density of a pre-trained diffusion model~\citep{Sohl-Dickstein2015thermodynamics,song2019generatiive,ho2020DDPM,song2021scorebased,vahdat2021latent}, while
$p_* \in \mathcal{P}$ is the target distribution of pre-training.
A ``noising'' process $\lbrace \bar{X}_t \rbrace_{t\geq 0}$ denotes the Ornstein-Uhlenbeck (OU) process from $p_*(x)$. 
% The law of $\bar{X}_t$ given $\bar{X}_0 \sim p_*$ is $\mathcal{N}(m_t \bar{X}_0,\sigma_t), \; m_t = e^{-t}, \sigma_t^2 =1 - e^{-2t}$. 
The law of $\bar{X}_t$ can be written as $p_t(x) = \int \mathcal{N}(m_t \bar{X}_0,\sigma_t) \mathrm{d}p_*(\bar{X}_0)$ with $m_t = e^{-t}, \sigma_t^2 =1 - e^{-2t}$.
Then, the reverse process $\{\bar{X}_t^\leftarrow\}_{0\leq t \leq T}$ ($T\geq 0$) can be defined as
\begin{align}
    \bar{X}_t^\leftarrow \sim p_t,\ \mathrm{d}\bar{X}_t^\leftarrow = \{\bar{X}_t^\leftarrow + 2 \nabla \log p_{T-t}(\bar{X}_t^\leftarrow)\}\mathrm{d}t + \sqrt{2} \mathrm{d}B_t.
\end{align}
Then it holds that $\mathrm{Law}(\bar{X}_t^\leftarrow)=\mathrm{Law}(\bar{X}_{T-t})$, which enables us to sample from $p_*$.
In practice, we approximate the score $\nabla \log p_{T-t}(\bar{X}_t^\leftarrow)$ function by a score network 
$s:\mathbb{R}^{d+1}\to \mathbb{R}^d:(x,t) \mapsto s(x,t)$.
In addition, we initialize $\Xref_0 \sim \mathcal{N}(0,I_d) \simeq p_T$ and the process is time-discretized.
The random variable generated by the following dynamics with step size $h$ is denoted by $\{\Xref_t\}_{0\leq t \leq T}$.
\begin{equation}
\Xref_0 \sim \mathcal{N}(0,I_d),\ 
        \mathrm{d}\Xref_t = \lbrace  \Xref_{\tprev}+ 2s(\Xref_{\tprev},\tprev) \rbrace \mathrm{d}t
        + \sqrt{2}\mathrm{d}B_t, \ t \in [\tprev, \tnext], \; l=1,...,L=T/\stepsize.
\end{equation}
In the same way, we define $q_t$ as the density of the diffusion process corresponding to $q_*$, and $\{{\bar{X}}_t\}_{t\geq 0}$ and $\{{\bar{X}^\leftarrow}_t\}_{0\leq t \leq T}$ as the corresponding noising and the backward process.


% The SDE is formulated as
% \begin{equation}
%         \mathrm{d}\Xref_t = \lbrace  \Xref_{\tprev}+ 2s(\Xref_{\tprev},\tprev) \rbrace \mathrm{d}t
%         + \sqrt{2}\mathrm{d}B_t, \quad t \in [\tprev, \tnext], \; l=1,...,L=T/\stepsize,
% \end{equation}
% where $\delta$ is the step size and $L=T/\delta$ is the number of steps, assumed to be integer for simplicity. 
% We define $\pref \simeq p_*$ as the law of the output $\Xref_T$, and $\pmREF$ as the path measure of $\lbrace\Xref_t\rbrace_t$.

Now we have two challenges in this problem (\ref{eq:setup-minimization}):
\begin{itemize}[topsep=0mm,itemsep=-1mm,leftmargin = 12mm] 
%[topsep=0mm,itemsep=-1mm,leftmargin = 5.5mm,labelsep=0.5mm]
\item[(A).]\textbf{Multimodality of $\pref$ and $\qstar$.} %\quad
Our goal is to obtain samples from $\qstar$.
Now we tackle with the case that $\pref$ and $\qstar$ have high \textit{multimodality}: $\pref$ and $\qstar$ have multiple modes or maxima, in other words, the potentials $-\log\pref$ and $-\log\qstar$ are extremely far from concave. Probability distributions of real-world data like images often have such complex structures, which implies that LSI is significantly weak.
\item[(B).]\textbf{Inaccesibility to the density $\pref$.} %\quad
We cannot directly calculate the density $\pref$ because diffusion models only have the score network. We only have information of $\pref$ as samples from $\pref$.
\end{itemize}

\textbf{Inapplicability of Mean-Field Langevin Dynamics.}\quad
Mean-field Langevin dynamics (MFLD)~\citep{mei2018mf} is the most standard particle-based optimization method tailored to solve the special case of the problem~\eqref{eq:setup-minimization} where $F$ is convex functional and $\pref$ is a strongly log-concave distribution such as Gaussian distribution. The convergence rate of MFLD~\citep{nitanda22mfld,chizat2022meanfield} has been established under the condition where the proximal distribution: $p_q \propto \pref \exp{\left(-\beta^{-1}\dFdq(q,\cdot)\right)}$ associated with MFLD iteration $q$ satisfies logarithmic Sobolev inequality (LSI), which says sufficient concentration of $p_q$. Typically, LSI for $p_q$ should rely on the isoperimetry of the reference distribution $\pref$ with Holley-Strook argument~\citep{holley1987logarithmic} since $\dFdq(q,\cdot)$ is not expected to encourage it in general. Therefore, the replacement of $\pref$ to a pre-trained distribution, which is highly complex and has multi-modality, leads to failure or weak satisfaction of LSI. As a result, the convergence of MFLD is significantly slowed down.
Additionally, MFLD is practically implemented so that finite particles approximately follow the ideal dynamics of MFLD, and it is not intended for resampling from the final distribution represented by these particles. In short, the lack of (i) a sampling guarantee and (ii) a isoperimetric condition limits the applicability of MFLD in our problem setting. 

\vspace{-2mm}
To address these challenges, we employ the following strategy, which will be detailed in Section~\ref{section:algorithm-introduction}.
\begin{itemize}[topsep=0mm,itemsep=-1mm,leftmargin = 12mm] 
%[topsep=0mm,itemsep=-1mm,leftmargin = 5.5mm,labelsep=0.5mm]
%[topsep=0mm,itemsep=-1mm,leftmargin = 12mm]
\item[(A).]\textbf{Modifications of diffusion models for sampling from a multimodal distribution.}\quad
First, we obtain samples from $\pref$ using a diffusion model which works for a broad range of probability distributions that have complex structures such as multimodality. 
\revisedStart
Then, we reconstruct the diffusion model to sample from 
$\qstar$ by simply adding a correction term.
\revisedEnd

\item[(B).]\textbf{A distributional optimization algorithm that does not require the density $\pref$.}\quad
Second, following the dual averaging (DA) method in the distribution optimization setting, we constract a sequence of distributions $\qstark$ converging to the optimum, which guides an aligned diffusion process in combination with the density ratio estimation using neural networks with Doob's $h$-transform. 
We only have to calculate the density ratio between $\qstark$ and $\pref$ and the samples from $\pref$ to run our algorithm.
\end{itemize}

\begin{comment}
    \textbf{Denoising Diffusion Probabilistic Model.}\quad
    Let $p_0 \in \mathcal{P}$ be a target distribution of pre-training, $T$ be the total diffusion time.
    A ``noising" process $\lbrace X_t \rbrace_{t\in[0,T]}$ from $p_0(x)$ is called as the Ornstein-Uhlenbeck (OU) process. The Stochastic Differential Equation (SDE) is
    \begin{equation}
        \mathrm{d}X_t = - X_t \mathrm{d}t + \sqrt{2}\mathrm{d}B_t, \quad X_0 \sim p_0,
    \end{equation}
    where $\lbrace B_t \rbrace_{t\in[0,T]}$ is $d$-dimensional Brownian motion. Then, we see $X_t|X_0 \sim \mathcal{N}(m_t X_0,\sigma_t), \; m_t = e^{-t}, \sigma_t^2 =1 - e^{-2t}$. We denote the law of $X_t$ as $p_t \coloneq \mathrm{Law}(X_t)$.
    A time-reversed ``denoising" process $\lbrace \Xrou_t \rbrace_{t\in[0,T]}$ is the reversed OU process
    \begin{equation}
        \mathrm{d}\Xrou_t = \lbrace \Xrou_t + 2 \nabla \log \bar{p}_{T-t}(\Xrou_t) \rbrace + \sqrt{2} \mathrm{d}B_t, \quad \Xrou_0 \sim p_T.
    \end{equation}
    Here, $\nabla \log p_{T-t}$ is called the score. To sample from $p_0$, it is known to be effective to run the modified backward process $\lbrace\Xref_t\rbrace_{t\in[0,T]}$ with the approximated score network $s:\mathbb{R}^{d+1}\to \mathbb{R}^d:(x,t) \mapsto s(x,t) \simeq \nabla \log p_t(x)$. In addition, we initialize $\Xref_0 \sim \mathcal{N}(0,I_d) \simeq p_T$ and the process is time-discretized.
    The SDE is formulated as
    \begin{equation}
        \mathrm{d}\Xref_t = \lbrace  \Xref_{\tprev}+ 2s(\Xref_{\tprev},T-\tprev) \rbrace
        + \sqrt{2}\mathrm{d}B_t, \quad t \in [l\delta, (l+1)\delta], \; l=1,...,L=T/\delta,
    \end{equation}
    where $\delta$ is the step size and $L=T/\delta$ is the number of steps, assumed to be integer for simplicity.  We define $\pref$ as the law of the output $\Xref_T$, and $\pmREF$ as the path measure of $\lbrace\Xref_t\rbrace_t$.
    
    \textbf{Score Matching.}\quad
    The score network is ideally selected to minimize the denoising score matching loss
    \begin{equation}
        \E_t [\E_{x_0}\E_{x_t|x_0}[\|s(x_t,t) - \nabla \log p_t(x_t|x_0)\|_2^2]],
    \end{equation}
    where $t \sim \mathrm{Unif}[0,T]$, $x_0 \sim p_0$, $x_t|x_0 \sim p_t(x_t|x_0)$. In practice, we have finite data $\lbrace x_{0,i}\rbrace_{i=1}^{n}$, $x_{0,i}\overset{\text{i.i.d.}}{\sim}p_0$. So the estimated score network $\hat{s}$ minimizes the empirical loss
    \begin{equation}
        \frac{1}{n}\sum_{i=1}^n \E_t \E_{x_t|x_{0,i}}[\|s(x_t,t) - \nabla \log p_t(x_t|x_{0,i})\|_2^2]].
    \end{equation}
\end{comment}

\vspace{-3mm}
\section{Applications}
\vspace{-3mm}

Our framework of the distributional optimization for the pre-trained diffusion models
includes important fine-tuning methods for diffusion models.
% We are mainly interested in when $F(q)$ is a functional of the form
%\begin{equation}
%    F(q) = G(q, \pref), \quad G: \mathcal{P}\times\mathcal{P}\to \mathbb{R}\text{ is also a functional.}
% \end{equation}
% Then, (\ref{eq:setup-minimization}) can be seen as an objective for fine-tunes of the diffusion model. 
The first term $F(q)$ in (\ref{eq:setup-minimization}) represents \textit{the human preference}, acting as feedback from the outputs of $\pref$. Note that $F(q)$ can be dependent of $\pref$. The second term $\beta \KL(q\|\pref)$ in (\ref{eq:setup-minimization}) provides regularization to prevent $q$ from collapsing.

\begin{example}[Reinforcement Learning]
    Our study includes the case $F$ is limited to be a linear functional:
    \vspace{-2mm}
    \begin{equation}
        \underset{q \in \mathcal{P}}{\min}\left\lbrace \E_q [-r(x)] + \beta \KL(q \| \pref) \right\rbrace,
    \end{equation}
    \vspace{-1mm}
    where $r(x)$ is a reward function. In this case, the optimal distribution is obtained as
    %\begin{equation}
    $\textstyle    q_*(x) \propto \exp \left(\frac{r(x)}{\beta} \right) \pref(x)$.
    %\end{equation}
    This type of the problems has been studied as the Reinforcement Learning~\citep{fan2023reward,black2024reward,clark2024reward,uehara2024reward,marion2024implicit}. 
    %For example, see ~\cite{uehara2024RLHF}.
\end{example}

\vspace{-1mm}
The following two examples have not been directly solved via diffusion models:

\begin{example}[DPO]
    % \textcolor{red}{the functional derivative is bounded if the initialization is bounded, but not smooth (not in TV, nor in KL. $\because |\dFdq(q,x)-\dFdq(q',x)|\simeq |\frac{1}{q(x)}-\frac{1}{q'(x)}|$)}
    Direct Preference Optimization (DPO) \citep{rafailov2023DPO} is an effective approach for learning from human preference for not only language models but also diffusion models.
    Our algorithm can directly minimize the DPO objective,
    while \cite{Wallace2024DiffusionDPO} tried applying DPO to diffusion models via minimization of an upper bound of the original objective.
    In DPO, humans decide which sample is more preferred given two samples from $\pref$. Let $x_w$ and $x_l$ be ``winning" and ``losing" samples from $\pref$. $x_w \succ x_l$ denote the event that $x_w$ is preferred to $x_l$.
    The DPO objective can be written as
    \vspace{-1mm}
    \begin{equation}\textstyle
        L_{\mathrm{DPO}}(q) \coloneq - \E_{x_w\sim p_{\mathrm{ref}}}\E_{x_l\sim p_{\mathrm{ref}}}
        \left[ 
             \log \sigma \left(\gamma \log \frac{q(x_w)}{\pref(x_w)}- \gamma \log \frac{q(x_l)}{\pref(x_l)}\right)\mathbbm{1}_{x_w \succ x_l}(x_w,x_l)
        \right],
    \end{equation}
    where $\E_{x\sim p}$ denotes expectation with respect to $x$ whose probability density is $p \in \mathcal{P}$, $\sigma$ is a sigmoid function, $\mathbbm{1}_{x \succ y}(x,y)$ is one if $x \succ y$ and is zero otherwise.
    % We have the functional derivatives as the form $\E_{\pref}[\psi(f)]$, where $q = e^{-f}{\pref} / \int e^{-f}d{\pref}$, $\psi:\mathbb{R}\to\mathbb{R}$. 
    Precisely, the functional derivative of $L_\mathrm{DPO}(q)$ is calculated as
    \begin{align}\textstyle 
    \frac{\delta L_\mathrm{DPO}}{\delta q}(q,x) 
    =& \textstyle 
    -\gamma\E_{x_l\sim p_{\mathrm{ref}}}
    \left[
        \left(
            1- \sigma
            \left(
                -\gamma f(x) + \gamma f(x_l)
            \right)
        \right)
        \frac{\int e^{-f} dp_\mathrm{ref}}{e^{-f(x)}}
        \mathbbm{1}_{x \succ x_l}(x,x_l)
    \right]\\
    &\textstyle  +     \gamma\E_{x_w\sim p_{\mathrm{ref}}}
    \left[
        \left(
            1- \sigma
            \left(
                - \gamma f(x_w) 
                + \gamma f(x)
            \right)
        \right)
        \frac{\int e^{-f} dp_\mathrm{ref}}{e^{-f(x)}}
        \mathbbm{1}_{x_w \succ x}(x_w,x)
    \right],\label{eq-main-DPO-derivative}
    \end{align}
    where $q = e^{-f}\pref / \int e^{-f}\mathrm{d}\pref$. See Appendix~\ref{section:appendix-functional-derivative} for the derivation.
    Therefore, we only need samples from $\pref$ and the log-density ratio or the potential $f$ to calculate the functional derivatives.
\end{example}
\begin{example}[KTO]
    % \textcolor{red}{Note: the functional derivative is not sufficiently smooth (although it is bounded)}

    Our algorithm can also minimize $L_\mathrm{KTO}$ directly.
    Assume that the whole data space $\mathbb{R}^d$ is split into a desirable domain $\mathcal{D}_\mathrm{D}$ and an undesirable domain $\mathcal{D}_\mathrm{U}$.
    The objective of the original KTO \citep{ethayarajh2024KTO} is formulated as
    \begin{align}
\textstyle        L_\mathrm{KTO}(q)=
        &\textstyle \E_{x\sim \pref}
        \left[ 
        \gamma_D \left(1 - \sigma \left(\kappa\log \frac{q}{\pref} - \KL(q\|\pref)\right)
        \right)\mathbbm{1}_{\lbrace x \in \mathcal{D}_\mathrm{D}\rbrace}
        \right. \\
        &\textstyle \quad \left.+
        \gamma_U \left(1 - \sigma \left(\KL(q\|\pref)-\kappa\log \frac{q}{\pref} \right)\right)\mathbbm{1}_{\lbrace x \in \mathcal{D}_\mathrm{U}\rbrace}
        \right],
    \end{align}
    where $\gamma_D, \; \gamma_U, \; \kappa$ are hyper parameters, and $\sigma$ is a sigmoid function. 
    \cite{li2024diffusionKTO} defined objectives compatible with diffusion models based on KTO, but our algorithm can directly minimize $L_\mathrm{KTO}$.
    Like the DPO objective, we only have to calculate samples from $\pref$ and the potential $f$ of the density ratio ($f$ is defined as $q = e^{-f}{\pref} / \int e^{-f}d{\pref}$) to calculate the functional derivatives. Please refer to Appendix~\ref{section:appendix-functional-derivative} for the concrete formulation of $\frac{\delta L_\mathrm{KTO}}{\delta q}$.
\end{example}

\vspace{-2mm}
\section{The Nonlinear Distributional Optimization Algorithm}
\label{section:algorithm-introduction}
\vspace{-1mm}

Now we make a concrete introduction of our proposed approach.
Our goal of the distribution optimization (\ref{eq:setup-minimization}) is to train a neural network that approximates $\fopt = \log \frac{\qstar}{p_{\rm ref}} + (\text{const.})$.
To achieve this, we utilize the \textit{Dual Averaging} (DA) algorithm \citep{Nesterov2009,NEURIPS2021_a34e1ddb,nishikawa2022twolayer}, and we iteratively construct a tentative local potential $f_k$ by approximating the update the DA algorithm.
After we obtain $f_K$, we estimate the diffusion model that generates the desired output  (approximately) following $q_*$, through \emph{Doob's h-transform technique}. 
%function $f_*$, we get the desired output $Y_T^\leftarrow \sim \exp(-f_*)\pref$ by Doob's h-transform.

\textbf{Phase 1: Dual Averaging.}\quad
Let $f_1$ be a randomly initialized potential. First, we initialize $\qone \propto \exp(-f_1)\pref$, where $f_1$ is a randomly initialized neural network.
Then, the distribution $q^{(k)}$ is updated recursively by pulling back the weighted sum of gradients from the dual space to the primal space.
There are two options of DA methods.
For a given hyper-parameter $\beta' > 0$, the update of Option 1 is given as 
\begin{flalign}
\text{\bf (Opt. 1)} &  & \qstarnext &    
\textstyle = \argmin\limits_{q \in \mathcal{P}}   
\Big\lbrace\frac{2}{k(k+1)} \sum\limits_{j=1}^k   j  \left( \E_q \left[ \dFdq(\qj) \right] + \beta \KL(q \| \pref) \right)  +  \frac{2 \beta'}{k} \KL(q \| \pref) \Big\rbrace  &  \\
 &  & &  =: \exp{(-\gbark)}\pref,   &   \label{eq:DAAlg1}
%    \qstarnext \coloneq \exp{(-\gbark)}\pref = \underset{q \in \mathcal{P}}{\min}\left\lbrace \E_q\left[\gbark \right] + \KL(q \| \pref) \right\rbrace,
\end{flalign}
where $\gbark(x) = \sum_{j=1}^k w_j^{(k)} \dFdq(\qj,x),~ w_j^{(k)} = \frac{j}{\beta k(k+1)/2+ \beta'(k+1)}~~(j=1,\dots,k)$. \revisedStart 
By Lemma~\ref{lem-da-nonconvex-nitanda} in Appendix~\ref{section:da-nonconvex-proof}, $\gbark$ can be explicitly determined.
\revisedEnd
% \begin{flalign} \textstyle 
% \text{where}~~
% \gbark(x) = \sum_{j=1}^k w_j^{(k)} \dFdq(\qj,x), ~~ w_j^{(k)} = \frac{j}{\beta k(k+1)/2+ \beta'(k+1)}~~(j=1,\dots,k),  & & 
% %    \gbark(x) = \frac{1}{\beta}\sum_{j=1}^k w_j \dFdq(\qj,x), \quad w_j = \frac{2j}{(k+1)(k+2)}, \text{ for $j=1,...,k$.}
% \end{flalign}
We train a neural network $f_{k+1}$ to approximate $\gbark$\footnote{
For DPO and KTO, it suffices to obtain the neural network $\gbark$ to minimize $\E_{\pref}[(f - \gbark)^2]$ where the expectation with respect to $\pref$ is simulated by generating data from $\pref$, while obtaining $\gbark$ for general settings requires Doob’s h-transform similar to Phase 2. Please also refer to Section~\ref{sec:numerical-experiments}.
% While it is possible to apply Doob’s h-transform to obtain $\gbark$ here for general settings, for DPO and KTO, it suffices to train the neural network by minimizing $\E_{\pref}[(f - \gbark)^2]$ where the expectation with respect to $\pref$ is simulated by generating data from $\pref$. 
% Although it is possible to perform a rigorous error analysis for this approximation, we leave it out in this work for simplicity.
} and define the next step as $q^{(k+1)} \propto \exp{(-f_{k+1})}\pref$. Similarly, the update of Option 2 is given as
\begin{flalign}
\text{\bf (Opt. 2)} &  \!\!\!& \qstarnext &    
\textstyle \!=\! \argmin\limits_{q \in \mathcal{P}}   
\Big\lbrace\frac{2}{k(k+1)} \sum\limits_{j=1}^k   j  \left( \E_q \left[ \dFdq(\qj) 
 - \beta \gbarj \right] \right) \! + \! \frac{2 \beta'}{k} \KL(q \| \pref) \Big\rbrace.   & \label{eq:DAAlg2}
 %\\ & & & \coloneq \exp{(-\gbark)}\pref.   & 
%    \qstarnext \coloneq \exp{(-\gbark)}\pref = \underset{q \in \mathcal{P}}{\min}\left\lbrace \E_q\left[\gbark \right] + \KL(q \| \pref) \right\rbrace,
\end{flalign}
Here, we again express as $\qstarnext(x) \propto \exp(-  \gbark(x)) \pref(x)$ where 
$\gbark(x) = \sum_{j=1}^k w_j^{(k)} ( \dFdq(\qj,x) -\beta \gbarj(x))$ with $w_j^{(k)} = \frac{j}{\beta' (k+1)}$. 
Then, $q^{(k+1)}$ is obtained in the same manner as Option 1. 
This phase of DA update is summarized in Algorithm~\ref{alg:DA-train-f-mainpart}. \revisedStart For the more detailed algorithm in Option 1, please refer to Algorithm~\ref{alg:DA-train-f}.\revisedEnd
\vspace{-2mm}
\begin{algorithm}[htbp]
\label{alg:DA-train-f-mainpart}
\caption{Dual Averaging (DA)}
\begin{algorithmic}
    \REQUIRE{
            $s$: pre-trained score,
            $f_1$: initialized neural networks
        }\\
    \ENSURE{
            $f_{K}$: a trained potential.
    }
    \STATE Set  $q^{(0)}=\pref$ and $q^{(1)} \propto \exp(- f_1)\pref$
    \FOR{$k = 1,...,K-1$}
        \STATE Obtain $\gbark$ via the DA algorithm with Option 1 (Eq.~\eqref{eq:DAAlg1}) or Option 2 (Eq.~\eqref{eq:DAAlg2}) using the recurrence formula (\ref{eq-appendix-experiment-recurrence}), where $\qstarnext \propto \exp(-\gbark)\pref$ is the ideal update. 
        \STATE Train a neural network $f_{k+1}$ to approximate $\gbark$, and set $q^{(k+1)} \propto \exp{(-f_{k+1})}\pref$.  % $\frac{k}{k+2}f_{k} + \frac{2k}{\beta k(k+2)}\dFdq(\qk)$
    \ENDFOR
    %\STATE \textbf{End}
\end{algorithmic}
\end{algorithm}

\vspace{-2mm}
\textbf{Phase 2: Sampling with Doob's h-transform.}\quad
After we obtain the solution $f_K$, we want to sample from $q_K \propto \exp(-f_K)p_{\rm ref}$, which approximates the optimal solution of (\ref{eq:setup-minimization}).
When sampling from $q_K$, it is necessary to obtain the score function related to this distribution. However, constructing the score function of $q_*$ only from the score function of $p_{\rm ref}$ and $f_K$ requires a particular technique.
Specifically, we apply Doob's h-transform~\citep{Rogers2000Doob,chopin2023doob,uehara2024reward,heng2024schrodingerbridge}.
By introducing the correction term $u_*\colon \R^{d+1}\to\R^d$ defined by 
\begin{equation}
u_*(y,t)= \nabla \log \E[\exp(-f_*(\bar{X}_T^\leftarrow))\mid \bar{X}_t^\leftarrow =y], % \Xref_{T-t}=y], 
%\E_{\pmREF}[\exp(-f_*(\Xref_T))\mid \Xref_{T-t}=y],
\end{equation}
the score function of $q_*$ at $(x,t)$ is written as $
 \nabla \log q_{t}(y)  =  \nabla \log p_{t}(X^\leftarrow_t,T-t)+u_*(y,t)$, where $q_{t}$ is the law of the backward process at time $t$ whose output distribution is the optimal solution $q_*$.
 We provide the derivation in Lemma~\ref{lem:H-transform} and refer readers to \cite{Rogers2000Doob,chopin2023doob} for more details and a formal
treatment of Doob’s h-transform.
By approximating $\log p_{t}(X^\leftarrow_t,T-t)$ by the score network $s(x,t)$ and the correction term $u_*(x,t)$ by $u(x,t)$ and discretizing the dynamics, 
we obtain the following update
\begin{equation}
    Y^\leftarrow_0 \sim \mathcal{N}(0,I_d),\
    \mathrm{d}Y^\leftarrow_t = \lbrace Y^\leftarrow_t+ 2(s(Y^\leftarrow_{lh},lh) + u(Y^\leftarrow_{lh}, lh)) \rbrace \mathrm{d}t
    + \sqrt{2}\mathrm{d}B_t,\ t\in [lh,(l+1)h],
    \label{eq:h-transformed-backward}
\end{equation}
% then it is known that $\rho_t$ yields the density ratio $\rho_t = q_{K,t}/p_{\mathrm{ref},t}$ where  $p_{\mathrm{ref},t}$ and $q_{K,t}$ are the densities of $\Xref_t$ and $\Ydoobd_t$ respectively. 
% In particular, we can see that the (unnormalized) density ratio at $T$ is $h_T(x) = \exp(-f_K(x))$ that indicates exactly what we want: $h_T(x) = \exp(-f_K(x)) \propto q_K/\pref$. 
% Indeed, if we let $p_{\mathrm{ref},t}$ and $q_{K,t}$ be the density of $\Xref_t$ and $\Ydoobd_t$ and write $h_t = q_{K,t}/p_{\mathrm{ref},t}$ that is the density ratio at time $t$, then the correction term $u$ is calculated as the ``score" of the density ratio:
%where $u(y,t) \simeq \nabla \log h_t(\Xref_t) = \nabla \log \E_{\pmREF}[\exp(-f_K(\Xref_T))\mid \Xref_t=y]$, 
% which we call Doob's h-transform. 
% Consequently, using the approximated score function $\nabla \log p_{\mathrm{ref},t} \simeq s(\Ydoobd_{t},T-t)$ of the original model, 
% the score function of the fine-tuned diffusion model obtained by the Doob's h-transform
% can be obtained by 
% \begin{equation}
%  \nabla \log q_{K,t}(y)  \simeq  s(y,T-t)+u(y,t),
% \end{equation}
% as an approximate of the score $\nabla \log q_{K,t}(y)$. 
where $u(x,t)$ can be computed as $u(x,t) = \nabla_x \log \E[\exp(-f_K(X_T^{\leftarrow}))|X_t^{\leftarrow} = x]$,
which can be estimated by running the reference diffusion model $(X_t^{\leftarrow})$.   
% To implement the dynamics above we discretize the process and we need to approximate the score function by, e.g., neural networks.
%We further need to approximate $u(x,t)$, because this involves expectation and differentiation regarding some expectation of $f_K(x)$. 
The practical treatment for this is discussed in Appendix~\ref{section:appendix-error-diffusion}. 
\revisedStart For experimental information, please have a look at Section~\ref{sec:numerical-experiments} and Algorithm~\ref{alg:doob-sampling} in Appendix~\ref{sec:Appendix-Experimennt}.\revisedEnd

\vspace{-2mm}
\section{Theoretical Analysis \label{section_theoretical}}
\vspace{-2mm}
In this section, we give theoretical justification of our proposed algorithm.
More concretely, we show the rate of convergence of the (inexact) DA method and give an approximation error bound on the diffusion model based on the h-transform. 

\vspace{-2mm}
\subsection{Convergence rate of the DA method}
\vspace{-1mm}

We give the convergence rate of the DA algorithm in the two settings: when $F$ is (I) 
 convex and (II) non-convex, respectively.  

\noindent \textbf{(I): Convex objective $F$.}
First, we show the rate when $F$ is convex. 
We basically follow the proof technique of \cite{NEURIPS2021_a34e1ddb,nishikawa2022twolayer}.
In the analysis, we put the following assumption on $F$. 
%for Theorem~\ref{thm-da-nishikawa} are as follows:
\begin{assumption}\label{ass:ConvexF}
The loss function $F$ satisfies the following conditions:     
\begin{enumerate}[topsep=0mm,itemsep=-1mm,leftmargin = 6mm]
    \item[(i)] $\dFdq$ is bounded: There exists $B_F > 0$ such that $\|\dFdq(q)\|_\infty \leq B_F$ for any $q \in \mathcal{P}$,
    \item[(ii)] $\dFdq$ is Lipshitz continuous with respect to the TV distance: There exists $L_{\mathrm{TV}}  > 0$ such that $\|\dFdq(q) - \dFdq(q')\|_\infty \leq L_\mathrm{TV} \TV(q,q')$ for any $q,q' \in \mathcal{P}$.
    \item[(iii)] $F$ is convex: $F(q) \geq F(q') + \int \dFdq(q')\mathrm{d}(q-q')$ for any $q,q' \in \mathcal{P}$,
\end{enumerate}
\end{assumption}

Then, we can show that the (inexact) DA algorithm achieves the following convergence rate that yields $\mathcal{O}(K^{-1})$ convergence of the objective.
\begin{thm}[\revisedStart Convergence of the objective in Option 1 \revisedEnd]
    \label{thm-da-nishikawa}
    Suppose that $\beta' \geq \beta$ and we train the potential $f_{k+1}$ so that it is sufficiently close to $\gbark$ as 
    %\begin{equation}
        $\TV(\qstark,\qk) \leq \epsilon_\mathrm{TV}$ for all $k$. 
    %\end{equation}
    Then, under Assumption \ref{ass:ConvexF}, Option 1 satisfies the following convergence guarantee: 
  \begin{equation}
    \frac{2}{K(K+1)}\sum_{k=1}^{K}k \left[ L(\qstark) - L(\qstar) \right]
\leq 
2L_{\mathrm{TV}}\epsilon_{\mathrm{TV}}+\left(\frac{2B_F}{K(K+1)}  + \frac{2 \beta' \KL(q_*\|\pref) + 2B_F^2 \beta^{-1}}{K} \right).
  \end{equation}
%    where
%  \begin{equation}
%    $\epsilon_{\mathrm{FD}} = \frac{2}{K(K+1)}\sum_{k=1}^{K} k \epsilon_{\mathrm{TV}}$. 
%  \end{equation}
    % \begin{align}
    %     &\frac{2}{K(K+1)}\sum_{k=1}^{K}
    %     k \left[
    %         L(\qstark) - L(\qstar)
    %     \right]\\
    %     \leq&
    %     2L_\mathrm{TV}\epsilon_\mathrm{TV}
    %     +\mathcal{O}
    %     \left(
    %         \frac{B_F}{K^2}
    %         +
    %         \frac{
    %         \beta \KL(\qstar\|\pref)
    %             + \beta^{-1}B_F^2
    %         }
    %         {
    %             K
    %         }
    %     \right).
    % \end{align}
\end{thm}
See Appendix~\ref{sec:AppendixConvProof} for the proof.
From this theorem, we see that the DA algorithm with Option 1 achieves $\gO(1/K)$ convergence. The assumption (ii) in Assumption \ref{ass:ConvexF} is required to bound an expectation of $\gbark$ in the bound.  
The assumption (iii) is required to bound the difference between the exact update $\qstark$ and the inexact update $q^{(k)}$. If the update is exact, we don't need this assumption. 

\noindent \textbf{(II): Non-convex objective $F$.}
We also give a convergence for a non-convex loss $F$.  
Here, we put the following assumption. 
\begin{assumption}\label{ass:NonconvexF}
The loss function $F$ satisfies the following conditions:     
\begin{enumerate}[topsep=0mm,itemsep=-1mm,leftmargin = 6mm]%\label{ass:NonconvexF}
    % \item[(i)] $F$ is smooth: There exists $S_F \geq 0$ such that $F(q) \leq F(q') + \int \dFdq(q')\mathrm{d}(q-q') + \frac{S_F}{2}\KL(q\|q')$ for any $q,q' \in \gP$,
    \item[(i)] $\dFdq$ is bounded: There exists $B_F > 0$ such that $\|\dFdq(q)\|_\infty \leq B_F$ for any $q \in \gP$,
    \item[(ii)] $\dFdq$ is Lipshitz continuous with respect to the TV distance: $\|\dFdq(q) - \dFdq(q')\|_\infty \leq L_{\mathrm{TV}} \TV(q,q')$ for any $q,q' \in \gP$, %\textcolor{red}{TV or KL}
    \item[(iii)] $F$ is lower bounded.
\end{enumerate}
\end{assumption}
Assumptions (i) and (ii) are the same as the convex case (Assumption \ref{ass:ConvexF}), and lower boundedness (iii) is weaker than convexity in Assumption \ref{ass:ConvexF}.
% The smoothness condition (i) is a quite standard assumption in the finite dimensional optimization literature. It is a just functional derivative version of its finite dimensional counterpart. 
Assumption (ii) induces the following type of smoothness commonly observed in standard optimization: (ii)' There exists $S_F \geq 0$ such that $F(q) \leq F(q') + \int \dFdq(q')\mathrm{d}(q-q') + \frac{S_F}{2}\KL(q\|q')$.
%\begin{itemize}
    %\item[(ii)'] (A weaker version of (ii)). \\
    %There exists $S_F \geq 0$ such that $F(q) \leq F(q') + \int %\dFdq(q')\mathrm{d}(q-q') + \frac{S_F}{2}\KL(q\|q')$
%\end{itemize}
When the inner-loop error is ignored, it is possible to prove convergence using only the smoothness assumption (ii)' instead of assumption (ii). For details, please refer to Appendix~\ref{sec:AppendixNonconvexConv}.

Under this assumption, we can show the following convergence guarantee with respect to the sequence $(\qstark)_{k=1}^K$ even in a non-convex setting.  
\begin{thm}\label{thm:NonconvexConv}
%%%%%%%%%%%% S_F = 2 * L_TV %%%%%%%%%%%%%%
Suppose that $\beta' > \beta + \LipTV$ and $\TV(\qstark,\qk) \leq \epsilon_\mathrm{TV}$ for all $k$, then under Assumption \ref{ass:NonconvexF}, both Option 1 and 2 yield the following convergence: 
    \begin{itemize}[topsep=0mm,itemsep=-1mm,leftmargin = 6mm]
        \item[(i)] $\lim_{k \to \infty} \KL(\qstarnext\|\qstark) = 0$.
        \item[(ii)] For all $K$, it holds that
        \begin{equation}
            \underset{k=1,...,K}{\min}\left\{c_k \KL(\qstarnext \|\qstark \right) \}
            \leq 
            \frac{(\tilde{L}_1(\qstarone - L(\qstar)) + (\LipTV +B_F) K \epsilon_\mathrm{TV} }{K\beta'} =: \Psi_K,
        \end{equation}
        where $c_k = \frac{\beta k + 2 \beta' }{2}$ for Option 1 and $c_k =1$ for Option 2. 
    \end{itemize}
\end{thm}
See Appendix~\ref{sec:AppendixNonconvexConv} for the proof.
The proof utilizes an analogous argument with \cite{LIU2023nonconvexDA} that analyzed convergence of a DA method in a finite dimensional non-convex optimization problem. However, we need to reconstruct a proof because we should work on the probability measure space, which is not a vector space, and carefully make use of the property of the KL-divergence. 
We see that $\Psi_K = \gO(1/K)$ if $\epsilon_\mathrm{TV}$ is sufficiently small as $\epsilon_\mathrm{TV} = \gO(1/K)$, and thus the discrepancy between $\qstarnext$ and $\qstark$ converges. 
Especially, the convergence of $\KL(\qstarnext\|\qstark)$ yields the convergence of the ``dual variable'' for Option 2 as in the following corollary. 
For that purpose, we define 
%\begin{equation}
$\tilde{L}_k(q) \coloneq L(q) + \frac{\beta'}{k}\KL(q\|\pref)$ 
%\end{equation}
(see its similarity to the inner objective of the DA update \eqref{eq:DAAlg1} and \eqref{eq:DAAlg2}), 
and define $\psi_{q}\left( g \right) = \log\left(\E_q[\exp( - g + \E_q[g])] \right)$ which is a ``moment generating function'' of a dual variable $g$. 
\begin{cor}[\revisedStart Convergence in Option 2 \revisedEnd] \label{cor:DualConv}
Under the same condition as Theorem \ref{thm:NonconvexConv}, we have that 
$$
\textstyle \min\limits_{1 \leq k \leq K} \psi_{\qstark}\left(\frac{k}{\beta' (k+1)}\frac{\delta \tilde{L}_k}{\delta q}(\qstark)  \right) 
\leq  \Psi_K.
$$
\end{cor}
\vspace{-0.3cm}
Roughly speaking, this corollary indicates that 
the variance of the gradient $\frac{\delta \tilde{L}_k}{\delta q}(\qstark)$ converges as $\mathrm{Var}_{\qstark}\left(\frac{\delta \tilde{L}_k}{\delta q}(\qstark)\right) = \gO(1/K)$ because we may approximate $\psi_{q}(g) \simeq \frac{1}{2}\mathrm{Var}_q(g)$ when $|g- \E_q[g]|$ is sufficiently small.
Therefore, we have a convergence guarantee of the dual variable (gradient) $\frac{\delta \tilde{L}_k}{\delta q}(\qstark,x) \to 0 \; \text{(up to a constant w.r.t. $x$)}$ even in a non-convex setting, which justifies usage of our method for general objective functions.  

\vspace{-1mm}
\subsection{Discretization error of Doob's h-transform}\label{subsection:doob}
\vspace{-1mm}

We now provide the sampling error analysis after obtaining the approximate solution $f_K$.
\revisedStart
In the analysis of Dual Averaging, $\pref$ was assumed to be known, and the goal was to obtain the optimal solution of (\ref{eq:setup-minimization}) that corresponds to $\qstar = \exp(-\fopt)\pref$. From here, considering that $\pref$ estimates $p_*$, we shift our focus to sample from the optimal alignment $q_* \propto \rho_* p_*$, \revisedEnd while the dynamics we implement involves several approximations:    
(i) time discretization approximation, (ii) approximation of the score $\nabla_x \log p_t$ by $s(x,T-t)$, (iii) approximation of $\rho^*$ by $\rho = \exp(-f_K)$, (iv) approximation of $u_*$ by $u$.  
%, (iv) empirical approximation of $\nabla \log \rho_t(y)$. 

%The first approximation is discretization of the dynamics (\ref{eq:h-transformed-backward}). 

% In addition to this, we need to approximate the correction term $u_*(x,t)=\nabla_x\mathbb{E}_{\mathbb{P}_{\rm ref}}[\exp (- f_*(X_0))|X_t=x]$ by using $u = \exp (- f_K)$, as
% \begin{align}
%     \hat{u}(x,kh) =  
% & 
% \frac{ \frac{1}{n} \sum_{i=1}^n \left[ 
% \rho'({x}_{i,(T - lh)/h}) 
% \frac{(I - 2 \nabla^\top s(x,lh))}{\sqrt{e^{2h}-1}} [{x}_{i,(l+1)h} - (x - 2 s(x,lh))] 
% \right]}{\frac{1}{n} \sum_{i=1}^n \rho'({x}_{(T - lh)/h}) },
% \end{align}
% where 
% $D_{n,l}(x) := (({x}_{i,kh})_{k=l+1}^{(T - lh)/h})_{i=1}^n$ is an i.i.d. sequence generated from our reference model with a initial condition $X_{lh} = x$.  
% By using $\hat{u}(x,kh)$, we obtain the implementable dynamics as
% \begin{equation}
%     \mathrm{d}\Ydoobc_t = [\Ydoobc_t+ 2 (s(\Ydoobc_{\tprev},\tprev) + \hat{u}(\Ydoobc_{\tprev},\tprev))] 
%     + \sqrt{2}\mathrm{d}B_t, \quad t \in [\tprev, \tnext].
% \end{equation}

% \begin{equation}
%     \mathrm{d}\bar{X}^{\leftarrow}_t = (\bar{X}_t^\leftarrow + 2\nabla_x p_{T-t}(\bar{X}^\leftarrow_{\tprev}) ) \rd t 
%     + \sqrt{2}\mathrm{d}B_t, \quad t \in [\tprev, \tnext].
% \end{equation}
% The SDE corresponding to our model: 





%Here, we want to bound the discrepancy between $q_* \propto \rho_* p_*$ and $\hat{q} \propto \rho \pref$ where $\rho = \exp(- f_K)$ obtained by the DA algorithm in Phase 1. 

% In the above SDE, $\nabla \log \rho_{t}(x)$ is calculated as
% \begin{align}
%     &\nabla \log\rho_t(x)\\
%     =&\nabla \log \E_{\pmREF}[\rho_T(\Xref_T) \mid \Xref_t = x]\\
%     =& 
%     \frac{
%         \int
%         \E_{\pmREF}[\rho_T(\Xref_T)\mid \Xref_{\tnext}=y]\frac{\partial}{\partial x}\mathcal{N}(\Xref_{\tnext}\mid m_{\tnext - t}(x), \sigma^2_{\tnext - t})\mathrm{d}y
%     }{
%         \int
%         \E_{\pmREF}[\rho_T(\Xref_T)\mid \Xref_{\tnext}=y]\mathcal{N}(y \mid m_{\tnext - t}(x), \sigma^2_{\tnext - t})\mathrm{d}y
%     },
% \end{align}
% where $m_{u}(x) = e^u x + (-2 s_{T-\tprev}(x))(1-e^u), \; \sigma^2_u = e^{2u} - 1$.
% The partial derivative $\frac{\partial}{\partial x}\mathcal{N}(y \mid m_{\tnext - t}(x), \sigma^2_{\tnext - t})$ is
% \begin{equation}
%     \frac{\left(\frac{\partial}{\partial x}m_{\tnext - t}(x)^\top \right) (m_{\tnext - t}(x) - y)}{\sigma_{\tnext - t}^2}\mathcal{N}(y \mid m_{\tnext - t}(x), \sigma^2_{\tnext - t}).
% \end{equation}
% Note that $\left(\frac{\partial}{\partial x}m_{\tnext - t}(x)^\top \right)$ is a $d \times d$ matrix and $\left(\frac{\partial}{\partial x}m_{\tnext - t}(x)^\top \right)_{ij} = \frac{\partial}{\partial x_i}m_{\tnext - t}(x)_j$.
% Finally, we discretize $\nabla \log \rho_t$ with respect to time $t$:
% \begin{equation}
%     \mathrm{d}\Ydoobd_t = \lbrace  \Ydoobd_t+ 2s(\Ydoobd_{\tprev},\tprev) + 2\nabla \log \rho_{\tprev}(\Ydoobd_{\tprev}) \rbrace
%     + \sqrt{2}\mathrm{d}B_t, \quad t \in [\tprev, \tnext].
% \end{equation}
%Let $u_*(x,t)$ be the optimal correction term as $u^*(x,t) = \nabla_x \log(\E[\rho_*(\bar{X}_T^\leftarrow)|\bar{X}_t^\leftarrow = x])$ where $\bar{X}_t^\leftarrow$ is the reverse process of the true reference distribution $p_t$. 


To evaluate how such approximation affects the final quality of our generative model, 
we will give a bound of the TV-distance between $q_*$ and $\hat{q} = \rho \pref$ by putting all those errors together. 
To do so, we put the following assumption.  
\begin{assumption}\label{assumption:TVBoundMainText}
\begin{enumerate}[topsep=0mm,itemsep=-1mm,leftmargin = 6mm]
\item 
    $\nabla \log p_t$ is $L_p$-smooth at every time $t$ and it has finite second moment $\mathbb{E}[\|\bar{X}_t\|^2_2] \leq \mathsf{m} < \infty$ for all $t\in \R_+$ and $x\in \R^d$. 
\item  $\nabla \log \rho_*$ is $L_\rho$-smooth and bounded as $C_\rho^{-1}\leq \rho_* \leq C_\rho$ for a constant $C_\rho$.
\item   The score estimation error is bounded by 
\revisedStart
$\E_{\bar{X}_{\cdot}^{\leftarrow}}[\|s(\bar{X}_{t}^{\leftarrow},t) \!\! - \!\! \nabla \log p_{T-t}(\bar{X}_{t}^{\leftarrow})\|^2]\!\leq \!\varepsilon$ 
\revisedEnd
at each time $t$. 
\item 
$\E_{\bar{X}_{\cdot}^{\leftarrow}}[\|u_*(\bar{X}_t^{\leftarrow},t) - u(\bar{X}_{lh}^{\leftarrow},lh)\|^2] \leq \varepsilon_{\rho,l}^2$~~for any $1 \leq l \leq T/h$
and $t \in [lh,(l+1)h)$.
%$\E_{X_{lh}}[\E_{\bar{D}_{n,l}(X_{lh})}[\|u_*(X_{lh},lh) - \hat{u}(X_{lh},lh)\|^2]] \leq \varepsilon_{\rho,l}^2$~~for any $1 \leq l \leq T/h$
\end{enumerate}
\end{assumption}
This assumption is rather standard, for example, \cite{chen2023sampling} employed these conditions except the last condition on $u$ and $u_*$.
The fourth assumption in Assumption~\ref{assumption:TVBoundMainText} is imposed to mathematically formulate the situation: $\qstar/\pref\simeq q_*/p_*$.  
%Under this assumption, 
Then, we obtain the following error bound: 
\begin{thm}\label{thm:Diffusion-1}
Suppose that Assumption \ref{assumption:TVBoundMainText} is satisfied. Then, we have the following bound on the distribution $\hat{q}$ of $Y^\leftarrow_T$: 
\begin{equation}
\textstyle
\TV(q_*,\hat{q})^2
\lesssim  T \varepsilon^2 + \sum_{l=1}^{T/h} h \varepsilon_{\rho,l}^2 + T (L_pC_\rho^2+L_\rho)^2(dh + \mathsf{m} h^2 )+ \exp(-2 T)\KL(q_* \| N(0,I)).
\end{equation}
\end{thm}
The proof is given in Appendix \ref{sec:AppendixCorrectionTermBound}. 
It basically follows \cite{chen2023sampling,chen2023improved}, but we have derived the smoothness of $\nabla \log(q_{*,t})$ from that of $\nabla \log(p_{t})$. 
In this bound, we did not give any evaluation on $\varepsilon_{\rho,l}^2$, however, this error term can be bounded as follows with additional conditions. 
\begin{assumption}\label{ass:BoundingHMainText}
%\begin{enumerate}[topsep=0mm,itemsep=-1mm,leftmargin = 6mm]\label{ass:NonconvexF}
    %\item[(i)] 
    (i) $\nabla_x s(\cdot,\cdot)$ is $H_s$-Lipschitz continuous in a sense that $\|\nabla_x s(x,t) - \nabla_y s(y,t)\|_{\mathrm{op}} \leq H_s \|x- y\|$
for any $x,y \in \sR^d$ and $0 \leq t \leq T$ and $\E[\| s(\tilde{X}_{kh}^\leftarrow,kh)\|^2] \leq Q^2$ for any $k$, 
    %\item[(ii)] 
    (ii) There exists $R > 0$ such that $\sup_{t,x}\{\|\nabla_x^2 \log p_t(x)\|_{\mathrm{op}},\|\nabla_x^2 \log s(x,t)\|_{\mathrm{op}}\} \leq R$.
%\end{enumerate}
\end{assumption}
\begin{thm}\label{thm:Diffusion-2}
Suppose that Assumptions \ref{assumption:TVBoundMainText} and \ref{ass:BoundingHMainText} hold 
%the same condition as Theorem \ref{thm:delhDiffXYExp} holds 
and $\|\rho_* - \rho\|_\infty \leq \varepsilon'$ and $\|\rho\|_\infty \leq C_\rho$. 
We also assume $\nabla \rho^*$ is bounded and Lipschitz continuous. 
%Let $L_\varphi$ and $R_\varphi$ be as given in Lemma \ref{lemm:phiYboundLip}.
Then, for any choice of $0 \leq h \leq \delta \leq 1/(1 + 2R)$, we have that 
\begin{align}
%& %\E_{\bar{Y}_{\cdot}^{\leftarrow}}[ \|\nabla_x \E[\rho(\tilde{X}_{T}^\leftarrow) |  \tilde{X}_{t}^\leftarrow = x] |_{x = \bar{Y}_{t}^{\leftarrow}} -  \hat{u}(\bar{Y}_{k_t h}^\leftarrow,t) \|^2]  
\textstyle \varepsilon_{\rho,l}^2 %\\
\lesssim & \textstyle 
C_\rho^3 \left\{ \Xi_{\delta,\varepsilon}
+  R_\varphi^2 \left(\varepsilon^2+ \Lipdp^2 d(\delta + \mathsf{m} \delta^2) \right) + 
[ L_\varphi^2 (\mathsf{m} + Q^2 + d h)  
+ R_\varphi^2  (1 + 2R)^2] h^2\right\} \\
& \textstyle +   \min\{T-lh,1/(2+2R)\}^{-1} \varepsilon'^2,
%+  C_\rho (1 + 2R) \sqrt{\frac{\log(T/h)}{n h}}, 
%= & \gO\left(\varepsilon^2 + \delta + \frac{\varepsilon_\TV^2}{\delta}\right), 
\end{align}
where 
%\begin{align}
$
\Xi_{\delta,\varepsilon} := C_\rho^2 (1+2R)^2 \delta  
+  C_\rho^2 \frac{1 + \delta R_\varphi^2}{\delta} [\varepsilon^2+ \Lipdp^2 d(h + \mathsf{m} h^2)], 
%\varepsilon_{\TV}^2
%+  R_\varphi^2  [\varepsilon^2 + \Lipdp^2 d(\delta + \mathsf{m} \delta^2)],
%&  = \gO\left(\delta + \varepsilon^2 + \frac{\varepsilon_\TV^2}{\delta}\right),
$ and 
$R_\varphi$ and $L_\varphi$ are constants introduced in Lemma \ref{lemm:phiYboundLip}. 
%\end{align} 
%and $c_\eta > 0$ is a universal constant. 
\end{thm}
The proof is given in Appendix~\ref{sec:proofOfCorrection}.
We applied the so-called {\it Bismut-Elworthy-Li} integration-by-parts formula \citep{MR755001,ELWORTHY1994252} to obtain the discretization error. 
By substituting $\delta \leftarrow \sqrt{h}$ to balance the terms related to $h$ and $\delta$, we obtain a simplified upper bound of $\sum_{l=1}^{T/h} h \varepsilon_{\rho,l}^2$ as 
$
\sum_{l=1}^{T/h} h \varepsilon_{\rho,l}^2 \lesssim 
\left( 1 + \frac{1}{\sqrt{h}}\right) \varepsilon^2 T + T \sqrt{h} + ( T + \log(1/h)) \varepsilon'^2. 
$
These results can be seen as $h$-Transform extension of the approximation error analysis given in \cite{chen2023sampling}. 
However, the approximation error corresponding to $u$ has worse dependency on $h$. This is because the computation of $u$ uses the discretized process $\pref$ and is affected by its error while the ordinary diffusion model does not require sampling to obtain the score. 

\vspace{-2mm}
\section{Numerical Experiments}\label{sec:numerical-experiments}
\vspace{-2mm}

We conducted numerical experiments to confirm the effectiveness of minimizing nonlinear objectives. We used Option 1 and $\beta$ was set to be 0.04.
We also compared the DPO objective~\citep{rafailov2023DPO} we minimized with the upper-bound~\citep{Wallace2024DiffusionDPO} using a toy case, Gaussian Mixture Model (GMM).
For detailed experimental setting, please refer to Appendix~\ref{section:appendix-experiments}.

\textbf{Alignment for Gaussian Mixture Model.}\quad
We aligned a score-based diffusion model to sample from a 2-dimensional GMM using the DPO objective. The reference density was defined as $\frac{1}{2}\left(\mathcal{N}(\mu_1,\Sigma) + \mathcal{N}(\mu_2,\Sigma)\right)$, $\mu_1 = [-2.5, 0], \; \mu_2 = [2.5, 0], \; \Sigma = [[1, 0],[0,5]]$.
The target point was $\mu_w := \mu_2$.
The preference of points $x_w$ and $x_l$ was determined by the Euclidean distance $d(\cdot, \mu_w) $ from $\mu_w$. $x_w \succ x_l$ if and only if $d(x_w,\mu_w) < d(x_l,\mu_w)$. \revisedStart We describe the implementation details below:

% \begin{itemize}[topsep=0mm,leftmargin = 8mm]
% \item[(1)]

\textit{Dual Averaging.}\quad
We set the hyperparameter $\beta'$ in [0.04, 0.2], which controls the learning speed illustrated at the middle in Figure~\ref{fig:da-summary}.
In $k$th DA iteration ($k=1,...,K$), using empirical samples $\lbrace x_i = (x_{1,i},x_{2,i})\rbrace_{i=1}^{1000}$ from $\pref$ and the previous potential $f_{k-1}$ ($q^{(k-1)} \propto e^{-f_{k-1}}\pref$) implemented by neural networks, we prepare $\lbrace (x_i, \dFdq(q^{(k-1)},x_i))\rbrace_{i=1}^{1000}$ to construct $f_{k} \simeq \bar{g}^{(k-1)}$ with the recurrence formula (\ref{eq-appendix-experiment-recurrence}) in Appendix~\ref{section:appendix-experiments}.
Please note that, to calculate $\dFdq$ for DPO, we only need the $f_k $ and empirical samples from $\pref$ to calculate the expectation $\E_{x\sim \pref}[\cdot]$ as described in Eq. (\ref{eq-main-DPO-derivative}).
The convergence of the true DPO loss is depicted in Figure~\ref{fig:da-summary}, compared to Diffusion-DPO~\citep{Wallace2024DiffusionDPO}. The true DPO loss, upperbound, and metric loss (mean of the Euclidean distance of the particles from $\mu_w$) attained by Diffusion-DPO and our method, with optimization time and GPU memory consumption (in Phase I for ours), are summarized in Table \ref{table:comparison}.
%In practice, it is hardly realistic to compute the true loss during Diffusion-DPO, but in this case, we forcefully carried out the computation: The densities $q$ and $\pref$ were estimated by repeatedly computing the denoising path and empirically obtaining their marginal densities.
Diffusion-DPO optimizes the approximate upperbound instead of the true loss, and therefore it failed to completely control the true loss.
Compared to Diffusion-DPO, our algorithm allowed us to optimize the loss to a smaller value within an acceptable computational budget. 
% \textit{Comparison with the upper-bound in the previous work.}\quad
% We see that all of them decreased during DA, however, a noticeable gap persisted between the two, as shown in Figure~\ref{fig:da-summary}. 
% This indicates that the upper bound is not tight enough to guarantee the minimization of the original objective.
\revisedEnd

% \item[(2)]
\textit{Doob's h-transform.}\quad
We constructed the aligned model by calculating the correction terms. The conditional expectations were calculated using naive Monte Carlo method with 30000 samples.
The number of the diffusion steps was 50. The histograms of aligned samples are shown at the right of Figure~\ref{fig:da-summary}. 
% The empirical densities of them reflect non-linearity of the objective. 
\revisedStart
In the simplest phase 2 that we present (Algorithm~\ref{alg:doob-sampling}), we used significant computational resources, with $\mathcal{O}(L^2)$ time complexity. To estimate the correction term $u$ for each sample simultaneously by Monte Carlo with $N$ samples, $\mathcal{O}(N)$ memory space is required. However, this Doob's h-transform technique itself has been used in image generation~\citep{uehara2024RLHF,uehara2024reward}, Bayesian samping~\citep{heng2024schrodingerbridge}, and filtering~\citep{chopin2023doob}. In particular, computational resources can be saved by approximating the correction term using a neural ODE~\citep{uehara2024reward}. 
\revisedEnd
% \end{itemize}
\begin{figure}[htbp]
    \centering
    \includegraphics[width=0.99\linewidth]{figures/MoG_summary_0305.png}
    %\caption{\revisedStart
    %\textbf{Upper Left and Upper Center.} The loss during optimization for Gaussian Mixture Model in Diffusion-DPO with/without regularization (left) and ours (center).
    %    ``True Objective'': the true DPO loss~\citep{rafailov2023DPO} whose target point was $\mu_w = [2.5, 0]$.
    %    %``Regularized Objective'': ``Objective" + $\beta \KL(q\|\pref)$,
    %    ``Upperbound'': An upperbound of ``Objective'' optimized in Diffusion-DPO~\citep{Wallace2024DiffusionDPO}.
    %    %``Regularized Upperbound'': ``Upperbound'' + $\beta \KL(q\|\pref)$, $\beta=0.04$
    %    \textbf{Upper Right.} Aligned samples by Doob's h-transform. ``iter" represents the iteration of DA. ``reference'' represents the empirical density of $\pref$.
    %    \textbf{Bottom.} The quantative comparison between Diffusion-DPO with/without regularization ($\beta=0.04$ or $0$) and our DDO. In addition to the true DPO loss, ``Metric Loss" is the mean of the Euclidean distance of the particles from $\mu_w$, as one of the benchmarks. The computational efficiency was evaluated during the optimization phase (our phase 1). \revisedEnd}
    \vspace{-2mm}
    \caption{\revisedStart
    \textbf{Left and Middle.} The smoothed loss during optimization for Gaussian Mixture Model in Diffusion-DPO with/without regularization (left) and ours (middle).
        ``True Objective'': the true DPO loss~\citep{rafailov2023DPO} whose target point was $\mu_w = [2.5, 0]$.
        %``Regularized Objective'': ``Objective" + $\beta \KL(q\|\pref)$,
        ``Upperbound'': An approximate upperbound of ``Objective'' optimized by Diffusion-DPO~\citep{Wallace2024DiffusionDPO}.
        %``Regularized Upperbound'': ``Upperbound'' + $\beta \KL(q\|\pref)$, $\beta=0.04$
        \textbf{Right.} Aligned samples by Doob's h-transform. ``Ref.'' represents the empirical density of $\pref$.
        \revisedEnd}
    \vspace{-3mm}
    \label{fig:da-summary}
\end{figure}
\vspace{-2mm}
\begin{table}[ht]
\vspace{-3mm}
    \centering
    \caption{The quantitative comparison between Diffusion-DPO and our method}
\scalebox{0.85}{
    \begin{tabular}{lcccc}
        \toprule
        & Ref. & \begin{tabular}{c}Diffusion-DPO\\(50 iter., w/o reg.)\end{tabular} & \begin{tabular}{c}Diffusion-DPO\\(200 iter., w/ reg.)\end{tabular} & \begin{tabular}{c}\textbf{Ours}\\($\beta'=0.04$)\end{tabular} \\
        \midrule
        True DPO loss             & 0.346 & 0.340 & 0.343 & \textbf{0.328} \\
        (Approx.) Upperbound      & -     & 0.311 & 0.337 & -              \\
        Metric loss               & 3.256 & 3.011 & 3.303 & \textbf{2.022} \\
        Opt. time (s)             & -     & 585   & 2280  & 1320 (Phase I)    \\
        GPU memory (\%)           & -     & 6.54  & 6.54  & 8.75 (Phase I)   \\
        \bottomrule
    \end{tabular}\label{table:comparison}
    }
\end{table}
% 実行時間, gpuの修正 (20250304)
% w/o: 34min * 50/200
% w/ : 38min
%ours: 22min (speed testのため, beta' = 0.04のみ, 念の為やり直した)

%\revisedEnd
\vspace{-1mm}
\textbf{Image Generation Alignment based on a Target Color.}\quad
We aligned the image generation of the basic pre-trained model available Diffusion Models Course (source: \cite{huggingfacecourse}) to simulate specific generations for data augmentation. The pre-trained model generates RGB-colored $32\times 32$ pixel images of butterflies. 
\revisedStart
In this experiment, the model was aligned based on a target color, which is (R,G,B)= (0.9, 0.9, 0.9) while (1,1,1) corresponds to white,
%cabbage butterflies would be more preferred,
using the DPO objective. 
Please refer to Figure~\ref{fig:BT-compare} for an illustration of the preference indicated by DPO.
\revisedEnd
We leveraged 1024 images generated by the pre-trained model to train $f_k$ in each DA iteration. 
We see that the (regularized) objective went down during DA and the generated images reflected the target density more, on the right side of Figure~\ref{fig:BT-and-CT}. Please refer to Figure~\ref{fig:BT-summary} in Appendix for the convergence results.

\textbf{Tilt Correction for Image Generation.}\quad
The goal of this experiment is data augmentation of images facing the same direction, given only a dataset of rotated images. We used 10000 images of Head CT in Medical MNIST~\citep{MedicalMNISTClassification} and augmented it by rotating images up to 40000. Data augmentation of brain images is useful for tumor analysis and anomaly detection~\citep{shen2024MRIdiffusion,fontanella2024BrainDiffusionGenerationAnomaly}. In each DA iteration, 6400 images were generated to train $f_k$.
The results are in Figure~\ref{fig:BT-and-CT}. For the convergence results, please refer to Figure~\ref{fig:CT-summary} in Appendix.
From the results, we see that the (regularized) objective decreased during DA and the generated images became more preferred.
\begin{figure}[h]
    \centering
    \vspace{-3mm}
    \includegraphics[width=0.5\linewidth]{figures/BT_and_CT.png}
    \vspace{-2mm}
    \caption{\revisedStart
    \textbf{Left.} Examples of aligned image generation. Our goal was to generate light-colored butterflies. ``iter=2'': ours with $k=2$ DA iterations, ``iter=2'': ours with $k=1$ DA iteration. ``Ref.'': samples from $\pref$. \textbf{Right.} Tilt-corrected Head CT image generation. ``iter=3'': ours with $k=3$ DA iterations, ``iter=1'': ours with $k=1$ DA iteration. ``Ref,'': samples from $\pref$.
    \revisedEnd
    }
    \label{fig:BT-and-CT}
\end{figure}


% \vspace{-7mm}
\vspace{-4mm}
\section{Conclusion}
\vspace{-2mm}

We proved that the distributional optimization can be solved even if $\pref$ and $\qstar$ are mutilmodal and we only have access to the score, not the density.
This setting includes important fine-tuning methodologies for diffusion models: Reinforcement Learning, DPO, and KTO.
Our algorithms are based on the DA algorithm and Doob's h-transform technique and it can solve them more directly than previous works.
They are guaranteed to be useful for general objective functions, the dual variable converges even in a non-convex setting.
While our framework has potential applications in general distributional optimization problems, such as density ratio estimation under covariate shift and climate change tracking, further exploration of these applications is left for future work.
% \subsubsection*{Author Contributions}
% If you'd like to, you may include  a section for author contributions as is done
% in many journals. This is optional and at the discretion of the authors.

% \subsubsection*{Acknowledgments}
% Use unnumbered third level headings for the acknowledgments. All
% acknowledgments, including those to funding agencies, go at the end of the paper.


\section*{Acknowledgments}
RK was partially supported by JST CREST (JPMJCR2015).
KO is partially supported by JST ACT-X (JP-MJAX23C4).
TS was partially supported by JSPS KAKENHI (24K02905, 20H00576) and JST CREST (JPMJCR2115).
% Use unnumbered third level headings for the acknowledgments. All
% acknowledgments, including those to funding agencies, go at the end of the paper.
This research is supported by the National Research Foundation, Singapore, Infocomm Media Development Authority under its Trust Tech Funding Initiative, and the Ministry of Digital Development and Information under the AI Visiting Professorship Programme (award number AIVP-2024-004). Any opinions, findings and conclusions or recommendations expressed in this material are those of the author(s) and do not reflect the views of National Research Foundation, Singapore, Infocomm Media Development Authority, and the Ministry of Digital Development and Information. 

\bibliography{iclr2025_conference}
\bibliographystyle{iclr2025_conference}

\appendix

\clearpage

\section*{Notations}


% \section*{List of Symbols}
\setlength{\LTpre}{0pt}
\setlength{\LTpost}{0pt}
%\begin{longtable}{|c|l|}
\begin{longtable}{|>{\raggedright\arraybackslash}p{3cm}|>{\raggedright\arraybackslash}p{9cm}|}
\caption{List of commonly used symbols}
%\centering
%\begin{tabular}
\\
\hline
Symbol & Description \\ \hline

$d$ & dimension of the sample space $\mathbb{R}^d$. \\ & \\ % \hline
$\mathcal{B}(\mathbb{R}^d)$ & Borel set on $\mathbb{R}^d$. \\ & \\ % \hline
$\mathcal{P}$ & The space of probability density functions to the Lebesgue measure on $(\mathbb{R}^d, \mathcal{B}(\mathbb{R}^d))$. \\  & \\ % \hline

% 
$F:\mathbb{R}^d:\mathcal{P}\to \mathbb{R}$ & A functional.  \\ & \\ % \hline

$\dFdq(q,x)$ & The functional derivative of $F$. We often abbreviate this as $\dFdq(q)$. \\ & \\ % \hline

$\KL(q\|p)$ & Kullback-Leibler divergence between $q$ and $p$.  \\ & \\ % \hline
$\mathrm{TV}(q,p)$ & Total Variation between $q$ and $p$.  \\ & \\ % \hline

$L(q)$ & The regularized loss $F(q) + \beta\KL(q\|\pref)$.  \\ & \\ % \hline
$\beta$ & The regularization coefficient of $L(q)$.  \\ & \\ % \hline


$p_*\in\gP$ & The target distribution of pre-training. \\ & \\ % \hline
$\pref \in \gP$ & The output distribution at the final denoising step of the pre-trained model. \\ & \\ % \hline
$q_* = \rho_* p_* \propto \exp(-f_*) p_* \in \gP$ & The target distribution of alignment. \\ & \\ % \hline
% $\rho_*$ & The target density ratio of alignment. \\ & \\ % \hline
$\qstar = \hat{\rho}_{\mathrm{opt}}\pref \propto \exp(-\hat{f}_\mathrm{opt})\pref\in \gP$ & The unique minimizer of $L(q)$, that corresponds to $q_*$ when $\pref=p_*$. 
We assume that $\rho_* \propto \exp(-f_*)$ can be estimated by minimizing $L(q)$, i.e. $\hat{\rho}_\mathrm{opt} \simeq \rho_*$.
\\ & \\ 

$\qstark \propto \exp(-\gbark)\pref$ & The optimal solution of the subproblem solved iteratively during Dual Averaging.  \\ & \\

$\qk \propto \exp(-f_{k})\pref$ & The implemented density that estimates $\qstark$. \\ & \\

$\beta'$ & The hyperparameter of Dual Averaging that controls the optimization speed. \\ & \\ % \hline

$u_*$ & The optimal correction term to sample from $q_*$. \\ & \\


$u$ & The implemented correction term using the estimated output density ratio $\rho \simeq \hat{\rho}_\mathrm{opt}$. \\ & \\

$\mathcal{N}(\mu,\sigma)$ & Gaussian with mean = $\mu$ and variance = $\sigma^2$ \\ & \\ % \hline

$T > 0$ & The diffusion time. \\ & \\ % \hline
$\lbrace B_t \rbrace_{t\in[0,T]}$ & Brownian motion on $\mathbb{R}^d$. \\ & \\ % \hline

$\lbrace\bar{X}_t\rbrace_{t\in[0,T]}$ & The Ornstein-Uhlenbeck (OU) process starts from $p_*$. \\ & \\ % \hline
$\lbrace\bar{X}^\leftarrow_t\rbrace_{t\in[0,T]}$ & The reversed Ornstein-Uhlenbeck (ROU) process of $\lbrace\bar{X}_t\rbrace_{t\in[0,T]}$, initialized at $p_T$. \\ & \\ % \hline
$\lbrace X^\leftarrow_t\rbrace_{t\in[0,T]}$ & The time-discretized, and score-estimated ROU process of $\lbrace\bar{X}_t\rbrace_{t\in[0,T]}$, initialized as $X^\leftarrow_0 \sim \mathcal{N}(0,I_d)$. \\ & \\ %\hline
$\lbrace\bar{Y}_t\rbrace_{t\in[0,T]}$ & The Ornstein-Uhlenbeck (OU) process starts from $q_*$. \\ & \\ %\hline
$\lbrace\bar{Y}^\leftarrow_t\rbrace_{t\in[0,T]}$ & The reversed Ornstein-Uhlenbeck (ROU) process starts from $q_T$. \\ & \\ %\hline
$\lbrace Y^\leftarrow_t\rbrace_{t\in[0,T]}$ & The time-discretized dynamics using $s$ and $u$. \\ & \\ %\hline

$p_t$ & The law of $\bar{X}_t$, and the law of $\bar{X}_{T-t}^\leftarrow$. \\ & \\ %\hline
$q_t$ & The law of the OU process $\bar{Y}_t$ corresponding to $q_*=\rho_*p_*$. \\ & \\ %\hline

$h > 0$ & The discretized time step of $\lbrace X^\leftarrow_t\rbrace_{t\in[0,T]}$. \\ & \\  %\hline
$L$ & The number of steps of $\lbrace X^\leftarrow_t\rbrace_{t\in[0,T]}$.  \\ & \\ %\hline

$s:\mathbb{R}^d\times\mathbb{R}\to\mathbb{R}^d$ & A score network to implement $\lbrace X^\leftarrow_t\rbrace_{t\in[0,T]}$ \\ & \\

$B_F$ & A constant that bounds $\dFdq$. \\ & \\

$\LipTV$ & Lipschitz constant of $\dFdq$. \\ & \\

$S_F$ & Smoothness constant of the weaker smoothness of $F$. \\ & \\


$\nabla_x f(x)$ & Gradient of $f$ with respect to $x$ \\ & \\ % \hline
$\nabla_x^2 f(x)$ & Hessian matrix of $f$ \\ & \\ % \hline
$\|\cdot\|$ & Euclid norm \\ & \\
$\|\cdot\|_\infty$ & $L^\infty$ norm \\ & \\
$\|\cdot\|_{\mathrm{op}}$ & The operator norm \\ & \\

$\E[\cdot|X]$ & the conditional expectation given a randam variable $X$. \\ & \\

$\E[\cdot|X=x]$ & the conditional expectation given a randam variable $X$, evaluated at $X=x$.\\ & \\

$\mathcal{O}$ & Big-O notation \\ & \\
$\lesssim$, $\gtrsim$ & inequalities ignoring constants. \\ & \\

%\end{tabular}

\hline

\end{longtable}

\newpage
% \clearpage

\revisedStart
\section{Convergence Analysis of Dual Averaging}
\revisedEnd

\textbf{Overview for convex loss}

In Section~\ref{sec:AppendixConvProof}, we are interested in the convergence of Option 1 when the objective functional $F$ is convex in a distributional sense:
\begin{equation}
    F(q) \geq F(q') + \int \dFdq(q')\mathrm{d}(q-q'), \quad \text{for any $q,q'\in\mathcal{P}$}.
\end{equation}
Then, the regularized objective $L(q) = F(q) + \beta \KL(q\|\pref)$ becomes strongly convex with $\beta>0$.
Formally, we assume that
\begin{enumerate}[topsep=0mm,itemsep=-1mm,leftmargin = 6mm]
    \item[(i)] $\dFdq$ is bounded: There exists $B_F > 0$ such that $\|\dFdq(q)\|_\infty \leq B_F$ for any $q \in \mathcal{P}$,
    \item[(ii)] $\dFdq$ is Lipshitz continuous with respect to the TV distance: There exists $L_{\mathrm{TV}}  > 0$ such that $\|\dFdq(q) - \dFdq(q')\|_\infty \leq L_\mathrm{TV} \TV(q,q')$ for any $q,q' \in \mathcal{P}$.
    \item[(iii)] $F$ is convex: $F(q) \geq F(q') + \int \dFdq(q')\mathrm{d}(q-q')$ for any $q,q' \in \mathcal{P}$,
\end{enumerate}
In the update of Option 1, we iteratively use the distributional proximal operator. We minimize
\begin{equation}
\mathrm{E}_{q}\left[\sum_{j=1}^{k}\frac{2j}{k (k+1)} g^{(j)}\right] + \beta \KL(q\|\pref)  
+ \beta' \frac{2}{k}\KL(q\|\pref) \label{eq:appendixA-summary-min}
\end{equation}
by the minimizer $\qstark$, where  $\gk$ denotes $\dFdq(\qk)$ with $\qk \simeq \qstark$ ($\qk$ is constructed by neural networks and samples from $\pref$, while we do not directly compute $\qk$.). Intuitively, with $k \to \infty$, we hope that $q$ is almost converged around the minimizer $\qstar$, i.e. $q \simeq \qstar$. Then, from the differentiability of $F$, each $\E_q[g^{(j)}]$ in the first term 
would be a linear functional that well approximates $F(q^{(j)})$, with suficiently large $j$. In addition, $\beta' \frac{2}{k}\KL(q\|\pref)$ vanishes, so the equation (\ref{eq:appendixA-summary-min}) is roughly written as
\begin{align}
    &\sum_{j=1}^{k}\frac{2j}{k (k+1)} \left[ \mathrm{E}_{q}\left[g^{(j)}\right] + \beta \KL(q\|\pref) \right] 
+ \beta' \frac{2}{k}\KL(q\|\pref) \\
\simeq &\sum_{j=1}^{k}\frac{2j}{k (k+1)} (F(\qj) +\beta \KL(q\|\pref))
+ \beta' \frac{2}{k}\KL(q\|\pref)\\
\simeq &\sum_{j=1}^{k}\frac{2j}{k (k+1)} (F(\qstarj) +\beta \KL(\qstarj\|\pref)).
\end{align}
This is the weighted average of the regularized losses $L(q) = F(q) + \beta \KL(q\|\pref)$.
From this concept, we will show that the convergence represented as
\begin{equation}
    \left[ \text{Weighted average of } L(\qstark) \right] = \left[ \text{Weighted average of } L(\qstar) \right]+ \mathcal{O}\left(\frac{1}{k}\right).
\end{equation}

\textbf{Overview for nonconvex loss}

In Section~\ref{section:da-nonconvex-proof}, we are interested in the case that $F$ is not necessarily convex (mainly) in Option 2. 
Alternatively, we use the assumption that $F$ is smooth in terms of KL-divergence:
\begin{itemize}
\item[(ii)'] (The weaker smoothness derived from (ii)). There exists $S_F \geq 0$ such that 
\begin{equation}
    F(q) \leq F(q') + \int \dFdq(q')\mathrm{d}(q-q') + \frac{S_F}{2}\KL(q\|q') \quad \text{for any}\quad  q,q' \in \gP, \label{eq:appendix-da-nonconvex-overview-smoothness}
\end{equation}
\end{itemize}
This is induced by Lipschitz continuity of $\dFdq$ ((ii) of Assumption~\ref{ass:ConvexF} and ~\ref{ass:NonconvexF}) and Pinsker's inequality. When the inner-loop error is ignored, it is possible to prove convergence using only the smoothness (\ref{eq:appendix-da-nonconvex-overview-smoothness}) instead of Lipschitz continuity of $\dFdq$.
Please note that KL-divergence would be the ``quadratic term" in the ordinary definition of  smoothness.
Following the theoretical analysis of standard nonconvex optimization, our goal is to show that the ``derivative" of the (regularized) objective $L(q)$ goes to zero in the form of functional derivative:
\begin{equation}
    \frac{\delta L}{\delta q}(\qstark,x) \to 0 \quad \text{(up to constant w.r.t. $x$.)}
\end{equation}
Please note that we can ignore $\frac{\delta L}{\delta q}(\qstark,x)$ if it is a constant because of the definition of the functional derivatives. 
% 

At the first step, both in Option 1 and 2, the regularized objective $L(q)$ (roughly) monotonically decrease during the Dual Averaging. Ignoring some terms, we want to show that
\begin{equation}
    L(\qstarnext) - L(\qstark) \lesssim - \KL(\qstark\|\qstarnext) < 0.
\end{equation}
Next, by taking a telescoping sum, we obtain that
\begin{equation}
    \frac{1}{K}\sum_{k=1}^K (\text{weight})_k \KL(\qstark\|\qstarnext) = \mathcal{O}\left(\frac{1}{K}\right).
\end{equation}
From the above equation, it is immediately shown that
\begin{equation}
    \min_{k=1,...,K} (\text{weight})_k \KL(\qstark\|\qstarnext) = \mathcal{O}\left(\frac{1}{K}\right),
\end{equation}
where $(\text{weight})_k \simeq k$ in the Option 1, while $(\text{weight})_k \simeq 1$ in the Option 2.
Therefore, when we choose the Option 2, it holds that
\begin{equation}
    \min_{k=1,...,K} \KL(\qstark\|\qstarnext) = \mathcal{O}\left(\frac{1}{K}\right)
\end{equation}
Finally, we get the conclusion $\frac{\delta L(\qstark)}{\delta q} \to 0\; \text{(up to constant w.r.t. $x$)}$ because $\KL(\qstark\|\qstarnext)$ is the approximate ``moment generation function" of $\frac{\delta L(\qstark)}{\delta q}$; when $k \to \infty$,
\begin{equation}
    \min_{k=1,..,K}\left(\text{``Variance" of }\frac{\delta L(\qstark)}{\delta q} \right) \simeq \min_{k=1,..,K} \KL(\qstark\|\qstarnext) = \mathcal{O}\left(\frac{1}{K}\right).
\end{equation}
This result aligns with the result for standard, non-distributinal Dual Averaging~\citep{LIU2023nonconvexDA}, $\min_{k=1,...,K}\|\text{(gradient of the objective)}_k\|^2 = \mathcal{O}(1/K)$ because the variance is the second moment.

\subsection{When $F$ is convex\label{section:da-convex-proof}}\label{sec:AppendixConvProof}
% We provide the convergence analysis done in \textcolor{red}{cite} for completeness. Note that there are minor changes in the results from prior works.

We will show that, when $F$ is convex in the distributional sense, the convergence of the Option 1 can be written as 
\begin{equation}
    \left[ \text{Weighted average of } L(\qstark) \right] - \left[ \text{Weighted average of } L(\qstar) \right] = \mathcal{O}\left(\frac{1}{K}\right).
\end{equation}
We require $\beta$ to be positive to make the regularized objective $F(q) + \beta\KL(q\|\pref)$ strongly convex.
In addition, one of the necessary conditions $\beta' \geq \beta$ implies that the ``learning rate" should be controlled to converge.
\begin{thm*}[restated - Theorem~\ref{thm-da-nishikawa}]
Assume that $F$ is convex, $|\dFdq|\leq B_F$, $\dFdq$ is $\LipTV$-Lipshitz with respect to $q$ in TV distance, $\TV(\qstark,\qk)\leq \epsilon_\TV$, and $\beta' \geq \beta$. Then,
    \begin{align}
    &\frac{2}{K(K+1)}\sum_{k=1}^{K}k \left[ F(\qstark) + \beta \KL(\qstark\|\pref) - F(\qstar) - \beta \KL(\qstar\|\pref)\right]\\
    \leq&\frac{2}{K(K+1)}\sum_{k=1}^{K}k \left[ 2 \LipTV\epsilon_\TV +\E_{\qstark}[g^{(k)}] - \E_{\qstar}[g^{(k)}] + \beta (\KL(\qstark\|\pref) - \KL(\qstar\|\pref))\right]\\
    \leq&2\LipTV\epsilon_{\mathrm{FD}}+\frac{2}{K(K+1)}\left[B_F + \beta(K+3)\KL(\qstar\|\pref) + \E_{\qstar}[g^{(1)}] + \frac{B_F^2 K}{\beta}\right].
  \end{align}
%    where
%  \begin{equation}
%    \epsilon_{\mathrm{FD}} = \frac{2}{K(K+1)}\sum_{k=1}^{K} k \epsilon_\TV.
%  \end{equation}
\end{thm*}

First, we want to show that the $\KL$ regularization term plays a role of the ``quadratic" regularization, which makes the regularized objective $F(q) + \beta\KL(q\|\pref)$ strongly convex.
We observe that the sum of the linear (convex and concave) objective and the $\KL$ regularization term
\begin{equation}
    \tilde{F}(q) = \mathrm{E}_q[r(x)] + \beta \KL(q\|p)
\end{equation}
is strongly convex in the distributional sense. In particular, if $q$ is the minimizer of $\tilde{F}$, then
\begin{equation}
    \tilde{F}(q) + \frac{\beta}{2}\mathrm{TV}(q,q')^2 \leq \tilde{F}(q') \text{  for all  } q' \in \gP.
\end{equation}
This result is useful to prove Lemma~\ref{lem:da-convex-vkprev-vk}, which controls the regularized and linearized objective $\E_q[\dFdq] + \beta \KL(q\|\pref)$:

\begin{lem}[\cite{NEURIPS2021_a34e1ddb}]
    \label{lem:da-convex}
  $\tilde{F}:\mathcal{P}\to \mathbb{R}$, $\tilde{F}(q) = \mathrm{E}_q[r(x)] + \beta \KL(q\|p)$.
  Assume that 
  $r$ is bounded. We put $r_1:=\frac{dq}{dp}, \; r_2' := \frac{dq'}{dp}$,
  \begin{equation}
    \tilde{F}(q) \leq \tilde{F}(q') + \mathrm{E}_q[(r_1-r_2)(r(x) + \beta \log r_1)] - \frac{\beta}{2}\|r_1-r_2\|^2_{L^1(p)}.
  \end{equation}
  Especially, if $q \propto \exp (-\frac{r}{\beta})p$, 
  \begin{equation}
    \tilde{F}(q) \leq \tilde{F}(q') - \frac{\beta}{2}\|r_1-r_2\|^2_{L^1(p)}.
  \end{equation}
\end{lem}
\begin{proof}
Omitted. See \cite{NEURIPS2021_a34e1ddb}. We use Pinsker's inequality
$\mathrm{TV}(q,q')^2 \lesssim \KL(q\|q')$.
\end{proof}
Given that $F(q) + \beta\KL(q\|\pref)$ is strongly convex, let us prove Theorem~\ref{thm-da-nishikawa}:
\begin{proof}
  The main parts of the proof follow existing papers~\citep{NEURIPS2021_a34e1ddb, nishikawa2022twolayer}. The key difference lies in that we are not using Langevin sampler in the inner loop, which slightly changes how the error of $\qk$ is handled. 

  We begin by linearizing the weighted sum of the losses.
  $\gk$ denotes $\dFdq(\qk)$, and $\qstar$ denotes the minimizer of $F$.
  From the convexity of $F$,
  \begin{align}
    &\frac{2}{K(K+1)}\sum_{k=1}^{K}k \left[ F(\qstark) + \beta \KL(\qstark\|\pref) - F(\qstar) - \beta \KL(\qstar\|\pref)\right]\\
    \leq& \frac{2}{K(K+1)}\sum_{k=1}^{K}k \left[ \int \frac{\delta F}{\delta q}(\qstark) d(\qstark - \qstar)  + \beta (\KL(\qstark\|\pref) - \KL(\qstar\|\pref))\right]\\
    \leq& \frac{2}{K(K+1)}\sum_{k=1}^{K}k \left[ \left| \int \frac{\delta F}{\delta q}(\qstark) - \frac{\delta F}{\delta q}(\qk) d(\qstark - \qstar)\right|+\E_{\qstark}[g^{(k)}] - \E_{\qstar}[g^{(k)}]\right. \\
    &\left.+ \beta (\KL(\qstark\|\pref) - \KL (\qstar\|\pref))\right].
  \end{align}
  Here, we bound the estimation error of $\gk = \dFdq(\qk)$ using the Lipschitz continuity of $\dFdq$ with respect to $q$:
  \begin{align}
    &\left| \int \frac{\delta F}{\delta q}(\qstark) - \gk d(\qstark - \qstar)\right| \\
    \leq& \TV(\qstark, \qstar) \sup_x \left|\frac{\delta F}{\delta q}(\qstark,x) - \gk(x) \right|\\
    \leq& 2L_\mathrm{q}\epsilon_\TV.
  \end{align}
  Then we obtain
  \begin{align}
      &\frac{2}{K(K+1)}\sum_{k=1}^{K}k \left[ F(\qstark) + \beta \KL(\qstark\|\pref) - F(\qstar) - \beta \KL(\qstar\|\pref)\right]\\
      \leq& \frac{2}{K(K+1)}\sum_{k=1}^{K}k \left[2L_\mathrm{q}\epsilon_\TV +\E_{\qstark}[g^{(k)}] - \E_{\qstar}[g^{(k)}]\right. \\
      &\left.+ \beta (\KL(\qstark \|\pref) - \KL (\qstar\|\pref))\right].\label{eq:appendix-da-ineq-weighted-sum}
  \end{align}
  This implies that it is sufficient to bound the weighted sum of the (regularized) linearized objectives $\E_{\qstark}[g^{(k)}] + \beta (\KL(\qstark \|\pref), \; k=1,...,K$.
  
  In each update of Option 1, $\qstarnext$ is obtained by maximizing 
  \begin{equation}
    V_{k+1}(q) = - \mathrm{E}_{q}\left[\sum_{j=1}^{k}j g^{(j)}\right] - \frac{\beta k (k+1)}{2}\KL(q\|\pref)  
    - \beta' (k+1)\KL(q\|\pref). 
  \end{equation}
  %by $\qstarnext$ in each update. 
  We also define
  \begin{equation}
    r_{*}^{(k+1)} = \frac{\mathrm{d} \qstarnext}{\mathrm{d}\pref}, \quad V_{k+1}^*= V_{k+1}(\qstarnext).
   \end{equation}

  Then, we can show that $V^*_k$ has the following recursive relation.
  Lemma~\ref{lem:da-convex-vkprev-vk} approximately implies
  \begin{equation}
      k \E_{\qstark}[g^{(k)}] + k \beta \KL(\qstarnext\|\pref)) \lesssim V_{k}^* - V_{k+1}^* + \mathcal{O}(1),
  \end{equation}
  then, summing from $k=1$ to $K$, we roughly obtain,
  \begin{equation}
      \frac{2}{K(K+1)}\left[ \sum_{k=1}^K k(\E_{\qstark}[g^{(k)}] + \beta \KL(\qstark\|\pref)) + V_{K+1}^* \right]\lesssim \mathcal{O}\left(\frac{1}{K}\right).\label{eq:appendix-da-lem2-implication}
  \end{equation}
  So, as long as we prove Lemma~\ref{lem:da-convex-vkprev-vk}, we can connect this inequality (\ref{eq:appendix-da-lem2-implication}) to the inequality (\ref{eq:appendix-da-ineq-weighted-sum}) because $V_{K+1}^* \gtrsim -\sum_{k=1}^K k(\E_{\qstar}[g^{(k)}] + \beta \KL(\qstar\|\pref))$:
   
  \begin{lem}
    \label{lem:da-convex-vkprev-vk}
      For $k\geq 1$,
      \begin{equation}
        V_{k+1}^* \leq V_{k}^* - k\E_{\qstark}[g^{(k)}] - (\beta k+\beta')\KL(\qstarnext\|\pref) + \frac{B_F^2k}{\beta (k-1) + 2 \beta'}.\label{eq:da-convex-lem-2}
      \end{equation}
  \end{lem}
  For the proof of Lemma~\ref{lem:da-convex-vkprev-vk}, please refer to Section~\ref{sec:da-convex-lem}, in which we use Lemma~\ref{lem:da-convex}.
  % \begin{rem*}[Interpretation of Lemma~\ref{lem:da-convex-vkprev-vk}]

%\end{rem*}

  From here, we will rigorously explain the result derived from Lemma~\ref{lem:da-convex-vkprev-vk}.
  When $k=1$, because $\|\gk\|_\infty\leq B_F$,
  \begin{equation}
    V_1^* \leq B_F - \beta \KL(\hat{q}^{(1)}\|\pref).
  \end{equation}
   By taking a telescoping sum of (\ref{eq:da-convex-lem-2}),
  \begin{align}
    V_{K+1}^* \leq&B_F - \sum_{k=1}^{K} k \left(\E_{\qstark}[g^{(k)}] + \beta \KL(\qstark\|\pref)\right) - [\beta (K+1)+ \beta']\KL(\hat{q}^{(k+2)}\|\pref)\\
    &+ \sum_{k=1}^{K}\frac{B_F^2k}{\beta(k-1) + 2\beta'}\\
    \leq&B_F - \sum_{k=1}^{K} k \left(\E_{\qstark}[g^{(k)}] + \beta \KL(\qstark\|\pref)\right) + \frac{B_F^2 K}{\beta},\label{eq:da-convex-vk-upperbound}
  \end{align}
  where we used $\beta' \geq \beta$ in the last inequality. 
  On the other hand, we see that
  \begin{align}
    V_{K+1}^* \geq& - \E_{\qstar}\left[\sum_{k=1}^{K}kg^{(k)}\right] - \frac{\beta K(K+1) + 2\beta' (K+1)}{2}\KL(\qstar\|\pref)\\
    =& - \E_{\qstar}\left[\sum_{k=1}^{K}k (g^{(k)} + \beta \KL(\qstar\|\pref))\right] - \beta'(K+1)\KL(\qstar\|\pref)  \label{eq:da-convex-vk-lowerbound}
  \end{align}
  because $V_{K+2}^*$ is the maximal value. Combining the upper bound (Eq.~\ref{eq:da-convex-vk-upperbound}) and the lower bound (Eq.~\ref{eq:da-convex-vk-lowerbound}),
  \begin{align}
    &\sum_{k=1}^{K}k\left(\E_{\qstark}[g^{(k)}] - \E_{\qstar}[g^{(k)}] + \beta (\KL(\qstark\|\pref)-\KL(\qstar\|\pref))\right)\\
    \leq& B_F + \beta' (K+1)\KL(\qstar\|\pref) + \frac{B_F^2 K}{\beta}.
  \end{align}
  Finally we get the convergence rate:
  \begin{align}
    &\frac{2}{K(K+1)}\sum_{k=1}^{K}k \left[ F(\qstark) + \beta \KL(\qstark\|\pref) - F(\qstar) - \beta \KL(\qstar\|\pref)\right]\\
    \leq&\frac{2}{K(K+1)}\sum_{k=1}^{K}k \left[ 2 \LipTV\epsilon_\TV +\E_{\qstark}[g^{(k)}] - \E_{\qstar}[g^{(k)}] + \beta (\KL(\qstark\|\pref) - \KL(\qstar\|\pref))\right]\\
    \leq&2\LipTV\epsilon_{\mathrm{FD}}+\frac{2}{K(K+1)}\left[B_F + \beta'(K+1)\KL(\qstar\|\pref)  + \frac{ B_F^2 K}{\beta}\right].
  \end{align}
  %   where
  % \begin{equation}
  %   \epsilon_{\mathrm{FD}} = \frac{2}{K(K+1)}\sum_{k=1}^{K} k \epsilon_\TV. 
  % \end{equation}
  This concludes the assertion. 
\end{proof}

\subsubsection{Proof of Lemma~\ref{lem:da-convex-vkprev-vk}\label{sec:da-convex-lem}}\label{sec:AppendixNonconvexConv}
   \begin{proof}
      We can calculate the relation between $V_{k+1}^* = V_{k+1}(\qstarnext)$ and $V_{k}(\qstarnext)$ as
      \begin{align}
        V_{k+1}^* =& - \E_{\qstarnext}\left[\sum_{j=1}^{k}j g^{(j)}\right] - \frac{(k+1)( \beta k + 2 \beta') }{2}\KL(\qstarnext \|\pref)\\
        =& - \E_{\qstarnext}\left[\sum_{j=1}^{k-1}j g^{(j)}\right] - \frac{\beta k (k-1)}{2}\KL(\qstarnext \|\pref) 
        - \beta' k \KL(\qstarnext\|\pref)
        \\
        &- k\E_{\qstarnext}[g^{(k)}]  - (\beta k + \beta') \KL(\qstarnext \|\pref) \\
        = & V_{k}(\qstarnext) - k \E_{\qstark}[g^{(k)}]+ k(\E_{\qstark}[g^{(k)}]-\E_{\qstarnext}[g^{(k)}]) - (\beta k+\beta')\KL(\qstarnext\|\pref),\label{eq:da-convex-vkp1-and-vk-0}
      \end{align}
where we used the definitions of $V_k$ and $V_{k+1}$.
Next, we upper bound the RHS of \Eqref{eq:da-convex-vkp1-and-vk-0} using Lemma~\ref{lem:da-convex} about the convexity of $\KL$:
      \begin{align}
        &(\text{RHS of \Eqref{eq:da-convex-vkp1-and-vk-0}})\\
        \leq& V_{k}^* - \frac{\beta k (k-1) + 2 \beta' k }{4}\|\rstarknext - \rstark\|_{L^1(\pref)}^2
        %\KL(\qstarnext \| \qstark ) 
        -k \E_{\qstark}[g^{(k)}]+ k(\E_{\qstark}[g^{(k)}]-\E_{\qstarnext}[g^{(k)}])\nonumber \\
        &- ( \beta k+\beta')\KL(\qstarnext\pref)\quad (\because \text{the second equation in Lemma~\ref{lem:da-convex} and the optimality of $V_k^*$})\\
        \leq& V_{k}^* -  \frac{\beta k (k-1) + 2 \beta' k }{4}\|\rstarknext - \rstark\|_{L^1(\pref)}^2  -k \E_{\qstark}[g^{(k)}]+ k\left|\E_{\qstark}[g^{(k)}]-\E_{\qstarnext}[g^{(k)}]\right|\nonumber  \\
        &- \beta(k+1)\KL(\qstarnext\|\pref)\\
        \leq& V_{k}^* - \frac{\beta k (k-1) + 2 \beta' k }{4}\|\rstarknext - \rstark\|_{L^1(\pref)}^2 -k \E_{\qstark}[g^{(k)}]+ B_F k\|\rstarknext - \rstark\|_{L^1(\pref)}\nonumber \\
        &- \beta(k+1)\KL(p_*^{(t+1)}\|q)\; \quad (\because \|g^{(k)}\|_\infty \leq B_F) \\
        \leq& V_{k}^* - k\E_{\qstark}[g^{(k)}] - \beta(k+1)\KL(\qstarnext\|\pref) + \frac{B_F^2k}{\beta (k-1) + 2\beta' }, %(\because\text{completing the square})
      \end{align}
      where we used the arithmetic-geometric mean inequality in the last inequality. 
      This concludes the proof. 
 % Then, we conclude the proof of the lemma.
  \end{proof}

\subsection{Convergence proof for non-convex loss $F$\label{section:da-nonconvex-proof}}

%We partially used the proof techniques in ~\cite{LIU2023nonconvexDA}.

Here, we give the proof of Theorem \ref{thm:NonconvexConv} and Corollary \ref{cor:DualConv}. In this section, we always assume Assumption \ref{ass:NonconvexF} holds. We are mainly interested in the property that $F$ is smooth with respect to KL-divergence instead of convexity:
\begin{enumerate}
\item[(ii)'] (A weaker version of (ii)). There exists $S_F \geq 0$ such that 
\begin{equation}
    F(q) \leq F(q') + \int \dFdq(q')\mathrm{d}(q-q') + \frac{S_F}{2}\KL(q\|q') \quad \text{for any}\quad  q,q' \in \gP. \label{eq:appendix-da-nonconvex-weak-smoothness}
\end{equation}
\end{enumerate}
This property can be derived from (ii) in Assumption~\ref{ass:NonconvexF} that is Lipschitz continuity of $\dFdq$:
\begin{lem}
    Assume that $\dFdq$ is Lipshitz continuous with respect to the TV distance: There exists $L_{\mathrm{TV}}  > 0$ such that $\|\dFdq(q) - \dFdq(q')\|_\infty \leq L_\mathrm{TV} \TV(q,q')$ for any $q,q' \in \mathcal{P}$. 
    Then, 
    \begin{equation}
        F(q) \leq F(q') + \int \dFdq(q')\mathrm{d}(q-q') + \frac{\LipTV}{2}\mathrm{TV}(q,q')^2 \quad \text{for any}\quad  q,q' \in \gP.
    \end{equation}
    From Pinsker's inequality,
    \begin{equation}
        F(q) \leq F(q') + \int \dFdq(q')\mathrm{d}(q-q') + \LipTV \KL(q\|q') \quad \text{for any}\quad  q,q' \in \gP.
    \end{equation}
\end{lem}
\begin{proof}
    The proof is almost identical to the proof of $\LipTV$-smoothness commonly discussed in the context of standard optimization.
    For $q,q'\in\gP$, we define $q_t = q' + t (q-q') \in \gP$. Then,
    \begin{align}
        &F(q) - F(q') - \int \dFdq(q')\mathrm{d}(q-q')\\
        = & \int_{t=0}^{t=1}  \int \dFdq(q_t) \mathrm{d}(q-q') \mathrm{d}t
        - \int \dFdq(q')\mathrm{d}(q-q')\\
        = & \int_{t=0}^{t=1}  \int \left[ \dFdq(q_t) - \dFdq(q') \right] \mathrm{d}(q-q') \mathrm{d}t\\
        \leq &  \int_{t=0}^{t=1} t \LipTV \mathrm{TV}(q,q')^2 \mathrm{d}t \quad (\text{$\LipTV$-Lipshitz continuity of $\dFdq$})\\
        = & \frac{\LipTV}{2}\mathrm{TV}(q,q')^2,
    \end{align}
    while we used the fundamental theorem of calculus in the first equation and the assumption in the inequality.
\end{proof}

We emphasize that when the inner-loop error is ignored, it is possible to prove convergence using only the smoothness (\ref{eq:appendix-da-nonconvex-weak-smoothness}) instead of (ii) in Assumption~\ref{ass:NonconvexF}. 
Therefore, we will use the notation $S_F$ for the parts that can be derived using Assumption (ii)' instead of Assumption (ii).

Under the assumptions, our goal is to show that the functional derivative of $L(q)=F(q)+\beta\KL(q\|\pref)$ goes to a constant:
\begin{equation}
    \frac{\delta L}{\delta q} \to 0 \quad (\text{up to constant w.r.t. $x$})
\end{equation}

We prepare the following Lemma about the convexity of $\KL$:
\begin{lem}
    \label{lem-da-nonconvex-nitanda}
    The following equations hold:
    \begin{itemize}
        \item[\textup{(i)}]
        $\KL(q\|p) = \KL(q'\|p) + \int \frac{\delta}{\delta q}\KL(q'\|p) \mathrm{d}(q-q') + \KL(q\|q')$,
        \item[\textup{(ii)}]
        Let $r: \mathbb{R}^d \to \mathbb{R}$ be a smooth function.
        For $\tilde{F}(q)=\E_q[r(x)] + \KL(q\|p), \; q' = \exp(-r)q$, it holds that
        \begin{equation}
            \tilde{F}(q)=\tilde{F}(q') + \KL(q\|q').
        \end{equation}
        $\tilde{F}(q)$ is uniquely minimized at $q=q'$.
    \end{itemize}
\end{lem}
We omit its proof because it is straight forward. 
Lemma~\ref{lem-da-nonconvex-nitanda} implies that $\KL$ also plays a important role as a ``quadratic" penalty term of the proximal operator whose output is 
\begin{equation}
    \underset{q\in \gP}{\mathrm{argmin}} \left\lbrace H(q) + \KL(q\|\pref) \right\rbrace, \text{  for any functional $H$}.
\end{equation}
This property allows for the use of standard nonconvex convergence analysis of Dual Averaging based on the proximal operator~\citep{LIU2023nonconvexDA}.

Roughly speaking, the intermediate goal is to show that $L(q)$ monotonically decreases in each $k$th iteration:
\begin{equation}
    L(\qstarnext) - L(\qstark) \lesssim 0.
\end{equation}
By the weaker smoothness of $F$ (Eq.(\ref{eq:appendix-da-nonconvex-weak-smoothness})), the left hand side is approximately bounded as
\begin{align}
    L(\qstarnext) - L(\qstark) \lesssim& \int \dFdq(\qstark)\mathrm{d}(\qstarnext - \qstark) + \KL(\qstarnext\|\qstark)\\
    \lesssim& \int \dLdq(\qstark)\mathrm{d}(\qstarnext - \qstark)+ \KL(\qstarnext\|\qstark),\label{eq-da-nonconvex-motivation-lem-1}
\end{align}
ignoring various minor terms and constants. To bound the right hand side in (\ref{eq-da-nonconvex-motivation-lem-1}), we show the following inequality regarding the Option 2 using Lemma~\ref{lem-da-nonconvex-nitanda}. 
\begin{lem}
    \label{lem-da-nonconvex-1}
    Assume that $\TV(\qstark, \qk) \leq \epsilon_\TV/2$ for all $k$. Then, the Option 2 achieves 
    \begin{align}
        &\int \dLdq(\qk)\mathrm{d}(\qstarnext - \qstark)\\
        \leq & B_F\epsilon_\TV + \frac{\beta'}{k}\left(\KL(\qstark\|\pref)-\KL(\qstarnext\|\pref)-k\KL(\qstarnext\|\qstark)
        - (k+1)\KL(\qstark \| \qstarnext) \right)
    \end{align}
\end{lem}
\begin{proof}
    In the Option 2, $\qstarnext$ minimizes
    \begin{equation}
        r_{k+1}(q) \coloneq \sum_{j=1}^k \int  j \dLdq(\qj)\mathrm{d}(q-\qstarj)
        + \beta' (k+1) \KL(q\|\pref).
    \end{equation}
    We interpret that Option 2 computes the proximal operator of the weighted sum of $\dLdq$ and this concept is justified by Lemma~\ref{lem-da-nonconvex-nitanda}.
    By the definition of $r_k$ and $r_{k+1}$, it holds that 
    \begin{equation}
        r_{k+1}(q) = r_k(q) + \int k \dLdq(\qk) \mathrm{d}(q-\qstark)
        +\beta' \KL(q\|\pref).\label{eq-da-noncovex-r_knext}
    \end{equation}
    From Lemma~\ref{lem-da-nonconvex-nitanda}, we also have 
    \begin{equation}\label{eq:rkupdateDiff}
        r_k(q) - r_k(\qstark) = \beta' k \KL(q\|\qstark), \text{  for all $q \in \gP$}
    \end{equation}
    because $r_k(q)$ is just the sum of the linear functional of $q$ and the KL-divergence ignoring the constant. 
    Letting $q = \qstarnext$ and we obtain
    \begin{align}\label{eq:rkinductionFirstSide}
        0 \leq& r_k(\qstarnext) - r_k(\qstark) - \beta' k \KL(\qstarnext\|\qstark)\\
        =& r_{k+1}(\qstarnext) -  \int k \dLdq(\qk) \mathrm{d}(\qstarnext-\qstark)
        -\beta' \KL(\qstarnext\|\pref)\\
        &-r_k(\qstark) - \beta' k \KL(\qstarnext\|\qstark).
    \end{align}
    In the last equality, we decomposed the sum using the equation (\ref{eq-da-noncovex-r_knext}).
    Thus,
    \begin{align}
        &\int k \dLdq(\qk) \mathrm{d}(\qstarnext-\qstark)\\
        \leq& r_{k+1}(\qstarnext) -r_k(\qstark) 
        -\beta' \KL(\qstarnext\|\pref)
        - \beta' k \KL(\qstarnext\|\qstark).
    \end{align}
    By using the same argument as \Eqref{eq:rkupdateDiff} for $k \leftarrow k+1$, we have 
    \begin{align}
        r_{k+1}(\qstarnext)
        + \beta' (k+1)\KL(\qstark\|\qstarnext) 
        =& r_{k+1}(\qstark)\\
        =&r_k(\qstark) + \beta' \KL(\qstark\|\pref).
    \end{align}
    Substituting this relation to \Eqref{eq:rkinductionFirstSide} and noticing $|\dFdq|\leq B_F$, we obtain the assertion. 
    % Combining the above inequalities and $|\dFdq|\leq B_F$, we get the desired result.
\end{proof}

We also show the result for Option 1 that is similar to Lemma~\ref{lem-da-nonconvex-1}:

\begin{lem}%[\ref{lem-da-nonconvex-1}']
    \label{lem-da-nonconvex-1-dash}
    Assume that $\TV(\qstark, \qk) \leq \epsilon_\TV/2$ for all $k$. Then, Algorithm 1 achieves 
    \begin{align}
        &\int \dFdq(\qk)\mathrm{d}(\qstarnext - \qstark)\\
        \leq & B_F\epsilon_\TV + \frac{\beta k + \beta'}{k}\left(\KL(\qstark\|\pref)-\KL(\qstarnext\|\pref) \right) \\
        & 
        -  \left( \frac{\beta (k-1) + 2 \beta' }{2}\right) \KL(\qstarnext \| \qstark) 
        -  \left( \frac{\beta (k+1) + 2 \beta'(1+1/k) }{2}\right) \KL(\qstark \| \qstarnext) 
    \end{align}
\end{lem}

\begin{proof}
    Please refer to Appendix~\ref{sec:da-nonconvex-proof-1-dash}. The proof is almost identical to the proof of Lamma~\ref{lem-da-nonconvex-1}.
\end{proof}

We mentioned that our rough intermediate goal was to show that $L(q)$ monotonically decreases in each $k$th iteration:
\begin{equation}
    L(\qstarnext) - L(\qstark) \lesssim 0.
\end{equation}
Rigorously, we will show that
\begin{equation}
    \tilde{L}_k(q) \coloneq L(q) + \frac{\beta'}{k}\KL(q\|\pref)
\end{equation}
decreases. This is an objective function with additional regularization imposed by the hyperparameter $\beta'$. 
From Lemma~\ref{lem-da-nonconvex-1}, we prove that $\tilde{L}_k(\qstark)$ decreases as in the following lemma. 
\begin{lem}\label{lem:LdiffInduction}
    Assume that $\TV(\qstark, \qk) \leq \epsilon_\TV/2$ for all $k$ and $2\beta + S_F \leq 2 \beta'$. Then, Option 2 satisfies
    %$S_F \leq 2\beta$. Then Algorithm 2 satisfies 
    \begin{align}
        \tilde{L}_{k+1}(\qstarnext) - \tilde{L}_k(\qstark)
         \leq 
         & (\LipTV +B_F)\epsilon_\TV    % +\frac{\beta'}{k}\left(\KL(\qstark\|\pref)-\KL(\qstarnext\|\pref) \right)  \\ &
         - \frac{\beta' (k+1)}{k} \KL(\qstark \|\qstarnext),
        \label{eq-da-nonconvex-monotone}
    \end{align}
    and Option 1 satisfies 
    \begin{align}
        \tilde{L}_{k+1}(\qstarnext) - \tilde{L}_k(\qstark)
         \leq 
         & (\LipTV +B_F)\epsilon_\TV    % +\frac{\beta'}{k}\left(\KL(\qstark\|\pref)-\KL(\qstarnext\|\pref) \right)  \\ &
         - \frac{\beta k(k+1)  +2 \beta' (k+1)}{2k} \KL(\qstark \|\qstarnext),
        \label{eq-da-nonconvex-monotone}
    \end{align}
%    \begin{equation}
%        \tilde{L}_{k}(q_*^{(k+1)}) -  \tilde{L}_{k}(q_*^{(k)})
%        \leq (\LipTV +B_F)\epsilon_\TV -\beta \KL(q_*^{(k+1)}\|q_*^{(k)}).\label{eq-da-nonconvex-monotone}
%    \end{equation}
\end{lem}
By this lemma, it can be shown that the sum of the KL divergences $\KL(\qstark \|\qstarnext)$ converges to 0 at a rate of $\mathcal{O}(1/K)$ through a telescoping sum:
we obtain that
\begin{equation}
    \frac{1}{K}\sum_{k=1}^K (\text{weight})_k \KL(\qstark\|\qstarnext) = \mathcal{O}\left(\frac{1}{K}\right).
\end{equation}
From the above equation, it is immediately shown that
\begin{equation}
    \min_{k=1,...,K} (\text{weight})_k \KL(\qstark\|\qstarnext) = \mathcal{O}\left(\frac{1}{K}\right),
\end{equation}
where $(\text{weight})_k \simeq k$ in Option 1, while $(\text{weight})_k \simeq 1$ in Option 2. In Option 2, we will be able to show that $\min_k \KL(\qstark\|\qstarnext) \to 0$, which implies that $\qstark$ converges to some point (in fact, this is the stationary point).

Now let us prove Lemma~\ref{lem:LdiffInduction}:
\begin{proof}
    We only prove the inequality for Option 2.
    By the smoothness of $F$ (the weaker smoothness (ii)' in Eq.(\ref{eq:appendix-da-nonconvex-weak-smoothness})), the Lipschitz continuity of $\dFdq$ and the property of the KL-divergence (Lemma \ref{lem-da-nonconvex-nitanda}), we have 
    \begin{align}
        L(\qstarnext) - L(\qstark)
        \leq& \int \dFdq(\qstark)\mathrm{d}(\qstarnext - \qstark) + \frac{S_F}{2}\KL(\qstarnext\|\qstark)\\
        & + \beta \KL(\qstarnext \| \qstark) + \int (- \beta  \gbarkprev) \mathrm{d} (\qstarnext - \qstark) \\
        \leq & \int \left( \dFdq(\qk) - \beta \gbarkprev \right)\mathrm{d}(\qstarnext - \qstark) \\
        & + \frac{S_F}{2}\KL(\qstarnext\|\qstark) + \beta \KL(\qstarnext \| \qstark) + 
        \LipTV \epsilon_\TV. \label{eq-da-nonconvex-monotone0}
    \end{align}
    We have used the induced smoothness of $F$ (Eq.(\ref{eq:appendix-da-nonconvex-weak-smoothness})) in the first inequality, and used the Lipschitz continuity in the second inequality.
    By Lemma~\ref{lem-da-nonconvex-1}, we can bound $\int \left( \dFdq(\qk) - \beta \gbarkprev \right)\mathrm{d}(\qstarnext - \qstark) = \int \dLdq(\qstark) \mathrm{d}(\qstarnext - \qstark)$.
    Thus the RHS of \Eqref{eq-da-nonconvex-monotone0} can be bounded as 
    \begin{align}
        (\text{RHS})
        \leq& \frac{\beta'}{k}\left(\KL(\qstark\|\pref)-\KL(\qstarnext\|\pref)
        -k\KL(\qstarnext\|\qstark)
        - (k+1)\KL(\qstark \| \qstarnext)
        \right)\\
        &+ \frac{2\beta + S_F}{2}\KL(\qstarnext\|\qstark)+ (\LipTV +B_F) \epsilon_\TV \\
        \leq& \frac{\beta'}{k}\left(\KL(\qstark\|\pref)-\KL(\qstarnext\|\pref) - (k+1)\KL(\qstark \| \qstarnext) \right) \\
        &+ \frac{(2\beta + S_F - 2 \beta')}{2}\KL(\qstarnext\|\qstark)+ (\LipTV +B_F) \epsilon_\TV,
    \end{align}
    which gives the assertion in Option 2 by noting  $\frac{\beta'}{k}\KL(\qstarnext\|\pref) \geq \frac{\beta'}{k+1}\KL(\qstarnext\|\pref)$. As for Option 1, we repeat the same argument to show the desired result.
%    From the definition of $\tilde{L}_k$,
 %   \begin{equation}
  %      \tilde{L}_{k}(\qstarnext) -  \tilde{L}_{k}(\qstark)
  %      \leq - \frac{2\beta-S_F}{2}\KL(\qstarnext\|\qstark)
  %      + (\LipTV +B_F) \epsilon_\TV.
  %  \end{equation}
\end{proof}

 
Combining Lemma \ref{lem-da-nonconvex-1} and Lemma \ref{lem:LdiffInduction}, 
we will prove that
\begin{equation}
    \frac{1}{K}\sum_{k=1}^K (\text{weight})_k \KL(\qstark\|\qstarnext) = \mathcal{O}\left(\frac{1}{K}\right)
\end{equation}
where $(\text{weight})_k \simeq k$ in Option 1 and $(\text{weight})_k \simeq 1$ in Option 2. 
When we take Option 2, it holds that $\min_k \KL(\qstark\|\qstarnext) = \mathcal{O}(1/K)$.
% Finally, we get the conclusion $\frac{\delta L(\qstark)}{\delta q} \to \mathrm{(constant.)}$ because 
It is important that $\KL(\qstark\|\qstarnext)$ is also the approximate ``moment generation function" $\psi_q(g) = \log (\E_q[\exp(-g+\E_q[g]])$ of $\frac{\delta \tilde{L}_k(\qstark)}{\delta q}$; when $k \to \infty$,
\begin{equation}
    \min_{k=1,..,K}\left(\text{``Variance" of }\frac{\delta \tilde{L}_k}{\delta q} \right)(\qstark,x) \simeq \min_{k=1,..,K}\KL(\qstark\|\qstarnext) = \mathcal{O}\left(\frac{1}{K}\right).
\end{equation}
This can be interpreted as $\frac{\delta \tilde{L}_k}{\delta q}(\qstark,x) \to 0\; \text{(up to constant w.r.t. $x$)}$.

This result is consistent with the findings for standard, non-distributional Dual Averaging~\citep{LIU2023nonconvexDA}, where $\min_{k=1,...,K}\|\text{(gradient of the objective)}_k\|^2 = \mathcal{O}(1/K)$ as the variance corresponds to the second moment..


We rigorously formulate the above discussion as the following proposition. 
\begin{prop}
Let $$\Psi_K := \frac{1}{K\beta'} (\tilde{L}_1(\hat{q}^{(1)})- L^*) %+ \beta'\KL(\qstar^{(1)}\| \pref))  
+ \frac{(\LipTV +B_F)}{K\beta'} \sum_{k=1}^K \epsilon_\TV. $$
Then, Algorithm 1 satisfies  
    \begin{align}
        & \frac{1}{K} \sum_{k=1}^K \frac{\beta k + 2 \beta' }{2} \KL(\qstark \| \qstarnext)  \leq \Psi_K 
    \end{align}
and Algorithm 2 satisfies  
    \begin{align}
        & \frac{1}{K} \sum_{k=1}^K \KL(\qstark \| \qstarnext)  \leq     
        \Psi_K.
    \end{align}
This also yields that the following bound holds for Algorithm 2:  
    \begin{align}
        & \min_{1 \leq k \leq K} \psi_{\qstark}\left(\frac{k}{\beta' (k+1)}\frac{\delta \tilde{L}_k}{\delta q}(\qstark)  \right)  \leq  
        \E_{k \sim \mathrm{Univ}([K])} \left[ \psi_{\qstark}\left(\frac{k}{\beta' (k+1)}\frac{\delta \tilde{L}_k}{\delta q}(\qstark)  \right)   \right]  \\
        & \leq  \Psi_K,
    \end{align}
    where the expectation in the middle term is taken over a random index $k$ uniformly chosen from $[K] =\{1,\dots,K\}$. 
\end{prop}
\begin{proof}
Summing up \eqref{eq-da-nonconvex-monotone} for $k=1,\dots,K$, we have that 
\begin{align}
\tilde{L}_{K+1}(\hat{q}^{(K+1)}) - \tilde{L}_1(\hat{q}^{(1)}) \leq &   (\LipTV +B_F) \sum_{k=1}^K \epsilon_\TV 
% + \beta' \KL(\qstar^{(1)}\| \pref)   
% - \sum_{k=2}^{K} \frac{\beta'}{k(k-1)} \KL(\qstark \| \pref)  \\&
% -   \frac{\beta'}{K}\KL(\qstar^{(K+1)}\| \pref)  
- \beta'  \sum_{k=1}^{K}\frac{k+1}{k}\KL(\qstark \| \qstarnext). 
\end{align}
This yields that 
\begin{align}
&  
\frac{\beta'}{K} \sum_{k=1}^{K} % \left[ 
\frac{k+1}{k}\KL(\qstark \| \qstarnext)   
%+ \frac{1}{K} \sum_{k=1}^{K} 
%\frac{1}{k}( \KL(\qstarnext \| \pref) - \KL(\qstark \| \pref)) 
%\frac{1}{k(k-1)} \KL(\qstark \| \pref)
%\right] 
% \\ & 
%+ \frac{\beta'}{K} \sum_{k=2}^{K}\frac{1}{k(k-1)} \KL(\qstark \| \pref)  \\ &
+ 
\frac{1}{K}( \tilde{L}_{K+1}(\hat{q}^{(K+1)}) - L^*)\\
 \leq  &
\frac{1}{K} (\tilde{L}_1(\hat{q}^{(1)})- L^*) %+ \beta'\KL(\qstar^{(1)}\| \pref))  
+ \frac{(\LipTV +B_F)}{K} \sum_{k=1}^K \epsilon_\TV. \label{eq-da-nonconvex-bound-kl-sum}
\end{align}
Please note that 
\begin{equation}
\frac{\beta'}{K} \sum_{k=1}^{K} % \left[ 
\KL(\qstark \| \qstarnext)   
\leq
\frac{\beta'}{K} \sum_{k=1}^{K} % \left[ 
\frac{k+1}{k}\KL(\qstark \| \qstarnext)  
+
\frac{1}{K}( \tilde{L}_{K+1}(\hat{q}^{(K+1)}) - L^*)
\end{equation}
by the optimality of $L^*$ and that the RHS in (\ref{eq-da-nonconvex-bound-kl-sum}) is $\mathcal{O}(1/K)$.
Here, we see that 
\begin{align}
&  \KL(\qstark \| \qstarnext)\\
= & 
\E_{\qstark}[ - \gbarkprev + \gbark ] -  \log(\E_{\pref}[\exp(-\gbarkprev)]) + \log(\E_{\pref}[\exp(-\gbark)])  \\
= & 
 \E_{\qstark}[ - \gbarkprev + \gbark ]  +  \log(\E_{\qstark}[\exp(-\gbark + \gbarkprev)]) \\  
%=&  \E_{\qstark}[ - \gbarkprev + \gbark ]  + 
=&  \log\left\{\E_{\qstark}\left[\exp\left(-\gbark + \gbarkprev - \E_{\qstark}\left[ - \gbark + \gbarkprev \right]\right)\right]\right\}, 
%\\ =& \psi_{\qstark}(\gbark - \gbarkprev). 
%= & 
% \E_{\qstark}\left[ \frac{k}{\beta' (k+1)} \dLdq(\qstark) - \frac{1}{k+1}  \gbarkprev \right] \\ 
% & -  \tfrac{k}{k+1}\log(\E_{\pref}[\exp(-\gbarkprev)]) + \tfrac{k}{k+1} \log[(\E_{\pref}[\exp(-\gbark)])^{\tfrac{k+1}{k}}] \\  
%\geq & 
% \frac{k}{k+1} \left\{   \E_{\qstark}\left[ \frac{1}{\beta'} \dLdq(\qstark) \right]  - 
% \log(\E_{\pref}[\exp(-\gbarkprev)]) + \log(\E_{\pref}[\exp(-\tfrac{k+1}{k}\gbark)]) \right\}, 
\end{align}
%where we used the convexity of $(\cdot)^{(k+1)/k}$ in the last inequality. 
Now, the term $\gbark - \gbarkprev$ can be evaluated as 
$$\gbark - \gbarkprev = \frac{k}{\beta' (k+1)} \dLdq(\qstark) - \frac{1}{k+1}  \gbarkprev
 =  \frac{k}{\beta' (k+1)} \frac{\delta \tilde{L}_k}{\delta q}(\qstark),
$$
which yields the assertion because 
\begin{align}
    &\KL(\qstark \| \qstarnext)\\
    =& \log\left\{\E_{\qstark}\left[\exp\left( \frac{k}{\beta' (k+1)} \frac{\delta \tilde{L}_k}{\delta q}(\qstark) -\E_{\qstark}\left[\frac{k}{\beta' (k+1)} \frac{\delta \tilde{L}_k}{\delta q}(\qstark)\right]\right)\right]\right\},
\end{align}
which is the ``moment generating function" $\psi_{\qstark}$ of $\frac{k}{\beta' (k+1)} \frac{\delta \tilde{L}_k}{\delta q}(\qstark)$.
%
%On the other hand, the right hand side can be also evaluated as   
%\begin{align}
%& 
%  \log\left\{\E_{\qstark}\left[\exp\left(- \frac{k}{\beta' (k+1)} \dLdq(\qstark) + \frac{1}{k+1}  \gbarkprev - \E_{\qstark}\left[ - \frac{k}{\beta' (k+1)} \dLdq(\qstark) + \frac{1}{k+1}  \gbarkprev  \right] \right)\right]\right\}
% \\ 
%\geq & 
%  \log\left\{\E_{\qstark}\left[\exp\left(- \frac{k}{\beta' (k+1)} \dLdq(\qstark) + \frac{1}{k+1}  \gbarkprev - \E_{\qstark}\left[ - \frac{k}\beta' (k+1)} \dLdq(\qstark) + \frac{1}{k+1}  \gbarkprev  \right] \right)\right]\right\} \\  
%\geq & 
% \frac{k}{k+1} \left\{   \E_{\qstark}\left[ \frac{1}{\beta'} \dLdq(\qstark) \right]  - 
% \log(\E_{\pref}[\exp(-\gbarkprev)]) + \log(\E_{\pref}[\exp(-\tfrac{k+1}{k}\gbark)]) \right\}, 
%\end{align}
%which yields the first assertion. 
%which yields the assertion. 
%Here, we see that 
%\begin{align}
%&  \KL(\qstark \| \qstarnext)-  
%\frac{1}{k+1}\KL(\qstark \| \pref) \\
%= & 
%\E_{\qstark}[ - \gbarkprev + \gbark ] -  \log(\E_{\pref}[\exp(-\gbarkprev)]) + \log(\E_{\pref}[\exp(-\gbark)]) \\
%& - \tfrac{1}{k+1} \left( \E_{\qstark}[ - \gbarkprev]   -  \log(\E_{\pref}[\exp(-\gbarkprev)]) \right) \\
%= & 
% \E_{\qstark}[ - \tfrac{k}{k+1}\gbarkprev + \gbark ] \\ 
% & -  \tfrac{k}{k+1}\log(\E_{\pref}[\exp(-\gbarkprev)]) + \log(\E_{\pref}[\exp(-\gbark)]) \\ 
%= & 
% \E_{\qstark}[ - \tfrac{k}{k+1}\gbarkprev + \gbark ] \\ 
% & - \tfrac{k}{k+1}\log(\E_{\pref}[\exp(-\gbarkprev)]) + \log(\E_{\pref}[\exp(-\gbark)]) \\  
%= & 
% \E_{\qstark}[ - \tfrac{k}{k+1}\gbarkprev + \gbark ] \\ 
% & -  \tfrac{k}{k+1}\log(\E_{\pref}[\exp(-\gbarkprev)]) + \log(\E_{\pref}[\exp(-\gbark)]) \\  
%= & 
% \E_{\qstark}\left[ \frac{k}{\beta' (k+1)} \dLdq(\qstark) \right] \\ 
% & -  \tfrac{k}{k+1}\log(\E_{\pref}[\exp(-\gbarkprev)]) + \tfrac{k}{k+1} \log[(\E_{\pref}[\exp(-\gbark)])^{\tfrac{k+1}{k}}] \\  
%\geq & 
% \frac{k}{k+1} \left\{   \E_{\qstark}\left[ \frac{1}{\beta'} \dLdq(\qstark) \right]  - 
% \log(\E_{\pref}[\exp(-\gbarkprev)]) + \log(\E_{\pref}[\exp(-\tfrac{k+1}{k}\gbark)]) \right\}, 
%\end{align}
%where we used the convexity of $(\cdot)^{(k+1)/k}$ in the last inequality. 
\end{proof}

\begin{comment}
We characterize $\KL(\qstarnext\|\qstark)$ by this proposition:
\begin{prop}
    For all $k\geq 1$, in the limit $\gbark \rightarrow \gbarkprev$ in KL divergence,
    \begin{align}
        &\left|\KL(\qstarnext\|\qstark) - \frac{1}{2}\mathrm{Var}_{\qstarnext}\left(\frac{k}{\beta(k+1)}\frac{\delta L_k}{\delta q}(q_*^{(k)})\right)\right|\\
        =& \mathcal{O}\left(\| \gbark - \gbarkprev\|_{L^3(\qstarnext)}^3 + \textcolor{red}{\text{(Bound of $\frac{\delta L_k}{\delta q}$$\simeq B_F?$)}}\frac{B_F \epsilon}{\beta}\right),
    \end{align}
    where $\Var_q$ is the variance with respect to $q$.
\end{prop}
\begin{proof}
By an asymptotic expansion of $\KL$,
\begin{align}
    &\KL(\qstarnext\|\qstark) \\
    =& \int -\gbark +\gbarkprev + \log \frac{\int e^{-\gbarkprev}d\pref}{\int e^{-\gbark}d\pref} \mathrm{d}\qstarnext\\
    =& \int \left[ \frac{1}{2}\|-\gbarkprev + \gbark\|^2(x)
    -\frac{1}{2}\left(\int e^{-\gbarkprev+\gbark}-1 \mathrm{d}\qstarnext \right)^2
    \right] \mathrm{d}\qstarnext\\
    &+ \mathcal{O}\left(\| \gbark - \gbarkprev\|_{L^3(\qstarnext)}^3 \right)\\
    =&\frac{1}{2}\mathrm{Var}_{\qstarnext}(-\gbarkprev + \gbark)
    + \mathcal{O}\left(\| \gbark - \gbarkprev\|_{L^3(\qstarnext)}^3 \right).
\end{align}
From the definition of $\gbark$,
\begin{equation}
    \gbark = \frac{k}{k+1}\gbarkprev + \frac{k}{\beta(k+1)}\dFdq(\qk).
\end{equation}
Because of the optimality of $\qstark \propto \exp(-\gbarkprev)\pref$,
\begin{equation}
    \gbarkprev = - \frac{k}{\beta} \left(\frac{\delta \tilde{L}_k}{\delta q}(\qstark) - \dFdq(\qstark) \right).
\end{equation}
Then we see that
\begin{equation}
    \gbark - \gbarkprev = \frac{k}{\beta(k+1)}\frac{\delta L_k}{\delta q}(q_*^{(k)}) + \Theta \left( \frac{B_F}{\beta} \epsilon \right).
\end{equation}
\end{proof}
\end{comment}

\subsubsection{Proof of Lemma~\ref{lem-da-nonconvex-1-dash}}\label{sec:da-nonconvex-proof-1-dash}

\begin{proof}
    The proof is almost identical to Lemma \ref{lem-da-nonconvex-1}. 
    In Algorithm 1, $\qstarnext$ is defined as the minimizer of the following quantity: 
    \begin{equation}
        r_{k+1}(q) \coloneq \sum_{j=1}^k j \left( \int   \dFdq(\qj)\mathrm{d}(q-\qstarj)  + \beta \KL(q \| \pref) \right)
        + \beta' (k+1) \KL(q\|\pref).
    \end{equation}
    Hence, we have 
    \begin{equation}
        r_{k+1}(q) = r_k(q) + \int k \dFdq(\qk) \mathrm{d}(q-\qstark)
        + \left(\beta k  + \beta' \right) \KL(q\|\pref) \label{eq:lem-da-nonconvex-proof_opt1}
    \end{equation}
    by the definition of $r_k$. Moreover, Lemma~\ref{lem-da-nonconvex-nitanda} and the optimality of $\qstark$ gives that  
    \begin{equation}\label{eq:rkupdateDiff2}
        r_k(q) - r_k(\qstark) =  \left( \frac{\beta k(k-1)}{2} + \beta' k \right) \KL(q\|\qstark), \quad \text{for all $q\in\gP$.}
    \end{equation}
    Substituting $q \leftarrow \qstarnext$, we obtain
    \begin{align}
        0 \leq& r_k(\qstarnext) - r_k(\qstark) -  \left( \frac{\beta k(k-1)}{2} + \beta' k \right) \KL(\qstarnext\|\qstark) \\
        =& r_{k+1}(\qstarnext) -  \int k \dFdq(\qk) \mathrm{d}(\qstarnext-\qstark)
        -( \beta k + \beta') \KL(\qstarnext\|\pref) \quad (\text{used (\ref{eq:lem-da-nonconvex-proof_opt1})})\\
        &-r_k(\qstark) -  \left( \frac{\beta k(k-1)}{2} + \beta' k \right) \KL(\qstarnext\|\qstark).\label{eq:rkinductionFirstSide2}
    \end{align}
    This is equivalent to 
    \begin{align}
        &\int k \dFdq(\qk) \mathrm{d}(\qstarnext-\qstark)\\
        \leq& r_{k+1}(\qstarnext) -r_k(\qstark) 
        -(\beta k + \beta') \KL(\qstarnext\|\pref)
        - \left( \frac{\beta k(k-1)}{2} + \beta' k \right) \KL(\qstarnext\|\qstark).
    \end{align}
    By using the same argument as \Eqref{eq:rkupdateDiff2} for $k \leftarrow k+1$, we have 
    \begin{align}
        & r_{k+1}(\qstarnext)
        + \left( \frac{\beta k(k+1)}{2} + \beta' (k+1) \right) \KL(\qstark\|\qstarnext)  \\
        =& r_{k+1}(\qstark)\\
        =&r_k(\qstark) + (\beta k + \beta') \KL(\qstark\|\pref).
    \end{align}
    Substituting this relation to \Eqref{eq:rkinductionFirstSide2} and noticing $|\dFdq|\leq B_F$, we obtain the assertion. 
    % Combining the above inequalities and $|\dFdq|\leq B_F$, we get the desired result.
\end{proof}

\section{Derivations of functional derivatives\label{section:appendix-functional-derivative}}

\subsection{Direct Preference Optimization}

We can apply the proposed algorithm to the (populational) objective of Direct Preference Optimization (DPO)~\citep{rafailov2023DPO}. 
DPO is an effective approach for learning from human preference for not only language models but also diffusion models.

\paragraph{Original DPO objective}
Let $x_w, x_l$ be ``winning" and ``losing" outputs independently sampled from the reference model $\pref$. The event $\lbrace x_w \succ x_l \rbrace$ is determined by the human preference. The original DPO objective is formulated as
\begin{equation}
    L_{\mathrm{DPO,original}}(q) = -\E_{x_w,x_l}
    \left[
        \log \sigma \left(\gamma \log \frac{q(x_w)}{\pref(x_w)}- \gamma \log \frac{q(x_l)}{\pref(x_l)}\right)
    \right],
\end{equation}
where $\gamma > 0$ is a hyperparameter.  The expectation is taken by $x_w, x_l$, that are ``winning" and ``losing" samples from $\pref$.
~\cite{Wallace2024DiffusionDPO} derived a new objective that is the upper bound of $ L_{\mathrm{DPO,original}}(q)$, but it is a specialized derivation of the optimization method for DPO.

\paragraph{Reformulating of the DPO Objective}
The goal is to directly minimize $L_{\mathrm{DPO,original}}(q)$, however, in  the above expression, we cannot apply DPO to diffusion models directly because the expectaion with tuple $(x_w,x_l)$ is not formulated well. We start with another expression of the objective of DPO:
\begin{equation}
    L_{\mathrm{DPO}}(q) \coloneq - \E_{x_w\sim p_{\mathrm{ref}}}\E_{x_l\sim p_{\mathrm{ref}}}
    \left[ 
         \log \sigma \left(\gamma \log \frac{q(x_w)}{\pref(x_w)}- \gamma \log \frac{q(x_l)}{\pref(x_l)}\right)\mathbbm{1}_{x_w \succ x_l}(x_w,x_l)
    \right],
\end{equation}
$\mathbbm{1}_{x \succ y}(x,y)$ is 1 if $x \succ y$, 0 otherwise.
This $L_\mathrm{DPO}$ is in the regime of our algorithm and the functional derivative can be derived:
\begin{prop}
    The functional derivative of $L_\mathrm{DPO}(q)$ is calculated as
    \begin{align}
    &\frac{\delta L_\mathrm{DPO}}{\delta q}(q,x) \\
    =&
    -\gamma\E_{x_l\sim p_{\mathrm{ref}}}
    \left[
        \left(
            1- \sigma
            \left(
               - \gamma f(x) + \gamma f(x_l)
            \right)
        \right)
        \frac{\int e^{-f} dp_\mathrm{ref}}{e^{-f(x)}}
        \mathbbm{1}_{x \succ x_l}(x,x_l)
    \right]\\
    &+     \gamma\E_{x_w\sim p_{\mathrm{ref}}}
    \left[
        \left(
            1- \sigma
            \left(
                - \gamma f(x_w)
                + \gamma f(x)
            \right)
        \right)
        \frac{\int e^{-f} dp_\mathrm{ref}}{e^{-f(x)}}
        \mathbbm{1}_{x_w \succ x}(x_w,x)
    \right],
    \end{align}
    where $q = e^{-f}\pref / \int e^{-f}\mathrm{d}\pref$. This functional derivative is tractable in our settings.
\end{prop}

\begin{proof}
In this proof, we use the following notations:
\begin{itemize}
    \item $p_\mathrm{ref}: $ the output distribution of a pre-trained model,
    \item $q \coloneq e^{-f} p_\mathrm{ref} / \int e^{-f} dp_\mathrm{ref}$: the output distribution of an aligned model,
    \item $\mathrm{LSL}(q_1, q_2) \coloneq \log \sigma (\gamma \log q_1/p_\mathrm{ref} - \gamma \log q_2/p_\mathrm{ref})$,
    \item $\partial_1 \mathrm{LSL}(q_1, q_2) = \gamma (1 - \sigma(\gamma \log q_1/p_\mathrm{ref} - \gamma \log q_2/p_\mathrm{ref})) \frac{1}{q_1} $,
    \item $\partial_2 \mathrm{LSL}(q_1, q_2) = -(1 - \sigma(\gamma \log q_1/p_\mathrm{ref} - \gamma \log q_2/p_\mathrm{ref})) \frac{1}{q_2} $,
    \item $\psi(r_1, r_2) \coloneq \gamma (\log r_1 - \log r_2)$,
    \item $\mathrm{Inv}(f,x) = \frac{\int e^{-f} dp_\mathrm{ref}}{e^{-f(x)}}$.
\end{itemize}
The objective of DPO is written as
\begin{equation}
    L_\mathrm{DPO}(q) \coloneq - \mathrm{E}_{x_w\sim p_{\mathrm{ref}}}\mathrm{E}_{x_l\sim p_{\mathrm{ref}}}
    \left[ 
        \mathrm{LSL}(q(x_w),qp(x_l)) \mathbbm{1}_{x_w \succ x_l}(x_w,x_l)
    \right].
\end{equation}
We obtain the first variation of the objective as follows:
\begin{align}
    &L_\mathrm{DPO}(q + \epsilon(\tilde{q}-q))\\
    =&
    L_\mathrm{DPO}(q) - \epsilon
    \mathrm{E}_{x_w\sim p_{\mathrm{ref}}}\mathrm{E}_{x_l\sim p_{\mathrm{ref}}}
    \left[ 
        \left(
            \partial_1 \mathrm{LSL}(q(x_w), q(x_l)) (\tilde{q}-q)(x_w)
            \right.\right.\\
            &\left.\left.\quad\quad\quad+\partial_2 \mathrm{LSL}(q(x_w), q(x_l)) (\tilde{q}-q)(x_l) 
        \right)
        \mathbbm{1}_{x_w \succ x_l}(x_w,x_l)
    \right]+\mathcal{O}(\epsilon^2)\\
    =&
    L_\mathrm{DPO}(p)\\
    &-
    \epsilon
    \left[
        \mathrm{E}_{x_l\sim p_{\mathrm{ref}}}
        \left[ 
            \int \gamma (1 - \sigma(\psi(\frac{q(x_w)}{p_\mathrm{ref}(x_w)}, \frac{q(x_l)}{p_\mathrm{ref}(x_l)}) ))(\tilde{q}-q)(x_w) \mathrm{Inv}(f,x_w)dx_w
            \mathbbm{1}_{x_w \succ x_l}
        \right]\right.\\
    &\left.+
    \mathrm{E}_{x_w\sim p_{\mathrm{ref}}}
        \left[ 
            \int \gamma (1 - \sigma(\psi(\frac{q(x_w)}{p_\mathrm{ref}(x_w)}, \frac{q(x_l)}{p_\mathrm{ref}(x_l)}) ))(\tilde{q}-q)(x_l) \mathrm{Inv}(f,x_l)dx_l
            \mathbbm{1}_{x_w \succ x_l}
        \right]
    \right]\\
    &+\mathcal{O}(\epsilon^2).
\end{align}
Then, the first derivative of $F$ is
\begin{align}
    \frac{\delta L_\mathrm{DPO}}{\delta p}(p,x) =&-
    \E_{x_l\sim p_{\mathrm{ref}}}
    \left[
        \gamma
        \left(
            1- \sigma
            \left(
                \psi(\frac{q(x)}{\pref(x)},\frac{q(x_l)}{\pref(x_l)})
            \right)
        \right)
        \mathbbm{1}_{x \succ x_l}(x,x_l)
    \right]\mathrm{Inv}(f,x)\\
    &+ \E_{x_w\sim p_{\mathrm{ref}}}
    \left[
        \gamma
        \left(
            1- \sigma
            \left(
                \psi(\frac{q(x_w)}{\pref(x_w)},\frac{q(x)}{\pref(x)})
            \right)
        \right)
        \mathbbm{1}_{x_w \succ x}(x_w,x)
    \right]\mathrm{Inv}(f,x).
\end{align}
\end{proof}

\subsection{Kahneman-Tversky optimization}

Assume that the whole data space $\mathbb{R}^d$ is split into a desirable domain $\mathcal{D}_\mathrm{D}$ and an undesirable domain $\mathcal{D}_\mathrm{U}$.
The objective of original KTO~\citep{ethayarajh2024KTO} is formulated as
\begin{align}
    L_\mathrm{KTO}(q)=
    &\E_{x\sim \pref}
    \left[ 
    \gamma_D \left(1 - \sigma \left(\kappa\log \frac{q}{\pref} - \KL(q\|\pref)\right)
    \right)\mathbbm{1}_{\lbrace x \in \mathcal{D}_\mathrm{D}\rbrace}
    \right. \\
    &\quad \left.+
    \gamma_U \left(1 - \sigma \left(\KL(q\|\pref)-\kappa\log \frac{q}{\pref} \right)\right)\mathbbm{1}_{\lbrace x \in \mathcal{D}_\mathrm{U}\rbrace}
    \right],
\end{align}
where $\gamma_D, \; \gamma_U, \; \kappa$ are hyper parameters, and $\sigma$ is a sigmoid function.

\begin{prop}
    The functional derivative of $L_\mathrm{KTO}$ is calculated as
    \begin{align}
        &\frac{\delta L_\mathrm{KTO}}{\delta q}(q,x)\\
        =& 
        - \kappa \gamma_D \sigma_\mathrm{deriv}(\phi(x))\frac{\int e^{-f}\mathrm{d}\pref}{e^{-f(x)}}\mathbbm{1}_{\lbrace x \in \mathcal{D}_\mathrm{D}\rbrace}\\
        &+ (-f(x) - \log \int e^{-f(x)}d\pref)
        %+ \log \frac{q(x)}{\pref(x)}
        \E_{y\sim \pref}
        \left[ 
        \gamma_D \sigma_{\mathrm{deriv}}\left(\phi(y)\right)\mathbbm{1}_{\lbrace y \in \mathcal{D}_\mathrm{D}\rbrace}
        \right]\\
        &+ \kappa \gamma_U \sigma_\mathrm{deriv}(-\phi(x))\frac{\int e^{-f}\mathrm{d}\pref}{e^{-f(x)}}\mathbbm{1}_{\lbrace x \in \mathcal{D}_\mathrm{U}\rbrace}\\
        &- (-f(x) - \log \int e^{-f(x)}d\pref)
        %\log \frac{q(x)}{\pref(x)}
        \E_{y\sim \pref}
        \left[ 
        \gamma_U \sigma_{\mathrm{deriv}}\left(-\phi(y)\right)\mathbbm{1}_{\lbrace y \in \mathcal{D}_\mathrm{U}\rbrace}
        \right]
    \end{align}
    where $\sigma_\mathrm{deriv}(\cdot) \coloneq \sigma(\cdot) (1- \sigma(\cdot))$, 
    $\phi(x) \coloneq \kappa\log \frac{q(x)}{\pref(x)} - \KL(q\|\pref)$, $q = \frac{e^{-f}\pref}{\int e^{-f}d\pref}$. 
\end{prop}
The functional derivative of KTO can be calculated if you have $f(x)$ and the samples from  $\pref$. Note that $\log q(x)/\pref(x) = -f(x) - \log \int e^{-f(x)}d\pref$.

\begin{proof}
    The first variation of $L_\mathrm{KTO}$ is
    \begin{align}
        &L_\mathrm{KTO}(q+\epsilon(\tilde{q}-q))-L_\mathrm{KTO}(q)\\
        \simeq&\epsilon
        \E_{x\sim \pref}
        \left[-
            \kappa \gamma_D \sigma_\mathrm{deriv}(\phi(x))
            \left(\frac{\tilde{q}(x)-q(x)}{q(x)}
        + \int \log \frac{q(y)}{\pref(y)} \mathrm{d}(\tilde{q}-q)(y)\right)
        \mathbbm{1}_{\lbrace x \in \mathcal{D}_\mathrm{D}\rbrace}
        \right]\\
    &+ \epsilon       
    \E_{x\sim \pref}
        \left[
            \kappa \gamma_U \sigma_\mathrm{deriv}(-\phi(x))
            \left(\frac{\tilde{q}(x)-q(x)}{q(x)}
        - \int \log \frac{q(y)}{\pref(y)} \mathrm{d}(\tilde{q}-q)(y)\right)
        \mathbbm{1}_{\lbrace x \in \mathcal{D}_\mathrm{U}\rbrace}
        \right]\\
    =&\epsilon
      \int
        \left\lbrace-
            \kappa \gamma_D \sigma_\mathrm{deriv}(\phi(x))
            \mathrm{Inv}(f,x)\mathbbm{1}_{\lbrace x \in \mathcal{D}_\mathrm{D}\rbrace}\right.\\
        &\left.+ \log \frac{q(x)}{\pref(x)}
        \E\left[
        \kappa \gamma_D \sigma_\mathrm{deriv}(\phi(y))
        \mathbbm{1}_{\lbrace y \in \mathcal{D}_\mathrm{D}\rbrace}
        \right]
        \right\rbrace
        \mathrm{d}(\tilde{q}-q)(y)
        \\
    +&        
    \epsilon
      \int
        \left\lbrace
            \kappa \gamma_U \sigma_\mathrm{deriv}(-\phi(x))
            \mathrm{Inv}(f,x)\mathbbm{1}_{\lbrace x \in \mathcal{D}_\mathrm{D}\rbrace}\right.\\
            &\left.- \log \frac{q(x)}{\pref(x)}
            \E\left[
            \kappa \gamma_D \sigma_\mathrm{deriv}(-\phi(y))
            \mathbbm{1}_{\lbrace y \in \mathcal{D}_\mathrm{U}\rbrace}
            \right]
        \right\rbrace
        \mathrm{d}(\tilde{q}-q)(y),
    \end{align}
    where $\mathrm{Inv}(f,x) = \frac{\int e^{-f} dp_\mathrm{ref}}{e^{-f(x)}}$.
    Now the desired result immediately follows.
\end{proof}


\section{Error Analysis of Diffusion Model}\label{section:appendix-error-diffusion}


\subsection{Overview}
Here, we analyze the sampling error caused by the diffusion model. Let us organize the settings and notations used in this section.

{\bf Target distribution.} The target distribution is $q_*=q_0$, which is decomposed as $q_*(x) = \rho_*(x) p_*(x)$ with $p_*=p_0$ and $\rho_*$. 
Here $p_*$ and $\rho_*$ represent the distribution of the original model and the density ratio obtained by fine-tuning, respectively. 

{\bf Sampling with score-based diffusion model.} In the score-based diffusion model, we start with the forward process, which is written as a stochastic differential equation (SDE).
Choosing the Ornstein–Uhlenbeck (OU) process, $\{\bar{X}_t\}_{t\geq 0}$ follows the following SDE:
\begin{align}
    \bar{X}_0 \sim p_*,\ \mathrm{d}\bar{X}_t = - \bar{X}_t\mathrm{d}t + \sqrt{2}\mathrm{d}B_t,
    \label{eq:Appendix-Diffusion-Preliminary-1}
\end{align}
where $\{B_t\}_{t\geq 0}$ is the standard Brownian motion.
At each time $t$, the law of $X_t$ is written as
\begin{align}
\label{eq:Appendix-Diffusion-Preliminary-3}
    p_t(x) = \int p_*(y)\exp\bigg(-\frac{1}{2\sigma_t^2}\|m_t y - x\|^2\bigg)\mathrm{d}x,
\end{align}
where $m_t = e^{-t}$ and $\sigma_t^2 = 1 - e^{-2t}$. 
In the same way, define $\{\bar{Y}_t\}_{t\geq 0}$ by replacing $p_*$ by $q_*$ in (\ref{eq:Appendix-Diffusion-Preliminary-1}) and let $q_t$ be the law of $\bar{Y}_t$. 

Then, for some $T\geq 0$, we can define the reverse process for $\{\bar{X}_{t}^\leftarrow\}_{0\leq t \leq T}$. (Let $B_t$ below be distinct from the one in (\ref{eq:Appendix-Diffusion-Preliminary-1}).)
\begin{align}
\label{eq:Appendix-Diffusion-Preliminary-2}
    \bar{X}_{0}^\leftarrow\sim p_{T},\ \mathrm{d}\bar{X}_{t}^\leftarrow = \{\bar{X}_{t}^\leftarrow + 2\nabla \log p_{T-t}(\bar{X}_{t}^\leftarrow)\}\mathrm{d}t + \sqrt{2}\mathrm{d}B_t.
\end{align}
Then, the law of $\bar{X}_{t}^\leftarrow$ equals $p_{T-t}$, which is why we call (\ref{eq:Appendix-Diffusion-Preliminary-2}) as the reverse process.
In the same way, we define $\{\bar{Y}_{t}^\leftarrow\}_{0\leq t \leq T}$ as the reverse process of $\{\bar{Y}_{t}\}_{t\geq 0}$. 
% We define $\{\bar{X}^\leftarrow_{t}\}_{0\leq t\leq T}$ by replacing $q$ by $p$ in (\ref{eq:Appendix-Diffusion-Preliminary-2}).

% Describing $\nabla p_{T-t}$ more precisely, 
% Define $p_t$ in the same way as $q_t$. 
% More precisely, 
% $p_{t}$ is obtained by replacing $q_*$ by $p_*$ in (\ref{eq:Appendix-Diffusion-Preliminary-3}), which is log likelihood of the distribution at time $t$ of the forward diffusion process starting from $p_*=p_0$. 

{\bf Doob's h-transform.} 
By applying Doob's h-transform to $\nabla \log q_{T-t}$, it can be decomposed into the original score $\nabla\log p_{T-t}$ and a correction term.
\begin{align}
    \nabla \log q_{T-t}(\bar{Y}_{t}^\leftarrow) = \nabla\log p_{T-t}(\bar{Y}_{t}^\leftarrow) + \nabla_x \log (\mathbb{E}[\rho_*(\bar{X}_0)|\bar{X}_{T-t}=x])|_{x=\bar{Y}_{t}^\leftarrow}.
    \label{eq:Appendix-Diffusion-Preliminary-4}
\end{align}
See Lemma~\ref{lem:H-transform} for derivation.

{\bf Approximation of the score and correction term.} 
Because we do not have access to the exact value of $p_{t}$ and $\rho_*$ and therefore cannot implement (\ref{eq:Appendix-Diffusion-Preliminary-4}) exactly, we consider approximating them by, e.g., neural networks.
We approximate $\nabla \log p_{T-t}(x)$ by $s(x,t)\colon \R^{d+1}\to \R^d$.
Also, we approximate
$u_*(x,t)=\nabla_x \log (\mathbb{E}[\rho_*(\bar{X}_0)|\bar{X}_{T-t}=x])$ by $u(x,t)\colon \R^{d+1}\to \R^d$.

{\bf Discretization.} 
Also, we need to discretize the stochastic differential equation.
We finally obtain the approximation of (\ref{eq:Appendix-Diffusion-Preliminary-2}), denoted by $\{ Y^\leftarrow_t\}_{0\leq t \leq T}$, as
\begin{align}
    Y^\leftarrow_0 \sim \mathcal{N}(0,I_d),\ 
    \mathrm{d} Y^\leftarrow_t = \{ Y^\leftarrow_t + 2s( Y^\leftarrow_{kh},kh)+2u( Y^\leftarrow_{kh},kh)\}\mathrm{d}t + \sqrt{2}\mathrm{d}B_t,\ t\in [kh,(k+1)h].
    \label{eq:Appendix-Diffusion-Preliminary-5}
\end{align}


{\bf Obtaining the correction term (approximately).} 
\label{sec:AppendixCorrectionTermBound}
Given the score network $s(x,t)$ approximating $\nabla \log p_t(x)$ and the function $\bar{h}$ that approximates $h_*$, we can approximate the correction term $u(x,t)$.
Remind that, for fixed $x\in \R^d,\ t=kh, s=k(h+1)\in \R$, the correction term is calculated as
\begin{align}
    \nabla_x \log (\mathbb{E}[\rho_*(\bar{X}_0)|\bar{X}_{T-t}=x]|)
    = 
    \frac{
        \int
        \mathbb{E}[\rho_*(\bar{X}_0)|\bar{X}_{T-s}=x']
        \frac{\partial}{\partial x}\mathbb{P}[\bar{X}_{T-s}=x'|\bar{X}_{T-t}=x]\mathrm{d}x'
    }{
        \mathbb{E}[\rho_*(\bar{X}_0)|\bar{X}_{T-t}=x]
    }.
    \label{eq:Appendix-Diffusion-Preliminary-6}
\end{align}
One way to approximate (\ref{eq:Appendix-Diffusion-Preliminary-6}) starts from approximating $\mathbb{E}[\rho_*(\bar{X}_0)|\bar{X}_{T-t}=x]$. 
If we run the reverse diffusion process
\begin{align}
    \tilde{X}_{t}^\leftarrow=x,\ \mathrm{d}\tilde{X}_{\tau}^\leftarrow = \{\tilde{X}_{\tau}^\leftarrow + 2\nabla \log p_{T-\tau}(\tilde{X}_{\tau}^\leftarrow)\}\mathrm{d}\tau + \sqrt{2}\mathrm{d}B_\tau,
\end{align}
we obtain that the law of $\tilde{X}_{T}^\leftarrow$ is equal to that of $\bar{X}_0|\bar{X}_{T-t}=x$. 
Therefore, by running the approximated reverse process (with a slight abuse of notation)
\begin{align}
    \tilde{X}_{t} = x,\ 
    \mathrm{d}\tilde{X}_{\tau} = \{\tilde{X}_\tau + 2s(\tilde{X}_{lh},lh)\}\mathrm{d}t + \sqrt{2}\mathrm{d}B_\tau,\ \tau\in [lh,(l+1)h],
    \label{eq:Appendix-Diffusion-Preliminary-9}
\end{align}
multiple times, the sample of $\tilde{X}_T$, denoted by $\{\tilde{x}_{T,i}\}_{i=1}^n$, can approximate $\mathbb{E}[\rho_*(\bar{X}_0)|\bar{X}_{T-t}=x]$ as
\begin{align}
    \mathbb{E}[\rho_*(\bar{X}_0)|\bar{X}_{T-t}=x] \approx
    \frac1n\sum_{i=1}^n \rho'(\tilde{x}_{T,i}),
      \label{eq:Appendix-Diffusion-Preliminary-8}
\end{align}
where $\rho'$ is the approximation of $\rho_*$. 

On the other hand, we approximate $\frac{\partial}{\partial x}\mathbb{P}[\bar{X}_{T-s}=x'|\bar{X}_{T-t}=x]$ by approximating $\mathbb{P}[\bar{X}_{T-s}=x'|\bar{X}_{T-t}=x]$ with a Gaussian distribution.
Specifically, because $\bar{X}_{T-s}=x'|\bar{X}_{T-t}=x$ is obtained by the following reverse diffusion process
\begin{align}
    \tilde{X}_{t}^\leftarrow=x,\ \mathrm{d}\tilde{X}_{\tau}^\leftarrow = \{\tilde{X}_{\tau}^\leftarrow + 2\nabla \log p_{T-\tau}(\tilde{X}_{\tau}^\leftarrow)\}\mathrm{d}\tau + \sqrt{2}\mathrm{d}B_\tau,
\end{align}
we approximate $\nabla \log p_{T-\tau}(\tilde{X}_{\tau}^\leftarrow)$ by $s(x_{kh},kh)$ to obtain
\begin{align}
    \dot{X}_{t}^\leftarrow=x_{kh},\ \mathrm{d}\dot{X}_{\tau}^\leftarrow = \{\dot{X}_{\tau}^\leftarrow + 2s(x_{kh},kh)\}\mathrm{d}\tau + \sqrt{2}\mathrm{d}B_\tau.
\end{align}
The distribution of $\tilde{X}_{s}^\leftarrow$ is denoted by
\begin{align}
    \mathcal{N}\big(e^h x_{kh} + 2(e^h-1)s(x_{kh},kh), e^{2h} - 1\big).
     \label{eq:Appendix-Diffusion-Preliminary-7}
\end{align}
Using this, our approximation is
\begin{align}
   & \frac{\partial}{\partial x}\mathbb{P}[\bar{X}_{T-s}=x'|\bar{X}_{T-t}=x]
   \\ & \approx 
    \frac{\partial}{\partial x}\frac{1}{(2\pi (e^{2h} - 1))^\frac{d}{2}}
    \exp\bigg(-\frac{(x'-(e^h x + 2(e^h-1)s(x_{kh},kh)))^2}{2(e^{2h} - 1)}\bigg)
    \\ & = \frac{e^h(x'-(e^h x + 2(e^h-1)s(x_{kh},kh)))}{(e^{2h} - 1)(2\pi (e^{2h} - 1))^\frac{d}{2}}
    \exp\bigg(-\frac{(x'-(e^h x + 2(e^h-1)s(x_{kh},kh)))^2}{2(e^{2h} - 1)}\bigg).
\end{align}
This implies that, if we sample $\{x'_{j}\}_{j=1}^m$ from (\ref{eq:Appendix-Diffusion-Preliminary-7}),
\begin{align}
  &  \int
        \mathbb{E}[\rho_*(\bar{X}_0)|\bar{X}_{T-s}=x']
        \frac{\partial}{\partial x}\mathbb{P}[\bar{X}_{T-s}=x'|\bar{X}_{T-t}=x]\mathrm{d}x'
  \\ &  \approx \frac{e^h}{(e^{2h} - 1)}\frac1m\sum_{j=1}^m \mathbb{E}[\rho_*(\bar{X}_0)|\bar{X}_{T-s}=x'_j] (x'_{j}-(e^h x + 2(e^h-1)s(x_{t},t))),
\end{align}
and approximate each $\mathbb{E}[\rho_*(\bar{X}_0)|\bar{X}_{T-s}=x'_j]$ in the same way as (\ref{eq:Appendix-Diffusion-Preliminary-8}), we can approximate the correction term (\ref{eq:Appendix-Diffusion-Preliminary-6}). 



Now, our goal is to bound the error of the whole pipeline under the following assumptions of $p_*$ and $\rho_*$ and the score approximation error.
\begin{assumption}[Assumption~\ref{assumption:TVBoundMainText}, restated]\label{assumption:TVBoundMainText-2}
\begin{enumerate}[topsep=0mm,itemsep=-1mm,leftmargin = 6mm]
\item 
    $\nabla \log p_t$ is $L_p$-smooth at every time $t$ and it has finite second moment $\mathbb{E}[\|\bar{X}_t\|^2_2] \leq \mathsf{m} < \infty$ for all $t\in \R_+$ and $x\in \R^d$. 
\item  $\nabla \log \rho_*$ is $L_\rho$-smooth and bounded as $C_\rho^{-1}\leq \rho_* \leq C_\rho$ for a constant $C_\rho$.
\item   The score estimation error is bounded by 
%$\|s(\cdot,t)-\nabla \log p_{T-t}\|_{L^\infty}\leq \varepsilon$ 
\revisedStart
$\E_{\bar{X}_{\cdot}^{\leftarrow}}[\|s(\bar{X}_{t}^{\leftarrow},t)-\nabla \log p_{T-t}(\bar{X}_{T-t}^{\leftarrow})\|^2]\leq \varepsilon$ 
\revisedEnd
at each time $t$. 
\item 
$\E_{p_t}[\|u_*(x,lh) - u(x,lh)\|^2] \leq \varepsilon_{\rho,l}^2$~~for any $1 \leq l \leq T/h$.
%$\E_{X_{lh}}[\E_{\bar{D}_{n,l}(X_{lh})}[\|u_*(X_{lh},lh) - \hat{u}(X_{lh},lh)\|^2]] \leq \varepsilon_{\rho,l}^2$~~for any $1 \leq l \leq T/h$
\end{enumerate}
\end{assumption}
% \begin{assumption}\label{assumption:smoothnessP}
%     For the original distribution $p_*$, we assume that its score $\nabla \log p_t$ is $L_p$-smooth at every time $t$ and it has finite second moment $\mathbb{E}[\|\bar{X}_t\|^2_2] \leq \mathsf{m} < \infty$ for all $t\in \R_+$ and $x\in \R^d$. 
% \end{assumption}
% \begin{assumption}\label{assumption:smoothnessH}
%     For the density ratio $\rho_*$, we assume that $\nabla \log \rho_*$ is $L_\rho$-smooth and bounded as $C_\rho^{-1}\leq \rho_* \leq C_\rho$ for a constant $C_\rho$. 
% \end{assumption}
% \begin{assumption}\label{assumption:ApproxError}
%     The score estimation error is bounded by $\|s(\cdot,t)-\nabla \log p_{T-t}\|_{L^\infty}\leq \varepsilon$ at each time $t$. 
% \end{assumption}
% \begin{align}
%     \| p_{0|t}(x|x') - \hat{p}_{0|t}(x|x')\|_1^2 \leq \sum_{l=k}^{N}\frac{T}{h}\mathbb{E}_{p_{0|t}(x|t')}[\|\nabla \log p_{T-lh}(x)-s(x,lh)\|^2]
% \end{align}

% Under this, Theorem~\ref{} is restated as follows.
% \begin{thm}
%     Under Assumptions~\ref{assumption:smoothnessP}, \ref{assumption:smoothnessH}, and \ref{assumption:ApproxError}, take
%     \begin{align}
%         T &\simeq \log \frac{d+\mathsf{m}}{\varepsilon}, 
%         \\
%         \varepsilon &\lesssim ,
%         \\
%         h & \lesssim,
%        \\ n &\lesssim ,
%       \\  m &\lesssim ,
%         .
%     \end{align}
%     Then, the distribution $\hat{q}$ of $Y_T$ satisfies
%     \begin{align}
%         \mathrm{KL}(q_*\|\hat{q}) \leq \varepsilon_*
%     \end{align}
% \end{thm}
\begin{thm}[Theorem \ref{thm:Diffusion-1}, restated]\label{thm:Diffusion-1-appendix}
Suppose that Assumption \ref{assumption:TVBoundMainText-2} is satisfied. Then, we have the following bound on the distribution $\hat{q}$ of $Y^\leftarrow_T$: 
\begin{align}
\TV(q_*,\hat{q})^2
\lesssim  T \varepsilon^2 + \sum_{l=1}^{T/h} h \varepsilon_{\rho,l}^2 + T (L_pC_\rho^2+L_\rho)^2(dh + \mathsf{m}^2 h^2 )+ \exp(-2 T)\KL(\qstar \| N(0,I)).
\end{align}
\end{thm}
\begin{proof}
    Suppose that $\|s(\cdot,t)+u(\cdot,t)-\nabla \log q_{T-t}\|_{L^\infty}\leq \varepsilon'$ and $\nabla \log q_{T-t}$ is $L_q$-smooth at every time $t$. 
    According to \cite{chen2023sampling} and Pinsker's inequality, the distribution $\hat{q}$ of $Y_T$ satisfies
    \begin{align}
        \mathrm{KL}(\hat{q}\|q_*)^2 \lesssim
       T \varepsilon^2 + \sum_{l=1}^{T/h} h \varepsilon_{\rho,l}^2 + T \Lipdp^2(dh + \mathsf{m}^2 h^2 )+ \exp(-2 T)\KL(\qstar \| N(0,I)).
    \end{align}
    According to \Cref{lem:Smoothness}, $L_q$ is bounded by
    \begin{align}
        L_q \lesssim L_pC_\rho^2+L_\rho,
    \end{align}
    which yields the assertion.
\end{proof}


In the bound, we assumed that the term $\varepsilon_{\rho,l}^2$ can be bounded, however this approximation error can be derived as in the following theorem with additional technical conditions. 
\begin{assumption}[Assumption~\ref{ass:BoundingHMainText} restated]
%\begin{enumerate}[topsep=0mm,itemsep=-1mm,leftmargin = 6mm]\label{ass:NonconvexF}
    %\item[(i)] 
    (i) $\nabla_x s(\cdot,\cdot)$ is $H_s$-Lipschitz continuous in a sense that $\|\nabla_x s(x,t) - \nabla_y s(y,t)\|_{\mathrm{op}} \leq H_s \|x- y\|$
for any $x,y \in \sR^d$ and $0 \leq t \leq T$ and $\E[\| s(\tilde{X}_{kh}^\leftarrow,kh)\|^2] \leq Q^2$ for any $k$, 
    %\item[(ii)] 
    (ii) There exists $R > 0$ such that $\sup_{t,x}\{\|\nabla_x^2 \log p_t(x)\|_{\mathrm{op}},\|\nabla_x^2 \log s(x,t)\|_{\mathrm{op}}\} \leq R$.
%\end{enumerate}
\end{assumption}
\begin{thm}[Theorem~\ref{thm:Diffusion-2} restated]
Suppose that Assumptions \ref{assumption:TVBoundMainText} and \ref{ass:BoundingHMainText} hold.  
%the same condition as Theorem \ref{thm:delhDiffXYExp} holds 
Assume that $\|\rho_* - \rho\|_\infty \leq \varepsilon'$, $\|\rho\|_\infty \leq C_\rho$,
and 
$\sup_x\|\nabla \rho_*(x)\|\leq R_\rho$, $\|\nabla \rho_*(x) - \nabla \rho_*(y)\| \leq L_\rho\|x-y\|~(\forall x,y)$.
%(and some additional technical conditions). 
%Let $L_\varphi$ and $R_\varphi$ be as given in Lemma \ref{lemm:phiYboundLip}.
Then, for any choice of $0 \leq h \leq \delta \leq 1/(1 + 2R)$, we have that 
\begin{align}
%& %\E_{\bar{Y}_{\cdot}^{\leftarrow}}[ \|\nabla_x \E[\rho(\tilde{X}_{T}^\leftarrow) |  \tilde{X}_{t}^\leftarrow = x] |_{x = \bar{Y}_{t}^{\leftarrow}} -  \hat{u}(\bar{Y}_{k_t h}^\leftarrow,t) \|^2]  
\varepsilon_{\rho,l}^2 %\\
\lesssim & 
C_\rho^3 \left\{ \Xi_{\delta,\varepsilon}
+  R_\varphi^2 \left(\varepsilon^2+ \Lipdp^2 d(\delta + \mathsf{m} \delta^2) \right) + 
[ L_\varphi^2 (\mathsf{m} + Q^2 + d h)  
+ R_\varphi^2  (1 + 2R)^2] h^2\right\} \\
& +   \min\{T-lh,1/(2+2R)\}^{-1} \varepsilon'^2,
%+  C_\rho (1 + 2R) \sqrt{\frac{\log(T/h)}{n h}}, 
%= & \gO\left(\varepsilon^2 + \delta + \frac{\varepsilon_\TV^2}{\delta}\right), 
\end{align}
where 
%\begin{align}
$
\Xi_{\delta,\varepsilon} := C_\rho^2 (1+2R)^2 \delta  
+  C_\rho^2 \frac{1 + \delta R_\varphi^2}{\delta} [\varepsilon^2+ \Lipdp^2 d(h + \mathsf{m} h^2)], 
%\varepsilon_{\TV}^2
%+  R_\varphi^2  [\varepsilon^2 + \Lipdp^2 d(\delta + \mathsf{m} \delta^2)],
%&  = \gO\left(\delta + \varepsilon^2 + \frac{\varepsilon_\TV^2}{\delta}\right),
$ and 
$R_\varphi$ and $L_\varphi$ are constants introduced in Lemma \ref{lemm:phiYboundLip}. 
\end{thm}


\subsection{Bounding the smoothness}
The proof of Theorem \ref{thm:Diffusion-1} (i.e., Theorem \ref{thm:Diffusion-1-appendix}) utilizes the smoothness of the density $q_t$ corresponding to the aligned model. The following lemma gives its bound. 

\begin{lem}\label{lem:Smoothness}
    Suppose that $\nabla \log p_t$ is $L_p$-Lipschitz for all $t$ and $\nabla \log \rho_*$ is $L_\rho$-Lipschitz, and $C_\rho^{-1}\leq \rho_*\leq C_\rho$. 
    Then, $\nabla\log  q_t$ is $L_q$-Lipschitz for all $t$, where $L_q$ is bounded by
    \begin{align}
        L_q \leq \min \bigg\{\frac{4(L_p+L_\rho)^2}{2(L_p+L_\rho)-1}, (4+C_\rho^2) L_p + 4L_\rho\bigg\}\lesssim L_pC_\rho^2+L_\rho.
    \end{align}
\end{lem}
\begin{proof}
    We divide the proof into two parts, with $\sigma_t^2 = \frac{1}{2(L_p + L_\rho)}$ as the boundary.
    
    First consider the case when $\sigma_t^2 \leq \frac{1}{2(L_p + L_\rho)}$. 
    Remind that
    \begin{align}
        q_t (x) = \int q_*(y)\frac{1}{(2\pi\sigma_t^2)^{\frac{d}{2}}}\exp\bigg(-\frac{1}{2\sigma_t^2}\|m_t y - x\|^2\bigg)\mathrm{d}y,
    \end{align}
    with $\sigma_t^2 = 1-e^{-2t}$ and $m_t = e^{-t}$. 
    From this, $\nabla_x q_t(x)$ and $\nabla_x^2 q_t(x)$ are computed as
    \begin{align}
        \nabla_x q_t(x) &= \nabla_x \int q_*(y)\frac{1}{(2\pi\sigma_t^2)^{\frac{d}{2}}}\exp\big(-\frac{1}{2\sigma_t^2}\|m_t y - x\|^2\big)\mathrm{d}y 
        % \\ &= \int q_*(y)\frac{1}{(2\pi\sigma_t^2)^{\frac{d}{2}}}\exp\big(-\frac{1}{2\sigma_t^2}\|m_t y - x\|^2\big)\frac{x-m_ty}{\sigma_t^2}\mathrm{d}y
        \\ &= m_t^{-1}\int(\nabla_y q_*(y))\frac{1}{(2\pi\sigma_t^2)^{\frac{d}{2}}}\exp\big(-\frac{1}{2\sigma_t^2}\|m_t y - x\|^2\big)\mathrm{d}y
    \end{align}
    and
    \begin{align}
        \nabla_x^2 q_t(x) 
       & = m_t^{-2}\int(\nabla_y^2 q_*(y))\frac{1}{(2\pi\sigma_t^2)^{\frac{d}{2}}}\exp\big(-\frac{1}{2\sigma_t^2}\|m_t y - x\|^2\big)\mathrm{d}y
       .
    \end{align}
    Thus, we can compute $\nabla^2_x \log q_t(x)$ as
    \begin{align}
      &  \nabla^2_x \log q_t (x) 
      \\ &= \frac{\nabla^2_x q_t (x)}{q_t (x)} - \frac{\nabla_x q_t (x)(\nabla_x q_t (x))^\top }{(q_t (x))^2}
    \\ &= \frac{ m_t^{-2} \int (\frac{\nabla_y^2 q_*(y)}{q_*(y)}-\frac{\nabla_y q_*(y)\nabla_y q_*(y)^\top}{q_*(y)^2})q_*(y)\exp\big(-\frac{1}{2\sigma_t^2}\|m_t y - x\|^2\big)\mathrm{d}y}{\int q_*(y)\exp\big(-\frac{1}{2\sigma_t^2}\|m_t y - x\|^2\big)\mathrm{d}y}
    \label{eq:Appendix-Diffusion-Smoothness-1}
    \\ &\quad + \frac{ m_t^{-2} \int \frac{\nabla_y q_*(y)\nabla_y q_*(y)^\top}{q_*(y)^2} q_*(y)\exp\big(-\frac{1}{2\sigma_t^2}\|m_t y - x\|^2\big)\mathrm{d}y}{\int q_*(y)\exp\big(-\frac{1}{2\sigma_t^2}\|m_t y - x\|^2\big)\mathrm{d}y}
    \label{eq:Appendix-Diffusion-Smoothness-2}
       \\ &\quad -\frac{m_t^{-2} \int \frac{\nabla_y q_*(y)}{q_*(y)}q_*(y)\exp\big(-\frac{1}{2\sigma_t^2}\|m_t y - x\|^2\big)\mathrm{d}y\int \frac{\nabla_y q_*(y)}{q_*(y)}q_*(y)\exp\big(-\frac{1}{2\sigma_t^2}\|m_t y - x\|^2\big)\mathrm{d}y}{(\int q_*(y)\exp\big(-\frac{1}{2\sigma_t^2}\|m_t y - x\|^2\big)\mathrm{d}y)^2}.
       \label{eq:Appendix-Diffusion-Smoothness-3}
        % \\
        % &= \frac{m_t^{-1} \int \nabla_y q_*(y)\exp\big(-\frac{1}{2\sigma_t^2}\|m_t y - x\|^2\big)\mathrm{d}y}{\int q_*(y)\exp\big(-\frac{1}{2\sigma_t^2}\|m_t y - x\|^2\big)\mathrm{d}y},
    \end{align}
    Eq.~(\ref{eq:Appendix-Diffusion-Smoothness-1}) is an expectation of $\nabla_y^2 \log q_*(y) = \frac{\nabla_y^2 q_*(y)}{q_*(y)}-\frac{\nabla_y q_*(y)\nabla_y q_*(y)^\top}{q_*(y)^2}$ with respect to a distribution 
    \begin{align}
        A(y|x) \propto q_*(y)\exp\big(-\frac{1}{2\sigma_t^2}\|m_t y - x\|^2\big).
    \end{align}
    Therefore, (\ref{eq:Appendix-Diffusion-Smoothness-1}) is bounded by $m_t^{-2}(L_p+L_\rho)$ from the assumption.
    On the other hand, the other two terms are regarded as the covariance of $\nabla \log q_*(y)$ with respect to that distribution.
    Because $\sigma_t^2 \leq \frac{1}{2(L_p + L_\rho)}$, $A(y|x)$
    is $(L_p + L_\rho)$-strongly concave, and therefore satisfies the Poincar\'e inequality with a constant $\frac{1}{L_p + L_\rho}$. Therefore, for any $a\in \R^d$, we have
    \begin{align}
  &  a^\top (\text{(\ref{eq:Appendix-Diffusion-Smoothness-2})}+\text{(\ref{eq:Appendix-Diffusion-Smoothness-3})})a
  \\ &  = m_t^{-2}a^\top(\mathbb{E}_{A(y|x)}[(\nabla \log q_*(y))(\nabla \log q_*(y))^\top] - \mathbb{E}_{A(y|x)}[\nabla \log q_*(y)]\mathbb{E}_{A(y|x)}[\nabla \log q_*(y)]^\top)a
      \\ &    \leq \frac{m_t^{-2}}{L_p + L_\rho} \cdot \mathbb{E}[\|a\nabla^2 \log q_*(y)\|^2]\leq \frac{m_t^{-2}}{L_p + L_\rho}(L_p+L_\rho)^2 = m_t^{-2}L_p+L_\rho.
    \end{align}
    This implies that (\ref{eq:Appendix-Diffusion-Smoothness-2})$+$(\ref{eq:Appendix-Diffusion-Smoothness-3}) is $m_t^{-2}(L_p+L_\rho)$-smooth and $\nabla_x^2 \log q_t(x)$ is $2m_t^{-2}(L_p+L_\rho)$-smooth.
    Because $m_t = \sqrt{1-\sigma_t^2}$, we have $m_t \geq \sqrt{1-\frac{1}{2(L_p+L_\rho)}}.$ By applying this, we have $2m_t^{-2}(L_p+L_\rho)\leq \frac{4(L_p+L_\rho)^2}{2(L_p+L_\rho)-1}$.

    Next, let us consider the case when $\sigma_t^2 \geq \frac{2}{(L_p + L_\rho)}$. 
    Note that  $\nabla_x q_t(x)$ and $\nabla_x^2 q_t(x)$ are also written as
    \begin{align}
        \nabla_x q_t(x) &= 
        -\int q_*(y)\frac{1}{(2\pi\sigma_t^2)^{\frac{d}{2}}}\exp\big(-\frac{1}{2\sigma_t^2}\|m_t y - x\|^2\big)\frac{x-m_ty}{\sigma_t^2}\mathrm{d}y
    \end{align}
    and
    \begin{align}
       & \nabla_x^2 q_t(x) 
     \\  & =-\int q_*(y)\frac{1}{(2\pi\sigma_t^2)^{\frac{d}{2}}}\exp\big(-\frac{1}{2\sigma_t^2}\|m_t y - x\|^2\big)\frac{I}{\sigma_t^2}\mathrm{d}y   
      \\ & \quad  + \int q_*(y)\frac{1}{(2\pi\sigma_t^2)^{\frac{d}{2}}}\exp\big(-\frac{1}{2\sigma_t^2}\|m_t y - x\|^2\big)\frac{(x-m_ty)(x-m_ty)^\top}{\sigma_t^4}\mathrm{d}y.
    \end{align}
    Thus, we can compute $\nabla_x^2 \log q_t(x)$ as
    \begin{align}
      &  \nabla^2_x \log q_t (x) 
     \\  &= \frac{\nabla^2_x q_t (x)}{q_t (x)} - \frac{\nabla_x q_t (x)(\nabla_x q_t (x))^\top }{(q_t (x))^2}
      \\ &=  -\frac{\int q_*(y)\exp\big(-\frac{1}{2\sigma_t^2}\|m_t y - x\|^2\big)\frac{I}{\sigma_t^2}\mathrm{d}y }{\int q_*(y)\exp\big(-\frac{1}{2\sigma_t^2}\|m_t y - x\|^2\big)\mathrm{d}y}
       \label{eq:Appendix-Diffusion-Smoothness-4}
      \\ & \quad  + \frac{\int q_*(y)\exp\big(-\frac{1}{2\sigma_t^2}\|m_t y - x\|^2\big)\frac{(x-m_ty)(x-m_ty)^\top}{\sigma_t^4}\mathrm{d}y}{\int q_*(y)\exp\big(-\frac{1}{2\sigma_t^2}\|m_t y - x\|^2\big)\mathrm{d}y}
       \label{eq:Appendix-Diffusion-Smoothness-5}
      \\ & \quad - \frac{\big(\int q_*(y)\exp\big(-\frac{1}{2\sigma_t^2}\|m_t y - x\|^2\big)\frac{(x-m_ty)}{\sigma_t^2}\mathrm{d}y\big)\big(\int q_*(y)\exp\big(-\frac{1}{2\sigma_t^2}\|m_t y - x\|^2\big)\frac{(x-m_ty)}{\sigma_t^2}\mathrm{d}y\big)^\top}{\big(\int q_*(y)\exp\big(-\frac{1}{2\sigma_t^2}\|m_t y - x\|^2\big)\mathrm{d}y\big)^2}.
       \label{eq:Appendix-Diffusion-Smoothness-6}
    \end{align}
    Eq.~(\ref{eq:Appendix-Diffusion-Smoothness-4}) is bounded by $\sigma_t^{-2}\leq 2(L_p+L_\rho)$. 
    On the other hand, (\ref{eq:Appendix-Diffusion-Smoothness-5})+(\ref{eq:Appendix-Diffusion-Smoothness-6}) are transformed into
    \begin{align}
       & \text{(\ref{eq:Appendix-Diffusion-Smoothness-5})+(\ref{eq:Appendix-Diffusion-Smoothness-6})}
       \\ & = \frac{m_t^2}{\sigma_t^4}\frac{\int q_*(y)\exp\big(-\frac{1}{2\sigma_t^2}\|m_t y - x\|^2\big)yy^\top\mathrm{d}y}{\int q_*(y)\exp\big(-\frac{1}{2\sigma_t^2}\|m_t y - x\|^2\big)\mathrm{d}y}
      \\ & \quad - \frac{m_t^2}{\sigma_t^4}\frac{\big(\int q_*(y)\exp\big(-\frac{1}{2\sigma_t^2}\|m_t y - x\|^2\big)y\mathrm{d}y\big)\big(\int q_*(y)\exp\big(-\frac{1}{2\sigma_t^2}\|m_t y - x\|^2\big)y\mathrm{d}y\big)^\top}{\big(\int q_*(y)\exp\big(-\frac{1}{2\sigma_t^2}\|m_t y - x\|^2\big)\mathrm{d}y\big)^2}
      % \\ &  = \frac{\int q_*(\frac{z-x}{m_t})\exp\big(-\frac{\|z\|^2}{2\sigma_t^2}\big)\frac{zz^\top}{\sigma_t^4}\mathrm{d}z}{\int q_*(\frac{z-x}{m_t})\exp\big(-\frac{\|z\|^2}{2\sigma_t^2}\big)\mathrm{d}z}
      % \\ & \quad - \frac{\big(\int q_*(\frac{z-x}{m_t})\exp\big(-\frac{\|z\|^2}{2\sigma_t^2}\big)\frac{z}{\sigma_t^2}\mathrm{d}z\big)\big(\int q_*(\frac{z-x}{m_t})\exp\big(-\frac{\|z\|^2}{2\sigma_t^2}\big)\frac{z}{\sigma_t^2}\mathrm{d}z\big)^\top}{\big(\int q_*(\frac{z-x}{m_t})\exp\big(-\frac{\|z\|^2}{2\sigma_t^2}\big)\mathrm{d}z\big)^2}
     \\ & =\frac{m_t^2}{\sigma_t^4}\mathrm{Var}_{q_{0|t}(y|x)}(y),
    \end{align}
    where $\mathrm{Var}_{q_{0|t}(y|x)}(y)$ means the variance of $X_0$ conditioned on $X_t=x$, with respect to $q_t$.

    Thus, bounding $\mathrm{Var}_{q_{0|t}(y|x)}(x)$ yields the conclusion. 
    Lemma~\ref{lem:Bounded-Discrepancy} implies that
    \begin{align}
        \mathrm{Var}_{q_{0|t}(y|x)}(y) \leq C_\rho^2 \mathrm{Var}_{p_{0|t}(y|x)}(y).
        \label{eq:Appendix-Diffusion-Smoothness-7}
    \end{align}
    Similarly to $\nabla_x^2 \log q_t(x)$, $\nabla_x^2 \log p_t(x)$ satisfies
    \begin{align}
        \nabla_x^2 \log p_t(x) = -\sigma_t^{-2}I_d + \frac{m_t^2}{\sigma_t^4}\mathrm{Var}_{p_{0|t}(y|x)}(y).
        \label{eq:Appendix-Diffusion-Smoothness-8}
    \end{align}
    By combining (\ref{eq:Appendix-Diffusion-Smoothness-7}) and (\ref{eq:Appendix-Diffusion-Smoothness-8}), we have
    \begin{align}
        \frac{m_t^2}{\sigma_t^4} \mathrm{Var}_{q_{0|t}(y|x)}(y) \leq C_\rho^2
        \sigma_t^{-2}I_d + C_\rho^2\nabla_x^2 \log p_t(x).
    \end{align}
    Therefore, from the assumption that $\nabla_x \log p_t(x)$ is $L_p$-Lipschitz and $\sigma_t^{-2}\geq 2(L_p+L_\rho)$, we obtain that $ \|\frac{m_t^2}{\sigma_t^4} \mathrm{Var}_{q_{0|t}(y|x)}(y)\|\leq (2+C_\rho^2) L_p + 2L_\rho$. 

    By putting it all together, $\nabla \log q_t$ is $((4+C_\rho^2) L_p + 4L_\rho)$-Lipschitz.
\end{proof}

\begin{lem}\label{lem:Bounded-Discrepancy}
    When $C_\rho^{-1}\leq h_*(x)\leq C_\rho$, we have
    \begin{align}
        \frac{q_{0|t}(x|x')}{p_{0|t}(x|x')}\leq C_\rho^2 
    \end{align}
    for all $x,x'$ and $t$.
\end{lem}
\begin{proof}
    We can write $q_{0|t}(x|x')$ as
    \begin{align}
        q_{0|t}(x|x') = \frac{q_{0,t}(x,x')}{\int q_{0}(x'')q_{t|0}(x|x'')\mathrm{d}x''}
        =  \frac{p_*(x)\rho_*(x)q_{t|0}(x'|x)}{\int p_*(x'')\rho_*(x'')q_{t|0}(x|x'')\mathrm{d}x''}.
    \end{align}
    Because $C_\rho^{-1}\leq \rho_*(x)\leq C_\rho$ and $q_{t|0}(x'|x)=p_{t|0}(x'|x)$, we have
    \begin{align}
        q_{0|t}(x|x') \leq  C_\rho^2\frac{p_*(x)p_{t|0}(x'|x)}{\int p_*(x'')p_{t|0}(x|x'')\mathrm{d}x''}
        = C_\rho^2p_{0|t}(x|x'),
    \end{align}
    which concludes the proof.
\end{proof}

\subsection{Estimation Error of the Correction Term}
\label{sec:proofOfCorrection}
% \begin{lem}
%     Suppose that $\nabla \log p_t$ is $L_p$-smooth.
%     Also assume that the error of the score function $s(x,t)$ is bounded by $\varepsilon$ in $L^\infty$-norm for all $t=kh$. 
%     If we independently sample $\{\tilde{y}_{T,i}\}_{i=1}^n$ using (\ref{eq:Appendix-Diffusion-Preliminary-9}), we have
%     \begin{align}
%      \bigg| \frac1n\sum_{i=1}^n \rho'(\tilde{y}_{T,i})-  \mathbb{E}[\rho_*(\bar{X}_0)|\bar{X}_{T-t}=x] \bigg|\lesssim  \|\rho'-\rho_*\|_{\infty}+
%     T^\frac12 \varepsilon +  TL_pd^\frac12 h^\frac12+\frac{C_\rho}{\sqrt{n}}\log (\delta^{-1}) 
%     \end{align}
%     with probability $1-\delta$. 
% \end{lem}
% \begin{proof}
%     According to \cite{chen2023improved} (Theorem 1 without the term on the convergence of forward process) and Pinsker's inequality, we have
%     \begin{align}
%         \TV(\tilde{y}_{T,i},\bar{X}_0|\bar{X}_{T-t}=x)
%         \lesssim \mathrm{KL}(\tilde{y}_{T,i},\bar{X}_0|\bar{X}_{T-t}=x)^\frac12 \lesssim T^\frac12 \varepsilon +  TL_pd^\frac12 h^\frac12.
%     \end{align}
%     Because $\rho_*$ is bounded by $C_\rho$, 
%     \begin{align}
%       |\mathbb{E}[\rho_*(\bar{X}_0)|\bar{X}_{T-t}=x]
%         -
%         \mathbb{E}[\rho_*(\tilde{y}_{T,i})]| 
%         &\leq \|h\|_\infty\mathrm{TV}(\tilde{y}_{T,i},\bar{X}_0|\bar{X}_{T-t}=x)
%      \\ &   \lesssim C_\rho(T^\frac12 \varepsilon +  TL_pd^\frac12 h^\frac12).
%         \label{eq:Appendix-Correction-Approximation-1}
%     \end{align}
%     Bounding $\frac1n\sum_{i=1}^n \rho_*(\tilde{y}_{T,i}) -\mathbb{E}[\rho_*(\tilde{y}_{T,i})]$ by Hoeffding's inequality, we have
%     \begin{align}
%         \bigg|\frac1n\sum_{i=1}^n \rho_*(\tilde{y}_{T,i}) -\mathbb{E}[\rho_*(\tilde{y}_{T,i})]\bigg|
%         \leq \frac{C_\rho}{\sqrt{2n}}\log (2\delta^{-1})
%         \label{eq:Appendix-Correction-Approximation-2}
%     \end{align}
%     with probability $1-\delta$. 
%     By combining (\ref{eq:Appendix-Correction-Approximation-1}) and (\ref{eq:Appendix-Correction-Approximation-2}) and replacing $\frac1n\sum_{i=1}^n \rho_*(\tilde{y}_{T,i})$ by $\frac1n\sum_{i=1}^n \rho'(\tilde{y}_{T,i})$, we obtain the assertion. 
% \end{proof}



Since we have shown the time discretization error in the previous section, what we remain to show is just an upper bound of $\varepsilon_{\rho,l}^2$. For that purpose, we put an additional assumption which is almost same as Assumption \ref{ass:BoundingHMainText} except the condition (iii). 
A bound of $\varepsilon_{\TV}^2$ in the third condition (iii) will be given as 
$\varepsilon_\TV^2 = \gO(\varepsilon^2 + h)$ by \cite{chen2023sampling}. 
\begin{assumption}\label{ass:BoundingH}
\begin{enumerate}[topsep=0mm,itemsep=-1mm,leftmargin = 6mm]
    \item[(i)] $\nabla_x s(\cdot,\cdot)$ is $H_s$-Lipschitz continuous in a sense that $\|\nabla_x s(x,t) - \nabla_y s(y,t)\|_{\mathrm{op}} \leq H_s \|x- y\|$
for any $x,y \in \sR^d$ and $0 \leq t \leq T$ and $\E[\| s(\bar{X}_{kh}^\leftarrow,kh)\|^2] \leq Q^2$ for any $k$.
    \item[(ii)] There exists $R > 0$ such that $\sup_{t,x}\{\|\nabla_x^2 \log p_t(x)\|_{\mathrm{op}},\|\nabla_x^2 \log s(x,t)\|_{\mathrm{op}}\} \leq R$.
    \item[(iii)] $\E_{\bar{X}_t}[\TV(\bar{X}_{T}^\leftarrow,{X}_{T}^\leftarrow|\bar{X}_{T-t}^\leftarrow = {X}_{T-t}^\leftarrow = \bar{X}_t)^2] \leq \varepsilon_{\TV}^2$ for any $t \in [0,T]$.
\end{enumerate}    
\end{assumption}

\begin{thm}\label{thm:delhDiffXYExp}
Suppose that $0 \leq h \leq \delta \leq 1/(1 + 2R)$
and Assumptions \ref{ass:BoundingH} and \ref{assumption:TVBoundMainText-2} hold. 
Let $L_\varphi$ and $R_\varphi$ be as given in Lemma \ref{lemm:phiYboundLip}.
Then, it holds that  
\begin{align}
& \E_{\bar{X}_{\cdot}^\leftarrow}[ \|\nabla_x \E[\rho_*(\bar{X}_{T}^\leftarrow) |  \bar{X}_{t}] % deleted!
% |_{x = \bar{X}_{t}^\leftarrow} 
-  \nabla_x \E[\rho_*({X}_{T}^\leftarrow) | {X}_{kh}^\leftarrow]
% |_{x = \bar{X}_{kh}^\leftarrow}
\|^2]  \\
\lesssim & 
R_\varphi^2 \left(\varepsilon^2+ \Lipdp^2 d(\delta + \mathsf{m} \delta^2) \right)
+ \Xi_{\delta,\varepsilon}
+
[ L_\varphi^2 (\mathsf{m} + 4Q^2 + d h)  
+ R_\varphi^2  (1 + 2R)^2] h^2 \\
= & \gO\left(\varepsilon^2 + \delta + \frac{\varepsilon_\TV^2}{\delta}\right), 
\end{align}
where 
\begin{align}
\Xi_{\delta,\varepsilon} & := \frac{4 c_\eta^2  C_\rho^2 (1+2R)^2}{3}\delta  
+ 2 \exp(2)
\left\{ C_\rho^2 \frac{\varepsilon_{\TV}^2}{\delta} + C R_\varphi^2  [\varepsilon^2 + \Lipdp^2 d(\delta + \mathsf{m} \delta^2)]\right\},
%&  = \gO\left(\delta + \varepsilon^2 + \frac{\varepsilon_\TV^2}{\delta}\right),
\end{align} 
and $c_\eta > 0$ is a universal constant. 
\end{thm}
By \cite{chen2023sampling}, $\varepsilon_\TV^2 = \gO(\varepsilon^2 + h)$, and thus by substituting $\delta \leftarrow \sqrt{h}$, we finally obtain an error estimate as 
$$
\left( 1 + \frac{1}{\sqrt{h}}\right) \varepsilon^2 + \sqrt{h}. 
$$
\begin{proof}
Let $t \in [kh,(k+1)h)$ and $t^* = kh + \delta$ where $\delta$ is larger than or equal to $h$: $\delta \geq h$. 
We only consider a situation where $T- t \geq \delta$. 
The situation where $\delta < T- t$ can be treated in the same manner by noticing 
a trivial relation $\nabla_x \E[\rho_*(\bar{X}_{T}^\leftarrow) | \bar{X}_{T}^\leftarrow = x]
= \nabla \rho_*(x)$.
Then, for a given initial state $x \in \sR^d$, we define the stochastic processes as 
\begin{align}
    \bar{X}_{t}^\leftarrow=x,\ \mathrm{d}\bar{X}_{\tau}^\leftarrow = \{\bar{X}_{\tau}^\leftarrow + 2\nabla \log p_{T-\tau}(\bar{X}_{\tau}^\leftarrow)\}\mathrm{d}\tau + \sqrt{2}\mathrm{d}B_\tau,
    ~~~(t \leq \tau \leq T), 
\end{align}
and its numerical approximation as  
\begin{align}
    {X}_{t}^\leftarrow=x,\ \mathrm{d}{X}_{\tau}^\leftarrow = \{{X}_{\tau}^\leftarrow + 2s({X}^\leftarrow_{k_\tau h},k_\tau h)\}\mathrm{d}\tau + \sqrt{2}\mathrm{d}B_\tau.
    ~~~(t \leq \tau \leq T), 
\end{align}
where $k_\tau$ is the integer such that $\tau \in [k_\tau h,(k_\tau +1)h)$. 

Note that 
\begin{align}
& \nabla_x \E[\rho_*(\bar{X}_{T}^\leftarrow) |  \bar{X}_{t}]
-  \nabla_x \E[\rho_*({X}_{T}^\leftarrow) | {X}_{kh}^\leftarrow] \\
=&  
\underbrace{(\nabla_x \E[\rho_*(\bar{X}_{T}^\leftarrow) |  \bar{X}_{t}]
-  \nabla_x \E[\rho_*({X}_{T}^\leftarrow) | {X}_{t}^\leftarrow])}_{(a)} 
+ \underbrace{(\nabla_x \E[\rho_*({X}_{T}^\leftarrow) | {X}_{t}^\leftarrow]
-  \nabla_x \E[\rho_*({X}_{T}^\leftarrow) | {X}_{kh}^\leftarrow])}_{(b)}. 
\label{eq:NablaRhotDecomp}
\end{align}
We first evaluate the term (a): 
$$
\nabla_x \E[\rho_*(\bar{X}_{T}^\leftarrow) | \bar{X}_{t}^\leftarrow = x] - \nabla_x \E[\rho_*({X}_{T}^\leftarrow) | {X}_{t}^\leftarrow = x]. 
$$
As we have seen above, the derivative can be expressed by the following recursive formula of the conditional expectation:  
$$
\nabla_x \E[\E[\rho_*(\bar{X}_{T}^\leftarrow) | \bar{X}_{t^*} ] | \bar{X}_{t}^\leftarrow = x].
$$
For a notation simplicity, we let $\varphi_X(x) := \E[\rho_*(\bar{X}_{T}^\leftarrow) | \bar{X}_{t^*}^\leftarrow = x]$ and $\varphi_Y(x) := \E[\rho_*({X}_{T}^\leftarrow) | {X}_{t^*}^\leftarrow = x]$. Then, the Bismut-Elworthy-Li formula \citep{MR755001,ELWORTHY1994252} yields that, for any $v \in \sR^d$,
\begin{align}
v^\top \nabla_x \E[\rho_*(\bar{X}_{0}^\leftarrow) | \bar{X}_{t}^\leftarrow = x] = 
\E\left[\frac{1}{\delta} \int_{0}^\delta \langle \eta_{\bar{X},\tau} , \mathrm{d} B_\tau \rangle  \varphi_X( \bar{X}_{t^*}^\leftarrow) ~|~\bar{X}_{t}^\leftarrow = x\right], 
\end{align}
where $\eta_{\bar{X},\tau}$ is  the solution of 
\begin{align}
& \rd \eta_{\bar{X},\tau}  = (I + 2 \nabla^2 \log p_{T- t -\tau}(\bar{X}_{t+\tau}^\leftarrow)) \eta_{\bar{X},\tau} \rd \tau,  \\
& \eta_{\bar{X},0} = v. 
\end{align} 
Similarly, we define $\eta_{{X},\tau}$ for the process ${X}_{\tau}^\leftarrow$ as
\begin{align}
& \rd \eta_{{X},\tau}  = (\eta_{{X},\tau} + 2 \nabla_x^\top s({X}_{k_\tau h}^\leftarrow,k_{\tau} h)\eta_{{X},k_\tau h - t})  \rd \tau,  \\
& \eta_{{X},0} = v. 
\end{align} 
%Note that $\eta_{{X},\tau}$ has no randomness once the initial state is conditioned as $\bar{X}_{t}^\leftarrow = {X}_{t}^\leftarrow = x$.  
Then, 
\begin{align}
& v^\top \nabla_x \E[\rho_*(\bar{X}_{T}^\leftarrow) | \bar{X}_{t}^\leftarrow = x] -  v^\top \nabla_x \E[\rho_*({X}_{T}^\leftarrow) | {X}_{t}^\leftarrow = x]  \\
 = &  \E\left[ \frac{1}{\delta} \int_{0}^\delta \langle \eta_{\bar{X},\tau} -\eta_{{X},\tau}, \mathrm{d} B_\tau \rangle  \varphi_Y( {X}_{t^*}^\leftarrow)   ~|~\bar{X}_{t}^\leftarrow = {X}_{t}^\leftarrow = x\right]  \\
 & + \E\left[ \frac{1}{\delta} \int_{0}^\delta \langle \eta_{\bar{X},\tau} , \mathrm{d} B_\tau \rangle  (\varphi_X( \bar{X}_{t^*}^\leftarrow) - \varphi_Y( {X}_{t^*}^\leftarrow))  ~|~ \bar{X}_{t}^\leftarrow = {X}_{t}^\leftarrow = x\right]. 
 \label{eq:BismutBoundFirst}
\end{align}
By the Ito isometry, the first term of the right hand side can be bounded as 
\begin{align}
& \left( \E\left[ \frac{1}{\delta} \int_{0}^\delta \langle \eta_{\bar{X},\tau} -\eta_{{X},\tau} , \mathrm{d} B_\tau \rangle  \varphi_Y( {X}_{t^*})   ~|~\bar{X}_{t}^\leftarrow = {X}_{t}^\leftarrow =x\right]\right)^2 \\
\leq & 
 \E\left[ \left(\frac{1}{\delta} \int_{0}^\delta \langle \eta_{\bar{X},\tau} -\eta_{{X},\tau} , \mathrm{d} B_\tau \rangle  \varphi_Y( {X}_{t^*}) \right)^2  ~|~\bar{X}_{t}^\leftarrow = {X}_{t}^\leftarrow =x\right] \\
\leq & 
C_\rho^2 \E\left[ \left(\frac{1}{\delta} \int_{0}^\delta \langle \eta_{\bar{X},\tau} -\eta_{{X},\tau} , \mathrm{d} B_\tau \rangle  \right)^2  ~|~\bar{X}_{t}^\leftarrow ={X}_{t}^\leftarrow = x\right] \\
= &
C_\rho^2  \E\left[ \frac{1}{h^2} \int_{\tau}^h \| \eta_{\bar{X},\tau} -\eta_{{X},\tau}\|^2 \rd \tau  ~|~\bar{X}_{t}^\leftarrow ={X}_{t}^\leftarrow = x\right] \\
\leq &
2 C_\rho^2  \E\left[ \frac{1}{\delta^2} \int_{0}^\delta (\| \eta_{\bar{X},\tau} - v\|^2 + \| \eta_{{X},\tau} - v\|^2) \rd \tau  ~|~\bar{X}_{t}^\leftarrow ={X}_{t}^\leftarrow = x\right].
\end{align}
Hence, we just need to bound $\|\eta_{\bar{X},\tau} - v\|^2$ in the right hand side. 
We note that it obeys the following differential equation:   
\begin{align}
&\frac{\rd \|\eta_{\bar{X},\tau} - v\|^2}{\rd \tau}  \\
= & 2 (\eta_{\bar{X},\tau} - v)^\top \frac{\rd \eta_{\bar{X},\tau}}{\rd \tau} \\
= & 2 (\eta_{\bar{X},\tau} - v)^\top (I + 2 \nabla^2 \log p_{T- t -\tau}(\bar{X}_{T - t -\tau}^\leftarrow)) \eta_{\bar{X},\tau} \\
= & 2 (\eta_{\bar{X},\tau} - v)^\top (I + 2 \nabla^2 \log p_{T- t -\tau}(\bar{X}_{T - t -\tau}^\leftarrow)) [(\eta_{\bar{X},\tau} - v) + v] \\
\leq & 2 (1 + 2 R) \|\eta_{\bar{X},\tau} - v\|^2 +   2 (1 + 2R) \|v\| \|\eta_{\bar{X},\tau} - v\|, 
\end{align}
which also yields that  
\begin{align}
&2 \|\eta_{\bar{X},\tau} - v\| \frac{\rd \|\eta_{\bar{X},\tau} - v\|}{\rd \tau}  \leq  2(1+ 2 R) \|\eta_{\bar{X},\tau} - v\|^2 +    2(1+ 2 R) \|v\| \|\eta_{\bar{X},\tau} - v\| \\
\Rightarrow~ 
&\frac{\rd \|\eta_{\bar{X},\tau} - v\|}{\rd \tau}  \leq   2(1+ 2 R) (\|\eta_{\bar{X},\tau} - v\| +  \|v\|)  \\
\Rightarrow ~& \|\eta_{\bar{X},\tau} - v\| \leq [\exp( 2 (1+ 2 R) \tau) - 1]\|v\| \\
\Rightarrow ~& \|\eta_{\bar{X},\tau} - v\|^2 \leq [\exp( 2 (1+ 2 R) \tau) - 1]^2\|v\|^2.
\end{align}
Therefore, if $\delta$ is sufficiently small (such as $\delta \leq 1/(1 + 2R)$), then we arrive at 
$$
\|\eta_{\bar{X},\tau} - v\|^2 \leq c_\eta (1+ 2 R)^2 \tau^2\|v\|^2,
$$
with a universal constant $c_\eta$, for any $0 \leq \tau \leq \delta$. 
In the same vein, we also have 
\begin{align}
& \|\eta_{{X},\tau} - v\|^2 \leq c_\eta (1+ 2 R)^2 \tau^2\|v\|^2,~ \\
& \|\eta_{\bar{X},\tau}\|^2 \leq \exp(4(1+2R)\tau) \|v\|^2,~
\|\eta_{{X},\tau}\|^2 \leq \exp(4(1+2R)\tau) \|v\|^2,
\end{align}
for $0 \leq \tau \leq \delta$. 
These bounds yield that  
\begin{align}
& 2 C_\rho^2  \E\left[ \frac{1}{\delta^2} \int_{0}^\delta (\| \eta_{\bar{X},\tau} - v\|^2 + \| \eta_{{X},\tau} - v\|^2) \rd \tau  ~|~\bar{X}_{t}^\leftarrow = {X}_{t}^\leftarrow =x\right] \\
\leq & 4 c_\eta^2  C_\rho^2 (1+2R)^2  \frac{\int_0^\delta \tau^2 \rd \tau}{\delta^2} = 
\frac{4 c_\eta^2  C_\rho^2 (1+2R)^2}{3}\delta. 
\end{align}
%Then, by taking integral of $0 \leq h \leq h'$, we have that the first term becomes $O(h^2)$.

Next, we bound the second term of the right hand side in \Eqref{eq:BismutBoundFirst}: 
\begin{align}
& \E\left[ \frac{1}{\delta} \int_{0}^\delta \langle \eta_{\bar{X},\tau} , \mathrm{d} B_\tau \rangle  (\varphi_X( \bar{X}_{t^*}^\leftarrow) - \varphi_Y( {X}_{t^*}^\leftarrow))  ~|~ \bar{X}_{t}^\leftarrow = {X}_{t}^\leftarrow = x\right]^2 \\
\leq &
\E\left[ \frac{1}{\delta^2} \int_{0}^\delta \|\eta_{\bar{X},\tau}\|^2 \rd \tau  ~|~ \bar{X}_{t}^\leftarrow = {X}_{t}^\leftarrow = x\right] 
\E\left[ (\varphi_X( \bar{X}_{t^*}^\leftarrow) - \varphi_Y( {X}_{t^*}^\leftarrow))^2  ~|~ \bar{X}_{t}^\leftarrow = {X}_{t}^\leftarrow = x\right] \\
\leq &
\frac{\exp(2(1 + 2R)\delta)\|v\|^2}{\delta}
\E\left[ (\varphi_X( \bar{X}_{t^*}^\leftarrow) - \varphi_Y( {X}_{t^*}^\leftarrow))^2  ~|~ \bar{X}_{t}^\leftarrow = {X}_{t}^\leftarrow = x\right]. 
\label{eq:varPhiXYdiff}
\end{align}
Otherwise, we also have the following inequality:  
\begin{align}
& \E\left[ \frac{1}{\delta} \int_{0}^\delta \langle \eta_{\bar{X},\tau} , \mathrm{d} B_\tau \rangle  (\varphi_X( \bar{X}_{t^*}^\leftarrow) - \varphi_Y( {X}_{t^*}^\leftarrow))  ~|~ \bar{X}_{t}^\leftarrow = {X}_{t}^\leftarrow = x\right]^2 \\
\leq & 2 \E\left[ \frac{1}{\delta} \int_{0}^\delta \langle \eta_{\bar{X},\tau} , \mathrm{d} B_\tau \rangle  (\varphi_Y( \bar{X}_{t^*}^\leftarrow) -\varphi_Y( {X}_{t^*}^\leftarrow)  )  ~|~ \bar{X}_{t}^\leftarrow = {X}_{t}^\leftarrow = x\right]^2 \\ & 
+ 2 \E\left[  \eta_{\bar{X},\delta}^\top \nabla (\varphi_X( \bar{X}_{t^*}^\leftarrow)) - \varphi_Y( \bar{X}_{t^*}^\leftarrow))  ~|~ \bar{X}_{t}^\leftarrow = {X}_{t}^\leftarrow = x\right]^2 \\
\leq &
\frac{\exp(2(1 + 2R)\delta)\|v\|^2}{\delta}
\E\left[ (\varphi_Y( \bar{X}_{t^*}^\leftarrow) - \varphi_Y( {X}_{t^*}^\leftarrow))^2  ~|~ \bar{X}_{t}^\leftarrow = {X}_{t}^\leftarrow = x\right] \\
& + 2 \E\left[ \| \eta_{\bar{X},\delta} \| \|\nabla (\varphi_X( \bar{X}_{t^*}^\leftarrow)) - \varphi_Y( \bar{X}_{t^*}^\leftarrow))\|  ~|~ \bar{X}_{t}^\leftarrow = {X}_{t}^\leftarrow = x\right]^2.
\label{eq:varPhiXYdiffSecond}
\end{align}

For bounding these quantities, we need to bound the discrepancy $\|\bar{X}_{\tau}^\leftarrow - {X}_{\tau}^\leftarrow\|^2$.  
Note that this quantity follows the following ODE: 
\begin{align}
& \frac{\rd \|\bar{X}_{\tau}^\leftarrow - {X}_{\tau}^\leftarrow\|^2 }{\rd \tau} \\
= & 2(\bar{X}_{\tau}^\leftarrow - {X}_{\tau}^\leftarrow)^\top [(\bar{X}_{\tau}^\leftarrow - 2 \nabla_x \log p_{T-\tau-t}(\bar{X}_{\tau}^\leftarrow)) - 
({X}_{\tau}^\leftarrow - 2 s(x,kh))]  \\
= & 2\|\bar{X}_{\tau}^\leftarrow - {X}_{\tau}^\leftarrow\|^2 
- 4(\bar{X}_{\tau}^\leftarrow - {X}_{\tau}^\leftarrow)^\top 
(\nabla_x \log p_{T-\tau-t}(\bar{X}_{\tau}^\leftarrow) - s(x,kh))  \\
% = & 2\|\bar{X}_{\tau}^\leftarrow - {X}_{\tau}^\leftarrow\|^2 
% - 4(\bar{X}_{\tau}^\leftarrow - {X}_{\tau}^\leftarrow)^\top 
% [\nabla_x \log(p_{T-\tau-t}(\bar{X}_{\tau}^\leftarrow)/p_{T-t-\tau}(\bar{X}_{t}^\leftarrow)) + 
% \nabla_x \log(p_{T-\tau-t}(\bar{X}_{t}^\leftarrow)/p_{T-t}(\bar{X}_{t}^\leftarrow)) \\
% & +
% (\nabla \log (p_{T-t}(\bar{X}_{t}^\leftarrow)-s(x,kh))] \\
\leq 
& 4\|\bar{X}_{\tau}^\leftarrow - {X}_{\tau}^\leftarrow\|^2  + 2 \|\nabla_x \log(p_{T-\tau-t}(\bar{X}_{\tau}^\leftarrow))-s(x,kh)\|^2.
\end{align}
Therefore, it satisfies that 
\begin{align}
\|\bar{X}_{\tau}^\leftarrow - {X}_{\tau}^\leftarrow\|^2
\leq 
4 \int_0^\tau  \|\bar{X}_{s}^\leftarrow - {X}_{s}^\leftarrow\|^2 \rd s 
+ 
2 \int_0^\tau \|\nabla_x \log(p_{T-\tau-t}(\bar{X}_{s}^\leftarrow))-s(x,kh)\|^2 \rd s.
\end{align}
Taking its expectation, we see that 
\begin{align}
\E[\|\bar{X}_{\tau}^\leftarrow - {X}_{\tau}^\leftarrow\|^2]
\leq 
4 \int_0^\tau  \E[\|\bar{X}_{s}^\leftarrow - {X}_{s}^\leftarrow\|^2] \rd s 
+ 
2 \underbrace{\int_0^\tau (\varepsilon^2 + \Lipdp^2 d s + \Lipdp^2 \mathsf{m} s^2) \rd s}_{=\gO(\varepsilon^2 \tau + \Lipdp^2 d (\tau^2 + \mathsf{m} \tau^3)) =: \xi(\tau)},
\end{align}
where we used Theorem 10 (and its proof) of \cite{chen2023improved} for obtaining $\xi(\tau)$. 
Then, Gronwall inequality yields 
\begin{align}
\E[\|\bar{X}_{\tau}^\leftarrow - {X}_{\tau}^\leftarrow\|^2]
\leq \xi(\tau) + \int 4 \xi(s) e^{4(\tau -s)} \rd s \lesssim \varepsilon^2 \tau + \Lipdp^2 d(\tau^2 + \mathsf{m} \tau^3),
\end{align}
(see \cite{Mischeler:Note:2019} for example). 
Then, the Lipschitz continuity of $\varphi_Y$ (Lemma \ref{lemm:phiYboundLip}) yields that 
\begin{align}
& \E\left[ (\varphi_Y( \bar{X}_{t^*}^\leftarrow) - \varphi_Y( {X}_{t^*}^\leftarrow))^2  ~|~ \bar{X}_{t}^\leftarrow = {X}_{t}^\leftarrow = x\right] 
\lesssim  L_\varphi^2 [\varepsilon^2 \tau + \Lipdp^2 d(\tau^2 + \mathsf{m} \tau^3)]. %(\varepsilon^2 \tau + \tau^2),
\end{align}


\paragraph{Bound for $t = kh$:} 
First, we show a bound for $t = kh$. %$\delta =h$. 
The right hand side of \Eqref{eq:varPhiXYdiff} with $\delta = h$ can be bounded by 
\begin{align}
& \frac{\exp(2(1 + 2R)\delta)\|v\|^2}{\delta}
\E_{\bar{X}_{t}^\leftarrow}\left[ \E\left[ (\varphi_X( \bar{X}_{t^*}^\leftarrow) - \varphi_Y( {X}_{t^*}^\leftarrow))^2  ~|~ \bar{X}_{t}^\leftarrow = {X}_{t}^\leftarrow \right] \right]\\
\leq &
 \frac{\exp(2(1 + 2R)\delta)\|v\|^2}{\delta}
2 \E_{\bar{X}_{t}^\leftarrow}\left[\E\left[ (\varphi_X( \bar{X}_{t^*}^\leftarrow) - 
\varphi_Y( \bar{X}_{t^*}^\leftarrow) )^2
+ (\varphi_Y( \bar{X}_{t^*}^\leftarrow) 
- \varphi_Y( {X}_{t^*}^\leftarrow))^2  ~|~ \bar{X}_{t}^\leftarrow = {X}_{t}^\leftarrow \right] \right]\\
\leq &
 \frac{\exp(2(1 + 2R)\delta)\|v\|^2}{\delta}
2\left\{ C_\rho^2 \varepsilon_{\TV}^2 + R_\varphi^2
\E_{\bar{X}_{t}^\leftarrow}\left[\E\left[ 
(\bar{X}_{t^*}^\leftarrow  
- {X}_{t^*}^\leftarrow)^2  ~|~ \bar{X}_{t}^\leftarrow = {X}_{t}^\leftarrow \right] \right] \right\},
\end{align}
where we used 
\begin{align}
 \E\left[ (\varphi_X( \bar{X}_{t^*}^\leftarrow) - 
\varphi_Y( \bar{X}_{t^*}^\leftarrow) )^2 \right]
\leq & C_\rho^2\E_{\bar{X}_{T-t^*}}\left[\TV(\bar{X}_{T}^\leftarrow,{X}_{T}^\leftarrow | {X}_{t^*}^\leftarrow = \bar{X}_{t^*}^\leftarrow = \bar{X}_{T-t^*})^2 \right]  \\
\leq & C_\rho^2 \varepsilon_{\TV}.
\end{align}
Here, by using Theorem 10 of \cite{chen2023sampling} again, 
the right hand side can be bounded as  
\begin{align}
& 2 \frac{\exp(2(1 + 2R)\delta)\|v\|^2}{\delta}
\left\{ C_\rho^2\varepsilon_{\TV}^2 + C R_\varphi^2  [\varepsilon^2 \delta + \Lipdp^2 d(\delta^2 + \mathsf{m} \delta^3)]\right\} \\
= & 
2 \exp(2(1 + 2R)\delta)\|v\|^2
\left\{ C_\rho^2\frac{\varepsilon_{\TV}^2}{\delta} + C R_\varphi^2 [\varepsilon^2 + \Lipdp^2 d(\delta + \mathsf{m} \delta^2)]\right\}.
\end{align}
Then, with the contraint $h \leq 1/(1+2R)$, it can be further simplified as 
$$
2 \exp(2)\|v\|^2
\left\{ C_\rho^2\frac{\varepsilon_{\TV}^2}{\delta} + C R_\varphi^2 [\varepsilon^2 + \Lipdp^2 d(\delta + \mathsf{m} \delta^2)]\right\}.
$$
Therefore, by taking maximum with respect to $v \in \sR^d$ with a constraint $\|v\| =1$, 
\begin{align}
& \E_{ \bar{X}_{t}^\leftarrow }[ \|\nabla_x \E[\rho_*(\bar{X}_{T}^\leftarrow) | \bar{X}_{t}^\leftarrow ] - \nabla_x \E[\rho_*({X}_{T}^\leftarrow) | {X}_{t}^\leftarrow =  \bar{X}_{t}^\leftarrow ]\|^2 ] \\
\leq & \frac{4 c_\eta^2  C_\rho^2 (1+2R)^2}{3}\delta  
+ 2 \exp(2)
\left\{ C_\rho^2\frac{\varepsilon_{\TV}^2}{\delta} + C R_\varphi^2  [\varepsilon^2 + \Lipdp^2 d(\delta + \mathsf{m} \delta^2)]\right\} =: \Xi_{\delta,\varepsilon}.
\label{eq:DeltavphiXwithh}
\end{align}
We see that $\Xi_{\delta,\varepsilon} = \gO(\delta + \varepsilon^2 + \varepsilon_\TV^2/\delta)$. 

\paragraph{Bound for general $t \in (kh,(k+1)h)$:} 
In this setting, we utilize the inequality \eqref{eq:varPhiXYdiffSecond}. 
Using the constraint $\delta \leq 1/(1 + 2R)$ and $\|v\|=1$, the right hand side of \eqref{eq:varPhiXYdiffSecond} can be bounded by 
\begin{align}
&   \frac{\exp(2)}{\delta}
\E\left[ (\varphi_Y( \bar{X}_{t^*}^\leftarrow) - \varphi_Y( {X}_{t^*}^\leftarrow))^2  ~|~ \bar{X}_{t}^\leftarrow = {X}_{t}^\leftarrow = x\right] \\
& + 2 \E\left[ \exp(2) \|\nabla (\varphi_X( \bar{X}_{t^*}^\leftarrow)) - \varphi_Y( \bar{X}_{t^*}^\leftarrow))\|  ~|~ \bar{X}_{t}^\leftarrow = {X}_{t}^\leftarrow = x\right]^2 \\ 
\leq & 
\frac{\exp(2)}{\delta} 
R_\varphi^2 \E\left[ (\bar{X}_{t^*}^\leftarrow -  {X}_{t^*}^\leftarrow)^2  ~|~ \bar{X}_{t}^\leftarrow = {X}_{t}^\leftarrow = x\right]
+ 2 \exp(2) \Xi_{\delta,\varepsilon}~~~(\because \Eqref{eq:DeltavphiXwithh}). 
\end{align}
By taking the expectation with respect to $x = \bar{X}_{t}^\leftarrow$, we arrive at 
\begin{align} 
& \E_{\bar{X}_{t}^\leftarrow}[ \|\nabla_x \E[\rho_*(\bar{X}_{T}^\leftarrow) |  \bar{X}_{t}^\leftarrow] 
% |_{x = \bar{X}_{t}^\leftarrow}
-  \nabla_x \E[\rho_*({X}_{T}^\leftarrow) | {X}_{t}^\leftarrow] 
%|_{x = \bar{X}_{t}^\leftarrow}
\|^2] \\
\leq & C \frac{\exp(2)}{\delta} 
R_\varphi^2 \left(\varepsilon^2 \delta  + \Lipdp^2 d(\delta^2 + \mathsf{m} \delta^3) \right)
+ 2 \exp(2) \Xi_{\delta,\varepsilon} \\
\leq & 
C \exp(2)
R_\varphi^2 \left(\varepsilon^2+ \Lipdp^2 d(\delta + \mathsf{m} \delta^2) \right)
+ 2 \exp(2) \Xi_{\delta,\varepsilon}. 
    \label{eq:phyYboundwithphiX}
\end{align}
This gives an upper bound of the term (a) in \Eqref{eq:NablaRhotDecomp}. 
Then, we just need to bound the remaining term (b) in \Eqref{eq:NablaRhotDecomp}: 
\begin{align}
    \E_{\bar{X}_{t}^\leftarrow,\bar{X}_{kh}^\leftarrow}[ \| \nabla_x \E[\rho_*({X}_{T}^\leftarrow) | {X}_{t}^\leftarrow] 
    %|_{x = \bar{X}_{t}^\leftarrow} 
    -  \nabla_x \E[\rho_*({X}_{T}^\leftarrow) | {X}_{kh}^\leftarrow ] 
    %|_{x = \bar{X}_{kh}^\leftarrow}
    \|^2].
\end{align}
For that purpose, we define $\varphi_{Y,t}(x) = \E[\rho_*({X}_{T}^\leftarrow)| {X}_{t}^\leftarrow = x]$.
Then, using the Bismut-Elworthy-Li formula again, 
\begin{align}
&    v^\top (\nabla \varphi_{Y,t}(x) - \nabla \varphi_{Y,kh}(x)) \\ 
= &  
v^\top \nabla \varphi_{Y,t}(x) - 
\E[ \eta_{{X},h(k+1)-t}^\top \nabla \varphi_{Y,t}({X}_{t}^\leftarrow) |
{X}_{kh}^\leftarrow = x] \\
= &   
\E[ v^\top( \nabla \varphi_{Y,t}(x) - \nabla \varphi_{Y,t}({X}_{t}^\leftarrow))
+ 
(\eta_{{X},(k+1)h - t}^\top - v ) \nabla \varphi_{Y,t}({X}_{t}^\leftarrow) |
{X}_{kh}^\leftarrow = x] \\
\leq &   
\E[ L_\varphi \|x -{X}_{t}^\leftarrow \|
+ 
R_\varphi \|\eta_{{X},(k+1)h-t}^\top - v \|  \mid
{X}_{(k+1)h - t}^\leftarrow = x]  \\
\leq & 
\E[ L_\varphi \| ((k+1)h - t)(x - 2 s(x,kh)) + \sqrt{(h - \delta})B_{kh} \|
+ 
R_\varphi \|\eta_{{X},(k+1)h - t}^\top - v \|  \mid
{X}_{kh}^\leftarrow = x],
\end{align}
which yields that 
\begin{align}
&  \E_{\bar{X}_{\cdot}^\leftarrow}[\|\nabla \varphi_{Y,t}(\bar{X}_{t}^\leftarrow) - \nabla \varphi_{Y,kh}(\bar{X}_{(k+1)h - t}^\leftarrow) \|^2 ] \\
\leq 
& 2 L_\varphi^2((k+1)h - t)^2 \E_{\bar{X}_{\cdot}^\leftarrow}[ \|\bar{X}_{kh}^\leftarrow\|^2
+ 4\| s(\bar{X}_{kh}^\leftarrow,kh)\|^2 + d ((k+1)h - t)]  \\
&
+ R_\varphi^2 c_\eta (1 + 2R)^2 ((k+1)h - t)^2 \\
\leq 
& 
2 L_\varphi^2 h^2 (\mathsf{m} + 4Q^2 + d h)  
+ R_\varphi^2 c_\eta (1 + 2R)^2 h^2 \\
= & 
[2 L_\varphi^2 (\mathsf{m} + 4Q^2 + d h)  
+ R_\varphi^2 c_\eta (1 + 2R)^2] h^2. 
\label{eq:phyYdifftbound}
\end{align}
Combining \eqref{eq:phyYboundwithphiX} and \eqref{eq:phyYdifftbound} gives the assertion. 
\end{proof}



\begin{lem}\label{lemm:phiYboundLip}
Suppose that
$\sup_x\|\nabla \rho_*(x)\|\leq R_\rho$, $\|\nabla \rho_*(x) - \nabla \rho_*(y)\| \leq L_\rho\|x-y\|~(\forall x,y)$, and 
$\nabla_x s(\cdot,\cdot)$ is $H_s$-Lipschitz continuous with respect to $x$. 
Let $\varphi_{Y,t}(x) = \E[\rho_*({X}_{T}^\leftarrow)| {X}_{t}^\leftarrow = x]$. 
Then, $\nabla_x \varphi_{Y,t}(x)$ is bounded by $R_\varphi$ and $L_\varphi$-Lipschitz continuous for any $0 \leq t \leq T$, where  
\begin{align}
R_\varphi &=  \max\{C_\rho 2\sqrt{(1+2R) e},e^{1/2} R_\rho\}, 
%,~L_\varphi = \left(2 C_\eta^2 H_s^2 R^2 + e^2 L_\rho^2 \right)^{1/2}.
\\
L_\varphi &= \max\left\{\left( \frac{2 C_\eta^2 H_s^2 C_\rho^2}{1+2R} + 
2 (1 + 2R) \exp(6)R_\varphi^2 \right)^{1/2},
\left(2 C_\eta^2 H_s^2 R^2 + e^2 L_\rho^2 \right)^{1/2}
\right\}, 
\end{align}
for a universal constant $C_\eta > 0$. 
\end{lem}
\begin{proof}
We show it only when $t = kh$ for a positive integer $k$ just for simplicity. The proof for a general $t$ can be obtained in the same manner. 

(i) First, we assume that $T - t \geq 1/4(1 + 2R)$. 
In the following, we let $v \in \sR^d$ be an arbitrary vector with $\|v\| =1$. 
We again utilize the Bismut-Elworthy-Li formula: 
\begin{align} 
& v^\top \nabla \varphi_{Y,t}(x) \\
& = v^\top \nabla_x \E[\rho_*({X}_{T}^\leftarrow) \mid {X}_{t}^\leftarrow = x]   \\
& =  \E\left[ \frac{1}{S}\int_{0}^S \langle \eta_\tau, \rd B_\tau \rangle \varphi_{Y,S}({X}_{S}^\leftarrow) \mid {X}_{t}^\leftarrow = x\right].
\end{align}
Hence, 
\begin{align} 
& (v^\top \nabla \varphi_{Y,t}(x))^2 \\
\leq & 
 C_\rho^2 \E\left[ \frac{1}{S^2} \left( \int_{0}^S \langle \eta_\tau, \rd B_\tau \rangle \right)^2
 \mid {X}_{t}^\leftarrow = x\right]  \\
\leq & 
 C_\rho^2 \E\left[ \frac{1}{S^2} \int_{0}^S \|\eta_\tau\|^2 \rd \tau 
 \mid {X}_{t}^\leftarrow = x\right].
\end{align}
Here, we know that $\|\eta_\tau\|^2 \leq \exp(4(1+2R)\tau) \|v\|^2$, and thus 
\begin{align} 
& (v^\top \nabla \varphi_{Y,t}(x))^2 \leq C_\rho^2 \frac{1}{S} \exp(4(1+2R)S) \|v\|^2.
\end{align}
Hence, by taking $S = \frac{1}{4(1+2R)}$, we have that 
$$
(v^\top \nabla \varphi_{Y,t}(x))^2 \leq C_\rho^2 4(1+2R) e \|v\|^2. 
$$
This shows that $\|\nabla \varphi_{Y,t}(x)\|$ is bounded by $R_\varphi =  C_\rho 2\sqrt{(1+2R) e}$. 

Next, we show its Lipschitz continuity. 
For that purpose, we define two stochastic processes 
\begin{align}
{X}_{t}^\leftarrow=x,\ \mathrm{d}{X}_{\tau}^\leftarrow = \{{X}_{\tau}^\leftarrow + 2s({X}_{kh}^\leftarrow,kh)\}\mathrm{d}\tau + \sqrt{2}\mathrm{d}B_\tau~~(\tau \in [kh,k(h+1)]), \\
\tilde{Z}_{t}^\leftarrow=y,\ \mathrm{d}\tilde{Z}_{\tau}^\leftarrow = \{\tilde{Z}_{\tau}^\leftarrow + 2s(\tilde{Z}_{kh}^\leftarrow,kh)\}\mathrm{d}\tau + \sqrt{2}\mathrm{d}B_\tau~~(\tau \in [kh,k(h+1)]),
\end{align}
where $x,y \in \sR^d$ with $\|x - y\| \leq \varepsilon$.
Accordingly, we also define 
\begin{align}
& \eta_{{X},0} = v,~~\frac{\rd \eta_{{X},\tau}}{\rd \tau}  = (I + 2 \nabla_x^\top s({X}_{kh}^\leftarrow,kh)) \eta_{{X},\tau},  \\
&  \eta_{Z,0} = v,~~\frac{\rd \eta_{Z,\tau}}{\rd \tau}  = (I + 2 \nabla_x^\top s(\tilde{Z}_{kh}^\leftarrow,kh)) \eta_{Z,\tau}.
\end{align} 
Thus, 
\begin{align}
{X}_{(k+1)h}^\leftarrow - \tilde{Z}_{(k+1)h}^\leftarrow
= 
{X}_{kh}^\leftarrow - \tilde{Z}_{k h}^\leftarrow
+ h [{X}_{kh}^\leftarrow - \tilde{Z}_{k h}^\leftarrow + 2(s({X}_{kh}^\leftarrow,kh) - s(\tilde{Z}_{kh}^\leftarrow,kh))],
\end{align}
which yields 
\begin{align}
\|{X}_{(k+1)h}^\leftarrow - \tilde{Z}_{(k+1)h}^\leftarrow\| 
\leq & 
(1 + h(1 + R)) \|{X}_{kh}^\leftarrow - \tilde{Z}_{k h}^\leftarrow\| \\
\leq & 
(1 + h(1 + R))^{k+1} \|x - y\|.
\end{align}
Now, we assume $k \leq S/h$ so that we have 
$\|{X}_{(k+1)h}^\leftarrow - \tilde{Z}_{(k+1)h}^\leftarrow\| 
\leq \exp(S(1+R)) \|x - y\|$ for $k=1,\dots,S/h$. 
Hence, 
\begin{align}
& \frac{\rd (\eta_{{X},\tau} - \eta_{Z,\tau})}{\rd \tau}  \\
=& ( \eta_{{X},\tau} -  \eta_{Z,\tau}) + 
2 \nabla_x^\top s({X}_{kh}^\leftarrow,kh)) \eta_{{X},\tau} - 2 \nabla_x^\top s(\tilde{Z}_{kh}^\leftarrow,kh)) \eta_{Z,\tau} \\
=& ( \eta_{{X},\tau} -  \eta_{Z,\tau}) + 
2 (\nabla_x^\top s({X}_{kh}^\leftarrow,kh) - \nabla_x^\top s(\tilde{Z}_{kh}^\leftarrow,kh)) 
\eta_{{X},\tau} - 
2 \nabla_x^\top s(\tilde{Z}_{kh}^\leftarrow,kh)(\eta_{Z,\tau} -  \eta_{{X},\tau}),
\end{align} 
which also yields that 
\begin{align}
& \frac{\rd \|\eta_{{X},\tau} - \eta_{Z,\tau}\|^2}{\rd \tau}  \\
=& 2 \| \eta_{{X},\tau} -  \eta_{Z,\tau}\|^2 + 
4 (\eta_{{X},\tau} -  \eta_{Z,\tau})^\top (\nabla_x^\top s({X}_{kh}^\leftarrow,kh) - \nabla_x^\top s(\tilde{Z}_{kh}^\leftarrow,kh)) 
\eta_{{X},\tau} \\
& - 
4 (\eta_{{X},\tau} -  \eta_{Z,\tau})^\top \nabla_x^\top s(\tilde{Z}_{kh}^\leftarrow,kh)(\eta_{Z,\tau} -  \eta_{{X},\tau}) \\
\leq & 2 \| \eta_{{X},\tau} -  \eta_{Z,\tau}\|^2 + 
4 \|\eta_{{X},\tau} -  \eta_{Z,\tau}\| H_s \exp(S(1+R)) \varepsilon  
\exp(2(1+2R)S) \\
& +  
4 R \|\eta_{{X},\tau} -  \eta_{Z,\tau}\|^2. 
\end{align} 
Therefore, 
\begin{align}
& \frac{\rd \|\eta_{{X},\tau} - \eta_{Z,\tau}\|}{\rd \tau}  \\
\leq & (1 + 2R) \left[ \| \eta_{{X},\tau} -  \eta_{Z,\tau}\| + 
2 H_s \exp(S(1+R))   
\exp(2(1+2R)S) \varepsilon / (1 + 2R) \right],
\end{align}
and thus by noticing $\|\eta_{{X},0} - \eta_{Z,0}\| = 0$, we have 
\begin{align}
\|\eta_{{X},\tau} - \eta_{Z,\tau}\| \leq 
[\exp(S (1 + 2R)) -1] \frac{2 H_s\exp(S(1+R))   
\exp(2(1+2R)S)  }{1 + 2R} \varepsilon,
\end{align}
for any $\tau \leq S$. Then, by setting $S = 1/(1+2R)$, the right hand side can be rewritten as 
\begin{align}
\|\eta_{{X},\tau} - \eta_{Z,\tau}\| \leq 
C_\eta  \frac{H_s}{1 + 2R} \varepsilon,
\end{align}
for an absolute constant $C_\eta$. 
Therefore, we arrive at 
\begin{align} 
& (v^\top (\nabla \varphi_{Y,t}(x) - \nabla \varphi_{Y,t}(y)))^2 \\
=  & \E\left[ \frac{1}{S}\int_{0}^S \langle \eta_{{X},\tau}, \rd B_\tau \rangle \varphi_{Y,S}({X}_{S}^\leftarrow) 
-
\frac{1}{S}\int_{0}^S \langle \eta_{Z,\tau}, \rd B_\tau \rangle \varphi_{Y,S}(\tilde{Z}_{S}^\leftarrow)
\right]^2 \\
\leq & 
  2 \E\left[ \frac{1}{S^2} \int_{0}^S (\eta_{{X},\tau} - \eta_{Z,\tau})^2 \rd \tau \right] 
  \E\left[\varphi_{Y,S}({X}_{S}^\leftarrow)^2\right]  \\
&   + 
   2  \E\left[ \frac{1}{S^2} \int_{0}^S \eta_{Z,\tau}^2 \rd \tau \right] 
  \E\left[(\varphi_{Y,S}({X}_{S}^\leftarrow) - \varphi_{Y,S}(\tilde{Z}_{S}^\leftarrow))^2\right] \\
\leq & 
\frac{2}{S} C_\eta^2  \frac{H_s^2}{(1 + 2R)^2} \varepsilon^2 \cdot C_\rho^2 
 + \frac{2}{S}  
 \exp(4(1+2R)S) \|v\|^2 
  R_\varphi^2 \exp(2 S(1+R)) \varepsilon^2.
\end{align}
Then, for the choice of $S = 1/(1+2R)$, the right hand side can be bounded by 
$$
\left( \frac{2 C_\eta^2 H_s^2 C_\rho^2}{1+2R} + 
2 (1 + 2R) \exp(6) R_\varphi^2 \right) 
\varepsilon^2. 
$$
This implies that $\nabla \varphi_{Y,t}(\cdot)$ is Lipschitz continuous with a constant $L_\varphi = \left( \frac{2 C_\eta^2 H_s^2 C_\rho^2}{1+2R} + 
2 (1 + 2R) \exp(6)R_\varphi^2 \right)^{1/2}$. 

(ii) Next, we assume that $T - t \leq S = 1/4(1 + 2R)$. 
In this situation, we may use the following relation: 
\begin{align} 
& v^\top \nabla \varphi_{Y,t}(x)  = \E[\eta_\tau^{T-t} \nabla \rho_*({X}_{T}^\leftarrow) \mid {X}_{t}^\leftarrow = x]. 
\end{align}
And, tracing an analogous argument by replacing $\varphi_{Y,S}$ with $h$, we obtain the assertion with 
$$
R_\varphi = e^{1/2} R_\rho,~L_\varphi = \left(2 C_\eta^2 H_s^2 R^2 + e^2 L_\rho^2 \right)^{1/2}.
$$
\end{proof}


\begin{lem}\label{lem:hhdashDiff}
If $\|\rho_* - \rho\|_\infty \leq \varepsilon'$, then % and $\|\nabla h- \nabla h'\|_\infty \leq  \varepsilon'$
%\footnote{The condition $\|\nabla h- \nabla h'\|_\infty \leq  \varepsilon'$ can be omitted. However we will have a bound of $\frac{e}{\sqrt{\min\{T-t,1/(2+2R)\}}}\varepsilon'$ instead}, then  
 $$\|\nabla_x \E[\rho_*({X}_{T}^\leftarrow) | {X}_{t}^\leftarrow = x] - 
 \nabla_x \E[\rho({X}_{T}^\leftarrow)  | {X}_{t}^\leftarrow = x] \| \leq 
 \frac{e}{\sqrt{\min\{T-t,1/(2+2R)\}}} \varepsilon'.$$   
\end{lem}
\begin{proof}
It can be proved by the Bismut-Elworthy-Li formula again. We omit the details. 
\end{proof}

% Noticing that ${X}_{(k+1)h} = h ({X}_{kh} - 2 s({X}_{kh},kh)) + \sqrt{2} \xi_{k}$ where $\xi_k \sim N(0,I)$, 
% %the Bismut-Elworthy-Li formula yields   
% a simple calculation yields that 
% $$\nabla_x \E[\rho({X}_{T}^\leftarrow) | {X}_{t}^\leftarrow = x]
% = \E_{\xi_k}\left[ \varphi'_{Y,(k+1)h}({X}_{kh}^\leftarrow - 2 s({X}_{kh}^\leftarrow,kh) + \sqrt{2} \xi_{k}) 
% \frac{(I - 2 \nabla^\top s(x,kh)|_{x = {X}_{kh}^\leftarrow} )}{\sqrt{e^{2h}-1}}  \xi_{k}
% \right],
% $$
% where $\varphi'_{Y,t}(y) = \E[\rho({X}_{T}^{\leftarrow}) | {X}_{t}^{\leftarrow} = y]$. 
% We notice that the empirical estimate of $\nabla_x \E[\rho_*({X}_{T}^\leftarrow) | {X}_{t}^\leftarrow = x]$ is obtained by replacing the expectation by a sample average. 
% Then, since $\|I - 2 \nabla^\top s(x,kh)|_{x = {X}_{kh}^\leftarrow}\|_{\mathrm{op}} \leq (1 + 2R)$ and $\|\rho\|_\infty \leq C_\rho$, we see that,
% for an i.i.d. sequence $({X}_i)_{i=1}^n$ obtained by simulating ${X}_T^{\leftarrow}$, the sub-Gaussian concentration inequality gives that 
% \begin{align}
% & \nabla_x \E[\rho({X}_{T}^\leftarrow) | {X}_{kh}^\leftarrow = x]
% - 
% \frac{1}{n} \sum_{i=1}^n \left[ 
% h'({X}_{i}) 
% \frac{(I - 2 \nabla^\top s(x,kh)|_{x = {X}_{kh}^\leftarrow} )}{\sqrt{e^{2h}-1}}  \xi_{k,i} 
% \right] \\ 
% & \leq C C_\rho (1 + 2R) \sqrt{\frac{\log(\delta)}{n (e^{2h} -1)}}
% \lesssim   C_\rho (1 + 2R) \sqrt{\frac{\log(\delta^{-1})}{n h}},  
% \label{eq:EmpiricalhDiff}
% \end{align}
% with probability $1- \delta$,
% where $\xi_{k,i}$ is the standard normal random variable to generate a sample of ${X}_{(k+1)h}^{\leftarrow}$ and $C$ is a universal constant. 
% This bound can be uniform over all $k=1,\dots,T/h$ by replacing $\delta$ with $\delta/(T/h)$.



Combining all inequalities, we arrive at (the formal version of) Theorem~\ref{thm:Diffusion-2}. 
\begin{thm}[Formal statement of Theorem~\ref{thm:Diffusion-2}] \label{thm:ustaruDiffFinal}
Assume that Assumptions \ref{ass:BoundingH} and \ref{assumption:TVBoundMainText-2} hold 
and the conditions in Lemma \ref{lemm:phiYboundLip} are satisfied. 
%the same condition as Theorem \ref{thm:delhDiffXYExp} holds 
and $\|\rho_* - \rho\|_\infty \leq \varepsilon'$ and $\|\rho\|_\infty \leq C_\rho$. 
Let $L_\varphi$ and $R_\varphi$ be as given in Lemma \ref{lemm:phiYboundLip}.
Then, for $0 \leq h \leq \delta \leq 1/(1 + 2R)$, we have that 
\begin{align}
& \E_{\bar{Y}_{\cdot}^{\leftarrow}}[ \|u_*(\bar{Y}_{t}^\leftarrow,t) -  u(\bar{Y}_{k_t h}^\leftarrow,t) \|^2]  \\
\lesssim & 
C_\rho^3 \left\{ R_\varphi^2 \left(\varepsilon^2+ \Lipdp^2 d(\delta + \mathsf{m} \delta^2) \right)
+ \Xi_{\delta,\varepsilon}
+
[ (R_\varphi^2 + L_\varphi^2) (\mathsf{m} + 4Q^2 + d h)  
+ R_\varphi^2  (1 + 2R)^2] h^2\right\} \\
& +  \frac{e^2}{\min\{T-t,1/(2+2R)\}} \varepsilon'^2 +  C_\rho (1 + 2R) \sqrt{\frac{\log(T/(h\delta))}{n h}}, 
%= & \gO\left(\varepsilon^2 + \delta + \frac{\varepsilon_\TV^2}{\delta}\right), 
\end{align}
where 
\begin{align}
\Xi_{\delta,\varepsilon} & := \frac{4 c_\eta^2  C_\rho^2 (1+2R)^2}{3}\delta  
+ 2 \exp(2)
\left\{ C_\rho^2 \left(R_\varphi^2 + \frac{1}{\delta}\right)\varepsilon_{\TV}^2 + C R_\varphi^2  [\varepsilon^2 + \Lipdp^2 d(\delta + \mathsf{m} \delta^2)]\right\},
%&  = \gO\left(\delta + \varepsilon^2 + \frac{\varepsilon_\TV^2}{\delta}\right),
\end{align} 
and $c_\eta > 0$ is a universal constant. 
\end{thm}

\begin{proof}
Define 
$$
\rho_{*,t}(x) = \E[\rho_*(\bar{X}_{T}^\leftarrow) |  \bar{X}_{t}^\leftarrow = x],~\rho_{t}(x) = \E[\rho(X_{T}^\leftarrow) |  {X}_{t}^\leftarrow = x].
$$
First, note that 
\begin{align}
\| u^*(x,t) - u(x,t) \|^2= & 
\left\| \frac{ \nabla \rho_{*,t}(x) - \nabla \rho_{t}(x)
}{ \rho_{*,t}(x) } + \frac{\nabla \rho_{t}(x) (\rho_{*,t}(x) -\rho_{t}(x))
}{ \rho_{*,t}(x) \rho_t(x) } \right\|^2 \\
\leq & 
2 \left\| \frac{ \nabla \rho_{*,t}(x) - \nabla \rho_{t}(x)
}{ \rho_{*,t}(x) }\right\|^2  +
2 \left\| \frac{\nabla \rho_{t}(x) (\rho_{*,t}(x) -\rho_{t}(x))
}{ \rho_{*,t}(x) \rho_t(x) } \right\|^2 \\ 
\leq & 2 C_\rho^2 \left\| \nabla \rho_{*,t}(x) - \nabla \rho_{t}(x)\right\|^2  +
 2 \frac{\left\| \nabla \rho_{t}(x) \right\|^2 |\rho_{*,t}(x) -\rho_{t}(x)|^2 }{ (\rho_{*,t}(x) \rho_t(x))^2 }  \\
\leq & 2 C_\rho^2 \left\| \nabla \rho_{*,t}(x) - \nabla \rho_{t}(x)\right\|^2  +
 2 R_\varphi^2 C_\rho^2 |\rho_{*,t}(x) -\rho_{t}(x)|^2. 
\end{align}
Therefore, the expectation of the right hand side with respect to $\bar{X}_t^\leftarrow$ can be bounded by 
\begin{align}
& \E_{\bar{X}_t^\leftarrow} \left[ \| u^*(\bar{X}_t^\leftarrow,t) - u(\bar{X}_t^\leftarrow,t) \|^2\right]  \\ 
\leq &
2C_\rho^2 \E_{\bar{X}_t^\leftarrow} \left[  \left\| \nabla \rho_{*,t}(x) - \nabla \rho_{t}(x) \right\|^2 \right]
+ 2 R_\varphi C_\rho^2  \E_{\bar{X}_t^\leftarrow}\left[ |\rho_{*,t}(x) -\rho_{t}(x)|^2 \right] \\
\leq &
2C_\rho^2 \E_{\bar{X}_t^\leftarrow} \left[  \left\| \nabla \rho_{*,t}(x) - \nabla \rho_{t}(x) \right\|^2 \right]
+ 2 R_\varphi C_\rho^2  
\E_{\bar{X}_t^\leftarrow} \left[  \TV(\bar{X}_T^\leftarrow,X_T^\leftarrow | \bar{X}_t^\leftarrow = X_t^\leftarrow =x) |_{x  = \bar{X}_t^\leftarrow}^2  \right].
\end{align}
The first term of the right hand side can be bounded by Theorem \ref{thm:delhDiffXYExp} and Lemma \ref{lem:hhdashDiff}.  
The second term can be bounded by $\varepsilon_\TV^2$ by Assumption \ref{ass:BoundingH}. 

In the same vein, we can bound the difference 
\begin{align}
&  \| u(x,t) - u(x,k_th) \|^2 \\
\leq &
2C_\rho^2 \E_{\bar{X}_t^\leftarrow} \left[  \left\| \nabla \rho_{t}(x) - \nabla \rho_{k_t h}(x) \right\|^2 \right]
+ 2 R_\varphi C_\rho^2  
\E_{\bar{X}_t^\leftarrow} \left[ |\rho_{t}(x) - \rho_{k_t h}(x)|  \right] \\
\leq &
[2 L_\varphi^2 (\mathsf{m} + 4Q^2 + d h)  
+ R_\varphi^2 c_\eta (1 + 2R)^2] h^2 \\
& +  2 R_\varphi^2 h^2 (\mathsf{m} + 4Q^2 + d h)   \\
\leq & 
[2 (L_\varphi^2+R_\varphi^2) (\mathsf{m} + 4Q^2 + d h)  
+ R_\varphi^2 c_\eta (1 + 2R)^2] h^2, 
\end{align}
where we used \eqref{eq:phyYdifftbound} and the same argument as \eqref{eq:phyYdifftbound} in the last inequality with $R_\varphi$ Lipschitz continuity of $\rho_{t}$. 


Finally, we convert the expectation w.r.t. $\bar{X_t}^\leftarrow$ to that w.r.t. $\bar{X_t}^\leftarrow$. However, the density ratio between $p_t$ and $q_t$ is bounded by $C_\rho$,
which yields the assertion. 
%The proof can be accomplished by combining Theorem \ref{thm:delhDiffXYExp}, Lemma \ref{lem:hhdashDiff} and \Eqref{eq:EmpiricalhDiff}. 
%Only thing we should notice is that the density ratio between $q_t$ and $p_t$ is bounded by $C_\rho$, which is required to transform the expectation w.r.t $\bar{X}_t$ to $\bar{Y}_t$. 
\end{proof}

\subsection{Doob's H-transform}
\begin{lem}\label{lem:H-transform}
    % Eq.~\ref{eq:Appendix-Diffusion-Preliminary-4} holds.
    For all $t\in [0,T]$, the following relationship holds.
    \begin{align}
    \nabla_x \log q_{t}(x) = \nabla_x \log p_{t}(x) + \nabla_x \log (\mathbb{E}[\rho_*(\bar{X}_0)|\bar{X}_{t}=x]).
    \label{eq:Appendix-Diffusion-Doob-1}
    \end{align}
\end{lem}
\begin{proof}
    Let us denote the joint distribution of $\bar{X}_0$ and $\bar{X}_t$ as $p_{0,t}(\bar{X}_0,\bar{X}_t)$, the conditional distributions of $\bar{X}_0$ given $\bar{X}_t$ as $p_{0|t}(\bar{X}_0|\bar{X}_t)$, and the conditional distributions of $\bar{X}_t$ given $\bar{X}_0$ as $p_{t|0}(\bar{X}_t|\bar{X}_0)$. Define $q_{t|0}$ in the same way. 
    It is straightforward to see that %verify (\ref{eq:Appendix-Diffusion-Doob-1}): 
    \begin{align}
        \log p_{t}(x) + \log (\mathbb{E}[\rho_*(\bar{X}_0)|\bar{X}_{t}=x])
        &= \log (p_{t}(x)\mathbb{E}[\rho_*(\bar{X}_0)|\bar{X}_{t}=x])
        \\ & = \log p_{t}(x)\int_{x'} \rho_*(x')p_{0|t}(x'|x)\mathrm{d}x'
        \\ & = \log \int_{x'} \rho_*(x')p_{0,t}(x',x)\mathrm{d}x'
        \\ & = \log \int_{x'} \rho_*(x')p_0(x')p_{t|0}(x|x')\mathrm{d}x'
        \\ & = \log \int_{x'} q_0(x')p_{t|0}(x|x')\mathrm{d}x'.
        \label{eq:Appendix-Diffusion-Doob-2}
    \end{align}
    Note that $p_{t|0}(x|x')$ and $q_{t|0}(x|x')$ are the same in (\ref{eq:Appendix-Diffusion-Doob-2}). Therefore,
    \begin{align}
        \log p_{t}(x) + \log (\mathbb{E}[\rho_*(\bar{X}_0)|\bar{X}_{t}=x]|)
       & = \log \int_{x'} q_0(x')q_{t|0}(x|x')\mathrm{d}x'
        \\ & = \log q_t(x), 
    \end{align}
    which concludes the proof.
\end{proof}




\section{Details of Numerical Experiments}
\label{section:appendix-experiments}
\section{Experiments}
\label{sec:exp}
Following the settings in Section \ref{sec:existing}, we evaluate \textit{NovelSum}'s correlation with the fine-tuned model performance across 53 IT datasets and compare it with previous diversity metrics. Additionally, we conduct a correlation analysis using Qwen-2.5-7B \cite{yang2024qwen2} as the backbone model, alongside previous LLaMA-3-8B experiments, to further demonstrate the metric's effectiveness across different scenarios. Qwen is used for both instruction tuning and deriving semantic embeddings. Due to resource constraints, we run each strategy on Qwen for two rounds, resulting in 25 datasets. 

\subsection{Main Results}

\begin{table*}[!t]
    \centering
    \resizebox{\linewidth}{!}{
    \begin{tabular}{lcccccccccc}
    \toprule
    \multirow{3}*{\textbf{Diversity Metrics}} & \multicolumn{10}{c}{\textbf{Data Selection Strategies}} \\
    \cmidrule(lr){2-11}
    & \multirow{2}*{\textbf{K-means}} & \multirow{2}*{\vtop{\hbox{\textbf{K-Center}}\vspace{1mm}\hbox{\textbf{-Greedy}}}}  & \multirow{2}*{\textbf{QDIT}} & \multirow{2}*{\vtop{\hbox{\textbf{Repr}}\vspace{1mm}\hbox{\textbf{Filter}}}} & \multicolumn{5}{c}{\textbf{Random}} & \multirow{2}{*}{\textbf{Duplicate}} \\ 
    \cmidrule(lr){6-10}
    & & & & & \textbf{$\mathcal{X}^{all}$} & ShareGPT & WizardLM & Alpaca & Dolly &  \\
    \midrule
    \rowcolor{gray!15} \multicolumn{11}{c}{\textit{LLaMA-3-8B}} \\
    Facility Loc. $_{\times10^5}$ & \cellcolor{BLUE!40} 2.99 & \cellcolor{ORANGE!10} 2.73 & \cellcolor{BLUE!40} 2.99 & \cellcolor{BLUE!20} 2.86 & \cellcolor{BLUE!40} 2.99 & \cellcolor{BLUE!0} 2.83 & \cellcolor{BLUE!30} 2.88 & \cellcolor{BLUE!0} 2.83 & \cellcolor{ORANGE!20} 2.59 & \cellcolor{ORANGE!30} 2.52 \\    
    DistSum$_{cosine}$  & \cellcolor{BLUE!30} 0.648 & \cellcolor{BLUE!60} 0.746 & \cellcolor{BLUE!0} 0.629 & \cellcolor{BLUE!50} 0.703 & \cellcolor{BLUE!10} 0.634 & \cellcolor{BLUE!40} 0.656 & \cellcolor{ORANGE!30} 0.578 & \cellcolor{ORANGE!10} 0.605 & \cellcolor{ORANGE!20} 0.603 & \cellcolor{BLUE!10} 0.634 \\
    Vendi Score $_{\times10^7}$ & \cellcolor{BLUE!30} 1.70 & \cellcolor{BLUE!60} 2.53 & \cellcolor{BLUE!10} 1.59 & \cellcolor{BLUE!50} 2.23 & \cellcolor{BLUE!20} 1.61 & \cellcolor{BLUE!30} 1.70 & \cellcolor{ORANGE!10} 1.44 & \cellcolor{ORANGE!20} 1.32 & \cellcolor{ORANGE!10} 1.44 & \cellcolor{ORANGE!30} 0.05 \\
    \textbf{NovelSum (Ours)} & \cellcolor{BLUE!60} 0.693 & \cellcolor{BLUE!50} 0.687 & \cellcolor{BLUE!30} 0.673 & \cellcolor{BLUE!20} 0.671 & \cellcolor{BLUE!40} 0.675 & \cellcolor{BLUE!10} 0.628 & \cellcolor{BLUE!0} 0.591 & \cellcolor{ORANGE!10} 0.572 & \cellcolor{ORANGE!20} 0.50 & \cellcolor{ORANGE!30} 0.461 \\
    \midrule    
    \textbf{Model Performance} & \cellcolor{BLUE!60}1.32 & \cellcolor{BLUE!50}1.31 & \cellcolor{BLUE!40}1.25 & \cellcolor{BLUE!30}1.05 & \cellcolor{BLUE!20}1.20 & \cellcolor{BLUE!10}0.83 & \cellcolor{BLUE!0}0.72 & \cellcolor{ORANGE!10}0.07 & \cellcolor{ORANGE!20}-0.14 & \cellcolor{ORANGE!30}-1.35 \\
    \midrule
    \midrule
    \rowcolor{gray!15} \multicolumn{11}{c}{\textit{Qwen-2.5-7B}} \\
    Facility Loc. $_{\times10^5}$ & \cellcolor{BLUE!40} 3.54 & \cellcolor{ORANGE!30} 3.42 & \cellcolor{BLUE!40} 3.54 & \cellcolor{ORANGE!20} 3.46 & \cellcolor{BLUE!40} 3.54 & \cellcolor{BLUE!30} 3.51 & \cellcolor{BLUE!10} 3.50 & \cellcolor{BLUE!10} 3.50 & \cellcolor{ORANGE!20} 3.46 & \cellcolor{BLUE!0} 3.48 \\ 
    DistSum$_{cosine}$ & \cellcolor{BLUE!30} 0.260 & \cellcolor{BLUE!60} 0.440 & \cellcolor{BLUE!0} 0.223 & \cellcolor{BLUE!50} 0.421 & \cellcolor{BLUE!10} 0.230 & \cellcolor{BLUE!40} 0.285 & \cellcolor{ORANGE!20} 0.211 & \cellcolor{ORANGE!30} 0.189 & \cellcolor{ORANGE!10} 0.221 & \cellcolor{BLUE!20} 0.243 \\
    Vendi Score $_{\times10^6}$ & \cellcolor{ORANGE!10} 1.60 & \cellcolor{BLUE!40} 3.09 & \cellcolor{BLUE!10} 2.60 & \cellcolor{BLUE!60} 7.15 & \cellcolor{ORANGE!20} 1.41 & \cellcolor{BLUE!50} 3.36 & \cellcolor{BLUE!20} 2.65 & \cellcolor{BLUE!0} 1.89 & \cellcolor{BLUE!30} 3.04 & \cellcolor{ORANGE!30} 0.20 \\
    \textbf{NovelSum (Ours)}  & \cellcolor{BLUE!40} 0.440 & \cellcolor{BLUE!60} 0.505 & \cellcolor{BLUE!20} 0.403 & \cellcolor{BLUE!50} 0.495 & \cellcolor{BLUE!30} 0.408 & \cellcolor{BLUE!10} 0.392 & \cellcolor{BLUE!0} 0.349 & \cellcolor{ORANGE!10} 0.336 & \cellcolor{ORANGE!20} 0.320 & \cellcolor{ORANGE!30} 0.309 \\
    \midrule
    \textbf{Model Performance} & \cellcolor{BLUE!30} 1.06 & \cellcolor{BLUE!60} 1.45 & \cellcolor{BLUE!40} 1.23 & \cellcolor{BLUE!50} 1.35 & \cellcolor{BLUE!20} 0.87 & \cellcolor{BLUE!10} 0.07 & \cellcolor{BLUE!0} -0.08 & \cellcolor{ORANGE!10} -0.38 & \cellcolor{ORANGE!30} -0.49 & \cellcolor{ORANGE!20} -0.43 \\
    \bottomrule
    \end{tabular}
    }
    \caption{Measuring the diversity of datasets selected by different strategies using \textit{NovelSum} and baseline metrics. Fine-tuned model performances (Eq. \ref{eq:perf}), based on MT-bench and AlpacaEval, are also included for cross reference. Darker \colorbox{BLUE!60}{blue} shades indicate higher values for each metric, while darker \colorbox{ORANGE!30}{orange} shades indicate lower values. While data selection strategies vary in performance on LLaMA-3-8B and Qwen-2.5-7B, \textit{NovelSum} consistently shows a stronger correlation with model performance than other metrics. More results are provided in Appendix \ref{app:results}.}
    \label{tbl:main}
    \vspace{-4mm}
\end{table*}


\begin{table}[t!]
\centering
\resizebox{\linewidth}{!}{
\begin{tabular}{lcccc}
\toprule
\multirow{2}*{\textbf{Diversity Metrics}} & \multicolumn{3}{c}{\textbf{LLaMA}} & \textbf{Qwen}\\
\cmidrule(lr){2-4} \cmidrule(lr){5-5} 
& \textbf{Pearson} & \textbf{Spearman} & \textbf{Avg.} & \textbf{Avg.} \\
\midrule
TTR & -0.38 & -0.16 & -0.27 & -0.30 \\
vocd-D & -0.43 & -0.17 & -0.30 & -0.31 \\
\midrule
Facility Loc. & 0.86 & 0.69 & 0.77 & 0.08 \\
Entropy & 0.93 & 0.80 & 0.86 & 0.63 \\
\midrule
LDD & 0.61 & 0.75 & 0.68 & 0.60 \\
KNN Distance & 0.59 & 0.80 & 0.70 & 0.67 \\
DistSum$_{cosine}$ & 0.85 & 0.67 & 0.76 & 0.51 \\
Vendi Score & 0.70 & 0.85 & 0.78 & 0.60 \\
DistSum$_{L2}$ & 0.86 & 0.76 & 0.81 & 0.51 \\
Cluster Inertia & 0.81 & 0.85 & 0.83 & 0.76 \\
Radius & 0.87 & 0.81 & 0.84 & 0.48 \\
\midrule
NovelSum & \textbf{0.98} & \textbf{0.95} & \textbf{0.97} & \textbf{0.90} \\
\bottomrule
\end{tabular}
}
\caption{Correlations between different metrics and model performance on LLaMA-3-8B and Qwen-2.5-7B.  “Avg.” denotes the average correlation (Eq. \ref{eq:cor}).}
\label{tbl:correlations}
\vspace{-2mm}
\end{table}

\paragraph{\textit{NovelSum} consistently achieves state-of-the-art correlation with model performance across various data selection strategies, backbone LLMs, and correlation measures.}
Table \ref{tbl:main} presents diversity measurement results on datasets constructed by mainstream data selection methods (based on $\mathcal{X}^{all}$), random selection from various sources, and duplicated samples (with only $m=100$ unique samples). 
Results from multiple runs are averaged for each strategy.
Although these strategies yield varying performance rankings across base models, \textit{NovelSum} consistently tracks changes in IT performance by accurately measuring dataset diversity. For instance, K-means achieves the best performance on LLaMA with the highest NovelSum score, while K-Center-Greedy excels on Qwen, also correlating with the highest NovelSum. Table \ref{tbl:correlations} shows the correlation coefficients between various metrics and model performance for both LLaMA and Qwen experiments, where \textit{NovelSum} achieves state-of-the-art correlation across different models and measures.

\paragraph{\textit{NovelSum} can provide valuable guidance for data engineering practices.}
As a reliable indicator of data diversity, \textit{NovelSum} can assess diversity at both the dataset and sample levels, directly guiding data selection and construction decisions. For example, Table \ref{tbl:main} shows that the combined data source $\mathcal{X}^{all}$ is a better choice for sampling diverse IT data than other sources. Moreover, \textit{NovelSum} can offer insights through comparative analyses, such as: (1) ShareGPT, which collects data from real internet users, exhibits greater diversity than Dolly, which relies on company employees, suggesting that IT samples from diverse sources enhance dataset diversity \cite{wang2024diversity-logD}; (2) In LLaMA experiments, random selection can outperform some mainstream strategies, aligning with prior work \cite{xia2024rethinking,diddee2024chasing}, highlighting gaps in current data selection methods for optimizing diversity.



\subsection{Ablation Study}


\textit{NovelSum} involves several flexible hyperparameters and variations. In our main experiments, \textit{NovelSum} uses cosine distance to compute $d(x_i, x_j)$ in Eq. \ref{eq:dad}. We set $\alpha = 1$, $\beta = 0.5$, and $K = 10$ nearest neighbors in Eq. \ref{eq:pws} and \ref{eq:dad}. Here, we conduct an ablation study to investigate the impact of these settings based on LLaMA-3-8B.

\begin{table}[ht!]
\centering
\resizebox{\linewidth}{!}{
\begin{tabular}{lccc}
\toprule
\textbf{Variants} & \textbf{Pearson} & \textbf{Spearman} & \textbf{Avg.} \\
\midrule
NovelSum & 0.98 & 0.96 & 0.97 \\
\midrule
\hspace{0.10cm} - Use $L2$ distance & 0.97 & 0.83 & 0.90\textsubscript{↓ 0.08} \\
\hspace{0.10cm} - $K=20$ & 0.98 & 0.96 & 0.97\textsubscript{↓ 0.00} \\
\hspace{0.10cm} - $\alpha=0$ (w/o proximity) & 0.79 & 0.31 & 0.55\textsubscript{↓ 0.42} \\
\hspace{0.10cm} - $\alpha=2$ & 0.73 & 0.88 & 0.81\textsubscript{↓ 0.16} \\
\hspace{0.10cm} - $\beta=0$ (w/o density) & 0.92 & 0.89 & 0.91\textsubscript{↓ 0.07} \\
\hspace{0.10cm} - $\beta=1$ & 0.90 & 0.62 & 0.76\textsubscript{↓ 0.21} \\
\bottomrule
\end{tabular}
}
\caption{Ablation Study for \textit{NovelSum}.}
\label{tbl:ablation}
\vspace{-2mm}
\end{table}

In Table \ref{tbl:ablation}, $\alpha=0$ removes the proximity weights, and $\beta=0$ eliminates the density multiplier. We observe that both $\alpha=0$ and $\beta=0$ significantly weaken the correlation, validating the benefits of the proximity-weighted sum and density-aware distance. Additionally, improper values for $\alpha$ and $\beta$ greatly reduce the metric's reliability, highlighting that \textit{NovelSum} strikes a delicate balance between distances and distribution. Replacing cosine distance with Euclidean distance and using more neighbors for density approximation have minimal impact, particularly on Pearson's correlation, demonstrating \textit{NovelSum}'s robustness to different distance measures.







\end{document}
