\pdfoutput=1

\documentclass[11pt]{article}

\usepackage[final]{acl}

\usepackage{times}
\usepackage{latexsym}

\usepackage[T1]{fontenc}

\usepackage[utf8]{inputenc}

\usepackage{microtype}

\usepackage{inconsolata}

\usepackage{graphicx}
\usepackage{amsthm}
\usepackage{amsmath}
\usepackage{amssymb}
\usepackage{mathtools}
\usepackage{subfigure}
\theoremstyle{plain}
\newtheorem{theorem}{Theorem}[section]
\newtheorem{proposition}[theorem]{Proposition}
\usepackage{booktabs, array} %
\newcolumntype{H}{>{\setbox0=\hbox\bgroup}c<{\egroup}@{}}
\usepackage[textsize=tiny]{todonotes}
\usepackage{makecell}
\usepackage{colortbl}
\usepackage{xcolor}
\definecolor{ForestGreen}{RGB}{34, 139, 34}

\NewDocumentCommand{\yz}
{ mO{} }{\textcolor{blue}{\textsuperscript{\textit{Yizhe}}\textsf{\textbf{\small[#1]}}}}

\NewDocumentCommand{\samy}
{ mO{} }{\textcolor{blue}{\textsuperscript{\textit{Samy}}\textsf{\textbf{\small[#1]}}}}

\NewDocumentCommand{\rxz}
{ mO{} }{\textcolor{teal}{\textsuperscript{\textit{rxz}}{{\small[#1]}}}}

\NewDocumentCommand{\rb}
{ mO{} }{\textcolor{ForestGreen}{\textsuperscript{\textit{rb}}{{\small[#1]}}}}




\title{Reversal Blessing: Thinking Backward May Outpace Thinking Forward in Multi-choice Questions}




\author{
 \textbf{Yizhe Zhang\textsuperscript{1}}\thanks{Equal contribution.}, 
 \textbf{Richard He Bai\textsuperscript{1}}\footnotemark[1], 
 \textbf{Zijin Gu\textsuperscript{1}}\thanks{Core contribution.}, 
 \textbf{Ruixiang Zhang\textsuperscript{1}}, 
\\
 \textbf{Jiatao Gu\textsuperscript{1}},
 \textbf{Emmanuel Abbe\textsuperscript{1}}, 
 \textbf{Samy Bengio\textsuperscript{1}}, 
 \textbf{Navdeep Jaitly\textsuperscript{1}}
\\
\\
 \textsuperscript{1}Apple
\\
}





\begin{document}
\maketitle
\begin{abstract}
Language models usually use left-to-right (L2R) autoregressive  factorization.
However, L2R factorization may not always be the best inductive bias.
Therefore, we investigate whether alternative factorizations of the text distribution could be beneficial in some tasks.
We investigate right-to-left (R2L) training as a compelling alternative, focusing on multiple-choice questions (MCQs) as a test bed for knowledge extraction and reasoning. Through extensive experiments across various model sizes (2B-8B parameters) and training datasets, we find that R2L models can significantly outperform L2R models on several MCQ benchmarks, including logical reasoning, commonsense understanding, and truthfulness assessment tasks. Our analysis reveals that this performance difference may be fundamentally linked to multiple factors including calibration, computability and directional conditional entropy.
We ablate the impact of these factors  through controlled simulation studies using arithmetic tasks, where the impacting factors can be better disentangled. 
Our work demonstrates that exploring alternative factorizations of the text distribution can lead to improvements in LLM capabilities and provides theoretical insights into optimal factorization towards approximating human language distribution, and when each reasoning order might be more advantageous.
\end{abstract}


\begin{figure*}[ht!]
    \centering
    \includegraphics[width=1.0\linewidth]{figures/figure1.pdf}
    \caption{Reverse Thinking in MCQs.
\textbf{Top left}: Standard forward thinking evaluates each answer choice based on the question and selects the one with the highest relevance score in a L2R LLM.
\textbf{Bottom left}: Reverse thinking evaluates the question based on each answer choice and selects the answer that maximizes the relevance score in a R2L LLM.
\textbf{Right}: Reverse thinking consistently outperforms forward thinking in certain MCQ tasks (Openbook QA in this figure), independent of training data and model size.
}
    \label{fig:r2l}
\end{figure*}

\section{Introduction}
Large Language Model (LLM) pretraining commonly employs left-to-right (L2R) next-token prediction, an approach that enables efficient parallelization and caching. This method models the text distribution $p(x)$ as a factorized autoregressive chain as $p(x_t|x_{<t})$. L2R naturally aligns with human cognitive processes of text generation and reasoning, making it well-suited for inference tasks.
However, while perfect modeling of each $p(x_t|x_{<t})$ would theoretically enable exact recovery of the data distribution $p(x)$, neural networks inevitably introduce approximation errors for each $p(x_t|x_{<t})$.
These errors compound over timestep $t$ during inference, potentially resulting in hallucinations and repetitions in generation \citep{bengio2015scheduled, zhang2023planner}.
Further, L2R factorization can result in inductive biases that lead to unwanted behaviors.
For example, \citet{allen2023physics_3_1} show that inverse search is challenging for L2R LLMs. 

We wonder whether L2R is optimal, and if alternative factorizations might capture unique aspects of the data distribution that complement L2R. Can specific factorizations achieve lower approximation errors compared to L2R, or reduce L2R's inherent bias in particular task domains? 


Autoregressive modeling in right-to-left (R2L) fashion factorizes $p(x)$ as $p(x_t|x_{>t})$, which presents a particularly promising alternative  that has been examined in previous work \citep{papadopoulos2024arrows, berglund2023reversal, zhang2024reverse}. 
This setup views the task as predicting the previous token, and it can achieve prediction losses comparable to the L2R next token prediction objective, due to its symmetry to L2R.
We leave the exploration of finding the optimal factorization for language data as future work. 




We investigates three questions: (1) How to evaluate R2L models on knowledge extraction and basic reasoning tasks? (2) Can R2L factorization match or surpass L2R's capabilities in knowledge extraction and reasoning for downstream tasks? (3) What are underlying factors determining the preference of L2R or R2L factorizations?

To address these questions, we conducted controlled experiments comparing L2R and R2L models trained with identical data and computational resources.
We evaluated both factorization approaches using standard LLM benchmarks with Multiple-Choice Questions (MCQs). For simplicity, we limit our comparison to MCQs, and leave the evaluations for generative tasks as future work.
For R2L models, we applied Bayesian inference to implement "reverse thinking," evaluating choices based on their likelihood of generating the prompt (Figure~\ref{fig:r2l}).
Our results demonstrate that R2L factorization consistently outperforms L2R across various model sizes and pretraining datasets on several MCQ reasoning and knowledge extraction tasks. 





We then investigate the underlying reasons that contribute to R2L's superior performance in certain scenarios and aim to establish principles for selecting between L2R and R2L factorizations. We hypothesize that the effectiveness of L2R and R2L may be fundamentally linked to several factors: \textit{calibration}, \textit{computability}, and the \textit{conditional entropy} of the factorization direction.
Specifically, we found that the factorization direction that achieves lower conditional perplexity generally yields better evaluation results.
Nevertheless, these factors are intricately interwoven in actual MCQs, complicating the analysis. To disentangle these factors and ablate on their impact to the performance of L2R or R2L factorization, we design a controlled simulation study using arithmetic tasks, revealing how various factors influence the effectiveness of certain factorization.
Our code and model checkpoints have been made available to facilitate future research.\footnote{\scriptsize \url{https://github.com/apple/ml-reversal-blessing}.}






\section{Thinking Backward in MCQs}
\subsection{Solving MCQs}
\label{sec:rev}
\paragraph{Solving MCQs with forward thinking}
As shown in Figure~\ref{fig:r2l}, in MCQs, LLM process a question $q$ alongside a set of answer choices $ A = \{a_1, a_2, \ldots, a_n\} $. Each (question, answer) pair $(q, a_i)$ is encoded to compute a relevance score $s_i$. The model then selects the answer $a_k$ corresponding to the highest score: $k = \arg\max_i s_i$. 


To compute $s_i$, the model evaluates the log-probability of generating the answer $a_i$ given the question $q$. This log-probability is often normalized to account for variations in answer length, preventing a bias toward shorter or longer responses. Various normalization techniques \citep{holtzman-etal-2021-surface} can be applied, however, we resort to the most common approach which divides the total log-probability by the length of the answer $N_i=\text{len}(a_i)$ in tokens or bytes, resulting in a normalized relevance score:
$s_i = \frac{\log p(a_i \mid q)}{N_i}$. The log-probability is factorized as 
\begin{align}
    \log p(a_i \mid q) = \sum_{l=1}^{N_i} \log p_{L2R}(a_i^{l} \mid q, a_i^{<l}),
    \label{eq:forward}
\end{align}
where $a_i^{l}$ represents the $l$-th token in $a_i$. 


\paragraph{Solving MCQs with reverse thinking}
If an R2L model is trained, $s_i$ can be computed using Bayes' rule:
\begin{align}
    s_i &= \log p(a_i \mid q) / M_i \nonumber  \\ 
    &= \frac{1}{M_i}(\log p_{R2L}(q \mid a_i) + \log p_{R2L}(a_i) - C), \nonumber
\end{align}
where $M_i=\text{len}(q,a_i)$,  $C=\log p_{R2L}(q)$ is a constant. $\log p_{R2L}(q \mid a_i)$ and $\log p_{R2L}(a_i)$ can be autoregressively factorized in R2L manner similar to the forward thinking process in Eq.~\eqref{eq:forward}. We consider 3 paradigms of the $s_i$ for reverse thinking: (1) normalized $s_i$ with $M_i =\text{len}(q, a_i)$ resembling the forward thinking; (2) unnormalized $s_i$ with $M_i = 1$; (3) unnormalized $s_i$ without prior, i.e. $s_i=\log p_{R2L}(q \mid a_i)$.




\subsection{Model evaluation}
We conduct our evaluation on standard LLM evaluation tasks with MCQs that cover different domains including commonsense reasoning, logical reasoning, truthfulness evaluation and more. 

Our evaluation tasks include HellaSwag~\citep{zellers-etal-2019-hellaswag}, ARC~\citep{clark2018think}, MMLU~\citep{hendrycks2021measuring}, Openbook QA~\citep{mihaylov2018openbookqa}, MathQA~\citep{amini-etal-2019-mathqa}, LogiQA~\citep{liu2020logiqa}, PIQA~\citep{bisk2019piqa}, Social IQA~\citep{sap-etal-2019-social}, Commonsense QA \citep{talmor2018commonsenseqa}, Truthful QA~\citep{lin2021truthfulqa}, and WinoGrande~\citep{sakaguchi2021winogrande}. 
For ARC (easy, hard) and MMLU, we combine all the subtasks to report the overall score. 
We use Eleuther-AI LM-eval harness \citep{eval-harness} for all the evaluations. 
For MMLU, LogiQA, and Commonsense QA, we modify the task templates to present full answer choices rather than just choice labels.






\begin{table*}[!htp]\centering
\caption{Comparing L2R and R2L on MCQs. All the models are trained on 350B non-repeating tokens. The HF-2B baseline is from \citet{penedo2024the}. We directly used their reported numbers. EDU-2B, EDU-8B and HF-2B models are trained with the same FineWeb-EDU 350B dataset. \textcolor{ForestGreen}{Green} indicates R2L wins, \textcolor{red}{red} indicates R2L loses.
}\label{tab:main_results}
\rowcolors{2}{gray!15}{white}
\small
\begin{tabular}{lcccccccccccc}
\toprule
&\multicolumn{3}{c}{\textbf{DCLM-2B}} &\multicolumn{3}{c}{\textbf{EDU-2B}} &\multicolumn{3}{c}{\textbf{EDU-8B}} & \textbf{HF-2B} \\\cmidrule{2-11}
&L2R &R2L &\% Change &L2R &R2L &\% Change &L2R &R2L &\% Change &L2R \\
\midrule
Training loss & \textbf{2.668} & 2.724 & \textcolor{red}{+2.10} &\textbf{2.345} &2.396 & \textcolor{red}{+2.17} & \textbf{2.087}& 2.138 & \textcolor{red}{+2.44} & - \\
\midrule
\textbf{LogiQA} &30.57 &\textbf{31.64} & \textcolor{ForestGreen}{+3.52} &27.96 &\textbf{31.49} & \textcolor{ForestGreen}{+12.64} &29.95 &\textbf{31.03} & \textcolor{ForestGreen}{+3.61} & - \\
\textbf{OpenbookQA} &36.00 &\textbf{38.40} & \textcolor{ForestGreen}{+6.67} &42.40 &\textbf{44.40} & \textcolor{ForestGreen}{+4.72} &45.00 &\textbf{48.40} & \textcolor{ForestGreen}{+7.56} & 41.04 \\
\textbf{TruthfulQA} &19.82 &\textbf{29.99} & \textcolor{ForestGreen}{+51.23} &24.36 &\textbf{28.76} & \textcolor{ForestGreen}{+18.09} &24.97 &\textbf{31.70} & \textcolor{ForestGreen}{+26.95} & - \\
\textbf{CommonsenseQA} &42.83 &\textbf{45.29} & \textcolor{ForestGreen}{+5.74} &42.92 &\textbf{45.13} & \textcolor{ForestGreen}{+5.15} &39.15 &\textbf{44.96} & \textcolor{ForestGreen}{+14.84} & 36.60 \\
Social IQA &\textbf{41.56} &40.94 & \textcolor{red}{-1.48} &\textbf{42.78} &42.22 & \textcolor{red}{-1.32} &\textbf{44.58} &43.50 & \textcolor{red}{-2.42} & 40.52 \\
ARC &\textbf{54.11} &43.88 & \textcolor{red}{-18.91} &\textbf{60.65} &52.31 & \textcolor{red}{-13.75} &\textbf{68.29} &56.22 & \textcolor{red}{-17.67} & 57.47 \\
HellaSwag &\textbf{60.87} &45.89 & \textcolor{red}{-24.62} &\textbf{60.57} &42.22 & \textcolor{red}{-26.78} &\textbf{71.60} &49.22 & \textcolor{red}{-31.26} & 59.34 \\
MathQA &\textbf{26.50} &22.21 & \textcolor{red}{-16.18} &\textbf{26.80} &24.86 & \textcolor{red}{-7.25} &\textbf{28.77} &25.33 & \textcolor{red}{-11.96} & - \\
MMLU &\textbf{31.66} &31.31 & \textcolor{red}{-1.10} &\textbf{34.57} &34.35 & \textcolor{red}{-0.62} &\textbf{38.90} &37.11 & \textcolor{red}{-4.60} & 37.35 \\
PIQA &\textbf{74.43} &58.05 & \textcolor{red}{-22.00} &\textbf{74.48} &57.13 & \textcolor{red}{-23.30} &\textbf{77.80} &59.14 & \textcolor{red}{-23.98} & 76.70 \\
Winogrande &\textbf{61.01} &53.51 & \textcolor{red}{-12.29} &\textbf{60.93} &54.85 & \textcolor{red}{-9.97} &\textbf{65.75} &54.70 & \textcolor{red}{-16.81} & 57.54 \\
\bottomrule
\end{tabular}
\end{table*}


\subsection{Model Pretraining}
To pretrain the model, we first tokenize each complete dataset. 
The R2L model is then trained by reversing all tokens within each training data instance. For a fair comparison between the R2L and L2R models, both models are pretrained from scratch using the same Fineweb-EDU subset dataset comprising 350B tokens \citep{penedo2024the}. Each model consists of 2B parameters (\textbf{EDU-2B}). This is the default setting in our experiments. 
Both the L2R and R2L models are trained for a single epoch, ensuring each training instance is seen only once, thus the training loss should align with the validation loss. More details for model architecture and training are provided in Appendix~\ref{app:arch}.

To validate the robustness of our findings, we further train two additional variants of settings. These include 8B L2R and R2L models trained with the same 350B Fineweb-EDU dataset (\textbf{EDU-8B}), and 2B L2R and R2L models trained with a random subset of the DCLM dataset \citep{li2024datacomplm} containing 350B tokens (\textbf{DCLM-2B}).




\subsection{Results}
We present our results in Table~\ref{tab:main_results}. To verify our pretraining pipeline, we first compare the performance of our pretrained model with the 2B model trained by Huggingface \citep{penedo2024the} (\textbf{HF-2B}) 
\footnote{\url{https://huggingface.co/spaces/HuggingFaceFW/blogpost-fineweb-v1}}. Under similar model size and the same dataset, our 2B model's performance (\textbf{EDU-2B}) is either comparable to or exceeds the L2R results reported by Huggingface \textbf{HF-2B}.



We then compare the L2R and R2L on all evaluated tasks.
As shown in Table~\ref{tab:main_results}, surprisingly, for 4 out of the 11 tasks (LogiQA, OpenbookQA, TruthfulQA and Commonsense QA), using R2L with reverse thinking actually improves reasoning performance.
In some cases, the improvement was substantial (e.g. 51.23\% on TruthfulQA).
These results held consistently across different model sizes (2B, 8B), datasets (DCLM, FineWeb EDU), and random seeds, suggesting it is not merely random fluctuation.

For reverse thinking with R2L, we use the paradigm 3 (\textit{i.e.}, unnormalized $s_i$ without prior) for downstream tasks evaluation. 
We compare the three paradigms for reverse thinking in Appendix~\ref{app:rev_comparison}, Table~\ref{tab:3variants}.
Ideally, $s_i$ should incorporate priors, as in paradigm 1 or 2. However, in practice, using $s_i$ without prior (paradigm 3) consistently yields the best performance except for Social IQA and PIQA. We hypothesize this may be due to intrinsic difficulty of estimating the prior probabilities $p(a)$ using LLMs, due to the "surface competition" calibration issues \citep{holtzman-etal-2021-surface}. We provide detailed explanation of our hypothesize using an illustrative example in Appendix~\ref{app:surface}. Since MCQ answer choices are generally designed to be reasonable text, assigning a uniform prior is probably a reasonable approach.


We also monitor the training loss for pretraining the models on both directions.
We observed findings similar to~\citet{papadopoulos2024arrows} in that L2R yields a lower loss compared to R2L, even though both model the same target data distribution.
In \citet{papadopoulos2024arrows}, the largest model that was trained had $405$M parameters while our models were trained at the popular small LLM size range of 2B-8B parameters.
At this size, we observe a similar percentage difference as reported by previous work, of about 2\%-2.5\% increase in loss when using R2L, indicating learning the R2L factorization is more challenging. This makes it particularly interesting that on a bunch of MCQ tasks we see the R2L is performing better, as elaborated above. 

 
\begin{figure*}[htp!]
    \centering
    \includegraphics[width=1.0\linewidth]{figures/figure2.pdf}
    \caption{L2R and R2L LLMs pretrained on the same data will generate opposite search graphs based on the order in which they process the information entities.}
    \label{fig:graph}
\end{figure*}

\section{What Makes The Preferred Order of Thinking?}
\label{sec:why}
We then seek to gain a deeper understanding of why there is a preferred orientation for the MCQs. We explore three main hypotheses (3\textbf{C}): \textit{\textbf{C}alibration}, \textit{\textbf{C}omputability}, and \textit{\textbf{C}onditional entropy}. 
Admittedly, there may be other factors that we have overlooked that contributes to this preference. 

\subsection{Calibration} The first potential explanation concerns the scoring mechanism in forward thinking, where $s_i=\log p_{L2R}(a_i \mid q)$. Eq.~\eqref{eq:forward} might not lead to an optimal estimation of $p(a|q)$ as it suffers from several calibration issues. Among the choices, some may contain more words that are highly predictable (e.g., "Hong Kong" or stop-words like "a"), potentially leading to spuriously inflated relevance scores. Additional, \citet{holtzman-etal-2021-surface} shows that simple probability normalization in MCQs is challenging because different surface forms of semantically equivalent answers compete for probability mass, potentially \textit{diluting} scores
for correct answers due to this "surface form competition".

In contrast, reverse thinking with paradigm 3, where $s_i=\log p_{R2L}(q \mid a_i)$, mitigates this issue since the target question $q$ remains constant across all choices.
We provide rationale analysis on how R2L paradigm 3 alleviates "surface competition" in Appendix~\ref{app:surface}.
In a nutshell, forward thinking suffers from surface form competition, where semantically similar words (e.g., "dog" and "puppy") split probability mass, reducing the likelihood of selecting the correct answer. Reverse thinking mitigates this by enforcing a uniform prior, eliminating competition in the prior distribution and allowing a fairer comparison between answer choices.

This suggests that reverse thinking inherently "auto-normalizes" different choices, resulting in more robust evaluation. However, this sole theory fails to explain why reverse thinking does not consistently outperform forward thinking across all tasks, instead showing superior performance only in specific MCQ scenarios.




\subsection{Computability} A second potential theoretical explanation, which echoes with \citet{papadopoulos2024arrows}, suggests that computational complexity may underlie these directional preferences. Drawing an analogy to number theory, where multiplying prime numbers is computationally straightforward, while the reverse operation of prime factorization is NP-hard. 

It is tempting to consider this computational complexity asymmetry as the main underlying cause for why L2R or R2L is preferred for specific tasks. However,
recent research  \citep{mirzadeh2024gsm,kambhampati2024can,valmeekam2024llms} find that LLMs may not actually perform genuine reasoning or computing,
as evidenced by their poor generalization when tasks undergo minor modifications. 
This implies that LLMs mainly emulate \textit{reasoning patterns} from their training data instead of carrying out actual logical computation, weakening the hypothesis that directional preferences stem from varying computability in different directions.

Furthermore, most MCQs primarily involve knowledge retrieval and basic reasoning, which might not reach the complexity threshold where computational hardness would become a significant factor. Therefore, acknowledging that computability may be a factor, we keep exploring alternative hypotheses.

\begin{figure*}[ht!]
    \centering
    \includegraphics[width=0.85\linewidth]{figures/l2r_r2l_acc_ce.pdf}
    \caption{
    Lower conditional entropy is typically associated with higher accuracy in the reasoning direction.
    }
    \label{fig:ppl_comparison}
\end{figure*}

\subsection{Conditional Entropy}
Our final hypothesis posits that the optimal direction of thinking is closely related to the \textit{conditional entropy} of the downstream task. 
Recent work has shown that learning knowledge extraction and simple multihop reasoning is more challenging for problems with higher degree of \textbf{branching factors} or "\textbf{globality degree}" compared to those with lower branching factors and more deterministic relationships~\citep{abbe2024far}.
It is conceivable that directionality of data can impact the branching degree and lead to different learning efficiencies in different directions (for example multiplication in left-to-right direction is factorization in the opposite direction, each with different branching factors).

Previous work \citep{berglund2023reversal,allen2023physics_3_2} has also demonstrated that LLMs suffer from the "reversal curse", indicating that inverse R2L search in LLMs is inherently challenging for L2R models - due the disconnect between training and inference directions.
Consider an LLM trained on sequences of knowledge/information name entities $(e_1,e_2,\cdots,e_n)$. LLM may effectively construct a \textbf{directed} search graph that maps the key $(e_1,\cdots,e_{i-1})$ to the value $e_i$ for any $i$.  Following this logic, the training data essentially forms a Bayesian network that can be represented as a \textit{directed acyclic graph} (DAG) of entities. Similarly, training an R2L model yields an analogous DAG but with reversed edge directions (see Figure~\ref{fig:graph} for an illustration).
The search efficiency between these two graphs may vary given different queries.




We hypothesize here that between two different factorizations of the data, \textbf{the direction yielding lower conditional entropy will perform better in MCQs}, as it reflects better efficiency in knowledge extraction and multi-hop search. 
We note however, that this is only true when models under both factorization directions have sufficiently low error, which seems to be true for our models here. 


More formally, for a downstream MCQ task $T$ with question and answer choices following task-specific data distribution $P_T(q, a)$, we compare the \textit{conditional entropy} in both directions under pretrained L2R and R2L models (Eq.~\eqref{eq:l2r_ent} for L2R and Eq.~\eqref{eq:r2l_ent} for R2L):
\begin{align}
&-\mathbb{E}_{q'\sim P_T(q)} \sum_a p_{L2R}(a|q') \log p_{L2R}(a|q').
\label{eq:l2r_ent}\\
&-\mathbb{E}_{a'\sim P_T(a)} \sum_q p_{R2L}(q|a') \log p_{R2L}(q|a').
\label{eq:r2l_ent}
\end{align}


We assume that the conditional entropy is a proxy for the quality of the learned model, and the direction with lower conditional entropy should perform better.
However, computing these summations in \eqref{eq:l2r_ent} and \eqref{eq:r2l_ent} is intractable due to the exponentially large candidate space. Therefore, we employ Monte Carlo estimation of \eqref{eq:l2r_ent} and \eqref{eq:r2l_ent}
as proxy measures, specifically computing \begin{align}
&-\mathbb{E}_{q' \sim P_t(q), a' \sim p_{L2R}(a|q')} \log p_{L2R}(a' | q'), \\
&-\mathbb{E}_{a' \sim P_t(a), q' \sim p_{R2L}(q|a')} \log p_{R2L}(q' | a'). 
\end{align}

Because of the extensive amount of evaluation datasets, due to limited computation budget, we only conducted a single sample rollout for $a' \sim p_{L2R}(a|q')$ and $q' \sim p_{R2L}(q|a')$. We recognize that this may not be a precise representation of the true conditional entropy, given that the candidate space grows exponentially with the maximum sequence length. 



\paragraph{Empirical Verification}
To verify this hypothesis, we estimate the conditional entropy for all the evaluation tasks. We provide more experimental details in Appendix~\ref{app:ce}.
Figure \ref{fig:ppl_comparison} presents our empirical results, which support this hypothesis that
lower conditional entropy is typically linked to greater task accuracy, except for CommonsenseQA and OpenbookQA which are outliers likely because of other confounding factors including the computability.
 
In Figure \ref{fig:ppl_comparison}, we observed that the conditional entropy of R2L is generally greater than L2R. This trend could be related to the findings presented in Table~\ref{tab:main_results}, indicating that R2L tends to have higher training loss too.
Complementing the rationale in \citet{papadopoulos2024arrows}, we hypothesize that the ease with which the language model can approximate the factorized distribution of L2R and R2L, may be also tied to which direction exhibits higher branching factors in that direction. We leave this exploration for future study. 









\begin{table*}[!htp]\centering
\caption{
Results of the controlled simulation study of 4-digits multiplication. Theoretical Conditional Entropy (Theo. Cond. Ent.) represents the expected conditional entropy under an ideal model. L2R consistently outperforms R2L in Forward X, while R2L is superior in Reverse X. Lower conditional entropy correlates with higher accuracy. 
}\label{tab:sim_results}
\small
\begin{tabular}{lcccHHccc}
\toprule
&\multicolumn{3}{c}{\textbf{Forward X}} &\multicolumn{2}{H}{Unique X} &\multicolumn{3}{c}{\textbf{Reverse X}} \\\cmidrule{2-4} \cmidrule{7-9}
&L2R &R2L(m,n) &R2L(m) &L2R &R2L &R2L &L2R(m,n) &L2R(n) \\\midrule
Test Accuracy (\%) & \textbf{99.81}$\pm$0.15 & 59.71$\pm$1.99 & 60.93 $\pm$ 0.88&\textbf{96.97}$\pm$0.48 & 53.91$\pm$1.09  & \textbf{100}$\pm$0 & 97.82$\pm$0.35 & 99.85$\pm$0.10\\
Train Accuracy (\%) & \textbf{99.76}$\pm$0.15 & 59.03 $\pm$ 1.66& 61.22$\pm$1.12& 95.84$\pm$0.46& 54.79 $\pm$1.32& \textbf{100}$\pm$0 & 97.90$\pm$0.42 & 99.98$\pm$0.04\\
\midrule
Test Cond. Ent. (nats) & 0.06 & 1.18 & 0.08& 1.04 & 1.50 & 0 & 0.84 & 0.01 \\
Train Cond. Ent. (nats) & 0.06 & 1.17 & 0.08 & 1.05& 1.51 & 0 & 0.83 & 0.01\\
Theo. Cond. Ent. (nats) & 0 & 1.49 & 0 & 0 & 0 & 0 & 1.49 & 0 \\
\midrule
Training loss & \textbf{0.86} & 0.94 & 0.94 & 0.85 & 0.92 & \textbf{0.86} & 0.94 & 0.94\\
\bottomrule
\end{tabular}
\end{table*}

\section{Controlled Simulation Study}
\label{sec:sim}
The three hypotheses discussed in Section~\ref{sec:why} are intricately entwined in actual MCQs, making it challenging to disentangle them.
To better investigate the hypotheses explaining the optimal direction for MCQs, we conducted a meticulously controlled simulation study (Figure~\ref{fig:sim}) focused on 4-digit multiplication. The L2R and R2L models were initialized \textbf{from scratch} and \textbf{exclusively} trained on this simulation dataset to eliminate any potential confounding factors. All data instances share the same format and length, removing the \textit{calibration} effect from the analysis and allowing us to concentrate on \textit{computability} and \textit{conditional entropy}. 


\begin{figure}
    \centering
    \includegraphics[width=1.0\linewidth]{figures/sim.pdf}
    \caption{Simulation Study. Forward multiplication simulates a \textbf{many-to-one} mapping scenario, while reverse multiplication simulates a \textbf{one-to-many} mapping.}
    \label{fig:sim}
\end{figure}
\paragraph{Experiment Setup}
We conduct two types of simulation experiments: Forward Multiplication (\textbf{Forward X}) and Reverse Multiplication (\textbf{Reverse X}). In Forward X, each training instance was represented as $m \times n = p$, where $m, n \in \{0, \ldots, 10^4\}$ and $p \in \{0, \ldots, 10^8\}$. The formatting included spaces between digits and mathematical operators to ensure a consistent single tokenization for both L2R and R2L models.
In Reverse X, the multiplication was in reverse order, such as $p = m \times n$. For each simulation type, L2R and R2L models were trained with a 2B model size with 1 epoch on all $10^8$ non-repeating equations except 1,000 test examples, totaling to almost 3.2B tokens.



The model performance was assessed using the held-out test set with 1,000 examples. These examples were converted into a multiple-choice format consisting of 4 choices (Figure~\ref{fig:sim}). Other than the correct answer, the remaining three hard-negative options were created by  altering a single digit in the correct answer to a random other digit, at a random position. The presenting order of the four choices are then randomly shuffled. We augmented the test set 10 times and calculated the average accuracy and conditional entropy.

As multiple pairs of $m$ and $n$ can be mapped to the same product $p$, Forward X is a \textbf{many-to-one} mapping. The theoretical conditional entropy for predicting the correct $p$ from $m \times n$ is 0 under an oracle model. However, as there are several paths from the product $p$ to the $m,n$ pairs, the theoretical conditional entropy for predicting the $m \times n$ from $p$ becomes 1.49 nats under an oracle model.
In the Reverse X task, which transitions into a \textbf{one-to-many} scenario, the analysis is inverted.

For Forward X, we explore an alternative R2L evaluation method, denoted as \textbf{R2L(m)}, where the relevance score of the i-th choice $p_i$ is calculated as $s_i=\log p_{R2L}(m \mid p_i, n)$, focusing on the conditional entropy of $m$ rather than $m \times n$ as in the standard \textbf{R2L(m,n)} method. Since R2L(m) is essentially division, it is deterministic with a theoretical conditional entropy of 0. Similarly, we have a variant for L2R in reverse X, called \textbf{L2R(n)}.








\paragraph{Results}
The results are presented in Table~\ref{tab:sim_results}. In Forward X scenarios, L2R models demonstrate significantly higher accuracies than R2L(m,n) models, with correspondingly lower conditional entropy and training loss. This observation aligns with our hypothesis in Section~\ref{sec:why}.
Conversely, in Reverse X scenarios, the R2L model outperforms the L2R(m,n) model. The training and test performance gaps are minimal.

Interestingly, R2L(m) achieves better accuracy than R2L(m,n) in Forward X as conditional entropy decreases. Similarly, L2R(m,n) surpasses L2R(n) in Reverse X. This suggests that \textbf{when maintaining the same thinking direction -- where computability should remain equivalent -- performance improvements can be achieved} by configuring $s_i$ to have lower conditional entropy. 
This hints that the R2L performance on MCQs can potentially be further improved by configuring the input to predict fewer tokens in the question $q$, so that the minimum conditional entropy is obtained.
We leave this for future exploration. 

On the other hand, comparing L2R with R2L(m), where theoretical conditional entropy equal 0, L2R maintains superiority, indicating that \textbf{computability likely remains as a key factor}. 
For the Reverse X task, the accuracy gap between R2L and L2R(n) is smaller than the accuracy gap between L2R(m,n) and L2R(n), suggesting that the conditional entropy may explain more of the performance gap than the computability.

Notably, models achieve higher accuracies on Reverse X compared to their Forward X counterparts, despite similar training loss and conditional entropy values. This disparity could probably be attributed to the closer proximity of choices in Forward X, which inherently increases task difficulty. We provide additional discussion and analysis comparing Forward X and Reverse X in Appendix~\ref{app:simulation}.























\section{Related Work}
\paragraph{Reversal Curse} 
\citet{berglund2023reversal} first investigates the "reversal curse" in LLMs, which refers to the phenomenon where models trained on forward text data struggle to perform well on inverse search tasks. \citet{allen2023physics_3_1} further discusses this issue and proposes that augmentation during the pretraining stage can help bridge the knowledge extraction performance gap in reverse entity mapping. In a similar vein, \citet{golovneva2024reverse} suggest training a unified model that combines text data with augmented reversed or partially reversed data can mitigate the reversal curse.
These studies imply that autoregressively-trained language models tend to have a linear and unidirectional thinking process, and certain types of augmentation can faciliate the model in making complex connections between pieces of learned information to enable more intricate cross-referencing. Our research also demonstrates that the autoregressive nature of LLMs may introduce inductive biases rooted from the pretraining corpus. Instead of focusing on the "reversal curse," we suggest that knowledge extraction and reasoning may be more straightforward in the direction with lower conditional entropy.


\paragraph{Order of Reasoning} 
Previous works have also been exploring the reasoning order's impact to the reasoning performance. \citet{vinyals2015order} first demonstrates that the sequence in which input and output data are organized significantly impacts the performance of sequence-to-sequence models and propose to search over possible orders during training to manage unstructured output sets.
Recently, \citet{papadopoulos2024arrows} reveals a surprisingly consistent lower log-perplexity when predicting in L2R versus R2L, despite theoretical expectations of symmetry. The authors attributes this asymmetry to factors like sparsity and computational complexity. We also observe this difference yet we have another hypothesis rationale beyond theirs. 
\citet{zhang2024reverse} shows that by reversing the digit order, prioritizing the least significant digit can improve LLMs's performance on arithmetic, which aligns with our findings in section~\ref{sec:sim}. 



Previous studies on sequence modeling have also delved into relaxing the conventional ``left-to-right'' autoregressive dependencies, primarily to facilitate rapid parallel generation \citep{gu2018non, ghazvininejad-etal-2019-mask,gu-kong-2021-fully,zhang-etal-2020-pointer} and non-monotonic sequential generation~\citep{welleck2019non,gu2019insertion}. Text diffusion has recently emerged as a promising approach in terms of planning and controllability \citep{Li-2022-DiffusionLM, zhang2023planner, gong2024scaling}. It has shown to be more effective than LLM than language model (LLM), particularly for tasks that require bidirectional reasoning strategies such as sudoku and countdown games \citep{ye2024beyond}.


\paragraph{Multiple-Choice Questions (MCQs) for LLM evaluation}
MCQs have been widely used for evaluation LLM's reasoning and knowledge extraction abilities. 
\citet{zheng2023large} demonstrates that LLMs exhibit a selection bias in MCQs, favoring certain option positions, and introduces a debiasing method to mitigate this issue.
\citet{pezeshkpour2023large} examines how LLMs’ performance on MCQs is influenced by the order of answer options, finding that reordering can lead to huge performance variations. 
\citet{ghosal2022two} proposes reframing MCQs as a series of binary classifications, demonstrating that this approach significantly improves performance across various models and datasets.
\citet{li2024can} highlights issues like positional biases and discrepancies compared to long-form generated responses, when using MCQs in evaluating LLMs.
\citet{wiegreffe2024answer} discovers that the prediction of specific answer symbols is primarily attributed to a single middle layer’s multi-head self-attention mechanism, with subsequent layers increasing the probability of the chosen answer in the model’s vocabulary space.
In contrast to the previous work, our work first shows the connection between the preferred reasoning direction and the direction that has lower conditional entropy in MCQ evaluations.



\section{Conclusion}
In this work, we investigated the potential benefits of R2L factorization in language modeling, focusing on MCQs. Through extensive experimentation with models of varying sizes and training datasets, we demonstrated that R2L factorization can outperform traditional L2R approaches in specific MCQs. Our analysis revealed that the effectiveness of each factorization direction may be intrinsically linked to several factors including calibration, computability, and conditional entropy of the downstream task distribution, with lower conditional entropy yielding better performance. We disentangle and validate these factors through controlled simulation studies using arithmetic tasks.
These findings may suggest the potential for future language model development,  by revealing the knowledge extraction and reasoning machinery of LLM and suggesting that alternative factorizations deserve serious consideration in model design. 
Future work could explore additional factorization strategies beyond L2R and R2L, investigate applications to other types of language tasks, and develop more sophisticated methods for combining different factorizations.

% \smallskip
% \myparagraph{Acknowledgments} We thank the reviewers for their comments.
% The work by Moshe Tennenholtz was supported by funding from the
% European Research Council (ERC) under the European Union's Horizon
% 2020 research and innovation programme (grant agreement 740435).



\section*{Limitation}
Our work has several limitations. While MCQs provide a controlled evaluation setting, they represent only a subset of language understanding tasks, and the applicability of our findings to other formats like open-ended generation remains unexplored. We are currently extending our investigation to include generative tasks and dialogue systems to validate the broader applicability of our directional factorization hypothesis. Our theoretical framework around conditional entropy, while supported by empirical observations, lacks formal proofs and relies on estimated conditional entropy as an imperfect proxy. The simulation studies focused primarily on arithmetic operations with well-defined properties, which may not fully generalize to more complex language understanding and reasoning scenarios. New controlled experiments yet to be designed with increasingly complex reasoning chains and various answer length that better mirror real-world language understanding tasks. Our experiments were limited to models in the 2B-8B parameter range, and the relationships we observed might vary with different model scales or architectures. 

\bibliography{main}
\bibliographystyle{acl_natbib}



\newpage
\subsection{Lloyd-Max Algorithm}
\label{subsec:Lloyd-Max}
For a given quantization bitwidth $B$ and an operand $\bm{X}$, the Lloyd-Max algorithm finds $2^B$ quantization levels $\{\hat{x}_i\}_{i=1}^{2^B}$ such that quantizing $\bm{X}$ by rounding each scalar in $\bm{X}$ to the nearest quantization level minimizes the quantization MSE. 

The algorithm starts with an initial guess of quantization levels and then iteratively computes quantization thresholds $\{\tau_i\}_{i=1}^{2^B-1}$ and updates quantization levels $\{\hat{x}_i\}_{i=1}^{2^B}$. Specifically, at iteration $n$, thresholds are set to the midpoints of the previous iteration's levels:
\begin{align*}
    \tau_i^{(n)}=\frac{\hat{x}_i^{(n-1)}+\hat{x}_{i+1}^{(n-1)}}2 \text{ for } i=1\ldots 2^B-1
\end{align*}
Subsequently, the quantization levels are re-computed as conditional means of the data regions defined by the new thresholds:
\begin{align*}
    \hat{x}_i^{(n)}=\mathbb{E}\left[ \bm{X} \big| \bm{X}\in [\tau_{i-1}^{(n)},\tau_i^{(n)}] \right] \text{ for } i=1\ldots 2^B
\end{align*}
where to satisfy boundary conditions we have $\tau_0=-\infty$ and $\tau_{2^B}=\infty$. The algorithm iterates the above steps until convergence.

Figure \ref{fig:lm_quant} compares the quantization levels of a $7$-bit floating point (E3M3) quantizer (left) to a $7$-bit Lloyd-Max quantizer (right) when quantizing a layer of weights from the GPT3-126M model at a per-tensor granularity. As shown, the Lloyd-Max quantizer achieves substantially lower quantization MSE. Further, Table \ref{tab:FP7_vs_LM7} shows the superior perplexity achieved by Lloyd-Max quantizers for bitwidths of $7$, $6$ and $5$. The difference between the quantizers is clear at 5 bits, where per-tensor FP quantization incurs a drastic and unacceptable increase in perplexity, while Lloyd-Max quantization incurs a much smaller increase. Nevertheless, we note that even the optimal Lloyd-Max quantizer incurs a notable ($\sim 1.5$) increase in perplexity due to the coarse granularity of quantization. 

\begin{figure}[h]
  \centering
  \includegraphics[width=0.7\linewidth]{sections/figures/LM7_FP7.pdf}
  \caption{\small Quantization levels and the corresponding quantization MSE of Floating Point (left) vs Lloyd-Max (right) Quantizers for a layer of weights in the GPT3-126M model.}
  \label{fig:lm_quant}
\end{figure}

\begin{table}[h]\scriptsize
\begin{center}
\caption{\label{tab:FP7_vs_LM7} \small Comparing perplexity (lower is better) achieved by floating point quantizers and Lloyd-Max quantizers on a GPT3-126M model for the Wikitext-103 dataset.}
\begin{tabular}{c|cc|c}
\hline
 \multirow{2}{*}{\textbf{Bitwidth}} & \multicolumn{2}{|c|}{\textbf{Floating-Point Quantizer}} & \textbf{Lloyd-Max Quantizer} \\
 & Best Format & Wikitext-103 Perplexity & Wikitext-103 Perplexity \\
\hline
7 & E3M3 & 18.32 & 18.27 \\
6 & E3M2 & 19.07 & 18.51 \\
5 & E4M0 & 43.89 & 19.71 \\
\hline
\end{tabular}
\end{center}
\end{table}

\subsection{Proof of Local Optimality of LO-BCQ}
\label{subsec:lobcq_opt_proof}
For a given block $\bm{b}_j$, the quantization MSE during LO-BCQ can be empirically evaluated as $\frac{1}{L_b}\lVert \bm{b}_j- \bm{\hat{b}}_j\rVert^2_2$ where $\bm{\hat{b}}_j$ is computed from equation (\ref{eq:clustered_quantization_definition}) as $C_{f(\bm{b}_j)}(\bm{b}_j)$. Further, for a given block cluster $\mathcal{B}_i$, we compute the quantization MSE as $\frac{1}{|\mathcal{B}_{i}|}\sum_{\bm{b} \in \mathcal{B}_{i}} \frac{1}{L_b}\lVert \bm{b}- C_i^{(n)}(\bm{b})\rVert^2_2$. Therefore, at the end of iteration $n$, we evaluate the overall quantization MSE $J^{(n)}$ for a given operand $\bm{X}$ composed of $N_c$ block clusters as:
\begin{align*}
    \label{eq:mse_iter_n}
    J^{(n)} = \frac{1}{N_c} \sum_{i=1}^{N_c} \frac{1}{|\mathcal{B}_{i}^{(n)}|}\sum_{\bm{v} \in \mathcal{B}_{i}^{(n)}} \frac{1}{L_b}\lVert \bm{b}- B_i^{(n)}(\bm{b})\rVert^2_2
\end{align*}

At the end of iteration $n$, the codebooks are updated from $\mathcal{C}^{(n-1)}$ to $\mathcal{C}^{(n)}$. However, the mapping of a given vector $\bm{b}_j$ to quantizers $\mathcal{C}^{(n)}$ remains as  $f^{(n)}(\bm{b}_j)$. At the next iteration, during the vector clustering step, $f^{(n+1)}(\bm{b}_j)$ finds new mapping of $\bm{b}_j$ to updated codebooks $\mathcal{C}^{(n)}$ such that the quantization MSE over the candidate codebooks is minimized. Therefore, we obtain the following result for $\bm{b}_j$:
\begin{align*}
\frac{1}{L_b}\lVert \bm{b}_j - C_{f^{(n+1)}(\bm{b}_j)}^{(n)}(\bm{b}_j)\rVert^2_2 \le \frac{1}{L_b}\lVert \bm{b}_j - C_{f^{(n)}(\bm{b}_j)}^{(n)}(\bm{b}_j)\rVert^2_2
\end{align*}

That is, quantizing $\bm{b}_j$ at the end of the block clustering step of iteration $n+1$ results in lower quantization MSE compared to quantizing at the end of iteration $n$. Since this is true for all $\bm{b} \in \bm{X}$, we assert the following:
\begin{equation}
\begin{split}
\label{eq:mse_ineq_1}
    \tilde{J}^{(n+1)} &= \frac{1}{N_c} \sum_{i=1}^{N_c} \frac{1}{|\mathcal{B}_{i}^{(n+1)}|}\sum_{\bm{b} \in \mathcal{B}_{i}^{(n+1)}} \frac{1}{L_b}\lVert \bm{b} - C_i^{(n)}(b)\rVert^2_2 \le J^{(n)}
\end{split}
\end{equation}
where $\tilde{J}^{(n+1)}$ is the the quantization MSE after the vector clustering step at iteration $n+1$.

Next, during the codebook update step (\ref{eq:quantizers_update}) at iteration $n+1$, the per-cluster codebooks $\mathcal{C}^{(n)}$ are updated to $\mathcal{C}^{(n+1)}$ by invoking the Lloyd-Max algorithm \citep{Lloyd}. We know that for any given value distribution, the Lloyd-Max algorithm minimizes the quantization MSE. Therefore, for a given vector cluster $\mathcal{B}_i$ we obtain the following result:

\begin{equation}
    \frac{1}{|\mathcal{B}_{i}^{(n+1)}|}\sum_{\bm{b} \in \mathcal{B}_{i}^{(n+1)}} \frac{1}{L_b}\lVert \bm{b}- C_i^{(n+1)}(\bm{b})\rVert^2_2 \le \frac{1}{|\mathcal{B}_{i}^{(n+1)}|}\sum_{\bm{b} \in \mathcal{B}_{i}^{(n+1)}} \frac{1}{L_b}\lVert \bm{b}- C_i^{(n)}(\bm{b})\rVert^2_2
\end{equation}

The above equation states that quantizing the given block cluster $\mathcal{B}_i$ after updating the associated codebook from $C_i^{(n)}$ to $C_i^{(n+1)}$ results in lower quantization MSE. Since this is true for all the block clusters, we derive the following result: 
\begin{equation}
\begin{split}
\label{eq:mse_ineq_2}
     J^{(n+1)} &= \frac{1}{N_c} \sum_{i=1}^{N_c} \frac{1}{|\mathcal{B}_{i}^{(n+1)}|}\sum_{\bm{b} \in \mathcal{B}_{i}^{(n+1)}} \frac{1}{L_b}\lVert \bm{b}- C_i^{(n+1)}(\bm{b})\rVert^2_2  \le \tilde{J}^{(n+1)}   
\end{split}
\end{equation}

Following (\ref{eq:mse_ineq_1}) and (\ref{eq:mse_ineq_2}), we find that the quantization MSE is non-increasing for each iteration, that is, $J^{(1)} \ge J^{(2)} \ge J^{(3)} \ge \ldots \ge J^{(M)}$ where $M$ is the maximum number of iterations. 
%Therefore, we can say that if the algorithm converges, then it must be that it has converged to a local minimum. 
\hfill $\blacksquare$


\begin{figure}
    \begin{center}
    \includegraphics[width=0.5\textwidth]{sections//figures/mse_vs_iter.pdf}
    \end{center}
    \caption{\small NMSE vs iterations during LO-BCQ compared to other block quantization proposals}
    \label{fig:nmse_vs_iter}
\end{figure}

Figure \ref{fig:nmse_vs_iter} shows the empirical convergence of LO-BCQ across several block lengths and number of codebooks. Also, the MSE achieved by LO-BCQ is compared to baselines such as MXFP and VSQ. As shown, LO-BCQ converges to a lower MSE than the baselines. Further, we achieve better convergence for larger number of codebooks ($N_c$) and for a smaller block length ($L_b$), both of which increase the bitwidth of BCQ (see Eq \ref{eq:bitwidth_bcq}).


\subsection{Additional Accuracy Results}
%Table \ref{tab:lobcq_config} lists the various LOBCQ configurations and their corresponding bitwidths.
\begin{table}
\setlength{\tabcolsep}{4.75pt}
\begin{center}
\caption{\label{tab:lobcq_config} Various LO-BCQ configurations and their bitwidths.}
\begin{tabular}{|c||c|c|c|c||c|c||c|} 
\hline
 & \multicolumn{4}{|c||}{$L_b=8$} & \multicolumn{2}{|c||}{$L_b=4$} & $L_b=2$ \\
 \hline
 \backslashbox{$L_A$\kern-1em}{\kern-1em$N_c$} & 2 & 4 & 8 & 16 & 2 & 4 & 2 \\
 \hline
 64 & 4.25 & 4.375 & 4.5 & 4.625 & 4.375 & 4.625 & 4.625\\
 \hline
 32 & 4.375 & 4.5 & 4.625& 4.75 & 4.5 & 4.75 & 4.75 \\
 \hline
 16 & 4.625 & 4.75& 4.875 & 5 & 4.75 & 5 & 5 \\
 \hline
\end{tabular}
\end{center}
\end{table}

%\subsection{Perplexity achieved by various LO-BCQ configurations on Wikitext-103 dataset}

\begin{table} \centering
\begin{tabular}{|c||c|c|c|c||c|c||c|} 
\hline
 $L_b \rightarrow$& \multicolumn{4}{c||}{8} & \multicolumn{2}{c||}{4} & 2\\
 \hline
 \backslashbox{$L_A$\kern-1em}{\kern-1em$N_c$} & 2 & 4 & 8 & 16 & 2 & 4 & 2  \\
 %$N_c \rightarrow$ & 2 & 4 & 8 & 16 & 2 & 4 & 2 \\
 \hline
 \hline
 \multicolumn{8}{c}{GPT3-1.3B (FP32 PPL = 9.98)} \\ 
 \hline
 \hline
 64 & 10.40 & 10.23 & 10.17 & 10.15 &  10.28 & 10.18 & 10.19 \\
 \hline
 32 & 10.25 & 10.20 & 10.15 & 10.12 &  10.23 & 10.17 & 10.17 \\
 \hline
 16 & 10.22 & 10.16 & 10.10 & 10.09 &  10.21 & 10.14 & 10.16 \\
 \hline
  \hline
 \multicolumn{8}{c}{GPT3-8B (FP32 PPL = 7.38)} \\ 
 \hline
 \hline
 64 & 7.61 & 7.52 & 7.48 &  7.47 &  7.55 &  7.49 & 7.50 \\
 \hline
 32 & 7.52 & 7.50 & 7.46 &  7.45 &  7.52 &  7.48 & 7.48  \\
 \hline
 16 & 7.51 & 7.48 & 7.44 &  7.44 &  7.51 &  7.49 & 7.47  \\
 \hline
\end{tabular}
\caption{\label{tab:ppl_gpt3_abalation} Wikitext-103 perplexity across GPT3-1.3B and 8B models.}
\end{table}

\begin{table} \centering
\begin{tabular}{|c||c|c|c|c||} 
\hline
 $L_b \rightarrow$& \multicolumn{4}{c||}{8}\\
 \hline
 \backslashbox{$L_A$\kern-1em}{\kern-1em$N_c$} & 2 & 4 & 8 & 16 \\
 %$N_c \rightarrow$ & 2 & 4 & 8 & 16 & 2 & 4 & 2 \\
 \hline
 \hline
 \multicolumn{5}{|c|}{Llama2-7B (FP32 PPL = 5.06)} \\ 
 \hline
 \hline
 64 & 5.31 & 5.26 & 5.19 & 5.18  \\
 \hline
 32 & 5.23 & 5.25 & 5.18 & 5.15  \\
 \hline
 16 & 5.23 & 5.19 & 5.16 & 5.14  \\
 \hline
 \multicolumn{5}{|c|}{Nemotron4-15B (FP32 PPL = 5.87)} \\ 
 \hline
 \hline
 64  & 6.3 & 6.20 & 6.13 & 6.08  \\
 \hline
 32  & 6.24 & 6.12 & 6.07 & 6.03  \\
 \hline
 16  & 6.12 & 6.14 & 6.04 & 6.02  \\
 \hline
 \multicolumn{5}{|c|}{Nemotron4-340B (FP32 PPL = 3.48)} \\ 
 \hline
 \hline
 64 & 3.67 & 3.62 & 3.60 & 3.59 \\
 \hline
 32 & 3.63 & 3.61 & 3.59 & 3.56 \\
 \hline
 16 & 3.61 & 3.58 & 3.57 & 3.55 \\
 \hline
\end{tabular}
\caption{\label{tab:ppl_llama7B_nemo15B} Wikitext-103 perplexity compared to FP32 baseline in Llama2-7B and Nemotron4-15B, 340B models}
\end{table}

%\subsection{Perplexity achieved by various LO-BCQ configurations on MMLU dataset}


\begin{table} \centering
\begin{tabular}{|c||c|c|c|c||c|c|c|c|} 
\hline
 $L_b \rightarrow$& \multicolumn{4}{c||}{8} & \multicolumn{4}{c||}{8}\\
 \hline
 \backslashbox{$L_A$\kern-1em}{\kern-1em$N_c$} & 2 & 4 & 8 & 16 & 2 & 4 & 8 & 16  \\
 %$N_c \rightarrow$ & 2 & 4 & 8 & 16 & 2 & 4 & 2 \\
 \hline
 \hline
 \multicolumn{5}{|c|}{Llama2-7B (FP32 Accuracy = 45.8\%)} & \multicolumn{4}{|c|}{Llama2-70B (FP32 Accuracy = 69.12\%)} \\ 
 \hline
 \hline
 64 & 43.9 & 43.4 & 43.9 & 44.9 & 68.07 & 68.27 & 68.17 & 68.75 \\
 \hline
 32 & 44.5 & 43.8 & 44.9 & 44.5 & 68.37 & 68.51 & 68.35 & 68.27  \\
 \hline
 16 & 43.9 & 42.7 & 44.9 & 45 & 68.12 & 68.77 & 68.31 & 68.59  \\
 \hline
 \hline
 \multicolumn{5}{|c|}{GPT3-22B (FP32 Accuracy = 38.75\%)} & \multicolumn{4}{|c|}{Nemotron4-15B (FP32 Accuracy = 64.3\%)} \\ 
 \hline
 \hline
 64 & 36.71 & 38.85 & 38.13 & 38.92 & 63.17 & 62.36 & 63.72 & 64.09 \\
 \hline
 32 & 37.95 & 38.69 & 39.45 & 38.34 & 64.05 & 62.30 & 63.8 & 64.33  \\
 \hline
 16 & 38.88 & 38.80 & 38.31 & 38.92 & 63.22 & 63.51 & 63.93 & 64.43  \\
 \hline
\end{tabular}
\caption{\label{tab:mmlu_abalation} Accuracy on MMLU dataset across GPT3-22B, Llama2-7B, 70B and Nemotron4-15B models.}
\end{table}


%\subsection{Perplexity achieved by various LO-BCQ configurations on LM evaluation harness}

\begin{table} \centering
\begin{tabular}{|c||c|c|c|c||c|c|c|c|} 
\hline
 $L_b \rightarrow$& \multicolumn{4}{c||}{8} & \multicolumn{4}{c||}{8}\\
 \hline
 \backslashbox{$L_A$\kern-1em}{\kern-1em$N_c$} & 2 & 4 & 8 & 16 & 2 & 4 & 8 & 16  \\
 %$N_c \rightarrow$ & 2 & 4 & 8 & 16 & 2 & 4 & 2 \\
 \hline
 \hline
 \multicolumn{5}{|c|}{Race (FP32 Accuracy = 37.51\%)} & \multicolumn{4}{|c|}{Boolq (FP32 Accuracy = 64.62\%)} \\ 
 \hline
 \hline
 64 & 36.94 & 37.13 & 36.27 & 37.13 & 63.73 & 62.26 & 63.49 & 63.36 \\
 \hline
 32 & 37.03 & 36.36 & 36.08 & 37.03 & 62.54 & 63.51 & 63.49 & 63.55  \\
 \hline
 16 & 37.03 & 37.03 & 36.46 & 37.03 & 61.1 & 63.79 & 63.58 & 63.33  \\
 \hline
 \hline
 \multicolumn{5}{|c|}{Winogrande (FP32 Accuracy = 58.01\%)} & \multicolumn{4}{|c|}{Piqa (FP32 Accuracy = 74.21\%)} \\ 
 \hline
 \hline
 64 & 58.17 & 57.22 & 57.85 & 58.33 & 73.01 & 73.07 & 73.07 & 72.80 \\
 \hline
 32 & 59.12 & 58.09 & 57.85 & 58.41 & 73.01 & 73.94 & 72.74 & 73.18  \\
 \hline
 16 & 57.93 & 58.88 & 57.93 & 58.56 & 73.94 & 72.80 & 73.01 & 73.94  \\
 \hline
\end{tabular}
\caption{\label{tab:mmlu_abalation} Accuracy on LM evaluation harness tasks on GPT3-1.3B model.}
\end{table}

\begin{table} \centering
\begin{tabular}{|c||c|c|c|c||c|c|c|c|} 
\hline
 $L_b \rightarrow$& \multicolumn{4}{c||}{8} & \multicolumn{4}{c||}{8}\\
 \hline
 \backslashbox{$L_A$\kern-1em}{\kern-1em$N_c$} & 2 & 4 & 8 & 16 & 2 & 4 & 8 & 16  \\
 %$N_c \rightarrow$ & 2 & 4 & 8 & 16 & 2 & 4 & 2 \\
 \hline
 \hline
 \multicolumn{5}{|c|}{Race (FP32 Accuracy = 41.34\%)} & \multicolumn{4}{|c|}{Boolq (FP32 Accuracy = 68.32\%)} \\ 
 \hline
 \hline
 64 & 40.48 & 40.10 & 39.43 & 39.90 & 69.20 & 68.41 & 69.45 & 68.56 \\
 \hline
 32 & 39.52 & 39.52 & 40.77 & 39.62 & 68.32 & 67.43 & 68.17 & 69.30  \\
 \hline
 16 & 39.81 & 39.71 & 39.90 & 40.38 & 68.10 & 66.33 & 69.51 & 69.42  \\
 \hline
 \hline
 \multicolumn{5}{|c|}{Winogrande (FP32 Accuracy = 67.88\%)} & \multicolumn{4}{|c|}{Piqa (FP32 Accuracy = 78.78\%)} \\ 
 \hline
 \hline
 64 & 66.85 & 66.61 & 67.72 & 67.88 & 77.31 & 77.42 & 77.75 & 77.64 \\
 \hline
 32 & 67.25 & 67.72 & 67.72 & 67.00 & 77.31 & 77.04 & 77.80 & 77.37  \\
 \hline
 16 & 68.11 & 68.90 & 67.88 & 67.48 & 77.37 & 78.13 & 78.13 & 77.69  \\
 \hline
\end{tabular}
\caption{\label{tab:mmlu_abalation} Accuracy on LM evaluation harness tasks on GPT3-8B model.}
\end{table}

\begin{table} \centering
\begin{tabular}{|c||c|c|c|c||c|c|c|c|} 
\hline
 $L_b \rightarrow$& \multicolumn{4}{c||}{8} & \multicolumn{4}{c||}{8}\\
 \hline
 \backslashbox{$L_A$\kern-1em}{\kern-1em$N_c$} & 2 & 4 & 8 & 16 & 2 & 4 & 8 & 16  \\
 %$N_c \rightarrow$ & 2 & 4 & 8 & 16 & 2 & 4 & 2 \\
 \hline
 \hline
 \multicolumn{5}{|c|}{Race (FP32 Accuracy = 40.67\%)} & \multicolumn{4}{|c|}{Boolq (FP32 Accuracy = 76.54\%)} \\ 
 \hline
 \hline
 64 & 40.48 & 40.10 & 39.43 & 39.90 & 75.41 & 75.11 & 77.09 & 75.66 \\
 \hline
 32 & 39.52 & 39.52 & 40.77 & 39.62 & 76.02 & 76.02 & 75.96 & 75.35  \\
 \hline
 16 & 39.81 & 39.71 & 39.90 & 40.38 & 75.05 & 73.82 & 75.72 & 76.09  \\
 \hline
 \hline
 \multicolumn{5}{|c|}{Winogrande (FP32 Accuracy = 70.64\%)} & \multicolumn{4}{|c|}{Piqa (FP32 Accuracy = 79.16\%)} \\ 
 \hline
 \hline
 64 & 69.14 & 70.17 & 70.17 & 70.56 & 78.24 & 79.00 & 78.62 & 78.73 \\
 \hline
 32 & 70.96 & 69.69 & 71.27 & 69.30 & 78.56 & 79.49 & 79.16 & 78.89  \\
 \hline
 16 & 71.03 & 69.53 & 69.69 & 70.40 & 78.13 & 79.16 & 79.00 & 79.00  \\
 \hline
\end{tabular}
\caption{\label{tab:mmlu_abalation} Accuracy on LM evaluation harness tasks on GPT3-22B model.}
\end{table}

\begin{table} \centering
\begin{tabular}{|c||c|c|c|c||c|c|c|c|} 
\hline
 $L_b \rightarrow$& \multicolumn{4}{c||}{8} & \multicolumn{4}{c||}{8}\\
 \hline
 \backslashbox{$L_A$\kern-1em}{\kern-1em$N_c$} & 2 & 4 & 8 & 16 & 2 & 4 & 8 & 16  \\
 %$N_c \rightarrow$ & 2 & 4 & 8 & 16 & 2 & 4 & 2 \\
 \hline
 \hline
 \multicolumn{5}{|c|}{Race (FP32 Accuracy = 44.4\%)} & \multicolumn{4}{|c|}{Boolq (FP32 Accuracy = 79.29\%)} \\ 
 \hline
 \hline
 64 & 42.49 & 42.51 & 42.58 & 43.45 & 77.58 & 77.37 & 77.43 & 78.1 \\
 \hline
 32 & 43.35 & 42.49 & 43.64 & 43.73 & 77.86 & 75.32 & 77.28 & 77.86  \\
 \hline
 16 & 44.21 & 44.21 & 43.64 & 42.97 & 78.65 & 77 & 76.94 & 77.98  \\
 \hline
 \hline
 \multicolumn{5}{|c|}{Winogrande (FP32 Accuracy = 69.38\%)} & \multicolumn{4}{|c|}{Piqa (FP32 Accuracy = 78.07\%)} \\ 
 \hline
 \hline
 64 & 68.9 & 68.43 & 69.77 & 68.19 & 77.09 & 76.82 & 77.09 & 77.86 \\
 \hline
 32 & 69.38 & 68.51 & 68.82 & 68.90 & 78.07 & 76.71 & 78.07 & 77.86  \\
 \hline
 16 & 69.53 & 67.09 & 69.38 & 68.90 & 77.37 & 77.8 & 77.91 & 77.69  \\
 \hline
\end{tabular}
\caption{\label{tab:mmlu_abalation} Accuracy on LM evaluation harness tasks on Llama2-7B model.}
\end{table}

\begin{table} \centering
\begin{tabular}{|c||c|c|c|c||c|c|c|c|} 
\hline
 $L_b \rightarrow$& \multicolumn{4}{c||}{8} & \multicolumn{4}{c||}{8}\\
 \hline
 \backslashbox{$L_A$\kern-1em}{\kern-1em$N_c$} & 2 & 4 & 8 & 16 & 2 & 4 & 8 & 16  \\
 %$N_c \rightarrow$ & 2 & 4 & 8 & 16 & 2 & 4 & 2 \\
 \hline
 \hline
 \multicolumn{5}{|c|}{Race (FP32 Accuracy = 48.8\%)} & \multicolumn{4}{|c|}{Boolq (FP32 Accuracy = 85.23\%)} \\ 
 \hline
 \hline
 64 & 49.00 & 49.00 & 49.28 & 48.71 & 82.82 & 84.28 & 84.03 & 84.25 \\
 \hline
 32 & 49.57 & 48.52 & 48.33 & 49.28 & 83.85 & 84.46 & 84.31 & 84.93  \\
 \hline
 16 & 49.85 & 49.09 & 49.28 & 48.99 & 85.11 & 84.46 & 84.61 & 83.94  \\
 \hline
 \hline
 \multicolumn{5}{|c|}{Winogrande (FP32 Accuracy = 79.95\%)} & \multicolumn{4}{|c|}{Piqa (FP32 Accuracy = 81.56\%)} \\ 
 \hline
 \hline
 64 & 78.77 & 78.45 & 78.37 & 79.16 & 81.45 & 80.69 & 81.45 & 81.5 \\
 \hline
 32 & 78.45 & 79.01 & 78.69 & 80.66 & 81.56 & 80.58 & 81.18 & 81.34  \\
 \hline
 16 & 79.95 & 79.56 & 79.79 & 79.72 & 81.28 & 81.66 & 81.28 & 80.96  \\
 \hline
\end{tabular}
\caption{\label{tab:mmlu_abalation} Accuracy on LM evaluation harness tasks on Llama2-70B model.}
\end{table}

%\section{MSE Studies}
%\textcolor{red}{TODO}


\subsection{Number Formats and Quantization Method}
\label{subsec:numFormats_quantMethod}
\subsubsection{Integer Format}
An $n$-bit signed integer (INT) is typically represented with a 2s-complement format \citep{yao2022zeroquant,xiao2023smoothquant,dai2021vsq}, where the most significant bit denotes the sign.

\subsubsection{Floating Point Format}
An $n$-bit signed floating point (FP) number $x$ comprises of a 1-bit sign ($x_{\mathrm{sign}}$), $B_m$-bit mantissa ($x_{\mathrm{mant}}$) and $B_e$-bit exponent ($x_{\mathrm{exp}}$) such that $B_m+B_e=n-1$. The associated constant exponent bias ($E_{\mathrm{bias}}$) is computed as $(2^{{B_e}-1}-1)$. We denote this format as $E_{B_e}M_{B_m}$.  

\subsubsection{Quantization Scheme}
\label{subsec:quant_method}
A quantization scheme dictates how a given unquantized tensor is converted to its quantized representation. We consider FP formats for the purpose of illustration. Given an unquantized tensor $\bm{X}$ and an FP format $E_{B_e}M_{B_m}$, we first, we compute the quantization scale factor $s_X$ that maps the maximum absolute value of $\bm{X}$ to the maximum quantization level of the $E_{B_e}M_{B_m}$ format as follows:
\begin{align}
\label{eq:sf}
    s_X = \frac{\mathrm{max}(|\bm{X}|)}{\mathrm{max}(E_{B_e}M_{B_m})}
\end{align}
In the above equation, $|\cdot|$ denotes the absolute value function.

Next, we scale $\bm{X}$ by $s_X$ and quantize it to $\hat{\bm{X}}$ by rounding it to the nearest quantization level of $E_{B_e}M_{B_m}$ as:

\begin{align}
\label{eq:tensor_quant}
    \hat{\bm{X}} = \text{round-to-nearest}\left(\frac{\bm{X}}{s_X}, E_{B_e}M_{B_m}\right)
\end{align}

We perform dynamic max-scaled quantization \citep{wu2020integer}, where the scale factor $s$ for activations is dynamically computed during runtime.

\subsection{Vector Scaled Quantization}
\begin{wrapfigure}{r}{0.35\linewidth}
  \centering
  \includegraphics[width=\linewidth]{sections/figures/vsquant.jpg}
  \caption{\small Vectorwise decomposition for per-vector scaled quantization (VSQ \citep{dai2021vsq}).}
  \label{fig:vsquant}
\end{wrapfigure}
During VSQ \citep{dai2021vsq}, the operand tensors are decomposed into 1D vectors in a hardware friendly manner as shown in Figure \ref{fig:vsquant}. Since the decomposed tensors are used as operands in matrix multiplications during inference, it is beneficial to perform this decomposition along the reduction dimension of the multiplication. The vectorwise quantization is performed similar to tensorwise quantization described in Equations \ref{eq:sf} and \ref{eq:tensor_quant}, where a scale factor $s_v$ is required for each vector $\bm{v}$ that maps the maximum absolute value of that vector to the maximum quantization level. While smaller vector lengths can lead to larger accuracy gains, the associated memory and computational overheads due to the per-vector scale factors increases. To alleviate these overheads, VSQ \citep{dai2021vsq} proposed a second level quantization of the per-vector scale factors to unsigned integers, while MX \citep{rouhani2023shared} quantizes them to integer powers of 2 (denoted as $2^{INT}$).

\subsubsection{MX Format}
The MX format proposed in \citep{rouhani2023microscaling} introduces the concept of sub-block shifting. For every two scalar elements of $b$-bits each, there is a shared exponent bit. The value of this exponent bit is determined through an empirical analysis that targets minimizing quantization MSE. We note that the FP format $E_{1}M_{b}$ is strictly better than MX from an accuracy perspective since it allocates a dedicated exponent bit to each scalar as opposed to sharing it across two scalars. Therefore, we conservatively bound the accuracy of a $b+2$-bit signed MX format with that of a $E_{1}M_{b}$ format in our comparisons. For instance, we use E1M2 format as a proxy for MX4.

\begin{figure}
    \centering
    \includegraphics[width=1\linewidth]{sections//figures/BlockFormats.pdf}
    \caption{\small Comparing LO-BCQ to MX format.}
    \label{fig:block_formats}
\end{figure}

Figure \ref{fig:block_formats} compares our $4$-bit LO-BCQ block format to MX \citep{rouhani2023microscaling}. As shown, both LO-BCQ and MX decompose a given operand tensor into block arrays and each block array into blocks. Similar to MX, we find that per-block quantization ($L_b < L_A$) leads to better accuracy due to increased flexibility. While MX achieves this through per-block $1$-bit micro-scales, we associate a dedicated codebook to each block through a per-block codebook selector. Further, MX quantizes the per-block array scale-factor to E8M0 format without per-tensor scaling. In contrast during LO-BCQ, we find that per-tensor scaling combined with quantization of per-block array scale-factor to E4M3 format results in superior inference accuracy across models. 


\end{document}
