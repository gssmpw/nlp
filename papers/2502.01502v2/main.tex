%%%%%%%%%%%%%%%%%%%%%%%%%%%%%%%%%%%%
% This is the template for submission to ISCA 2025
%%%%%%%%%%%%%%%%%%%%%%%%%%%%%%%%%%%%
\PassOptionsToPackage{letterspace=-20,tracking=true,final}{microtype}
\documentclass[sigconf]{acmart}

%% \BibTeX command to typeset BibTeX logo in the docs
\AtBeginDocument{%
  \providecommand\BibTeX{{%
    \normalfont B\kern-0.5em{\scshape i\kern-0.25em b}\kern-0.8em\TeX}} }
%% Rights management information.
\setcopyright{none} % Camera Ready: \setcopyright{acmcopyright}
\copyrightyear{2025}
\acmYear{2025}
\acmDOI{XX.XXXX}
%% These commands are for a PROCEEDINGS abstract or paper.
\acmConference[arXiv]{February, 2025}{February,2025}{Barcelona, Spain}
\acmISBN{978-1-4503-XXXX-X/18/06}
% % Removes citation information below abstract
\settopmatter{printacmref=false}
% Page Number
\settopmatter{printfolios=true}

% *** CITATION PACKAGES ***
% \usepackage{cite}

% *** MATH PACKAGES ***
% \usepackage{amsmath,amssymb,amsfonts}
\usepackage{mathtools}

% *** SPECIALIZED LIST PACKAGES ***
\usepackage{algorithm}
\usepackage{algpseudocode}
% \usepackage{algorithmic}

% *** GRAPHICS RELATED PACKAGES ***
\usepackage{graphicx}
\usepackage{textcomp}
\usepackage[normalem]{ulem}
\usepackage{color}
\usepackage{xcolor}
\usepackage{caption}
\usepackage{subcaption}

% *** ALIGNMENT PACKAGES ***
\usepackage{array}
\usepackage{booktabs}
\usepackage{colortbl}
\usepackage{multirow}

% BibTeX
\def\BibTeX{{\rm B\kern-.05em{\sc i\kern-.025em b}\kern-.08em
    T\kern-.1667em\lower.7ex\hbox{E}\kern-.125emX}}

% Ensure letter paper
\pdfpagewidth=8.5in
\pdfpageheight=11in


%%%%%%%%%%%---SETME-----%%%%%%%%%%%%%
% \newcommand{\hpcasubmissionnumber}{783}
%%%%%%%%%%%%%%%%%%%%%%%%%%%%%%%%%%%%

\pagenumbering{arabic}

%%%%%%%%%%%---SETME-----%%%%%%%%%%%%%
\title{Hamun: An Approximate Computation Method to Prolong the Lifespan of ReRAM-Based Accelerators}
% \author{\normalsize{HPCA 2025 Submission
%     \textbf{\#\hpcasubmissionnumber} -- Confidential Draft -- Do NOT Distribute!!}}
% \author{\normalsize{ISCA 2025 Submission
%      \textbf{\#783} -- Confidential Draft -- Do NOT Distribute!!}}

\author{Mohammad Sabri, Marc Riera, Antonio González}
% \email{{msabri, mriera, antonio}@ac.upc.edu}
\email{mohammad.sabri@upc.edu, marc.riera.villanueva@upc.edu, antonio@ac.upc.edu}
\affiliation{
  \institution{Universitat Polit\`{e}cnica de Catalunya (UPC)}
  \city{Barcelona}
  \country{Spain}
}
\renewcommand{\shortauthors}{Sabri et al.}

%%%%%%%%%%%%%%%%%%%%%%%%%%%%%%%%%%%%

%%%%%%%%%%%---Make Space-----%%%%%%%%%%%%%
% shrink horizontally (letterspace=-25 suggested by julian)
% \usepackage[letterspace=-25,tracking=true,final]{microtype}
\usepackage{microtype}
%%%%%%%%%%%%%%%%%%%%%%%%%%%%%%%%%%%%

% %For the captions of the tables:
% \usepackage{caption}
% \DeclareCaptionLabelFormat{lc}{\MakeLowercase{#1}~#2}
% %\captionsetup{labelfont=sc,labelformat=lc}
% \captionsetup{labelfont=sc}
% \usepackage[font=footnotesize,labelfont=bf, textfont=bf]{caption}

\begin{document}
% \thispagestyle{plain}
% \pagestyle{plain}

\begin{abstract}
ReRAM-based accelerators exhibit enormous potential to increase computational efficiency for DNN inference tasks, delivering significant performance and energy savings over traditional platforms. By incorporating adaptive scheduling, these accelerators dynamically adjust to DNN requirements, optimizing allocation of constrained hardware resources. However, ReRAM cells have limited endurance cycles due to wear-out from multiple updates for each inference execution, which shortens the lifespan of ReRAM-based accelerators and presents a practical challenge in positioning them as alternatives to conventional platforms like TPUs. Addressing these endurance limitations is essential for making ReRAM-based solutions viable for long-term, high-performance DNN inference.

% Addressing these endurance limitations is essential to fully realize the potential of ReRAM-based accelerators as viable, long-term solutions for high-performance, energy-efficient computation.

To address the lifespan limitations of ReRAM-based accelerators, we introduce \textit{Hamun}, an approximate computing method designed to extend the lifespan of ReRAM-based accelerators through a range of optimizations. Hamun incorporates a novel mechanism that detects faulty cell due to wear-out and retires them, avoiding in this way their otherwise adverse impact on DNN accuracy. Moreover, Hamun extends the lifespan of ReRAM-based accelerators by adapting wear-leveling techniques across various abstraction levels of the accelerator and implementing a batch execution scheme to maximize ReRAM cell usage for multiple inferences. Additionally, Hamun introduces a new approximation method that leverages the fault tolerance characteristics of DNNs to delay the retirement of worn-out cells, reducing the performance penalty of retired  and further extending the accelerator’s lifespan. On average, evaluated on a set of popular DNNs, Hamun demonstrates an improvement in lifespan of $13.2\times$ over a state-of-the-art baseline. The main contributors to this improvement are the fault handling and batch execution schemes, which provide $4.6\times$ and $2.6\times$ lifespan improvements respectively.
\end{abstract}

\maketitle
% \begin{IEEEkeywords}
% Deep Neural Networks (DNNs), Hardware Accelerators, Processing-In-Memory (PIM), ReRAM
% \end{IEEEkeywords}
% }

\textls{
\vspace{-.5cm}
%%%%%% -- PAPER CONTENT STARTS-- %%%%%%%%
\section{Introduction}

In recent years, with advancements in generative models and the expansion of training datasets, text-to-speech (TTS) models \cite{valle, voicebox, ns3} have made breakthrough progress in naturalness and quality, gradually approaching the level of real recordings. However, low-latency and efficient dual-stream TTS, which involves processing streaming text inputs while simultaneously generating speech in real time, remains a challenging problem \cite{livespeech2}. These models are ideal for integration with upstream tasks, such as large language models (LLMs) \cite{gpt4} and streaming translation models \cite{seamless}, which can generate text in a streaming manner. Addressing these challenges can improve live human-computer interaction, paving the way for various applications, such as speech-to-speech translation and personal voice assistants.

Recently, inspired by advances in image generation, denoising diffusion \cite{diffusion, score}, flow matching \cite{fm}, and masked generative models \cite{maskgit} have been introduced into non-autoregressive (NAR) TTS \cite{seedtts, F5tts, pflow, maskgct}, demonstrating impressive performance in offline inference.  During this process, these offline TTS models first add noise or apply masking guided by the predicted duration. Subsequently, context from the entire sentence is leveraged to perform temporally-unordered denoising or mask prediction for speech generation. However, this temporally-unordered process hinders their application to streaming speech generation\footnote{
Here, “temporally” refers to the physical time of audio samples, not the iteration step $t \in [0, 1]$ of the above NAR TTS models.}.


When it comes to streaming speech generation, autoregressive (AR) TTS models \cite{valle, ellav} hold a distinct advantage because of their ability to deliver outputs in a temporally-ordered manner. However, compared to recently proposed NAR TTS models,  AR TTS models have a distinct disadvantage in terms of generation efficiency \cite{MEDUSA}. Specifically, the autoregressive steps are tied to the frame rate of speech tokens, resulting in slower inference speeds.  
While advancements like VALL-E 2 \cite{valle2} have boosted generation efficiency through group code modeling, the challenge remains that the manually set group size is typically small, suggesting room for further improvements. In addition,  most current AR TTS models \cite{dualsteam1} cannot handle stream text input and they only begin streaming speech generation after receiving the complete text,  ignoring the latency caused by the streaming text input. The most closely related works to SyncSpeech are CosyVoice2 \cite{cosyvoice2.0} and IST-LM \cite{yang2024interleaved}, both of which employ interleaved speech-text modeling to accommodate dual-stream scenarios. However, their autoregressive process generates only one speech token per step, leading to low efficiency.



To seamlessly integrate with  upstream LLMs and facilitate dual-stream speech synthesis, this paper introduces \textbf{SyncSpeech}, designed to keep the generation of streaming speech in synchronization with the incoming streaming text. SyncSpeech has the following advantages: 1) \textbf{low latency}, which means it begins generating speech in a streaming manner as soon as the second text token is received,
and
2) \textbf{high efficiency}, 
which means for each arriving text token, only one decoding step is required to generate all the corresponding speech tokens.

SyncSpeech is based on the proposed \textbf{T}emporal \textbf{M}asked generative \textbf{T}ransformer (TMT).
During inference, SyncSpeech adopts the Byte Pair Encoding (BPE) token-level duration prediction, which can access the previously generated speech tokens and performs top-k sampling. 
Subsequently, mask padding and greedy sampling are carried out based on  the duration prediction from the previous step. 

Moreover, sequence input is meticulously constructed to incorporate duration prediction and mask prediction into a single decoding step.
During the training process, we adopt a two-stage training strategy to improve training efficiency and model performance. First, high-efficiency masked pretraining is employed to establish a rough alignment between text and speech tokens within the sequence, followed by fine-tuning the pre-trained model to align with the inference process.

Our experimental results demonstrate that, in terms of generation efficiency, SyncSpeech operates at 6.4 times the speed of the current dual-stream TTS model for English and at 8.5 times the speed for Mandarin. When integrated with LLMs, SyncSpeech achieves latency reductions of 3.2 and 3.8 times, respectively, compared to the current dual-stream TTS model for both languages.
Moreover, with the same scale of training data, SyncSpeech performs comparably to traditional AR models in terms of the quality of generated English speech. For Mandarin, SyncSpeech demonstrates superior quality and robustness compared to current dual-stream TTS models. This showcases the potential of  SyncSpeech as a foundational model to integrate with upstream LLMs.



\section{Background and Related Work}
Our study integrates concepts from multilingual large language models, charitable giving, (persuasive) human-AI co-writing, expected utility theory, and human perceptions of AI-generated content. We present related literature in these realms and we position our work and contributions at their confluence.




\subsection{Multilingual LLMs}
Large Language Models have demonstrated strong multilingual capabilities across numerous tasks \cite{zhao2023survey,bang2023multitask,le2023bloom}, which has led to their widespread proliferation across various countries \cite{Kaddour_2023}. Many multinational companies rely on their translations to accelerate cross-national team cooperation, and individual workers across the globe benefit from individualized LLM assistance that enhances their productivity -- both in their native and the English language. Yet, despite their immense promise, LLMs still differ substantially across languages \cite{Ahuja2023-zu,Zhang2024-mp,huang2023not,bang2023multitask,joshi2020state,jiao2023chatgpt,hada2024akal}. LLMs not only show poorer text generation and problem-solving performance for low-resource languages, but also heightened security vulnerabilities, safety challenges, and tokenizer biases \cite{Shen_2024,ahia2023languagescostsametokenization,Ahuja2023-zu}. 




These caveats are not strictly limited to languages conventionally considered low-resource. Multiple open-source LLMs are disproportionately trained on English data. For example, PaLM2's training corpus consists of 78.99\% English data, compared to only 2.11\% for Spanish data \cite{chowdhery2023palm}, and LLAMA-2 exhibits a mere 0.11\% Spanish representation \cite{touvron2023llama}. \revision{Despite these gaps, Spanish is generally considered a relatively high-resource language. The GPT-4 technical report \cite{achiam2023gpt} shows benchmark accuracies of 85.5\% and 84\% for English and Spanish respectively in the multilingual version of the MMLU \cite{hendrycks2021ethics}. Spanish also performs relatively well in the QWEN2 technical report for professional annotator tasks \cite{yang2024qwen2}.}


\revision{The practical performance of multilingual LLMs in the Spanish language, however, is often relatively poor, especially in contextual usage and practical applications \cite{jin2024better, conde2024opensourceconversationalllms, zhang2024dolares}. A particularly striking finding is highlighted by \citet{conde2024opensourceconversationalllms}, who show that most open-source LLMs exhibit significant comprehension deficiencies for the Spanish vocabulary. Two-thirds of the models, including the Llama-2 series (a predecessor of the Llama 3.1 model used in our experiments), fail to provide valid definitions for more than 50\% of tested Spanish words. Moreover, when evaluated for contextual word usage, most models fall below 10\%. For instance, the Llama-2-7b model correctly defines only 42 out of 100 words and uses just 3 out of 100 words correctly in context. Importantly, these failures are not confined to low-frequency words; even highly frequent words like "minuto" fail in meaning across 8 out of 12 evaluated models. While their work highlights linguistic limitations, our research extends this by exploring the behavioural and real-world impacts of such deficiencies.}
Particularly in co-creation environments that span linguistic and cultural boundaries, varying performance levels can disrupt the collaborative process, leading to asymmetric misunderstandings of system capabilities and thereby in inappropriate reliance that hurts efficiency \cite{Dhillon2024-ol}. This may be particularly problematic in sensitive domains such as persuasive writing, which depend on subtle combinations of tangible and non-tangible elements, like emotional appeals and accurate fact specificities \cite{Hibbert2007-ty,Choi2020-eh,Wymer2023-pg,Dafouz-Milne2008-iy}. 
\revision{However, the ground reality is that multilingual LLMs are largely being deployed and used immaterial of their performance in specific languages. And while companies often do evaluate their models across languages, traditional technical benchmarks may not capture the full picture due to downstream consequences for user behaviour -- even when performance across languages appears comparable. Inspired by this real-world context, we explore the utilization of multilingual LLMs in a co-writing task and study how exposure effects with LLMs in different languages influence user interaction and behavior.}
  

\subsection{Charitable Persuasive Writing}

Persuasive writing is a form of communication that seeks to convince the reader to adopt a particular viewpoint or take a specific action \cite{jonsen2018convincing}. It can be viewed through the lens of sender-receiver games, a concept generated from economic theory frequently applied in computer science across domains such as recommender systems \cite{Apel2020-tt}, reinforcement learning \cite{hiraoka2014reinforcement}, multi-agent interaction \cite{meta2022human}, and most recently LLMs \cite{Shin_Kim_2024}.

Advertisers often leverage this sender-receiver model (e.g., by employing relevant recommender systems) to influence a receiver’s behaviour by appealing to emotions or raising awareness, and charitable advertisement writing is no exception \cite{salvi2024conversational, furumai2024zero,Murphy2001-yn}.

With the rise of LLMs, the traditional dynamic of a human agent influencing a human receiver has evolved. Now, LLM agents can serve as persuasive entities, sometimes more efficiently and effectively than humans \cite{Breum2024-xf, Matz2024-pl, Zeng2024-lp, Xu2023-bs, Goldstein2023-wr,durmus2024measuring}. LLM-generated text has been shown to influence political attitudes \cite{voelkel2023artificial}, vaccine uptake \cite{karinshak2023working}, strategic negotiations \cite{meta2022human}, personal beliefs \cite{salvi2024conversational}, or even romantic conversations \cite{zhou2020design}. However, LLMs also exhibit many limitations -- such as hallucination, a lack of contextual knowledge and wordiness \cite{Myers_2024}--- prompting a shift towards co-creation where human and AI agents collaborate as persuasive agents to ensure both effectiveness and reliability \cite{Dhillon2024-ol, zhang2023human}. Specifically in the context of altruistic social preferences such as charitable giving, little is known about the effectiveness of LLMs in eliciting donations. Even the psychological literature is fragmented, lacking a coherent model about which factors specifically increase the effectiveness of charity advertisement \cite{xu2020relative,saeri2023works,schamp2023effectiveness,fan2020factors}. This makes charitable giving not only a novel but potentially high-value field of application for persuasive LLMs. We, therefore, consider the task of charitable persuasive writing as a lens to study user behavior with multilingual LLMs in this paper.




\subsection{LLM Augmented Co-writing}
There has been considerable interest from the HCI community in analyzing and fostering collaborative human-AI writing. Many popular real-world applications like Microsoft Word and Gmail already utilize smart features that, e.g., predict the next words a user is likely to write or provide context-dependent phrasing advice (auto-completion suggestions). LLMs themselves have demonstrated exceptional performance in open-ended writing, with great potential benefit for a wide variety of tasks \cite{lee2022coauthor,macneil2022automatically,meyer2022we,mirowski2023co,yuan2022wordcraft,zhao2023more,herbold2023ai}. 



Beyond evaluating the output of foundational LLMs, the HCI community has begun to create tools designed to support human writers. \citet{Kim2023-wn} introduce a framework that augments ``object-oriented'' interaction with LLMs. Their approach enables end-users to ``track, compare, and combine'' configurations, fostering inspiration and creativity in the design process. Based on this framework, there has been an ongoing development of novel and innovative systems, focusing on integral parts of the writing process such as non-linearity, rewriting or idea-generation \cite{Reza2023-hp,lu2024corporate, teufelberger2024llm}.

In this paper, we utilize and slightly modify ABScribe, a tool created by \citet{Reza2023-hp}, which allows users to ``track, compare, and modify variations" while interacting directly with an LLM system to generate new ideas during the writing process. The system is purposefully designed to allow for a seamless writing flow by providing users with flexibility and autonomy throughout the co-creating process. Through different functions that go beyond mere text generation, it provides users with a more targeted approach to exploit the various use cases of LLMs. 

The integration of LLMs into the writing process has the potential to transform how content is created across various fields. However, the effectiveness of LLM-augmented co-writing depends on several factors, including the quality of the LLM, the nature of the task, the expertise of the human writer, the interaction of the human writer with the LLM, and the perceptions of consumers towards LLMs. In this paper, we focus on the latter two aspects, as both appropriate reliance and consumer attitudes have been previously shown to be important facets in human-AI interaction  \cite{He_2023,sara_23,schemmer2023appropriate,erlei2024understanding,wester2024exploring,grassini2024understanding}.


\subsection{Choice Independence and Human-AI Interaction}
Our study focuses on choice independence as a particularly important aspect of human-AI interaction.\footnote{The von Neumann-Morgenstern's utility theory (or expected utility theory), provides a key mechanism for understanding the behaviour of a rational agent under uncertainty. In this framework, a rational agent makes decisions that maximize the subjective value of their utility when faced with stochastic outcomes \cite{vnm-original}. The theory is built upon four main axioms: \textit{completeness}, \textit{transitivity}, \textit{independence}, and \textit{continuity}. Among these, the independence axiom is the most contentious and has significant implications for rational decision-making \cite{Holt1986-jp}. 
Mathematically, this axiom is represented as follows,
\[
X \succ Y \implies pX + (1-p)Z \succ pY + (1-p)Z \quad \text{for} \quad 0 < p \leq 1
\]

Here, $X$, $Y$, and $Z$ represent lotteries, which can be thought of as stochastic processes or uncertain events that yield probability distributions over a set of outcomes. If amongst the lotteries $X$ and $Y$ a rational agent is said to prefer
lottery $X$ over $Y$ their preference should remain unchanged even if an irrelevant lottery is introduced and mixed with both $X$ and $Y$ in equal proportion. This axiom asserts the stability of preferences and is considered a cornerstone of rationality in decision theory.} 
In the context of multilingual LLMs, it postulates that users who experience two or more languages should evaluate them independently, adjusting their usage according to their language-specific experiences. Choice independence is a foundational axiom of expected utility theory \cite{vnm-original}, and often implicitly assumed when deploying novel technologies, systems and products \cite{erlei2024understanding,Ethayarajh2022-gj}. We argue that this is one of the reasons why companies tend to simultaneously deploy their AI assistants globally, despite the documented differences in performance across languages. The HCI literature has only recently begun to empirically scrutinize its applicability in the context of algorithmic and AI systems. \cite{pareek2024trust,erlei2024understanding}.



\revision{So far, evidence from HCI work is constrained to a limited number of abstract %laboratory-like 
decision tasks. For example, \citet{erlei2024understanding} uses an online experiment to explore how humans delegate decisions to a superior AI system across two independent abstract prediction tasks. They manipulate the performance of the AI system such that participants either observe an AI system that provides the best-possible prediction in both tasks or one that exhibits a systematic error in one of the two. Results show that the induced error in one task significantly reduces trust in and delegation to the AI system even for the second prediction, indicating that participants erroneously generalize AI errors across tasks, violating choice independence and undermining appropriate reliance. Similarly, \citet{pareek2024trust} conducted an online experiment to study trust dynamics in the context of complementary Human-AI expertise. Participants engage in classification tasks involving familiar (High Human Expertise, HHE) and unfamiliar (Low Human Expertise, LHE) stimuli. The paper shows that people calibrate trust in the AI for LHE tasks based on its performance in HHE tasks, demonstrating a spillover of trust judgments across tasks. While these studies are informative, abstract decision tasks are not only specifically designed to artificially test a specific hypothesis while controlling for the entire context (e.g., expertise, task), but also benefit from very focused attention towards the specific problem a researcher is interested in. Real-world decisions are often more complicated, limiting the extent to which certain laboratory results can be generalized or scaled \cite{list2022voltage,brandon2022human,sara_23,salimzadeh2024dealing}. For example, in \citet{erlei2024understanding}, AI errors are precisely quantifiable and codified, effectively alleviating any participant uncertainty about model performance, and facilitating easy comparisons of the AI's performance across tasks. In real-world settings, many people learn by updating their beliefs solely through experience and noisy feedback, while always being uncertain about the model's ``true'' performance.
Therefore, in this paper, we extend the analysis of human behaviour in the context of heterogeneous AI output across distinct tasks to the applied context of human-AI co-writing}. Multilingual LLMs represent a prime example to test whether humans tend to rationally learn about and evaluate LLMs, because (1) multilingual writing is everywhere, (2) writing and text generation belong to the most common use-cases of LLMs, (3) writing is a complex and non-linear activity for which humans possess intimate familiarity and expertise, and (4) producing persuasive text in one language is distinct from the problem of producing persuasive text in a second language.


\subsection{Human Attitudes Towards AI Generated Content} There has been a lot of 
interdisciplinary literature that has analyzed how humans react to AI-generated output while articulating the notions of algorithmic affinity and aversion~\cite{dietvorst2015algorithm,dietvorst2018overcoming}. Within the scope of this paper, we are primarily interested in textual or persuasive content. In addition, eliciting donations through advertisements is closely related to negotiation scenarios. Recent studies provide mixed results on human perceptions of AI-generated content. In \citet{lim2024effect}, disclosing AI as the source of communication negatively impacts human perceptions of messages. People may prefer AI advertisements depending on which kind of appeal is made \cite{chen2024consumer}, but can react negatively towards AI use by charities \cite{arango2023consumer} and generally appear to denigrate creators who transparently use AI \cite{rae2024effects,bruns2024you}. Other research finds positive effects of revealing the use of AI technology in the context of influencing and persuasion \cite{wang2024positive} and no creator loss in credibility \cite{huschens2023you}. In general, the literature documents several divergent effects, currently lacking a parsimonious explanation \cite{ferraro2024paradoxes}. Interestingly, people often appear unable to identify AI-generated content \cite{clark2021all} and only reveal preferences against it upon disclosure \cite{kobis2021artificial}, possibly due to inherent pro-human attitudes \cite{grassini2024understanding,zhang2023human}. In bargaining and negotiations, humans also tend to exhibit preferences for other humans \cite{erlei2022s}, and behave more self-interested \cite{erlei2020impact,shen2024bargaining,chugunova2022we,von2023social}. We add to this existing bed of literature by examining how peoples' beliefs about the origin of a charitable advertisement affect donation behaviour, and whether these beliefs correlate with true LLM usage.

\section{Hamun Accelerator}\label{s:Accelerator}
In this section, we describe the hardware architecture of our proposed ReRAM-based accelerator, which is organized into multiple hierarchical levels. We then explain the execution flow, detailing the weight-writing and computation procedures.

\subsection{Architecture}\label{subs:Architecture}
Most prior works (e.g.,~\cite{RAELLA, ISAAC}) design accelerators under the assumption that all DNN weights are pre-stored in ReRAM crossbars. This approach demands substantial resources, including large numbers of ReRAM crossbars and on-chip SRAM buffers, making it inefficient for large networks. In contrast, proposals like ARAS~\cite{ARAS} operate with limited resources by adopting a layer-by-layer execution, where the computation of one layer overlaps with the writing of weights for subsequent layer(s). However, the frequent updates to ReRAM cells for executing DNN layers reduce the lifespan of the accelerator due to ReRAM's limited endurance cycles. This is one of the key challenges that Hamun addresses by proposing an approximate computing scheme along with some wear-leveling optimizations.

Figure~\ref{fig:Top_view} presents a top-down view of the Hamun accelerator architecture. Figure~\ref{fig:Top_view}(a) shows a high-level schematic of the chip architecture. A single chip comprises several key components: an External IO Interface (Ext-IO), multiple Processing Elements (PEs), a Special Function Unit (SFU), Accumulation Units (ACC), and a Global Buffer (Gbuffer). The Ext-IO facilitates communication with Main Memory (MM), loading network weights and inputs, and storing the final outputs. The ACC units aggregate partial results from neural operations for a given layer, which may be generated across multiple PEs when a layer spans several PEs due to its size. The SFU handles transitional operations such as pooling, non-linear activations (e.g., sigmoid or ReLU), and normalization, ensuring support for the full range of computations required for state-of-the-art DNNs. The Global Buffer (Gbuffer) acts as a storage unit for intermediate activations produced during the execution of each layer.

\begin{figure}[t!]
    \centering
    \includegraphics[width=0.95\columnwidth]{images/Top_view.pdf}
    \vskip -0.10in
    \caption{Architecture of the Hamun accelerator including the organization of: (a) Chip, (b) Processing Element (PE), (c) Analog Processing Unit (APU), and (d) Transposing Banks.}
    \vskip -0.15in
    \label{fig:Top_view}
\end{figure}

Figure~\ref{fig:Top_view}(b) illustrates the structure of a Processing Element (PE), which consists of $m \times n$ Analog Processing Units (APUs), $m$ buffers to store either input activations or weights, an output buffer for storing partial sums, $n$ accumulation modules to sum the partial outputs from the APUs in each column, shift registers to serialize activations, and a multiplexer that switches between the weight-writing and computation phases. Figure~\ref{fig:Top_view}(c) displays the main components of an Analog Processing Unit (APU), which includes a ReRAM-based crossbar array of 1T1R cells used to store synaptic weights. Each APU also includes an input register, which is used to store activations, a write register to store weights, a Mask register to disable an entire crossbar column in the writing/computing procedure, a WL/BL (Wordline/Bitline) driver to control the wordlines and bitlines, and an SL (Sourceline) driver to generate required voltage pulses in the sourcelines. Moreover, each APU has an analog multiplexer, a shared pool of Analog-to-Digital Converters (ADCs), and functional units for accumulating and shifting the partial results from different bitlines across iterations. Further details on the analog dot-product operation procedure using ReRAM crossbars, as well as the ReRAM cell writing process, can be found in Section~\ref{subs:Execution Dataflow}.

The Gbuffer is divided into different bank groups. While one bank group operates in a standard mode, the other bank group follows a distinct writing procedure that results in a transposed matrix format when reading data. As discussed in Section~\ref{subs:Transformers}, in the attention block, calculating the score matrix requires performing matrix multiplication between the query matrix and the transposed key matrix. This scheme facilitates the transposition of the key matrix in-situ, without introducing any additional latency overhead.

Figure~\ref{fig:Example_In_Transposition} provides a simplified example illustrating our proposed in-situ transposition scheme. In (a) we present an original matrix that needs transposing. The matrix elements are generated sequentially in a row-by-row manner and transferred via the NoC to storage. For this example, the NoC transmits only two elements per transaction. These are then stored in two memory banks for each transfer. Figure~\ref{fig:Example_In_Transposition}(b) depicts the resulting memory layout post-storage. During writing, consecutive elements in a row are distributed across two banks, while consecutive elements in a column are written in the same entry in adjacent banks, enabling the transposed matrix to be retrieved by reading the banks entry-by-entry. In some NoC transactions, element pairs need to be swapped to ensure correct storage order; for instance, elements $(a_{20}, a_{21})$ are swapped before writing, so $a_{20}$ is written into bank 1 and $a_{21}$ into bank 0. Similar swaps occur during reading to ensure accurate transposition during transfer, e.g., when reading $(a_{21}, a_{11})$.

\begin{figure}[t!]
    \centering
    \includegraphics[width=0.9\columnwidth]{images/Example_In_Transposition.pdf}
    \vskip -0.10in
    \caption{(a) Original matrix and (b) Post-storage matrix layout.}
    \vskip -0.25in
    \label{fig:Example_In_Transposition}
\end{figure}

Figure~\ref{fig:Top_view}(d) illustrates our hardware proposal for the transposing bank group, which consists of 16 banks, each with a data width of one byte, and a 16-byte swapping register capable of holding 16 elements of the key matrix. As previously discussed, each attention block in Transformer models includes a FC layer to generate the key matrix. In the Hamun architecture, FC layers operate token-by-token, resulting in a row-by-row generation of the key matrix. Each row is divided into multiple NoC transactions, with each transaction consisting of 16 bytes. These transactions are transferred through the chip's mesh network. Each transaction is further split into 16 distinct elements, and these elements are distributed across the 16 banks in the transposing bank group. The swapping register reorders the elements based on the desired transposed matrix configuration and aligns each elements with its corresponding banks. Equation~\ref{eq:Transposition} (in section~\ref{subs:Transposition}) maps how each element in the original flattened matrix is assigned to its proper position in the transposed matrix. Therefore, to store elements in transposed order, first, the reordering of the elements is executed according to the corresponding bank for each element, as outlined by Equation~\ref{eq:id} below. Next, all elements are simultaneously stored in their respective banks at the entries determined by Equation~\ref{eq:entry}.

% Therefore, here we define Equation~\ref{eq:id} and Equation~\ref{eq:entry} to show how each element is assigned to a specific bank and how it is placed within an entry in that bank.

\vskip -0.10in
\begin{equation}
    id = P(\alpha) \, mod \, (\#banks)
    \label{eq:id}
\end{equation}

\vskip -0.10in
\begin{equation}
    entry = P(\alpha) \///(\#banks)
    \label{eq:entry}
\end{equation}

The function $P(\alpha)$ defines the correct index in the flattened transposed matrix for an element with index $\alpha$ in the original matrix, as detailed in Equation~\ref{eq:Transposition}. Also, the number of banks, represented by $\#banks$, is set to 16 in the transposing bank group. This in-situ approach eliminates the overhead for transposing the key matrix in Transformer DNNs, as the scheduler statically pre-generates the required swap instructions and bank entry mappings based on the key matrix size and number of banks in the transposing bank group.

\subsection{Execution Dataflow}\label{subs:Execution Dataflow}
n this section, we explain the dataflow of the Hamun accelerator, which involves two distinct procedures: writing weights into the ReRAM crossbars and performing dot-product computations. As previously detailed in Section~\ref{subs:Writing}, the dot products in each crossbar are computed between the input vector and all weights stored in each column of the crossbar. Consequently, when a ReRAM cell becomes worn out, the resulting dot product in the column containing the faulty cell will be incorrect. The most effective solution is to retire the faulty cell, which requires disabling the entire column that includes the damaged cell. In Hamun, this process is handled by masking the faulty column when the MUX in each crossbar connects the column to the ADC, avoiding using the faulty column.

% Unlike traditional crossbars where the word-line (WL) enables all cells in a row, in this architecture, the word-line connects all cells in a column (see Figure~\ref{fig:New_WeightUpdate}).

% A significant distinguishing feature of Hamun compared to previous proposals, is its adoption of a column-by-column scheme for weight updates, as opposed to the traditional row-by-row approach.

In this paper, a PE row refers to a group of APUs organized in a single row within a PE and share a common buffer for input activations and weights. The PE row serves as the smallest granularity for mapping a DNN layer. Each layer of the DNN is mapped to at least one PE row, where different output neurons (or kernels in convolutional layers) are assigned across distinct APUs within the row. The computations for the same input activation set, across various output neurons, are carried out simultaneously within the PE row. If the number of output neurons exceeds the capacity of a PE row, or if the size of each output neuron surpasses the number of ReRAM cells in an APU column, the layer mapping extends to additional PE rows. This hierarchical organization ensures efficient utilization of resources while accommodating the diverse computational demands of DNN layers.

\emph{Weight Writing Procedure:} As indicated in Figure~\ref{fig:WeightUpdating}, the ReRAM weight writing procedure follows the row-by-row approach within each crossbar. The weight-writing process for a ReRAM crossbar row begins by fetching the weights from main memory. These weights are transferred through the chip NoC to the corresponding PE, where they are stored directly in the destination buffer, bypassing the shift registers. The target APU then reads these weights from the buffer and loads them into its Writing Registers. Simultaneously, mask signals — one bit per column — are fetched from the host and stored in the Mask Register. Note that mask signals are only fetched during the writing of the first row and reused for subsequent rows. The APU drivers are configured to adjust the weights by applying either increasing or decreasing pulses to the ReRAM cells in two phases (Figure~\ref{fig:WeightUpdating}). The SL drivers generate these pulses with varying amplitudes based on the new weights stored in the Writing Registers and the current cell values in the ReRAM according to P\&V scheme. If the mask signal indicates that the column contains a faulty cell, the SL driver is disabled for that column since that column will be masked in computation stage. Meanwhile, the BL driver controls the polarity of the programming signals, as the weight increase and decrease phases require opposite polarities for proper adjustment of the ReRAM cells. The process is repeated for all rows in the crossbar, while simultaneously other crossbars follow the same procedure. Since all cells in a crossbar row are written concurrently, the row's writing latency is governed by the slowest cell in each phase. In other words, the longest latency cell in a row determines the overall write time for that row. Additionally, Hamun takes into account the main memory bandwidth as a limiting factor for the number of APUs can perform weight updates concurrently.
 % which restricts APUs writing across multiple APUs.
 
In the P\&V scheme, after each programming pulse, the value of each ReRAM cell is read to check whether it has reached the desired state. If a cell fails to change after successive pulses and becomes stuck at a specific value, it is identified as faulty. Whenever a new fault is detected, the accelerator sends a notification to the host, which initiates a routine to isolate the faulty cell that is discussed in more detail in Section~\ref{subs:Scheduler Fault handling}.

% A key distinguishing feature of Hamun, compared to prior proposals, is its use of a column-by-column scheme for weight updates, rather than the conventional row-by-row approach. The weight-writing process for a ReRAM crossbar column begins by fetching the weights from main memory. These weights are transferred through the NoC to the corresponding PE, where they are stored directly in the destination buffer, bypassing the shift registers. The target APU then reads these weights from the buffer and loads them into its Writing Registers. Simultaneously, mask signals—one bit per column—are fetched from the host and stored in the Mask Register. Importantly, mask signals are only fetched during the writing of the first column and reused for subsequent columns. The APU drivers are configured to adjust the weights by applying either increasing or decreasing pulses to the ReRAM cells in two phases (Figure~\ref{fig:New_WeightUpdate}). The BL drivers generate these pulses with varying amplitudes based on the new weights stored in the Writing Registers and the current cell values in the ReRAM. The WL driver activates or deactivates all cells in a column based on the mask signal corresponding to that column. If the mask signal indicates the column contains a faulty cell, the WL driver disables the entire column to prevent further operations. Meanwhile, the SL driver controls the polarity of the programming signals, as the weight increase and decrease phases require opposite polarities for proper adjustment of the ReRAM cells. The process is repeated for all columns in the crossbar, and if a column’s mask signal is disabled, the weight update is skipped to avoid generating erroneous dot products due to faulty cells.

% \begin{figure}[t!]
%     \centering
%     \includegraphics[width=0.8\columnwidth]{images/New_WeightUpdate.pdf}
%     \vskip -0.10in
%     \caption{Hamun adopts a column-by-column weight update mechanism, which enables the retiring of amortized ReRAM cells by simply disabling the corresponding WL, effectively prolonging the operational lifespan of the accelerator.}
%     \vskip -0.15in
%     \label{fig:New_WeightUpdate}
% \end{figure}

\emph{Dot-Product Computations Procedure:} The dot-product computation in Hamun begins by fetching the input activations from on-chip memory (or the main memory in the case of the first layer) and distributing them across the PEs according to the placement of the stored weights. All the weights within a PE row correspond to a single layer and share the same input activation. The activations are then serialized within the PEs, where the PE controller activates the corresponding buffer, and the activations are streamed bit-serially into the buffer. The shift register set, which is responsible to serialize activations, is shared for different PE rows to reduce area overhead. Like other architectures~\cite{ISAAC, PRIME, ReDy}, Hamun iterates over the bits of the activations to perform the dot-product operations (as illustrated in Figure~\ref{fig:CrossBars}(a)). The crossbars compute the partial sums for each activation bit in every iteration, which are then shifted and accumulated across iterations to form the final result. For columns that contain faulty cells, the masking signal is used to disable these columns, preventing incorrect dot-product calculations. The controller, responsible for providing the select signal to the MUX, will skip any column with an active mask signal, ensuring that faulty columns are excluded from the computation process.

In the final iteration, each APU returns its portion of the neural computation, and the complete result for each kernel or filter is computed by aggregating all the APUs' partial sums. Depending on the kernel size and the mapping of the neural network layer, this accumulation can be performed either within the PEs or in the global chip-level accumulator. Once the dot-product calculations are complete, the results are passed through activation, normalization, and pooling functions, after which they are stored in the Global Buffer. The accelerator is then ready to compute the next convolution window in a convolutional layer or process the next token in a Fully Connected layer for tasks like Large Language Models (LLMs).

% 1- scheduler compilation flow and execution scheme
% 2- aligning and retiring a col and mask signal still the other APU canused the masked one.
% 3- overhead in counter and wearout level in scheduler

\section{Hamun Scheduler}\label{s:Scheduler}
Figure~\ref{fig:Compilation} shows the Hamun scheduler, which is divided into multiple main components. The offline scheduler manages both computation and weight-writing tasks by efficiently assigning necessary resources to each task in the Binding procedure, and also orchestrates tasks in the Scheduling procedure. The offline scheduler focuses on two key optimizations aimed at prolonging the accelerator’s lifespan. In contrast, the online component continuously monitors the appearance of worn-out cells and notifies the host of their location. The offline scheduler generates a set of instructions that control both the weight writing procedure and the dot-product computations procedure during each inference. These instructions are then executed by the accelerator for each inference task.

To maintain the \textit{adaptability} of the accelerator as the size of the network grows, Hamun follows a similar execution scheme to those used in previous works~\cite{ARAS, MNEMOSENE}, which emphasize efficient resource utilization. However, Hamun focuses on addressing the primary limitation of these designs, the restricted lifespan of the accelerator due to frequent ReRAM cell updating. Hamun scheduler adopts a systematic layer-by-layer computation strategy and overlaps the computation of the current layer with the writing of weights for subsequent layers. As soon as the computation for a layer is done, the scheduler reallocates the resources used for this computation for writing the weights of subsequent layer(s). Depending on the number of available crossbars, the accelerator can either update the weights for part of a layer, an entire layer, or multiple layers simultaneously.

% Our scheduler can efficiently accommodate different layers of weights within a single PE, yet each row of APUs in a PE belongs to one layer, since the same activations are broadcast to all APUs in the row.

\begin{figure}[t!]
    \centering
    \includegraphics[width=1.0\columnwidth]{images/NewCompilationFlow.pdf}
    \vskip -0.10in
    \caption{Hamun Compilation Flow.}
    \label{fig:Compilation}
    \vskip -0.25in
\end{figure}

Figure~\ref{fig:Execution_Flow} shows an example of the Hamun execution flow for an encoder block commonly used in LLMs. Initially, based on the available accelerator ReRAM crossbar resources, the weights for multiple layers are written into the ReRAM crossbars. In this example, the first three FC layers responsible for generating the Query, Key, and Value matrices, along with the final FC layer in the attention block, are written simultaneously. Hamun performs matrix multiplication using ReRAM crossbars, where one matrix play the role of weights and the other play the role of activations. Once the Query, Key, and Value matrices are produced, the resources used for their respective FC layers are released, and the Key and Value matrices are written into ReRAM memory to perform the matrix multiplication required to get the score matrix. It is important to note that before writing, the Key matrix must be transposed. Hamun utilizes in-situ transposition for this purpose, as described in section~\ref{subs:Architecture}. Once the matrix multiplications are complete, the assigned ReRAM crossbars are released, allowing the weights for the final two feed-forward FC layers to be written. As illustrated, Hamun follows a layer-by-layer computation approach, respecting the data dependencies between layers to ensure correct execution. Moreover, Hamun overlaps the weight writing process with the computation of dot products to effectively hide the latency associated with costly ReRAM writes, optimizing both time and resource utilization.

The key insight in Hamun's offline scheduling and binding process lies in its dual strategy of overlapping layer computations with the writing of weights for the subsequent layers, alongside efficient resource management to maximize parallelism in the weight update process. By executing the computation of one layer while concurrently writing the weights for the next layer(s), Hamun reduces latency and ensures that weight writing doesn't become a bottleneck. Moreover, efficient resource allocation allows multiple weights to be updated simultaneously, which not only increases performance but also ensures faster transitions between layers.

\begin{figure}[t!]
    \centering
    \includegraphics[width=1.0\columnwidth]{images/Scheduler_Execution_Flow.pdf}
    \vskip -0.10in
    \caption{An example of the execution flow in Hamun for an encoder block typically found in LLMs, showcasing its layer-by-layer computation scheme, which efficiently overlaps weight updates with ongoing computations.}
    \label{fig:Execution_Flow}
    \vskip -0.25in
\end{figure}

\subsection{Hamun Fault Handling}\label{subs:Scheduler Fault handling}
In this section, we describe the wear-out fault-handling mechanism employed by Hamun. As mentioned earlier, the Program \& Verify (P\&V) scheme used for updating ReRAM cells can detect cells that become stuck at a specific value (permanent failure) and do not respond to programming pulses. If a ReRAM cell fails due to reaching its endurance limit while being updated, the accelerator immediately alerts the host system via the online monitoring component (Figure~\ref{fig:Compilation}). Hamun's fault-handling routine estimates the impact of this faulty cell in the final accuracy, and if it is above the user-specified threshold it retires the column where the faulty cell resides and re-does the scheduling and binding process to ensure accurate DNN inference despite the fault.

Figure~\ref{fig:Compilation} illustrates the Hamun compilation process when a new cell wears out during inference execution. The process starts with analyzing the PyTorch model to extract the network structure and data dependencies. Following this, the scheduling and binding tasks are performed offline, based on the network structure and its dependencies. Two key optimizations are applied during this phase. The first is network batch execution, where weights are reused across different inferences within a batch, reducing the overhead of multiple memory writing operations for each inference. The second optimization focuses on resource management during the binding process, where resources are prioritized according to their wear level. The wear level is tracked using counters that monitor the frequency of resource usage, this data being stored in the host side to avoid adding extra area overhead to the accelerator. To ensure balanced writing distribution across the ReRAM crossbars, those crossbars that have been utilized less frequently are given higher priority in the binding process to assign new computations to them. This approach optimizes resource usage and prolongs the lifespan of the ReRAM cells by evenly distributing the wear.

After the offline scheduler generates the instructions for executing network inferences, the accelerator proceeds with performing inferences based on this configuration. When a new ReRAM cell wears out, the accelerator notifies the host of the faulty cell's location and the current inference execution stops. At this point, during the "Request" stage, the wear levels of resources are updated based on the number of inferences executed with the current scheduling and binding configuration. A new request is then sent to the scheduler, prompting it to redo the scheduling and binding processes to generate a new configuration that may exclude the worn-out cell while maintaining full network accuracy.

% It's important to note that the affected crossbar may be utilized for multiple layers of computation.

Hamun introduces an approximation optimization that, instead of retiring a faulty cell immediately after its detection, it determines its impact on accuracy together with previously detected and not retired cells. If the collective impact do not significantly degrade accuracy, the current scheduling and binding configuration is maintained, allowing continued operation. However, if the expected accuracy loss exceeds a user-defined acceptable threshold, all faulty cells are retired simultaneously, and a new configuration is computed. This strategy ensures minimal interruptions and maximizes the accelerator's operational lifespan by only retiring cells when absolutely necessary.

When a ReRAM cell becomes worn out, the entire column containing the faulty cell in the crossbar is deactivated by disabling the corresponding mask signal to prevent incorrect dot-product computations. Consequently, during the re-binding process, no weights from any layer are assigned to that specific column. This ensures that the faulty column is not used in future computations. Multiple PE rows are utilized for a layer when the size of the input exceeds the number of cells available in a crossbar column. In such cases, since the final result of the dot-product is obtained by accumulating partial results from all APUs within each PE column, it is crucial that the number of crossbar columns to which the layer's weights are mapped remains consistent across all APUs in a PE column. Therefore, the number of crossbar columns used in each APU is aligned with the minimum number of available columns across all crossbars within a PE column, providing balanced resource utilization and accurate accumulation of results.

At first glance, aligning the number of APUs' columns within each PE column may appear to underutilize the accelerator's resources. However, if the number of columns used in a crossbar is lower than the total number of available columns on that crossbar, this alignment introduces an advantage in the event of a cell in the used columns becoming faulty. In such scenarios, the scheduler can reassign one of the unused columns within the crossbar to replace the faulty one.

This process repeats until the accelerator executes the specified number of inferences set by the user. Throughout the operation, the system continually monitors the health of the ReRAM cells and dynamically updates the scheduler whenever a new worn-out cell is detected.

\subsection{Wear Leveling Techniques}\label{subs:WL}
As illustrated in Figure~\ref{fig:Top_view}, the Hamun system architecture is organized into multiple levels, enabling the implementation of Wear Leveling (WL) techniques in different levels of abstraction. Each PE row includes a counter that tracks the frequency of its use, referred to as the "wear level" of that row. These counters are stored on the host side, and during the binding process, the offline scheduler uses these wear levels to efficiently assign resources for each layer. These counters are updated at "Request" stage according to number of inferences that are executed with the current binding and scheduling configuration.

Hamun employs a counter per PE row rather than per ReRAM crossbar row or per cell for two main reasons. First, having a counter for each individual crossbar cell or row would significantly increase the complexity of the scheduling and binding process, thereby slowing down the offline scheduler's execution time. Managing counters at such a fine-grained level would introduce substantial overhead in updating each cell's usage, especially as faults emerge. Second, Hamun includes a crossbar-level WL technique that eliminates the need to monitor the frequency of cell or row utilization within each crossbar. This approach ensures balanced wear across cells without adding per-cell tracking, streamlining both system efficiency and durability.

The wear level of each PE is determined by the maximum wear level among all of its rows. During the binding process, the scheduler first sorts the available PEs based on their wear level index. Then, for each layer, the scheduler selects the PEs with the lowest wear levels to evenly distribute wear across the system. Similarly, when assigning rows within a PE, Hamun also applies WL techniques, prioritizing rows with lower wear level counters.

As mentioned before, Hamun also employs WL techniques at the crossbar level to enhance the longevity of ReRAM cells. During the weight-writing process in the ReRAM crossbar, cells storing the least significant bits (LSBs) experience a higher frequency of updates compared to those storing the most significant bits (MSBs). To address this imbalance, Hamun utilizes a byte-granularity WL scheme. For each inference, this scheme remaps bit positions to different ReRAM cells in a round robin manner. Figure~\ref{fig:wl}(a) provides an example, in which each cell stores two bits, meaning that four cells are required to store an 8-bit weight. In this example, the two LSB bits of an 8-bit weight are initially stored in \textit{cell 0} of a given set of cells. Then, the WL technique reassigns these bits to \textit{cell 3} in the subsequent inference, so update frequency is distributed evenly across all cells. In addition, this remapping is seamlessly integrated with the dot-product computation by aligning the ADC column sampling order with the remapped bit positions.

% Marc: Try to reorganize this sentence, right now it is a bit confusing.
% When a layer’s weights cannot fully utilize all rows within a subarray, assigning the same physical address sets repeatedly for each inference results in uneven wear across rows.
% subarray
Figure~\ref{fig:wl}(b) illustrates another wear leveling (WL) technique employed by Hamun, designed to evenly distribute wear across ReRAM crossbar rows. When a layer’s weights do not fully occupy all available rows in a crossbar, utilizing the same set of physical address for each inference leads to uneven wear in specific rows. This repeated ReRAM update causes certain rows to undergo more frequent write cycles, leading to premature degradation of those crossbar rows while underutilizing others. To mitigate this, Hamun shifts the starting point of the weight-writing procedure to a different physical address for each inference. As shown in Figure~\ref{fig:wl}(b), for each inference, the starting row of the weight-writing process is incremented by one. This dynamic shifting ensures that all rows wears out at the same pace, extending the life of the ReRAM crossbar. To ensure accuracy in the dot-product computation, the order of activations fetched into the APU’s input register is adapted so that each activation is correctly multiplied by its corresponding weight.

\begin{figure}[t!]
    \centering
    \includegraphics[width=1.0\columnwidth]{images/WL.pdf}
    \vskip -0.1in
    \caption{Wear leveling techniques at the crossbar abstraction level. (a): First scheme remaps bit positions to different ReRAM cells for each inference. (b): Second scheme shifts the starting row of the weight-writing procedure in each crossbar for each inference.}
    \label{fig:wl}
    \vskip -0.20in
\end{figure}

The proposed WL schemes in Hamun aim to evenly distribute write operations across all cells in a ReRAM crossbar. Another related approach, proposed by ARAS~\cite{ARAS}, takes a different route by increasing the similarity across the weights of different layers during inference, which reduces the number of weight updates required. While this method successfully decreases the total weight update frequency, it leaves the "hotspot" issue unresolved, especially in cells storing the LSBs, which still experience most of the updates. However, when Hamun's WL techniques are applied on top of ARAS's scheme, the reduced number of total weight updates results in even greater improvements in the accelerator's lifespan. This is because Hamun's WL scheme ensures that the reduced number of weight updates is more evenly distributed across all cells, addressing the imbalance between LSB and MSB cells, and mitigating the wear hotspot issue. Consequently, the combination of both approaches maximizes the lifespan of ReRAM cells by reducing and evenly distributing the frequency of updates.

\subsection{Batch Execution}\label{subs:Batch}
% Batch execution model
% On-chip memory constraint
The weight updating procedure in ReRAM-based accelerators is a costly process in terms of energy consumption, latency, and its impact on the lifespan of memory cells. To mitigate this cost, Hamun employs a batch execution technique. This method allows the accelerator to execute multiple inferences without reallocating resources or rewriting weights between each inference. Essentially, the weights for a given layer are written into the ReRAM crossbars once, and the same set of weights is reused for consecutive inferences. The scheduler in Hamun still follows the same execution flow as described above, but instead of immediately reallocating resources to the next layer after completing one, it processes multiple inferences using the current layer's weights. While Hamun’s batch execution technique helps reduce the cost of frequent weight updates, it does generate more partial results during the computation process, which increases pressure on the on-chip SRAM memory.

\begin{figure}[t!]
    \centering
    \includegraphics[width=0.95\columnwidth]{images/Batching.pdf}
    \vskip -0.10in
    \caption{Batch Execution with size two. Blue blocks represent computation steps for a new inference, while patterned blue and gray blocks indicate additional writes required by the second inference.}
    \label{fig:Batching}
    \vskip -0.25in
\end{figure}

Figure~\ref{fig:Batching} illustrates the batch execution of the previous example (Figure~\ref{fig:Execution_Flow}). For layers with static weights, such as Fully Connected and Convolutional layers, batch execution allows the reuse of written weights across multiple inferences, with the corresponding partial results stored separately in on-chip memory. This means that once a set of weights is written, they are reused for all computations associated with multiple inferences before being overwritten by new weights. Even in cases where the accelerator only has enough ReRAM storage to accommodate a portion of a layer, all computations for that portion across different inferences are completed before reallocating the ReRAM for the next portion.

In cases where a layer does not have static, pre-known weights such as matrix multiplication in an attention block, batch execution becomes more complex. Specifically, executing that layer for multiple inferences requires a separate weight update for each inference, as one of the operands in the matrix multiplication must be written to ReRAM for every inference (shown in patterned blue and gray color in the figure). This weight updating procedure begins as soon as the operand is computed by the preceding layer. Unlike layers with static weights, where Hamun achieves significant lifespan improvements by reusing the same weights across multiple inferences, non-static layers do not benefit from this optimization.

The number of inferences that can be batched together in Hamun depends on the available on-chip SRAM memory. A larger batch size leads to more inferences being processed simultaneously, which generates more partial results, requiring additional on-chip memory for storage. To optimize this, Hamun employs an offline iterative procedure to determine the maximum number of DNN inferences that can be included in a batch. In each iteration, the procedure increments the batch size and calculates the required on-chip memory. The process continues until the memory requirement exceeds the accelerator's on-chip memory capacity. Once the optimal batch size is identified, Hamun sets this value for its offline scheduler, which then generates the appropriate instructions for executing the batch. This approach maximizes the number of inferences processed in parallel, increasing the reuse of written weights in the ReRAM crossbar. As a result, it significantly improves the lifespan of the accelerator by allowing it to execute a greater number of DNN inferences.

\subsection{Approximate Computing}\label{subs:Approximation}
The inherent fault tolerance of deep neural networks (DNNs) allows them to accommodate some inaccuracies in their computations without significantly affecting performance. This resilience, stemming from the redundant and distributed nature of DNN learning processes, enables DNNs to tolerate occasional weight errors. Such fault tolerance is particularly advantageous in ReRAM-based accelerators, where cell wear-related inaccuracies can often be tolerated without requiring immediate retirement of faulty cells. This tolerance can extend the accelerator’s lifespan by reducing the need for frequent reconfiguration.

% Because each network’s fault tolerance varies according to its unique learning procedures, Hamun leverages this property to delay the retirement of faulty cells until their effects surpass the network’s tolerance threshold. To determine each network’s specific tolerance to faults, Hamun employs an offline procedure to estimate the acceptable level of inaccuracy based on user-defined requirements. This process identifies the maximum number of faults that can affect layer weights before impacting overall performance.

As detailed in Section~\ref{subs:Scheduler Fault handling}, when a ReRAM cell wears out, the scheduler notifies the host via the online monitoring component. Depending on the DNN layer size and binding configuration, a worn-out cell can impact one or multiple layers or even introduce multiple faults within a single layer. To extend the accelerator’s lifespan, Hamun introduces a fault-tolerant strategy that optimizes fault handling. Rather than retiring each faulty cell immediately, Hamun assesses the fault’s impact on inference accuracy. If the accuracy loss is greater than a user-defined threshold, the faulty cell is retired, and a new scheduling and binding configuration is computed. Otherwise, the retirement is deferred, allowing the accelerator to continue with the existing configuration.

As additional cells wear out, Hamun repeats this process by evaluating the cumulative effect of these multiple faults. This iterative assessment continues until the accumulated faults degrade accuracy beyond the acceptable threshold, at which point all faulty cells are retired, and the system is reconfigured to maintain accuracy. This approach minimizes interruptions by retiring cells only when absolutely necessary, thereby maximizing the operational lifespan of the accelerator.

To gauge the impact of faulty cells on accuracy, Hamun employs an offline estimation method to ensure that accuracy loss remains within the acceptable threshold. During this process, Hamun randomly introduces an equal number of faults across all layers and evaluates the inference accuracy using the entire validation or test dataset. These faults are applied to random weights and random bit positions within those weights, ensuring a general assessment of network robustness. Moreover, to evaluate diverse fault scenarios, for each inference in the validation or test dataset, a different set of random faults is imposed.

Through a iterative process, Hamun progressively increases the fault count uniformly across all layers until accuracy degradation exceeds the user-defined threshold. This provides an \textbf{optimal fault tolerance threshold}, representing the network’s resilience to inaccuracies. This threshold is used at runtime to guide decisions on whether to postpone retirement of faulty cells without compromising network accuracy.

At runtime, when a fault occurs, Hamun’s host system identifies the affected layer(s) based on the fault’s location. To evaluate potential accuracy loss from multiple faults, Hamun tracks the number of weights impacted within each layer. If this number surpasses the \textbf{fault tolerance threshold} for any layer, Hamun flags the accuracy impact as unacceptable. In such cases, all faulty cells are retired, and the system redoes the binding and scheduling to maintain inference accuracy. This method ensures that accuracy loss remains within the user-specified limits, optimizing both performance and lifespan.

% Baseline
% Modeling
% Endurance
% Benchmarks

\section{Methodology}\label{s:Methodology}
We developed an event-driven simulator to accurately model the lifespan of Hamun, and compare it against ARAS~\cite{ARAS}, which serves as our baseline. Lifespan is defined as the number of inferences a ReRAM-based accelerator can perform while maintaining an acceptable level of throughput. In ARAS, the simulation ends as soon as any ReRAM cell reaches a wear-out point, as it lacks a fault-handling mechanism to deal with faulty cells. In contrast, Hamun is designed with a fault-tolerant approach that allows it to continue functioning even when individual cells wear out. It maintains operation until throughput degradation reaches a user-defined threshold, allowing for greater longevity and resilience in performance. For this simulation, we set the acceptable throughput drop to $40\%$ of the maximum accelerator throughput.

The evaluation of area, latency, and energy consumption for the proposed ReRAM-based accelerator leverages a multi-tool methodology for detailed component modeling. ReRAM crossbars are simulated using NeuroSim~\cite{Neurosim_github}. For on-chip buffers, CACTI-P~\cite{cacti-p} is used, and logic components, such as control and computation units, are implemented in Verilog and synthesized using Synopsys Design Compiler~\cite{Design_compiler} with a 28/32nm technology library. Main memory is assumed to be an LPDDR4 module with 8 GB capacity and 19.2 GB/s bandwidth (single-channel) and is simulated using DRAMSim3~\cite{DRAMsim3}.

To evaluate the accelerator's lifespan, each ReRAM cell is initialized with a specific endurance (number of writes before wear-out). The initial endurance of the cells follow a normal distribution, as established in \cite{Realizing}, with a Coefficient of Variation (CoV) of $0.2$ and a mean value of $2.5\times10^9$.

For a fair comparison with the ARAS baseline, we configured the accelerator with similar parameters, as outlined in Table~\ref{tab:Param}. Specifically, we set the accelerator to include 64 PEs, each composed of 6x4 APUs. Each APU features a crossbar array of $128\times128$ ReRAM cells, and each ReRAM cell has a 2-bit storage resolution. Consequently, representing an 8-bit weight requires four consecutive ReRAM cells. Furthermore, the batch size, or the number of inferences processed concurrently, depends on the available on-chip SRAM memory as explained above. To consider adequate on-chip memory capacity and enable fair comparison, we configure this parameter at 8 MB, in line with the Google Edge TPU’s on-chip SRAM size.

\begin{table}[t!]
\caption{Hamun accelerator configuration parameters.}
\label{tab:Param}
\centering
\resizebox{0.6\columnwidth}{!}{%
    \centering
    \begin{tabular}{|c|c|}
    \hline
    \cellcolor[gray]{0.9} Technology & 32 nm \\
    \cellcolor[gray]{0.9} Frequency & 1 GHz \\ 
    \cellcolor[gray]{0.9} Number of ADCs per APUs & 16 \\
    \cellcolor[gray]{0.9} ADCs Sampling Precision & 6-bits \\
    \cellcolor[gray]{0.9} PE Buffers Size & 1.5 KB \\
    \cellcolor[gray]{0.9} Crossbar Computation Latency & 96 Cycles \\
    \cellcolor[gray]{0.9} Crossbar Row Writing Latency & 6000 Cycles \\
    \hline
    \end{tabular}%
}
\vskip -0.20in
\end{table}

We evaluate our scheme on three prominent DNNs with distinct architectures and applications: Vision Transformer (ViT)~\cite{ViT} for image classification, BERT~\cite{BERT} for question-answering, and GPT-2~\cite{GPT2} for text classification. The Vision Transformer is assessed on the ImageNet~\cite{ImageNet} dataset, which is widely used in image classification tasks. BERT is evaluated on the SQuAD v1.1~\cite{SQuAD} dataset, a benchmark for question-answering tasks that requires the model to identify answers within a context passage. GPT-2, a decoder-based language model, is tested on the IMDB~\cite{imdb} dataset for text classification, specifically for sentiment analysis. Both BERT and ViT contain 12 encoder blocks, while GPT-2 is composed of 12 decoder blocks. The INT8 accuracy (F1) for GPT-2 and BERT are $91.37\%$ and $79.73\%$ respectively, while the accuracy (Top-1) for ViT is $80.96\%$. In the Hamun approximation scheme, the maximum acceptable accuracy loss is set at $1\%$, although this threshold can be adjusted according to user requirements.

\section{Results}\label{sec:results}

We focus our analysis on the L2 processor design and performance.
For the evaluations, HITNet \cite{hitnet} has been chosen as a representative of the latest stereo depth algorithms.
For comparison with the L2 processor, we evaluate HITNet on the Jetson Orin Nano, which is a representative off-the-shelf mobile compute system.
A public implementation of HITNet is used \cite{tinyhitnet}, and it is compiled on a per-ROI size basis using ONNX and TensorRT. This approach provides optimistic estimates for the device's performance.
To measure system power, we utilize Jetson Stats and isolate the power consumption of the GPU and CPU for comparison.

Our system simulator is made of multiple parts. 
Firstly, we implement a counter-based architecture model to estimate the L2 processor performance.
The dynamic and static energy of PE, SCU, \ca{and} SRAM I/O are estimated based on post-APR simulation using the TSMC 28nm PDK and 16-bit operations.
This L2 simulator also accounts for DRAM I/O \cite{lpddr4x, lpddr5_est} and NoC \cite{ansa} energy and latency.
% In our energy estimation, we take into account all parts of the system including DRAM I/O \cite{lpddr4x, lpddr5_est}, NoC \cite{ansa}, sensor \cite{gs_cis1}, uTSV, and MIPI interface \cite{gomez_distributed_2022}.
We base the L1 energy and latency on \cite{marsellus} and estimate the energy required for object detection on images of size $384\times1280$.
We also evaluate system level sensor \cite{gs_cis1}, uTSV, and MIPI interface \cite{gomez_distributed_2022} energy.
This complete simulator is used to conduct design space sweeps of the \projname{} system running HITNet for stereo depth and TinyYOLOv3 for object detection, according to Section ~\ref{sec:mapping}.

\subsection{Ablation Studies}

% \begin{figure}
%     \centering
%     \includegraphics[width=0.8\linewidth]{Figures/ResultAxisAblation_2024_07_04.pdf}
%     \caption{Binning with each design space axis disabled.}
%     \label{fig:result_axis_ablation}
% \end{figure}

% \textbf{Design Space Axes}: In Fig.~\ref{fig:result_axis_ablation}, we evaluate binnings generated by disabling each of the design axes from Section ~\ref{subsec:phase1}, and compare these to the ones generated with all axes enabled.
% For example, disabling the processor utilization axis forces all bins to use one uniform processor utilization, i.e., the entire L2 processor without power gating.
% By comparing the uniform processor utilization curve with the independent bins curve, one can recognize the significance of disabling the processor utilization axis.
% From the comparisons in Fig.~\ref{fig:result_axis_ablation}, we can see that the design-time ROI axis has the least impact on energy.
% The DRAM modes axis is situationally important.
% Large processors have sufficient on-chip SRAM and do not need DRAM I/O caching.
% On the other hand, for very small processors, other mapping inefficiencies dominate the energy for average ROI sizes.
% However, medium-sized processors rely on DRAM caching to support large ROIs, but suffer up to $3.15\times$ inefficiency by using it for average ROIs.
% Finally, the processor utilization axis is the most significant axis across processor sizes, showing that the ability to operate efficiently on small and large ROI sizes by tuning processor utilization within the same silicon is critical for this system design.

\begin{figure}
    \centering
    \includegraphics[width=0.8\linewidth]{Figures/ResultBinCountAblation_2024_07_04.pdf}
    \caption{Effect of number of bins. Increasing bin count results in marginal gains, with highest energy savings on intermediate sized processors.}
    \label{fig:result_bincount_ablation}
\end{figure}

\textbf{Bin Count}: In Fig.~\ref{fig:result_bincount_ablation}, we evaluate the results using different number of bins, which corresponds to the number of runtime ROI intervals.
Moving from one to two bins, the processor can use different mappings for large and small ROI sizes, optimized for different objectives.
This results in a significant gain in energy.
However, as the number of bins increases beyond 2 bins, the marginal gain per additional bin diminishes and varies slightly across processor areas.
In some cases, adding an extra bin can still be beneficial to better adapt to the specific ROI distribution being used.

\begin{figure}
    \centering
    \includegraphics[width=0.8\linewidth]{Figures/ResultDistributionCompare_2024_07_04.pdf}
    \caption{Results of different ROI distributions. The pareto curve of an ROI distribution is dictated primarily by the ROI mean, though variance also plays a minor role.}
    \label{fig:result_distribution_compare}
\end{figure}

\textbf{ROI Distributions}: We compare results for multiple ROI probability distributions to evaluate the generality of our design methodology, as seen in Fig.~\ref{fig:result_distribution_compare}.
Interestingly, the average energy required to run the various ROI distributions is ordered in the same sequence as their mean ROI size, as reported in Fig.~\ref{fig:roi_distribution}.
This suggests that our design method minimizes nonlinear overheads caused by the variable ROI sizes.
Even high variance bimodal distributions, such as ``Vegetables'' \cite{epic_kitchens}, can be efficiently handled by adequately parameterized ROI binnings on \projname{}.

\subsection{Design Benchmarking}

Next, we compare designs generated by this methodology, with existing edge system and baseline ASIC designs.

\begin{figure}
    \centering
    \includegraphics[width=0.8\linewidth]{Figures/ResultJetsonCompare_2024_07_04.pdf}
    \caption{Energy consumption of Jetson Orin Nano running different ROI sizes and \projname{} systems optimized for different ROI distributions. \projname{} achieves superior granularity in energy and lower overall energy.}
    \label{fig:jetson_compare}
    % Distribution: KITTI
    %    Config: ((4, 16), 16, 4)
    %    SRAM Dictionary: {'core_v0': 164864.0, 'core_v1': 123648.0, 'vmm_vec': 61632.0, 'vmm_mat': 248112.0}
    % Distribution: Aubergine
    %    Config: ((8, 16), 6, 2)
    %    SRAM Dictionary: {'core_v0': 245760.0, 'core_v1': 491520.0, 'vmm_vec': 256000.0, 'vmm_mat': 1024000.0}
    % Distribution: Olive
    %    Config: ((4, 16), 6, 2)
    %    SRAM Dictionary: {'core_v0': 1310720.0, 'core_v1': 491520.0, 'vmm_vec': 109568.0, 'vmm_mat': 1382400.0}
    % Distribution: Chopping Board
    %    Config: ((16, 16), 4, 1)
    %    SRAM Dictionary: {'core_v0': 370944.0, 'core_v1': 491520.0, 'vmm_vec': 1966080.0, 'vmm_mat': 4748928.0}
    % Distribution: Left Hand
    %    Config: ((16, 16), 4, 1)
    %    SRAM Dictionary: {'core_v0': 513280.0, 'core_v1': 491520.0, 'vmm_vec': 1966080.0, 'vmm_mat': 4799664.0}
    % Distribution: Pan
    %    Config: ((16, 16), 4, 1)
    %    SRAM Dictionary: {'core_v0': 368640.0, 'core_v1': 491520.0, 'vmm_vec': 1966080.0, 'vmm_mat': 4748928.0}
    % Distribution: Vegetable
    %    Config: ((8, 16), 4, 1)
    %    SRAM Dictionary: {'core_v0': 1966080.0, 'core_v1': 491520.0, 'vmm_vec': 61440.0, 'vmm_mat': 4055040.0}
\end{figure}

\textbf{Comparison with Jetson Orin Nano}: 
In Fig.~\ref{fig:jetson_compare}, we compare the per-ROI energy of the Jetson Orin Nano with \projname{} systems optimized for different ROI distributions.
The \projname{} designs demonstrate two prominent advantages.
Firstly, a \projname{} design can be optimized based on the ROI distribution, whereas the Jetson Orin Nano requires statically compiled binaries for each ROI size in the distribution, which makes it less practical. 
Secondly, 
\projname{} processor dynamically scales performance and energy according to the size of ROI being processed.
In contrast, the Jetson Orin Nano appears to suffer from a coarse-grained reconfigurability of its tensor cores, resulting in an energy pattern characterized by stair steps\ca{; compute latency also suffers, and the Jetson Orin Nano does not exceed 15 FPS operation.}
\projname{}, on the other hand, uses tiles, PEs and SCUs to enable fine-grained optimization.

\begin{figure}
    \centering
    \includegraphics[width=0.9\linewidth]{Figures/ResultBinningBreakdown_2024_07_11_KITTI.NORM.pdf}
    \caption{Breakdown of \projname{} L2 processor energy by ROI size. Dashed lines represent the boundaries of mapping bins.}
    \label{fig:binning_breakdown}
    % 29 January 2025
    % 4 Tiles, 16 PEs / Tile, 1 Core / Tile
    % 16 Vectors, 4 Long / PE
    % 90kB Vector1 SRAM / Core
    % 480kB Vector2 SRAM / Core
    % 18kB Vector SRAM / PE
    % 240kB Matrix SRAM / PE
\end{figure}

\textbf{Energy Breakdown by ROI Size}: We analyze the breakdown of energy by ROI size in a \projname{} L2 processor, as illustrated in Fig.~\ref{fig:binning_breakdown}.
The energy consumption is primarily influenced by static power and DRAM access, with their proportions varying accordingly.
For very small ROIs, static power draw is dominant. 
For extremely large ROIs, the energy is dominated by DRAM I/O.
DRAM I/O is used to minimize the L2 SRAM and reduce static power for small ROIs.
This insight underscores the challenge of optimizing energy across ROI distribution in the L2 processor design. 

\begin{figure}
    \centering
    \includegraphics[width=0.9\linewidth]{Figures/Piechart_2024_07_11.pdf}
    \caption{Comparison of energy in baseline design (no ROI exploitation) with \projname{} design, assuming object detection runs every 5 frames and KITTI ROI distribution.}
    \label{fig:baseline_compare}
    % 29 January 2025
    % Naive Pipeline:
    %   8 Tiles, 16 PEs / Tile, 1 Core / Tile
    %   16 Vectors, 4 Long / PE
    %   960kB Vector1 SRAM / Core
    %   240kB Vector2 SRAM / Core
    %   16kB Vector SRAM / PE
    %   600kB Matrix SRAM / PE
    % ROI Sparsity Exploitation:
    %   4 Tiles, 16 PEs / Tile, 1 Core / Tile
    %   16 Vectors, 4 Long / PE
    %   480kB Vector1 SRAM / Core
    %   241.5kB Vector2 SRAM / Core
    %   288kB Vector SRAM / PE
    %   487.5kB Matrix SRAM / PE
\end{figure}

\textbf{Comparison with Baseline System}: In Fig.~\ref{fig:baseline_compare}, we compare a baseline system with no ROI or temporal sparsity with a \projname{} system.
In this comparison, we optimize both processors to have an area under 100 mm\textsuperscript{2} and a frame rate of at least 30 FPS.
The exploitation of ROI requires object detection and object tracking.
Energy savings are limited by these two costs, which do not scale with the ROI.
Nevertheless, \projname{} still achieves \textbf{$4.35\times$} per-inference energy savings by effectively leveraging ROI-based processing.

\section{Conclusions}\label{s:Conclusion}
In this paper, we demonstrate that frequent ReRAM cell updates needed for DNN inference significantly shorten the lifespan of ReRAM-based accelerators due to the limited endurance cycles of ReRAM cells. To address this challenge, we introduce \textit{Hamun}, an approximate computation method designed to extend the lifespan of ReRAM-based accelerators through multiple optimizations. Hamun features a novel fault-handling scheme that identifies worn-out cells and retire them to prevent their impact on DNN accuracy. Additionally, Hamun employs wear-leveling and batch execution techniques to further increase longevity. To reduce the overhead of retiring cells, Hamun also incorporates an approximation method, thereby extending the accelerator’s lifespan with minimal degradation in DNN accuracy. Across a set of popular DNNs, Hamun achieves a $13.2\times$ lifespan improvement over the baseline on average, highlighting its potential in making ReRAM-based accelerators more viable for long-term use.

\section{Acknowledgments}\label{s:Acknowledgments}

This work has been supported by the CoCoUnit ERC Advanced Grant of the EU’s Horizon 2020 program (grant No 833057), the Spanish State Research Agency (MCIN/AEI) under grant PID2020-113172RB-I00, the Catalan Agency for University and Research (AGAUR) under grant 2021SGR00383, and the ICREA Academia program. 
}
%%%%%%% -- PAPER CONTENT ENDS -- %%%%%%%%

%%%%%%%%% -- BIB STYLE AND FILE -- %%%%%%%%
\bibliographystyle{ACM-Reference-Format}
\bibliography{refs}
%%%%%%%%%%%%%%%%%%%%%%%%%%%%%%%%%%%%

\end{document}
\endinput