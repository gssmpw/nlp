


\begin{table}[!t]
  \centering
  \setlength{\tabcolsep}{2.5pt}
  \scalebox{0.8}{
    \begin{tabular}{l|c|cccccc|c}
      \toprule
      {\textbf{Method}} & {\textbf{KV}} & \multicolumn{7}{c}{\textbf{LongBench}} \\
      \cline{3-9}
       & \textbf{Budget} & SQA & MQA & Sum & FSL & Ret & Cod & \textbf{Avg.} \\
      \midrule
      % InfLLM & 2k & 38.5 & 36.9 & 27.0 & 69.0 & 84.0 & 53.2 & 47.0 \\
      TSLLM & 2.5k & 37.0 & 35.4 & 28.3 & \textbf{67.3} & 87.0 & 51.2 & 46.7 \\
      EMLLM & 8k & 39.3 & 37.7 & 27.0 & \textbf{69.2} & 87.5 & 50.3 & 47.2 \\
      w/ APQ & 2k & \textbf{40.3} & 40.7 & \textbf{27.5} & 63.1 & \textbf{98.0} & 61.5 & 48.8 \\
      w/ DCM & 2k & 39.7 & 42.1 & 27.4 & 64.3 & 94.5 & 61.7 & 49.2 \\
      \name & 2k & 39.4 & \textbf{42.3} & 27.4 & 65.1 & 94.5 & \textbf{62.0} & \textbf{49.4} \\
      \bottomrule
    \end{tabular}
}
\caption{
    \label{tab:ablation}
    The ablation study of our method \name, where \textbf{A}ctivated \textbf{P}robe-\textbf{Q}uery for KV matching and \textbf{D}ynamic \textbf{C}ut-off \textbf{M}echanism  for KV recall. We use the mean pooling to represent \pq in w/ APQ as same as InfLLM and QLLM.
}
%\vspace{-1em}
\end{table}
