\section{Literature Review}
\label{2sec:lit}

Discrete choice modeling is a vast research area that aims to predict customer choices given different sets of alternatives. A common approach to discrete choice modeling suggests that each customer selects at most one alternative. This applies to the multinomial logit model (MNL) introduced by \cite{Luce1959}, which is arguably the most prominent discrete choice model. Despite being widely adopted in practice, choice models that assume a single-item customer shopping behavior are less suitable for industries where customers purchase products primarily in baskets. As a result, a number of models of customers' basket shopping behavior have been developed.
These models can be divided into three categories: Multi-purchase choice models (sometimes referred to as menu or subset selection choice models), multiple-discrete choice models, and multiple discrete-continuous choice models. In multi-purchase choice models, several alternatives can be chosen at the same time, but at most one unit of each alternative can be selected.
Multiple-discrete choice models allow an integer number of units of each of the selected alternatives to be chosen. Finally, in multiple discrete-continuous choice models, noninteger amounts of several alternatives can be selected simultaneously. 
Our focus is on the setting in which customers purchase products in sets, i.e., where each shopping basket consists of unique products. This assumption is fairly justified for many retail industries such as consumer electronics, fashion, toys, etc. It also applies if the goal is to study product categories rather than products themselves. Therefore, theory on multiple-discrete and multiple discrete-continuous choice models is less relevant to our research. We refer the reader to the notable works of \cite{Hendel1}, \cite{Dube1}, \cite{Kim1}, \cite{bhat1} and \cite{bhat2} in those fields.



With regard to multi-purchase choice modeling, one of the most prevalent models is the multivariate logit model (MVL). \cite{Hruschka1} used the MVL to analyze how cross-category sales promotion effects influence purchase probabilities.
The theoretical justification for the MVL was developed by \cite{Russell1}, who derived basket choice probabilities from conditional probabilities of purchasing each product given the purchase decisions related to all
other products. The authors showed that the only joint distribution that is consistent with the specified conditional choice probabilities is the multivariate logistic distribution \citep{Cox1}. \cite{Russell1} applied the MVL to analyze basket purchases in a fairly small setting with four product categories. Subsequently, \cite{Boztug1} proposed a way of applying the MVL in settings of a larger scale. In their paper, the authors first determine prototypical baskets that are used for segmentation of the customer base and then estimate a separate MVL for each customer segment. 
This approach also makes it possible to account for customer heterogeneity by considering a mixture of MVL models. 
\cite{Song1} presented the idea of leveraging the MVL framework to model customer choices across different product categories assuming that customers choose at most one product within each category (i.e., products in one category are considered to be strong substitutes). Such an extension of the MVL is often referred to as the multivariate MNL, or MVMNL (see, e.g., \citealp{Chen1} and \citealp{Jasin1}). The MVL can thus be viewed as a special case of the MVMNL where each product category contains a single product. It can, however, model an arbitrarily strong substitution effect between any two products by assigning a sufficiently large negative value to the parameter representing the pairwise demand dependency between these products (see Section~\ref{2sec:ising_intro} for parameter definition). An interesting model similar to the MVL was proposed by \cite{Benson1}. In their work,
the utility of each basket equals the sum of utilities of individual products in that basket plus an optional correction term. 
The authors proved that the problem of determining the optimal set of baskets receiving corrective utilities is NP-hard and developed several heuristic algorithms for finding such sets. Overall, the MVL and MVMNL models have been applied in various contexts such as recommendation systems \citep{Moon1} and pricing under competition \citep{Richards1} (see \cite{Jasin1} for a brief overview of different application areas of these models). 

Another classic approach to multi-purchase choice modeling is based on the multivariate probit model (MVP) introduced by \cite{Manchanda1}. In the MVP framework, vectors of unobserved parts of product utilities are assumed to follow the multivariate normal distribution. 
The MVP makes it possible to capture the correlation of customer product preferences across different baskets. However, this modeling approach does not consider the complementarity and substitution effects occurring within one purchased basket. In other words, unlike the MVL, the MVP does not account for the fact that purchasing one product may change the marginal value of adding other products to the same basket \citep{Kamakura1}. 




To the best of our knowledge, there are only three papers that address assortment optimization under the MVL model and its variations. \cite{Tulabandhula1} studied the assortment optimization problem under the BundleMVL-K model -- a version of the MVL in which the size of each basket is restricted by an exogenously given constant $K$. The authors focused on a setting in which each basket consists of at most two products, i.e., $K=2$. They showed that the decision version of the assortment problem under this model is NP-complete. This powerful result might seem somewhat counterintuitive. Indeed, the BundleMVL-2 specified for $n$ products can be viewed as the MNL defined over the set of $n^2$ basket alternatives, meaning that the optimal set of baskets (which is generally not translated into the optimal set of products) can be found in polynomial time. The authors proposed a binary search-based heuristic algorithm to solve the assortment problem and compared it against a mixed-integer programming benchmark, as well as two other heuristic algorithms: a greedy approach and the revenue-ordered heuristic (see, e.g., \citealp{Rusmevichientong1}). 
 
 \cite{Chen1} studied the assortment optimization problem under the MVMNL with two product categories, i.e., where the size of each basket is at most two. However, in contrast to the work of \cite{Tulabandhula1}, \cite{Chen1} explicitly separated the product portfolio into two disjoint product categories. The authors proved that the assortment optimization problem in this setting is strongly NP-hard. They proposed the concept of adjusted-revenue-ordered assortments and showed that the assortment with the highest revenue provides a $0.5$-approximation. 
They also developed a $0.74$-approximation algorithm based on a linear programming relaxation of the assortment optimization problem. Lastly, they considered three extensions of the original setting: with capacity constraints, with generally defined basket prices, and with three product categories. They proved that the assortment optimization problems in all these settings do not admit constant-factor approximation algorithms assuming the Exponential Time Hypothesis.

More recently, \cite{Jasin1} addressed the assortment optimization problem under the MVMNL assuming that the parameters representing cross-category demand dependencies are the same for all products in one category. The authors showed that 
 the decision version of the assortment optimization problem is NP-hard even if each category comprises no more than two products and there are no cross-category interaction terms (i.e., the utility of each basket is the sum of utilities of the corresponding product categories). They developed a fully polynomial-time approximation scheme (FPTAS) that can be used to solve this optimization problem. The authors also proposed the generalization of the FPTAS for the case with nonzero cross-category interactions assuming that the maximum number of interacting categories is two. Finally, they considered a setting with capacity constraints and developed an FPTAS for the capacitated assortment problem by building on the ideas presented for the uncapacitated case. 




As detailed in Section~\ref{2sec:intro}, we advance existing research on assortment optimization under basket shopping behavior by adopting methodologies and theoretical results developed for the Ising model. The necessary theoretical background related to the Ising model is provided in Sections~\ref{2subsec:mrf_background} and \ref{2sec:estim}.



\vspace{-0.04cm}