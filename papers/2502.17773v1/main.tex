\pdfoutput=1

\documentclass[english,11pt]{article}
\newcommand{\thought}[1]{{\color[rgb]{0.2,0.39,0.66}(#1)}}
\newcommand{\todo}[1]{{\color[rgb]{1.0,0.0,0.0}(#1)}}
\newcommand{\hsh}[1]{{\color{green!50!black} Henrik: #1}}
\newcommand{\st}[1]{{\color{red!50!black} Sebastian: #1}}

\newcommand{\ulm}[1]{_{\scaleto{\mathrm{#1}}{3pt}}}
\newcommand\at[2]{\left.#1\right|_{#2}}











\newtheorem{assumption}{Assumption}

\DeclareMathOperator*{\argmax}{arg\,max}
\DeclareMathOperator*{\argmin}{arg\,min}

\newcommand{\swname}[1]{\texttt{#1}}
\newcommand{\ie}{i\/.\/e\/.,\/~}
\newcommand{\eg}{e\/.\/g\/.,\/~}
\newcommand{\cf}{cf\/.\/~}

\newcommand{\fig}{Fig\/.\/~}
\newcommand{\defn}{Def\/.\/~}
\newcommand{\sect}{Sec\/.\/~}
\newcommand{\tabl}{Tab\/.\/~}
\newcommand{\algo}{Algorithm~}
\newcommand{\theo}{Theorem~}

\newcommand{\bnnl}{3 hidden layers}
\newcommand{\bnnn}{50 neurons}
\newcommand{\bnna}{tanh activations}

\newcommand{\capt}[1]{\mdseries{\emph{#1}}}

\newcommand{\videolink}{at \url{https://youtu.be/_d7AqTRjz6g}}
\newcommand{\codelink}{\url{https://github.com/wheelbot/mini-wheelbot}}

\newcommand{\fakepar}[1]{\vspace{0mm}\noindent\textbf{#1.}}

\newcommand{\needref}{\textcolor{red}{[REF]}}

\newcommand{\plotfontsize}{9pt}





\begin{document}


\title{Uncertainty Quantification for LLM-Based Survey Simulations}




\author{Chengpiao Huang\thanks{Department of IEOR, Columbia University. Email: \texttt{chengpiao.huang@columbia.edu}.}
	\and Yuhang Wu\thanks{Decision, Risk, and Operations Division, Columbia Business School. Email: \texttt{yuhang.wu@columbia.edu}}
	\and Kaizheng Wang\thanks{Department of IEOR and Data Science Institute, Columbia University. Email: \texttt{kaizheng.wang@columbia.edu}.}
}


\date{This version: \today}

\maketitle

\begin{abstract}
We investigate the reliable use of simulated survey responses from large language models (LLMs) through the lens of uncertainty quantification. Our approach converts synthetic data into confidence sets for population parameters of human responses, addressing the distribution shift between the simulated and real populations. A key innovation lies in determining the optimal number of simulated responses: too many produce overly narrow confidence sets with poor coverage, while too few yield excessively loose estimates. To resolve this, our method adaptively selects the simulation sample size, ensuring valid average-case coverage guarantees. It is broadly applicable to any LLM, irrespective of its fidelity, and any procedure for constructing confidence sets. Additionally, the selected sample size quantifies the degree of misalignment between the LLM and the target human population. We illustrate our method on real datasets and LLMs. 
\end{abstract}
\noindent{\bf Keywords:} Synthetic data, Large language models, Uncertainty quantification, Simulation




\section{Introduction}

Large language models (LLMs) have demonstrated remarkable capabilities in mimicking human behaviors. Recent studies have leveraged LLMs to simulate human responses in various domains, including economic and social science experiments \citep{AAK23, Hor23, CLS23, BCD24, HYZ24, YLW24, ZHS24}, market research \citep{BIN23, GTo23, GSi24, WZZ24}, education \citep{ZMT23, LWa24}, and so on. Compared to traditional survey methods that recruit and query real people, LLM simulations offer significant advantages in terms of time and cost efficiency, enabling the generation of large-scale synthetic responses with minimal effort.

However, a growing body of evidence suggests that LLMs are not perfectly aligned with the human population, and in some cases, the misalignment can be substantial \citep{AAK23, SDL23}. This raises critical concerns about the reliability of insights derived from LLM-generated data. It remains a challenge how to properly simulate human responses using LLMs and how to account for their imperfections when using the simulated samples to make inference about the true human population.

We propose to address this challenge through the lens of \emph{uncertainty quantification}. Specifically, we seek to construct confidence sets for population statistics of human responses based on LLM-generated data. A central question in this process is:
\begin{center}
\emph{How many synthetic samples should be generated?}
\end{center}
On one hand, generating too many samples risks overfitting the synthetic distribution, which may deviate from the real human population. On the other hand, generating too few samples yields overly large and uninformative confidence sets. The optimal sample size depends on the discrepancy between the synthetic and real populations --- a quantity that is unknown in practice. This necessitates a data-driven approach to determine the appropriate number of simulated responses.


\paragraph{Main contributions.} In this paper, we develop a general framework to address these challenges. Our key contributions are as follows:

\begin{itemize}
\item (Formulation) We provide a rigorous mathematical framework for uncertainty quantification in LLM-based survey simulations.

\item (Methodology) We propose a flexible methodology that transforms simulated responses into valid confidence sets for population parameters of human responses. Our approach adaptively selects the simulation sample size based on the observed misalignment between the LLM and human populations. It is applicable to any LLM, regardless of its fidelity, and can be combined with any method for confidence set construction.
\end{itemize}


\paragraph{Related works.} 

Our work relates to research on assessing the fidelity of LLM simulations and measuring their alignment with real human populations. Prior studies have explored similarity metrics between synthetic and human distributions \citep{SDL23, HMG24, DHM24, DNL24, CRD25} and Turing-type tests \citep{ABFG23, MXY24} to evaluate LLM reliability. While these approaches provide valuable insights into LLM misalignment, they do not offer methods for leveraging imperfect LLM simulations to draw reliable conclusions about human populations. In contrast, our work provides a principled approach for constructing confidence sets that account for the inherent discrepancies between LLM-generated and human responses.

Additionally, our work connects to the line of work on model-free statistical inference \citep{SVo08, BAL21, ABF23}. At a high level, these methods use labeled data from the true distribution to calibrate imperfect point predictions from an arbitrary black-box model and then construct valid set estimates.
Our approach follows a similar spirit. The ``features'' and ``labels'' in our setting correspond to the survey questions and their population statistics of human responses, respectively. 
However, our labels are not directly observable. As a result, the labeled calibration data needed for these methods is not available.
Moreover, for every simulation sample size $k$, one can produce a point prediction of the label using $k$ synthetic responses generated by the LLM. As the optimal sample size is not known a priori, there are infinitely many candidate point predictions to choose from. This makes it difficult to apply existing methods.

\paragraph{Outline.} The rest of the paper is organized as follows. \Cref{sec-warmup} studies one-dimensional mean estimation as a warm-up example. \Cref{sec-general} presents the general problem setup and methodology. \Cref{sec-experiments} illustrates our proposed method on real datasets. \Cref{sec-discussions} concludes the paper.

\paragraph{Notation.} We use $\ZZ_+$ to denote the set of positive integers. For $n\in\ZZ_+$, define $[n]=\{1,2,...,n\}$. For $a,b\in\RR$, define $a\wedge b = \min\{a,b\}$ and $a\vee b = \max\{a,b\}$. For non-negative sequences $\{a_n\}_{n=1}^{\infty}$ and $\{b_n\}_{n=1}^{\infty}$, we write $a_n=O(b_n)$ if there exists $C>0$ such that for all $n$, it holds that $a_n \le C b_n$. We write $a_n = \Omega(b_n)$ if $b_n = O(a_n)$. We write $a_n = \Theta(b_n)$ if $a_n=O(b_n)$ and $a_n=\Omega(b_n)$. The notation $\Bernoulli(p)$ denotes the Bernoulli distribution with mean $p$. The notation $N(\mu,\sigma^2)$ denotes the normal distribution with mean $\mu$ and variance $\sigma^2$.
\section{Warm-up: Simulation of Binary Responses}\label{sec-warmup}

To motivate our problem and methodology, we will start with a simple setting where an LLM simulates binary responses to a survey question. In \Cref{sec-general}, we will present the general problem setup and the general methodology.

\subsection{Motivating Example: Educational Test}\label{sec-example-education}

Suppose a school wants to estimate the proportion $\mu \in [0,1]$ of students that can answer a newly designed test question correctly. It will not only provide insights into student progress but also evaluate the question's effectiveness in differentiating among students with varying levels of understanding. Such information can guide the school in tailoring teaching strategies to better address student needs.

The most direct approach is to give the test to $n$ students and collect their results $y_1,...,y_n\in\{0,1\}$, where $y_i$ indicates whether student $i$ answers the question correctly. A point estimate for $\mean$ is the sample mean $\responsebar = \frac{1}{n} \sum_{i=1}^n \response_i$. Given $\alpha\in(0,1)$, we can construct a confidence interval for $\mean$:
\begin{equation}\label{eqn-CI-intro-standard}
\left[ \responsebar - \frac{\samplesd}{\sqrt{n}} \Phi^{-1}\left( 1- \frac{\alpha}{2} \right) , ~  \responsebar + \frac{\samplesd}{\sqrt{n}} \Phi^{-1}\left( 1 - \frac{\alpha}{2} \right) \right],
\end{equation}
where $\samplesd = \sqrt{\responsebar (1-\responsebar) }$ is the sample standard deviation, and $\Phi$ is the cumulative distribution function (CDF) of $\Normal(0,1)$. By the Central Limit Theorem (CLT), this interval has asymptotic coverage probability $1-\alpha$ as $n\to\infty$. As a different approach, one can also use Hoeffding's concentration inequality (e.g., Theorem 2.8 in \cite{BLM13}) to construct a finite-sample confidence interval
\begin{equation}
\left[ \responsebar - \sqrt{\frac{\log(2/\alpha)}{2n}}  , ~  \responsebar + \sqrt{\frac{\log(2/\alpha)}{2n}} \right],
\end{equation}
which has at least $(1-\alpha)$ coverage probability for every $n\in\ZZ_+$. For simplicity, we will stick to \eqref{eqn-CI-intro-standard} in this section.

Alternatively, the school may use an LLM to simulate students' responses to the question. Compared with directly testing on real students, this approach is more time-efficient and cost-saving. If we prompt the LLM $k$ times with random student profiles, then it generates $k$ synthetic responses, which leads to synthetic outcomes $\simresponse_1,...,\simresponse_k\in\{0,1\}$. We may also compute the sample mean $\simresponsebar_k = \frac{1}{k} \sum_{i=1}^k \simresponse_i$ and the CLT-based confidence interval
\begin{equation}\label{eqn-CI-intro-sim}
\simCI(k) =  \bigg[ \simresponsebar_k - \frac{c\cdot \simsamplesd_k}{\sqrt{k}} \Phi^{-1}\left( 1- \frac{\alpha}{2} \right) , 
~
\simresponsebar_k + \frac{c\cdot \simsamplesd_k}{\sqrt{k}} \Phi^{-1}\left( 1 - \frac{\alpha}{2} \right) \bigg],
\end{equation}
where $\simsamplesd_k = \sqrt{\simresponsebar_k (1-\simresponsebar_k)}$, and $c>1$ is a scaling parameter. Such a dilation by $c$ is necessary; without it, whenever the LLM-generated data deviates from the student population (even by the slightest amount), the interval $\simCI(k)$ may never achieve $(1-\alpha)$ coverage regardless of $k$. We give an example in \Cref{sec-impossibility-exact-CLT}.

Due to the misalignment between the LLM and students, the distribution of the synthetic data $\{\simresponse_i\}_{i=1}^k$ may be very different from the true response distribution. In this case, the sample mean $\simresponsebar$ can be a poor estimate of 
$\mean$, and $\simCI(k)$ is generally not a valid confidence interval for $\mean$. In particular, as $k\to\infty$, the interval concentrates tightly around the synthetic mean $\EE [\simresponse_1]$ and fails to cover the true mean $\mean$. When $k$ is small, the interval becomes too wide to be informative, even though it may cover $\mean$ with high probability. We provide an illustration in \Cref{fig-tradeoff}.

\begin{figure}[h]
\centering
\includegraphics[scale=0.6]{figures/tradeoff.pdf}
\caption{The coverage-width trade-off for the simulation sample size $k$. The true distribution is $\Ber(0.4)$ and the synthetic distribution is $\Ber(0.6)$. The red dotted horizontal line plots the true mean $\mean=0.4$. The blue line plots the sample mean $\simresponsebar_k$ of the synthetic data, and the blue shaded region visualizes the confidence interval $\simCI(k)$, for $k\in[40]$. When $k$ is too small (say $k\le 6$), the interval $\simCI(k)$ is too wide. When $k$ is too large (say $k\ge 18$), interval $\simCI(k)$ fails to cover $\mean$.}\label{fig-tradeoff}
\end{figure}

\paragraph{Main insights.} Our goal in this work is to develop a principled approach for choosing a good simulation sample size $\widehat{k}$, so that $\simCI( \widehat{k} )$ is a valid confidence interval for $\mean$ while having a modest width. Solving this problem has the following important implications. 
\begin{enumerate}
\item The choice of $\widehat{k}$ offers valuable information for future simulation tasks on the appropriate number of synthetic samples to generate, so as to produce reliable confidence intervals. It also helps avoid generating excessive samples and improves computational efficiency.
\item The width of $\simCI( \widehat{k} )$ provides an assessment of the alignment between the LLM and the human population. A wide confidence interval indicates high uncertainty of its estimate of the true $\mean$, and thus a large gap between the synthetic data distribution and the true population.
\item The sample size $\widehat{k}$ also reflects the richness of the human population captured by the LLM. We make an analogy using the classical theory of parametric bootstrap. If a model is trained via maximum likelihood estimation over $k$ i.i.d.~human samples, then when performing bootstrap for uncertainty quantification, the bootstrap sample size is usually set to be $k$. In this regard, our approach can be thought of as uncovering the size of human population that the LLM can represent:
\begin{center}
	\emph{Using the LLM to simulate survey responses is roughly as accurate as surveying $\widehat{k}$ real people.}
\end{center}
We provide an illustration in \Cref{fig-LLMTurk}. 
The larger $\widehat{k}$ is, the more diversity that the LLM appears to capture. In contrast, a small $\widehat{k}$ could imply the peculiarity of the LLM compared to the major population. %Hence, $\widehat{k}$ gauges the alignment between the LLM and the human population.
\end{enumerate}

\begin{remark}[Comparison with existing works]
Existing works typically measure LLM misalignment using integral probability metrics and $f$-divergences \citep{SDL23, DHM24, DNL24}, which do not carry operational meanings themselves and at times can be hard to interpret. In contrast, our simulation sample size $\widehat{k}$ is scale-free and has a clear operational meaning.
\end{remark} 

\begin{figure}
\centering
\includegraphics[scale=0.15]{figures/LLMTurk}
\caption{An interpretation of the simulation sample size $k$ as the size of human population that an LLM can represent. Generating outputs from the LLM can be thought of as ``resampling'' from $k$ human samples that make up the LLM. The figure is generated by DALL-E \citep{RPG21}, and borrows ideas from the \emph{Mechanical Turk}, a chess-playing machine from the 18th century with a human player hidden inside.}\label{fig-LLMTurk}
\end{figure}


\subsection{Methodology for Selecting the Simulation Sample Size}\label{sec-method-1D}

We now introduce our method for choosing a good simulation sample size $\widehat{k}$. It makes use of similar test questions for which real students' results are available. If such data is available, we can compare LLM simulations with real students' results on these questions, and use it to guide the choice of $k$. 

Specifically, we assume access to $m$ test questions similar to the question of interest. For example, they can come from previous tests or a question bank. For $j\in[m]$, the $j$-th test question has been tested on $n_j$ real students, with test results $\dataset_j = \{ \response_{j,i} \}_{i=1}^{n_j}$. We also simulate LLM responses $\simdataset_j = \{ \simresponse_{j,i} \}_{i=1}^K$ to the $j$-th test question, and $\simdataset = \{ \simresponse_{i} \}_{i=1}^K$ to the new test question. Here $K\in\ZZ_+$ is the simulation budget.

For each question $j\in[m]$, we form confidence intervals similar to \eqref{eqn-CI-intro-sim} using the synthetic data $\simdataset_j$, aiming to cover the true proportion $\mean_j$ of students that can answer the $j$-th question correctly:
\begin{equation}\label{eqn-CI-intro-sim-calibrate}
\simCI_j(k) = \bigg[ \simresponsebar_{j,k} - \frac{c\cdot \simsamplesd_{j,k}}{\sqrt{k}} \Phi^{-1}\left( 1- \frac{\alpha}{2} \right) , 
~  \simresponsebar_{j,k} + \frac{c\cdot \simsamplesd_{j,k}}{\sqrt{k}} \Phi^{-1}\left( 1 - \frac{\alpha}{2} \right) \bigg], 
\end{equation}
where $\simresponsebar_{j,k} = \frac{1}{k} \sum_{i=1}^k \simresponse_{j,i}$ is the sample mean of the first $k$ samples in $\simdataset_j$, and $\simsamplesd_{j,k} = \sqrt{\simresponsebar_{j,k} ( 1 - \simresponsebar_{j,k} )}$ is the estimated standard deviation. We also set the convention $\simCI(0) = \simCI_j(0) = \RR$, as nothing can be said about the true parameter without data. We will pick $\widehat{k}\in\{0,1...,K\}$ such that $\simCI_j(\widehat{k})$ covers $\mean_j$ with high probability. We expect this choice of $\widehat{k}$ to be also good for $\simCI(k)$, as the test questions are similar.

Ideally, we would like to pick $k$ such that $(1-\alpha)$-coverage is achieved empirically over the $m$ test questions:
\begin{equation}\label{eqn-oracle-criterion-1D}
\frac{1}{m} \sum_{j=1}^m \ind \{ \mean_j \not\in \simCI_j(k) \} \le \alpha.
\end{equation}
As the true $\{\mean_j\}_{j=1}^m$ are not available, we use the real data $\{ \dataset_j \}_{j=1}^m$ to compute the sample means $\responsebar_j = \frac{1}{n_j} \sum_{i=1}^{n_j} \response_{j,i}$ as proxies for $\mean_j$. The empirical miscoverage can be approximated by
\begin{equation}\label{eqn-proxy-1D}
\coverage(k) = \frac{1}{m} \sum_{j=1}^m \ind \{ \responsebar_j \not\in \simCI_j(k) \}.
\end{equation}
Our criterion for selecting $k$ is given by
\begin{equation}\label{eqn-empirical-criterion-1D}
\widehat{k} = \max \left\{ 0\le k \le K : \coverage(i) \le \alpha/2 ~~\forall i\le k \right\}.
\end{equation}
Note that $\widehat{k}$ is well-defined because $\coverage(0) = 0$.

The choice of the threshold $\alpha/2$ in \eqref{eqn-empirical-criterion-1D} can be explained as follows. By CLT, when $n_j$ is large, $\PP( \mean_{j} \ge \responsebar_j ) \approx 1/2$. Suppose $\mean_{j} \not\in \simCI_j(k)$, then  $\mean_{j}$ is either on the left or the right of $\simCI_j(k)$. In the former case, $\responsebar_j$ is on the left of $\mean_{j}$ with probability around $1/2$, which implies $\responsebar_j \not\in \simCI_j(k)$. Similarly, in the latter case, $\responsebar_j$ is on the right of $\mean_{j}$ with probability around $1/2$, and then $\responsebar_j \not\in \simCI_j(k)$. Roughly speaking, the frequency of having $\responsebar_j \not\in \simCI_j(k)$ is at least half of the frequency of having $\mean_{j} \not\in \simCI_j(k)$. In other words, the lower bound
\begin{equation}\label{eqn-proxy-to-oracle-1D}
G(k) \geq \frac{1}{2} \cdot \frac{1}{m} \sum_{j=1}^m \ind \{ \mean_{j} \not\in \simCI_j(k) \} 
\end{equation}
approximately holds. Substituting \eqref{eqn-proxy-to-oracle-1D} into \eqref{eqn-oracle-criterion-1D} yields the threshold $\alpha/2$ for choosing $\widehat{k}$. 


\subsection{Theoretical Analysis}\label{sec-theory-1D}

In this section, we present a theoretical analysis of our proposed method. To do so, we first describe the setup in \Cref{sec-example-education} and \Cref{sec-method-1D} in mathematical terms.

The student population can be represented by a distribution $\distribution$ over a space $\profilespace$ of possible \emph{student profiles}, say, vectors of background information, classes taken, grades, etc. To simulate student responses from the LLM, synthetic student profiles are generated from a synthetic student population $\simdistribution$ over $\profilespace$, and then fed to the LLM.

We use $\testfunction$ and $\{ \testfunction_j \}_{j=1}^m$ to refer to the test question of interest and the $m$ similar ones, respectively. 
Students' performance on test questions are characterized by a \emph{performance function} $\performancefunction$: a student with profile $\profile\in\profilespace$ answers a question $\testfunction$ correctly with probability $\performancefunction ( \profile , \testfunction ) \in [0, 1]$. The average student performance on the test questions $\testfunction$ and $\{ \testfunction_j \}_{j=1}^m$ are then $\mean = \EE_{\profile \sim \distribution } \performancefunction ( \profile ,  \testfunction ) $ and $\mean_j = \EE_{\profile \sim \distribution } \performancefunction (  \profile, \testfunction_j ) $, respectively.
In addition, the LLM generates synthetic student performance from a \emph{synthetic performance function} $\simperformancefunction$: when prompted with a synthetic profile $\simprofile\in\profilespace$, the LLM answers a question $\testfunction$ correctly with probability $\simperformancefunction ( \simprofile , \testfunction ) \in [0, 1]$. 

The collection of the real dataset $\dataset_j = \{ \response_{j,i} \}_{i=1}^{n_j}$ can be thought of as drawing $n_j$ i.i.d.~student profiles $\{ \profile_{j,i} \}_{i=1}^{n_j} \sim \distribution$ and then sampling $\response_{j,i} \sim \Bernoulli (  \performancefunction ( \profile_{j,i} , \testfunction_j  ) )$ for each $i\in[n_j]$. Similarly, the generation of the synthetic dataset $\simdataset_j = \{ \simresponse_{j,i} \}_{i=1}^{K}$ can be thought of as drawing i.i.d.~synthetic profiles $\{ \simprofile_{j,i} \}_{i=1}^{K} \sim \simdistribution$ and then sampling $\simresponse_{j,i} \sim \Bernoulli ( 
\simperformancefunction
( \simprofile_{j,i} , \testfunction_j  ) )$ for each $i\in [K]$. For  $\simdataset = \{ \simresponse_i \}_{i=1}^K$, we adopt a similar notation $\{ \simprofile_i \}_{i=1}^K$ for the synthetic profiles. We note that when collecting real or synthetic samples, the performance functions never appear explicitly. They are introduced only to facilitate the problem formulation.

Finally, we assume that the test questions are drawn randomly from a question bank, and that the datasets are independent.

\begin{assumption}[Randomly sampled questions]\label{assumption-iid-test-1D}
The questions $\testfunction,\testfunction_1,...,\testfunction_m$ are independently sampled from a distribution over a space $\testfunctionfamily$.
\end{assumption}

\begin{assumption}[Independent data]\label{assumption-indep-data-1D}
For each $j\in[m]$, conditioned on $\testfunction_j$, the datasets $\dataset_j$ and $\simdataset_j$ are independent. Conditioned on $\testfunction_1,...,\testfunction_m$, the dataset tuples $(\dataset_1,\simdataset_1),...,(\dataset_m,\simdataset_m)$ are independent. Finally, $(\testfunction,\simdataset)$ is independent of $\big\{ (\testfunction_j,\dataset_j,\simdataset_j) \big\}_{j=1}^m$.
\end{assumption}

We are now ready to state the theoretical guarantee of our approach. Its proof is deferred to \Cref{sec-thm-coverage-1D-proof}. We note that the assumption $\PP( \mean_{j} \ge \responsebar_j \mid \testfunction_j ) = 1/2$ is a CLT approximation and is for mathematical convenience only.

\begin{theorem}[Coverage guarantee]\label{thm-coverage-1D}
Let Assumptions \ref{assumption-iid-test-1D} and \ref{assumption-indep-data-1D} hold. Assume that $\PP( \responsebar_j \leq \mean_{j} \mid \testfunction_j ) = 1/2$ for each $j\in[m]$. Fix $\alpha\in(0,1)$. Then the simulation sample size $\widehat{k}$ defined by \eqref{eqn-empirical-criterion-1D} satisfies
\[
\PP\Big( \mean \in \simCI(\widehat{k}) \Big) \ge 1-\alpha - \sqrt{\frac{2}{m}}.
\]
The probability is taken with respect to randomness of $\big\{ ( \testfunction_j , \dataset_j, \simdataset_j ) \big\}_{j=1}^m$, $\testfunction$ and $\simdataset$.
\end{theorem}

On average, the chosen simulation sample size $\widehat{k}$ leads to a confidence interval $\simCI(\widehat{k})$ that covers the true mean $\mean$ with probability at least $1-\alpha-O(\sqrt{1/m})$. As $m\to\infty$, the aforementioned lower bound converges to $1-\alpha$.


\subsection{Sharpness of Sample Size Selection}

We have seen that the chosen interval $\simCI(\widehat{k})$ has good coverage properties. In this section, we complement this result by showing that the interval is not overly conservative. To simplify computation, we slightly modify the setting.

\begin{example}[Gaussian performance score]\label{example-gaussian}
Consider the setting in \Cref{sec-theory-1D} with the following modifications. On a test question $\testfunction \in\testfunctionfamily$, the performance (e.g., score) of a real student with profile $\profile$ follows the distribution $N( \performancefunction ( \profile , \testfunction ),1)$ instead of $\Bernoulli ( F ( \profile , \testfunction ) )$; similarly, the performance of the LLM prompted with synthetic profile $\simprofile$ has distribution $N( \simperformancefunction (  \simprofile, \testfunction ),1)$. Moreover, the confidence intervals $\simCI(k)$ defined in \eqref{eqn-CI-intro-sim} and $\simCI_j(k)$ defined in \eqref{eqn-CI-intro-sim-calibrate} are changed to
\begin{align*}
& \simCI(k) = \left[ \simresponsebar_k - \frac{C}{\sqrt{k}},~ \simresponsebar_k + \frac{C}{\sqrt{k}} \right], \\[4pt]
& \simCI_j(k) = \left[ \simresponsebar_{j,k} - \frac{C}{\sqrt{k}},~ \simresponsebar_{j,k} + \frac{C}{\sqrt{k}} \right],
\end{align*}
respectively, where $C = 2 \Phi^{-1} (1 - \alpha / 4)$. For simplicity, we suppose that the real datasets have the same size: $n_j=n$ for all $j\in[m]$. Finally, we define
\[
\Delta = \sup_{\testfunction \in \testfunctionfamily} \big| \EE_{\profile \sim \distribution} \performancefunction ( \profile , \testfunction ) - \EE_{\simprofile \sim \simdistribution} \simperformancefunction ( \simprofile , \testfunction ) \big|.
\]
\end{example}

In \Cref{example-gaussian}, the quantity $\Delta$ measures the discrepancy between the distributions of the real students' performance and of the simulated students' performance. The following theorem presents a lower bound on the chosen simulation sample size $\widehat{k}$. Its proof can be found in \Cref{sec-thm-sharpness-1D-proof}.

\begin{theorem}[Sharpness of chosen sample size]\label{thm-sharpness-1D}
Consider the setting of \Cref{example-gaussian}. Let $\widehat{k}$ be chosen by the procedure \eqref{eqn-empirical-criterion-1D}. Choose $\delta \in (0, 1)$. There exists a constant $C'>0$ determined by $\alpha$ such that when $m  > C' \log(n / \delta)$, the following holds with probability at least $1 - \delta$:
\[
\widehat{k} \ge \min \left\{
\left(
\frac{C}{5 \Delta}
\right)^2,
n,  K
\right\}.
\]
When this happens, the selected confidence interval $\simCI(\widehat{k})$ has width $O\left( \max\{\Delta, n^{-1/2}, K^{-1/2} \} \right)$.
\end{theorem}

\Cref{thm-sharpness-1D} implies that the chosen $\widehat{k}$ is the largest possible, or equivalently, the interval $\simCI(\widehat{k})$ is the shortest possible. To see this, suppose that the simulation budget $K$ is large. When $\Delta$ is small, there is little distribution shift between the real and simulated student performance, so it is optimal to use as many synthetic samples as possible. \Cref{thm-sharpness-1D} shows that in this case, the procedure \eqref{eqn-empirical-criterion-1D} chooses $\widehat{k}=n$ with high probability. This is the maximum effective sample size, since the $n$ real data points can identify the true mean only up to an error of $O(n^{-1/2})$ and thus there is no statistical benefit in simulating more than $n$ samples. On the other hand, when the discrepancy $\Delta$ between the real and simulated student performance is large, any $\simCI(k)$ that covers the true mean with high probability must have width $\Omega(\Delta)$. Thus, the chosen confidence interval $\simCI(\widehat{k})$, which has width $O(\Delta)$ with high probability, is the shortest possible.


\section{General Setup and Methodology}\label{sec-general}

In this section, we study the more general setting where survey responses and confidence sets can be multi-dimensional.


\subsection{Problem Formulation}

Let $\profilespace$ be a profile space, $\distribution$ a probability distribution over $\profilespace$ which represents the true population, and $\simdistribution$ a synthetic distribution over $\profilespace$ used to generate synthetic profiles. 

Let $\testfunctionfamily$ be a collection of survey questions, and $\responsespace$ be the space of possible responses to the survey questions. When a person with profile $\profile\in\profilespace$ is asked a survey question $\testfunction\in\testfunctionfamily$, the person gives a response $\response$ following a distribution $\responsedist( ~ \cdot \mid \profile, \testfunction)$ over $\responsespace$. We are interested in the distribution of the population's response to the survey question $\testfunction$, which is given by $\responsedistalt(~ \cdot \mid \testfunction )
=
\int_{\profilespace} \responsedist( ~ \cdot \mid \profile, \testfunction) \, \distribution ( d \profile )$. In particular, we seek to construct a confidence set for some statistic $\statistic( \testfunction )$ of $\responsedistalt(~ \cdot \mid \testfunction )$, which can be multi-dimensional, say in $\RR^d$. Below we revisit the educational test example in \Cref{sec-warmup} in this framework. More examples are provided in \Cref{sec-examples}.

\begin{example}[Educational test evaluation]\label{example-education}
In the educational test evaluation example in \Cref{sec-warmup}, each $\profile\in\profilespace$ is a student profile, each $\testfunction\in\testfunctionfamily$ is a test question, the response space is $\responsespace=\{0,1\}$, and $\responsedist( ~ \cdot \mid \profile, \testfunction) = \Bernoulli ( \performancefunction( \profile , \testfunction) )$. The statistic $\statistic( \testfunction )$ is the probability of a student answering the question correctly: $\EE_{\response \sim \responsedistalt( \cdot \mid \testfunction )} [ y ] = \EE_{\profile \sim \distribution} [ \performancefunction( \profile , \testfunction) ]$.
\end{example}

We consider constructing the confidence set by using simulated responses from an LLM. Given a profile $\profile$, a survey question $\testfunction$ and a prompt $\prompt$, the LLM simulates a response $\simresponse$ from a distribution $\simresponsedist(~\cdot\mid \profile , \testfunction, \prompt)$ which aims to mimic $\responsedist(~\cdot\mid \profile , \testfunction)$. We can generate synthetic profiles $\{ \simprofile_i \}_{i=1}^K$ from some distribution $\simdistribution$, then feed them into the LLM along with $\testfunction$ and $\prompt$. The LLM then generates synthetic responses $\{ \simresponse_i \}_{i=1}^K$, where $\simresponse_i \sim \simresponsedist(~\cdot\mid \simprofile_i , \testfunction, \prompt)$. Here $K$ is the simulation budget.


Using the simulated samples $\simdataset = \{ \simresponse_i \}_{i=1}^K$, we can construct a family of candidate confidence sets such as the one-dimensional CLT-based confidence interval \eqref{eqn-CI-intro-sim}. More generally, the statistics literature has developed a variety of approaches such as inverting hypothesis tests \citep{CBe02}, the bootstrap \citep{Efr79}, and the empirical likelihood ratio function \citep{OWe90}. We will assume access to a black-box procedure $\setmap$ that takes as input a dataset $\dataset$ and outputs a confidence set $\setmap(\dataset)\subseteq\RR^d$. Then, we can construct a family of confidence sets $\{\simCIalt(k)\}_{k=1}^{K}$ by
\begin{equation}\label{eqn-CI-sim}
\simCIalt(k) = \setmap \left( \{ \simresponse_i \}_{i=1}^k \right).
\end{equation}
We also set $\simCIalt(0) = \RR^d$, so $\statistic ( \testfunction ) \in \simCIalt(0)$ always. We will not impose any assumptions on the quality of the confidence sets produced by $\setmap$.

As the LLM may not be a faithful reflection of the true human population, we will make use of real data to choose a good confidence set from $\{\simCIalt(k)\}_{k=1}^{K}$. We assume that we have collected real human responses from $m$ surveys $\testfunction_1,...,\testfunction_m\in\testfunctionfamily$. For each $j\in[m]$, we have responses $\dataset_j = \{ \response_{j,i} \}_{i=1}^{n_j}$ from $n_j$ i.i.d.~surveyees $\{ \profile_{j,i} \}_{i=1}^{n_j} \sim \distribution$, with $\response_{j,i} \sim \responsedist(~\cdot\mid  \profile_{j,i}, \testfunction_j )$. 

We also simulate LLM responses to these $m$ survey questions. For each $j\in[m]$, we feed synthetic profiles $\{ \simprofile_{j,i} \}_{i=1}^K$, the question $\testfunction_j$ and the prompt $\prompt$ into the LLM, which simulates responses $\simdataset_j = \{ \simresponse_{j,i} \}_{i=1}^{K}$ with $\simresponse_{j,i} \sim \simresponsedist (~\cdot \mid \simprofile_{j,i}, \testfunction_j, \prompt )$. The datasets $\{ \dataset_j \}_{j=1}^m $ and $\{\simdataset_j \}_{j=1}^m $ will be used to select a confidence set from $\{ \simCIalt(k) \}_{k=1}^K$. We make the same Assumptions \ref{assumption-iid-test-1D} and \ref{assumption-indep-data-1D} as in \Cref{sec-warmup}.

We are now ready to formally state our problem.

\begin{problem}[Uncertainty quantification]\label{problem-general}
Given $\alpha \in (0,1)$, how to use $\{ \dataset_j \}_{j=1}^m$ and $\{ \simdataset_j \}_{j=1}^m$ to choose $\widehat{k}\in[K]$ such that
\[
\PP \Big( \statistic ( \testfunction ) \in \simCIalt(\widehat{k}) \Big) \approx 1-\alpha?
\]
\end{problem}


\subsection{General Methodology for Sample Size Selection}

We now present our general methodology for Problem \ref{problem-general}. For each $j\in[m]$, we form confidence sets similar to \eqref{eqn-CI-sim} using the synthetic data $\simdataset_j$:
\begin{equation}\label{eqn-CI-sim-calibrate}
\simCIalt_j(k) = \setmap \left( \{ \simresponse_{j,i} \}_{i=1}^k \right),\quad \forall k\in[K].
\end{equation}
We also set $\simCIalt_j(0) = \RR^d$. We will pick $\widehat{k}\in\{0,1,...,K\}$ such that $\simCIalt_j(\widehat{k})$ is a good confidence interval for $\statistic (\testfunctionalt_j)$ for each $j\in[m]$. This choice of $\widehat{k}$ will also be good for $\simCIalt(k)$, thanks to the i.i.d.~assumption on the survey questions.

Ideally, we would like to pick $k$ such that $(1-\alpha)$ coverage is achieved empirically over the $m$ survey functions:
\begin{equation}\label{eqn-oracle-criterion}
\frac{1}{m} \sum_{j=1}^m \ind \{  \statistic( \testfunction_j )  \not\in \simCIalt_j(k) \} \le \alpha.
\end{equation}
However, the population-level quantities $\{ \statistic( \testfunction_j ) \}_{j=1}^m$ are not available, so we must approximate them by the real data $\{ \dataset_j \}_{j=1}^m$. In \Cref{sec-warmup}, we have taken the approach of constructing unbiased point estimates, but it does not directly extend to the more general case. 

Instead, we will use the real data $\{ \dataset_j \}_{j=1}^m$ to construct confidence sets for $\{  \statistic( \testfunction_j )  \}_{j=1}^m$. Choose a confidence level $\gamma\in(0,1)$. For each $j \in [m]$, we use $\dataset_j$ to construct a confidence set $\CIalt_j$ that satisfies
\begin{equation}\label{eqn-CI-calibrate}
\PP\Big( \statistic ( \testfunction_j)  \in \CIalt_j \Bigm| \testfunction_j \Big) \ge \gamma.
\end{equation}
These confidence sets are easy to construct as the samples in $\dataset_j$ follow the true response distribution. When $\statistic ( \testfunction_j)  \in \CIalt_j$, the condition $\CIalt_j\subseteq \simCIalt_j(k)$ is sufficient for $\statistic ( \testfunction_j)  \in \simCIalt_j(k)$. Equivalently, when $\statistic ( \testfunction_j)  \in \CIalt_j$, the condition $\statistic ( \testfunction_j)  \not\in \simCIalt_j(k)$ must imply $\CIalt_j\not\subseteq \simCIalt_j(k)$. Thus, we take
\begin{equation}\label{eqn-proxy}
\coveragealt(k) = \frac{1}{m} \sum_{j=1}^m \ind \{ \CIalt_j\not\subseteq \simCIalt_j(k) \}
\end{equation}
as a proxy for the empirical miscoverage. Since $\statistic ( \testfunction_j)  \in \CIalt_j$ happens with probability $\gamma$, then the frequency of having $\CIalt_j\not\subseteq \simCIalt_j(k)$ is at least $\gamma$ times the frequency of $\statistic ( \testfunction_j)  \not\in \simCIalt_j(k)$. Roughly speaking,
\begin{equation}\label{eqn-proxy-to-oracle}
\coveragealt(k) \ge \gamma \cdot\left( \frac{1}{m} \sum_{j=1}^m \ind \{  \statistic( \testfunction_j )  \not\in \simCIalt_j(k) \} \right).
\end{equation}

Combining \eqref{eqn-oracle-criterion} and \eqref{eqn-proxy-to-oracle} leads to the following criterion for selecting $k$:
\begin{equation}\label{eqn-empirical-criterion}
\widehat{k} = \max \left\{ 0\le k \le K : ~ L(i) \le \gamma\alpha, ~\forall i\le k \right\}.
\end{equation}
Note that $\widehat{k}$ is well-defined because $\coveragealt(0) = 0$. The full procedure for sample size selection is summarized in \Cref{alg-general}. 

\begin{algorithm}[h]
	\begin{algorithmic}
	\STATE {\bf Input:} Survey questions with real and simulated responses $\big\{(\testfunction_j, \dataset_j, \simdataset_j)\big\}_{j=1}^m$, prescribed miscoverage probability $\alpha$, confidence set construction procedure $\setmap$, confidence level $\gamma$, simulation budget $K$. \\[4pt]
	\FOR{$j=1,...,m$}
		\STATE Use $\setmap$ and $\simdataset_j$ to construct synthetic confidence sets $\{\simCIalt_j(k)\}_{k=0}^K$ by \eqref{eqn-CI-sim-calibrate}. \\[4pt]
		\STATE Use $\dataset_j$ to construct a confidence set $\CIalt_j$ satisfying \eqref{eqn-CI-calibrate}.
	\ENDFOR
	\STATE Define
	\[
	\coveragealt(k) = \frac{1}{m} \sum_{j=1}^m \ind \{ \CIalt_j\not\subseteq \simCIalt_j(k) \}.
	\]
	\STATE Set $\widehat{k} = \max \left\{ 0\le k \le K :~ L(i) \le \gamma\alpha ,~\forall i\le k \right\}$. \\[4pt]
%	\STATE Use $\setmap$ and $\simdataset$ to construct synthetic confidence sets $\{\simCIalt ( \widehat{k} )\}_{k=0}^K$ by \eqref{eqn-CI-sim}. \\[4pt]
	\STATE {\bf Output:} Sample size $\widehat{k}$
.	\caption{Simulation Sample Size Selection}
	\label{alg-general}
	\end{algorithmic}
\end{algorithm}

We now present the coverage guarantee for our method, which shows that the chosen confidence set $\simCIalt(\widehat{k})$ has coverage probability at least $1-\alpha-O(1/\sqrt{m})$. Its proof is deferred to \Cref{sec-thm-coverage-proof}.


\begin{theorem}\label{thm-coverage}
Let Assumptions \ref{assumption-iid-test-1D} and \ref{assumption-indep-data-1D} hold. Fix $\alpha\in(0,1)$. The output $\widehat{k}$ of \Cref{alg-general} satisfies
\[
\PP\Big( \statistic ( \testfunction ) \in \simCIalt(\widehat{k}) \Big) \ge 1-\alpha - \gamma^{-1} \sqrt{\frac{1}{2m}}.
\]
\end{theorem}

It is worth noting that our method achieves this coverage without any assumptions on the qualities of the LLM and the procedure $\setmap$ for confidence set construction. Nevertheless, the size of the chosen confidence set $\simCIalt(\widehat{k})$, in terms of the true coverage rate and size, depend on these factors. If there is a large alignment gap between the LLM and the human population, then $\simCIalt(\widehat{k})$ will inevitably be large.





\section{Numerical Experiments}\label{sec-experiments}

In this section, we apply our general method in \Cref{sec-general} to LLMs over real datasets. The code and data are available at \url{https://github.com/yw3453/uq-llm-survey-simulation}.

\subsection{Experiment Setup}


\paragraph{LLMs.} We consider $8$ LLMs: GPT-3.5-Turbo (\texttt{gpt-3.5-turbo}), GTP-4o (\texttt{gpt-4o}), and GPT-4o-mini (\texttt{gpt-4o-mini}) \citep{GPT3.5, GPT4o, GPT4omini}; Claude 3.5 Haiku (\texttt{claude-3-5-haiku-20241022}) \cite{Ant24}; Llama 3.1 8B (\texttt{Llama-3-8B-Instruct-Turbo}) and Llama 3.3 70B (\texttt{Llama-3.3-70B-Instruct-Turbo}) \citep{DJP24}; Mistral 7B (\texttt{Mistral-7B-Instruct-v0.3}) \citep{JSM23}; DeepSeek-V3 (\texttt{DeepSeek-V3}) \citep{LFX24}.


\paragraph{Datasets.} We use two datasets for survey questions, each corresponding to one uncertainty quantification task. The first dataset is the OpinionQA dataset created by \cite{SDL23}. It was built from Pew Research's American Trends Panel\footnote{\url{https://www.pewresearch.org/the-american-trends-panel/}}, and contains the general US population's responses to survey questions spanning topics such as science, politics, and health. After pre-processing we have 385 unique questions and 1,476,868 responses to these questions from at least 32,864 people. These questions have $5$ choices corresponding to ordered sentiments which we map to sentiment scores $-1,-\frac{1}{3},0,\frac{1}{3},1$. Each question has at least $400$ responses. For each response, we have information on their political profile, religious affiliation, educational background, socio-economic status, etc. This information is used as their profiles to generate synthetic profiles. See \Cref{sec-opinion} for more information on this dataset, including example questions, profile features, and the generation of synthetic profiles and answers. We consider the task of constructing a confidence interval for the US population's average sentiment score for a survey question. This is the setup in \Cref{example-public-survey-1D}.


The second dataset is the EEDI dataset created by \cite{HMG24}, which was built upon the NeurIPS 2020 Education Challenge dataset \citep{WLS21}. It consists of students' responses to mathematics multiple-choice questions on the Eedi online educational platform\footnote{\url{https://eedi.com/}}. The dataset contains 573 unique questions and 443,433 responses to these questions from 2,287 students. All questions have four choices (A, B, C, D). Out of these questions, we use questions that have at least $100$ student responses. Excluding questions with graphs or diagrams, we are left with a total of $412$ questions. For each student, we have information on their gender, age, and socioeconomic status. This information is used as their profiles to generate synthetic profiles. See \Cref{sec-eedi} for more information on this dataset, including example questions, profile distribution, and the generation of synthetic profiles and answers. We consider the task of constructing a confidence interval for the probability of a student answering a question correctly. This is similar to the setup in \Cref{sec-warmup} and \Cref{example-education}.


\paragraph{Confidence set construction.} Our method can be built upon any arbitrary confidence set construction procedure $\setmap$. We use a construction procedure based on Hoeffding's concentration inequality (e.g., Theorem 2.8 in \cite{BLM13}), as it has valid coverage guarantee for any finite sample size. Given $\alpha\in(0,1)$ and responses $\{\simresponse_i\}_{i=1}^k$, we construct the confidence interval
\begin{equation}\label{eqn:synthCI}
\setmap \big(\{\simresponse_i\}_{i=1}^k\big)
= 
\bigg[ \simresponsebar_k - cM\sqrt{\frac{\log(2/\alpha)}{2k}}  ,  ~  \simresponsebar_k + cM\sqrt{\frac{\log(2/\alpha)}{2k}} \bigg],
\end{equation}
where $\simresponsebar_k = \frac{1}{k}\sum_{i=1}^k \simresponse_i$ is the sample mean, $M>0$ is an upper bound on the range of the responses, and $c>1$ is a scaling constant. 

\paragraph{Hyperparameters.} We consider $\alpha\in\{0.05\cdot\ell:\ell\in[10]\}$, $c=\sqrt{2}$ and $\gamma=0.5$. For the EEDI dataset, we set the simulation budget $K=50$ and take $M=1$ since the responses are binary. For the OpinionQA dataset, we set the simulation budget $K=100$ and take $M=2$ since the responses range within $[-1,1]$.


\subsection{Experiment Procedure}

We now describe our experiment procedure for applying the method in \Cref{sec-general} to each dataset. Denote the dataset by $\{ ( \testfunction_j , \dataset_j ) \}_{j=1}^{J}$, where $\testfunction_j$ is a survey question and $\dataset_j = \{ \response_{j,i} \}_{i=1}^{n_j}$ is a collection of human responses. For each $j\in[J]$, we simulate $K$ responses $\simdataset_j$ from an LLM. We then randomly split $ \datasetmeta = \{ ( \dataset_j, \simdataset_j ) \}_{j=1}^{J}$ into a training set $\datasetmeta^{\train} = \{ ( \dataset_j, \simdataset_j ) \}_{j\in\cJ_{\train}}$ and a testing set $\datasetmeta^{\test} = \{ ( \dataset_j, \simdataset_j ) \}_{j\in\cJ_{\test}}$, with $|\datasetmeta^{\train}| : |\datasetmeta^{\test}| = 3 : 2$.


\paragraph{Selection of simulation sample size.} We apply the approach \eqref{eqn-empirical-criterion} with the training set $\datasetmeta^{\train}$ to select a simulation sample size $\widehat{k}$. For the confidence set $\CIalt_j$ in \eqref{eqn-CI-calibrate} constructed from the real data $\dataset_j$, we use the standard CLT-based confidence interval:
\begin{equation}\label{eqn-CI-calibrate-CLT}
\CIalt_j = \bigg[ \responsebar_j - \frac{\samplesd_j}{\sqrt{n_j}} \Phi^{-1}\left( \frac{1+\gamma}{2} \right) ,
~  \responsebar_j + \frac{\samplesd_j}{\sqrt{n_j}} \Phi^{-1}\left( \frac{1+\gamma}{2} \right) \bigg],
\end{equation}
where $\responsebar_j = \frac{1}{n_j} \sum_{i=1}^{n_j} \response_{j,i}$ and $\samplesd_j = \sqrt{\responsebar_j (1-\responsebar_j) }$. Since $n_j$ is at least $100$, $\CIalt_j$ has approximately $\gamma$ coverage probability. In \Cref{fig:k-upcross-eg} of \Cref{sec-appendix-experiments-results}, we provide a visualization for the selection of $\widehat{k}$.

\paragraph{Evaluation of selected sample size.} We use $\datasetmeta^{\test}$ to evaluate the quality of the chosen simulation sample size $\widehat{k}$. As the true population mean $\statistic ( \testfunction )$ is unavailable, the true coverage probability $\PP\big( \statistic ( \testfunction ) \in \simCIalt(\widehat{k}) \big)$ cannot be computed. However, we can apply the same idea as \eqref{eqn-proxy} in \Cref{sec-general} to compute a proxy for the miscoverage level. For each survey question $j\in\cJ_2$, the selected sample size $\widehat{k}$ leads to the synthetic confidence set $\simCIalt_j(\widehat{k}) = \setmap\big( \{ \simresponse_{j,i} \}_{i=1}^{\widehat{k}} \big)$. We form the confidence set $\CIalt_j$ from real data $\dataset_j$ as in \eqref{eqn-CI-calibrate-CLT} and define
\begin{equation}\label{eqn:proxy}
\widetilde{\coveragealt}(\widehat{k}) = \frac{1}{\gamma} \cdot \frac{1}{|\cJ_{\test}|} \sum_{j\in\cJ_{\test}} \ind \{ \CIalt_j\not\subseteq \simCIalt_j(\widehat{k}) \}.
\end{equation}
The proof of \Cref{thm-coverage} shows that, for every $k\in[K]$ and survey question $j$,
\[
\gamma \cdot \PP\big( \statistic ( \testfunction ) \not\in \simCIalt(k) \big) \le \PP\big( \CIalt_j\not\subseteq \simCIalt_j(k) \big) = \gamma \cdot \EE \big[ \widetilde{\coveragealt}(\widehat{k}) \big] .
\]
Thus, if $\EE \big[ \widetilde{\coveragealt}(\widehat{k}) \big] \le \alpha$, then $\PP\big( \statistic ( \testfunction ) \not\in \simCIalt(\widehat{k}) \big) \le \alpha$ must hold. To that end, we will test a hypothesis $H_0: \; \EE \big[ \widetilde{\coveragealt}(\widehat{k}) \big] \le \alpha$ against its alternative $H_1: \; \EE \big[ \widetilde{\coveragealt}(\widehat{k}) \big] > \alpha$.


\subsection{Experiment Results}\label{sec-experiments-results}

For both datasets, we evaluate three metrics as $\alpha$ varies: the miscoverage probability proxy \eqref{eqn:proxy}, the selected simulation sample size $\widehat{k}$, and the half-width of the synthetic confidence interval $\simCIalt(\widehat{k})$. We consider $100$ random train-test splits of the questions. For compactness, we present results on the miscoverage probability proxy and $\widehat{k}$, and defer the results on the half-width of the synthetic confidence interval as well as more experiment details to \Cref{sec-appendix-experiments-results}. We omit Llama 3.1 8B for the EEDI dataset experiment because it frequently failed to answer EEDI questions in required formats. As a baseline, we also include a na\"{i}ve response generator (\texttt{random}) that chooses an available answer uniformly at random. 

In \Cref{fig:combined}, we present histograms of $p$-values for the hypothesis test $\EE \big[ \widetilde{\coveragealt}(\widehat{k}) \big] \le \alpha$ against $\EE \big[ \widetilde{\coveragealt}(\widehat{k}) \big] > \alpha$ across various LLMs and $\alpha$'s over the OpinionQA and EEDI datasets. The $p$-values are computed using a one-sided $z$-test over the $100$ random splits. As can be seen from the histograms, all $p$-values are reasonably large, indicating that the hypothesis $\EE \big[ \widetilde{\coveragealt}(\widehat{k}) \big] \le \alpha$ cannot be rejected (e.g.,~at the 0.05 significance level) for any LLM and $\alpha$ across both datasets. These experiment results verify the theoretical guarantees in \Cref{sec-general}, showing that the miscoverage rate is effectively controlled by our method. 

\begin{figure}[h]
	\centering
	\begin{minipage}{0.45\textwidth}
		\centering
		\includegraphics[width=0.8\linewidth]{figures/pval_OpinionQA.pdf}
	\end{minipage}%\hspace{em}%\hfill
	\begin{minipage}{0.45\textwidth}
		\centering
		\includegraphics[width=0.8\linewidth]{figures/pval_EEDI.pdf}
	\end{minipage}
	\caption{Histograms of $p$-values for the hypothesis test $\EE \big[ \widetilde{\coveragealt}(\widehat{k}) \big] \le \alpha$ against $\EE \big[ \widetilde{\coveragealt}(\widehat{k}) \big] > \alpha$ across various LLMs and $\alpha$'s over the OpinionQA (left) and EEDI (right) datasets.}
	\label{fig:combined}
\end{figure}



In \Cref{fig:k-hat}, we plot the average $\widehat{k}$ over the $100$ random splits for various LLMs on the OpinionQA and EEDI datasets. The error bars represent 95\% confidence intervals.
In general, a larger $\widehat{k}$ means that the LLM has stronger simulation power. On the OpinionQA dataset, GPT-4o has the best performance. On the EEDI dataset, DeepSeek-V3 has the best performance. Interestingly, on the OpinionQA dataset all LLMs clearly outperform the random benchmark, while on the EEDI dataset only DeepSeek-V3 and GPT-4o seem to outperform the random benchmark. Moreover, LLMs exhibit uniformly higher $\widehat{k}$ on the OpinionQA dataset than on the EEDI dataset, suggesting higher fidelity in simulating subjective opinions than in simulating answers to mathematics questions.


\begin{figure}[h]
	\centering
	\begin{minipage}{0.45\textwidth}
		\centering
		\includegraphics[width=0.8\linewidth]{figures/k_hat_OpinionQA.pdf}
	\end{minipage}
	\begin{minipage}{0.45\textwidth}
		\centering
		\includegraphics[width=0.8\linewidth]{figures/k_hat_EEDI.pdf}
	\end{minipage}
	\caption{Average $\widehat{k}$ for various LLMs and $\alpha$ over the OpinionQA (left) and EEDI (right) datasets.}
	\label{fig:k-hat}
\end{figure}



The experiment results demonstrate the importance of a disciplined approach to using synthetic samples. The ease of LLM-based simulation makes it tempting to generate a large number of responses per question. However, as can be seen from the figures, there is great heterogeneity in the simulation power of different LLMs over different datasets: the largest $\widehat{k}$ is below 100, while the smallest $\widehat{k}$ could be in the single digits. This means that there is real peril in using too many synthetic samples and being overly confident in the results.

\section{Discussions}\label{sec-discussions}

We developed a general approach for converting imperfect LLM-based survey simulations into statistically valid confidence sets for population statistics of human responses. It identifies a simulation sample size which is useful for future simulation tasks and which reveals the degree of misalignment between the LLM and the target human population. 

Several future directions are worth exploring. First, our approach does not explicitly minimize the size of the prediction set. A natural question is whether we can incorporate a size minimization procedure to produce smaller confidence sets with good coverage. Second, it would be interesting to see if our approach can be combined with debiasing methods to give more informative confidence sets. Finally, as prompt engineering is known to have crucial effects on the quality of LLM generations, it is worth investigating the impacts of prompts on the selected simulation sample size $\widehat{k}$, and how prompt engineering can be leveraged to reduce the misalignment gap between LLM simulations and true human responses.



\section*{Acknowledgement}
Chengpiao Huang and Kaizheng Wang's research is supported by an NSF grant DMS-2210907 and a Data Science Institute seed grant SF-181 at Columbia University.


\newpage 
\appendix


\section{More Examples for \Cref{sec-general}}\label{sec-examples}

In this section, we provide more examples for the general problem framework in \Cref{sec-general}. In these examples, the survey responses can be real-valued or multi-dimensional.


\begin{example}[Market research]
Suppose a company is interested in learning its customers' willingness-to-pay (WTP) for a new product, which is the highest price a customer is willing to pay for the product. Then, each $\profile \in \profilespace$ can represent a customer profile (e.g., age, gender, occupation), each survey question $\testfunction$ is about a certain product, and a customer's response $\response$ is a noisy observation of the customer's WTP. Then $\responsedistalt(~\cdot\mid \testfunction )$ is the distribution of the customer population's WTP. We may take $\statistic (\testfunction)$ as the $\tau$-quantile of the WTP distribution $\responsedistalt(~\cdot\mid \testfunction )$, for some $\tau\in(0,1)$:
\[
\statistic (\testfunction) = \inf \left\{ q\in[0,\infty) : \PP_{\response \sim \responsedistalt(\cdot\mid \testfunction )} (\response\le q) \ge \tau \right\}.
\] 
An LLM can be used to simulate customers' WTP for the product.
\end{example}

\begin{example}[Public survey, multi-dimensional]\label{example-public-survey}
Suppose an organization is interested in performing a public survey in a city. Each survey question $\testfunction$ is a multiple-choice question with $5$ options. An example is ``How often do you talk to your neighbors?'', with $5$ choices ``Basically every day'', ``A few times a week'', ``A few times a month'', ``Once a month'', and ``Less than once a month''. Every $\profile \in \profilespace$ is a person's profile (e.g., age, gender, occupation), the response space $\responsespace$ is the standard orthonormal basis $\{e_i\}_{i=1}^5$ in $\RR^5$, where $\response = e_i$ indicates that a person chooses the $i$-th option. We can take $\statistic ( \testfunction ) = \EE_{\response \sim \responsedist(\cdot\mid\testfunction)}[y]\in\RR^5$, which summarizes the proportion of people that choose the $i$-th option. An LLM can be used to simulate people's answers to the survey question.
\end{example}

\begin{example}[Public survey, one-dimensional]\label{example-public-survey-1D}
Consider the setup in \Cref{example-public-survey}. When the $5$ choices in a survey question correspond to ordered sentiments, we can map them to numeric scores, say, $v = (-1,-\frac{1}{3},0,\frac{1}{3},1)^\top$. Then the statistic $\widetilde{\statistic} (\testfunction) = \langle v,\statistic (\testfunction)  \rangle$ reflects the population's average sentiment in the survey question $\testfunction$.
\end{example}


\section{Proofs}

\subsection{Failure of Exact CLT-Based Intervals under Distribution Shift}\label{sec-impossibility-exact-CLT}

Consider a true distribution $N(\mean,1)$ and a synthetic distribution $N(\simmean,1)$. Suppose we draw $k$ i.i.d.~synthetic samples $\{ x_i \}_{i=1}^k \sim N(\simmean,1)$ and construct the standard $(1-\alpha)$ confidence interval for $\mean$:
\[
\simCI(k) = \left[ \bar{x}_k - \frac{\Phi^{-1}(1-\alpha/2)}{\sqrt{k}}, ~ \bar{x}_k + \frac{\Phi^{-1}(1-\alpha/2)}{\sqrt{k}} \right],
\]
where $\bar{x}_k = \frac{1}{k} \sum_{i=1}^k x_i$. Let $\Delta = \mean - \simmean$, which represents the discrepancy between the true distribution and the synthetic distribution. Then
\begin{align*}
\PP\Big( \mean \in \simCI(k) \Big)
&=
\PP\left( | \bar{x}_k - \mean | \le \frac{\Phi^{-1}(1-\alpha/2)}{\sqrt{k}} \right) \\[4pt]
&=
\PP\left( | \sqrt{k} (\bar{x}_k - \simmean) - \sqrt{k} (\mean - \simmean) | \le \Phi^{-1}(1-\alpha/2) \right) \\[4pt]
&=
\PP\left( \sqrt{k} \Delta - \Phi^{-1}(1-\alpha/2) \le \sqrt{k} (\bar{x}_k - \simmean) \le \sqrt{k} \Delta + \Phi^{-1}(1-\alpha/2) \right).
\end{align*}
Note that $\sqrt{k} (\bar{x}_k - \simmean) \sim N(0,1)$. Thus, whenever $\Delta\neq 0$,
\[
\PP\Big( \mean \in \simCI(k) \Big) < 1-\alpha,
\]
failing to attain $(1-\alpha)$ coverage probability, regardless of the sample size $k$.


\subsection{Proof of \Cref{thm-coverage-1D}}\label{sec-thm-coverage-1D-proof}

We will prove the following stronger guarantee.

\begin{lemma}[Conditional coverage]\label{lem-conditional-coverage-1D}
Consider the setting of \Cref{thm-coverage-1D}. Let $\delta\in(0,1)$. With probability at least $1-\delta$,
\begin{equation}\label{eqn-conditional-coverage-1D}
\PP\Big( \mean \in \simCI(\widehat{k}) \Bigm| \widehat{k} \Big) \ge 1 - \alpha - \sqrt{\frac{2\log(1/\delta)}{m}}.
\end{equation}
\end{lemma}

By \Cref{lem-conditional-coverage-1D}, we obtain
\begin{align*}
\PP\Big( \mean \in \simCI(\widehat{k}) \Big)
&=
\EE\left[ \PP \Big( \mean \in \simCI(\widehat{k}) \Bigm| \widehat{k} \Big) \right] \\[4pt]
&=
\int_0^{\infty} \PP \left( \PP \Big( \mean \in \simCI(\widehat{k}) \Bigm| \widehat{k} \Big) > t \right) \,dt \\[4pt]
&\ge 
\int_0^{1-\alpha} \left[ 1 - \exp\left( -\frac{m}{2} (t - (1-\alpha) )^2 \right) \right] \,dt \\[4pt]
&\ge 
1-\alpha - \int_{-\infty}^{1-\alpha} \exp\left( -\frac{m}{2} (t - (1-\alpha) )^2 \right) \,dt \\[4pt]
&\ge 
1-\alpha - \sqrt{\frac{2}{m}}.
\end{align*}

We will now prove \Cref{lem-conditional-coverage-1D}. Define $\varepsilon = \sqrt{2\log(1/\delta) / m}$ and a deterministic oracle sample size
\begin{equation}\label{eqn-oracle-k-1D}
\bar{k} = \inf \left\{ k\in [K] : \PP \Big( \mean \not\in \simCI(k) \Big)  
> \alpha + \varepsilon \right\}.
\end{equation}
If $\bar{k} = \inf\emptyset$ does not exist, then there is nothing to prove. Now suppose that $\bar{k} \in [K]$ exists. We will prove that with probability at least $1-\delta$, it holds that $\coverage(\bar{k}) > \alpha/2$. When this event happens, we have $\widehat{k} < \bar{k}$, which implies $\PP\big( \mean \not\in \simCI(\widehat{k}) \bigm| \widehat{k} \big) \le \alpha+\varepsilon$ and thus \eqref{eqn-conditional-coverage-1D}, thanks to the independence of $\widehat{k}$ and $(\testfunction,\simdataset)$.

By Hoeffding's inequality (e.g., Theorem 2.8 in \cite{BLM13}) and the conditional independence of $(\dataset_1,\simdataset_1),...,(\dataset_m,\simdataset_m)$ given $(\testfunction_1,...,\testfunction_m)$,
\begin{equation}\label{eqn-Hoeffding-1D}
\PP\left( \coverage(\bar{k}) \ge \frac{1}{m} \sum_{j=1}^m \PP\Big( \responsebar_j \not\in \simCI_j(\bar{k}) \Big) -  \sqrt{\frac{\log(1/\delta)}{2m}} \right) \ge 1-\delta.
\end{equation}
We now bound $\PP\big( \responsebar_j \not\in \simCI_j(\bar{k}) \big)$. For each $j\in[m]$ and $k\in[K]$,
\[
\ind \left\{ \responsebar_j \not\in \simCI_j(k) \right\}
\ge 
\ind \left\{ \responsebar_j < \mean_{j} \text{ and } \mean_{j} < \min \simCI_j(k) \right\}
+
\ind \left\{ \responsebar_j \ge \mean_{j} \text{ and } \mean_{j} > \max \simCI_j(k) \right\}.
\]
By the conditional independence of $\dataset_j$ and $\simdataset_j$ given $\testfunction_j$,
\begin{align*}
& \PP \Big( \responsebar_j < \mean_{j} \text{ and } \mean_{j} < \min \simCI_j(k) \Big) \\[4pt]
&=
\EE\Big[ \PP \Big( \responsebar_j < \mean_{j} \Bigm| \testfunction_j \Big) \cdot \PP \Big( \mean_{j} < \min \simCI_j(k)  \Bigm| \testfunction_j \Big) \Big] \\[4pt]
&=
\EE\left[ \frac{1}{2} \cdot \PP \Big( \mean_{j} < \min \simCI_j(k)  \Bigm| \testfunction_j \Big) \right] \\[4pt]
&=
\frac{1}{2} \PP\Big( \mean_{j} < \min \simCI_j(k)  \Big).
\end{align*}
Similarly,
\[
\PP \Big( \responsebar_j \ge \mean_{j} \text{ and } \mean_{j} > \max \simCI_j(k) \Big)
=
\frac{1}{2} \PP\Big( \mean_{j} > \max \simCI_j(k) \Big).
\]
Therefore,
\begin{align}
\PP \Big( \responsebar_j \not\in \simCI_j(k) \Big)
&\ge 
\frac{1}{2} \left[ \PP\Big( \mean_{j} < \min \simCI_j(k) \Big) + \PP\Big( \mean_{j} > \max \simCI_j(k) \Big) \right] \notag \\[4pt]
&=
\frac{1}{2} \PP \Big( \mean_{j} \not\in \simCI_j(k) \Big)
=
\frac{1}{2} \PP \Big( \mean \not\in \simCI(k) \Big), \label{eqn-proxy-to-oracle-precise-1D}
\end{align}
where the last equality is due to Assumption \ref{assumption-iid-test-1D}. When the event in \eqref{eqn-Hoeffding-1D} happens,
\begin{align*}
\coverage(\bar{k}) 
&
\ge \frac{1}{m} \sum_{j=1}^m \PP\Big( \responsebar_j \not\in \simCI_j(\bar{k}) \Big) -  \sqrt{\frac{\log(1/\delta)}{2m}} \\[4pt]
&\ge 
\frac{1}{2} \PP \Big( \mean \not\in \simCI(\bar{k}) \Big) -  \sqrt{\frac{\log(1/\delta)}{2m}} \tag{by \eqref{eqn-proxy-to-oracle-precise-1D}} \\[4pt]
&> 
\frac{\alpha}{2}. \tag{by definition of $\bar{k}$}
\end{align*}
This completes the proof.


\subsection{Proof of \Cref{thm-sharpness-1D}}\label{sec-thm-sharpness-1D-proof}

By Hoeffding's inequality (e.g., Theorem 2.8 in \cite{BLM13}) and a union bound, the following happens with probability at least $1-\delta$:
\begin{equation}\label{eqn-sharpness-proof-1}
\coverage(k) \le \frac{1}{m} \sum_{j=1}^m \PP\Big( \bar{y}_j \not\in \simCI_j(k) \Big)  +  \sqrt{\frac{\log(n/\delta)}{2m}}
,\quad \forall k \le \min \bigg\{ n,K,~ \bigg(\frac{C}{5 \Delta}\bigg)^2 \bigg\}
\end{equation}
We now show that the right hand side of \eqref{eqn-sharpness-proof-1} is at most $\alpha/2$ for $m$ large. For all $j\in[m]$ and $k \in [K]$,
\[
\PP \Big( \simresponse_{j} \not\in \simCI_{j}(k) \Big)
= \PP \left(|\simresponse_j -\simresponsebar_{j, k} | > \frac{C}{\sqrt{k}} \right).
\]
Since $\simresponsebar_{j, k} \sim N( \simmean_{j} , 1/k )$ and $\bar\response_j \sim N( \mean_{j}, 1/n )$, then
\[
\simresponse_j -
\simresponsebar_{j, k} 
\sim N \left( 
\mean_{j} - \simmean_{j}
, ~ \frac{1}{k} + \frac{1}{n} \right).
\]
When $k \leq \min \{ n,K,~ (\frac{C}{5 \Delta})^2 \}$, we have
\begin{align*}
\PP \Big( \bar{\response}_{j} \not\in \simCI_{j}(k) \Big) 
&\le \PP \left(
|
(\bar\response_j - \simresponsebar_{j, k} )
- (\mean_{j} - \simmean_{j})
|
+ \varepsilon > \frac{C}{\sqrt{k}}
\right)
\\[4pt]
& = 2 \Phi\left(
- \frac{C / \sqrt{k} - \varepsilon}{\sqrt{ k^{-1} + n^{-1} }}
\right)
\le 
2  \Phi\bigg(
- \frac{4	C / (5 \sqrt{k} )}{\sqrt{ k^{-1} + k^{-1} }}\bigg) \tag{$\Delta\le C/(5\sqrt{k})$} \\[4pt]
&= 2  \Phi \bigg(
- \frac{
2 \sqrt{2}	C 
}{
5
}
\bigg)
= 2  \Phi \bigg(
- \frac{
	4 \sqrt{2}	\Phi^{-1}(1 - \alpha/4)
}{
	5
}
\bigg)
=
2  \Phi \bigg(
\frac{
	4 \sqrt{2}	\Phi^{-1}(\alpha/4)
}{
	5
}
\bigg) < \frac{\alpha}{2}.
\end{align*}
Let
\[
\xi = \frac{\alpha}{2} - 2  \Phi \bigg(
\frac{
	4 \sqrt{2}	\Phi^{-1}(\alpha/4)
}{
	5
}
\bigg),
\]
then $\xi > 0$ and
\begin{equation}\label{eqn-sharpness-proof-2}
\PP \Big(\bar{\response}_{j} \notin \simCI_{j}(k) \Big) \le  \frac{\alpha}{2} - \xi.
\end{equation}

When $m > \log(n / \delta) / (2\xi^2)$, substituting \eqref{eqn-sharpness-proof-2} into \eqref{eqn-sharpness-proof-1} yields that for all $k \le \min\{ n , K, ~ ( \frac{C}{5\Delta})^2 \}$,
\begin{align*}
\coverage(k) 
&\le 
\frac{1}{m} \sum_{j=1}^m \PP\Big( \bar{y}_j \not\in \simCI_j(k) \Big)  +  \sqrt{\frac{\log(n/\delta)}{2m}} \\[4pt]
&<
\left( \frac{\alpha}{2} - \xi \right) + \xi = \frac{\alpha}{2}.
\end{align*}
When this happens, we have $\widehat{k} \ge \min  \{ n , K,~  (\frac{C}{5 \Delta} )^2  \}$. 


\subsection{Proof of \Cref{thm-coverage}}\label{sec-thm-coverage-proof}

We will prove the following stronger guarantee.

\begin{lemma}[Conditional coverage]\label{lem-conditional-coverage}
Consider the setting of \Cref{thm-coverage}. Let $\delta\in(0,1)$. With probability at least $1-\delta$,
\begin{equation}\label{eqn-conditional-coverage}
\PP\Big( \statistic(\testfunction)  \in \simCIalt(\widehat{k}) \Bigm| \widehat{k} \Big) \ge 1 - \alpha - \gamma^{-1} \sqrt{\frac{\log(1/\delta)}{2m}}.
\end{equation}
\end{lemma}

By \Cref{lem-conditional-coverage-1D}, we obtain
\begin{align*}
\PP\Big( \statistic (\testfunction)  \in \simCIalt(\widehat{k}) \Big)
&=
\EE\left[ \PP \Big( \statistic (\testfunction)  \in \simCIalt(\widehat{k}) \Bigm| \widehat{k} \Big) \right] \\[4pt]
&=
\int_0^{\infty} \PP \left( \PP \Big( \statistic (\testfunction)  \in \simCIalt(\widehat{k}) \Bigm| \widehat{k} \Big) > t \right) \,dt \\[4pt]
&\ge 
\int_0^{1-\alpha} \left[ 1 - \exp\left( -2m\gamma^2 (t - (1-\alpha) )^2 \right) \right] \,dt \\[4pt]
&\ge 
1-\alpha - \int_{-\infty}^{1-\alpha} \exp\left( -2m\gamma^2 (t - (1-\alpha) )^2 \right) \,dt \\[4pt]
&\ge 
1-\alpha - \gamma^{-1} \sqrt{\frac{1}{2m}}.
\end{align*}

We now prove \Cref{lem-conditional-coverage}. Define $\varepsilon = \gamma^{-1} \sqrt{\frac{\log(1/\delta)}{2m}} $ and a deterministic oracle sample size
\begin{equation}\label{eqn-oracle-k}
\bar{k} = \inf \left\{ k\in [K] : \PP \Big( \statistic (\testfunction)  \not\in \simCIalt(k) \Big)  
> \alpha + \varepsilon \right\}.
\end{equation}
If $\bar{k} = \inf \emptyset$ does not exist, then there is nothing to prove. Now suppose $\bar{k} \in [K]$ exists. We will prove that with probability at least $1-\delta$, it holds that $L(\bar{k}) > \gamma\alpha$. When this event happens, we have $\widehat{k} < \bar{k}$, which implies $\PP\big( \statistic (\testfunction)  \not\in \simCIalt(\widehat{k}) \bigm| \widehat{k} \big) \le \alpha+\varepsilon$ and thus \eqref{eqn-conditional-coverage}, thanks to the independence of $\widehat{k}$ and $(\testfunction,\simdataset)$.

By Hoeffding's inequality (e.g., Theorem 2.8 in \cite{BLM13}) and the conditional independence of $(\dataset_1,\simdataset_1),...,(\dataset_m,\simdataset_m)$ given $(\testfunction_1,...,\testfunction_m)$,
\begin{equation}\label{eqn-Hoeffding}
\PP\left( \coveragealt(\bar{k}) \ge \frac{1}{m} \sum_{j=1}^m \PP\Big( \CIalt_j \not\subseteq \simCIalt_j(\bar{k}) \Big) -  \sqrt{\frac{\log(1/\delta)}{2m}} \right) \ge 1-\delta.
\end{equation}
We now bound $\PP\big( \CIalt_j \not\subseteq \simCIalt_j(\bar{k}) \big)$. For each $j\in[m]$ and $j\in[K]$,
\[
\ind \left\{ \CIalt_j \not\subseteq \simCIalt_j(k) \right\}
\ge 
\ind \left\{ \statistic (\testfunction_j) \in \CIalt_j \text{ and } \statistic (\testfunction_j) \not\in \simCIalt_j(k) \right\}.
\]
By the conditional independence of $\dataset_j$ and $\simdataset_j$ given $\testfunction_j$,
\begin{align}
\PP \Big( \CIalt_j \not\subseteq \simCIalt_j(k) \Big)
&\ge
\EE\Big[ \PP \Big( \statistic (\testfunction_j) \in \CIalt_j \text{ and } \statistic (\testfunction_j) \not\in \simCIalt_j(k)  \Bigm| \testfunction_j \Big) \Big] \notag \\[4pt]
&=
\EE\Big[ \PP \Big( \statistic (\testfunction_j) \in \CIalt_j \Bigm| \testfunction_j \Big) \cdot \PP \Big( \statistic (\testfunction_j) \not\in \simCIalt_j(k)  \Bigm| \testfunction_j \Big) \Big] \notag \\[4pt]
&\ge
\EE\Big[ \gamma \cdot \PP \Big( \statistic (\testfunction_j) \not\in \simCIalt_j (k) \Bigm| \testfunction_j \Big) \Big] \notag \\[4pt]
&=
\gamma \cdot \PP\Big( \statistic (\testfunction_j) \not\in \simCIalt_j(k)  \Big) \notag \\[4pt]
&=
\gamma \cdot \PP\Big( \statistic (\testfunction)  \not\in \simCIalt(k)  \Big), \label{eqn-proxy-to-oracle-precise}
\end{align}
where the last equality is due to Assumption \ref{assumption-iid-test-1D}. Therefore, when the event in \eqref{eqn-Hoeffding} happens,
\begin{align*}
\coveragealt(\bar{k}) 
&
\ge \frac{1}{m} \sum_{j=1}^m \PP \Big( \CIalt_j \not\subseteq \simCIalt_j(\bar{k}) \Big) -  \sqrt{\frac{\log(1/\delta)}{2m}} \\[4pt]
&\ge 
\gamma \cdot \PP\Big( \statistic (\testfunction)  \not\in \simCIalt(\bar{k}) \Big) -  \sqrt{\frac{\log(1/\delta)}{2m}} \tag{by \eqref{eqn-proxy-to-oracle-precise}} \\[4pt]
&> 
\gamma\alpha. \tag{by definition of $\bar{k}$}
\end{align*}
This completes the proof.
\section{Experiments}\label{sec:experiments_extra}
The code used in this work will be made publicly available later.
%{The code to replicate our results can be found in this public Github repositroy:}
%\AL{make a new public github repo and include link later}
%\AN{We should adhere to dual blind policy during review process. Please refer to https://icml.cc/Conferences/2025/AuthorInstructions}
%\AN{This mean it is safe to submit the code as a supplement file (zipped). After the acceptance, we can make the code open on github.}
%\AL{Yes, I will consult you on this tomorrow, thank you.}

\subsection{Pseudocode and training settings for mean-field experiments} \label{subsec:pseudocode}

For experiments concerning MFNNs, the output of a neuron in a two-layer MFNN is modelled by: $h(x_i, z_i) = R\tanh(x_i^3) \tanh(x_i^{1\top} z_i + x_i^2)$, where $x_i = (x_i^1, x_i^2, x_i^3) \in \bR^{d + 1 + 1}$ is its parameter, $z_i$ is the given input and $R$ is a scaling constant. The $\tanh$ activation function is placed on the second layer as boundedness of the model is crucial for our analysis. Noisy gradient descent is then used to train neural networks for $T$ epochs each. We omit the pseudocode for training MFNNs with MFLD since it is identical to the backpropagation with noisy gradient descent algorithm.

\paragraph{Algorithm \ref{alg:circle_data}} Generate the double circle data: $\mathcal{D} = \left( z_i, y_i\right)^n_{i=1}$, $z_i \in \bR^2, y_i \in \bR$ before splitting it into $\mathcal{D}_\text{train}$ and $\mathcal{D}_{\text{test}}$. We set $n=200$, $r_\text{inner}=1$, $r_\text{outer}=2$ and use an 80-20 train-test split for the data.

\paragraph{Algorithm \ref{alg:multi_index_data}} Generate the $k$ multi-index data: $\mathcal{D} = \left( z_i, y_i\right)^n_{i=1}$, $z_i \in \bR^d, y_i \in \bR$. A key step is normalizing and projecting $z_i$ to the inside of a $d$-dimensional hypersphere. We set $n = 500$, $d = 100$, $r=5$, $k=100$ and $\bar{R} = 100$. 

\paragraph{Algorithm \ref{alg:classification}} Describes how we obtain and test the performance of merged MFNNs against (an approximation to) the mean-field limit by computing the sup-norm between both outputs. The relevant results are stored into a dictionary for plotting the heatmaps. The training procedure is identical for both the classification and regression problem. Let $M_\text{max} = 20$ and $N_\text{list} = \{50, 100, \dots ,500 \}$ denote the maximum number of networks to merge and list of neuron settings to train in parallel respectively. We set the hyperparameters for training as follows:
\begin{itemize}
    \item Classification: $R=10$, $N_\infty=10000$, $\eta = 0.1$, $\lambda' = 0.1$, $\lambda = 0.01$, $T=200$ and loss function: logistic loss
    \item Regression: $R=10$, $N_\infty=10000$, $\eta = 0.01$, $\lambda' = 0.1$, $\lambda = 0.01$, $T=100$ and loss function: mean squared error
\end{itemize}


\begin{algorithm} 
\caption{Generate data points along cocentric 2D circles}\label{alg:circle_data}
\begin{algorithmic}[1]
\REQUIRE $n$, $r_{\text{inner}}$, $r_{\text{outer}}$
\ENSURE Dataset $\mathcal{D} = \{(z_i, y_i)\}_{i=1}^n$
\STATE Initialize $\mathcal{D} \gets \emptyset$
\FOR{$i = 1$ to $n$}
    \STATE Sample $\theta \sim \text{Uniform}(0, 2\pi)$
    \STATE Sample $\xi_1, \xi_2 \sim \text{Normal}(0, 0.1)$
    \IF{$i <  n/2$}
        \STATE $r \gets r_{\text{inner}}$
        \STATE $y_i \gets -1$
    \ELSE
        \STATE $r \gets r_{\text{outer}}$
        \STATE $y_i \gets +1$
    \ENDIF
    \STATE Compute Cartesian coordinates: $z_i = (r \cos(\theta) + \xi_1, r \sin(\theta) + \xi_2)$
    \STATE Add $(z_i, y_i)$ to $\mathcal{D}$
\ENDFOR
\STATE Randomly shuffle $\mathcal{D}$
\STATE Split $\mathcal{D}$ into $\mathcal{D}_{\text{train}}$ (80\%) and $\mathcal{D}_{\text{test}}$ (20\%)
\STATE \textbf{return} $\mathcal{D}_{\text{train}}, \mathcal{D}_{\text{test}}$
\end{algorithmic}
\end{algorithm}

\clearpage
\begin{figure}[H]
\vspace{-1.5em}
\begin{algorithm}[H] 
\caption{Generate $k$ multi-index data}
\begin{algorithmic}[1] \label{alg:multi_index_data}
\REQUIRE $n$, $d$, $r$, $k$, $\bar{R}$
\ENSURE Dataset $\mathcal{D} = \{(z_i, y_i)\}_{i=1}^n$
\STATE Initialize $\mathcal{D} \gets \emptyset$
\FOR{$i = 1$ to $n$}
    \STATE Sample $\zeta \sim \text{Normal}(0,1)$
    \STATE $\zeta \gets \zeta^{(1/d)} \times r$ \hfill \COMMENT{Get scaling constant}
    \STATE Sample $z \sim \text{Normal}\left(0, \text{I}_d \right)$
    
    \STATE $z_i \gets z/ |z|$ \hfill \COMMENT{Normalize} 
    \STATE $z_i \gets z_i \times \zeta$ \hfill \COMMENT{Project}
    \STATE $y_i \gets 0$
    \FOR{$j = 1$ to $k$}
        \STATE $y_i \gets y_i + \tanh \left(z_i^j \right)$
    \ENDFOR
\STATE $y_i \gets y_i \times (\bar{R}/k)$
\STATE Add $(z_i, y_i)$ to $\mathcal{D}$
\ENDFOR
\STATE Split $\mathcal{D}$ into $\mathcal{D}_{\text{train}}$ (80\%) and $\mathcal{D}_{\text{test}}$ (20\%)
\STATE \textbf{return} $\mathcal{D}_{\text{train}}, \mathcal{D}_{\text{test}}$
\end{algorithmic}
\end{algorithm}
\end{figure}

\begin{algorithm} 
\caption{Training and merging MFNNs}\label{alg:classification}
\begin{algorithmic}[H]
\REQUIRE $\mathcal{D}_{\text{train}}, \mathcal{D}_{\text{test}} = (z_\text{test}, y_\text{test})$, $N_\infty$, $N_\text{list}$, $M_\text{max}$
\ENSURE Dictionary \textit{sup\_norm\_dic} maps $N$ to the average sup\_norm
\STATE $h_\infty \gets$ Train a MFNN with $N_\infty$ neurons on $\mathcal{D}_{\text{train}}$
\STATE $\hat{y}_\infty \gets$Use $h_\infty$ to predict on $\mathcal{D}_{\text{test}}$
\STATE Initialize \textit{sup\_norm\_dic} $\gets \{\}$
\FOR{$N \in N_\text{list}$}
    \STATE  $\{h_N^{1}, h_N^{2}, \dots h^{M_\text{max}}_N \} \gets$Train $M_\text{max}$ MFNNs with $N$ neurons on $\mathcal{D}_{\text{train}}$
    \STATE Initialize \textit{sup\_norm\_lst} $\gets []$

    \FOR{$M \in \{1, 2, \dots M_\text{max} \}$}
        \STATE \textit{sup\_norm\_total} $\gets 0$
        \FOR{50 iterations}
            \STATE Randomly sample $M$ networks from $\left \{ h_N^1, h_N^2, \dots, h^{M_\text{max}}_N \right \}$
            \STATE $h_{MN} \gets$Merge the $M$ networks to form a new neural network 
            \STATE $\hat{y} \gets$Use $h_{MN}$ to predict on $\mathcal{D}_{\text{test}}$
            \STATE \textit{sup\_norm} $\gets \text{max}\left(|\hat{y} - \hat{y}_\infty |\right)$
            \STATE $\text{\textit{sup\_norm\_total}} \gets \text{\textit{sup\_norm\_total}} + \text{\textit{sup\_norm}}$
        \ENDFOR
        \STATE Append \textit{sup\_norm\_total}/ 50 to \textit{sup\_norm\_lst}
    \ENDFOR
    \STATE \textit{sup\_norm\_dic}[\textit{N}] $\gets$ \textit{sup\_norm\_lst}
\ENDFOR
\STATE \textbf{return} \textit{sup\_norm\_dic}
\end{algorithmic}
\end{algorithm}


\clearpage

\subsection{Additional MFNN experiments} \label{subsec:additional_experiments}
Beyond examining the effect of both $M$ and $N$ on sup norm, we also compare the convergence rate of MFNNs using different $\lambda \in \{10^{-1}, 10^{-2}, 10^{-3}, 10^{-4} \}$ on the multi-index regression problem. Since the training dataset is small and we intend to investigate high $\lambda$, we have to consider the low epoch setting to prevent deterioration of generalization capabilities. We train 20 networks in parallel and average the MSE (in log-scale) at each epoch, repeating this for $N\in \{300, 400, \dots, 800\}$. Figure \ref{fig:experiments_extra} shows that higher $\lambda$ improves the convergence speed of particles and makes training more stable. Finally, we merge networks with the same hyperparameters for comparison across different $\lambda$. A similar trend is observed in Table \ref{table:experiments_extra}, highlighting the efficacy of PoC-based ensembling when training for fewer epochs with a high $\lambda$.

\begin{figure}[H]
\vskip 0.2in
\begin{center}
\centerline{\includegraphics[width=\columnwidth]{extra_experiment.png}}
\caption{Averaged test ln(MSE) of singular MFNNs, across different $N$ and $\lambda$ for 5 epochs}  
\label{fig:experiments_extra}
\end{center}
\end{figure}

\begin{table*}[th]
    \centering
    \caption{MSE comparison between merging $M=20$ networks across different $N$ and $\lambda$ after 5 epochs.}
    \label{table:experiments_extra}
    \begin{footnotesize}
    \begin{tabular}{ccccccc} 
    \toprule
    & \multicolumn{6}{c}{$\boldsymbol{N}$} \\
    $\boldsymbol{\lambda}$ & 300 & 400 & 500 & 600 & 700 & 800\\
    \midrule
    $10^{-1}$ & \underline{0.9132253} & \underline{0.9040508}& \underline{0.9075238}& \underline{0.9044338}& \underline{0.9030165}&  \underline{0.9022377}\\
    $10^{-2}$ & 1.2325489& 1.2229528& 1.2166352& 1.1978958 & 1.1921849& 1.1654898\\
    $10^{-3}$ & 1.5718020& 1.5668763& 1.5607907& 1.5581368& 1.5282313& 1.5234329\\
    $10^{-4}$ & 1.6987042& 1.6887244& 1.6631799& 1.6135653& 1.5860944& 1.5821924\\
    \bottomrule
    \end{tabular}
    \end{footnotesize}
\end{table*}


\clearpage


\subsection{LoRA for finetuning language models}\label{subsec:experiments_extra_lora}
To examine the effect of $\lambda$, we perform LoRA and PoC-based merging by varying $\lambda \in \{0,10^{-5},10^{-4}\}$ with one-epoch training. We optimize eight LoRA parameters of rank $N=32$ in parallel using noisy AdamW with the speficied $\lambda$. Table \ref{table:LoRA_comparison_1epoch} summarizes the results. For LoRA, the table lists the best result among the eight LoRA parameters based on the average accuracy across all datasets and also provides the average accuracies of the eight parameters for each dataset. We observed that for Llama2-7B with $\lambda=0$ and  $\lambda=10^{-5}$, the chances of the optimization converging are very low. Consequently, both the average accuracy of eight LoRAs and the accuracy of PoC-based merging are also low. This is because the regularization strength $\lambda$ controls the optimization speed as seen in Theorem \ref{theorem:mfld_convergence}. On the other hand, by using a high constant $\lambda=10^{-4}$ the average performance was improved, and PoC-based merging achieved quite high accuracy even with only one-epoch of training. This result suggests using high $\lambda$ to reduce the training costs, provided it does not negatively affect generalization error. For Llama3-8B, one-epoch training is sufficient to converge, and while LoRA performed well and PoC-based merging further improved the accuracies.

\begin{table*}[th]
    \centering
    \caption{Accuracy comparison of LoRA and PoC-based merging for finetuning Llama models (1 epoch).}
    \label{table:LoRA_comparison_1epoch}
    \begin{footnotesize}
    \begin{tabular}{cccccccccccc}
        \toprule
        \textbf{Model} & \textbf{Method} & \textbf{$\lambda$} & \textbf{SIQA} & \textbf{PIQA} & \textbf{WinoGrande} & \textbf{OBQA} & \textbf{ARC-c} & \textbf{ARC-e} & \textbf{BoolQ} & \textbf{HellaSwag} & \textbf{Ave.} \\
        \midrule
        \multirow{11}{*}{\begin{tabular}{c}Llama2\\7B\end{tabular}}
            & LoRA (best) & $0$  & $80.55$ & $82.86$ & $83.19$ & $81.60$ & $71.08$ & $84.51$ & $71.90$ & $90.21$ & $80.74$ \\
            & LoRA (ave.) & $0$ & $64.73$  & $76.31$ & $77.76$ & $68.70$ & $57.02$ & $69.02$ & $69.04$ & $70.63$ & $69.15$ \\
            & \textbf{PoC merge} & $0$ & $32.29$ & $62.57$ & $83.58$ & $22.20$ & $28.41$ & $29.42$ & $61.53$ & $28.50$ & $43.56$ \\
            \cmidrule(lr){2-12}
            & LoRA (best) & $10^{-5}$ & $80.14$ & $82.37$ & $83.43$ & $80.40$ & $68.86$ & $83.42$ & $71.68$ & $89.94$ & $80.03$ \\
            & LoRA (ave.) & $10^{-5}$  & $74.37$ & $74.12$ & $80.55$ & $67.50$ & $58.34$ & $71.98$ & $69.43$ & $66.25$ & $70.32$ \\
            & \textbf{PoC merge} & $10^{-5}$ & $74.56$ & $83.84$ & $85.16$ & $60.00$ & $63.14$ & $78.37$ & $68.72$ & $92.77$ & $75.82$ \\
            \cmidrule(lr){2-12}
            & LoRA (best) & $10^{-4}$ & $78.20$ & $80.90$ & $81.22$ & $78.40$ & $65.19$ & $79.00$ & $69.97$ & $86.50$ & $77.42$ \\
            & LoRA (ave.) & $10^{-4}$  & $74.42$ & $77.70$ & $76.08$ & $75.93$ & $60.93$ & $76.25$ & $65.68$ & $66.71$ & $71.71$ \\
            & \textbf{PoC merge} & $10^{-4}$ & $80.76$ & $82.15$ & $84.85$ & $84.80$ & $71.25$ & $85.35$ & $72.26$ & $91.65$ & $81.63$ \\
        \midrule
        \multirow{11}{*}{\begin{tabular}{c}Llama3\\8B\end{tabular}}
            & LoRA (best) & $0$ & $80.45$ & $88.47$ & $86.82$ & $87.60$ & $82.25$ & $90.87$ & $73.85$ & $95.78$ & $85.76$ \\
            & LoRA (ave.) & $0$  & $80.51$ & $88.87$ & $86.85$ & $87.00$ & $80.78$ & $90.98$ & $73.71$ & $95.84$ & $85.57$ \\
            & \textbf{PoC merge} & $0$ & $81.73$ & $88.96$ & $87.77$ & $88.00$ & $81.40$ & $91.71$ & $74.46$ & $96.45$ & $86.31$ \\
            \cmidrule(lr){2-12}
            & LoRA (best) & $10^{-5}$ & $80.50$ & $88.68$ & $86.98$ & $86.80$ & $81.48$ & $91.12$ & $75.14$ & $95.97$ & $85.83$ \\
            & LoRA (ave.) & $10^{-5}$  & $80.83$ & $88.64$ & $86.85$ & $87.05$ & $80.39$ & $90.76$ & $71.54$ & $95.87$ & $85.24$ \\
            & \textbf{PoC merge} & $10^{-5}$ & $81.53$ & $89.45$ & $87.92$ & $87.80$ & $82.25$ & $91.79$ & $75.54$ & $96.44$ & $86.59$ \\
            \cmidrule(lr){2-12}
            & LoRA (best) & $10^{-4}$ & $80.30$ & $88.57$ & $86.42$ & $87.20$ & $78.07$ & $89.81$ & $73.61$ & $95.14$ & $84.89$ \\
            & LoRA (ave.) & $10^{-4}$ & $80.00$ & $88.20$ & $85.69$ & $86.23$ & $78.86$ & $89.48$ & $73.08$ & $95.05$ & $84.57$ \\
            & \textbf{PoC merge} & $10^{-4}$ & $80.71$ & $89.72$ & $88.08$ & $89.00$ & $82.17$ & $91.79$ & $74.56$ & $96.36$ & $86.55$ \\
        \bottomrule
    \end{tabular}
    \end{footnotesize}
\end{table*}






{
\bibliographystyle{ims}
\bibliography{bib}
}



\end{document}
