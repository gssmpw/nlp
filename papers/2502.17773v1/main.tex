\pdfoutput=1

\documentclass[english,11pt]{article}

%
\setlength\unitlength{1mm}
\newcommand{\twodots}{\mathinner {\ldotp \ldotp}}
% bb font symbols
\newcommand{\Rho}{\mathrm{P}}
\newcommand{\Tau}{\mathrm{T}}

\newfont{\bbb}{msbm10 scaled 700}
\newcommand{\CCC}{\mbox{\bbb C}}

\newfont{\bb}{msbm10 scaled 1100}
\newcommand{\CC}{\mbox{\bb C}}
\newcommand{\PP}{\mbox{\bb P}}
\newcommand{\RR}{\mbox{\bb R}}
\newcommand{\QQ}{\mbox{\bb Q}}
\newcommand{\ZZ}{\mbox{\bb Z}}
\newcommand{\FF}{\mbox{\bb F}}
\newcommand{\GG}{\mbox{\bb G}}
\newcommand{\EE}{\mbox{\bb E}}
\newcommand{\NN}{\mbox{\bb N}}
\newcommand{\KK}{\mbox{\bb K}}
\newcommand{\HH}{\mbox{\bb H}}
\newcommand{\SSS}{\mbox{\bb S}}
\newcommand{\UU}{\mbox{\bb U}}
\newcommand{\VV}{\mbox{\bb V}}


\newcommand{\yy}{\mathbbm{y}}
\newcommand{\xx}{\mathbbm{x}}
\newcommand{\zz}{\mathbbm{z}}
\newcommand{\sss}{\mathbbm{s}}
\newcommand{\rr}{\mathbbm{r}}
\newcommand{\pp}{\mathbbm{p}}
\newcommand{\qq}{\mathbbm{q}}
\newcommand{\ww}{\mathbbm{w}}
\newcommand{\hh}{\mathbbm{h}}
\newcommand{\vvv}{\mathbbm{v}}

% Vectors

\newcommand{\av}{{\bf a}}
\newcommand{\bv}{{\bf b}}
\newcommand{\cv}{{\bf c}}
\newcommand{\dv}{{\bf d}}
\newcommand{\ev}{{\bf e}}
\newcommand{\fv}{{\bf f}}
\newcommand{\gv}{{\bf g}}
\newcommand{\hv}{{\bf h}}
\newcommand{\iv}{{\bf i}}
\newcommand{\jv}{{\bf j}}
\newcommand{\kv}{{\bf k}}
\newcommand{\lv}{{\bf l}}
\newcommand{\mv}{{\bf m}}
\newcommand{\nv}{{\bf n}}
\newcommand{\ov}{{\bf o}}
\newcommand{\pv}{{\bf p}}
\newcommand{\qv}{{\bf q}}
\newcommand{\rv}{{\bf r}}
\newcommand{\sv}{{\bf s}}
\newcommand{\tv}{{\bf t}}
\newcommand{\uv}{{\bf u}}
\newcommand{\wv}{{\bf w}}
\newcommand{\vv}{{\bf v}}
\newcommand{\xv}{{\bf x}}
\newcommand{\yv}{{\bf y}}
\newcommand{\zv}{{\bf z}}
\newcommand{\zerov}{{\bf 0}}
\newcommand{\onev}{{\bf 1}}

% Matrices

\newcommand{\Am}{{\bf A}}
\newcommand{\Bm}{{\bf B}}
\newcommand{\Cm}{{\bf C}}
\newcommand{\Dm}{{\bf D}}
\newcommand{\Em}{{\bf E}}
\newcommand{\Fm}{{\bf F}}
\newcommand{\Gm}{{\bf G}}
\newcommand{\Hm}{{\bf H}}
\newcommand{\Id}{{\bf I}}
\newcommand{\Jm}{{\bf J}}
\newcommand{\Km}{{\bf K}}
\newcommand{\Lm}{{\bf L}}
\newcommand{\Mm}{{\bf M}}
\newcommand{\Nm}{{\bf N}}
\newcommand{\Om}{{\bf O}}
\newcommand{\Pm}{{\bf P}}
\newcommand{\Qm}{{\bf Q}}
\newcommand{\Rm}{{\bf R}}
\newcommand{\Sm}{{\bf S}}
\newcommand{\Tm}{{\bf T}}
\newcommand{\Um}{{\bf U}}
\newcommand{\Wm}{{\bf W}}
\newcommand{\Vm}{{\bf V}}
\newcommand{\Xm}{{\bf X}}
\newcommand{\Ym}{{\bf Y}}
\newcommand{\Zm}{{\bf Z}}

% Calligraphic

\newcommand{\Ac}{{\cal A}}
\newcommand{\Bc}{{\cal B}}
\newcommand{\Cc}{{\cal C}}
\newcommand{\Dc}{{\cal D}}
\newcommand{\Ec}{{\cal E}}
\newcommand{\Fc}{{\cal F}}
\newcommand{\Gc}{{\cal G}}
\newcommand{\Hc}{{\cal H}}
\newcommand{\Ic}{{\cal I}}
\newcommand{\Jc}{{\cal J}}
\newcommand{\Kc}{{\cal K}}
\newcommand{\Lc}{{\cal L}}
\newcommand{\Mc}{{\cal M}}
\newcommand{\Nc}{{\cal N}}
\newcommand{\nc}{{\cal n}}
\newcommand{\Oc}{{\cal O}}
\newcommand{\Pc}{{\cal P}}
\newcommand{\Qc}{{\cal Q}}
\newcommand{\Rc}{{\cal R}}
\newcommand{\Sc}{{\cal S}}
\newcommand{\Tc}{{\cal T}}
\newcommand{\Uc}{{\cal U}}
\newcommand{\Wc}{{\cal W}}
\newcommand{\Vc}{{\cal V}}
\newcommand{\Xc}{{\cal X}}
\newcommand{\Yc}{{\cal Y}}
\newcommand{\Zc}{{\cal Z}}

% Bold greek letters

\newcommand{\alphav}{\hbox{\boldmath$\alpha$}}
\newcommand{\betav}{\hbox{\boldmath$\beta$}}
\newcommand{\gammav}{\hbox{\boldmath$\gamma$}}
\newcommand{\deltav}{\hbox{\boldmath$\delta$}}
\newcommand{\etav}{\hbox{\boldmath$\eta$}}
\newcommand{\lambdav}{\hbox{\boldmath$\lambda$}}
\newcommand{\epsilonv}{\hbox{\boldmath$\epsilon$}}
\newcommand{\nuv}{\hbox{\boldmath$\nu$}}
\newcommand{\muv}{\hbox{\boldmath$\mu$}}
\newcommand{\zetav}{\hbox{\boldmath$\zeta$}}
\newcommand{\phiv}{\hbox{\boldmath$\phi$}}
\newcommand{\psiv}{\hbox{\boldmath$\psi$}}
\newcommand{\thetav}{\hbox{\boldmath$\theta$}}
\newcommand{\tauv}{\hbox{\boldmath$\tau$}}
\newcommand{\omegav}{\hbox{\boldmath$\omega$}}
\newcommand{\xiv}{\hbox{\boldmath$\xi$}}
\newcommand{\sigmav}{\hbox{\boldmath$\sigma$}}
\newcommand{\piv}{\hbox{\boldmath$\pi$}}
\newcommand{\rhov}{\hbox{\boldmath$\rho$}}
\newcommand{\upsilonv}{\hbox{\boldmath$\upsilon$}}

\newcommand{\Gammam}{\hbox{\boldmath$\Gamma$}}
\newcommand{\Lambdam}{\hbox{\boldmath$\Lambda$}}
\newcommand{\Deltam}{\hbox{\boldmath$\Delta$}}
\newcommand{\Sigmam}{\hbox{\boldmath$\Sigma$}}
\newcommand{\Phim}{\hbox{\boldmath$\Phi$}}
\newcommand{\Pim}{\hbox{\boldmath$\Pi$}}
\newcommand{\Psim}{\hbox{\boldmath$\Psi$}}
\newcommand{\Thetam}{\hbox{\boldmath$\Theta$}}
\newcommand{\Omegam}{\hbox{\boldmath$\Omega$}}
\newcommand{\Xim}{\hbox{\boldmath$\Xi$}}


% Sans Serif small case

\newcommand{\Gsf}{{\sf G}}

\newcommand{\asf}{{\sf a}}
\newcommand{\bsf}{{\sf b}}
\newcommand{\csf}{{\sf c}}
\newcommand{\dsf}{{\sf d}}
\newcommand{\esf}{{\sf e}}
\newcommand{\fsf}{{\sf f}}
\newcommand{\gsf}{{\sf g}}
\newcommand{\hsf}{{\sf h}}
\newcommand{\isf}{{\sf i}}
\newcommand{\jsf}{{\sf j}}
\newcommand{\ksf}{{\sf k}}
\newcommand{\lsf}{{\sf l}}
\newcommand{\msf}{{\sf m}}
\newcommand{\nsf}{{\sf n}}
\newcommand{\osf}{{\sf o}}
\newcommand{\psf}{{\sf p}}
\newcommand{\qsf}{{\sf q}}
\newcommand{\rsf}{{\sf r}}
\newcommand{\ssf}{{\sf s}}
\newcommand{\tsf}{{\sf t}}
\newcommand{\usf}{{\sf u}}
\newcommand{\wsf}{{\sf w}}
\newcommand{\vsf}{{\sf v}}
\newcommand{\xsf}{{\sf x}}
\newcommand{\ysf}{{\sf y}}
\newcommand{\zsf}{{\sf z}}


% mixed symbols

\newcommand{\sinc}{{\hbox{sinc}}}
\newcommand{\diag}{{\hbox{diag}}}
\renewcommand{\det}{{\hbox{det}}}
\newcommand{\trace}{{\hbox{tr}}}
\newcommand{\sign}{{\hbox{sign}}}
\renewcommand{\arg}{{\hbox{arg}}}
\newcommand{\var}{{\hbox{var}}}
\newcommand{\cov}{{\hbox{cov}}}
\newcommand{\Ei}{{\rm E}_{\rm i}}
\renewcommand{\Re}{{\rm Re}}
\renewcommand{\Im}{{\rm Im}}
\newcommand{\eqdef}{\stackrel{\Delta}{=}}
\newcommand{\defines}{{\,\,\stackrel{\scriptscriptstyle \bigtriangleup}{=}\,\,}}
\newcommand{\<}{\left\langle}
\renewcommand{\>}{\right\rangle}
\newcommand{\herm}{{\sf H}}
\newcommand{\trasp}{{\sf T}}
\newcommand{\transp}{{\sf T}}
\renewcommand{\vec}{{\rm vec}}
\newcommand{\Psf}{{\sf P}}
\newcommand{\SINR}{{\sf SINR}}
\newcommand{\SNR}{{\sf SNR}}
\newcommand{\MMSE}{{\sf MMSE}}
\newcommand{\REF}{{\RED [REF]}}

% Markov chain
\usepackage{stmaryrd} % for \mkv 
\newcommand{\mkv}{-\!\!\!\!\minuso\!\!\!\!-}

% Colors

\newcommand{\RED}{\color[rgb]{1.00,0.10,0.10}}
\newcommand{\BLUE}{\color[rgb]{0,0,0.90}}
\newcommand{\GREEN}{\color[rgb]{0,0.80,0.20}}

%%%%%%%%%%%%%%%%%%%%%%%%%%%%%%%%%%%%%%%%%%
\usepackage{hyperref}
\hypersetup{
    bookmarks=true,         % show bookmarks bar?
    unicode=false,          % non-Latin characters in AcrobatÕs bookmarks
    pdftoolbar=true,        % show AcrobatÕs toolbar?
    pdfmenubar=true,        % show AcrobatÕs menu?
    pdffitwindow=false,     % window fit to page when opened
    pdfstartview={FitH},    % fits the width of the page to the window
%    pdftitle={My title},    % title
%    pdfauthor={Author},     % author
%    pdfsubject={Subject},   % subject of the document
%    pdfcreator={Creator},   % creator of the document
%    pdfproducer={Producer}, % producer of the document
%    pdfkeywords={keyword1} {key2} {key3}, % list of keywords
    pdfnewwindow=true,      % links in new window
    colorlinks=true,       % false: boxed links; true: colored links
    linkcolor=red,          % color of internal links (change box color with linkbordercolor)
    citecolor=green,        % color of links to bibliography
    filecolor=blue,      % color of file links
    urlcolor=blue           % color of external links
}
%%%%%%%%%%%%%%%%%%%%%%%%%%%%%%%%%%%%%%%%%%%






\begin{document}


\title{Uncertainty Quantification for LLM-Based Survey Simulations}




\author{Chengpiao Huang\thanks{Department of IEOR, Columbia University. Email: \texttt{chengpiao.huang@columbia.edu}.}
	\and Yuhang Wu\thanks{Decision, Risk, and Operations Division, Columbia Business School. Email: \texttt{yuhang.wu@columbia.edu}}
	\and Kaizheng Wang\thanks{Department of IEOR and Data Science Institute, Columbia University. Email: \texttt{kaizheng.wang@columbia.edu}.}
}


\date{This version: \today}

\maketitle

\begin{abstract}
We investigate the reliable use of simulated survey responses from large language models (LLMs) through the lens of uncertainty quantification. Our approach converts synthetic data into confidence sets for population parameters of human responses, addressing the distribution shift between the simulated and real populations. A key innovation lies in determining the optimal number of simulated responses: too many produce overly narrow confidence sets with poor coverage, while too few yield excessively loose estimates. To resolve this, our method adaptively selects the simulation sample size, ensuring valid average-case coverage guarantees. It is broadly applicable to any LLM, irrespective of its fidelity, and any procedure for constructing confidence sets. Additionally, the selected sample size quantifies the degree of misalignment between the LLM and the target human population. We illustrate our method on real datasets and LLMs. 
\end{abstract}
\noindent{\bf Keywords:} Synthetic data, Large language models, Uncertainty quantification, Simulation




\vspace{-2mm}
\section{Introduction}


Diffusion models have achieved significant success across various domains, including computer vision and scientific fields \citep{ramesh2021zero,watson2023novo}.
These models enable sampling from complex natural image spaces or molecular spaces that resemble natural structures. Beyond the capabilities of such pre-trained diffusion models, there is often a need to optimize downstream reward functions. For instance, in text-to-image diffusion models, the reward function may be the alignment score \citep{black2023training,fan2023dpok,uehara2024feedback}, while in protein sequence diffusion models, it could include metrics such as stability, structural constraints, or binding affinity \citep{verkuil2022language}, and in DNA sequence diffusion models, it may involve activity levels \citep{sarkar2024designing,lal2024reglm}.






 
Building on the motivation above, we focus on optimizing downstream reward functions while preserving the naturalness of the designs. (e.g., a natural-like protein sequence exhibiting strong binding affinity) by seamlessly integrating these reward functions with pre-trained diffusion models during inference. While numerous studies have proposed to incorporate rewards 
into the generation process of diffusion models (e.g., classifier guidance \citep{dhariwal2021diffusion} by setting rewards as classifiers, derivative-free methods \citep{wu2024practical,li2024derivative}), they rely on a \emph{single-shot} denoising pass for generation. However, a natural question arises:

\emph{Can we further leverage inference-time computation during generation to refine the model’s output?}

\begin{figure}[!t]
    \centering
    \includegraphics[width=\linewidth]{images/main_all_proposal.png}
    \caption{
    Our proposed framework follows an iterative process, with each iteration injecting noise into the sample and then denoising it while optimizing rewards. For sequences, this can be implemented via masked diffusion, initialized from pre-trained diffusion models (left). Our algorithm can continuously refine the outputs by gradually correcting errors introduced during reward-guided denoising, improving the design over successive iterations (middle). For instance, for the task of optimizing the similarity (RMSD) of a protein to a target structure (\textcolor{red}{Red}), we can progressively minimize the RMSD through refinement, optimizing the design from an initial (\textcolor{orange}{Orange}) fit to a better final fit (\textcolor{green}{Green}), as shown on the right.} 
    \label{fig:propsoal}
\end{figure}

+
In this study, we observe that diffusion models can inherently support an \emph{iterative} generation procedure, where the design can be progressively refined through successive cycles of masking and noise removal. This allows us to utilize arbitrarily large amounts of computation during generation to continuously improve the design. 



Motivated by the above observations, we propose a novel framework for test-time reward optimization with diffusion models. Our approach employs an iterative refinement algorithm consisting of two steps in each iteration: partial noising and reward-guided denoising as in \pref{fig:propsoal}. The reward-guided denoising step transitions from partially noised states to denoised states using techniques such as classifier guidance or derivative-free guidance. Unlike existing \emph{single-shot} methods, our approach offers several advantages. First, our sequential refinement process allows for the gradual correction of errors introduced during reward-guided denoising, enabling us to optimize complex reward functions, such as structural properties in protein sequence design. In particular, this correction is expected to be crucial in recent successful masked diffusion models \citep{sahoo2024simple,shi2024simplified}, as once a token is demasked, it remains unchanged until the end of the denoising step. Besides, for reward functions with hard constraints, commonly encountered in biological sequence or molecular design (e.g., cell-type-specific DNA design \citep{gosai2023machine,lal2024reglm} or binders with high specificity), our framework can effectively optimize such reward functions by initializing seed sequences within feasible regions that satisfy these constraints.

{ Our contribution is summarized as follows. First, we propose a new reward-guided generation framework for diffusion models that sequentially refines the generated outputs (\pref{sec:iterative}). Our algorithm addresses two major issues in existing methods such as the lack of a correction mechanism and difficulties of handling hard constraints. Secondly, we provide a theoretical formulation demonstrating that our algorithm samples from the desirable distribution $\exp(r(x))p^{\pre}(\cdot)$, where $p^{\pre}(\cdot)$ is a pre-trained distribution (\pref{sec:analysis}) and $r(\cdot)$ is a reward function.
Finally, we present a specific instantiation of our unified framework by carefully designing the reward-guided denoising stage in each iteration, which bears similarities to evolutionary algorithms (\pref{sec:practical}). Using this approach, we experimentally demonstrate that our algorithm effectively optimizes reward functions, outperforming existing methods in computational protein and DNA design (\pref{sec:experiment}).} 

\section{Warm-up: Simulation of Binary Responses}\label{sec-warmup}

To motivate our problem and methodology, we will start with a simple setting where an LLM simulates binary responses to a survey question. In \Cref{sec-general}, we will present the general problem setup and the general methodology.

\subsection{Motivating Example: Educational Test}\label{sec-example-education}

Suppose a school wants to estimate the proportion $\mu \in [0,1]$ of students that can answer a newly designed test question correctly. It will not only provide insights into student progress but also evaluate the question's effectiveness in differentiating among students with varying levels of understanding. Such information can guide the school in tailoring teaching strategies to better address student needs.

The most direct approach is to give the test to $n$ students and collect their results $y_1,...,y_n\in\{0,1\}$, where $y_i$ indicates whether student $i$ answers the question correctly. A point estimate for $\mean$ is the sample mean $\responsebar = \frac{1}{n} \sum_{i=1}^n \response_i$. Given $\alpha\in(0,1)$, we can construct a confidence interval for $\mean$:
\begin{equation}\label{eqn-CI-intro-standard}
\left[ \responsebar - \frac{\samplesd}{\sqrt{n}} \Phi^{-1}\left( 1- \frac{\alpha}{2} \right) , ~  \responsebar + \frac{\samplesd}{\sqrt{n}} \Phi^{-1}\left( 1 - \frac{\alpha}{2} \right) \right],
\end{equation}
where $\samplesd = \sqrt{\responsebar (1-\responsebar) }$ is the sample standard deviation, and $\Phi$ is the cumulative distribution function (CDF) of $\Normal(0,1)$. By the Central Limit Theorem (CLT), this interval has asymptotic coverage probability $1-\alpha$ as $n\to\infty$. As a different approach, one can also use Hoeffding's concentration inequality (e.g., Theorem 2.8 in \cite{BLM13}) to construct a finite-sample confidence interval
\begin{equation}
\left[ \responsebar - \sqrt{\frac{\log(2/\alpha)}{2n}}  , ~  \responsebar + \sqrt{\frac{\log(2/\alpha)}{2n}} \right],
\end{equation}
which has at least $(1-\alpha)$ coverage probability for every $n\in\ZZ_+$. For simplicity, we will stick to \eqref{eqn-CI-intro-standard} in this section.

Alternatively, the school may use an LLM to simulate students' responses to the question. Compared with directly testing on real students, this approach is more time-efficient and cost-saving. If we prompt the LLM $k$ times with random student profiles, then it generates $k$ synthetic responses, which leads to synthetic outcomes $\simresponse_1,...,\simresponse_k\in\{0,1\}$. We may also compute the sample mean $\simresponsebar_k = \frac{1}{k} \sum_{i=1}^k \simresponse_i$ and the CLT-based confidence interval
\begin{equation}\label{eqn-CI-intro-sim}
\simCI(k) =  \bigg[ \simresponsebar_k - \frac{c\cdot \simsamplesd_k}{\sqrt{k}} \Phi^{-1}\left( 1- \frac{\alpha}{2} \right) , 
~
\simresponsebar_k + \frac{c\cdot \simsamplesd_k}{\sqrt{k}} \Phi^{-1}\left( 1 - \frac{\alpha}{2} \right) \bigg],
\end{equation}
where $\simsamplesd_k = \sqrt{\simresponsebar_k (1-\simresponsebar_k)}$, and $c>1$ is a scaling parameter. Such a dilation by $c$ is necessary; without it, whenever the LLM-generated data deviates from the student population (even by the slightest amount), the interval $\simCI(k)$ may never achieve $(1-\alpha)$ coverage regardless of $k$. We give an example in \Cref{sec-impossibility-exact-CLT}.

Due to the misalignment between the LLM and students, the distribution of the synthetic data $\{\simresponse_i\}_{i=1}^k$ may be very different from the true response distribution. In this case, the sample mean $\simresponsebar$ can be a poor estimate of 
$\mean$, and $\simCI(k)$ is generally not a valid confidence interval for $\mean$. In particular, as $k\to\infty$, the interval concentrates tightly around the synthetic mean $\EE [\simresponse_1]$ and fails to cover the true mean $\mean$. When $k$ is small, the interval becomes too wide to be informative, even though it may cover $\mean$ with high probability. We provide an illustration in \Cref{fig-tradeoff}.

\begin{figure}[h]
\centering
\includegraphics[scale=0.6]{figures/tradeoff.pdf}
\caption{The coverage-width trade-off for the simulation sample size $k$. The true distribution is $\Ber(0.4)$ and the synthetic distribution is $\Ber(0.6)$. The red dotted horizontal line plots the true mean $\mean=0.4$. The blue line plots the sample mean $\simresponsebar_k$ of the synthetic data, and the blue shaded region visualizes the confidence interval $\simCI(k)$, for $k\in[40]$. When $k$ is too small (say $k\le 6$), the interval $\simCI(k)$ is too wide. When $k$ is too large (say $k\ge 18$), interval $\simCI(k)$ fails to cover $\mean$.}\label{fig-tradeoff}
\end{figure}

\paragraph{Main insights.} Our goal in this work is to develop a principled approach for choosing a good simulation sample size $\widehat{k}$, so that $\simCI( \widehat{k} )$ is a valid confidence interval for $\mean$ while having a modest width. Solving this problem has the following important implications. 
\begin{enumerate}
\item The choice of $\widehat{k}$ offers valuable information for future simulation tasks on the appropriate number of synthetic samples to generate, so as to produce reliable confidence intervals. It also helps avoid generating excessive samples and improves computational efficiency.
\item The width of $\simCI( \widehat{k} )$ provides an assessment of the alignment between the LLM and the human population. A wide confidence interval indicates high uncertainty of its estimate of the true $\mean$, and thus a large gap between the synthetic data distribution and the true population.
\item The sample size $\widehat{k}$ also reflects the richness of the human population captured by the LLM. We make an analogy using the classical theory of parametric bootstrap. If a model is trained via maximum likelihood estimation over $k$ i.i.d.~human samples, then when performing bootstrap for uncertainty quantification, the bootstrap sample size is usually set to be $k$. In this regard, our approach can be thought of as uncovering the size of human population that the LLM can represent:
\begin{center}
	\emph{Using the LLM to simulate survey responses is roughly as accurate as surveying $\widehat{k}$ real people.}
\end{center}
We provide an illustration in \Cref{fig-LLMTurk}. 
The larger $\widehat{k}$ is, the more diversity that the LLM appears to capture. In contrast, a small $\widehat{k}$ could imply the peculiarity of the LLM compared to the major population. %Hence, $\widehat{k}$ gauges the alignment between the LLM and the human population.
\end{enumerate}

\begin{remark}[Comparison with existing works]
Existing works typically measure LLM misalignment using integral probability metrics and $f$-divergences \citep{SDL23, DHM24, DNL24}, which do not carry operational meanings themselves and at times can be hard to interpret. In contrast, our simulation sample size $\widehat{k}$ is scale-free and has a clear operational meaning.
\end{remark} 

\begin{figure}
\centering
\includegraphics[scale=0.15]{figures/LLMTurk}
\caption{An interpretation of the simulation sample size $k$ as the size of human population that an LLM can represent. Generating outputs from the LLM can be thought of as ``resampling'' from $k$ human samples that make up the LLM. The figure is generated by DALL-E \citep{RPG21}, and borrows ideas from the \emph{Mechanical Turk}, a chess-playing machine from the 18th century with a human player hidden inside.}\label{fig-LLMTurk}
\end{figure}


\subsection{Methodology for Selecting the Simulation Sample Size}\label{sec-method-1D}

We now introduce our method for choosing a good simulation sample size $\widehat{k}$. It makes use of similar test questions for which real students' results are available. If such data is available, we can compare LLM simulations with real students' results on these questions, and use it to guide the choice of $k$. 

Specifically, we assume access to $m$ test questions similar to the question of interest. For example, they can come from previous tests or a question bank. For $j\in[m]$, the $j$-th test question has been tested on $n_j$ real students, with test results $\dataset_j = \{ \response_{j,i} \}_{i=1}^{n_j}$. We also simulate LLM responses $\simdataset_j = \{ \simresponse_{j,i} \}_{i=1}^K$ to the $j$-th test question, and $\simdataset = \{ \simresponse_{i} \}_{i=1}^K$ to the new test question. Here $K\in\ZZ_+$ is the simulation budget.

For each question $j\in[m]$, we form confidence intervals similar to \eqref{eqn-CI-intro-sim} using the synthetic data $\simdataset_j$, aiming to cover the true proportion $\mean_j$ of students that can answer the $j$-th question correctly:
\begin{equation}\label{eqn-CI-intro-sim-calibrate}
\simCI_j(k) = \bigg[ \simresponsebar_{j,k} - \frac{c\cdot \simsamplesd_{j,k}}{\sqrt{k}} \Phi^{-1}\left( 1- \frac{\alpha}{2} \right) , 
~  \simresponsebar_{j,k} + \frac{c\cdot \simsamplesd_{j,k}}{\sqrt{k}} \Phi^{-1}\left( 1 - \frac{\alpha}{2} \right) \bigg], 
\end{equation}
where $\simresponsebar_{j,k} = \frac{1}{k} \sum_{i=1}^k \simresponse_{j,i}$ is the sample mean of the first $k$ samples in $\simdataset_j$, and $\simsamplesd_{j,k} = \sqrt{\simresponsebar_{j,k} ( 1 - \simresponsebar_{j,k} )}$ is the estimated standard deviation. We also set the convention $\simCI(0) = \simCI_j(0) = \RR$, as nothing can be said about the true parameter without data. We will pick $\widehat{k}\in\{0,1...,K\}$ such that $\simCI_j(\widehat{k})$ covers $\mean_j$ with high probability. We expect this choice of $\widehat{k}$ to be also good for $\simCI(k)$, as the test questions are similar.

Ideally, we would like to pick $k$ such that $(1-\alpha)$-coverage is achieved empirically over the $m$ test questions:
\begin{equation}\label{eqn-oracle-criterion-1D}
\frac{1}{m} \sum_{j=1}^m \ind \{ \mean_j \not\in \simCI_j(k) \} \le \alpha.
\end{equation}
As the true $\{\mean_j\}_{j=1}^m$ are not available, we use the real data $\{ \dataset_j \}_{j=1}^m$ to compute the sample means $\responsebar_j = \frac{1}{n_j} \sum_{i=1}^{n_j} \response_{j,i}$ as proxies for $\mean_j$. The empirical miscoverage can be approximated by
\begin{equation}\label{eqn-proxy-1D}
\coverage(k) = \frac{1}{m} \sum_{j=1}^m \ind \{ \responsebar_j \not\in \simCI_j(k) \}.
\end{equation}
Our criterion for selecting $k$ is given by
\begin{equation}\label{eqn-empirical-criterion-1D}
\widehat{k} = \max \left\{ 0\le k \le K : \coverage(i) \le \alpha/2 ~~\forall i\le k \right\}.
\end{equation}
Note that $\widehat{k}$ is well-defined because $\coverage(0) = 0$.

The choice of the threshold $\alpha/2$ in \eqref{eqn-empirical-criterion-1D} can be explained as follows. By CLT, when $n_j$ is large, $\PP( \mean_{j} \ge \responsebar_j ) \approx 1/2$. Suppose $\mean_{j} \not\in \simCI_j(k)$, then  $\mean_{j}$ is either on the left or the right of $\simCI_j(k)$. In the former case, $\responsebar_j$ is on the left of $\mean_{j}$ with probability around $1/2$, which implies $\responsebar_j \not\in \simCI_j(k)$. Similarly, in the latter case, $\responsebar_j$ is on the right of $\mean_{j}$ with probability around $1/2$, and then $\responsebar_j \not\in \simCI_j(k)$. Roughly speaking, the frequency of having $\responsebar_j \not\in \simCI_j(k)$ is at least half of the frequency of having $\mean_{j} \not\in \simCI_j(k)$. In other words, the lower bound
\begin{equation}\label{eqn-proxy-to-oracle-1D}
G(k) \geq \frac{1}{2} \cdot \frac{1}{m} \sum_{j=1}^m \ind \{ \mean_{j} \not\in \simCI_j(k) \} 
\end{equation}
approximately holds. Substituting \eqref{eqn-proxy-to-oracle-1D} into \eqref{eqn-oracle-criterion-1D} yields the threshold $\alpha/2$ for choosing $\widehat{k}$. 


\subsection{Theoretical Analysis}\label{sec-theory-1D}

In this section, we present a theoretical analysis of our proposed method. To do so, we first describe the setup in \Cref{sec-example-education} and \Cref{sec-method-1D} in mathematical terms.

The student population can be represented by a distribution $\distribution$ over a space $\profilespace$ of possible \emph{student profiles}, say, vectors of background information, classes taken, grades, etc. To simulate student responses from the LLM, synthetic student profiles are generated from a synthetic student population $\simdistribution$ over $\profilespace$, and then fed to the LLM.

We use $\testfunction$ and $\{ \testfunction_j \}_{j=1}^m$ to refer to the test question of interest and the $m$ similar ones, respectively. 
Students' performance on test questions are characterized by a \emph{performance function} $\performancefunction$: a student with profile $\profile\in\profilespace$ answers a question $\testfunction$ correctly with probability $\performancefunction ( \profile , \testfunction ) \in [0, 1]$. The average student performance on the test questions $\testfunction$ and $\{ \testfunction_j \}_{j=1}^m$ are then $\mean = \EE_{\profile \sim \distribution } \performancefunction ( \profile ,  \testfunction ) $ and $\mean_j = \EE_{\profile \sim \distribution } \performancefunction (  \profile, \testfunction_j ) $, respectively.
In addition, the LLM generates synthetic student performance from a \emph{synthetic performance function} $\simperformancefunction$: when prompted with a synthetic profile $\simprofile\in\profilespace$, the LLM answers a question $\testfunction$ correctly with probability $\simperformancefunction ( \simprofile , \testfunction ) \in [0, 1]$. 

The collection of the real dataset $\dataset_j = \{ \response_{j,i} \}_{i=1}^{n_j}$ can be thought of as drawing $n_j$ i.i.d.~student profiles $\{ \profile_{j,i} \}_{i=1}^{n_j} \sim \distribution$ and then sampling $\response_{j,i} \sim \Bernoulli (  \performancefunction ( \profile_{j,i} , \testfunction_j  ) )$ for each $i\in[n_j]$. Similarly, the generation of the synthetic dataset $\simdataset_j = \{ \simresponse_{j,i} \}_{i=1}^{K}$ can be thought of as drawing i.i.d.~synthetic profiles $\{ \simprofile_{j,i} \}_{i=1}^{K} \sim \simdistribution$ and then sampling $\simresponse_{j,i} \sim \Bernoulli ( 
\simperformancefunction
( \simprofile_{j,i} , \testfunction_j  ) )$ for each $i\in [K]$. For  $\simdataset = \{ \simresponse_i \}_{i=1}^K$, we adopt a similar notation $\{ \simprofile_i \}_{i=1}^K$ for the synthetic profiles. We note that when collecting real or synthetic samples, the performance functions never appear explicitly. They are introduced only to facilitate the problem formulation.

Finally, we assume that the test questions are drawn randomly from a question bank, and that the datasets are independent.

\begin{assumption}[Randomly sampled questions]\label{assumption-iid-test-1D}
The questions $\testfunction,\testfunction_1,...,\testfunction_m$ are independently sampled from a distribution over a space $\testfunctionfamily$.
\end{assumption}

\begin{assumption}[Independent data]\label{assumption-indep-data-1D}
For each $j\in[m]$, conditioned on $\testfunction_j$, the datasets $\dataset_j$ and $\simdataset_j$ are independent. Conditioned on $\testfunction_1,...,\testfunction_m$, the dataset tuples $(\dataset_1,\simdataset_1),...,(\dataset_m,\simdataset_m)$ are independent. Finally, $(\testfunction,\simdataset)$ is independent of $\big\{ (\testfunction_j,\dataset_j,\simdataset_j) \big\}_{j=1}^m$.
\end{assumption}

We are now ready to state the theoretical guarantee of our approach. Its proof is deferred to \Cref{sec-thm-coverage-1D-proof}. We note that the assumption $\PP( \mean_{j} \ge \responsebar_j \mid \testfunction_j ) = 1/2$ is a CLT approximation and is for mathematical convenience only.

\begin{theorem}[Coverage guarantee]\label{thm-coverage-1D}
Let Assumptions \ref{assumption-iid-test-1D} and \ref{assumption-indep-data-1D} hold. Assume that $\PP( \responsebar_j \leq \mean_{j} \mid \testfunction_j ) = 1/2$ for each $j\in[m]$. Fix $\alpha\in(0,1)$. Then the simulation sample size $\widehat{k}$ defined by \eqref{eqn-empirical-criterion-1D} satisfies
\[
\PP\Big( \mean \in \simCI(\widehat{k}) \Big) \ge 1-\alpha - \sqrt{\frac{2}{m}}.
\]
The probability is taken with respect to randomness of $\big\{ ( \testfunction_j , \dataset_j, \simdataset_j ) \big\}_{j=1}^m$, $\testfunction$ and $\simdataset$.
\end{theorem}

On average, the chosen simulation sample size $\widehat{k}$ leads to a confidence interval $\simCI(\widehat{k})$ that covers the true mean $\mean$ with probability at least $1-\alpha-O(\sqrt{1/m})$. As $m\to\infty$, the aforementioned lower bound converges to $1-\alpha$.


\subsection{Sharpness of Sample Size Selection}

We have seen that the chosen interval $\simCI(\widehat{k})$ has good coverage properties. In this section, we complement this result by showing that the interval is not overly conservative. To simplify computation, we slightly modify the setting.

\begin{example}[Gaussian performance score]\label{example-gaussian}
Consider the setting in \Cref{sec-theory-1D} with the following modifications. On a test question $\testfunction \in\testfunctionfamily$, the performance (e.g., score) of a real student with profile $\profile$ follows the distribution $N( \performancefunction ( \profile , \testfunction ),1)$ instead of $\Bernoulli ( F ( \profile , \testfunction ) )$; similarly, the performance of the LLM prompted with synthetic profile $\simprofile$ has distribution $N( \simperformancefunction (  \simprofile, \testfunction ),1)$. Moreover, the confidence intervals $\simCI(k)$ defined in \eqref{eqn-CI-intro-sim} and $\simCI_j(k)$ defined in \eqref{eqn-CI-intro-sim-calibrate} are changed to
\begin{align*}
& \simCI(k) = \left[ \simresponsebar_k - \frac{C}{\sqrt{k}},~ \simresponsebar_k + \frac{C}{\sqrt{k}} \right], \\[4pt]
& \simCI_j(k) = \left[ \simresponsebar_{j,k} - \frac{C}{\sqrt{k}},~ \simresponsebar_{j,k} + \frac{C}{\sqrt{k}} \right],
\end{align*}
respectively, where $C = 2 \Phi^{-1} (1 - \alpha / 4)$. For simplicity, we suppose that the real datasets have the same size: $n_j=n$ for all $j\in[m]$. Finally, we define
\[
\Delta = \sup_{\testfunction \in \testfunctionfamily} \big| \EE_{\profile \sim \distribution} \performancefunction ( \profile , \testfunction ) - \EE_{\simprofile \sim \simdistribution} \simperformancefunction ( \simprofile , \testfunction ) \big|.
\]
\end{example}

In \Cref{example-gaussian}, the quantity $\Delta$ measures the discrepancy between the distributions of the real students' performance and of the simulated students' performance. The following theorem presents a lower bound on the chosen simulation sample size $\widehat{k}$. Its proof can be found in \Cref{sec-thm-sharpness-1D-proof}.

\begin{theorem}[Sharpness of chosen sample size]\label{thm-sharpness-1D}
Consider the setting of \Cref{example-gaussian}. Let $\widehat{k}$ be chosen by the procedure \eqref{eqn-empirical-criterion-1D}. Choose $\delta \in (0, 1)$. There exists a constant $C'>0$ determined by $\alpha$ such that when $m  > C' \log(n / \delta)$, the following holds with probability at least $1 - \delta$:
\[
\widehat{k} \ge \min \left\{
\left(
\frac{C}{5 \Delta}
\right)^2,
n,  K
\right\}.
\]
When this happens, the selected confidence interval $\simCI(\widehat{k})$ has width $O\left( \max\{\Delta, n^{-1/2}, K^{-1/2} \} \right)$.
\end{theorem}

\Cref{thm-sharpness-1D} implies that the chosen $\widehat{k}$ is the largest possible, or equivalently, the interval $\simCI(\widehat{k})$ is the shortest possible. To see this, suppose that the simulation budget $K$ is large. When $\Delta$ is small, there is little distribution shift between the real and simulated student performance, so it is optimal to use as many synthetic samples as possible. \Cref{thm-sharpness-1D} shows that in this case, the procedure \eqref{eqn-empirical-criterion-1D} chooses $\widehat{k}=n$ with high probability. This is the maximum effective sample size, since the $n$ real data points can identify the true mean only up to an error of $O(n^{-1/2})$ and thus there is no statistical benefit in simulating more than $n$ samples. On the other hand, when the discrepancy $\Delta$ between the real and simulated student performance is large, any $\simCI(k)$ that covers the true mean with high probability must have width $\Omega(\Delta)$. Thus, the chosen confidence interval $\simCI(\widehat{k})$, which has width $O(\Delta)$ with high probability, is the shortest possible.


\section{General Setup and Methodology}\label{sec-general}

In this section, we study the more general setting where survey responses and confidence sets can be multi-dimensional.


\subsection{Problem Formulation}

Let $\profilespace$ be a profile space, $\distribution$ a probability distribution over $\profilespace$ which represents the true population, and $\simdistribution$ a synthetic distribution over $\profilespace$ used to generate synthetic profiles. 

Let $\testfunctionfamily$ be a collection of survey questions, and $\responsespace$ be the space of possible responses to the survey questions. When a person with profile $\profile\in\profilespace$ is asked a survey question $\testfunction\in\testfunctionfamily$, the person gives a response $\response$ following a distribution $\responsedist( ~ \cdot \mid \profile, \testfunction)$ over $\responsespace$. We are interested in the distribution of the population's response to the survey question $\testfunction$, which is given by $\responsedistalt(~ \cdot \mid \testfunction )
=
\int_{\profilespace} \responsedist( ~ \cdot \mid \profile, \testfunction) \, \distribution ( d \profile )$. In particular, we seek to construct a confidence set for some statistic $\statistic( \testfunction )$ of $\responsedistalt(~ \cdot \mid \testfunction )$, which can be multi-dimensional, say in $\RR^d$. Below we revisit the educational test example in \Cref{sec-warmup} in this framework. More examples are provided in \Cref{sec-examples}.

\begin{example}[Educational test evaluation]\label{example-education}
In the educational test evaluation example in \Cref{sec-warmup}, each $\profile\in\profilespace$ is a student profile, each $\testfunction\in\testfunctionfamily$ is a test question, the response space is $\responsespace=\{0,1\}$, and $\responsedist( ~ \cdot \mid \profile, \testfunction) = \Bernoulli ( \performancefunction( \profile , \testfunction) )$. The statistic $\statistic( \testfunction )$ is the probability of a student answering the question correctly: $\EE_{\response \sim \responsedistalt( \cdot \mid \testfunction )} [ y ] = \EE_{\profile \sim \distribution} [ \performancefunction( \profile , \testfunction) ]$.
\end{example}

We consider constructing the confidence set by using simulated responses from an LLM. Given a profile $\profile$, a survey question $\testfunction$ and a prompt $\prompt$, the LLM simulates a response $\simresponse$ from a distribution $\simresponsedist(~\cdot\mid \profile , \testfunction, \prompt)$ which aims to mimic $\responsedist(~\cdot\mid \profile , \testfunction)$. We can generate synthetic profiles $\{ \simprofile_i \}_{i=1}^K$ from some distribution $\simdistribution$, then feed them into the LLM along with $\testfunction$ and $\prompt$. The LLM then generates synthetic responses $\{ \simresponse_i \}_{i=1}^K$, where $\simresponse_i \sim \simresponsedist(~\cdot\mid \simprofile_i , \testfunction, \prompt)$. Here $K$ is the simulation budget.


Using the simulated samples $\simdataset = \{ \simresponse_i \}_{i=1}^K$, we can construct a family of candidate confidence sets such as the one-dimensional CLT-based confidence interval \eqref{eqn-CI-intro-sim}. More generally, the statistics literature has developed a variety of approaches such as inverting hypothesis tests \citep{CBe02}, the bootstrap \citep{Efr79}, and the empirical likelihood ratio function \citep{OWe90}. We will assume access to a black-box procedure $\setmap$ that takes as input a dataset $\dataset$ and outputs a confidence set $\setmap(\dataset)\subseteq\RR^d$. Then, we can construct a family of confidence sets $\{\simCIalt(k)\}_{k=1}^{K}$ by
\begin{equation}\label{eqn-CI-sim}
\simCIalt(k) = \setmap \left( \{ \simresponse_i \}_{i=1}^k \right).
\end{equation}
We also set $\simCIalt(0) = \RR^d$, so $\statistic ( \testfunction ) \in \simCIalt(0)$ always. We will not impose any assumptions on the quality of the confidence sets produced by $\setmap$.

As the LLM may not be a faithful reflection of the true human population, we will make use of real data to choose a good confidence set from $\{\simCIalt(k)\}_{k=1}^{K}$. We assume that we have collected real human responses from $m$ surveys $\testfunction_1,...,\testfunction_m\in\testfunctionfamily$. For each $j\in[m]$, we have responses $\dataset_j = \{ \response_{j,i} \}_{i=1}^{n_j}$ from $n_j$ i.i.d.~surveyees $\{ \profile_{j,i} \}_{i=1}^{n_j} \sim \distribution$, with $\response_{j,i} \sim \responsedist(~\cdot\mid  \profile_{j,i}, \testfunction_j )$. 

We also simulate LLM responses to these $m$ survey questions. For each $j\in[m]$, we feed synthetic profiles $\{ \simprofile_{j,i} \}_{i=1}^K$, the question $\testfunction_j$ and the prompt $\prompt$ into the LLM, which simulates responses $\simdataset_j = \{ \simresponse_{j,i} \}_{i=1}^{K}$ with $\simresponse_{j,i} \sim \simresponsedist (~\cdot \mid \simprofile_{j,i}, \testfunction_j, \prompt )$. The datasets $\{ \dataset_j \}_{j=1}^m $ and $\{\simdataset_j \}_{j=1}^m $ will be used to select a confidence set from $\{ \simCIalt(k) \}_{k=1}^K$. We make the same Assumptions \ref{assumption-iid-test-1D} and \ref{assumption-indep-data-1D} as in \Cref{sec-warmup}.

We are now ready to formally state our problem.

\begin{problem}[Uncertainty quantification]\label{problem-general}
Given $\alpha \in (0,1)$, how to use $\{ \dataset_j \}_{j=1}^m$ and $\{ \simdataset_j \}_{j=1}^m$ to choose $\widehat{k}\in[K]$ such that
\[
\PP \Big( \statistic ( \testfunction ) \in \simCIalt(\widehat{k}) \Big) \approx 1-\alpha?
\]
\end{problem}


\subsection{General Methodology for Sample Size Selection}

We now present our general methodology for Problem \ref{problem-general}. For each $j\in[m]$, we form confidence sets similar to \eqref{eqn-CI-sim} using the synthetic data $\simdataset_j$:
\begin{equation}\label{eqn-CI-sim-calibrate}
\simCIalt_j(k) = \setmap \left( \{ \simresponse_{j,i} \}_{i=1}^k \right),\quad \forall k\in[K].
\end{equation}
We also set $\simCIalt_j(0) = \RR^d$. We will pick $\widehat{k}\in\{0,1,...,K\}$ such that $\simCIalt_j(\widehat{k})$ is a good confidence interval for $\statistic (\testfunctionalt_j)$ for each $j\in[m]$. This choice of $\widehat{k}$ will also be good for $\simCIalt(k)$, thanks to the i.i.d.~assumption on the survey questions.

Ideally, we would like to pick $k$ such that $(1-\alpha)$ coverage is achieved empirically over the $m$ survey functions:
\begin{equation}\label{eqn-oracle-criterion}
\frac{1}{m} \sum_{j=1}^m \ind \{  \statistic( \testfunction_j )  \not\in \simCIalt_j(k) \} \le \alpha.
\end{equation}
However, the population-level quantities $\{ \statistic( \testfunction_j ) \}_{j=1}^m$ are not available, so we must approximate them by the real data $\{ \dataset_j \}_{j=1}^m$. In \Cref{sec-warmup}, we have taken the approach of constructing unbiased point estimates, but it does not directly extend to the more general case. 

Instead, we will use the real data $\{ \dataset_j \}_{j=1}^m$ to construct confidence sets for $\{  \statistic( \testfunction_j )  \}_{j=1}^m$. Choose a confidence level $\gamma\in(0,1)$. For each $j \in [m]$, we use $\dataset_j$ to construct a confidence set $\CIalt_j$ that satisfies
\begin{equation}\label{eqn-CI-calibrate}
\PP\Big( \statistic ( \testfunction_j)  \in \CIalt_j \Bigm| \testfunction_j \Big) \ge \gamma.
\end{equation}
These confidence sets are easy to construct as the samples in $\dataset_j$ follow the true response distribution. When $\statistic ( \testfunction_j)  \in \CIalt_j$, the condition $\CIalt_j\subseteq \simCIalt_j(k)$ is sufficient for $\statistic ( \testfunction_j)  \in \simCIalt_j(k)$. Equivalently, when $\statistic ( \testfunction_j)  \in \CIalt_j$, the condition $\statistic ( \testfunction_j)  \not\in \simCIalt_j(k)$ must imply $\CIalt_j\not\subseteq \simCIalt_j(k)$. Thus, we take
\begin{equation}\label{eqn-proxy}
\coveragealt(k) = \frac{1}{m} \sum_{j=1}^m \ind \{ \CIalt_j\not\subseteq \simCIalt_j(k) \}
\end{equation}
as a proxy for the empirical miscoverage. Since $\statistic ( \testfunction_j)  \in \CIalt_j$ happens with probability $\gamma$, then the frequency of having $\CIalt_j\not\subseteq \simCIalt_j(k)$ is at least $\gamma$ times the frequency of $\statistic ( \testfunction_j)  \not\in \simCIalt_j(k)$. Roughly speaking,
\begin{equation}\label{eqn-proxy-to-oracle}
\coveragealt(k) \ge \gamma \cdot\left( \frac{1}{m} \sum_{j=1}^m \ind \{  \statistic( \testfunction_j )  \not\in \simCIalt_j(k) \} \right).
\end{equation}

Combining \eqref{eqn-oracle-criterion} and \eqref{eqn-proxy-to-oracle} leads to the following criterion for selecting $k$:
\begin{equation}\label{eqn-empirical-criterion}
\widehat{k} = \max \left\{ 0\le k \le K : ~ L(i) \le \gamma\alpha, ~\forall i\le k \right\}.
\end{equation}
Note that $\widehat{k}$ is well-defined because $\coveragealt(0) = 0$. The full procedure for sample size selection is summarized in \Cref{alg-general}. 

\begin{algorithm}[h]
	\begin{algorithmic}
	\STATE {\bf Input:} Survey questions with real and simulated responses $\big\{(\testfunction_j, \dataset_j, \simdataset_j)\big\}_{j=1}^m$, prescribed miscoverage probability $\alpha$, confidence set construction procedure $\setmap$, confidence level $\gamma$, simulation budget $K$. \\[4pt]
	\FOR{$j=1,...,m$}
		\STATE Use $\setmap$ and $\simdataset_j$ to construct synthetic confidence sets $\{\simCIalt_j(k)\}_{k=0}^K$ by \eqref{eqn-CI-sim-calibrate}. \\[4pt]
		\STATE Use $\dataset_j$ to construct a confidence set $\CIalt_j$ satisfying \eqref{eqn-CI-calibrate}.
	\ENDFOR
	\STATE Define
	\[
	\coveragealt(k) = \frac{1}{m} \sum_{j=1}^m \ind \{ \CIalt_j\not\subseteq \simCIalt_j(k) \}.
	\]
	\STATE Set $\widehat{k} = \max \left\{ 0\le k \le K :~ L(i) \le \gamma\alpha ,~\forall i\le k \right\}$. \\[4pt]
%	\STATE Use $\setmap$ and $\simdataset$ to construct synthetic confidence sets $\{\simCIalt ( \widehat{k} )\}_{k=0}^K$ by \eqref{eqn-CI-sim}. \\[4pt]
	\STATE {\bf Output:} Sample size $\widehat{k}$
.	\caption{Simulation Sample Size Selection}
	\label{alg-general}
	\end{algorithmic}
\end{algorithm}

We now present the coverage guarantee for our method, which shows that the chosen confidence set $\simCIalt(\widehat{k})$ has coverage probability at least $1-\alpha-O(1/\sqrt{m})$. Its proof is deferred to \Cref{sec-thm-coverage-proof}.


\begin{theorem}\label{thm-coverage}
Let Assumptions \ref{assumption-iid-test-1D} and \ref{assumption-indep-data-1D} hold. Fix $\alpha\in(0,1)$. The output $\widehat{k}$ of \Cref{alg-general} satisfies
\[
\PP\Big( \statistic ( \testfunction ) \in \simCIalt(\widehat{k}) \Big) \ge 1-\alpha - \gamma^{-1} \sqrt{\frac{1}{2m}}.
\]
\end{theorem}

It is worth noting that our method achieves this coverage without any assumptions on the qualities of the LLM and the procedure $\setmap$ for confidence set construction. Nevertheless, the size of the chosen confidence set $\simCIalt(\widehat{k})$, in terms of the true coverage rate and size, depend on these factors. If there is a large alignment gap between the LLM and the human population, then $\simCIalt(\widehat{k})$ will inevitably be large.





\section{Numerical Experiments}\label{sec-experiments}

In this section, we apply our general method in \Cref{sec-general} to LLMs over real datasets. The code and data are available at \url{https://github.com/yw3453/uq-llm-survey-simulation}.

\subsection{Experiment Setup}


\paragraph{LLMs.} We consider $8$ LLMs: GPT-3.5-Turbo (\texttt{gpt-3.5-turbo}), GTP-4o (\texttt{gpt-4o}), and GPT-4o-mini (\texttt{gpt-4o-mini}) \citep{GPT3.5, GPT4o, GPT4omini}; Claude 3.5 Haiku (\texttt{claude-3-5-haiku-20241022}) \cite{Ant24}; Llama 3.1 8B (\texttt{Llama-3-8B-Instruct-Turbo}) and Llama 3.3 70B (\texttt{Llama-3.3-70B-Instruct-Turbo}) \citep{DJP24}; Mistral 7B (\texttt{Mistral-7B-Instruct-v0.3}) \citep{JSM23}; DeepSeek-V3 (\texttt{DeepSeek-V3}) \citep{LFX24}.


\paragraph{Datasets.} We use two datasets for survey questions, each corresponding to one uncertainty quantification task. The first dataset is the OpinionQA dataset created by \cite{SDL23}. It was built from Pew Research's American Trends Panel\footnote{\url{https://www.pewresearch.org/the-american-trends-panel/}}, and contains the general US population's responses to survey questions spanning topics such as science, politics, and health. After pre-processing we have 385 unique questions and 1,476,868 responses to these questions from at least 32,864 people. These questions have $5$ choices corresponding to ordered sentiments which we map to sentiment scores $-1,-\frac{1}{3},0,\frac{1}{3},1$. Each question has at least $400$ responses. For each response, we have information on their political profile, religious affiliation, educational background, socio-economic status, etc. This information is used as their profiles to generate synthetic profiles. See \Cref{sec-opinion} for more information on this dataset, including example questions, profile features, and the generation of synthetic profiles and answers. We consider the task of constructing a confidence interval for the US population's average sentiment score for a survey question. This is the setup in \Cref{example-public-survey-1D}.


The second dataset is the EEDI dataset created by \cite{HMG24}, which was built upon the NeurIPS 2020 Education Challenge dataset \citep{WLS21}. It consists of students' responses to mathematics multiple-choice questions on the Eedi online educational platform\footnote{\url{https://eedi.com/}}. The dataset contains 573 unique questions and 443,433 responses to these questions from 2,287 students. All questions have four choices (A, B, C, D). Out of these questions, we use questions that have at least $100$ student responses. Excluding questions with graphs or diagrams, we are left with a total of $412$ questions. For each student, we have information on their gender, age, and socioeconomic status. This information is used as their profiles to generate synthetic profiles. See \Cref{sec-eedi} for more information on this dataset, including example questions, profile distribution, and the generation of synthetic profiles and answers. We consider the task of constructing a confidence interval for the probability of a student answering a question correctly. This is similar to the setup in \Cref{sec-warmup} and \Cref{example-education}.


\paragraph{Confidence set construction.} Our method can be built upon any arbitrary confidence set construction procedure $\setmap$. We use a construction procedure based on Hoeffding's concentration inequality (e.g., Theorem 2.8 in \cite{BLM13}), as it has valid coverage guarantee for any finite sample size. Given $\alpha\in(0,1)$ and responses $\{\simresponse_i\}_{i=1}^k$, we construct the confidence interval
\begin{equation}\label{eqn:synthCI}
\setmap \big(\{\simresponse_i\}_{i=1}^k\big)
= 
\bigg[ \simresponsebar_k - cM\sqrt{\frac{\log(2/\alpha)}{2k}}  ,  ~  \simresponsebar_k + cM\sqrt{\frac{\log(2/\alpha)}{2k}} \bigg],
\end{equation}
where $\simresponsebar_k = \frac{1}{k}\sum_{i=1}^k \simresponse_i$ is the sample mean, $M>0$ is an upper bound on the range of the responses, and $c>1$ is a scaling constant. 

\paragraph{Hyperparameters.} We consider $\alpha\in\{0.05\cdot\ell:\ell\in[10]\}$, $c=\sqrt{2}$ and $\gamma=0.5$. For the EEDI dataset, we set the simulation budget $K=50$ and take $M=1$ since the responses are binary. For the OpinionQA dataset, we set the simulation budget $K=100$ and take $M=2$ since the responses range within $[-1,1]$.


\subsection{Experiment Procedure}

We now describe our experiment procedure for applying the method in \Cref{sec-general} to each dataset. Denote the dataset by $\{ ( \testfunction_j , \dataset_j ) \}_{j=1}^{J}$, where $\testfunction_j$ is a survey question and $\dataset_j = \{ \response_{j,i} \}_{i=1}^{n_j}$ is a collection of human responses. For each $j\in[J]$, we simulate $K$ responses $\simdataset_j$ from an LLM. We then randomly split $ \datasetmeta = \{ ( \dataset_j, \simdataset_j ) \}_{j=1}^{J}$ into a training set $\datasetmeta^{\train} = \{ ( \dataset_j, \simdataset_j ) \}_{j\in\cJ_{\train}}$ and a testing set $\datasetmeta^{\test} = \{ ( \dataset_j, \simdataset_j ) \}_{j\in\cJ_{\test}}$, with $|\datasetmeta^{\train}| : |\datasetmeta^{\test}| = 3 : 2$.


\paragraph{Selection of simulation sample size.} We apply the approach \eqref{eqn-empirical-criterion} with the training set $\datasetmeta^{\train}$ to select a simulation sample size $\widehat{k}$. For the confidence set $\CIalt_j$ in \eqref{eqn-CI-calibrate} constructed from the real data $\dataset_j$, we use the standard CLT-based confidence interval:
\begin{equation}\label{eqn-CI-calibrate-CLT}
\CIalt_j = \bigg[ \responsebar_j - \frac{\samplesd_j}{\sqrt{n_j}} \Phi^{-1}\left( \frac{1+\gamma}{2} \right) ,
~  \responsebar_j + \frac{\samplesd_j}{\sqrt{n_j}} \Phi^{-1}\left( \frac{1+\gamma}{2} \right) \bigg],
\end{equation}
where $\responsebar_j = \frac{1}{n_j} \sum_{i=1}^{n_j} \response_{j,i}$ and $\samplesd_j = \sqrt{\responsebar_j (1-\responsebar_j) }$. Since $n_j$ is at least $100$, $\CIalt_j$ has approximately $\gamma$ coverage probability. In \Cref{fig:k-upcross-eg} of \Cref{sec-appendix-experiments-results}, we provide a visualization for the selection of $\widehat{k}$.

\paragraph{Evaluation of selected sample size.} We use $\datasetmeta^{\test}$ to evaluate the quality of the chosen simulation sample size $\widehat{k}$. As the true population mean $\statistic ( \testfunction )$ is unavailable, the true coverage probability $\PP\big( \statistic ( \testfunction ) \in \simCIalt(\widehat{k}) \big)$ cannot be computed. However, we can apply the same idea as \eqref{eqn-proxy} in \Cref{sec-general} to compute a proxy for the miscoverage level. For each survey question $j\in\cJ_2$, the selected sample size $\widehat{k}$ leads to the synthetic confidence set $\simCIalt_j(\widehat{k}) = \setmap\big( \{ \simresponse_{j,i} \}_{i=1}^{\widehat{k}} \big)$. We form the confidence set $\CIalt_j$ from real data $\dataset_j$ as in \eqref{eqn-CI-calibrate-CLT} and define
\begin{equation}\label{eqn:proxy}
\widetilde{\coveragealt}(\widehat{k}) = \frac{1}{\gamma} \cdot \frac{1}{|\cJ_{\test}|} \sum_{j\in\cJ_{\test}} \ind \{ \CIalt_j\not\subseteq \simCIalt_j(\widehat{k}) \}.
\end{equation}
The proof of \Cref{thm-coverage} shows that, for every $k\in[K]$ and survey question $j$,
\[
\gamma \cdot \PP\big( \statistic ( \testfunction ) \not\in \simCIalt(k) \big) \le \PP\big( \CIalt_j\not\subseteq \simCIalt_j(k) \big) = \gamma \cdot \EE \big[ \widetilde{\coveragealt}(\widehat{k}) \big] .
\]
Thus, if $\EE \big[ \widetilde{\coveragealt}(\widehat{k}) \big] \le \alpha$, then $\PP\big( \statistic ( \testfunction ) \not\in \simCIalt(\widehat{k}) \big) \le \alpha$ must hold. To that end, we will test a hypothesis $H_0: \; \EE \big[ \widetilde{\coveragealt}(\widehat{k}) \big] \le \alpha$ against its alternative $H_1: \; \EE \big[ \widetilde{\coveragealt}(\widehat{k}) \big] > \alpha$.


\subsection{Experiment Results}\label{sec-experiments-results}

For both datasets, we evaluate three metrics as $\alpha$ varies: the miscoverage probability proxy \eqref{eqn:proxy}, the selected simulation sample size $\widehat{k}$, and the half-width of the synthetic confidence interval $\simCIalt(\widehat{k})$. We consider $100$ random train-test splits of the questions. For compactness, we present results on the miscoverage probability proxy and $\widehat{k}$, and defer the results on the half-width of the synthetic confidence interval as well as more experiment details to \Cref{sec-appendix-experiments-results}. We omit Llama 3.1 8B for the EEDI dataset experiment because it frequently failed to answer EEDI questions in required formats. As a baseline, we also include a na\"{i}ve response generator (\texttt{random}) that chooses an available answer uniformly at random. 

In \Cref{fig:combined}, we present histograms of $p$-values for the hypothesis test $\EE \big[ \widetilde{\coveragealt}(\widehat{k}) \big] \le \alpha$ against $\EE \big[ \widetilde{\coveragealt}(\widehat{k}) \big] > \alpha$ across various LLMs and $\alpha$'s over the OpinionQA and EEDI datasets. The $p$-values are computed using a one-sided $z$-test over the $100$ random splits. As can be seen from the histograms, all $p$-values are reasonably large, indicating that the hypothesis $\EE \big[ \widetilde{\coveragealt}(\widehat{k}) \big] \le \alpha$ cannot be rejected (e.g.,~at the 0.05 significance level) for any LLM and $\alpha$ across both datasets. These experiment results verify the theoretical guarantees in \Cref{sec-general}, showing that the miscoverage rate is effectively controlled by our method. 

\begin{figure}[h]
	\centering
	\begin{minipage}{0.45\textwidth}
		\centering
		\includegraphics[width=0.8\linewidth]{figures/pval_OpinionQA.pdf}
	\end{minipage}%\hspace{em}%\hfill
	\begin{minipage}{0.45\textwidth}
		\centering
		\includegraphics[width=0.8\linewidth]{figures/pval_EEDI.pdf}
	\end{minipage}
	\caption{Histograms of $p$-values for the hypothesis test $\EE \big[ \widetilde{\coveragealt}(\widehat{k}) \big] \le \alpha$ against $\EE \big[ \widetilde{\coveragealt}(\widehat{k}) \big] > \alpha$ across various LLMs and $\alpha$'s over the OpinionQA (left) and EEDI (right) datasets.}
	\label{fig:combined}
\end{figure}



In \Cref{fig:k-hat}, we plot the average $\widehat{k}$ over the $100$ random splits for various LLMs on the OpinionQA and EEDI datasets. The error bars represent 95\% confidence intervals.
In general, a larger $\widehat{k}$ means that the LLM has stronger simulation power. On the OpinionQA dataset, GPT-4o has the best performance. On the EEDI dataset, DeepSeek-V3 has the best performance. Interestingly, on the OpinionQA dataset all LLMs clearly outperform the random benchmark, while on the EEDI dataset only DeepSeek-V3 and GPT-4o seem to outperform the random benchmark. Moreover, LLMs exhibit uniformly higher $\widehat{k}$ on the OpinionQA dataset than on the EEDI dataset, suggesting higher fidelity in simulating subjective opinions than in simulating answers to mathematics questions.


\begin{figure}[h]
	\centering
	\begin{minipage}{0.45\textwidth}
		\centering
		\includegraphics[width=0.8\linewidth]{figures/k_hat_OpinionQA.pdf}
	\end{minipage}
	\begin{minipage}{0.45\textwidth}
		\centering
		\includegraphics[width=0.8\linewidth]{figures/k_hat_EEDI.pdf}
	\end{minipage}
	\caption{Average $\widehat{k}$ for various LLMs and $\alpha$ over the OpinionQA (left) and EEDI (right) datasets.}
	\label{fig:k-hat}
\end{figure}



The experiment results demonstrate the importance of a disciplined approach to using synthetic samples. The ease of LLM-based simulation makes it tempting to generate a large number of responses per question. However, as can be seen from the figures, there is great heterogeneity in the simulation power of different LLMs over different datasets: the largest $\widehat{k}$ is below 100, while the smallest $\widehat{k}$ could be in the single digits. This means that there is real peril in using too many synthetic samples and being overly confident in the results.

\section{Discussions}\label{sec-discussions}

We developed a general approach for converting imperfect LLM-based survey simulations into statistically valid confidence sets for population statistics of human responses. It identifies a simulation sample size which is useful for future simulation tasks and which reveals the degree of misalignment between the LLM and the target human population. 

Several future directions are worth exploring. First, our approach does not explicitly minimize the size of the prediction set. A natural question is whether we can incorporate a size minimization procedure to produce smaller confidence sets with good coverage. Second, it would be interesting to see if our approach can be combined with debiasing methods to give more informative confidence sets. Finally, as prompt engineering is known to have crucial effects on the quality of LLM generations, it is worth investigating the impacts of prompts on the selected simulation sample size $\widehat{k}$, and how prompt engineering can be leveraged to reduce the misalignment gap between LLM simulations and true human responses.



\section*{Acknowledgement}
Chengpiao Huang and Kaizheng Wang's research is supported by an NSF grant DMS-2210907 and a Data Science Institute seed grant SF-181 at Columbia University.


\newpage 
\appendix


\section{PEW Surveys Experiments: Details and Additional Examples} \label{sec:pew_additional_examples} 
In this section, we expand on our PEW surveys experiment where we used polling data on key political and social issues to show: (i) how $\nbc$ rankings can differ from those maximizing the average reward; (ii) How sensitive $\nbc$ is to the sampling distribution of the pairwise preference data. 

\niparagraph{Data.} We use several Pew Research Center surveys, specifically the American Trends Panel surveys number 35, 52, 79, 83, 99,  109, 
111, 112,  114, 119, 120, 121, 126, 127, 128, 129, 130, 131, and 132. The choices are a mix of recent surveys and  those relevant to science, technology, data and AI. 
%
Each survey include questions asked to thousands of participants. We categorize participants by political party leanings to define types.
%
When processing the questions, we discard responses that are empty, as well as discarding the option "Refused". We note that discarding the option "Refused" had no effect on the results as it is not frequently chosen. 

\niparagraph{Reward Estimation.}
Although we observe how often each group selects a particular option, we don't directly observe respondents' internal rewards. To estimate this, we apply the Luce-Shepherd model~\citep{shepard_stimulus_1957, luce1959individual}:
 \begin{equation}
 \label{eq:luce-shep}
        \Pr\big(\text{option } i \text{ is chosen from } \mathcal{S}\big) =  \frac{\exp{(r(i;u))}}{\sum_{{j\in \mathcal{S}}} \exp{(r(j;u))}},
\end{equation}
where $\mathcal{S}$ is the set of options, and $r(\cdot; u)$ is the reward for type $u$. 
This allows us to estimate each option's reward (up to a constant additive term) for each type. From these estimates and observed probabilities, we compute both the expected reward and the $\nbc$ metric, where in the latter we assume the uniform probability for alternatives unless specified otherwise.

\niparagraph{Sensitivity Experiments.} In \cref{sec:sensitivity}, we estimate the sensitivity of $\nbc$ rankings to the sampling distribution of pairwise preference data by determining the minimum Total Variation (TV) distance required, if possible, to alter $\nbc$ rankings from that attained under the uniform distribution.

Recall that $\nbc$ is defined as:
\[
\nbc(\vy; \gD) \coloneqq \E_{\vy' \sim \gD(\cdot)} \big[\Pr(\vy \succ \vy' \mid \vx; r)\big].
\]
To compute this, we first estimate the reward function \(r\), then evaluate \(\Pr(\vy \succ \vy'; r)\) for all \((\vy, \vy') \in \gY \times \gY\), where \(\gY\) is the set of alternatives. 
Next, consider the feasibility of swapping the ranking induced by the uniform distribution \(\gD_U\) for alternatives \(\vy_i\) and \(\vy_j\) with a new distribution \(\gD_a\), assuming \(\nbc(\vy_i; \gD_U) > \nbc(\vy_j; \gD_U)\). This is equivalent to solving the following linear program:
\[
\begin{aligned}
     \text{minimize}& \quad \frac{1}{2} \mathbf{1}^\top \mathbf{s}, \\
    \text{subject to:} 
    & \quad \mathbf{q} > \epsilon \mathbf{1}, \\
    & \quad \mathbf{s} \geq \frac{1}{N} \mathbf{1} - \mathbf{q}, \\
    & \quad \mathbf{s} \geq \mathbf{q} - \frac{1}{N} \mathbf{1}, \\
    & \quad \mathbf{P}_{i} \mathbf{q} < \mathbf{P}_{j} \mathbf{q} + \delta, \\
    & \quad \mathbf{1}^\top \mathbf{q} = 1, \\
    & \quad \mathbf{q} > \mathbf{0}.
\end{aligned}
\]
Here, \(N = |\gY|\), and labeling the alternatives as \(\vy_1, \dots, \vy_N\), we define \(\mathbf{P}_{ij} = \Pr(\vy_i \succ \vy_j; r)\), \(q_i = \gD_a(\vy_i)\), and \(\mathbf{P}_k\) as the \(k\)-th row of \(\mathbf{P}\). The parameter \(\epsilon > 0\) ensures support for all alternatives, and \(\delta > 0\) controls the required magnitude of change in \(\nbc\) beyond what is required for the swap. We set \(\epsilon = \delta = 10^{-5}\). 

To compute the minimum TV distance, we solve the program for all pairs \((i, j)\) where \(\nbc(\vy_i; \gD_U) > \nbc(\vy_j; \gD_U)\) and record the smallest objective value. In this analysis, we group respondents by political leaning (specifically, the column \texttt{F\_PARTYSUM\_FINAL}). We also note that in this analysis we exclude survey questions with fewer than three options. 


% \niparagraph{Sensitivity Experiments.} In \cref{sec:sensitiviy}, to estimate the sensitivity of $\nbc$ to the sampling distribution of the pairwise preference data, we find the minimum TV distance required to alter $\nbc$ ranking under uniform. 
% First recall that $\nbc$ is defined as,
% \begin{equation}
%     \nbc(\vy; \gD) \coloneqq \E_{\vy' \sim \gD(\cdot)} \Big[\Pr(\vy \succ \vy' \mid \vx; r)\Big].
% \end{equation}
% Hence, to estimate it, we first estimate the reward $r$ and then compute the quantity $\Pr(\vy \succ \vy'; r) \forall (\vy, \vy') \in \gY \times \gY$, where $\gY$ is the set of alternatives. Note that the feasibility of swapping the ranking induced by the uniform distribution $\gD_U$ of options $\vy_i$ and $\vy_j$ with a new distribution $\gD_a$, and assuming $\nbc(\vy_i; \gD_u) > \nbc(\vy_j; \gD_U)$, is equivalent to the feasibility of the following linear program,
% \[
% \begin{aligned}
%     & \text{minimize} \quad \frac{1}{2} \mathbf{1}^\top \mathbf{s} \\
%     \text{subject to:} 
%     & \quad \mathbf{q} > \epsilon \mathbf{1}, \\
%     & \quad \mathbf{s} \geq \frac{1}{N} \mathbf{1} - \mathbf{q}, \\
%     & \quad \mathbf{s} \geq \mathbf{q} - \frac{1}{N} \mathbf{1}, \\
%     & \quad \mathbf{P}_{i} \mathbf{q} < \mathbf{P}_{j} \mathbf{q} + \delta, \\
%     & \quad \mathbf{1}^\top \mathbf{q} = 1, \\
%     & \quad \mathbf{q} > \mathbf{0}.
% \end{aligned}
% \]
% Where in the above,  $N = |\gY|$, and with labeling of the alternatives as $\vy_1, \dots, \vy_N$, let   $\mathbf{P}_{ij} = \Pr(\vy_i \succ \vy_j; r)$, $q_i = \gD_a(y_i)$, and $\mathbf{P}_k$ be the $k$-th row of $\mathbf{P}$. Note that $\epsilon > 0$ is used to retain the support for all alternatives, and $\delta > 0$ controls the magnitude of the change  required in $\nbc$. We set both $\delta$ and $\epsilon$ to $10^{-5}$. 
% %
% Now to find the minimum TV distance required, we find the minimum value attained by this program across all pairs $(i,j)$ such that $\nbc(\vy_i; \gD_u) > \nbc(\vy_j); \gD_u$.
% Note that in the analysis above, we use the political leaning (specifically the column \texttt{F\_PARTYSUM\_FINAL}) to group respondents.

\niparagraph{Additional Examples.} We highlight some of the examples of discrepancies that we find in some of the surveys listed above in Figures \ref{fig:example5}, \ref{fig:example6}, \ref{fig:example7}, \ref{fig:example8},  and \ref{fig:example9}. 
%
 We note though that while we indeed find a few examples of the discrepancy, $\nbc$ rankings is actually aligned with average reward rankings in most cases.  One possible explanation for this is that in many questions, the distributions of responses across types were very similar, suggesting that a homogeneous reward model would have been appropriate, causing $\nbc$ and average reward to align.


\begin{figure}[!htbp]
    \centering
    \begin{minipage}{0.6\textwidth}
        \centering
    % First Row
    \subfigure[Rewards per Type]{%
        \includegraphics[width=0.98\linewidth]{figs/PEW/rew_BIDENADM_W83.pdf}
        \label{fig:prop_rew5}
    }%
    \hfill
    \subfigure[Avg Reward vs. NBC]{%
        \includegraphics[width=0.98\linewidth]{figs/PEW/reward_vs_NBC_BIDENADM_W83.pdf}
        \label{fig:reward_vs_nbc5}
    }
    \end{minipage} 
    \begin{minipage}{0.2\textwidth}
        \subfigure[Ranking]{%
        \includegraphics[width=0.95\linewidth]{figs/PEW/ranking_comparison_BIDENADM_W83.pdf}
        \label{fig:ranking5}
    }
    \end{minipage}
    \caption{Do you think the plans and policies of the Biden administration will make the country’s response to the coronavirus outbreak:
A: A lot better; B: A little better; C: Not much different; D: A little worse; E: A lot worse}
\label{fig:example5}
\end{figure}

\begin{figure}[!htbp]
    \centering
    \begin{minipage}{0.6\textwidth}
        \centering
    % First Row
    \subfigure[Rewards per Type]{%
        \includegraphics[width=0.98\linewidth]{figs/PEW/rew_COVIDEGFP_a_W114.pdf}
        \label{fig:prop_rew6}
    }%
    \hfill
    \subfigure[Avg Reward vs. NBC]{%
        \includegraphics[width=0.98\linewidth]{figs/PEW/reward_vs_NBC_COVIDEGFP_a_W114.pdf}
        \label{fig:reward_vs_nbc6}
    }
    \end{minipage} 
    \begin{minipage}{0.2\textwidth}
        \subfigure[Ranking]{%
        \includegraphics[width=0.95\linewidth]{figs/PEW/ranking_comparison_COVIDEGFP_a_W114.pdf}
        \label{fig:ranking6}
    }
    \end{minipage}
    
    \caption{ How would you rate the job Joe Biden is doing responding to the coronavirus outbreak?    
A: Excellent; B: Good; C: Only fair; D: Poor}
\label{fig:example6}
\end{figure}

\begin{figure}[!htbp]
    \centering
    \begin{minipage}{0.6\textwidth}
        \centering
    % First Row
    \subfigure[Rewards per Type]{%
        \includegraphics[width=0.98\linewidth]{figs/PEW/rew_EVCAR2_W128.pdf}
        \label{fig:prop_rew7}
    }%
    \hfill
    \subfigure[Avg Reward vs. NBC]{%
        \includegraphics[width=0.98\linewidth]{figs/PEW/reward_vs_NBC_EVCAR2_W128.pdf}
        \label{fig:reward_vs_nbc7}
    }
    \end{minipage} 
    \begin{minipage}{0.2\textwidth}
        \subfigure[Ranking]{%
        \includegraphics[width=0.95\linewidth]{figs/PEW/ranking_comparison_EVCAR2_W128.pdf}
        \label{fig:ranking7}
    }
    \end{minipage}
    
    \caption{The next time you purchase a vehicle, how likely are you to seriously consider purchasing an electric vehicle?
A: Very likely; B: Somewhat likely; C: Not too likely; D: Not at all likely; E: I do not expect to purchase a vehicle
}
\label{fig:example7}
\end{figure}

\begin{figure}[!htbp]
    \centering
    \begin{minipage}{0.6\textwidth}
        \centering
    % First Row
    \subfigure[Rewards per Type]{%
        \includegraphics[width=0.98\linewidth]{figs/PEW/rew_FAVPOL_HARRIS_W130.pdf}
        \label{fig:prop_rew8}
    }%
    \hfill
    \subfigure[Avg Reward vs. NBC]{%
        \includegraphics[width=0.98\linewidth]{figs/PEW/reward_vs_NBC_FAVPOL_HARRIS_W130.pdf}
        \label{fig:reward_vs_nbc8}
    }
    \end{minipage} 
    \begin{minipage}{0.2\textwidth}
        \subfigure[Ranking]{%
        \includegraphics[width=0.95\linewidth]{figs/PEW/ranking_comparison_FAVPOL_HARRIS_W130.pdf}
        \label{fig:ranking8}
    }
    \end{minipage}
    
    \caption{ What is your overall opinion of    Kamala Harris?
A: Very favorable; B: Mostly favorable; C: Mostly unfavorable; D: Very unfavorable; E: Never heard of this person.
}
\label{fig:example8}
\end{figure}

\begin{figure}[!htbp]
    \centering
    \begin{minipage}{0.6\textwidth}
        \centering
    % First Row
    \subfigure[Rewards per Type]{%
        \includegraphics[width=0.98\linewidth]{figs/PEW/rew_FAVPOL_BIDEN_W130.pdf}
        \label{fig:prop_rew9}
    }%
    \hfill
    \subfigure[Avg Reward vs. NBC]{%
        \includegraphics[width=0.98\linewidth]{figs/PEW/reward_vs_NBC_FAVPOL_BIDEN_W130.pdf}
        \label{fig:reward_vs_nbc9}
    }
    \end{minipage} 
    \begin{minipage}{0.2\textwidth}
        \subfigure[Ranking]{%
        \includegraphics[width=0.95\linewidth]{figs/PEW/ranking_comparison_FAVPOL_BIDEN_W130.pdf}
        \label{fig:ranking9}
    }
    \end{minipage}
    
    \caption{ What is your overall opinion of Joe Biden?
A: Very favorable; B: Mostly favorable; C: Mostly unfavorable; D: Very unfavorable; E: Never heard of this person.
}
\label{fig:example9}
\end{figure}


% \begin{figure}[!htbp]
%     \centering
%     \begin{minipage}{0.6\textwidth}
%         \centering
%     % First Row
%     \subfigure[Rewards per Type]{%
%         \includegraphics[width=0.98\linewidth]{figs/PEW/rew_POLPROB_W127.pdf}
%         \label{fig:prop_rew10}
%     }%
%     \hfill
%     \subfigure[Avg Reward vs. NBC]{%
%         \includegraphics[width=0.98\linewidth]{figs/PEW/reward_vs_NBC_POLPROB_W127.pdf}
%         \label{fig:reward_vs_nbc10}
%     }
%     \end{minipage} 
%     \begin{minipage}{0.2\textwidth}
%         \subfigure[Ranking]{%
%         \includegraphics[width=0.95\linewidth]{figs/PEW/ranking_comparison_POLPROB_W127.pdf}
%         \label{fig:ranking10}
%     }
%     \end{minipage}
    
%     \caption{ How much of a problem do you think police violence against Black people is in the United States today? A: Major problem; B: Minor problem; C: Not a problem.
% }
% \label{fig:example10}
% \end{figure}


\section{Proof of Fact~\ref{fact:inv:soln:exists}} \label{appendix:proof:inv:soln}

\noindent
Fix $g \in \G$ with its action on $x \in \R^3$ represented by $g(x) = A x + b$. Given that $(\psi, E)$ solves \eqref{eq:schrodinger}, we seek to show that under the stated conditions, $\psi_g(\bx) \coloneqq \psi(g(\bx))$ is also an anti-symmetric solution to \eqref{eq:schrodinger} with respect to the same energy $E$. For $\bx \in \R^{3n}$ and $g \in \G$, denote $\bx_g = g(\bx)$. Then for every $\bx \in \R^{3n}$, 
\begin{align*}
    \Big( - \mfrac{1}{2} \nabla^2 + V(\bx) \Big) \, \psi_{g}( \bx )
    \;=&\; 
    \Big( - \mfrac{1}{2} \nabla^2 + V(\bx) \Big) \, \psi(  A x_1 + b, \ldots, A x_n + b )
    \\
    \;=&\;  
    - \mfrac{1}{2} (A^\top A) \nabla^2  \psi( \bx_{g} )
    + V( \bx ) \psi( \bx_{g} )
    \\
    \;\overset{(a)}{=}&\;
    - \mfrac{1}{2} \nabla^2  \psi( \bx_{g})
    + V( \bx_{g} ) \psi( \bx_{g})
    \;\overset{(b)}{=}\; 
    E \psi( \bx_{g} )
    \;=\; 
    E \psi_{g}( \bx )
    \;.
\end{align*}
In $(a)$, we have used the $\G_\diag$-invariance of $V$;  in $(b)$, we used that $(\psi, E)$ solve \eqref{eq:schrodinger}. Since the above holds for all $\bx \in \R^{3n} $, we get that $(\psi_{g},E)$ also solves the Schr\"odinger's equation. To verify the anti-symmetric requirement, we write the wavefunction $\psi$ as $\psi(\tilde \bx) = \psi(\tilde x_1, \ldots, \tilde x_n)$, where each $\tilde x_i = (x_i, \sigma_i)$ now additionally depends on the spin $\sigma_i \{ \uparrow, \downarrow\}$. Since $\G$ only acts on the spatial position in $\R^3$ and leaves the spins invariant, we can WLOG express the $g$-transformed version of $\tilde x_i$ as $g(\tilde x_i) = (g(x_i), \sigma_i)$. Moreover, since $\sigma \in P_n$ commutes with the diagonal action of $g \in \G$ on $(\R^3 \times \{\uparrow, \downarrow\})^n$, 
\begin{align*}
    \psi_g( \sigma(\tilde \bx) )
    \;=\; 
    \psi( g \circ \sigma (\tilde \bx) )
    \;=\;
    \psi( \sigma \circ g(\tilde \bx) ) 
    \;\overset{(c)}{=}\; 
    \textrm{sgn}(\sigma) \, \psi( g(\tilde \bx) ) 
    \;=\; 
    \textrm{sgn}(\sigma) \, \psi_g( \tilde \bx ) \;.
\end{align*}
In $(c)$, we have used the anti-symmetry of $\psi$. This proves that $(\psi_{g},E)$ solves \eqref{eq:schrodinger} with the correct anti-symmetric requirement. To prove the theorem statement, we see that $\psi^\G$ is a linear combination of finitely many eigenfunctions $(\psi_{g})_{g \in \cG}$ with the same eigenvalue $E$ and therefore yields an anti-symmetric solution with the same energy $E$. 
\qed 


\section{Proofs for data augmentation and group-averaging}  \label{appendix:proof:DA:GA}

\subsection{Proof of Proposition~\ref{prop:DA}}
We first compute the difference in expectation as
\begin{align*}
    \| \mean[ \delta \theta^{(\rm DA)}] - \mean[\delta \theta^{(\rm OG)}] \|
    \;=&\;
    \Big\| 
        \mean\Big[ \mfrac{1}{N} \msum_{i \leq N/k} \msum_{j \leq k}
    F_{\bg_{i,j}(\bX_i); \psi_\theta} \Big] 
        - 
        \mean\Big[ \mfrac{1}{N} \msum_{i \leq N} F_{\bX_i; \psi_\theta} 
        \Big] 
    \Big\|
    \\
    \;=&\;
    \| \mean[  F_{\bg_{1,1}(\bX_1); \psi_\theta} ] - \mean[  F_{\bX_1; \psi_\theta} ] \|
    \;=\; 
    \Big\| \mean_{\bX \sim p^{(m)}_{\psi_\theta; {\rm DA}}}[  F_{\bX; \psi_\theta} ] - \mean_{\bY \sim p^{(m)}_{\psi_\theta}}[  F_{\bY; \psi_\theta} ] \Big\|
    \;.
\end{align*}
In the last line, we used linearity of expectation, the fact that $\bg_{i,j}(\bX_i)$'s are identically distributed and that $\bX_i$'s are identically distributed. Recall that $ \big\{ 
    \bx \mapsto 
    \big( 
        F_{\bx;\psi_\theta}
        \,,\,
        F_{\bx;\psi_\theta}^{\otimes 2}
    \big)
    \,\big|\, 
    \theta \in \R^q
    \big\} 
    \,\subseteq\, \cF$
and $d_\cF(p,q) = \sup_{f \in \cF} \| \mean_{\bX \sim p}[f(\bX)] -  \mean_{\bY \sim q}[f(\bY)] \|$. This implies that
\begin{align*}
    \| \mean[ \delta \theta^{(\rm DA)}] - \mean[ \delta \theta^{(\rm OG)}] \|
    \;=\;
    \Big\| \mean_{\bX \sim p^{(m)}_{\psi_\theta; {\rm DA}}}[  F_{\bX; \psi_\theta} ] - \mean_{\bY \sim p^{(m)}_{\psi_\theta}}[  F_{\bY; \psi_\theta} ] \Big\|
    \;\leq\; 
    d_\cF \big(  p^{(m)}_{\psi_\theta; {\rm DA}},  p^{(m)}_{\psi_\theta} \big),
\end{align*}
which proves the first bound. For the second bound, notice that 
\begin{align*}
    \Var[ \delta \theta^{(\rm DA)} ]
    \;=&\; 
    \Var\Big[ 
        \mfrac{1}{N} \msum_{i \leq N/k} \msum_{j \leq k}
        F_{\bg_{i,j}(\bX_i); \psi_\theta}
    \Big]
    \\
    \;\overset{(a)}{=}&\;
    \mfrac{1}{N/k}
    \Var\Big[ 
        \mfrac{1}{k} \msum_{j \leq k}
        F_{\bg_{1,j}(\bX_1); \psi_\theta}
    \Big]
    \\
    \;\overset{(b)}{=}&\;
    \mfrac{1}{N}
    \Var[ F_{\bg_{1,1}(\bX_1); \psi_\theta}]
    +
    \mfrac{k-1}{N} 
    \Cov[  F_{\bg_{1,1}(\bX_1); \psi_\theta},  F_{\bg_{1,2}(\bX_1); \psi_\theta} ]
    \\
    \;\overset{(c)}{=}&\;
    \mfrac{1}{N}
    \Var[ F_{\bg_{1,1}(\bX_1); \psi_\theta}]
    +
    \mfrac{k-1}{N} 
    \Var \, \mean\big[ F_{\bg_{1,1}(\bX_1); \psi_\theta} \big| \bX_1 \big]
    \;.
\end{align*}
In $(a)$, we have noted that the summands are i.i.d.~across $i \leq N/k$; in $(b)$, we have computed the variance of the sum explicitly by expanding the expectation of a double-sum; in $(c)$, we have applied the law of total covariance to obtain that 
\begin{align*}
    &\;\Cov[  F_{\bg_{1,1}(\bX_1); \psi_\theta},  F_{\bg_{1,2}(\bX_1); \psi_\theta} ]
    \\
    &\hspace{5em}
    \;=\;
    \underbrace{\Cov\big[ 
        \mean[ F_{\bg_{1,1}(\bX_1); \psi_\theta} | \bX_1 ]
        \,,\,  
        \mean[ F_{\bg_{1,2}(\bX_1); \psi_\theta} | \bX_1 ] 
    \big]}_{ =  \, \Var \, \mean[ F_{\bg_{1,1}(\bX_1); \psi_\theta} | \bX_1 ] }
    +
        \mean \, 
        \underbrace{\Cov\big[ F_{\bg_{1,1}(\bX_1); \psi_\theta} \,,\, F_{\bg_{1,2}(\bX_1); \psi_\theta} \,\big|\, \bX_1 \big]
    }_{ = 0}
    \;.
\end{align*}
Meanwhile, the same calculation with $k=1$ and $\bg_{1,1}$ replaced by identity gives 
\begin{align*}
    \Var[ \delta \theta^{(\rm OG)} ]
    \;=\;
    \mfrac{1}{N}
    \Var[ F_{\bX_1; \psi_\theta}]
    \;.
\end{align*}
Taking a difference and applying the triangle inequality twice, we have 
\begin{align*}
    &\; 
    \Big\| \Var[ \delta \theta^{(\rm DA)} ] - \Var[ \delta \theta^{(\rm OG)} ] - \mfrac{k-1}{N} 
    \Var \, \mean\big[ F_{\bg_{1,1}(\bX_1); \psi_\theta} \big| \bX_1 \big] 
    \Big\| 
    \;=\;
    \mfrac{1}{N} \big\|  \Var[ F_{\bg_{1,1}(\bX_1); \psi_\theta}] - \Var[ F_{\bX_1; \psi_\theta}] \big\|
    \\
    &\;\leq\;
    \mfrac{1}{N} \big\|  \mean\big[ F_{\bg_{1,1}(\bX_1); \psi_\theta}^{\otimes 2}\big] - \mean\big[ F_{\bX_1; \psi_\theta}^{\otimes 2} \big] \big\|
    +
    \mfrac{1}{N} \big\|  \mean\big[ F_{\bg_{1,1}(\bX_1); \psi_\theta}\big]^{\otimes 2} - \mean\big[ F_{\bX_1; \psi_\theta} \big]^{\otimes 2} \big\|
    \\
    &\;\leq\;
    \mfrac{1}{N} \big\|  \mean\big[ F_{\bg_{1,1}(\bX_1); \psi_\theta}^{\otimes 2}\big] - \mean\big[ F_{\bX_1; \psi_\theta}^{\otimes 2} \big] \big\|
    +
    \mfrac{1}{N} \big\|  \mean\big[ F_{\bg_{1,1}(\bX_1); \psi_\theta}\big]+  \mean\big[ F_{\bX_1; \psi_\theta} \big] \big\| \, \big\|  \mean\big[ F_{\bg_{1,1}(\bX_1); \psi_\theta}\big] -  \mean\big[ F_{\bX_1; \psi_\theta} \big] \big\|
    \\
    &\;\leq\;
    \mfrac{d_\cF \big(  p^{(m)}_{\psi_\theta; {\rm DA}},  p^{(m)}_{\psi_\theta} \big)}{N}
    +
    \mfrac{1}{N} \big( 2 \big\| \mean\big[ F_{\bX_1; \psi_\theta} \big] \big\| + d_\cF \big(  p^{(m)}_{\psi_\theta; {\rm DA}},  p^{(m)}_{\psi_\theta} \big) \big) \, \times \, d_\cF \big(  p^{(m)}_{\psi_\theta; {\rm DA}},  p^{(m)}_{\psi_\theta} \big)
    \\
    &\;=\;
    \mfrac{ 1 + 2 \| \mean[ \delta \theta^{(\rm OG)} ] \| + d_\cF (  p^{(m)}_{\psi_\theta; {\rm DA}},  p^{(m)}_{\psi_\theta} )   }{N} \, \times \,  d_\cF \big(  p^{(m)}_{\psi_\theta; {\rm DA}},  p^{(m)}_{\psi_\theta} \big)
    \;. \tagaligneq \label{eq:var:analysis:DA}
\end{align*} 
This proves the second bound. In the case when $\bg(\bX_1) \overset{d}{=} \bX_1$ for all $\bg \in \Gdiag$, we have $p^{(m)}_{\psi_\theta; {\rm DA}} = p^{(m)}_{\psi_\theta}$ and therefore the bounds above all evaluate to zero. In this case we have $\mean[ \delta \theta^{(\rm DA)}] = \mean[\delta \theta^{(\rm OG)}]$ and 
\begin{align*}
    \Var[ \delta  \theta^{(\rm DA)}] - \Var[ 
        \delta \theta^{(\rm OG)}] \;=\;  \mfrac{k-1}{N} 
    \Var \, \mean\big[ F_{\bg_{1,1}(\bX_1); \psi_\theta} \big| \bX_1 \big]\;,
\end{align*}
which is positive semi-definite. \qed

\subsection{Proof of Lemma~\ref{lem:GA}}

By construction, $\delta \theta^{(\rm GA)} = \frac{1}{N / k} \sum_{i \leq N/k} F_{\bX^\cG_i; \psi^\cG_\theta}$ is a size-$N/k$ empirical average of i.i.d.~quantities. The mean and variance formulas thus follows directly from a standard computation:
\begin{align*}
    \mean[ \delta \theta^{(\rm GA)} ]
    \;=&\;
    \mean[  F_{\bX^\cG_1; \psi^\cG_\theta}  ]
    &\text{ and }&&
    \Var[\delta \theta^{(\rm GA)}]
    \;=&\;
    \mfrac{\Var[  F_{\bX^\cG_1; \psi^\cG_\theta}  ]}{N/k}
    \;. \qedhere
\end{align*}

\subsection{Proof of Theorem~\ref{thm:DA:CLT}} 
To prove the coordinate-wise bound, fix $l \leq p$. Note that if $\sigma^{(\rm DA)}_l = 0$, then $ \delta \theta^{(\rm DA)}_{1l} = \mean[ \delta \theta^{(\rm DA)}_{1l}]$ with probability $1$, implying that the distribution difference is zero and hence the bound is satisfied. In the case $\sigma^{(\rm DA)}_l > 0$, $\delta \theta^{(\rm DA)}_{1l} = \frac{1}{N / k} \msum_{i \leq N / k} F^{(\rm DA)}_{il}$ is an average of i.i.d.~univariate random variables with positive variance. By renormalizing $t$ and applying the Berry-Ess\'een theorem (see Theorem 3.7 of \citet{chen2011normal} or \citet{shevtsova2013optimization} for the version with a tight constant $C_1 = 0.469$) applied to $\frac{1}{N / k} \msum_{i \leq N / k} F_{il}$, we get that
\begin{align*}
    &\;
    \sup_{t \in \R} 
    \,
    \Big|
    \,
        \P\big( \, 
        \delta \theta^{(\rm DA)}_{1l}
        \,\leq\, t 
        \big)
        -
        \P\Big( \, \mean\big[ \delta \theta^{(\rm DA)}_{1l} \big]  
        + (\Var[\delta \theta^{(\rm DA)}_{1l}  ])^{1/2} \, Z_l \, 
        \,\leq\, t 
        \Big)
    \,
    \Big|
    \\
    &=
    \sup_{t \in \R} 
    \,
    \Big|
    \,
        \P\Big( \, 
        \mfrac{1}{N / k} \msum_{i \leq N / k} \big( F^{(\rm DA)}_{il}  - \mean\big[F^{(\rm DA)}_{il}\big] \big)
        \,\leq\, t 
        \Big)
        -
        \P\Big( \, \mfrac{\sigma^{(\rm DA)}_l}{\sqrt{N/k}} \, Z_l \, 
        \,\leq\, t 
        \Big)
    \,
    \Big|
    \leq
    \mfrac{C_1 \, \mean | F^{(\rm DA)}_{1l} |^3}{ \sqrt{N/k} \, \big(\sigma^{(\rm DA)}_l\big)^3}
    \;.
\end{align*}
This proves the first set of bounds. To prove the second set of bounds, we first denote the mean-zero variable $\bar F_{il} \coloneqq - F^{(\rm DA)}_{il} + \mean[F^{(\rm DA)}_{il}]$, and let $(\bar Z_{11}, \ldots, \bar Z_{np})$ be an $\R^{np}$-valued Gaussian vector with the same mean and variance as $(\bar F_{11}, \ldots, \bar F_{np})$. The difference in distribution function can be re-expressed as 
\begin{align*}
    &\;
    \sup_{t \in \R} 
    \,
    \Big|
    \,
        \P\Big( \, 
            \max_{l \leq p} \Big|\delta  \theta^{(\rm DA)}_{1l} -  \mean\big[\delta  \theta^{(\rm DA)}_{1l} \big]   \Big|
        \,\leq\, t 
        \big)
        -
        \P\Big( \, 
        \max_{l \leq p} \Big|
            (\Var[ \delta \theta^{(\rm DA)}_{1l}  ])^{1/2} \, Z_l \,
        \Big| 
        \,\leq\, t 
        \Big)
    \,
    \Big|
    \\
    &
    \;=\;
    \sup_{t \in \R} 
    \,
    \Big|
    \,
        \P\Big( \, 
            \max_{l \leq p} \Big| \mfrac{1}{\sqrt{ N / k }} \msum_{i \leq N/k} \bar F_{il} \Big|
        \,\leq\, t 
        \big)
        -
        \P\Big( \, 
        \max_{l \leq p} \Big|
        \mfrac{1}{\sqrt{ N / k}} \msum_{i \leq N/k} \bar Z_{il} 
        \Big| 
        \,\leq\, t 
        \Big)
    \,
    \Big|
    \;,
    \tagaligneq \label{eq:DA:diff:intermediate}
\end{align*}
where we have used the definition of $\theta^{(\rm DA)}_1$ and also replaced $t$ by $t  / \sqrt{N k}$. Note that $\bar F_{il}$'s are i.i.d.~mean-zero across $1 \leq i \leq N/k$ and $\bar Z_{il}$'s are i.i.d.~mean-zero across $1 \leq i \leq N/k$. As before, if $\sigma^{(\rm DA)}_l = 0$ for all $1 \leq l \leq p$, the two random variables to be compared are both $0$ with probability $1$ and the distributional difference above evaluates to zero. If there is at least one $l$ such that $\sigma^{(\rm DA)}_l = 0$, we can restrict both maxima above to be over $l \leq p$ such that $\sigma^{(\rm DA)}_l = 0$ and ignore the coordinates with zero variance. As such, we can WLOG assume that $ \sigma^{(\rm DA)}_l > 0$ for all $l \leq p$. Now write 
\begin{align*}
    \tilde F_i \;\coloneqq\; ( \sigma_1^{-1} \bar F_{i1}, \ldots, \sigma_p^{-1} \bar F_{ip})\;,
    \qquad 
    \tilde Z_i \;\coloneqq\; ( \sigma_1^{-1} \bar Z_{i1}, \ldots,  \sigma_p^{-1} \bar Z_{ip})\;,
\end{align*}
and denote the hyper-rectangular set $\cA(t) \coloneqq [ - \sigma_1^{-1} t, + \sigma_1^{-1} t ]
\times  \cdots  \times 
[ - \sigma_p^{-1} t, + \sigma_p^{-1} t ] \subseteq \R^q$. We can now express the difference above further as 
\begin{align*}
    \eqref{eq:DA:diff:intermediate}
    \;=&\;
    \sup_{t \in \R} 
    \,
    \Big|
    \,
        \P\Big( \, 
        \mfrac{1}{\sqrt{ N / k }} \msum_{i \leq N/k} \tilde F_i 
        \in 
        \cA(t) 
        \Big)
        -
        \P\Big( \, 
        \mfrac{1}{\sqrt{ N / k }} \msum_{i \leq N/k} \tilde Z_i 
        \in 
        \cA(t) 
        \Big)
    \,
    \Big|\;.
\end{align*}
This is a difference in distribution functions between a normalized empirical average of $p$-dimensional vectors with zero mean and identity covariance and a standard Gaussian in $\R^q$, measured through a subset of hyperrectangles. In particular, this is the quantity controlled by \citet{chernozhukov2017central}: Under the stated moment conditions and applying their Proposition 2.1, we have that for some absolute constant $C_2 > 0$,
\begin{align*}
    \eqref{eq:DA:diff:intermediate}
    \;\leq\;
    C_2
    \Big( 
    \mfrac{  (\tilde F^{(\rm DA)})^2 \, (\log (p N / k) )^7}
    { N / k }  
    \Big)^{1/6}\;.
    \tag*{\qed}
\end{align*}

\section{Proofs for results on canonicalization}  \label{appendix:proof:canon}

\subsection{Proof of Theorem~\ref{thm:SC}}

To prove (i), we fix any $\tilde g \in \G$, and WLOG let $k=1$. Then by the definition of $\psiSCone_{\theta;\epsilon}$,
\begin{align*}
    &\; \psiSCone_{\theta;\epsilon}
    (\tilde g(x_1), \ldots, \tilde g(x_n)) 
    \\
    & \hspace{4em} 
    \;=\;  
    \sum_{\substack{g \in \G \text{ s.t. } \\ d(\tilde g(x_1), g(\Pi_0)) \leq \epsilon}} 
    \Big(
    \, 
    \mfrac{ \lambda_\epsilon\big( d(\tilde g(x_1), g(\Pi_0)) \big) }{\sum_{\substack{g' \in \G \text{ s.t. } \\ d(\tilde g(x_1), g'(\Pi_0)) \leq \epsilon }}  \lambda_\epsilon\big( d(\tilde g(x_1), g'(\Pi_0)) \big)  }
    \times 
        \psi_\theta\big( g^{-1} \tilde g(x_1), \ldots, g^{-1} \tilde g(x_n) \big)
    \Big)\;.
\end{align*}
Relabelling $g$ by $\tilde g g$ and $g'$ by $\tilde g g'$ in the sums above, and noting that $d(\tilde g(x_1), \tilde g g(\Pi_0)) = d(x_1, g(\Pi_0))$, we obtain that 
\begin{align*}
    \psiSCone_{\theta;\epsilon}
    (\tilde g(x_1), \ldots, \tilde g(x_n)) 
    \;=&\;
    \sum_{\substack{g \in \G \text{ s.t. } \\ d(\tilde g(x_1), \tilde g g(\Pi_0)) \leq \epsilon}} 
    \Big(
    \, 
    \mfrac{ \lambda_\epsilon\big( d(\tilde g(x_1), \tilde g g(\Pi_0)) \big) }{\sum_{\substack{g' \in \G \text{ s.t. } \\ d(\tilde g(x_1),  \tilde g g'(\Pi_0)) \leq \epsilon}}  \lambda_\epsilon\big( d(\tilde g(x_1), \tilde g g'(\Pi_0)) \big)  }
    \,
    \\ 
    &\hspace{8em}
    \times 
        \psi_\theta\big( g^{-1} \tilde g^{-1} \tilde g(x_1), \ldots, g^{-1} \tilde g^{-1} \tilde g(x_n) \big)
    \Big)
    \\
    \;=&\;
    \sum_{\substack{g \in \G \text{ s.t. } \\ d(x_1, g(\Pi_0)) \leq \epsilon}} 
    \, 
    \mfrac{ \lambda_\epsilon\big(  d(x_1, g(\Pi_0)) \big) }{\sum_{\substack{g' \in \G \text{ s.t. } \\ d(x_1,  g'(\Pi_0)) \leq \epsilon}}  \lambda\big(  d(x_1,  g'(\Pi_0))\big)  }
    \,\;
    \psi_\theta\big( g^{-1} (x_1), \ldots, g^{-1} (x_n) \big)
    \\
    \;=&\;
    \psiSCone_{\theta;\epsilon}
    (x_1, \ldots, x_n)
    \;.
\end{align*}
This proves diagonal $\G$-invariance. 

\vspace{.5em}

To prove (ii), we shall make the spin-dependence explicit and write $\tilde x_i = (x_i, \sigma_i)$ and $g(\tilde x_i) = (g(x_i), \sigma_i)$. First consider $\pi$, a transposition that swaps the indices $1$ and $2$. Then
\begin{align*}
    \psiSC_{\theta;\epsilon}(\tilde x_{\pi(1)}, \ldots, \tilde  x_{\pi(n)} )
    \;=&\;
    \mfrac{1}{n} \msum_{k=1}^n
    \psiSCk_{\theta;\epsilon}(\tilde  x_2, \tilde x_1, \tilde  x_3 \ldots, \tilde x_n )
    \;.
\end{align*}
By the anti-symmetry of $\psi_\theta$, we see that for $k \geq 3$,
\begin{align*}
    \psiSCk_{\theta;\epsilon}(\tilde x_2, \tilde x_1, \tilde x_3 \ldots, \tilde x_n )
    \;=&\;
    \msum_{g \in \cG_\epsilon(x_k)} 
    \, 
    w^g_\epsilon ( x_k)
    \, 
    \psi_\theta \big( g^{-1}(\tilde x_2), g^{-1}(\tilde x_1), g^{-1}(\tilde x_3), \ldots, g^{-1}(\tilde x_n) \big)
    \\
    \;=&\;
    -
    \msum_{g \in \cG_\epsilon(x_k)} 
    \, 
    w^g_\epsilon (x_k)
    \, 
    \psi_\theta \big( g^{-1}(\tilde x_1), g^{-1}(\tilde x_2), g^{-1}(\tilde x_3), \ldots, g^{-1}(\tilde x_n) \big)
    \\
    \;=&\;
    -
    \psiSCk_{\theta;\epsilon}(\tilde x_1, \ldots, \tilde x_n )
    \;.
\end{align*}
For the case $k=1,2$, we can apply a similar calculation while noting that the weights remain unchanged, and obtain
\begin{align*}
    \psiSCone_{\theta;\epsilon}(\tilde x_2, \tilde x_1, \tilde x_3 \ldots, \tilde x_n )
    \;=&\;
    - \psiSCtwo_\theta(\tilde x_1, \tilde x_2, \tilde x_3 \ldots, \tilde  x_n ) 
    \;,
    \\
    \psiSCtwo_\theta(\tilde x_2, \tilde x_1, \tilde x_3 \ldots, \tilde x_n ) 
    \;=&\;
    - \psiSCone_{\theta;\epsilon}(\tilde x_1, \tilde x_2, \tilde x_3 \ldots, \tilde x_n ) 
    \;.
\end{align*}
This implies that 
\begin{align*}
    \psiSC_{\theta;\epsilon}(\tilde x_{\pi(1)}, \ldots, \tilde x_{\pi(n)} )
    \;=&\;
    -
    \mfrac{1}{n} \msum_{k=1}^n
    \psiSCk_{\theta;\epsilon}(\tilde x_1, \tilde x_2, \tilde x_3 \ldots, \tilde  x_n )
    \;=\;
    {\rm sgn}(\pi)
    \,
    \psiSC_{\theta;\epsilon}(\tilde x_1,  \ldots, \tilde x_n )
    \;.
\end{align*}
Since the choice of indices $1$ and $2$ are arbitrary, the above in fact holds for all transpositions $\pi$, which implies 
\begin{align*}
    \psiSC_{\theta;\epsilon}(\tilde x_{\tau(1)}, \ldots,\tilde  x_{\tau(n)} )
    \;=\;
    {\rm sgn}(\tau)
    \,
    \psiSC_{\theta;\epsilon}(\tilde x_1, \ldots, \tilde x_n )
\end{align*}
for all permutations $\tau$ of the $n$ electrons. This proves (ii).

\vspace{.5em}

To prove (iii), it suffices to show that for every $k \leq n$, the function $\psiSCk_{\theta;\epsilon}$ defined above is $p$-times continuously differentiable at $\bx$, i.e.~$\nabla^p \psiSCk_{\theta;\epsilon}$ exists and is continuous at $\bx$. Again it suffices to show this for the case $k=1$. Let $\tilde \bx \coloneqq (\tilde x_1, \ldots, \tilde x_n)$ be a vector in a sufficiently small neighborhood of a fixed $\bx \coloneqq (x_1, \ldots, x_n)$, and recall that 
\begin{align*}
    \cG_\epsilon( \tilde x_1 )
    \;=\; 
    \big\{ g \in \G \,\big|\, d( \tilde x_1, g(\Pi_0)) \leq \epsilon \big\}
    \;.
\end{align*}
For $\tilde \bx$ in a sufficiently small neighborhood of $\bx$, $\cG_\epsilon( \tilde x_1 )$ takes value in $\{ \cG_l \}_{0 \leq l \leq M}$, where 
\begin{align*}
    \cG_l
    \;\coloneqq\;
    \cG_\epsilon(x_1)
    \,\setminus\, 
    \{ g^\epsilon_1, \ldots, g^\epsilon_l \} 
    \;,
\end{align*}
and $g^\epsilon_1, \ldots, g^\epsilon_M \in \G$ is an enumeration of all group elements such that 
\begin{align*}
    d\big( x_1 \,,\, g^\epsilon_l(\Pi_0) \big)
    \;=\; 
    \epsilon
    \qquad 
    \text{ for } 0 \leq l \leq M\;.
\end{align*}
Therefore $\psiSCone_{\theta;\epsilon}(\tilde \bx)$ takes values in $\{\psi_{\cG_l}(\tilde \bx) \}_{0 \leq l \leq M}$, where 
\begin{align*}
    \psi_{\cG_l}(\tilde \bx)
    \coloneqq
    \msum_{g \in \cG_l} 
    \mfrac{ \lambda_\epsilon\big( \frac{\epsilon - d(  \tilde x_1, g(\Pi_0))}{\epsilon} \big) }{\sum_{g' \in \cG_l} \lambda_\epsilon\big( \frac{\epsilon - d(\tilde x_1, g'(\Pi_0))}{\epsilon} \big)  }
    \, \psi_\theta\big(
        g^{-1}  (\tilde x_1)
        \,,\,
        \ldots 
        \,,\,
        g^{-1} (\tilde x_n)
    \big) 
    \;.
\end{align*}
Notice that at $\tilde \bx=\bx$, by the definition of $g^\epsilon_l$, we have
\begin{align*}
    \psi_{\cG_l}(\bx)
    \;=&\;
    \msum_{g \in \cG_l} 
    \mfrac{ \lambda_\epsilon \big( d(  x_1, g(\Pi_0)) \big) }{\sum_{g' \in \cG_l}  \lambda_\epsilon \big( d(  x_1, g'(\Pi_0)) \big)  }
    \times
    \, \psi_\theta\big(
        g^{-1}  ( x_1)
        \,,\,
        \ldots 
        \,,\,
        g^{-1}  ( x_n)
    \big) 
    \\ 
    \;\overset{(a)}{=}&\;
    \msum_{g \in \cG_\epsilon(x_1)} 
    \mfrac{ \lambda_\epsilon \big(  d(   x_1, g(\Pi_0))  \big) }{\sum_{g' \in \cG_\epsilon(x_1)} \lambda_\epsilon\big( d( x_1, g'(\Pi_0)) \big)  }
    \times
    \, \psi_\theta\big(
        g^{-1} ( x_1)
        \,,\,
        \ldots 
        \,,\,
        g^{-1}  ( x_n)
    \big) 
    \;=\; \psiSCone_{\theta;\epsilon}(\bx)
    \;.
\end{align*}
Since the above argument works with $\bx$ replaced by $\tilde \bx$, we also have 
\begin{align*}
    \psiSCone_{\theta;\epsilon}(\tilde \bx)
    \;=\;
    \msum_{g \in \tilde \cG_l} 
    \mfrac{ \lambda_\epsilon \big(  d(  x_1, g(\Pi_0)) \big) }{\sum_{g' \in \tilde \cG_l} \lambda_\epsilon\big( d( x_1, g'(\Pi_0)) \big)  }
    \times
    \, \psi_\theta\big(
        g^{-1}  ( x_1)
        \,,\,
        \ldots 
        \,,\,
        g^{-1}  ( x_n)
    \big) 
    \tagaligneq \label{eq:continuity:F1:eps}
\end{align*}
for $0 \leq l \leq \tilde M$, where we have defined 
\begin{align*}
    \tilde \cG_l \;\coloneqq\; \cG_\epsilon(\tilde x_1) \,\setminus\, 
    \{ \tilde g^\epsilon_1, \ldots, \tilde g^\epsilon_l \} 
    \;,
\end{align*}
and $\tilde g^\epsilon_1, \ldots, \tilde g^\epsilon_M \in \G$ is an enumeration of all group elements such that 
\begin{align*}
    d\big( \tilde x_1 \,,\, \tilde g^\epsilon_l(\Pi_0) \big)
    \;=\; 
    \epsilon
    \qquad 
    \text{ for } 0 \leq l \leq M\;.
\end{align*}
Notice that for $\tilde \bx$ in a sufficiently small neighborhood of $\bx$, we have $\{ \tilde \cG_l \}_{l \leq \tilde M} = \{ \cG_l \}_{l \leq M}$, in which case \eqref{eq:continuity:F1:eps} implies 
\begin{align*}
    \psiSCone_{\theta;\epsilon}(\tilde \bx) 
    \;=\; 
    \psi_{\cG_l}(\tilde \bx)
    \qquad 
    \text{ for all } 0 \leq l \leq M\;.
\end{align*}
In other words, we have shown that in a sufficiently small neighborhood of $\bx$, $F_1$ equals $\psi_{\cG_l}$ for all $1 \leq l \leq M$. Recall that $\lambda$ and $d(\argdot, g(\Pi_0))$ are $p$-times continuously differentiable for all $g \in \G$, and $\psi_\theta$ is $p$-times continuously differentiable at $g(\bx)$ for all $g \in \G$ by assumption. This implies that $\psi_{\cG_l}$ is also $p$-times continuously differentiable at $\bx$ and so is $\psiSCone_{\theta;\epsilon}$. Moreover, for $0 \leq q \leq p$, the derivative can be computed as
\begin{align*}
    \nabla^q \psiSCone_{\theta;\epsilon}(\bx)
    \;=\;
    \nabla^q \psi_{\cG_l}(\bx)
    \;=\;
    \msum_{g \in \cG_\epsilon(x_1)} 
    \nabla^q 
    \psiSCk_{\theta; g, x_1}
    (\bx)
    \;,
\end{align*}
where we recall that for $1 \leq k \leq n$, $g \in \G$ and $x, y_1, \ldots, y_n \in \R^3$, 
\begin{align*}
    \psiSCk_{\theta,\epsilon;g,x}
    (y_1, \ldots, y_n)
    \;\coloneqq&\;
    \mfrac{
        \lambda_\epsilon
        \big(  d( y_k, g(\Pi_0)) \big) 
    }{
        \sum_{g' \in \cG_\epsilon(x)}
        \lambda_\epsilon
        \big(  d( y_k, g'(\Pi_0))  \big) 
    }
    \,
    \psi_\theta( g^{-1}(y_1) \,,\, \ldots \,,\, g^{-1}(y_n) )
    \;.
\end{align*}
The same argument applies to all $\psiSCk_{\theta;\epsilon}$'s with $k \leq n$ and therefore for $0 \leq q \leq p$,
\begin{align*}
    \nabla^q 
    \psiSC_{\theta;\epsilon}(x_1, \ldots, x_n)
    \;=&\;
    \mfrac{1}{n} \msum_{k=1}^n 
    \nabla^q 
    \psiSCk_{\theta;\epsilon}(x_1, \ldots, x_n)
    \;=\;
    \mfrac{1}{n}
    \msum_{k=1}^n
    \msum_{g \in \cG_\epsilon(x_k)}
    \,
    \nabla^q 
    \psiSCk_{\theta,\epsilon;g,x_k}
    (\bx)\;,
\end{align*}
which proves (iii). \qed

\subsection{Proof of Lemma~\ref{lem:eps:lamb:blowup}} 

Since $\lambda_\epsilon(0) = 1$, $\lambda_\epsilon(\epsilon) = 0$ and $\lambda_\epsilon$ is continuously differentiable, by the mean value theorem, there is $y_1 \in (0, \epsilon)$ such that 
\begin{align*}
    \partial\lambda_\epsilon(y_1) \;=\; (1-0)/ \epsilon \;=\; \epsilon^{-1}   \;.
\end{align*}    
Meanwhile, since $\lambda_\epsilon(w) = 1$ for  all $w \leq 0$, $\partial \lambda_\epsilon(w) = 0$ for  all $w < 0$, and since $\partial \lambda_\epsilon$ is continuous, $\partial \lambda_\epsilon(0) = 0$. By the twice continuous differentiability of $\lambda_\epsilon$ and the mean value theorem again, there exists $y'_1 \in (0, y_1) \subset (0, \epsilon)$ such that 
\begin{align*}
    \partial^2 \lambda_\epsilon(y'_1) \;=\; (\epsilon^{-1}-0)/ y_1 \;=\; \mfrac{1}{\epsilon y'_1} \;\geq\; \mfrac{1}{\epsilon^2}  \;.
\end{align*}    
Since $\partial^2 \lambda_\epsilon(w) = 0$ for all $w < 0$ and $\partial^2 \lambda_\epsilon$ is continuous, $\partial^2 \lambda_\epsilon(0) = 0$. As $\partial^2 \lambda_\epsilon(y'_1) \geq  \mfrac{1}{\epsilon^2}$, by the intermediate value theorem, there exists some $y_2 \in [0, y_1] \subseteq [0, \epsilon]$ such that  $\partial^2 \lambda_\epsilon(y_2) = \epsilon^{-2}$. \qed
% \section{Reproducibility Checklist}
% Please refer to the technical appendix titled 'Reproducibility Checklist', which is attached in the supplementary material of this submission.

\section{Experimental Setup \label{sec:hyperParams}} 
All training experiments were performed
on public datasets using a single A100 40GB GPU for a
maximum of two days. All experiments were conducted using PyTorch, and results are averaged over three seeds. All hyperparameters are detailed in Tab.~\ref{tab:NLPhyperpams} and Tab.~\ref{tab:Vsionhyperpams}.

\begin{table}[h]
\centering
\small
\begin{tabular}{l c}
\toprule
\textbf{Parameter} & \textbf{Value} \\
\midrule
Model-width & 192 \\
Number of layers & 24 \\
Number of patches & 196 \\
%Scan Mode {\color{red} ???} & one directional \\
Batch-size & 512 \\
Optimizer & AdamW \\
Momentum & \( \beta_1, \beta_2 = 0.9, 0.999 \) \\
Base learning rate & $5e-4$ \\
Weight decay & 0.1 \\
Dropout & 0 \\
Training epochs & 300 \\
Learning rate schedule & cosine decay \\
Warmup epochs & 5 \\
Warmup schedule & linear \\ 
Degree of Taylor approx. (Eq.~\ref{eq:simplifiedModel}) & 3 \\
\bottomrule
\end{tabular}
\caption{Hyperparameters for image-classification via Vision Mamba variants} 
\label{tab:Vsionhyperpams}
\end{table}

\begin{table}[h]
\centering
\small
\begin{tabular}{l c}
\toprule
\textbf{Parameter} & \textbf{Value} \\
\midrule
Model-width & 386 \\
Number of layers & 12 \\
Context-length (training) & 1024 \\
Batch-size & 32 \\
Optimizer & AdamW \\
Momentum & \( \beta_1, \beta_2 = 0.9, 0.999 \) \\
Base learning rate & $1.5e-3$ \\
Weight decay & 0.01 \\
Dropout & 0 \\
Training epochs & 20 \\
Learning rate schedule & cosine decay  \\
Warmup epochs & 1  \\
Warmup schedule & linear  \\ 
Degree of Taylor approx. (Eq.~\ref{eq:simplifiedModel}) &  3\\
\bottomrule
\end{tabular}
\caption{Hyperparameters for language modeling via Mamba-based LMs} 
\label{tab:NLPhyperpams}
\end{table}

% \section{Additional Background Material}

% {\noindent\textbf{Rademacher Complexities}}
% We explore the generalization capabilities of overparameterized NNs by analyzing their Rademacher complexity. This measure provides an upper bound on the worst-case generalization gap, which represents the difference between training and testing errors within a specific hypothesis class. It is defined as the expected performance of the class averaged over all possible data labelings, with labels independently and uniformly drawn from the set $\{\pm 1\}$. For further details, refer to \citep{mohri2018foundations, Shalev-Shwartz2014, bartlett2002rademacher}.


% \begin{definition}[Rademacher Complexity] Let $\mathcal{F}$ be a set of real-valued functions $f_w:\mathcal{X} \to \mathbb{R}^\mathcal{C}$ defined over a set $\mathcal{X}$. Given a fixed sample $X = \{ x_j\}_{j=1}^m \in \mathcal{X}^m$, the empirical Rademacher complexity of $\mathcal{F}$ is defined as follows: 
% \begin{equation*}
% \mathcal{R}_{X}(\mathcal{F}) ~:=~ \frac{1}{m} \mathbb{E}_{\xi: \xi_{ic} \sim U[\{\pm 1\}]} \left[ \sup_{f_w \in \mathcal{F}} \sum^{m}_{j=1}\sum^{\mathcal{C}}_{c=1} \xi_{ic} f_w(x_j)_c  \right].
% \end{equation*}
% \end{definition}

% {\color{red}
% It should be moved:
% We focus on training models for classification tasks. The problem is formally defined by a distribution $P$ over pairs $(x,y)\in \cX\times \cY$, where $\cX \subset \R^{\mathcal{D}}$ represents the input space, and $\cY \subset \R^\mathcal{C}$ is the label space containing one-hot encoded labels for the integers $1,\ldots,\mathcal{C}$. 

% We define a hypothesis class $\cF \subset \{f_w:\cX\to \R^\mathcal{C}\}$ (such as a neural network architecture), where each function \( f_w \in \cF \) is parameterized by a vector of trainable weights $w$. Given any input \( x \in \cX \), \( f_w \) provides a predicted label, and the performance is evaluated based on the \emph{expected error}, \(\err_P(f_w) := \mathbb{E}_{(x, y) \sim P}\left[\mathbb{I}\left[\max_{j \neq y}(f_w(x)_j) \geq f_w(x)_{y}\right]\right]\). Here, the indicator function \(\mathbb{I}\) returns 1 for True and 0 for False.

% Since the full distribution \( P \) is not directly accessible, our objective is to train a model \( f_w \) using a training dataset \( S = \{(x_i, y_i)\}_{i = 1}^m \) consisting of independent and identically distributed (i.i.d.) samples from \( P \). We aim to achieve accurate predictions while applying regularization to control the complexity of \( f_w \). 

% To denote the entire $n$th row and $n$th column of a matrix $M$, we use the notation:
% \begin{align*}
% M_{n*} & \text{ denotes the entire } n \text{th row of matrix } M. \\
% M_{*n} & \text{ denotes the entire } n \text{th column of matrix } M. \\
% M_{nk} & \text{ denotes the element in the } n \text{th row and } k \text{th column.}
% \end{align*}
% {\bf Selective State Space Models.\enspace} 
% Time-variant SSMs, namely, the matrices $A,B,C$ of each channel are modified over $L$ time steps. We are focusing on selective SSMs of the following form. 
% A neural network $f_w : \mathbb{R}^{D \times L} \rightarrow \mathbb{R}^{\mathcal{C}}$ takes a sequence $x=(x_{*1},...,x_{*L}) \in \mathbb{R}^{D \times L}$ as input where $L$ is the length of the sequence and $D$ is the dimension of the tokens $x_i$. We denote
% \begin{align*}
%     B_i &= B x_{*i}, B \in \mathbb{R}^{N \times D} \\
%     C_i &= C x_{*i}, C \in \mathbb{R}^{N \times D} \\
%     \Delta_{*i} &= S_{\Delta} x_{*i}, S_{\Delta} \in \mathbb{R}^{D \times D} \rightarrow \Delta \in \mathbb{R}^{D \times L} \\
%     \bar{A}_{j,i} &= (z^2 + \alpha z)(\Delta_{j,i} * \underbrace{A_{j*}}_{\textbf{ $j$th row of A } } ), A \in \mathbb{R}^{D \times N}, \text{$(z^2+\alpha z)$ is an activation function, } \alpha \geq 0 \\ 
%     W &\in \mathbb{R}^{\mathcal{C} \times D} 
% \end{align*}
% }





{
\bibliographystyle{ims}
\bibliography{bib}
}



\end{document}
