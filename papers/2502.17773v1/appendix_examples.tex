\section{More Examples for \Cref{sec-general}}\label{sec-examples}

In this section, we provide more examples for the general problem framework in \Cref{sec-general}. In these examples, the survey responses can be real-valued or multi-dimensional.


\begin{example}[Market research]
Suppose a company is interested in learning its customers' willingness-to-pay (WTP) for a new product, which is the highest price a customer is willing to pay for the product. Then, each $\profile \in \profilespace$ can represent a customer profile (e.g., age, gender, occupation), each survey question $\testfunction$ is about a certain product, and a customer's response $\response$ is a noisy observation of the customer's WTP. Then $\responsedistalt(~\cdot\mid \testfunction )$ is the distribution of the customer population's WTP. We may take $\statistic (\testfunction)$ as the $\tau$-quantile of the WTP distribution $\responsedistalt(~\cdot\mid \testfunction )$, for some $\tau\in(0,1)$:
\[
\statistic (\testfunction) = \inf \left\{ q\in[0,\infty) : \PP_{\response \sim \responsedistalt(\cdot\mid \testfunction )} (\response\le q) \ge \tau \right\}.
\] 
An LLM can be used to simulate customers' WTP for the product.
\end{example}

\begin{example}[Public survey, multi-dimensional]\label{example-public-survey}
Suppose an organization is interested in performing a public survey in a city. Each survey question $\testfunction$ is a multiple-choice question with $5$ options. An example is ``How often do you talk to your neighbors?'', with $5$ choices ``Basically every day'', ``A few times a week'', ``A few times a month'', ``Once a month'', and ``Less than once a month''. Every $\profile \in \profilespace$ is a person's profile (e.g., age, gender, occupation), the response space $\responsespace$ is the standard orthonormal basis $\{e_i\}_{i=1}^5$ in $\RR^5$, where $\response = e_i$ indicates that a person chooses the $i$-th option. We can take $\statistic ( \testfunction ) = \EE_{\response \sim \responsedist(\cdot\mid\testfunction)}[y]\in\RR^5$, which summarizes the proportion of people that choose the $i$-th option. An LLM can be used to simulate people's answers to the survey question.
\end{example}

\begin{example}[Public survey, one-dimensional]\label{example-public-survey-1D}
Consider the setup in \Cref{example-public-survey}. When the $5$ choices in a survey question correspond to ordered sentiments, we can map them to numeric scores, say, $v = (-1,-\frac{1}{3},0,\frac{1}{3},1)^\top$. Then the statistic $\widetilde{\statistic} (\testfunction) = \langle v,\statistic (\testfunction)  \rangle$ reflects the population's average sentiment in the survey question $\testfunction$.
\end{example}

