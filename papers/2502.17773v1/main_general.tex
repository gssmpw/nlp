\section{General Setup and Methodology}\label{sec-general}

In this section, we study the more general setting where survey responses and confidence sets can be multi-dimensional.


\subsection{Problem Formulation}

Let $\profilespace$ be a profile space, $\distribution$ a probability distribution over $\profilespace$ which represents the true population, and $\simdistribution$ a synthetic distribution over $\profilespace$ used to generate synthetic profiles. 

Let $\testfunctionfamily$ be a collection of survey questions, and $\responsespace$ be the space of possible responses to the survey questions. When a person with profile $\profile\in\profilespace$ is asked a survey question $\testfunction\in\testfunctionfamily$, the person gives a response $\response$ following a distribution $\responsedist( ~ \cdot \mid \profile, \testfunction)$ over $\responsespace$. We are interested in the distribution of the population's response to the survey question $\testfunction$, which is given by $\responsedistalt(~ \cdot \mid \testfunction )
=
\int_{\profilespace} \responsedist( ~ \cdot \mid \profile, \testfunction) \, \distribution ( d \profile )$. In particular, we seek to construct a confidence set for some statistic $\statistic( \testfunction )$ of $\responsedistalt(~ \cdot \mid \testfunction )$, which can be multi-dimensional, say in $\RR^d$. Below we revisit the educational test example in \Cref{sec-warmup} in this framework. More examples are provided in \Cref{sec-examples}.

\begin{example}[Educational test evaluation]\label{example-education}
In the educational test evaluation example in \Cref{sec-warmup}, each $\profile\in\profilespace$ is a student profile, each $\testfunction\in\testfunctionfamily$ is a test question, the response space is $\responsespace=\{0,1\}$, and $\responsedist( ~ \cdot \mid \profile, \testfunction) = \Bernoulli ( \performancefunction( \profile , \testfunction) )$. The statistic $\statistic( \testfunction )$ is the probability of a student answering the question correctly: $\EE_{\response \sim \responsedistalt( \cdot \mid \testfunction )} [ y ] = \EE_{\profile \sim \distribution} [ \performancefunction( \profile , \testfunction) ]$.
\end{example}

We consider constructing the confidence set by using simulated responses from an LLM. Given a profile $\profile$, a survey question $\testfunction$ and a prompt $\prompt$, the LLM simulates a response $\simresponse$ from a distribution $\simresponsedist(~\cdot\mid \profile , \testfunction, \prompt)$ which aims to mimic $\responsedist(~\cdot\mid \profile , \testfunction)$. We can generate synthetic profiles $\{ \simprofile_i \}_{i=1}^K$ from some distribution $\simdistribution$, then feed them into the LLM along with $\testfunction$ and $\prompt$. The LLM then generates synthetic responses $\{ \simresponse_i \}_{i=1}^K$, where $\simresponse_i \sim \simresponsedist(~\cdot\mid \simprofile_i , \testfunction, \prompt)$. Here $K$ is the simulation budget.


Using the simulated samples $\simdataset = \{ \simresponse_i \}_{i=1}^K$, we can construct a family of candidate confidence sets such as the one-dimensional CLT-based confidence interval \eqref{eqn-CI-intro-sim}. More generally, the statistics literature has developed a variety of approaches such as inverting hypothesis tests \citep{CBe02}, the bootstrap \citep{Efr79}, and the empirical likelihood ratio function \citep{OWe90}. We will assume access to a black-box procedure $\setmap$ that takes as input a dataset $\dataset$ and outputs a confidence set $\setmap(\dataset)\subseteq\RR^d$. Then, we can construct a family of confidence sets $\{\simCIalt(k)\}_{k=1}^{K}$ by
\begin{equation}\label{eqn-CI-sim}
\simCIalt(k) = \setmap \left( \{ \simresponse_i \}_{i=1}^k \right).
\end{equation}
We also set $\simCIalt(0) = \RR^d$, so $\statistic ( \testfunction ) \in \simCIalt(0)$ always. We will not impose any assumptions on the quality of the confidence sets produced by $\setmap$.

As the LLM may not be a faithful reflection of the true human population, we will make use of real data to choose a good confidence set from $\{\simCIalt(k)\}_{k=1}^{K}$. We assume that we have collected real human responses from $m$ surveys $\testfunction_1,...,\testfunction_m\in\testfunctionfamily$. For each $j\in[m]$, we have responses $\dataset_j = \{ \response_{j,i} \}_{i=1}^{n_j}$ from $n_j$ i.i.d.~surveyees $\{ \profile_{j,i} \}_{i=1}^{n_j} \sim \distribution$, with $\response_{j,i} \sim \responsedist(~\cdot\mid  \profile_{j,i}, \testfunction_j )$. 

We also simulate LLM responses to these $m$ survey questions. For each $j\in[m]$, we feed synthetic profiles $\{ \simprofile_{j,i} \}_{i=1}^K$, the question $\testfunction_j$ and the prompt $\prompt$ into the LLM, which simulates responses $\simdataset_j = \{ \simresponse_{j,i} \}_{i=1}^{K}$ with $\simresponse_{j,i} \sim \simresponsedist (~\cdot \mid \simprofile_{j,i}, \testfunction_j, \prompt )$. The datasets $\{ \dataset_j \}_{j=1}^m $ and $\{\simdataset_j \}_{j=1}^m $ will be used to select a confidence set from $\{ \simCIalt(k) \}_{k=1}^K$. We make the same Assumptions \ref{assumption-iid-test-1D} and \ref{assumption-indep-data-1D} as in \Cref{sec-warmup}.

We are now ready to formally state our problem.

\begin{problem}[Uncertainty quantification]\label{problem-general}
Given $\alpha \in (0,1)$, how to use $\{ \dataset_j \}_{j=1}^m$ and $\{ \simdataset_j \}_{j=1}^m$ to choose $\widehat{k}\in[K]$ such that
\[
\PP \Big( \statistic ( \testfunction ) \in \simCIalt(\widehat{k}) \Big) \approx 1-\alpha?
\]
\end{problem}


\subsection{General Methodology for Sample Size Selection}

We now present our general methodology for Problem \ref{problem-general}. For each $j\in[m]$, we form confidence sets similar to \eqref{eqn-CI-sim} using the synthetic data $\simdataset_j$:
\begin{equation}\label{eqn-CI-sim-calibrate}
\simCIalt_j(k) = \setmap \left( \{ \simresponse_{j,i} \}_{i=1}^k \right),\quad \forall k\in[K].
\end{equation}
We also set $\simCIalt_j(0) = \RR^d$. We will pick $\widehat{k}\in\{0,1,...,K\}$ such that $\simCIalt_j(\widehat{k})$ is a good confidence interval for $\statistic (\testfunctionalt_j)$ for each $j\in[m]$. This choice of $\widehat{k}$ will also be good for $\simCIalt(k)$, thanks to the i.i.d.~assumption on the survey questions.

Ideally, we would like to pick $k$ such that $(1-\alpha)$ coverage is achieved empirically over the $m$ survey functions:
\begin{equation}\label{eqn-oracle-criterion}
\frac{1}{m} \sum_{j=1}^m \ind \{  \statistic( \testfunction_j )  \not\in \simCIalt_j(k) \} \le \alpha.
\end{equation}
However, the population-level quantities $\{ \statistic( \testfunction_j ) \}_{j=1}^m$ are not available, so we must approximate them by the real data $\{ \dataset_j \}_{j=1}^m$. In \Cref{sec-warmup}, we have taken the approach of constructing unbiased point estimates, but it does not directly extend to the more general case. 

Instead, we will use the real data $\{ \dataset_j \}_{j=1}^m$ to construct confidence sets for $\{  \statistic( \testfunction_j )  \}_{j=1}^m$. Choose a confidence level $\gamma\in(0,1)$. For each $j \in [m]$, we use $\dataset_j$ to construct a confidence set $\CIalt_j$ that satisfies
\begin{equation}\label{eqn-CI-calibrate}
\PP\Big( \statistic ( \testfunction_j)  \in \CIalt_j \Bigm| \testfunction_j \Big) \ge \gamma.
\end{equation}
These confidence sets are easy to construct as the samples in $\dataset_j$ follow the true response distribution. When $\statistic ( \testfunction_j)  \in \CIalt_j$, the condition $\CIalt_j\subseteq \simCIalt_j(k)$ is sufficient for $\statistic ( \testfunction_j)  \in \simCIalt_j(k)$. Equivalently, when $\statistic ( \testfunction_j)  \in \CIalt_j$, the condition $\statistic ( \testfunction_j)  \not\in \simCIalt_j(k)$ must imply $\CIalt_j\not\subseteq \simCIalt_j(k)$. Thus, we take
\begin{equation}\label{eqn-proxy}
\coveragealt(k) = \frac{1}{m} \sum_{j=1}^m \ind \{ \CIalt_j\not\subseteq \simCIalt_j(k) \}
\end{equation}
as a proxy for the empirical miscoverage. Since $\statistic ( \testfunction_j)  \in \CIalt_j$ happens with probability $\gamma$, then the frequency of having $\CIalt_j\not\subseteq \simCIalt_j(k)$ is at least $\gamma$ times the frequency of $\statistic ( \testfunction_j)  \not\in \simCIalt_j(k)$. Roughly speaking,
\begin{equation}\label{eqn-proxy-to-oracle}
\coveragealt(k) \ge \gamma \cdot\left( \frac{1}{m} \sum_{j=1}^m \ind \{  \statistic( \testfunction_j )  \not\in \simCIalt_j(k) \} \right).
\end{equation}

Combining \eqref{eqn-oracle-criterion} and \eqref{eqn-proxy-to-oracle} leads to the following criterion for selecting $k$:
\begin{equation}\label{eqn-empirical-criterion}
\widehat{k} = \max \left\{ 0\le k \le K : ~ L(i) \le \gamma\alpha, ~\forall i\le k \right\}.
\end{equation}
Note that $\widehat{k}$ is well-defined because $\coveragealt(0) = 0$. The full procedure for sample size selection is summarized in \Cref{alg-general}. 

\begin{algorithm}[h]
	\begin{algorithmic}
	\STATE {\bf Input:} Survey questions with real and simulated responses $\big\{(\testfunction_j, \dataset_j, \simdataset_j)\big\}_{j=1}^m$, prescribed miscoverage probability $\alpha$, confidence set construction procedure $\setmap$, confidence level $\gamma$, simulation budget $K$. \\[4pt]
	\FOR{$j=1,...,m$}
		\STATE Use $\setmap$ and $\simdataset_j$ to construct synthetic confidence sets $\{\simCIalt_j(k)\}_{k=0}^K$ by \eqref{eqn-CI-sim-calibrate}. \\[4pt]
		\STATE Use $\dataset_j$ to construct a confidence set $\CIalt_j$ satisfying \eqref{eqn-CI-calibrate}.
	\ENDFOR
	\STATE Define
	\[
	\coveragealt(k) = \frac{1}{m} \sum_{j=1}^m \ind \{ \CIalt_j\not\subseteq \simCIalt_j(k) \}.
	\]
	\STATE Set $\widehat{k} = \max \left\{ 0\le k \le K :~ L(i) \le \gamma\alpha ,~\forall i\le k \right\}$. \\[4pt]
%	\STATE Use $\setmap$ and $\simdataset$ to construct synthetic confidence sets $\{\simCIalt ( \widehat{k} )\}_{k=0}^K$ by \eqref{eqn-CI-sim}. \\[4pt]
	\STATE {\bf Output:} Sample size $\widehat{k}$
.	\caption{Simulation Sample Size Selection}
	\label{alg-general}
	\end{algorithmic}
\end{algorithm}

We now present the coverage guarantee for our method, which shows that the chosen confidence set $\simCIalt(\widehat{k})$ has coverage probability at least $1-\alpha-O(1/\sqrt{m})$. Its proof is deferred to \Cref{sec-thm-coverage-proof}.


\begin{theorem}\label{thm-coverage}
Let Assumptions \ref{assumption-iid-test-1D} and \ref{assumption-indep-data-1D} hold. Fix $\alpha\in(0,1)$. The output $\widehat{k}$ of \Cref{alg-general} satisfies
\[
\PP\Big( \statistic ( \testfunction ) \in \simCIalt(\widehat{k}) \Big) \ge 1-\alpha - \gamma^{-1} \sqrt{\frac{1}{2m}}.
\]
\end{theorem}

It is worth noting that our method achieves this coverage without any assumptions on the qualities of the LLM and the procedure $\setmap$ for confidence set construction. Nevertheless, the size of the chosen confidence set $\simCIalt(\widehat{k})$, in terms of the true coverage rate and size, depend on these factors. If there is a large alignment gap between the LLM and the human population, then $\simCIalt(\widehat{k})$ will inevitably be large.




