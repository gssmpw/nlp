\section{Related Works}
In this section, we first briefly review the background of the 5 representative MAD methods, with a special focus on their unique mechanisms. Next, we introduce some recent works that also review MAD methods and explain how our work differs from theirs.

\subsection{MAD frameworks}

~\citet{duimproving} presented the first LLM-based MAD framework. In this framework, multiple agents independently generate proposals and collaboratively engage in deliberation on their respective reasoning processes to reach a consensus on a unified response. To avoid misunderstanding, in the following part of this paper we refer to ~\citet{duimproving} as \textbf{SoM}.

\textbf{Multi-Persona (MP)} \cite{liang2023encouraging} leverages the role-playing capabilities of LLMs to encourage diverse thinking. In this approach, an affirmative agent(angel) and a negative agent(devil) present their answer to a judge agent, which ultimately determines the final solution.

While leveraging multi-persona role-playing, \textbf{ChatEval (CE)} \cite{chanchateval} explores communication strategies among agents through three frameworks, focusing on the impact of asynchronous responses and round-by-round summarization on agent performance.


\textbf{Exchange-of-Thought (EoT)} \cite{yin2023exchange} similarly emphasizes the study of communication strategies. It introduces four distinct communication paradigms and undertakes a comprehensive analysis of communication volume and information propagation speed. Additionally, it implements a confidence evaluation mechanism designed to reduce the adverse effects of erroneous reasoning processes.


\textbf{AgentVerse (AGV)} \cite{chen2023agentverse} is a meticulously designed and sophisticated multi-agent collaboration framework, distinct from conventional MAD frameworks. In AgentVerse, an agent will take the role of HR to hire experts to collaboratively draft the solution, allowing dynamic adjustment of group members based on current progress.

\textbf{FORD}~\cite{xiong-FORD} mitigates the inconsistency as the debate processes, and introduces a judge agent to summarize the debate results.

\textbf{ReConcile}~\cite{chen2024reconcile} adopts confidence-weighted voting to help consensus seeking. 

\textbf{COMM}~\cite{chen2024comm} encourage diverse thinking in the debate by assigning different reasoning paths to agents with different roles.

\textbf{IoA}~\cite{chen2024IoA} organizes agents in a network structure, splitting agents into blocks for better collaboration. 

In \cite{li2024Sparse}, agents communicate through a sparse topological structure. This sparse configuration enhances the depth of collaborative reasoning, enabling more complex inference.

\citet{qian2024scaling} investigated the scaling effects of MAD systems. By constructing various topological structures, the authors found that MAD systems can achieve consistent performance improvements as more agents participate in the reasoning process.

\citet{pham2023let} introduced a novel approach where agents interact using token embeddings instead of natural language. Compared to natural language, token embeddings can convey additional information, such as an agent’s confidence in specific key information.

\subsection{Recent works reviewing MAD}

\citet{smit2023we} evaluated several MAD methods, particularly their performance on medical benchmarks. The authors observed that the level of agreement among agents within MAD frameworks is a critical factor influencing performance. When agents are more inclined to agree with one another, the overall performance improves. However, this study does not comprehensively compare these MAD methods with single-agent inference techniques, leaving the scalability of MAD inconclusive. 

\citet{khan2024debating} analyzed the performance of MAD systems from the perspective of persuasiveness. Instead of testing existing MAD methods, the study examined the impact of model persuasiveness on the overall performance of the MAD framework in a generic scenario involving two debater agents and one judger agent. The authors found that more persuasive models (often stronger models) could enhance the overall performance of the MAD system, even when the judger was a weaker model. While this study delves into the underlying mechanisms of MAD, its focus does not align with the primary goals of our research.

\citet{wang2024rethinking} compared single-agent methods with MAD methods and found that when a single agent is provided with sufficiently detailed problem descriptions and demonstrations to assist its reasoning, single-agent inference can achieve approximate performance to MAD methods. However, this study evaluated only three datasets rather than a comprehensive set of commonly used ones. Furthermore, the single-agent inference method used for comparison was not a widely adopted approach, limiting the generalizability of the findings.

\citet{zhang2023exploring} explores the behavioral logic of MAD from the perspective of social psychology. By combining agents with different individual traits (e.g., stubbornness or conformity ) within an experimental framework, the authors found that certain combinations of personality can enhance the overall performance of the MAD system.


In summary, while some recent works established an initial review of existing MAD frameworks, they have certain limitations that prohibit us from understanding the gap between MAD and other single-agent inference methods. Furthermore, these works lack special attention to MAD's scalability performance, which is a significant limitation considering the community-wide expectation of MAD to be an efficient way to utilize aggregated inference strength. Therefore, it is necessary to establish a more comprehensive evaluation of MAD frameworks, with a special focus on their scaling behavior.
