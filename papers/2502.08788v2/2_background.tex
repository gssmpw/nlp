\section{Background}
In this section, we briefly review the background of MAD methods, with a special focus on their unique mechanisms. We also present a systematic comparison between these MAD methods from several commonly shared dimensions. 

% Based on the information flow through agents, current MAD methods can be represented as a directed graph, where each node corresponds to invoking an LLM agent, and each edge denotes the transfer of outcomes between agents. Specifically, the action performed at each node is defined as follows:
% Based on the information flow through agents, current MAD methods can be represented as a directed graph, where each node corresponds to invoking an LLM agent, and each edge denotes the transfer of outcomes between agents. Specifically, the action performed at each node is defined as follows:

% \begin{equation}
%     a \sim F(G(q,x;h,\bm{\theta},\beta,p))
% \end{equation}
% \\
% where $h$ denotes the chat history of the agent, $\theta$ denotes parameters of the pre-trained base model, $\beta$ denotes inference hyper-parameters, $p$ denotes formatting prompts guiding the debate, $q$ denotes the input question, and $x$ denotes the answers from other agents (output from input nodes). $G$ represents invoking an LLM agent with specified configurations and $F$ denotes the post-processing of the agent's output. For instance, in EoT, the post-processing function $F$ assigns a confidence score to each agent's output, enabling other agents to assess whether the outcome was generated with high confidence. These dimensions are considered since they can be flexibly adjusted in different MAD frameworks and are enough to cover the main components of MAD. 

% The final response, $a$, is sampled from the conditional distribution $F(G(\circ; \circ))$. A digraph denoted as $G= (V,E)$ can describe the architecture of the MAD framework. A directed edge $e = (v_i, v_j) \in E$ will transmit an output $a_i$ from node $v_i$ to its subsequent neighbor $v_j$, which gives $a_i \in x_j$. If $(v_i, v_j)$ is the only incoming edge for $v_j$, then $x_j=\{a_i\}$. Additionally, the workflow of MAD frameworks can sometimes be conditional, depending on specific outputs from certain agents. For instance, in AgentVerse, the verifier’s output may determine whether the solver is invoked again to provide a refined solution.

Inspired by \textbf{The Society of Mind}, ~\citet{duimproving} presented the first LLM-based MAD framework. In this framework, multiple agents independently generate proposals and collaboratively engage in deliberation on their respective reasoning processes to reach a consensus on a unified response. The answers may be optionally summarized before being added to the history that is available to the agents in future rounds. To avoid misunderstanding, in the following part of this paper we refer to ~\citet{duimproving} as \textbf{SoM}.

\textbf{Multi-Persona (MP)} \cite{liang2023encouraging} leverages the role-playing capabilities of LLMs to encourage diverse thinking. In this approach, an affirmative agent(angel) and a negative agent(devil) present their answer to a judge agent, which ultimately determines the final solution.

While leveraging multi-persona role-playing, \textbf{ChatEval (CE)} \cite{chanchateval} explores communication strategies among agents through three frameworks, focusing on the impact of asynchronous responses and round-by-round summarization on agent performance.


\textbf{Exchange-of-Thought (EoT)} \cite{yin2023exchange} similarly emphasizes the study of communication strategies. It introduces four distinct communication paradigms and undertakes a comprehensive analysis of communication volume and information propagation speed. Additionally, it implements a confidence evaluation mechanism designed to reduce the adverse effects of erroneous reasoning processes.


\textbf{AgentVerse (AGV)} \cite{chen2023agentverse} is a meticulously designed and sophisticated multi-agent collaboration framework, distinct from conventional MAD frameworks. In AgentVerse, an agent will take the role of HR to hire experts to collaboratively draft the solution, allowing dynamic adjustment of group members based on current progress.


In Table \ref{tab:mad_features} (in Appendix \ref{appendix:mad_conf}) we compare these MAD frameworks from different dimensions of their implementation. As the first MAD framework, SoM simply aggregates information from agents without much processing. Following works 