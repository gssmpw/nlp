\section{Relation to separator logic}
\label{sec:separator}

In this section we prove \cref{thm:main-separator}. By \cref{thm:low-rank-quantifier-elimination}, it suffices to show that for every weakly sparse graph class $\Cc$, separator logic has low rank definability property on $\Cc$. The plan is as follows. We first prove a combinatorial characterization of low rank sets in weakly sparse graph classes. Next, we use this characterization to prove low rank definability property.

For the characterization, we prove that every low rank set in a graph from a weakly sparse graph class is very close to a (vertex) separation of small order. Let us first recall some graph-theoretic terminology. A {\em{separation}} of a graph $G$ is a pair $(L,R)$ of vertex subsets such that $L\cup R=V(G)$ and there is no edge with one endpoint in $L\setminus R$ and the other in $R\setminus L$; see \cref{fig:sep}. The {\em{order}} of the separation $(L,R)$ is $|L\cap R|$, i.e. the number of vertices the two sides have in common. We shall say that a set of vertices $X\subseteq V(G)$ is {\em{captured}} by a separation $(L,R)$ if we have
\[L\setminus R \subseteq X\subseteq L.\]
That is, $X$ has to contain all the vertices that lie strictly in the left side of the separation (set $L\setminus R$) and may additionally contain a subset of the separator (set $L\cap R$).

	\begin{figure}
\centering
		\includegraphics[page=2,scale=0.22]{low-rank-logic}
		\caption{A separation.}\label{fig:sep}

	\end{figure}

In this terminology, the combinatorial characterization  reads as follows.

\begin{lemma}\label{lem:characterization-separator}
 Let $G$ be a graph that does not contain the complete bipartite graph $K_{t,t}$ as a subgraph. Let $X$ be a set of rank at most $r$ in $G$. Then $X$ is captured by some separation of $G$ of order at most $2^{r+1}(t-1)$.
\end{lemma}
\begin{proof}
 Denote $\wh X\coloneqq V(G)\setminus X$ and consider the bipartite adjacency matrix $M\coloneqq \Adj_G[X,\wh X]$. Call a row of $M$ {\em{frequent}} if it appears in $M$ at least $t$ times, that is, there are at least $t-1$ other rows equal to it. Define {\em{frequent columns}} of $M$ analogously. Note that the entry at the intersection of every frequent row and every frequent column must be $0$, for otherwise the at least $t$ equal rows and the at least $t$ equal columns would induce a $t\times t$ submatrix of $M$ entirely filled with $1$s, which in turn would yield a $K_{t,t}$ subgraph in $G$. This proves that the following sets $L$ and $R$ form a separation of $G$:
 \begin{itemize}[nosep]
  \item $L$ comprises all the vertices of $X$ and those vertices of $\wh X$ whose columns are non-frequent; and
  \item $R$ comprises all the vertices of $\wh X$ and those vertices of $X$ whose rows are non-frequent.
 \end{itemize}
 Clearly, $(L,R)$ captures $X$. Further, $L\cap R$ comprises all the vertices of $X$ whose rows are non-frequent and all the vertices of $\wh X$ whose columns are non-frequent. Since $\rk(X)\leq r$, the rank of $M$ over $\F_2$ is at most $r$, hence $M$ has at most $2^r$ distinct rows and at most $2^r$ distinct columns. Since every non-frequent row occurs at most $t-1$ times in $M$, $M$ has at most $2^r(t-1)$ non-frequent rows; and similarly $M$ has at most $2^r(t-1)$ non-frequent columns. We conclude that $|L\cap R|\leq 2^{r+1}(t-1)$, as required.
\end{proof}

In the next lemma we verify the low rank definability property for $r=0$ and formulas without additional free variables. This is the key step of the proof, where we use the idempotence property of separator logic in order to turn set quantification into first-order quantification. Note here that if $X$ is a set of rank $0$ in a graph $G$, then every connected component of $G$ is either entirely contained in $X$ or is disjoint with $X$. Hence, $X$ is the union of the vertex set of a collection of connected components of $G$.

\begin{lemma}\label{lem:rank0-separator}
 Let $\varphi(X)$ be a formula of separator logic with one free set variable $X$. Then there exists a formula $\varphi'(x,\tup t)$ of separator logic, where $x$ and $\tup t$ are free vertex variables, such that the following holds. For every graph $G$, if there exists a set $A\subseteq V(G)$ of rank $0$ such that $G\models \varphi(A)$, then there exists $A'\subseteq V(G)$ of rank $0$ and an evaluation $\tup d$ of variables $\tup t$ such that
 \[G \models \varphi(A')\qquad \textrm{and}\qquad A' = \varphi'(G, \tup d).\]
\end{lemma}
\begin{proof}
  Let $\Sigma$ be the set of unary predicates used by $\varphi$, and let $q$ be the quantifier rank of $\varphi$. We will use the standard notion of {\em{types}} tailored to separator logic. Concretely, for a $\Sigma$-colored graph $H$, the {\em{type}} of $H$ is the set of all sentences of separator logic in the language of $\Sigma$-colored graphs with quantifier rank at most $q+1$ that are satisfied in $H$. A standard induction argument shows that for fixed $q$ and $\Sigma$, there are only boundedly many (in terms of $q$ and $\Sigma$) non-equivalent such sentences, hence also only boundedly many types. Let $\cal T$ be the set of all types; then $|\cal T|$ is bounded by a function of $q$ and $\Sigma$. Observe that for each $\tau \in \cal T$, there is a sentence $\psi_\tau$ such that $H\models \psi_\tau$ if and only if $H$ has type $\tau$. Indeed, it suffices to take $\psi_\tau$ to be the conjunction of all the sentences belonging to $\tau$.

 Let $\cal C$ be the set of connected components of $G$ and let $\{\cal C_\tau\colon \tau\in \cal T\}$ be the partition of $\cal C$ according to the types of the individual components; that is, a component $C\in \cal C$ belongs to $\cal C_\tau$ iff the type of $C$ is $\tau$. The following claim follows from a standard argument involving Ehrenfeucht--Fra\"iss\'e games.
 
 \begin{claim}\label{cl:EF}
  There exists a constant $p$, depending only on $q$, such that the following holds. Suppose for some type $\tau\in \cal T$, $A$ contains more than $p$ components from $\cal C_\tau$, and $A$ is also disjoint with more than $p$ components from $\cal C_\tau$. Then for every $C\in \cal C_\tau$ contained in $A$, we have $G\models \varphi(A\setminus C)$.
 \end{claim}

 An immediate corollary of \Cref{cl:EF} is the following.

 \begin{claim}\label{cl:idemp}
  Suppose there exists $A\subseteq V(G)$ of rank $0$ such that $G\models \varphi(A)$.
  Then there also exists $A'\subseteq V(G)$ of rank $0$ such that the following holds:
  \begin{itemize}[nosep]
   \item $G\models \varphi(A')$; and
   \item for each $\tau \in \cal T$, $A'$ contains at most $p$ components of $\cal C_\tau$ or $A'$ contains all but at most $p$ components~of~$\cal C_\tau$.
  \end{itemize}
 \end{claim}

 It remains to construct a sentence $\varphi'(x,\tup t)$ that will define some set $A'$ whose existence is asserted by \Cref{cl:idemp}. For this, for each $\tau \in \cal T$ construct a $p$-tuple of variables $\tup t_\tau$, and let $\tup t$ be the union of $\tup t_\tau$ over all $\tau\in \cal T$. For every function $g\colon {\cal T}\to \{0,1\}$, we may use the sentences $\{\psi_\tau\colon \tau\in \cal T\}$ to write a formula $\alpha_g(x,\tup t)$ expressing the following assertion:
 supposing $x$ belongs to a component $C$ whose type is $\tau$, we have that $C$ contains a vertex of $\tup t_\tau$ if $g(\tau)=0$, and $C$ does not contain a vertex of $\tup t_\tau$ if $g(\tau)=1$. Then, we may write $\varphi'(x,\tup t)$ as the formula expressing the following:
 \begin{itemize}[nosep]
  \item for some $g\colon {\cal T}\to \{0,1\}$, we have $G\models \varphi(\alpha_g(G,\tup t))$; and
  \item $\alpha_{g^\circ}(x,\tup t)$, where $g^\circ$ is the lexicographically smallest function from $\cal T$ to $\{0,1\}$ satisfying the above.
 \end{itemize}
 By \Cref{cl:idemp}, for some evaluation $\tup d$ of $\tup t$ and some function $g\colon {\cal T}\to \{0,1\}$, we have $A'=\alpha_g(G,\tup d)$. As $G\models \varphi(A')$, the first point above holds. Then we have $\varphi'(G,\tup d)=\alpha_{g^\circ}(G,\tup d)$, where $g^{\circ}$ is as in the second point. And by construction, this set has rank $0$ and satisfies~$\varphi$.
\end{proof}

We now generalize the conclusion of \cref{lem:rank0-separator} to low rank definability property using \cref{lem:characterization-separator} in combination with \cref{lem:operation-forawrd,lem:operation-backward}.

\begin{lemma}\label{lem:lrdp-separator}
 For every weakly sparse graph class $\Cc$, separator logic has low rank definability property on $\Cc$.
\end{lemma}
\begin{proof}
 Let $r\in \N$ be a bound on the rank and $\varphi(X,\tup Y,\tup z)$ be a formula of separator logic, as in the definition of low rank definability property. We may assume that $\wtup Y=\emptyset$ and $\tup z=\emptyset$.
 Indeed, otherwise we may expand the signature by $|\wtup Y|+|\tup z|$ unary predicates marking the free variables of $\varphi$, and perform the reasoning on the modified formula $\varphi$ that treats the free variables of $\wtup Y$ and $\tup z$ as unary predicates present in the structure.
 Once we obtain a suitable formula $\varphi'$, the additional unary predicates can be interpreted back as free variables.
 Hence, from now $\varphi$ has only one free set variable $X$.

 Since $\Cc$ is weakly sparse, there is some $t\in \N$ such that no graph from $\Cc$ contains $K_{t,t}$ as a subgraph. Consider any $G\in \Cc$ and $A\subseteq V(G)$ with $\rk(A)\leq r$. By \cref{lem:characterization-separator}, there is a separation $(L,R)$ of $G$ of order at most $p\coloneqq 2^{r+1}\cdot (t-1)$ that captures $A$. Let $S\coloneqq L\cap R$; then $|S|\leq p$. Observe that since $(L,R)$ captures $A$, $A$ is the union of a subset of $S$ and a collection of connected components of $G-S$. In other words, $A$ is a set of rank $0$ in $G'$ --- the $S$-operation of $G$ with respect to the separator logic.

 By \cref{lem:operation-forawrd}, there is a formula $\psi(X)$ depending only on $\varphi$ and $p$ such that for every $B\subseteq V(G)$,
 \[G\models \varphi(B)\qquad\textrm{if and only if}\qquad G'\models \psi(B).\]
 In particular, $G'\models \psi(A)$. By \cref{lem:rank0-separator}, we may find a formula $\psi'(x,\tup t)$ depending only on $\psi$ and $p$, an evaluation $\tup d$ of variables $\tup t$, and a set $A'\subseteq V(G)$ of rank $0$ in $G'$, such that
 \[G'\models \psi(A')\qquad \textrm{and}\qquad A'=\psi'(G',\tup d).\]
 By \cref{lem:operation-backward}, there is a formula $\varphi'(x,\tup t,\tup t')$, depending only on $\psi'$ and $p$, such that for every $u\in V(G)$ and evaluation $\tup c$ of $\tup t$, we have
 \[G\models \varphi'(u,\tup c, \tup s)\qquad \textrm{if and only if}\qquad G'\models \psi'(u,\tup c),\]
 where $\tup s$ is the tuple of elements from $S$.
 All in all, we have
 \[G\models \varphi(A')\qquad \textrm{and}\qquad A'=\varphi'(G',\tup d, \tup s).\]
 Therefore, all the conditions required from $\varphi',A'$ are~met.
\end{proof}


Now \cref{thm:main-separator} follows immediately by combining \cref{lem:lrdp-separator} with \cref{thm:low-rank-quantifier-elimination}.
