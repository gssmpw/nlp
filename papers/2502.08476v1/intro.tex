\section{Introduction}\label{sec:intro}

Much of the contemporary work on logic on graphs revolves around the first-order logic \fo and variants of the monadic second-order logic \mso. The general understanding is that \mso behaves well, in terms of computational aspects, on graphs that have a tree-like structure~\cite{courcelleMonadicSecondorderLogic1990,courcelle2000linear}.
The tameness of \fo is the subject of an ongoing line of work and is expected to be connected with the non-existence of complicated local structures~\cite{DreierEMMPT24,DreierMST23,DreierMT24,flippergame,Torunczyk23}. A general overview of this area can be found in the recent survey of Pilipczuk~\cite{Pilipczuk25}.

Naturally, there are interesting concepts of logic on graphs besides \fo, \mso, and their variants. One such logic was proposed independently by Boja\'nczyk~\cite{separator-logic2021} (under the name {\em{separator logic}}) and by Schirrmacher, Siebertz, and Vigny~\cite{schirrmacher2023first} (under the name \foconn). The idea is to extend \fo by a simple mechanism for expressing connectivity: we allow the usage of predicates $\conn_k(s,t,a_1,\ldots,a_k)$, for every $k\in \N$, that verify the existence of a path connecting vertices $s$ and $t$ that avoids vertices $a_1,\ldots,a_k$. Thus, the expressive power of separator logic lies strictly between \fo and \mso: the logic is not bound only to local properties, like \fo, but full set quantification is not available. Boja\'nczyk~\cite{separator-logic2021} argued that on classes of graphs of bounded pathwidth, separator logic characterizes a suitable analogue of star-free languages.
Pilipczuk, Schirrmacher, Siebertz, Toru\'nczyk, and Vigny~\cite{separatorModelChecking} showed that computational tameness of separator logic (precisely, fixed-parameter tractability of the model-checking problem) coincides on subgraph-closed classes with a natural dividing line in structural graph theory: exclusion of a fixed topological~minor.

The drawback of separator logic is that the connectivity predicates are designed with separators consisting of a bounded number of vertices in mind. Such separators are a fundamental concept in the structural theory of sparse graphs, but become ill-suited for the treatment of dense graphs. In contrast, both \fo and \mso can be naturally deployed on graphs regardless of their sparsity, and much of the recent developments around these logics concern understanding their expressive power and computational aspects on classes of well-structured dense graphs. What would be then a dense analogue of separator logic?

\vspace{-0.1cm}
\paragraph*{Low rank \mso.} We introduce a new logic on graphs called {\em{low rank \mso}} with expressive power lying strictly between \fo and \mso. We argue that low rank \mso is a natural dense analogue of separator logic. The main idea is the following: as separator logic allows quantification over small separators as understood in the theory of sparse graphs, its dense analogue should allow quantification over small separators as understood in the theory of dense graphs. The latter concept is delivered by the notion of~{\em{cutrank}}.

Let $G$ be a graph with vertex set $V$ and $X$ be a subset of $V$; denote $\wh X\coloneqq V\setminus X$ for brevity. The idea is to measure the complexity of the cut $(X,\wh X)$ by examining the adjacency matrix $\Adj_G[X,\wh X]$: the $\{0,1\}$-matrix with rows indexed by $X$ and columns indexed by $\wh X$, where the entry at the intersection of the row of $u\in X$ and the column of $v\in \wh X$ is $1$ if $u$ and $v$ are adjacent, and $0$ otherwise. The {\em{cutrank}} of~$X$, denoted $\rk(X)$, is the rank of $\Adj_G[X,\wh X]$ over the binary field $\F_2$. Cutrank was introduced by Oum and Seymour in their work on {\em{rankwidth}}~\cite{oum2005rank,OumS06}. This graph parameter, together with the functionally equivalent {\em{cliquewidth}}~\cite{courcelle2000linear} and {\em{NLC-width}}~\cite{Wanke94}, is understood as a suitable notion of tree-likeness for dense graphs. In particular, in graph theory there is an ongoing work on constructing a theory of {\em{vertex-minors}}, which is expected to be a dense counterpart of the theory of graph minors. In this theory, rankwidth is the analogue of treewidth, and cuts of low cutrank play the same role as vertex separators of bounded size. See the survey of Kim and Oum~\cite{KimO24} and the PhD thesis of McCarty~\cite{RoseThesis} for an introduction to this area.

With the motivation behind cutrank understood, the definition of low rank \mso becomes natural: it is the fragment of \mso on (undirected) graphs, where every use of the set quantifier must be accompanied by an explicit bound on the cutrank of the quantified set. The syntax uses the following~constructors:
\begin{align*}
\myunderbrace{
    \forall x \quad \exists x
}{quantification \\
over vertices}
\qquad\qquad
\myunderbrace{
    \forall X \colon r \quad \exists X \colon r
}{quantification over subsets of \\
vertices  with cutrank at most $r$}
\qquad\qquad
\myunderbrace{
    \land \quad \lor \quad \neg
}{Boolean \\ combinations}
\qquad\qquad
\myunderbrace{
    E(x,y)
}{edge relation}
\qquad\qquad
\myunderbrace{
    x \in X
}{set membership}
\end{align*}
The semantics are defined in the expected way, as a fragment of \mso with quantification over sets of vertices. (This variant of \mso is often called $\textsc{mso}_1$ in the literature, as opposed to $\textsc{mso}_2$, where quantification over edge sets is also allowed. We do not consider $\textsc{mso}_2$ here.)

The goal of this work is to robustly introduce low rank \mso, study its expressive power, and pose multiple questions about its various aspects. While we are mostly concerned about the setting of undirected graphs\footnote{Concretely, in all our results we consider the setting of vertex-colored undirected graphs, where the vertex set may be additionally equipped with a number of unary predicates.}, low rank \mso can be naturally defined on any structures where a meaningful notion of rank can be considered. We expand on this in \cref{sec:conclusions}.

\paragraph*{Our results.} It is not hard to see that predicates $\conn_k$ can be expressed in low rank \mso, hence low rank \mso is at least as expressive as separator logic (see \cref{lem:easy-comparison}). We first prove that on classes of sparse graphs, the expressive power of the two logics actually coincide. Here, we say that a class of graphs $\Cc$ is {\em{weakly sparse}} if there is $t\in \N$ such that all graphs from $\Cc$ do not contain the biclique $K_{t,t}$ as a subgraph.

\begin{theorem}
    \label{thm:main-separator}
    Let $\Cc$ be a weakly sparse class of graphs. Then for every formula of low rank \mso there exists a formula of separator logic such that the two formulas are equivalent on all graphs from $\Cc$.
\end{theorem}

\cref{thm:main-separator} corroborates the claim that low rank \mso is a dense analogue of separator logic. There is, however, an important difference. Low rank \mso is defined as a fragment of \mso, and not as an extension of \fo, like the separator logic. This has a particular impact on the computational aspects. For instance, observe that for every fixed sentence $\varphi$ of separator logic, whether $\varphi$ holds in a given a graph $G$ can be decided in polynomial time. Indeed, it suffices to apply a brute-force recursive algorithm that, for every next vertex variable quantified, examines all possible evaluations of this variable in the graph; and in the leaves of the recursion, adjacency and connectivity predicates can be checked in polynomial time. This approach does not work for low rank \mso, for the number of sets of bounded rank in a graph can be~exponential.

To bridge this gap, it would be useful to find a logic closer in spirit to \fo that would be equivalent to low rank~\mso. For this, we use the concept of {\em{definable flips}}, which has played an important role in the recent advances in understanding \fo on dense graphs, see~\cite{incremental-lemma,DreierMST23,DreierMT24,flippergame,Torunczyk23}. Formal definitions are given in \cref{sec:logics}, but in a nutshell, a {\em{definable flip}} of a graph $G$ is any graph that can be obtained from $G$ as~follows:
\begin{itemize}[nosep]
 \item Select a tuple $\tup a$ of vertices of $G$. These are the {\em{parameters}} of the flip.
 \item Classify the vertices of $G$ according to their adjacency to the vertices of $\tup a$. The vertices of $\tup a$ belong to their own singleton classes.
 \item For every pair of classes $K$ and $L$, either leave the adjacency relation between $K$ and $L$ intact, or {\em{flip}} it: exchange all edges with non-edges, and vice versa. This can be applied also for $K=L$, which amounts to complementing the adjacency relation within $K$.
\end{itemize}
Note that a single flip applied in the last point can remove a large biclique, so a definable flip of a dense graph may be much sparser. In the aforementioned works~\cite{incremental-lemma,DreierMST23,DreierMT24,flippergame,Torunczyk23}, definable flips have been identified as a useful notion of separations in dense graphs that can be described succinctly, by specifying a constant number of parameters and a constant-size piece of information on which pairs of classes should be~flipped.

This leads to the following candidate for an extension of \fo that could be equivalent to low rank~\mso. We define {\em{flip-connectivity logic}} as the extension of \fo on undirected graphs obtained by additionally allowing the usage of predicates $\flipconn_{k, A}(s, t, a_1, \ldots, a_k)$. Such a predicate verifies whether $s$ and $t$ are in the same connected component of the flip of $G$ defined by the tuple $\tup a=(a_1,\ldots,a_k)$ and the relation $A$ specifying pairs of classes to be flipped. Again, it is not hard to prove that these predicates can be expressed in low rank \mso, implying that low rank \mso is at least as expressive as flip-connectivity logic (see \cref{lem:easy-comparison}). Somewhat surprisingly, we show that the expressive power of low rank \mso is in fact strictly larger on all graphs, but the two logics are equivalent on every class of graphs with bounded Vapnik--Chervonenkis (VC) dimension (see \cref{ssec:vcdim} for the definition).

\begin{restatable}{theorem}{mainFconnNegative}\label{thm:main-fconn-negative}
    There is a sentence of low rank \mso that cannot be expressed in flip-connectivity logic.
\end{restatable}

\vspace{-0.3cm}

\begin{restatable}{theorem}{mainFconnPositive}\label{thm:main-fconn-positive}
    Let $\Cc$ be a graph class of bounded VC dimension. Then for every formula of low rank \mso there exists a formula of flip-connectivity logic such that the two formulas are equivalent on all graphs from~$\Cc$.
\end{restatable}

Finally, we show that flip-connectivity logic can be amended to a stronger logic so that it becomes equivalent to low rank \mso on all graphs. Concretely, in {\em{flip-reachability logic}}, we consider a more general notion of directed flips that work naturally in the setting of directed graphs. The definition is essentially the same as in the undirected setting, except that flipping the arc relation between classes $K$ and $L$ exchanges arcs with non-arcs in the set $K\times L$. Note that thus, directed flips work on {\em{ordered pairs}} of classes: exchanging arcs with non-arcs in $K\times L$ and in $L\times K$ are two different operations, and they can be applied or not independently. Flip-reachability logic is defined by extending \fo by predicates $\flipreach_{k, A}(s, t, a_1, \ldots, a_k)$ that test whether $t$ is {\em{reachable}} from $s$ in the directed flip defined by $\tup a=(a_1,\ldots,a_k)$ and $A$.

While flip-reachability logic naturally works in the domain of directed graphs, we can deploy it also on undirected graphs by replacing every edge with a pair of arcs directed oppositely. Note, however, that even if we start with an undirected graph $G$, a definable directed flip of $G$ is not necessarily undirected. Hence, access to the reachability relation in directed flips of $G$ may provide a larger expressive power than access to the connectivity relation in undirected flips of $G$. We prove that this is indeed the case, and the expressive power of flip-reachability logic meets that of low rank \mso.

\begin{theorem}\label{thm:main-freach}
For every formula of low rank \mso there exists a formula of flip-reachability logic such that the two formulas are equivalent on all undirected graphs.
\end{theorem}

Similarly to flip-connectivity logic, also the flip-reachability predicates are expressible in low rank \mso (see \cref{lem:easy-comparison}), hence the expressive power of the two logics is indeed equal. We also remark that in \cref{thm:main-freach} it is important that we consider low rank \mso and flip-reachability logic in the domain of undirected graphs. We currently do not know whether also on all directed graphs, the expressive power of flip-reachability logic coincides with that of (naturally defined) low rank \mso.

Similarly to separator logic, both flip-connectivity logic and flip-reachability logic are defined as extensions of \fo by predicates whose satisfaction on given arguments can be checked in polynomial time. Hence, for every fixed sentence $\varphi$ of flip-reachability logic, it can be decided in polynomial time whether a given graph $G$ satisfies $\varphi$. By combining this observation with \cref{thm:main-freach} we conclude the following.

\begin{corollary}\label{cor:xp}
 Every graph property definable in low rank \mso can be decided in polynomial time.
\end{corollary}

In the terminology of parameterized complexity, this proves that the model-checking problem for low rank \mso is {\em{slice-wise polynomial}}, that is, belongs\footnote{Formally, \cref{cor:xp} places the problem only in non-uniform $\mathsf{XP}$, because for every sentence $\varphi$ of low rank \mso we obtain a different algorithm for deciding the satisfaction of $\varphi$. However, our proof of \cref{thm:main-freach} can be turned into an effective algorithm for translating $\varphi$ into an equivalent sentence $\varphi'$ of flip-reachability logic, which can be subsequently decided by brute force. This yields a uniform $\mathsf{XP}$ algorithm for model-checking low rank \mso. We omit the details.} to the complexity class $\mathsf{XP}$.

\paragraph*{Organization.} In \cref{sec:logics} we formally introduce all the considered logics and establish some basic properties. In \cref{sec:framework} we propose a simple framework for proving equivalence with low rank \mso that is reused for all the three positive results: \cref{thm:main-separator,thm:main-fconn-positive,thm:main-freach}. Main results comparing low rank \mso with separator logic (\cref{thm:main-separator}), flip-connectivity logic (\cref{thm:main-fconn-negative,thm:main-fconn-positive}), and flip-reachability logic (\cref{thm:main-freach}) are proved in \cref{sec:separator,sec:fconn,,sec:freach}, respectively. In \cref{sec:conclusions} we discuss several open questions and possible directions for future work.
