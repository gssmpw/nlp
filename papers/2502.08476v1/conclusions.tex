\section{Conclusions}
\label{sec:conclusions}

We conclude by outlining several possible directions for future work.

\paragraph*{Beyond graphs.} In this work we considered low rank \mso on undirected graphs. However, the definition can be naturally extended to any kind of structures in which a meaningful notion of the rank of a set can be considered. Take for example binary relational structures. For a partition $(X,\wh X)$ of the universe of a structure $\mathbb A$, we can define the bipartite adjacency matrix $\Adj_{\mathbb A}[X,\wh X]$ by placing at the intersection of the row of $u\in X$ and the column of $v\in X$, the atomic type of the pair $(u,v)$. This way, the adjacency matrix is no longer binary, but as discussed in \cref{sec:logics}, we can still measure its diversity: the number of distinct rows plus the number of distinct columns. Now, in low rank \mso over binary structures we would stipulate the set quantification to range only over sets that induce cuts of bounded diversity. For structures of higher arity, say bounded by $k$, one natural definition would be to index rows of $\Adj_{\mathbb A}[X,\wh X]$ by $(<k)$-tuples of elements of $X$, the columns by $(<k)$-tuples of elements of $Y$, and at the intersection of row $\tup u\in X^{<k}$ and column $\tup v\in \wh X^{<k}$ put the atomic type of $\tup u \tup v$.

To see an example of this definition in practice, consider the setting of finite words over a finite alphabet~$\Sigma$. There are two ways of encoding such words as relational structures: either by equipping the set of positions by the total order, or only by the successor relation. Deploying first-order logic \fo on these two encodings yields two logics with different expressive power. It is customary to consider the ordered one as the right notion of \fo on words, mainly due to classic connections with star-free expressions~\cite{McNaughtonPapert71} and aperiodic monoids~\cite{Schutzenberger65}. In contrast, it is not difficult to see that in both encodings, sets of bounded rank can be characterized as unions of a bounded number of intervals of positions. Consequently, regardless of the encoding, low rank \mso on finite words is equivalent to \fo with access to the total order on positions.

Another case is that of directed graphs, which can be seen as binary structures consisting of the vertex set equipped with a (not necessarily symmetric) arc relation. Recall that we proved that in undirected graphs, properties expressible in low rank \mso can be decided in polynomial time (\cref{cor:xp}), and this was a consequence of the equivalence of low rank \mso and flip-reachability logic (\cref{thm:main-freach}). We do not know whether this equivalence holds in general directed graphs, hence the following question is open.

\begin{question}
 Can every property of directed graphs definable in low rank \mso be decided in polynomial~time?
\end{question}

Finally, another interesting class of structures on which low rank \mso can be deployed are matroids. This setting is slightly tricky, as it is natural to take the connectivity function as the concept of the rank of a set, instead of the standard matroid rank function (i.e., maximum size of an independent subset). That is, if $\cal M$ is a matroid with ground set $E$ and $(X,\wh X)$ is a partition of $E$, then the connectivity function of $X$ is $\rkk_{\cal M}(X)+\rkk_{\cal M}(\wh X)-\rkk_{\cal M}(E)$, where $\rkk_{\cal M}(\cdot)$ is the standard rank function of $\cal M$; this way the connectivity function of $X$ and of $\wh X$ are equal. Given the tight links between the structural theory of matroids and the theory of vertex-minors, where sets of small cutrank play a key role, one might suspect that low rank \mso has interesting properties on matroids as well.

\paragraph*{Monadic dependence.} The recent developments on computational aspects of \fo have highlighted the notion of {\em{monadically dependent}} graph classes as a key structural dividing line. In a nutshell, a class of graphs $\Cc$ is monadically dependent if one cannot interpret the class of all graphs in vertex-colored graphs from $\Cc$ using a fixed \fo interpretation; see~\cite[Section 4.1]{Pilipczuk25} for a formal definition and a discussion. The notion of monadic dependence can be naturally defined also for logics other than \fo. What would be then monadic dependence with respect to low rank \mso? It is not hard to prove using the results of Pilipczuk et al.~\cite{separatorModelChecking} that on subgraph-closed classes, monadic dependence with respect to separator logic coincides with the property of excluding a fixed graph as a topological minor. So by analogy, monadic dependence with respect to low rank \mso should be a dense, logically-motivated analogue of excluding topological minors. It must be, however, different from the property of excluding a fixed vertex-minor, for the latter is known not to imply monadic dependence with respect to \fo~\cite{HlinenyP22}.

We pose the following question as an excuse to explore combinatorial properties of graph classes that are monadically dependent with respect to low rank \mso. It is a low rank \mso counterpart of the analogous question posed for \fo (see e.g.~\cite[Conjecture~2]{Pilipczuk25}), which is currently under intensive investigation.

\begin{question}
 Is the model-checking problem for low rank \mso fixed-parameter tractable on every class of graphs that is monadically dependent with respect to low rank \mso?
\end{question}

\paragraph*{Space complexity.} Note that every graph property definable in \fo can be decided in $\mathsf{L}$ (deterministic logarithmic space): a brute-force model-checking algorithm needs only to remember an evaluation of constantly many variables present in the verified sentence. Due to Reingold's result that undirected reachability is in $\mathsf{L}$~\cite{Reingold08}, the same can be said about separator logic and about flip-connectivity logic; hence also about low rank \mso on any graph class of bounded VC dimension, by \cref{thm:main-fconn-positive}. However, the argument breaks for flip-reachability logic, because testing flip-reachability predicates requires solving directed reachability, which is $\mathsf{NL}$-complete. Consequently, thanks to Immerman--Szelepcs\'enyi
 Theorem~\cite{Immerman88,Szelepcsenyi88} and \cref{thm:main-freach}, we can place the model-checking problem for low rank \mso in slicewise $\mathsf{NL}$, but membership in slicewise $\mathsf{L}$ is unclear. We ask the following.

 \begin{question}
  Can every property of undirected graphs definable in low rank \mso be decided in deterministic logarithmic space?
 \end{question}

