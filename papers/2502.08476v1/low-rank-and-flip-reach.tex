\subsection{Proof of the equivalence}\label{sec:flip-reach-proof}

With the combinatorial characterization of sets of low rank we can prove that flip-reachability logic has the low rank definability property on the class of all graphs (\Cref{def:low-rank-definability}).

\begin{lemma}
    \label{lem:low-rank-definability-freach}
    Flip-reachability logic has low rank definability property on the class of all graphs.
\end{lemma}
\begin{proof}
    We proceed similarly to the proof of~\Cref{lem:lrdp-separator} and~\Cref{lem:property-for-fconn}.
    Let $r\in\mathbb{N}$ be a bound on the rank and $\varphi(X, \wtup Y, \tup z)$ be a formula of flip-reachability logic, as in the definition of low rank definability property. As in the proofs of~\Cref{lem:lrdp-separator} and~\Cref{lem:property-for-fconn}, we may assume that $\wtup Y=\emptyset$ and $\tup z=\emptyset$ (and therefore $\varphi$ can be assumed to have only one free set variable $X$).

    In the remainder of the proof we show that there is a formula $\varphi'(x,\tup t)$ of flip-reachability logic, where $x$ and $\tup t$ are free vertex variables, such that the following holds.
    For every graph $G$ and set $A \subseteq V(G)$ such that $G\models \varphi(A)$, there exists a set $A'\subseteq V(G)$ and an evaluation $\tup d$ of variables $\tup t$ such that
    \[G \models \varphi(A')\qquad \textrm{and}\qquad A' = \setof{v \in V(G)}{G \models \varphi'(v, \tup d)}.\]


    The core of our construction is~\Cref{thm:low-rank-structure}, from which we get the formulas $\varphi_+, \varphi_-,\varphi_\sim$ of flip-reachability logic such that the following holds. For every graph $G$ and set $A\subseteq V(G)$ with $\rk(A)\leq r$,
    there is a tuple $\tup a$ such that $A\in \Span(\Seed(\varphi_+, \varphi_-, \varphi_{\sim}, \tup{a}))$. Note that these formulas depend only on $r$.

    We will also use the \emph{types} of flip-reachability logic.
    Let $\Sigma$ be a finite set of unary predicates including the unary predicates contained in $\varphi$.
    Given an $\Sigma$-colored graph $H$, its \emph{$q$-flip-reachability type} is the set of all flip-reachability sentences of quantifier rank at most $q$ that hold in $H$.
    Similarly as for the flip-connectivity logic, there are only finitely many such types and they can be defined in flip-reachability logic.

    Let $q$ be the quantifier rank of formula $\varphi$.
    \begin{claim}
        Let $G$ be a graph. Assume that there is a set $A \subseteq V(G)$ and a tuple $\tup a$ such that $G\models \varphi(A)$ and $A\in \Span(X_+, X_-, \Xx_{\sim})$, where $(X_+, X_-, \Xx_{\sim}) \coloneqq \Seed(\varphi_+, \varphi_-, \varphi_{\sim}, \tup{a})$. Also, assume that $G$ is marked with unary predicates added for atomic types over $\tup a$. Then there also exists a set $A'\subseteq V(G)$ such that the following holds:
        \begin{itemize}[nosep]
            \item $G\models \varphi(A')$;
            \item $A'\in \Span(X_+, X_-, \Xx_{\sim})$; and
            \item for each $q$-flip-connectivity type $\tau$, $A'$ contains at most $q$ parts from $\Xx_{\sim}$ with $q$-flip-reachability type $\tau$ or $A'$ contains all but at most $q$ parts from $\Xx_{\sim}$ with $q$-flip-reachability type $\tau$.
        \end{itemize}
    \end{claim}
    \begin{claimproof}
        We consider a set $A' \in \Span(X_+, X_-, \Xx_{\sim})$ as follows:
        \begin{itemize}[nosep]
            \item if $A$ contains at most $q$ parts from $\Xx_{\sim}$ with $q$-flip-reachability type $\tau$, then $A'$ contains the same parts from $\Xx_{\sim}$ with $q$-flip-connectivity type $\tau$, and
            \item if $A$ contains all but at most $q$ parts from $\Xx_{\sim}$ with $r$-flip-connectivity type $\tau$, then $A'$ contains the same parts from $\Xx_{\sim}$ with $q$-flip-connectivity type $\tau$, and
            \item in all the other cases $A'$ contains $q$ arbitrary parts from $\Xx_{\sim}$ with $q$-flip-connectivity type $\tau$.
        \end{itemize}


        Now observe that $G \models \varphi(A)$ if and only if $G \models \varphi(A')$.
        Indeed, this is a simple Ehrenfeucht--Fraïssé argument.
        Namely, if Duplicator has a winning strategy for a $q$-round EF game in the structure $G_A \coloneqq (G, A)$, then clearly she can modify her strategy to win in the structure $G_{A'} \coloneqq (G, A')$ by always playing in the part of $\Xx_{\sim}$ in $G_{A'}$ that has the same $q$-flip-reachability type as the part of $\Xx_\sim$ in $G_A$ that she would play in the original game.
        Whenever at the end of the $q$ round game we get two tuples $\tup e$ of vertices played in $G_A$ and $\tup f$ of vertices played in $G_{A'}$ then since the seed $(X_+, X_-, \Xx_{\sim})$ is $\tup a$ uniform and we evaluated $q$-flip-reachability types in $G^+$ that contains predicates for atomic types on $\tup a$, then $\tup e$ and $\tup f$ have the same atomic types with respect to the edge relation and flip-reachability predicates.
    \end{claimproof}
    

    It remains to construct the formula $\phi'(x, \tup t)$ of flip-reachability logic with a tuple of parameters $\tup t$ of bounded size that defines such a set $A'$.
    Indeed, it is enough to use the parameters from $\tup a$ and for every $q$-flip-reachability type $\tau$ we need at most $q$ additional free variables -- one for each part of $\Xx_{\sim}$ with $q$-flip-reachability type $\tau$ which is in $A'$ (or is not in $A'$ in the case when $A'$ contains all but at most $q$ parts of $\Xx_{\sim}$ with $q$-flip-reachability type $\tau$).
    This way, we can define for every $q$-flip-reachability type $\tau$, we can consider a formula $\xi_\tau$ and there is a bounded number of such formulas.
    Therefore, we can write a single formula $\phi'(x, \tup t)$ by extending the tuple $\tup t$ with a finite number of dummy variables; the formula $\psi$ chooses the appropriate formula $\xi$ by testing the equality type of these dummy variables.   
\end{proof}

\Cref{thm:main-freach} follows from \Cref{lem:low-rank-definability-freach} and \Cref{thm:low-rank-quantifier-elimination}.
