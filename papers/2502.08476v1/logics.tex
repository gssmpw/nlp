\section{Definitions and basic properties}\label{sec:logics}

For $k\in \N$, we denote $[k]\coloneqq \{1,\ldots,k\}$.
All graphs are finite, simple (i.e. without loops or parallel edges), and undirected, unless explicitly stated. They are also possibly {\em{vertex-colored}}: the vertex set is equipped with a number of distinguished subsets -- colors.
We treat such graphs as relational structures consisting of the vertex set, the binary symmetric edge relation (denoted $E$), and a number of unary predicates signifying colors. Note that a vertex may belong to several different colors simultaneously, or to no color at~all.

For a set $A$ of vertices of $G$, we denote by $G[A]$ the subgraph of $G$ induced by $A$, i.e. the graph with vertex set $A$, edge set consisting of all edges of $G$ with both endpoints in $A$, and colors inherited naturally.
For a vertex $v$ of $G$ and a set of vertices $A \subseteq V(G)$, we denote by $E(v, S)$ the set of neighbors of $v$ in $S$, i.e. $E(v, S) = \setof{u \in S}{uv \in E(G)}$. The {\em{neighborhood}} of a vertex is $N(v)\coloneqq E(v,V(G))$.

We assume reader's familiarity with the first-order logic \fo and monadic second-order logic \mso. For convenience, in all formulas of all the considered logics, including \fo and its extensions, we allow free set variables. If $X$ is such a set variable and $x$ is a vertex variable, then we allow membership tests of the form~$x\in X$.
By $\tup x$ we denote a tuple of vertex variables and by $\wtup X$ we denote a tuple of set variables.
For a graph $G$ and a tuple of vertices $\tup x$, we denote by $G^{\tup x}$ the set of evaluations of $\tup x$ in $G$, i.e. functions from the variables in $\tup x$ to the vertices of $G$. Similarly, for a tuple of set variables $\wtup X$, we denote by $G^{\wtup X}$ the set of evaluations of $\wtup X$ in $G$.
For brevity we might identify tuples of vertices with respective evaluations.
We say that a logic $\Ll$ is an {\em{extension}} of \fo if $\Ll$ contains \fo as its fragment.

For a formula $\varphi(x, \wtup Y, \tup z)$ of a logic $\Ll$, a graph $G$, and evaluations $\wtup B$ of $\wtup Y$ and $\tup c$ of $\tup z$, we say that $\varphi(x, \wtup B, \tup c)$ {\em{defines}} a set $A \subseteq V(G)$ in $G$ if
\[
    A = \setof{v \in V(G)}{G \models \varphi(v, \wtup B, \tup c)}.
\]
We also denote by $\varphi(G, \wtup B, \tup c)$ the set defined by $\varphi(x, \wtup B, \tup c)$ in $G$.

\paragraph*{Low rank \mso.} We have already defined low rank \mso in \cref{sec:intro}. Let us make here a few simple remarks about the choices made in the definition.

For readers not familiar with the notions of rankwidth and of cutrank, measuring the complexity of a binary matrix by its rank over $\F_2$ may not be the most intuitive choice. Let us explain that the  selection of rank over $\F_2$ is immaterial, as any similar choice would lead to a logic with the same expressive~power.

For a $\{0,1\}$-matrix $M$ and a field $\F$, by $\rkk_\F(M)$ we denote the rank of $M$ over~$\F$. Further, let the {\em{diversity}} of $M$, denoted $\dv(M)$, be the number of different rows of $M$ plus the number of different columns of~$M$. We have the following simple algebraic~fact.

\begin{lemma}\label{lem:equiv-measures}
 Let $M$ be a matrix with entries in $\{0,1\}$ and $\F$ be a finite field. Then
 \[\rkk_\F(M)\leq \rkk_\Q(M)\leq \dv(M)/2\leq |\F|^{\rkk_\F(M)}.\]
\end{lemma}
\begin{proof}
 For the first inequality, every set of columns of $M$ that is dependent over $\Q$ is also dependent over~$\F$. For the second inequality, the rank of a matrix over any field is always bounded by the number different rows, as well as by the number of different columns. For the last inequality, if the rank of $M$ over $\F$ is $k$, then the columns of $M$ are contained in the span of a base of size $k$. This span has at most $|\F|^k$ different vectors, hence $M$ has at most $|\F|^k$ different columns. A symmetric argument shows that $M$ has at most $|\F|^k$ different rows as well.
\end{proof}

\cref{lem:equiv-measures} implies that for a $\{0,1\}$-matrix $M$, whether we measure the diversity of $M$, or its rank over~$\F_2$, or its rank over any other finite field $\F$, or its rank over $\Q$, all these measurements yield values that are bounded by functions of each other. Hence, if $M$ has one of those measures bounded, then all the other measures are bounded as well. Let us also observe that testing the value of any of the considered measures can be defined in \fo.

\begin{lemma}\label{lem:inter-def}
 For every $k\in \N$ and a finite field $\F$, there is an \fo formula $\varphi_{k,\F}(X)$ that for a graph $G$ and $X\subseteq V(G)$, tests whether $\rk_\F(\Adj_G[X,\wh X])\leq k$, where $\wh X=V(G)\setminus X$. Similarly, there are formulas $\varphi_{k,\Q}(X)$ and $\varphi_{k,\dv}(X)$ that test whether $\rk_\Q(\Adj_G[X,\wh X])\leq k$ and $\dv(\Adj_G[X,\wh X])\leq k$,~respectively.
\end{lemma}
\begin{proof}
 We first construct the formula $\varphi_\F(X)$. Call two vertices $u,u'\in X$ {\em{twins}} if $u$ and $u'$ have the same neighborhood in $\wh X$; equivalently, $u$ and $u'$ define equal rows in $M\coloneqq \Adj_G[X,\wh X]$. Similarly, $v,v'\in \wh X$ are twins if they have the same neighborhood in $X$, or equivalently they define equal columns of $M$. Let $C$ be an inclusionwise maximal subset of $X$ consisting of pairwise non-twins, and similarly let $D\subseteq \wh X$ be an inclusionwise maximal subset of $\wh X$ consisting of pairwise non-twins. Note that $|C|+|D|=\dv(M)\leq 2 \cdot |\F|^{\rkk_\F(M)}$ by \cref{lem:equiv-measures}. Therefore, formula $\varphi_\F(X)$ can be constructed by (i) existentially quantifying $C$ and $D$ as sets of total size at most $|\F|^k$, (ii) checking that $C$ and $D$ are inclusionwise maximal subsets of $X$ and of $\wh X$, respectively, consisting of pairwise non-twins, and (iii) verifying that the rank over $\F$ of the minor of $M$ induced by the rows of $C$ and the columns of $D$ is at most $k$, by making a disjunction over all possible adjacency relations between the vertices of $C$ and of $D$. Formula $\varphi_{k,\Q}(X)$ can be defined in the same way (here we have $|C|+|D|\leq 2\cdot 2^k$ by \cref{lem:equiv-measures} for $\F=\F_2$), while in formula $\varphi_{k,\dv}(X)$ we only need to make sure that $|C|+|D|\leq k$.
\end{proof}

From \cref{lem:equiv-measures,lem:inter-def} we conclude that in the definition of low rank \mso, regardless whether in set quantification we require providing an explicit bound on the rank of the bipartite adjacency matrix over~$\F_2$, or over any other fixed finite field $\F$, or over $\Q$, or even on the diversity of the adjacency matrix, all these logics will have the same expressive power. This is because if we have two measures $\mu_1,\mu_2$ among the above, to quantify over $X$ with $\mu_2(\Adj_G[X,\wh X])\leq k$, it suffices to quantify over $X$ with $\mu_1(\Adj_G[X,\wh X])\leq f(k)$, where $f\colon \N\to \N$ is such that $\mu_1(M)\leq f(\mu_2(M))$ for every $\{0,1\}$-matrix $M$, and verify that indeed $\mu_2(\Adj_G[X,\wh X])\leq k$ using a formula provided by \cref{lem:inter-def}. Therefore, following the literature on rankwidth we make the arbitrary choice of defining low rank \mso using the cutrank function that relies on ranks over $\F_2$. In the remainder of this paper, we denote $\rkk\coloneqq \rkk_{\F_2}$ for brevity. Also, the cutrank of a set will be called just {\em{rank}}.

\paragraph*{Separator logic, flip-connectivity logic, and flip-reachability logic.}
Separator logic has also been introduced in \cref{sec:intro}. Recall that it is defined as the extension of \fo on graphs by predicates $\conn_k$ for $k\in \N$, each of arity $k+2$, with the following semantics: if $G$ is a graph and $s,t,a_1,\ldots,a_k$ are vertices of $G$, then $\conn_k(s,t,a_1,\ldots,a_k)$ holds in $G$ if and only if there is a path with endpoints $s$ and $t$ that does not pass through any of the vertices $a_1,\ldots,a_k$.

We now define flip-connectivity logic and flip-reachability logic. Flip-reachability logic works naturally on directed graphs ({\em{digraphs}}) and flip-connectivity logic will be a special case of the definition, hence we need to introduce some terminology on digraphs.

A~directed graph (digraph) is a~pair $G=(V, E)$ consisting of a~set $V=V(G)$ of vertices and a~set $E=E(G)$ of \emph{arcs}.
An~arc from $u$ to $v$ is denoted $\vec{uv}$ and has tail $u$ and head $v$.
We specify that digraphs do not contain self-loops, so $u \neq v$ for each arc $\vec{uv}$, and neither do they contain multiple copies of an~arc.
However, we allow parallel arcs connecting two vertices in the opposite directions.

The \emph{atomic type} of a~$k$-tuple $\tup{v} = (v_1, \ldots, v_k)$ of vertices of an~undirected graph $G$, denoted $\atp(\tup{v}) = \atp(v_1, \ldots, v_k)$, is the set of all atomic formulas of the form $x_i = x_j$, $E(x_i, x_j)$, and $U(x_i)$ for some $i,j\in \{1,\ldots,k\}$ and a unary predicate $U$ in the language of $G$, satisfied by $\tup{v}$ in $G$.
Naturally, two $p$-tuples $\tup{u}$, $\tup{v}$ of vertices satisfy the same quantifier-free formulas of first-order logic with no parameters if and only if $\atp(\tup{u}) = \atp(\tup{v})$.
We define \emph{edge type} of a~$k$-tuple $\tup v$ similarly to the atomic type, but we do not consider the unary predicates.

Let $\atp^k$ denote the set of all possible atomic types of $k$-tuples of vertices of an~undirected graph. Note that $\atp^k$ is finite and of size bounded by a function of $k$ and the number of unary predicates interpreted in $G$. Atomic types could be also naturally defined for directed graphs, but we will use them only in the undirected context.

Next, we define flips of (undirected) graphs with respect to a~tuple of parameters.
Let $k \in \N$, $A \subseteq \atp^{k+1} \times \atp^{k+1}$, $G$ be an~undirected graph and $\tup{a} = (a_1, \ldots, a_k)$ be a~$k$-tuple of vertices of $G$ -- the \emph{parameters} of the flip.
Then the \emph{$A$-flip of $G$ with parameters $\tup{a}$}, denoted $G \oplus_{\tup{a}} A$, is the directed graph with the same vertex set as~$G$, where for distinct $u, v \in V(G)$, we have
\[
    \vec{uv} \in E(G \oplus_{\tup{a}} A)\qquad\textrm{if and only if}\qquad[uv \in E(G)]\ \textrm{xor}\ [(\atp(u, \tup{a}), \atp(v, \tup{a})) \in A].
\]
Note that if the relation $A$ is symmetric, then the resulting graph is always undirected, in the sense that every arc $\vec{uv}$ is accompanied with the opposite arc $\vec{vu}$.
In this case, we will say that the $A$-flip is \emph{symmetric} and consider its result to be an undirected graph.

This definition now allows us to formally introduce the flip-reachability logic and its weaker flip-connectivity counterpart.

\begin{definition}[Flip-reachability logic]
    For $k \in \N$ and $A \subseteq \atp^{k+1} \times \atp^{k+1}$, we introduce the flip-reachability predicate
    \[
        \flipreach_{k, A}(s, t, a_1, \ldots, a_k)
    \]
    as the relation on vertices in a~graph that holds precisely when there exists a~directed path from $s$ to $t$ in $G \oplus_{\tup{a}} A$.
    Flip-reachability logic is first-order logic over graphs in which the universe is formed by the vertices of a~graph, and the available relations are the binary edge relation, the unary predicates in the language, as well as the flip-reachability predicates.
\end{definition}

\begin{definition}[Flip-connectivity logic]
    For $k \in \N$ and symmetric $A \subseteq \atp^{k+1} \times \atp^{k+1}$, we introduce the flip-connectivity predicate
    \[
        \flipconn_{k, A}(s, t, a_1, \ldots, a_k)
    \]
    as the relation on vertices in a~graph that holds precisely when $s$ and $t$ are in the same connected component of $G \oplus_{\tup{a}} A$.
    Flip-connectivity logic is first-order logic over graphs in which the universe is formed by the vertices of a~graph, and the available relations are the binary edge relation, the unary predicates in the language, as well as the flip-connectivity predicates.
\end{definition}

\paragraph*{Easy comparisons.} Let us first establish the straightforward relations between the considered logics in terms of expressive power. We first note the following.

\begin{lemma}\label{lem:freach-in-lrmso}
    Every flip-reachability predicate can be expressed in low rank \mso.
\end{lemma}
\begin{proof}
    Fix $k\in \N$, $A \subseteq \atp^{k+1} \times \atp^{k+1}$, a graph $G$, and vertices $s,t,a_1,\ldots,a_k$; denote $\tup a=(a_1,\ldots,a_k)$.
    Observe that the predicate $\flipreach_{k, A}(s, t, a_1, \ldots, a_k)$ is false if and only if there is a set $X$ of vertices of $G$ such that $s\in X$, $t\notin X$, and there are no arcs from $X$ to $\wh X$ in $G \oplus_{\tup{a}} A$, where we denote $\wh X=V(G)\setminus X$.
    Note that these conditions, for a given set $X$, can be encoded by a first-order formula taking $X,s,t,a_1,\ldots,a_k$ as free variables.
    So it remains to show that quantification over such sets $X$ can be done using a low-rank quantifier, that is, that $X$ has rank bounded by a function of~$k$.

    For this, we argue that the set $X$ of vertices of $G$ that are reachable from $s$ in $G \oplus_{\tup{a}} A$ has rank at most $|\atp^{k+1}|$.
    To see this, consider two vertices $u,u'\in X$ such that $\atp(u, \tup{a}) = \atp(u', \tup{a})$ (in $G$), as well as a vertex $v\in \wh X$. Note that $v$ is adjacent in $G$ to either both $u$ and $u'$ or to none of them, for otherwise either $\vec{uv}$ or $\vec{u'v}$ would appear as an arc in $G \oplus_{\tup{a}} A$, which would contradict the assumption that $v$ is not reachable from $s$.
    Hence, whether there is an edge in $G$ between a vertex $u$ in $X$ and a vertex $v$ in $\wh X$ depends only on the atomic type $\atp(u,\tup{a})$. It follows that the adjacency matrix $\Adj_G[X, \wh X]$ has at most $|\atp^{k+1}|$ distinct rows, hence its rank is at most $|\atp^{k+1}|$.
\end{proof}

We conclude the following.

\begin{proposition}\label{lem:easy-comparison}
 The following holds:
 \begin{itemize}[nosep]
  \item For every formula of separator logic there is an equivalent formula of flip-connectivity logic.
  \item For every formula of flip-connectivity logic there is an equivalent formula of flip-reachability logic.
  \item For every formula of flip-reachability logic there is an equivalent formula of low rank \mso.
 \end{itemize}
\end{proposition}
\begin{proof}
 For the first point, it suffices to observe that there is a~symmetric $A\subseteq \atp^{k+1} \times \atp^{k+1}$ such that for every graph $G$ and $\tup a\in V(G)^k$, the flip $G\oplus_{\tup a} A$ is equal to $G$ with all edges incident to the vertices of $\tup a$ removed; then the predicate $\conn_k(s,t,a_1,\ldots,a_k)$ is equivalent to $\flipconn_{k,A}(s,t,a_1,\ldots,a_k)$. The second point is obvious: flip-connectivity predicates are special cases of flip-reachability predicates obtained by restricting attention to symmetric relations $A$. The third point follows immediately from \cref{lem:freach-in-lrmso}.
\end{proof}

Finally, we note that flip-connectivity logic has a strictly larger expressive power than separator logic. A distinguishing property is {\em{co-connectivity}}: connectivity of the complement of the graph.

\begin{proposition}
 There is a sentence of flip-connectivity logic that verifies whether a graph is co-connected. However, there is no such sentence in separator logic.
\end{proposition}
\begin{proof}
 For the sentence of flip-connectivity logic verifying co-connectivity, we can take \[\forall s\,\forall t\, \flipconn_{0,A}(s,t),\qquad\textrm{where }A=\atp^1\times \atp^1.\]
 We are left with proving that the property cannot be expressed in separator logic.

 For the sake of contradiction, suppose there is a sentence $\varphi$ of separator logic that holds exactly in graphs whose complements are connected. Since $\varphi$ is finite, there is some number $k\in \N$ such that all connectivity predicates present in $\varphi$ are of arity at most $k+2$. Now, for every integer $n>k+2$, consider the following two graphs: $G_n$ is the complement of a cycle on $n$ vertices, and $H_n$ is the complement of the disjoint union of two cycles on $n$ vertices. Note that $G_n$ is co-connected while $H_n$ is not, hence $\varphi$ holds in $G_n$ and does not hold in $H_n$. Further, it can be easily verified that both $G_n$ and $H_n$ are $k$-connected, that is, no pair of vertices can be disconnected by a $k$-tuple of other vertices. Hence, every predicate $\conn_\ell(s,t,a_1,\ldots,a_\ell)$, for $\ell\leq k$, is equivalent on $\{G_n,H_n\colon n>k+2\}$ to the first-order formula
 \[\bigwedge_{i\in [\ell]} (s\neq a_i \wedge t\neq a_i).\]
 By replacing all the connectivity predicates with the formulas above, we turn $\varphi$ into a first-order sentence $\varphi'$ such that for all $n>k+2$, $\varphi'$ holds in $G_n$ and does not hold in $H_n$. However, a standard argument based on Ehrenfeucht--Fra\"isse games for first-order logic shows that such a sentence $\varphi'$ distinguishing $G_n$ from $H_n$ does not exist.
\end{proof}
