\section{Relation to flip-connectivity logic}\label{sec:fconn}

In this section we compare the expressive power of low rank \mso with flip-connectivity logic.

\subsection{Negative result}
As our first result we show that on the class of all graphs low rank \mso is strictly more expressive than flip-connectivity logic. That is, we prove \Cref{thm:main-fconn-negative}, which we restate here for convenience.

\mainFconnNegative*
\begin{proof}
    In the proof we construct a family of graphs $G_n$ and $G_n'$ that can be distinguished by a sentence of low rank \mso, but for every sentence $\psi$ of flip-connectivity logic there is some $n$ such that $G_n$ and $G_n'$ are indistinguishable by $\psi$.

    We begin with an auxiliary graph, which we call $F_n$. In this graph, there are $n+1$ layers of vertices. There are $10n^2$ vertices in layer $0$. The vertices in layer $i \in \set{1,\ldots,n}$ are all subsets of vertices in layer $i-1$ that have size $5n$, and the edge relation between layers $i$ and $i-1$ represents membership. Thus, layer $1$ has $\binom{10n^2}{5n}$ vertices, layer $2$ has $\binom{\binom{10n^2}{5n}}{5n}$ vertices, and so on. Here is a picture:
    \mypic{4}

    
    Using the auxiliary graph $F_n$, we define the two colored graphs $G_n$ and $G'_n$. Actually, the underlying uncolored graphs of $G_n$ and $G'_n$ will be exactly the same; the difference between them is only in the placement of colors (unary predicates). The uncolored graph underlying $G_n$ and $G_n'$ is constructed as~follows:
        \begin{itemize}[nosep]
            \item take four disjoint copies of $H_n$;
            \item merge the first two copies of layer 0;
            \item merge the last two copies of layer 0;
            \item create a biclique between the merged copies from the previous two items.
        \end{itemize}
    Here is a picture of the graph: 
    \mypic{3}
    This completes the description of the uncolored graph underlying  $G_n$ and $G'_n$. To differentiate them, we add four unary predicates that mark vertices of $n$'th layer in each copy of $F_n$. These, call them $A, B, C, D$, are distributed as follows:
    \mypic{5}

    To simplify the notation in our proof, we denote sets of vertices marked with predicates $A, B, C,$ and $D$ by $V_n^A, V_n^B, V_n^C,$ and $V_n^D$, respectively.
    Also, the set of vertices in the $j$'th layer of the copy of $F_n$ (for $j \in [n]$) whose last layer is marked with a predicate $L \in \set{A, B, C, D}$ is denoted by $V_j^L$.
    Finally, in $G_n$ we call the set of vertices adjacent to $V_1^A$ and $V_1^B$ by $U_1$ and the set of vertices adjacent to $V_1^C$ and $V_1^D$ by $U_2$.
    Similarly, in $G_n'$ we call the set of vertices adjacent to $V_1^A$ and $V_1^C$ by $U_1$ and the set of vertices adjacent to $V_1^B$ and $V_1^D$ by $U_2$.
    This way we can assume that $G_n$ and $G_n'$ have the same vertex set (so we interpret $V_j^L$ as the same set of vertices in both $G_n$ and $G_n'$), but the edge relation is different.
    Also, we call each set $V_j^L$ or $U_1, U_2$ a \emph{block}.
    For convenience, we will say that a~block $Q$ is an~\emph{expansion} of a~block $P$ if for every subset $P' \subseteq P$ of cardinality $5n$ there exists exactly one vertex $v \in Q$ whose neighborhood in $P$ is precisely $P'$.
    We will call this vertex a~\emph{witness} of $P'$ (in $P$).
    

    We start with the following claim.
    \begin{claim}
        There exists a sentence $\varphi$ of low rank \mso such that $G_n \models \varphi$ and $G_n' \not \models \varphi$ for every $n \ge 3$.
    \end{claim}
    \begin{claimproof}
        Let $\varphi$ be a sentence of low rank \mso that stipulates:
        \begin{center}
            \textit{There is a set of rank $1$ that contains all the vertices marked with predicate $A$ but no vertices marked with predicate $C$.}
        \end{center}
        This sentence is clearly true in $G_n$: the sought set is $U_1 \cup \left(\bigcup_{i \in [n]} V_i^A\right) \cup \left(\bigcup_{i \in [n]} V_i^B\right)$.
        Now, we will show that $\varphi$ does not hold in $G_n'$.
        By contradiction, assume there is a set $X \subseteq V(G_n')$ of rank $1$ that contains all the vertices marked with predicate $A$ but no vertex marked with predicate $C$.
        Consider blocks $V_n^C, V_{n-1}^C, \ldots, V_1^C, U_1, V_1^A, V_2^A, \ldots, V_n^A$ ordered in this way.
        Since $V_n^C \cap X = \emptyset$ and $V_n^A \subseteq X$, there is the~earliest block $P$ in this ordering which contains at least two vertices of $X$.
        Take any two distinct vertices $u, v \in P \cap X$.
        If $P$ is $V_i^C$ for some $i < n$ then there is at most one vertex in $V_{i+1}^C \cap X$.
        However, for each subset $T$ of vertices in $V_i^C \setminus \set{u, v}$ of size $5n - 1$, there is a vertex in $V_{i+1}^C$ that is connected in $V_{i}^C$ precisely to $T \cup \set{u}$ (i.e., a~witness of $T \cup \set{u}$).
        Since we have more than $2$ such sets $T$, we get a vertex in the complement of $X$ that is connected to $u$ but not to $v$.
        Similarly, we have a vertex in the complement of $X$ that is connected to $v$ but not to $u$, so the rank of $X$ is at least $2$.
        The case of $P$ being $U_1$ is similar.

        Now consider the case when $P$ is $V_1^A$.
        Since both $u$ and $v$ have different neighborhoods in $U_1$ each of size $5n$, there are two vertices $w$ and $w'$ in $U_1$ such that $E(u, w)$, $\neg E(v, w)$, $E(v, w')$, and $\neg E(u, w')$.
        If neither $w$ nor $w'$ is in $X$ then the rank of $X$ is at least $2$.
        Therefore, assume that one of $w, w'$ (without loss of generality $w$) is in $X$.
        Then, since $w$ has more than one neighbor in $V_1^C$, it has a neighbor $z \in V_1^C\setminus X$.
        Then we have $E(w, z)$, $\neg E(v, z)$, $E(v, w')$, and $\neg E(w, w')$, so the rank of $X$ is at least $2$.
        The case of $P$ being $V_i^A$ for some $i > 1$ is analogous.
        This shows that $G_n' \not \models \varphi$.
    \end{claimproof}

    Now assume by contradiction that $\varphi$ is equivalent to a sentence $\psi$ of flip-connectivity logic.
    Assume that $\psi$ has quantifier rank less than $q$ and uses flip-connectivity predicates with at most $q$ arguments.
    We will show that $\psi$ does not distinguish $G_n$ and $G_n'$ for $n$ large enough; in particular we require $n \ge q$.

    Our first goal is to show that in $G_n$, flip-connectivity predicates can be expressed in first-order logic (and similarly in $G_n'$).
    Consider any set $S$ of vertices of $G_n$ with $|S| \leq q$ and a symmetric flip $H_n$ of $G_n$ with parameters $S$.
    We will show that all the vertices in $V(G_n) \setminus (V_n^A \cup V_n^B \cup V_n^C \cup V_n^D \cup S)$ are in the same connected component of $H_n$.
    For this we need to show a number of claims.

    \begin{claim}
        \label{cl:limited-adjacency}
        Suppose $P, Q$ are two blocks such that either $Q$ is an~expansion of $P$ or $\{P, Q\} = \{U_1, U_2\}$.
        Then every vertex $v \notin P \cup Q$ is adjacent to at most $5n$ vertices of $Q$.
        In particular, there are at most $5nq \leq 5n^2$ vertices of $Q$ that either belong to $S$ or are adjacent to $S \setminus P$.
    \end{claim}
    \begin{claimproof}
        Straightforward case analysis.
    \end{claimproof}

    \begin{claim}
        \label{cl:expansion-adjacency}
        Let a~block $Q$ be an~expansion of a~block $P$ in $G_n$.
        Then for every two vertices $u, v \in P \setminus S$ there is a~vertex $w \in Q \setminus S$ such that $w$ is adjacent to both $u$ and $v$ in $H_n$.
    \end{claim}
    \begin{claimproof}
        We will show that there are vertices $w_1, w_2, w_3, w_4 \in Q \setminus S$ that do not neighbor any vertex of $S$ in $G_n$ and exhibit all four possible neighborhoods on $\set{u, v}$.
        Therefore $w_1, w_2, w_3,$ and $w_4$ have the same atomic type on $S$ and no matter how we flip their neighborhood class with the classes of $u$ and $v$, one of them is adjacent to both $u$ and $v$ in $H_n$.

        Since all the cases are similar, we will show that there is a vertex $w_1 \in Q \setminus S$ not neighboring any vertex of $S$ in $G_n$ that is adjacent to $u$ but not to $v$.
        Observe that there are at least $\binom{|P| - 2 - q}{5n - 1} > 5n^2$ vertices of $Q$ adjacent to $u$ and non-adjacent to $\set{v} \cup (P \cap S)$: this is because for every $T \subseteq P \setminus (\{u, v\} \cup S)$ of size $5n - 1$, $Q$ contains a~witness of $T \cup \set{u}$.
        Out of these, at most $5n^2$ vertices belong to $S$ or are adjacent to $S \setminus P$ by \Cref{cl:limited-adjacency}.
        Hence there exists a~vertex $w_1 \in Q \setminus S$ that is adjacent to $u$ and non-adjacent to $\set{v} \cup (P \cap S) \cup (S \setminus P) = \set{v} \cup S$. The other cases can be argued similarly.
    \end{claimproof}

    \begin{claim}
        \label{cl:center-adjacency}
        In $H_n$ there is either an edge between $U_1 \setminus S$ and $U_2 \setminus S$ or there is an edge between $V_1^A \setminus S$ and $V_1^D \setminus S$.
    \end{claim}
    \begin{claimproof}
        Assume that in $H_n$ there is no edge between $U_1 \setminus S$ and $U_2 \setminus S$.
        Denote $T_1 = U_1 \cap S$ and $T_2 = U_2 \cap S$.
        Consider the set $P_2$ of vertices outside of $S$ neighboring (in $G_n$) all the vertices in $T_1$ and no other vertex of $S$.
        By \Cref{cl:limited-adjacency} there are at most $5n^2$ vertices in $U_2$ that neighbor a vertex in $S \setminus T_1$, so $P_2 \cap U_2$ is non-empty.
        In the same way we show that $P_2 \cap V_1^A$ is non-empty.
        Similarly, for the set $P_1$ of vertices outside of $S$ neighboring all the vertices in $T_2$ and no other vertex of $S$, both $P_1 \cap U_1$ and $P_1 \cap V_1^D$ are non-empty.
        Note that $P_1$ and $P_2$ are two (possibly equal) adjacency classes of vertices in $G_n$.
        Since there are all possible edges between $U_1$ and $U_2$ in $G_n$, but no such edges in $H_n$, we get that during the construction of $H_n$ we flipped the adjacency relation between $P_1$ and $P_2$.
        Hence in $H_n$ there is an edge between $V_1^A \setminus S$ and~$V_1^D \setminus S$.
    \end{claimproof}

    Combining all the claims above we get the following.

    \begin{claim}
        \label{cl:one-large-component}
        All vertices of $V(H_n) \setminus \left(S \cup V_n^A \cup V_n^B \cup V_n^C \cup V_n^D\right)$ are in the same connected component of~$H_n$.
    \end{claim}
    \begin{claimproof}
        For every $i \in [n - 1]$ and $L \in \set{A, B, C, D}$, the vertices of $V_i^L \setminus S$ are in the same connected component of $H_n$: by \Cref{cl:expansion-adjacency}, each pair of vertices in $V_i^L \setminus S$ shares a~neighbor in $V_{i+1}^L \setminus S$.
        The same conclusion is true also for $U_1 \setminus S$ and $U_2 \setminus S$.
        By the same claim, there exists at least one edge between $V_i^L \setminus S$ and $V_{i+1}^L \setminus S$; at least one edge between $U_1 \setminus S$ and each of $V_1^A \setminus S$ and $V_1^B \setminus S$; and at least one edge between $U_2 \setminus S$ and each of $V_1^C \setminus S$ and $V_1^D \setminus S$.
        Finally, by \Cref{cl:center-adjacency} there is also either an edge between $U_1 \setminus S$ and $U_2 \setminus S$ or an edge between $V_1^A \setminus S$ and $V_1^D \setminus S$.
    \end{claimproof}

    Note that all the claims above also work in the setting where we consider a~symmetric flip $H'_n$ of $G'_n$ with parameters $S$. This is because the uncolored graph underlying $G_n$ and $G_n'$ is the same.

    \begin{claim}
        \label{cl:flipconn-to-fo}
        For every predicate $\flipconn_{q, \Pi}(s, t, a_1, \ldots, a_q)$ there is an \fo formula $\xi_{q, \Pi}(s, t, a_1, \ldots, a_q)$ such that for any $n$ large enough, for any vertices $s, t, a_1, \ldots, a_q$ in $G_n$ we have
        \[
            G_n \models \flipconn_{q, \Pi}(s, t, a_1, \ldots, a_q)\qquad \textrm{if and only if}\qquad G_n \models \xi_{q, \Pi}(s, t, a_1, \ldots, a_q);
        \]
        and also for any vertices $s', t', a_1', \ldots, a_q'$ in $G_n'$, we have
        \[
            G_n' \models \flipconn_{q, \Pi}(s', t', a_1', \ldots, a_q')\qquad \textrm{if and only if}\qquad  G_n' \models \xi_{q, \Pi}(s', t', a_1', \ldots, a_q').
        \]
    \end{claim}
    \begin{claimproof}
        Let $S = \{a_1, \ldots, a_q\}$ and $H_n$ be the $\Pi$-flip of $G_n$ with parameters $a_1, \ldots, a_q$.
        As in \Cref{cl:one-large-component}, all vertices of $X \coloneqq V(H_n) \setminus \left(S \cup V_n^A \cup V_n^B \cup V_n^C \cup V_n^D\right)$ belong to a~single connected component of $H_n$.
        Next, vertices of each $V_n^L \setminus S$ for $L \in \set{A, B, C, D}$ can be partitioned into at most $2^q$ parts according to the atomic types on $\set{a_1, \ldots, a_q}$; hence $(V_n^A \cup V_n^B \cup V_n^C \cup V_n^D) \setminus S$ is partitioned into $4 \cdot 2^q$ parts in total.
        All these parts are homogeneous (i.e. fully adjacent or fully non-adjacent) with respect to the adjacency in~$H_n$.
        It follows that if there exists a~path in $H_n$ between two vertices of $(V_n^A \cup V_n^B \cup V_n^C \cup V_n^D) \setminus S$ whose all vertices are in $(V_n^A \cup V_n^B \cup V_n^C \cup V_n^D) \setminus S$, then the shortest such path has length at most $4 \cdot 2^q$.

        Note now that the diameter of every connected component $Y$ of $H_n$ that is disjoint from $X$ is bounded from above by $(q + 1)(4 \cdot 2^q + 1)$. Indeed, suppose $u, v \in Y$ are connected in $H_n$ and consider a~shortest path $P$ between $u$ and $v$.
        Then every subpath of $P$ of length $4 \cdot 2^q + 1$ must contain a~vertex outside of $V_n^A \cup V_n^B \cup V_n^C \cup V_n^D \cup X$; but such a~vertex must belong to $S$ and so there are at most $q$ such vertices.
        
        Now, we can easily see that $G_n \not\models \flipconn_{q, \Pi}(s, t, a_1, \ldots, a_q)$ if and only if $s$ and $t$ reside in different connected components of $H_n$, at least one of which has diameter at most $(q + 1)(4 \cdot 2^q + 1)$; this property can be easily tested by an \fo formula, whose negation can be taken as $\xi_{q, \Pi}$.
        Also, it can be verified that the same formula $\neg \xi_{q, \Pi}$ checks if $s'$ and $t'$ reside in different connected components of $H'_n$, where $H'_n$ is the $\Pi$-flip of $G'_n$ with parameters $a_1', \ldots, a_q'$.
        Applying the same argumentation as in $H_n$, we conclude that $G_n' \models \flipconn_{q, \Pi}(s', t', a_1', \ldots, a_q')$ if and only if $G_n' \models \xi_{q, \Pi}(s', t', a_1', \ldots, a_q')$.
    \end{claimproof}

    To finish the argument, observe that if $G_n$ and $G_n'$ were distinguishable by a sentence of flip-connectivity logic, then by \Cref{cl:flipconn-to-fo} they would be distinguishable by a sentence of first-order logic.
    However, by a standard argument using Ehrenfeucht--Fraïssé games we know that $G_n$ and $G_n'$ are indistinguishable by \fo sentences of quantifier rank $o(\log n)$.
    This finishes the example and shows that on all graphs, low rank \mso is strictly more expressive than flip-connectivity logic.
\end{proof}

\subsection{VC dimension and related notions} \label{ssec:vcdim}
Before we continue to our positive result for flip-connectivity logic, i.e. \cref{thm:main-fconn-positive} that states that for every class of bounded Vapnik--Chervonenkis (VC) dimension, low rank \mso and flip-connectivity logic are equivalent, we start with a number of definitions that explain this notion.

A {\em{set system}} over a universe $U$ is just a family $\Ff$ of subsets of $U$. For a subset of the universe $X\subseteq U$, we say that $X$ is {\em{shattered}} by $\Ff$ if for every $Y\subseteq X$ there exists $F\in \Ff$ such that $F\cap X=Y$. The {\em{VC dimension}} of $\Ff$ is the largest cardinality of a set shattered by $\Ff$. The VC dimension of a graph $G$ is the VC dimension of the set system of neighborhoods $\{N(v)\colon v\in V(G)\}$; and a graph class $\Cc$ has {\em{bounded VC dimension}} if there is $d\in \N$ such that every member of $\Cc$ has VC dimension at most $d$.

Next, we will need the following definition of duality of a binary relation, and its connection to the notion of VC dimension. The following definitions and results are taken from~\cite{incremental-lemma}.

\begin{definition}
    Let $E \subseteq A \times B$ be a binary relation.
    We say that $E$ has a {\em{duality}} of order $k$ if at least one of two cases holds:
    \begin{itemize}[nosep]
        \item there exists $A_0 \subseteq A$ of size at most $k$ such that for every $b \in B$ there is some $a \in A_0$ with $\neg E(a, b)$, or
        \item there exists $B_0 \subseteq B$ of size at most $k$ such that for every $a \in A$ there is some $b \in B_0$ with $E(a, b)$.
    \end{itemize}
\end{definition}

\begin{theorem}[see {\cite{incremental-lemma}}]
    \label{thm:vc-dim-duality}
    For every $d \in \N$ there is some $k \in \N$ such that the following holds.
    Let $E \subseteq A \times B$ be a binary relation with both $A$ and $B$ finite such that the set system $\{\setof{b \in B}{E(a, b)}\colon a \in A\}$ has VC dimension at most $d$.
    Then $E$ has a duality of order $k$.
\end{theorem}

In the proof of equivalence of low rank \mso and flip-connectivity logic, we will use the following result proven by Bonnet, Dreier, Gajarsk\'y, Kreutzer, M\"ahlmann, Simon, and Toru\'nczyk~\cite{incremental-lemma}.
Here, a {\em{pseudometric}} is a symmetric function $\delta\colon V \times V \to \R_{\geq 0} \cup \set{+\infty}$ satisfying the triangle inequality; and for a partition $\Pp$ of a set $V$, by $\Pp(v)$ we denote the unique part of $\Pp$ containing $v$.

\begin{theorem}
    [{\cite[Theorem~3.5]{incremental-lemma}}]
    \label{thm:incremental-lemma}
    Fix $r, k, t \in \N$.
    Let $V$ be a finite set equipped with:
    \begin{itemize}[nosep]
        \item a binary relation $E \subseteq V \times V$ such that for all $A \subseteq V$ and $B \subseteq V$, $E \cap (A \times B)$ has a duality of order~$k$,
        \item a pseudometric $\dist\colon V \times V \to \R_{\geq 0} \cup \set{+\infty}$, and
        \item a partition $\Pp$ of $V$ with $|\Pp| \leq t$.
    \end{itemize}
    Suppose further that $E(u, v)$ depends only on $\Pp(u)$ and $\Pp(v)$, for all $u, v$ with $\dist(u, v) > r$. (That is, whenever $\dist(u,v)>r$, $\dist(u',v')>r$, $\Pp(u)=\Pp(u')$, and $\Pp(v)=\Pp(v')$, we have $E(u,v)\iff E(u',v')$.)
    Then there is a set $S \subseteq V$ of size $\Oh(kt^2)$ such that $E(u, v)$ depends only on $E(u, S)$ and $E(S, v)$ for all $u, v \in V$ with $\dist(u, v) > 5r$.
\end{theorem}

\subsection{Positive result}

Now we proceed to the proof of \Cref{thm:main-fconn-positive}.
We restate it here for convenience.
\mainFconnPositive*

We start with the following combinatorial lemma that is the analogue of \cref{lem:characterization-separator}. Intuitively, it characterizes any set of low rank in a graph $G$ of bounded VC dimension as the union of a collection of connected components in a graph obtained from $G$ by a small flip.
\begin{lemma}
    \label{lem:low-rank-sets-in-flipconn}
    Fix a constant $r \in \N$ and a class  $\Cc$ of graphs of bounded VC dimension.
    Then there exists a constant $\ell \in \N$ such that for every graph $G \in \Cc$ and every set $A \subseteq V(G)$ of rank at most $r$, there is a set $S \subseteq V(G)$ of size at most $\ell$ and a symmetric flip $G'$ of $G$ with parameters $S$ such that $A$ is the union of the vertex sets of a collection of connected components of $G'$.
\end{lemma}

\begin{proof}
    Since $\Cc$ has bounded VC dimension, by \cref{thm:vc-dim-duality} there is some $k \in \N$ such that the edge relation of every graph $G \in \Cc$ has a duality of order $k$.
    Take a graph $G \in \Cc$ and a subset $A \subseteq V(G)$ of rank at most $r$.
    Partition $A$ according to the edge types on the complement of $A$, i.e. two vertices $u, v \in A$ are in the same part if for every $w \in V(G) \setminus A$ we have $E(u, w) \iff E(v, w)$.
    Since $A$ has rank at most $r$, the number of parts is at most $2^r$.
    Similarly, partition the complement of $A$ according to the edge types on $A$; again we get at most $2^r$ parts.
    In this way we obtain a partition $\Pp$ of $V(G)$ into at most $2^{r + 1}$ parts.

    Define a pseudometric on $V(G)$ as follows:
    \[
        \dist(u, v) =
        \begin{cases}
            0 & \text{if $u, v \in A$,} \\
            0 & \text{if $u, v \in V(G) \setminus A$,} \\
            +\infty & \text{otherwise.}
        \end{cases}
    \]

    Note that $E(u, v)$ depends only on $\Pp(u)$ and $\Pp(v)$, for all $u, v$ with $\dist(u, v) > 1$.
    Therefore, by \Cref{thm:incremental-lemma} there is a set $S \subseteq V(G)$ of size at most $\ell$ for some constant $\ell = \ell(k, r)$ such that $E(u, v)$ depends only on $E(u, S)$ and $E(v, S)$ for all $u, v \in V(G)$ with $\dist(u, v) > 5$.
    In particular, for every $u \in A$ and $v \in V(G) \setminus A$ we have that $E(u, v)$ depends only on $E(u, S)$ and $E(v, S)$.
    Therefore, there exists a symmetric flip $G'$ of $G$ with parameters $S$ that has no edges between $A$ and $V(G) \setminus A$.
    Naturally, then $A$ is the union of the vertex sets of a collection of connected components of $G'$.
\end{proof}

With the combinatorial characterization of low rank sets in graphs of bounded VC dimension given by \Cref{lem:low-rank-sets-in-flipconn}, we can prove that flip-connectivity logic has low rank definability property over any class of bounded VC dimension, as postulated in \Cref{def:low-rank-definability}.
Similarly as in \cref{sec:separator}, we first prove this property just for rank $0$ sets, i.e.\ unions of the vertex sets of connected components.

\begin{lemma}
    \label{lem:rank-0-definable}
    Let $\varphi(X, \wtup Y, \tup z)$ be a formula of flip-connectivity logic with free set variables $X, \wtup Y$ and free vertex variables $\tup z$.
    Then there is a formula $\psi(x, \tup t, \wtup Y, \tup z)$ of flip-connectivity logic such that the following holds:
    For every graph $G$ and evaluations $\wtup B$ of $\wtup Y$ and $\tup c$ of $\tup z$, if there is a set $A \subseteq V(G)$ of rank $0$ such that $G \models \varphi(A, \wtup B, \tup c)$, then there exists a set $A' \subseteq V(G)$ of rank $0$ and an evaluation $\tup d$ of variables $\tup t$ such that
    \[G \models \varphi(A', \wtup B, \tup c)\qquad \textrm{and}\qquad A' = \psi(G, \tup d, \wtup B, \tup c).\]
\end{lemma}
\begin{proof}
    The proof of this lemma is essentially the same as that of \Cref{lem:rank0-separator}.
    The only substantial difference is that we need to tailor the standard notion of types to flip-connectivity logic.
    Concretely, instead of considering the set of all sentences of separator logic over a given signature and of bounded quantifier rank, we consider the set of all flip-connectivity sentences  over that signature and with the same bound on their quantifier rank.
\end{proof}
    
Now we can prove that flip-connectivity logic has low rank definability property on every class of graphs of bounded VC dimension. The proof essentially repeats the reasoning of \cref{lem:lrdp-separator}.
\begin{lemma}
    \label{lem:property-for-fconn}
    For every graph class $\Cc$ of bounded VC dimension, flip-connectivity logic has low rank definability property on $\Cc$.
\end{lemma}
\begin{proof}
    Let $r \in \N$ be a bound on the rank and $\varphi(X, \wtup Y, \tup z)$ be a formula of flip-connectivity logic, as in the definition of low rank definability property.
    By \Cref{lem:operation-forawrd} there is a formula $\varphi'(X, \wtup Y, \tup z)$ depending only on $\varphi$ and $r$ such that for every subset $S$ of vertices of $G$ of size at most $\ell$ and every symmetric flip $G'$ of $G$ with parameters $S$, we have
    \[
        G \models \varphi(A, \wtup B, \tup c)\qquad\textrm{if and only if}\qquad G' \models \varphi'(A, \wtup B, \tup c).
    \]
    Fix an evaluation $\wtup B$ of $\wtup Y$ and $\tup c$ of $\tup z$.
    Assume that there is a set $A \subseteq V(G)$ of rank at most $r$ such that $G \models \varphi(A, \wtup B, \tup c)$.
    By \Cref{lem:low-rank-sets-in-flipconn} there is a set $S \subseteq V(G)$ of size at most $\ell$ and an $S$-flip $G'$ of $G$ such that $A$ is a set of rank $0$ in $G'$.
    Also, we know that $G' \models \varphi'(A, \wtup B, \tup c)$.
    So, by \Cref{lem:rank-0-definable} there is a set $A'$ of rank $0$ in $G'$ such that $G' \models \varphi'(A', \wtup B, \tup c)$ and $A' = \psi'(G', \tup d, \wtup B, \tup c)$ for some tuple of parameters $\tup d$ and a formula $\psi'$ that depends only on $\varphi'$ (so only on $\varphi$ and $\ell$).
    By \Cref{lem:operation-backward}, there is a formula $\psi(x, \tup t, \tup t', \wtup Y, \tup z)$ such that $A' = \psi(G, \tup d, \tup s, \wtup B, \tup c)$.
    As all the conditions required from $A', \psi$ are~met,  flip-connectivity logic has low rank definability property on $\Cc$.
\end{proof}

\Cref{thm:main-fconn-positive} follows now from \Cref{lem:property-for-fconn} and \Cref{thm:low-rank-quantifier-elimination}.