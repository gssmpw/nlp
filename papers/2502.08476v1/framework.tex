\section{A general framework for proving equivalence with low rank \mso}\label{sec:framework}

In the upcoming sections we will prove that low rank \mso has the same expressive power as separator logic on weakly sparse graph classes (\cref{thm:main-separator}), as flip-connectivity logic on graph classes with bounded VC-dimension (\cref{thm:main-fconn-positive}), and as flip-reachability logic on all graphs (\cref{thm:main-freach}). All these proofs follow a similar scheme: we prove that quantification over sets of low rank can be emulated in the considered logic. The goal of this section is to provide a framework that captures common features of all three proofs, reducing the task to verifying a concrete property of the considered logic on the considered graph classes.

This key property is captured in the following definition.

\begin{definition}
    \label{def:low-rank-definability}
    We say that a logic $\Ll$ has \emph{low rank definability property} on a class of (colored) graphs $\Cc$ if for every $r \in \N$ and formula $\varphi(X, \wtup Y, \tup z)$ of $\Ll$ with free set variables $X, \wtup Y$ and free vertex variables $\tup z$, there is a formula $\psi(x, \tup t, \wtup Y, \tup z)$ of $\Ll$ with the following property: For every $G \in \Cc$, evaluation $\wtup B$ of $\wtup Y$ and evaluation $\tup c$ of $\tup z$, if there is a set $A \subseteq V(G)$ with $\rk(A) \le r$ such that $G \models \varphi(A, \wtup B, \tup c)$, then there is a set $A' \subseteq V(G)$ and a tuple $\tup d \in V(G)^{\tup t}$ such that
    \[G \models \varphi(A', \wtup B, \tup c)\qquad \textrm{and}\qquad A' = \psi(G, \tup d, \wtup B, \tup c).\]
\end{definition}

\begin{theorem}
    \label{thm:low-rank-quantifier-elimination}
    Suppose a logic $\Ll$ is an extension of \fo and has low rank definability property on a class of (colored) graphs $\Cc$. Then for every formula of low rank \mso there is a formula of $\Ll$ so that the two formulas are equivalent on all graphs from $\Cc$.
\end{theorem}
\begin{proof}
    We show that over our class of graphs $\Cc$, every formula of low rank \mso is equivalent to some formula of $\Ll$ by induction on the structure of a formula of low rank \mso.
    The only interesting part of the induction is low rank set quantification, since all other formula constructors of low rank \mso are already present in \fo.
    Consider a formula that begins with such a quantifier:
    \begin{align}
    \label{eq:existential-formula}
    \exists X \colon r\quad \varphi(X, \wtup Y, \tup z).
    \end{align}
    By induction assumption, the inner formula $\varphi$ can be expressed in $\Ll$.
    We may also assume that for every graph $G$, a set $A \subseteq V(G)$, and evaluations $\wtup B$ of $\wtup Y$ and $\tup c$ of $\tup z$, the condition $G \models \varphi(A, \wtup B, \tup c)$ implies that the rank of $A$ is at most $r$.
    Indeed, by \Cref{lem:inter-def} we know that the rank of a set $A$ in a graph $G$ is at most $r$ can be verified in \fo, and we can always add this assertion to $\varphi$.
    Then, by low rank definability property we have a formula $\psi(x, \tup t, \wtup Y, \tup z)$ of $\Ll$ such that for any graph $G \in \Cc$, whenever there exists a set $A \subseteq V(G)$ of rank at most $r$ satisfying $G \models \varphi(A, \wtup B, \tup c)$ for some evaluation $\wtup B$ of $\wtup Y$ and $\tup c$ of $\tup z$, then we have a tuple $\tup d \in V(G)^{\tup t}$ and a set $A' = \psi(G, \tup d, \wtup B, \tup c)$ such that $G \models \varphi(A', \wtup B, \tup c)$.
    Since the satisfaction of $\varphi$ implies that the rank is at most $r$, we can express the formula \eqref{eq:existential-formula} in $\Ll$.
\end{proof}

Next, we propose tools for streamlining the verification of the low rank definability property for the considered extensions of \fo. Intuitively, for each considered extension and relevant graph class $\Cc$, we will show a structure theorem showing that every set $A$ of low rank in a graph $G\in \Cc$ admits a certain structure guarded by a small set of parameters $S\subseteq V(G)$. It will be useful to consider $A$ in a suitable ``simplification'' of~$G$, which we call the {\em{$S$-operation}} of $G$. The definition of $S$-operation depends on the considered logic, as explained formally~below.

\begin{definition}
    Let $\Ll \in \set{\text{separator logic}, \text{flip-connectivity logic}, \text{flip-reachability logic}}$, $G$ be a colored graph and $S \subseteq V(G)$ be a subset of its vertices.
    We say that a graph $G'$ is an \emph{$S$-operation} with respect to $\Ll$ of $G$ if the following holds:
    \begin{itemize}[nosep]
        \item If $\Ll$ is separator logic, then $G'$ is the graph obtained from $G$ by isolating vertices from $S$ (i.e. removing all edges incident to vertices in $S$). Further, for each $s\in S$ add a unary predicate that marks the neighborhood of $S$ in $G$, and a unary predicate that marks only $s$.
        \item If $\Ll$ is flip-connectivity logic, then $G'$ is a symmetric flip of $G$ with parameters $S$. Further, for every atomic type over $S$ in $G$ (with an arbitrary fixed enumeration of $S$), add a unary predicate that marks the vertices of $G$ of this type.
        \item If $\Ll$ is flip-reachability logic, and $G'$ is a flip of $G$ with parameters $S$. Again, for every atomic type over $S$ in $G$, add a unary predicate that marks the vertices of $G$ of this type.
    \end{itemize}
\end{definition}

Observe that if we fix an arbitrary enumeration of $S$, then the unary predicates that we add to $G'$ (but not their interpretations) depend only on $|S|$.
Therefore, for a given class of graphs $\Cc$ we may talk about the language of $k$-operations, for a fixed $k \in \N$.
This is the language that consists of all the relations used in graphs from $\Cc$ and all the unary predicates added to $S$-operations of graphs from $\Cc$ for $S$ of size $k$.

Next, for each considered extension $\Ll$ of \fo, we  show that we can freely translate $\Ll$-formulas working on the graph and on its $S$-operation.

\begin{lemma}
    \label{lem:operation-forawrd}
    Let $\Ll \in \set{\text{separator logic}, \text{flip-connectivity logic}, \text{flip-reachability logic}}$, $G$ be a graph, $S \subseteq V(G)$ be a subset of its vertices, $G'$ be an $S$-operation with respect to $\Ll$ of $G$, and $\varphi(\wtup X, \tup y)$ be a formula of $\Ll$ with free set variables $\wtup X$ and free vertex variables $\tup y$. Then there is a formula $\varphi'(\wtup X, \tup y)$ of $\Ll$, depending only on $\varphi$ and $|S|$, such that for every evaluation $\wtup A$ of $\wtup X$ and $\tup b$ of $\tup y$ we have
    \[
        G \models \varphi(\wtup A, \tup b) \qquad\textrm{if and only if}\qquad G' \models \varphi'(\wtup A, \tup b).
    \]
\end{lemma}
\begin{proof}
    We start with the case when $\Ll$ is separator logic.
    Observe that in $G'$ we can define the adjacency relation of $G$ as a quantifier-free formula that uses the unary predicates for the neighborhoods of vertices in $S$ and the unary predicates for the individual vertices of $S$.
    Indeed, if we want to check whether $G \models E(u, v)$ for some vertices $u, v \in V(G)$, then first we check if any of them is in $S$.
    If not, then we know $G \models E(u, v)$ if and only if $G' \models E(u, v)$.
    In the other case, i.e.\ when (at least) one of them is in $S$, we check if the other one is marked with the respective unary predicate.

    Next, observe that we can define each connectivity predicate of $G$ in $G'$.
    Indeed, to check whether $G \models \conn_k(u, v, \tup a)$ for some vertices $u, v \in V(G)$ and vertices $\tup a \in V(G)^k$, we can guess (by making a disjunction over all cases) if the path goes through vertices from $S$ and in which order they appear.
    Then in $G'$ we can express that there are consecutive paths between $u$, guessed vertices from $S$, and $v$ after removing $\tup a$.
    In this way we get a disjunction over quantifier-free formulas that use connectivity predicates in $G'$.

    To obtain the formula $\varphi'$ we replace all occurrences of the edge relation and connectivity predicates in $\varphi$ with the respective quantifier-free formulas.

    The cases of flip-connectivity logic and flip-reachability logic are similar.
    The only interesting case is when we need to express flip-connectivity (respectively flip-reachability) predicates of $G$ in $G'$.
    For this note that since we added unary predicates to $G'$, $G$ is a symmetric flip (respectively flip) of $G'$ without parameters.
    So every symmetric flip (respectively flip) of $G$ is a symmetric flip (respectively flip) of~$G'$.
\end{proof}

\begin{lemma}
    \label{lem:operation-backward}
    Let $\Ll \in \set{\text{separator logic}, \text{flip-connectivity logic}, \text{flip-reachability logic}}$, $G$ be a graph, $S \subseteq V(G)$ be a subset of its vertices, $\tup s$ be a tuple enumerating all the elements of $S$, $G'$ be an $S$-operation of $G$ with respect to $\Ll$, and $\varphi(\wtup X, \tup y)$ be a formula of $\Ll$ with free set variables $\wtup X$ and free vertex variables $\tup y$.
    Then there is a formula $\varphi'(\wtup X, \tup y, \tup z)$ of $\Ll$, depending only on $\varphi$ and $|S|$ with $|\tup z| = |\tup s|$, such that for every evaluation $\wtup A$ of $\wtup X$ and $\tup b$ of $\tup y$ we have
    \[
        G' \models \varphi(\wtup A, \tup b) \qquad\textrm{if and only if}\qquad G \models \varphi'(\wtup A, \tup b, \tup s).
    \]
\end{lemma}
\begin{proof}
    We start with the case when $\Ll$ is separator logic.
    The proof is similar to the proof of \cref{lem:operation-forawrd}.
    We clearly see that we can define the adjacency relation of $G'$ in $G$ using the provided tuple $\tup s$.
    The same is true for the unary predicates marking the neighborhoods of vertices from $S$ in $G'$.
    Finally, for each connectivity predicate $\conn_k(u, v, \tup a)$ in $G'$ we can replace it with predicate $\conn_{k+|S|}(u, v, \tup a\tup s)$ of $G$.

    The cases of flip-connectivity logic and flip-reachability logic are similar.
\end{proof}

Intuitively,
\cref{lem:operation-forawrd,lem:operation-backward} allow us to reduce verification of the low rank definability property from the original graph to its $S$-operation, for any set of parameters $S$ of bounded size. As the $S$-operation will be a much simpler graph, arguing definability there is significantly easier. We will use this scheme in the proofs of \cref{thm:main-separator,thm:main-fconn-positive}. In the proof of \cref{thm:main-freach}, low rank definability property will be argued directly.