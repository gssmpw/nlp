% \subsection{Instruction Prompts}
\label{sec:appendix_prompt}

The prompts used by the WildLong framework can be seen from Table~\ref{tab:prompt_extract_meta}, Table~\ref{tab:prompt_path_to_instruct}, Table~\ref{tab:prompt_instruct_response}, and Table~\ref{tab:prompt_single_to_multi}.

\begin{table}[h!]\centering
\begin{minipage}{\textwidth}
%\vspace{0mm}    
\centering
\begin{tcolorbox} 
    \centering
   
     %\hspace{-4mm}
      \small
    \begin{tabular}{p{0.95\textwidth}}
    Below is a conversation between a user and an AI Language Model, likely involving a long document.\\
    \\ \textbf{Conversation}
         \\ {\tt \{conversation\}} \\ 
         \\ \textbf{Your Tasks}
         \\ Based on the conversation above, try to finish the following tasks.
         \\ - Determine whether the query of the user involves a long document (or any form of long text). 
         \\ - If the conversation involves a long document, analyse the conversation and provide the following information using concise phrases. 
         \\ \hspace{0.4em} - Document Type: Specify the format or category of the document, such as a research paper, technical report, fictional story, instruction manual, etc. Ideally, extract one document type. However, if you believe there are multiple types, limit the number to two.
         \\ \hspace{0.4em} - Tasks or Requests: Identify 1 to 3 the specific tasks the user wants the chatbot to perform given the long context. This may include summarizing key points, integrating multiple pieces of information, continuing the dialogue or story, providing an analysis, or any other specific task relevant to the long text.
         - Purpose of Query: Define the objective behind the user's query, such as educational purposes, decision-making, research, entertainment, etc. List 1 to 3 items.
         \\ \hspace{0.4em} - User Intention: Determine the underlying goal or reason behind the user's request, such as completing an assignment, preparing for a debate, gaining a general understanding, etc. List 1 to 3 items.
         \\ \hspace{0.4em} - User Profile: Describe the possible characteristics and background of the user in 1 to 5 phrases.
         \\ \hspace{0.4em} - User's Language Style: Identify the language style of the user. List 1 to 3 items.
         \\ \hspace{0.4em} - Context: Describe the situational background influencing the query, such as working on a group project, preparing for an exam, etc. List 1 to 3 context items.
         \\ \hspace{0.4em} - Knowledge/Commonsense Involved for User: Identify the prior knowledge or commonsense the user is expected to have. List 1 to 5 items.
         \\ \hspace{0.4em} - Knowledge/Commonsense Involved for Chatbot: Identify the prior knowledge or commonsense the chatbot is expected to have to address the query. List 1 to 5 items.
         \\ \hspace{0.4em} - Long Context Capability Involved: Determine the comprehension and information processing skills required to address the user's request, such as long document comprehension, key information retrieval, handling multiple perspectives, etc. List 1 to 3 items.
         \\ \hspace{0.4em} - Output Format: Identify the desired format of the response. List 1 to 3 items.
         \\ \hspace{0.4em} - Sentiment: Determine the expected emotional tone or attitude in the response. List 1 to 3 items.
         \\ \hspace{0.4em} - Constraint of the Request: Identify the limitations or additional requirements that the user has for the chatbot's response. List 0 to 3 constraints, if any.
         \\ \hspace{0.4em} - Simplified Instruction by User: Provide a simplified version of the user's request, removing any context or background information.

    \\ \textbf{Output Format}
        \\ Document Type:
        \\ 1. doc type 1 ...
        \\ 2. doc type 2 ...
        \\
        \\ Task or Request:
        \\ 1. request type 1 ...
        \\ 2. request type 2 ...
        \\ ...

    \\ \textbf{Additional Requirements for Output}
        \\ - Analyze the entire conversation to produce your answers, taking into account both the user's and the chatbot's contributions. Do not limit your analysis to just one side.
        \\ - If the user query does not involve a long document (or any form of long text), output only "No long document involved".
        \\ - For each output field, output commonly used phrases or short sentences in academic or industry if applicable.
        \\ - If you cannot extract anything for a particular field, output "NA" for that field.
    \end{tabular}
\end{tcolorbox}
%\vspace{-2mm}
\caption{The prompt to extract meta information with GPT-4.}
    \label{tab:prompt_extract_meta}
\end{minipage}
\end{table}



\begin{table}[h!]\centering
\begin{minipage}{\textwidth}
%\vspace{0mm}    
\centering
\begin{tcolorbox} 
    \centering
   
     %\hspace{-4mm}
      \small
    \begin{tabular}{p{0.95\textwidth}}
    You are tasked with generating 3 realistic user queries or instructions for a chatbot about a long document. The user is interacting with a long {\tt \{doc{\_}type\}}, but you do not have access to the exact content of the document. Your task is to create reasonable user queries or instructions that meet specific meta information criteria. There are 12 meta information categories that define the characteristics of a user query or instruction. You will be provided with 6 key meta information fields that must be incorporated into each of your generated queries or instructions. For the remaining 6 categories, you have the flexibility to explore different possibilities to create varied and diverse queries or instructions. You will be given an example meta information criteria and a corresponding sample query or instruction to help you understand the context and how to apply the meta information.\\
    \\
    \textbf{Additional requirements}\\
    - Incorporate All Key Fields: Aim to integrate all 6 key meta information fields into each query or instruction you create. If a field is particularly challenging to include, substitute it with a reasonable alternative.\\
    - Ensure Coherence and Creativity: Your generated queries or instructions should be coherent, natural, and flow smoothly. They should not appear as a direct combination of the meta information fields, instead aiming for a realistic scenario that a user in the given context might actually encounter.\\
    - Creative Interpretation: The meta information criteria represent high-level characteristics of a user's query or instruction. You can interpret and apply them creatively to generate a range of realistic and diverse outputs.\\
    - Output Format: Present your generated queries or instructions in bullet points, formatted as follows:\\
    1. {{query 1}}\\
    2. {{query 2}}\\
    3. {{query 3}}\\
    \\
    \textbf{Definitions of the 12 meta information categories}\\
    - Tasks or Requests: tasks the user wants the chatbot to perform given the long context.\\ 
    - Purpose of Query: the objective behind the user's query, such as educational purposes, decision-making, research, entertainment, etc.\\
    - User Intention: the underlying goal or reason behind the user's request, such as completing an assignment, preparing for a debate, gaining a general understanding, etc.\\
    - User Profile: the possible characteristics and background of the user.\\
    - User's Language Style: the language style of the user.\\
    - Context: the situational background influencing the query, such as working on a group project, preparing for an exam, etc.\\
    - Knowledge/Commonsense Involved for User: the prior knowledge or commonsense the user is expected to have.\\
    - Knowledge/Commonsense Involved for Chatbot: the prior knowledge or commonsense the chatbot is expected to have to address the query.\\
    - Long Context Capability Involved: the comprehension and information processing skills required to address the user's request, such as long document comprehension, key information retrieval, handling multiple perspectives, etc.\\
    - Output Format: the desired format of the response.\\
    - Sentiment: the expected emotional tone or attitude in the response.\\
    - Constraint of the Request: the limitations or additional requirements that the user has for the chatbot's response.\\
    \\
    \textbf{Example meta information criteria}\\
    {\tt \{example{\_}meta{\_}info\}}\\
    \\
    \textbf{Example query/instruction}\\
    {\tt \{example{\_}instruction\}}\\
    \\
    \textbf{Your task}\\
    Generate a new query or instruction that aligns with the given meta information criteria:\\
    {\tt \{path{\_}meta{\_}info\}}\\
    \end{tabular}
\end{tcolorbox}
%\vspace{-2mm}
\caption{The prompt to generate instruction given a sampled meta-information path.}
    \label{tab:prompt_path_to_instruct}
\end{minipage}
\end{table}



\begin{table}[h!]\centering
\begin{minipage}{\textwidth}
%\vspace{0mm}    
\centering
\begin{tcolorbox} 
    \centering
   
     %\hspace{-4mm}
      \small
    \begin{tabular}{p{0.95\textwidth}}
    \textbf{Long Document:}\\
    {\tt \{long{\_}doc\}}\\
    \\
    \textbf{Example Query/Instruction:}\\
    {\tt \{example{\_}instruct\}}\\
    \\
    \textbf{Your Task:}\\
    You have been provided with a long document above, along with an example query or instruction that was formulated for another similar long document.\\
    
    Your task is to create a new query or instruction that can be addressed using the information contained within the long document provided.\\
    
    The new query or instruction should be inspired by the structure and intent of the given example but is not a direct copy. You should adapt the query or instruction to fit the context of the long document while still addressing a similar type of task.\\
    
    Once you have formulated the query or instruction, provide a response based on the content of the long document.\\
    \\
    Please format your output as follows:\\
    Query/Instruction: \{\{query{\_}or{\_}instruction\}\}\\
    Response: \{\{response\}\}
    \end{tabular}
\end{tcolorbox}
%\vspace{-2mm}
\caption{The prompt to generate instruction-response pairs.}
    \label{tab:prompt_instruct_response}
\end{minipage}
\end{table}



\begin{table}[h!]\centering
\begin{minipage}{\textwidth}
%\vspace{0mm}    
\centering
\begin{tcolorbox} 
    \centering
   
     %\hspace{-4mm}
      \small
    \begin{tabular}{p{0.95\textwidth}}
    The following are tasks or requests made by users when querying a chatbot about a single document. Modify the tasks or requests as if the user is querying multiple documents. Ensure that the modifications reflect a realistic need to handle information across multiple sources, incorporating cognitive operations usually applied to multiple documents.\\
    The document type is {\tt \{doc{\_}type\}}. Avoid simply adding phrases like "across multiple documents." Instead, adapt each task to reflect a more complex interaction with multiple sources, focusing on the cognitive operation that makes sense in the multi-document context.\\
    \\
    \textbf{Cognitive operations}
    - Comparison: identifying similarities, differences, or evaluating multiple documents\\
    - Synthesis: integrating information from multiple sources to create a new, cohesive understanding\\
    - Aggregation: collecting and presenting information from multiple sources without integrating or interpreting\\
    - Verification and Validation: cross-referencing and fact-checking across documents\\
    - Consensus Analysis: identifying agreement across documents\\
    - Divergence Analysis: recognizing conflicting or differing points of view\\
    - Problem Solving: formulating solutions based on multiple documents\\
    - Decision Making: formulating decisions based on multiple documents\\
    - Exploration: discovery across multiple sources without a predefined goal\\
    - Trend and Pattern Identification: detecting larger patterns or trends from multiple documents\\
    - Hypothesis Generation: forming new hypotheses through integrated data\\
    - Creative Synthesis: fostering novel ideas or concepts from the documents\\
    \\
    \textbf{Original tasks or requests}
    {\tt \{original{\_}tasks{\_}or{\_}requests\}}\\
    \\
    \textbf{Output format}\\
    1. { \{original{\_}tasks{\_}or{\_}requests\}}: { \{modified{\_}tasks{\_}or{\_}requests\}}\\
    2. { \{original{\_}tasks{\_}or{\_}requests\}}: { \{modified{\_}tasks{\_}or{\_}requests\}}\\
    ...\\
    \end{tabular}
\end{tcolorbox}
%\vspace{-2mm}
\caption{The prompt to convert single-document tasks to multi-document tasks. }
    \label{tab:prompt_single_to_multi}
\end{minipage}
\end{table}

% The following are tasks or requests made by users when querying a chatbot about a single document. Modify the tasks or requests as if the user is querying multiple documents. Ensure that the modifications reflect a realistic need to handle information across multiple sources, incorporating cognitive operations usually applied to multiple documents.
% The document type is {doc_type}. Avoid simply adding phrases like "across multiple documents." Instead, adapt each task to reflect a more complex interaction with multiple sources, focusing on the cognitive operation that makes sense in the multi-document context.

% ** Cognitive operations:
% - Comparison: identifying similarities, differences, or evaluating multiple documents
% - Synthesis: integrating information from multiple sources to create a new, cohesive understanding
% - Aggregation: collecting and presenting information from multiple sources without integrating or interpreting
% - Verification and Validation: cross-referencing and fact-checking across documents
% - Consensus Analysis: identifying agreement across documents
% - Divergence Analysis: recognizing conflicting or differing points of view
% - Problem Solving: formulating solutions based on multiple documents
% - Decision Making: formulating decisions based on multiple documents
% - Exploration: discovery across multiple sources without a predefined goal
% - Trend and Pattern Identification: detecting larger patterns or trends from multiple documents
% - Hypothesis Generation: forming new hypotheses through integrated data
% - Creative Synthesis: fostering novel ideas or concepts from the documents

% ** Original tasks or requests:
% {orig_tasks_or_requests}

% ** Output format:
% 1. {{original_task_or_request}}: {{modified_task_or_request}}
% 2. {{original_task_or_request}}: {{modified_task_or_request}}
% ...

% ** Your output: