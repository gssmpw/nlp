% By default, a switch is "false". Use \NAME_OF_SWITCHtrue to set to true
% Uncomment the line below to set the switch to "true"
\newif\ifARXIV
%\ARXIVtrue

\newif\ifHAL
\HALtrue

\ifHAL
\documentclass[11pt,a4paper]{article}
\usepackage{natbib}
\usepackage{a4wide}
\else
%\documentclass[authoryear,preprint,5p]{elsarticle}
%\documentclass[authoryear,preprint]{elsarticle}
\documentclass[preprint]{elsarticle}
\fi

\usepackage{moreverb,url}

\usepackage[colorlinks,bookmarksopen,bookmarksnumbered,allcolors=blue]{hyperref}

\ifHAL 
\usepackage[hyperpageref]{backref} 
\renewcommand*{\backref}[1]{}  
\renewcommand*{\backrefalt}[4]{
  \ifcase #1 
  No cited.
  \or
  Cited on page #2.
  \else
  Cited on pages #2.
  \fi}

\usepackage{doi} %maybe put it back in the hal version...
\fi
\usepackage[ruled,vlined]{algorithm2e}
\SetAlFnt{\scriptsize} % or footnotesize

\usepackage{amsmath,amssymb,amsfonts}
\usepackage[utf8]{inputenc}
\usepackage{graphicx}
\usepackage[scriptsize]{subfigure}
\usepackage{xspace}
\usepackage{genom}
\usepackage{listings}
\usepackage{lstlang3}
\usepackage{xcolor}
\usepackage{makecell}
\usepackage{placeins}
\usepackage[binary-units]{siunitx}


\newcommand{\fiacre}{{\sc Fiacre}}
\newcommand{\hfiacre}{{\sc H-Fiacre}}
\newcommand{\tina}{{\sc Tina}}
\newcommand{\hippo}{{\sc Hippo}}
\newcommand{\ps}{{\sc ProSkill}}
\newcommand{\proskill}{{\sc ProSkill}}
\newcommand{\ProSkill}{{\sc ProSkill}}
\newcommand{\bttf}{{\sc BT2Fiacre}}

% misc Commandes
\newcommand{\task}[1]{\textsl{#1}}
\newcommand{\event}[1]{\textsl{#1}}
\newcommand{\port}[1]{\textsf{#1}}
\newcommand{\modu}[1]{\textsc{#1}}
\newcommand{\bt}[1]{\textbf{#1}}
\newcommand{\btn}[1]{\textsl{#1}}
\newcommand{\process}[1]{\textbf{#1}}
\newcommand{\sv}[1]{\textsf{#1}}
\newcommand{\svv}[1]{\textit{#1}}
\newcommand{\state}[1]{\textit{#1}}
\newcommand{\codel}[1]{\small\texttt{#1}}
\newcommand{\code}[1]{{\small\ttfamily{#1}}}
\newcommand{\tool}[1]{\textsf{#1}}
\newcommand{\field}[1]{\code{:#1}}

\newcommand{\running}{\code{running}}
\newcommand{\success}{\code{success}}
\newcommand{\failure}{\code{failure}}


\usepackage{inconsolata}
% Define Colors
\definecolor{eclipseBlue}{RGB}{42,0.0,255}
\definecolor{eclipseGreen}{RGB}{63,127,95}
\definecolor{eclipsePurple}{RGB}{127,0,85}

\definecolor{colKeys}{rgb}{1,0,0.75}
\definecolor{colKeys2}{rgb}{0,0,1}
\definecolor{colKeys3}{RGB}{63,127,95}
\definecolor{colIdentifier}{rgb}{0,0,0}
\definecolor{colComments}{rgb}{0,0.5,0}
\definecolor{colString}{rgb}{0.6,0.1,0.1} 

\definecolor{bluekeywords}{rgb}{0.13,0.13,1}
\definecolor{greencomments}{rgb}{0,0.5,0}
\definecolor{redstrings}{rgb}{0.9,0,0}

\definecolor{Maroon}{rgb}{0.5,0,0}
\definecolor{darkgreen}{rgb}{0,0.5,0}

\lstset{%configuration de listings
float=hbp,%
basicstyle=\scriptsize\ttfamily, %
identifierstyle=\color{colIdentifier}, %
keywordstyle=\color{colKeys}, %
keywordstyle=[2]{\color{colKeys2}},%  
keywordstyle=[3]{\color{colKeys3}},%  
stringstyle=\color{colString}, %
commentstyle=\color{colComments}, %
numberstyle=\tiny\ttfamily, %
escapeinside={(*@}{@*)},%
columns=flexible, %
tabsize=2, %
extendedchars=true, %
showspaces=false, %
showstringspaces=false, %
numbers=none, %
breaklines=true, %
breakautoindent=true, %
captionpos=t}

\lstdefinelanguage{XML}
{
  basicstyle=\scriptsize\ttfamily,%\ttfamily\footnotesize,
  morestring=[b]",
  moredelim=[s][\color{bluekeywords}]{<}{\ },
  moredelim=[s][\color{bluekeywords}]{</}{>},
  moredelim=[l][\color{bluekeywords}]{/>},
  moredelim=[l][\color{bluekeywords}]{>},
  morecomment=[s]{<?}{?>},
  morecomment=[s]{<!--}{-->},
  commentstyle=\color{colComments},
  stringstyle=\color{colString},
  identifierstyle=\color{colIdentifier}
}

\def\volumeyear{2024}


\setcounter{secnumdepth}{3}

\newcommand{\from}[2]
{\textcolor{red}{\textbf{//}{note from #1 : #2}\textbf{ //}}}

\newcommand{\fromf}[1]
{\from{Felix}{\textcolor{blue}{#1}}}

\newcommand{\fromb}[1]
{\from{Bernard}{\textcolor{orange}{#1}}}

\newcommand{\frompe}[1]
{\from{Pierre Emmanuel}{\textcolor{green}{#1}}}

\newcommand{\froms}[1]
{\from{Silvano}{\textcolor{purple}{#1}}}


\begin{document}

\renewcommand{\cite}[1]{\citep{#1}}

\ifHAL
\title{A formal implementation of Behavior Trees to act in robotics}%\\
%\textcolor{red}{DRAFT, do no distribute!}}
\else
\title{A formal implementation of Behavior Trees to act in robotics}
\fi

\ifHAL
\author{Félix Ingrand\\
  felix@laas.fr\\
  LAAS-CNRS, Universit\'e de Toulouse\\
  Toulouse, France}
\date{}

\maketitle

\else
\author[1]{Félix Ingrand\corref{cor1}} \ead{felix@laas.fr}
\cortext[cor1]{Corresponding author}
\address[1]{LAAS-CNRS, Universit\'e de Toulouse, Toulouse, France}
\fi


\begin{abstract}
  Behavior Trees (BT) are becoming quite popular as an \emph{Acting} component of autonomous robotic systems. We propose to define a formal
  semantics to BT by translating them to a formal language which enables us to perform verification of programs written with BT, as well as
  runtime verification while these BT execute. This allows us to formally verify BT correctness without requiring BT programmers to master
  formal languages and without compromising BT most valuable features: modularity, flexibility and reusability. We present the formal
  framework we use: \fiacre{}, its language and the produced TTS model; \tina{}, its model checking tools and \hippo{}, its runtime
  verification engine. We then show how the translation from BT to \fiacre{} is automatically done, the type of formal LTL and CTL
  properties we can check offline and how to execute the formal model online in place of a regular BT engine. We illustrate our approach on
  two robotics applications, and show how BT can be extended with state variables, \btn{eval} nodes, node evaluation results and benefit of
  other features available in the \fiacre{} formal framework (e.g., time).
\end{abstract}

\ifHAL
\else
\begin{keyword}
Behavior Tree, Acting in Robotics, Formal model, Validation and Verification, Runtime Verification
\end{keyword}

\maketitle 

\fi

\ifHAL
%\clearpage 
%\begin{small}                   %To fit on one page
%\tableofcontents 
%\end{small}
%\clearpage 
\else
%\begin{small}                   %To fit on one page
%\tableofcontents 
%\end{small}
\fi
\section{Introduction and motivation}
\label{sec:intro}
\label{sec:btgeneral}

Behavior Trees (BT) were initially developed and deployed to program Non-Player Characters (NPCs) in video games.  BT are a powerful and
modular framework for designing and implementing decision-making and control systems. They provide a structured way to define complex
behaviors by combining smaller, reusable behavior modules into a hierarchical tree structure. BT have gained popularity in robotics due to
their flexibility, readability, and ability to handle dynamic environments (e.g., Nav2, a popular navigation stack used in ROS2, is now
deployed using BT~\cite{ros2nav2:2024aa}, similarly, BT are a keystone of projects such as~\cite{Street:2024aa} where they aim at verifying the
whole development toolchain).

Why yet another BT implementation? To provide a formal representation of BT which can then be used to formally verify interesting properties
of the BT (can this node succeed, or fail, is this node reachable or not, can we prove that if we reach this state, eventually we will reach
this one within some time interval, etc), and also to execute the formal model which implements the BT in lieu of a regular BT engine.

A behavior tree consists of nodes arranged hierarchically. Parent nodes tick their children to pass them the execution
``token''.  Children return either \success{}, \running{} or \failure{} to their parent when the current execution tick is done, and the
parent pursues the execution following a specified behavior.
\begin{description}
\item[Root node] The entry point of the tree that initiates the behavior evaluation and generates the successive ticks.
\item[\btn{Leaf} nodes] The terminal nodes that perform specific functions when ticked:
  \begin{description}
  \item[\btn{Condition} nodes] Evaluate conditions, such as checking sensor data, battery level, or environmental states.  They return
    \success{} or \failure{}.
  \item[\btn{Action} nodes] Trigger actions in the environment, such as moving a joint, picking up an object, or sending a command. Action,
    on top of \success{} or \failure{}, 
    may also return \running{} (e.g., when it takes \emph{some} time to complete the action such as a robot motion).
  \end{description}
\item[\btn{Control} nodes] These modes specify the execution logic (behavior) of their children nodes (i.e., how to execute them):
  \begin{description}
  \item[\btn{Sequence}] Executes its children from left to right. If any child fails, the \btn{Sequence} fails, halting further
    execution. If the last one succeeds, the \btn{Sequence} succeeds. If a child returns \running{}, so does the \btn{Sequence}, which will,
    when ticked again,
    either call the last running child or restart the \btn{Sequence}, depending on its type: \btn{Sequence} or \btn{ReactiveSequence}.
  \item[\btn{Fallback}] (a mirror of \btn{Sequence}) Executes its children from left to right. If any child succeeds, the \btn{Fallback}
    succeeds, skipping the rest. If a child returns \failure{}, we tick the next child, unless it was the last one, in which case the
    \btn{Fallback} returns \failure{}. When a child returns \running{}, the \btn{Fallback} returns \running{}, and, when ticked again, will
    tick either the last ticked child or the first one depending on its type: \btn{Fallback} or \btn{ReactiveFallback}.
  \item[\btn{Parallel}] Runs/ticks multiple children simultaneously and succeeds or fails based on some success threshold (e.g., either all or some
    of the children must succeed for \success{}).
  \end{description}
\item[\btn{Decorator} nodes] perform a specific operation on a single child (e.g., the \btn{Invert} node returns \success{} when its child returns
  \failure{}, and vice versa and returns \running{} when the child returns \running{}). Here is a  non-exhaustive list of \btn{Decorator}
  nodes: \btn{Repeat}, \btn{ForceFailure}, \btn{ForceSuccess}, \btn{RetryUntilSuccessful}, etc. 
\end{description}

This is a very quick and shallow description of BT, and we invite the reader to check books (e.g., \cite{Colledanchise:2018ub}) and online
tutorials (e.g., \url{https://www.behaviortree.dev/}, which introduces the popular \code{BehaviorTree.CPP}), to get a complete picture of
the BT ``programming'' ecosystem,

Note that the BT specifications are not ``closed''. If most implementations propose the BT nodes described above, some offer additional
\btn{Control} nodes which implement more ``specific'' control algorithms (e.g., Nav2~\cite{ros2nav2:2024aa} proposes \btn{Recovery},
\btn{PipelineSequence}, \btn{RoundRobin}, ...) or additional \btn{Decorator} nodes (e.g., Nav2 proposes \btn{RateController}, etc).  Overall,
BT are making explicit the \emph{control} of the execution of the nodes. However, for the leaves of the tree (\btn{Action} and
\btn{Condition} nodes), the specification remains minimal and silent on some features (e.g. can one pass arguments to nodes? returning values?
asynchronous calls? time taken by the real execution? etc). Nevertheless, most implementations specify how these features are handled (e.g.,
with black board for variables, or in the C/C++ code called to implement them, multi-threading, etc).

Considering that more and more robotics applications, using BT, may be deployed in critical applications (autonomous drones or vehicles, etc)
or in an environment with human (service robots), we need to be able to prove some safety properties on the BT, and to trust their execution will
remain faithful to the programmer's intentions. For this we believe we need to harness some formal models to the BT language and its execution
engine to enable some formal offline and online validation and verification of BT.

\section{State of the art and proposed approach}

We split the state of the art in two parts, on one side, the approaches and the papers concerned with BT in robotic applications, on the other
side, the ones which study formal models jointly with BT.

\subsection{BT in robotics}
\label{sec:soa}

BT are praised for their modularity, readability, scalability, flexibility, robustness, and supposedly being easy to debug and test. Even
if these are questionable and somewhat subjective, one cannot deny their rising success and interest in robotics for autonomous navigation,
human-robot interaction, manipulation tasks and multi-agent systems.

The book~\cite{Colledanchise:2018ub} covers most, if not all, aspects of BT in robotics. They make an extensive presentation of BT, how they
compare to FSM, how they can be linked to the planning activity, etc. They mention the importance of safety and formalism, although not much
is said on formal proof and verification.

In~\cite{Iovino:2022aa} the authors make a comprehensive and large survey of the topic of BT in AI and robotic applications. The existing
literature is described and categorized based on methods, application areas and contributions, and the paper concludes with a list of open
research challenges: explainable AI, human–robot interaction, safe AI, and the combination of learning and BT.

The work presented in~\cite{Marzinotto:2014tg} shows the equivalence between BT and Controlled Hybrid Dynamical Systems.  Similarly the
authors of~\cite{Ogren:2022aa} study how to deploy Behavior Trees in Robot Control Systems, and they propose an interesting formal analysis
regarding convergence and regions of attraction.

The authors of~\cite{Schulz-Rosengarten:2024aa} address one of the BT ``shortsight'' and propose to add a cleaner communication extension,
but lack formalism and proof on the LF.  The input/output mechanism is inspiring.

As for evaluating BT, the authors of \cite{Gugliermo:2024aa} propose a set of metrics (some static, some gathered from real runs), to
evaluate some BT properties, as to evaluate and analyze them.

\paragraph{Implementation considerations}

Deploying BT in robotic applications requires addressing implementation issues which may not be present in Video Game
programming. In~\cite{Colledanchise:2018vt} the authors present an original approach to handle parallelism and concurrency in BT (CBT) with
execution progress and resources management.  In~\cite{Colledanchise:2021aa} , the same authors point out the issues on memory nodes (to
avoid reevaluating), asynchronous action calls, parameters, halt (blocking or not), etc.

\paragraph{BT and planning}
\label{sec:soabtplanning}

In many robotics architectures, BT is considered as the ``acting'' component of the decisional layer~\cite{Ingrand:2015ue}. This is the case for
example in PlanSys2~\cite{Martin:2021aa} (now 
part of the AIPlan4EU platform~\cite{Micheli:2025aa}) where the planner produces plans as BT which can be deployed for plan execution.  It
is also interesting to study how BT may also be extended to perform some planning.  In \cite{Colledanchise:2019vf} the authors propose a
dynamic modification of BT for planning (planning with back chaining), so they can perform robust acting, without resorting to
replanning. On another yet different type of planning/BT interaction, in~\cite{Kockemann:2023aa} planning is used to produce testing plans
for BT, whose testing participates to increase the  trust we  put in these BT.

\subsection{BT and formal models}
\label{sec:soabtfm}

This paper main subject is about BT and formal V\&V, so we now examine the state of the art in this area.

In~\cite{Klockner:2013aa} the authors propose to interface BT mission plans and a simulation of the world using the description logic
($\mathcal{ALC(D)}$).  So the description logic formal model acts as a safety check between the plan execution, and the simulation.  Even if
this does not provide a formal proof of the mission plans BT, it improves the trust we can have in these plans by formally validating their
execution (in simulation) before deploying them in the real world.

The authors of \cite{Colledanchise:2017ac} propose to synthesize correct by construction BT from an environment specification along the
agent model and an objective expressed in LTL.  From a standpoint, the approach is clearly sound and synthesizes correct BT, but requires
the programmer to write LTL goal specifications to get started which may be seen as a deterrent to non formal ``programmers''.
 

\vspace{0.5em}

The last four approaches we present here have strong similarities with the one we propose.

In~\cite{Biggar:2020aa} the authors propose to synthesize LTL from  BT and then show that the obtained model satisfies some   LTL
specifications.  The paper goes in depth to explain the translation process, although it is not clear it can be automated, and a priori, the
produced formal model cannot be directly executed in place of a regular BT engine.

The authors of~\cite{Colledanchise:2021ab} focus on the formalisation of the  execution context of BT to be able to perform runtime
verification. They propose \emph{channel systems} to model the ``surroundings'' of the BT and then to check at runtime that some
specifications, written in the SCOPE language, are satisfied while the BT executes. So the approach, which is not limited to BT,  provides a
very strong and formal execution framework sitting between the BT and the robot, behaving like a safety bag. The battery example they present
has some similarity with the one we deploy on our UAV in section~\ref{sec:uav-bt}.

Similarly, the authors of~\cite {Serbinowska:2024ab} focus primarily on runtime verification of BT with contingency monitors (BTM) written with a DSL:
BehaVerify.  These monitors can be used to correct an undesirable behavior when it is detected and can handle LTL specifications. Yet, they
can also check the BT at design time, by checking these BTM with model checking.

In~\cite{Wang:2024aa}, the authors present an approach where they use the BIP formal framework to model BT and propose an implementation of
their tool: \tool{xml2bip}. They then use model checking (not D-Finder as the original BIP implementation did) to check for formal
properties. Although some versions of BIP come with a runtime engine (e.g., the one used in~\cite{Bensalem:2011uf}), they do not yet propose
a ``fornal'' execution of the BT with the BIP engine.

\subsection{Proposed approach}

Our approach has one main goal: to provide a formal semantics for BT, by translating it to a formal model, which can then be used offline to
check formal properties, but also online to implement and enforce this semantics.

We propose to achieve this objective by following these steps.

\begin{itemize}
\item Define a clear complete and unequivocal translation of all BT to a formal model in \fiacre{}~\cite{Berthomieu:2007ab,Berthomieu:2008vo}.
\item The obtained BT formal model can be checked and analyzed to prove logical and temporal properties (LTL and CTL).
\item The \emph{same}  BT formal model can be linked to actual code and executed like other BT framework engines (e.g. \code{Behaviortree.CPP},
  \code{BT.py}) do. This shows that the operational semantic of the BT formal model is the expected one, while guaranteeing the property
  proven offline.
\end{itemize}

Moreover, implementing this approach leads to some interesting side effects and features.  It clarifies the BT semantics when needed, e.g.,
the wait/halt semantics when \running{} nodes must be halted. It also enables time representation extensions and enriches
the BT language with state variables and functions evaluation.

\vspace{1em}

The rest of the paper is organized as follows. After introducing above the BT, the state of the art and our approach, we first present in
Section~\ref{sec:formalfw} the \fiacre{}/\tina{}/\hippo{} formal suite we use.  Section~\ref{sec:btformal} introduces how each BT node is mapped in
\fiacre{} (this is implemented in our \bttf{} tool\footnote{The \bttf{} tool developed for this study can be downloaded from the repository:
  \url{https://redmine.laas.fr/projects/bt2fiacre/pages/index}.}). Then Section ~\ref{sec:deploy} presents how these \fiacre{} BT nodes are
put together to build the complete formal model of the whole BT, and we then show what are the type of formal properties one can prove
offline but also at runtime. Two examples are presented: in Section~\ref{sec:uav-bt}, we introduce a drone controller written in BT for
which we successfully deployed our approach, and; in Section~\ref{sec:nav2}, we show how the Nav2 BT~\cite{ros2nav2:2024aa} can be deployed
with \bttf{}. A discussion in Section~\ref{sec:conclusion} reassesses the pros and cons of the \bttf{} tool and the use of \fiacre{} as an
underlying formal language to provide a formal model and a formal semantics to BT, followed by future work section and the conclusion of the
paper.

\section{A Formal Framework for Offline and Runtime Verification: The \fiacre{} Language, Models, and Tools}
\label{sec:formalfw}

While this paper does not aim to exhaustively present the formal framework we use, some terminology and explanations are essential for
clarity and make the paper self contained. Readers interested in more details may consult the specific papers and websites cited below.
\footnote{Note that a similar \fiacre{} presentation can be found in this paper~\cite{Ingrand:2024aa} (from the same author). We include it
  almost as is in this paper as to make the paper self contained, nevertheless, If the reviewers believe this section should be shortened and
  replaced by a pointer to the other paper, this is perfectly fine for us.}

\subsection{Terminology, Models, Languages, and Tools}

We define the following terms:

\begin{description}

\item[Time Petri Nets] \cite{Berthomieu:1991wv} are an extension of traditional \emph{Petri nets} where each transition has an associated
  time interval (typically $[0,\infty)$) specifying the time range within which an enabled transition can be fired.

\item[TTS] Time Transition Systems extend \emph{Time Petri nets} by adding data-handling capabilities, allowing transitions to invoke data
  processing functions.

\item[\tina{}] (short for "TIme Petri Net Analyzer") is a toolkit for editing, simulating, and analyzing \emph{Petri nets}, \emph{Time Petri
    nets}, and \emph{TTS}. Within this toolkit, \tool{sift} and \tool{selt} enable the construction of reachable state sets and the
  verification of LTL properties.\footnote{\url{https://projects.laas.fr/tina/index.php}}

\item[\fiacre{}] stands for "Intermediate Format for Embedded Distributed Component Architectures" (in French). It is a formally defined
  language designed to represent the behavioral and timing aspects of embedded and distributed systems for purposes of formal verification
  and simulation. \fiacre{} specifications can be compiled into a \emph{TTS} using the \texttt{frac}
  compiler.\footnote{\url{https://projects.laas.fr/fiacre/index.php}}

\item[\hfiacre{}] adds \emph{Event Ports} and \emph{Tasks} linked to C/C++ functions, to make the \fiacre{} models “executable”.
  
\item[\hippo{}] is an engine for executing \emph{TTS} resulting from the \hfiacre{}
  specifications compilation~\citep{Hladik:2021vt}.\footnote{\url{https://projects.laas.fr/hippo/index.php}}

\end{description}

This framework has been applied across various projects and applications,\footnote{\url{https://projects.laas.fr/fiacre/papers.php}}
including the validation and verification of functional components in our robotics experiments~\cite{Dal-Zilio:2023aa}, but also to the
validation and verification of robotic skills programmed in \proskill{}~\cite{Ingrand:2024aa}.

\subsection{\fiacre{} Semantics}
\label{sec:fiacreprocess}

Although the formal model and tools are detailed in specific papers and websites (see above), we include a brief example to illustrate the
semantics of the \fiacre{} language. The example, a triple-click detector for a mouse, is shown in
Listing~\ref{lst:ftct}\code{p}\pageref{lst:ftct}\footnote{All floating Listing and Figures numbers are given with the page number. In this case,
  Listing~\ref{lst:ftct}\code{p}\pageref{lst:ftct} is Listing~\ref{lst:ftct}, page~\pageref{lst:ftct}.} and illustrated in
Figure~\ref{fig:tcd-tina}. It defines three \fiacre{} processes, each represented by an automaton. The first process, \process{clicker},
generates a \event{click} at any time, waiting between $0$ and $\infty$, then synchronizes on the \port{click} port with
\process{detect\_triple\_click}. This second process has four states, waiting for synchronization on \port{click} or until the maximum
allowed time between clicks (\qty{0.2}{\sec}) has passed. Note the \code{select} option in \state{wait\_second} and \state{wait\_third}
states, introducing a non-deterministic choice for exploration by the model checker. Upon reaching \state{detected}, a synchronization on
\port{triple\_click} enables the transition of the \process{triple\_click\_receiver} process to \state{received\_tc}.

Following these specifications, a component is defined by placing three process instances in parallel (line~\ref{lst:ftct}.\ref{ll:par})\footnote{Listing lines are referenced with the $<$listing
  number$>$.$<$line number$>$, example:~\ref{lst:ftct}.\ref{ll:par} is Listing~\ref{lst:ftct}, line:~\ref{ll:par}.} and
linking them through two ports (line~\ref{lst:ftct}.\ref{ll:ports}). This example is simple by design, though the \fiacre{} language
supports complex data types, bidirectional ports, local and global variables, conditions, switch/case statements, transition guards, and
function calls (internal to \fiacre{} or external in C/C++ code) for advanced computation. More complex \fiacre{} specifications can be
found in \ref{app:app1} and \ref{app:app2}.

\begin{lstlisting}[caption={\fiacre{} specification for a triple click detector (\fiacre{} offline version).}, numbers=left, xleftmargin=15pt, label={lst:ftct}, language=fiacre]
process clicker [click:sync] is // synthesize clicks and sync them on its port at any time
states wait_click, make_click

from wait_click
   wait [0, ...[; // wait any time from zero to infinity
   to make_click

from make_click
   click; // issue a click sync on the Fiacre port
   to wait_click

process detect_triple_click [click:sync,triple_click:sync] is
states wait_first, wait_second, wait_third, detected

from wait_first
   click;         // first click  (*@ \label{ll:click1} @*) 
   to wait_second

from wait_second
   select        // we wait either
     wait [0.2,0.2]; //  exactly 0.2 second
     to wait_first // then reset the detector
   []
     click;       // or for the second click   (*@ \label{ll:click2} @*) 
     to wait_third // whichever comes first
   end

from wait_third
   select        // again for the third click
     wait [0.2,0.2];
     to wait_first
   []
     click;       // third  (*@ \label{ll:click3} @*) 
     to detected
   end

from detected
   triple_click; // sync on the triple_click port
   to wait_first

process triple_click_receiver[triple_click:sync] is
states waiting_tc, received_tc

from waiting_tc
   triple_click; // just wait for a sync on this port
   to received_tc

from received_tc
   /* do what needs to be done when a TC has been detected */
   to waiting_tc

component comp_tc is //we now specifiy the component

port click:sync in [0,0], triple_click:sync in [0,0] // two ports  (*@ \label{ll:ports} @*) 

par * in // 3 processes composed in parallel  (*@ \label{ll:par} @*) 
   detect_triple_click[click, triple_click] // process 1
|| clicker[click]   // process 2
|| triple_click_receiver[triple_click] //process 3
end

comp_tc // this instantiates the component

// some properties to check
property ddlf is deadlockfree  // deadlock free (TRUE) (*@ \label{ll:ddf} @*) 
assert ddlf
// in the next property comp_tc/3/state designates the state in the 3rd process of
property cannot_receveice_tc is absent comp_tc/3/state received_tc // the comp_tc component (*@ \label{ll:rtc} @*) 
assert cannot_receveice_tc // we cannot detect a triple click (FALSE)
\end{lstlisting}

\begin{figure*}[!ht]
\begin{center}
\includegraphics[width=0.75\textwidth]{./tcd-tina}
\caption{The \fiacre{} processes modeling the \fiacre{} specification on Listing~\ref{lst:ftct}\code{p}\pageref{lst:ftct}.}
\label{fig:tcd-tina}
\end{center}
\end{figure*}

\subsection{Offline Formal Verification}

The \tool{frac} compiler is used to compile the \fiacre{} specifications shown in Listing~\ref{lst:ftct}\code{p}\pageref{lst:ftct} into an equivalent TTS. With the
\tool{sift} tool from the \tina{} toolbox, one can then construct the system's set of reachable states, and by using \tool{selt}, the
properties outlined in the initial \fiacre{} specifications, as well as any additional properties, can be verified. The \tina{} toolbox
offers a variety of other tools that readers may explore for further analysis.

Listing~\ref{lst:ftct}\code{p}\pageref{lst:ftct} suggests several properties to check for this specification: Is the model deadlock-free
(line~\ref{lst:ftct}.\ref{ll:ddf})? This is confirmed to be TRUE. Can the model successfully detect a triple click
(line~\ref{lst:ftct}.\ref{ll:rtc})? By verifying that reaching the \state{received\_tc} state is possible, the model confirms that it can
indeed detect a triple click. Additional complex properties could be added, such as ensuring there is at most \qty{0.4}{\sec} between the
first and last clicks.

The verification approach used by the \tina{} tools relies on model checking, which can be affected by state explosion~\cite{Clarke:2012uv},
potentially limiting its effectiveness. However, as demonstrated in section~\ref{sec:results}, the results from our example remain both
insightful and non-trivial.

\subsection{\hfiacre{} Runtime Extensions}

Although \fiacre{} was originally designed for offline verification, it has been extended with two primitives that enable runtime
verification~\citep{Hladik:2021vt}. These extensions allow the model to connect with C/C++ functions that send events or execute commands,
forming what we call \hfiacre{}, to distinguish it from the base \fiacre{} language.

The purpose of the \hfiacre{} runtime version is to make the model "executable" in connection with real-world interactions.

Listing~\ref{lst:ftch}\code{p}\pageref{lst:ftch} (along with Figure~\ref{fig:tcd-hippo}\code{p}\pageref{fig:tcd-hippo}) presents the executable version of the specification given in Listing~\ref{lst:ftct}\code{p}\pageref{lst:ftct}.

\begin{description}
\item[Event Ports] are defined in the specification's preamble (see line~\ref{lst:ftch}.\ref{ll:event_port}), linking a C function to a
  \fiacre{} port. In this case, the event \event{click} is linked to the \code{c\_click} function in C/C++. When this port is one of the
  possible transitions (lines~\ref{lst:ftct}.\ref{ll:click1},~\ref{lst:ftct}.\ref{ll:click2}, and~\ref{lst:ftct}.\ref{ll:click3}), the C/C++
  function is called, and the port becomes active upon the function's return. These C/C++ functions can accept and return values typed in
  \fiacre{}. 
\item[Tasks] are also defined in the preamble (see line~\ref{lst:ftch}.\ref{ll:task}), associating a task (in this case,
  \task{report\_triplec}) with a C/C++ function (here \code{c\_report\_triple\_click}), which is called asynchronously upon a \code{start}
  (see line~\ref{lst:ftch}.\ref{ll:start}). This enables the corresponding \code{sync} (line~\ref{lst:ftch}.\ref{ll:sync}) once the C/C++
  function completes. Values can be passed to the task at call time and returned when it completes. 
\end{description}

\begin{lstlisting}[caption={\hfiacre{} processes implementing a triple click detector.}, numbers=left, xleftmargin=15pt, label={lst:ftch}, language=fiacre]
event click : sync is c_click // declare the Fiacre event port which transmits click (*@ \label{ll:event_port} @*) 
task report_triplec () : nat is // declare the task and 
	c_report_triple_click  // the C/C++ function called by this task  (*@ \label{ll:task} @*) 

process detect_triple_click [triple_click:sync] is 
// this process is exactly the same than in the regular Fiacre version
// only the click port is now an event port

process triple_click_receiver[triple_click:sync] is
states waiting_tc, received_tc, sync_report
var ignore : nat

from waiting_tc
   triple_click;
   to received_tc

from received_tc // show an example of an external call
   start report_triplec();  (*@ \label{ll:start} @*) 
   to sync_report

from sync_report
   sync report_triplec ignore;  // wait until the call return (*@ \label{ll:sync} @*) 
   to waiting_tc

component comp_tc is
port triple_click:sync

par * in
   detect_triple_click[triple_click]
|| triple_click_receiver[triple_click]
end

comp_tc
\end{lstlisting}

In this example, we replace the \process{clicker} process, which previously synchronized with \port{click} at any moment, with the
\event{click} event port (highlighted in purple). Additionally, we introduce a task (\task{report\_triplec} shown in light blue) to execute
when synchronizing with a \event{triple\_click} in the \process{triple\_click\_receiver} process. The remainder of the model remains
unchanged, transforming it from a model specifying a triple-click detector to an actual program or controller that implements it. In this
way, the specification itself becomes an executable program.

\begin{figure*}[!ht]
\begin{center}
\includegraphics[width=0.97\textwidth]{./tcd-hippo}
\caption{Illustration of the \hfiacre{} program on Listing~\ref{lst:ftch}\code{p}\pageref{lst:ftch}.}
\label{fig:tcd-hippo}
\end{center}
\end{figure*}

\subsection{Runtime (Online) Verification}
\label{sec:rv}

The \hfiacre{} model, once compiled into TTS format using \tool{frac}, is linked with the \hippo{} engine and the C/C++ functions needed to
run it (e.g., \code{c\_click} and \code{c\_report\_triple\_click}). The \hippo{} engine executes the TTS model in real-time and initiates
the appropriate C/C++ function calls (in separate threads) connected to event ports and tasks. Note that the properties checked offline also
hold in the online version (as the reachable states set of the former include the one form the latter). 

Additionally, the model can be extended with runtime verification properties by incorporating a monitoring process. For instance, if this
controller is applied in a video game where rapid sequences of triple clicks are suspicious (indicating potential use of a cheating device),
a new \process{cheat\_detector} process could be added (shown on the left in Figure~\ref{fig:tcd-hippo}\code{p}\pageref{fig:tcd-hippo} and in Listing~\ref{lst:fcd}\code{p}\pageref{lst:fcd}). This
process could have three states and use a shared Boolean variable, \code{cheat}. This variable would be set to `false` upon transitioning to
the \state{waiting\_second} state in the \process{detect\_triple\_click} process, and switched to `true` upon reaching the \state{detected}
state if the click sequence timing suggests non-human activity.


\begin{lstlisting}[caption={\process{cheat\_detector} process detecting a cheating device by monitoring the \code{cheat} Boolean variable.}, numbers=left, xleftmargin=15pt, label={lst:fcd}, language=fiacre]
process cheat_detector(&cheat:bool) is

states state1, state2, cheat_detected

from state1
  on (not cheat); // guard on (not cheat)
  to state2
  
from state2 // cheat was set to false
  select // either 
     wait [0.05,0.05]; // 50 ms elapsed
     cheat := true; // reset the cheat variable
     to state1 // go back to monitoring
  []
     on (cheat); // cheat became true again before the 50ms above.
     to cheat_detected //caught cheating
  end

from cheat_detected
  // the player is cheating, do what needs to be done.
  to state1
\end{lstlisting}

Within the \process{cheat\_detector} process, in \state{state1}, the system sets a guard on \code{(not cheat)} before transitioning to
\state{state2}, where it then waits for either \qty{50}{\ms} or until \code{cheat} becomes true. If \code{cheat} becomes true before
\qty{50}{\ms} has elapsed (indicating a suspiciously fast triple-click), it transitions to \state{cheat\_detected} and flags this unusual
activity. If \qty{50}{\ms} passes without the \code{cheat} variable being set to true, the system sets \code{cheat} to true and returns to
\state{state1}.

This approach allows us to synthesize a controller that directly runs the specification. This dual capability is a major strength of the
\fiacre{} framework: the same formal model can be verified offline and executed online. In practical terms, this means that the controller’s
real-time behavior aligns with the initial model specifications, validating that the offline-verifiable properties are applied consistently
in the live system. While observing expected behavior is a necessary, though not entirely sufficient, indicator of correctness, it
significantly strengthens the link between specification and execution.

Moreover, if runtime behavior diverges from expectations, you can debug it as you would with any programs. From a formal perspective, the possible
traces of the \hfiacre{} version (also called the \hippo{} version) are contained within those of the \fiacre{} model
(also called the \tina{} version), ensuring consistency between the
runtime model and its  offline counterpart.

\section{The mapping of BT in \fiacre{}}
\label{sec:btformal}

Before getting into the details of the produced formal models, we present on Figure~\ref{fig:workflow}\code{p}\pageref{fig:workflow} the
overall workflow from BT to the formal executable version (top part in green), and the formal verifiable version and its analyzed properties
report (bottom part in purple).  One should keep in mind, that the BT programmers only provide the various BT in \code{.btf} format (like
the one on Listing~\ref{lst:dronebt}\code{p}\pageref{lst:dronebt}), the C/C++ codes which glue \btn{Action} and \btn{Condition} BT to the
real robot commands and perception primitives (all in blue) and, optionally, LTL properties to verify, and monitors written in \fiacre{} (in
slanted blue). The rest is fully synthesized and automatically compiled.

\begin{figure*}[!ht]
\begin{center}
\includegraphics[width=0.97\textwidth]{./workflow}
\caption{The \bttf{} (\hippo{}/\tina{}) workflow. Only the data in the blue boxes need to be provided by the programmer. The \bttf{} tool (in light
  red) synthesizes the two models, the rest is fully automated.  In light green, the workflow for the \hippo{} runtime verification
  version, and in light purple, the workflow for the \tina{} offline verification version.}
\label{fig:workflow}
\end{center} 
\end{figure*}

\begin{lstlisting}[caption={A simple  drone survey  BT. Note that for historical reasons and as we are reusing some of our existing tools, we do not use an XML syntax but rather a Lisp like syntax (called the \code{.btf} format), but this has no consequence on the content and the interpretation of BT (See Appendix~\ref{app:ros2nav2}, Listing~\ref{lst:ros2nav2xml}\code{p}\pageref{lst:ros2nav2xml} and~\ref{lst:ros2nav2btf}\code{p}\pageref{lst:ros2nav2btf} for an example in both formats).}, numbers=left, xleftmargin=15pt, label={lst:dronebt}, language={[btf]Lisp}]
((BehaviorTree :name drone
  (Sequence
   (ParallelAll :wait 1 :halt 0 ; if wait is 1, will wait the running branch if one fails
    (Action :ID start_drone)
    (Action :ID start_camera))
   (ReactiveSequence
    (Fallback
     (Condition :ID battery_ok)  ; check if the battery is OK
     (ForceFailure :ID fail  ; if not, just land and fail
      (Action :ID land)))
    (Fallback
     (Condition :ID localization_ok) ;same for localization
     (ForceFailure :ID fail
      (Action :ID land)))
    (Sequence
     (Action :ID takeoff :args (height 1.0 duration 0)) 
     (Parallel :success 1 :wait 0 :halt 1 ; If the survey or the nav succeed, we are done.
      (Action :ID camera_survey)
      (Sequence
       (Action :ID goto_waypoint :args (x -3 y -3 z 5)) 
       (Action :ID goto_waypoint :args (x -1.5 y 3 z 5))
       (Action :ID goto_waypoint :args (x 0 y -3 z 5))
       (Action :ID goto_waypoint :args (x 1.5 y 3 z 5))
       (Action :ID goto_waypoint :args (x 3 y -3 z 5))
       (Action :ID goto_waypoint :args (x 3 y -3 z 5))))
     (Action :ID goto_waypoint :args (x 0 y 0 z 5))
     (Action :ID land)
     (Action :ID shutdown_drone))))))
\end{lstlisting}

As mentioned above, the semantics of BT is not formally defined. By translating BT to \fiacre{}, we define a formal semantics, hopefully
consistent with the operational semantics people expect from BT.

Another goal we pursue is to model as much BT ``additional information'' as possible in the \fiacre{} model (for example if the variable used in
BT can be modelled in \fiacre{}, the better). 

We shall first present the overall mapping and then we will point out where the semantics had to be clarified and what are the
``additional'' information we gather in the \fiacre{} model.

\subsection{The general mapping}

Each BT node is automatically mapped in a \fiacre{} process (See Section~\ref{sec:fiacreprocess}), whose automata is modeled in accordance to its node type (\btn{control},
\btn{decorator}). Then, all these \fiacre{} processes are instantiated and composed in a component which provides a shared array of \fiacre{}
records. This array \code{BTnode[]} has one element for each BT and is a shared variable between all the \fiacre{} processes. In the following figures \code{BTnode[self]} is
the record for the current BT node. Each \code{BTnode[]} record has two fields of interest: \code{caller} and \code{rstatus}. \code{caller} is
initialized at \code{None} and will be set to the BT node which ticks/calls it. It is set back to \code{None} when its execution (for the
current tick) is finished. \code{rstatus} contains the last returned status (\success{}, \failure{} or \running) of the BT
node.  \code{rstatus} may also be set to \code{halt\_me} (by its caller), when a \running{} BT node must be halted.

\begin{figure*}[!ht]
\begin{center}
\includegraphics[width=0.7\textwidth]{./node}
\caption{Preamble and postamble of BT nodes in \fiacre{}.}
\label{fig:node}
\end{center}
\end{figure*}

As shown on Figure~\ref{fig:node}\code{p}\pageref{fig:node}, a BT node \fiacre{} process %, except the one modelling the root of the BT,
starts with a preamble (on Listing~\ref{lst:factionh}\code{p}\pageref{lst:factionh} or~\ref{lst:fsequenceh}\code{p}\pageref{lst:fsequenceh}
from \state{start\_} to \state{tick\_node}) with a guard on its activation on \code{BTnode[self].caller} (i.e. presumably its parent node
has ticked its by setting its \code{caller} field). Follow a test to check if it has been asked to halt (in case it had previously returned
\running{} and now its parent asks it to halt). If it needs to run, it transitions to the \state{tick\_node} state where the particular of
this node type is handled.

Similarly, the postamble mostly consists of the four automata states (\success{}, \failure{}, \running{} and \code{halted}), all returning the proper
return status \code{rstatus} and then transitioning to a \state{done} state, in which the control is relinquished by setting \code{caller}
to \code{None} (the parent node has a guard on this to check that the node has finished this tick).

In the following \btn{Action} and \btn{Condition} nodes in \fiacre{} are presented in their \tina{} version (i.e., the offline verification) but also \hippo{}
version, (i.e. the runtime version). All other BT nodes \fiacre{} models are strictly the same between the two versions.

\subsection{\btn{Condition} Node}
\label{sec:btncondition}

The \btn{Condition} node is the simplest node, it only returns \success{} or \failure. The \tina{} version will just return either values
(See Figure~\ref{fig:conditiont}\code{p}\pageref{fig:conditiont}), while the \hippo{} version will call the \emph{external} C/C++ function
which performs the test and return its value (See Figure~\ref{fig:conditionh}\code{p}\pageref{fig:conditionh}). The C/C++ function is
expected to be fast and should not delay execution unnecessarily. Note that in our formalism, one can pass arguments to the C/C++ call. This
is very much welcome to make the BT language more expressive and to some extent to have these values available in the formal model.

\begin{figure*}[!ht]
\begin{center}
\includegraphics[width=0.7\textwidth]{./conditiont}
\caption{The \fiacre{} process modeling the \btn{Condition} node for the verification (\tina{}) model.}
\label{fig:conditiont}
\end{center}
\end{figure*}

\begin{figure*}[!ht]
\begin{center}
\includegraphics[width=0.7\textwidth]{./conditionh}
\caption{The \fiacre{} process modeling the \btn{Condition} node for the runtime (\hippo{}) model.}
\label{fig:conditionh}
\end{center}
\end{figure*}

\subsection{\btn{Action} Node}
\label{sec:btnaction}

The \btn{Action} node is slightly more complex, as it can also return \running{}. Indeed, actions may take some time and not return
\success{} or \failure{} right away. To preserve the reactivity of the overall execution, and keep the tick short, it is often advised to
return \running{} while the action is still executing in its own thread. As a consequence, it means that an \bt{Action} node may be \emph{halted}
(i.e. at some point, it is in a \running{} state but its parent node wants to halt it). In the \tina{} version (See
Figure~\ref{fig:actiont}\code{p}\pageref{fig:actiont}), all the return values are possible, while in the \hippo{} version (See Figure~\ref{fig:actionh}\code{p}\pageref{fig:actionh}), a \fiacre{}
\emph{task} is started calling the C/C++ \code{action\_task}, and subsequent ticks will return \running{} until the C/C++ \code{action\_task} finishes
and returns \success{} or \failure{}. An \emph{external} C/C++ \code{action\_halt} is also defined and is called if the action must be
halted (it then returns \failure{}). While the \code{action\_task} may take some time in its own thread, the \code{action\_halt} is expected to
be fast. In an \btn{Action} node too, one may define and pass some arguments to the C/C++ function.  Listing~\ref{lst:factionh}\code{p}\pageref{lst:factionh} in
Appendix~\ref{app:app1} shows the \fiacre{} code for the \btn{takeoff} action.

\begin{figure*}[!ht]
\begin{center}
\includegraphics[width=0.7\textwidth]{./actiont}
\caption{The \fiacre{} process modeling the \btn{Action} node for the verification (\tina{}) model.}
\label{fig:actiont}
\end{center}
\end{figure*}

\begin{figure*}[!ht]
\begin{center}
\includegraphics[width=0.7\textwidth]{./actionh}
\caption{The \fiacre{} process modeling the \btn{Action} node for the runtime (\hippo{}) model.}
\label{fig:actionh}
\end{center}
\end{figure*}

Note that only the \btn{Action} and \btn{Condition} nodes have a slightly different \tina{} (offline) and \hippo{} (online) versions. For all the other BT
nodes, the model is strictly the same.

\subsection{\btn{Sequence} Nodes}

The \btn{Sequence} node ticks/executes its child in sequence as they succeed, until one fails. It then returns \failure{}. If all succeed, it
returns \success{}. If one child returns \running{},  the \btn{Sequence} returns \running{}. Upon return of the tick this particular child is ticked/called again.

Listing~\ref{lst:fsequenceh}\code{p}\pageref{lst:fsequenceh}, shows the complete \fiacre{} code of a \btn{Sequence} node with three children. The \fiacre{} process has a local variable
\code{next\_seq} initialized at 1, which holds which node will be ticked/called when the current node is ticked/called again. From the
\state{Tick Node} state (See Figure~\ref{fig:sequence}\code{p}\pageref{fig:sequence}), one proceeds to the \state{$Child_{\texttt{next\_seq}}$} state, which ticks/calls the proper
child. We wait until this child is done \code{caller=None} and check its returned status \code{rstatus}:

\begin{itemize}
\item \success{}, if it is the last child to succeed, we return \success{}, otherwise, we proceed to the next child
  $Child_{\texttt{next\_seq+1}}$.

\item \failure{}, we return \failure{}.

\item \running{}, we set \code{next\_seq} to the proper value and return \running{}.

\end{itemize}

There exist variants of \btn{Sequence:} \btn{ReactiveSequence}, \btn{SequenceWithMemory}. They alter the way the sequence is ticked after
\failure{} or \running{} are returned. For example \btn{ReactiveSequence} always restarts the sequence from the beginning after one child
has returned \running{}. The goal here is not to list all the variants, just to give a hint on how these are transformed in \fiacre{}. Of
course, the generated \fiacre{} code implements the proper semantics for each variant by properly setting the \code{next\_seq} value.

\begin{figure*}[!ht]
\begin{center}
\includegraphics[width=1\textwidth]{./sequence}
\caption{The \fiacre{} process modeling the \btn{Sequence} node.}
\label{fig:sequence}
\end{center}
\end{figure*}

\subsection{\btn{Fallback} Nodes}

The \btn{Fallback} node executes/ticks its child in sequence as they fail, until one succeeds. It then returns \success{}. If all fail, it
returns \failure{}. If one child returns \running{}, it returns \running{}. Upon return of the tick this child is again ticked/called. As
shown on the Figure~\ref{fig:fallback}\code{p}\pageref{fig:fallback}, it is the symmetric of the \btn{Sequence} node, thus its behavior is just the mirror of the
\btn{Sequence} one described above. There is a variant of \btn{Fallback}: \btn{ReactiveFallback}. It alters the way the \btn{Fallback} is
restarted/reticked after \running{} is returned. \btn{ReactiveFallback} always restarts from the beginning after one child has returned
\running{}. This is achieved in \fiacre{} by properly setting the \code{next\_fb} value.


\subsection{\btn{Parallel} Nodes}

The \btn{Parallel} node specifies how many children \code{m} out of all \code{n} children must succeed for the \btn{Parallel} node to
succeed (with \btn{ParallelAll} they must all succeed: \code{n = m}). From this, we see that the implementation just ticks all
the node, and then keep track of how many return \failure{}, \success{} or \running{} (See Figure~\ref{fig:parallel}\code{p}\pageref{fig:parallel}). If any final
condition for \success{} or \failure{} is met (enough \code{BTnode[child1].rstatus = \success{}})\footnote{(enough... \success{}) is true
  when \code{\success{} >= m}} or (enough \code{BTnode[childn].rstatus = \failure{}})\footnote{(enough...  \failure{}) is true when
  \code{\failure{} > n - m}}, then it proceeds to the corresponding state, otherwise, this node
returns \running{}.  Note that \btn{Parallel} nodes can lead to children being halted (if the condition for \success{} or \failure{} are met
while some children are still \running{}).

\subsection{\btn{Decorator} Nodes}

\btn{Decorator} nodes have only one child. When this child is \state{done} the transitions to \success{}, \failure{} or \running{} depend on
the particular \btn{decorator} type (\btn{Inverter}, \btn{ForceFailure}, \btn{ForceSuccess}, \btn{KeepRunningUntilFailure},
\btn{RetryUntilSuccessful}, \btn{RateController}, \btn{Repeat}, etc). For example, the \btn{Inverter} (See Figure~\ref{fig:decorator}\code{p}\pageref{fig:decorator}) one
will swap the \success{} and \failure{} transitions, while leaving untouched the \running{} one, \btn{Repeat} will call its child a number
of times, etc.

\begin{figure*}[!ht]
\begin{center}
\includegraphics[width=1\textwidth]{./decorator}
\caption{The \fiacre{} process modeling a \btn{Decorator} (\btn{Inverter}) node.}
\label{fig:decorator}
\end{center}
\end{figure*}

\subsection{Semantics clarification and added Features}

As mentioned earlier, the translation to \fiacre{} defines a formal semantics of BT. But there are particular choices which need to be
clarified and specified. Many of these choices have already been identified~\cite{Ghiorzi:2024aa}, we try here to settle them with a formal
proposition.

\subsubsection{Halting running BT nodes}

We have seen there are a number of situations where one may have to handle ``orphan'' running branches. They are not really orphans, but
the situation is such that their parents have already decided the \success{} or \failure{}, independently of their own final outcome other than
\running{}.  This happens when a \btn{Parallel} fails (or succeeds) and there are still \running{} children, similarly when a
\btn{ReactiveSequence} fails or a \btn{ReactiveFallback} succeeds, one child may still be \running{}. So for these nodes we added 2
keywords \code{:wait} and \code{:halt} to their specification. If \code{:halt} is true, we explicitly halt the still \running{}
children. Note that this is propagated all the way to leaves nodes, in particular \btn{Action} nodes will explicitly halt whatever they are
doing by calling their \code{action\_halt} C/C++ function (see Section~\ref{sec:btnaction}). If \code{:wait} is true, the node waits until
all the children return something else than \running{}. These choices can be discussed and modified, but at least, the semantics is clarified,
defined and implemented in the formal model.

\subsubsection{Ticks and BT root}
\label{sec:tick}

\fiacre{} supports time, and so does the produced TTS and the \tina{} verification tools. We consider that a BT tick takes one \fiacre{}
unit of time\footnote{In the regular \fiacre{} verification language, time is ``unitless'', but in the \hippo{} engine, we set the ``tick''
  (\qty{100}{ms} in the experiments described in this paper).}. By default, only the BT root generates ticks, one after another,
independently of ``how long'' the tick traversal took. So the default BT model produced in \fiacre{} has just one transition $[1,1]$ in the
root BT, and all other transitions are timeless: $[0,0]$. This is a perfectly ``fine'' assumption if one considers the tick more as an
execution ``token'' traversing the BT in no time, but if one wants to perform more advanced validation and verification, it may be a good
idea to propose different tick semantics. For now we propose two other semantics, one is to have all the \btn{Action} and \btn{Condition} BT
nodes to have a $[1,1]$ transition on their \state{tick\_node} state, and the other one is to have all BT nodes with such $[1,1]$ transition.  Of
course, the chosen semantics among the three possibilities can be specified when producing the \fiacre{}
model.  
Note that from a temporal model checking point of view, the last semantics is preferred, as it minimizes simultaneous transitions
interleaving, hence the branching factor to the reachable states exploration.

The BT root is also responsible for keeping the BT alive. What should it do when its child returns \running{}, \success{} or \failure{}?  Some
implementations keep running on \success{}, others do not. To clarify the model we produce, the root BT keeps \running{} while its child returns
\running{}, and stops when it returns \success{} or \failure{} (e.g., in the drone experiment we present in Section~\ref{sec:uav-bt}, this is
the expected behavior).

\subsubsection{State Variables, Expression Evaluation and Node Status}

BT tend to rely on ``external'' \btn{Actions} and \btn{Conditions} to compute values, tests them, etc. One of our objectives is to get as much as possible of the BT and
its associated \btn{Condition} and \btn{Action} nodes modelled in \fiacre{}. The more we get, the more properties we can prove on the reachable state of the
BT, and the more we control execution at runtime.

Hence we introduce \emph{state variables} which can be used in the BT model and will end up in the \fiacre{} model as \fiacre{} variables.

For now, state variables can hold a natural number or an enumeration. See the example below with \sv{flight\_level} which can take an
\code{int} between 0 and 3, or \sv{battery} which can take three values (\svv{Good}, \svv{Low} and \svv{Critical}) (note that for an
enumeration, one can specify the acceptable transitions from one value to others).

We also added two new leaves nodes to the BT specification:
\begin{description}
\item [\btn{SetSV}] nodes can be used to set a state variable passed with the :SV keyword, by calling the :ID function (defined in
  \fiacre{}). \btn{SetSV} nodes only return \success{}. 
\item[\btn{Eval}] nodes evaluate the condition they hold and return \success{} or \failure{}.
\end{description}

Last, we also make available for evaluation the resulting status of any BT, e.g.  \code{(Eval (= camera\_track.rstatus success))} will return
\success{} if the execution of the \code{camera\_track} \btn{Action} node was a \success{}.

\section{Deploying, Model Checking and Running the \fiacre{} BT}
\label{sec:deploy}

The final \fiacre{} model of the BT is produced by instantiating all the BT nodes \fiacre{} processes and combining them in parallel.

 %\subsection{The \tool{frac} compilation}

 As shown on Figure~\ref{fig:workflow}\code{p}\pageref{fig:workflow} both \fiacre{} models (offline and online) are compiled in their
 respective TTS (i.e., a Time Petri Net with data) with the \code{frac} compiler.

\subsection{Offline  formal verification of  BT}
\label{sec:ofv}
\label{sec:ltlprop}


The \tina{} toolbox used here mostly rely on LTL (SE-LTL state/event LTL~\cite{Chaki:2004aa} to be more accurate). LTL already offers a rich
language and a lot of flexibility to write and prove some logical and temporal properties on the formal BT. Using model checking, one can
either check properties on the fly (reachability of a state), or show the absence of a state (but for this, the whole reachable state set has to
be built).  By default, the \tool{sift} tool computes the reachable states set of the BT. The resulting \code{.ktz} file (binary kripke
transition system) can then be analyzed with default or additional properties with the \code{selt} tools\footnote{The \tina{} toolbox
  provides many tools, we just focus here on \code{selt} and \code{sift}.}.

For each BT node, we define and check some default properties:  can the BT execute and complete,
can it succeed, fail, or return running? Can it be halted?  Most of these properties correspond to either a particular \state{state} in the fiacre
model, or to some explicit value in the \code{btnode[self].rstatus} record. So the properties can be synthesized automatically as follow
(by checking if their negation is reachable, i.e. we try to show that the state cannot be reached, and expect \code{selt} to return a
counter example):

\begin{lstlisting}[language=LTL]
// For the  Action_takeoff btnode
prove absent drone/12/state done 
prove absent drone/12/value (btnode[takeoff].rstatus=success)
prove absent drone/12/value (btnode[takeoff].rstatus=failure) 
prove absent drone/12/state halted 
// the 12 is the index of the Action_takeoff btnode process instance in all the processes 
// combined in parallel to build the drone component
\end{lstlisting}

One can also check more complex properties. For example, if we reach a particular state, then we will eventually reach another one. 

We also modelled some examples found in some papers presented in Section~\ref{sec:soa}.

In \cite{Biggar:2020aa} the authors present an example of a Mars rover using unfolded solar panels to charge its battery, but should fold
them when there is a storm. The same example is studied in ~\cite{Wang:2024aa} with the BT transformed to BIP on which they  check the
same properties. The equivalent BT in the \code{.btf} formalism can be found in Appendix~\ref{app:mars}, Listing~\ref{lst:mars}\code{p}\pageref{lst:mars}. After
translation to \fiacre{}, we can build the set of reachable states:
\begin{lstlisting}[language=bash]
sift -stats mars_rover.tts -rsd mars_rover.tts/mars_rover.ktz  
20783 classe(s), 20275 marking(s), 72 domain(s), 103111 transition(s)
0.520s
\end{lstlisting}
and to prove the property  (check that the robot cannot be hit by a storm while its  solar panels are unfolded):
\begin{lstlisting}[language=LTL]
prove absent (mars_rover/3/state Unfolded and mars_rover/1/state Storm)
\end{lstlisting}
we get (as they do):
\begin{lstlisting}[language=fiacre]
never (sv__panel__automata_1_sUnfolded /\ sv__meteo__automata_1_sStorm) 
FALSE
\end{lstlisting}

\code{selt} finds a counter example, i.e., a state where this can happen (which is a problem and need to be fixed).

Similarly, in~\cite{Wang:2024aa} they develop an example with a train having a non null speed while a door is open, we modelled it and reach the same 
formal proof results.

\subsection{Online (or runtime) verification}

The formal executable model is obtained by producing the TTS with the \code{frac} compiler, and then linking the result with the \hippo{}
engine library and the C/C++ code implementing the \btn{Action} and \btn{Condition} \fiacre{} tasks and \fiacre{} externals
(Section~\ref{sec:btnaction} and~\ref{sec:btncondition}). In most case these C/C++ code use the client library which allows to call actions,
services or check topics on ROS nodes~\cite{Quigley:2009tg}; or to make requests to or read ports from \GenoM{} modules~\cite{Dal-Zilio:2023aa}.

As mentioned earlier, our translation from \code{.btf} BT to \fiacre{} defines the formal semantics of the language. We have seen above that we
can prove properties on this formal model. But executing the formal model with \hippo{} is also a way to ``validate'' that the formal
semantics we defined is correct with respect to the operational semantics one expects from BT. In other words, the execution of the formal
model produces what is expected by the original BT programmer, as if he/she were using a regular BT execution engine.

As seen on Figures~\ref{fig:screen-bt}\code{p}\pageref{fig:screen-bt}
and~\ref{fig:archi-uav}\code{p}\pageref{fig:archi-uav}, %and~\ref{fig:screen}\code{p}\pageref{fig:screen},
our BT show changing color while executing: white, the node has not yet been visited; dark blue, the tick is currently in this node; light
blue, the tick has been passed to child(ren) node(s) below and the node is waiting for the children to return; yellow, the node has returned
\running{}; green, the node has returned \success{}; red, the node has returned \failure{}; purple, the node was running and has been halted
by one of its parent nodes; pink, the node has been instructed to halt itself (and its running branches).
  
The white square node close to the root of the BT indicates the \hippo{} tick %, which is set at \qty{10}{Hz} in this experiment, and
which is also the BT tick.

\section{Some illustrating examples}
\label{sec:examples}

We now illustrate our approach with two examples, a drone controller and the Nav2 navigation ROS2 stack.

\subsection{An UAV surveying an area}
\label{sec:uav-bt}

We program an UAV to perform a survey mission with a BT. The functional layer of this experiment (Figure~\ref{fig:archi-uav}\code{p}\pageref{fig:archi-uav}) has already
been presented in~\cite{Dal-Zilio:2023aa}, but suffices to say that it provides robust localization, navigation, flight control and allows us
to command the drone and its camera. It is deployed using the \GenoM{} specification language (which also maps in the same  formal framework to validate and verify
the functional components~\cite{Dal-Zilio:2023aa}).

\begin{figure*}[!ht]
\begin{center}
\centerline{\includegraphics[width=1.1\textwidth]{./archi-uav}}
\caption{Architecture of the drone experiment. On the left, all the functional components involved, on the right a Behavior Tree in charge
  of the overall mission.}
\label{fig:archi-uav}
\end{center}
\end{figure*}

Eight functional components are deployed for this experiment. The set of primitive commands available for the \emph{acting} component are:
\bt{start\_drone}, \bt{start\_camera}, \bt{shutdown\_drone}, \bt{takeoff}, \bt{land}, \bt{goto\_waypoint}, and \bt{camera\_survey}. Each of
them has a corresponding \btn{Action} node in the larger BT \bt{drone} on listing~\ref{lst:drone3bt}\code{p}\pageref{lst:drone3bt} (e.g,
\bt{takeoff} (line~\ref{ll:takeoff}), \bt{goto\_waypoint} (lines~\ref{ll:gw1}-\ref{ll:gw2}), etc),

The state variables handled by the model are \sv{battery}: \svv{Good}, \svv{Low} and \svv{Critical} (line~\ref{ll:battery}); and
\sv{flight\_level}: a natural number between 0 and 3 (line~\ref{ll:fls}).

\begin{lstlisting}[caption={A more complex drone survey  BT in the \code{.btf} format, with \emph{state variables} declaration, \btn{SetSV}
and \btn{Eval} nodes. The $:=$ is the assignment operator, and $\sim$ is the logical negation (See also Figure~\ref{fig:bt-drone3}\code{p}\pageref{fig:bt-drone3}).}, numbers=left, xleftmargin=15pt, label={lst:drone3bt}, language={[btf]Lisp}]
((defsv fls (*@ \label{ll:fls} @*) ; the flight level of the drone
  :init 0
  :min 0
  :max 3)

(defsv battery (*@ \label{ll:battery} @*) ; the battery level of the drone
  :states (Good Low Critical)        ;Self explanatory
  :init Good
  :transitions :all)

(BehaviorTree :name drone
    (Sequence
      (ParallelAll :wait 1 :halt 0 (*@ \label{ll:wait} @*) 
        (Action :ID start_drone)
        (Action :ID start_camera))
      (ReactiveSequence :halt 1 (*@ \label{ll:halt} @*) 
        (Sequence
          (Fallback
            (ForceFailure :ID fail
              (SetSV :ID measure_battery :sv battery)) (*@ \label{ll:setsv} @*) 
            (Eval (~ (= battery critical))) (*@ \label{ll:eval} @*) 
            (ForceFailure :ID fail_mission
              (Action :ID land)))
          (Eval (~ (= battery critical)))
          (Fallback
            (Eval (= battery good))
            (Action :ID goto_waypoint :args (x -1 y -1 z 0.5)))) ; goto charging station
        (Fallback
          (Condition :ID localization_ok)
          (ForceFailure :ID fail
            (Action :ID land)))
        (Sequence
          (Action :ID takeoff :args (height 1.0 duration 0)) (*@ \label{ll:takeoff} @*)
          (Parallel :success 1 :wait 0 :halt 1 ; if the tracking or the nav success... we are done.
            (Action :ID camera_tracking :name camera_track)
            (Repeat :repeat 3 (*@ \label{ll:repeat} @*) 
              (Sequence
                (Eval (:= fls (+ 1 fls))) (*@ \label{ll:add1} @*) 
                (Action :ID goto_waypoint :args (x -3 y -3 z (* 2 $fls)))  (*@ \label{ll:gw1} @*) 
                (Action :ID goto_waypoint :args (x -1.5 y 3 z (* 2 $fls)))
                (Action :ID goto_waypoint :args (x 0 y -3 z (* 2 $fls)))
                (Action :ID goto_waypoint :args (x 1.5 y 3 z (* 2 $fls)))
                (Action :ID goto_waypoint :args (x 3 y -3 z (* 2 $fls)))
                (Action :ID goto_waypoint :args (x 3 y -3 z (* 2 $fls))))))
          (Action :ID goto_waypoint :args (x 0 y 0 z 5))  (*@ \label{ll:gw2} @*) 
          (Action :ID land)
          (Action :ID shutdown_drone)
          (Eval (= camera_track.rstatus success))))))) (*@ \label{ll:finaleval} @*) 
\end{lstlisting}

\begin{figure*}[!ht]
\begin{center}
\includegraphics[width=0.97\textwidth]{./bt-drone3}
\caption{The graphical representation of the BT Listing~\ref{lst:drone3bt}\code{p}\pageref{lst:drone3bt}.}
\label{fig:bt-drone3}
\end{center}
\end{figure*}
 
Calling \bttf{} results in a \fiacre{} model with two processes for the state variables, and thirty eight processes for the BT
nodes.\footnote{The resulting code can be found in the \code{examples} subdirectory of
  \url{https://redmine.laas.fr/projects/bt2fiacre/repository}.}

  
\subsubsection{Offline verification results}
\label{sec:results}
\label{sec:cpu}

To compute the reachable states set of the drone BT (Listing~\ref{lst:drone3bt}\code{p}\pageref{lst:drone3bt}) we call the  \code{sift} tool:

\begin{lstlisting}[language=bash]
sift -stats drone.tts drone.tts/drone.ktz
49 196 302 classe(s), 49 145 735 marking(s), 152 domain(s), 267 702 880 transition(s)
4552.704s
\end{lstlisting}

which takes around 1 hour and 16 minutes to build on an Intel(R) Xeon(R) Silver 4110 CPU @ 2.10 GHz with 256 GB of memory.  The resulting
\code{.ktz} has $49 196 302$ classes, $49 145 735$ markings, $152$ domains, $267 702 880$ transitions.  In this context, a marking is a
particular set of states and values for all the processes and variables in the system. A class is a state extended with timing information
on the enabled transitions (therefore we can have several classes with the same marking).

All the default properties presented in Section~\ref{sec:ltlprop} have been checked and all the results are the expected ones. Most of them
take between less than a second and 300 seconds to be verified. As expected, the ones for which the model checker finds a counter example
are usually quick to be verified while the ones which require to explore the whole set take more time.

Among the default proven properties, some are still interesting to comment on. For example, despite the \btn{ParallelAll} on the
\btn{start\_drone} and the \btn{start\_camera}, they cannot be halted (because the  \btn{ParallelAll}  has \code{:halt 0}). Unlike the BT
children of the \btn{ReactiveSequence} which has \code{:halt 1}, they can all be halted (except the first branch)\footnote{As only a branch returning
\running{} on the previous tick may be halted if one branch on its left fails, hence, never the first branch.}.

But on the top of the default properties, we can add new ones specific to this particular experiment. Can we prove that if the \sv{battery}
is \svv{Critical} then the drone will attempt to land, and if it is \svv{Low} it will return to the charging station?  Can we show that if
the camera survey fails then the whole drone mission will fail? Can the drone fly higher than 6 meters?

These properties can easily be written in \fiacre{} pointing to states of BT \fiacre{} nodes (e.g., \state{tick\_node}, \state{failure}),
values of state variable (e.g. \sv{fls}, \sv{battery}), or even returned status from executed BT (e.g.,
\code{btnode[localization\_ok\_btn18].rstatus = failure})\footnote{The $\Box,\Rightarrow,\diamondsuit$ operators have the usual LTL
  semantics.}:


\begin{lstlisting}[language=LTL]
property attempt_to_land_if_battery_Critical is
ltl ([] ((drone/5/value (battery=Critical)) and (drone/5/state done)) =>
    	<> (drone/8/state tick_node)) // land_btn12

property attempt_to_go_to_charging_station_if_battery_Low is
ltl ([] ((drone/12/value (battery=Low))  and (drone/5/state done)) =>
    	<> (drone/13/state tick_node)) // goto_waypoint_btn16

property attempt_to_land_if_localization_failure is
ltl ([] ((drone/16/value (btnode[localization_ok_btn18].rstatus = failure) and (drone/16/state done))) =>
    	<> (drone/17/state tick_node)) // land_btn20

property camera_fails_implies_mission_fails is
ltl ([] (drone/21/state failure) => 	    // camera_tracking_camera_track 
    	<> (drone/39/state failure))        // BehaviorTree1_drone

property fly_not_higher_than_6m is absent (drone/35/value (fls > 3))
\end{lstlisting}

The result are proven true for all these properties:

\begin{lstlisting}[language=fiacre]
operator attempt_to_land_if_battery_Critical : prop
TRUE
0.001s
operator attempt_to_go_to_charging_station_if_battery_Low : prop
TRUE
0.001s
operator attempt_to_land_if_localization_failure : prop
TRUE
0.001s
operator camera_fails_implies_mission_fails : prop
TRUE
0.001s
operator fly_not_higher_than_6m : prop
TRUE
242.462s
\end{lstlisting}

More interestingly, we can also take advantage of \fiacre{}/\tina{} timed models. For example, if one considers the one tick per node semantics
(See Section~\ref{sec:tick}), we can prove that the a critical battery will always leads to a landing within a $[0,2]$ ticks interval
(everything else considered in the BT):

\begin{lstlisting}[language=LTL]
property attempt_to_land_if_battery_Critical_timed is
((drone/5/value (battery=Critical))and (drone/5/state done)) leadsto (drone/8/state tick_node) within [0,2]
\end{lstlisting}

which results in: 

\begin{lstlisting}[language=fiacre]
operator attempt_to_land_if_battery_Critical_timed : prop
TRUE
49.170s
\end{lstlisting}

\subsubsection{Runtime verification results}
\label{sec:hresults}

The \bt{Drone} BT execution by \hippo{} runs as expected, the survey mission is executed, and by adding random fault (on \sv{battery} level,
or failing the \btn{localization\_ok} \btn{Condition} node), we show that the drone behaves as expected (perform land to prevent further
problems). If the survey finds the object, it returns \success{} and the navigation is halted (because the \btn{Parallel} is \code{:success 1
  :wait 0 :halt 1}. Note that at the end, we check with an \btn{Eval} mode, that the returned status of the \code{camera\_track}
\btn{Action} node is \success{}. So when the overall \btn{Drone} BT completes, it will return \success{} if and only if the target was found,
\failure{} otherwise.

Again, the fact that the BT formal model execution exhibit what is expected from the original programmer is a good sign that the formal
semantics is consistent with the operational one, and that the offline proofs we did on the very same model hold for the original
BT.\footnote{Some videos of these runs and the BT animation are available here:
  \url{https://redmine.laas.fr/projects/bt2fiacre/pages/index\#yet-another-more-complex-example-drone3}.}

From a performance point of view, the overhead of the \hippo{} execution is negligible. After all, the \hippo{} engine is a Petri Net (TTS)
``execution'' engine, and has 497 places and 1113 transitions and manages up to 7 task execution threads (one for each \btn{Action} BT node).

\begin{figure*}[!ht]
\begin{center}
\centerline{\includegraphics[width=1.1\textwidth]{./screen}}
\caption{The screen dump of the drone BT mission~\ref{lst:drone3bt}\code{p}\pageref{lst:drone3bt} executing in \hippo{}. This shows that the top \btn{BehaviorTree} and
  its child \btn{Sequence} child have passed the tick to the \btn{ReactiveSequence} currently executing while its \btn{Sequence} has
  returned \running{} from its \btn{Land} child, while the \btn{camera\_track} has returned \success{}, as a result, the \btn{Parallel} has
  halted the \btn{Repeat} and its children. Note the tick counter near the root which indicates the mission is in its 20th second of execution.}
\label{fig:screen-bt-drone}
\end{center}
\end{figure*}

\subsection{ROS2 Nav2 BT}
\label{sec:nav2}

One of the most popular navigation stacks in robotics is the Nav2 one. It is now part of ROS2 and is proposed with a number of BT to
implement it~\cite{ros2nav2:2024aa}.  So to also properly test our approach we implemented the Nav2 BT in our framework.  Note that these BT
define some new \btn{Decorator} and \btn{Control} nodes: \btn{Recovery}, \btn{PipelineSequence}, \btn{RateController}, \btn{RoundRobin},
etc.  They were implemented in our \bttf{} tool and are being  translated to \fiacre{} along the regular ones. One interesting aspect of our approach is that the
operational semantics of these new nodes can be proven in our formal semantic of \fiacre{} using the \tina{} toolbox. For example, one can
formally prove that our implementation of \btn{Recovery} is correct by showing that the second node can only be called when the first one
has returned \failure{}, and that the first one can only be called again when the second has returned \success{}, etc.

% We were able to produce the  reachable states of some of the Nav2 BT, but others remain difficult to synthesize, mostly due to the
% \btn{RateController} nodes. Indeed, this node is implemented with a natural number which counts the number of ticks required to reach the
% tick rate divided by the controller rate. As a consequence, this multiply the number of reachable states of the BT by the maximum value of
% this counter (10, with a \qty{10}{\Hz} tick).

We are able to check the reachable state of the Nav2 BT presented in Appendix, Listing~\ref{lst:ros2nav2btf}\code{p}\pageref{lst:ros2nav2btf} and Figure~\ref{fig:screen-nav2}\code{p}\pageref{fig:screen-nav2}.

\begin{lstlisting}[language=bash]
sift -stats ros2-nav2.tts ros2-nav2.tts/ros2-nav2.ktz
171 656 098 classe(s), 171 656 098 marking(s), 66 domain(s), 422 970 818 transition(s)
5776.872s

selt ros2-nav2.tts/ros2-nav2.ktz ros2-nav2.tts/ros2-nav2.ltl -b
Selt version 3.8.0 -- 05/02/24 -- LAAS/CNRS
ktz loaded, 171 656 098 states, 422 970 818 transitions
2212.781s
\end{lstlisting}

Building the set of reachable states of this BT takes \qty{1}{\hour} \qty{46}{\minute} on the same CPU as the one used in
Section~\ref{sec:cpu} and results in reachable state sets larger than the one we had with the drone experiment. Similarly, all the default
formal properties were checked (taking between 0 and 85 seconds for each property) without any unexpected results.

More interestingly is to check that our implementation of the added BT (e.g., \btn{Recovery}, \btn{RoundRobin}, etc) is correct.

\paragraph{Recovery}
The \btn{Recovery} node\footnote{\url{https://docs.nav2.org/behavior_trees/overview/nav2_specific_nodes.html}} is a control flow
node with two children. It returns \success{} if and only if the first child returns \success{}. The second child will be executed only if
the first child returns \failure{}. If the second child succeeds, then the first child will be executed again. The user can specify how
many \code{:retry} times the recovery actions should be taken before returning \failure{}.

We define a simple generic example:
\begin{lstlisting}[caption={A simple \btn{Recovery} node.}, numbers=left, xleftmargin=15pt, label={lst:recoverybtf}, language={[btf]Lisp}]
((BehaviorTree :name bt_recovery
     (Recovery :num_retries 1 :name recovery
          (Action :name action)
          (Action :name recov))))
\end{lstlisting}

for which the model checker \tool{selt} is able to prove the following properties:

\begin{lstlisting}[language=LTL]
// Action_action		// 1 
// Action_recov			// 2
// Recovery_recovery		// 3
// BehaviorTree_bt_recovery	// 4

property failure_recov_leads_to_failure is // TRUE if the recovery action fails, than the recovery fails
(bt_recovery/2/state failure) leadsto (bt_recovery/3/state failure) within [0,0]

property failure_action_leads_to_recovery_failure is  // FALSE because retry is still 0 (not retried yet)
(bt_recovery/1/state failure and bt_recovery/3/value (retry = 0))  leadsto  (bt_recovery/3/state failure) within [0,0]

property failure_action_leads_to_recovery_failure is // TRUE because we already retied
(bt_recovery/1/state failure and bt_recovery/3/value (retry = 1))  leadsto  (bt_recovery/3/state failure) within [0,0]

property success_action_leads_to_success is // TRUE
(bt_recovery/1/state success) leadsto (bt_recovery/3/state success) within [0,0]
\end{lstlisting}

We present here a proof on a simple instance, but similar proofs can be made on an instance of \btn{Recovery} embedded in a larger BT, for
example the ones in the Nav2 BT in Appendix, Listing~\ref{lst:ros2nav2btf}\code{p}\pageref{lst:ros2nav2btf} and
Figure~\ref{fig:screen-nav2}\code{p}\pageref{fig:screen-nav2}.

\paragraph{RoundRobin}
The \btn{RoundRobin} node\footnote{\url{https://docs.nav2.org/behavior_trees/overview/nav2_specific_nodes.html}} is a control node which
ticks its children in a round robin fashion until a child returns \success{}, in which case the parent node will also return \success{}. If all
children return \failure{} so will the parent \btn{RoundRobin}.

Similarly, we define a simple generic example:
\begin{lstlisting}[caption={A simple \btn{RoundRobin} node.}, numbers=left, xleftmargin=15pt, label={lst:roundrobinbtf}, language={[btf]Lisp}]
((BehaviorTree :name bt_roundrobin
    (KeepRunningUntilFailure :name kr
      (RoundRobin :name RR
          (Action :name A1)
          (Action :name A2)
          (Action :name A3)
          (Action :name A4)))))
\end{lstlisting}

for which the model checker is able to prove the following properties:

\begin{lstlisting}[language=LTL]
// Action_A1	// 1
// Action_A2	// 2
// Action_A3	// 3
// Action_A4	// 4
// RoundRobin_RR // 5

property failure_a1_leads_to_failure is // FALSE a failure of one child does not necessary leads to rr failure
(bt_roundrobin/1/state failure) leadsto (bt_roundrobin/5/state failure) within [0,1]

property failure_a4_leads_to_a1_ticked is  // TRUE a failure of a child with less than 3 failure so far leads to the next child to be ticked
((bt_roundrobin/4/state failure ) and (bt_roundrobin/5/value (failed < 3))) leadsto (bt_roundrobin/1/state tick_node) within [0,0]

property success_a2_leads_a3_ticked_exp_true is // TRUE success of a child leads to the next one to be ticked upon rr reticked
(bt_roundrobin/2/state success) leadsto (bt_roundrobin/3/state tick_node) within [0,3] // 3 because the tick has to go all the way up

property success_a2_leads_rr_success_exp_true is // TRUE
(bt_roundrobin/2/state success) leadsto (bt_roundrobin/5/state success) within [0,0]
\end{lstlisting}

Similar proofs can be made with any BT nodes to show that our \fiacre{} implementation satisfies the expected formal properties
defining the BT node considered.

\subsubsection{Runtime}

All the Nav2 BT have been translated to \code{.btf} and run with Hippo and simulated ROS2 actions. One of the features we added and we
believe goes beyond the ``nice graphical touch'' aspect, is to dynamically color the BT nodes while they execute. Of course, there are text
traces of the Hippo engine running the BT, but being able to  follow the execution ticks as well as the last returned status for each BT is rather
informative.  Some videos of these runs and the BT animation are available here:
\url{https://redmine.laas.fr/projects/bt2fiacre/pages/index\#playing-with-ros2-nav2-bt}.
Figure~\ref{fig:screen-nav2}\code{p}\pageref{fig:screen-nav2} shows the Nav2 BT (See Listing~\ref{lst:ros2nav2btf}\code{p}\pageref{lst:ros2nav2btf}) executed by \hippo{}.

\section{Limits, discussion, future work, and conclusion}
\label{sec:conclusion}

Before concluding, we propose to examine the limits of our work, to discuss its pros and cons and consider future work.

\subsection{Limits}

Some limits of our work are mostly due to choices currently made, which could be reconsidered, if needed. For example, we do not implement a
black board to handle variables. Instead we propose to handle BT variables in \fiacre{} directly. This has the advantage of being able to
include these variables in the formal model, hence in the proof and the runtime verification (e.g. the \sv{fls} flight level in the drone
experiment). The disadvantage is that only \fiacre{} supported types can be used.

An area where our approach would suffer, is when BT are dynamically modified, or transformed. For example in
Section~\ref{sec:soabtplanning} we consider BT used for planning. In these approaches, BT are being modified on the fly, clearly, we could not
perform model checking on the fly, yet assuming the dynamic of the planner remains slow (below one second) we could still synthesize and
compile the model (both are almost instantaneous), and jump in its execution on the fly. 

Although all our examples are based on one BT, absolutely nothing prevents us from having multiple BT executing together. Yet from a
verification point of view, this would probably lead to an intractable model as models running in parallel tend to multiply the size of their
reachable states set, to account for all the possible transition interleavings.

This brings us to the main limiting factor of our approach. Offline verification with model checking may lead to large intractable reachable
states set. This is a well known limit of these approaches and we do not have any magic bullet. It is hard to predict what will be the size
of the reachable states set for a given BT. Some very large BT may still produce a rather small reachable states set, while a simple BT
(e.g., with a lot of parallelism, loops, etc) are untractable. Yet we have seen in our examples that we can verify reasonably complex
robotics BT skills, that we can prove ``individual'' BT node behavior (e.g. the new control nodes added in Nav2), and that we can also
handle complex properties on missions written with BT. For the models too complex to be verified offline, we can still use the complete online
version with some added monitors, or work on an abstracted model for offline verification (but we would then loose the offline/online
equivalence).

\subsection{Discussion}
\label{sec:discussion}

Our approach to improve the trust one can put in BT completely relies on a on a well established formal language and framework (\fiacre{}
and TTS), a formal verification toolbox (\tina{} to check LTL, CTL, properties, patterns~\cite{Abid:2014aa}\footnote{Patterns allow the
  verification of properties with explicit time (e.g. to prove that at most x units of time will elapse between two states.).}) and
\hippo{}, a runtime engine able to execute the formal
model.  We have seen in Section~\ref{sec:soabtfm} that there are other approaches which transform BT in a formal framework, or harness a
formal model around BT, but to our knowledge, none of them support the \emph{automatic} translation of BT to an equivalent formal model
\emph{and} the execution of the same formal model in place of the original BT. This is a very strong argument as the same formal model,
which clarifies and specifies the formal semantics of BT, can be used to prove properties and also execute and show, while running, that it
properly implements the expected operational semantic, moreover knowing that the proof made offline also holds online.

Despite the formal and proven tools deployed, our translation tool \bttf{} may still contain errors and bugs, but again, the fact that we
can use theses tools to prove the resulting translated model against specifications done on the BT is reassuring.  One could take every
single BT node type, write its formal specification and prove that the \fiacre{} translation satisfies them with \tina{} (as we did for the
ones added in the Nav2 experiment).

% Again, stress
% that having a model both for verification and to run the experiment is a very strong argument in favor of this approach. While writting the
% BT, we can check that it does what its programmer intended it to do, and it can run verification tools on the model.

Furthermore, \fiacre{} offers some features which could be valuable to be added to BT. Explicit time representation is the foremost feature which could
come handy if added to the BT. Indeed, robot planning, acting and control is usually handling explicit time representation. Control loops
have frequency, plans have duration and deadline, actions take some time to execute, etc. So it would be perfectly logical to add explicit
time representation to BT. As a consequence, this would also have an impact on the tick semantic which would probably become more an
execution token than a time elapsing counter.

Another domain where our approach and \fiacre{} could offer some valuable addition is with environment modelling. BT are intrinsically
embedded operational models, they really exist to be executed in the real world (or in simulation). But now that we can consider model
checking them, one needs to take into account (when possible) the outcomes of actions or conditions in the environment. Without any
particular information, the model checker consider all possibilities (\success{}, \failure{} or \running{}), but the real world often prove
to be more ``constrained'', and one could consider a more accurate and explicit model of the environment (e.g., with external asynchronous
events, state variable values changes, etc).

Parallelism is clearly allowed and properly modelled in our framework. Moreover the model checker can verify properties even if it needs to
consider all the possible execution interleavings of parallel nodes (but at some cost). Meanwhile, the semantic of transition in
\fiacre{} is such that resource checking and reservation on a transition are atomic. As a consequence, resources management, even
considering parallelism comes for free with \fiacre{}.

Last, our approach has been implemented in a software suite~\cite{Ingrand:2024ab}  which has been  applied to many BT examples (available
here: \url{https://redmine.laas.fr/projects/bt2fiacre/pages/index}). Moreover, all the examples presented here are available in the
\code{examples} directory and we invite other formal approach to test their systems on these examples and report their results for comparison. 

%\cite{Ingrand:2024aa}


\subsection{Future work}

Considering the tight links between the \fiacre{} framework and our approach/tool \bttf{}, it is clear that any improvement in the former
may also improve the latter. The \fiacre{} developers are considering extending the data types handled in the language by adding rational
numbers and strings. Both types would still allow model checking and enable the deployment of richer state variables in our \code{.btf}
format.

Similarly, there are many existing features in \fiacre{} which could be used in \code{.btf} BT (some have already been used in
\ps{}~\cite{Ingrand:2024aa}, another robotics acting language mapping to \fiacre{}). As already mentioned in Section~\ref{sec:discussion}
above, time would be a valuable addition to model, with time interval $[min,max]$, how long an \btn{Action} is expected to take, or to wait
a given amount of time before returning from a \btn{Wait} node, etc. Not only would this enrich the \code{.btf} format and language, but it
would also be taken into account by the \tina{} model checking tools, and enforced by the \hippo{} engine.

External asynchronous events and state variables asynchronous value changes are features also available in \fiacre{} and its tools. This
again could enrich the \code{.btf} format and allow the programmer to account for ``uncontrollable''  state transitions.

%The \fiacre{} language is getting new basic types, string and rational number. This should open the language to more complex BT model

\subsection{Conclusion}


BT are more and more popular in robotics. To encourage their deployment and improve the trust one has in robotic applications using BT, we
propose an approach and an automatic tool to transform any BT in a formal model with a formal semantics. The resulting models can then be
used offline with model checking to prove some properties of the BT, but also linked to the real actions and perceptions of the robots and
executed online on the robot.  Of course, this can be deployed in BT applications outside of robotics, and also participates to define
extensions to BT and to better formalize them.

\section*{Acknowledgement}

 We thank Bernard Berthomieu, Silvano Dal Zilio and Pierre-Emmanuel Hladik for their help while developing and deploying the work presented here.

\ifHAL
\bibliographystyle{abbrvnat}
\else
\bibliographystyle{elsarticle-harv}
\fi

\ifARXIV
%\bibliography{master}
\documentclass{MITstyle}

%\usepackage[table]{xcolor}
\usepackage{chngcntr}
\usepackage{hyperref}
\usepackage{microtype}

\title{A Lightweight and Extensible Cell Segmentation and Classification Model for Whole Slide Images}

\author{Nikita Shvetsov~$^{1, }$\footnote{Correspondence e-mail: nikita.shvetsov@uit.no}, Thomas K. Kilvaer~$^{2, 3}$, Masoud Tafavvoghi~$^{4}$, Anders Sildnes~$^{1}$, \\ Kajsa Møllersen~$^{4}$, Lill-Tove Rasmussen Busund~$^{5, 6}$, Lars Ailo Bongo~$^{1}$ \\
%
\vspace{1em} % Space between authors and afilliations
%
\normalfont{\small $^{1}$Department of Computer Science, UiT The Arctic University of Norway}\\
\normalfont{\small $^{2}$Department of Oncology, University Hospital of North Norway}\\
\normalfont{\small $^{3}$Department of Clinical Medicine, UiT The Arctic University of Norway}\\
\normalfont{\small $^{4}$Department of Community Medicine, UiT The Arctic University of Norway}\\
\normalfont{\small $^{5}$Department of Medical Biology, UiT The Arctic University of Norway} \\
\normalfont{\small $^{6}$Department of Clinical Pathology, University Hospital of North Norway} %\vspace{2em}
}

\begin{document}
\maketitle

\section*{Abstract}

% \begin{abstract}
% Developing clinically useful cell-level analysis tools in digital pathology remains challenging due to limitations in dataset granularity, inconsistent annotations, computational demands of advanced models, and difficulties in integrating new technologies into clinical workflows. To address these challenges, we propose a multi-faceted solution that enhances data quality, model performance, and usability to create a lightweight and extensible cell segmentation and classification model.

% First, we update data labels by employing a cross-relabeling process that refines the labels of two existing datasets, PanNuke and MoNuSAC, to create a new unified dataset with enhanced granularity, encompassing seven distinct cell types. Second, we leverage the H-Optimus foundation model as a fixed encoder to improve feature representation for simultaneous cell segmentation and classification tasks. Third, to address the computational demands of foundation models, we employ knowledge distillation to reduce model size and complexity while maintaining comparable performance. Finally, to facilitate integration into clinical workflows, we integrate the distilled model into the QuPath software, a widely used open-source platform in digital pathology.

% Our results demonstrate improvements in cell segmentation and classification performance using the H‑Optimus-based model compared to a CNN-based model. Specifically, the average $R^2$ improved from 0.575 to 0.871, and the average $PQ$ score improved from 0.450 to 0.492, indicating better alignment with actual cell counts and enhanced segmentation and classification quality. Furthermore, the distilled student model maintains performance comparable to the larger foundation model while reducing the parameter count by a factor of 48.
% Overall, by reducing computational complexity and integrating it into existing workflows, the proposed approach may significantly impact diagnostic processes, reduce the workload of pathologists, and contribute to improved patient outcomes. Though our approach shows potential enhancements in efficiency and usability of cell segmentation and classification models in digital pathology, extensive validation is needed to deploy these models in clinical practice.
% \end{abstract}

%%% shortened abstract
\begin{abstract}
Developing clinically useful cell-level analysis tools in digital pathology remains challenging due to limitations in dataset granularity, inconsistent annotations, high computational demands, and difficulties integrating new technologies into workflows. To address these issues, we propose a solution that enhances data quality, model performance, and usability by creating a lightweight, extensible cell segmentation and classification model. 

First, we update data labels through cross-relabeling to refine annotations of PanNuke and MoNuSAC, producing a unified dataset with seven distinct cell types. Second, we leverage the H-Optimus foundation model as a fixed encoder to improve feature representation for simultaneous segmentation and classification tasks. Third, to address foundation models' computational demands, we distill knowledge to reduce model size and complexity while maintaining comparable performance. Finally, we integrate the distilled model into QuPath, a widely used open-source digital pathology platform. 

Results demonstrate improved segmentation and classification performance using the H-Optimus-based model compared to a CNN-based model. Specifically, average $R^2$ improved from 0.575 to 0.871, and average $PQ$ score improved from 0.450 to 0.492, indicating better alignment with actual cell counts and enhanced segmentation quality. The distilled model maintains comparable performance while reducing parameter count by a factor of 48. By reducing computational complexity and integrating into workflows, this approach may significantly impact diagnostics, reduce pathologist workload, and improve outcomes. Although the method shows promise, extensive validation is necessary prior to clinical deployment.
\end{abstract}
\clearpage

\section{Introduction}
In digital pathology, accurate segmentation and classification of cells are crucial for many diagnostic, prognostic, and predictive analyses \cite{Jaber_Beziaeva_etal._2019,Lin_Pan_etal._2022,Park_Ock_etal._2022,Shen_Choi_etal._2024}. Nowadays, developments in computational pathology offer multiple solutions \cite{H._Qu_P._Wu_etal._2020,Javed_Mahmood_etal._2020} to utilize cell-level datasets to train machine learning models that solve these problems. The quality and specificity of training datasets are critical for robust and accurate models. Adhering to the principle of "garbage in, garbage out", it is essential to ensure that these datasets are extensively and accurately labeled with distinct classes that reflect the diverse biological characteristics of different cell types. Unfortunately, the number of open-source datasets comprising such high-quality annotations is limited. Existing cell segmentation datasets \cite{Gamper_Koohbanani_etal._2019,Graham_Vu_etal._2019,Verma_Kumar_etal._2021} may offer extensive annotations for certain cell types while providing more general labels for others. For example, in PanNuke, which is one of the largest open-source datasets comprising labeled cells, various types of morphologically and functionally different inflammatory cells like macrophages and lymphocytes are clustered in a broad "inflammatory" class. Consequently, these classes are frequently omitted from analyses or aggregated into broader meta-classes \cite{Gamper_Koohbanani_etal._2020} and likely interfere with other cell classes included in the dataset. This and similar inconsistencies in annotation granularity limit the ability of machine learning models to learn the comprehensive and nuanced features necessary for accurate cell segmentation and classification. To address these challenges, methods for refining and standardizing dataset annotations are essential to enhance the quality of training data.

A complementary approach to mitigate the absence of high-quality training data is the use of foundation models. Foundation models as encoders are defined as large-scale, versatile networks pre-trained on vast, diverse datasets using self-supervised learning, contrasting with convolutional neural network (CNN) pre-trained encoders that rely on supervised learning with labeled data. In practice, foundation models leverage enormous amounts of weakly or unlabeled data from millions of whole slide images (WSIs) and employ self-attention mechanisms to capture long-range dependencies and global context \cite{Chen_Ding_etal._2024,Saillard_Jenatton_etal._2024,Vorontsov_Bozkurt_etal._2024,Xu_Usuyama_etal._2024}. As a consequence, foundation models are able to produce transferable feature representations across different cell types and tissue environments. The feature representations can be leveraged by decoder networks to produce segmentation masks and pixel-level classifications. Because foundation models have comprehensive feature representations, they can be effectively fine-tuned using much smaller amounts of cell-level data compared to the large datasets needed to train models from scratch. Furthermore, foundation models incorporate adversarial training elements or contrastive learning \cite{Chen_Ding_etal._2024,Xu_Usuyama_etal._2024}, enhancing their resilience and adaptability by exposing them to challenging and varied scenarios during training. This may result in more generalizable models, often making them well-suited for diverse and complex tasks in digital pathology.

Despite the inherent advantages of foundation models, their deployment for practical use faces its own obstacles. In particular, they require substantial computational power, financial investments and rigorous testing to ensure reliability and efficacy for a given task \cite{Akkus_Dangott_etal._2022,Dragomir_Cocuz_etal._2022,Go_2022,Jafri_Farooqui_etal._2024}. Moreover, while foundation models enhance feature representation and performance, they depend on the quality of available annotations for decoder fine-tuning and, like any other model, cannot resolve existing inconsistencies or ambiguities in data labels. Therefore, there remains a critical need for solutions that address both data quality and practical deployment considerations.
Further, integrating new technologies into existing clinical workflows often encounters resistance, as it necessitates adjustments to established diagnostic processes. So, there is a need to develop solutions that could be integrated into current practices, minimizing the burden on medical professionals to adopt new tools \cite{King_Williams_etal._2023}.

Existing solutions \cite{Goldsborough_Philps_etal._2024,Hörst_Rempe_etal._2024}, while addressing some aspects of these challenges, fall short in providing a comprehensive approach. To address the data quality and clinical deployment issues, we propose a multi-faceted solution that encompasses data refinement, model optimization, and integration with existing pathology tools (\hyperref[fig:fig1]{Figure 1}). The outcome is a lightweight cell segmentation and classification model that can be integrated into digital pathology workflows for practical clinical use.

\begin{figure}[h!]
    \centering
    \includegraphics[width=\textwidth, height=0.82\textheight, keepaspectratio]{images/Figure_1.pdf}
    \caption{Overview of the proposed solution, including 1) Data refinement using cross-relabeling, 2) Teacher model development and fine tuning, 3) Student model optimization with knowledge distillation and 4) Student model and QuPath integration}
    \label{fig:fig1}
\end{figure}
\clearpage

Our approach begins with preparing the data for the fine-tuning and training of the machine learning models. We create a refined dataset, acquired via cross-relabeling two cell-level datasets, enhancing annotation specificity and consistency of the labeled data. Subsequently, we create a cell segmentation and classification model based on the foundation model. We leverage the foundation model as a fixed encoder and fine-tune a decoder using the refined dataset to improve generalization across diverse tissue- and cell types.
To ensure that the model remains lightweight and deployable in a possibly resource-constrained environment, we employ knowledge distillation to approximate the functionality of the foundation model. Finally, to facilitate the practical application of our model in digital pathology workflows, we integrate it with the QuPath \cite{Bankhead_Loughrey_etal._2017} application. Each methodological component contributes to the overarching goal of enhancing model performance, generalizability, and usability in clinical settings.

The primary contributions of this paper are:
\begin{enumerate}
    \item \textit{Data labels refinement through cross-relabeling:}
    
    We propose a new method for refining labels of cell-level datasets through cross-relabeling. This method employs classification models to re-label broad and ambiguous instances, resulting in a more diverse dataset. Our evaluation demonstrates that these classification models achieve high accuracy on test subsets, indicating the reliability of the method for label refinement.

    \item \textit{Enhanced model performance via foundation models:}
    
    We employ a foundation model as a feature extractor for the cell segmentation and classification task. In comparison with training a CNN model from scratch, the foundation model backbone only needs fine-tuning, which significantly reduces training time, computational resources and data requirements. We show that using a foundation model encoder leads to better performance in cell segmentation and classification networks than using a CNN-based encoder. This improvement may enable the model to generalize more effectively across various tissue types and imaging methods.
    
    \item \textit{Model optimization through knowledge distillation:}
    
    We show that a smaller student model trained using knowledge distillation on the refined dataset obtained via our cross-relabeling approach from a foundation model achieves comparable performance in cell segmentation and quantification tasks. As a result, this model is more suitable for deployment in environments without high-performance computing resources.
    
    \item \textit{Integration with QuPath:}
    
    We integrate the distilled cell segmentation and classification model into QuPath, a widely used open-source digital pathology platform, to accelerate clinical adaptation by enabling pathologists to more easily incorporate advanced computational tools into their existing workflows.
\end{enumerate}

Through these methodological steps, we aim to bridge the gap between advanced machine learning techniques and practical clinical applications, making accurate and efficient digital pathology accessible in a broader range of healthcare settings.

\section{Refining Existing Datasets Using Cross-Relabeling}
To address the limitations of sparse and ambiguous labeling of cell-level datasets, we propose a generalizable cross-relabeling strategy that can be applied to any dataset containing broadly categorized or imprecisely labeled cell types. This approach involves training and subsequently leveraging classification models to refine broad categories into more specific or biologically relevant classes.
When applied to cell-level data, the methodology includes extracting individual cell images from the dataset patches, preprocessing these images to standardize the size and accommodate partial cells, and then training deep learning classifiers capable of distinguishing between the finer cell subtypes within the coarser categories. 
To illustrate our approach, we focus on the PanNuke \cite{Gamper_Koohbanani_etal._2020, Gamper_Koohbanani_etal._2019} and MoNuSAC \cite{Verma_Kumar_etal._2021} datasets that we have used to train models for cell quantification in our previous works \cite{Shvetsov_Grønnesby_etal._2022,Shvetsov_Sildnes_etal._2024}. We find that for better cell differentiation we have to introduce more granular labels. PanNuke includes a broad classification of "inflammatory" cells, encompassing lymphocytes, macrophages, and neutrophils. Each cell type differs significantly in structure, function, and clinical relevance. Conversely, MoNuSAC uses the label "epithelial" for a class that comprises both benign epithelial cells and malignant neoplastic cells. This practice makes it challenging to differentiate between benign and malignant epithelial cells in the dataset, which is a critical distinction when identifying tumor areas within tissue samples. To address these issues, we implement a cross-relabeling strategy as shown in \hyperref[fig:fig2]{Figure 2}. The key components are two classification models: one is trained on singular cell images from PanNuke data to classify the epithelial meta-class into epithelial and neoplastic classes. The other is trained on MoNuSAC to refine the inflammatory class into lymphocytes, neutrophils, and macrophages.

\begin{figure}[h!]
    \centering
    \includegraphics[width=\textwidth]{images/Figure_2.pdf}
    \caption{Refined dataset generation via cross relabeling}
    \label{fig:fig2}
\end{figure}

The refining approach consists of three consecutive steps. The first is the preprocessing step, in which we extract individual cells from both datasets (\hyperref[fig:fig3]{Figure 3}). The specifics of PanNuke and MoNuSAC patch preparation before cell preprocessing are provided in \hyperref[chap:S1]{Appendix S1}.

\begin{figure}[h!]
    \centering
    \includegraphics[width=\textwidth]{images/Figure_3.pdf}
    \caption{Cell instances preprocessing including (1) cell map extraction, (2) bounding box delineation, (3) adjusting cell boxes and (4) cropping and resizing of cell images}
    \label{fig:fig3}
\end{figure}

During preprocessing, we extract cell type maps from the ground truth label mask and calculate bounding boxes around each cell instance. To accommodate partial cells at patch borders, a common issue in cropped patch images, we employ mirror padding and extend the field of view of the cell label by 15 pixels to capture adjacent cells. We then crop and resize the identified regions to $64 \times 64$ pixels using bicubic interpolation.

The preprocessed PanNuke dataset comprises 68,031 neoplastic and 23,207 epithelial cell images, while MoNuSAC comprises  33,104 lymphocytes, 1,252 neutrophils, and 1,695 macrophages, which we subsequently use in training cell classification models and classifying the cell image data \hyperref[fig:S2]{Appendix Figure S2 (1)}. 

The next step is to train two distinct ResNet50-based classifiers tailored to address the specific labeling challenges inherent in each dataset. We use ResNet50 for classification models due to its proven effectiveness for image classification tasks in histopathology \cite{pan2022reviewmachinelearningapproaches}, and its compatibility with small images. For the PanNuke dataset, we design the classifier, trained on MoNuSAC data, to disaggregate the heterogeneous "inflammatory" cell category into distinct subtypes: lymphocytes, macrophages, and neutrophils. Similarly, for the MoNuSAC dataset, the classifier is trained on PanNuke data and distinguishes between benign and malignant epithelial cells within the overarching "epithelial" label. By applying these targeted classifiers to their respective datasets, we assign more specific labels to individual cell instances, thus enabling us to create a unified dataset.
To ensure a balanced representation of classes, we train both models on datasets that had been equalized to match the size of the least represented class. Thus, we obtain datasets comprising 23,207 samples per class for PanNuke and 1,252 samples per class for MoNuSAC data. Next, we partition both of them into training (70\%), validation (20\%), and testing (10\%) subsets. To mitigate the risk of overfitting, we use a single dropout layer with a rate of p=0.5 in both models and data augmentation using randomized color perturbations, rotation, and horizontal and vertical flipping. We employ AdamW optimizer and the cross-entropy loss function for the training criterion.

To evaluate the two trained models, we measure the classification accuracy on the respective test subsets. The accuracies on the test subset for both classifiers are presented in \hyperref[tab:1]{Table 1}. The PanNuke model achieves an average accuracy of 93.57\%, with higher accuracy for neoplastic cells (96.06\%) compared to epithelial cells (86.26\%). The confusion matrix in Figure A3.1 shows that the model predominantly distinguishes accurately between epithelial and neoplastic tissues, with a substantial number of correct classifications and relatively few misclassifications. The MoNuSAC model demonstrates an average accuracy of 98.92\%, excelling in classifying lymphocytes (99.67\%) and macrophages (94.12\%), with lower performance for neutrophils (85.71\%). The confusion matrix in Figure A3.2 shows that the model identifies lymphocytes and performs reasonably well with macrophages and neutrophils.

\begin{table}[h!]
\renewcommand{\arraystretch}{1.5}
  \centering
  \caption{Cell classification results for PanNuke and MoNuSAC trained models (CI 95\%).}
  \label{tab:1}
  \begin{tabular}{|l|c|c|}
   \hline
   %\rowcolor{gray!30}
    Accuracy               & PanNuke model              & MoNuSAC model              \\
    \hline
    Average      & 0.936 (0.931--0.941)         & 0.989 (0.986--0.993)        \\
    \hline
    Neoplastic   & 0.961 (0.956--0.965)         & -                          \\
    \hline
    Epithelial   & 0.863 (0.849--0.877)         & -                          \\
    \hline
    Lymphocytes  & -                          & 0.997 (0.995--0.999)        \\
    \hline
    Neutrophils  & -                          & 0.857 (0.796--0.918)        \\
    \hline
    Macrophages  & -                          & 0.941 (0.906--0.976)        \\
    \hline
  \end{tabular}
\end{table}

Finally, during the last step, we use the model trained on PanNuke data for epithelial cells in MoNuSAC and the model trained on MoNuSAC for the inflammatory cells class in PanNuke. Specifically, we use classifier models to relabel epithelial cells in MoNuSAC and inflammatory cells in PanNuke data. Then we combine cells with refined labels and the rest of the cells in both datasets to create a refined dataset (\hyperref[fig:S2]{Appendix Figure S2 (2)}). The process of relabeling cells and visualizing them on a patch is shown in \hyperref[fig:fig4]{Figure 4}. The cell counts in the refined dataset are provided in \hyperref[tab:S4]{Appendix Table S4}.

\begin{figure}[h!]
    \centering
    \includegraphics[width=\textwidth, height=0.42\textheight, keepaspectratio]{images/Figure_4.pdf}
    \caption{Cell relabeling procedure for epithelial and inflammatory cell classes}
    \label{fig:fig4}
\end{figure}

%\hfill

Relabeling and combining datasets have been explored in a prior study \cite{Parulekar_Kanwat_etal._2023}, where consecutive fine-tuning on multiple datasets was employed to account for hierarchical class label structures. While the method presented in \cite{Parulekar_Kanwat_etal._2023} is intuitive, it often lacks consistency and requires multiple fine-tuning runs, which can be cumbersome and time-consuming. 
In contrast, cross-relabeling simplifies this process by using specialized classification models tailored to each dataset's specific labeling challenges. This approach provides better transparency and produces a unified dataset encompassing seven distinct cell types across multiple tissue samples, enhancing data diversity for further model training or fine-tuning.

Despite these improvements, cross-relabeling does not entirely resolve issues related to poor labeling quality or the amount of labeled data. Specifically, our results show lower accuracies persist for underrepresented classes, such as macrophages, which may stem from a limited sample availability and intrinsic challenges in distinguishing these cells based solely on H\&E staining. Furthermore, while our method enhances label specificity, it relies on the initial quality of the broad labels; thus, any fundamental inaccuracies in the original annotations can propagate through the relabeling process. Addressing the overall problem of limited data labels may require integrating additional data sources or utilizing complementary immunohistochemical staining methods.
Although the reported performance metrics are obtained from evaluations on the native test sets of each dataset, it is important to note that the primary application of these classifiers is to perform cross-relabeling, where a model trained on one dataset (e.g., PanNuke) is applied to another (e.g., MoNuSAC) and vice versa. We acknowledge that a more systematic evaluation of cross-dataset generalization is needed and could be performed in future work.

Overall, the refined dataset produced by our approach can enhance the supervised training or fine-tuning of cell segmentation and classification models, especially those that utilize pre-trained foundation models to improve feature extraction robustness. In addition, these models can detect nuanced classes that enable researchers to conduct more detailed analyses of biological processes in computational pathology.

\section{Foundation models for robust cell segmentation and classification}

Accurate cell segmentation and classification in digital pathology are hindered by limited labeled data and the fact that conventional CNNs are unable to capture global contextual information due to their local receptive field constraints \cite{Gheflati_Rivaz_2022,Yang_Marcus_etal.}. Traditional approaches in cell quantification have predominantly relied on CNN encoders, such as ResNet50, given their proven effectiveness in semantic segmentation tasks \cite{Deshmane_2023,Graham_Vu_etal._2019,Mukasheva_Koishiyeva_etal._2024,Stringer_Wang_etal._2021}. However, approaches that include fine-tuning of pretrained CNNs, data augmentation, and stain normalization to partially increase data variability and address staining differences often fail to achieve the necessary generalization and robustness across diverse tissue types and staining conditions \cite{G._Wang_W._Li_etal._2018,Gao_Bagci_etal._2018,Karim_El_Khoury_Martin_Fockedey_etal._2021}.

To overcome these challenges, we leverage an encoder-decoder network that uses a foundation model as the encoder and a CNN upsampling decoder (\hyperref[fig:fig5]{Figure 5}) for simultaneous cell segmentation and classification in 2D patches extracted from WSIs. Foundation models with transformer-based architectures are viable alternatives to CNN-based encoders \cite{Shamshad_Khan_etal._2023,Sourget_2023}. They enable the creation of more advanced architectures that can decode or transform learned features more effectively \cite{Chen_Duan_etal._2023,Cheng_Misra_etal._2022,Xie_Wang_etal._2021}.

\begin{figure}[h!]
    \centering
    \includegraphics[width=\textwidth]{images/Figure_5.pdf}
    \caption{UNETR-like model with foundational model as backbone}
    \label{fig:fig5}
\end{figure}

By utilizing a transformer-based encoder, we incorporate global contextual information into the feature extraction process, which is a key advantage of such architectures \cite{Chen_Lu_etal._2021}. This foundation model integration facilitates accurate pixel-wise segmentation and classification without the need for extensive encoder training, thereby potentially improving generalization across varied cellular structures and tissue types.
In our implementation, we employ a modified UNETR \cite{Hatamizadeh_Tang_etal._2021} architecture that combines a vision transformer (ViT) \cite{Dosovitskiy_Beyer_etal._2021} encoder with a CNN-based decoder. The encoder utilizes the pretrained H-Optimus foundation model, which contains 1.1 billion parameters and is trained on over 500,000 H\&E stained WSIs \cite{Saillard_Jenatton_etal._2024}. We extract outputs from four evenly spaced transformer blocks $Z_i$, where $i \in [1, 14, 26, 38]$, to serve as residual connections for the CNN decoder. We select these blocks based on our observation that features from non-adjacent levels of the encoder lead to better overall performance on the test subset.

The CNN decoder upsamples the feature representations, acquired from the transformer blocks, to generate an intermediate vector that is handled by two task-specific layers that generate cell segmentation and classification masks. The first task-specific layer is the ‘Cellpose head’,  which is used to delineate cell instances. The layer generates horizontal and vertical gradient maps to form vector fields that are refined through gradient tracking in a post-processing step using the Cellpose algorithm \cite{Stringer_Wang_etal._2021}, known for its efficacy in cell segmentation tasks and generalizability across multiple domains \cite{Pachitariu_Stringer_2022,Stringer_Pachitariu_2024}. The second task-specific layer is the "Cell type head", which assigns labels to individual pixels. In the post-processing step, we determine the output classification label of each segmented cell instance by majority voting over the labeled pixels that comprise the cell in the segmentation map.

To evaluate model performance and measure the impact of adding a foundation model as backbone, we compare it to a ResNet50-based model. ResNet50 is a widely used solution for encoders in segmentation architectures in the medical domain \cite{Deshmane_2023,Graham_Vu_etal._2019,Mukasheva_Koishiyeva_etal._2024,Stringer_Wang_etal._2021}. For the H-Optimus-based model, we utilize frozen weights for the encoder and only fine-tune the decoder to take advantage of the extensive pre-training of the foundation model. For the ResNet50-based model we start with ImageNet \cite{Deng_Dong_etal.} weights and train both encoder and decoder parts. Hyperparameters for the training step are set to be identical, where possible, for comparable evaluation. 
For this evaluation, we deliberately use the PanNuke dataset to provide a standardized and controlled comparison between the H‑Optimus and ResNet50-based models (\hyperref[fig:S2]{Appendix Figure S2 (3)}). Specifically, we use two of the default PanNuke dataset splits (66\%) for training and validation, and reserve the third split (33\%) for testing.

To address the challenge of cell class imbalance in the PanNuke dataset, which is a common characteristic in most cell-level H\&E patch datasets, both models’ training processes employ a weighted loss function comprising cross-entropy and focal loss \cite{Lin_Goyal_etal._2018}. The focal loss component is adjusted with coefficients derived from each cell class' instance frequency, emphasizing learning from underrepresented classes and enhancing the model's sensitivity to rare but significant cellular patterns. The cross-entropy loss is augmented with spectral decoupling regularization \cite{Pezeshki_Kaba_etal._2021,Pohjonen_Stürenberg_etal._2022} and spatially varying label smoothing \cite{Islam_Glocker_2021}, which potentially stabilizes training and improves generalization in case of complex tissue morphologies. For optimization, we employ the \textit{AdamW} \cite{Loshchilov_Hutter_2019} to counter unbalanced class scenarios, with cosine annealing learning rate scheduler.

We utilize the scikit-learn library \cite{Van_der_Walt_Schönberger_etal._2014} and HoVer-Net \cite{Graham_Vu_etal._2019} implementations of $R^2$ (the coefficient of determination) and $PQ$ (panoptic quality) to evaluate our experiments. Complete mathematical formulations and detailed explanations of these metrics are provided in \hyperref[chap:S5]{Appendix S5}. To compute confidence intervals, we use nonparametric bootstrapping, where after calculating the metric on the full sample, we generated 1000 bootstrap replicates by resampling with replacement and then determined the 95\% confidence intervals as the 2.5th and 97.5th percentiles of the resulting empirical distribution.

%\hfill

The model comparisons are summarized in \hyperref[tab:2]{Table 2}. The H‑Optimus-based model achieves higher $R^2$ across all cell classes compared to the ResNet50-based model, which means that its predictions are more closely aligned with the PanNuke cell counts, indicating a stronger correlation with the observed data. Notably, the improvement of $R^2_{dead}$ may be an indicator of better global contextual representations provided by the foundation model backbone. In terms of segmentation and classification quality combined, measured by the PQ score, the H‑Optimus-based model demonstrates notable improvements across most cell classes. Overall, the average $R^2$ improved from 0.575 to 0.871, while the average $PQ$ score improved from 0.450 to 0.492, demonstrating better performance of the H-Optimus-based model.

\begin{table}[h!]
\renewcommand{\arraystretch}{1.5}
  \centering
  \caption{Cell quantification metrics for baseline and proposed models (CI 95\%).}
  \label{tab:2}
  \begin{tabular}{|l|c|c|}
    \hline
    %\rowcolor{gray!30}
    Metric             & Resnet50-based            & H-optimus-based              \\
    \hline
    $R^2_{neoplastic}$    & 0.681 (0.576--0.769)       & \textbf{0.941 (0.917--0.960)} \\
    \hline
    $R^2_{inflammatory}$  & 0.863 (0.778--0.903)       & \textbf{0.949 (0.918--0.966)} \\
    \hline
    $R^2_{connective}$    & 0.600 (0.488--0.698)       & 0.609 (0.436--0.772)          \\
    \hline
    $R^2_{dead}$          & 0.097 (-11.389--0.669)     & 0.925 (0.404--0.982)          \\
    \hline
    $R^2_{epithelial}$    & 0.635 (0.490--0.747)       & \textbf{0.930 (0.886--0.964)} \\
    \hline
    $PQ_{neoplastic}$       & 0.517 (0.499--0.535)       & \textbf{0.589 (0.575--0.604)} \\
    \hline
    $PQ_{inflammatory}$     & 0.455 (0.429--0.482)       & \textbf{0.528 (0.507--0.549)} \\
    \hline
    $PQ_{connective}$       & 0.416 (0.400--0.431)       & \textbf{0.451 (0.436--0.465)} \\
    \hline
    $PQ_{dead}$             & 0.374 (0.342--0.408)       & 0.292 (0.209--0.365)          \\
    \hline
    $PQ_{epithelial}$       & 0.488 (0.460--0.519)       & \textbf{0.599 (0.579--0.618)} \\
    \hline
  \end{tabular}
\end{table}

Our results  show that integrating the H‑Optimus foundation model within the UNETR architecture enhances the model's ability to segment and classify cells across diverse tissues from PanNuke data. The pretrained transformer encoder provides robust feature representations, resulting in higher average $R^2$ and $PQ$ scores compared to the CNN-based model. This leads to more reliable cell quantification and more accurate downstream analysis. Additionally, the streamlined fine-tuning process reduces computational overhead and training time, making the model more adaptable for new data.

Despite these advancements, the foundation model-based approach does not fully resolve all challenges related to cell segmentation and classification. We observe lower metric scores for underrepresented classes in the training data. Furthermore, foundation models typically encompass billions of parameters, resulting in substantial computational and memory requirements. It therefore poses challenges for deployment in resource-constrained environments, limiting their practical applicability in certain clinical settings.

\section{Model optimization via Knowledge Distillation}

To address the limitations posed by the extensive size of foundation models, we implement knowledge distillation — a model compression technique that leverages the teacher-student paradigm \cite{Hinton_Vinyals_etal._2015}. By training a smaller, more efficient student model to replicate the output of a larger, pre-trained teacher model, we retain performance while significantly reducing the model's complexity and resource requirements (\hyperref[fig:fig6]{Figure 6}).

\begin{figure}[h!]
    \centering
    \includegraphics[width=\textwidth, height=0.45\textheight, keepaspectratio]{images/Figure_6.pdf}
    \caption{Knowledge distillation framework for training a student model using a pre-trained teacher}
    \label{fig:fig6}
\end{figure}

We employ knowledge distillation to compress the H‑Optimus-based teacher model into a more efficient student model. The teacher model is the modified UNETR architecture with the H‑Optimus foundation model described in the previous chapter. The student model is based on a UNet architecture augmented with residual connections and incorporates a smaller ViT encoder with 9 million parameters \cite{Steiner_Kolesnikov_etal._2022,Wightman_2019}. 

First, we fine-tune the teacher model using the refined dataset from the cross-relabeling procedure (Section 2). Initially we train the decoder of the teacher model while keeping the encoder weights frozen. We split the refined dataset into train (70\%), validation (20\%) and test (10\%) subsets (\hyperref[fig:S2]{Appendix Figure S2 (4)}). During fine-tuning, we use the train and validation subsets, while leaving the test subset for model evaluation. We set the training procedure and model hyperparameters to be identical to those that were used to demonstrate the utility of foundation models for the simultaneous cell segmentation and classification task.

Next, we perform knowledge distillation from teacher to student using the refined dataset used to fine-tune the teacher model. The student model is trained to replicate the teacher model's outputs. We utilize a specialized loss function that aligns the student's predicted probability distribution with the teacher's, incorporating the teacher's class probability distribution derived from the output. Following the methodology of Hinton et al. \cite{Hinton_Vinyals_etal._2015}, we experiment with various hyperparameter settings for the temperature ($T$) and the balancing coefficients ($\alpha$ and $\beta$) in the loss function. We vary $T$ from 1 to 20 and adjust $\alpha$ and $\beta$ to balance the distillation and student losses. Through iterative tuning and evaluation, we identify that setting $T=14$, $\alpha=0.3$, and $\beta=0.7$ yields a configuration that converges and closely approximates the teacher model's performance during training.

Finally, we assess the performance of both models using the $R^2$ and $PQ$ (defined in \hyperref[chap:S5]{Appendix S5}) on the test set of the refined dataset (\hyperref[tab:3]{Table 3}). We observe that the 95\% confidence intervals overlap for most cell types, so we cannot claim statistically significant performance differences between the teacher and student models. One exception appears in the neoplastic class. The teacher model produces an $R^2$ of 0.919, while the student model shows an $R^2$ of 0.852. In addition, the student model achieves higher $PQ$ values for the neoplastic and connective classes, though the confidence intervals show overlap.

\begin{table}[h!]
\renewcommand{\arraystretch}{1.5}
  \centering
  \caption{Cell quantification metrics for teacher and distilled student models (CI 95\%).}
  \label{tab:3}
  \begin{tabular}{|l|c|c|}
    \hline
    %\rowcolor{gray!30}
    Metric & Teacher & Student \\
    \hline
    $R^2_{neoplastic}$    & \textbf{0.919} (0.898--0.939) & 0.852 (0.800--0.891) \\
    \hline
    $R^2_{lymphocyte}$    & 0.969 (0.956--0.977)         & 0.969 (0.956--0.978) \\
    \hline
    $R^2_{connective}$    & 0.694 (0.548--0.809)         & 0.618 (0.469--0.741) \\
    \hline
    $R^2_{dead}$          & 0.755 (0.400--0.908)         & 0.424 (0.100--0.731) \\
    \hline
    $R^2_{epithelial}$    & 0.922 (0.870--0.958)         & 0.843 (0.738--0.917) \\
    \hline
    $R^2_{macrophage}$    & 0.384 (-0.369--0.724)        & 0.704 (0.352--0.859) \\
    \hline
    $R^2_{neutrofil}$     & 0.854 (0.578--0.929)         & 0.833 (0.502--0.925) \\
    \hline
    $PQ_{neoplastic}$       & 0.581 (0.569--0.593)         & 0.601 (0.588--0.613) \\
    \hline
    $PQ_{lymphocyte}$       & 0.536 (0.520--0.553)         & 0.563 (0.544--0.579) \\
    \hline
    $PQ_{connective}$       & 0.436 (0.421--0.451)         & 0.457 (0.441--0.474) \\
    \hline
    $PQ_{dead}$             & 0.272 (0.235--0.315)         & 0.279 (0.201--0.369) \\
    \hline
    $PQ_{epithelial}$       & 0.522 (0.500--0.545)         & 0.530 (0.506--0.555) \\
    \hline
    $PQ_{macrophage}$       & 0.524 (0.459--0.588)         & 0.474 (0.405--0.543) \\
    \hline
    $PQ_{neutrofil}$        & 0.541 (0.490--0.592)         & 0.565 (0.522--0.607) \\
    \hline
  \end{tabular}
\end{table}


We further decompose the $PQ$ metric into its $SQ$ and $DQ$ components (\hyperref[tab:S6]{Appendix Table S6}). Both models produce nearly identical $SQ$ values, which indicates that they predict instance boundaries with similar precision. Although the student model shows some improvement in $DQ$ scores for certain classes, the confidence intervals overlap and do not confirm a statistically significant difference.

We observe that the student and teacher models yield comparable detection performance despite the student model using a much smaller and simpler architecture. A model with fewer parameters reduces the risk of overfitting when training data are scarce relative to the model’s complexity \cite{Farias_Ludermir_etal._2022}. The knowledge distillation process also encourages the student model to focus on the most generalizable detection features learned from the teacher. These factors enable the student model to achieve similar detection performance across different cell types.

Additionally, considering the model sizes reported in \hyperref[tab:4]{Table 4}, the distilled model achieves a significant reduction compared to the teacher model, with a 48-fold decrease in parameter count and a 5.5-fold reduction in on-disk size. In inference mode, the teacher model requires 16 GB of VRAM for a batch size of 32, while the distilled model only needs 3 GB of VRAM for the same batch size. These reductions make the distilled model significantly more practical for fine-tuning and deployment in resource-constrained environments.

\begin{table}[h!]
\renewcommand{\arraystretch}{1.5}
  \centering
  \caption{Parameter counts and size of teacher and distilled model}
  \label{tab:4}
  \adjustbox{max width=\textwidth}{%
  \begin{tabular}{|l|c|c|c|}
    \hline
    %\rowcolor{gray!30}
    Metric & H-optimus-based (Teacher) & mobileViT-based (Student) & Magnitude of difference \\
    \hline
    Parameters count       & 1,158,917,906   & \textbf{24,093,393}   & \textbf{48x}  \\
    \hline
    Estimated Total Size (MB) & 87,912       & \textbf{15,935}    & \textbf{5.5x} \\
    \hline
  \end{tabular}%
}
\end{table}

%\hfill

With recent advancements in complex network architectures and the use of pretrained encoders to achieve state-of-the-art performance \cite{Baumann_Dislich_etal._2024,Hörst_Rempe_etal._2024} in cell segmentation and classification tasks, model size, computational complexity, and processing times have increased. This limits the scalability and accessibility of these models. As we demonstrate, this may be mitigated using knowledge distillation. Studies in the field of natural language processing have demonstrated the efficacy of knowledge distillation in retaining the capabilities of the teacher model while achieving significant reductions in size and complexity \cite{Huangpu_Gao_2024,Sun_Yu_etal.}. 

We demonstrate the feasibility of knowledge distillation in digital pathology, specifically for cell segmentation and classification tasks. Moreover, we achieve this performance while also significantly reducing the parameter count. In addressing the challenge of knowledge transfer, we found that distillation from a transformer-based model to a smaller transformer is more straightforward than attempting to map transformer features to CNN blocks. In our experiments, using a CNN-based network as a student results in worse cell quantification performance due to the structural constraints of CNN feature space dimensions. 

Although our primary approach relies on a transformer-based student model that performs well, it can be further optimized to incorporate advantages from CNN architectures. For example, employing alternative techniques such as using ViT adapters \cite{Chen_Duan_etal._2023} or $1 \times 1$ convolutions to adjust feature map sizes may be beneficial for harnessing CNN advantages like enhanced local feature extraction. Moreover, if additional performance improvements are desired, the process can be further enhanced by applying supplementary knowledge distillation techniques, such as self-distillation \cite{Zhang_Song_etal._2019} or online distillation \cite{Houyon_Cioppa_etal._2023}.

Despite these promising results, further validation on independent datasets is necessary to fully understand the model's limitations. Underrepresented classes may pose challenges when addressing complex cases. Pathologists need to validate these models to adopt them in clinical settings. While the distilled models are smaller and more deployable, a technological gap persists because pathologists traditionally rely on established methods for inspecting WSIs and diagnosing diseases. Addressing the complexities involved in deploying models for inference and supporting pathologists in adopting new tools is essential for integrating these models into clinical workflows.

\section{Model integration with QuPath}
Digital pathology tools with graphical user interfaces are essential for visualizing and analyzing WSIs. To make our student model useful in clinical pathology workflows, it needs to be integrated into a tool that enables inspecting regions, creating annotations, and providing quantitative analyses of biomarkers. Therefore, we integrate the trained student model from the previous chapter into the QuPath open‑source platform \cite{Bankhead_Loughrey_etal._2017}. QuPath provides the required annotation, visualization, and analysis tools to interpret complex histological data, including workflows for cell segmentation, classification, and quantification (\hyperref[fig:fig7]{Figure 7}). 

\begin{figure}[h!]
    \centering
    \includegraphics[width=\textwidth]{images/Figure_7.pdf}
    \caption{Visualization of model-generated cell quantification annotations (left) and the corresponding unannotated slide (right) in QuPath}
    \label{fig:fig7}
\end{figure}

To identify the regions in a WSI critical for prognosticating tumor development, such as specific tumor areas or border regions without overlapping healthy tissue, the pathologist uses QuPath to outline these regions. Then, the pathologist initiates a cell segmentation and classification script through the QuPath interface for the selected regions. The resulting annotations and quantified cell information are then directly overlaid onto the WSI in the QuPath interface. Additional design and implementation details are in \hyperref[chap:S7]{Appendix S7}. 

Two common approaches for integrating deep learning models into QuPath are Java‑based native QuPath extensions \cite{Goldsborough_Philps_etal._2024} and the execution of RESTful API requests to a model server coupled with handling the response via an extension, as demonstrated in the application of cell segmentation models applied to immunofluorescence images \cite{Sugawara_2023}. While the community is actively working on these integration strategies, there is currently no universal solution that fully addresses all integration and performance requirements.

Extensions may offer better integration with QuPath, allowing slightly improved performance and more widespread usage of the built-in QuPath models, but they lack the flexibility to customize models and modify their behavior. For example, the newest version of QuPath includes models such as StarDist \cite{Weigert_Schmidt} and InstanSeg \cite{Goldsborough_Philps_etal._2024} that can perform cell segmentation. Both models pose limitations when applied to simultaneous cell segmentation and classification. StarDist performs well only on convex, round shapes by design, whereas some neoplastic, inflammatory, and connective cells exhibit complex and non-convex shapes. InstanSeg provides only semantic segmentation without assigning classes to the segmented cells.

%\hfill

In contrast, our approach offers an alternative integration strategy. It utilizes the paquo library to directly interact with QuPath’s internal application programming interface from within Python. This enables data exchange and processing without the need for intermediate conversion steps and provides greater control over model customization, retraining, and the incorporation of custom processing steps.

The integration of our custom model with QuPath underscores its potential to significantly enhance the diagnostic process by reducing the time burden on pathologists and enabling them to focus on more complex interpretative tasks using familiar software. Leveraging a tool that is already well-established among pathologists increases the likelihood of its adoption into daily clinical workflows. The quantitative data generated through the automated workflow is critical for both clinical decision-making and research, facilitating more accurate biomarker analysis, enabling robust statistical evaluations, and supporting hypothesis generation and testing. Additionally, by streamlining cell segmentation and classification, the tool enhances the scalability and reproducibility of pathological assessments, ultimately contributing to improved diagnostic accuracy and patient outcomes.

\section{Conclusion and future work}

In this study, we address critical challenges in digital pathology and tackle the usability and deployment issues of the developed models in standard computing environments without the need for high-performance computing systems. Our multi-faceted approach encompasses data refinement through cross-relabeling, leveraging foundation models for robust cell segmentation and classification, optimizing model performance via knowledge distillation, and integrating the optimized model into the QuPath software for practical application. This approach is used to construct a capable, versatile, and adjustable model for cell segmentation and classification, with enhanced performance and usability.

\begin{sloppypar}
While our approach shows potential in the field of computational pathology, certain limitations persist. 
For example, our implementation currently exhibits lower performance in detecting macrophages. 
This serves as an instance of the broader challenge of accurately identifying complex cell types. In order to address this issue, extending our approach to incorporate additional data sources, exploring alternative modeling approaches, and integrating other imaging modalities such as immunohistochemical staining may help improve detection accuracy. Moreover, although the distilled model reduces computational demands, integrating advanced deep learning models into clinical practice requires addressing technological gaps and potential resistance to adopting new tools within established diagnostic processes.
\end{sloppypar}

Future work could focus on several key areas to refine the proposed approach and facilitate its adoption in clinical environments. Enhancing the cell-relabeling process with additional datasets \cite{Graham_Jahanifar_etal._2021} could improve the representation of underrepresented cell types and enhance overall model performance. Also, incorporating additional data sources, such as multi-modal imaging or complementary staining methods, may address limitations related to cell type differentiation and class imbalance. Exploring other foundation models \cite{Vorontsov_Bozkurt_etal._2024,Zimmermann_Vorontsov_etal._2024} or introducing additional modalities \cite{Ding_Wagner_etal._2024,Vaidya_Zhang_etal._2025} may provide alternative architectures better suited to specific tasks or offer improved efficiency. Implementing more complex knowledge distillation techniques \cite{Houyon_Cioppa_etal._2023,Zhang_Song_etal._2019} could further optimize the model's performance and adaptability. Additionally, deeper integration with QuPath or other digital pathology software could provide pathologists more control over cell quantification analysis directly within the QuPath interface, thereby increasing accessibility and usability. Such enhancements would not only refine model performance but also ensure greater adaptability and scalability within various clinical environments. Finally, extensive validation of the model by pathologists and benchmarking against independent datasets are essential steps toward establishing the model's reliability and fostering confidence in its clinical utility.

\section*{Acknowledgments} 
This work was funded in part by the Research Council of Norway grant no. 309439 SFI Visual Intelligence, and the North Norwegian Health Authority grant no. HNF1521-20.

\bibliographystyle{IEEEtran}
\begin{sloppypar}
\begin{thebibliography}{99}

\bibitem{chaplot2020neural} Chaplot, Devendra Singh, et al. "Neural topological slam for visual navigation." Proceedings of the IEEE/CVF conference on computer vision and pattern recognition. 2020.

\bibitem{maksymets2021thda} Maksymets, Oleksandr, et al. "Thda: Treasure hunt data augmentation for semantic navigation." Proceedings of the IEEE/CVF International Conference on Computer Vision. 2021.

\bibitem{mezghan2022memory} Mezghan, Lina, et al. "Memory-augmented reinforcement learning for image-goal navigation." 2022 IEEE/RSJ International Conference on Intelligent Robots and Systems (IROS). IEEE, 2022.

\bibitem{al2022zero} Al-Halah, Ziad, Santhosh Kumar Ramakrishnan, and Kristen Grauman. "Zero experience required: Plug \& play modular transfer learning for semantic visual navigation." Proceedings of the IEEE/CVF Conference on Computer Vision and Pattern Recognition. 2022.

\bibitem{ye2021auxiliary} Ye, Joel, et al. "Auxiliary tasks and exploration enable objectgoal navigation." Proceedings of the IEEE/CVF international conference on computer vision. 2021.

\bibitem{chaplot2020object} Chaplot, Devendra Singh, et al. "Object goal navigation using goal-oriented semantic exploration." Advances in Neural Information Processing Systems 33 (2020)

\bibitem{ramakrishnan2022poni} Ramakrishnan, Santhosh Kumar, et al. "Poni: Potential functions for objectgoal navigation with interaction-free learning." Proceedings of the IEEE/CVF Conference on Computer Vision and Pattern Recognition. 2022.

\bibitem{ramrakhya2022habitat} Ramrakhya, Ram, et al. "Habitat-web: Learning embodied object-search strategies from human demonstrations at scale." Proceedings of the IEEE/CVF Conference on Computer Vision and Pattern Recognition. 2022.

\bibitem{mousavian2019visual} Mousavian, Arsalan, et al. "Visual representations for semantic target driven navigation." 2019 International Conference on Robotics and Automation (ICRA). IEEE, 2019.

\bibitem{dhariwal2021diffusion} Dhariwal, Prafulla, and Alexander Nichol. "Diffusion models beat gans on image synthesis." Advances in neural information processing systems 34 (2021)

\bibitem{ho2022classifier} Ho, Jonathan, and Tim Salimans. "Classifier-free diffusion guidance." arXiv preprint arXiv:2207.12598 (2022).

\bibitem{nichol2021glide} Nichol, Alex, et al. "Glide: Towards photorealistic image generation and editing with text-guided diffusion models." arXiv preprint arXiv:2112.10741 (2021)

\bibitem{brooks2023instructpix2pix} Brooks, Tim, Aleksander Holynski, and Alexei A. Efros. "Instructpix2pix: Learning to follow image editing instructions." Proceedings of the IEEE/CVF Conference on Computer Vision and Pattern Recognition. 2023.

\bibitem{fu2023guiding} Fu, Tsu-Jui, et al. "Guiding instruction-based image editing via multimodal large language models." arXiv preprint arXiv:2309.17102 (2023).

\bibitem{geng2024instructdiffusion} Geng, Zigang, et al. "Instructdiffusion: A generalist modeling interface for vision tasks." Proceedings of the IEEE/CVF Conference on Computer Vision and Pattern Recognition. 2024.

\bibitem{zhou2024minedreamer} Zhou, Enshen, et al. "Minedreamer: Learning to follow instructions via chain-of-imagination for simulated-world control." arXiv preprint arXiv:2403.12037 (2024).

\bibitem{zhou2023esc} Zhou, Kaiwen, et al. "Esc: Exploration with soft commonsense constraints for zero-shot object navigation." International Conference on Machine Learning. PMLR, 2023.

\bibitem{yu2023l3mvn} Yu, Bangguo, Hamidreza Kasaei, and Ming Cao. "L3mvn: Leveraging large language models for visual target navigation." 2023 IEEE/RSJ International Conference on Intelligent Robots and Systems (IROS). IEEE, 2023.

\bibitem{gadre2023cows} Gadre, Samir Yitzhak, et al. "Cows on pasture: Baselines and benchmarks for language-driven zero-shot object navigation." Proceedings of the IEEE/CVF Conference on Computer Vision and Pattern Recognition. 2023.

\bibitem{shah2023navigation} Shah, Dhruv, et al. "Navigation with large language models: Semantic guesswork as a heuristic for planning." Conference on Robot Learning. PMLR, 2023.

\bibitem{cai2024bridging} Cai, Wenzhe, et al. "Bridging zero-shot object navigation and foundation models through pixel-guided navigation skill." 2024 IEEE International Conference on Robotics and Automation (ICRA). IEEE, 2024.

\bibitem{yu2023co} Yu, Bangguo, Hamidreza Kasaei, and Ming Cao. "Co-NavGPT: Multi-robot cooperative visual semantic navigation using large language models." arXiv preprint arXiv:2310.07937 (2023).

\bibitem{wu2024voronav} Wu, Pengying, et al. "Voronav: Voronoi-based zero-shot object navigation with large language model." arXiv preprint arXiv:2401.02695 (2024).

\bibitem{qin2023mp5} Qin, Yiran, et al. "Mp5: A multi-modal open-ended embodied system in minecraft via active perception." arXiv preprint arXiv:2312.07472 (2023).

\bibitem{du2024learning} Du, Yilun, et al. "Learning universal policies via text-guided video generation." Advances in Neural Information Processing Systems 36 (2024).

\bibitem{ajay2024compositional} Ajay, Anurag, et al. "Compositional foundation models for hierarchical planning." Advances in Neural Information Processing Systems 36 (2024).

\bibitem{liang2024skilldiffuser} Liang, Zhixuan, et al. "Skilldiffuser: Interpretable hierarchical planning via skill abstractions in diffusion-based task execution." Proceedings of the IEEE/CVF Conference on Computer Vision and Pattern Recognition. 2024.

\bibitem{heusel2017gans} Heusel, Martin, et al. "Gans trained by a two time-scale update rule converge to a local nash equilibrium." Advances in neural information processing systems 30 (2017).

\bibitem{zhang2018unreasonable} Zhang, Richard, et al. "The unreasonable effectiveness of deep features as a perceptual metric." Proceedings of the IEEE conference on computer vision and pattern recognition. 2018.

\bibitem{brown2020language} Brown, Tom B. "Language models are few-shot learners." arXiv preprint arXiv:2005.14165 (2020).

\bibitem{podell2023sdxl} Podell, Dustin, et al. "Sdxl: Improving latent diffusion models for high-resolution image synthesis." arXiv preprint arXiv:2307.01952 (2023).

\bibitem{brohan2022rt} Brohan, Anthony, et al. "Rt-1: Robotics transformer for real-world control at scale." arXiv preprint arXiv:2212.06817 (2022).

\bibitem{brohan2023rt} Brohan, Anthony, et al. "Rt-2: Vision-language-action models transfer web knowledge to robotic control." arXiv preprint arXiv:2307.15818 (2023).

\bibitem{li2024manipllm} Li, Xiaoqi, et al. "Manipllm: Embodied multimodal large language model for object-centric robotic manipulation." Proceedings of the IEEE/CVF Conference on Computer Vision and Pattern Recognition. 2024.

\bibitem{shah2023vint} Shah, Dhruv, et al. "ViNT: A foundation model for visual navigation." arXiv preprint arXiv:2306.14846 (2023).

\bibitem{liu2024visual} Liu, Haotian, et al. "Visual instruction tuning." Advances in neural information processing systems 36 (2024).

\bibitem{hu2021lora} Hu, Edward J., et al. "Lora: Low-rank adaptation of large language models." arXiv preprint arXiv:2106.09685 (2021).

\bibitem{qin2023supfusion} Qin, Yiran, et al. "SupFusion: Supervised LiDAR-camera fusion for 3D object detection." Proceedings of the IEEE/CVF International Conference on Computer Vision. 2023.

\bibitem{qin2024worldsimbench} Qin, Yiran, et al. "Worldsimbench: Towards video generation models as world simulators." arXiv preprint arXiv:2410.18072 (2024).

\bibitem{yu2025gamefactory} Yu, Jiwen, et al. "GameFactory: Creating New Games with Generative Interactive Videos." arXiv preprint arXiv:2501.08325 (2025).

\bibitem{zhou2024code} Zhou, Enshen, et al. "Code-as-Monitor: Constraint-aware Visual Programming for Reactive and Proactive Robotic Failure Detection." arXiv preprint arXiv:2412.04455 (2024).

\bibitem{zhang2024ad} Zhang, Zaibin, et al. "AD-H: Autonomous Driving with Hierarchical Agents." arXiv preprint arXiv:2406.03474 (2024).

\bibitem{wang2024toward} Wang, Chaoqun, et al. "Toward Accurate Camera-based 3D Object Detection via Cascade Depth Estimation and Calibration." arXiv preprint arXiv:2402.04883 (2024).

\bibitem{huang2024story3d} Huang, Yuzhou, et al. "Story3d-agent: Exploring 3d storytelling visualization with large language models." arXiv preprint arXiv:2408.11801 (2024).

\bibitem{savinov2018semi} Savinov, Nikolay, Alexey Dosovitskiy, and Vladlen Koltun. "Semi-parametric topological memory for navigation." arXiv preprint arXiv:1803.00653 (2018).

\bibitem{majumdar2022zson} Majumdar, Arjun, et al. "Zson: Zero-shot object-goal navigation using multimodal goal embeddings." Advances in Neural Information Processing Systems 35 (2022): 32340-32352.

\bibitem{yadav2023offline} Yadav, Karmesh, et al. "Offline visual representation learning for embodied navigation." Workshop on Reincarnating Reinforcement Learning at ICLR 2023. 2023.

\bibitem{yadav2023ovrl} Yadav, Karmesh, et al. "Ovrl-v2: A simple state-of-art baseline for imagenav and objectnav." arXiv preprint arXiv:2303.07798 (2023).

\bibitem{sun2024fgprompt} Sun, Xinyu, et al. "FGPrompt: fine-grained goal prompting for image-goal navigation." Advances in Neural Information Processing Systems 36 (2024).

\bibitem{zhu2017target} Zhu, Yuke, et al. "Target-driven visual navigation in indoor scenes using deep reinforcement learning." 2017 IEEE international conference on robotics and automation (ICRA). IEEE, 2017.

\bibitem{koh2024generating} Koh, Jing Yu, Daniel Fried, and Russ R. Salakhutdinov. "Generating images with multimodal language models." Advances in Neural Information Processing Systems 36 (2024).

\bibitem{krantz2022instance} Krantz, Jacob, et al. "Instance-specific image goal navigation: Training embodied agents to find object instances." arXiv preprint arXiv:2211.15876 (2022).

\bibitem{schulman2017proximal} Schulman, John, et al. "Proximal policy optimization algorithms." arXiv preprint arXiv:1707.06347 (2017).

\bibitem{anderson2018evaluation} Anderson, Peter, et al. "On evaluation of embodied navigation agents." arXiv preprint arXiv:1807.06757 (2018).

\bibitem{lin2024navcot} Lin, Bingqian, et al. "NavCoT: Boosting LLM-Based Vision-and-Language Navigation via Learning Disentangled Reasoning." arXiv preprint arXiv:2403.07376 (2024).

\bibitem{NavGPT} Zhou, Gengze, Yicong Hong, and Qi Wu. "Navgpt: Explicit reasoning in vision-and-language navigation with large language models." Proceedings of the AAAI Conference on Artificial Intelligence.

\bibitem{hahn2021no} Hahn, Meera, et al. "No rl, no simulation: Learning to navigate without navigating." Advances in Neural Information Processing Systems 34 (2021): 26661-26673.

\bibitem{li2025t2isafety} Li, Lijun, et al. "T2ISafety: Benchmark for Assessing Fairness, Toxicity, and Privacy in Image Generation." arXiv preprint arXiv:2501.12612 (2025).

\bibitem{an2024agfsync} An, Jingkun, et al. "AGFSync: Leveraging AI-Generated Feedback for Preference Optimization in Text-to-Image Generation." arXiv preprint arXiv:2403.13352 (2024).


\end{thebibliography}
\end{sloppypar}

\clearpage
\beginsupplement
\section*{Appendix}
\renewcommand{\thesubsection}{S\arabic{subsection}}

\subsection{\label{chap:S1}PanNuke and MoNuSAC preprocessing}
The PanNuke dataset comprises a set of 7,901 RGB patches, each with dimensions of $256 \times 256$ pixels, which we set as the standard patch size for our analysis. In contrast, the MoNuSAC dataset encompasses 294 images of heterogeneous dimensions. To standardize the MoNuSAC images with our experiments, we implement a standardization protocol. Specifically, for images exceeding the dimensions of $256 \times 256$ pixels, we segment them into equal-sized patches and apply mirror padding to the remaining portions to avoid information loss at the peripherals. Patches with dimensions less than $128 \times 128$ pixels are excluded from the dataset due to the insufficient resolution to capture relevant cellular details. For patches where either dimension falls between 128 and 256 pixels, we employ upsampling to achieve the standard patch size. As a result, we obtain a total of 2,823 RGB patches derived from the MoNuSAC dataset for subsequent analysis. For additional details on the MoNuSAC data preparation process, refer to the source code \cite{Shvetsov_2025a}.
\clearpage

\subsection{\label{chap:S2}Data usage for the methodology}

\counterwithin{figure}{subsection}
\renewcommand{\thefigure}{S\arabic{subsection}}

\begin{figure}[h!]
    \centering
    \includegraphics[width=\textwidth, height=0.85\textheight, keepaspectratio]{images/A2.pdf}
    \caption{Overview of the methodology for cross-labeling, dataset refinement, and model comparison. (1) Cross-relabeling - training and testing cell classification models, (2) Cross-relabeling - using cell classification models to create refined dataset, (3) Fine-tuning and training models for comparison, (4) Student knowledge distillation with refined dataset}
    \label{fig:S2}
\end{figure}
\clearpage

\subsection{\label{chap:S3}Confusion matrices for classification models}
\counterwithin{figure}{subsection}
\renewcommand{\thefigure}{S\arabic{subsection}.\arabic{figure}}

\begin{figure}[h!]
    \centering
    \includegraphics[width=\textwidth, height=0.4\textheight, keepaspectratio]{images/A3_1.pdf}
    \caption{Confusion matrix for PanNuke trained model}
    \label{fig:S3.1}
\end{figure}

\begin{figure}[h!]
    \centering
    \includegraphics[width=\textwidth, height=0.4\textheight, keepaspectratio]{images/A3_2.pdf}
    \caption{Confusion matrix for MoNuSAC trained model}
    \label{fig:S3.2}
\end{figure}

\clearpage

\subsection{\label{chap:S4}Datasets cell counts}

\counterwithin{table}{subsection}
\renewcommand{\thetable}{S\arabic{subsection}}

\begin{table}[h!]
\renewcommand{\arraystretch}{2.0}
\centering
\caption{\label{tab:S4}Cell counts for PanNuke, MoNuSAC and refined datasets. Numbers in parentheses indicate preprocessed cell counts for cell classifier models training and testing.}
%\adjustbox{max width=\textwidth}{%
\begin{tabular}{|l|c|c|c|}
\hline
%\rowcolor{gray!30}
Cell type & PanNuke & MoNuSAC & Refined \\
\hline
Neoplastic & 77,403 (68,031) & - & 105,451 \\
\hline
Epithelial & 26,572 (23,207) & - & 29,926 \\
\hline
Epithelial (benign and malignant) & - & 31,402 & - \\
\hline
Inflammatory & 32,276 & - & - \\
\hline
Lymphocytes & - & 37,045 (33,104) & 65,275 \\
\hline
Neutrophils & - & 1,355 (1,252) & 3,833 \\
\hline
Macrophage & - & 1,842 (1,695) & 3,410 \\
\hline
Dead & 2,908 & - & 2,908 \\
\hline
Connective & 50,585 & - & 50,585 \\
\hline
\end{tabular}
%
%}
\end{table}



\clearpage

\subsection{\label{chap:S5}Definition of validation metrics}
\counterwithin{equation}{subsection}
\renewcommand{\theequation}{\arabic{equation}}

\subsubsection{\label{chap:S5.1}R\textsuperscript{2}}
The coefficient of determination, denoted as $R^2$, is a statistical measure that represents the proportion of variance in the dependent variable that is predictable from the independent variables. In the context of cell quantification in pathology, $R^2$ is used to assess how well the predicted quantities of different cell types in a patch align with the actual quantities observed in the ground truth data, with higher values representing more accurate quantification. $R^2$ is defined as
\begin{equation*}
R^2 = 1 - \frac{\sum_{i=1}^n (y_i - \hat{y}_i)^2}{\sum_{i=1}^n (y_i - \bar{y})^2},
\end{equation*}
where $y_i$ represents the actual number of cells of a specific type in the $i$-th image, $\hat{y}_i$ represents the predicted number of cells of that type in the $i$-th image, $\bar{y}$ is the mean of the actual numbers across all images, and $n$ is the total number of images in the dataset.

The $R^2$ metric has a range of $(-\infty, 1]$. An $R^2$ of 1 indicates perfect prediction, where all predicted values exactly match the actual values. An $R^2$ of 0 suggests that the model explains none of the variability of the response data around its mean. If $R^2$ is negative, it indicates that the model performs worse than a model that simply predicts the mean of the actual values for all observations.

\subsubsection{\label{chap:S5.2}PQ}
Panoptic Quality ($PQ$) is a comprehensive metric used to evaluate the performance of segmentation models in tasks that require both instance segmentation and classification. $PQ$ provides a single score that encapsulates both the detection accuracy (i.e., how many objects were correctly identified) and the segmentation quality (i.e., how accurately the objects' boundaries were delineated). This metric is particularly useful in multiclass scenarios where each pixel is classified into distinct categories, such as different cell types in pathology images.

$PQ$ is calculated as the product of two terms: Detection Quality ($DQ$) and Segmentation Quality ($SQ$). It can be expressed as
\begin{equation*}
PQ = DQ \cdot SQ,
\end{equation*}
where
\begin{equation*}
DQ = \frac{TP}{TP + 0.5\, FP + 0.5\, FN},
\end{equation*}
\begin{equation*}
SQ = \frac{\sum_{(p, g) \in \mathcal{M}} IoU(p, g)}{TP}.
\end{equation*}
In these formulas, $TP$ denotes the number of correctly matched instances between ground truth and prediction, $FP$ denotes the predicted instances that have no corresponding ground truth, $FN$ denotes the ground truth instances that were not detected, $IoU(p, g)$ is the Intersection over Union for a pair of matched instances $p$ (prediction) and $g$ (ground truth), and $\mathcal{M}$ is the set of matched pairs.

The $PQ$ metric is calculated for each class and is averaged across classes to provide a global performance measure.

The $PQ$ score has a range of $[0, 1.0]$, where a higher score indicates better performance in both detecting and segmenting the instances correctly. A $PQ$ of 1 signifies perfect identification and segmentation of all instances, whereas a $PQ$ of 0 indicates that no instances were correctly identified and segmented.

\clearpage

\subsection{\label{chap:S6}Segmentation and Detection quality metrics for teacher and student models}

\begin{table}[h!]
\renewcommand{\arraystretch}{2.0}
\centering
\caption{Segmentation and detection quality for student and teacher models (CI 95\%)}
\label{tab:S6}
%\adjustbox{max width=\textwidth}{%
\begin{tabular}{|l|c|c|}
\hline
%\rowcolor{gray!30}
Metric & Teacher & Student \\
\hline
$SQ_{neoplastic}$ & 0.819 (0.815--0.823) & 0.824 (0.819--0.828) \\
\hline
$SQ_{lymphocyte}$ & 0.795 (0.788--0.802) & 0.790 (0.783--0.796) \\
\hline
$SQ_{connective}$ & 0.770 (0.762--0.776) & 0.780 (0.772--0.786) \\
\hline
$SQ_{dead}$ & 0.659 (0.623--0.688) & 0.657 (0.624--0.695) \\
\hline
$SQ_{epithelial}$ & 0.780 (0.770--0.790) & 0.788 (0.779--0.797) \\
\hline
$SQ_{macrophage}$ & 0.788 (0.760--0.810) & 0.757 (0.730--0.783) \\
\hline
$SQ_{neutrofil}$ & 0.782 (0.761--0.801) & 0.775 (0.759--0.792) \\
\hline
$DQ_{neoplastic}$ & 0.706 (0.692--0.719) & 0.727 (0.712--0.741) \\
\hline
$DQ_{lymphocyte}$ & 0.675 (0.656--0.698) & 0.713 (0.691--0.734) \\
\hline
$DQ_{connective}$ & 0.566 (0.546--0.584) & 0.583 (0.565--0.602) \\
\hline
$DQ_{dead}$ & 0.410 (0.361--0.465) & 0.435 (0.306--0.561) \\
\hline
$DQ_{epithelial}$ & 0.668 (0.639--0.694) & 0.673 (0.644--0.702) \\
\hline
$DQ_{macrophage}$ & 0.657 (0.583--0.727) & 0.615 (0.531--0.703) \\
\hline
$DQ_{neutrofil}$ & 0.691 (0.625--0.753) & 0.729 (0.679--0.778) \\
\hline
\end{tabular}
%
%}
\end{table}

\clearpage

\subsection{\label{chap:S7}QuPath integration method}
We adopt an integration strategy leveraging the paquo \cite{Bayer_AG} library, a Python package that enables direct interaction with QuPath’s internal API, thereby facilitating seamless data exchange without intermediate conversion steps. The data processing pipeline (\hyperref[fig:S7]{Appendix Figure S7}) begins with the acquisition of WSIs and their associated annotations from QuPath, which are represented as Shapely \cite{Gillies_Wel_etal._2024} polygons. Utilizing paquo, we directly read, create, and modify these annotations and detections within a QuPath project in the Python environment. Images are then cropped using these polygons and processed by cell segmentation and classification models employing standard vision processing toolkits such as OpenCV, pyvips, and PyTorch. Additionally, QuPath employs Groovy scripts to initiate a Python process that starts the entire pipeline from QuPath graphical interface: fetching polygons, extracting images from them, and running deep learning model inference on the cropped images. 
The results are returned to QuPath, leveraging paquo's Python bindings to manipulate QuPath data while minimizing the computational overhead typically associated with cross-environment communication.

\counterwithin{figure}{subsection}
\renewcommand{\thefigure}{S\arabic{subsection}}

\begin{figure}[h!]
    \centering
    \includegraphics[width=\textwidth]{images/A7.pdf}
    \caption{QuPath integration workflow using Python environment}
    \label{fig:S7}
\end{figure}

Compared to traditional workflows that involve exporting annotations as GeoJSON, classifying them in Python, and reimporting them into QuPath, our approach offers several advantages. We eliminate the need to switch between programming languages, providing a cohesive and streamlined development process entirely within QuPath software and removing the necessity to use other tools. Meanwhile, we avoid storing annotations as intermediate JSON files unless required for external use or archiving. By conducting the entire inference and post-processing workflow within the Python environment, we leverage the power and flexibility of Python libraries for image processing and machine learning. This approach also enables adjustments to any set of labels and models, thereby improving its applicability.

%\hfill

The distilled model and QuPath integration code are packaged into a Docker container, enabling streamlined execution with the Docker engine. Detailed integration code and deployment instructions can be found in the GitHub repository \cite{Shvetsov_2025b}.

Despite these benefits, we acknowledge that the paquo library is a proof‑of‑concept project in its early development stage and has not been tested across all versions of QuPath.

\clearpage

\subsection{\label{chap:S8}Data and code availability statement}
All datasets, models, and code used in this study are publicly available and can be obtained from the repositories listed below. 
The PanNuke \cite{Gamper_Koohbanani_etal._2019} and MoNuSAC \cite{Verma_Kumar_etal._2021} datasets are publicly accessible, and download information along with detailed descriptions can be found in their respective articles. Preprocessing scripts for PanNuke and MoNuSAC data, as well as individual cell extraction scripts, are available on GitHub \cite{Shvetsov_2025a}. The H-Optimus foundation model used in our experiments can be downloaded from the HuggingFace repository \cite{hoptimus2024}, and model information is available on GitHub \cite{Saillard_Jenatton_etal._2024}. In addition, the integration code for QuPath and the distilled model packaged in a Docker container are provided in the repository \cite{Shvetsov_2025b}, and paquo Python library is available from the authors GitHub repository \cite{Bayer_AG}.
\clearpage

\end{document}

\else
\bibliography{master}
\fi

\newpage
\appendix
\section{A BT \btn{Action} \hfiacre{} process}

\label{app:app1}
\begin{lstlisting}[caption={The \hfiacre{} process specification of the \btn{takeoff} BT \btn{Action} node.}, numbers=left, xleftmargin=15pt, label={lst:factionh}, language=fiacre]
process btnode_takeoff_btn21 (&btnode: btnode_array, &fls: sv_fls, &battery: sv_battery) is

states start_, tick_node, success, failure, halt, halted, running, error,
   Action_takeoff, dispatch, Action_takeoff_sync, done

var callb: bool,
   report_halted:bool := false, ret_val: ret_status

from start_
   wait [0,0];
   on (btnode[takeoff_btn21].caller <> None); // Wait until we are called
   report_halted := false;
   if (btnode[takeoff_btn21].rstatus = halt_me) then // are we being instructed to halt?
       report_halted := true;
       to halt
   end;
   // Btnode Action 'takeoff_btn21' has been called (:height 1.000000  :duration 0 ) 
   btnode[takeoff_btn21].rstatus := no_ret_status; // just initializing the rstatus
to tick_node

from halt
   wait [0,0];
   // Btnode Action 'takeoff_btn21' is being halted (:height 1.000000  :duration 0 ) 
   callb := Fiacre_Action_takeoff_halt (btnode[takeoff_btn21]); //This call the external which
to halted // halts the action

from tick_node
   wait [0,0];
   to Action_takeoff

from Action_takeoff
   // synthesized action arg index  2
   btnode[takeoff_btn21].ArgIndex := 2;
   // Btnode Action 'takeoff_btn21' calling its action (start task)
   start Fiacre_Action_takeoff_task (btnode[takeoff_btn21]); // this call the Fiacre task
to Action_takeoff_sync // which handles this action.

from Action_takeoff_sync
   sync Fiacre_Action_takeoff_task ret_val; // wait until the Fiacre task returns
   // Btnode Action 'takeoff_btn21' returned (sync task)
   to dispatch

from dispatch
   wait [0,0]; // we dispatch to the proper state according to the return values
   if (ret_val = success) then
       to success
   elsif (ret_val = failure) then
       to failure
   elsif (ret_val = running) then
       to running
   else
       to error // a priori unreachable
   end

from success
   wait [0,0];
   // Btnode Action 'takeoff_btn21' (:height 1.000000  :duration 0 ) returns success.
   btnode[takeoff_btn21].rstatus := success;
   to done

from failure
   wait [0,0];
   if (report_halted) then // this is mostly for traces
       // Btnode Action 'takeoff_btn21' (:height 1.000000  :duration 0 ) returns halted failure.
       null
   else
       // Btnode Action 'takeoff_btn21' (:height 1.000000  :duration 0 ) returns failure.
       null
   end;
   btnode[takeoff_btn21].rstatus := failure;
   to done

from halted
   wait [0,0];
   // Btnode Action 'takeoff_btn21' (:height 1.000000  :duration 0 ) has been halted.
   report_halted := true;
   to failure

from running
   wait [0,0];
   // Btnode Action 'takeoff_btn21' (:height 1.000000  :duration 0 ) returns running.
   btnode[takeoff_btn21].rstatus := running;
   to done

from done
   wait [0,0];
   // Btnode Action 'takeoff_btn21' returning control to caller and back to 'start_'
   btnode[takeoff_btn21].caller := None; // we relinquish the tick and go back waiting
to start_
\end{lstlisting}

\section{A BT \btn{Sequence}  \hfiacre{} process}

\label{app:app2}
\begin{lstlisting}[caption={The \hfiacre{} process specification of a BT \btn{Sequence} node.}, numbers=left, xleftmargin=15pt, label={lst:fsequenceh}, language=fiacre]
process btnode_Sequence13_btn6 (&btnode: btnode_array) is

states start_, tick_node, success, failure, halt, halted, halt_wait, running, error,
    Fallback15_btn7, Fallback15_btn7_done, Eval25_btn12, Eval25_btn12_done,
    Fallback27_btn13, Fallback27_btn13_done, done

var next_seq: 1..3 := 1

from start_
    wait [0,0];
    on (btnode[Sequence13_btn6].caller <> None);
    if (btnode[Sequence13_btn6].rstatus = halt_me) then to halt end;
    // Btnode Sequence 'Sequence13_btn6' has been called 
    btnode[Sequence13_btn6].rstatus := no_ret_status;
to tick_node

from halt
    wait [0,0];
    // Btnode Sequence 'Sequence13_btn6' is being halted 
    if (btnode[Fallback15_btn7].rstatus = running) then
        // Btnode Sequence 'Sequence13_btn6' halting Fallback 'Fallback15_btn7'
        btnode[Fallback15_btn7].rstatus := halt_me;
        btnode[Fallback15_btn7].caller := caller_Sequence13_btn6;
        to halt_wait
    end;
    if (btnode[Eval25_btn12].rstatus = running) then
        // Btnode Sequence 'Sequence13_btn6' halting Eval 'Eval25_btn12'
        btnode[Eval25_btn12].rstatus := halt_me;
        btnode[Eval25_btn12].caller := caller_Sequence13_btn6;
        to halt_wait
    end;
    if (btnode[Fallback27_btn13].rstatus = running) then
        // Btnode Sequence 'Sequence13_btn6' halting Fallback 'Fallback27_btn13'
        btnode[Fallback27_btn13].rstatus := halt_me;
        btnode[Fallback27_btn13].caller := caller_Sequence13_btn6;
        to halt_wait
    end;
    to halted

from halt_wait
    on ((btnode[Fallback15_btn7].caller = None) and
        (btnode[Eval25_btn12].caller = None) and
        (btnode[Fallback27_btn13].caller = None));
    to halted

from tick_node
    wait [0,0];
    if (next_seq = 1) then to Fallback15_btn7 end;
    if (next_seq = 2) then to Eval25_btn12 end;
    if (next_seq = 3) then to Fallback27_btn13 end;
    to error

from Fallback15_btn7
    wait [0,0];
    // Btnode Sequence 'Sequence13_btn6' calling Fallback 'Fallback15_btn7'
    btnode[Fallback15_btn7].caller := caller_Sequence13_btn6;
    to Fallback15_btn7_done

from Fallback15_btn7_done
    wait [0,0];
    on (btnode[Fallback15_btn7].caller = None);
    // Btnode Sequence 'Sequence13_btn6' getting control back from Fallback 'Fallback15_btn7'
    if (btnode[Fallback15_btn7].rstatus = success) then
        to Eval25_btn12
    elsif (btnode[Fallback15_btn7].rstatus = failure) then
        next_seq := 1;
        to failure
    elsif (btnode[Fallback15_btn7].rstatus = running) then
        next_seq := 1;
        to running
    else
        to error
    end

from Eval25_btn12
    wait [0,0];
    // Btnode Sequence 'Sequence13_btn6' calling Eval 'Eval25_btn12'
    btnode[Eval25_btn12].caller := caller_Sequence13_btn6;
    to Eval25_btn12_done

from Eval25_btn12_done
    wait [0,0];
    on (btnode[Eval25_btn12].caller = None);
    // Btnode Sequence 'Sequence13_btn6' getting control back from Eval 'Eval25_btn12'
    if (btnode[Eval25_btn12].rstatus = success) then
        to Fallback27_btn13
    elsif (btnode[Eval25_btn12].rstatus = failure) then
        next_seq := 1;
        to failure
    elsif (btnode[Eval25_btn12].rstatus = running) then
        next_seq := 2;
        to running
    else
        to error
    end

from Fallback27_btn13
    wait [0,0];
    // Btnode Sequence 'Sequence13_btn6' calling Fallback 'Fallback27_btn13'
    btnode[Fallback27_btn13].caller := caller_Sequence13_btn6;
    to Fallback27_btn13_done

from Fallback27_btn13_done
    wait [0,0];
    on (btnode[Fallback27_btn13].caller = None);
    // Btnode Sequence 'Sequence13_btn6' getting control back from Fallback 'Fallback27_btn13'
    if (btnode[Fallback27_btn13].rstatus = success) then
        next_seq := 1;
        to success
    elsif (btnode[Fallback27_btn13].rstatus = failure) then
        next_seq := 1;
        to failure
    elsif (btnode[Fallback27_btn13].rstatus = running) then
        next_seq := 3;
        to running
    else
        to error
    end

from success
    wait [0,0];
    // Btnode Sequence 'Sequence13_btn6' returns success.
    btnode[Sequence13_btn6].rstatus := success;
    to done

from failure
    wait [0,0];
    // Btnode Sequence 'Sequence13_btn6' returns failure.
    btnode[Sequence13_btn6].rstatus := failure;
    to done

from halted
    wait [0,0];
    // Btnode Sequence 'Sequence13_btn6' has been halted.
    btnode[Sequence13_btn6].rstatus := failure;
    to done

from running
    wait [0,0];
    // Btnode Sequence 'Sequence13_btn6' returns running.
    btnode[Sequence13_btn6].rstatus := running;
    to done

from done
    wait [0,0];
    // Btnode Sequence 'Sequence13_btn6' returning control to caller and back to '_start'
    btnode[Sequence13_btn6].caller := None;
to start_
\end{lstlisting}

\clearpage
\section{\btn{Fallback} and \btn{Parallel} nodes transformation in \fiacre{}}

\begin{figure*}[!ht]
\begin{center}
\includegraphics[angle=-90,origin=c,height=0.6\textheight]{./fallback}
\caption{The \fiacre{} process modeling the \btn{Fallback} node.}
\label{fig:fallback}
\end{center}
\end{figure*}
  
\begin{figure*}[!ht]
\begin{center}
\includegraphics[angle=-90,origin=c,height=0.6\textheight]{./parallel}
\caption{The \fiacre{} process modeling the \btn{Parallel} node.}
\label{fig:parallel}
\end{center}
\end{figure*}


\clearpage
\section{The Drone BT.}

\begin{figure*}[!ht]
\begin{center}
\includegraphics[width=0.97\textwidth]{./screen-bt}
\caption{The graphical representation of the BT Listing~\ref{lst:dronebt}\code{p}\pageref{lst:dronebt}.}
\label{fig:screen-bt}
\end{center}
\end{figure*}


%\clearpage
\section{A ROS2 Nav2 BT and its BTF equivalent.}

\label{app:ros2nav2}
\begin{lstlisting}[caption={One of the ROS2 Nav2 Behavior Tree (Navigate To Pose With Replanning and Recovery)~\cite{ros2nav2:2024aa}.}, numbers=left, xleftmargin=15pt, label={lst:ros2nav2xml}, language=XML]
<root main_tree_to_execute="MainTree">
  <BehaviorTree ID="MainTree">
    <RecoveryNode number_of_retries="6" name="NavigateRecovery">
      <PipelineSequence name="NavigateWithReplanning">
        <RateController hz="1.0">
          <RecoveryNode number_of_retries="1" name="ComputePathToPose">
            <ComputePathToPose goal="{goal}" path="{path}" planner_id="GridBased"/>
            <ReactiveFallback name="ComputePathToPoseRecoveryFallback">
              <GoalUpdated/>
              <ClearEntireCostmap name="ClearGlobalCostmap-Context" 
                                service_name="global_costmap/clear_entirely_global_costmap"/>
            </ReactiveFallback>
          </RecoveryNode>
        </RateController><
        <RecoveryNode number_of_retries="1" name="FollowPath">
          <FollowPath path="{path}" controller_id="FollowPath"/>
          <ReactiveFallback name="FollowPathRecoveryFallback">
            <GoalUpdated/>
            <ClearEntireCostmap name="ClearLocalCostmap-Context" 
                                service_name="local_costmap/clear_entirely_local_costmap"/>
          </ReactiveFallback>
        </RecoveryNode>
      </PipelineSequence>
      <ReactiveFallback name="RecoveryFallback">
        <GoalUpdated/>
        <RoundRobin name="RecoveryActions">
          <Sequence name="ClearingActions">
            <ClearEntireCostmap name="ClearLocalCostmap-Subtree" 
                                service_name="local_costmap/clear_entirely_local_costmap"/>
            <ClearEntireCostmap name="ClearGlobalCostmap-Subtree" 
                                service_name="global_costmap/clear_entirely_global_costmap"/>
          </Sequence>
          <Spin spin_dist="1.57"/>
          <Wait wait_duration="5"/>
          <BackUp backup_dist="0.15" backup_speed="0.025"/>
        </RoundRobin>
      </ReactiveFallback>
    </RecoveryNode>
  </BehaviorTree>
</root>
\end{lstlisting}

\begin{lstlisting}[caption={The \code{.btf} version of the ROS2 Nav2 BT above (Listing~\ref{lst:ros2nav2xml}\code{p}\pageref{lst:ros2nav2xml}).}, numbers=left, xleftmargin=15pt, label={lst:ros2nav2btf}, language={[btf]Lisp}]
((BehaviorTree :ID MainTree ; The top level root node
     (Recovery :num_retries 6 :name NavigateRecovery
         (PipelineSequence :name NavigateWithReplanning
          (RateController :args (hz 1)
              (Recovery :num_retries 1 :name ComputePathToPose
               (Action :ID ComputePathToPose :args (goal $goal path $path planner_id GridBased))
               (ReactiveFallback :name ComputePathToPoseRecoveryFallback
                (Condition :ID GoalUpdated)
                (Action :ID ClearEntireCostmap :name ClearGlobalCostmap_Context1
                        :args (service_name global_costmap/clear_entirely_global_costmap)))))
          (Recovery :num_retries 1 :name FollowPath
              (Action :ID FollowPath :args (path $path controller_id FollowPath))
              (ReactiveFallback :name FollowPathRecoveryFallback
               (Condition :ID GoalUpdated)
               (Action :ID ClearEntireCostmap :name ClearLocalCostmap_Context2
                       :args (service_name local_costmap/clear_entirely_local_costmap)))))
         (ReactiveFallback :name RecoveryFallback
          (Condition :ID GoalUpdated)
          (RoundRobin :name RecoveryActions
              (Sequence :name ClearingActions
               (Action :ID ClearEntireCostmap :name ClearLocalCostmap_Subtree3 
                       :args ( service_name local_costmap/clear_entirely_local_costmap))
               (Action :ID ClearEntireCostmap :name ClearGlobalCostmap_Subtree4 
                       :args (service_name global_costmap/clear_entirely_global_costmap)))
              (Action :ID Spin :args (spin_dist 1.57))
              (Action :ID Wait :args (wait_duration 5))
              (Action :ID BackUp :args (backup_dist 0.15 backup_speed 0.025)))))))
\end{lstlisting}

\begin{figure*}[!ht]
\begin{center}
\centerline{\includegraphics[width=1.1\textwidth]{./screen-nav2}}
\caption{The screen dump of the Nav2 BT (See Listing~\ref{lst:ros2nav2btf}\code{p}\pageref{lst:ros2nav2btf} executing.}
\label{fig:screen-nav2}
\end{center}
\end{figure*}
 
\clearpage
\section{The Mars rover example from~\cite{Biggar:2020aa,Wang:2024aa}.}
\label{app:mars}

\begin{lstlisting}[caption={The BTF model of the Mars rover example from~\cite{Biggar:2020aa,Wang:2024aa}.}, numbers=left, xleftmargin=15pt, label={lst:mars}, language={[btf]Lisp}]
( ;example from 
  ; A framework for formal verification of behavior trees with linear temporal logic.(2020)
  ; O. Biggar and M. Zamani.
  ; also presented in 
  ; Enabling Behaviour Tree Verification via a Translation to BIP. (2024)
  ; Q Wang, H Dai, Y Zhao, M Zhang, S Bliudze
  (defsv meteo ; this defines the meteo state variable
    :states (MInit Normal Storm) ; MInit an undefined init state
    :init MInit  ; the following transitions forbid coming back to MInit
    :transitions ((MInit Normal) (MInit Storm) (Storm Normal) (Normal Storm)))

  (defsv battery
    :states (BInit Good Low) ; BInit just to says that we do not know
    :init BInit  ; the following transitions forbid coming back to BInit
    :transitions ((BInit Good) (BInit Low) (Low Good) (Good Low)))

  (defsv panel
    :states (PInit Folded Unfolded) ; PInit just to says that we do not know
    :init PInit  ; the following transitions forbid coming back to PInit
    :transitions ((PInit Folded) (PInit Unfolded) (Unfolded Folded) (Folded Unfolded)))

(BehaviorTree :name mars_rover
  (Fallback
   (Sequence
    (Eval (= battery Low)) ; if the battery is low
    (Action :ID UnfoldPanels :name unfold_panels :SF) ; we unfold the panel
    (Eval (:= panel Unfolded))) ; and the panel SV becomes unfolded
   (Sequence
    (Eval (= meteo Storm)) ; if the meteo is a storm
    (Action :ID Hibernate :name hibernate :SF) ; we hibernate
    (Eval (:= panel Folded))) ; and we fold the panel
   (Sequence
    (Action :ID DataReady :name dataready :SF) 
    (Action :ID Send :name send :SF)))))

; prove absent ( mars_rover/3/state Unfolded and mars_rover/1/state Storm ) 
\end{lstlisting}


\end{document}

%%% Local Variables:
%%% mode: latex
%%% TeX-master: t
%%% End:

