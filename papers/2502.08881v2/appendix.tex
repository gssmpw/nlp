\section{Discretization and Test Functions}\label{sec:appendix}
In order to satisfy the weak form, a solution must satisfy Eq.~\eqref{eq:ode-weak} for all possible test functions. However, in this work, we are not concerned with the solution itself, but with estimating the parameters in Eq.~\eqref{eq:ode-weak}. As the efficacy of the WENDy algorithm depends strongly on wisely choosing test function hyperparameters (smoothness, radius, etc.), in this Appendix we provide an exposition of the test function creation process in \cite{BortzMessengerDukic2023BullMathBiol}.
\subsection{Discretization of the Weak Form} \label{appendix:discWeak}

Starting from Eq.~\eqref{eq:ode-weak}, we must choose a finite set of test functions to create a vector of weak form EE residuals to be minimized. In approaches such as the Finite Element Method, this set is chosen based upon the expected properties of the solution as well as the class of equations being solved. Conversely, to perform weak form EE-based parameter estimation, we assume the availability of samples of the (possibly noisy) solution. In pursuit of the estimation goal, however, we note that how to optimally choose the properties, number, and location of the test functions remains an open problem.
Nonetheless, in \cite{MessengerBortz2021MultiscaleModelSimul}, it was reported that smoother test functions result in a more accurate convolution integral (via the Euler–Maclaurin formula), while later research revealed that test functions with wider support frequently performed better, and subsequently an algorithm to choose the support size was developed in  \cite{MessengerBortz2021JComputPhys}. These empirical observations regarding smoothness and support size have partially guided the design of the following algorithm.

From a numerical linear algebra perspective, it would be best if the test functions were orthogonal. Towards this goal, we begin by considering the following $C^\infty$ bump function
\begin{equation}
	\label{eq:test-fun}
	\varphi_k(t ; \eta, m_t)=C \exp \left(-\frac{\eta}{\left[1-(\tfrac{t-t_k}{m_t\Delta t} )^2\right]_{+}}\right)
\end{equation}
where $t_k$ is the center of the function, $m_t$ controls the radius of the support, the constant $C$ normalizes the test function such that $\|\varphi_k\|_2=1$, $\eta$ is a shape parameter, and $[\cdot]_{+}\eqdef\max (\cdot, 0)$, so that $\varphi_k(t ; m_t\Delta t)$ is supported only on $[-m_t\Delta t, m_t\Delta t]$. 
Using the algorithm below in \ref{appendix:minRadiusSelect}, we identify the minimal value for $m_t$ such that the numerical error in the convolution integral does not dominate the residual (denoted as $\underline{m}_t$). We define the set of test functions to be all test functions such that they are compactly supported over the measured time domain $(t_0,t_M)$, and have the following features: 1) the $t_k$ values coincide with the sampled timepoints and 2) allowable radii are larger than the minimal one $\underline{m}_t$. Motivated by empirical results, the shape parameter is arbitrarily fixed at $\eta=9$. This gives us a set of $K_\textrm{full}$ test functions $\{\testFun_k\}_{k=1}^{K_\text{full}}$ and we evaluate them on the grid to obtain the matrices:
\[\dTestFun_\text{full} \eqdef \begin{bmatrix}
	\testFun_1(t_0) & \cdots & \testFun_{K_\text{full}}(t_0)\\
	\vdots & \ddots & \vdots\\
	\testFun_1(t_M) & \cdots & \testFun_{K_\text{full}}(t_M)
\end{bmatrix}, \; \dot{\dTestFun}_\text{full} \eqdef \begin{bmatrix}
	\dot{\varphi}_1(t_0) & \cdots & \dot{\varphi}_{K_\text{full}}(t_0)\\
	\vdots & \ddots & \vdots\\
	\dot{\varphi}_1(t_M) & \cdots & \dot{\varphi}_{K_\text{full}}(t_M)
\end{bmatrix}\]

\subsubsection{Minimum Radius Selection.} \label{appendix:minRadiusSelect}

The algorithm by which these test functions are chosen is more fully described in \citep{BortzMessengerDukic2023BullMathBiol}, but a short summary is given here. 

Our analysis follows the equation $\dot{u}=\rhs(u)$ where $u$ is a one dimensional state variable. The result is generalized to a $D$ dimensional system by applying the radius selection in each dimension independently. We see that using integration by parts for a test function $\varphi$ gives the equality $-\langle \dot{\varphi}, u\rangle = \langle \varphi, \rhs(u) \rangle$. Integrating, we obtain
\[0 =  \int_0^T \rhs(u(t))\varphi(t) + \varphi'(t)u(t) dt = \int_0^T \frac{d}{dt}\bigl(u(t)\varphi(t)\bigr) dt. \]
Thus this integral should be zero, or at least small accounting for the presence of numerical error and noise. In this work, we use the trapezoidal rule on a uniform time grid. Because the $\varphi$ are test functions with compact support on the interior of the domain, we can define the integration error as in Proposition \ref{prop:int-error}: 
\begin{equation} \label{eq:eint} 
    e_\text{int}(M) = \frac{T}{M}\sum_{m=0}^M \frac{d}{dt}(u \varphi) (t_m)
\end{equation}
By expanding $\frac{d}{dt}\bigl(u(t) \varphi(t)\bigr)$ into its Fourier Series, where the coefficients are given by $\mathcal{F}_n[\cdot] := \frac{1}{\sqrt{T}}\int_0^T (\cdot) \exp\left[\frac{-2\pi i n }{T}t\right] dt$, we can make the following simplification:
\begin{align*} 
	\frac{d}{dt}\bigl(u(t)\varphi(t)\bigr) &= \sum_{n\in\mathbb{Z}} \mathcal{F}_n \left[\frac{d}{dt}\bigl(u(t)\varphi(t)\bigr)\right] \exp \left[\frac{ 2\pi i n}{T}t\right]\\
	&= \sum_{n\in\mathbb{Z}} \left(\frac{1}{T} \int_0^T \frac{d}{dt}\bigl(u(t)\varphi(t)\bigr)  \exp \left[\frac{ -2\pi i n}{T}t\right] dt\right)  \exp \left[\frac{ 2\pi i n}{T}t\right]\\
	&\stackrel{\text{IBP}}{=} \frac{ -2\pi i }{T} \sum_{n\in\mathbb{Z}} n\underbrace{\left( \frac{1}{T} \int_0^T u(t)\varphi(t)  \exp \left[\frac{ -2\pi i n}{T}t\right] dt \right)}_{=\mathcal{F}_n\bigl[\varphi u\bigr]}  \exp \left[\frac{ 2\pi i n}{T}t\right].
\end{align*} 
The last simplification in the integration by parts relies on the fact that $\varphi$ is a test function for the domain $[0,T]$, thus $\operatorname{supp}(\varphi) \subset (0,T)$. Now if we return to the quantity of interest, $e_{\text{int}}(M)$, by integrating both sides over the time interval we obtain: 
\[e_\text{int} = \frac{2\pi i}{M} \sum_{n\in\mathbb{Z}} n \mathcal{F}_n\bigl[\varphi u\bigr]\underbrace{\sum_{m=0}^M  \exp \left[\frac{ 2\pi n i }{M}m\right] }_{\eqdef I_n}.\] 
Notice that $\forall n \neq 0$, $I_n$ is 0 because $\exp\left[\tfrac{2\pi i}{T}t\right]dt$ is periodic on $[0,T]$. Also, $\mathcal{F}_0$ is 0 because it corresponds to the true integral. Notice that when $n = \ell M$ for some integer $\ell$, then we have that $I_{n} = M$ because the quadrature nodes align with the roots of unity for all the Fourier modes. Because the Fourier modes should decay as $M$ gets larger, we can say 
\[e_\text{int} \approx \operatorname{imag}\left\{ 2\pi n \mathcal{F}_M\bigl[\varphi u\bigr]\right\}.\]

Because we do not know $u$ analytically, the true Fourier coefficients are approximated by the Discrete Fourier Transform, $\hat{\mathcal{F}}_{n - \left\lfloor \tfrac{M}{2} \right\rfloor}[\dTrueState] = \frac{T}{M} \sum_{m=0}^{M} \dTrueState_m \exp \left[\frac{ 2\pi i n}{M} m\right]$. Thus, the largest mode we can approximate is the $\left\lfloor\frac{M}{2}\right\rfloor$ mode. Also, note that applying the same quadrature rule to a sub-sampled grid gives us an upper bound on the error: 
\[e_\text{int}(M) \leq e_\text{int}\Bigl(\bigl\lfloor\tfrac{M}{2}\bigr\rfloor\Bigr).\]
This motivates subsampling in time to inspect the error. Letting $\tilde{M} \eqdef \bigl\lfloor \frac{M}{s} \bigr\rfloor$ for some scaling factor $s>2$, we approximate the integration error when subsampled.

The last complication is that we do not have access to $\dTrueState$, so we must approximate by substituting $\dstate$: 
\[e_\text{int}(\tilde{M}) \leq \hat{e}_\text{int}(\tilde{M}) = \frac{2 pi}{\sqrt{T}} \hat{F}_{\tilde{M}}(\dstate).\]

This motivates us to search over the space of possible radii, and then select the smallest radius possible when the effects due to noise become dominate.
Qualitatively, our reasoning is that we can approximate the integration error by using the data, but as noise levels becomes more extreme the effects to noise become dominate for smaller radii, thus for higher noise we expect to select smaller radii in general. This can be seen in Figure \ref{fig:minRad}.
\begin{figure}[H] 
	\centering
	\includegraphics[width=0.9\textwidth]{minRadiusDetection.pdf}
	\caption{The results for the logistic growth function on the time domain $(0,10)$, initial condition of $0.01$ and $M=512$, looking at $\hat{\mathcal{F}}_{256}$ for a variety of radii  $m_t \in [0.01, 1]$. A vertical line indicates where we have detected a corner in the integration error surrogate}
	\label{fig:minRad}
\end{figure}

\subsubsection{Reducing the Size of the Test Function Matrix}
\newcommand{\leftSingVals}{\mathbf{Q}}
\newcommand{\rightSingVals}{\mathbf{V}}
The test function matrix $\dTestFun_\text{full}$ is often overdetermined, so in order to improve the conditioning of the system we use an SVD reduction technique to obtain a matrix $\dTestFun$ that has a better condition number $\kappa(\dTestFun) \eqdef \sigma_1(\dTestFun)  / \sigma_K(\dTestFun)$, while still retaining as much information as possible. This is done by looking for a corner in the singular values.
Let the SVD of $\dTestFun_\text{full}$ by 
\begin{align*} 
	\dTestFun_\text{full} &= \leftSingVals \operatorname{diag}([\sigma_1, \cdots, \sigma_K, \cdots]) \rightSingVals^T  
	\shortintertext{ where $\leftSingVals, \rightSingVals$ are unitary and $\sigma_1 \ge \sigma_2 \ge \ldots \ge 0$.
    We then define} \\
	\dTestFun &= \underbrace{\operatorname{diag}\Bigl(\bigl[\tfrac{1}{\sigma_1}, \cdots, \tfrac{1}{\sigma_K}\bigr]\Bigr)\leftSingVals^T}_{\mathbf{P}} \dTestFun_\text{full} 
\end{align*} 
where the cutoff $K$ is found by looking for a corner in the plot of $\sigma_i$ vs $i$.

The approach in \citep{BortzMessengerDukic2023BullMathBiol} approximates $\dot{\dTestFun}$ with a spectral method. In contrast, we compute  $\dot{\dTestFun}_\text{full}$ analytically, and then leverage the SVD of $\dTestFun$ to apply the same linear operators, $\dot{\dTestFun}=\mathbf{P}\dot{\dTestFun}_\text{full}$, to obtain a result that only has error from numerical precision,
leading to slightly better numerical integration error than in \citep{BortzMessengerDukic2023BullMathBiol}.

\section{Highlighting the Effects of Linearity in Parameters} \label{sec:linvsnonlin}
When $\rhs$ is linear in $\params$ then then we can write the residual as follows: 
\begin{equation}
	\label{eq:resLin}
	\wRes(\params;\dstatemat,\dt) = \wRhsLin(\dstatemat, \dt) \params - \wLhs(\dstatemat)
\end{equation}
where $\wRhsLin(\dstatemat, \dt) \in \R^{KD\times J}$ is a matrix-valued function that is constant with respect to $\params$. This leads to simplifications in the derivative information
which can improve computational efficiency. 

In the linear case, the Jacobian of the weak form right-hand side with respect to the parameters is the matrix $\wRhsLin(\dstatemat, \dt)$. Formally, $\nabla_{\params} \wRhs(\params; \dstatemat, \dt) = \wRhsLin(\dstatemat,\dt)$. This also simplifies implies $\nabla_{\params} \nabla_{u} \wRhs(\params; \dstatemat, \dt) = \nabla_{u} \wRhsLin(\dstatemat, \dt)$. We again drop explicit dependence on $\dstatemat$ and $\dt$ for simplicity of notation. Feeding this into our existing expression for the gradient of the weak form negative logarithm, we have
\begin{align*}
	\nabla_{\params} \ell(\params) &= 2 \wRhsLin^T \wCov(\params)^{-1} (\wRhsLin\params- \wLhs) \\
	&+ (\wRhsLin\params-\wLhs)^T(\partial_{\params_j} \wCov(\params)^{-1})(\wRhsLin\params-\wLhs).
\end{align*}
Also, notice that $\wCov(\params)$ now becomes quadratic in $\params$:
\[\wCov(\params) = \bigl(\nabla_{u} \wRhsLin [\params]+\dot{\dTestFun} \otimes \id_D\bigr) \bigl(\Sigma \otimes \id_{M+1} \bigr)\bigl(\nabla_{u} \wRhsLin [\params] + \dot{\dTestFun} \otimes \id_D\bigr)^T \]
In fact $\nabla_{u} \wRhsLin$ is the Jacobian of a matrix-valued function, and thus it is a three-dimensional tensor in $\R^{KD \times MD \times J}$. We treat it  as a linear operator acting on $\params$.  In practice this is a page-wise mat-vec across the third dimension of $\nabla_{u} \wRhsLin$. 

This causes the following simplification in the derivative of $\wCov(\params)$:
\[\partial_{\params_j} \wCov(\params) = 2\nabla_{u} \wRhsLin [\mathbf{e}_j] \bigl(\Sigma \otimes \id_{M+1} \bigr) \bigl(\nabla_{u}\wRhsLin[\params]+\dot{\dTestFun} \otimes \id_D\bigr)^T \]
where $\mathbf{e}_j \in \R^J$ is the $j^\text{th}$ canonical basis vector. This is equivalent to indexing into the $j^\text{th}$ page of $\nabla_{u}\wRhsLin$. 

The second order derivative computations also simplify beyond the evaluation of $\wCov(\params)$. Critically, observe that several terms in the Hessian are now guaranteed to be $\mathbf{0}$. In particular, we have $\forall \params$:
\[\partial_{\params_i,\params_j} \wRhsLin \params = \mathbf{0} \in \R^J \text{ and }  
\partial_{\params_i,\params_j} \nabla_{u} \wRhsLin[\params] = \mathbf{0} \in \R^{KD\times MD}. \]
Furthermore because $\nabla_{u} \wRhsLin$ is constant with respect to $\params$, then $\partial_{\params_i \params_j}\wCov(\params)^{-1}$ is constant with respect to $\params$, so this can be computed once and then reused. 

\section{Iterative Re-weighted Least Squares}\label{sec:WENDy-IRLS}

In the previous work \citep{BortzMessengerDukic2023BullMathBiol}, the covariance information was incorporated by solving the generalized least squares problem 
\[ \estim = \underset{\params \in \R^J}{\operatorname{argmin}} (\wRhs(\params;\dstatemat,\dt) - \wLhs(\dstatemat))^T\wCov(\trueParams; \dstatemat)^{-1}(\wRhs(\params;\dstatemat,\dt) - \wLhs(\dstatemat)).\]
Because $\wCov(\trueParams; \dstatemat)$ is not known, it has to be approximated with the current value of $\params$. This gives rise the iterative re-weighted least squares algorithm, which iterates as follows 
\begin{equation}
	\label{eq:WENDy-IRLSIter}
	\estim^{(i+1)} = \underset{\params \in \R^J}{\operatorname{argmin}}(\wRhs(\params) - \wLhs)^T\wCov(\estim^{(i)})^{-1}(\wRhs(\params) - \wLhs)
\end{equation}
until the iterates are sufficiently close together (and again, we drop explicit dependence on $\dstatemat$ and $\dt$).
This iteration involves computing the covariance estimate $\wCov(\params)$ then solving a weighted least squares problem. The previous work \citep{BortzMessengerDukic2023BullMathBiol} only considered problems that were linear in parameters, so in the nonlinear case, we have extended this algorithm by solving a nonlinear weighted least squares problem at each iteration where the Jacobian is computed using analytic derivative information described in Equations \eqref{eq:wnll-grad} and \eqref{eq:wnll-hess}. 

The previous work also chose the same initial guess based on the ordinary least square solution:
\[\estim^{(0)} = \underset{\params \in \R^J}{\operatorname{argmin}} \tfrac{1}{2}\|\wRhs(\params) - \wLhs\|^2_2\]
When the ODE was linear, this was done explicitly through the linear algebra. In the nonlinear case, this can no longer be done, so an initial guess is necessary. In this work, we pass all algorithms the same initial guess that is randomly sampled from parameter range specified in Table \ref{tab:odes}.
\section{Supplemental Material}
\subsection{Plots}
The following plots give detailed information on the bias, variance, MSE and coverage for all parameters for all ODEs, grouped by noise level.
\begin{figure}[H]
	\includegraphics[width=0.9\columnwidth]{logisticGrowth_UQ2.pdf}
	\caption{Left: the squared bias, variance and MSE for $p_1$ (top) and $p_2$ (bottom), as a function of noise level. Right: the coverage levels for $p_1$ (top) and $p_2$ (bottom), as a function of noise level.}
	\label{fig:logisticSup}
\end{figure}
\begin{figure}[H]
	\includegraphics[width=0.9\columnwidth]{hindmarshRose_UQ2.pdf}
	\caption{Left: the squared bias, variance and MSE for $p_1 - p_{10}$ top to bottom respectively, as a function of noise level. Right: the coverage levels for $p_1 - p_{10}$ top to bottom respectively, as a function of noise level.}
	\label{fig:hindmarshSup}
\end{figure}
\begin{figure}[H]
	\includegraphics[width=0.9\columnwidth]{lorenz_UQ2.pdf}
	\caption{Left: the squared bias, variance and MSE for $p_1 - p_3$ top to bottom respectively, as a function of noise level. Right: the coverage levels for $p_1 - p_3$ top to bottom respectively, as a function of noise level.}
	\label{fig:lorenzSup}
\end{figure}
\begin{figure}[H]
	\includegraphics[width=0.9\columnwidth]{multimodal_UQ2.pdf}
	\caption{Left: the squared bias, variance and MSE for $p_1 - p_5$ top to bottom respectively, as a function of noise level. Right: the coverage levels for $p_1 - p_5$ top to bottom respectively, as a function of noise level.}
	\label{fig:multimodalSup}
\end{figure}
\begin{figure}[H]
	\includegraphics[width=0.9\columnwidth]{goodwin_UQ2.pdf}
	\caption{Left: the squared bias, variance and MSE for $p_1 - p8$ top to bottom respectively, as a function of noise level. Right: the coverage levels for $p_1 - p_8$ top to bottom respectively, as a function of noise level.}
	\label{fig:goodwinSup}
\end{figure}
\begin{figure}[H]
	\includegraphics[width=0.9\columnwidth]{sir_UQ2.pdf}
	\caption{Left: the squared bias, variance and MSE for $p_1 - p_5$ top to bottom respectively, as a function of noise level. Right: the coverage levels for $p_1 - p_5$ top to bottom respectively, as a function of noise level.}
	\label{fig:sirSup}
\end{figure}

\subsection{Tables}
The following tables give detailed information on the bias, variance, MSE and coverage for all parameters for all ODEs, grouped by noise level.
\setlength{\tabcolsep}{6pt}
\renewcommand{\arraystretch}{1}
\begin{table}[H]
	\caption{Logistic Growth} \label{tab:logUQ1}	
	\resizebox{\textwidth}{!}{
	\begin{tabular}{lccccccccccccccccccccccccccccccccccccccccccccccccccccccccccccccccccccccccccccccc}
 &  & \multicolumn{3}{c}{1$\%$} & \multicolumn{3}{c}{2$\%$} & \multicolumn{3}{c}{3$\%$} & \multicolumn{3}{c}{4$\%$} & \multicolumn{3}{c}{5$\%$} \\
\cmidrule(l){3-5} \cmidrule(l){6-8} \cmidrule(l){9-11} \cmidrule(l){12-14} \cmidrule(l){15-17}  &  & M=1024 & M=512 & M=256 & M=1024 & M=512 & M=256 & M=1024 & M=512 & M=256 & M=1024 & M=512 & M=256 & M=1024 & M=512 & M=256 \\
\toprule
  &  Bias & 0.0021 & 0.002 & 0.0015 & 0.002 & 0.0018 & 0.00093 & 0.0018 & 0.0014 & 0.00013 & 0.0015 & 0.001 & -0.00063 & 0.0011 & 0.0006 & -0.0022 \\
  &  Variance & 3.1e-06 & 6.7e-06 & 1.2e-05 & 1.3e-05 & 2.7e-05 & 4.6e-05 & 2.8e-05 & 5.9e-05 & 0.0001 & 5e-05 & 0.0001 & 0.00018 & 8e-05 & 0.00016 & 0.00028 \\
  &  MSE & 1.1e-05 & 1.7e-05 & 2.5e-05 & 2.9e-05 & 5.7e-05 & 9.3e-05 & 5.9e-05 & 0.00012 & 0.0002 & 0.0001 & 0.00021 & 0.00036 & 0.00016 & 0.00032 & 0.00056 \\
\multirow[c]{-4}{*}{$p_{1}$} &  Coverage & 1 & 1 & 0.99 & 1 & 1 & 1 & 1 & 1 & 1 & 1 & 1 & 1 & 1 & 1 & 1 \\
\cmidrule(l){2-17}   &  Bias & 0.0016 & 0.0016 & 0.0012 & 0.0016 & 0.0014 & 0.00053 & 0.0013 & 0.00092 & -0.00023 & 0.00094 & 0.00054 & -0.00094 & 0.00055 & 6.5e-05 & -0.0024 \\
  &  Variance & 4.1e-06 & 9.3e-06 & 1.8e-05 & 1.6e-05 & 3.7e-05 & 7e-05 & 3.6e-05 & 8e-05 & 0.00015 & 6.5e-05 & 0.00014 & 0.00027 & 0.0001 & 0.00022 & 0.00043 \\
  &  MSE & 1.1e-05 & 2.1e-05 & 3.7e-05 & 3.5e-05 & 7.6e-05 & 0.00014 & 7.4e-05 & 0.00016 & 0.00031 & 0.00013 & 0.00028 & 0.00055 & 0.00021 & 0.00043 & 0.00086 \\
\multirow[c]{-4}{*}{$p_{2}$} &  Coverage & 1 & 1 & 1 & 1 & 1 & 1 & 1 & 1 & 1 & 1 & 1 & 1 & 1 & 1 & 1 \\
\bottomrule
 &  & \multicolumn{3}{c}{6$\%$} & \multicolumn{3}{c}{7$\%$} & \multicolumn{3}{c}{8$\%$} & \multicolumn{3}{c}{9$\%$} & \multicolumn{3}{c}{10$\%$} \\
\cmidrule(l){3-5} \cmidrule(l){6-8} \cmidrule(l){9-11} \cmidrule(l){12-14} \cmidrule(l){15-17}  &  & M=1024 & M=512 & M=256 & M=1024 & M=512 & M=256 & M=1024 & M=512 & M=256 & M=1024 & M=512 & M=256 & M=1024 & M=512 & M=256 \\
\toprule
  &  Bias & 0.00061 & -0.00012 & -0.0037 & 6.4e-05 & -0.00097 & -0.0052 & -0.00054 & -0.0018 & -0.0069 & -0.0011 & -0.0027 & -0.0087 & -0.0018 & -0.0037 & -0.011 \\
  &  Variance & 0.00012 & 0.00023 & 0.0004 & 0.00016 & 0.00031 & 0.00054 & 0.00021 & 0.00039 & 0.00071 & 0.00026 & 0.0005 & 0.00089 & 0.00033 & 0.00061 & 0.0011 \\
  &  MSE & 0.00023 & 0.00045 & 0.00081 & 0.00031 & 0.00062 & 0.0011 & 0.00041 & 0.00079 & 0.0015 & 0.00053 & 0.001 & 0.0019 & 0.00066 & 0.0012 & 0.0023 \\
\multirow[c]{-4}{*}{$p_{1}$} &  Coverage & 1 & 1 & 1 & 1 & 1 & 1 & 1 & 1 & 1 & 1 & 1 & 1 & 1 & 1 & 1 \\
\cmidrule(l){2-17}   &  Bias & 5.3e-05 & -0.00063 & -0.004 & -0.00052 & -0.0015 & -0.0055 & -0.0011 & -0.0023 & -0.0072 & -0.0018 & -0.0033 & -0.009 & -0.0025 & -0.0043 & -0.011 \\
  &  Variance & 0.00015 & 0.00031 & 0.00061 & 0.0002 & 0.00042 & 0.00083 & 0.00027 & 0.00054 & 0.0011 & 0.00034 & 0.00068 & 0.0014 & 0.00042 & 0.00084 & 0.0017 \\
  &  MSE & 0.0003 & 0.00062 & 0.0012 & 0.00041 & 0.00085 & 0.0017 & 0.00053 & 0.0011 & 0.0022 & 0.00068 & 0.0014 & 0.0028 & 0.00085 & 0.0017 & 0.0035 \\
\multirow[c]{-4}{*}{$p_{2}$} &  Coverage & 1 & 1 & 1 & 1 & 1 & 1 & 1 & 1 & 1 & 1 & 1 & 1 & 1 & 1 & 1 \\
\bottomrule
 &  & \multicolumn{3}{c}{11$\%$} & \multicolumn{3}{c}{12$\%$} & \multicolumn{3}{c}{13$\%$} & \multicolumn{3}{c}{14$\%$} & \multicolumn{3}{c}{15$\%$} \\
\cmidrule(l){3-5} \cmidrule(l){6-8} \cmidrule(l){9-11} \cmidrule(l){12-14} \cmidrule(l){15-17}  &  & M=1024 & M=512 & M=256 & M=1024 & M=512 & M=256 & M=1024 & M=512 & M=256 & M=1024 & M=512 & M=256 & M=1024 & M=512 & M=256 \\
\toprule
  &  Bias & -0.0026 & -0.0047 & -0.013 & -0.0033 & -0.0056 & -0.015 & -0.004 & -0.0061 & -0.017 & -0.0045 & -0.006 & -0.02 & -0.0052 & -0.0058 & -0.023 \\
  &  Variance & 0.0004 & 0.00073 & 0.0013 & 0.00048 & 0.00087 & 0.0016 & 0.00057 & 0.001 & 0.0018 & 0.00066 & 0.0012 & 0.0021 & 0.00074 & 0.0014 & 0.0024 \\
  &  MSE & 0.0008 & 0.0015 & 0.0028 & 0.00097 & 0.0018 & 0.0034 & 0.0012 & 0.0021 & 0.004 & 0.0013 & 0.0024 & 0.0046 & 0.0015 & 0.0028 & 0.0053 \\
\multirow[c]{-4}{*}{$p_{1}$} &  Coverage & 1 & 1 & 1 & 1 & 1 & 1 & 1 & 1 & 1 & 1 & 1 & 1 & 1 & 1 & 1 \\
\cmidrule(l){2-17}   &  Bias & -0.0032 & -0.0054 & -0.013 & -0.0041 & -0.0064 & -0.015 & -0.0047 & -0.007 & -0.018 & -0.0054 & -0.0073 & -0.02 & -0.0061 & -0.0073 & -0.023 \\
  &  Variance & 0.00051 & 0.001 & 0.002 & 0.00061 & 0.0012 & 0.0024 & 0.00072 & 0.0014 & 0.0028 & 0.00083 & 0.0016 & 0.0032 & 0.00094 & 0.0019 & 0.0037 \\
  &  MSE & 0.001 & 0.002 & 0.0042 & 0.0012 & 0.0024 & 0.005 & 0.0015 & 0.0029 & 0.0059 & 0.0017 & 0.0033 & 0.0069 & 0.0019 & 0.0038 & 0.0079 \\
\multirow[c]{-4}{*}{$p_{2}$} &  Coverage & 1 & 1 & 1 & 1 & 1 & 1 & 1 & 1 & 1 & 1 & 1 & 1 & 1 & 1 & 1 \\
\bottomrule
 &  & \multicolumn{3}{c}{16$\%$} & \multicolumn{3}{c}{17$\%$} & \multicolumn{3}{c}{18$\%$} & \multicolumn{3}{c}{19$\%$} & \multicolumn{3}{c}{20$\%$} \\
\cmidrule(l){3-5} \cmidrule(l){6-8} \cmidrule(l){9-11} \cmidrule(l){12-14} \cmidrule(l){15-17}  &  & M=1024 & M=512 & M=256 & M=1024 & M=512 & M=256 & M=1024 & M=512 & M=256 & M=1024 & M=512 & M=256 & M=1024 & M=512 & M=256 \\
\toprule
  &  Bias & -0.0055 & -0.0052 & -0.025 & -0.00079 & -0.0042 & -0.028 & 9.4e-06 & -0.0037 & -0.031 & -0.00041 & -0.0066 & -0.034 & -0.0034 & -0.01 & -0.037 \\
  &  Variance & 0.00088 & 0.0016 & 0.0027 & 0.001 & 0.0019 & 0.003 & 0.0013 & 0.002 & 0.0034 & 0.0017 & 0.0022 & 0.0037 & 0.0017 & 0.0023 & 0.0043 \\
  &  MSE & 0.0018 & 0.0032 & 0.0061 & 0.0021 & 0.0038 & 0.0069 & 0.0027 & 0.0039 & 0.0077 & 0.0033 & 0.0044 & 0.0086 & 0.0034 & 0.0048 & 0.0099 \\
\multirow[c]{-4}{*}{$p_{1}$} &  Coverage & 1 & 1 & 1 & 1 & 1 & 1 & 1 & 1 & 1 & 1 & 1 & 1 & 1 & 1 & 1 \\
\cmidrule(l){2-17}   &  Bias & -0.0066 & -0.0071 & -0.025 & -0.0028 & -0.0065 & -0.028 & -0.0023 & -0.0065 & -0.031 & -0.0029 & -0.0094 & -0.035 & -0.0057 & -0.012 & -0.038 \\
  &  Variance & 0.0011 & 0.0022 & 0.0042 & 0.0013 & 0.0025 & 0.0047 & 0.0015 & 0.0026 & 0.0052 & 0.0019 & 0.0029 & 0.0058 & 0.002 & 0.0031 & 0.0066 \\
  &  MSE & 0.0023 & 0.0044 & 0.009 & 0.0025 & 0.0051 & 0.01 & 0.0031 & 0.0053 & 0.011 & 0.0038 & 0.006 & 0.013 & 0.004 & 0.0064 & 0.015 \\
\multirow[c]{-4}{*}{$p_{2}$} &  Coverage & 1 & 1 & 1 & 1 & 1 & 1 & 1 & 1 & 1 & 1 & 1 & 1 & 1 & 1 & 1 \\
\bottomrule
 &  & \multicolumn{3}{c}{25$\%$} & \multicolumn{3}{c}{30$\%$} & \multicolumn{3}{c}{35$\%$} & \multicolumn{3}{c}{40$\%$} & \multicolumn{3}{c}{45$\%$} \\
\cmidrule(l){3-5} \cmidrule(l){6-8} \cmidrule(l){9-11} \cmidrule(l){12-14} \cmidrule(l){15-17}  &  & M=1024 & M=512 & M=256 & M=1024 & M=512 & M=256 & M=1024 & M=512 & M=256 & M=1024 & M=512 & M=256 & M=1024 & M=512 & M=256 \\
\toprule
  &  Bias & -0.011 & -0.036 & -0.057 & -0.024 & -0.045 & -0.079 & -0.033 & -0.07 & -0.1 & -0.044 & -0.1 & -0.13 & -0.068 & -0.15 & -0.17 \\
  &  Variance & 0.0026 & 0.0035 & 0.0061 & 0.0041 & 0.0054 & 0.0081 & 0.0068 & 0.0091 & 0.01 & 0.011 & 0.014 & 0.012 & 0.016 & 0.018 & 0.014 \\
  &  MSE & 0.0053 & 0.0083 & 0.015 & 0.0088 & 0.013 & 0.022 & 0.015 & 0.023 & 0.031 & 0.024 & 0.039 & 0.041 & 0.037 & 0.059 & 0.055 \\
\multirow[c]{-4}{*}{$p_{1}$} &  Coverage & 1 & 1 & 1 & 1 & 1 & 1 & 1 & 1 & 1 & 1 & 1 & 1 & 1 & 1 & 1 \\
\cmidrule(l){2-17}   &  Bias & -0.014 & -0.036 & -0.057 & -0.026 & -0.047 & -0.08 & -0.035 & -0.07 & -0.11 & -0.047 & -0.1 & -0.14 & -0.071 & -0.15 & -0.17 \\
  &  Variance & 0.0029 & 0.0048 & 0.0095 & 0.0046 & 0.0071 & 0.013 & 0.0072 & 0.011 & 0.016 & 0.011 & 0.016 & 0.019 & 0.015 & 0.02 & 0.022 \\
  &  MSE & 0.006 & 0.011 & 0.022 & 0.0098 & 0.016 & 0.032 & 0.016 & 0.027 & 0.043 & 0.024 & 0.042 & 0.056 & 0.035 & 0.061 & 0.072 \\
\multirow[c]{-4}{*}{$p_{2}$} &  Coverage & 1 & 1 & 1 & 1 & 1 & 1 & 1 & 1 & 1 & 1 & 1 & 1 & 1 & 1 & 1 \\
\bottomrule
\end{tabular}

	}
\end{table}
\begin{table}[H]
	\caption{Logistic Growth (continued)} \label{tab:logUQ2}	
	\resizebox{\textwidth}{!}{
	\begin{tabular}{lccccccccccccccccccccccccccccccccccccccccccccccccccccccccccccccccccccccccccccccc}
 &  & \multicolumn{3}{c}{50$\%$} \\
\cmidrule(l){3-5}  &  & M=1024 & M=512 & M=256 \\
\toprule
  &  Bias & -0.092 & -0.19 & -0.2 & \hspace{20pt} & \hspace{20pt} & \hspace{20pt} & \hspace{20pt} & \hspace{20pt} & \hspace{20pt} & \hspace{20pt} & \hspace{20pt} & \hspace{20pt} & \hspace{20pt} & \hspace{20pt} & \hspace{20pt} \\
  &  Variance & 0.024 & 0.022 & 0.016 & \hspace{20pt} & \hspace{20pt} & \hspace{20pt} & \hspace{20pt} & \hspace{20pt} & \hspace{20pt} & \hspace{20pt} & \hspace{20pt} & \hspace{20pt} & \hspace{20pt} & \hspace{20pt} & \hspace{20pt} \\
  &  MSE & 0.057 & 0.081 & 0.074 & \hspace{20pt} & \hspace{20pt} & \hspace{20pt} & \hspace{20pt} & \hspace{20pt} & \hspace{20pt} & \hspace{20pt} & \hspace{20pt} & \hspace{20pt} & \hspace{20pt} & \hspace{20pt} & \hspace{20pt} \\
\multirow[c]{-4}{*}{$p_{1}$} &  Coverage & 1 & 1 & 1 & \hspace{20pt} & \hspace{20pt} & \hspace{20pt} & \hspace{20pt} & \hspace{20pt} & \hspace{20pt} & \hspace{20pt} & \hspace{20pt} & \hspace{20pt} & \hspace{20pt} & \hspace{20pt} & \hspace{20pt} \\
\cmidrule(l){2-17}   &  Bias & -0.095 & -0.19 & -0.21 & \hspace{20pt} & \hspace{20pt} & \hspace{20pt} & \hspace{20pt} & \hspace{20pt} & \hspace{20pt} & \hspace{20pt} & \hspace{20pt} & \hspace{20pt} & \hspace{20pt} & \hspace{20pt} & \hspace{20pt} \\
  &  Variance & 0.022 & 0.023 & 0.025 & \hspace{20pt} & \hspace{20pt} & \hspace{20pt} & \hspace{20pt} & \hspace{20pt} & \hspace{20pt} & \hspace{20pt} & \hspace{20pt} & \hspace{20pt} & \hspace{20pt} & \hspace{20pt} & \hspace{20pt} \\
  &  MSE & 0.052 & 0.081 & 0.092 & \hspace{20pt} & \hspace{20pt} & \hspace{20pt} & \hspace{20pt} & \hspace{20pt} & \hspace{20pt} & \hspace{20pt} & \hspace{20pt} & \hspace{20pt} & \hspace{20pt} & \hspace{20pt} & \hspace{20pt} \\
\multirow[c]{-4}{*}{$p_{2}$} &  Coverage & 1 & 1 & 1 & \hspace{20pt} & \hspace{20pt} & \hspace{20pt} & \hspace{20pt} & \hspace{20pt} & \hspace{20pt} & \hspace{20pt} & \hspace{20pt} & \hspace{20pt} & \hspace{20pt} & \hspace{20pt} & \hspace{20pt} \\
\bottomrule
\end{tabular}

	}
\end{table}

\begin{table}[H]
	\caption{Hindmarsh-Rose} \label{tab:hindUQ1}	
	\resizebox{\textwidth}{!}{
	\begin{tabular}{lccccccccccccccccccccccccccccccccccccccccccccccccccccccccccccc}
 &  & \multicolumn{3}{c}{1$\%$} & \multicolumn{3}{c}{2$\%$} & \multicolumn{3}{c}{3$\%$} & \multicolumn{3}{c}{4$\%$} & \multicolumn{3}{c}{5$\%$} \\
\cmidrule(l){3-5} \cmidrule(l){6-8} \cmidrule(l){9-11} \cmidrule(l){12-14} \cmidrule(l){15-17}  &  & M=1024 & M=512 & M=256 & M=1024 & M=512 & M=256 & M=1024 & M=512 & M=256 & M=1024 & M=512 & M=256 & M=1024 & M=512 & M=256 \\
\toprule
  &  Bias & -0.29 & -0.043 & -0.4 & -0.27 & -0.12 & -0.67 & -0.11 & -0.23 & -1.2 & -0.18 & -0.41 & -1.6 & -0.083 & -0.64 & -2 \\
  &  Variance & 2.3 & 0.0029 & 0.18 & 2.1 & 0.011 & 0.18 & 0.51 & 0.024 & 0.48 & 0.83 & 0.043 & 0.58 & 0.012 & 0.066 & 0.43 \\
  &  MSE & 4.8 & 0.0075 & 0.52 & 4.2 & 0.036 & 0.8 & 1 & 0.1 & 2.4 & 1.7 & 0.26 & 3.8 & 0.031 & 0.54 & 4.9 \\
\multirow[c]{-4}{*}{$p_{1}$} &  Coverage & 0.97 & 1 & 0.99 & 0.97 & 1 & 1 & 0.99 & 1 & 1 & 0.98 & 1 & 0.94 & 1 & 1 & 0.79 \\
\cmidrule(l){2-17}   &  Bias & -0.35 & -0.067 & -0.22 & -0.39 & -0.16 & -0.42 & -0.27 & -0.32 & -0.74 & -0.35 & -0.55 & -1.1 & -0.37 & -0.84 & -1.5 \\
  &  Variance & 2.9 & 0.0041 & 0.28 & 2.9 & 0.017 & 0.069 & 0.96 & 0.035 & 0.18 & 0.95 & 0.061 & 0.45 & 0.017 & 0.091 & 0.62 \\
  &  MSE & 5.9 & 0.013 & 0.6 & 5.9 & 0.059 & 0.32 & 2 & 0.17 & 0.9 & 2 & 0.42 & 2.1 & 0.17 & 0.88 & 3.5 \\
\multirow[c]{-4}{*}{$p_{2}$} &  Coverage & 0.97 & 1 & 0.99 & 0.97 & 1 & 1 & 0.99 & 1 & 1 & 0.99 & 0.99 & 0.98 & 1 & 0.97 & 0.9 \\
\cmidrule(l){2-17}   &  Bias & 0.81 & -0.19 & -1.3 & 0.77 & -0.56 & -2.6 & -0.088 & -1.1 & -5 & -0.32 & -1.9 & -6.8 & -0.64 & -2.9 & -8.3 \\
  &  Variance & 26 & 0.05 & 0.85 & 26 & 0.22 & 3.5 & 4 & 0.44 & 9.7 & 3 & 0.77 & 12 & 0.15 & 1.1 & 7.5 \\
  &  MSE & 53 & 0.14 & 3.4 & 53 & 0.76 & 14 & 8 & 2.1 & 44 & 6.1 & 5.2 & 70 & 0.71 & 11 & 84 \\
\multirow[c]{-4}{*}{$p_{3}$} &  Coverage & 0.97 & 1 & 1 & 0.97 & 1 & 1 & 0.99 & 1 & 0.97 & 0.98 & 0.99 & 0.8 & 1 & 0.98 & 0.61 \\
\cmidrule(l){2-17}   &  Bias & 0.27 & 0.059 & 0.026 & 0.21 & 0.31 & 0.72 & 0.0065 & 0.68 & 2.4 & -0.1 & 1.3 & 3.9 & -0.2 & 2 & 5.5 \\
  &  Variance & 2.9 & 0.11 & 0.9 & 3 & 0.52 & 3.2 & 1.3 & 1.1 & 10 & 1.4 & 2 & 12 & 0.38 & 3.1 & 10 \\
  &  MSE & 6 & 0.23 & 1.8 & 6.1 & 1.1 & 7 & 2.5 & 2.6 & 26 & 2.7 & 5.6 & 39 & 0.81 & 10 & 52 \\
\multirow[c]{-4}{*}{$p_{4}$} &  Coverage & 0.97 & 1 & 1 & 0.97 & 1 & 0.99 & 0.98 & 1 & 0.95 & 0.99 & 1 & 0.9 & 1 & 1 & 0.91 \\
\cmidrule(l){2-17}   &  Bias & -0.29 & -0.021 & -0.086 & -0.27 & -0.12 & -0.39 & -0.1 & -0.27 & -0.98 & -0.035 & -0.5 & -1.5 & -0.0022 & -0.78 & -1.9 \\
  &  Variance & 2.8 & 0.011 & 0.12 & 2.7 & 0.045 & 0.23 & 1 & 0.097 & 0.74 & 0.58 & 0.17 & 0.92 & 0.041 & 0.24 & 2.3 \\
  &  MSE & 5.7 & 0.022 & 0.26 & 5.4 & 0.11 & 0.62 & 2 & 0.27 & 2.4 & 1.2 & 0.58 & 4.1 & 0.082 & 1.1 & 8.2 \\
\multirow[c]{-4}{*}{$p_{5}$} &  Coverage & 0.97 & 1 & 1 & 0.97 & 1 & 0.99 & 0.98 & 1 & 0.91 & 0.99 & 0.98 & 0.79 & 1 & 0.97 & 0.66 \\
\cmidrule(l){2-17}   &  Bias & 0.53 & -0.18 & -1.1 & 1.2 & -0.83 & -3.4 & -0.48 & -1.8 & -6.5 & -0.32 & -3.4 & -8.7 & -1.4 & -5.3 & -11 \\
  &  Variance & 9.2 & 0.23 & 26 & 57 & 0.93 & 6 & 1 & 1.9 & 15 & 13 & 3.1 & 37 & 1.6 & 4.3 & 68 \\
  &  MSE & 19 & 0.49 & 53 & 1.2e+02 & 2.5 & 24 & 2.3 & 7.2 & 72 & 26 & 18 & 1.5e+02 & 5.2 & 36 & 2.5e+02 \\
\multirow[c]{-4}{*}{$p_{6}$} &  Coverage & 0.97 & 1 & 0.99 & 0.97 & 0.99 & 0.91 & 0.98 & 0.89 & 0.61 & 0.95 & 0.78 & 0.44 & 0.96 & 0.57 & 0.24 \\
\cmidrule(l){2-17}   &  Bias & -0.29 & -0.022 & -0.33 & -0.28 & -0.13 & -0.71 & -0.14 & -0.31 & -1.3 & -0.19 & -0.59 & -1.7 & -0.18 & -0.93 & -2.1 \\
  &  Variance & 2.9 & 0.01 & 0.13 & 2.5 & 0.041 & 0.28 & 0.65 & 0.084 & 0.64 & 0.8 & 0.14 & 1.7 & 0.064 & 0.2 & 2 \\
  &  MSE & 5.9 & 0.021 & 0.36 & 5.1 & 0.099 & 1.1 & 1.3 & 0.26 & 2.9 & 1.6 & 0.62 & 6.2 & 0.16 & 1.3 & 8.4 \\
\multirow[c]{-4}{*}{$p_{7}$} &  Coverage & 0.97 & 1 & 0.99 & 0.97 & 0.99 & 0.89 & 0.98 & 0.95 & 0.65 & 0.96 & 0.86 & 0.49 & 1 & 0.76 & 0.33 \\
\cmidrule(l){2-17}   &  Bias & 0.015 & -0.00073 & -0.00099 & 0.015 & -0.002 & -0.0033 & 0.0032 & -0.0039 & -0.0079 & 0.0045 & -0.0065 & -0.012 & -0.002 & -0.0095 & -0.018 \\
  &  Variance & 0.0075 & 2.7e-06 & 2.6e-05 & 0.0072 & 1.1e-05 & 4.4e-05 & 0.0019 & 2.2e-05 & 0.0001 & 0.0035 & 3.7e-05 & 0.00012 & 1.4e-05 & 5.2e-05 & 0.00012 \\
  &  MSE & 0.015 & 5.9e-06 & 5.3e-05 & 0.015 & 2.6e-05 & 9.9e-05 & 0.0037 & 6e-05 & 0.00027 & 0.0071 & 0.00012 & 0.0004 & 3.2e-05 & 0.00019 & 0.00055 \\
\multirow[c]{-4}{*}{$p_{8}$} &  Coverage & 0.97 & 1 & 0.99 & 0.97 & 1 & 1 & 0.99 & 1 & 1 & 0.99 & 1 & 1 & 1 & 1 & 1 \\
\cmidrule(l){2-17}   &  Bias & -0.001 & -0.0012 & -0.0035 & -0.0019 & -0.0041 & -0.0055 & -0.0023 & -0.0071 & -0.01 & -0.0031 & -0.01 & -0.012 & -0.0047 & -0.013 & -0.015 \\
  &  Variance & 3.6e-05 & 1.7e-05 & 8.1e-05 & 4.5e-05 & 5.5e-05 & 0.00012 & 4.4e-05 & 0.0001 & 0.00015 & 8.3e-05 & 0.00014 & 0.00014 & 9.1e-05 & 0.00015 & 0.00011 \\
  &  MSE & 7.2e-05 & 3.6e-05 & 0.00018 & 9.3e-05 & 0.00013 & 0.00027 & 9.3e-05 & 0.00026 & 0.00041 & 0.00018 & 0.00039 & 0.00041 & 0.0002 & 0.00049 & 0.00044 \\
\multirow[c]{-4}{*}{$p_{9}$} &  Coverage & 1 & 1 & 1 & 1 & 1 & 1 & 1 & 1 & 1 & 1 & 1 & 1 & 1 & 1 & 1 \\
\cmidrule(l){2-17}   &  Bias & 0.03 & 0.0033 & 0.012 & 0.034 & 0.011 & 0.013 & 0.017 & 0.019 & 0.017 & 0.025 & 0.026 & 0.0091 & 0.019 & 0.031 & 0.011 \\
  &  Variance & 0.029 & 0.0003 & 0.0014 & 0.029 & 0.00094 & 0.0019 & 0.01 & 0.0017 & 0.0022 & 0.011 & 0.0023 & 0.0015 & 0.0016 & 0.0024 & 0.0014 \\
  &  MSE & 0.058 & 0.00061 & 0.0029 & 0.058 & 0.002 & 0.0039 & 0.021 & 0.0038 & 0.0048 & 0.022 & 0.0052 & 0.003 & 0.0035 & 0.0057 & 0.0029 \\
\multirow[c]{-4}{*}{$p_{10}$} &  Coverage & 1 & 1 & 1 & 1 & 1 & 1 & 1 & 1 & 1 & 1 & 1 & 1 & 1 & 1 & 1 \\
\bottomrule
\end{tabular}

	}
\end{table}

\begin{table}[H]
	\caption{Hindmarsh-Rose (continued)} \label{tab:hindUQ2}	
	\resizebox{\textwidth}{!}{
	\begin{tabular}{lccccccccccccccccccccccccccccccccccccccccccccccccccccccccccccc}
 &  & \multicolumn{3}{c}{6$\%$} & \multicolumn{3}{c}{7$\%$} & \multicolumn{3}{c}{8$\%$} & \multicolumn{3}{c}{9$\%$} & \multicolumn{3}{c}{10$\%$} \\
\cmidrule(l){3-5} \cmidrule(l){6-8} \cmidrule(l){9-11} \cmidrule(l){12-14} \cmidrule(l){15-17}  &  & M=1024 & M=512 & M=256 & M=1024 & M=512 & M=256 & M=1024 & M=512 & M=256 & M=1024 & M=512 & M=256 & M=1024 & M=512 & M=256 \\
\toprule
  &  Bias & -0.13 & -0.9 & -2.5 & -0.26 & -1.2 & -3 & -0.34 & -1.6 & -3.4 & -0.37 & -1.7 & -3.8 & -0.48 & -2 & -4.2 \\
  &  Variance & 0.018 & 0.088 & 0.41 & 0.62 & 0.11 & 0.41 & 0.26 & 0.72 & 0.4 & 0.043 & 0.14 & 0.4 & 0.054 & 0.15 & 0.41 \\
  &  MSE & 0.053 & 0.98 & 7.1 & 1.3 & 1.6 & 9.7 & 0.63 & 3.9 & 12 & 0.22 & 3.3 & 15 & 0.34 & 4.3 & 18 \\
\multirow[c]{-4}{*}{$p_{1}$} &  Coverage & 1 & 0.99 & 0.52 & 0.99 & 0.92 & 0.26 & 0.98 & 0.77 & 0.11 & 1 & 0.57 & 0.04 & 1 & 0.37 & 0.03 \\
\cmidrule(l){2-17}   &  Bias & -0.53 & -1.2 & -2.1 & -0.8 & -1.5 & -2.5 & -0.9 & -1.9 & -3 & -1.2 & -2.3 & -3.4 & -1.5 & -2.7 & -3.8 \\
  &  Variance & 0.024 & 0.12 & 0.43 & 0.89 & 0.15 & 0.49 & 0.16 & 0.18 & 0.53 & 0.049 & 0.2 & 0.57 & 0.059 & 0.22 & 0.61 \\
  &  MSE & 0.33 & 1.6 & 5.1 & 2.4 & 2.7 & 7.4 & 1.1 & 4 & 10 & 1.5 & 5.7 & 13 & 2.2 & 7.6 & 16 \\
\multirow[c]{-4}{*}{$p_{2}$} &  Coverage & 1 & 0.93 & 0.75 & 0.99 & 0.72 & 0.47 & 0.93 & 0.4 & 0.31 & 0.76 & 0.19 & 0.19 & 0.44 & 0.09 & 0.08 \\
\cmidrule(l){2-17}   &  Bias & -0.99 & -4.1 & -10 & -1.5 & -5.3 & -12 & -2.1 & -6.8 & -14 & -2.6 & -7.8 & -15 & -3.3 & -8.9 & -17 \\
  &  Variance & 0.22 & 1.5 & 6.4 & 3.3 & 1.7 & 5.7 & 1 & 6.6 & 4.8 & 0.48 & 2 & 4.4 & 0.57 & 2.1 & 4.2 \\
  &  MSE & 1.4 & 20 & 1.2e+02 & 8.9 & 32 & 1.6e+02 & 6.4 & 60 & 2e+02 & 7.7 & 64 & 2.4e+02 & 12 & 83 & 2.8e+02 \\
\multirow[c]{-4}{*}{$p_{3}$} &  Coverage & 1 & 0.9 & 0.29 & 0.98 & 0.67 & 0.13 & 0.98 & 0.39 & 0.03 & 0.97 & 0.16 & 0.02 & 0.86 & 0.08 & 0.01 \\
\cmidrule(l){2-17}   &  Bias & -0.079 & 2.9 & 7 & 0.23 & 3.9 & 8 & 0.7 & 5 & 8.7 & 1.2 & 6 & 9.1 & 1.9 & 6.9 & 9.3 \\
  &  Variance & 0.6 & 4.4 & 8.4 & 2.1 & 5.7 & 6.9 & 2.3 & 6.7 & 4 & 1.8 & 7.5 & 2.8 & 2.4 & 7.8 & 2.3 \\
  &  MSE & 1.2 & 17 & 66 & 4.3 & 27 & 78 & 5 & 38 & 83 & 4.9 & 51 & 88 & 8.4 & 64 & 91 \\
\multirow[c]{-4}{*}{$p_{4}$} &  Coverage & 1 & 0.99 & 0.92 & 0.99 & 0.97 & 0.95 & 0.99 & 0.94 & 0.95 & 1 & 0.95 & 0.93 & 1 & 0.94 & 0.93 \\
\cmidrule(l){2-17}   &  Bias & -0.06 & -1.1 & -2.6 & -0.1 & -1.4 & -3 & -0.19 & -1.7 & -3.3 & -0.42 & -2 & -3.6 & -0.6 & -2.2 & -3.8 \\
  &  Variance & 0.061 & 0.31 & 0.71 & 0.19 & 0.37 & 0.61 & 0.5 & 0.56 & 0.5 & 0.14 & 0.43 & 0.5 & 0.17 & 0.44 & 0.55 \\
  &  MSE & 0.13 & 1.8 & 8.4 & 0.4 & 2.7 & 10 & 1 & 3.9 & 12 & 0.46 & 4.8 & 14 & 0.7 & 5.8 & 15 \\
\multirow[c]{-4}{*}{$p_{5}$} &  Coverage & 1 & 0.94 & 0.55 & 1 & 0.91 & 0.52 & 0.99 & 0.92 & 0.5 & 1 & 0.91 & 0.52 & 1 & 0.88 & 0.51 \\
\cmidrule(l){2-17}   &  Bias & -2.2 & -7.4 & -15 & -2.5 & -9.6 & -17 & -3.8 & -11 & -20 & -5.5 & -14 & -21 & -6.8 & -16 & -23 \\
  &  Variance & 2.3 & 5.4 & 13 & 28 & 6.1 & 13 & 22 & 18 & 12 & 4.3 & 6.8 & 12 & 4.9 & 6.8 & 12 \\
  &  MSE & 9.5 & 65 & 2.4e+02 & 62 & 1e+02 & 3.3e+02 & 57 & 1.6e+02 & 4.1e+02 & 39 & 2e+02 & 4.8e+02 & 56 & 2.6e+02 & 5.6e+02 \\
\multirow[c]{-4}{*}{$p_{6}$} &  Coverage & 0.9 & 0.31 & 0.12 & 0.79 & 0.13 & 0.09 & 0.62 & 0.08 & 0.05 & 0.44 & 0.05 & 0.03 & 0.31 & 0.01 & 0 \\
\cmidrule(l){2-17}   &  Bias & -0.3 & -1.3 & -2.8 & -0.46 & -1.7 & -3.3 & -0.6 & -2.1 & -3.8 & -0.79 & -2.5 & -4.2 & -0.98 & -2.8 & -4.5 \\
  &  Variance & 0.092 & 0.25 & 0.62 & 0.91 & 0.29 & 0.65 & 0.18 & 0.36 & 0.67 & 0.18 & 0.35 & 0.69 & 0.21 & 0.36 & 0.71 \\
  &  MSE & 0.27 & 2.2 & 9.3 & 2 & 3.5 & 12 & 0.71 & 5.2 & 16 & 0.98 & 6.9 & 19 & 1.4 & 8.8 & 22 \\
\multirow[c]{-4}{*}{$p_{7}$} &  Coverage & 0.99 & 0.49 & 0.15 & 0.92 & 0.27 & 0.1 & 0.91 & 0.19 & 0.06 & 0.83 & 0.09 & 0.05 & 0.72 & 0.06 & 0.04 \\
\cmidrule(l){2-17}   &  Bias & -0.0032 & -0.013 & -0.022 & 0.0022 & -0.016 & -0.024 & -0.0073 & -0.02 & -0.027 & -0.0093 & -0.023 & -0.028 & -0.012 & -0.025 & -0.029 \\
  &  Variance & 2.2e-05 & 6.9e-05 & 0.00011 & 0.0044 & 8.7e-05 & 0.00013 & 6.7e-05 & 0.00011 & 0.00013 & 6e-05 & 0.00013 & 0.00014 & 7.4e-05 & 0.00014 & 0.00015 \\
  &  MSE & 5.5e-05 & 0.0003 & 0.00069 & 0.0089 & 0.00043 & 0.00086 & 0.00019 & 0.00062 & 0.00097 & 0.00021 & 0.00076 & 0.0011 & 0.0003 & 0.00092 & 0.0012 \\
\multirow[c]{-4}{*}{$p_{8}$} &  Coverage & 1 & 1 & 1 & 0.99 & 1 & 1 & 1 & 1 & 1 & 1 & 1 & 1 & 1 & 1 & 1 \\
\cmidrule(l){2-17}   &  Bias & -0.0065 & -0.016 & -0.018 & -0.0083 & -0.018 & -0.02 & -0.011 & -0.021 & -0.022 & -0.013 & -0.022 & -0.023 & -0.016 & -0.024 & -0.024 \\
  &  Variance & 0.00012 & 0.00015 & 9.6e-05 & 0.00015 & 0.00015 & 9.2e-05 & 0.00017 & 0.00013 & 8.5e-05 & 0.00017 & 0.00011 & 8.5e-05 & 0.00016 & 0.0001 & 8.7e-05 \\
  &  MSE & 0.00028 & 0.00057 & 0.00053 & 0.00037 & 0.00064 & 0.0006 & 0.00046 & 0.00068 & 0.00066 & 0.00051 & 0.00073 & 0.0007 & 0.00058 & 0.00078 & 0.00074 \\
\multirow[c]{-4}{*}{$p_{9}$} &  Coverage & 1 & 1 & 1 & 1 & 1 & 1 & 1 & 1 & 1 & 1 & 1 & 1 & 1 & 1 & 1 \\
\cmidrule(l){2-17}   &  Bias & 0.024 & 0.034 & 0.012 & 0.04 & 0.035 & 0.014 & 0.037 & 0.036 & 0.017 & 0.039 & 0.036 & 0.02 & 0.043 & 0.038 & 0.022 \\
  &  Variance & 0.0021 & 0.0023 & 0.0013 & 0.012 & 0.0022 & 0.0012 & 0.0029 & 0.0019 & 0.0012 & 0.0028 & 0.0018 & 0.0012 & 0.0028 & 0.0017 & 0.0012 \\
  &  MSE & 0.0047 & 0.0057 & 0.0027 & 0.025 & 0.0056 & 0.0026 & 0.0072 & 0.0052 & 0.0028 & 0.0072 & 0.0049 & 0.0028 & 0.0074 & 0.0047 & 0.003 \\
\multirow[c]{-4}{*}{$p_{10}$} &  Coverage & 1 & 1 & 1 & 1 & 1 & 1 & 1 & 1 & 1 & 1 & 1 & 1 & 1 & 1 & 1 \\
\bottomrule
\end{tabular}

	}
\end{table}
\begin{table}[H]
	\caption{Hindmarsh-Rose (continued)} \label{tab:hindUQ3}	
	\resizebox{\textwidth}{!}{
	\begin{tabular}{lccccccccccccccccccccccccccccccccccccccccccccccccccccccccccccc}
 &  & \multicolumn{3}{c}{11$\%$} & \multicolumn{3}{c}{12$\%$} & \multicolumn{3}{c}{13$\%$} & \multicolumn{3}{c}{14$\%$} & \multicolumn{3}{c}{15$\%$} \\
\cmidrule(l){3-5} \cmidrule(l){6-8} \cmidrule(l){9-11} \cmidrule(l){12-14} \cmidrule(l){15-17}  &  & M=1024 & M=512 & M=256 & M=1024 & M=512 & M=256 & M=1024 & M=512 & M=256 & M=1024 & M=512 & M=256 & M=1024 & M=512 & M=256 \\
\toprule
  &  Bias & -0.61 & -2.3 & -4.6 & -0.75 & -2.5 & -4.9 & -0.9 & -2.8 & -5.3 & -1.1 & -2.9 & -5.5 & -1.2 & -3.2 & -5.8 \\
  &  Variance & 0.067 & 0.15 & 0.66 & 0.08 & 0.16 & 0.65 & 0.1 & 0.27 & 0.63 & 0.11 & 0.2 & 0.46 & 0.13 & 0.21 & 0.49 \\
  &  MSE & 0.5 & 5.4 & 23 & 0.73 & 6.6 & 26 & 1 & 8.1 & 29 & 1.4 & 9 & 31 & 1.8 & 10 & 35 \\
\multirow[c]{-4}{*}{$p_{1}$} &  Coverage & 1 & 0.24 & 0.01 & 0.98 & 0.1 & 0.01 & 0.91 & 0.08 & 0.01 & 0.84 & 0.03 & 0 & 0.69 & 0.01 & 0 \\
\cmidrule(l){2-17}   &  Bias & -1.8 & -3 & -4.3 & -2.1 & -3.4 & -4.7 & -2.4 & -3.7 & -5.1 & -2.7 & -4 & -5.4 & -3.1 & -4.4 & -5.7 \\
  &  Variance & 0.07 & 0.22 & 0.96 & 0.081 & 0.23 & 0.92 & 0.092 & 0.28 & 0.88 & 0.1 & 0.27 & 0.62 & 0.11 & 0.27 & 0.59 \\
  &  MSE & 3.2 & 9.7 & 20 & 4.5 & 12 & 24 & 6 & 14 & 27 & 7.8 & 17 & 30 & 9.8 & 20 & 34 \\
\multirow[c]{-4}{*}{$p_{2}$} &  Coverage & 0.13 & 0.05 & 0.05 & 0.04 & 0.03 & 0.04 & 0.01 & 0.01 & 0.04 & 0.01 & 0 & 0.04 & 0 & 0 & 0.02 \\
\cmidrule(l){2-17}   &  Bias & -4.1 & -10 & -18 & -4.9 & -11 & -19 & -5.8 & -12 & -20 & -6.7 & -13 & -21 & -7.6 & -13 & -22 \\
  &  Variance & 0.65 & 1.9 & 5.6 & 0.73 & 1.9 & 5.8 & 0.89 & 2.7 & 5.6 & 0.82 & 2.2 & 4.8 & 0.84 & 2.2 & 5.4 \\
  &  MSE & 18 & 1e+02 & 3.3e+02 & 26 & 1.2e+02 & 3.7e+02 & 35 & 1.5e+02 & 4.2e+02 & 47 & 1.6e+02 & 4.5e+02 & 60 & 1.9e+02 & 5e+02 \\
\multirow[c]{-4}{*}{$p_{3}$} &  Coverage & 0.55 & 0.02 & 0.01 & 0.3 & 0 & 0 & 0.09 & 0 & 0 & 0.02 & 0 & 0 & 0 & 0 & 0 \\
\cmidrule(l){2-17}   &  Bias & 2.8 & 7.7 & 9.3 & 3.9 & 8.3 & 9.2 & 5 & 8.6 & 9.3 & 6.3 & 8.7 & 9.4 & 7.5 & 8.8 & 9.4 \\
  &  Variance & 3.2 & 6.9 & 3.2 & 4 & 6.8 & 5.3 & 5.2 & 6.8 & 5.4 & 5 & 9.5 & 4.6 & 4.6 & 11 & 4.8 \\
  &  MSE & 14 & 73 & 94 & 23 & 82 & 96 & 35 & 87 & 97 & 50 & 95 & 97 & 65 & 99 & 98 \\
\multirow[c]{-4}{*}{$p_{4}$} &  Coverage & 1 & 0.93 & 0.91 & 1 & 0.93 & 0.88 & 0.98 & 0.95 & 0.85 & 0.95 & 0.96 & 0.82 & 0.88 & 0.96 & 0.78 \\
\cmidrule(l){2-17}   &  Bias & -0.8 & -2.4 & -3.9 & -1 & -2.5 & -4.1 & -1.2 & -2.5 & -4.2 & -1.5 & -2.6 & -4.4 & -1.6 & -2.6 & -4.6 \\
  &  Variance & 0.21 & 0.39 & 0.63 & 0.24 & 0.4 & 0.88 & 0.32 & 0.7 & 1.1 & 0.28 & 0.56 & 1.3 & 0.28 & 0.64 & 1.8 \\
  &  MSE & 1.1 & 6.6 & 17 & 1.5 & 7.2 & 18 & 2.1 & 7.9 & 20 & 2.7 & 8.1 & 22 & 3.2 & 8.3 & 25 \\
\multirow[c]{-4}{*}{$p_{5}$} &  Coverage & 1 & 0.89 & 0.51 & 1 & 0.91 & 0.55 & 1 & 0.93 & 0.55 & 1 & 0.93 & 0.55 & 1 & 0.94 & 0.56 \\
\cmidrule(l){2-17}   &  Bias & -8.2 & -17 & -25 & -9.6 & -19 & -26 & -11 & -20 & -27 & -12 & -21 & -28 & -14 & -23 & -30 \\
  &  Variance & 5.4 & 6.6 & 15 & 5.8 & 6.5 & 14 & 7 & 22 & 13 & 6.1 & 6.8 & 12 & 6.2 & 6.9 & 12 \\
  &  MSE & 78 & 3.1e+02 & 6.3e+02 & 1e+02 & 3.7e+02 & 7e+02 & 1.3e+02 & 4.4e+02 & 7.7e+02 & 1.7e+02 & 4.8e+02 & 8.3e+02 & 2e+02 & 5.3e+02 & 8.9e+02 \\
\multirow[c]{-4}{*}{$p_{6}$} &  Coverage & 0.15 & 0 & 0.01 & 0.08 & 0 & 0.01 & 0.04 & 0 & 0 & 0.02 & 0 & 0 & 0 & 0 & 0 \\
\cmidrule(l){2-17}   &  Bias & -1.2 & -3.2 & -4.8 & -1.4 & -3.5 & -5.1 & -1.6 & -3.7 & -5.4 & -1.8 & -4 & -5.6 & -2.1 & -4.2 & -5.8 \\
  &  Variance & 0.23 & 0.37 & 0.72 & 0.26 & 0.38 & 0.73 & 0.34 & 0.65 & 0.74 & 0.3 & 0.4 & 0.76 & 0.32 & 0.42 & 0.78 \\
  &  MSE & 1.9 & 11 & 25 & 2.5 & 13 & 28 & 3.2 & 15 & 30 & 4 & 17 & 33 & 4.9 & 19 & 35 \\
\multirow[c]{-4}{*}{$p_{7}$} &  Coverage & 0.57 & 0.04 & 0.01 & 0.44 & 0.03 & 0 & 0.34 & 0.03 & 0 & 0.23 & 0.01 & 0 & 0.12 & 0 & 0 \\
\cmidrule(l){2-17}   &  Bias & -0.015 & -0.028 & -0.025 & -0.019 & -0.029 & -0.025 & -0.022 & -0.03 & -0.025 & -0.026 & -0.03 & -0.03 & -0.029 & -0.03 & -0.03 \\
  &  Variance & 8.6e-05 & 0.00014 & 0.0024 & 9.7e-05 & 0.00015 & 0.0023 & 0.0001 & 0.00016 & 0.0022 & 0.0001 & 0.00021 & 0.00033 & 9.7e-05 & 0.00023 & 0.00036 \\
  &  MSE & 0.00041 & 0.001 & 0.0054 & 0.00055 & 0.0011 & 0.0052 & 0.00071 & 0.0012 & 0.005 & 0.00088 & 0.0013 & 0.0015 & 0.001 & 0.0014 & 0.0016 \\
\multirow[c]{-4}{*}{$p_{8}$} &  Coverage & 1 & 1 & 0.99 & 1 & 1 & 1 & 1 & 1 & 1 & 1 & 1 & 1 & 1 & 1 & 1 \\
\cmidrule(l){2-17}   &  Bias & -0.019 & -0.026 & -0.022 & -0.021 & -0.026 & -0.022 & -0.023 & -0.027 & -0.023 & -0.024 & -0.027 & -0.025 & -0.026 & -0.027 & -0.026 \\
  &  Variance & 0.00015 & 8.2e-05 & 0.00045 & 0.00013 & 7.8e-05 & 0.00043 & 0.00011 & 7.7e-05 & 0.0004 & 8.7e-05 & 9.3e-05 & 0.00012 & 7.1e-05 & 9.7e-05 & 0.00012 \\
  &  MSE & 0.00065 & 0.00082 & 0.0014 & 0.00069 & 0.00085 & 0.0014 & 0.00073 & 0.00087 & 0.0013 & 0.00077 & 0.00091 & 0.00086 & 0.00081 & 0.00094 & 0.00089 \\
\multirow[c]{-4}{*}{$p_{9}$} &  Coverage & 1 & 1 & 1 & 1 & 1 & 1 & 1 & 1 & 1 & 1 & 1 & 1 & 1 & 1 & 1 \\
\cmidrule(l){2-17}   &  Bias & 0.046 & 0.041 & 0.034 & 0.047 & 0.042 & 0.037 & 0.047 & 0.045 & 0.04 & 0.046 & 0.047 & 0.035 & 0.047 & 0.051 & 0.041 \\
  &  Variance & 0.0027 & 0.0016 & 0.011 & 0.0023 & 0.0016 & 0.011 & 0.002 & 0.0016 & 0.011 & 0.0018 & 0.0017 & 0.0016 & 0.0016 & 0.0018 & 0.0016 \\
  &  MSE & 0.0074 & 0.0049 & 0.022 & 0.0067 & 0.0049 & 0.022 & 0.0063 & 0.0053 & 0.023 & 0.0058 & 0.0057 & 0.0044 & 0.0054 & 0.0062 & 0.0049 \\
\multirow[c]{-4}{*}{$p_{10}$} &  Coverage & 1 & 1 & 1 & 1 & 1 & 1 & 1 & 1 & 1 & 1 & 1 & 1 & 1 & 1 & 1 \\
\bottomrule
\end{tabular}

	}
\end{table}
\begin{table}[H]
	\caption{Hindmarsh-Rose (continued)} \label{tab:hindUQ4}	
	\resizebox{\textwidth}{!}{
	\begin{tabular}{lccccccccccccccccccccccccccccccccccccccccccccccccccccccccccccc}
 &  & \multicolumn{3}{c}{16$\%$} & \multicolumn{3}{c}{17$\%$} & \multicolumn{3}{c}{18$\%$} & \multicolumn{3}{c}{19$\%$} & \multicolumn{3}{c}{20$\%$} \\
\cmidrule(l){3-5} \cmidrule(l){6-8} \cmidrule(l){9-11} \cmidrule(l){12-14} \cmidrule(l){15-17}  &  & M=1024 & M=512 & M=256 & M=1024 & M=512 & M=256 & M=1024 & M=512 & M=256 & M=1024 & M=512 & M=256 & M=1024 & M=512 & M=256 \\
\toprule
  &  Bias & -1.4 & -3.4 & -6.2 & -1.6 & -3.6 & -6.5 & -1.8 & -3.8 & -6.8 & -2 & -4 & -7.1 & -2.2 & -4.2 & -7.3 \\
  &  Variance & 0.14 & 0.22 & 0.53 & 0.16 & 0.24 & 0.58 & 0.18 & 0.25 & 0.66 & 0.2 & 0.26 & 0.7 & 0.22 & 0.28 & 0.64 \\
  &  MSE & 2.3 & 12 & 39 & 2.9 & 13 & 43 & 3.5 & 15 & 47 & 4.2 & 16 & 51 & 5.1 & 18 & 54 \\
\multirow[c]{-4}{*}{$p_{1}$} &  Coverage & 0.5 & 0 & 0 & 0.39 & 0 & 0 & 0.28 & 0 & 0 & 0.21 & 0 & 0 & 0.13 & 0 & 0 \\
\cmidrule(l){2-17}   &  Bias & -3.4 & -4.7 & -6.1 & -3.7 & -4.9 & -6.4 & -4.1 & -5.2 & -6.7 & -4.4 & -5.5 & -7 & -4.7 & -5.7 & -7.3 \\
  &  Variance & 0.12 & 0.28 & 0.54 & 0.12 & 0.28 & 0.48 & 0.12 & 0.28 & 0.48 & 0.13 & 0.28 & 0.48 & 0.13 & 0.28 & 0.45 \\
  &  MSE & 12 & 22 & 38 & 14 & 25 & 42 & 17 & 28 & 46 & 19 & 31 & 51 & 22 & 34 & 54 \\
\multirow[c]{-4}{*}{$p_{2}$} &  Coverage & 0 & 0 & 0 & 0 & 0 & 0 & 0 & 0 & 0 & 0 & 0 & 0 & 0 & 0 & 0 \\
\cmidrule(l){2-17}   &  Bias & -8.5 & -14 & -23 & -9.3 & -15 & -24 & -10 & -16 & -25 & -11 & -16 & -26 & -12 & -17 & -26 \\
  &  Variance & 0.84 & 2.2 & 6 & 0.87 & 2.2 & 6.7 & 0.88 & 2.2 & 8.2 & 0.9 & 2.1 & 8.1 & 0.94 & 2.1 & 7.1 \\
  &  MSE & 74 & 2.1e+02 & 5.5e+02 & 88 & 2.3e+02 & 6e+02 & 1e+02 & 2.5e+02 & 6.5e+02 & 1.2e+02 & 2.7e+02 & 6.9e+02 & 1.4e+02 & 3e+02 & 7.2e+02 \\
\multirow[c]{-4}{*}{$p_{3}$} &  Coverage & 0 & 0 & 0 & 0 & 0 & 0 & 0 & 0 & 0 & 0 & 0 & 0 & 0 & 0 & 0 \\
\cmidrule(l){2-17}   &  Bias & 8.4 & 8.9 & 9.4 & 9 & 8.9 & 9.4 & 9.5 & 9 & 9.1 & 9.7 & 9 & 9 & 9.9 & 9 & 8.9 \\
  &  Variance & 3.5 & 11 & 5 & 2.2 & 12 & 5.3 & 1.2 & 12 & 7.9 & 0.58 & 12 & 8.7 & 0.26 & 12 & 9.5 \\
  &  MSE & 77 & 1e+02 & 98 & 85 & 1e+02 & 98 & 92 & 1e+02 & 99 & 96 & 1.1e+02 & 99 & 98 & 1.1e+02 & 99 \\
\multirow[c]{-4}{*}{$p_{4}$} &  Coverage & 0.79 & 0.96 & 0.75 & 0.75 & 0.96 & 0.68 & 0.66 & 0.94 & 0.63 & 0.65 & 0.93 & 0.54 & 0.59 & 0.9 & 0.49 \\
\cmidrule(l){2-17}   &  Bias & -1.8 & -2.7 & -4.8 & -1.8 & -2.6 & -5.1 & -1.9 & -2.6 & -5.5 & -1.8 & -2.6 & -5.7 & -1.8 & -2.6 & -5.9 \\
  &  Variance & 0.25 & 0.71 & 2.2 & 0.25 & 0.77 & 2.7 & 0.26 & 0.83 & 3.7 & 0.29 & 0.88 & 4 & 0.33 & 0.94 & 3.9 \\
  &  MSE & 3.6 & 8.5 & 28 & 3.8 & 8.5 & 31 & 4 & 8.6 & 37 & 4 & 8.6 & 41 & 3.9 & 8.6 & 42 \\
\multirow[c]{-4}{*}{$p_{5}$} &  Coverage & 1 & 0.96 & 0.53 & 1 & 0.96 & 0.5 & 1 & 0.97 & 0.42 & 1 & 0.97 & 0.39 & 1 & 0.98 & 0.38 \\
\cmidrule(l){2-17}   &  Bias & -15 & -24 & -31 & -16 & -25 & -31 & -17 & -26 & -32 & -19 & -27 & -33 & -20 & -28 & -34 \\
  &  Variance & 6.3 & 7 & 11 & 7.7 & 7 & 10 & 6.5 & 7 & 9.4 & 6.6 & 7 & 8.8 & 6.6 & 6.9 & 8.3 \\
  &  MSE & 2.4e+02 & 5.8e+02 & 9.5e+02 & 2.8e+02 & 6.3e+02 & 1e+03 & 3.2e+02 & 6.8e+02 & 1.1e+03 & 3.6e+02 & 7.3e+02 & 1.1e+03 & 4e+02 & 7.8e+02 & 1.2e+03 \\
\multirow[c]{-4}{*}{$p_{6}$} &  Coverage & 0 & 0 & 0 & 0.01 & 0 & 0 & 0 & 0 & 0 & 0 & 0 & 0 & 0 & 0 & 0 \\
\cmidrule(l){2-17}   &  Bias & -2.3 & -4.4 & -5.9 & -2.4 & -4.7 & -6.1 & -2.7 & -4.9 & -6.2 & -2.9 & -5 & -6.4 & -3.1 & -5.2 & -6.5 \\
  &  Variance & 0.33 & 0.43 & 0.8 & 0.43 & 0.44 & 0.82 & 0.38 & 0.45 & 0.86 & 0.39 & 0.46 & 0.83 & 0.41 & 0.46 & 0.8 \\
  &  MSE & 5.8 & 21 & 37 & 6.9 & 23 & 39 & 8 & 24 & 41 & 9.2 & 26 & 42 & 10 & 28 & 44 \\
\multirow[c]{-4}{*}{$p_{7}$} &  Coverage & 0.08 & 0 & 0 & 0.06 & 0 & 0 & 0.04 & 0 & 0 & 0.04 & 0 & 0 & 0.03 & 0 & 0 \\
\cmidrule(l){2-17}   &  Bias & -0.031 & -0.03 & -0.029 & -0.033 & -0.03 & -0.029 & -0.034 & -0.03 & -0.028 & -0.034 & -0.03 & -0.028 & -0.035 & -0.029 & -0.028 \\
  &  Variance & 8.6e-05 & 0.00026 & 0.00039 & 7.6e-05 & 0.00029 & 0.00041 & 6.7e-05 & 0.00032 & 0.00049 & 6.2e-05 & 0.00035 & 0.00054 & 6.2e-05 & 0.00038 & 0.00061 \\
  &  MSE & 0.0011 & 0.0014 & 0.0016 & 0.0012 & 0.0015 & 0.0017 & 0.0013 & 0.0015 & 0.0018 & 0.0013 & 0.0016 & 0.0019 & 0.0013 & 0.0016 & 0.002 \\
\multirow[c]{-4}{*}{$p_{8}$} &  Coverage & 1 & 1 & 1 & 1 & 1 & 1 & 1 & 1 & 1 & 1 & 1 & 1 & 1 & 1 & 1 \\
\cmidrule(l){2-17}   &  Bias & -0.027 & -0.028 & -0.026 & -0.027 & -0.028 & -0.027 & -0.028 & -0.028 & -0.027 & -0.028 & -0.028 & -0.027 & -0.028 & -0.028 & -0.027 \\
  &  Variance & 5.9e-05 & 9.6e-05 & 0.00012 & 5.1e-05 & 1e-04 & 0.00012 & 4.5e-05 & 0.0001 & 0.00014 & 4e-05 & 0.00011 & 0.00016 & 3.7e-05 & 0.00011 & 0.00017 \\
  &  MSE & 0.00083 & 0.00095 & 0.00093 & 0.00084 & 0.00096 & 0.00096 & 0.00085 & 0.00097 & 0.001 & 0.00087 & 0.00098 & 0.001 & 0.00087 & 0.001 & 0.0011 \\
\multirow[c]{-4}{*}{$p_{9}$} &  Coverage & 1 & 1 & 1 & 1 & 1 & 1 & 1 & 1 & 1 & 1 & 1 & 1 & 1 & 1 & 1 \\
\cmidrule(l){2-17}   &  Bias & 0.047 & 0.054 & 0.046 & 0.047 & 0.058 & 0.051 & 0.05 & 0.062 & 0.056 & 0.053 & 0.066 & 0.059 & 0.056 & 0.071 & 0.062 \\
  &  Variance & 0.0015 & 0.0019 & 0.0017 & 0.0014 & 0.0021 & 0.0018 & 0.0012 & 0.0023 & 0.002 & 0.0012 & 0.0026 & 0.002 & 0.0013 & 0.0031 & 0.0022 \\
  &  MSE & 0.0051 & 0.0068 & 0.0056 & 0.005 & 0.0076 & 0.0063 & 0.005 & 0.0085 & 0.0071 & 0.0053 & 0.0097 & 0.0075 & 0.0058 & 0.011 & 0.0083 \\
\multirow[c]{-4}{*}{$p_{10}$} &  Coverage & 1 & 1 & 1 & 1 & 1 & 1 & 1 & 1 & 1 & 1 & 1 & 1 & 1 & 1 & 1 \\
\bottomrule
\end{tabular}

	}
\end{table}

\begin{table}[H]
	\caption{Lorenz} \label{tab:lorenzUQ1}	
	\resizebox{\textwidth}{!}{
	\begin{tabular}{lccccccccccccccccccccccccccccccccccccccccccccccccccccccccccccccccccc}
 &  & \multicolumn{3}{c}{1$\%$} & \multicolumn{3}{c}{2$\%$} & \multicolumn{3}{c}{3$\%$} & \multicolumn{3}{c}{4$\%$} & \multicolumn{3}{c}{5$\%$} \\
\cmidrule(l){3-5} \cmidrule(l){6-8} \cmidrule(l){9-11} \cmidrule(l){12-14} \cmidrule(l){15-17}  &  & M=1000 & M=500 & M=250 & M=1000 & M=500 & M=250 & M=1000 & M=500 & M=250 & M=1000 & M=500 & M=250 & M=1000 & M=500 & M=250 \\
\toprule
  &  Bias & -0.075 & -0.07 & -0.077 & -0.099 & -0.09 & -0.096 & -0.13 & -0.11 & -0.13 & -0.15 & -0.14 & -0.16 & -0.18 & -0.16 & -0.2 \\
  &  Variance & 8.2e-05 & 0.00018 & 0.00043 & 0.00038 & 0.00085 & 0.002 & 0.001 & 0.0023 & 0.0052 & 0.0021 & 0.0049 & 0.011 & 0.0038 & 0.0088 & 0.018 \\
  &  MSE & 0.0057 & 0.0052 & 0.0067 & 0.011 & 0.0097 & 0.013 & 0.018 & 0.017 & 0.026 & 0.027 & 0.028 & 0.047 & 0.038 & 0.043 & 0.077 \\
\multirow[c]{-4}{*}{$p_{1}$} &  Coverage & 0.75 & 1 & 1 & 1 & 1 & 1 & 1 & 1 & 1 & 1 & 1 & 1 & 1 & 1 & 1 \\
\cmidrule(l){2-17}   &  Bias & -0.035 & -0.035 & -0.034 & -0.022 & -0.022 & -0.021 & -0.013 & -0.012 & -0.011 & -0.0078 & -0.0051 & -0.005 & -0.0049 & -0.00056 & -0.002 \\
  &  Variance & 3.2e-05 & 6.6e-05 & 0.00013 & 0.00012 & 0.00026 & 0.00052 & 0.00025 & 0.00062 & 0.0012 & 0.00045 & 0.0011 & 0.0022 & 0.00071 & 0.0018 & 0.0035 \\
  &  MSE & 0.0013 & 0.0014 & 0.0014 & 0.00071 & 0.001 & 0.0015 & 0.00067 & 0.0014 & 0.0025 & 0.00096 & 0.0023 & 0.0044 & 0.0014 & 0.0037 & 0.0069 \\
\multirow[c]{-4}{*}{$p_{2}$} &  Coverage & 0 & 0.06 & 0.38 & 0.82 & 0.92 & 0.98 & 0.97 & 0.99 & 1 & 0.98 & 0.98 & 0.99 & 1 & 0.98 & 0.99 \\
\cmidrule(l){2-17}   &  Bias & -0.038 & -0.037 & -0.037 & -0.036 & -0.036 & -0.036 & -0.033 & -0.034 & -0.033 & -0.031 & -0.031 & -0.03 & -0.028 & -0.029 & -0.026 \\
  &  Variance & 6.1e-07 & 1.3e-06 & 2.4e-06 & 2.6e-06 & 6e-06 & 9.6e-06 & 6.8e-06 & 1.6e-05 & 2.5e-05 & 1.5e-05 & 3.5e-05 & 5e-05 & 2.7e-05 & 6.4e-05 & 8.8e-05 \\
  &  MSE & 0.0014 & 0.0014 & 0.0014 & 0.0013 & 0.0013 & 0.0013 & 0.0011 & 0.0012 & 0.0011 & 0.00097 & 0.0011 & 0.001 & 0.00085 & 0.00096 & 0.00087 \\
\multirow[c]{-4}{*}{$p_{3}$} &  Coverage & 0 & 0 & 0 & 0 & 0 & 0 & 0 & 0 & 0.5 & 0.02 & 0.45 & 0.96 & 0.44 & 0.88 & 1 \\
\bottomrule
 &  & \multicolumn{3}{c}{6$\%$} & \multicolumn{3}{c}{7$\%$} & \multicolumn{3}{c}{8$\%$} & \multicolumn{3}{c}{9$\%$} & \multicolumn{3}{c}{10$\%$} \\
\cmidrule(l){3-5} \cmidrule(l){6-8} \cmidrule(l){9-11} \cmidrule(l){12-14} \cmidrule(l){15-17}  &  & M=1000 & M=500 & M=250 & M=1000 & M=500 & M=250 & M=1000 & M=500 & M=250 & M=1000 & M=500 & M=250 & M=1000 & M=500 & M=250 \\
\toprule
  &  Bias & -0.2 & -0.19 & -0.25 & -0.22 & -0.21 & -0.3 & -0.25 & -0.25 & -0.37 & -0.28 & -0.29 & -0.45 & -0.31 & -0.34 & -0.54 \\
  &  Variance & 0.0062 & 0.014 & 0.028 & 0.0095 & 0.021 & 0.041 & 0.014 & 0.03 & 0.057 & 0.018 & 0.041 & 0.076 & 0.024 & 0.054 & 0.098 \\
  &  MSE & 0.052 & 0.063 & 0.12 & 0.068 & 0.089 & 0.18 & 0.088 & 0.12 & 0.25 & 0.11 & 0.17 & 0.35 & 0.14 & 0.22 & 0.49 \\
\multirow[c]{-4}{*}{$p_{1}$} &  Coverage & 1 & 1 & 1 & 1 & 1 & 1 & 1 & 1 & 1 & 1 & 1 & 1 & 1 & 1 & 1 \\
\cmidrule(l){2-17}   &  Bias & -0.0032 & 0.0028 & -0.00029 & -0.0021 & 0.0055 & 0.00069 & -0.0013 & 0.0079 & 0.0013 & -0.00067 & 0.01 & 0.0019 & -0.00013 & 0.012 & 0.0024 \\
  &  Variance & 0.001 & 0.0027 & 0.0051 & 0.0014 & 0.0038 & 0.0071 & 0.0019 & 0.005 & 0.0094 & 0.0024 & 0.0064 & 0.012 & 0.003 & 0.008 & 0.015 \\
  &  MSE & 0.0021 & 0.0054 & 0.01 & 0.0028 & 0.0076 & 0.014 & 0.0038 & 0.01 & 0.019 & 0.0048 & 0.013 & 0.024 & 0.0061 & 0.016 & 0.031 \\
\multirow[c]{-4}{*}{$p_{2}$} &  Coverage & 1 & 0.98 & 0.99 & 1 & 0.98 & 0.99 & 1 & 0.98 & 0.99 & 1 & 0.98 & 0.99 & 1 & 0.99 & 0.99 \\
\cmidrule(l){2-17}   &  Bias & -0.026 & -0.026 & -0.022 & -0.024 & -0.024 & -0.018 & -0.021 & -0.021 & -0.014 & -0.019 & -0.019 & -0.0098 & -0.017 & -0.016 & -0.0054 \\
  &  Variance & 4.3e-05 & 0.00011 & 0.00014 & 6.6e-05 & 0.00016 & 0.00021 & 9.7e-05 & 0.00024 & 0.00031 & 0.00014 & 0.00034 & 0.00043 & 0.00018 & 0.00046 & 0.00059 \\
  &  MSE & 0.00075 & 0.0009 & 0.00079 & 0.00069 & 0.00089 & 0.00077 & 0.00065 & 0.00092 & 0.00082 & 0.00065 & 0.001 & 0.00096 & 0.00067 & 0.0012 & 0.0012 \\
\multirow[c]{-4}{*}{$p_{3}$} &  Coverage & 0.88 & 0.98 & 1 & 0.98 & 0.99 & 1 & 0.99 & 1 & 1 & 0.99 & 1 & 1 & 0.99 & 1 & 1 \\
\bottomrule
 &  & \multicolumn{3}{c}{11$\%$} & \multicolumn{3}{c}{12$\%$} & \multicolumn{3}{c}{13$\%$} & \multicolumn{3}{c}{14$\%$} & \multicolumn{3}{c}{15$\%$} \\
\cmidrule(l){3-5} \cmidrule(l){6-8} \cmidrule(l){9-11} \cmidrule(l){12-14} \cmidrule(l){15-17}  &  & M=1000 & M=500 & M=250 & M=1000 & M=500 & M=250 & M=1000 & M=500 & M=250 & M=1000 & M=500 & M=250 & M=1000 & M=500 & M=250 \\
\toprule
  &  Bias & -0.35 & -0.4 & -0.64 & -0.39 & -0.46 & -0.76 & -0.44 & -0.54 & -0.89 & -0.5 & -0.63 & -1 & -0.56 & -0.72 & -1.2 \\
  &  Variance & 0.031 & 0.069 & 0.12 & 0.039 & 0.086 & 0.15 & 0.049 & 0.11 & 0.18 & 0.059 & 0.13 & 0.21 & 0.07 & 0.15 & 0.24 \\
  &  MSE & 0.18 & 0.3 & 0.66 & 0.23 & 0.39 & 0.87 & 0.29 & 0.5 & 1.1 & 0.36 & 0.64 & 1.5 & 0.45 & 0.82 & 1.9 \\
\multirow[c]{-4}{*}{$p_{1}$} &  Coverage & 1 & 1 & 1 & 1 & 1 & 1 & 1 & 1 & 1 & 1 & 1 & 0.98 & 1 & 1 & 0.97 \\
\cmidrule(l){2-17}   &  Bias & 0.00036 & 0.014 & 0.003 & 0.0008 & 0.017 & 0.0036 & 0.0012 & 0.019 & 0.0043 & 0.0017 & 0.021 & 0.0051 & 0.0021 & 0.024 & 0.0059 \\
  &  Variance & 0.0037 & 0.0099 & 0.019 & 0.0045 & 0.012 & 0.023 & 0.0054 & 0.014 & 0.028 & 0.0063 & 0.017 & 0.033 & 0.0074 & 0.019 & 0.038 \\
  &  MSE & 0.0075 & 0.02 & 0.038 & 0.009 & 0.024 & 0.046 & 0.011 & 0.029 & 0.055 & 0.013 & 0.034 & 0.066 & 0.015 & 0.039 & 0.077 \\
\multirow[c]{-4}{*}{$p_{2}$} &  Coverage & 1 & 0.99 & 0.99 & 1 & 0.99 & 0.99 & 1 & 0.99 & 0.99 & 1 & 0.99 & 0.99 & 1 & 0.99 & 0.99 \\
\cmidrule(l){2-17}   &  Bias & -0.015 & -0.013 & -0.001 & -0.014 & -0.011 & 0.0032 & -0.012 & -0.0083 & 0.0073 & -0.01 & -0.006 & 0.011 & -0.009 & -0.0039 & 0.014 \\
  &  Variance & 0.00024 & 0.00062 & 0.00078 & 0.00031 & 0.00081 & 0.001 & 0.0004 & 0.001 & 0.0013 & 0.0005 & 0.0013 & 0.0016 & 0.00061 & 0.0016 & 0.002 \\
  &  MSE & 0.00073 & 0.0014 & 0.0016 & 0.00082 & 0.0017 & 0.002 & 0.00094 & 0.0022 & 0.0027 & 0.0011 & 0.0027 & 0.0034 & 0.0013 & 0.0033 & 0.0043 \\
\multirow[c]{-4}{*}{$p_{3}$} &  Coverage & 1 & 1 & 1 & 1 & 1 & 1 & 1 & 1 & 1 & 1 & 1 & 1 & 1 & 1 & 1 \\
\bottomrule
 &  & \multicolumn{3}{c}{16$\%$} & \multicolumn{3}{c}{17$\%$} & \multicolumn{3}{c}{18$\%$} & \multicolumn{3}{c}{19$\%$} & \multicolumn{3}{c}{20$\%$} \\
\cmidrule(l){3-5} \cmidrule(l){6-8} \cmidrule(l){9-11} \cmidrule(l){12-14} \cmidrule(l){15-17}  &  & M=1000 & M=500 & M=250 & M=1000 & M=500 & M=250 & M=1000 & M=500 & M=250 & M=1000 & M=500 & M=250 & M=1000 & M=500 & M=250 \\
\toprule
  &  Bias & -0.62 & -0.83 & -1.4 & -0.7 & -0.95 & -1.5 & -0.78 & -1.1 & -1.7 & -0.86 & -1.2 & -1.9 & -0.96 & -1.4 & -2.1 \\
  &  Variance & 0.081 & 0.17 & 0.27 & 0.094 & 0.2 & 0.3 & 0.11 & 0.22 & 0.32 & 0.12 & 0.25 & 0.35 & 0.13 & 0.27 & 0.37 \\
  &  MSE & 0.55 & 1 & 2.4 & 0.67 & 1.3 & 2.9 & 0.82 & 1.6 & 3.6 & 0.99 & 2 & 4.3 & 1.2 & 2.4 & 5.2 \\
\multirow[c]{-4}{*}{$p_{1}$} &  Coverage & 1 & 1 & 0.93 & 1 & 1 & 0.92 & 1 & 1 & 0.88 & 1 & 1 & 0.76 & 1 & 1 & 0.69 \\
\cmidrule(l){2-17}   &  Bias & 0.0025 & 0.026 & 0.0067 & 0.0028 & 0.028 & 0.0075 & 0.0032 & 0.031 & 0.0083 & 0.0034 & 0.034 & 0.0089 & 0.0036 & 0.036 & 0.0093 \\
  &  Variance & 0.0086 & 0.022 & 0.045 & 0.0099 & 0.026 & 0.052 & 0.011 & 0.029 & 0.059 & 0.013 & 0.033 & 0.067 & 0.014 & 0.037 & 0.076 \\
  &  MSE & 0.017 & 0.045 & 0.089 & 0.02 & 0.052 & 0.1 & 0.022 & 0.059 & 0.12 & 0.026 & 0.067 & 0.13 & 0.029 & 0.076 & 0.15 \\
\multirow[c]{-4}{*}{$p_{2}$} &  Coverage & 1 & 0.99 & 0.99 & 1 & 0.99 & 0.99 & 1 & 0.99 & 0.99 & 1 & 1 & 1 & 1 & 1 & 1 \\
\cmidrule(l){2-17}   &  Bias & -0.0078 & -0.002 & 0.017 & -0.0067 & -0.00041 & 0.02 & -0.0059 & 0.00086 & 0.021 & -0.0052 & 0.0018 & 0.022 & -0.0047 & 0.0023 & 0.023 \\
  &  Variance & 0.00074 & 0.002 & 0.0025 & 0.00088 & 0.0025 & 0.0031 & 0.001 & 0.003 & 0.0037 & 0.0012 & 0.0035 & 0.0045 & 0.0014 & 0.0042 & 0.0053 \\
  &  MSE & 0.0015 & 0.0041 & 0.0053 & 0.0018 & 0.0049 & 0.0065 & 0.0021 & 0.0059 & 0.0079 & 0.0025 & 0.007 & 0.0094 & 0.0029 & 0.0083 & 0.011 \\
\multirow[c]{-4}{*}{$p_{3}$} &  Coverage & 1 & 1 & 1 & 1 & 1 & 1 & 1 & 1 & 1 & 1 & 1 & 1 & 1 & 1 & 1 \\
\bottomrule
\end{tabular}

	}
\end{table}

\begin{table}[H]
	\caption{Lorenz (continued)} \label{tab:lorenzUQ2}	
	\resizebox{\textwidth}{!}{
	\begin{tabular}{lccccccccccccccccccccccccccccccccccccccccccccccccccccccccccccccccccc}
 &  & \multicolumn{3}{c}{25$\%$} & \multicolumn{3}{c}{30$\%$} \\
\cmidrule(l){3-5} \cmidrule(l){6-8}  &  & M=1000 & M=500 & M=250 & M=1000 & M=500 & M=250 \\
\toprule
  &  Bias & -1.5 & -2.1 & -3.2 & -2.1 & -3 & -4.2 & \hspace{20pt} & \hspace{20pt} & \hspace{20pt} & \hspace{20pt} & \hspace{20pt} & \hspace{20pt} & \hspace{20pt} & \hspace{20pt} & \hspace{20pt} \\
  &  Variance & 0.2 & 0.38 & 0.45 & 0.26 & 0.45 & 0.48 & \hspace{20pt} & \hspace{20pt} & \hspace{20pt} & \hspace{20pt} & \hspace{20pt} & \hspace{20pt} & \hspace{20pt} & \hspace{20pt} & \hspace{20pt} \\
  &  MSE & 2.6 & 5.4 & 11 & 4.9 & 9.9 & 19 & \hspace{20pt} & \hspace{20pt} & \hspace{20pt} & \hspace{20pt} & \hspace{20pt} & \hspace{20pt} & \hspace{20pt} & \hspace{20pt} & \hspace{20pt} \\
\multirow[c]{-4}{*}{$p_{1}$} &  Coverage & 1 & 0.97 & 0.22 & 1 & 0.81 & 0.02 & \hspace{20pt} & \hspace{20pt} & \hspace{20pt} & \hspace{20pt} & \hspace{20pt} & \hspace{20pt} & \hspace{20pt} & \hspace{20pt} & \hspace{20pt} \\
\cmidrule(l){2-17}   &  Bias & 0.0033 & 0.048 & 0.0078 & -0.00046 & 0.057 & -0.0038 & \hspace{20pt} & \hspace{20pt} & \hspace{20pt} & \hspace{20pt} & \hspace{20pt} & \hspace{20pt} & \hspace{20pt} & \hspace{20pt} & \hspace{20pt} \\
  &  Variance & 0.025 & 0.063 & 0.13 & 0.04 & 0.099 & 0.21 & \hspace{20pt} & \hspace{20pt} & \hspace{20pt} & \hspace{20pt} & \hspace{20pt} & \hspace{20pt} & \hspace{20pt} & \hspace{20pt} & \hspace{20pt} \\
  &  MSE & 0.05 & 0.13 & 0.27 & 0.079 & 0.2 & 0.43 & \hspace{20pt} & \hspace{20pt} & \hspace{20pt} & \hspace{20pt} & \hspace{20pt} & \hspace{20pt} & \hspace{20pt} & \hspace{20pt} & \hspace{20pt} \\
\multirow[c]{-4}{*}{$p_{2}$} &  Coverage & 1 & 1 & 1 & 1 & 1 & 1 & \hspace{20pt} & \hspace{20pt} & \hspace{20pt} & \hspace{20pt} & \hspace{20pt} & \hspace{20pt} & \hspace{20pt} & \hspace{20pt} & \hspace{20pt} \\
\cmidrule(l){2-17}   &  Bias & -0.0056 & -0.0011 & 0.014 & -0.012 & -0.015 & -0.012 & \hspace{20pt} & \hspace{20pt} & \hspace{20pt} & \hspace{20pt} & \hspace{20pt} & \hspace{20pt} & \hspace{20pt} & \hspace{20pt} & \hspace{20pt} \\
  &  Variance & 0.0028 & 0.0085 & 0.011 & 0.0047 & 0.015 & 0.02 & \hspace{20pt} & \hspace{20pt} & \hspace{20pt} & \hspace{20pt} & \hspace{20pt} & \hspace{20pt} & \hspace{20pt} & \hspace{20pt} & \hspace{20pt} \\
  &  MSE & 0.0055 & 0.017 & 0.022 & 0.0096 & 0.03 & 0.041 & \hspace{20pt} & \hspace{20pt} & \hspace{20pt} & \hspace{20pt} & \hspace{20pt} & \hspace{20pt} & \hspace{20pt} & \hspace{20pt} & \hspace{20pt} \\
\multirow[c]{-4}{*}{$p_{3}$} &  Coverage & 1 & 1 & 1 & 1 & 1 & 1 & \hspace{20pt} & \hspace{20pt} & \hspace{20pt} & \hspace{20pt} & \hspace{20pt} & \hspace{20pt} & \hspace{20pt} & \hspace{20pt} & \hspace{20pt} \\
\bottomrule
\end{tabular}

	}
\end{table}

\begin{table}[H]
	\caption{Goodwin 2D} \label{tab:goodwin2DUQ1}	
	\resizebox{\textwidth}{!}{
	\begin{tabular}{lcccccccccccccccccccccccccccccccccccccccccccccccccccccccccccccccc}
 &  & \multicolumn{3}{c}{1$\%$} & \multicolumn{3}{c}{2$\%$} & \multicolumn{3}{c}{3$\%$} & \multicolumn{3}{c}{4$\%$} & \multicolumn{3}{c}{5$\%$} \\
\cmidrule(l){3-5} \cmidrule(l){6-8} \cmidrule(l){9-11} \cmidrule(l){12-14} \cmidrule(l){15-17}  &  & M=1024 & M=512 & M=256 & M=1024 & M=512 & M=256 & M=1024 & M=512 & M=256 & M=1024 & M=512 & M=256 & M=1024 & M=512 & M=256 \\
\toprule
  &  Bias & -3.9 & -5.9 & -6 & -4.7 & -4.7 & -5.1 & -3.4 & -4.7 & -4.7 & -3.8 & -5.6 & -6.1 & -4.5 & -5.5 & -5.6 \\
  &  Variance & 42 & 40 & 46 & 43 & 43 & 47 & 41 & 44 & 49 & 41 & 41 & 43 & 42 & 40 & 41 \\
  &  MSE & 1e+02 & 1.1e+02 & 1.3e+02 & 1.1e+02 & 1.1e+02 & 1.2e+02 & 94 & 1.1e+02 & 1.2e+02 & 98 & 1.1e+02 & 1.2e+02 & 1e+02 & 1.1e+02 & 1.1e+02 \\
\multirow[c]{-4}{*}{$p_{1}$} &  Coverage & 0.86 & 0.84 & 0.83 & 0.87 & 0.89 & 0.87 & 0.95 & 0.91 & 0.87 & 1 & 0.99 & 0.99 & 1 & 1 & 1 \\
\cmidrule(l){2-17}   &  Bias & 0.92 & 1 & 0.94 & 0.9 & 1.1 & 0.93 & 0.9 & 1.1 & 0.96 & 0.89 & 1 & 1 & 0.9 & 0.99 & 1.1 \\
  &  Variance & 0.23 & 0.28 & 0.26 & 0.2 & 0.35 & 0.28 & 0.29 & 0.36 & 0.28 & 0.24 & 0.35 & 0.31 & 0.24 & 0.35 & 0.41 \\
  &  MSE & 1.3 & 1.6 & 1.4 & 1.2 & 1.8 & 1.4 & 1.4 & 1.9 & 1.5 & 1.3 & 1.8 & 1.6 & 1.3 & 1.7 & 2 \\
\multirow[c]{-4}{*}{$p_{2}$} &  Coverage & 0.1 & 0.07 & 0.08 & 0.07 & 0.11 & 0.12 & 0.16 & 0.11 & 0.09 & 0.1 & 0.09 & 0.09 & 0.1 & 0.13 & 0.14 \\
\cmidrule(l){2-17}   &  Bias & -0.42 & -0.56 & -0.47 & -0.39 & -0.54 & -0.5 & -0.46 & -0.51 & -0.51 & -0.39 & -0.53 & -0.58 & -0.4 & -0.48 & -0.58 \\
  &  Variance & 0.15 & 0.21 & 0.12 & 0.18 & 0.18 & 0.17 & 0.16 & 0.25 & 0.18 & 0.21 & 0.22 & 0.19 & 0.17 & 0.23 & 0.22 \\
  &  MSE & 0.49 & 0.72 & 0.46 & 0.51 & 0.65 & 0.59 & 0.53 & 0.76 & 0.61 & 0.57 & 0.72 & 0.73 & 0.5 & 0.7 & 0.78 \\
\multirow[c]{-4}{*}{$p_{3}$} &  Coverage & 0.84 & 0.79 & 0.72 & 0.87 & 0.85 & 0.81 & 0.95 & 0.88 & 0.82 & 0.94 & 0.89 & 0.89 & 1 & 0.98 & 0.93 \\
\cmidrule(l){2-17}   &  Bias & 0.41 & 0.46 & 0.26 & 0.35 & 0.56 & 0.39 & 0.37 & 0.64 & 0.47 & 0.35 & 0.52 & 0.49 & 0.34 & 0.52 & 0.57 \\
  &  Variance & 0.35 & 0.37 & 0.22 & 0.31 & 0.42 & 0.35 & 0.34 & 0.56 & 0.39 & 0.29 & 0.46 & 0.33 & 0.33 & 0.46 & 0.44 \\
  &  MSE & 0.86 & 0.95 & 0.51 & 0.75 & 1.2 & 0.86 & 0.82 & 1.5 & 1 & 0.7 & 1.2 & 0.9 & 0.78 & 1.2 & 1.2 \\
\multirow[c]{-4}{*}{$p_{4}$} &  Coverage & 0 & 0 & 0.02 & 0.01 & 0 & 0.05 & 0.01 & 0.02 & 0.01 & 0.05 & 0.05 & 0.08 & 0.05 & 0.11 & 0.13 \\
\cmidrule(l){2-17}   &  Bias & 0.49 & 0.61 & 0.44 & 0.5 & 0.65 & 0.59 & 0.49 & 0.77 & 0.6 & 0.48 & 0.61 & 0.61 & 0.41 & 0.69 & 0.7 \\
  &  Variance & 0.43 & 0.51 & 0.43 & 0.42 & 0.6 & 0.52 & 0.44 & 0.63 & 0.55 & 0.42 & 0.59 & 0.53 & 0.44 & 0.65 & 0.58 \\
  &  MSE & 1.1 & 1.4 & 1.1 & 1.1 & 1.6 & 1.4 & 1.1 & 1.9 & 1.5 & 1.1 & 1.6 & 1.4 & 1 & 1.8 & 1.7 \\
\multirow[c]{-4}{*}{$p_{5}$} &  Coverage & 0.03 & 0.04 & 0.1 & 0.06 & 0.13 & 0.13 & 0.15 & 0.16 & 0.14 & 0.14 & 0.18 & 0.19 & 0.18 & 0.23 & 0.27 \\
\bottomrule
 &  & \multicolumn{3}{c}{6$\%$} & \multicolumn{3}{c}{7$\%$} & \multicolumn{3}{c}{8$\%$} & \multicolumn{3}{c}{9$\%$} & \multicolumn{3}{c}{10$\%$} \\
\cmidrule(l){3-5} \cmidrule(l){6-8} \cmidrule(l){9-11} \cmidrule(l){12-14} \cmidrule(l){15-17}  &  & M=1024 & M=512 & M=256 & M=1024 & M=512 & M=256 & M=1024 & M=512 & M=256 & M=1024 & M=512 & M=256 & M=1024 & M=512 & M=256 \\
\toprule
  &  Bias & -4.3 & -3.7 & -5.4 & -4.6 & -4.5 & -5.9 & -4 & -4 & -5.6 & -4.5 & -4.8 & -4.4 & -5.1 & -5.9 & -4.9 \\
  &  Variance & 42 & 42 & 44 & 41 & 48 & 37 & 40 & 44 & 39 & 43 & 39 & 40 & 45 & 37 & 44 \\
  &  MSE & 1e+02 & 98 & 1.2e+02 & 1e+02 & 1.2e+02 & 1.1e+02 & 97 & 1e+02 & 1.1e+02 & 1e+02 & 1e+02 & 99 & 1.2e+02 & 1.1e+02 & 1.1e+02 \\
\multirow[c]{-4}{*}{$p_{1}$} &  Coverage & 1 & 1 & 1 & 1 & 1 & 1 & 1 & 1 & 1 & 1 & 1 & 1 & 1 & 1 & 1 \\
\cmidrule(l){2-17}   &  Bias & 0.94 & 1.1 & 0.97 & 0.93 & 1.1 & 0.98 & 0.94 & 1.1 & 0.9 & 0.87 & 1.1 & 0.92 & 0.94 & 1.1 & 0.93 \\
  &  Variance & 0.22 & 0.37 & 0.38 & 0.27 & 0.38 & 0.42 & 0.26 & 0.42 & 0.41 & 0.27 & 0.43 & 0.5 & 0.28 & 0.41 & 0.43 \\
  &  MSE & 1.3 & 1.8 & 1.7 & 1.4 & 1.9 & 1.8 & 1.4 & 2 & 1.6 & 1.3 & 2 & 1.8 & 1.4 & 1.9 & 1.7 \\
\multirow[c]{-4}{*}{$p_{2}$} &  Coverage & 0.09 & 0.13 & 0.18 & 0.14 & 0.14 & 0.17 & 0.14 & 0.16 & 0.23 & 0.18 & 0.15 & 0.27 & 0.13 & 0.15 & 0.23 \\
\cmidrule(l){2-17}   &  Bias & -0.39 & -0.43 & -0.52 & -0.4 & -0.48 & -0.52 & -0.34 & -0.48 & -0.5 & -0.36 & -0.52 & -0.49 & -0.31 & -0.49 & -0.53 \\
  &  Variance & 0.19 & 0.26 & 0.18 & 0.16 & 0.24 & 0.24 & 0.22 & 0.23 & 0.2 & 0.14 & 0.28 & 0.19 & 0.22 & 0.24 & 0.16 \\
  &  MSE & 0.54 & 0.71 & 0.63 & 0.49 & 0.7 & 0.74 & 0.54 & 0.7 & 0.64 & 0.41 & 0.83 & 0.62 & 0.54 & 0.72 & 0.6 \\
\multirow[c]{-4}{*}{$p_{3}$} &  Coverage & 1 & 1 & 0.99 & 1 & 0.99 & 0.98 & 1 & 1 & 0.99 & 1 & 0.99 & 1 & 1 & 1 & 1 \\
\cmidrule(l){2-17}   &  Bias & 0.38 & 0.52 & 0.48 & 0.35 & 0.54 & 0.51 & 0.37 & 0.54 & 0.47 & 0.29 & 0.63 & 0.57 & 0.17 & 0.57 & 0.49 \\
  &  Variance & 0.43 & 0.47 & 0.42 & 0.35 & 0.49 & 0.54 & 0.36 & 0.48 & 0.43 & 0.32 & 0.55 & 0.5 & 0.19 & 0.51 & 0.45 \\
  &  MSE & 1 & 1.2 & 1.1 & 0.83 & 1.3 & 1.3 & 0.85 & 1.2 & 1.1 & 0.73 & 1.5 & 1.3 & 0.4 & 1.4 & 1.1 \\
\multirow[c]{-4}{*}{$p_{4}$} &  Coverage & 0.12 & 0.11 & 0.16 & 0.17 & 0.13 & 0.2 & 0.25 & 0.18 & 0.26 & 0.37 & 0.21 & 0.26 & 0.48 & 0.23 & 0.29 \\
\cmidrule(l){2-17}   &  Bias & 0.48 & 0.77 & 0.67 & 0.4 & 0.72 & 0.61 & 0.5 & 0.64 & 0.58 & 0.4 & 0.64 & 0.58 & 0.37 & 0.71 & 0.69 \\
  &  Variance & 0.42 & 0.65 & 0.61 & 0.42 & 0.65 & 0.59 & 0.43 & 0.58 & 0.52 & 0.37 & 0.6 & 0.6 & 0.4 & 0.63 & 0.55 \\
  &  MSE & 1.1 & 1.9 & 1.7 & 1 & 1.8 & 1.6 & 1.1 & 1.6 & 1.4 & 0.9 & 1.6 & 1.5 & 0.93 & 1.8 & 1.6 \\
\multirow[c]{-4}{*}{$p_{5}$} &  Coverage & 0.19 & 0.24 & 0.34 & 0.28 & 0.24 & 0.35 & 0.37 & 0.34 & 0.36 & 0.36 & 0.34 & 0.46 & 0.5 & 0.33 & 0.32 \\
\bottomrule
 &  & \multicolumn{3}{c}{11$\%$} & \multicolumn{3}{c}{12$\%$} & \multicolumn{3}{c}{13$\%$} & \multicolumn{3}{c}{14$\%$} & \multicolumn{3}{c}{15$\%$} \\
\cmidrule(l){3-5} \cmidrule(l){6-8} \cmidrule(l){9-11} \cmidrule(l){12-14} \cmidrule(l){15-17}  &  & M=1024 & M=512 & M=256 & M=1024 & M=512 & M=256 & M=1024 & M=512 & M=256 & M=1024 & M=512 & M=256 & M=1024 & M=512 & M=256 \\
\toprule
  &  Bias & -4.8 & -5.2 & -5.9 & -5.5 & -5.9 & -6.1 & -6 & -5 & -5.5 & -6.6 & -5.2 & -5.6 & -6.6 & -5.2 & -6.7 \\
  &  Variance & 41 & 43 & 39 & 37 & 37 & 40 & 40 & 44 & 46 & 40 & 40 & 42 & 38 & 43 & 38 \\
  &  MSE & 1e+02 & 1.1e+02 & 1.1e+02 & 1e+02 & 1.1e+02 & 1.2e+02 & 1.2e+02 & 1.1e+02 & 1.2e+02 & 1.2e+02 & 1.1e+02 & 1.1e+02 & 1.2e+02 & 1.1e+02 & 1.2e+02 \\
\multirow[c]{-4}{*}{$p_{1}$} &  Coverage & 1 & 1 & 1 & 1 & 1 & 1 & 1 & 1 & 1 & 1 & 1 & 1 & 1 & 1 & 1 \\
\cmidrule(l){2-17}   &  Bias & 0.95 & 1 & 1 & 0.89 & 1 & 1 & 0.99 & 1.1 & 1 & 1 & 1.1 & 0.99 & 0.99 & 1.1 & 1.1 \\
  &  Variance & 0.3 & 0.43 & 0.46 & 0.33 & 0.41 & 0.44 & 0.29 & 0.48 & 0.44 & 0.29 & 0.42 & 0.5 & 0.31 & 0.41 & 0.49 \\
  &  MSE & 1.5 & 1.9 & 2 & 1.5 & 1.9 & 1.9 & 1.6 & 2.1 & 1.9 & 1.7 & 2 & 2 & 1.6 & 2 & 2.1 \\
\multirow[c]{-4}{*}{$p_{2}$} &  Coverage & 0.17 & 0.2 & 0.21 & 0.21 & 0.19 & 0.23 & 0.12 & 0.23 & 0.22 & 0.14 & 0.19 & 0.26 & 0.16 & 0.18 & 0.23 \\
\cmidrule(l){2-17}   &  Bias & -0.34 & -0.51 & -0.52 & -0.32 & -0.52 & -0.49 & -0.4 & -0.47 & -0.49 & -0.42 & -0.44 & -0.47 & -0.36 & -0.42 & -0.49 \\
  &  Variance & 0.16 & 0.25 & 0.2 & 0.18 & 0.23 & 0.28 & 0.11 & 0.24 & 0.2 & 0.11 & 0.29 & 0.21 & 0.12 & 0.29 & 0.19 \\
  &  MSE & 0.44 & 0.76 & 0.67 & 0.47 & 0.74 & 0.81 & 0.39 & 0.7 & 0.63 & 0.39 & 0.78 & 0.64 & 0.36 & 0.76 & 0.61 \\
\multirow[c]{-4}{*}{$p_{3}$} &  Coverage & 1 & 1 & 1 & 1 & 1 & 1 & 1 & 1 & 1 & 1 & 1 & 1 & 1 & 1 & 1 \\
\cmidrule(l){2-17}   &  Bias & 0.24 & 0.54 & 0.56 & 0.22 & 0.61 & 0.55 & 0.3 & 0.49 & 0.51 & 0.18 & 0.52 & 0.46 & 0.13 & 0.53 & 0.44 \\
  &  Variance & 0.22 & 0.39 & 0.48 & 0.29 & 0.48 & 0.46 & 0.29 & 0.47 & 0.47 & 0.19 & 0.48 & 0.47 & 0.13 & 0.39 & 0.4 \\
  &  MSE & 0.5 & 1.1 & 1.3 & 0.62 & 1.3 & 1.2 & 0.67 & 1.2 & 1.2 & 0.41 & 1.2 & 1.2 & 0.28 & 1.1 & 1 \\
\multirow[c]{-4}{*}{$p_{4}$} &  Coverage & 0.5 & 0.26 & 0.31 & 0.6 & 0.27 & 0.28 & 0.56 & 0.34 & 0.31 & 0.64 & 0.27 & 0.3 & 0.73 & 0.33 & 0.29 \\
\cmidrule(l){2-17}   &  Bias & 0.43 & 0.68 & 0.64 & 0.33 & 0.66 & 0.73 & 0.34 & 0.66 & 0.68 & 0.32 & 0.61 & 0.59 & 0.27 & 0.71 & 0.64 \\
  &  Variance & 0.4 & 0.63 & 0.58 & 0.37 & 0.6 & 0.63 & 0.38 & 0.56 & 0.63 & 0.36 & 0.61 & 0.56 & 0.3 & 0.54 & 0.56 \\
  &  MSE & 0.98 & 1.7 & 1.6 & 0.86 & 1.6 & 1.8 & 0.87 & 1.6 & 1.7 & 0.83 & 1.6 & 1.5 & 0.67 & 1.6 & 1.5 \\
\multirow[c]{-4}{*}{$p_{5}$} &  Coverage & 0.46 & 0.36 & 0.45 & 0.57 & 0.42 & 0.39 & 0.56 & 0.44 & 0.44 & 0.61 & 0.41 & 0.44 & 0.66 & 0.42 & 0.45 \\
\bottomrule
\end{tabular}

	}
\end{table}

\begin{table}[H]
	\caption{Goodwin 2D (continued)} \label{tab:goodwin2DUQ2}	
	\resizebox{\textwidth}{!}{
	\begin{tabular}{lcccccccccccccccccccccccccccccccccccccccccccccccccccccccccccccccc}
 &  & \multicolumn{3}{c}{16$\%$} & \multicolumn{3}{c}{17$\%$} & \multicolumn{3}{c}{18$\%$} & \multicolumn{3}{c}{19$\%$} & \multicolumn{3}{c}{20$\%$} \\
\cmidrule(l){3-5} \cmidrule(l){6-8} \cmidrule(l){9-11} \cmidrule(l){12-14} \cmidrule(l){15-17}  &  & M=1024 & M=512 & M=256 & M=1024 & M=512 & M=256 & M=1024 & M=512 & M=256 & M=1024 & M=512 & M=256 & M=1024 & M=512 & M=256 \\
\toprule
  &  Bias & -6.7 & -6.4 & -6.5 & -7.3 & -4.6 & -6 & -7.5 & -5 & -6.9 & -6.9 & -5.1 & -5.9 & -7.3 & -5.3 & -7.2 \\
  &  Variance & 40 & 41 & 43 & 36 & 43 & 44 & 37 & 42 & 40 & 43 & 43 & 43 & 39 & 47 & 41 \\
  &  MSE & 1.3e+02 & 1.2e+02 & 1.3e+02 & 1.3e+02 & 1.1e+02 & 1.2e+02 & 1.3e+02 & 1.1e+02 & 1.3e+02 & 1.3e+02 & 1.1e+02 & 1.2e+02 & 1.3e+02 & 1.2e+02 & 1.3e+02 \\
\multirow[c]{-4}{*}{$p_{1}$} &  Coverage & 1 & 1 & 1 & 1 & 1 & 1 & 1 & 1 & 1 & 1 & 1 & 1 & 1 & 1 & 1 \\
\cmidrule(l){2-17}   &  Bias & 1 & 1.1 & 1 & 0.99 & 1.1 & 1.1 & 1 & 1.1 & 1 & 0.99 & 1.1 & 1.1 & 1 & 1 & 1 \\
  &  Variance & 0.35 & 0.44 & 0.47 & 0.34 & 0.45 & 0.44 & 0.41 & 0.46 & 0.51 & 0.39 & 0.42 & 0.47 & 0.4 & 0.48 & 0.42 \\
  &  MSE & 1.7 & 2.1 & 2 & 1.7 & 2 & 2.1 & 1.8 & 2 & 2 & 1.8 & 2 & 2.1 & 1.8 & 2 & 1.9 \\
\multirow[c]{-4}{*}{$p_{2}$} &  Coverage & 0.16 & 0.17 & 0.23 & 0.17 & 0.22 & 0.18 & 0.18 & 0.24 & 0.25 & 0.2 & 0.2 & 0.19 & 0.18 & 0.25 & 0.21 \\
\cmidrule(l){2-17}   &  Bias & -0.39 & -0.47 & -0.48 & -0.37 & -0.41 & -0.46 & -0.43 & -0.39 & -0.49 & -0.37 & -0.48 & -0.46 & -0.38 & -0.47 & -0.45 \\
  &  Variance & 0.15 & 0.23 & 0.22 & 0.14 & 0.25 & 0.19 & 0.13 & 0.24 & 0.18 & 0.15 & 0.2 & 0.19 & 0.15 & 0.17 & 0.17 \\
  &  MSE & 0.45 & 0.67 & 0.67 & 0.42 & 0.66 & 0.59 & 0.45 & 0.64 & 0.6 & 0.44 & 0.63 & 0.6 & 0.45 & 0.57 & 0.54 \\
\multirow[c]{-4}{*}{$p_{3}$} &  Coverage & 1 & 1 & 1 & 1 & 1 & 1 & 1 & 1 & 1 & 1 & 1 & 1 & 1 & 1 & 1 \\
\cmidrule(l){2-17}   &  Bias & 0.22 & 0.46 & 0.49 & 0.26 & 0.47 & 0.39 & 0.22 & 0.45 & 0.4 & 0.14 & 0.45 & 0.37 & 0.16 & 0.44 & 0.4 \\
  &  Variance & 0.27 & 0.45 & 0.53 & 0.27 & 0.44 & 0.44 & 0.25 & 0.46 & 0.42 & 0.13 & 0.41 & 0.42 & 0.2 & 0.35 & 0.47 \\
  &  MSE & 0.59 & 1.1 & 1.3 & 0.61 & 1.1 & 1 & 0.55 & 1.1 & 0.99 & 0.29 & 1 & 0.98 & 0.43 & 0.9 & 1.1 \\
\multirow[c]{-4}{*}{$p_{4}$} &  Coverage & 0.69 & 0.31 & 0.31 & 0.72 & 0.38 & 0.27 & 0.77 & 0.44 & 0.36 & 0.76 & 0.39 & 0.36 & 0.74 & 0.43 & 0.33 \\
\cmidrule(l){2-17}   &  Bias & 0.23 & 0.52 & 0.65 & 0.25 & 0.64 & 0.61 & 0.22 & 0.62 & 0.53 & 0.26 & 0.57 & 0.44 & 0.25 & 0.56 & 0.47 \\
  &  Variance & 0.22 & 0.57 & 0.69 & 0.32 & 0.45 & 0.67 & 0.27 & 0.58 & 0.58 & 0.36 & 0.51 & 0.56 & 0.32 & 0.54 & 0.59 \\
  &  MSE & 0.49 & 1.4 & 1.8 & 0.71 & 1.3 & 1.7 & 0.58 & 1.5 & 1.4 & 0.78 & 1.3 & 1.3 & 0.71 & 1.4 & 1.4 \\
\multirow[c]{-4}{*}{$p_{5}$} &  Coverage & 0.75 & 0.53 & 0.46 & 0.79 & 0.48 & 0.48 & 0.81 & 0.53 & 0.55 & 0.79 & 0.56 & 0.59 & 0.79 & 0.56 & 0.61 \\
\bottomrule
 &  & \multicolumn{3}{c}{25$\%$} \\
\cmidrule(l){3-5}  &  & M=1024 & M=512 & M=256 \\
\toprule
  &  Bias & -8.1 & -5.9 & -6.9 & \hspace{20pt} & \hspace{20pt} & \hspace{20pt} & \hspace{20pt} & \hspace{20pt} & \hspace{20pt} & \hspace{20pt} & \hspace{20pt} & \hspace{20pt} & \hspace{20pt} & \hspace{20pt} & \hspace{20pt} \\
  &  Variance & 31 & 54 & 48 & \hspace{20pt} & \hspace{20pt} & \hspace{20pt} & \hspace{20pt} & \hspace{20pt} & \hspace{20pt} & \hspace{20pt} & \hspace{20pt} & \hspace{20pt} & \hspace{20pt} & \hspace{20pt} & \hspace{20pt} \\
  &  MSE & 1.3e+02 & 1.4e+02 & 1.4e+02 & \hspace{20pt} & \hspace{20pt} & \hspace{20pt} & \hspace{20pt} & \hspace{20pt} & \hspace{20pt} & \hspace{20pt} & \hspace{20pt} & \hspace{20pt} & \hspace{20pt} & \hspace{20pt} & \hspace{20pt} \\
\multirow[c]{-4}{*}{$p_{1}$} &  Coverage & 1 & 1 & 1 & \hspace{20pt} & \hspace{20pt} & \hspace{20pt} & \hspace{20pt} & \hspace{20pt} & \hspace{20pt} & \hspace{20pt} & \hspace{20pt} & \hspace{20pt} & \hspace{20pt} & \hspace{20pt} & \hspace{20pt} \\
\cmidrule(l){2-17}   &  Bias & 1 & 1.1 & 1.1 & \hspace{20pt} & \hspace{20pt} & \hspace{20pt} & \hspace{20pt} & \hspace{20pt} & \hspace{20pt} & \hspace{20pt} & \hspace{20pt} & \hspace{20pt} & \hspace{20pt} & \hspace{20pt} & \hspace{20pt} \\
  &  Variance & 0.48 & 0.42 & 0.51 & \hspace{20pt} & \hspace{20pt} & \hspace{20pt} & \hspace{20pt} & \hspace{20pt} & \hspace{20pt} & \hspace{20pt} & \hspace{20pt} & \hspace{20pt} & \hspace{20pt} & \hspace{20pt} & \hspace{20pt} \\
  &  MSE & 2 & 2.1 & 2.3 & \hspace{20pt} & \hspace{20pt} & \hspace{20pt} & \hspace{20pt} & \hspace{20pt} & \hspace{20pt} & \hspace{20pt} & \hspace{20pt} & \hspace{20pt} & \hspace{20pt} & \hspace{20pt} & \hspace{20pt} \\
\multirow[c]{-4}{*}{$p_{2}$} &  Coverage & 0.31 & 0.29 & 0.33 & \hspace{20pt} & \hspace{20pt} & \hspace{20pt} & \hspace{20pt} & \hspace{20pt} & \hspace{20pt} & \hspace{20pt} & \hspace{20pt} & \hspace{20pt} & \hspace{20pt} & \hspace{20pt} & \hspace{20pt} \\
\cmidrule(l){2-17}   &  Bias & -0.37 & -0.44 & -0.48 & \hspace{20pt} & \hspace{20pt} & \hspace{20pt} & \hspace{20pt} & \hspace{20pt} & \hspace{20pt} & \hspace{20pt} & \hspace{20pt} & \hspace{20pt} & \hspace{20pt} & \hspace{20pt} & \hspace{20pt} \\
  &  Variance & 0.1 & 0.16 & 0.21 & \hspace{20pt} & \hspace{20pt} & \hspace{20pt} & \hspace{20pt} & \hspace{20pt} & \hspace{20pt} & \hspace{20pt} & \hspace{20pt} & \hspace{20pt} & \hspace{20pt} & \hspace{20pt} & \hspace{20pt} \\
  &  MSE & 0.34 & 0.51 & 0.65 & \hspace{20pt} & \hspace{20pt} & \hspace{20pt} & \hspace{20pt} & \hspace{20pt} & \hspace{20pt} & \hspace{20pt} & \hspace{20pt} & \hspace{20pt} & \hspace{20pt} & \hspace{20pt} & \hspace{20pt} \\
\multirow[c]{-4}{*}{$p_{3}$} &  Coverage & 1 & 1 & 1 & \hspace{20pt} & \hspace{20pt} & \hspace{20pt} & \hspace{20pt} & \hspace{20pt} & \hspace{20pt} & \hspace{20pt} & \hspace{20pt} & \hspace{20pt} & \hspace{20pt} & \hspace{20pt} & \hspace{20pt} \\
\cmidrule(l){2-17}   &  Bias & 0.12 & 0.33 & 0.41 & \hspace{20pt} & \hspace{20pt} & \hspace{20pt} & \hspace{20pt} & \hspace{20pt} & \hspace{20pt} & \hspace{20pt} & \hspace{20pt} & \hspace{20pt} & \hspace{20pt} & \hspace{20pt} & \hspace{20pt} \\
  &  Variance & 0.18 & 0.29 & 0.4 & \hspace{20pt} & \hspace{20pt} & \hspace{20pt} & \hspace{20pt} & \hspace{20pt} & \hspace{20pt} & \hspace{20pt} & \hspace{20pt} & \hspace{20pt} & \hspace{20pt} & \hspace{20pt} & \hspace{20pt} \\
  &  MSE & 0.36 & 0.69 & 0.97 & \hspace{20pt} & \hspace{20pt} & \hspace{20pt} & \hspace{20pt} & \hspace{20pt} & \hspace{20pt} & \hspace{20pt} & \hspace{20pt} & \hspace{20pt} & \hspace{20pt} & \hspace{20pt} & \hspace{20pt} \\
\multirow[c]{-4}{*}{$p_{4}$} &  Coverage & 0.84 & 0.58 & 0.41 & \hspace{20pt} & \hspace{20pt} & \hspace{20pt} & \hspace{20pt} & \hspace{20pt} & \hspace{20pt} & \hspace{20pt} & \hspace{20pt} & \hspace{20pt} & \hspace{20pt} & \hspace{20pt} & \hspace{20pt} \\
\cmidrule(l){2-17}   &  Bias & 0.12 & 0.41 & 0.58 & \hspace{20pt} & \hspace{20pt} & \hspace{20pt} & \hspace{20pt} & \hspace{20pt} & \hspace{20pt} & \hspace{20pt} & \hspace{20pt} & \hspace{20pt} & \hspace{20pt} & \hspace{20pt} & \hspace{20pt} \\
  &  Variance & 0.18 & 0.43 & 0.6 & \hspace{20pt} & \hspace{20pt} & \hspace{20pt} & \hspace{20pt} & \hspace{20pt} & \hspace{20pt} & \hspace{20pt} & \hspace{20pt} & \hspace{20pt} & \hspace{20pt} & \hspace{20pt} & \hspace{20pt} \\
  &  MSE & 0.37 & 1 & 1.5 & \hspace{20pt} & \hspace{20pt} & \hspace{20pt} & \hspace{20pt} & \hspace{20pt} & \hspace{20pt} & \hspace{20pt} & \hspace{20pt} & \hspace{20pt} & \hspace{20pt} & \hspace{20pt} & \hspace{20pt} \\
\multirow[c]{-4}{*}{$p_{5}$} &  Coverage & 0.87 & 0.71 & 0.59 & \hspace{20pt} & \hspace{20pt} & \hspace{20pt} & \hspace{20pt} & \hspace{20pt} & \hspace{20pt} & \hspace{20pt} & \hspace{20pt} & \hspace{20pt} & \hspace{20pt} & \hspace{20pt} & \hspace{20pt} \\
\bottomrule
\end{tabular}

	}
\end{table}

\begin{table}[H]
	\caption{Goodwin 3D} \label{tab:goodwin3DUQ1}	
	\resizebox{\textwidth}{!}{
	\begin{tabular}{lcccccccccccccccccccccccccccccccccccccccccccccccccccccccccccccccc}
 &  & \multicolumn{3}{c}{1$\%$} & \multicolumn{3}{c}{2$\%$} & \multicolumn{3}{c}{3$\%$} & \multicolumn{3}{c}{4$\%$} & \multicolumn{3}{c}{5$\%$} \\
\cmidrule(l){3-5} \cmidrule(l){6-8} \cmidrule(l){9-11} \cmidrule(l){12-14} \cmidrule(l){15-17}  &  & M=1024 & M=512 & M=256 & M=1024 & M=512 & M=256 & M=1024 & M=512 & M=256 & M=1024 & M=512 & M=256 & M=1024 & M=512 & M=256 \\
\toprule
  &  Bias & -0.15 & -0.048 & -0.0073 & -0.09 & 0.036 & 0.051 & -0.11 & -0.062 & -0.17 & -0.32 & -0.39 & -0.66 & -0.66 & -0.89 & -1.2 \\
  &  Variance & 0.072 & 0.098 & 0.25 & 0.19 & 0.45 & 0.79 & 0.46 & 0.92 & 1.2 & 0.64 & 1 & 1.3 & 0.66 & 0.9 & 1 \\
  &  MSE & 0.17 & 0.2 & 0.5 & 0.39 & 0.9 & 1.6 & 0.93 & 1.8 & 2.5 & 1.4 & 2.2 & 3 & 1.7 & 2.6 & 3.4 \\
\multirow[c]{-4}{*}{$p_{1}$} &  Coverage & 0.98 & 1 & 1 & 1 & 1 & 1 & 1 & 1 & 1 & 1 & 1 & 0.99 & 1 & 0.99 & 0.94 \\
\cmidrule(l){2-17}   &  Bias & 0.00025 & -0.00018 & -0.00022 & -0.00014 & -0.00018 & -0.00027 & -0.00015 & -0.00021 & -0.00046 & -0.00021 & -0.00034 & -0.00073 & -0.00031 & 0.00044 & -0.001 \\
  &  Variance & 1.9e-05 & 7.3e-08 & 1.4e-07 & 1.7e-07 & 3.1e-07 & 5.3e-07 & 3.9e-07 & 8.6e-07 & 1.3e-06 & 7.6e-07 & 9.7e-06 & 2.6e-06 & 1.3e-06 & 0.00011 & 4.5e-06 \\
  &  MSE & 3.8e-05 & 1.8e-07 & 3.3e-07 & 3.5e-07 & 6.6e-07 & 1.1e-06 & 8e-07 & 1.8e-06 & 2.8e-06 & 1.6e-06 & 1.9e-05 & 5.6e-06 & 2.8e-06 & 0.00022 & 1e-05 \\
\multirow[c]{-4}{*}{$p_{2}$} &  Coverage & 0.99 & 1 & 1 & 1 & 1 & 1 & 1 & 1 & 1 & 1 & 0.99 & 1 & 1 & 0.99 & 1 \\
\cmidrule(l){2-17}   &  Bias & -0.054 & -0.016 & -0.0017 & -0.027 & 0.014 & 0.017 & -0.027 & -0.018 & -0.065 & -0.093 & -0.13 & -0.24 & -0.21 & -0.3 & -0.41 \\
  &  Variance & 0.011 & 0.013 & 0.032 & 0.026 & 0.059 & 0.1 & 0.063 & 0.12 & 0.15 & 0.089 & 0.14 & 0.16 & 0.095 & 0.13 & 0.13 \\
  &  MSE & 0.024 & 0.026 & 0.065 & 0.052 & 0.12 & 0.2 & 0.13 & 0.24 & 0.3 & 0.19 & 0.31 & 0.37 & 0.23 & 0.35 & 0.43 \\
\multirow[c]{-4}{*}{$p_{3}$} &  Coverage & 0.98 & 1 & 1 & 1 & 1 & 1 & 1 & 1 & 1 & 1 & 1 & 0.99 & 1 & 0.99 & 0.97 \\
\cmidrule(l){2-17}   &  Bias & 0.018 & 0.031 & 0.037 & 0.051 & 0.029 & 0.053 & 0.075 & 0.045 & 0.12 & 0.14 & 0.12 & 0.24 & 0.24 & 0.25 & 0.39 \\
  &  Variance & 0.24 & 0.003 & 0.0063 & 0.007 & 0.013 & 0.023 & 0.017 & 0.039 & 0.054 & 0.033 & 0.29 & 0.1 & 0.058 & 0.35 & 0.18 \\
  &  MSE & 0.48 & 0.0069 & 0.014 & 0.017 & 0.026 & 0.048 & 0.039 & 0.08 & 0.12 & 0.085 & 0.59 & 0.27 & 0.17 & 0.76 & 0.51 \\
\multirow[c]{-4}{*}{$p_{4}$} &  Coverage & 0.98 & 1 & 1 & 1 & 1 & 1 & 1 & 1 & 1 & 1 & 0.99 & 1 & 1 & 0.99 & 1 \\
\cmidrule(l){2-17}   &  Bias & -0.00016 & -0.00031 & -0.00032 & -0.00025 & -0.00033 & -0.0003 & -0.00024 & -0.00023 & -0.00027 & -0.00023 & -0.00027 & -0.00023 & -0.00022 & -0.00037 & -0.00018 \\
  &  Variance & 2.7e-06 & 6.5e-08 & 1.3e-07 & 1.1e-07 & 2.6e-07 & 4.9e-07 & 2.6e-07 & 1.2e-06 & 1.1e-06 & 4.6e-07 & 1.4e-06 & 2e-06 & 7.1e-07 & 2.6e-06 & 3.1e-06 \\
  &  MSE & 5.4e-06 & 2.3e-07 & 3.6e-07 & 2.9e-07 & 6.3e-07 & 1.1e-06 & 5.7e-07 & 2.4e-06 & 2.3e-06 & 9.6e-07 & 2.9e-06 & 4e-06 & 1.5e-06 & 5.3e-06 & 6.3e-06 \\
\multirow[c]{-4}{*}{$p_{5}$} &  Coverage & 0.98 & 1 & 0.99 & 1 & 1 & 1 & 1 & 0.99 & 1 & 1 & 1 & 1 & 1 & 1 & 1 \\
\cmidrule(l){2-17}   &  Bias & -0.00014 & -0.00023 & -0.00023 & -0.00018 & -0.00024 & -0.00022 & -0.00017 & -0.0002 & -0.00021 & -0.00014 & -0.00024 & -0.00019 & -0.00012 & -2e-05 & -0.00017 \\
  &  Variance & 7.4e-07 & 2.1e-08 & 4e-08 & 3.6e-08 & 8.7e-08 & 1.5e-07 & 8.3e-08 & 3.8e-07 & 3.5e-07 & 1.5e-07 & 4.7e-07 & 6.3e-07 & 2.3e-07 & 5.5e-06 & 9.9e-07 \\
  &  MSE & 1.5e-06 & 9.6e-08 & 1.3e-07 & 1.1e-07 & 2.3e-07 & 3.6e-07 & 1.9e-07 & 8e-07 & 7.4e-07 & 3.2e-07 & 9.9e-07 & 1.3e-06 & 4.8e-07 & 1.1e-05 & 2e-06 \\
\multirow[c]{-4}{*}{$p_{6}$} &  Coverage & 0.98 & 1 & 1 & 1 & 1 & 1 & 1 & 1 & 1 & 1 & 1 & 1 & 1 & 0.99 & 1 \\
\cmidrule(l){2-17}   &  Bias & -9.4e-05 & 2e-05 & 2.7e-06 & -0.00013 & 8.2e-05 & 0.00013 & -0.00015 & 0.00031 & 0.00032 & -0.0001 & 0.0006 & 0.00061 & 1.9e-06 & 0.00087 & 0.00091 \\
  &  Variance & 5.3e-07 & 1.4e-07 & 3.3e-07 & 3.8e-07 & 6.4e-07 & 1.3e-06 & 9.1e-07 & 1.5e-06 & 3.2e-06 & 1.7e-06 & 3.2e-06 & 5.9e-06 & 2.8e-06 & 4.8e-06 & 9.1e-06 \\
  &  MSE & 1.1e-06 & 2.9e-07 & 6.6e-07 & 7.8e-07 & 1.3e-06 & 2.7e-06 & 1.8e-06 & 3.1e-06 & 6.5e-06 & 3.4e-06 & 6.7e-06 & 1.2e-05 & 5.5e-06 & 1e-05 & 1.9e-05 \\
\multirow[c]{-4}{*}{$p_{7}$} &  Coverage & 0.99 & 1 & 1 & 1 & 1 & 1 & 1 & 1 & 1 & 1 & 1 & 1 & 1 & 1 & 1 \\
\cmidrule(l){2-17}   &  Bias & -0.00017 & 9.9e-06 & -1.6e-06 & -9.1e-05 & 5.9e-05 & 1e-04 & -9.1e-05 & 0.00023 & 0.00025 & -2.7e-05 & 0.00043 & 0.00048 & 8.8e-05 & 0.00067 & 0.00072 \\
  &  Variance & 8.4e-07 & 9.3e-08 & 2e-07 & 2.3e-07 & 4.2e-07 & 8.2e-07 & 5.5e-07 & 9.3e-07 & 2e-06 & 1e-06 & 1.9e-06 & 3.6e-06 & 1.7e-06 & 3.2e-06 & 5.6e-06 \\
  &  MSE & 1.7e-06 & 1.9e-07 & 4.1e-07 & 4.7e-07 & 8.5e-07 & 1.7e-06 & 1.1e-06 & 1.9e-06 & 4e-06 & 2.1e-06 & 4e-06 & 7.5e-06 & 3.4e-06 & 6.8e-06 & 1.2e-05 \\
\multirow[c]{-4}{*}{$p_{8}$} &  Coverage & 0.98 & 1 & 1 & 1 & 1 & 1 & 1 & 1 & 1 & 1 & 1 & 1 & 1 & 1 & 1 \\
\bottomrule
 &  & \multicolumn{3}{c}{6$\%$} & \multicolumn{3}{c}{7$\%$} & \multicolumn{3}{c}{8$\%$} & \multicolumn{3}{c}{9$\%$} & \multicolumn{3}{c}{10$\%$} \\
\cmidrule(l){3-5} \cmidrule(l){6-8} \cmidrule(l){9-11} \cmidrule(l){12-14} \cmidrule(l){15-17}  &  & M=1024 & M=512 & M=256 & M=1024 & M=512 & M=256 & M=1024 & M=512 & M=256 & M=1024 & M=512 & M=256 & M=1024 & M=512 & M=256 \\
\toprule
  &  Bias & -1 & -1.2 & -1.6 & -1.4 & -1.5 & -1.7 & -1.6 & -1.7 & -1.9 & -1.7 & -1.8 & -2 & -1.8 & -1.8 & -2.1 \\
  &  Variance & 0.46 & 0.69 & 0.55 & 0.31 & 0.43 & 0.43 & 0.22 & 0.35 & 0.27 & 0.21 & 0.3 & 0.2 & 0.15 & 0.24 & 0.2 \\
  &  MSE & 2 & 2.9 & 3.6 & 2.4 & 3.1 & 3.9 & 2.9 & 3.5 & 4.2 & 3.3 & 3.7 & 4.4 & 3.6 & 3.9 & 4.6 \\
\multirow[c]{-4}{*}{$p_{1}$} &  Coverage & 1 & 1 & 0.86 & 1 & 0.92 & 0.77 & 0.99 & 0.93 & 0.64 & 0.94 & 0.89 & 0.61 & 0.9 & 0.85 & 0.52 \\
\cmidrule(l){2-17}   &  Bias & -0.00045 & -0.00086 & -0.0013 & -0.00063 & -0.0011 & -0.0015 & -0.00074 & -0.0014 & -0.0018 & -0.0009 & -0.0016 & -0.0019 & -0.0011 & -0.0018 & -0.002 \\
  &  Variance & 2.2e-06 & 5.6e-06 & 7.3e-06 & 3.5e-06 & 8e-06 & 1.2e-05 & 5.2e-06 & 1.3e-05 & 1.5e-05 & 7.7e-06 & 1.8e-05 & 2e-05 & 1.1e-05 & 2.3e-05 & 2.7e-05 \\
  &  MSE & 4.5e-06 & 1.2e-05 & 1.6e-05 & 7.4e-06 & 1.7e-05 & 2.6e-05 & 1.1e-05 & 2.8e-05 & 3.4e-05 & 1.6e-05 & 3.8e-05 & 4.4e-05 & 2.2e-05 & 5e-05 & 5.8e-05 \\
\multirow[c]{-4}{*}{$p_{2}$} &  Coverage & 1 & 1 & 1 & 1 & 1 & 1 & 1 & 1 & 1 & 1 & 1 & 1 & 1 & 1 & 1 \\
\cmidrule(l){2-17}   &  Bias & -0.34 & -0.41 & -0.54 & -0.45 & -0.5 & -0.58 & -0.52 & -0.53 & -0.63 & -0.56 & -0.55 & -0.65 & -0.6 & -0.56 & -0.65 \\
  &  Variance & 0.07 & 0.11 & 0.08 & 0.05 & 0.076 & 0.076 & 0.04 & 0.07 & 0.057 & 0.042 & 0.065 & 0.039 & 0.035 & 0.066 & 0.045 \\
  &  MSE & 0.26 & 0.39 & 0.45 & 0.3 & 0.4 & 0.49 & 0.35 & 0.42 & 0.51 & 0.4 & 0.43 & 0.5 & 0.43 & 0.44 & 0.52 \\
\multirow[c]{-4}{*}{$p_{3}$} &  Coverage & 1 & 1 & 0.94 & 1 & 0.97 & 0.89 & 1 & 1 & 0.88 & 0.99 & 1 & 0.86 & 0.98 & 1 & 0.89 \\
\cmidrule(l){2-17}   &  Bias & 0.37 & 0.28 & 0.51 & 0.53 & 0.4 & 0.54 & 0.68 & 0.28 & 0.55 & 0.81 & 0.21 & 0.5 & 0.96 & 0.11 & 0.41 \\
  &  Variance & 0.094 & 0.18 & 0.27 & 0.15 & 0.32 & 0.41 & 0.22 & 0.32 & 0.46 & 0.34 & 0.4 & 0.51 & 0.49 & 0.49 & 0.62 \\
  &  MSE & 0.33 & 0.44 & 0.79 & 0.57 & 0.79 & 1.1 & 0.9 & 0.71 & 1.2 & 1.3 & 0.84 & 1.3 & 1.9 & 0.98 & 1.4 \\
\multirow[c]{-4}{*}{$p_{4}$} &  Coverage & 1 & 1 & 1 & 1 & 1 & 1 & 1 & 1 & 1 & 1 & 1 & 1 & 1 & 1 & 1 \\
\cmidrule(l){2-17}   &  Bias & -0.00021 & -0.00022 & -0.00014 & -0.00022 & -0.00025 & -8.1e-05 & -0.00023 & -0.00014 & -4.1e-06 & -0.00023 & -5.5e-05 & 0.00011 & -0.00023 & 5.2e-05 & 0.00025 \\
  &  Variance & 1e-06 & 2.4e-06 & 4.5e-06 & 1.4e-06 & 3.2e-06 & 6.2e-06 & 1.8e-06 & 4.2e-06 & 8e-06 & 2.3e-06 & 5.4e-06 & 1e-05 & 2.8e-06 & 6.6e-06 & 1.2e-05 \\
  &  MSE & 2.1e-06 & 4.8e-06 & 9e-06 & 2.8e-06 & 6.5e-06 & 1.2e-05 & 3.6e-06 & 8.5e-06 & 1.6e-05 & 4.6e-06 & 1.1e-05 & 2e-05 & 5.6e-06 & 1.3e-05 & 2.5e-05 \\
\multirow[c]{-4}{*}{$p_{5}$} &  Coverage & 1 & 1 & 1 & 1 & 1 & 1 & 1 & 1 & 1 & 1 & 1 & 1 & 1 & 1 & 1 \\
\cmidrule(l){2-17}   &  Bias & -9.6e-05 & -0.00023 & -0.00016 & -7.2e-05 & -0.00022 & -0.00013 & -4.4e-05 & -0.00021 & -0.0001 & -6.9e-06 & -0.00019 & -5.7e-05 & 3.5e-05 & -0.00015 & 7.1e-06 \\
  &  Variance & 3.3e-07 & 8e-07 & 1.4e-06 & 4.5e-07 & 1.1e-06 & 2e-06 & 5.9e-07 & 1.4e-06 & 2.6e-06 & 7.4e-07 & 1.8e-06 & 3.2e-06 & 9.1e-07 & 2.3e-06 & 4e-06 \\
  &  MSE & 6.8e-07 & 1.7e-06 & 2.9e-06 & 9.1e-07 & 2.2e-06 & 4e-06 & 1.2e-06 & 2.9e-06 & 5.1e-06 & 1.5e-06 & 3.7e-06 & 6.5e-06 & 1.8e-06 & 4.5e-06 & 8e-06 \\
\multirow[c]{-4}{*}{$p_{6}$} &  Coverage & 1 & 1 & 1 & 1 & 1 & 1 & 1 & 1 & 1 & 1 & 1 & 1 & 1 & 1 & 1 \\
\cmidrule(l){2-17}   &  Bias & 0.00014 & 0.0012 & 0.0012 & 0.00028 & 0.0014 & 0.0014 & 0.0004 & 0.0018 & 0.0017 & 0.00053 & 0.0021 & 0.002 & 0.00069 & 0.0025 & 0.0023 \\
  &  Variance & 4.1e-06 & 7.1e-06 & 1.3e-05 & 5.6e-06 & 9.8e-06 & 1.8e-05 & 7.5e-06 & 1.3e-05 & 2.2e-05 & 9.5e-06 & 1.6e-05 & 2.7e-05 & 1.2e-05 & 2e-05 & 3.2e-05 \\
  &  MSE & 8.2e-06 & 1.6e-05 & 2.7e-05 & 1.1e-05 & 2.1e-05 & 3.7e-05 & 1.5e-05 & 2.9e-05 & 4.7e-05 & 1.9e-05 & 3.7e-05 & 5.9e-05 & 2.4e-05 & 4.6e-05 & 7e-05 \\
\multirow[c]{-4}{*}{$p_{7}$} &  Coverage & 1 & 1 & 1 & 1 & 1 & 1 & 1 & 1 & 1 & 1 & 1 & 1 & 1 & 1 & 1 \\
\cmidrule(l){2-17}   &  Bias & 0.00023 & 0.00092 & 0.00093 & 0.00039 & 0.0011 & 0.0011 & 0.00053 & 0.0014 & 0.0013 & 0.0007 & 0.0016 & 0.0016 & 0.00088 & 0.0019 & 0.0018 \\
  &  Variance & 2.5e-06 & 4.6e-06 & 8e-06 & 3.4e-06 & 6.4e-06 & 1.1e-05 & 4.6e-06 & 8.5e-06 & 1.4e-05 & 5.8e-06 & 1.1e-05 & 1.7e-05 & 7.2e-06 & 1.3e-05 & 2e-05 \\
  &  MSE & 5e-06 & 1e-05 & 1.7e-05 & 7e-06 & 1.4e-05 & 2.3e-05 & 9.4e-06 & 1.9e-05 & 2.9e-05 & 1.2e-05 & 2.4e-05 & 3.6e-05 & 1.5e-05 & 3e-05 & 4.3e-05 \\
\multirow[c]{-4}{*}{$p_{8}$} &  Coverage & 1 & 1 & 1 & 1 & 1 & 1 & 1 & 1 & 1 & 1 & 1 & 1 & 1 & 1 & 1 \\
\bottomrule
\end{tabular}

	}
\end{table}
\begin{table}[H]
	\caption{Goodwin 3D (continued)} \label{tab:goodwin3DUQ2}	
	\resizebox{\textwidth}{!}{
	\begin{tabular}{lcccccccccccccccccccccccccccccccccccccccccccccccccccccccccccccccc}
 &  & \multicolumn{3}{c}{11$\%$} & \multicolumn{3}{c}{12$\%$} & \multicolumn{3}{c}{13$\%$} & \multicolumn{3}{c}{14$\%$} & \multicolumn{3}{c}{15$\%$} \\
\cmidrule(l){3-5} \cmidrule(l){6-8} \cmidrule(l){9-11} \cmidrule(l){12-14} \cmidrule(l){15-17}  &  & M=1024 & M=512 & M=256 & M=1024 & M=512 & M=256 & M=1024 & M=512 & M=256 & M=1024 & M=512 & M=256 & M=1024 & M=512 & M=256 \\
\toprule
  &  Bias & -1.8 & -1.9 & -2.1 & -1.9 & -1.9 & -2.1 & -1.9 & -1.9 & -2.1 & -1.9 & -1.9 & -2.1 & -1.8 & -2 & -2.1 \\
  &  Variance & 0.19 & 0.23 & 0.2 & 0.16 & 0.22 & 0.21 & 0.17 & 0.21 & 0.23 & 0.2 & 0.24 & 0.23 & 0.46 & 0.2 & 0.23 \\
  &  MSE & 3.7 & 4.1 & 4.7 & 4 & 4.2 & 4.8 & 4.1 & 4.2 & 4.9 & 4.2 & 4.2 & 4.9 & 4.3 & 4.3 & 4.9 \\
\multirow[c]{-4}{*}{$p_{1}$} &  Coverage & 0.87 & 0.8 & 0.49 & 0.76 & 0.77 & 0.41 & 0.68 & 0.74 & 0.4 & 0.66 & 0.73 & 0.39 & 0.69 & 0.71 & 0.4 \\
\cmidrule(l){2-17}   &  Bias & 0.0013 & -0.001 & -0.002 & -0.00011 & -0.0011 & -0.0019 & -0.00092 & -0.0012 & -0.0018 & 0.00042 & -0.0024 & -0.0017 & 0.0029 & -0.0023 & -0.0016 \\
  &  Variance & 0.00023 & 0.00014 & 3.6e-05 & 0.00013 & 0.00015 & 4.5e-05 & 2.6e-05 & 0.00016 & 5.3e-05 & 0.00014 & 5.7e-05 & 6.2e-05 & 0.00029 & 6.6e-05 & 7.2e-05 \\
  &  MSE & 0.00046 & 0.00028 & 7.5e-05 & 0.00025 & 0.0003 & 9.3e-05 & 5.2e-05 & 0.00032 & 0.00011 & 0.00028 & 0.00012 & 0.00013 & 0.00058 & 0.00014 & 0.00015 \\
\multirow[c]{-4}{*}{$p_{2}$} &  Coverage & 0.98 & 0.99 & 1 & 0.99 & 0.99 & 1 & 1 & 0.99 & 1 & 0.99 & 1 & 1 & 0.98 & 1 & 1 \\
\cmidrule(l){2-17}   &  Bias & -0.58 & -0.56 & -0.64 & -0.62 & -0.54 & -0.61 & -0.6 & -0.51 & -0.58 & -0.58 & -0.46 & -0.56 & -0.51 & -0.45 & -0.53 \\
  &  Variance & 0.057 & 0.073 & 0.056 & 0.054 & 0.08 & 0.071 & 0.069 & 0.091 & 0.091 & 0.094 & 0.11 & 0.11 & 0.15 & 0.11 & 0.13 \\
  &  MSE & 0.46 & 0.46 & 0.52 & 0.49 & 0.45 & 0.52 & 0.5 & 0.44 & 0.52 & 0.52 & 0.44 & 0.53 & 0.56 & 0.42 & 0.53 \\
\multirow[c]{-4}{*}{$p_{3}$} &  Coverage & 0.94 & 0.98 & 0.91 & 0.93 & 0.98 & 0.92 & 0.92 & 0.98 & 0.93 & 0.9 & 0.99 & 0.95 & 0.93 & 0.99 & 0.96 \\
\cmidrule(l){2-17}   &  Bias & 0.96 & 0.069 & 0.27 & 1.2 & -0.067 & 0.11 & 1.2 & -0.22 & -0.069 & 1.3 & -0.43 & -0.24 & 1.1 & -0.57 & -0.43 \\
  &  Variance & 1.3 & 0.86 & 0.74 & 1.2 & 1 & 0.86 & 1.5 & 1.2 & 0.97 & 2.1 & 1 & 1.1 & 2.5 & 1.1 & 1.2 \\
  &  MSE & 3.4 & 1.7 & 1.6 & 3.9 & 2 & 1.7 & 4.6 & 2.4 & 1.9 & 5.9 & 2.3 & 2.2 & 6.3 & 2.6 & 2.5 \\
\multirow[c]{-4}{*}{$p_{4}$} &  Coverage & 0.99 & 0.99 & 1 & 0.99 & 0.99 & 1 & 1 & 0.99 & 1 & 0.99 & 1 & 1 & 0.99 & 1 & 1 \\
\cmidrule(l){2-17}   &  Bias & -0.00028 & 4.6e-05 & 0.00042 & -0.00038 & 0.00019 & 0.00062 & -0.00023 & 0.00035 & 0.00085 & -0.0004 & 0.00065 & 0.0011 & -0.00039 & 0.00085 & 0.0014 \\
  &  Variance & 5.5e-06 & 8.2e-06 & 1.5e-05 & 5.4e-06 & 9.6e-06 & 1.8e-05 & 4.6e-06 & 1.1e-05 & 2.1e-05 & 6.8e-06 & 1.3e-05 & 2.4e-05 & 7.5e-06 & 1.5e-05 & 2.8e-05 \\
  &  MSE & 1.1e-05 & 1.6e-05 & 3e-05 & 1.1e-05 & 1.9e-05 & 3.6e-05 & 9.3e-06 & 2.2e-05 & 4.3e-05 & 1.4e-05 & 2.7e-05 & 4.9e-05 & 1.5e-05 & 3.1e-05 & 5.7e-05 \\
\multirow[c]{-4}{*}{$p_{5}$} &  Coverage & 1 & 1 & 1 & 1 & 1 & 1 & 1 & 1 & 1 & 1 & 1 & 1 & 1 & 1 & 1 \\
\cmidrule(l){2-17}   &  Bias & 0.00034 & 0.00012 & 8.5e-05 & 0.00033 & 0.00017 & 0.00018 & 0.00019 & 0.00024 & 0.00029 & 0.00043 & 8e-05 & 0.0004 & 0.0005 & 0.00016 & 0.00053 \\
  &  Variance & 5.4e-06 & 8.8e-06 & 4.8e-06 & 5.4e-06 & 9.5e-06 & 5.7e-06 & 1.5e-06 & 1e-05 & 6.6e-06 & 5.9e-06 & 4.5e-06 & 7.7e-06 & 6.1e-06 & 5.1e-06 & 8.7e-06 \\
  &  MSE & 1.1e-05 & 1.8e-05 & 9.6e-06 & 1.1e-05 & 1.9e-05 & 1.1e-05 & 3.1e-06 & 2e-05 & 1.3e-05 & 1.2e-05 & 9e-06 & 1.5e-05 & 1.2e-05 & 1e-05 & 1.8e-05 \\
\multirow[c]{-4}{*}{$p_{6}$} &  Coverage & 0.99 & 1 & 1 & 1 & 1 & 1 & 1 & 1 & 1 & 1 & 1 & 1 & 1 & 1 & 1 \\
\cmidrule(l){2-17}   &  Bias & 0.00059 & 0.0028 & 0.0026 & 0.00095 & 0.0031 & 0.003 & 0.0011 & 0.0034 & 0.0034 & 0.0013 & 0.0037 & 0.0038 & 0.00085 & 0.004 & 0.0043 \\
  &  Variance & 2.6e-05 & 2.4e-05 & 3.8e-05 & 1.8e-05 & 2.9e-05 & 4.5e-05 & 2.1e-05 & 3.5e-05 & 5.3e-05 & 2.4e-05 & 4e-05 & 6.2e-05 & 4.2e-05 & 4.7e-05 & 7.1e-05 \\
  &  MSE & 5.2e-05 & 5.7e-05 & 8.3e-05 & 3.6e-05 & 6.8e-05 & 1e-04 & 4.2e-05 & 8.1e-05 & 0.00012 & 5e-05 & 9.4e-05 & 0.00014 & 8.4e-05 & 0.00011 & 0.00016 \\
\multirow[c]{-4}{*}{$p_{7}$} &  Coverage & 1 & 1 & 1 & 1 & 1 & 1 & 1 & 1 & 1 & 1 & 1 & 1 & 1 & 1 & 1 \\
\cmidrule(l){2-17}   &  Bias & 0.00079 & 0.0021 & 0.002 & 0.0012 & 0.0024 & 0.0023 & 0.0014 & 0.0026 & 0.0026 & 0.0016 & 0.0028 & 0.0029 & 0.0014 & 0.003 & 0.0032 \\
  &  Variance & 1.9e-05 & 1.6e-05 & 2.4e-05 & 1.1e-05 & 1.9e-05 & 2.8e-05 & 1.3e-05 & 2.3e-05 & 3.3e-05 & 1.5e-05 & 2.7e-05 & 3.8e-05 & 2.7e-05 & 3.1e-05 & 4.3e-05 \\
  &  MSE & 3.8e-05 & 3.7e-05 & 5.1e-05 & 2.3e-05 & 4.4e-05 & 6.1e-05 & 2.8e-05 & 5.2e-05 & 7.2e-05 & 3.3e-05 & 6.2e-05 & 8.4e-05 & 5.7e-05 & 7.1e-05 & 9.7e-05 \\
\multirow[c]{-4}{*}{$p_{8}$} &  Coverage & 1 & 1 & 1 & 1 & 1 & 1 & 1 & 1 & 1 & 1 & 1 & 1 & 1 & 1 & 1 \\
\bottomrule
 &  & \multicolumn{3}{c}{16$\%$} & \multicolumn{3}{c}{17$\%$} & \multicolumn{3}{c}{18$\%$} & \multicolumn{3}{c}{19$\%$} & \multicolumn{3}{c}{20$\%$} \\
\cmidrule(l){3-5} \cmidrule(l){6-8} \cmidrule(l){9-11} \cmidrule(l){12-14} \cmidrule(l){15-17}  &  & M=1024 & M=512 & M=256 & M=1024 & M=512 & M=256 & M=1024 & M=512 & M=256 & M=1024 & M=512 & M=256 & M=1024 & M=512 & M=256 \\
\toprule
  &  Bias & -1.9 & -2 & -2.1 & -1.9 & -2 & -2.1 & -1.9 & -2 & -2.1 & -1.9 & -2 & -2.1 & -1.9 & -2 & -2.1 \\
  &  Variance & 0.25 & 0.2 & 0.23 & 0.26 & 0.19 & 0.23 & 0.27 & 0.21 & 0.22 & 0.27 & 0.21 & 0.21 & 0.26 & 0.2 & 0.16 \\
  &  MSE & 4.2 & 4.3 & 5 & 4.2 & 4.3 & 5 & 4.1 & 4.4 & 4.9 & 4.2 & 4.4 & 4.9 & 4.2 & 4.4 & 4.9 \\
\multirow[c]{-4}{*}{$p_{1}$} &  Coverage & 0.62 & 0.68 & 0.43 & 0.6 & 0.68 & 0.45 & 0.61 & 0.64 & 0.44 & 0.59 & 0.62 & 0.43 & 0.56 & 0.6 & 0.42 \\
\cmidrule(l){2-17}   &  Bias & 0.0014 & -0.002 & -0.0014 & 0.0011 & -0.002 & -0.0013 & 0.0017 & -0.0018 & -0.0012 & 0.0033 & -0.0015 & -0.0012 & 0.004 & -0.0011 & -0.00063 \\
  &  Variance & 0.00015 & 7.6e-05 & 8.5e-05 & 5.3e-05 & 8.7e-05 & 9.9e-05 & 5.9e-05 & 9.8e-05 & 0.00012 & 0.00017 & 0.00011 & 0.00014 & 0.00018 & 0.00012 & 0.00019 \\
  &  MSE & 0.0003 & 0.00016 & 0.00017 & 0.00011 & 0.00018 & 0.0002 & 0.00012 & 0.0002 & 0.00024 & 0.00035 & 0.00022 & 0.00028 & 0.00037 & 0.00025 & 0.00039 \\
\multirow[c]{-4}{*}{$p_{2}$} &  Coverage & 0.99 & 1 & 1 & 1 & 1 & 1 & 1 & 1 & 1 & 0.99 & 1 & 1 & 0.99 & 1 & 1 \\
\cmidrule(l){2-17}   &  Bias & -0.5 & -0.42 & -0.49 & -0.44 & -0.38 & -0.46 & -0.38 & -0.34 & -0.42 & -0.34 & -0.3 & -0.38 & -0.3 & -0.28 & -0.36 \\
  &  Variance & 0.18 & 0.13 & 0.15 & 0.22 & 0.15 & 0.17 & 0.27 & 0.19 & 0.18 & 0.32 & 0.22 & 0.2 & 0.36 & 0.24 & 0.19 \\
  &  MSE & 0.6 & 0.43 & 0.54 & 0.63 & 0.44 & 0.55 & 0.69 & 0.49 & 0.55 & 0.76 & 0.53 & 0.55 & 0.82 & 0.55 & 0.52 \\
\multirow[c]{-4}{*}{$p_{3}$} &  Coverage & 0.9 & 1 & 0.97 & 0.91 & 1 & 0.99 & 0.92 & 1 & 0.99 & 0.92 & 1 & 0.99 & 0.91 & 1 & 0.98 \\
\cmidrule(l){2-17}   &  Bias & 1.2 & -0.74 & -0.61 & 1.1 & -0.89 & -0.77 & 1 & -1 & -0.95 & 0.94 & -1.2 & -1.1 & 0.92 & -1.2 & -1.2 \\
  &  Variance & 2.9 & 1.3 & 1.3 & 3.1 & 1.3 & 1.3 & 3.6 & 1.3 & 1.3 & 4.1 & 1.4 & 1.3 & 4.5 & 1.5 & 1.6 \\
  &  MSE & 7.4 & 3.1 & 2.9 & 7.5 & 3.3 & 3.2 & 8.3 & 3.7 & 3.5 & 9 & 4.2 & 4 & 9.8 & 4.5 & 4.7 \\
\multirow[c]{-4}{*}{$p_{4}$} &  Coverage & 0.99 & 1 & 1 & 1 & 1 & 1 & 1 & 1 & 1 & 0.99 & 1 & 1 & 0.99 & 1 & 1 \\
\cmidrule(l){2-17}   &  Bias & -0.00037 & 0.0011 & 0.0017 & -0.00022 & 0.0013 & 0.002 & -0.00026 & 0.0016 & 0.0023 & -0.00034 & 0.0018 & 0.0027 & -0.00034 & 0.0021 & 0.0031 \\
  &  Variance & 8.3e-06 & 1.7e-05 & 3.1e-05 & 7.7e-06 & 1.9e-05 & 3.5e-05 & 8.7e-06 & 2.1e-05 & 3.9e-05 & 1.1e-05 & 2.4e-05 & 4.4e-05 & 1.2e-05 & 2.6e-05 & 4.9e-05 \\
  &  MSE & 1.7e-05 & 3.5e-05 & 6.5e-05 & 1.6e-05 & 4e-05 & 7.4e-05 & 1.8e-05 & 4.5e-05 & 8.4e-05 & 2.3e-05 & 5.1e-05 & 9.5e-05 & 2.5e-05 & 5.7e-05 & 0.00011 \\
\multirow[c]{-4}{*}{$p_{5}$} &  Coverage & 1 & 1 & 1 & 1 & 1 & 1 & 1 & 1 & 1 & 1 & 1 & 1 & 1 & 1 & 1 \\
\cmidrule(l){2-17}   &  Bias & 0.00058 & 0.00027 & 0.00067 & 0.00047 & 0.00033 & 0.00082 & 0.00052 & 0.00043 & 0.00097 & 0.00083 & 0.00054 & 0.0011 & 0.00091 & 0.00064 & 0.0013 \\
  &  Variance & 6.3e-06 & 5.7e-06 & 9.9e-06 & 2.6e-06 & 6.5e-06 & 1.1e-05 & 2.9e-06 & 7.3e-06 & 1.2e-05 & 6.8e-06 & 8.1e-06 & 1.4e-05 & 7.1e-06 & 9e-06 & 1.5e-05 \\
  &  MSE & 1.3e-05 & 1.2e-05 & 2e-05 & 5.4e-06 & 1.3e-05 & 2.3e-05 & 6.1e-06 & 1.5e-05 & 2.6e-05 & 1.4e-05 & 1.7e-05 & 2.9e-05 & 1.5e-05 & 1.8e-05 & 3.2e-05 \\
\multirow[c]{-4}{*}{$p_{6}$} &  Coverage & 1 & 1 & 1 & 1 & 1 & 1 & 1 & 1 & 1 & 1 & 1 & 1 & 1 & 1 & 1 \\
\cmidrule(l){2-17}   &  Bias & 0.0015 & 0.0042 & 0.0048 & 0.0017 & 0.0048 & 0.0053 & 0.0019 & 0.0051 & 0.0059 & 0.002 & 0.0054 & 0.0066 & 0.0022 & 0.0057 & 0.007 \\
  &  Variance & 3.1e-05 & 5.2e-05 & 8.2e-05 & 3.4e-05 & 6.1e-05 & 9.4e-05 & 3.8e-05 & 6.8e-05 & 0.00011 & 4.1e-05 & 7.8e-05 & 0.00012 & 4.5e-05 & 8.7e-05 & 0.00014 \\
  &  MSE & 6.5e-05 & 0.00012 & 0.00019 & 7.1e-05 & 0.00014 & 0.00022 & 8e-05 & 0.00016 & 0.00025 & 8.7e-05 & 0.00018 & 0.00028 & 9.5e-05 & 0.00021 & 0.00034 \\
\multirow[c]{-4}{*}{$p_{7}$} &  Coverage & 1 & 1 & 1 & 1 & 1 & 1 & 1 & 1 & 1 & 1 & 1 & 1 & 1 & 1 & 1 \\
\cmidrule(l){2-17}   &  Bias & 0.002 & 0.003 & 0.0035 & 0.0023 & 0.0034 & 0.0039 & 0.0026 & 0.0036 & 0.0042 & 0.0027 & 0.0037 & 0.0047 & 0.0029 & 0.0038 & 0.0049 \\
  &  Variance & 2e-05 & 3.5e-05 & 5e-05 & 2.2e-05 & 4.1e-05 & 5.7e-05 & 2.4e-05 & 4.5e-05 & 6.5e-05 & 2.6e-05 & 5.1e-05 & 7.3e-05 & 2.9e-05 & 5.7e-05 & 8.9e-05 \\
  &  MSE & 4.3e-05 & 7.8e-05 & 0.00011 & 4.8e-05 & 9.3e-05 & 0.00013 & 5.4e-05 & 0.0001 & 0.00015 & 6e-05 & 0.00012 & 0.00017 & 6.6e-05 & 0.00013 & 0.0002 \\
\multirow[c]{-4}{*}{$p_{8}$} &  Coverage & 1 & 1 & 1 & 1 & 1 & 1 & 1 & 1 & 1 & 1 & 1 & 1 & 1 & 1 & 1 \\
\bottomrule
\end{tabular}

	}
\end{table}
\begin{table}[H]
	\caption{Goodwin 3D (continued)} \label{tab:goodwin3DUQ3}	
	\resizebox{\textwidth}{!}{
	\begin{tabular}{lcccccccccccccccccccccccccccccccccccccccccccccccccccccccccccccccc}
 &  & \multicolumn{3}{c}{25$\%$} \\
\cmidrule(l){3-5}  &  & M=1024 & M=512 & M=256 \\
\toprule
  &  Bias & -1.9 & -2.1 & -2.1 & \hspace{20pt} & \hspace{20pt} & \hspace{20pt} & \hspace{20pt} & \hspace{20pt} & \hspace{20pt} & \hspace{20pt} & \hspace{20pt} & \hspace{20pt} & \hspace{20pt} & \hspace{20pt} & \hspace{20pt} \\
  &  Variance & 0.35 & 0.17 & 0.17 & \hspace{20pt} & \hspace{20pt} & \hspace{20pt} & \hspace{20pt} & \hspace{20pt} & \hspace{20pt} & \hspace{20pt} & \hspace{20pt} & \hspace{20pt} & \hspace{20pt} & \hspace{20pt} & \hspace{20pt} \\
  &  MSE & 4.3 & 4.6 & 4.8 & \hspace{20pt} & \hspace{20pt} & \hspace{20pt} & \hspace{20pt} & \hspace{20pt} & \hspace{20pt} & \hspace{20pt} & \hspace{20pt} & \hspace{20pt} & \hspace{20pt} & \hspace{20pt} & \hspace{20pt} \\
\multirow[c]{-4}{*}{$p_{1}$} &  Coverage & 0.57 & 0.55 & 0.44 & \hspace{20pt} & \hspace{20pt} & \hspace{20pt} & \hspace{20pt} & \hspace{20pt} & \hspace{20pt} & \hspace{20pt} & \hspace{20pt} & \hspace{20pt} & \hspace{20pt} & \hspace{20pt} & \hspace{20pt} \\
\cmidrule(l){2-17}   &  Bias & 0.0067 & 0.00089 & 0.00025 & \hspace{20pt} & \hspace{20pt} & \hspace{20pt} & \hspace{20pt} & \hspace{20pt} & \hspace{20pt} & \hspace{20pt} & \hspace{20pt} & \hspace{20pt} & \hspace{20pt} & \hspace{20pt} & \hspace{20pt} \\
  &  Variance & 0.00025 & 0.0003 & 0.00046 & \hspace{20pt} & \hspace{20pt} & \hspace{20pt} & \hspace{20pt} & \hspace{20pt} & \hspace{20pt} & \hspace{20pt} & \hspace{20pt} & \hspace{20pt} & \hspace{20pt} & \hspace{20pt} & \hspace{20pt} \\
  &  MSE & 0.00055 & 0.0006 & 0.00091 & \hspace{20pt} & \hspace{20pt} & \hspace{20pt} & \hspace{20pt} & \hspace{20pt} & \hspace{20pt} & \hspace{20pt} & \hspace{20pt} & \hspace{20pt} & \hspace{20pt} & \hspace{20pt} & \hspace{20pt} \\
\multirow[c]{-4}{*}{$p_{2}$} &  Coverage & 1 & 0.99 & 0.99 & \hspace{20pt} & \hspace{20pt} & \hspace{20pt} & \hspace{20pt} & \hspace{20pt} & \hspace{20pt} & \hspace{20pt} & \hspace{20pt} & \hspace{20pt} & \hspace{20pt} & \hspace{20pt} & \hspace{20pt} \\
\cmidrule(l){2-17}   &  Bias & -0.055 & -0.12 & -0.087 & \hspace{20pt} & \hspace{20pt} & \hspace{20pt} & \hspace{20pt} & \hspace{20pt} & \hspace{20pt} & \hspace{20pt} & \hspace{20pt} & \hspace{20pt} & \hspace{20pt} & \hspace{20pt} & \hspace{20pt} \\
  &  Variance & 0.47 & 0.37 & 0.34 & \hspace{20pt} & \hspace{20pt} & \hspace{20pt} & \hspace{20pt} & \hspace{20pt} & \hspace{20pt} & \hspace{20pt} & \hspace{20pt} & \hspace{20pt} & \hspace{20pt} & \hspace{20pt} & \hspace{20pt} \\
  &  MSE & 0.93 & 0.76 & 0.69 & \hspace{20pt} & \hspace{20pt} & \hspace{20pt} & \hspace{20pt} & \hspace{20pt} & \hspace{20pt} & \hspace{20pt} & \hspace{20pt} & \hspace{20pt} & \hspace{20pt} & \hspace{20pt} & \hspace{20pt} \\
\multirow[c]{-4}{*}{$p_{3}$} &  Coverage & 0.97 & 0.98 & 0.97 & \hspace{20pt} & \hspace{20pt} & \hspace{20pt} & \hspace{20pt} & \hspace{20pt} & \hspace{20pt} & \hspace{20pt} & \hspace{20pt} & \hspace{20pt} & \hspace{20pt} & \hspace{20pt} & \hspace{20pt} \\
\cmidrule(l){2-17}   &  Bias & 0.55 & -1.6 & -2 & \hspace{20pt} & \hspace{20pt} & \hspace{20pt} & \hspace{20pt} & \hspace{20pt} & \hspace{20pt} & \hspace{20pt} & \hspace{20pt} & \hspace{20pt} & \hspace{20pt} & \hspace{20pt} & \hspace{20pt} \\
  &  Variance & 5.3 & 3.1 & 2.5 & \hspace{20pt} & \hspace{20pt} & \hspace{20pt} & \hspace{20pt} & \hspace{20pt} & \hspace{20pt} & \hspace{20pt} & \hspace{20pt} & \hspace{20pt} & \hspace{20pt} & \hspace{20pt} & \hspace{20pt} \\
  &  MSE & 11 & 8.6 & 8.9 & \hspace{20pt} & \hspace{20pt} & \hspace{20pt} & \hspace{20pt} & \hspace{20pt} & \hspace{20pt} & \hspace{20pt} & \hspace{20pt} & \hspace{20pt} & \hspace{20pt} & \hspace{20pt} & \hspace{20pt} \\
\multirow[c]{-4}{*}{$p_{4}$} &  Coverage & 1 & 0.99 & 0.99 & \hspace{20pt} & \hspace{20pt} & \hspace{20pt} & \hspace{20pt} & \hspace{20pt} & \hspace{20pt} & \hspace{20pt} & \hspace{20pt} & \hspace{20pt} & \hspace{20pt} & \hspace{20pt} & \hspace{20pt} \\
\cmidrule(l){2-17}   &  Bias & -0.00035 & 0.0037 & 0.0056 & \hspace{20pt} & \hspace{20pt} & \hspace{20pt} & \hspace{20pt} & \hspace{20pt} & \hspace{20pt} & \hspace{20pt} & \hspace{20pt} & \hspace{20pt} & \hspace{20pt} & \hspace{20pt} & \hspace{20pt} \\
  &  Variance & 1.6e-05 & 4e-05 & 8.6e-05 & \hspace{20pt} & \hspace{20pt} & \hspace{20pt} & \hspace{20pt} & \hspace{20pt} & \hspace{20pt} & \hspace{20pt} & \hspace{20pt} & \hspace{20pt} & \hspace{20pt} & \hspace{20pt} & \hspace{20pt} \\
  &  MSE & 3.3e-05 & 9.3e-05 & 0.0002 & \hspace{20pt} & \hspace{20pt} & \hspace{20pt} & \hspace{20pt} & \hspace{20pt} & \hspace{20pt} & \hspace{20pt} & \hspace{20pt} & \hspace{20pt} & \hspace{20pt} & \hspace{20pt} & \hspace{20pt} \\
\multirow[c]{-4}{*}{$p_{5}$} &  Coverage & 1 & 1 & 1 & \hspace{20pt} & \hspace{20pt} & \hspace{20pt} & \hspace{20pt} & \hspace{20pt} & \hspace{20pt} & \hspace{20pt} & \hspace{20pt} & \hspace{20pt} & \hspace{20pt} & \hspace{20pt} & \hspace{20pt} \\
\cmidrule(l){2-17}   &  Bias & 0.0012 & 0.0015 & 0.0025 & \hspace{20pt} & \hspace{20pt} & \hspace{20pt} & \hspace{20pt} & \hspace{20pt} & \hspace{20pt} & \hspace{20pt} & \hspace{20pt} & \hspace{20pt} & \hspace{20pt} & \hspace{20pt} & \hspace{20pt} \\
  &  Variance & 5.6e-06 & 2.2e-05 & 2.3e-05 & \hspace{20pt} & \hspace{20pt} & \hspace{20pt} & \hspace{20pt} & \hspace{20pt} & \hspace{20pt} & \hspace{20pt} & \hspace{20pt} & \hspace{20pt} & \hspace{20pt} & \hspace{20pt} & \hspace{20pt} \\
  &  MSE & 1.3e-05 & 4.7e-05 & 5.2e-05 & \hspace{20pt} & \hspace{20pt} & \hspace{20pt} & \hspace{20pt} & \hspace{20pt} & \hspace{20pt} & \hspace{20pt} & \hspace{20pt} & \hspace{20pt} & \hspace{20pt} & \hspace{20pt} & \hspace{20pt} \\
\multirow[c]{-4}{*}{$p_{6}$} &  Coverage & 1 & 1 & 1 & \hspace{20pt} & \hspace{20pt} & \hspace{20pt} & \hspace{20pt} & \hspace{20pt} & \hspace{20pt} & \hspace{20pt} & \hspace{20pt} & \hspace{20pt} & \hspace{20pt} & \hspace{20pt} & \hspace{20pt} \\
\cmidrule(l){2-17}   &  Bias & 0.0022 & 0.0077 & 0.01 & \hspace{20pt} & \hspace{20pt} & \hspace{20pt} & \hspace{20pt} & \hspace{20pt} & \hspace{20pt} & \hspace{20pt} & \hspace{20pt} & \hspace{20pt} & \hspace{20pt} & \hspace{20pt} & \hspace{20pt} \\
  &  Variance & 6.9e-05 & 0.00014 & 0.00021 & \hspace{20pt} & \hspace{20pt} & \hspace{20pt} & \hspace{20pt} & \hspace{20pt} & \hspace{20pt} & \hspace{20pt} & \hspace{20pt} & \hspace{20pt} & \hspace{20pt} & \hspace{20pt} & \hspace{20pt} \\
  &  MSE & 0.00014 & 0.00033 & 0.00053 & \hspace{20pt} & \hspace{20pt} & \hspace{20pt} & \hspace{20pt} & \hspace{20pt} & \hspace{20pt} & \hspace{20pt} & \hspace{20pt} & \hspace{20pt} & \hspace{20pt} & \hspace{20pt} & \hspace{20pt} \\
\multirow[c]{-4}{*}{$p_{7}$} &  Coverage & 1 & 1 & 1 & \hspace{20pt} & \hspace{20pt} & \hspace{20pt} & \hspace{20pt} & \hspace{20pt} & \hspace{20pt} & \hspace{20pt} & \hspace{20pt} & \hspace{20pt} & \hspace{20pt} & \hspace{20pt} & \hspace{20pt} \\
\cmidrule(l){2-17}   &  Bias & 0.0036 & 0.0048 & 0.0065 & \hspace{20pt} & \hspace{20pt} & \hspace{20pt} & \hspace{20pt} & \hspace{20pt} & \hspace{20pt} & \hspace{20pt} & \hspace{20pt} & \hspace{20pt} & \hspace{20pt} & \hspace{20pt} & \hspace{20pt} \\
  &  Variance & 4.6e-05 & 8.8e-05 & 0.00013 & \hspace{20pt} & \hspace{20pt} & \hspace{20pt} & \hspace{20pt} & \hspace{20pt} & \hspace{20pt} & \hspace{20pt} & \hspace{20pt} & \hspace{20pt} & \hspace{20pt} & \hspace{20pt} & \hspace{20pt} \\
  &  MSE & 0.0001 & 0.0002 & 0.0003 & \hspace{20pt} & \hspace{20pt} & \hspace{20pt} & \hspace{20pt} & \hspace{20pt} & \hspace{20pt} & \hspace{20pt} & \hspace{20pt} & \hspace{20pt} & \hspace{20pt} & \hspace{20pt} & \hspace{20pt} \\
\multirow[c]{-4}{*}{$p_{8}$} &  Coverage & 1 & 1 & 1 & \hspace{20pt} & \hspace{20pt} & \hspace{20pt} & \hspace{20pt} & \hspace{20pt} & \hspace{20pt} & \hspace{20pt} & \hspace{20pt} & \hspace{20pt} & \hspace{20pt} & \hspace{20pt} & \hspace{20pt} \\
\bottomrule
\end{tabular}

	}
\end{table}
\begin{table}[H]
	\caption{Goodwin 3D (continued)} \label{tab:goodwin3DUQ4}	
	\resizebox{\textwidth}{!}{
	\begin{tabular}{lcccccccccccccccccccccccccccccccccccccccccccccccccccccccccccccccc}
 &  & \multicolumn{3}{c}{16$\%$} & \multicolumn{3}{c}{17$\%$} & \multicolumn{3}{c}{18$\%$} & \multicolumn{3}{c}{19$\%$} & \multicolumn{3}{c}{20$\%$} \\
\cmidrule(l){3-5} \cmidrule(l){6-8} \cmidrule(l){9-11} \cmidrule(l){12-14} \cmidrule(l){15-17}  &  & M=1024 & M=512 & M=256 & M=1024 & M=512 & M=256 & M=1024 & M=512 & M=256 & M=1024 & M=512 & M=256 & M=1024 & M=512 & M=256 \\
\toprule
  &  Bias & -1.9 & -2 & -2.1 & -1.9 & -2 & -2.1 & -1.9 & -2 & -2.1 & -1.9 & -2 & -2.1 & -1.9 & -2 & -2.1 \\
\cmidrule(l){2-17}   &  Variance & 0.25 & 0.2 & 0.23 & 0.26 & 0.19 & 0.23 & 0.27 & 0.21 & 0.22 & 0.27 & 0.21 & 0.21 & 0.26 & 0.2 & 0.16 \\
\cmidrule(l){2-17}   &  MSE & 4.2 & 4.3 & 5 & 4.2 & 4.3 & 5 & 4.1 & 4.4 & 4.9 & 4.2 & 4.4 & 4.9 & 4.2 & 4.4 & 4.9 \\
\cmidrule(l){2-17} \multirow[c]{-4}{*}{$p_{1}$} &  Coverage & 0.62 & 0.68 & 0.43 & 0.6 & 0.68 & 0.45 & 0.61 & 0.64 & 0.44 & 0.59 & 0.62 & 0.43 & 0.56 & 0.6 & 0.42 \\
\cmidrule(l){2-17}   &  Bias & 0.0014 & -0.002 & -0.0014 & 0.0011 & -0.002 & -0.0013 & 0.0017 & -0.0018 & -0.0012 & 0.0033 & -0.0015 & -0.0012 & 0.004 & -0.0011 & -0.00063 \\
\cmidrule(l){2-17}   &  Variance & 0.00015 & 7.6e-05 & 8.5e-05 & 5.3e-05 & 8.7e-05 & 9.9e-05 & 5.9e-05 & 9.8e-05 & 0.00012 & 0.00017 & 0.00011 & 0.00014 & 0.00018 & 0.00012 & 0.00019 \\
\cmidrule(l){2-17}   &  MSE & 0.0003 & 0.00016 & 0.00017 & 0.00011 & 0.00018 & 0.0002 & 0.00012 & 0.0002 & 0.00024 & 0.00035 & 0.00022 & 0.00028 & 0.00037 & 0.00025 & 0.00039 \\
\cmidrule(l){2-17} \multirow[c]{-4}{*}{$p_{2}$} &  Coverage & 0.99 & 1 & 1 & 1 & 1 & 1 & 1 & 1 & 1 & 0.99 & 1 & 1 & 0.99 & 1 & 1 \\
\cmidrule(l){2-17}   &  Bias & -0.5 & -0.42 & -0.49 & -0.44 & -0.38 & -0.46 & -0.38 & -0.34 & -0.42 & -0.34 & -0.3 & -0.38 & -0.3 & -0.28 & -0.36 \\
\cmidrule(l){2-17}   &  Variance & 0.18 & 0.13 & 0.15 & 0.22 & 0.15 & 0.17 & 0.27 & 0.19 & 0.18 & 0.32 & 0.22 & 0.2 & 0.36 & 0.24 & 0.19 \\
\cmidrule(l){2-17}   &  MSE & 0.6 & 0.43 & 0.54 & 0.63 & 0.44 & 0.55 & 0.69 & 0.49 & 0.55 & 0.76 & 0.53 & 0.55 & 0.82 & 0.55 & 0.52 \\
\cmidrule(l){2-17} \multirow[c]{-4}{*}{$p_{3}$} &  Coverage & 0.9 & 1 & 0.97 & 0.91 & 1 & 0.99 & 0.92 & 1 & 0.99 & 0.92 & 1 & 0.99 & 0.91 & 1 & 0.98 \\
\cmidrule(l){2-17}   &  Bias & 1.2 & -0.74 & -0.61 & 1.1 & -0.89 & -0.77 & 1 & -1 & -0.95 & 0.94 & -1.2 & -1.1 & 0.92 & -1.2 & -1.2 \\
\cmidrule(l){2-17}   &  Variance & 2.9 & 1.3 & 1.3 & 3.1 & 1.3 & 1.3 & 3.6 & 1.3 & 1.3 & 4.1 & 1.4 & 1.3 & 4.5 & 1.5 & 1.6 \\
\cmidrule(l){2-17}   &  MSE & 7.4 & 3.1 & 2.9 & 7.5 & 3.3 & 3.2 & 8.3 & 3.7 & 3.5 & 9 & 4.2 & 4 & 9.8 & 4.5 & 4.7 \\
\cmidrule(l){2-17} \multirow[c]{-4}{*}{$p_{4}$} &  Coverage & 0.99 & 1 & 1 & 1 & 1 & 1 & 1 & 1 & 1 & 0.99 & 1 & 1 & 0.99 & 1 & 1 \\
\cmidrule(l){2-17}   &  Bias & -0.00037 & 0.0011 & 0.0017 & -0.00022 & 0.0013 & 0.002 & -0.00026 & 0.0016 & 0.0023 & -0.00034 & 0.0018 & 0.0027 & -0.00034 & 0.0021 & 0.0031 \\
\cmidrule(l){2-17}   &  Variance & 8.3e-06 & 1.7e-05 & 3.1e-05 & 7.7e-06 & 1.9e-05 & 3.5e-05 & 8.7e-06 & 2.1e-05 & 3.9e-05 & 1.1e-05 & 2.4e-05 & 4.4e-05 & 1.2e-05 & 2.6e-05 & 4.9e-05 \\
\cmidrule(l){2-17}   &  MSE & 1.7e-05 & 3.5e-05 & 6.5e-05 & 1.6e-05 & 4e-05 & 7.4e-05 & 1.8e-05 & 4.5e-05 & 8.4e-05 & 2.3e-05 & 5.1e-05 & 9.5e-05 & 2.5e-05 & 5.7e-05 & 0.00011 \\
\cmidrule(l){2-17} \multirow[c]{-4}{*}{$p_{5}$} &  Coverage & 1 & 1 & 1 & 1 & 1 & 1 & 1 & 1 & 1 & 1 & 1 & 1 & 1 & 1 & 1 \\
\cmidrule(l){2-17}   &  Bias & 0.00058 & 0.00027 & 0.00067 & 0.00047 & 0.00033 & 0.00082 & 0.00052 & 0.00043 & 0.00097 & 0.00083 & 0.00054 & 0.0011 & 0.00091 & 0.00064 & 0.0013 \\
\cmidrule(l){2-17}   &  Variance & 6.3e-06 & 5.7e-06 & 9.9e-06 & 2.6e-06 & 6.5e-06 & 1.1e-05 & 2.9e-06 & 7.3e-06 & 1.2e-05 & 6.8e-06 & 8.1e-06 & 1.4e-05 & 7.1e-06 & 9e-06 & 1.5e-05 \\
\cmidrule(l){2-17}   &  MSE & 1.3e-05 & 1.2e-05 & 2e-05 & 5.4e-06 & 1.3e-05 & 2.3e-05 & 6.1e-06 & 1.5e-05 & 2.6e-05 & 1.4e-05 & 1.7e-05 & 2.9e-05 & 1.5e-05 & 1.8e-05 & 3.2e-05 \\
\cmidrule(l){2-17} \multirow[c]{-4}{*}{$p_{6}$} &  Coverage & 1 & 1 & 1 & 1 & 1 & 1 & 1 & 1 & 1 & 1 & 1 & 1 & 1 & 1 & 1 \\
\cmidrule(l){2-17}   &  Bias & 0.0015 & 0.0042 & 0.0048 & 0.0017 & 0.0048 & 0.0053 & 0.0019 & 0.0051 & 0.0059 & 0.002 & 0.0054 & 0.0066 & 0.0022 & 0.0057 & 0.007 \\
\cmidrule(l){2-17}   &  Variance & 3.1e-05 & 5.2e-05 & 8.2e-05 & 3.4e-05 & 6.1e-05 & 9.4e-05 & 3.8e-05 & 6.8e-05 & 0.00011 & 4.1e-05 & 7.8e-05 & 0.00012 & 4.5e-05 & 8.7e-05 & 0.00014 \\
\cmidrule(l){2-17}   &  MSE & 6.5e-05 & 0.00012 & 0.00019 & 7.1e-05 & 0.00014 & 0.00022 & 8e-05 & 0.00016 & 0.00025 & 8.7e-05 & 0.00018 & 0.00028 & 9.5e-05 & 0.00021 & 0.00034 \\
\cmidrule(l){2-17} \multirow[c]{-4}{*}{$p_{7}$} &  Coverage & 1 & 1 & 1 & 1 & 1 & 1 & 1 & 1 & 1 & 1 & 1 & 1 & 1 & 1 & 1 \\
\cmidrule(l){2-17}   &  Bias & 0.002 & 0.003 & 0.0035 & 0.0023 & 0.0034 & 0.0039 & 0.0026 & 0.0036 & 0.0042 & 0.0027 & 0.0037 & 0.0047 & 0.0029 & 0.0038 & 0.0049 \\
\cmidrule(l){2-17}   &  Variance & 2e-05 & 3.5e-05 & 5e-05 & 2.2e-05 & 4.1e-05 & 5.7e-05 & 2.4e-05 & 4.5e-05 & 6.5e-05 & 2.6e-05 & 5.1e-05 & 7.3e-05 & 2.9e-05 & 5.7e-05 & 8.9e-05 \\
\cmidrule(l){2-17}   &  MSE & 4.3e-05 & 7.8e-05 & 0.00011 & 4.8e-05 & 9.3e-05 & 0.00013 & 5.4e-05 & 0.0001 & 0.00015 & 6e-05 & 0.00012 & 0.00017 & 6.6e-05 & 0.00013 & 0.0002 \\
\cmidrule(l){2-17} \multirow[c]{-4}{*}{$p_{8}$} &  Coverage & 1 & 1 & 1 & 1 & 1 & 1 & 1 & 1 & 1 & 1 & 1 & 1 & 1 & 1 & 1 \\
\bottomrule
\end{tabular}

	}
\end{table}

\begin{table}[H]
	\caption{Goodwin 3D (continued)} \label{tab:goodwin3DUQ5}	
	\resizebox{\textwidth}{!}{
	\input{fig/goodwin_UQ_table_5.tex}
	}
\end{table}

\begin{table}[H]
	\caption{SIR-TDI} \label{tab:sirUQ1}	
	\resizebox{\textwidth}{!}{
	\begin{tabular}{lccccccccccccccccccccccccccccccccccccccccccccccccccccccccccccc}
 &  & \multicolumn{3}{c}{1$\%$} & \multicolumn{3}{c}{2$\%$} & \multicolumn{3}{c}{3$\%$} & \multicolumn{3}{c}{4$\%$} & \multicolumn{3}{c}{5$\%$} \\
\cmidrule(l){3-5} \cmidrule(l){6-8} \cmidrule(l){9-11} \cmidrule(l){12-14} \cmidrule(l){15-17}  &  & M=1024 & M=512 & M=256 & M=1024 & M=512 & M=256 & M=1024 & M=512 & M=256 & M=1024 & M=512 & M=256 & M=1024 & M=512 & M=256 \\
\toprule
  &  Bias & 0.00071 & 0.00023 & 9.2e-05 & 0.00067 & 0.00029 & 0.00033 & 0.00057 & 0.00035 & 0.00029 & 0.00041 & 0.00034 & 0.00023 & 0.00022 & 0.00029 & 0.0001 \\
  &  Variance & 8.6e-08 & 2e-07 & 7.9e-07 & 2.8e-07 & 7.5e-07 & 2.3e-06 & 6.2e-07 & 1.7e-06 & 5.2e-06 & 1.1e-06 & 2.9e-06 & 9.3e-06 & 1.7e-06 & 4.6e-06 & 1.5e-05 \\
  &  MSE & 6.8e-07 & 4.6e-07 & 1.6e-06 & 1e-06 & 1.6e-06 & 4.7e-06 & 1.6e-06 & 3.6e-06 & 1.1e-05 & 2.3e-06 & 5.9e-06 & 1.9e-05 & 3.4e-06 & 9.3e-06 & 3e-05 \\
\multirow[c]{-4}{*}{$p_{1}$} &  Coverage & 1 & 1 & 1 & 1 & 1 & 1 & 1 & 1 & 1 & 1 & 1 & 1 & 1 & 1 & 1 \\
\cmidrule(l){2-17}   &  Bias & 0.00063 & 0.00056 & 0.00043 & 0.00084 & 0.00059 & 0.00011 & 0.0013 & 0.00074 & 0.0014 & 0.0019 & 0.0013 & 0.0031 & 0.0029 & 0.0023 & 0.0053 \\
  &  Variance & 4.4e-06 & 8.6e-06 & 2.4e-05 & 1.7e-05 & 3.3e-05 & 8.2e-05 & 3.7e-05 & 7.2e-05 & 0.00019 & 6.5e-05 & 0.00012 & 0.00033 & 0.0001 & 0.00019 & 0.00052 \\
  &  MSE & 9.3e-06 & 1.7e-05 & 4.9e-05 & 3.4e-05 & 6.6e-05 & 0.00016 & 7.6e-05 & 0.00015 & 0.00037 & 0.00013 & 0.00025 & 0.00067 & 0.00021 & 0.00039 & 0.0011 \\
\multirow[c]{-4}{*}{$p_{2}$} &  Coverage & 1 & 1 & 1 & 1 & 1 & 1 & 1 & 1 & 1 & 1 & 1 & 1 & 1 & 1 & 1 \\
\cmidrule(l){2-17}   &  Bias & 0.00052 & 0.00017 & 9e-05 & 0.0005 & 0.0002 & 0.00025 & 0.00045 & 0.00023 & 0.00028 & 0.00038 & 0.00022 & 0.00031 & 0.0003 & 0.00019 & 0.00031 \\
  &  Variance & 5.5e-08 & 9.4e-08 & 3.4e-07 & 1.9e-07 & 3.7e-07 & 1.1e-06 & 4.3e-07 & 8.4e-07 & 2.5e-06 & 7.6e-07 & 1.4e-06 & 4.5e-06 & 1.2e-06 & 2.2e-06 & 7.3e-06 \\
  &  MSE & 3.8e-07 & 2.2e-07 & 6.9e-07 & 6.4e-07 & 7.7e-07 & 2.3e-06 & 1.1e-06 & 1.7e-06 & 5.1e-06 & 1.7e-06 & 2.9e-06 & 9.1e-06 & 2.4e-06 & 4.5e-06 & 1.5e-05 \\
\multirow[c]{-4}{*}{$p_{3}$} &  Coverage & 1 & 1 & 1 & 1 & 1 & 1 & 1 & 1 & 1 & 1 & 1 & 1 & 1 & 1 & 1 \\
\cmidrule(l){2-17}   &  Bias & -6.4e-05 & -3.3e-05 & -3.2e-05 & -0.00013 & -3.7e-05 & 1.3e-05 & -0.00024 & -1.4e-05 & -6.6e-05 & -0.0004 & -1.3e-05 & -0.00016 & -0.00061 & -2.1e-05 & -0.0003 \\
  &  Variance & 4.1e-08 & 8.3e-08 & 2e-07 & 1.6e-07 & 3.2e-07 & 7.3e-07 & 3.5e-07 & 7e-07 & 1.6e-06 & 6.1e-07 & 1.2e-06 & 2.9e-06 & 9.4e-07 & 1.9e-06 & 4.5e-06 \\
  &  MSE & 8.5e-08 & 1.7e-07 & 4e-07 & 3.3e-07 & 6.4e-07 & 1.5e-06 & 7.5e-07 & 1.4e-06 & 3.3e-06 & 1.4e-06 & 2.4e-06 & 5.9e-06 & 2.3e-06 & 3.8e-06 & 9.1e-06 \\
\multirow[c]{-4}{*}{$p_{4}$} &  Coverage & 1 & 1 & 1 & 1 & 1 & 1 & 1 & 1 & 1 & 1 & 1 & 1 & 1 & 1 & 1 \\
\cmidrule(l){2-17}   &  Bias & -3.6e-06 & -2e-06 & -4.1e-07 & -1.6e-06 & -2.1e-06 & -1.5e-06 & 1.8e-06 & -3.4e-06 & -1.5e-06 & 6.7e-06 & -4.7e-06 & -1.4e-06 & 1.3e-05 & -6.7e-06 & -1e-06 \\
  &  Variance & 1.1e-11 & 2.6e-11 & 5.7e-11 & 4.4e-11 & 1e-10 & 2.2e-10 & 1e-10 & 2.3e-10 & 5e-10 & 1.8e-10 & 4.1e-10 & 8.9e-10 & 2.8e-10 & 6.3e-10 & 1.4e-09 \\
  &  MSE & 3.5e-11 & 5.5e-11 & 1.1e-10 & 9.1e-11 & 2.1e-10 & 4.5e-10 & 2e-10 & 4.8e-10 & 1e-09 & 4e-10 & 8.4e-10 & 1.8e-09 & 7.3e-10 & 1.3e-09 & 2.8e-09 \\
\multirow[c]{-4}{*}{$p_{5}$} &  Coverage & 1 & 1 & 1 & 1 & 1 & 1 & 1 & 1 & 1 & 1 & 1 & 1 & 1 & 1 & 1 \\
\bottomrule
 &  & \multicolumn{3}{c}{6$\%$} & \multicolumn{3}{c}{7$\%$} & \multicolumn{3}{c}{8$\%$} & \multicolumn{3}{c}{9$\%$} & \multicolumn{3}{c}{10$\%$} \\
\cmidrule(l){3-5} \cmidrule(l){6-8} \cmidrule(l){9-11} \cmidrule(l){12-14} \cmidrule(l){15-17}  &  & M=1024 & M=512 & M=256 & M=1024 & M=512 & M=256 & M=1024 & M=512 & M=256 & M=1024 & M=512 & M=256 & M=1024 & M=512 & M=256 \\
\toprule
  &  Bias & -2.2e-05 & 0.00019 & -0.00046 & -0.00035 & -0.00012 & -0.00027 & -0.00068 & -1.3e-05 & -0.00058 & -0.00096 & 4.5e-05 & -0.00087 & -0.0013 & -8.7e-05 & -0.0013 \\
  &  Variance & 2.4e-06 & 6.6e-06 & 2.3e-05 & 3.3e-06 & 9.6e-06 & 2.9e-05 & 4.5e-06 & 1.2e-05 & 3.9e-05 & 5.5e-06 & 1.4e-05 & 4.7e-05 & 6.6e-06 & 1.8e-05 & 5.8e-05 \\
  &  MSE & 4.9e-06 & 1.3e-05 & 4.6e-05 & 6.8e-06 & 1.9e-05 & 5.7e-05 & 9.4e-06 & 2.3e-05 & 7.9e-05 & 1.2e-05 & 2.9e-05 & 9.6e-05 & 1.5e-05 & 3.5e-05 & 0.00012 \\
\multirow[c]{-4}{*}{$p_{1}$} &  Coverage & 1 & 1 & 1 & 1 & 1 & 1 & 1 & 1 & 1 & 1 & 1 & 1 & 1 & 1 & 1 \\
\cmidrule(l){2-17}   &  Bias & 0.0041 & 0.0034 & 0.0099 & 0.0057 & 0.0057 & 0.011 & 0.0074 & 0.0061 & 0.015 & 0.0093 & 0.007 & 0.019 & 0.011 & 0.009 & 0.024 \\
  &  Variance & 0.00015 & 0.00029 & 0.0008 & 0.00021 & 0.00041 & 0.001 & 0.00026 & 0.00049 & 0.0014 & 0.00033 & 0.00062 & 0.0017 & 0.00041 & 0.00076 & 0.0022 \\
  &  MSE & 0.00031 & 0.00058 & 0.0017 & 0.00045 & 0.00085 & 0.0022 & 0.00058 & 0.001 & 0.003 & 0.00075 & 0.0013 & 0.0038 & 0.00096 & 0.0016 & 0.0049 \\
\multirow[c]{-4}{*}{$p_{2}$} &  Coverage & 1 & 1 & 1 & 1 & 1 & 1 & 1 & 1 & 1 & 1 & 1 & 1 & 1 & 1 & 1 \\
\cmidrule(l){2-17}   &  Bias & 0.00019 & 0.00014 & 9.5e-05 & 4.1e-05 & -1.6e-05 & 0.00026 & -0.00011 & 3.5e-05 & 0.00018 & -0.00021 & 7.2e-05 & 0.00012 & -0.00034 & 1.5e-05 & -4.1e-06 \\
  &  Variance & 1.7e-06 & 3.2e-06 & 1e-05 & 2.3e-06 & 4.5e-06 & 1.4e-05 & 3.1e-06 & 5.6e-06 & 1.9e-05 & 3.9e-06 & 7.1e-06 & 2.3e-05 & 4.8e-06 & 8.7e-06 & 2.8e-05 \\
  &  MSE & 3.4e-06 & 6.4e-06 & 2.1e-05 & 4.6e-06 & 9e-06 & 2.8e-05 & 6.2e-06 & 1.1e-05 & 3.8e-05 & 7.9e-06 & 1.4e-05 & 4.6e-05 & 9.6e-06 & 1.7e-05 & 5.6e-05 \\
\multirow[c]{-4}{*}{$p_{3}$} &  Coverage & 1 & 1 & 1 & 1 & 1 & 1 & 1 & 1 & 1 & 1 & 1 & 1 & 1 & 1 & 1 \\
\cmidrule(l){2-17}   &  Bias & -0.00086 & -4e-05 & -0.00067 & -0.0012 & -0.00023 & -0.00064 & -0.0015 & -0.00013 & -0.00088 & -0.0018 & -7.8e-05 & -0.0011 & -0.0022 & -0.00014 & -0.0014 \\
  &  Variance & 1.4e-06 & 2.8e-06 & 6.9e-06 & 1.9e-06 & 4e-06 & 8.9e-06 & 2.4e-06 & 4.9e-06 & 1.2e-05 & 3e-06 & 6.2e-06 & 1.5e-05 & 3.7e-06 & 7.6e-06 & 1.8e-05 \\
  &  MSE & 3.5e-06 & 5.6e-06 & 1.4e-05 & 5.1e-06 & 8.1e-06 & 1.8e-05 & 7e-06 & 9.9e-06 & 2.4e-05 & 9.3e-06 & 1.2e-05 & 3.1e-05 & 1.2e-05 & 1.5e-05 & 3.8e-05 \\
\multirow[c]{-4}{*}{$p_{4}$} &  Coverage & 1 & 1 & 1 & 1 & 1 & 1 & 1 & 1 & 1 & 1 & 1 & 1 & 1 & 1 & 1 \\
\cmidrule(l){2-17}   &  Bias & 2e-05 & -8.6e-06 & 3.3e-06 & 2.8e-05 & -5.8e-06 & -7.8e-07 & 3.7e-05 & -1e-05 & 8e-07 & 4.7e-05 & -1.5e-05 & 2.9e-07 & 5.6e-05 & -1.6e-05 & 1.6e-06 \\
  &  Variance & 4.2e-10 & 9.2e-10 & 2.2e-09 & 5.8e-10 & 1.4e-09 & 2.8e-09 & 7.8e-10 & 1.7e-09 & 3.9e-09 & 9.8e-10 & 2.1e-09 & 4.8e-09 & 1.2e-09 & 2.6e-09 & 6e-09 \\
  &  MSE & 1.2e-09 & 1.9e-09 & 4.4e-09 & 2e-09 & 2.8e-09 & 5.6e-09 & 3e-09 & 3.5e-09 & 7.7e-09 & 4.1e-09 & 4.4e-09 & 9.5e-09 & 5.6e-09 & 5.4e-09 & 1.2e-08 \\
\multirow[c]{-4}{*}{$p_{5}$} &  Coverage & 1 & 1 & 1 & 1 & 1 & 1 & 1 & 1 & 1 & 1 & 1 & 1 & 1 & 1 & 1 \\
\bottomrule
 &  & \multicolumn{3}{c}{11$\%$} & \multicolumn{3}{c}{12$\%$} & \multicolumn{3}{c}{13$\%$} & \multicolumn{3}{c}{14$\%$} & \multicolumn{3}{c}{15$\%$} \\
\cmidrule(l){3-5} \cmidrule(l){6-8} \cmidrule(l){9-11} \cmidrule(l){12-14} \cmidrule(l){15-17}  &  & M=1024 & M=512 & M=256 & M=1024 & M=512 & M=256 & M=1024 & M=512 & M=256 & M=1024 & M=512 & M=256 & M=1024 & M=512 & M=256 \\
\toprule
  &  Bias & -0.0017 & -0.00027 & -0.0018 & -0.0021 & -0.00045 & -0.0024 & -0.0026 & -0.00067 & -0.003 & -0.003 & -0.00093 & -0.0038 & -0.0035 & -0.0012 & -0.0046 \\
  &  Variance & 8.1e-06 & 2.1e-05 & 7.1e-05 & 9.6e-06 & 2.5e-05 & 8.4e-05 & 1.1e-05 & 2.9e-05 & 9.8e-05 & 1.3e-05 & 3.4e-05 & 0.00011 & 1.5e-05 & 3.9e-05 & 0.00013 \\
  &  MSE & 1.9e-05 & 4.3e-05 & 0.00014 & 2.4e-05 & 5.1e-05 & 0.00017 & 2.9e-05 & 5.9e-05 & 0.00021 & 3.5e-05 & 6.9e-05 & 0.00024 & 4.2e-05 & 7.9e-05 & 0.00028 \\
\multirow[c]{-4}{*}{$p_{1}$} &  Coverage & 1 & 1 & 1 & 1 & 1 & 1 & 1 & 1 & 1 & 1 & 1 & 1 & 1 & 1 & 1 \\
\cmidrule(l){2-17}   &  Bias & 0.014 & 0.011 & 0.03 & 0.017 & 0.014 & 0.036 & 0.02 & 0.016 & 0.043 & 0.024 & 0.019 & 0.051 & 0.028 & 0.023 & 0.06 \\
  &  Variance & 0.0005 & 0.00093 & 0.0027 & 0.0006 & 0.0011 & 0.0032 & 0.00072 & 0.0013 & 0.0038 & 0.00084 & 0.0015 & 0.0045 & 0.00098 & 0.0018 & 0.0052 \\
  &  MSE & 0.0012 & 0.002 & 0.0062 & 0.0015 & 0.0024 & 0.0077 & 0.0018 & 0.0029 & 0.0095 & 0.0022 & 0.0034 & 0.012 & 0.0027 & 0.0041 & 0.014 \\
\multirow[c]{-4}{*}{$p_{2}$} &  Coverage & 1 & 1 & 1 & 1 & 1 & 1 & 1 & 1 & 1 & 1 & 1 & 1 & 1 & 1 & 1 \\
\cmidrule(l){2-17}   &  Bias & -0.0005 & -6.6e-05 & -0.00016 & -0.00068 & -0.00015 & -0.00035 & -0.00086 & -0.00024 & -0.00058 & -0.001 & -0.00035 & -0.00084 & -0.0012 & -0.00047 & -0.0011 \\
  &  Variance & 5.8e-06 & 1.1e-05 & 3.4e-05 & 6.9e-06 & 1.3e-05 & 4.1e-05 & 8.1e-06 & 1.5e-05 & 4.7e-05 & 9.4e-06 & 1.7e-05 & 5.5e-05 & 1.1e-05 & 1.9e-05 & 6.3e-05 \\
  &  MSE & 1.2e-05 & 2.1e-05 & 6.8e-05 & 1.4e-05 & 2.5e-05 & 8.1e-05 & 1.7e-05 & 2.9e-05 & 9.5e-05 & 2e-05 & 3.4e-05 & 0.00011 & 2.3e-05 & 3.9e-05 & 0.00013 \\
\multirow[c]{-4}{*}{$p_{3}$} &  Coverage & 1 & 1 & 1 & 1 & 1 & 1 & 1 & 1 & 1 & 1 & 1 & 1 & 1 & 1 & 1 \\
\cmidrule(l){2-17}   &  Bias & -0.0026 & -0.00022 & -0.0018 & -0.003 & -0.00031 & -0.0021 & -0.0035 & -0.00041 & -0.0026 & -0.004 & -0.00055 & -0.0031 & -0.0044 & -0.00071 & -0.0036 \\
  &  Variance & 4.5e-06 & 9.2e-06 & 2.2e-05 & 5.3e-06 & 1.1e-05 & 2.6e-05 & 6.2e-06 & 1.3e-05 & 3.1e-05 & 7.2e-06 & 1.5e-05 & 3.5e-05 & 8.3e-06 & 1.7e-05 & 4.1e-05 \\
  &  MSE & 1.6e-05 & 1.8e-05 & 4.7e-05 & 2e-05 & 2.2e-05 & 5.7e-05 & 2.5e-05 & 2.6e-05 & 6.8e-05 & 3e-05 & 3e-05 & 8e-05 & 3.6e-05 & 3.5e-05 & 9.4e-05 \\
\multirow[c]{-4}{*}{$p_{4}$} &  Coverage & 1 & 1 & 1 & 1 & 1 & 1 & 1 & 1 & 1 & 1 & 1 & 1 & 1 & 1 & 1 \\
\cmidrule(l){2-17}   &  Bias & 6.6e-05 & -1.8e-05 & 3.4e-06 & 7.6e-05 & -1.9e-05 & 5.9e-06 & 8.7e-05 & -2e-05 & 9.1e-06 & 9.7e-05 & -2.1e-05 & 1.3e-05 & 0.00011 & -2.1e-05 & 1.8e-05 \\
  &  Variance & 1.5e-09 & 3.1e-09 & 7.3e-09 & 1.8e-09 & 3.7e-09 & 8.8e-09 & 2.2e-09 & 4.4e-09 & 1.1e-08 & 2.6e-09 & 5.1e-09 & 1.2e-08 & 3e-09 & 5.9e-09 & 1.5e-08 \\
  &  MSE & 7.4e-09 & 6.5e-09 & 1.5e-08 & 9.5e-09 & 7.8e-09 & 1.8e-08 & 1.2e-08 & 9.2e-09 & 2.1e-08 & 1.5e-08 & 1.1e-08 & 2.5e-08 & 1.8e-08 & 1.2e-08 & 2.9e-08 \\
\multirow[c]{-4}{*}{$p_{5}$} &  Coverage & 1 & 1 & 1 & 1 & 1 & 1 & 1 & 1 & 1 & 1 & 1 & 1 & 1 & 1 & 1 \\
\bottomrule
\end{tabular}

	}
\end{table}
\begin{table}[H]
	\caption{SIR-TDI (continued)} \label{tab:sirUQ2}	
	\resizebox{\textwidth}{!}{
	\begin{tabular}{lccccccccccccccccccccccccccccccccccccccccccccccccccccccccccccc}
 &  & \multicolumn{3}{c}{16$\%$} & \multicolumn{3}{c}{17$\%$} & \multicolumn{3}{c}{18$\%$} & \multicolumn{3}{c}{19$\%$} & \multicolumn{3}{c}{20$\%$} \\
\cmidrule(l){3-5} \cmidrule(l){6-8} \cmidrule(l){9-11} \cmidrule(l){12-14} \cmidrule(l){15-17}  &  & M=1024 & M=512 & M=256 & M=1024 & M=512 & M=256 & M=1024 & M=512 & M=256 & M=1024 & M=512 & M=256 & M=1024 & M=512 & M=256 \\
\toprule
  &  Bias & -0.004 & -0.0015 & -0.0058 & -0.0045 & -0.0018 & -0.0069 & -0.005 & -0.0022 & -0.0078 & -0.0055 & -0.0026 & -0.0093 & -0.006 & -0.003 & -0.01 \\
  &  Variance & 1.7e-05 & 4.4e-05 & 0.00016 & 1.9e-05 & 4.9e-05 & 0.00017 & 2.2e-05 & 5.5e-05 & 0.00019 & 2.5e-05 & 6.1e-05 & 0.00022 & 2.7e-05 & 6.7e-05 & 0.00023 \\
  &  MSE & 5e-05 & 9e-05 & 0.00035 & 5.9e-05 & 0.0001 & 0.00039 & 6.9e-05 & 0.00011 & 0.00043 & 7.9e-05 & 0.00013 & 0.00052 & 9.1e-05 & 0.00014 & 0.00057 \\
\multirow[c]{-4}{*}{$p_{1}$} &  Coverage & 1 & 1 & 1 & 1 & 1 & 1 & 1 & 1 & 1 & 1 & 1 & 1 & 1 & 1 & 1 \\
\cmidrule(l){2-17}   &  Bias & 0.032 & 0.026 & 0.073 & 0.036 & 0.03 & 0.089 & 0.041 & 0.035 & 0.1 & 0.046 & 0.039 & 0.12 & 0.052 & 0.044 & 0.13 \\
  &  Variance & 0.0011 & 0.002 & 0.0079 & 0.0013 & 0.0023 & 0.01 & 0.0014 & 0.0026 & 0.011 & 0.0016 & 0.003 & 0.014 & 0.0018 & 0.0034 & 0.015 \\
  &  MSE & 0.0032 & 0.0048 & 0.021 & 0.0039 & 0.0056 & 0.029 & 0.0046 & 0.0065 & 0.032 & 0.0054 & 0.0075 & 0.041 & 0.0064 & 0.0087 & 0.046 \\
\multirow[c]{-4}{*}{$p_{2}$} &  Coverage & 1 & 1 & 1 & 1 & 1 & 1 & 1 & 1 & 1 & 1 & 1 & 1 & 1 & 1 & 1 \\
\cmidrule(l){2-17}   &  Bias & -0.0014 & -0.00059 & -0.0017 & -0.0016 & -0.00073 & -0.0019 & -0.0017 & -0.00088 & -0.0022 & -0.0019 & -0.001 & -0.0028 & -0.002 & -0.0012 & -0.0031 \\
  &  Variance & 1.2e-05 & 2.2e-05 & 7.4e-05 & 1.4e-05 & 2.5e-05 & 8e-05 & 1.6e-05 & 2.8e-05 & 8.8e-05 & 1.8e-05 & 3.1e-05 & 0.0001 & 2e-05 & 3.4e-05 & 0.00011 \\
  &  MSE & 2.7e-05 & 4.5e-05 & 0.00015 & 3.1e-05 & 5e-05 & 0.00016 & 3.5e-05 & 5.7e-05 & 0.00018 & 3.9e-05 & 6.3e-05 & 0.00021 & 4.4e-05 & 7e-05 & 0.00023 \\
\multirow[c]{-4}{*}{$p_{3}$} &  Coverage & 1 & 1 & 1 & 1 & 1 & 1 & 1 & 1 & 1 & 1 & 1 & 1 & 1 & 1 & 1 \\
\cmidrule(l){2-17}   &  Bias & -0.0049 & -0.00089 & -0.0046 & -0.0055 & -0.0011 & -0.0058 & -0.006 & -0.0013 & -0.0065 & -0.0066 & -0.0016 & -0.0077 & -0.0072 & -0.0019 & -0.0085 \\
  &  Variance & 9.4e-06 & 2e-05 & 6.3e-05 & 1e-05 & 2.2e-05 & 8.7e-05 & 1.2e-05 & 2.5e-05 & 9.2e-05 & 1.3e-05 & 2.8e-05 & 0.00011 & 1.5e-05 & 3.1e-05 & 0.00012 \\
  &  MSE & 4.3e-05 & 4e-05 & 0.00015 & 5.1e-05 & 4.5e-05 & 0.00021 & 6e-05 & 5.1e-05 & 0.00023 & 6.9e-05 & 5.8e-05 & 0.00029 & 8e-05 & 6.5e-05 & 0.00031 \\
\multirow[c]{-4}{*}{$p_{4}$} &  Coverage & 1 & 1 & 0.99 & 1 & 1 & 0.98 & 1 & 1 & 0.98 & 1 & 1 & 0.97 & 1 & 1 & 0.97 \\
\cmidrule(l){2-17}   &  Bias & 0.00012 & -2.1e-05 & 0.01 & 0.00013 & -2.1e-05 & 0.02 & 0.00014 & -2e-05 & 0.02 & 0.00015 & -1.8e-05 & 0.03 & 0.00016 & -1.6e-05 & 0.03 \\
  &  Variance & 3.5e-09 & 6.8e-09 & 0.0099 & 4.1e-09 & 7.7e-09 & 0.02 & 4.6e-09 & 8.7e-09 & 0.02 & 5.3e-09 & 9.7e-09 & 0.029 & 6e-09 & 1.1e-08 & 0.029 \\
  &  MSE & 2.1e-08 & 1.4e-08 & 0.02 & 2.4e-08 & 1.6e-08 & 0.039 & 2.8e-08 & 1.8e-08 & 0.039 & 3.2e-08 & 2e-08 & 0.059 & 3.7e-08 & 2.2e-08 & 0.059 \\
\multirow[c]{-4}{*}{$p_{5}$} &  Coverage & 1 & 1 & 0.99 & 1 & 1 & 0.99 & 1 & 1 & 0.99 & 1 & 1 & 0.99 & 1 & 1 & 0.99 \\
\bottomrule
\end{tabular}

	}
\end{table}
