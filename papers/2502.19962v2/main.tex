% CVPR 2025 Paper Template; see https://github.com/cvpr-org/author-kit

\documentclass[10pt,twocolumn,letterpaper]{article}

%%%%%%%%% PAPER TYPE  - PLEASE UPDATE FOR FINAL VERSION
% \usepackage{cvpr}              % To produce the CAMERA-READY version
% \usepackage[review]{cvpr}      % To produce the REVIEW version
\usepackage[pagenumbers]{cvpr} % To force page numbers, e.g. for an arXiv version

% Import additional packages in the preamble file, before hyperref
%
% --- inline annotations
%
\newcommand{\red}[1]{{\color{red}#1}}
\newcommand{\todo}[1]{{\color{red}#1}}
\newcommand{\TODO}[1]{\textbf{\color{red}[TODO: #1]}}
% --- disable by uncommenting  
% \renewcommand{\TODO}[1]{}
% \renewcommand{\todo}[1]{#1}



\newcommand{\VLM}{LVLM\xspace} 
\newcommand{\ours}{PeKit\xspace}
\newcommand{\yollava}{Yo’LLaVA\xspace}

\newcommand{\thisismy}{This-Is-My-Img\xspace}
\newcommand{\myparagraph}[1]{\noindent\textbf{#1}}
\newcommand{\vdoro}[1]{{\color[rgb]{0.4, 0.18, 0.78} {[V] #1}}}
% --- disable by uncommenting  
% \renewcommand{\TODO}[1]{}
% \renewcommand{\todo}[1]{#1}
\usepackage{slashbox}
% Vectors
\newcommand{\bB}{\mathcal{B}}
\newcommand{\bw}{\mathbf{w}}
\newcommand{\bs}{\mathbf{s}}
\newcommand{\bo}{\mathbf{o}}
\newcommand{\bn}{\mathbf{n}}
\newcommand{\bc}{\mathbf{c}}
\newcommand{\bp}{\mathbf{p}}
\newcommand{\bS}{\mathbf{S}}
\newcommand{\bk}{\mathbf{k}}
\newcommand{\bmu}{\boldsymbol{\mu}}
\newcommand{\bx}{\mathbf{x}}
\newcommand{\bg}{\mathbf{g}}
\newcommand{\be}{\mathbf{e}}
\newcommand{\bX}{\mathbf{X}}
\newcommand{\by}{\mathbf{y}}
\newcommand{\bv}{\mathbf{v}}
\newcommand{\bz}{\mathbf{z}}
\newcommand{\bq}{\mathbf{q}}
\newcommand{\bff}{\mathbf{f}}
\newcommand{\bu}{\mathbf{u}}
\newcommand{\bh}{\mathbf{h}}
\newcommand{\bb}{\mathbf{b}}

\newcommand{\rone}{\textcolor{green}{R1}}
\newcommand{\rtwo}{\textcolor{orange}{R2}}
\newcommand{\rthree}{\textcolor{red}{R3}}
\usepackage{amsmath}
%\usepackage{arydshln}
\DeclareMathOperator{\similarity}{sim}
\DeclareMathOperator{\AvgPool}{AvgPool}

\newcommand{\argmax}{\mathop{\mathrm{argmax}}}     



% It is strongly recommended to use hyperref, especially for the review version.
% hyperref with option pagebackref eases the reviewers' job.
% Please disable hyperref *only* if you encounter grave issues, 
% e.g. with the file validation for the camera-ready version.
%
% If you comment hyperref and then uncomment it, you should delete *.aux before re-running LaTeX.
% (Or just hit 'q' on the first LaTeX run, let it finish, and you should be clear).
\definecolor{cvprblue}{rgb}{0.21,0.49,0.74}
\usepackage[pagebackref,breaklinks,colorlinks,allcolors=cvprblue]{hyperref}

\usepackage{multirow}
\usepackage{multicol}
\usepackage{makecell}
% \usepackage{ulem}

%%%%%%%%% PAPER ID  - PLEASE UPDATE
\def\paperID{11263} % *** Enter the Paper ID here
\def\confName{CVPR}
\def\confYear{2025}

%%%%%%%%% TITLE - PLEASE UPDATE
% \title{\LaTeX\ Author Guidelines for \confName~Proceedings}
% \title{Have Positive Samples Been Fully Exploited? Relation-aware Discrimination for Positive Pair Mining in Noisy Correspondence}
% \title{Preserving Relation Consistency for Enhanced Positive Pair Discrimination in Noisy Correspondence Learning}
\title{ReCon: Enhancing True Correspondence Discrimination through Relation Consistency for Robust Noisy Correspondence Learning}

%%%%%%%%% AUTHORS - PLEASE UPDATE

\author{Quanxing Zha$^1$,~~Xin Liu$^{1,2}$\thanks{Corresponding author},~~~Shu-Juan Peng$^1$,~~Yiu-ming Cheung$^{2}$,~~Xing Xu$^3$,~~Nannan Wang$^4$\\
$^1$Huaqiao University, $^2$Hong Kong Baptist University\\
$^3$University of Electronic Science and Technology of
China, $^4$Xidian University\\
{quanxing.zha@gmail.com, xliu@hqu.edu.cn}
}



%\author{Quanxing Zha\\
%Institution1\\
%Institution1 address\\
%{\tt\small quanxing.zha@gmail.com}
% For a paper whose authors are all at the same institution,
% omit the following lines up until the closing ``}''.
% Additional authors and addresses can be added with ``\and'',
% just like the second author.
% To save space, use either the email address or home page, not both
%\and
%Second Author\\
%Institution2\\
%First line of institution2 address\\
%{\tt\small secondauthor@i2.org}
%}

\begin{document}
\maketitle
% \begin{abstract}


The choice of representation for geographic location significantly impacts the accuracy of models for a broad range of geospatial tasks, including fine-grained species classification, population density estimation, and biome classification. Recent works like SatCLIP and GeoCLIP learn such representations by contrastively aligning geolocation with co-located images. While these methods work exceptionally well, in this paper, we posit that the current training strategies fail to fully capture the important visual features. We provide an information theoretic perspective on why the resulting embeddings from these methods discard crucial visual information that is important for many downstream tasks. To solve this problem, we propose a novel retrieval-augmented strategy called RANGE. We build our method on the intuition that the visual features of a location can be estimated by combining the visual features from multiple similar-looking locations. We evaluate our method across a wide variety of tasks. Our results show that RANGE outperforms the existing state-of-the-art models with significant margins in most tasks. We show gains of up to 13.1\% on classification tasks and 0.145 $R^2$ on regression tasks. All our code and models will be made available at: \href{https://github.com/mvrl/RANGE}{https://github.com/mvrl/RANGE}.

\end{abstract}

    
% \section{Introduction}
Backdoor attacks pose a concealed yet profound security risk to machine learning (ML) models, for which the adversaries can inject a stealth backdoor into the model during training, enabling them to illicitly control the model's output upon encountering predefined inputs. These attacks can even occur without the knowledge of developers or end-users, thereby undermining the trust in ML systems. As ML becomes more deeply embedded in critical sectors like finance, healthcare, and autonomous driving \citep{he2016deep, liu2020computing, tournier2019mrtrix3, adjabi2020past}, the potential damage from backdoor attacks grows, underscoring the emergency for developing robust defense mechanisms against backdoor attacks.

To address the threat of backdoor attacks, researchers have developed a variety of strategies \cite{liu2018fine,wu2021adversarial,wang2019neural,zeng2022adversarial,zhu2023neural,Zhu_2023_ICCV, wei2024shared,wei2024d3}, aimed at purifying backdoors within victim models. These methods are designed to integrate with current deployment workflows seamlessly and have demonstrated significant success in mitigating the effects of backdoor triggers \cite{wubackdoorbench, wu2023defenses, wu2024backdoorbench,dunnett2024countering}.  However, most state-of-the-art (SOTA) backdoor purification methods operate under the assumption that a small clean dataset, often referred to as \textbf{auxiliary dataset}, is available for purification. Such an assumption poses practical challenges, especially in scenarios where data is scarce. To tackle this challenge, efforts have been made to reduce the size of the required auxiliary dataset~\cite{chai2022oneshot,li2023reconstructive, Zhu_2023_ICCV} and even explore dataset-free purification techniques~\cite{zheng2022data,hong2023revisiting,lin2024fusing}. Although these approaches offer some improvements, recent evaluations \cite{dunnett2024countering, wu2024backdoorbench} continue to highlight the importance of sufficient auxiliary data for achieving robust defenses against backdoor attacks.

While significant progress has been made in reducing the size of auxiliary datasets, an equally critical yet underexplored question remains: \emph{how does the nature of the auxiliary dataset affect purification effectiveness?} In  real-world  applications, auxiliary datasets can vary widely, encompassing in-distribution data, synthetic data, or external data from different sources. Understanding how each type of auxiliary dataset influences the purification effectiveness is vital for selecting or constructing the most suitable auxiliary dataset and the corresponding technique. For instance, when multiple datasets are available, understanding how different datasets contribute to purification can guide defenders in selecting or crafting the most appropriate dataset. Conversely, when only limited auxiliary data is accessible, knowing which purification technique works best under those constraints is critical. Therefore, there is an urgent need for a thorough investigation into the impact of auxiliary datasets on purification effectiveness to guide defenders in  enhancing the security of ML systems. 

In this paper, we systematically investigate the critical role of auxiliary datasets in backdoor purification, aiming to bridge the gap between idealized and practical purification scenarios.  Specifically, we first construct a diverse set of auxiliary datasets to emulate real-world conditions, as summarized in Table~\ref{overall}. These datasets include in-distribution data, synthetic data, and external data from other sources. Through an evaluation of SOTA backdoor purification methods across these datasets, we uncover several critical insights: \textbf{1)} In-distribution datasets, particularly those carefully filtered from the original training data of the victim model, effectively preserve the model’s utility for its intended tasks but may fall short in eliminating backdoors. \textbf{2)} Incorporating OOD datasets can help the model forget backdoors but also bring the risk of forgetting critical learned knowledge, significantly degrading its overall performance. Building on these findings, we propose Guided Input Calibration (GIC), a novel technique that enhances backdoor purification by adaptively transforming auxiliary data to better align with the victim model’s learned representations. By leveraging the victim model itself to guide this transformation, GIC optimizes the purification process, striking a balance between preserving model utility and mitigating backdoor threats. Extensive experiments demonstrate that GIC significantly improves the effectiveness of backdoor purification across diverse auxiliary datasets, providing a practical and robust defense solution.

Our main contributions are threefold:
\textbf{1) Impact analysis of auxiliary datasets:} We take the \textbf{first step}  in systematically investigating how different types of auxiliary datasets influence backdoor purification effectiveness. Our findings provide novel insights and serve as a foundation for future research on optimizing dataset selection and construction for enhanced backdoor defense.
%
\textbf{2) Compilation and evaluation of diverse auxiliary datasets:}  We have compiled and rigorously evaluated a diverse set of auxiliary datasets using SOTA purification methods, making our datasets and code publicly available to facilitate and support future research on practical backdoor defense strategies.
%
\textbf{3) Introduction of GIC:} We introduce GIC, the \textbf{first} dedicated solution designed to align auxiliary datasets with the model’s learned representations, significantly enhancing backdoor mitigation across various dataset types. Our approach sets a new benchmark for practical and effective backdoor defense.



% \section{Related work}
\label{sec:formatting}

\subsection{Text-to-Video Generation}

T2V generation has made notable progress, evolving from early GAN-based models \cite{saito2017temporal,tulyakov2018mocogan,fu2023tell,li2018video,wu2022nuwa,yu2022generating} to newer transformer \cite{yan2021videogpt,arnab2021vivit,esser2021taming,ramesh2021zero,yu2022scaling} and diffusion models \cite{kirkpatrick2017overcoming,sohl2015deep,song2020denoising,zhang2022gddim}. Early efforts like MoCoGAN~\cite{tulyakov2018mocogan} focused on short video clips but faced issues with stability and coherence. The introduction of transformers improved sequential data handling, enhancing video generation, while diffusion models further improved video quality by progressively denoising the input. 
Despite these advances, T2V models still struggle to reflect human preferences, with the generated videos generally lacking aesthetic quality. Additionally, the scarcity of paired video preference data hinders effective model training and may lead to insufficient flexibility and poor quality in the generated videos.


\subsection{RLHF}

\iffalse
Aligning LLMs \cite{dai1901transformer,radford2019language,zhang2023opt} typically involves two steps: supervised fine-tuning followed by Reinforcement Learning with Human Feedback (RLHF) \cite{gao2023scaling,stiennon2020learning,rafailov2024direct}. Although effective, RLHF is computationally expensive and can lead to issues like reward hacking. Methods like DPO have streamlined alignment by leveraging feedback data directly, improving efficiency.

In contrast, diffusion model alignment is still evolving, focusing mainly on enhancing visual quality through curated datasets. Techniques like DOODL \cite{wallace2023end} and AlignProp \cite{prabhudesai2023aligning} target aesthetic improvements but face challenges with complex tasks such as text-image alignment. Reinforcement learning methods like DPOK \cite{fan2024reinforcement} and DDPO \cite{black2023training} improve reward optimization but struggle with scalability. DPO-SDXL integrates DPO into T2I generation, boosting both alignment and aesthetics.

However, aligning video generation remains a largely unaddressed challenge, especially when dealing with motion consistency and semantic coherence across frames.
\fi

RLHF \cite{gao2023scaling,stiennon2020learning,rafailov2024direct} is a method that utilizes human feedback to guide machine learning models. Early RLHF algorithms, such as DDPG~\cite{lillicrap2015continuous} and PPO~\cite{schulman2017proximal}, typically relied on complex reward models to quantify human feedback. These reward models require a large amount of annotated data and face challenges during tuning. As research has progressed, more efficient preference learning methods have emerged, among which DPO has become a new framework. DPO does not depend on a separate reward model; instead, it obtains human preferences through pairwise comparisons and directly optimizes these preferences. This shift not only simplifies the application of RLHF but also enhances the alignment of models with human values. Furthermore, DPO has been successfully introduced into T2I tasks~\cite{wallace2024diffusion,yang2024using}, providing new insights for generative models in addressing the alignment of human preferences and showcasing DPO's potential in the field of AIGC~\cite{shi2024instantbooth,
qing2024hierarchical,menapace2024snap,koley2024s}. However, there remains a gap in current research regarding the application of DPO strategies to T2V tasks. Effectively integrating DPO into T2V tasks presents a challenging endeavor.


% \section{Preliminary}
\label{sec:preliminary}
In this section, we first introduce the mathematical formulation of flow-based text-to-image generative models~\cite{Xingchao_2022,Lipman_2022}, which forms the foundation of modern T2I systems~\cite{sd3,sdxl,imagen3,imagen}. We then describe classifier-free guidance~\cite{ho2022classifier}, a key technique to control the generation process through text conditioning.

\subsection{Flow-based text-to-image generative models}
In state-of-the-art T2I models~\cite{sd3}, the image generation process is modeled by learning, through a neural network, a flow $\psi$ that generates a probability path $(p_t)_{0\le t\le 1}$ bridging the source distribution $p_0$ and the target distribution $p_1$ ~\cite{Xingchao_2022,Lipman_2022}. This framework encompasses diffusion models~\cite{sohl2015deep,ddpm} as a special case. In particular, a commonly used formulation sets a Gaussian distribution as the source: $p_0 = \mathcal{N}(\mathbf{0}, \mathbf{I})$ and a delta distribution centered on a sample $\mathbf{x}_1$ from the data distribution $q$ as the target: $p_1 = \delta_{\mathbf{x}_1}$.
Then, it incorporates an affine conditional flow $\psi_t(\mathbf{x} | \mathbf{x}_1) = a_t \mathbf{x}_1 + b_t \mathbf{x}$ with the boundary condition $(a_0, b_0) = (0, 1),\ (a_1, b_1) = (1, 0)$ to bridge them. The neural network typically approximates quantities such as velocity fields, $x_0$ prediction or $x_1$ prediction. In this modeling, these quantities can be viewed as affine transformations of the marginal probability path score $\nabla_{\mathbf{x}} \log p_t(\mathbf{x})$.

\subsection{Classifier-free guidance in flow-based models}
Classifier-free guidance~\cite{ho2022classifier} is a method for sampling from a model conditioned by a text input $\mathbf{y}$ by guiding an unconditional image generation model modeled using the score $\nabla_{\mathbf{x}} \log p_t(\mathbf{x})$. This enables the sampling from
\[
q_w(\mathbf{x}, \mathbf{y}) \propto q(\mathbf{x})q(\mathbf{y}|\mathbf{x})^w \propto q(\mathbf{x})^{1-w}q(\mathbf{x}|\mathbf{y})^w
\]
where $w \in \mathbb{R}$ is the guidance scale typically used with $w > 1$. The score satisfies
\[
\nabla_{\mathbf{x}} \log q_w(\mathbf{x}, \mathbf{y}) = (1-w)\nabla_{\mathbf{x}} \log q(\mathbf{x}) + w\nabla_{\mathbf{x}} \log q(\mathbf{x}|\mathbf{y})
\]
so by training the network to learn both the unconditional score $\nabla_{\mathbf{x}} \log q(\mathbf{x})$ and conditional score $\nabla_{\mathbf{x}} \log q(\mathbf{x}|\mathbf{y})$, flexible sampling from the conditional distribution can be achieved through a weighted sum of the network outputs.

\begin{abstract}
Can we accurately identify the true correspondences from multimodal datasets containing mismatched data pairs? Existing methods primarily emphasize the similarity matching between the representations of objects across modalities, potentially neglecting the crucial relation consistency within modalities that are particularly important for distinguishing the true and false correspondences. Such an omission often runs the risk of misidentifying negatives as positives, thus leading to unanticipated performance degradation. To address this problem, we propose a general \textbf{Re}lation \textbf{Con}sistency learning framework, namely \textbf{ReCon}, to accurately discriminate the true correspondences among the multimodal data and thus effectively mitigate the adverse impact caused by mismatches. Specifically, ReCon leverages a novel relation consistency learning to ensure the dual-alignment, respectively of, the cross-modal relation consistency between different modalities and the intra-modal relation consistency within modalities. Thanks to such dual constrains on relations, ReCon significantly enhances its effectiveness for true correspondence discrimination and therefore reliably filters out the mismatched pairs to mitigate the risks of wrong supervisions. Extensive experiments on three widely-used benchmark datasets, including Flickr30K, MS-COCO, and Conceptual Captions, are conducted to demonstrate the effectiveness and superiority of ReCon compared with other SOTAs. The code is available at: \href{https://github.com/qxzha/ReCon}{https://github.com/qxzha/ReCon}.

% In this work, we tackle this challenging issue in noisy correspondence learning.
% Such mismatches, i.e., noisy correspondence, are often introduced due to the expensive data collection and non-expert annotations, which 


% Learning discriminative features between distinct pairs is essential in noisy correspondence. Existing methods primarily emphasize the matching between representations of objects, neglecting the relations within modalities that are crucial for accurately distinguishing between true and false pairs. Such omission often risks misleading by misidentified negatives and thus leads to unexpected performance degradation. To address this problem, we propose a novel \textbf{Re}lation \textbf{Con}sistency learning framework, namely \textbf{ReCon}, that effectively enforces the alignment both between objects across modalities and relations within modalities, which significantly enhances the discriminability for true correspondences. Thanks to this dual consistencies, distinct data pairs can be accurately divided into appropriated partitions, ensuring that the corresponding optimization functions are employed to facilitate robust cross-modal retrieval. Extensive experiments on three widely-used benchmark datasets, including Flickr30K, MS-COCO, and Conceptual Captions, are conducted to demonstrate the effectiveness and superiority of ReCon compared with other SOTA methods. The code is available at: \href{https://anonymous.4open.science/r/ReCon-NCL}{https://anonymous.4open.science/r/ReCon-NCL}.
\end{abstract}

% \begin{abstract}
% Learning discriminative features between distinct multi-modal data pairs is essential in noisy correspondence. Existing methods primarily emphasize the matching between representations of objects, potentially neglecting the crucial relations within modalities that are  beneficial for accurately distinguishing the true and false pairs. Such an omission often risks misleading by misidentified negatives and thus leads to unanticipated performance degradation. To address this problem, we propose a novel \textbf{Re}lation \textbf{Con}sistency learning framework, namely \textbf{ReCon}, that effectively enforces the alignment both from the objects across modalities and relations within modalities, thereby significantly enhances the discriminability for true correspondences. Thanks to this dual consistencies, distinct data pairs can be accurately divided into appropriated partitions, ensuring that the corresponding optimization functions are employed to facilitate various cross-modal retrieval. Extensive experiments on three widely-used benchmark datasets, including Flickr30K, MS-COCO, and Conceptual Captions, are conducted to demonstrate the effectiveness and superiority of ReCon compared with other SOTA methods. The code is available at: \href{https://anonymous.4open.science/r/ReCon-NCL}{https://anonymous.4open.science/r/ReCon-NCL}.
% \end{abstract}

\section{Introduction}
Cross-modal retrieval is dedicated to understanding the semantic correspondences between multimedia data, aiming to recall the most relevant candidates for a given query \cite{scan,cmgm,dscmr,pecmr}. While existing approaches have achieved remarkable success by associating the heterogeneous data in a common latent space, they often neglect to provide an explicit consideration of semantically irrelevant data. Such mismatches, a.k.a., \textit{noisy correspondence} (NC) \cite{ncr}, would be inadvertently introduced due to the notoriously labor-intensive data collection and the unreliable non-expert annotations \cite{cc, scaling}, which inevitably impedes the semantic correspondences between modalities and consequently results in a decline of retrieval performance \cite{BiCro,crcl,MSCN}. 

\begin{figure}
    \centering
    \includegraphics[width=0.98\linewidth]{figures/new_motivation.pdf}
    % \caption{Illustration of  motivation and difference with SOTAs.}
    \caption{Illustration of relation discrepancy. The relation-aware alignment correctly identifies mismatched pair as negatives, while relation-agnostic alignment fails to detect such inconsistency.}
    \label{motivation}
\end{figure}

To tackle the NC problem, a core consensus is to enhance the discriminability for positives/matches and to mine local correspondences from negatives/mismatches. Several priors \cite{ncr,MSCN,BiCro} leverage the \textit{memory effect} \cite{coteaching}, wherein DNNs learn simple dominant patterns first, to identify matches in the early training stage. Subsequently, they estimate soft correspondence labels to describe the matching degree of mismatches, thus down-weighting their contributions and enforcing learning the local correspondences. In order to avoid the misleading caused by easily-determined noisy pairs, some attempts \cite{trip,cream,UGNCL} propose more refined data division strategies to filter out these mismatches. To further mitigate the wrong supervisions of mismatches, recent efforts \cite{l2rm,PC2} are presented to utilize pseudo counterparts for these mismatches to excavate informative correspondences. Furthermore, some works \cite{esc,gsc} achieve notable performance improvements by leveraging intrinsic properties observed within data, and methods \cite{decl,rcl,crcl} based on robust loss functions also effectively confront with the challenge of NCs. Nevertheless, they all neglect the relations within modalities, risking the misidentification of negatives as positives, particularly in cases of mismatched pairs that manifest high similarity scores, i.e., hard NCs. 

%\sout{As clarified in IAIS work~\cite{iais}, the relation consistency often enhances the contextualized representation of image-text pairs, which is beneficial to the model alignment for cross-modal retrieval.}

 As mentioned in IAIS work~\cite{iais}, the relation consistency often enhances the contextualized representation of image-text pairs. Inspired by this finding, we consider the relation discrepancy to mitigate the adverse impacts caused by mismatches among dataset.  As shown in Fig. \ref{motivation}(a), whether through relation-agnostic or relation-aware alignment, the true correspondence is expected to consistently assigned a high similarity score due to its perfect matching of both objects across modalities and relations within modalities. However, such matching is irreversibly compromised by the presence of untouchable noisy correspondence, thus narrows the distance between mismatches. Specifically, the unwanted misalignment erroneously reduces the distance of unassociated objects, inadvertently confusing retrieval models and thus undermining their discriminability for true correspondences. Besides, such misalignment also impairs the relations between objects within modalities, which significantly disrupts the contextual semantic consistency that is essential for true correspondences despite the nuance from objects. As shown in Fig. \ref{motivation}(b), the noisy pair with similar objects cannot be correctly identified by the relation-agnostic alignment due to its inability to recognize the discrepancies of relations within modalities. Such misidentification inevitably introduces false supervisions, which misleads the model towards further wrong optimization direction. In contrast, the relation-aware alignment accurately identifies it as a negative pair, benefiting from its dual consideration of both cross-modal and intra-modal relations.

%\textcolor{green}{was first explored by previous work} \cite{iais}, 

% As illustrated in Fig. \ref{fig:network},
Motivated by the above observations, we propose a general \textbf{Re}lation \textbf{Con}sistency learning framework, namely \textbf{ReCon}, to effectively mitigate the adverse impact caused by NCs, as shown in Fig.~\ref{fig:network}. The main motivation of ReCon is to \textit{enhance the discriminability for true correspondences}. Specifically, an effective relation consistency alignment strategy is introduced to enable alignment not only between objects across modalities but relations within modalities. In details, the cross-modal relation consistency is presented to maximize the similarity scores of positive pairs while minimizing the negatives, ensuring that aligned objects have similar semantic representations. Meanwhile, the intra-modal relation consistency is employed to minimize the distance of relation matrices that describe the contextualized semantics of objects within modalities, which further enlarges the distinguish between positives and negatives. In practice, due to the lack of explicit annotations of objects, we propose to align the relation matrix extracted from one selected anchor modality with the proxy relation matrix extracted from another modality. Subsequently, such dual constraints of relations are employed to divide the noisy training data, wherein the divided partitions will be trained with corresponding strategies to achieve robust cross-modal retrieval, which significantly enhances the discriminability for true correspondences and effectively avoids the wrong supervisions of misidentified negatives. 
% Finally, the divided partitions will be trained with corresponding strategies to 
% effectively avoids the misidentification of negatives as positives and 
% wherein the pairs with cross- and intra-modal consistencies are identified as true correspondences. Such division strategy effectively avoids the misidentification of negatives as positives and thus significantly enhances the discriminability of models for true correspondences. Finally, the divided partitions are trained with corresponding loss functions to achieve robust noisy correspondence learning. 

%In summary, the main contributions of our work are as follows: (1) We propose a general \textbf{Re}lation \textbf{Con}sistency learning framework, namely \textbf{ReCon}, to mitigate the adverse impact caused by NCs within multimodal dataset. (2) We introduce a novel relation consistency learning to enable alignment not only between objects across modalities but relations within modalities. (3) A reliable discrimination strategy is presented to leverage the dual constrains of relations to accurately identify true correspondences. (4) Extensive experiments on synthetic and real-world benchmark datasets are conducted to jointly demonstrate the robustness and effectiveness of our proposed ReCon.

In summary, the main contributions of our work are as follows: (1) A general \textbf{Re}lation \textbf{Con}sistency learning framework, namely \textbf{ReCon}, is robustly proposed to identify the true correspondences and therefore mitigate the adverse impact caused by NCs within multimodal dataset. (2) An effective relation consistency alignment strategy is explicitly employed to jointly enforce the alignment of the cross-modal relation consistency and the intra-modal relation consistency. (3) A reliable true correspondence discrimination strategy is effectively presented to accurately partition the noisy data pairs, which therefore, seamlessly minimizes the risk caused by wrong supervisions and mitigates the misidentification of mismatches. (4) Extensive experiments highlight the advantages of the proposed ReCon in comparison with other SOTA methods and demonstrate its outstanding performances in challenging NCs scenario.


% that leverages the dual constrains of relations to identify true correspondences, which effectively avoids the misleading of misidentified negatives and significantly mitigates the adverse impact caused by noisy correspondence. (2) We introduce a novel relation consistency learning that contains cross- and intra-modal relation consistency to jointly enforce the alignment both between representations across modalities while preserving the contextualized semantics within modalities, which encourages the distinct divided partitions to utilize corresponding optimization objectives in order to robust cross-modal retrieval. (3) Extensive experiments on synthetic and real-world benchmark datasets are conducted to demonstrate the robustness and effectiveness of our ReCon.

% Specifically, the cross-modal relation consistency is presented to maintain coherence between objects across modalities, ensuring that they have similar semantic representations. Meanwhile, the intra-modal relation consistency aims to preserve contextual semantic consistency within modalities, 

\section{Related Work}
\subsection{Cross-Modal Retrieval}
% with representatives like cross-attention \cite{scan} and similarity graph \cite{sgraf, cmgm},
Cross-modal retrieval (CMR) aims to search the most relevant items across different modalities in response to query modality. The core of CMR is to minimize the semantic discrepancies by projecting different modalities into a common comparable space, wherein the matched items manifest higher similarity or closer feature distance and vice versa. Current efforts, from the perspective of similarity calculation, can be roughly classified into two categories: 1) Coarse-grained measurement \cite{vse++,vsrn,gpo,vtsr}, which represents an efficient solution with the key idea of associating the correspondence holistically among features extracted by distinct modality-specific encoders. 2) Fine-grained measurement \cite{scan,naaf,cmgm, sgraf, lsra, cross-graph,chan}, which focuses on assessing the semantic relationships at a more granular level to learn and reason latent alignments between fragments. Unfortunately, the promising performance of all these methods relies heavily on an implicit assumption that all training data pairs are correctly matched while neglecting the presence of NC. Such NC inevitably undermines the alignments and complicates the accurate measurement of similarity, ultimately leading to inferior performance.

% by well-designed consistency interactions, 
%, which is often introduced due to the extensive data collection and annotation expenses
\subsection{Noisy Correspondence Learning}
% , which, recent years, has been widely explored in multi-modal tasks, including image-text matching \cite{ncr,rcl}, multi-view clustering \cite{multiview-1,multiview-2}, image captioning \cite{imagecap1,imagecap2}, and infrared re-identification \cite{reident1,reident2}.
Noisy correspondence learning refers to noise-tolerant approaches well-designed to effectively mitigate the adverse impacts caused by mismatches among dataset. Unlike traditional category-level mistaken annotations, this instance-level semantic inconsistency, first recognized as a new paradigm of noisy labels in \cite{ncr}, significantly affects the performance of retrieval models. Thus, some prior attempts \cite{ncr, BiCro,MSCN} employ the small-loss criterion \cite{dividemix} to identify matched pairs from the corrupted datasets and subsequently rectify soft correspondence labels for those mismatches. Following this, several works \cite{cream,trip,UGNCL} introduce novel criteria to enable more fine-grained data division, such as inconsistent predictions \cite{cream,trip} and uncertainty \cite{UGNCL}. To avoid inaccurate label predictions, some approaches \cite{l2rm,PC2} aim to refine alignments through alternative strategies like rematched mismatches \cite{l2rm} and pseudo captions \cite{PC2}. Besides these methods based data sanitized, other efforts \cite{decl,rcl,crcl} retort to robust loss functions to adaptively downweight the contributions of mismatches, e.g., evidential loss \cite{decl}, complementary contrast loss \cite{rcl}, and active complementary loss \cite{crcl}. Recently, some works \cite{esc,gsc} utilize the intrinsic properties observed within data to estimate accurate soft correspondence labels. However, thery all neglect the intra-modal relations, which is significantly crucial for accurately identify true correspondences.


\section{Methodology}

\begin{figure*}
    \centering
    \includegraphics{figures/network_v2.pdf}
    \caption{The schematic pipeline of the proposed ReCon learning framework.}
    \label{fig:network}
\end{figure*}

\subsection{Preliminaries}
\paragraph{Problem Definition}
In line with previous work, we take visual-text retrieval as a proxy task to discuss the noisy correspondence problem in cross-modal retrieval. Consider a multimodal dataset $\mathcal{D}=\left\{\left(V_{i}, L_{i}, y_{i}\right)\right\}_{i=1}^{N}$ containing of $N$ training pairs, where each $\left(V_{i}, L_{i}\right)$ denotes the $i$-th visual-text pair and $y_{i} \in \left\{0,1\right\}$ indicates whether the pair matched ($y_{i}=1$) or not ($y_{i}=0$). Typically, all pairs are assumed to be semantically associated with high similarity scores in the common representation space. However, due to the substantial costs of data collection and annotations, an unknown portion of mismatched pairs may be inadvertently labeled as matched ones. Such misalignment, a.k.a., noisy correspondence, without specific treatment, would severely disrupt the alignment between modalities and ultimately lead to performance degradation. The goal of our method is to effectively address the challenge of NCs within multimodal datasets, thus enabling robust cross-modal retrieval.

\paragraph{Intra-Modal Relation Alignment}
\label{sec:relation}
Given a sequence $\mathbf{O}=[o_{1}, \cdots, o_{N_{o}}]$ containing $N_{o}$ objects appeared in a visual-text pair, the sequences of visual and linguistic can be denoted as $\overline{\mathbf{V}}=[\overline{v}_{1},\cdots,\overline{v}_{N_{o}}]$ and $\overline{\mathbf{L}}=[\overline{l}_{1},\cdots,\overline{l}_{N_{o}}]$, respectively. Here, each item with same index corresponds to a same object. Note that an object may be described by one or more words in the sentence and one or more regions in the image, such that the linguistic item and visual item may represent a collocation of words and regions, respectively. The relation $\mathbf{C}_{o_{i}}=[c_{o_{i}\rightarrow o_{1}},\cdots,c_{o_{i}\rightarrow o_{N_{o}}}]$ of one object to others can also be depicted in both visual and linguistic modalities, i.e., $\mathbf{C}_{\overline{v}_{i}}=[c_{\overline{v}_{i}\rightarrow \overline{v}_{1}},\cdots,c_{\overline{v}_{i}\rightarrow \overline{v}_{N_{o}}}]$ and $\mathbf{C}_{\overline{l}_{i}}=[c_{\overline{l}_{i}\rightarrow \overline{l}_{1}},\cdots,c_{\overline{l}_{i}\rightarrow \overline{l}_{N_{o}}}]$, respectively.  Consequently, the alignment of such relations can be preserved by minimizing the expected risk for the distance objective~\cite{iais}, as expressed in the following equation:
\begin{equation}
    \mathcal{R}_{\mathcal{L}_{SD}} = \min\mathbb{E}_{(\mathbf{C}_{\overline{v}_{i}},\mathbf{C}_{\overline{l}_{i}}) \sim \mathcal{D}}[\mathcal{L}_{SD}(\mathbf{C}_{\overline{v}_{i}},\mathbf{C}_{\overline{l}_{i}})],
    \label{eq:relation}
\end{equation}
where $\mathcal{L}_{SD}$ is the loss function that utilized for narrowing semantic distance, e.g., symmetric matrix-based Kullback-Leibler Divergence (m-LK). Note that, IAIS \cite{iais} represents such relations within modalities using cross-modal attention matrix. Differently, ReCon obtains these relations by computing the similarity between objects within modalities.
 
\subsection{Relation Consistency Learning}
Let $\mathbf{V}{=}\left[v_{1},\cdots,v_{N_{\mathcal{V}}}\right]$ and $\mathbf{L}{=}\left[l_{1},\cdots,l_{N_{\mathcal{L}}}\right]$ be the original visual and linguistic input sequences, which respectively contains $N_{\mathcal{V}}$ visual regions and $N_{\mathcal{L}}$ linguistic words. The relation consistency learning aims not only to enforce alignment between objects across modalities, but also to ensure consistency of relations within modalities. Such dual constraints allow to comprehend more nuanced contextualized semantics compared to the relation-agnostic alignment and significantly improve the discriminability for true correspondences, which can effectively mitigate the risks of misleading caused by misidentified false supervisions, particularly in the presence of hard NCs. 

\paragraph{Cross-Modal Relation Consistency.} Cross-modal relation consistency refers to the semantic similarities between representations across modalities. To this end, two modal-specific networks $f_{\mathcal{V}}(\cdot, \Theta_{\mathcal{V}})$ and $f_{\mathcal{L}}(\cdot, \Theta_{\mathcal{L}})$ are first employed to project the visual and linguistic sequences into a common comparable space, where $\Theta_{\mathcal{V}}$ and $\Theta_{\mathcal{L}}$ are the parameterized models for visual and linguistic modalities, respectively. In the common space, the similarity of the given visual-linguistic pair is measured through similarity function $S=g(f_{\mathcal{V}}(\cdot),f_{\mathcal{L}}(\cdot),\Theta_{\mathcal{G}})$, where $\Theta_{\mathcal{G}}$ is the parameterized modal of similarity function $g$. Note that $g$ can be parametric \cite{sgraf, cmgm} or non-parametric \cite{gpo, vse++} function. For convenience, we denote $g(f_{\mathcal{V}}(\cdot),f_{\mathcal{L}}(\cdot))$ as $g(\cdot,\cdot)$ in the following. Intuitively, the goal of cross-modal consistency relation learning is to encourage the semantic gap between matches and mismatches as large as possible, which can be equivalent to maximizing the bidirectional matching probabilities of true correspondences. Consider a batch size $N_{b}$ pairs $\mathcal{D}_{N_{b}}=\{(V_{i}, L_{i}, y_{i})\}_{i=1}^{N_{b}}$, the matching probability of $i$-th visual query is defined as $p^{v2l}_{ij}=\frac{\exp(g\left(V_{i},L_{j}\right) / \tau)}{\sum_{k=1}^{N_{b}} \exp\left(g(V_{i},L_{k}\right) / \tau)}$, where $\tau$ is a temperature parameter. Likewise, the matching probability of $i$-th linguistic query is defined as $p^{l2v}_{ij}=\frac{\exp(g\left(V_{i},L_{j}\right) / \tau)}{\sum_{k=1}^{N_{b}} \exp(g\left(V_{k},L_{j}\right) / \tau)}$. Consequently, the cross-modal relation consistency can be preserved by minimizing the expected risk of bidirectional matching probabilities with the supervision of $y_{i}$, as expressed in the following equation:
\begin{equation}
% \left(f_{\mathcal{V}},f_{\mathcal{L}},g\right)
    \mathcal{R}_{\mathcal{L}_{CM}} = \min \mathbb{E}_{(V_{i},L_{i},y_{i} ) \sim \mathcal{D}_{N_{b}}} [\mathcal{L}_{CM}\left(V_{i},L_{i},y_{i}\right)],
\end{equation}
where $\mathcal{L}_{CM}$ denotes the cross-modal InfoNCE loss \cite{clip}, which encourages the similarity gap between matched and mismatched pairs as large as possible. Note that, the contributions of different data pairs will be adaptively adjusted according to their corresponding supervisions $y$. 

\paragraph{Intra-Modal Relation Consistency.}
Intra-modal relation consistency refers to the matching of semantics between visual contexts among regions and linguistic contexts among words. Unfortunately, the absence of explicit object annotations presents a particularly intricate and demanding challenge, which is quite common in real-world scenarios. Consequently, we cannot access to the sequences containing objects, which means that each visual/linguistic item now corresponds to only one region/word. Undoubtedly, Eq. \eqref{eq:relation} cannot be directly applied to such input sequences due to the lack of one-to-one correspondence properties found in object sequences. To address this problem, like~\cite{iais}, we first select an anchor modality, e.g., visual modality, containing regions sequence, and then construct a proxy sequence containing the most corresponding words sequence from the opposite modality, such that the relations of distinct modalities can be comparable. Gven a sequence $V$ with visual modality as anchor, the relations to sequence $L$ from the opposite modality can be obtained by $\mathbf{C}_{\mathcal{VL}}=g(V,L)$, wherein the relations between every visual item $v_{i}$ to all the linguistic items are depicted in $\mathbf{C}_{\mathcal{VL}}[i,:]$, i.e., the $i$-th row of this relation matrix. Subsequently, we can obtain the most relevant item $l_{i^*}$ for $v_{i}$, wherein the index can be calculated as $i^*=\arg \max \mathbf{C}_{\mathcal{VL}}[i,:]$. Likewise, we can obtain the most relevant linguistic item $l_{j^*}$ for the visual item $v_{j}$. Therefore, the intra-modal relations $c_{v_{i} \rightarrow v_{j}}$ within visual modality can be depicted by the intra-modal relations $c_{l_{i^*} \rightarrow l_{j^*}}$ within linguistic modality, which can be formulated in the following equation:
\begin{equation}
    % \mathbf{C}_{\mathcal{VV}}^{p}=\{c_{l_{i^*}\rightarrow l_{j^*}} | i^*,j^* \in [1,N_{\mathcal{L}}] \} = \Psi(\mathbf{C}_{\mathcal{LL}}, i^*,j^*)
    \mathbf{C}_{\mathcal{VV}}^{p} = \{c^{p}_{v_{i} \rightarrow v_{j}} | i,j\in[1,N_{\mathcal{V}}] \} = \Psi(\mathbf{C}_{\mathcal{LL}},l_{i^*},l_{j^*}),
    \label{eq:proxy}
\end{equation}
where $\Psi$ represents a reconstruction operation that form the proxy relation matrix. Here, the $\mathbf{C}_{\mathcal{VV}}^{p}$ can be regarded as a representation of the original visual relation matrix $\mathbf{C}_{\mathcal{VV}}$ from the linguistic view. Similarly, with linguistic modality as anchor, the reconstructed proxy relation matrix $\mathbf{C}_{\mathcal{LL}}^{p}$ from the visual view, which depicts the relations within linguistic modality, can also be obtained through above Eq. \eqref{eq:proxy}. As discussed in Sec. \ref{sec:relation}, we can now employ the m-LK to compute the distances between relation matrix and its proxy relation matrix, which can be defined as:
\begin{equation}
    \mathcal{L}_{IM} = D_{KL}(\mathbf{C}_{\mathcal{VV}}||\mathbf{C}_{\mathcal{VV}}^{p}) + D_{KL}(\mathbf{C}_{\mathcal{LL}}||\mathbf{C}_{\mathcal{LL}}^{p}).
\end{equation}

Therefore, the preservation of relations within modalities can be achieved through minimizing the expected risk of the above $\mathcal{L}_{IM}$, as formulated in the following equation:
\begin{equation}
    \mathcal{R}_{\mathcal{L}_{IM}} = \min \mathbb{E}_{(V_{i},L_{i},y_{i}) \sim \mathcal{D}_{N_{b}}}[ \mathcal{L}_{IM}(V_{i},L_{i},y_{i})].
\end{equation}

\subsection{True Correspondence Discrimination}
Due to the existence of NC, we can only have access to the noisy training dataset $\Tilde{\mathcal{D}}$ containing an unknown proportion of mismatched pairs. Thus, directly optimizing models on such dataset using the above loss functions may risks misleading by the unwanted mismatched pairs, potentially causing significant performance degradation or even leading to training collapse. To address this problem, a common strategy \cite{ncr, BiCro, esc} is to leverage the small-loss criterion \cite{dividemix} to divide the noisy dataset into clean and noisy partitions, wherein the different partitions will be processed with corresponding training strategies. In details, the clean partition can be directly used for model optimization, while the corresponding strategy might be exploited to learn all available and informative knowledge from the noisy partition, e.g., locally-associated correspondences, avoiding insufficient utilization for dataset. However, the previous methods may misidentifies the mismatched pairs as matches, thus declining the discriminability for true correspondences and resulting in suboptimal performance due to the misleading of mismatches. Thanks to the dual constrains of relations, ReCon provides more refined and reliable data division and effectively mitigates the misidentification of mismatches, especially in the existence of hard NCs.

\paragraph{Noisy Data Division.}
Inspired by the previous success \cite{ncr,BiCro}, we also leverage the small-loss criterion to achieve a rough division for the corrupted training data. Specifically, we first compute the per-sample cross-modal relation loss by $\mathcal{L}_{CM}$, denoted as $\{l_{i}^{CM}\}_{i=1}^{N}=\{\mathcal{L}_{CM}(V_{i},L_{i})\}_{i=1}^{N}$. Next, a two-component Gaussian Mixture Model (GMM) \cite{gmm} would be employed to fit the per-sample loss distribution of all training pairs, which can be expressed as $p(l_{i}^{CM})=\sum_{k=1}^{K}\lambda_{k}\phi(l_{i}^{CM} | k)$. Here $K=2$, $\lambda_{k}$ is the corresponding mixture coefficient, and $\phi(l_{i}^{CM} | k)$ indicates the probability density function of the $k$-th component. Besides, the Expectation-Maximization algorithm is employed to optimize the GMM. Finally, we use the component with smaller mean to obtain the estimated probability:
\begin{equation}
    y_{i}^{CM}=p(k|l_{i}^{CM})=p(k)p(l_{i}^{CM}|k) / p(l_{i}^{CM}).
\end{equation}

By setting a threshold $\omega_{1}$, we can roughly divide the dataset $\tilde{\mathcal{D}}$ into rough clean partition $\tilde{\mathcal{D}}_{c}=\{(V_{i},L_{i})|y_{i}^{CM} > \omega_{1}\}$ and noisy partition $\mathcal{D}_{n}=\{(V_{i},L_{i})|y_{i}^{CM} \leq \omega_{1}\}$. Theoretically, the probability of positive pairs should approach 1, while for negative pairs, it should tend toward 0. 

\paragraph{True Positives Identification.}
% To further accurately identify the true positives from $\tilde{\mathcal{D}}_{c}$ that may contain potential mismatches manifesting high similarity, the intra-modal relation loss is employed to filter the pairs with inconsistent intra-modal relations. 
To ensure that the model learns accurate representations of matched data pairs and their relations, establishing a reliable division for true positives is crucial. In practice, the accurate discrimination for positives contributes more than negatives, for only true positives can effectively guide model optimization and further enhance its discriminability, thus minimizing the risk caused by wrong supervisions. Even if some positive pairs are wrongly divided into the noisy partition, they can still be learned through the corresponding strategy of handling the noisy partition. However, the misidentified negatives would directly compromise the discriminability of model, which further increase the risk of learning from the false correspondences. Consequently, we employ the cross-modal and intra-modal relation consistency to jointly discriminate the true positives, and the discrepancies of relations within modalities can be measured through the following equation:
\begin{equation}
    y_{i}^{IM} = \frac{ \log(1 + \mathcal{L}_{IM}(V_{i},L_{i}))}{1 + \log(1 + \mathcal{L}_{IM}(V_{i},L_{i}))}.
\end{equation}

Theoretically, the discrepancies for true correspondences should approach zero, while others will exhibit significantly larger discrepancies due to their inconsistent intra-modal relations. Thus, we can distinguish such pairs from the $\tilde{\mathcal{D}}_{c}$ by a fixed threshold $\omega_{2}$ to form two refined partitions:
\begin{equation}
    \begin{cases}
        \mathcal{D}_{c} = \{(V_{i},L_{i})|y_{i}^{IM} < \omega_{2}, \forall(V_{i},L_{i}) \in \tilde{\mathcal{D}}_{c} \} \\
        \mathcal{D}_{l} = \{(V_{i},L_{i})|y_{i}^{IM} \geq \omega_{2}, \forall(V_{i},L_{i}) \in \tilde{\mathcal{D}}_{c} \}
    \end{cases},
\end{equation}
where $\mathcal{D}_{c}$ denotes the clean partition containing true correspondences that can be directly employed to subsequent training and $\mathcal{D}_{l}$ contains pairs of local-associated correspondences. To fully learn all available local-associated correspondences and  enhance the discriminability of models, while simultaneously enlarge the semantic distance between true correspondences and others, we penalize the weight of pairs belonging to $\mathcal{D}_{l}$ based on their discrepancies of intra-modal relations. The specific penalization factor can be calculated as follows:
\begin{equation}
\label{penalization}
    \lambda = \exp\{y_{i}^{IM} / \alpha\},
\end{equation}
where $\alpha$ is an empirical scale parameter. For the pairs belonging to $\mathcal{D}_{n}$, we estimate pseudo labels through the predictions of models to replace the original unreliable labels, which can be expressed as:
\begin{equation}
    \tilde{y}_{i}^{t} = \beta \tilde{y}_{i}^{t-1} + (1-\beta)p^{t}(V_{i},L_{i}), \forall (V_{i},L_{i}) \in \mathcal{D}_{n},
\end{equation}
where $\beta$ is the momentum coefficient, $\tilde{y}_{i}^{t}$ represents the estimated labels at $t$-th epoch and $p(V_{i},L_{i})=(p_{ii}^{v2t} + p_{ii}^{t2v}) / 2$ denotes the average matching probability. Thus, the final recasted labels of all pairs can be summarized as follows:
\begin{equation}
    \hat{y}_{i} = \begin{cases}
        1, \forall \left(V_{i},L_{i} \right) \in \mathcal{D}_{c} \cup \mathcal{D}_{l} \\
        \tilde{y}_{i}, \forall \left(V_{i},L_{i} \right) \in \mathcal{D}_{n}
    \end{cases}.
\end{equation}

\subsection{Overall Optimization Objective}
To ensure the initial stability and convergence for subsequent training, we first conduct $\eta$ epochs warmup process using the triplet loss \cite{vsrn}, which can be denoted as follows:
\begin{equation}
    \begin{aligned}
        \mathcal{L}_{w} &= \sum_{\tilde{L}}[\gamma - g(V_{i},L_{i}) + g(V_{i},\tilde{L})]_{+} \\
    &+ \sum_{\tilde{V}}[\gamma - g(V_{i},L_{i}) + g(\tilde{V},L_{i})]_{+}
    \end{aligned},
\end{equation}
where $\gamma$ is the fixed margin that controls the distance between positives and negatives, $[x]_{+}=\max(x,0)$, $\tilde{L}$ and $\tilde{V}$ are the negative samples in a given mini batch. Afterwards, the different partitions will be trained with corresponding optimization strategies. For pairs belonging to the $\mathcal{D}_{c}$, we aim to learn correct representations of matched pairs and relations, thus enhancing the discriminability for true correspondences.
Hence, the loss function for $\mathcal{D}_{c}$ is a combination of $\mathcal{L}_{CM}$ and $\mathcal{L}_{IM}$:
\begin{equation}
    \mathcal{L}_{c} = \xi \mathcal{L}_{CM} + \mathcal{L}_{IM},
\end{equation}
where $\xi$ is the balance factor that adjusts the contributions of cross-modal relations. As for the pairs belonging to $\mathcal{D}_{l}$, the penalization factor calculated by Eq. \eqref{penalization} will be employed to downweight the contributions of intra-modal relations:
\begin{equation}
    \mathcal{L}_{l} = \xi \mathcal{L}_{CM} + \frac{1}{\lambda} \mathcal{L}_{IM}.
\end{equation}

Finally, for the pairs belonging to $\mathcal{D}_{n}$, the estimated pseudo labels will be employed to adjust their contributions in cross-modal relations, while the intra-modal relations will be excluded to avoid incorrect supervisions:
\begin{equation}
    \mathcal{L}_{n} = \hat{y}_{i}\mathcal{L}_{CM} = \mathcal{H}(\hat{y}_{i},p_{ii}^{v2l}) + \mathcal{H}(\hat{y}_{i},p_{ii}^{l2v}),
\end{equation}
where $\mathcal{H}$ denotes the batched cross-entropy function.

\section{Experiments}

\subsection{Datasets and Protocols}
\paragraph{Datasets.}
We evaluate our method on three widely-used benchmarks, following the settings in \cite{ncr}. Specifically, Flickr30K \cite{flickr} contains 31K images with five textual descriptions, collected from the Flickr website. We split 1K image-text pairs for validation, 1K pairs for testing, and the rest are assigned for training. MS-COCO \cite{coco} includes 123, 287 images with five associated captions each. We assign 113, 287 image-text pairs for model training, 5K pairs for validation, and the rest for testing. Both results are reported in our experiments by averaging over 5 folds of 1K test pairs and on the whole 5K test pairs. Conceptual Captions (CC) \cite{cc} is a web-crawled large-scale dataset automatically sourced from the Internet, which inadvertently contains about 3\%$\sim$20\% mismatched or weakly-matched pairs, i.e., noisy correspondence. In our experiments, CC152K, a subset of CC, is utilized for model evaluation, which comprises 1K image-text pairs designated for validation, 1K pairs for testing, and the remaining 150K pairs for training.
% with specific descriptions are provided as follows:
% \begin{itemize}
%     \item \textbf{Flickr30K} \cite{flickr} contains 31K images with five textual descriptions, collected from the Flickr website. We split 1K image-text pairs for validation, 1K pairs for testing, and the rest are assigned for training.
%     \item \textbf{MS-COCO} \cite{coco} includes 123, 287 images with five associated captions each. We assign 113, 287 image-text pairs for model training, 5K pairs for validation, and the rest for testing. Both results are reported in our experiments by averaging over 5 folds of 1K test pairs and on the whole 5K test pairs.
%     \item \textbf{Conceptual Captions} (CC) \cite{cc} is a web-crawled large-scale dataset automatically sourced from the Internet, which inadvertently contains about 3\%$\sim$20\% mismatched or weakly-matched pairs, i.e., real-world noisy correspondence. In our experiments, CC152K, a subset of CC, is utilized for model evaluation, which comprises 1K image-text pairs designated for validation, 1K pairs for testing, and the remaining 150K pairs for training.
% \end{itemize}

\paragraph{Evaluation Protocols.}
Recall at K (R@K) is a widely-used metric to measure the retrieval performance, defined as the percentage of matched items successfully retrieved from the top K candidates~\cite{PAMI2021}. In our experiments, the R@1, R@5, R@10, and the sum of three recalls for image-to-text and text-to-image retrieval are all reported to provide a comprehensive performance evaluation for our method.

\begin{table}
    \centering
    \caption{Comparisons with real-world NCs on CC152K. The \textbf{Best} and \underline{second-best} results are respectively marked in each column.}
    \label{tab:real_world}
    \setlength{\tabcolsep}{0.98mm}
    \begin{tabular}{c|ccccccc}
    \toprule
        \multirow{2}{*}{Methods} & \multicolumn{3}{c}{Image to Text} & \multicolumn{3}{c}{Text to Image} & \\
         \cline{2-8}
        & R@1 & R@5 &R@10 & R@1 & R@5 &R@10 & rSum \\
       \bottomrule
       SCAN & 30.5 & 55.3 & 65.3 & 26.9 & 53.0 & 64.7 & 295.7 \\
       NCR & 39.5 & 64.5 & 73.5 & 40.3 & 64.6 & 73.2 & 355.6 \\
       DECL & 39.0 & 66.1 & 75.5 & 40.7 & 66.3 & 76.7 & 364.3 \\
       MSCN & 40.1 & 65.7 & 76.6 & 40.6 & 67.4 & 76.3 & 366.7 \\
       BiCro & 40.8 & 67.2 & 76.1 & 42.1 & 67.6 & 76.4 & 370.2 \\
       RCL & 41.7 & 66.0 & 73.6 & 41.6 & 66.4 & 75.1 & 364.4 \\
       CRCL & 41.8 & 67.4 & 76.5 & 41.6 & 68.0 & \textbf{78.4} & 373.7 \\
       SREM & 40.9 & 67.5 & 77.1 & 41.5 & \underline{68.2} & 77.0 & 372.2 \\
       PC$^2$ & 39.3 & 66.4 & 75.4 & 39.8 & 66.4 & 76.8 & 364.1 \\
       L2RM & \underline{43.0} & 67.5 & 75.7 & 42.8 & 68.0 & 77.2 & 374.2 \\
       ESC & 42.8 & 67.3 & 76.9 & \underline{44.8} & \underline{68.2} & 75.9 & \underline{375.9} \\
       GSC & 42.1 & \underline{68.4} & \underline{77.7} & 42.2 & 67.6 & 77.1 & 375.1 \\
       \textbf{ReCon} & \textbf{43.1} & \textbf{68.7} & \textbf{78.1} & \textbf{44.9} & \textbf{68.3} & \underline{77.4} & \textbf{380.5} \\
       \bottomrule
    \end{tabular}
\end{table}

\subsection{Implementation Details}
For fair comparisons, all experiments are conducted using the same backbone SGRAF \cite{sgraf} and all experimental settings are consistent with NCR \cite{ncr}, except for the specific parameters of ReCon. Specifically, the batch size $N_{b}$ is set to 128 and the temperature coefficients $\tau$ is 0.1. The division thresholds $\omega_{1}$ and $\omega_{2}$ are both set to 0.5, the scale parameter $\alpha$ for penalization factor is set to 0.1, and the momentum coefficient $\beta$ is 0.6. Moreover, the fixed margin $\gamma$ is set to 0.2 and the balance factor $\xi$ is 5. Before training models, we conduct a $\eta=5$ epochs warmup process for initial convergence. Besides, all experiments are conducted without any additional preprocessing or the use of external data sources.

%Due to the space limitation, more analysis and implementation details are given in our \textit{supplementary material}.

\begin{table*}
    \centering
    \setlength{\tabcolsep}{0.9mm}
    \caption{Cross-modal retrieval performance comparison under synthetic noise rates of 20\%, 40\%, and 60\% on Flickr30K and MS-COCO 1K. The best and the second best results are respectively marked by \textbf{bold} and \underline{underline}.}
    \label{tab:simulated}
    \begin{tabular}{c|c|ccc|ccc|c|ccc|ccc|c}
        \toprule
        \multirow{3}{*}{\makecell[c]{Noise \\ Ratio}} & \multirow{3}{*}{Methods} & \multicolumn{7}{|c}{Flickr30K} & \multicolumn{7}{|c}{MS-COCO 1K} \\
        & & \multicolumn{3}{|c|}{Image to Text} & \multicolumn{3}{|c|}{Text to Image} & & \multicolumn{3}{|c|}{Image to Text} & \multicolumn{3}{|c|}{Text to Image} & \\
        \cline{3-16}
         &  & R@1 & R@5 & R@10 & R@1 & R@5 & R@10 & rSum & R@1 & R@5 & R@10 & R@1 & R@5 & R@10 & rSum\\
        \bottomrule
        \multirow{13}{*}{20\%} 
        & SCAN (ECCV'18) & 59.1 & 83.4 & 90.4 & 36.6 & 67.0 & 77.5 & 414.0 & 66.2 & 91.0 & 96.4 & 45.0 & 80.2 & 89.3 & 468.1 \\
        & NCR (NIPS'21) & 73.5 & 93.2 & 96.6 & 56.9 & 82.4 & 88.5 & 491.1 & 76.6 & 95.6 & 98.2 & 60.8 & 88.8 & 95.0 & 515.0 \\
        & DECL (ACM MM'22) & 77.5 & 93.8 & 97.0 & 56.1 & 81.8 & 88.5 & 494.7 & 77.5 & 95.9 & 98.4 & 61.7 & 89.3 & 95.4 & 518.2 \\
        & MSCN (CVPR'23) & 77.4 & 94.9 & 97.6 & 59.6 & 83.2 & 89.2 & 501.9 & 78.1 & \textbf{97.2} & 98.8 & 64.3 & 90.4 & 95.8 & 524.6 \\
        & BiCro (CVPR'23) & 78.1 & 94.4 & 97.5 & 60.4 & 84.4 & 89.9 & 504.7 & 78.8 & 96.1 & 98.6 & 63.7 & 90.3 & 95.7 & 523.2 \\
        & RCL (TPAMI'23) & 75.9 & 94.5 & 97.3 & 57.9 & 82.6 & 88.6 & 496.8 & 78.9 & 96.0 & 98.4 & 62.8 & 89.9 & 95.4 & 521.4 \\
        & CRCL (NIPS'23) & 78.9 & 94.8 & \textbf{97.9} & 58.7 & 83.0 & 89.2 & 502.5 & 77.8 & 96.1 & 98.5 & 63.4 & 90.3 & \underline{95.9} & 522.0 \\
        & SREM (AAAI'24) & \underline{79.5} & 94.2 & \textbf{97.9} & \underline{61.2} & \underline{84.8} & 90.2 & \underline{507.8} & 78.5 & 96.8 & 98.8 & 63.8 & 90.4 & 95.8 & 524.1 \\
        & PC$^2$ (ACM MM'24) & 78.7 & 94.9 & 96.9 & 59.8 & 83.9 & 89.6 & 503.8 & 77.8 & 95.7 & 98.4 & 62.8 & 89.7 & 95.3 & 519.7 \\
        & L2RM (CVPR'24) & 77.9 & \underline{95.2} & \underline{97.8} & 59.8 & 83.6 & 89.5 & 503.8 & \underline{80.2} & 96.3 & 98.5 & 64.2 & 90.1 & 95.4 & 524.7 \\
        & ESC (CVPR'24) & 79.0 & 94.8 & 97.5 & 59.1 & 83.8 & 89.1 & 503.3 & 79.2 & \underline{97.0} & \textbf{99.1} & \underline{64.8} & \underline{90.7} & \textbf{96.0} & \underline{526.8} \\
        & GSC (CVPR'24) & 78.3 & 94.6 & \underline{97.8} & 60.1 & 84.5 & \underline{90.5} & 505.8 & 79.5 & 96.4 & \underline{98.9} & 64.4 & 90.6 & \underline{95.9} & 525.7 \\
        
        & \textbf{ReCon} & \textbf{80.3} & \textbf{95.3} & \underline{97.8} & \textbf{61.6} & \textbf{85.5} & \textbf{91.3} & \textbf{511.8} & \textbf{80.9} & 96.6 & 98.8 & \textbf{65.2} & \textbf{91.0} & \textbf{96.0} & \textbf{528.6} \\
        \bottomrule
        
        \multirow{13}{*}{40\%}
        & SCAN (ECCV'18) & 29.9 & 60.5 & 72.5 & 16.4 & 38.5 & 48.6 & 266.4 & 30.1 & 65.2 & 79.2 & 18.9 & 51.1 & 69.9 & 314.4 \\
        & NCR (NIPS'21) & 75.3 & 92.1 & 95.2 & 56.2 & 80.6 & 87.4 & 486.8 & 76.5 & 95.0 & 98.2 & 60.7 & 88.5 & 95.0 & 513.9 \\
        & DECL (ACM MM'22) & 72.7 & 92.3 & 95.4 & 53.4 & 79.4 & 86.4 & 479.6 & 75.6 & 95.5 & 98.3 & 59.5 & 88.3 & 94.8 & 512.0 \\
        & MSCN (CVPR'23) & 74.4 & \textbf{94.4} & \underline{96.9} & 57.2 & 81.7 & 87.6 & 492.2 & 74.8 & 94.9 & 98.0 & 60.3 & 88.5 & 94.4 & 510.9 \\
        & BiCro (CVPR'23) & 74.6 & 92.7 & 96.2 & 55.5 & 81.1 & 87.4 & 487.5 & 77.0 & 95.9 & 98.3 & 61.8 & 89.2 & 94.9 & 517.1 \\
        & RCL (TPAMI'23) & 72.7 & 92.7 & 96.1 & 54.8 & 80.0 & 87.1 & 483.4 & 77.0 & 95.5 & 98.3 & 61.2 & 88.5 & 94.8 & 515.3 \\
        & CRCL (NIPS'23) & 74.1 & 92.6 & \underline{96.9} & 55.5 & 80.9 & 87.6 & 487.6 & 76.6 & 95.6 & 98.5 & 62.3 & 89.7 & \underline{95.4} & 518.1 \\
        & SREM (AAAI'24) & \underline{76.5} & 93.9 & 96.3 & \underline{57.5} & \underline{82.7} & 88.5 & 495.4 & 77.2 & 96.0 & 98.5 & 62.1 & 89.3 & 95.3 & 518.4 \\
        & PC$^2$ (ACM MM'24) & 75.8 & 93.5 & \underline{96.9} & \underline{57.5} & 81.9 & 88.2 & 493.8 & 77.4 & 95.8 & 98.4 & 62.1 & 89.4 & 95.1 & 518.2 \\
        & L2RM (CVPR'24) & 75.8 & 93.2 & \underline{96.9} & 56.3 & 81.0 & 87.3 & 490.5 & 77.5 & 95.8 & 98.4 & 62.0 & 89.1 & 94.9 & 517.7\\
        & ESC (CVPR'24) & 76.1 & 93.1 & 96.4 & 56.0 & 80.8 & 87.2 & 489.6 & \underline{78.6} & \textbf{96.6} & \textbf{99.0} & \underline{63.2} & \textbf{90.6} & \textbf{95.9} & \underline{523.9} \\
        & GSC (CVPR'24) & \underline{76.5} & 94.1 & \textbf{97.6} & \underline{57.5} & \underline{82.7} & \underline{88.9} & \underline{497.3} & 78.2 & 95.9 & 98.2 & 62.5 & 89.7 & \underline{95.4} & 519.9 \\
        
        & \textbf{ReCon} & \textbf{79.4} & \underline{94.3} & \textbf{97.6} & \textbf{59.9} & \textbf{83.9} & \textbf{90.1} & \textbf{505.2} & \textbf{79.9} & \underline{96.2} & \underline{98.6} & \textbf{63.5} & \underline{90.5} & \textbf{95.9} & \textbf{524.5} \\
        \bottomrule
        
        \multirow{13}{*}{60\%}
        & SCAN (ECCV'18) & 16.9 & 39.3 & 53.9 & 2.8 & 7.4 & 11.4 & 131.7 & 27.8 & 59.8 & 74.8 & 16.8 & 47.8 & 66.4 & 293.4 \\
        & NCR (NIPS'21) & 68.7 & 89.9 & 95.5 & 52.0 & 77.6 & 84.9 & 468.6 & 72.7 & 94.0 & 97.6 & 57.9 & 87.0 & 94.1 & 503.3 \\
        & DECL (ACM MM'22) & 65.2 & 88.4 & 94.0 & 46.8 & 74.0 & 82.2 & 450.6 & 73.0 & 94.2 & 97.9 & 57.0 & 86.6 & 93.8 & 502.5 \\
        & MSCN (CVPR'23) & 70.4 & 91.0 & 94.9 & 53.4 & 77.8 & 84.1 & 471.6 & 74.4 & \underline{95.1} & 97.9 & 59.2 & 87.1 & 92.8 & 506.5 \\
        & BiCro (CVPR'23) & 67.6 & 90.8 & 94.4 & 51.2 & 77.6 & 84.7 & 466.3 & 73.9 & 94.4 & 97.8 & 58.3 & 87.2 & 93.9 & 505.5 \\
        & RCL (TPAMI'23) & 67.7 & 89.1 & 93.6 & 48.0 & 74.9 & 83.3 & 456.6 & 74.0 & 94.3 & 97.5 & 57.6 & 86.4 & 93.5 & 503.3 \\
        & CRCL (NIPS'23) & 70.4 & 90.4 & 94.9 & 52.6 & 78.1 & 85.1 & 471.5 & 75.2 & 94.9 & 98.0 & 60.1 & 88.5 & 94.8 & 511.5 \\
        & SREM (AAAI'24) & 71.0 & \underline{92.1} & \underline{96.1} & \underline{54.0} & \underline{80.1} & \underline{87.0} & \underline{480.3} & 74.5 & 94.5 & 97.9 & 58.7 & 87.5 & 93.9 & 506.9 \\
        & PC$^2$ (ACM MM'24) & 70.8 & 90.3 & 94.4 & 53.1 & 79.0 & 85.9 & 473.5 & 74.2 & 94.4 & 97.8 & 58.9 & 87.5 & 93.8 & 506.6 \\
        & L2RM (CVPR'24) & 70.0 & 90.8 & 95.4 & 51.3 & 76.4 & 83.7 & 467.6 & 75.4 & 94.7 & 97.9 & 59.2 & 87.4 & 93.8 & 508.4 \\
        & ESC (CVPR'24) & \underline{72.6} & 90.9 & 94.6 & 53.0 & 78.6 & 85.3 & 475.0 & \textbf{77.2} & \underline{95.1} & \underline{98.1} & \underline{61.1} & \underline{88.6} & \underline{94.9} & \underline{515.0} \\
        & GSC (CVPR'24) & 70.8 & 91.1 & 95.9 & 53.6 & 79.8 & 86.8 & 478.0 & \underline{75.6} & \underline{95.1} & 98.0 & 60.0 & 88.3 & 94.6 & 511.7 \\

        & \textbf{ReCon} & \textbf{74.3} & \textbf{93.6} & \textbf{96.6} & \textbf{55.7} & \textbf{81.6} & \textbf{88.1} & \textbf{489.9} & \textbf{77.2} & \textbf{95.9} & \textbf{98.4} & \textbf{61.8} & \textbf{89.3} & \textbf{95.2} & \textbf{517.8} \\
        \bottomrule
    \end{tabular}
\end{table*}

\subsection{Comparison with State-of-the-Arts}
% the results reported in this paper strictly reference those reported in the corresponding previous studies, and for cases where results were not reported, we retrained the models following the official recommended configurations. Furthermore, 
In this section, we carry out a comprehensive evaluation to present the effectiveness of ReCon, benchmarking it against SOTA baselines across three widely-used datasets above. The baselines comprise SCAN \cite{scan}, NCR \cite{ncr}, DECL \cite{decl}, MSCN \cite{MSCN}, BiCro \cite{BiCro}, RCL \cite{rcl}, CRCL \cite{crcl}, SREM \cite{srem}, PC$^2$ \cite{PC2}, L2RM \cite{l2rm}, ESC \cite{esc} and GSC \cite{gsc}. For the well-established Flickr30K and MS-COCO, the simulated NCs with varying noise rates, namely 20\%, 40\%, and 60\%, obtained by randomly shuffling the captions like \cite{ncr} are exploited to assess the robustness of ReCon. In addition to simulated NCs, we also validate the performance of ReCon with real-world noisy conditions using the web-crawled CC152K naturally containing 3\% $\sim$ 20\% unknown NCs. Note that the presented results of ReCon on the testing set are obtained through the checkpoints that achieved optimal performance on the validation set.

\textbf{Results on Simulated NCs.}
For quantitative evaluation the performance and robustness of all baselines under different noise ratios, we conduct all tested baselines on the Flickr30K and MS-COCO 1K with 20\%, 40\%, and 60\% of simulated noisy correspondence, where the results of MS-COCO are averaged on 5 folds of 1K test pairs as in previous works \cite{ncr, crcl}. The details are recorded in Table \ref{tab:simulated}, which demonstrates that our ReCon remarkably outperforms other baselines by a large margin on most of metrics. Notably, ReCon gains the highest R@1 score for both image-to-text and text-to-image retrieval across all noise rates on these two datasets, indicating that our method has significant potential to effectively deal with NCs. This promising performance can be attributed to the accurate identification for true positives, which avoids the misleading of wrongly introduced mismatches and enhances the discrimination between matched and mismatched pairs, thus achieving further performance improvement. Besides, ReCon performs competitive performance than other baselines under severely noise, proving its stability and reliability to facilitate robust cross-modal retrieval.

% \begin{table}
%     \centering
%     \caption{Comparisons with real-world NCs on CC152K. The \textbf{Best} and \underline{second-best} results are respectively marked in each column.}
%     \label{tab:real_world}
%     \setlength{\tabcolsep}{0.98mm}
%     \begin{tabular}{c|ccc|ccc|c}
%     \toprule
%         \multirow{2}{*}{Methods} & \multicolumn{3}{c|}{Image to Text} & \multicolumn{3}{c|}{Text to Image} & \\
%          \cline{2-8}
%         & R@1 & R@5 &R@10 & R@1 & R@5 &R@10 & rSum \\
%        \bottomrule
%        SCAN & 30.5 & 55.3 & 65.3 & 26.9 & 53.0 & 64.7 & 295.7 \\
%        NCR & 39.5 & 64.5 & 73.5 & 40.3 & 64.6 & 73.2 & 355.6 \\
%        DECL & 39.0 & 66.1 & 75.5 & 40.7 & 66.3 & 76.7 & 364.3 \\
%        MSCN & 40.1 & 65.7 & 76.6 & 40.6 & 67.4 & 76.3 & 366.7 \\
%        BiCro & 40.8 & 67.2 & 76.1 & 42.1 & 67.6 & 76.4 & 370.2 \\
%        RCL & 41.7 & 66.0 & 73.6 & 41.6 & 66.4 & 75.1 & 364.4 \\
%        CRCL & 41.8 & 67.4 & 76.5 & 41.6 & 68.0 & \textbf{78.4} & 373.7 \\
%        SREM & 40.9 & 67.5 & 77.1 & 41.5 & \underline{68.2} & 77.0 & 372.2 \\
%        PC$^2$ & 39.3 & 66.4 & 75.4 & 39.8 & 66.4 & 76.8 & 364.1 \\
%        L2RM & \underline{43.0} & 67.5 & 75.7 & 42.8 & 68.0 & 77.2 & 374.2 \\
%        ESC & 42.8 & 67.3 & 76.9 & \underline{44.8} & \underline{68.2} & 75.9 & \underline{375.9} \\
%        GSC & 42.1 & \underline{68.4} & \underline{77.7} & 42.2 & 67.6 & 77.1 & 375.1 \\
%        \textbf{ReCon} & \textbf{43.1} & \textbf{68.7} & \textbf{78.1} & \textbf{44.9} & \textbf{68.3} & \underline{77.4} & \textbf{380.5} \\
%        \bottomrule
%     \end{tabular}
% \end{table}

\begin{table}
    \centering
    \caption{Performance comparison with CLIP on MS-COCO 5K. The \textbf{best} results are highlighted in \textbf{bold}.}
    \label{clip}
      \setlength{\tabcolsep}{0.5mm}
    \begin{tabular}{c|c|ccccccc}
    \toprule
        \multirow{2}{*}{Noise} & \multirow{2}{*}{Methods} & \multicolumn{3}{c}{Image to Text} & \multicolumn{3}{c}{Text to Image} & \\
    \cline{3-9}
         &  & R@1 & R@5 &R@10 & R@1 & R@5 &R@10 & rSum \\
    \midrule
        \multirow{3}{*}{0\%} & CLIP-14 & 58.4 & 81.5 & 88.1 & 37.8 & 62.4 & 72.2 & 400.4 \\
         & CLIP-32 & 50.2 & 74.6 & 83.6 & 30.4 & 56.0 & 66.8 & 361.6 \\
         & \textbf{ReCon} & \textbf{61.6} & \textbf{86.7} & \textbf{92.7} & \textbf{44.4} & \textbf{73.1} & \textbf{83.1} & \textbf{441.6} \\
    \midrule
        \multirow{3}{*}{20\%} & CLIP-14 & 36.1 & 61.3 & 72.5 & 22.6 & 43.2 & 53.7 & 289.4 \\
         & CLIP-32 & 21.4 & 49.6 & 63.3 & 14.8 & 37.6 & 49.6 & 236.3 \\
         & \textbf{ReCon} & \textbf{61.1} & \textbf{85.7} & \textbf{92.2} & \textbf{43.5} & \textbf{72.4} & \textbf{82.7} & \textbf{437.6} \\
    \midrule
        \multirow{2}{*}{50\%} & CLIP-32 & 10.9 & 27.8 & 38.3 & 7.8 & 19.5 & 26.8 & 131.1 \\
         & \textbf{ReCon} & \textbf{58.1} & \textbf{85.1} & \textbf{91.9} & \textbf{41.5} & \textbf{70.7} & \textbf{81.0} & \textbf{428.3} \\
    \bottomrule
    \end{tabular}
\end{table}



\begin{figure}
    \centering
        \begin{minipage}{0.25\textwidth}
            \centering
            \includegraphics[width=\textwidth]{figures/omega.pdf}
        \end{minipage}
        \begin{minipage}{0.21\textwidth}
            \centering
            \includegraphics[width=\textwidth]{figures/xi.pdf}
        \end{minipage}
    \caption{Performance under different hyper-parameters of ReCon on Flickr30K with 40\% NCs.}
    \label{fig:hyper}
\end{figure}

\textbf{Results on Real-World NCs.}
% We directly train and evaluate ReCon on CC152K without any additional simulated noise injection. 
For substantiating the comprehensive performance assessment, we also provide the quantitative results that evaluated on CC152K containing real-world NCs, which better mirrors real-world industry scenarios. According to the results shown in Table \ref{tab:real_world}, it can be observed that ReCon outperforms the baselines by a considerable margin with the overall score 4.4\% performance improvement compared to the second-best ESC of 375.9\%. Besides, ReCon exhibits competitive performance across all metrics, consistently indicating its robustness and effectiveness in handling real-world NCs.

\textbf{Comparison to Pre-trained Model.}
% In detail, in zero-shot, the released pre-trained models of CLIP are directly employed to perform inference. In fine-tune, we first fine-tune the released CLIP model and perform the inference on the testing set. 
To further present the superiority and necessity of ReCon, we perform comparisons to the large pre-trained vision-language model, i.e., CLIP \cite{clip}, which is a powerful baseline trained on massive image-text pairs collected from the Internet with a large number of real NCs. In line with \cite{ncr}, we compare our ReCon to the CLIP on MS-COCO dataset under the following two settings: zero-shot and fine-tune, and the two baselines: CLIP-14 (ViT-L/14) and CLIP-32 (ViT-B/32). From the results shown in Table \ref{clip}, the significant performance degradation of CLIP can be attribute to the lack of effective mechanism to handle noisy correspondence. In contrast, the performance of ReCon under 50\% noise even surpasses the zero-shot results achieved of CLIP, indicating the effectiveness and necessity of our ReCon.

% \begin{table}
%     \centering
%     \caption{Performance comparison with CLIP on MS-COCO 5K. The \textbf{best} results are highlighted in bold.}
%     \label{clip}
%       \setlength{\tabcolsep}{0.5mm}
%     \begin{tabular}{c|c|cccccc|c}
%     \toprule
%         \multirow{2}{*}{Noise} & \multirow{2}{*}{Methods} & \multicolumn{3}{c}{Image to Text} & \multicolumn{3}{c}{Text to Image} & \\
%     \cline{3-9}
%          &  & R@1 & R@5 &R@10 & R@1 & R@5 &R@10 & rSum \\
%     \midrule
%         \multirow{4}{*}{0\%} & CLIP-14 & 58.4 & 81.5 & 88.1 & 37.8 & 62.4 & 72.2 & 400.4 \\
%          & CLIP-32 & 50.2 & 74.6 & 83.6 & 30.4 & 56.0 & 66.8 & 361.6 \\
%          & NCR & 58.2 & 84.2 & 91.5 & 41.7 & 71.0 & 81.3 & 427.9 \\
%          & \textbf{ReCon} & \textbf{61.6} & \textbf{86.7} & \textbf{92.7} & \textbf{44.4} & \textbf{73.1} & \textbf{83.1} & \textbf{441.6} \\
%     \midrule
%         \multirow{4}{*}{20\%} & CLIP-14 & 36.1 & 61.3 & 72.5 & 22.6 & 43.2 & 53.7 & 289.4 \\
%          & CLIP-32 & 21.4 & 49.6 & 63.3 & 14.8 & 37.6 & 49.6 & 236.3 \\
%          & NCR & 56.9 & 83.6 & 91.0 & 40.6 & 69.8 & 80.1 & 422.0 \\
%          & \textbf{ReCon} & \textbf{61.1} & \textbf{85.7} & \textbf{92.2} & \textbf{43.5} & \textbf{72.4} & \textbf{82.7} & \textbf{437.6} \\
%     \midrule
%         \multirow{3}{*}{50\%} & CLIP-32 & 10.9 & 27.8 & 38.3 & 7.8 & 19.5 & 26.8 & 131.1 \\
%         & NCR & 53.1 & 80.7 & 88.5 & 37.9 & 66.6 & 77.8 & 404.6 \\
%          & \textbf{ReCon} & \textbf{58.1} & \textbf{85.1} & \textbf{91.9} & \textbf{41.5} & \textbf{70.7} & \textbf{81.0} & \textbf{428.3} \\
%     \bottomrule
%     \end{tabular}
% \end{table}

\begin{table}
    \centering
      \caption{Ablation studies on Flick30K with 40\% noise with different components in ReCon. The \textbf{best} results are marked in \textbf{bold}.}
    \label{tab:ablation}
    \setlength{\tabcolsep}{0.8mm}
    \begin{tabular}{ccc|ccccccc}
    \toprule
        \multicolumn{3}{c}{Components} & \multicolumn{3}{|c}{Image to Text} & \multicolumn{3}{|c}{Text to Image} &  \\
        \midrule
        Tru. & $\mathcal{L}_{IM}$ & $\lambda$ & R@1 & R@5 &R@10 & R@1 & R@5 &R@10 & rSum \\
        \midrule
        \checkmark & \checkmark & \checkmark & \textbf{79.4} & \textbf{94.3} & \textbf{97.6} & \textbf{59.9} & \textbf{83.9} & \textbf{90.1} & \textbf{505.2} \\
        \checkmark & \checkmark &  & 77.3 & 94.1 & 97.3 & 58.7 & 83.3 & 89.5 & 500.2 \\
        & \checkmark & \checkmark & 77.2 & \textbf{94.3} & 97.2 & 57.9 & 83.1 & 89.3 & 499.1 \\
        & \checkmark &  & 77.0 & 94.1 & 97.0 & 57.6 & 82.8 & 89.0 & 497.5 \\
        \checkmark & & & 74.1 & 93.2 & 96.7 & 57.4 & 83.1 & 88.9 & 493.3 \\
         \bottomrule
    \end{tabular}
\end{table}

\begin{figure}
    \centering
    \includegraphics[width=1.0\linewidth]{figures/case.pdf}
    \caption{Examples of detected mismatched pairs on Flickr30K.}
    \label{fig:case}
\end{figure}

\subsection{Ablation Study}
\textbf{Impact of components.}
% For $\mathcal{L}_{IM}$, we just divide the training data. 
We conducted ablation studies on the Flickr30K with 40\% noise to validate the individual contributions of each component within ReCon, as detailed in Table \ref{tab:ablation}. For the true correspondence discrimination, all pairs are divided into clean and noisy partitions based on the $\mathcal{L}_{CM}$, and the intra-modal relation $\mathcal{L}_{IM}$ is directly employed to the clean partition. From the table, ReCon achieves the optimal performance by integrating all these components. This substantial improvement not only confirms the effectiveness of each individual component but also indicates their collective contributions in enhancing the robustness of models to address noisy correspondence.
\textbf{Impact of hyper-parameters.}
Fig. \ref{fig:hyper} shows the effects of the main hyper-parameters including division thresholds and balance factor. From the results, ReCon obtains better performance with $\omega_{1},\omega_{2} \in [0.4,0.6]$ and the $\xi \in [3,7]$.
\textbf{Detected noisy correspondences.}
Fig. \ref{fig:case} visualizes some detected mismatched pairs on Flickr30K by ReCon. These pairs exhibit high matching probabilities with local correspondences, yet are correctly identified as mismatched pairs due to their inconsistencies of intra-modal relations.

\section{Conclusion}
This paper introduces a general \textbf{Re}lation \textbf{Con}sistency learning framework, namely \textbf{ReCon}, to effectively mitigate the adverse impact caused by NCs. The main motivation of our ReCon is to \textit{enhance the discriminability of models for true correspondences in noisy multimodal dataset} and thus effectively avoids the wrong supervisions of false correspondences, especially in the presence of hard NCs. Specifically, we leverage the dual constrains, which simultaneously consider the cross- and intra-modal relations, to jointly divide the corrupted training data into different partitions. Extensive experiments conducted on three widely-used cross-modal benchmarks validate the effectiveness and robustness of ReCon in handling both simulated and real-world NCs.

{
    \small
    \bibliographystyle{ieeenat_fullname}
    \bibliography{main}
}

% WARNING: do not forget to delete the supplementary pages from your submission 
% \clearpage
\pagenumbering{gobble}
\maketitlesupplementary

\section{Additional Results on Embodied Tasks}

To evaluate the broader applicability of our EgoAgent's learned representation beyond video-conditioned 3D human motion prediction, we test its ability to improve visual policy learning for embodiments other than the human skeleton.
Following the methodology in~\cite{majumdar2023we}, we conduct experiments on the TriFinger benchmark~\cite{wuthrich2020trifinger}, which involves a three-finger robot performing two tasks: reach cube and move cube. 
We freeze the pretrained representations and use a 3-layer MLP as the policy network, training each task with 100 demonstrations.

\begin{table}[h]
\centering
\caption{Success rate (\%) on the TriFinger benchmark, where each model's pretrained representation is fixed, and additional linear layers are trained as the policy network.}
\label{tab:trifinger}
\resizebox{\linewidth}{!}{%
\begin{tabular}{llcc}
\toprule
Methods       & Training Dataset & Reach Cube & Move Cube \\
\midrule
DINO~\cite{caron2021emerging}         & WT Venice        & 78.03     & 47.42     \\
DoRA~\cite{venkataramanan2023imagenet}          & WT Venice        & 81.62     & 53.76     \\
DoRA~\cite{venkataramanan2023imagenet}          & WT All           & 82.40     & 48.13     \\
\midrule
EgoAgent-300M & WT+Ego-Exo4D      & 82.61    & 54.21      \\
EgoAgent-1B   & WT+Ego-Exo4D      & \textbf{85.72}      & \textbf{57.66}   \\
\bottomrule
\end{tabular}%
}
\end{table}

As shown in Table~\ref{tab:trifinger}, EgoAgent achieves the highest success rates on both tasks, outperforming the best models from DoRA~\cite{venkataramanan2023imagenet} with increases of +3.32\% and +3.9\% respectively.
This result shows that by incorporating human action prediction into the learning process, EgoAgent demonstrates the ability to learn more effective representations that benefit both image classification and embodied manipulation tasks.
This highlights the potential of leveraging human-centric motion data to bridge the gap between visual understanding and actionable policy learning.



\section{Additional Results on Egocentric Future State Prediction}

In this section, we provide additional qualitative results on the egocentric future state prediction task. Additionally, we describe our approach to finetune video diffusion model on the Ego-Exo4D dataset~\cite{grauman2024ego} and generate future video frames conditioned on initial frames as shown in Figure~\ref{fig:opensora_finetune}.

\begin{figure}[b]
    \centering
    \includegraphics[width=\linewidth]{figures/opensora_finetune.pdf}
    \caption{Comparison of OpenSora V1.1 first-frame-conditioned video generation results before and after finetuning on Ego-Exo4D. Fine-tuning enhances temporal consistency, but the predicted pixel-space future states still exhibit errors, such as inaccuracies in the basketball's trajectory.}
    \label{fig:opensora_finetune}
\end{figure}

\subsection{Visualizations and Comparisons}

More visualizations of our method, DoRA, and OpenSora in different scenes (as shown in Figure~\ref{fig:supp pred}). For OpenSora, when predicting the states of $t_k$, we use all the ground truth frames from $t_{0}$ to $t_{k-1}$ as conditions. As OpenSora takes only past observations as input and neglects human motion, it performs well only when the human has relatively small motions (see top cases in Figure~\ref{fig:supp pred}), but can not adjust to large movements of the human body or quick viewpoint changes (see bottom cases in Figure~\ref{fig:supp pred}).

\begin{figure*}
    \centering
    \includegraphics[width=\linewidth]{figures/supp_pred.pdf}
    \caption{Retrieval and generation results for egocentric future state prediction. Correct and wrong retrieval images are marked with green and red boundaries, respectively.}
    \label{fig:supp pred}
\end{figure*}

\begin{figure*}[t]
    \centering
    \includegraphics[width=0.9\linewidth]{figures/motion_prediction.pdf}
    \vspace{-0.5mm}
    \caption{Motion prediction results in scenes with minor changes in observation.}
    \vspace{-1.5mm}
    \label{fig:motion_prediction}
\end{figure*}

\subsection{Finetuning OpenSora on Ego-Exo4D}

OpenSora V1.1~\cite{opensora}, initially trained on internet videos and images, produces severely inconsistent results when directly applied to infer future videos on the Ego-Exo4D dataset, as illustrated in Figure~\ref{fig:opensora_finetune}.
To address the gap between general internet content and egocentric video data, we fine-tune the official checkpoint on the Ego-Exo4D training set for 50 epochs.
OpenSora V1.1 proposed a random mask strategy during training to enable video generation by image and video conditioning. We adopted the default masking rate, which applies: 75\% with no masking, 2.5\% with random masking of 1 frame to 1/4 of the total frames, 2.5\% with masking at either the beginning or the end for 1 frame to 1/4 of the total frames, and 5\% with random masking spanning 1 frame to 1/4 of the total frames at both the beginning and the end.

As shown in Fig.~\ref{fig:opensora_finetune}, despite being trained on a large dataset, OpenSora struggles to generalize to the Ego-Exo4D dataset, producing future video frames with minimal consistency relative to the conditioning frame. While fine-tuning improves temporal consistency, the moving trajectories of objects like the basketball and soccer ball still deviate from realistic physical laws. Compared with our feature space prediction results, this suggests that training world models in a reconstructive latent space is more challenging than training them in a feature space.


\section{Additional Results on 3D Human Motion Prediction}

We present additional qualitative results for the 3D human motion prediction task, highlighting a particularly challenging scenario where egocentric observations exhibit minimal variation. This scenario poses significant difficulties for video-conditioned motion prediction, as the model must effectively capture and interpret subtle changes. As demonstrated in Fig.~\ref{fig:motion_prediction}, EgoAgent successfully generates accurate predictions that closely align with the ground truth motion, showcasing its ability to handle fine-grained temporal dynamics and nuanced contextual cues.

\section{OpenSora for Image Classification}

In this section, we detail the process of extracting features from OpenSora V1.1~\cite{opensora} (without fine-tuning) for an image classification task. Following the approach of~\cite{xiang2023denoising}, we leverage the insight that diffusion models can be interpreted as multi-level denoising autoencoders. These models inherently learn linearly separable representations within their intermediate layers, without relying on auxiliary encoders. The quality of the extracted features depends on both the layer depth and the noise level applied during extraction.


\begin{table}[h]
\centering
\caption{$k$-NN evaluation results of OpenSora V1.1 features from different layer depths and noising scales on ImageNet-100. Top1 and Top5 accuracy (\%) are reported.}
\label{tab:opensora-knn}
\resizebox{0.95\linewidth}{!}{%
\begin{tabular}{lcccccc}
\toprule
\multirow{2}{*}{Timesteps} & \multicolumn{2}{c}{First Layer} & \multicolumn{2}{c}{Middle Layer} & \multicolumn{2}{c}{Last Layer} \\
\cmidrule(r){2-3}   \cmidrule(r){4-5}  \cmidrule(r){6-7}  & Top1           & Top5           & Top1            & Top5           & Top1           & Top5          \\
\midrule
32        &  6.10           & 18.20             & 34.04               & 59.50             & 30.40             & 55.74             \\
64        & 6.12              & 18.48              & 36.04               & 61.84              & 31.80         & 57.06         \\
128       & 5.84             & 18.14             & 38.08               & 64.16              & 33.44       & 58.42 \\
256       & 5.60             & 16.58              & 30.34               & 56.38              &28.14          & 52.32        \\
512       & 3.66              & 11.70            & 6.24              & 17.62              & 7.24              & 19.44  \\ 
\bottomrule
\end{tabular}%
}
\end{table}

As shown in Table~\ref{tab:opensora-knn}, we first evaluate $k$-NN classification performance on the ImageNet-100 dataset using three intermediate layers and five different noise scales. We find that a noise timestep of 128 yields the best results, with the middle and last layers performing significantly better than the first layer.
We then test this optimal configuration on ImageNet-1K and find that the last layer with 128 noising timesteps achieves the best classification accuracy.

\section{Data Preprocess}
For egocentric video sequences, we utilize videos from the Ego-Exo4D~\cite{grauman2024ego} and WT~\cite{venkataramanan2023imagenet} datasets.
The original resolution of Ego-Exo4D videos is 1408×1408, captured at 30 fps. We sample one frame every five frames and use the original resolution to crop local views (224×224) for computing the self-supervised representation loss. For computing the prediction and action loss, the videos are downsampled to 224×224 resolution.
WT primarily consists of 4K videos (3840×2160) recorded at 60 or 30 fps. Similar to Ego-Exo4D, we use the original resolution and downsample the frame rate to 6 fps for representation loss computation.
As Ego-Exo4D employs fisheye cameras, we undistort the images to a pinhole camera model using the official Project Aria Tools to align them with the WT videos.

For motion sequences, the Ego-Exo4D dataset provides synchronized 3D motion annotations and camera extrinsic parameters for various tasks and scenes. While some annotations are manually labeled, others are automatically generated using 3D motion estimation algorithms from multiple exocentric views. To maximize data utility and maintain high-quality annotations, manual labels are prioritized wherever available, and automated annotations are used only when manual labels are absent.
Each pose is converted into the egocentric camera's coordinate system using transformation matrices derived from the camera extrinsics. These matrices also enable the computation of trajectory vectors for each frame in a sequence. Beyond the x, y, z coordinates, a visibility dimension is appended to account for keypoints invisible to all exocentric views. Finally, a sliding window approach segments sequences into fixed-size windows to serve as input for the model. Note that we do not downsample the frame rate of 3D motions.

\section{Training Details}
\subsection{Architecture Configurations}
In Table~\ref{tab:arch}, we provide detailed architecture configurations for EgoAgent following the scaling-up strategy of InternLM~\cite{team2023internlm}. To ensure the generalization, we do not modify the internal modules in InternML, \emph{i.e.}, we adopt the RMSNorm and 1D RoPE. We show that, without specific modules designed for vision tasks, EgoAgent can perform well on vision and action tasks.

\begin{table}[ht]
  \centering
  \caption{Architecture configurations of EgoAgent.}
  \resizebox{0.8\linewidth}{!}{%
    \begin{tabular}{lcc}
    \toprule
          & EgoAgent-300M & EgoAgent-1B \\
          \midrule
    Depth & 22    & 22 \\
    Embedding dim & 1024  & 2048 \\
    Number of heads & 8     & 16 \\
    MLP ratio &    8/3   & 8/3 \\
    $\#$param.  & 284M & 1.13B \\
    \bottomrule
    \end{tabular}%
    }
  \label{tab:arch}%
\end{table}%

Table~\ref{tab:io_structure} presents the detailed configuration of the embedding and prediction modules in EgoAgent, including the image projector ($\text{Proj}_i$), representation head/state prediction head ($\text{MLP}_i$), action projector ($\text{Proj}_a$) and action prediction head ($\text{MLP}_a$).
Note that the representation head and the state prediction head share the same architecture but have distinct weights.

\begin{table}[t]
\centering
\caption{Architecture of the embedding ($\text{Proj}_i$, $\text{Proj}_a$) and prediction ($\text{MLP}_i$, $\text{MLP}_a$) modules in EgoAgent. For details on module connections and functions, please refer to Fig.~2 in the main paper.}
\label{tab:io_structure}
\resizebox{\linewidth}{!}{%
\begin{tabular}{lcl}
\toprule
       & \multicolumn{1}{c}{Norm \& Activation} & \multicolumn{1}{c}{Output Shape}  \\
\midrule
\multicolumn{3}{l}{$\text{Proj}_i$ (\textit{Image projector})} \\
\midrule
Input image  & -          & 3$\times$224$\times$224 \\
Conv 2D (16$\times$16) & -       & Embedding dim$\times$14$\times$14    \\
\midrule
\multicolumn{3}{l}{$\text{MLP}_i$ (\textit{State prediction head} \& \textit{Representation head)}} \\
\midrule
Input embedding  & -          & Embedding dim \\
Linear & GELU       & 2048          \\
Linear & GELU       & 2048          \\
Linear & -          & 256           \\
Linear & -          & 65536     \\
\midrule
\multicolumn{3}{l}{$\text{Proj}_a$ (\textit{Action projector})} \\
\midrule
Input pose sequence  & -          & 4$\times$5$\times$17 \\
Conv 2D (5$\times$17) & LN, GELU   & Embedding dim$\times$1$\times$1    \\
\midrule
\multicolumn{3}{l}{$\text{MLP}_a$ (\textit{Action prediction head})} \\
\midrule
Input embedding  & -          & Embedding dim$\times$1$\times$1 \\
Linear & -          & 4$\times$5$\times$17     \\
\bottomrule
\end{tabular}%
}
\end{table}


\subsection{Training Configurations}
In Table~\ref{tab:training hyper}, we provide the detailed training hyper-parameters for experiments in the main manuscripts.

\begin{table}[ht]
  \centering
  \caption{Hyper-parameters for training EgoAgent.}
  \resizebox{0.86\linewidth}{!}{%
    \begin{tabular}{lc}
    \toprule
    Training Configuration & EgoAgent-300M/1B \\
    \midrule
    Training recipe: &  \\
    optimizer & AdamW~\cite{loshchilov2017decoupled} \\
    optimizer momentum & $\beta_1=0.9, \beta_2=0.999$ \\
    \midrule
    Learning hyper-parameters: &  \\
    base learning rate & 6.0E-04 \\
    learning rate schedule & cosine \\
    base weight decay & 0.04 \\
    end weight decay & 0.4 \\
    batch size & 1920 \\
    training iters & 72,000 \\
    lr warmup iters & 1,800 \\
    warmup schedule & linear \\
    gradient clip & 1.0 \\
    data type & float16 \\
    norm epsilon & 1.0E-06 \\
    \midrule
    EMA hyper-parameters: &  \\
    momentum & 0.996 \\
    \bottomrule
    \end{tabular}%
    }
  \label{tab:training hyper}%
\end{table}%

\clearpage



\end{document}
