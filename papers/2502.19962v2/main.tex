% CVPR 2025 Paper Template; see https://github.com/cvpr-org/author-kit

\documentclass[10pt,twocolumn,letterpaper]{article}

%%%%%%%%% PAPER TYPE  - PLEASE UPDATE FOR FINAL VERSION
% \usepackage{cvpr}              % To produce the CAMERA-READY version
% \usepackage[review]{cvpr}      % To produce the REVIEW version
\usepackage[pagenumbers]{cvpr} % To force page numbers, e.g. for an arXiv version

% Import additional packages in the preamble file, before hyperref
\newcommand{\CG}{\mathcal{G}\xspace}
\newcommand{\CV}{\mathcal{V}\xspace}
\newcommand{\CE}{\mathcal{E}\xspace}
\newcommand{\CA}{\mathcal{A}\xspace}
\newcommand{\CF}{\mathcal{F}\xspace}
\newcommand{\CR}{\mathcal{R}\xspace}
\newcommand{\CB}{\mathcal{B}\xspace}
\newcommand{\CX}{\mathcal{X}\xspace}
\newcommand{\CK}{\mathcal{K}\xspace}
\newcommand{\CM}{\mathcal{M}\xspace}
\newcommand{\CC}{\mathcal{C}\xspace}
\newcommand{\CL}{\mathcal{L}\xspace}
\newcommand{\CI}{\mathcal{I}\xspace}
\newcommand{\CQ}{\mathcal{Q}\xspace}
\newcommand{\CO}{\mathcal{O}\xspace}
\newcommand{\CP}{\mathcal{P}\xspace}
\newcommand{\CS}{\mathcal{S}\xspace}
\newcommand{\CT}{\mathcal{T}\xspace}
\newcommand{\CJ}{\mathcal{J}\xspace}
\usepackage[para]{footmisc}
\usepackage{subfig}
% \usepackage{subcaption}
% \usepackage{array}
% \usepackage{colortbl}



% It is strongly recommended to use hyperref, especially for the review version.
% hyperref with option pagebackref eases the reviewers' job.
% Please disable hyperref *only* if you encounter grave issues, 
% e.g. with the file validation for the camera-ready version.
%
% If you comment hyperref and then uncomment it, you should delete *.aux before re-running LaTeX.
% (Or just hit 'q' on the first LaTeX run, let it finish, and you should be clear).
\definecolor{cvprblue}{rgb}{0.21,0.49,0.74}
\usepackage[pagebackref,breaklinks,colorlinks,allcolors=cvprblue]{hyperref}

\usepackage{multirow}
\usepackage{multicol}
\usepackage{makecell}
% \usepackage{ulem}

%%%%%%%%% PAPER ID  - PLEASE UPDATE
\def\paperID{11263} % *** Enter the Paper ID here
\def\confName{CVPR}
\def\confYear{2025}

%%%%%%%%% TITLE - PLEASE UPDATE
% \title{\LaTeX\ Author Guidelines for \confName~Proceedings}
% \title{Have Positive Samples Been Fully Exploited? Relation-aware Discrimination for Positive Pair Mining in Noisy Correspondence}
% \title{Preserving Relation Consistency for Enhanced Positive Pair Discrimination in Noisy Correspondence Learning}
\title{ReCon: Enhancing True Correspondence Discrimination through Relation Consistency for Robust Noisy Correspondence Learning}

%%%%%%%%% AUTHORS - PLEASE UPDATE

\author{Quanxing Zha$^1$,~~Xin Liu$^{1,2}$\thanks{Corresponding author},~~~Shu-Juan Peng$^1$,~~Yiu-ming Cheung$^{2}$,~~Xing Xu$^3$,~~Nannan Wang$^4$\\
$^1$Huaqiao University, $^2$Hong Kong Baptist University\\
$^3$University of Electronic Science and Technology of
China, $^4$Xidian University\\
{quanxing.zha@gmail.com, xliu@hqu.edu.cn}
}



%\author{Quanxing Zha\\
%Institution1\\
%Institution1 address\\
%{\tt\small quanxing.zha@gmail.com}
% For a paper whose authors are all at the same institution,
% omit the following lines up until the closing ``}''.
% Additional authors and addresses can be added with ``\and'',
% just like the second author.
% To save space, use either the email address or home page, not both
%\and
%Second Author\\
%Institution2\\
%First line of institution2 address\\
%{\tt\small secondauthor@i2.org}
%}

\begin{document}
\maketitle
% \begin{abstract}

% Recent works to jointly reconstruct 3D human and object from a single RGB image, are mostly model-based, that fail to capture the fine details of the clothed human body and object surface. In this paper, we introduce ReCHOR, a novel, model-free, first-method to produce realistic clothed human-object reconstructions from a monocular view. This is extremely challenging due to human-object occlusions, diverse interactions and depth ambiguity, as it needs to infer both 3D spatial awareness and high resolution details. Our core idea is based on estimating neural implicit representations for human and object respectively by an attention-based neural implicit model that attends to pixel-aligned features from both the global human-object image for spatial awareness and  the local separate view of human and object images for high quality details. Additionally, the network is conditioned on semantic features from an initial estimated human-object pose prior and a generative diffusion model that inpaints occluded regions, thus enabling the retrieval of details from them.
% We also propose a synthetic dataset with rendered scenes of diverse, inter-occluded 3D human and object scans, to train our network. We evaluate our method on the synthetic and real world BEHAVE dataset. Our experiments show that our method outperforms the SOTA in achieving realistic clothed human-object reconstructions.
Recent approaches to jointly reconstruct 3D humans and objects from a single RGB image represent 3D shapes with template-based or coarse models, which fail to capture details of loose clothing on human bodies. In this paper, we introduce a novel implicit approach for jointly reconstructing realistic 3D clothed humans and objects from a monocular view. For the first time, we model both the human and the object with an implicit representation, allowing to capture more realistic details such as clothing. This task is extremely challenging due to human-object occlusions and the lack of 3D information in 2D images, often leading to poor detail reconstruction and depth ambiguity. To address these problems, we propose a novel attention-based neural implicit model that leverages image pixel alignment from both the input human-object image for a global understanding of the human-object scene and from local separate views of the human and object images to improve realism with, for example, clothing details. Additionally, the network is conditioned on semantic features derived from an estimated human-object pose prior, which provides 3D spatial information about the shared space of humans and objects. To handle human occlusion caused by objects, we use a generative diffusion model that inpaints the occluded regions, recovering otherwise lost details. For training and evaluation, we introduce a synthetic dataset featuring rendered scenes of inter-occluded 3D human scans and diverse objects. Extensive evaluation on both synthetic and real-world datasets demonstrates the superior quality of the proposed human-object reconstructions over competitive methods.
\end{abstract}    
% \section{Introduction}
\label{sec:intro}
% Image editing methods in diffusion models depend on user-defined control directions - users can unlock their creativity using these methods by specifying the desired manipulation through prompts~\cite{gandikota2023concept}, reference images~\cite{ruiz2022dreambooth, kumari2022customdiffusion, gal2022image, chen2024trainingfreeregionalpromptingdiffusion}, or attribute vectors~\cite{parmar2023zero,hertz2022prompt}. In this work, we ask a fundamentally different question: \emph{Can we automatically discover the underlying visual structure of a concept within diffusion model's knowledge?} %Rather than requiring user-specified controls, we aim to decompose the model's internal knowledge into meaningful directions.

% This question touches on a fundamental limitation in how we interact with diffusion models. Current control methods ~\cite{zhang2023addingconditionalcontroltexttoimage, gandikota2023concept, ye2023ipadaptertextcompatibleimage,ye2023ipadaptertextcompatibleimage, hertz2024stylealignedimagegeneration, li2023photomaker, shi2024instantbooth, chen2024trainingfreeregionalpromptingdiffusion} require users to specify their desired manipulations in advance, limiting interactive creativity. This contrasts with natural human artistic workflows, where creators dynamically explore creative ideas while jointly refining them toward meaningful artistic outcomes~\cite{hoffmann2016modeling}. This synergy between specification and exploration is not new to generative models. Early GAN architectures naturally developed disentangled latent spaces that enabled continuous\cite{harkonen2020ganspace,radford2015unsupervised, wu2021stylespace, shen2020interfacegan}, compositional control over generated images. Users could explore these spaces to discover interesting variations that would be difficult to describe in words~\cite{wu2021stylespace}, then combine them to achieve their creative goals~\cite{grabe2022towards}. 


% While diffusion models have largely superseded GANs in conditional image synthesis~\cite{dhariwal2021diffusion},  their underlying structure remains less understood. Diffusion models achieve remarkable diversity through high-dimensional latents, unlike GANs' compact latent spaces.  With a single prompt, diffusion models can generate radically different variations through different random initializations of input noise. We ask - Is it possible to discover interpretable structure within this vast space of variations?

Text-to-image diffusion models are capable of generating remarkable visual variations from a single prompt through different random initializations. However, this vast creative potential remains largely opaque to users---while we can generate diverse images, we lack understanding of the underlying structure of these variations. This presents a fundamental challenge: how can we discover and expose the latent visual capabilities encoded within these models?

\let\thefootnote\relax \footnote{$^{*}$Correspondence to \texttt{gandikota.ro@northeastern.edu}}

The challenge touches on a key limitation in how we interact with diffusion models today. Current control methods require users to explicitly specify their desired edits in advance through prompts~\cite{gandikota2023concept}, reference images~\cite{zhang2023addingconditionalcontroltexttoimage, chen2024trainingfreeregionalpromptingdiffusion, ruiz2022dreambooth,kumari2022customdiffusion, Ryu_lora, hu2021lora}, or attribute vectors~\cite{ye2023ipadaptertextcompatibleimage, hertz2024stylealignedimagegeneration, li2023photomaker, shi2024instantbooth,parmar2023zero,hertz2022prompt}. That contrasts sharply with natural human creative workflows, where artists dynamically explore creative ideas and jointly refine them toward meaningful artistic outcomes~\cite{hoffmann2016modeling}. The need for pre-specified controls creates a barrier between users and the full creative potential of these models.

Interestingly, earlier generative models like GANs~\cite{gans,karras2019style,brock2018large} naturally developed more interpretable internal structures. Their compact latent spaces often exhibited emergent disentanglement~\cite{harkonen2020ganspace,radford2015unsupervised, wu2021stylespace, shen2020interfacegan}, enabling continuous and compositional control over generated images. Users could explore these spaces to discover interesting variations that would be difficult to describe in words~\cite{wu2021stylespace}, then combine them to achieve their creative goals~\cite{grabe2022towards}.

Diffusion models have largely superseded GANs in conditional image synthesis~\cite{dhariwal2021diffusion}, achieving greater diversity through much higher-dimensional latents. And yet an understanding of the underlying structure of these larger latent spaces has remained elusive. In this work, we ask a fundamental question: \emph{Can we automatically discover the visual structure within a diffusion model's knowledge of a concept?} Rather than requiring user-specified controls, we aim to decompose the model's internal representations into expressive directions that users can explore and combine.

To address these needs, we present \textbf{SliderSpace}, a framework that brings systematic explorability to diffusion models. Given just a text prompt, SliderSpace discovers a canonical set of meaningful, diverse, and controllable directions within the model's knowledge of that concept. Each direction is implemented as a low-rank adapter~\cite{hu2021lora} that can be scaled and composed with others, allowing users to explore and smoothly combine different aspects of variation, as shown in Figure~\ref{fig:intro}.

We ground SliderSpace discovery in three key requirements for meaningful decomposition of a diffusion model's visual manifold: 
\begin{enumerate}
    \item \textbf{Unsupervised Discovery:} The decomposition process should emerge from the intrinsic structure of the model's learned representation, rather than being guided by predefined attributes. This ensures we capture the true topology of the model's knowledge space rather than projecting our assumptions onto it.
    
    \item \textbf{Semantic Orthogonality:} Each discovered control must represent a distinct semantic direction. This is enforced in a semantic feature space, like CLIP, where every slider has an orthogonal effect in embeddings. This prevents discovering multiple controls that create similar semantic effects, making the system more efficient and easier.
    
    \item \textbf{Distribution Consistency:} Directions must induce consistent transformations across both random seeds and prompt variations. 
\end{enumerate}

These requirements naturally lead to our proposed framework, which we formalize in Section~\ref{sec:method}. As we show in our experiments, SliderSpace is architecture-agnostic, working with both conventional U-Net based models like Stable Diffusion~\cite{rombach2022high, rombach2022sd20, podell2023sdxl, turbo, dmd} and recent transformer-based architectures like Flux~\cite{flux}.

We demonstrate the expressiveness of SliderSpace through three applications: First, we show how SliderSpace can decompose high-level concepts into diverse and expressive components, revealing the natural axes of variation in the model's understanding. Second, we explore artistic style variation, where SliderSpace discovers directions that match or exceed the diversity of manually curated artist lists while being judged more useful by human evaluators. Finally, we show how SliderSpace can help reverse the mode collapse commonly observed in distilled diffusion models, restoring diversity while maintaining generation speed.

Beyond providing practical creative control, SliderSpace opens new avenues for understanding and utilizing the latent capabilities of diffusion models. By mapping these models' visual potential into intuitive, composable directions, we take a step toward making their creative possibilities more accessible and interpretable to users.

% Image editing methods in diffusion models unlock the creativity of users. In this work we ask an alternate question: \emph{Can we organize and expose what of the diffusion model is already capable of?}.
% Existing methods for controlling image generation typically require users to manually specify edit directions for desired changes. This process is time-consuming, requires technical expertise, and limits the spontaneity of the creative process. For instance, if a user wants to adjust the smile of a generated person, they must explicitly request this edit, often through imprecise prompt engineering or model fine-tuning. This approach of predefined controls or manual specifications restricts users from fully exploring the latent capabilities of the model. There may be interesting stylistic variations or attributes that the model can generate, but users have no easy way to discover or utilize these.

% Natural visual disentanglement was an emergent property in the latent space of Generative Adversarial Models (GANs) \cite{harkonen2020ganspace,radford2015unsupervised, wu2021stylespace, shen2020interfacegan}. In particular, it has been observed that StyleGAN~\cite{karras2019style} stylespace neurons offer detailed control over many meaningful aspects of images that would be difficult to describe in words~\cite{wu2021stylespace}. However, diffusion models do not share such a compact latent space~\cite{park2023unsupervised}; and efforts to uncover such a space in the semantic embeddings of the text conditioning have met with limited success \nik{Nick - is there a specific citation you were thinking about?}.

% In this work we introduce \textbf{SliderSpace}, which takes a step towards uncovering an analogous low dimensional representation of diffusion models' visual breadth; in essence treating the diffusion model as many generators sharing parameters, where a particular generator is defined by a specific prompt. For a given prompt we sample many random seeds (and optionally prompt expansions using an LLM), generate the corresponding images, and apply an off the shelf feature extractor (in this work CLIP, but our method can be applied to any differentiable feature extractor). We use PCA to analyze these features, and for each of the leading $k$ principal components we train a LoRA \cite{} which causes the diffusion model to produces images which increase the feature magnitude along that component when passed back through the same feature extractor. This leads to a 'Slider' for each principal component, because each LoRA can be scaled and applied to the original diffusion model, continuously varying those visual features in the generated results (as measured, in our case, by CLIP).

% There are many other works that enhance the controllability of diffusion models. One common approach is enabling users to add spatial constraints to a generation either manually, or via a reference image \cite{zhang2023addingconditionalcontroltexttoimage, chen2024trainingfreeregionalpromptingdiffusion}, a second is leveraging more abstract embeddings (e.g. identity, style) extracted from a reference image \cite{ye2023ipadaptertextcompatibleimage, hertz2024stylealignedimagegeneration, li2023photomaker, shi2024instantbooth}, a third is finetuning a foundation model to better generate a concept important to the user \cite{ruiz2022dreambooth, kumari2022customdiffusion, Ryu_lora, hu2021lora}, and a fourth (most relevant to this work) is finding low-rank adaptors of the model based on a prompt or small training set which can be scaled to provide continous control over one aspect of generated image (e.g. night vs day, basic vs luxury, etc.) \cite{gandikota2023concept}. SliderSpace is complementary to all of these methods and offers something distinct. All of the other methods we are aware require the user (and / or model designer) to know in advance what type of control they want. In contrast SliderSpace assists users in discovering and controlling hidden capabilities present in the diffusion model's distribution of possible generations.

%We propose that truly intuitive creative control in a text-to-image model should meet three key criteria: \emph{discoverability}, \emph{intuitiveness}, and \emph{specificity}. The model should reveal controllable attributes that may not be immediately obvious, offer controls that are easy to understand and manipulate, and ensure each control affects a distinct attribute of the generated image.

% We demonstrate the utility and power of SliderSpace using three applications built on top of SDXL-DMD \cite{dmd}, because its fast generation speed lends itself well to the continuous control offered by SliderSpace.

% First, we study concept decomposition (Section \ref{sec:concept_exp}), where we learn sliders for a specific concept (e.g. 'monster', 'waterfall', 'car'). Through quantitative metrics of diversity and text alignment we demonstrate that the learned sliders dramatically boost the diversity of generations when randomly applied without harming text alignment; we also ask humans to qualitatively judge these results in a user study where they find the SliderSpace results to be more 'Diverse', 'Useful', and 'Creative' than our baselines.

% Second, we attempt to compare the automatic discoveries of SliderSpace to a large scale manual study of artistic styles (Section \ref{sec:art_exp}), open-sourced by ParrotZone \cite{parrotzone}. In this study SDXL was prompted with over 4300 artist names,  and based on visual inspection the cases of successful stylistic mimicry recorded. Quantitatively SliderSpace more closely matches the distribution of artistic variation discovered by ParrotZone than other baselines, and in our user studies was judged to be significantly more 'Diverse' and 'Useful' than the baselines. To our surprise humans even judged SliderSpace results to be slightly more 'Diverse' than the results generated by the manually discovered artist names of \cite{parrotzone}.

% Third, we attempt to use SliderSpace to reverse the mode collapse commonly observed in distilled few-step diffusion models relative to the original teacher model (Section \ref{sec:diverse_exp}). We quantitatively demonstrate that applying SliderSpace to SDXL-DMD leads to more closely matching the distribution of images by the original teacher, SDXL.

%Through extensive experiments on various state-of-the-art text-to-image models, we demonstrate that SliderSpace significantly enhances user control and creative expression in AI-assisted image generation tasks. Our method enables a range of applications, including concept decomposition and control, diversity improvement in generated images, customization dissection and edits, and the exploration of artistic styles inherent in the model.

% SliderSpace goes beyond providing a practical tool for enhanced creative control. By mapping the visual potential of diffusion models it can open new avenues for generative creativity and deepens our understanding of each model's hidden potential.
% \section{Related work}
\label{sec:formatting}

\subsection{Text-to-Video Generation}

T2V generation has made notable progress, evolving from early GAN-based models \cite{saito2017temporal,tulyakov2018mocogan,fu2023tell,li2018video,wu2022nuwa,yu2022generating} to newer transformer \cite{yan2021videogpt,arnab2021vivit,esser2021taming,ramesh2021zero,yu2022scaling} and diffusion models \cite{kirkpatrick2017overcoming,sohl2015deep,song2020denoising,zhang2022gddim}. Early efforts like MoCoGAN~\cite{tulyakov2018mocogan} focused on short video clips but faced issues with stability and coherence. The introduction of transformers improved sequential data handling, enhancing video generation, while diffusion models further improved video quality by progressively denoising the input. 
Despite these advances, T2V models still struggle to reflect human preferences, with the generated videos generally lacking aesthetic quality. Additionally, the scarcity of paired video preference data hinders effective model training and may lead to insufficient flexibility and poor quality in the generated videos.


\subsection{RLHF}

\iffalse
Aligning LLMs \cite{dai1901transformer,radford2019language,zhang2023opt} typically involves two steps: supervised fine-tuning followed by Reinforcement Learning with Human Feedback (RLHF) \cite{gao2023scaling,stiennon2020learning,rafailov2024direct}. Although effective, RLHF is computationally expensive and can lead to issues like reward hacking. Methods like DPO have streamlined alignment by leveraging feedback data directly, improving efficiency.

In contrast, diffusion model alignment is still evolving, focusing mainly on enhancing visual quality through curated datasets. Techniques like DOODL \cite{wallace2023end} and AlignProp \cite{prabhudesai2023aligning} target aesthetic improvements but face challenges with complex tasks such as text-image alignment. Reinforcement learning methods like DPOK \cite{fan2024reinforcement} and DDPO \cite{black2023training} improve reward optimization but struggle with scalability. DPO-SDXL integrates DPO into T2I generation, boosting both alignment and aesthetics.

However, aligning video generation remains a largely unaddressed challenge, especially when dealing with motion consistency and semantic coherence across frames.
\fi

RLHF \cite{gao2023scaling,stiennon2020learning,rafailov2024direct} is a method that utilizes human feedback to guide machine learning models. Early RLHF algorithms, such as DDPG~\cite{lillicrap2015continuous} and PPO~\cite{schulman2017proximal}, typically relied on complex reward models to quantify human feedback. These reward models require a large amount of annotated data and face challenges during tuning. As research has progressed, more efficient preference learning methods have emerged, among which DPO has become a new framework. DPO does not depend on a separate reward model; instead, it obtains human preferences through pairwise comparisons and directly optimizes these preferences. This shift not only simplifies the application of RLHF but also enhances the alignment of models with human values. Furthermore, DPO has been successfully introduced into T2I tasks~\cite{wallace2024diffusion,yang2024using}, providing new insights for generative models in addressing the alignment of human preferences and showcasing DPO's potential in the field of AIGC~\cite{shi2024instantbooth,
qing2024hierarchical,menapace2024snap,koley2024s}. However, there remains a gap in current research regarding the application of DPO strategies to T2V tasks. Effectively integrating DPO into T2V tasks presents a challenging endeavor.


% \section{Preliminary}
\label{sec:preliminary}
In this section, we first introduce the mathematical formulation of flow-based text-to-image generative models~\cite{Xingchao_2022,Lipman_2022}, which forms the foundation of modern T2I systems~\cite{sd3,sdxl,imagen3,imagen}. We then describe classifier-free guidance~\cite{ho2022classifier}, a key technique to control the generation process through text conditioning.

\subsection{Flow-based text-to-image generative models}
In state-of-the-art T2I models~\cite{sd3}, the image generation process is modeled by learning, through a neural network, a flow $\psi$ that generates a probability path $(p_t)_{0\le t\le 1}$ bridging the source distribution $p_0$ and the target distribution $p_1$ ~\cite{Xingchao_2022,Lipman_2022}. This framework encompasses diffusion models~\cite{sohl2015deep,ddpm} as a special case. In particular, a commonly used formulation sets a Gaussian distribution as the source: $p_0 = \mathcal{N}(\mathbf{0}, \mathbf{I})$ and a delta distribution centered on a sample $\mathbf{x}_1$ from the data distribution $q$ as the target: $p_1 = \delta_{\mathbf{x}_1}$.
Then, it incorporates an affine conditional flow $\psi_t(\mathbf{x} | \mathbf{x}_1) = a_t \mathbf{x}_1 + b_t \mathbf{x}$ with the boundary condition $(a_0, b_0) = (0, 1),\ (a_1, b_1) = (1, 0)$ to bridge them. The neural network typically approximates quantities such as velocity fields, $x_0$ prediction or $x_1$ prediction. In this modeling, these quantities can be viewed as affine transformations of the marginal probability path score $\nabla_{\mathbf{x}} \log p_t(\mathbf{x})$.

\subsection{Classifier-free guidance in flow-based models}
Classifier-free guidance~\cite{ho2022classifier} is a method for sampling from a model conditioned by a text input $\mathbf{y}$ by guiding an unconditional image generation model modeled using the score $\nabla_{\mathbf{x}} \log p_t(\mathbf{x})$. This enables the sampling from
\[
q_w(\mathbf{x}, \mathbf{y}) \propto q(\mathbf{x})q(\mathbf{y}|\mathbf{x})^w \propto q(\mathbf{x})^{1-w}q(\mathbf{x}|\mathbf{y})^w
\]
where $w \in \mathbb{R}$ is the guidance scale typically used with $w > 1$. The score satisfies
\[
\nabla_{\mathbf{x}} \log q_w(\mathbf{x}, \mathbf{y}) = (1-w)\nabla_{\mathbf{x}} \log q(\mathbf{x}) + w\nabla_{\mathbf{x}} \log q(\mathbf{x}|\mathbf{y})
\]
so by training the network to learn both the unconditional score $\nabla_{\mathbf{x}} \log q(\mathbf{x})$ and conditional score $\nabla_{\mathbf{x}} \log q(\mathbf{x}|\mathbf{y})$, flexible sampling from the conditional distribution can be achieved through a weighted sum of the network outputs.

\begin{abstract}
Can we accurately identify the true correspondences from multimodal datasets containing mismatched data pairs? Existing methods primarily emphasize the similarity matching between the representations of objects across modalities, potentially neglecting the crucial relation consistency within modalities that are particularly important for distinguishing the true and false correspondences. Such an omission often runs the risk of misidentifying negatives as positives, thus leading to unanticipated performance degradation. To address this problem, we propose a general \textbf{Re}lation \textbf{Con}sistency learning framework, namely \textbf{ReCon}, to accurately discriminate the true correspondences among the multimodal data and thus effectively mitigate the adverse impact caused by mismatches. Specifically, ReCon leverages a novel relation consistency learning to ensure the dual-alignment, respectively of, the cross-modal relation consistency between different modalities and the intra-modal relation consistency within modalities. Thanks to such dual constrains on relations, ReCon significantly enhances its effectiveness for true correspondence discrimination and therefore reliably filters out the mismatched pairs to mitigate the risks of wrong supervisions. Extensive experiments on three widely-used benchmark datasets, including Flickr30K, MS-COCO, and Conceptual Captions, are conducted to demonstrate the effectiveness and superiority of ReCon compared with other SOTAs. The code is available at: \href{https://github.com/qxzha/ReCon}{https://github.com/qxzha/ReCon}.

% In this work, we tackle this challenging issue in noisy correspondence learning.
% Such mismatches, i.e., noisy correspondence, are often introduced due to the expensive data collection and non-expert annotations, which 


% Learning discriminative features between distinct pairs is essential in noisy correspondence. Existing methods primarily emphasize the matching between representations of objects, neglecting the relations within modalities that are crucial for accurately distinguishing between true and false pairs. Such omission often risks misleading by misidentified negatives and thus leads to unexpected performance degradation. To address this problem, we propose a novel \textbf{Re}lation \textbf{Con}sistency learning framework, namely \textbf{ReCon}, that effectively enforces the alignment both between objects across modalities and relations within modalities, which significantly enhances the discriminability for true correspondences. Thanks to this dual consistencies, distinct data pairs can be accurately divided into appropriated partitions, ensuring that the corresponding optimization functions are employed to facilitate robust cross-modal retrieval. Extensive experiments on three widely-used benchmark datasets, including Flickr30K, MS-COCO, and Conceptual Captions, are conducted to demonstrate the effectiveness and superiority of ReCon compared with other SOTA methods. The code is available at: \href{https://anonymous.4open.science/r/ReCon-NCL}{https://anonymous.4open.science/r/ReCon-NCL}.
\end{abstract}

% \begin{abstract}
% Learning discriminative features between distinct multi-modal data pairs is essential in noisy correspondence. Existing methods primarily emphasize the matching between representations of objects, potentially neglecting the crucial relations within modalities that are  beneficial for accurately distinguishing the true and false pairs. Such an omission often risks misleading by misidentified negatives and thus leads to unanticipated performance degradation. To address this problem, we propose a novel \textbf{Re}lation \textbf{Con}sistency learning framework, namely \textbf{ReCon}, that effectively enforces the alignment both from the objects across modalities and relations within modalities, thereby significantly enhances the discriminability for true correspondences. Thanks to this dual consistencies, distinct data pairs can be accurately divided into appropriated partitions, ensuring that the corresponding optimization functions are employed to facilitate various cross-modal retrieval. Extensive experiments on three widely-used benchmark datasets, including Flickr30K, MS-COCO, and Conceptual Captions, are conducted to demonstrate the effectiveness and superiority of ReCon compared with other SOTA methods. The code is available at: \href{https://anonymous.4open.science/r/ReCon-NCL}{https://anonymous.4open.science/r/ReCon-NCL}.
% \end{abstract}

\section{Introduction}
Cross-modal retrieval is dedicated to understanding the semantic correspondences between multimedia data, aiming to recall the most relevant candidates for a given query \cite{scan,cmgm,dscmr,pecmr}. While existing approaches have achieved remarkable success by associating the heterogeneous data in a common latent space, they often neglect to provide an explicit consideration of semantically irrelevant data. Such mismatches, a.k.a., \textit{noisy correspondence} (NC) \cite{ncr}, would be inadvertently introduced due to the notoriously labor-intensive data collection and the unreliable non-expert annotations \cite{cc, scaling}, which inevitably impedes the semantic correspondences between modalities and consequently results in a decline of retrieval performance \cite{BiCro,crcl,MSCN}. 

\begin{figure}
    \centering
    \includegraphics[width=0.98\linewidth]{figures/new_motivation.pdf}
    % \caption{Illustration of  motivation and difference with SOTAs.}
    \caption{Illustration of relation discrepancy. The relation-aware alignment correctly identifies mismatched pair as negatives, while relation-agnostic alignment fails to detect such inconsistency.}
    \label{motivation}
\end{figure}

To tackle the NC problem, a core consensus is to enhance the discriminability for positives/matches and to mine local correspondences from negatives/mismatches. Several priors \cite{ncr,MSCN,BiCro} leverage the \textit{memory effect} \cite{coteaching}, wherein DNNs learn simple dominant patterns first, to identify matches in the early training stage. Subsequently, they estimate soft correspondence labels to describe the matching degree of mismatches, thus down-weighting their contributions and enforcing learning the local correspondences. In order to avoid the misleading caused by easily-determined noisy pairs, some attempts \cite{trip,cream,UGNCL} propose more refined data division strategies to filter out these mismatches. To further mitigate the wrong supervisions of mismatches, recent efforts \cite{l2rm,PC2} are presented to utilize pseudo counterparts for these mismatches to excavate informative correspondences. Furthermore, some works \cite{esc,gsc} achieve notable performance improvements by leveraging intrinsic properties observed within data, and methods \cite{decl,rcl,crcl} based on robust loss functions also effectively confront with the challenge of NCs. Nevertheless, they all neglect the relations within modalities, risking the misidentification of negatives as positives, particularly in cases of mismatched pairs that manifest high similarity scores, i.e., hard NCs. 

%\sout{As clarified in IAIS work~\cite{iais}, the relation consistency often enhances the contextualized representation of image-text pairs, which is beneficial to the model alignment for cross-modal retrieval.}

 As mentioned in IAIS work~\cite{iais}, the relation consistency often enhances the contextualized representation of image-text pairs. Inspired by this finding, we consider the relation discrepancy to mitigate the adverse impacts caused by mismatches among dataset.  As shown in Fig. \ref{motivation}(a), whether through relation-agnostic or relation-aware alignment, the true correspondence is expected to consistently assigned a high similarity score due to its perfect matching of both objects across modalities and relations within modalities. However, such matching is irreversibly compromised by the presence of untouchable noisy correspondence, thus narrows the distance between mismatches. Specifically, the unwanted misalignment erroneously reduces the distance of unassociated objects, inadvertently confusing retrieval models and thus undermining their discriminability for true correspondences. Besides, such misalignment also impairs the relations between objects within modalities, which significantly disrupts the contextual semantic consistency that is essential for true correspondences despite the nuance from objects. As shown in Fig. \ref{motivation}(b), the noisy pair with similar objects cannot be correctly identified by the relation-agnostic alignment due to its inability to recognize the discrepancies of relations within modalities. Such misidentification inevitably introduces false supervisions, which misleads the model towards further wrong optimization direction. In contrast, the relation-aware alignment accurately identifies it as a negative pair, benefiting from its dual consideration of both cross-modal and intra-modal relations.

%\textcolor{green}{was first explored by previous work} \cite{iais}, 

% As illustrated in Fig. \ref{fig:network},
Motivated by the above observations, we propose a general \textbf{Re}lation \textbf{Con}sistency learning framework, namely \textbf{ReCon}, to effectively mitigate the adverse impact caused by NCs, as shown in Fig.~\ref{fig:network}. The main motivation of ReCon is to \textit{enhance the discriminability for true correspondences}. Specifically, an effective relation consistency alignment strategy is introduced to enable alignment not only between objects across modalities but relations within modalities. In details, the cross-modal relation consistency is presented to maximize the similarity scores of positive pairs while minimizing the negatives, ensuring that aligned objects have similar semantic representations. Meanwhile, the intra-modal relation consistency is employed to minimize the distance of relation matrices that describe the contextualized semantics of objects within modalities, which further enlarges the distinguish between positives and negatives. In practice, due to the lack of explicit annotations of objects, we propose to align the relation matrix extracted from one selected anchor modality with the proxy relation matrix extracted from another modality. Subsequently, such dual constraints of relations are employed to divide the noisy training data, wherein the divided partitions will be trained with corresponding strategies to achieve robust cross-modal retrieval, which significantly enhances the discriminability for true correspondences and effectively avoids the wrong supervisions of misidentified negatives. 
% Finally, the divided partitions will be trained with corresponding strategies to 
% effectively avoids the misidentification of negatives as positives and 
% wherein the pairs with cross- and intra-modal consistencies are identified as true correspondences. Such division strategy effectively avoids the misidentification of negatives as positives and thus significantly enhances the discriminability of models for true correspondences. Finally, the divided partitions are trained with corresponding loss functions to achieve robust noisy correspondence learning. 

%In summary, the main contributions of our work are as follows: (1) We propose a general \textbf{Re}lation \textbf{Con}sistency learning framework, namely \textbf{ReCon}, to mitigate the adverse impact caused by NCs within multimodal dataset. (2) We introduce a novel relation consistency learning to enable alignment not only between objects across modalities but relations within modalities. (3) A reliable discrimination strategy is presented to leverage the dual constrains of relations to accurately identify true correspondences. (4) Extensive experiments on synthetic and real-world benchmark datasets are conducted to jointly demonstrate the robustness and effectiveness of our proposed ReCon.

In summary, the main contributions of our work are as follows: (1) A general \textbf{Re}lation \textbf{Con}sistency learning framework, namely \textbf{ReCon}, is robustly proposed to identify the true correspondences and therefore mitigate the adverse impact caused by NCs within multimodal dataset. (2) An effective relation consistency alignment strategy is explicitly employed to jointly enforce the alignment of the cross-modal relation consistency and the intra-modal relation consistency. (3) A reliable true correspondence discrimination strategy is effectively presented to accurately partition the noisy data pairs, which therefore, seamlessly minimizes the risk caused by wrong supervisions and mitigates the misidentification of mismatches. (4) Extensive experiments highlight the advantages of the proposed ReCon in comparison with other SOTA methods and demonstrate its outstanding performances in challenging NCs scenario.


% that leverages the dual constrains of relations to identify true correspondences, which effectively avoids the misleading of misidentified negatives and significantly mitigates the adverse impact caused by noisy correspondence. (2) We introduce a novel relation consistency learning that contains cross- and intra-modal relation consistency to jointly enforce the alignment both between representations across modalities while preserving the contextualized semantics within modalities, which encourages the distinct divided partitions to utilize corresponding optimization objectives in order to robust cross-modal retrieval. (3) Extensive experiments on synthetic and real-world benchmark datasets are conducted to demonstrate the robustness and effectiveness of our ReCon.

% Specifically, the cross-modal relation consistency is presented to maintain coherence between objects across modalities, ensuring that they have similar semantic representations. Meanwhile, the intra-modal relation consistency aims to preserve contextual semantic consistency within modalities, 

\section{Related Work}
\subsection{Cross-Modal Retrieval}
% with representatives like cross-attention \cite{scan} and similarity graph \cite{sgraf, cmgm},
Cross-modal retrieval (CMR) aims to search the most relevant items across different modalities in response to query modality. The core of CMR is to minimize the semantic discrepancies by projecting different modalities into a common comparable space, wherein the matched items manifest higher similarity or closer feature distance and vice versa. Current efforts, from the perspective of similarity calculation, can be roughly classified into two categories: 1) Coarse-grained measurement \cite{vse++,vsrn,gpo,vtsr}, which represents an efficient solution with the key idea of associating the correspondence holistically among features extracted by distinct modality-specific encoders. 2) Fine-grained measurement \cite{scan,naaf,cmgm, sgraf, lsra, cross-graph,chan}, which focuses on assessing the semantic relationships at a more granular level to learn and reason latent alignments between fragments. Unfortunately, the promising performance of all these methods relies heavily on an implicit assumption that all training data pairs are correctly matched while neglecting the presence of NC. Such NC inevitably undermines the alignments and complicates the accurate measurement of similarity, ultimately leading to inferior performance.

% by well-designed consistency interactions, 
%, which is often introduced due to the extensive data collection and annotation expenses
\subsection{Noisy Correspondence Learning}
% , which, recent years, has been widely explored in multi-modal tasks, including image-text matching \cite{ncr,rcl}, multi-view clustering \cite{multiview-1,multiview-2}, image captioning \cite{imagecap1,imagecap2}, and infrared re-identification \cite{reident1,reident2}.
Noisy correspondence learning refers to noise-tolerant approaches well-designed to effectively mitigate the adverse impacts caused by mismatches among dataset. Unlike traditional category-level mistaken annotations, this instance-level semantic inconsistency, first recognized as a new paradigm of noisy labels in \cite{ncr}, significantly affects the performance of retrieval models. Thus, some prior attempts \cite{ncr, BiCro,MSCN} employ the small-loss criterion \cite{dividemix} to identify matched pairs from the corrupted datasets and subsequently rectify soft correspondence labels for those mismatches. Following this, several works \cite{cream,trip,UGNCL} introduce novel criteria to enable more fine-grained data division, such as inconsistent predictions \cite{cream,trip} and uncertainty \cite{UGNCL}. To avoid inaccurate label predictions, some approaches \cite{l2rm,PC2} aim to refine alignments through alternative strategies like rematched mismatches \cite{l2rm} and pseudo captions \cite{PC2}. Besides these methods based data sanitized, other efforts \cite{decl,rcl,crcl} retort to robust loss functions to adaptively downweight the contributions of mismatches, e.g., evidential loss \cite{decl}, complementary contrast loss \cite{rcl}, and active complementary loss \cite{crcl}. Recently, some works \cite{esc,gsc} utilize the intrinsic properties observed within data to estimate accurate soft correspondence labels. However, thery all neglect the intra-modal relations, which is significantly crucial for accurately identify true correspondences.


\section{Methodology}

\begin{figure*}
    \centering
    \includegraphics{figures/network_v2.pdf}
    \caption{The schematic pipeline of the proposed ReCon learning framework.}
    \label{fig:network}
\end{figure*}

\subsection{Preliminaries}
\paragraph{Problem Definition}
In line with previous work, we take visual-text retrieval as a proxy task to discuss the noisy correspondence problem in cross-modal retrieval. Consider a multimodal dataset $\mathcal{D}=\left\{\left(V_{i}, L_{i}, y_{i}\right)\right\}_{i=1}^{N}$ containing of $N$ training pairs, where each $\left(V_{i}, L_{i}\right)$ denotes the $i$-th visual-text pair and $y_{i} \in \left\{0,1\right\}$ indicates whether the pair matched ($y_{i}=1$) or not ($y_{i}=0$). Typically, all pairs are assumed to be semantically associated with high similarity scores in the common representation space. However, due to the substantial costs of data collection and annotations, an unknown portion of mismatched pairs may be inadvertently labeled as matched ones. Such misalignment, a.k.a., noisy correspondence, without specific treatment, would severely disrupt the alignment between modalities and ultimately lead to performance degradation. The goal of our method is to effectively address the challenge of NCs within multimodal datasets, thus enabling robust cross-modal retrieval.

\paragraph{Intra-Modal Relation Alignment}
\label{sec:relation}
Given a sequence $\mathbf{O}=[o_{1}, \cdots, o_{N_{o}}]$ containing $N_{o}$ objects appeared in a visual-text pair, the sequences of visual and linguistic can be denoted as $\overline{\mathbf{V}}=[\overline{v}_{1},\cdots,\overline{v}_{N_{o}}]$ and $\overline{\mathbf{L}}=[\overline{l}_{1},\cdots,\overline{l}_{N_{o}}]$, respectively. Here, each item with same index corresponds to a same object. Note that an object may be described by one or more words in the sentence and one or more regions in the image, such that the linguistic item and visual item may represent a collocation of words and regions, respectively. The relation $\mathbf{C}_{o_{i}}=[c_{o_{i}\rightarrow o_{1}},\cdots,c_{o_{i}\rightarrow o_{N_{o}}}]$ of one object to others can also be depicted in both visual and linguistic modalities, i.e., $\mathbf{C}_{\overline{v}_{i}}=[c_{\overline{v}_{i}\rightarrow \overline{v}_{1}},\cdots,c_{\overline{v}_{i}\rightarrow \overline{v}_{N_{o}}}]$ and $\mathbf{C}_{\overline{l}_{i}}=[c_{\overline{l}_{i}\rightarrow \overline{l}_{1}},\cdots,c_{\overline{l}_{i}\rightarrow \overline{l}_{N_{o}}}]$, respectively.  Consequently, the alignment of such relations can be preserved by minimizing the expected risk for the distance objective~\cite{iais}, as expressed in the following equation:
\begin{equation}
    \mathcal{R}_{\mathcal{L}_{SD}} = \min\mathbb{E}_{(\mathbf{C}_{\overline{v}_{i}},\mathbf{C}_{\overline{l}_{i}}) \sim \mathcal{D}}[\mathcal{L}_{SD}(\mathbf{C}_{\overline{v}_{i}},\mathbf{C}_{\overline{l}_{i}})],
    \label{eq:relation}
\end{equation}
where $\mathcal{L}_{SD}$ is the loss function that utilized for narrowing semantic distance, e.g., symmetric matrix-based Kullback-Leibler Divergence (m-LK). Note that, IAIS \cite{iais} represents such relations within modalities using cross-modal attention matrix. Differently, ReCon obtains these relations by computing the similarity between objects within modalities.
 
\subsection{Relation Consistency Learning}
Let $\mathbf{V}{=}\left[v_{1},\cdots,v_{N_{\mathcal{V}}}\right]$ and $\mathbf{L}{=}\left[l_{1},\cdots,l_{N_{\mathcal{L}}}\right]$ be the original visual and linguistic input sequences, which respectively contains $N_{\mathcal{V}}$ visual regions and $N_{\mathcal{L}}$ linguistic words. The relation consistency learning aims not only to enforce alignment between objects across modalities, but also to ensure consistency of relations within modalities. Such dual constraints allow to comprehend more nuanced contextualized semantics compared to the relation-agnostic alignment and significantly improve the discriminability for true correspondences, which can effectively mitigate the risks of misleading caused by misidentified false supervisions, particularly in the presence of hard NCs. 

\paragraph{Cross-Modal Relation Consistency.} Cross-modal relation consistency refers to the semantic similarities between representations across modalities. To this end, two modal-specific networks $f_{\mathcal{V}}(\cdot, \Theta_{\mathcal{V}})$ and $f_{\mathcal{L}}(\cdot, \Theta_{\mathcal{L}})$ are first employed to project the visual and linguistic sequences into a common comparable space, where $\Theta_{\mathcal{V}}$ and $\Theta_{\mathcal{L}}$ are the parameterized models for visual and linguistic modalities, respectively. In the common space, the similarity of the given visual-linguistic pair is measured through similarity function $S=g(f_{\mathcal{V}}(\cdot),f_{\mathcal{L}}(\cdot),\Theta_{\mathcal{G}})$, where $\Theta_{\mathcal{G}}$ is the parameterized modal of similarity function $g$. Note that $g$ can be parametric \cite{sgraf, cmgm} or non-parametric \cite{gpo, vse++} function. For convenience, we denote $g(f_{\mathcal{V}}(\cdot),f_{\mathcal{L}}(\cdot))$ as $g(\cdot,\cdot)$ in the following. Intuitively, the goal of cross-modal consistency relation learning is to encourage the semantic gap between matches and mismatches as large as possible, which can be equivalent to maximizing the bidirectional matching probabilities of true correspondences. Consider a batch size $N_{b}$ pairs $\mathcal{D}_{N_{b}}=\{(V_{i}, L_{i}, y_{i})\}_{i=1}^{N_{b}}$, the matching probability of $i$-th visual query is defined as $p^{v2l}_{ij}=\frac{\exp(g\left(V_{i},L_{j}\right) / \tau)}{\sum_{k=1}^{N_{b}} \exp\left(g(V_{i},L_{k}\right) / \tau)}$, where $\tau$ is a temperature parameter. Likewise, the matching probability of $i$-th linguistic query is defined as $p^{l2v}_{ij}=\frac{\exp(g\left(V_{i},L_{j}\right) / \tau)}{\sum_{k=1}^{N_{b}} \exp(g\left(V_{k},L_{j}\right) / \tau)}$. Consequently, the cross-modal relation consistency can be preserved by minimizing the expected risk of bidirectional matching probabilities with the supervision of $y_{i}$, as expressed in the following equation:
\begin{equation}
% \left(f_{\mathcal{V}},f_{\mathcal{L}},g\right)
    \mathcal{R}_{\mathcal{L}_{CM}} = \min \mathbb{E}_{(V_{i},L_{i},y_{i} ) \sim \mathcal{D}_{N_{b}}} [\mathcal{L}_{CM}\left(V_{i},L_{i},y_{i}\right)],
\end{equation}
where $\mathcal{L}_{CM}$ denotes the cross-modal InfoNCE loss \cite{clip}, which encourages the similarity gap between matched and mismatched pairs as large as possible. Note that, the contributions of different data pairs will be adaptively adjusted according to their corresponding supervisions $y$. 

\paragraph{Intra-Modal Relation Consistency.}
Intra-modal relation consistency refers to the matching of semantics between visual contexts among regions and linguistic contexts among words. Unfortunately, the absence of explicit object annotations presents a particularly intricate and demanding challenge, which is quite common in real-world scenarios. Consequently, we cannot access to the sequences containing objects, which means that each visual/linguistic item now corresponds to only one region/word. Undoubtedly, Eq. \eqref{eq:relation} cannot be directly applied to such input sequences due to the lack of one-to-one correspondence properties found in object sequences. To address this problem, like~\cite{iais}, we first select an anchor modality, e.g., visual modality, containing regions sequence, and then construct a proxy sequence containing the most corresponding words sequence from the opposite modality, such that the relations of distinct modalities can be comparable. Gven a sequence $V$ with visual modality as anchor, the relations to sequence $L$ from the opposite modality can be obtained by $\mathbf{C}_{\mathcal{VL}}=g(V,L)$, wherein the relations between every visual item $v_{i}$ to all the linguistic items are depicted in $\mathbf{C}_{\mathcal{VL}}[i,:]$, i.e., the $i$-th row of this relation matrix. Subsequently, we can obtain the most relevant item $l_{i^*}$ for $v_{i}$, wherein the index can be calculated as $i^*=\arg \max \mathbf{C}_{\mathcal{VL}}[i,:]$. Likewise, we can obtain the most relevant linguistic item $l_{j^*}$ for the visual item $v_{j}$. Therefore, the intra-modal relations $c_{v_{i} \rightarrow v_{j}}$ within visual modality can be depicted by the intra-modal relations $c_{l_{i^*} \rightarrow l_{j^*}}$ within linguistic modality, which can be formulated in the following equation:
\begin{equation}
    % \mathbf{C}_{\mathcal{VV}}^{p}=\{c_{l_{i^*}\rightarrow l_{j^*}} | i^*,j^* \in [1,N_{\mathcal{L}}] \} = \Psi(\mathbf{C}_{\mathcal{LL}}, i^*,j^*)
    \mathbf{C}_{\mathcal{VV}}^{p} = \{c^{p}_{v_{i} \rightarrow v_{j}} | i,j\in[1,N_{\mathcal{V}}] \} = \Psi(\mathbf{C}_{\mathcal{LL}},l_{i^*},l_{j^*}),
    \label{eq:proxy}
\end{equation}
where $\Psi$ represents a reconstruction operation that form the proxy relation matrix. Here, the $\mathbf{C}_{\mathcal{VV}}^{p}$ can be regarded as a representation of the original visual relation matrix $\mathbf{C}_{\mathcal{VV}}$ from the linguistic view. Similarly, with linguistic modality as anchor, the reconstructed proxy relation matrix $\mathbf{C}_{\mathcal{LL}}^{p}$ from the visual view, which depicts the relations within linguistic modality, can also be obtained through above Eq. \eqref{eq:proxy}. As discussed in Sec. \ref{sec:relation}, we can now employ the m-LK to compute the distances between relation matrix and its proxy relation matrix, which can be defined as:
\begin{equation}
    \mathcal{L}_{IM} = D_{KL}(\mathbf{C}_{\mathcal{VV}}||\mathbf{C}_{\mathcal{VV}}^{p}) + D_{KL}(\mathbf{C}_{\mathcal{LL}}||\mathbf{C}_{\mathcal{LL}}^{p}).
\end{equation}

Therefore, the preservation of relations within modalities can be achieved through minimizing the expected risk of the above $\mathcal{L}_{IM}$, as formulated in the following equation:
\begin{equation}
    \mathcal{R}_{\mathcal{L}_{IM}} = \min \mathbb{E}_{(V_{i},L_{i},y_{i}) \sim \mathcal{D}_{N_{b}}}[ \mathcal{L}_{IM}(V_{i},L_{i},y_{i})].
\end{equation}

\subsection{True Correspondence Discrimination}
Due to the existence of NC, we can only have access to the noisy training dataset $\Tilde{\mathcal{D}}$ containing an unknown proportion of mismatched pairs. Thus, directly optimizing models on such dataset using the above loss functions may risks misleading by the unwanted mismatched pairs, potentially causing significant performance degradation or even leading to training collapse. To address this problem, a common strategy \cite{ncr, BiCro, esc} is to leverage the small-loss criterion \cite{dividemix} to divide the noisy dataset into clean and noisy partitions, wherein the different partitions will be processed with corresponding training strategies. In details, the clean partition can be directly used for model optimization, while the corresponding strategy might be exploited to learn all available and informative knowledge from the noisy partition, e.g., locally-associated correspondences, avoiding insufficient utilization for dataset. However, the previous methods may misidentifies the mismatched pairs as matches, thus declining the discriminability for true correspondences and resulting in suboptimal performance due to the misleading of mismatches. Thanks to the dual constrains of relations, ReCon provides more refined and reliable data division and effectively mitigates the misidentification of mismatches, especially in the existence of hard NCs.

\paragraph{Noisy Data Division.}
Inspired by the previous success \cite{ncr,BiCro}, we also leverage the small-loss criterion to achieve a rough division for the corrupted training data. Specifically, we first compute the per-sample cross-modal relation loss by $\mathcal{L}_{CM}$, denoted as $\{l_{i}^{CM}\}_{i=1}^{N}=\{\mathcal{L}_{CM}(V_{i},L_{i})\}_{i=1}^{N}$. Next, a two-component Gaussian Mixture Model (GMM) \cite{gmm} would be employed to fit the per-sample loss distribution of all training pairs, which can be expressed as $p(l_{i}^{CM})=\sum_{k=1}^{K}\lambda_{k}\phi(l_{i}^{CM} | k)$. Here $K=2$, $\lambda_{k}$ is the corresponding mixture coefficient, and $\phi(l_{i}^{CM} | k)$ indicates the probability density function of the $k$-th component. Besides, the Expectation-Maximization algorithm is employed to optimize the GMM. Finally, we use the component with smaller mean to obtain the estimated probability:
\begin{equation}
    y_{i}^{CM}=p(k|l_{i}^{CM})=p(k)p(l_{i}^{CM}|k) / p(l_{i}^{CM}).
\end{equation}

By setting a threshold $\omega_{1}$, we can roughly divide the dataset $\tilde{\mathcal{D}}$ into rough clean partition $\tilde{\mathcal{D}}_{c}=\{(V_{i},L_{i})|y_{i}^{CM} > \omega_{1}\}$ and noisy partition $\mathcal{D}_{n}=\{(V_{i},L_{i})|y_{i}^{CM} \leq \omega_{1}\}$. Theoretically, the probability of positive pairs should approach 1, while for negative pairs, it should tend toward 0. 

\paragraph{True Positives Identification.}
% To further accurately identify the true positives from $\tilde{\mathcal{D}}_{c}$ that may contain potential mismatches manifesting high similarity, the intra-modal relation loss is employed to filter the pairs with inconsistent intra-modal relations. 
To ensure that the model learns accurate representations of matched data pairs and their relations, establishing a reliable division for true positives is crucial. In practice, the accurate discrimination for positives contributes more than negatives, for only true positives can effectively guide model optimization and further enhance its discriminability, thus minimizing the risk caused by wrong supervisions. Even if some positive pairs are wrongly divided into the noisy partition, they can still be learned through the corresponding strategy of handling the noisy partition. However, the misidentified negatives would directly compromise the discriminability of model, which further increase the risk of learning from the false correspondences. Consequently, we employ the cross-modal and intra-modal relation consistency to jointly discriminate the true positives, and the discrepancies of relations within modalities can be measured through the following equation:
\begin{equation}
    y_{i}^{IM} = \frac{ \log(1 + \mathcal{L}_{IM}(V_{i},L_{i}))}{1 + \log(1 + \mathcal{L}_{IM}(V_{i},L_{i}))}.
\end{equation}

Theoretically, the discrepancies for true correspondences should approach zero, while others will exhibit significantly larger discrepancies due to their inconsistent intra-modal relations. Thus, we can distinguish such pairs from the $\tilde{\mathcal{D}}_{c}$ by a fixed threshold $\omega_{2}$ to form two refined partitions:
\begin{equation}
    \begin{cases}
        \mathcal{D}_{c} = \{(V_{i},L_{i})|y_{i}^{IM} < \omega_{2}, \forall(V_{i},L_{i}) \in \tilde{\mathcal{D}}_{c} \} \\
        \mathcal{D}_{l} = \{(V_{i},L_{i})|y_{i}^{IM} \geq \omega_{2}, \forall(V_{i},L_{i}) \in \tilde{\mathcal{D}}_{c} \}
    \end{cases},
\end{equation}
where $\mathcal{D}_{c}$ denotes the clean partition containing true correspondences that can be directly employed to subsequent training and $\mathcal{D}_{l}$ contains pairs of local-associated correspondences. To fully learn all available local-associated correspondences and  enhance the discriminability of models, while simultaneously enlarge the semantic distance between true correspondences and others, we penalize the weight of pairs belonging to $\mathcal{D}_{l}$ based on their discrepancies of intra-modal relations. The specific penalization factor can be calculated as follows:
\begin{equation}
\label{penalization}
    \lambda = \exp\{y_{i}^{IM} / \alpha\},
\end{equation}
where $\alpha$ is an empirical scale parameter. For the pairs belonging to $\mathcal{D}_{n}$, we estimate pseudo labels through the predictions of models to replace the original unreliable labels, which can be expressed as:
\begin{equation}
    \tilde{y}_{i}^{t} = \beta \tilde{y}_{i}^{t-1} + (1-\beta)p^{t}(V_{i},L_{i}), \forall (V_{i},L_{i}) \in \mathcal{D}_{n},
\end{equation}
where $\beta$ is the momentum coefficient, $\tilde{y}_{i}^{t}$ represents the estimated labels at $t$-th epoch and $p(V_{i},L_{i})=(p_{ii}^{v2t} + p_{ii}^{t2v}) / 2$ denotes the average matching probability. Thus, the final recasted labels of all pairs can be summarized as follows:
\begin{equation}
    \hat{y}_{i} = \begin{cases}
        1, \forall \left(V_{i},L_{i} \right) \in \mathcal{D}_{c} \cup \mathcal{D}_{l} \\
        \tilde{y}_{i}, \forall \left(V_{i},L_{i} \right) \in \mathcal{D}_{n}
    \end{cases}.
\end{equation}

\subsection{Overall Optimization Objective}
To ensure the initial stability and convergence for subsequent training, we first conduct $\eta$ epochs warmup process using the triplet loss \cite{vsrn}, which can be denoted as follows:
\begin{equation}
    \begin{aligned}
        \mathcal{L}_{w} &= \sum_{\tilde{L}}[\gamma - g(V_{i},L_{i}) + g(V_{i},\tilde{L})]_{+} \\
    &+ \sum_{\tilde{V}}[\gamma - g(V_{i},L_{i}) + g(\tilde{V},L_{i})]_{+}
    \end{aligned},
\end{equation}
where $\gamma$ is the fixed margin that controls the distance between positives and negatives, $[x]_{+}=\max(x,0)$, $\tilde{L}$ and $\tilde{V}$ are the negative samples in a given mini batch. Afterwards, the different partitions will be trained with corresponding optimization strategies. For pairs belonging to the $\mathcal{D}_{c}$, we aim to learn correct representations of matched pairs and relations, thus enhancing the discriminability for true correspondences.
Hence, the loss function for $\mathcal{D}_{c}$ is a combination of $\mathcal{L}_{CM}$ and $\mathcal{L}_{IM}$:
\begin{equation}
    \mathcal{L}_{c} = \xi \mathcal{L}_{CM} + \mathcal{L}_{IM},
\end{equation}
where $\xi$ is the balance factor that adjusts the contributions of cross-modal relations. As for the pairs belonging to $\mathcal{D}_{l}$, the penalization factor calculated by Eq. \eqref{penalization} will be employed to downweight the contributions of intra-modal relations:
\begin{equation}
    \mathcal{L}_{l} = \xi \mathcal{L}_{CM} + \frac{1}{\lambda} \mathcal{L}_{IM}.
\end{equation}

Finally, for the pairs belonging to $\mathcal{D}_{n}$, the estimated pseudo labels will be employed to adjust their contributions in cross-modal relations, while the intra-modal relations will be excluded to avoid incorrect supervisions:
\begin{equation}
    \mathcal{L}_{n} = \hat{y}_{i}\mathcal{L}_{CM} = \mathcal{H}(\hat{y}_{i},p_{ii}^{v2l}) + \mathcal{H}(\hat{y}_{i},p_{ii}^{l2v}),
\end{equation}
where $\mathcal{H}$ denotes the batched cross-entropy function.

\section{Experiments}

\subsection{Datasets and Protocols}
\paragraph{Datasets.}
We evaluate our method on three widely-used benchmarks, following the settings in \cite{ncr}. Specifically, Flickr30K \cite{flickr} contains 31K images with five textual descriptions, collected from the Flickr website. We split 1K image-text pairs for validation, 1K pairs for testing, and the rest are assigned for training. MS-COCO \cite{coco} includes 123, 287 images with five associated captions each. We assign 113, 287 image-text pairs for model training, 5K pairs for validation, and the rest for testing. Both results are reported in our experiments by averaging over 5 folds of 1K test pairs and on the whole 5K test pairs. Conceptual Captions (CC) \cite{cc} is a web-crawled large-scale dataset automatically sourced from the Internet, which inadvertently contains about 3\%$\sim$20\% mismatched or weakly-matched pairs, i.e., noisy correspondence. In our experiments, CC152K, a subset of CC, is utilized for model evaluation, which comprises 1K image-text pairs designated for validation, 1K pairs for testing, and the remaining 150K pairs for training.
% with specific descriptions are provided as follows:
% \begin{itemize}
%     \item \textbf{Flickr30K} \cite{flickr} contains 31K images with five textual descriptions, collected from the Flickr website. We split 1K image-text pairs for validation, 1K pairs for testing, and the rest are assigned for training.
%     \item \textbf{MS-COCO} \cite{coco} includes 123, 287 images with five associated captions each. We assign 113, 287 image-text pairs for model training, 5K pairs for validation, and the rest for testing. Both results are reported in our experiments by averaging over 5 folds of 1K test pairs and on the whole 5K test pairs.
%     \item \textbf{Conceptual Captions} (CC) \cite{cc} is a web-crawled large-scale dataset automatically sourced from the Internet, which inadvertently contains about 3\%$\sim$20\% mismatched or weakly-matched pairs, i.e., real-world noisy correspondence. In our experiments, CC152K, a subset of CC, is utilized for model evaluation, which comprises 1K image-text pairs designated for validation, 1K pairs for testing, and the remaining 150K pairs for training.
% \end{itemize}

\paragraph{Evaluation Protocols.}
Recall at K (R@K) is a widely-used metric to measure the retrieval performance, defined as the percentage of matched items successfully retrieved from the top K candidates~\cite{PAMI2021}. In our experiments, the R@1, R@5, R@10, and the sum of three recalls for image-to-text and text-to-image retrieval are all reported to provide a comprehensive performance evaluation for our method.

\begin{table}
    \centering
    \caption{Comparisons with real-world NCs on CC152K. The \textbf{Best} and \underline{second-best} results are respectively marked in each column.}
    \label{tab:real_world}
    \setlength{\tabcolsep}{0.98mm}
    \begin{tabular}{c|ccccccc}
    \toprule
        \multirow{2}{*}{Methods} & \multicolumn{3}{c}{Image to Text} & \multicolumn{3}{c}{Text to Image} & \\
         \cline{2-8}
        & R@1 & R@5 &R@10 & R@1 & R@5 &R@10 & rSum \\
       \bottomrule
       SCAN & 30.5 & 55.3 & 65.3 & 26.9 & 53.0 & 64.7 & 295.7 \\
       NCR & 39.5 & 64.5 & 73.5 & 40.3 & 64.6 & 73.2 & 355.6 \\
       DECL & 39.0 & 66.1 & 75.5 & 40.7 & 66.3 & 76.7 & 364.3 \\
       MSCN & 40.1 & 65.7 & 76.6 & 40.6 & 67.4 & 76.3 & 366.7 \\
       BiCro & 40.8 & 67.2 & 76.1 & 42.1 & 67.6 & 76.4 & 370.2 \\
       RCL & 41.7 & 66.0 & 73.6 & 41.6 & 66.4 & 75.1 & 364.4 \\
       CRCL & 41.8 & 67.4 & 76.5 & 41.6 & 68.0 & \textbf{78.4} & 373.7 \\
       SREM & 40.9 & 67.5 & 77.1 & 41.5 & \underline{68.2} & 77.0 & 372.2 \\
       PC$^2$ & 39.3 & 66.4 & 75.4 & 39.8 & 66.4 & 76.8 & 364.1 \\
       L2RM & \underline{43.0} & 67.5 & 75.7 & 42.8 & 68.0 & 77.2 & 374.2 \\
       ESC & 42.8 & 67.3 & 76.9 & \underline{44.8} & \underline{68.2} & 75.9 & \underline{375.9} \\
       GSC & 42.1 & \underline{68.4} & \underline{77.7} & 42.2 & 67.6 & 77.1 & 375.1 \\
       \textbf{ReCon} & \textbf{43.1} & \textbf{68.7} & \textbf{78.1} & \textbf{44.9} & \textbf{68.3} & \underline{77.4} & \textbf{380.5} \\
       \bottomrule
    \end{tabular}
\end{table}

\subsection{Implementation Details}
For fair comparisons, all experiments are conducted using the same backbone SGRAF \cite{sgraf} and all experimental settings are consistent with NCR \cite{ncr}, except for the specific parameters of ReCon. Specifically, the batch size $N_{b}$ is set to 128 and the temperature coefficients $\tau$ is 0.1. The division thresholds $\omega_{1}$ and $\omega_{2}$ are both set to 0.5, the scale parameter $\alpha$ for penalization factor is set to 0.1, and the momentum coefficient $\beta$ is 0.6. Moreover, the fixed margin $\gamma$ is set to 0.2 and the balance factor $\xi$ is 5. Before training models, we conduct a $\eta=5$ epochs warmup process for initial convergence. Besides, all experiments are conducted without any additional preprocessing or the use of external data sources.

%Due to the space limitation, more analysis and implementation details are given in our \textit{supplementary material}.

\begin{table*}
    \centering
    \setlength{\tabcolsep}{0.9mm}
    \caption{Cross-modal retrieval performance comparison under synthetic noise rates of 20\%, 40\%, and 60\% on Flickr30K and MS-COCO 1K. The best and the second best results are respectively marked by \textbf{bold} and \underline{underline}.}
    \label{tab:simulated}
    \begin{tabular}{c|c|ccc|ccc|c|ccc|ccc|c}
        \toprule
        \multirow{3}{*}{\makecell[c]{Noise \\ Ratio}} & \multirow{3}{*}{Methods} & \multicolumn{7}{|c}{Flickr30K} & \multicolumn{7}{|c}{MS-COCO 1K} \\
        & & \multicolumn{3}{|c|}{Image to Text} & \multicolumn{3}{|c|}{Text to Image} & & \multicolumn{3}{|c|}{Image to Text} & \multicolumn{3}{|c|}{Text to Image} & \\
        \cline{3-16}
         &  & R@1 & R@5 & R@10 & R@1 & R@5 & R@10 & rSum & R@1 & R@5 & R@10 & R@1 & R@5 & R@10 & rSum\\
        \bottomrule
        \multirow{13}{*}{20\%} 
        & SCAN (ECCV'18) & 59.1 & 83.4 & 90.4 & 36.6 & 67.0 & 77.5 & 414.0 & 66.2 & 91.0 & 96.4 & 45.0 & 80.2 & 89.3 & 468.1 \\
        & NCR (NIPS'21) & 73.5 & 93.2 & 96.6 & 56.9 & 82.4 & 88.5 & 491.1 & 76.6 & 95.6 & 98.2 & 60.8 & 88.8 & 95.0 & 515.0 \\
        & DECL (ACM MM'22) & 77.5 & 93.8 & 97.0 & 56.1 & 81.8 & 88.5 & 494.7 & 77.5 & 95.9 & 98.4 & 61.7 & 89.3 & 95.4 & 518.2 \\
        & MSCN (CVPR'23) & 77.4 & 94.9 & 97.6 & 59.6 & 83.2 & 89.2 & 501.9 & 78.1 & \textbf{97.2} & 98.8 & 64.3 & 90.4 & 95.8 & 524.6 \\
        & BiCro (CVPR'23) & 78.1 & 94.4 & 97.5 & 60.4 & 84.4 & 89.9 & 504.7 & 78.8 & 96.1 & 98.6 & 63.7 & 90.3 & 95.7 & 523.2 \\
        & RCL (TPAMI'23) & 75.9 & 94.5 & 97.3 & 57.9 & 82.6 & 88.6 & 496.8 & 78.9 & 96.0 & 98.4 & 62.8 & 89.9 & 95.4 & 521.4 \\
        & CRCL (NIPS'23) & 78.9 & 94.8 & \textbf{97.9} & 58.7 & 83.0 & 89.2 & 502.5 & 77.8 & 96.1 & 98.5 & 63.4 & 90.3 & \underline{95.9} & 522.0 \\
        & SREM (AAAI'24) & \underline{79.5} & 94.2 & \textbf{97.9} & \underline{61.2} & \underline{84.8} & 90.2 & \underline{507.8} & 78.5 & 96.8 & 98.8 & 63.8 & 90.4 & 95.8 & 524.1 \\
        & PC$^2$ (ACM MM'24) & 78.7 & 94.9 & 96.9 & 59.8 & 83.9 & 89.6 & 503.8 & 77.8 & 95.7 & 98.4 & 62.8 & 89.7 & 95.3 & 519.7 \\
        & L2RM (CVPR'24) & 77.9 & \underline{95.2} & \underline{97.8} & 59.8 & 83.6 & 89.5 & 503.8 & \underline{80.2} & 96.3 & 98.5 & 64.2 & 90.1 & 95.4 & 524.7 \\
        & ESC (CVPR'24) & 79.0 & 94.8 & 97.5 & 59.1 & 83.8 & 89.1 & 503.3 & 79.2 & \underline{97.0} & \textbf{99.1} & \underline{64.8} & \underline{90.7} & \textbf{96.0} & \underline{526.8} \\
        & GSC (CVPR'24) & 78.3 & 94.6 & \underline{97.8} & 60.1 & 84.5 & \underline{90.5} & 505.8 & 79.5 & 96.4 & \underline{98.9} & 64.4 & 90.6 & \underline{95.9} & 525.7 \\
        
        & \textbf{ReCon} & \textbf{80.3} & \textbf{95.3} & \underline{97.8} & \textbf{61.6} & \textbf{85.5} & \textbf{91.3} & \textbf{511.8} & \textbf{80.9} & 96.6 & 98.8 & \textbf{65.2} & \textbf{91.0} & \textbf{96.0} & \textbf{528.6} \\
        \bottomrule
        
        \multirow{13}{*}{40\%}
        & SCAN (ECCV'18) & 29.9 & 60.5 & 72.5 & 16.4 & 38.5 & 48.6 & 266.4 & 30.1 & 65.2 & 79.2 & 18.9 & 51.1 & 69.9 & 314.4 \\
        & NCR (NIPS'21) & 75.3 & 92.1 & 95.2 & 56.2 & 80.6 & 87.4 & 486.8 & 76.5 & 95.0 & 98.2 & 60.7 & 88.5 & 95.0 & 513.9 \\
        & DECL (ACM MM'22) & 72.7 & 92.3 & 95.4 & 53.4 & 79.4 & 86.4 & 479.6 & 75.6 & 95.5 & 98.3 & 59.5 & 88.3 & 94.8 & 512.0 \\
        & MSCN (CVPR'23) & 74.4 & \textbf{94.4} & \underline{96.9} & 57.2 & 81.7 & 87.6 & 492.2 & 74.8 & 94.9 & 98.0 & 60.3 & 88.5 & 94.4 & 510.9 \\
        & BiCro (CVPR'23) & 74.6 & 92.7 & 96.2 & 55.5 & 81.1 & 87.4 & 487.5 & 77.0 & 95.9 & 98.3 & 61.8 & 89.2 & 94.9 & 517.1 \\
        & RCL (TPAMI'23) & 72.7 & 92.7 & 96.1 & 54.8 & 80.0 & 87.1 & 483.4 & 77.0 & 95.5 & 98.3 & 61.2 & 88.5 & 94.8 & 515.3 \\
        & CRCL (NIPS'23) & 74.1 & 92.6 & \underline{96.9} & 55.5 & 80.9 & 87.6 & 487.6 & 76.6 & 95.6 & 98.5 & 62.3 & 89.7 & \underline{95.4} & 518.1 \\
        & SREM (AAAI'24) & \underline{76.5} & 93.9 & 96.3 & \underline{57.5} & \underline{82.7} & 88.5 & 495.4 & 77.2 & 96.0 & 98.5 & 62.1 & 89.3 & 95.3 & 518.4 \\
        & PC$^2$ (ACM MM'24) & 75.8 & 93.5 & \underline{96.9} & \underline{57.5} & 81.9 & 88.2 & 493.8 & 77.4 & 95.8 & 98.4 & 62.1 & 89.4 & 95.1 & 518.2 \\
        & L2RM (CVPR'24) & 75.8 & 93.2 & \underline{96.9} & 56.3 & 81.0 & 87.3 & 490.5 & 77.5 & 95.8 & 98.4 & 62.0 & 89.1 & 94.9 & 517.7\\
        & ESC (CVPR'24) & 76.1 & 93.1 & 96.4 & 56.0 & 80.8 & 87.2 & 489.6 & \underline{78.6} & \textbf{96.6} & \textbf{99.0} & \underline{63.2} & \textbf{90.6} & \textbf{95.9} & \underline{523.9} \\
        & GSC (CVPR'24) & \underline{76.5} & 94.1 & \textbf{97.6} & \underline{57.5} & \underline{82.7} & \underline{88.9} & \underline{497.3} & 78.2 & 95.9 & 98.2 & 62.5 & 89.7 & \underline{95.4} & 519.9 \\
        
        & \textbf{ReCon} & \textbf{79.4} & \underline{94.3} & \textbf{97.6} & \textbf{59.9} & \textbf{83.9} & \textbf{90.1} & \textbf{505.2} & \textbf{79.9} & \underline{96.2} & \underline{98.6} & \textbf{63.5} & \underline{90.5} & \textbf{95.9} & \textbf{524.5} \\
        \bottomrule
        
        \multirow{13}{*}{60\%}
        & SCAN (ECCV'18) & 16.9 & 39.3 & 53.9 & 2.8 & 7.4 & 11.4 & 131.7 & 27.8 & 59.8 & 74.8 & 16.8 & 47.8 & 66.4 & 293.4 \\
        & NCR (NIPS'21) & 68.7 & 89.9 & 95.5 & 52.0 & 77.6 & 84.9 & 468.6 & 72.7 & 94.0 & 97.6 & 57.9 & 87.0 & 94.1 & 503.3 \\
        & DECL (ACM MM'22) & 65.2 & 88.4 & 94.0 & 46.8 & 74.0 & 82.2 & 450.6 & 73.0 & 94.2 & 97.9 & 57.0 & 86.6 & 93.8 & 502.5 \\
        & MSCN (CVPR'23) & 70.4 & 91.0 & 94.9 & 53.4 & 77.8 & 84.1 & 471.6 & 74.4 & \underline{95.1} & 97.9 & 59.2 & 87.1 & 92.8 & 506.5 \\
        & BiCro (CVPR'23) & 67.6 & 90.8 & 94.4 & 51.2 & 77.6 & 84.7 & 466.3 & 73.9 & 94.4 & 97.8 & 58.3 & 87.2 & 93.9 & 505.5 \\
        & RCL (TPAMI'23) & 67.7 & 89.1 & 93.6 & 48.0 & 74.9 & 83.3 & 456.6 & 74.0 & 94.3 & 97.5 & 57.6 & 86.4 & 93.5 & 503.3 \\
        & CRCL (NIPS'23) & 70.4 & 90.4 & 94.9 & 52.6 & 78.1 & 85.1 & 471.5 & 75.2 & 94.9 & 98.0 & 60.1 & 88.5 & 94.8 & 511.5 \\
        & SREM (AAAI'24) & 71.0 & \underline{92.1} & \underline{96.1} & \underline{54.0} & \underline{80.1} & \underline{87.0} & \underline{480.3} & 74.5 & 94.5 & 97.9 & 58.7 & 87.5 & 93.9 & 506.9 \\
        & PC$^2$ (ACM MM'24) & 70.8 & 90.3 & 94.4 & 53.1 & 79.0 & 85.9 & 473.5 & 74.2 & 94.4 & 97.8 & 58.9 & 87.5 & 93.8 & 506.6 \\
        & L2RM (CVPR'24) & 70.0 & 90.8 & 95.4 & 51.3 & 76.4 & 83.7 & 467.6 & 75.4 & 94.7 & 97.9 & 59.2 & 87.4 & 93.8 & 508.4 \\
        & ESC (CVPR'24) & \underline{72.6} & 90.9 & 94.6 & 53.0 & 78.6 & 85.3 & 475.0 & \textbf{77.2} & \underline{95.1} & \underline{98.1} & \underline{61.1} & \underline{88.6} & \underline{94.9} & \underline{515.0} \\
        & GSC (CVPR'24) & 70.8 & 91.1 & 95.9 & 53.6 & 79.8 & 86.8 & 478.0 & \underline{75.6} & \underline{95.1} & 98.0 & 60.0 & 88.3 & 94.6 & 511.7 \\

        & \textbf{ReCon} & \textbf{74.3} & \textbf{93.6} & \textbf{96.6} & \textbf{55.7} & \textbf{81.6} & \textbf{88.1} & \textbf{489.9} & \textbf{77.2} & \textbf{95.9} & \textbf{98.4} & \textbf{61.8} & \textbf{89.3} & \textbf{95.2} & \textbf{517.8} \\
        \bottomrule
    \end{tabular}
\end{table*}

\subsection{Comparison with State-of-the-Arts}
% the results reported in this paper strictly reference those reported in the corresponding previous studies, and for cases where results were not reported, we retrained the models following the official recommended configurations. Furthermore, 
In this section, we carry out a comprehensive evaluation to present the effectiveness of ReCon, benchmarking it against SOTA baselines across three widely-used datasets above. The baselines comprise SCAN \cite{scan}, NCR \cite{ncr}, DECL \cite{decl}, MSCN \cite{MSCN}, BiCro \cite{BiCro}, RCL \cite{rcl}, CRCL \cite{crcl}, SREM \cite{srem}, PC$^2$ \cite{PC2}, L2RM \cite{l2rm}, ESC \cite{esc} and GSC \cite{gsc}. For the well-established Flickr30K and MS-COCO, the simulated NCs with varying noise rates, namely 20\%, 40\%, and 60\%, obtained by randomly shuffling the captions like \cite{ncr} are exploited to assess the robustness of ReCon. In addition to simulated NCs, we also validate the performance of ReCon with real-world noisy conditions using the web-crawled CC152K naturally containing 3\% $\sim$ 20\% unknown NCs. Note that the presented results of ReCon on the testing set are obtained through the checkpoints that achieved optimal performance on the validation set.

\textbf{Results on Simulated NCs.}
For quantitative evaluation the performance and robustness of all baselines under different noise ratios, we conduct all tested baselines on the Flickr30K and MS-COCO 1K with 20\%, 40\%, and 60\% of simulated noisy correspondence, where the results of MS-COCO are averaged on 5 folds of 1K test pairs as in previous works \cite{ncr, crcl}. The details are recorded in Table \ref{tab:simulated}, which demonstrates that our ReCon remarkably outperforms other baselines by a large margin on most of metrics. Notably, ReCon gains the highest R@1 score for both image-to-text and text-to-image retrieval across all noise rates on these two datasets, indicating that our method has significant potential to effectively deal with NCs. This promising performance can be attributed to the accurate identification for true positives, which avoids the misleading of wrongly introduced mismatches and enhances the discrimination between matched and mismatched pairs, thus achieving further performance improvement. Besides, ReCon performs competitive performance than other baselines under severely noise, proving its stability and reliability to facilitate robust cross-modal retrieval.

% \begin{table}
%     \centering
%     \caption{Comparisons with real-world NCs on CC152K. The \textbf{Best} and \underline{second-best} results are respectively marked in each column.}
%     \label{tab:real_world}
%     \setlength{\tabcolsep}{0.98mm}
%     \begin{tabular}{c|ccc|ccc|c}
%     \toprule
%         \multirow{2}{*}{Methods} & \multicolumn{3}{c|}{Image to Text} & \multicolumn{3}{c|}{Text to Image} & \\
%          \cline{2-8}
%         & R@1 & R@5 &R@10 & R@1 & R@5 &R@10 & rSum \\
%        \bottomrule
%        SCAN & 30.5 & 55.3 & 65.3 & 26.9 & 53.0 & 64.7 & 295.7 \\
%        NCR & 39.5 & 64.5 & 73.5 & 40.3 & 64.6 & 73.2 & 355.6 \\
%        DECL & 39.0 & 66.1 & 75.5 & 40.7 & 66.3 & 76.7 & 364.3 \\
%        MSCN & 40.1 & 65.7 & 76.6 & 40.6 & 67.4 & 76.3 & 366.7 \\
%        BiCro & 40.8 & 67.2 & 76.1 & 42.1 & 67.6 & 76.4 & 370.2 \\
%        RCL & 41.7 & 66.0 & 73.6 & 41.6 & 66.4 & 75.1 & 364.4 \\
%        CRCL & 41.8 & 67.4 & 76.5 & 41.6 & 68.0 & \textbf{78.4} & 373.7 \\
%        SREM & 40.9 & 67.5 & 77.1 & 41.5 & \underline{68.2} & 77.0 & 372.2 \\
%        PC$^2$ & 39.3 & 66.4 & 75.4 & 39.8 & 66.4 & 76.8 & 364.1 \\
%        L2RM & \underline{43.0} & 67.5 & 75.7 & 42.8 & 68.0 & 77.2 & 374.2 \\
%        ESC & 42.8 & 67.3 & 76.9 & \underline{44.8} & \underline{68.2} & 75.9 & \underline{375.9} \\
%        GSC & 42.1 & \underline{68.4} & \underline{77.7} & 42.2 & 67.6 & 77.1 & 375.1 \\
%        \textbf{ReCon} & \textbf{43.1} & \textbf{68.7} & \textbf{78.1} & \textbf{44.9} & \textbf{68.3} & \underline{77.4} & \textbf{380.5} \\
%        \bottomrule
%     \end{tabular}
% \end{table}

\begin{table}
    \centering
    \caption{Performance comparison with CLIP on MS-COCO 5K. The \textbf{best} results are highlighted in \textbf{bold}.}
    \label{clip}
      \setlength{\tabcolsep}{0.5mm}
    \begin{tabular}{c|c|ccccccc}
    \toprule
        \multirow{2}{*}{Noise} & \multirow{2}{*}{Methods} & \multicolumn{3}{c}{Image to Text} & \multicolumn{3}{c}{Text to Image} & \\
    \cline{3-9}
         &  & R@1 & R@5 &R@10 & R@1 & R@5 &R@10 & rSum \\
    \midrule
        \multirow{3}{*}{0\%} & CLIP-14 & 58.4 & 81.5 & 88.1 & 37.8 & 62.4 & 72.2 & 400.4 \\
         & CLIP-32 & 50.2 & 74.6 & 83.6 & 30.4 & 56.0 & 66.8 & 361.6 \\
         & \textbf{ReCon} & \textbf{61.6} & \textbf{86.7} & \textbf{92.7} & \textbf{44.4} & \textbf{73.1} & \textbf{83.1} & \textbf{441.6} \\
    \midrule
        \multirow{3}{*}{20\%} & CLIP-14 & 36.1 & 61.3 & 72.5 & 22.6 & 43.2 & 53.7 & 289.4 \\
         & CLIP-32 & 21.4 & 49.6 & 63.3 & 14.8 & 37.6 & 49.6 & 236.3 \\
         & \textbf{ReCon} & \textbf{61.1} & \textbf{85.7} & \textbf{92.2} & \textbf{43.5} & \textbf{72.4} & \textbf{82.7} & \textbf{437.6} \\
    \midrule
        \multirow{2}{*}{50\%} & CLIP-32 & 10.9 & 27.8 & 38.3 & 7.8 & 19.5 & 26.8 & 131.1 \\
         & \textbf{ReCon} & \textbf{58.1} & \textbf{85.1} & \textbf{91.9} & \textbf{41.5} & \textbf{70.7} & \textbf{81.0} & \textbf{428.3} \\
    \bottomrule
    \end{tabular}
\end{table}



\begin{figure}
    \centering
        \begin{minipage}{0.25\textwidth}
            \centering
            \includegraphics[width=\textwidth]{figures/omega.pdf}
        \end{minipage}
        \begin{minipage}{0.21\textwidth}
            \centering
            \includegraphics[width=\textwidth]{figures/xi.pdf}
        \end{minipage}
    \caption{Performance under different hyper-parameters of ReCon on Flickr30K with 40\% NCs.}
    \label{fig:hyper}
\end{figure}

\textbf{Results on Real-World NCs.}
% We directly train and evaluate ReCon on CC152K without any additional simulated noise injection. 
For substantiating the comprehensive performance assessment, we also provide the quantitative results that evaluated on CC152K containing real-world NCs, which better mirrors real-world industry scenarios. According to the results shown in Table \ref{tab:real_world}, it can be observed that ReCon outperforms the baselines by a considerable margin with the overall score 4.4\% performance improvement compared to the second-best ESC of 375.9\%. Besides, ReCon exhibits competitive performance across all metrics, consistently indicating its robustness and effectiveness in handling real-world NCs.

\textbf{Comparison to Pre-trained Model.}
% In detail, in zero-shot, the released pre-trained models of CLIP are directly employed to perform inference. In fine-tune, we first fine-tune the released CLIP model and perform the inference on the testing set. 
To further present the superiority and necessity of ReCon, we perform comparisons to the large pre-trained vision-language model, i.e., CLIP \cite{clip}, which is a powerful baseline trained on massive image-text pairs collected from the Internet with a large number of real NCs. In line with \cite{ncr}, we compare our ReCon to the CLIP on MS-COCO dataset under the following two settings: zero-shot and fine-tune, and the two baselines: CLIP-14 (ViT-L/14) and CLIP-32 (ViT-B/32). From the results shown in Table \ref{clip}, the significant performance degradation of CLIP can be attribute to the lack of effective mechanism to handle noisy correspondence. In contrast, the performance of ReCon under 50\% noise even surpasses the zero-shot results achieved of CLIP, indicating the effectiveness and necessity of our ReCon.

% \begin{table}
%     \centering
%     \caption{Performance comparison with CLIP on MS-COCO 5K. The \textbf{best} results are highlighted in bold.}
%     \label{clip}
%       \setlength{\tabcolsep}{0.5mm}
%     \begin{tabular}{c|c|cccccc|c}
%     \toprule
%         \multirow{2}{*}{Noise} & \multirow{2}{*}{Methods} & \multicolumn{3}{c}{Image to Text} & \multicolumn{3}{c}{Text to Image} & \\
%     \cline{3-9}
%          &  & R@1 & R@5 &R@10 & R@1 & R@5 &R@10 & rSum \\
%     \midrule
%         \multirow{4}{*}{0\%} & CLIP-14 & 58.4 & 81.5 & 88.1 & 37.8 & 62.4 & 72.2 & 400.4 \\
%          & CLIP-32 & 50.2 & 74.6 & 83.6 & 30.4 & 56.0 & 66.8 & 361.6 \\
%          & NCR & 58.2 & 84.2 & 91.5 & 41.7 & 71.0 & 81.3 & 427.9 \\
%          & \textbf{ReCon} & \textbf{61.6} & \textbf{86.7} & \textbf{92.7} & \textbf{44.4} & \textbf{73.1} & \textbf{83.1} & \textbf{441.6} \\
%     \midrule
%         \multirow{4}{*}{20\%} & CLIP-14 & 36.1 & 61.3 & 72.5 & 22.6 & 43.2 & 53.7 & 289.4 \\
%          & CLIP-32 & 21.4 & 49.6 & 63.3 & 14.8 & 37.6 & 49.6 & 236.3 \\
%          & NCR & 56.9 & 83.6 & 91.0 & 40.6 & 69.8 & 80.1 & 422.0 \\
%          & \textbf{ReCon} & \textbf{61.1} & \textbf{85.7} & \textbf{92.2} & \textbf{43.5} & \textbf{72.4} & \textbf{82.7} & \textbf{437.6} \\
%     \midrule
%         \multirow{3}{*}{50\%} & CLIP-32 & 10.9 & 27.8 & 38.3 & 7.8 & 19.5 & 26.8 & 131.1 \\
%         & NCR & 53.1 & 80.7 & 88.5 & 37.9 & 66.6 & 77.8 & 404.6 \\
%          & \textbf{ReCon} & \textbf{58.1} & \textbf{85.1} & \textbf{91.9} & \textbf{41.5} & \textbf{70.7} & \textbf{81.0} & \textbf{428.3} \\
%     \bottomrule
%     \end{tabular}
% \end{table}

\begin{table}
    \centering
      \caption{Ablation studies on Flick30K with 40\% noise with different components in ReCon. The \textbf{best} results are marked in \textbf{bold}.}
    \label{tab:ablation}
    \setlength{\tabcolsep}{0.8mm}
    \begin{tabular}{ccc|ccccccc}
    \toprule
        \multicolumn{3}{c}{Components} & \multicolumn{3}{|c}{Image to Text} & \multicolumn{3}{|c}{Text to Image} &  \\
        \midrule
        Tru. & $\mathcal{L}_{IM}$ & $\lambda$ & R@1 & R@5 &R@10 & R@1 & R@5 &R@10 & rSum \\
        \midrule
        \checkmark & \checkmark & \checkmark & \textbf{79.4} & \textbf{94.3} & \textbf{97.6} & \textbf{59.9} & \textbf{83.9} & \textbf{90.1} & \textbf{505.2} \\
        \checkmark & \checkmark &  & 77.3 & 94.1 & 97.3 & 58.7 & 83.3 & 89.5 & 500.2 \\
        & \checkmark & \checkmark & 77.2 & \textbf{94.3} & 97.2 & 57.9 & 83.1 & 89.3 & 499.1 \\
        & \checkmark &  & 77.0 & 94.1 & 97.0 & 57.6 & 82.8 & 89.0 & 497.5 \\
        \checkmark & & & 74.1 & 93.2 & 96.7 & 57.4 & 83.1 & 88.9 & 493.3 \\
         \bottomrule
    \end{tabular}
\end{table}

\begin{figure}
    \centering
    \includegraphics[width=1.0\linewidth]{figures/case.pdf}
    \caption{Examples of detected mismatched pairs on Flickr30K.}
    \label{fig:case}
\end{figure}

\subsection{Ablation Study}
\textbf{Impact of components.}
% For $\mathcal{L}_{IM}$, we just divide the training data. 
We conducted ablation studies on the Flickr30K with 40\% noise to validate the individual contributions of each component within ReCon, as detailed in Table \ref{tab:ablation}. For the true correspondence discrimination, all pairs are divided into clean and noisy partitions based on the $\mathcal{L}_{CM}$, and the intra-modal relation $\mathcal{L}_{IM}$ is directly employed to the clean partition. From the table, ReCon achieves the optimal performance by integrating all these components. This substantial improvement not only confirms the effectiveness of each individual component but also indicates their collective contributions in enhancing the robustness of models to address noisy correspondence.
\textbf{Impact of hyper-parameters.}
Fig. \ref{fig:hyper} shows the effects of the main hyper-parameters including division thresholds and balance factor. From the results, ReCon obtains better performance with $\omega_{1},\omega_{2} \in [0.4,0.6]$ and the $\xi \in [3,7]$.
\textbf{Detected noisy correspondences.}
Fig. \ref{fig:case} visualizes some detected mismatched pairs on Flickr30K by ReCon. These pairs exhibit high matching probabilities with local correspondences, yet are correctly identified as mismatched pairs due to their inconsistencies of intra-modal relations.

\section{Conclusion}
This paper introduces a general \textbf{Re}lation \textbf{Con}sistency learning framework, namely \textbf{ReCon}, to effectively mitigate the adverse impact caused by NCs. The main motivation of our ReCon is to \textit{enhance the discriminability of models for true correspondences in noisy multimodal dataset} and thus effectively avoids the wrong supervisions of false correspondences, especially in the presence of hard NCs. Specifically, we leverage the dual constrains, which simultaneously consider the cross- and intra-modal relations, to jointly divide the corrupted training data into different partitions. Extensive experiments conducted on three widely-used cross-modal benchmarks validate the effectiveness and robustness of ReCon in handling both simulated and real-world NCs.

{
    \small
    \bibliographystyle{ieeenat_fullname}
    \bibliography{main}
}

% WARNING: do not forget to delete the supplementary pages from your submission 
% 
\clearpage
% \setcounter{page}{1}
% \maketitlesupplementary
\begin{center}
Supplementary Material
\end{center}

% {
%     \onecolumn
%     \centering
%     \Large
%     \textbf{\thetitle}\\
%     \vspace{0.5em}Supplementary Material \\
%     \vspace{1.0em}
% }

\section{Proof of \cref{theorem:dr}}
We require some additional regularity assumptions:
\begin{assumption} 1) The number of classes $C$ is bounded w.r.t the number of samples $N$, 2) the missingness mechanism $P(A=1|Y,\theta)$, as well as its estimated counterpart $P(A=1|Y,\theta)$, are bounded below by some constant $\epsilon > 0$, 3) the quantities $P(Y|X,\theta)$ and $P(A|Y,\theta)$ are estimated using auxiliary samples independent of samples used for the sample averaging.
\label{assumption:extra}
\end{assumption}
Assumptions 1 and 2 are natural. For the missingness mechanism, the ground truth being bounded means that there is a non-vanishing proportion of samples for every class. The boundedness of the estimate can be enforced by clipping the estimate. Assumption 3 is called sample splitting in \cite{kennedy-dr}.

For convenience we use operator $\E_N$ to denote the average of $N$ samples i.e. $\frac{1}{N}\sum_{i=1}^N$. Note that this is by itself a random variable, in contrast to $\E$ which is a fixed number.

\begin{proof}[Proof of \cref{theorem:dr}] Because $C$ is bounded (assumption \ref{assumption:extra}), we can fix a class $c$ and prove the theorem.
Let us define the influence function $\phi$, parameterized by $\theta$, as
\begin{equation}
\phi(O | \theta)(c) = P(Y=c|X,\theta) + \frac{\one(A=1)}{P(A=1|Y,\theta)} (\one(Y=c) - P(Y=c|X,\theta)) - P(Y=c)
\end{equation}
As we have done in the main text, we use $\phi(O)$ to denote the same function but all estimated quantities are replaced with their truths. In other words, we use $\phi(O)$ for $\phi(O|\theta_0)$ where $\theta_0$ is the truth, given that our model contains $\theta_0$ e.g. when the model is consistent.

Recall that:
\begin{equation}
\begin{aligned}
\Psi_{dr}(\theta)(c) &= \frac{1}{N}\sum_{i=1}^N \left\{P(Y=c|X,\theta) + \frac{\one(A=1)}{P(A=1|Y,\theta)} (\one(Y=c) - P(Y=c|X,\theta))\right\}\\
&= \E_N [\phi(O|\theta)(c)] + P(Y=c)
\end{aligned}
\end{equation}

We will show that:
\begin{equation}
\Psi_{dr}(\theta)(c) - P(Y=c) = (\E_N - \E)[\phi(O)(c)] + o_P(N^{-1/2})
\label{eq:proof-linearity}
\end{equation}
To do that, we use the following decomposition
\begin{equation}
\begin{aligned}
\Psi_{dr}(\theta)(c) - P(Y=c) &= \E_N [\phi(O|\theta)(c)] \\
&= (\E_N - \E)[\phi(O)(c)] + (\E_N - \E)[\phi(O|\theta)(c) - \phi(O)(c)] + \E[\phi(O|\theta)(c)]
% &+ (\E_n - \E)[\phi(O;\theta) - \phi(O)]\\
% &+ \E[P(Y=c|X,\theta)] - \E[P(Y=c|X)] + \E[\phi(O,\theta)]
\end{aligned}
\end{equation}
and analyze the second and third term. The third term is:
\begin{equation}
\begin{aligned}
\E[\phi(O|\theta)(c)] &= \E[P(Y=c|X,\theta)] + \E\left[\frac{\one(A=1)}{P(A=1|Y,\theta)}(\one(Y=c) - P(Y=c|X,\theta))\right]- P(Y=c) \\
&= \E\left[P(Y=c|X,\theta) + \frac{P(A=1|Y)}{P(A=1|Y,\theta)}(P(Y=c|X) - P(Y=c|X,\theta))\right] - \E[P(Y=c|X)]\\
&= \E\left[(P(Y=c|X,\theta) - P(Y=c|X)) (P(A=1|Y,\theta) -P(A=1|Y)) \frac{1}{P(A=1|Y,\theta)}\right]\\
\end{aligned}
\end{equation}
by Cauchy-Schwarz inequality:
\begin{equation}
\begin{aligned}
\E[\phi(O|\theta)(c)] &\le \frac{1}{\epsilon} \|P(A=1|Y,\theta) - P(A=1|Y)\|_2 \|P(Y=c|X,\theta) - P(Y=c|X)\|_{L_2(P)}\\
&= \frac{1}{\epsilon} o_P(N^{-1/4} N^{-1/4}) = o_P(N^{-1/2})
\end{aligned}
\end{equation}
by assumption \ref{assumption:4th-root-n} and that $P(A=1|Y,\theta) > \epsilon$ (assumption \ref{assumption:extra}). The second term can be bounded by Chebyshev inequality
% \begin{equation}
% \begin{aligned}
% \E[\E_N[\phi(O|\theta)(c) - \phi(O)(c)]] &= \E[\phi(O|\theta)(c) - \phi(O)(c)]\\
% \var[\E_N[\phi(O|\theta)(c) - \phi(O)(c)]] &= \frac{1}{N}\var[\phi(O|\theta)(c) - \phi(O)(c)] \le 
% \end{aligned}
% \end{equation}
\begin{equation}
P(|(\E_N - \E)[\phi(O|\theta)(c) - \phi(O)(c)]| \ge t) \le \frac{\var[\E_N[\phi(O|\theta)(c) - \phi(O)(c)]]}{t^2} = \frac{\var[\phi(O|\theta)(c) - \phi(O)(c)]}{Nt^2}
\end{equation}
note here that $\theta$ is independent of the samples used for $\E_N$ by assumption \ref{assumption:extra}. For any $\varepsilon > 0$, by picking $t = \frac{1}{\sqrt{N\varepsilon}}$ we get
\begin{equation}
P\left(\left|\frac{(\E_N - \E)[\phi(O|\theta)(c) - \phi(O)(c)]}{N^{-1/2}}\right| \ge \frac{1}{\sqrt{\varepsilon}}\right) \le \varepsilon \var[\phi(O|\theta)(c) - \phi(O)(c)]
\end{equation}
by the definition of $O_P$, we then get
\begin{equation}
(\E_N - \E)[\phi(O|\theta)(c) - \phi(O)(c)] = O_P(N^{-1/2}\var[\phi(O|\theta)(c) - \phi(O)(c)])
\end{equation}
Because $\phi$ is a continuous function of $P(Y|X,\theta)$ and $P(A|Y,\theta)$ (given $P(A|Y,\theta) > \epsilon$, assumption \ref{assumption:extra}), by the continuous mapping theorem and the fact that $P(Y|X,\theta)$ and $P(A|Y,\theta)$ are convergent in probability (assumption \ref{assumption:4th-root-n}), we get $\var[\phi(O|\theta)(c) - \phi(O)(c)] = o_P(1)$. This gives
\begin{equation}
(\E_N - \E)[\phi(O|\theta)(c) - \phi(O)(c)] = o_P(N^{-1/2})
\end{equation}
Therefore, we have shown that the second and third term are both $o_P(N^{-1/2})$, proving \cref{eq:proof-linearity}. As the final step, multiply both sides of this equation by $\sqrt{N}$ we get:
\begin{equation}
\sqrt{N}(\Psi_{dr}(\theta)(c) - P(Y=c)) = \sqrt{N} (\E_N - \E)[\phi(O)(c)] + o_P(1) \rightsquigarrow \mathcal{N}(0, \var[\phi(O)(c)])
\end{equation}
by the central limit theorem, and $\var[\phi(O)(c)] = \E[\phi(O)(c)^2]$ because $\E[\phi(O)(c)] = 0$.
\end{proof}

While we started with the definition of $\phi$, \cref{eq:proof-linearity} shows that $\phi$ is indeed an influence function. Now we show that $\phi$ is also the efficient influence function, by using the characterization of the model's tangent space \cite{tsiatis-missingdata}. Note that the joint probability factorizes as $P(X,A,Y) = P(X)P(Y|X)P(A|Y)$, therefore the tangent space $\mathcal{T}$ factorizes as $\mathcal{T} = \mathcal{T}_{X} \oplus \mathcal{T}_{Y|X} \oplus \mathcal{T}_{A|Y}$ where $\mathcal{T}_X = \{h(X): \E[h] = 0\}$, $\mathcal{T}_{Y|X} = \{h(X,Y): \E[h|X] = 0\}$, $\mathcal{T}_{A|Y} = \{h(A,Y): \E[h|Y] = 0\}$, and the 3 subspaces are pairwise orthogonal. All influence functions are orthogonal to the tangent space, but the influence function that is also in the tangent space has the smallest variance and is called the efficient influence function. As $\phi$ is already an influence function, we need only show that $\phi$ is in $\mathcal{T}$. We write $\phi$ as
\begin{equation}
\phi(O)(c) = (P(Y=c|X) - P(Y=c)) + \left[\frac{\one(A=1)}{P(A=1|Y)} - 1\right](\one(Y=c) - P(Y=c|X)) + (\one(Y=c) - P(Y=c|X))
\end{equation}
and note that the first, second and third term are in $\mathcal{T}_X$, $\mathcal{T}_{A|Y}$ and $\mathcal{T}_{Y|X}$ respectively. Therefore, $\phi$ is indeed in $\mathcal{T}$. The efficient influence function has the smallest variance of all influence function, and therefore our estimator being asymptotically linear in $\phi$ (\cref{eq:proof-linearity}) has the smallest mean squared error in a local asymptotic minimax sense \cite{kennedy-dr, asymptoticstatistics}

\section{Further background and related work}
\paragraph{Discussion on semi-supervised EM.}
It appears that semi-supervised EM was first used for parameter estimation when the missingness mechanism is non-ignorable in \cite{ibrahim1996parameter}, but has not been used for label shift estimation.
Perhaps this is because the semi-supervised situation where additional unlabeled data is available during training is rarer than the test-time adaptation case. EM is well suited to take advantage of the extra unlabeled data to improve the classifier under very scarce and long-tailed labeled data. While the connection between pseudo-labeling and EM has been explored before \cite{entropyminimization}, the situation with label shift has not until recently \cite{simpro}. Here the application of EM is much more interesting, because other than simply giving pseudo-labeling a rigorous formulation, EM also estimates the missingness mechanism (equivalently the label distribution shift), which is important for shift correction and thus high-quality pseudo-labels \cite{acr}. The application of confidence thresholding can be seen as a sparse variant of EM \cite{neal1998view}.

\paragraph{The doubly-robust risk.} 
\label{subsec:dr-risk}
A technique that also derives from the theory of semi-parametric efficiency is orthogonal statistical learning \citep{foster2023orthogonal}. The idea is to minimize the doubly-robust risk:
\label{subsec:method-dr-risk}
\begin{equation}
\label{eq:dr-risk}
\mathcal{R}(\theta_2) = \frac{1}{N} \sum_{i=1}^N \Bigg[ l(x_i, \hat y_i|\theta_2) + \frac{\one(a_i=1)}{P(A=a_i|Y=y_i, \theta_1)} (l(x_i, y_i | \theta_2) - l(x_i, \hat y_i | \theta_2))\Bigg]
\end{equation}
where $l(x,y|\theta) = -\sum_{c=1}^C [y]_c \log P(Y=c|X=x,\theta)$ is the negative cross-entropy. 
The notation $[y]_c$ means that we are using the $c$-entry in a C-dimension probability vector $y$. 
Thus, $y_i$ denotes the one-hot label of observation $i$, while $\hat y_i$ denotes the pseudo-label, which can be one-hot or all-zero. 
Finally, we use $\theta_1$ to denote that $P(a|y,\theta_1)$ is an estimation from a previous stage, but it can be estimated with $\theta_2$ as well. 
The risk $\mathcal{R}(\theta_2)$ can be used as a training loss in a straightforward fashion. 
Similar to the doubly robust estimation of $P(Y)$, the doubly robust risk provides approximately unbiased estimation of the risk. 
This property has been used in \citep{arelabelsinformative, onnonrandommissinglabels, drst} also in the semi-supervised learning setting.
More broadly, it is at the heart of one of the core techniques in heterogenous treatment effect estimation in causal estimation \cite{kennedy2023towards, foster2023orthogonal, wager2018estimation}. 
The focus here is not the estimation of $\mathcal{R}(\theta_2)$ per se, but the quality of the learned model \cite{foster2023orthogonal}.
By using the doubly-robust risk, we can achieve an optimality result similar in spirit to our theorem \cref{theorem:dr}, but for the generalization error.
While this is appealing, in practice there are 2 problems with this approach. First, the inverse probability weight $P(A=a_i|Y=y_i,\theta_1)$ can be very large if the class ratio is highly unlabeled, making training unstable \cite{kallus2020deepmatch, pham2023stable}. 
This problem exists for our estimation as well. However, it is much easier to control for estimation than for training because of the iterative nature of model update. Secondly, we can further write $\mathcal{R}$ as:
\begin{equation}
\mathcal{R}(\theta_2) = \frac{1}{N}\sum_{i=1}^N l\left(x_i, \hat y_i + \frac{\one(a_i=1)}{P(A=a_i|Y=y_i,\theta_1)} (y_i - \hat y_i)\Bigg\vert\theta_2\right)
\end{equation}
which is a cross-entropy loss with new meta-pseudo-labels. However, these labels are not meant to be learned exactly, and furthermore they can be negative. Thus, theoretical works have to put stringent assumptions on the models. In \cref{subsec:ablation-1}, we show that experimentally that the instability problem makes doubly-robust risk performance worse than our 2-stage approach.

\section{Training and hyperparameter settings.}
\label{subsec:training-setting}
For neural network training, we follow the implementation and hyperparameter settings of \cite{simpro}. In particular, we adapt the core code of SimPro for Supervised, MLE and EM. For MLE, we update $P(A|Y)$ using the Adam optimizer with learning rate 1e-3, while for EM we use a momentum update similar to SimPro's update of $P(Y|A)$ because it has a a closed-form solution at each mini-batch. We use Wide ResNet-28-2 on all methods and all datasets in this section, including Imagenet-127, because we are motivated by the fact that stage-1's goal is not classification accuracy but the estimation of a finite-dimensional parameter. When using Wide ResNet-28-2 for Imagenet-127, we use the hyperparameters of CIFAR-100, except we lower the batch size of unlabeled data to 2 times that of labeled data instead of 8 for memory reason. We do not perform additional hyperparameter tuning. All experiments can be performed on 1 A6000 RTX GPU, and are run 3 times. We report the total variation distance between the estimated and the ground truth unlabeled class distribution, similar to its usage in Theorem 3.1 of \cite{lsc}, and the top-1 classification accuracy.

In the second stage of our algorithm, we freeze our estimation and plug it in SimPro and BOAT.
We keep exactly the same hyperparameter settings that SimPro and BOAT use. In particular, for Imagenet-127, we now use ResNet-50 and run each experiment once.
In SimPro, we set the unlabeled class distribution $P(Y|A=0)$ at the E-step;  however, we still keep a running estimate of the class distribution $P(Y)$ in the logit adjustment loss \cref{eq:simpro-la-loss}. While it is possible to use the first stage estimate in the logit adjustment loss, we observe that doing so results in lower accuracy than using the the running average. This is conceptually consistent with the role of the running average - serving not as an accurate estimate of $P(Y)$ but to make the classifier's class distribution uniform through the logit adjustment loss, which is good for the test set. Similarly, in BOAT, we only replace $\Delta_c = \log P(Y|A=1) - \log P(Y|A=0)$ in equation (4) of \cite{boat}, which is adjusting a classifier's predictions from the labeled to the unlabeled class distribution, with our SimPro + DR estimate instead of their on-the-fly estimate. 


% \section{Additional experiments}
% % \begin{table*}[t]
\centering
\caption{Total Variation Distance on CIFAR-10-LT ($N_l = 500$, $M_l = 4000$) with different class imbalance ratios $\gamma_l$ and $\gamma_u$ under five different unlabeled class distributions.}
\label{tab:cifar10-tv}
\resizebox{\textwidth}{!}{
\begin{tabular}{lccccccccccc}
\toprule
& & \multicolumn{2}{c}{consistent} & \multicolumn{2}{c}{uniform} & \multicolumn{2}{c}{reversed} & \multicolumn{2}{c}{middle} & \multicolumn{2}{c}{head-tail} \\
\cmidrule(lr){3-4} \cmidrule(lr){5-6} \cmidrule(lr){7-8} \cmidrule(lr){9-10} \cmidrule(lr){11-12}
& & $\gamma_l = 150$ & $\gamma_l = 100$ & $\gamma_l = 150$ & $\gamma_l = 100$ & $\gamma_l = 150$ & $\gamma_l = 100$ & $\gamma_l = 150$ & $\gamma_l = 100$ & $\gamma_l = 150$ & $\gamma_l = 100$ \\
Model & Estimator & $\gamma_u = 150$ & $\gamma_u = 100$ & $\gamma_u = 1$ & $\gamma_u = 1$ & $\gamma_u = 1/150$ & $\gamma_u = 1/100$ & $\gamma_u = 150$ & $\gamma_u = 100$ & $\gamma_u = 150$ & $\gamma_u = 100$ \\
\midrule
Supervised & MLLS & 0.269 ± 0.252 & 0.038 ± 0.006 & 0.251 ± 0.046 & 0.255 ± 0.060 & 0.429 ± 0.028 & 0.493 ± 0.050 & 0.333 ± 0.042 & 0.320 ± 0.009 & 0.457 ± 0.034 & 0.444 ± 0.043 \\
Supervised & RLLS & 0.043 ± 0.001 & 0.044 ± 0.010 & 0.348 ± 0.034 & 0.305 ± 0.068 & 0.769 ± 0.016 & 0.678 ± 0.028 & 0.430 ± 0.008 & 0.368 ± 0.013 & 0.539 ± 0.018 & 0.503 ± 0.020 \\
\midrule
MLE & IPW & 0.027 ± 0.001 & 0.027 ± 0.000 & 0.319 ± 0.072 & 0.243 ± 0.010 & 0.674 ± 0.020 & 0.646 ± 0.041 & 0.438 ± 0.020 & 0.454 ± 0.026 & 0.547 ± 0.049 & 0.491 ± 0.059 \\
MLE & OR & 0.045 ± 0.004 & 0.042 ± 0.000 & 0.215 ± 0.026 & 0.203 ± 0.032 & 0.433 ± 0.017 & 0.395 ± 0.033 & 0.193 ± 0.006 & 0.209 ± 0.037 & 0.307 ± 0.147 & 0.249 ± 0.130 \\
MLE & DR & 0.090 ± 0.002 & 0.079 ± 0.000 & 0.407 ± 0.027 & 0.360 ± 0.007 & 0.425 ± 0.007 & 0.421 ± 0.029 & 0.256 ± 0.001 & 0.286 ± 0.031 & 0.435 ± 0.136 & 0.362 ± 0.122 \\
\midrule
EM & IPW & 0.035 ± 0.002 & 0.040 ± 0.001 & 0.021 ± 0.001 & 0.029 ± 0.015 & 0.303 ± 0.187 & 0.091 ± 0.010 & 0.119 ± 0.011 & 0.105 ± 0.022 & 0.104 ± 0.026 & 0.104 ± 0.051 \\
EM & OR & 0.037 ± 0.003 & 0.042 ± 0.002 & 0.016 ± 0.001 & 0.024 ± 0.012 & 0.269 ± 0.183 & 0.090 ± 0.008 & 0.122 ± 0.012 & 0.103 ± 0.022 & 0.072 ± 0.012 & 0.073 ± 0.024 \\
EM & DR & 0.034 ± 0.004 & 0.037 ± 0.001 & 0.014 ± 0.001 & 0.027 ± 0.020 & 0.264 ± 0.191 & 0.092 ± 0.005 & 0.111 ± 0.019 & 0.097 ± 0.026 & 0.077 ± 0.016 & 0.073 ± 0.028 \\
\midrule
SimPro & IPW & 0.070 ± 0.011 & 0.058 ± 0.000 & 0.046 ± 0.001 & 0.049 ± 0.005 & 0.254 ± 0.074 & 0.223 ± 0.098 & 0.097 ± 0.025 & 0.067 ± 0.002 & 0.105 ± 0.066 & 0.110 ± 0.079 \\
SimPro & OR & 0.071 ± 0.012 & 0.058 ± 0.000 & 0.045 ± 0.001 & 0.049 ± 0.006 & 0.040 ± 0.003 & 0.059 ± 0.017 & 0.074 ± 0.006 & 0.075 ± 0.002 & 0.033 ± 0.003 & 0.033 ± 0.003 \\
SimPro & DR & 0.017 ± 0.004 & 0.026 ± 0.001 & 0.019 ± 0.002 & 0.018 ± 0.003 & 0.039 ± 0.003 & 0.058 ± 0.025 & 0.091 ± 0.007 & 0.031 ± 0.001 & 0.015 ± 0.003 & 0.019 ± 0.007 \\
\bottomrule
\end{tabular}
}
\end{table*}
% 

\begin{table*}[t]
\centering
\caption{Total Variation Distance on CIFAR-100-LT ($N_l = 50$, $M_l = 400$) with different class imbalance ratios $\gamma_l$ and $\gamma_u$ under five different unlabeled class distributions.}
\label{tab:cifar100-tv}
\resizebox{\textwidth}{!}{
\begin{tabular}{lccccccccccc}
\toprule
& & \multicolumn{2}{c}{consistent} & \multicolumn{2}{c}{uniform} & \multicolumn{2}{c}{reversed} & \multicolumn{2}{c}{middle} & \multicolumn{2}{c}{head-tail} \\
\cmidrule(lr){3-4} \cmidrule(lr){5-6} \cmidrule(lr){7-8} \cmidrule(lr){9-10} \cmidrule(lr){11-12}
& & $\gamma_l = 20$ & $\gamma_l = 10$ & $\gamma_l = 20$ & $\gamma_l = 10$ & $\gamma_l = 20$ & $\gamma_l = 10$ & $\gamma_l = 20$ & $\gamma_l = 10$ & $\gamma_l = 20$ & $\gamma_l = 10$ \\
Model & Estimator & $\gamma_u = 20$ & $\gamma_u = 10$ & $\gamma_u = 1$ & $\gamma_u = 1$ & $\gamma_u = 1/20$ & $\gamma_u = 1/10$ & $\gamma_u = 20$ & $\gamma_u = 10$ & $\gamma_u = 20$ & $\gamma_u = 10$ \\
\midrule
Supervised & MLLS & 0.707 ± 0.016 & 0.313 ± 0.100 & 0.445 ± 0.172 & 0.309 ± 0.119 & 0.383 ± 0.075 & 0.397 ± 0.006 & 0.570 ± 0.001 & 0.373 ± 0.107 & 0.543 ± 0.009 & 0.231 ± 0.057 \\
Supervised & RLLS & 0.520 ± 0.007 & 0.133 ± 0.003 & 0.337 ± 0.125 & 0.253 ± 0.082 & 0.424 ± 0.060 & 0.463 ± 0.003 & 0.454 ± 0.021 & 0.306 ± 0.074 & 0.460 ± 0.028 & 0.241 ± 0.040 \\
\midrule
MLE & IPW & 0.075 ± 0.000 & 0.071 ± 0.001 & 0.229 ± 0.001 & 0.167 ± 0.002 & 0.565 ± 0.005 & 0.443 ± 0.007 & 0.415 ± 0.000 & 0.311 ± 0.005 & 0.343 ± 0.000 & 0.280 ± 0.001 \\
MLE & OR & 0.065 ± 0.002 & 0.061 ± 0.001 & 0.200 ± 0.007 & 0.143 ± 0.001 & 0.526 ± 0.011 & 0.399 ± 0.023 & 0.360 ± 0.003 & 0.256 ± 0.012 & 0.328 ± 0.003 & 0.266 ± 0.005 \\
MLE & DR & 0.149 ± 0.019 & 0.145 ± 0.010 & 0.243 ± 0.004 & 0.214 ± 0.019 & 0.568 ± 0.005 & 0.464 ± 0.014 & 0.403 ± 0.014 & 0.309 ± 0.012 & 0.365 ± 0.007 & 0.320 ± 0.004 \\
\midrule
EM & IPW & 0.097 ± 0.008 & 0.092 ± 0.004 & 0.239 ± 0.007 & 0.179 ± 0.003 & 0.478 ± 0.012 & 0.329 ± 0.020 & 0.262 ± 0.016 & 0.202 ± 0.003 & 0.312 ± 0.002 & 0.227 ± 0.001 \\
EM & OR & 0.121 ± 0.007 & 0.108 ± 0.005 & 0.261 ± 0.007 & 0.189 ± 0.004 & 0.489 ± 0.013 & 0.335 ± 0.020 & 0.274 ± 0.016 & 0.211 ± 0.004 & 0.336 ± 0.003 & 0.235 ± 0.001 \\
EM & DR & 0.125 ± 0.005 & 0.111 ± 0.004 & 0.269 ± 0.007 & 0.194 ± 0.005 & 0.497 ± 0.010 & 0.336 ± 0.024 & 0.281 ± 0.019 & 0.219 ± 0.008 & 0.336 ± 0.007 & 0.233 ± 0.004 \\
\midrule
SimPro & IPW & 0.125 ± 0.001 & 0.100 ± 0.005 & 0.166 ± 0.007 & 0.141 ± 0.009 & 0.353 ± 0.023 & 0.261 ± 0.008 & 0.202 ± 0.003 & 0.158 ± 0.005 & 0.277 ± 0.009 & 0.197 ± 0.003 \\
SimPro & OR & 0.133 ± 0.005 & 0.100 ± 0.004 & 0.160 ± 0.007 & 0.138 ± 0.010 & 0.322 ± 0.014 & 0.253 ± 0.008 & 0.202 ± 0.003 & 0.156 ± 0.005 & 0.269 ± 0.006 & 0.191 ± 0.004 \\
SimPro & DR & 0.122 ± 0.003 & 0.106 ± 0.006 & 0.188 ± 0.009 & 0.149 ± 0.006 & 0.343 ± 0.023 & 0.257 ± 0.007 & 0.219 ± 0.010 & 0.172 ± 0.002 & 0.279 ± 0.007 & 0.198 ± 0.004 \\
\bottomrule
\end{tabular}
}
\end{table*}


\end{document}
