\section{Related Work}
\subsection{Cross-Modal Retrieval}
% with representatives like cross-attention \cite{scan} and similarity graph \cite{sgraf, cmgm},
Cross-modal retrieval (CMR) aims to search the most relevant items across different modalities in response to query modality. The core of CMR is to minimize the semantic discrepancies by projecting different modalities into a common comparable space, wherein the matched items manifest higher similarity or closer feature distance and vice versa. Current efforts, from the perspective of similarity calculation, can be roughly classified into two categories: 1) Coarse-grained measurement \cite{vse++,vsrn,gpo,vtsr}, which represents an efficient solution with the key idea of associating the correspondence holistically among features extracted by distinct modality-specific encoders. 2) Fine-grained measurement \cite{scan,naaf,cmgm, sgraf, lsra, cross-graph,chan}, which focuses on assessing the semantic relationships at a more granular level to learn and reason latent alignments between fragments. Unfortunately, the promising performance of all these methods relies heavily on an implicit assumption that all training data pairs are correctly matched while neglecting the presence of NC. Such NC inevitably undermines the alignments and complicates the accurate measurement of similarity, ultimately leading to inferior performance.

% by well-designed consistency interactions, 
%, which is often introduced due to the extensive data collection and annotation expenses
\subsection{Noisy Correspondence Learning}
% , which, recent years, has been widely explored in multi-modal tasks, including image-text matching \cite{ncr,rcl}, multi-view clustering \cite{multiview-1,multiview-2}, image captioning \cite{imagecap1,imagecap2}, and infrared re-identification \cite{reident1,reident2}.
Noisy correspondence learning refers to noise-tolerant approaches well-designed to effectively mitigate the adverse impacts caused by mismatches among dataset. Unlike traditional category-level mistaken annotations, this instance-level semantic inconsistency, first recognized as a new paradigm of noisy labels in \cite{ncr}, significantly affects the performance of retrieval models. Thus, some prior attempts \cite{ncr, BiCro,MSCN} employ the small-loss criterion \cite{dividemix} to identify matched pairs from the corrupted datasets and subsequently rectify soft correspondence labels for those mismatches. Following this, several works \cite{cream,trip,UGNCL} introduce novel criteria to enable more fine-grained data division, such as inconsistent predictions \cite{cream,trip} and uncertainty \cite{UGNCL}. To avoid inaccurate label predictions, some approaches \cite{l2rm,PC2} aim to refine alignments through alternative strategies like rematched mismatches \cite{l2rm} and pseudo captions \cite{PC2}. Besides these methods based data sanitized, other efforts \cite{decl,rcl,crcl} retort to robust loss functions to adaptively downweight the contributions of mismatches, e.g., evidential loss \cite{decl}, complementary contrast loss \cite{rcl}, and active complementary loss \cite{crcl}. Recently, some works \cite{esc,gsc} utilize the intrinsic properties observed within data to estimate accurate soft correspondence labels. However, thery all neglect the intra-modal relations, which is significantly crucial for accurately identify true correspondences.