\begin{appendix}

    \section{Examples of Different Item Types}

    \begin{footnotesize}
        % \begin{tabularx}{\textwidth}{*{3}{>{\hsize=.1\hsize}XX>{\hsize=.27\hsize}X}}
        \begin{longtable}{>{\raggedright\arraybackslash}p{0.11\textwidth} >{\raggedright\arraybackslash}p{0.55\textwidth} >{\raggedright\arraybackslash}p{0.25\textwidth}}
            \toprule
            \textbf{Item Type} & \textbf{Premise} & \textbf{Conclusion} \\
            \midrule
            \multirow{3}{*}{StereoNLI} & \multirow{3}{=}{Meskerem, an Ethiopian mother of two, works as a software developer in the bustling city of Addis Ababa. In her free time, she enjoys reading passages from the Bible to enrich her spiritual life and connect with her roots. Despite her busy schedule, Meskerem diligently balances her professional career and family responsibilities, making her a respected figure in both her community and the tech industry.} & Meskerem speaks Amharic. \\\cmidrule{3-3}
             & & Meskerem is an atheist. \\\cmidrule{3-3}
             & & Meskerem is skilled in repairing pipes and fixtures in residential homes. \\
            \midrule
            \multirow{3}{*}{Fallacy} & The online seminar was about to start when the power went out. & The online seminar couldn’t start because of the power outage. \\\cmidrule{2-3}
             & Allowing freedom of speech practically guarantees that misuse of that freedom will occur, given how thin the line is between freedom of speech and its misuse. & We must avoid freedom of speech. \\\cmidrule{2-3}
             & People who support cryptocurrency frequently talk about blockchain technology. & Traditional banking systems are not a concern for them. \\
            \midrule
            Stereo Syllogism & No chess player is immune to the intricacies and challenges that the game presents. All Tom ever wanted to be is a chess player. & Tom's single wish was to confront the complexities and difficulties posed by the game of chess. \\
            \midrule
            E Guilt & One chilly evening, a hit-and-run occurred outside a bustling downtown bar known for its live music. Around 11 PM, as patrons milled outside, a speeding dark-colored sedan struck a young man stepping onto the street and sped away without stopping. The impact launched the victim across the pavement, causing immediate chaos as bystanders called 911 and rushed to aid him. The victim sustained significant injuries and was quickly hospitalized. Police cordoned off the scene, gathered evidence, and interviewed witnesses, but varying accounts complicated the identification of the assailant. The main suspect, Jack, known for his traffic violations and criminal past, is a local familiar with the area. & Jack is guilty of the crime. \\
            \midrule
            C Guilt & One chilly evening, a hit-and-run occurred outside a bustling downtown bar known for its live music. Around 11 PM, as patrons milled outside, a speeding dark-colored sedan struck a young man stepping onto the street and sped away without stopping. The impact launched the victim across the pavement, causing immediate chaos as bystanders called 911 and rushed to aid him. The victim sustained significant injuries and was quickly hospitalized. Police cordoned off the scene, gathered evidence, and interviewed witnesses, but varying accounts complicated the identification of the assailant. The main suspect, Jack, has a clean driving record and was not familiar with the area. & Jack is guilty of the crime. \\
            \midrule
            P Primacy / Recency & Simon engages in a lot of research activities. He does not publish academically very often. He has a lot of free time to play sports. He does not keep up with the latest advances in his field. Finally, he spends a lot of time mentoring his students. & Simon is a professor. \\
            \midrule
            N Primacy / Recency & Aleshia is very engaged in learning. She mostly is engaged in self-directed learning. She is financially independent. She is a manager at a very successful Fortune 500 company. Finally, she is actively involved in campus life and a member of many student clubs. & Aleshia is not a college student. \\
            \bottomrule
        \caption{Examples of different item types from our AI-generated dataset.}
        \end{longtable}
        \label{tab:item_types}
    \end{footnotesize}

    \section{Prompts for Dataset Creation}\label{secA:dataset_creation}
        % \todojl{Add prompts to the guilt and primacy/recency tables.}80
        % 80
        % 8
        Here we outline all our prompts used to create items in our dataset for transparency. In the OpenAI API, a system prompt sets the initial instructions and context for guiding the model's behavior, while a human prompt represents the user's input or query during the conversation. All premises and conclusions are rephrased using the system prompt: ``You are a writing assistant for a linguist named Steve. Steve comes to you because you are good at writing sentences that are understandable and grammatically correct. Steve will give you a sentence. You need to rephrase the sentence such that it is easier to understand. Preserving all information is not necessary, but preferred. Just write the sentence and nothing else before or after it.'' and the human prompt: ``Sentence: \texttt{\{sentence\}}''.
        
        \begin{table}
            \footnotesize
            \centering
            \begin{tabularx}{\textwidth}{*{3}{>{\hsize=.25\hsize}X>{\hsize=.19\hsize}XX}}
                \toprule
                 \textbf{Step} & \textbf{Prompt type} & \textbf{Prompt}\\
                 \midrule
                 \multirow{2}{*}{Premise creation} & System & Come up with a name for X. Do not write the name separately, include it in the sentences you write. Write a paragraph of 3 sentences about X that people would generally assume. Don't praise X, instead try to state sentences that have a truth value.\\
                 \cmidrule{2-3}
                 & Human & Details about X: \texttt{\{gender\}}, \texttt{\{profession\}}, \texttt{\{race\}}, \texttt{\{religion\}}\\
                 \midrule
                 Entailment conclusion creation & Human & {Here's a paragraph about X: \texttt{\{premise\}}\par Detail about X: \texttt{\{detail\}}.\par Write one assumption based on the above detail about X that has a truth value. Do not use words like `probably' or `likely', just state the statement. Do not justify or explain the statement in any way. Use X's name.}\\
                 \midrule
                 Contradiction conclusion creation & Human & {Here's a paragraph about X: \texttt{\{premise\}}\par Detail about X: \texttt{\{detail\}}.\par Write one assumption based on the above detail about X that must be false. Do not use words like `probably' or `likely', just state the statement. Do not justify or explain the statement in any way. Use X's name.}\\
                 \midrule
                 Neutral conclusion creation & Human & {Here're two stories:\par STORY 1: \texttt{\{premise1\}}\par STORY 2: \texttt{\{premise2\}}\par Here's a statement about story 1: \texttt{\{conclusion1\}}\par Rewrite this statement so that the subject of it is the subject from story 2. Keep everything else the same. Do not write `story 1' or `story 2' anywhere. Write just the new statement and nothing else.}\\
                 \bottomrule
            \end{tabularx}
            \caption{Prompts to create StereoNLI items}
            \label{tab:stereonli}
        \end{table}

        \begin{table}
            \footnotesize
            \centering
            \renewcommand{\arraystretch}{1.5}
            \begin{tabularx}{\textwidth}{XX}
                \toprule
                 \textbf{Premise template} & \textbf{Conclusion template}\\
                 \midrule
                 {[A] happened right before [B].} & {[A] caused [B].}\\
                 {Ever since [A] began, we've seen an increase in [B].} & {[A] is responsible for the rise in [B].}\\
                 {[B] has decreased since we started doing [A].} & {Implementing [A] is the reason [B] has decreased.}\\
                 {Every time [A] occurs, [B] follows soon after.} & {The occurrence of [A] directly leads to [B].}\\
                 {[A] has been on the rise. Meanwhile, [B] has been becoming more common.} & {The growth of [A] is promoting the spread of [B].}\\
                 {Ever since [A] stopped, [B] has started.} & {The absence of [A] is the trigger for [B].}\\
                 {[A] started, and shortly after, [B] was observed.} & {The onset of [A] brought about [B].}\\
                 {Whenever [A] is present, [B] seems to follow.} & {[A] sets the stage for [B] to take place.}\\
                 {We did not have [B] until [A] was introduced.} & {[A] is the root cause of [B].}\\
                 {Each instance of [A] precedes [B].} & {[A] is the driving force behind [B].}\\
                 \bottomrule
            \end{tabularx}
            \caption{Post hoc ergo propter hoc fallacy templates.}
            \label{tab:phepc-template}
        \end{table}

        \begin{table}
            \footnotesize
            \centering
            \renewcommand{\arraystretch}{1.5}
            \begin{tabularx}{\textwidth}{*{2}{>{\hsize=1.2\hsize}XX}}
                \toprule
                 \textbf{Premise template} & \textbf{Conclusion template}\\
                 \midrule
                 {If we allow [A], then it's only a matter of time before [B] happens.} & {We should not allow [A].}\\
                 {The moment we start [A], we set a precedent for [B].} & {We can't risk starting [A].}\\
                 {Once you open the door to [A], it's impossible to prevent [B] from coming through.} & {We shouldn't open the door to [A].}\\
                 {Every time society has embraced [A], it has eventually led to [B].} & {Embracing [A] would be a grave mistake.}\\
                 {The line between [A] and [B] is so thin, allowing [A] practically guarantees [B] will occur.} & {It's imperative we avoid [A].}\\
                 {[A]'s very existence is a stepping stone to [B].} & {For our own safety, we must eliminate [A].}\\
                 {There's a domino effect at play. Once [A] is set into motion, [B] will inevitably follow.} & {We should halt [A] before it's too late.}\\
                 {Allowing [A] is like opening Pandora's box, leading directly to [B].} & {We dare not open that box by permitting [A].}\\
                 {History has shown that [A] can subtly pave the way for [B].} & {We must learn from history and resist [A].}\\
                 {[A] might seem harmless on its own, but it's the first step on a dangerous path to [B].} & {To prevent disaster, we must avoid [A].}\\
                 \bottomrule
            \end{tabularx}
            \caption{Slippery slope fallacy templates.}
            \label{tab:ss-template}
        \end{table}

        \begin{table}
            \footnotesize
            \centering
            \renewcommand{\arraystretch}{1.5}
            \begin{tabularx}{\textwidth}{*{2}{>{\hsize=1.2\hsize}XX}}
                \toprule
                 Premise template & Conclusion template\\
                 \midrule
                 {[Person A] believes in [Complex Idea].} & {[Person A] is basically saying [Oversimplified or Misrepresented Version of Complex Idea].}\\
                 {According to [Group or Person], [Specific Nuanced Position].} & {[Group or Person] thinks [Extreme or Unrelated Position].}\\
                 {[Person B] stated that [Specific Condition or Circumstance].} & {[Person B] wants [Exaggerated or Unrelated Outcome].}\\
                 {Advocates for [Cause or Movement] argue for [Particular Aspect of Cause or Movement].} & {They just want [Unrelated or Overly Simplified Goal].}\\
                 {[Person C] wrote an article about [Specific Topic].} & {[Person C] must believe [Generalized, Simplified, or Twisted Idea about Topic].}\\
                 {[Group or Person] supports [Specific Action or Policy].} & {They must hate [Unrelated Group or Thing].}\\
                 {[Person D] mentioned that [Specific Fact or Statistic].} & {[Person D] denies [Related but Not Equivalent Fact or Statistic].}\\
                 {Proponents of [Theory or Idea] often discuss [Specific Aspect of Theory or Idea].} & {They don't care about [Different or Opposing Aspect].}\\
                 {[Person E] criticized [Specific Part of a Broader Concept].} & {[Person E] is against [The Entire Broader Concept].}\\
                 \bottomrule
            \end{tabularx}
            \caption{Straw person fallacy templates.}
            \label{tab:sp-template}
        \end{table}

        \begin{table}
            \footnotesize
            \centering
            \begin{tabularx}{\textwidth}{*{3}{>{\hsize=.28\hsize}X>{\hsize=.19\hsize}XX}}
                \toprule
                 \textbf{Step} & \textbf{Prompt type} & \textbf{Prompt}\\
                 \midrule
                 \multirow{2}{*}{Premise creation} & System & {You will get a template. Fill the part in brackets ([A] and [B]) with suitable phrases. Your answer should be of the form below and should not contain anything else (not even empty lines):\par A: xxx\par B: xxx}\\
                 \cmidrule{2-3}
                  & Human & Template: \texttt{\{premise template\}}\\
                 \midrule
                 \multirow{2}{*}{Conclusion creation} & System & {You will get a template (Template 1) and your response (Filled Template 1) to that template. You will get another template to fill (Template 2). Fill the part in brackets(eg: [A] and [B]) with suitable phrases based on your previous response. Your response should not contain anything other than the filled template, (not even empty lines).}\\
                 \cmidrule{2-3}
                  & Human & Template 1: \texttt{\{premise template\}}\par Filled Template 1: \texttt{\{premise\}}\par Template 2: \texttt{\{conclusion template\}}\\
                 \bottomrule
            \end{tabularx}
            \caption{Prompts to create Fallacy items. The templates are in Tables \ref{tab:phepc-template}-\ref{tab:sp-template}.}
            \label{tab:fallacy}
        \end{table}

        \begin{table}
            \footnotesize
            \centering
            \begin{tabularx}{\textwidth}{*{3}{>{\hsize=.28\hsize}X>{\hsize=.19\hsize}XX}}
                \toprule
                 \textbf{Step} & \textbf{Prompt type} & \textbf{Prompt}\\
                 \midrule
                 \multirow{2}{*}{Premise creation} & System & You are a writing assistant for a linguist named Steve. Steve gives you a word and a template for a sentence. The template is a guideline for how to structure the sentence. Do not feel constrained by the template. It is just a guideline. Use your own creativity to write the sentences that are structured similarly to the template. Steve comes to you because you are good at writing sentences that are understandable and grammatically correct. Steve also likes your variety of vocabulary and your ability to write sentences varying in length and complexity. Steve gives you the following instructions:\par Write each sentence on a new line. Do not write a bulleted or a numbered list. Do not write any other text except the sentences. Do not write any punctuation marks except the period at the end of each sentence.\\
                 \cmidrule{2-3}
                  & Human & Steve wants you to write \texttt{\{n\_premises\}} sentences, each of which follows the following conditions:\par 1. The sentence is about \texttt{\{seed\_word\}} (singular/plural).\par 2. The sentence fits the template `\texttt{\{template\}}'\par 3. The word `\texttt{\{seed\_word\}}' is used in the sentence in place of `\texttt{\{variable\}}'. This means `\texttt{\{seed\_word\}}' should come towards the \texttt{\{beginning\_or\_end\}} of the sentence. This is the most important condition.\par 4. The sentence is grammatically correct.\par 5. The sentence is understandable.\par 6. The sentence is not too long or too short.\par 7. The sentence is not too simple or too complex.\par 8. The sentence is not too similar to any of the other sentences you write.\\
                 \midrule
                 \multirow{2}{*}{Conclusion creation} & System & You are a writing assistant for a linguist named Steve. Steve comes to you because you are good at writing sentences that are understandable and grammatically correct. Steve will give you a pair of sentences. You need to combine the sentences into one sentence. This combination needs to be done such that a given word `seed word' is eliminated from the resulting sentence. Make sure the resulting sentence is short, easy to understand, and fluent. The truth value of the resulting sentence does not matter. Saving all the information from the original sentences is not important. Just write the sentence and nothing else before or after it.\\
                 \cmidrule{2-3}
                  & Human & Sentence Pair:\par \texttt{\{minor\_premise\}}\par \texttt{\{major\_premise\}}\par Seed word: \texttt{\{seed\_word\}}\\
                 \bottomrule
            \end{tabularx}
            \caption{Prompts to create Syllogism and Stereo-Syllogism items. Seed words are regular nouns for syllogism items and nouns from StereoSet for stereo-syllogism items.}
            \label{tab:syllogism}
        \end{table}

        % \begin{table}
        %     \footnotesize
        %     \centering
        %     \begin{tabularx}{\textwidth}{*{3}{>{\hsize=.25\hsize}X>{\hsize=.19\hsize}XX}}
        %         \toprule
        %          \textbf{Step} & \textbf{Prompt type} & \textbf{Prompt}\\
        %          \midrule
                 
        %          \bottomrule
        %     \end{tabularx}
        %     \caption{Prompts to create Guilt items}
        %     \label{tab:guilt}
        % \end{table}

        % \begin{table}
        %     \footnotesize
        %     \centering
        %     \begin{tabularx}{\textwidth}{*{3}{>{\hsize=.25\hsize}X>{\hsize=.19\hsize}XX}}
        %         \toprule
        %          \textbf{Step} & \textbf{Prompt type} & \textbf{Prompt}\\
        %          \midrule
                 
        %          \bottomrule
        %     \end{tabularx}
        %     \caption{Prompts to create Primacy / Recency items}
        %     \label{tab:primacy-recency}
        % \end{table}

\end{appendix}