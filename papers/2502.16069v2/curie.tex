%%%%%%%% ICML 2025 EXAMPLE LATEX SUBMISSION FILE %%%%%%%%%%%%%%%%%

\documentclass{article}

% Recommended, but optional, packages for figures and better typesetting:
\usepackage{microtype}
\usepackage{graphicx}
% \usepackage{subfigure}
\usepackage{booktabs} % for professional tables

% hyperref makes hyperlinks in the resulting PDF.
% If your build breaks (sometimes temporarily if a hyperlink spans a page)
% please comment out the following usepackage line and replace
% \usepackage{icml2025} with \usepackage[nohyperref]{icml2025} above.
\usepackage{hyperref}

% Personal/user packages:
\usepackage{multirow}
\usepackage{subcaption}
\usepackage{tcolorbox}
\usepackage{verbatim}
\usepackage{multicol} % for multiple columns
\usepackage{circledsteps}
\usepackage{xspace}
\newcommand{\sys}{\texttt{Curie}\xspace}

% Attempt to make hyperref and algorithmic work together better:
\newcommand{\theHalgorithm}{\arabic{algorithm}}

% Use the following line for the initial blind version submitted for review:
% \usepackage{icml2025}

% If accepted, instead use the following line for the camera-ready submission:
\usepackage[accepted]{icml2025}

% For theorems and such
\usepackage{amsmath}
\usepackage{amssymb}
\usepackage{mathtools}
\usepackage{amsthm}

% if you use cleveref..
\usepackage[capitalize,noabbrev]{cleveref}

%%%%%%%%%%%%%%%%%%%%%%%%%%%%%%%%
% THEOREMS
%%%%%%%%%%%%%%%%%%%%%%%%%%%%%%%%
\theoremstyle{plain}
\newtheorem{theorem}{Theorem}[section]
\newtheorem{proposition}[theorem]{Proposition}
\newtheorem{lemma}[theorem]{Lemma}
\newtheorem{corollary}[theorem]{Corollary}
\theoremstyle{definition}
\newtheorem{definition}[theorem]{Definition}
\newtheorem{assumption}[theorem]{Assumption}
\theoremstyle{remark}
\newtheorem{remark}[theorem]{Remark} 

\newenvironment{packeditemize}{
\begin{itemize}
  \setlength{\itemsep}{0.3pt}
  \setlength{\parskip}{2pt}
  \setlength{\parsep}{0pt}
}{\end{itemize}}

\newenvironment{packedenumerate}{
\begin{enumerate}
  \setlength{\itemsep}{0.3pt}
  \setlength{\parskip}{2pt}
  \setlength{\parsep}{0pt}
}{\end{enumerate}}

% Todonotes is useful during development; simply uncomment the next line
%    and comment out the line below the next line to turn off comments
%\usepackage[disable,textsize=tiny]{todonotes}
\usepackage[textsize=tiny]{todonotes}


% The \icmltitle you define below is probably too long as a header.
% Therefore, a short form for the running title is supplied here:
% \icmltitlerunning{Submission and Formatting Instructions for ICML 2025}
\icmltitlerunning{\sys: Toward Rigorous and Automated Scientific Experimentation with AI Agents}
\begin{document}

\twocolumn[
\icmltitle{\sys: Toward Rigorous and Automated \\ Scientific Experimentation with AI Agents  
}

% It is OKAY to include author information, even for blind
% submissions: the style file will automatically remove it for you
% unless you've provided the [accepted] option to the icml2025
% package.

% List of affiliations: The first argument should be a (short)
% identifier you will use later to specify author affiliations
% Academic affiliations should list Department, University, City, Region, Country
% Industry affiliations should list Company, City, Region, Country

% You can specify symbols, otherwise they are numbered in order.
% Ideally, you should not use this facility. Affiliations will be numbered
% in order of appearance and this is the preferred way.
\icmlsetsymbol{equal}{*}



\begin{icmlauthorlist}
\icmlauthor{Patrick Tser Jern Kon}{equal,yyy}
\icmlauthor{Jiachen Liu}{equal,yyy}
\icmlauthor{Qiuyi Ding}{yyy}
\icmlauthor{Yiming Qiu}{yyy}
\icmlauthor{Zhenning Yang}{yyy}
\icmlauthor{Yibo Huang}{yyy}
\icmlauthor{Jayanth Srinivasa}{comp}
%\icmlauthor{}{sch}
\icmlauthor{Myungjin Lee}{comp}
\icmlauthor{Mosharaf Chowdhury}{yyy}
\icmlauthor{Ang Chen}{yyy}
%\icmlauthor{}{sch}
%\icmlauthor{}{sch}
\end{icmlauthorlist}

\icmlaffiliation{yyy}{Department of Computer Science and Engineering, University of Michigan}
\icmlaffiliation{comp}{Cisco Systems}



\icmlcorrespondingauthor{Patrick Tser Jern Kon}{patkon@umich.edu}
\icmlcorrespondingauthor{Jiachen Liu}{amberljc@umich.edu}



% You may provide any keywords that you
% find helpful for describing your paper; these are used to populate
% the "keywords" metadata in the PDF but will not be shown in the document
% \icmlkeywords{Machine Learning, ICML}

\vskip 0.3in
]

% this must go after the closing bracket ] following \twocolumn[ ...

% This command actually creates the footnote in the first column
% listing the affiliations and the copyright notice.
% The command takes one argument, which is text to display at the start of the footnote.
% The \icmlEqualContribution command is standard text for equal contribution.
% Remove it (just {}) if you do not need this facility.

% \printAffiliationsAndNotice{}
\printAffiliationsAndNotice{\icmlEqualContribution} % otherwise use the standard text.

\begin{abstract}
Scientific experimentation, a cornerstone of human progress, demands rigor in reliability, methodical control, and interpretability to yield meaningful results. Despite the growing capabilities of large language models (LLMs) in automating different aspects of the scientific process, automating rigorous experimentation remains a significant challenge.
To address this gap, we propose \sys, an AI agent framework designed to embed rigor into the experimentation process through three key components: an intra-agent rigor module to enhance reliability, an inter-agent rigor module to maintain methodical control, and an experiment knowledge module to enhance interpretability. 
To evaluate \sys, we design a novel experimental benchmark composed of 46 questions across four computer science domains, derived from influential research papers, and widely adopted open-source projects. 
% Compared to the strongest agent framework baseline, the evaluation results show that \sys achieves 1.2$\times$ improvement in experiment design correctness, 2.4$\times$ improvement in execution reproducibility and 3.4$\times$ improvement in conclusion accuracy. 
% Compared to the strongest agent baseline, we achieve a 3.4$\times$ improvement in correctly answering experimental questions.
Compared to the strongest baseline tested, we achieve a 3.4$\times$ improvement in correctly answering experimental questions.
\sys is open-sourced at \url{https://github.com/Just-Curieous/Curie}.

% Scientific research is the foundation of human progress, and its importance has spurred a plethora of solutions leveraging the power of LLMs to automate aspects of the scientific process. 
% Yet, experimentation—a fundamental pillar of research—has been largely overlooked, often relegated to simplistic, prompt-based approaches. This neglect undermines the precision, reproducibility, and systematic control that collectively define experimental rigor—the bedrock of scientific experimentation.

\if 0
Scientific experimentation is important, and hence many papers are starting to work on it. However they lack rigor, which is X. To address this, we propose Curie, a AI agent framework designed to provide rigor; through 3 main components, an intra-agent rigor that ensures, an inter-agent that ensures control, and an interface to make sure progress is X. To evaluate Curie, we design an experimental benchmark derived from X, Our results show that Curie XYZ. 

% Scientific experimentation, a critical component of scientific research 
Scientific research drives innovation and discovery, yet the experimentation process remains complex, resource-intensive, and prone to errors.
Recent advancements in AI agents offer promising opportunities to automate scientific research, but ensuring rigor and reproducibility remains an unresolved challenge.

We propose \sys, an AI agent framework designed to fully automate the experimentation process while systematically ensuring rigor and reproducibility. 
Beyond automating basic workflows, Curie incorporates a dedicated Rigor Module that enforces correctness at both intra-agent and inter-agent levels, ensuring consistency throughout experiment design, execution, and result analysis.


To evaluate Curie, we introduce an experimentation benchmark derived from real-world research papers and widely adopted open-source projects. 
The benchmark consists of tasks designed to evaluate whether a framework can reproduce, extend, or even challenge insights from established works across three key topics: LLM reasoning, vector indexing, cloud computing and ML training.
Our results demonstrate that Curie consistently outperforms state-of-the-art baselines, achieving superior rigor, reproducibility, and overall performance, paving the way for AI-driven automation of complex scientific workflows.
\todo{add quantitive numbers.}
\fi

\end{abstract}


\section{Introduction}
\label{sec:intro} 

\if 0
Thoughts:

- if not then not rigor... its reckels.  

- Existing work has delved into scientific research end to end, but they are not doing experimentation actually, treat experimentation as a byproduct because that requires rigor, but science requires critical thinking. 
\fi

Scientific research drives human progress, advancing medicine, technology, and our understanding of the universe. 
At the heart of this endeavor lies experimentation—a disciplined intellectual pursuit that transforms human curiosity, expressed through bold hypotheses, into verifiable knowledge. 
Experimentation thrives on creativity, as new ideas fuel discovery. 
Yet it also depends on rigor—ensuring that research is methodologically sound and its findings are trustworthy~\cite{rigor2, rigor3}.
\textit{If science isn’t rigorous, it’s reckless}~\cite{rigor1}.
% Yet it also depends on rigor—if science isn’t rigorous, it’s reckless.~\cite{rigor1} Ensuring that research is methodologically sound and its findings are trustworthy preserves the power of scientific inquiry.

\if 0
A significant driver of this acceleration is the rise of large language models (LLMs) and LLM-driven agents, which have evolved from low-stakes applications, such as chatbots~\cite{openai} and gameplay~\cite{voyager}, to high-stakes domains, including cloud management~\cite{cloud1, terrafault, lilac} and autonomous scientific discovery~\cite{zhang2024comprehensive,auto-science1,lu2024ai}. 
% LLMs themselves are products of this rapid research cycle, continuously refined through experimentation. 
% LLM-driven agents actively participate in automating experimentation, assisting in data analysis~\cite{hong2024data, chen2024scienceagentbench}, and even generating hypotheses~\cite{sourati2023accelerating, hypothesis1, hypothesis2, hypothesis3}. 
As these AI agents become integral to scientific exploration, they demand the same level of scrutiny as human researchers and must be held to the same rigorous standards as human researchers. 
\fi

In recent years, numerous works~\cite{zhang2024comprehensive,auto-science1,lu2024ai} leveraging large language models (LLMs) to automate scientific research have emerged (\S\ref{subsec:related-work}). 
These solutions typically rely on ad-hoc prompt-based methods to mimic scientific workflows, which are prone to hallucination.
While effective for creative tasks such as literature review and brainstorming, these approaches remain limited in their ability to support rigorous experimentation, a largely unexplored capability.

\begin{figure}
    \centering
    \includegraphics[width=0.99\linewidth, trim=50 60 50 70, clip]{figures/overview.png}
    \caption{Curie overview.}
    % \vspace{-7mm}
    \label{fig:curie-workflow}
\end{figure} 

\begin{figure*}
  \centering
  \includegraphics[width=0.99\linewidth]{figures/case_study.png}
\caption{
Case Study. \sys can help researchers validate, expand, and critique existing research on the benefits of repeated sampling in LLM reasoning~\cite{brown2024large}. 
The first panel (Original Finding) presents a result from the original paper. 
The second panel (Reproduce) has \sys confirming this finding through rigorous experimentation.
The third panel (Extend) has \sys exploring the impact of sampling temperature on repeated sampling.
The final panel (Challenge) shows \sys identifying a limitation in the original methodology, suggesting an avenue for future research.
}
  \label{fig:case-study}
\end{figure*} 
 

\if 0
Case Study. \sys can help researchers reproduce, extend, and challenge existing research on repeated sampling in LLM reasoning~\cite{brown2024large}. 
 The first panel (Original Finding) presents a result from the original paper. 
 The second panel (Reproduce) demonstrates that \sys successfully validates this claim by following a structured experimental process.
 The third panel (Extend) shows that \sys assists the researcher to explore the impact of sampling temperature on repeated sampling.
  The final panel (Challenge) identifies a limitation in the original methodology, suggesting an avenue for future research.
\fi

\if 0
Scientific research drives human progress, advancing medicine, technology, and our understanding of the universe. 
At the heart of this endeavor lies experimentation—a disciplined intellectual pursuit that demands a high degree of rigor, to transform human curiosity, expressed through various hypotheses, into verifiable knowledge.

In recent years, numerous works~\cite{zhang2024comprehensive,auto-science1,lu2024ai} leveraging large language models (LLMs) to automate scientific research have emerged (\S\ref{subsec:related-work}). 
These solutions typically rely on ad-hoc prompt-based methods and predefined workflows to mimic scientific workflows, which are prone to hallucination.
While effective for tasks such as literature review, brainstorming, and data analysis, these approaches remain limited in their ability to support rigorous experimentation.
\fi 
More specifically, rigorous experimentation (\S\ref{subsec:rigor}) involves a \textit{methodical procedure} that includes formulating hypotheses, designing experiments, executing controlled trials, and analyzing results. 
Achieving \textit{reliability} at every step is essential to ensure that the results are accurate, reproducible, and scientifically meaningful. 
Finally, all procedures and results must be documented in a well-structured and \textit{interpretable} manner, facilitating verification, reproducibility, and collaboration across the scientific community.

% To address these challenges,\lee{the last couple of sentences in the previous para look like requirements, not challenges; in order to make them as challenges, the previous para needs additional claims? e.g., ensuring methodical procedure, reliability, interpretability is challenging because ...; or we can simply say "to meet these requirements".} 
To meet these requirements, we propose \sys, an AI agent framework representing the first step toward rigorous and automated experimentation (\S\ref{sec:curie}). 
As shown in Fig.~\ref{fig:curie-workflow}, \sys takes an experimental question and relevant context (e.g., domain-specific knowledge or starter code) as input.
The Architect Agent generates high-level experimental plans, coordinates the process, and reflects on findings to guide subsequent steps.
Working in unison, our Technician Agents focus on carefully implementing and executing controlled experiments following these plans.

% \begin{figure}
%         \centering
%         \includegraphics[width=0.99\linewidth, trim=50 60 50 70, clip] {overview-2.png}
%     \caption{Curie overview.}
%     % \vspace{-7mm}
%     \label{fig:curie-workflow}
% \end{figure} 

% \begin{figure*}
%   \centering
%   \includegraphics[width=0.99\linewidth]{figures/case-study.png}
% \caption{Case Study. \sys is able to reproduce, extend, and challenge existing research in LLM reasoning.}
%   \label{fig:case-study}
% \end{figure*} 
% \lee{what is the relationship between this module and other entities (e.g., architect and technician)? Reading the description, the module seems to take an exp plan from the architect and validate the plan. So, does the architect use the module (meaning that the architect is a consumer of the module) or is there a third entity (or agent) that encompasses the module?} 
% At the core of \sys lies the \textbf{Experimental Rigor Module}, which ensures reliability, methodical control, and interpretability by validating agent behaviors and orchestrating their interactions throughout the experimentation process.


% At the core of \sys, the \textbf{Experimental Rigor Engine} validates agents' behaviors and coordinates their interactions throughout the experimentation process to ensure reliability, methodical control, and interpretability.
At the core of \sys, the \textbf{Experimental Rigor Engine} preserves agent creativity while embedding rigor seamlessly throughout the experimentation process.
% It validates agent behaviors and coordinates their interactions to ensure reliability, control, and interpretability.
% \amber{this can be improved.}
This is achieved via three key modules:
(1) The \textit{Intra-Agent Rigor Module} safeguards \textit{reliability} within individual agents by enforcing a set of extensible rigor policies (e.g., validating that experiment plans align with objectives and setups are reproducible).
(2) The \textit{Inter-Agent Rigor Module} 
maintains methodical control over agent coordination, ensuring correct task transitions and efficient task scheduling.
(3) Finally, the \textit{Experiment Knowledge Module} 
% ensures interpretability by maintaining well-structured documentation, facilitating collaboration within our multi-agent framework on large-scale experiments.
enhances interpretability by maintaining well-structured documentation, enabling seamless collaboration in large-scale experiments.
 

% Though this architecture suggests applications in various subjects, this paper focuses on addressing research problems in computer science, leveraging existing LLM-friendly interfaces for computer access~\cite{claude-computer-use, yang2024swe}.
Though our architecture suggests applications across various disciplines, 
this paper focuses on addressing research problems in computer science by leveraging existing LLM-friendly interfaces for computer access~\cite{claude-computer-use, yang2024swe}.
% In this work, we focus on addressing research problems in computer science by leveraging existing LLM-friendly interfaces for computer access~\cite{claude-computer-use, yang2024swe}.
% Our proposed architecture is highly adaptable and can be easily generalized to applications across various disciplines.
To evaluate \sys, we introduce an \textbf{Experimentation Benchmark} comprising 46 tasks of varying complexity across multiple domains within computer science (\S\ref{sec:benchmark}). 
% We designed an \textbf{Experimentation Benchmark} to evaluate \sys on 40 experimentation tasks of varying complexity across multiple key domains in computer science.
We derive these questions directly from influential research papers and widely adopted open-source projects, in order to reflect real-world challenges and practical significance. 
As shown in Fig.~\ref{fig:case-study}, \sys enables researchers to reproduce, extend, and
% critically evaluate 
challenge
existing research through rigorous experimentation.

We benchmarked \sys (\S\ref{sec:experiments}) against several state-of-the-art agents: OpenHands~\cite{wang2024openhands} (a top-performing coding agent on SWE-Bench~\cite{jimenez2023swe}), and Microsoft Magentic~\cite{fourney2024magentic} (a state-of-the-art generalist multi-agent system).
% and The AI Scientist’s~\cite{lu2024ai} experimentation module (\S\ref{sec:experiments}). 
% Our empirical findings show that \sys improves the experiment design correctness by 1.2$\times$, improves execution reproducibility $2.4\times$ and improves conclusion accuracy by 3.4$\times$, compared to our strongest agent baseline.
Our empirical findings show that \sys achieves a 3.4$\times$ improvement in correctly answering experimental questions, compared to the strongest baseline tested, among other aspects. 
These results underscore \sys's ability to automate complex and rigorous experimentation tasks, making it a promising step toward accelerating scientific research.

\section{Background}\label{sec:backgrnd}

\subsection{Cold Start Latency and Mitigation Techniques}

Traditional FaaS platforms mitigate cold starts through snapshotting, lightweight virtualization, and warm-state management. Snapshot-based methods like \textbf{REAP} and \textbf{Catalyzer} reduce initialization time by preloading or restoring container states but require significant memory and I/O resources, limiting scalability~\cite{dong_catalyzer_2020, ustiugov_benchmarking_2021}. Lightweight virtualization solutions, such as \textbf{Firecracker} microVMs, achieve fast startup times with strong isolation but depend on robust infrastructure, making them less adaptable to fluctuating workloads~\cite{agache_firecracker_2020}. Warm-state management techniques like \textbf{Faa\$T}~\cite{romero_faa_2021} and \textbf{Kraken}~\cite{vivek_kraken_2021} keep frequently invoked containers ready, balancing readiness and cost efficiency under predictable workloads but incurring overhead when demand is erratic~\cite{romero_faa_2021, vivek_kraken_2021}. While these methods perform well in resource-rich cloud environments, their resource intensity challenges applicability in edge settings.

\subsubsection{Edge FaaS Perspective}

In edge environments, cold start mitigation emphasizes lightweight designs, resource sharing, and hybrid task distribution. Lightweight execution environments like unikernels~\cite{edward_sock_2018} and \textbf{Firecracker}~\cite{agache_firecracker_2020}, as used by \textbf{TinyFaaS}~\cite{pfandzelter_tinyfaas_2020}, minimize resource usage and initialization delays but require careful orchestration to avoid resource contention. Function co-location, demonstrated by \textbf{Photons}~\cite{v_dukic_photons_2020}, reduces redundant initializations by sharing runtime resources among related functions, though this complicates isolation in multi-tenant setups~\cite{v_dukic_photons_2020}. Hybrid offloading frameworks like \textbf{GeoFaaS}~\cite{malekabbasi_geofaas_2024} balance edge-cloud workloads by offloading latency-tolerant tasks to the cloud and reserving edge resources for real-time operations, requiring reliable connectivity and efficient task management. These edge-specific strategies address cold starts effectively but introduce challenges in scalability and orchestration.

\subsection{Predictive Scaling and Caching Techniques}

Efficient resource allocation is vital for maintaining low latency and high availability in serverless platforms. Predictive scaling and caching techniques dynamically provision resources and reduce cold start latency by leveraging workload prediction and state retention.
Traditional FaaS platforms use predictive scaling and caching to optimize resources, employing techniques (OFC, FaasCache) to reduce cold starts. However, these methods rely on centralized orchestration and workload predictability, limiting their effectiveness in dynamic, resource-constrained edge environments.



\subsubsection{Edge FaaS Perspective}

Edge FaaS platforms adapt predictive scaling and caching techniques to constrain resources and heterogeneous environments. \textbf{EDGE-Cache}~\cite{kim_delay-aware_2022} uses traffic profiling to selectively retain high-priority functions, reducing memory overhead while maintaining readiness for frequent requests. Hybrid frameworks like \textbf{GeoFaaS}~\cite{malekabbasi_geofaas_2024} implement distributed caching to balance resources between edge and cloud nodes, enabling low-latency processing for critical tasks while offloading less critical workloads. Machine learning methods, such as clustering-based workload predictors~\cite{gao_machine_2020} and GRU-based models~\cite{guo_applying_2018}, enhance resource provisioning in edge systems by efficiently forecasting workload spikes. These innovations effectively address cold start challenges in edge environments, though their dependency on accurate predictions and robust orchestration poses scalability challenges.

\subsection{Decentralized Orchestration, Function Placement, and Scheduling}

Efficient orchestration in serverless platforms involves workload distribution, resource optimization, and performance assurance. While traditional FaaS platforms rely on centralized control, edge environments require decentralized and adaptive strategies to address unique challenges such as resource constraints and heterogeneous hardware.



\subsubsection{Edge FaaS Perspective}

Edge FaaS platforms adopt decentralized and adaptive orchestration frameworks to meet the demands of resource-constrained environments. Systems like \textbf{Wukong} distribute scheduling across edge nodes, enhancing data locality and scalability while reducing network latency. Lightweight frameworks such as \textbf{OpenWhisk Lite}~\cite{kravchenko_kpavelopenwhisk-light_2024} optimize resource allocation by decentralizing scheduling policies, minimizing cold starts and latency in edge setups~\cite{benjamin_wukong_2020}. Hybrid solutions like \textbf{OpenFaaS}~\cite{noauthor_openfaasfaas_2024} and \textbf{EdgeMatrix}~\cite{shen_edgematrix_2023} combine edge-cloud orchestration to balance resource utilization, retaining latency-sensitive functions at the edge while offloading non-critical workloads to the cloud. While these approaches improve flexibility, they face challenges in maintaining coordination and ensuring consistent performance across distributed nodes.


\section{\sys: Rigorous Experimentation}
\label{sec:curie}
\subsection{Architectural Overview} 

As shown in Fig.~\ref{fig:workflow}, \sys is composed of two types of LLM-based agents (an \textbf{Architect} Agent and a host of \textbf{Technician} Agents), 
sandwiched between them is our main innovation, the \textbf{Experimental Rigor Engine} that injects rigor throughout the experimental process. 

\noindent\textbf{High-level workflow.} Given an experimental question, our Architect will \circled{1} designs high-level \textit{experimental plans} (e.g., defining hypotheses, variables), completing its turn. Our Inter-Agent Rigor Module (\textbf{\textit{Inter-\texttt{ARM}}}) will \circled{A} intercept and enforce \textit{methodical procedure}. Since the plan is new, it is broken into smaller partitions for finer-grained execution. 
\textit{Inter-\texttt{ARM}} applies control flow policies to determine the next step for each partition. 
In this case, it decides go through the \circled{B} the Intra-Agent Rigor Module (\textbf{\textit{Intra-\texttt{ARM}}}) validation, which enhances \textit{reliability} by verifying partition integrity (e.g., assessing relevance to the experimental question).
Similarly, \textit{Inter-\texttt{ARM}} repeats this process based on the validation results, eventually \circled{C} forwarding the partition to a Technician to \circled{2} set up the controlled experiment. 
The remaining steps are omitted for brevity, but at a high level, every agent action follows the same structured workflow: \circled{A} interception by \textit{Inter-\texttt{ARM}}, \circled{B} validation by \textit{Intra-\texttt{ARM}}, and \circled{C} forwarding to the next appropriate agent. 
Finally, all of the above components will make use of our \textbf{Experiment Knowledge Module} for storing and tracking experimental progress, providing \textit{interpretability}. 
For example, the Architect stores refined experimental plans in a structured, metadata-enriched format, making them easier to analyze, track, and validate over time.
% For instance, the architect stores new plans which are internally heavily restructured and laden with metadata useful for interpretability, before storage. 

\if 0
As illustrated in Figure~\ref{fig:workflow}, \sys is designed to automate rigorous experimentation by employing two basic experimental agents and an advanced Rigor Module:

\begin{itemize}
    \item The \textbf{Architect} serves as the strategic planner, responsible for processing experimental questions and generating high-level experimental plans for the Technicians.
    Once the experiment results are obtained from technicians, the Architect evaluates results to refine hypotheses, adjust variables, or initiate new workflows. % under the help of the Rigor Module.
    
    \item The \textbf{Technician} executes the experimental plans from the architect by managing low-level tasks such as experiment setup, experiment execution, and data collection. 
    Technicians report all intermediate results back to the Architect for reflection and decision-making.

    \item The \textbf{Rigor Module} is triggered whenever an agent finishes its action. It consists of three key components:
    % , and it consists three key components to ensure rigor:
     The \textit{Intra-Agent Rigor Primitive} is triggered after each basic agent finishes its task to ensure reliability (\S\ref{subsec:intra-agent-primitive}).
     The \textit{Inter-Agent Rigor Primitive} is triggered during transitions between agents to ensure methodical control (\S\ref{subsec:inter-agent-primitive}). 
    The \textit{Experiment Knowledge Manager} regulates all interactions with the knowledge bank to ensure interpretability (\S\ref{subsec:interface}).
    \lee{how the rigor module is related to architect and/or technician? I mentioned in my earlier comments. Perhaps the details come in subsections. But until this point, the interactions between agents and the rigor module are unclear.}
% \item The \textbf{Intra-Agent Rigor Primitive} ensures reliability by verifying each agent’s internal processes after each agent finishes its task. 
% It validates that experimental setups adhere to planned workflows, checks the correctness of generated code, and ensures reproducibility.
% % It comprises a series of internal validators to enforce these checks, guaranteeing correctness and reproducibility within each agent.

% \item The \textbf{Inter-Agent Rigor Primitive} enforces methodical control by coordinating every transition among agents and validators.
% It decides subsequent actions, schedule experiments and allocates resources after each agent or validator completes its tasks.
% Therefore, it prevents incomplete workflows or steps, maintaining the integrity of the experimentation process.  
% \item The \textbf{Experiment Knowledge Manager} promotes interpretability by enforcing comprehensive documentation of experiment processes, intermediate results, and conclusions. 
% By organizing experimental knowledge in a structured and consistent format, it facilitates collaboration between agents and ensures transparency for human researchers.
\end{itemize} 
\fi
 


% We begin with a high-level overview of \sys, illustrated in Fig.~\ref{fig:curie-workflow}. 
% \Circled{1} 
% The workflow starts with the Architect, which serves as the entry point for processing experimental questions posed by the researcher. The architect generates high-level \textit{experimental plans} that includes the hypothesis, question, a high-level workflow for experiment setup, and definitions of independent, dependent, and constant variables along with their values. 
% \Circled{2} These plans are then handed off to the \textit{Technicians}, who construct and execute detailed experimental workflows to generate real experimental data. 
% \Circled{3} The results are then returned to the architect, who evaluates them to determine the next steps. The architect may conclude the experiment, refine hypotheses or variables, propose new plans, or request a re-execution of specific tasks.
% \Circled{4} Upon conclusion, a set of interpretable knowledge encompassing all key aspects of the experimental process, is outputted.
% \todo{may need new figures to show 1,2,3,4.}
% % The architect’s decisions ultimately determine the outcome of the experiment, allowing \sys to dynamically address the user’s question.

% Throughout the process, our Experimental Rigor Module (Fig.~\ref{fig:rigor-overview}) operates behind the scenes, integrating structured mechanisms that work in unison to ensure interpretability (\S\ref{subsec:interface}), reliability (\S\ref{subsec:intra-agent-primitive}), and methodical control (\S\ref{subsec:inter-agent-primitive}). We detail each of these primitives in the subsequent sections.
% \todo{we might need to elaborate these 3 if they have dependencies.}

% \todo{should we introduce different tools that our agent use.}

\subsection{Intra-Agent Rigor Module - Reliability}
\label{subsec:intra-agent-primitive}

\begin{figure}
    \centering
    \includegraphics[width=1\linewidth]{figures/intra-arm.png}
    \caption{\textit{Intra-\texttt{ARM}} setup validation high-level workflow.}
    \label{fig:intra-arm}
\end{figure}


% \begin{figure}
%     \centering
%     \includegraphics[width=1\linewidth]{setup-validator-examples.png}
%     \caption{Setup validator examples}
%     \label{fig:align}
% \end{figure}

% \begin{figure}
%     \centering
%     \includegraphics[width=1\linewidth]{exec-validator-examples.png}
%     \caption{Exec validator examples}
%     \label{fig:exec-validator-alignment}
% \end{figure}
 


\begin{figure}[t!] %\vspace{-0.4cm}
\centering
\begin{subfigure}{0.5\textwidth}
\centering
\includegraphics[width=\textwidth]{figures/setup-validator-examples.png}
% \vspace{-3mm}
\caption{Example errors that can be captured by the setup validator. 
}
\label{fig:setup-validator-examples}
\end{subfigure} 
\hfill
\begin{subfigure}{0.5\textwidth}
\centering
\includegraphics[width=\textwidth]{figures/exec-validator-examples.png}
% \vspace{-3mm}
\caption{Example errors that can be captured by the execution validator.
}
\label{fig:exec-validator-examples}
\end{subfigure}
\caption{Errors detected by two of \textit{Intra-\texttt{ARM}}'s many validators.
}
\vspace{-3mm}
\label{fig:intra-arm-error-examples}
\end{figure}

 

Large-scale and long-running experiments involve complex, interdependent steps where early-stage errors can propagate and compromise final results. This is especially critical to LLM-based experimentation since: (1) LLM-based agents are prone to hallucination, and (2) experimental processes are inherently exploratory, requiring iterative refinements to hypotheses, setups, and designs in response to new or unexpected findings.
Despite this, existing works~\cite{lu2024ai, schmidgall2025agent} largely overlook the need for continuous validation throughout the experimental process. A naive approach is to perform end-to-end validation only after an experiment concludes. However, this lacks the ability to backtrack to intermediate stages, preventing error isolation and correction, and forcing researchers to either discard progress or rerun the entire experiment—an inefficient and costly approach.
To address this, we introduce \textit{Intra-\texttt{ARM}}, a validation module that verifies the assigned tasks of our Architect and Technicians step by step, improving reliability and reproducibility to align with the overarching experimental objectives.
Inspired by process supervision~\cite{lightman2023let}, 
% we advocate for 
\textit{Intra-\texttt{ARM}} utilizes
\textbf{modular validation}, where a suite of validators continuously verifies each stage of the experiment (Fig.\ref{fig:workflow}), so that errors can be proactively detected and addressed early.
% This proactive approach catches errors early, minimizes propagation, and preserves experimental progress. 
Moreover, \textit{Intra-\texttt{ARM}}'s validators are extensible, allowing new ones to be incorporated as needed. We focus on two key validators here for brevity:

\if 0
% \textit{Intra-\texttt{ARM}} is responsible for ensuring the reliability of the experiment process within our Architect and Technicians. 
% \lee{architect and technicians are the consumer/user of the module. this explanation can come in the earlier part of the paper to ease the understanding.}
\textit{Intra-\texttt{ARM}} verifies that our Architect and Technicians perform their assigned tasks correctly, producing reliable and reproducible results that align with the overarching goals of the experiment question.
This module is particularly important for two reasons:
(1) LLM-based agents are prone to hallucinate, necessitating robust mechanisms to safeguard experimental integrity and prevent cascading errors.
% Additionally, due to the inherent uncertainty in research processes, [some examples], iterative refinements of experimental setups, hypotheses, and designs are often required. 
(2) Experimental processes are inherently exploratory, which requires iterative refinements to experimental setups, hypotheses, and designs based on new or unexpected findings.
% or unexpected outcomes.
% We now describe its two key components to address these challenges:

\noindent\textbf{Modular Experimental Validation.} 
Experiments, often large-scale and long-running, consist of complex chains of interdependent steps where early-stage errors can propagate and compromise final results. Despite this, existing works~\cite{lu2024ai, schmidgall2025agent} largely overlook the need for continuous validation throughout the experimental process.
A naive approach is to perform end-to-end validation only after an experiment concludes. However, this lacks the ability to backtrack to intermediate stages, preventing error isolation and correction, forcing researchers to either discard progress or rerun the entire experiment—an inefficient and costly approach.
Instead, we advocate for modular validation, drawing inspiration from process supervision~\cite{lightman2023let}. By employing a suite of validators at every step of the experimental process (Fig.~\ref{fig:workflow}), errors can be proactively detected and addressed early.
% , minimizing propagation while ensuring robustness and preserving progress. 
Moreover, \textit{Intra-\texttt{ARM}}'s validators are extensible, allowing new ones to be incorporated as needed.
Although our framework includes a suite of validators, we focus on two key components here for brevity:
\fi


% To address these challenges, this primitive employs a series of verifiers that systematically validate key aspects of the experimental workflow.

\begin{figure*}[t]
    \centering
    \includegraphics[width=\linewidth, trim=40 100 50 70, clip]{figures/inter-arm-v2.png}
    \caption{Simplified \textit{Inter-\texttt{ARM}} workflow with a partition state snapshot. Partition, control flow, and scheduling policies are customizable.}
    \label{fig:inter-arm}
\end{figure*}

\paragraph{Experimental Setup Validator.}
% \lee{again, is this component used by architect, technician, or both? additional details would help understanding this para.}
This component (Fig.~\ref{fig:intra-arm}) 
% is responsible for verifying that the experimental setup constructed by our Technicians is aligned with the experimental plan before execution (e.g., that it is methodologically sound, logically consistent). 
verifies that the experimental setup by our technicians aligns with the plan before execution, ensuring methodological soundness and logical consistency.
Each enforced policy checks alignment within a specific part of the experiment setup. 
This includes (Fig.~\ref{fig:setup-validator-examples}): (1) confirming the setup aligns with the experimental plan, including the research question and all specified variables (independent, dependent, and constant). 
(2) Analyzing all procedures for correct handling of input/output arguments; and detecting placeholders, hardcoded values, or incomplete variables to ensure meaningful results. 
(3) Checking that the setup documents all intermediate steps and expected results, including any identified issues for future analysis.

\if 0
This step prevents downstream errors by identifying flaws early in the workflow. 
Each policy enforced by this validator corresponds to a specific step in the experiment setup.
(1) \textit{Alignment with the Experimental Plan:}
Confirms that the setup reflects the experimental plan, including the research question and all specified variables (independent, dependent, and constant).
(2) \textit{Inspection of Scripts:}
Analyzes all scripts to ensure proper handling of input and output arguments as specified by the plan. Detects placeholders, hardcoded values, or incomplete variables to ensure the setup generates genuine, meaningful results.
(3) \textit{Results Logging and Documentation:}
Ensures that the setup logs all intermediate steps and expected results, including any identified issues for future analysis.
\fi
% \yiming{How is step (1) and (2) achieved? Right now it is unclear}

\paragraph{Execution Validator.}
% Once the setup passes the experimental setup validator without error, we run this validator to enforce reproducibility, by running the setup in a controlled environment to detect and address potential execution errors.
Once the setup passes the experimental setup validator, this validator enhances reproducibility by executing it in a controlled and clean environment to detect and resolve potential errors, a sample of which is illustrated in Fig.~\ref{fig:exec-validator-examples}.
(1) \textit{Error-Free Execution:}
The setup is executed in a clean environment, verifying that it operates without errors. Any encountered errors are logged in detail, providing actionable feedback for debugging and iterative refinement.
(2) \textit{Reproducibility Checks:}
The workflow is also run multiple times to enhance consistency in outputs and detect anomalies or hidden dependencies. Finally, the results are validated to ensure alignment with the experimental plan and compliance with predefined quality standards.
 
% \amber{add description about program-based verifier}
% \amber{add out-of-distribution error handling}

\subsection{Inter-Agent Rigor Module - Methodical Control}
\label{subsec:inter-agent-primitive}

% The Inter-Agent Rigor Primitive ensures methodical control over the experimentation lifecycle by coordinating interactions between agents (e.g., architects and technicians) and managing the execution of tasks. It addresses critical challenges such as task transitions, prioritization, and resource constraints, enabling seamless collaboration and efficient experimentation.

Experimental processes must follow a methodical precedure (\S\ref{subsec:rigor}) while balancing resource constraints (e.g., GPU availability), and experiment priorities.
% and prioritizing more important experiments. 
Traditional agentic conversational patterns~\cite{autogen-conv-patterns}—such as naive LLM-based coordination, sequential, or round-robin execution—are thus ill-suited for such a workflow. 
% These approaches lack the necessary control to prioritize tasks, adapt to dynamic constraints, and prevent inefficiencies in large-scale experimentation.
To \textit{ensure task coordination} and \textit{optimize resource efficiency}, \textit{Inter-\texttt{ARM}} enables seamless collaboration between our Architect, Technicians and \textit{Intra-\texttt{ARM}} 
through three key functions (illustrated in Fig.~\ref{fig:inter-arm}). We discuss each in turn. 
\if 0
% This module is essential for two reasons:
(1) \textit{Ensuring Task Coordination} – In complex experimentation workflows, logical task transitions between agents are critical to maintaining consistent, error-free progress. Without structured coordination, tasks may be executed out of order or without necessary dependencies, leading to wasted effort and erroneous conclusions.
(2) \textit{Optimizing Limited Resources}: Experimentation often operates under constrained resources, requiring careful scheduling and prioritization of tasks to improve efficiency.
\fi
% We address each of the above through two components:

\if 0
Our \textit{Inter-\texttt{ARM}} enables seamless collaboration and coordination between agents (e.g., architects and technicians).
% \lee{now I am confused about the rigor module. Is the module a separate process that interacts with architect and technicians? Then, calling it a module can be misleading. module can be seen as a library or package, which can be used by processes. But the way this part is described seems to suggest the rigor module is an independent entity (e.g., a process). If so, then it may be better to have a bounding box for the rigor process in figures 1-3 for better visualization.}
This module is essential for two main reasons:
(1) \textit{Ensuring Task Coordination}: In complex experimentation workflows, logical task transitions between agents are critical for maintaining meaningful progress and avoiding errors.
(2) \textit{Optimizing Limited Resources}: Experimentation often operates under constrained resources, requiring careful scheduling and prioritization of tasks to improve efficiency.
% We now describe its two key mechanisms to address these challenges:
We address each of the above through two key components:
% To address these challenges, the Inter-Agent Rigor Primitive employs two key mechanisms: Control Flow Enforcement and Experiments Scheduling. [simplify the transition]
\fi

\paragraph{Fine-grained Plan Partitioning.}
% To address these challenges,
% % \lee{what challenges? resource-intensiveness and time-consuming nature?} 
% our scheduler first breaks down new experiment plans into manageable partitions and then decides the resource scheduling plan for these partitions, in order to improves resource utilization, accelerates progress, and aligns task execution with the overarching goals of the experimental plan.
% \zy{It would be great if we could add an example here. A visual one showing parallelism would be even better. Too much dense text up to this point, some sort of visual aid (tables/figs) is needed.}
\textit{Inter-\texttt{ARM}} first breaks down new complex experimental plans generated by the Architect into smaller, independent partitions: defined as a distinct subset of independent variable values within the plan. 
% , enabling fine-grained scheduling and efficient execution.
% A partition is defined as a distinct subset of the experimental workflow, defined by a specific configuration of variables, hypotheses, or experimental conditions.
% Each partitions is executed in isolated environments, producing separate results.
% For example, in an experiment of testing the effect of multiple setup conditions, each condition forms a separate partition, with its own environment, execution, and results.
% Each partition will be executed independently, with 
By creating smaller, self-contained tasks, this facilitates modular execution and enables parallelization, making experimentation more scalable. 
In addition, this enables our Architect to track intermediate progress and results, making real-time decisions as new insights emerge (e.g., reprioritizing partitions by updating their execution priority).
% The architect, acting as the central coordinator, continuously tracks the progress of all partitions, dynamically updating their priorities as new progress is made or new insights are gained.  
% In addition, this simplifies the tracking of intermediate progress and results, enabling our Architect to monitor the state of individual tasks efficiently.
% (2)\lee{this para is length. split it into two here?} 

\paragraph{Control Flow Enforcement.}
This component ensures that transitions between our Architect, Technicians, and \textit{Intra-\texttt{ARM}}
% \texttt{Intra-\textit{ARM}} 
follow a logical sequence aligned with the experimentation lifecycle. 
This is critical to maintaining consistent, error-free progress. Without structured coordination, tasks may be executed out of order or without necessary dependencies, leading to wasted effort and erroneous conclusions.
% For instance, it prevents Technicians from passing experiment results to the Architect before validation by \textit{Intra-\texttt{ARM}}, to reduce the risk of erroneous data propagation.
For instance, it prevents Technicians from directly executing experiment setups before validation by \textit{Intra-\texttt{ARM}}'s setup validator, to reduce the risk of erroneous data propagation.
% It governs task progression by first evaluating the current state of a task and then determining the next permissible actions.
This is done in two steps:
% operating at the granularity of an experimental plan partition:
(1) \textit{State Evaluation}: First, it evaluates the current state of each partition (within an experimental plan) that has been modified by any given agent, e.g., a Technician who produced experimental results and recorded its progress via the Experiment Knowledge Module.
% It applies strict control flow transitions to enforce forward progress and prevent premature or erroneous transitions.
% For example, it will enforce a task completed by a technician to first pass through the Intra-Agent Rigor Primitive before being forwarded to the architect.
% In this way, our framework disallows the termination of the experiment until all necessary tasks are completed. 
(2) \textit{Permissible State Transitions}: Based on the current state of the partition(s), this component produces a set of allowed state transitions for the given partition, e.g., newly produced experimental results for a given partition need to be validated by \textit{Intra-\texttt{ARM}} first. It also gathers relevant context that would be useful if the transition were to be executed. 
% (e.g., previously validated partitions)
This state transition information will be consumed by our scheduler (defined below). 
% (2) Permissible State Transitions: This component determines the allowed state transitions for each partition based on its current state. For example, newly produced experimental results must first be validated by Intra-\texttt{ARM}} before proceeding. These transitions serve as scheduling parameters for the next component. Additionally, it gathers and passes relevant context, including prior experiment progress, to ensure continuity and informed decision-making.
% Control the flow of tasks between agents to maintain alignment with the common experimentation lifecycle. Meanwhile, it
% \lee{what is it? task transitions? that doesn't seem suitable as a subject of this sentence since it doesn't look like an active entity that can actively do something (e.g., collect and pass)} 
% It produces a set of permissible state transitions
% For example, our architect
% \lee{can we have more than one architect for one experimentation?} 
% design the experiment and pass the experimental plan to technicians, who setup the controlled experiment. Once the technician completes the task, it transitions back to architects along with the setups for analysis and next steps.
% \lee{this para is a bit hard to understand.}


\paragraph{Partition Scheduling.} 
% Experiments are often large-scale and operate under constrained resources, requiring careful scheduling and prioritization of tasks to improve efficiency.
Executing large-scale experiments can be resource-intensive and time-consuming, requiring careful scheduling and prioritization of tasks to improve efficiency.
Our scheduler currently utilizes three key parameters for partition scheduling: (1) partition execution priorities set by our Architect, (2) allowed partition state transitions, and (3) the availability of our agents (that may be busy handling other partitions).
% Technician along with its resource availability. 
% For instance, hyperparameter tuning often involves exploring a vast search space, requiring thousands of GPUs hours to evaluate various configurations.
% Such scenarios highlight the importance of an efficient scheduler that minimizes cost, accelerates execution, and ensures systematic progress. 
% As part of the Inter-Agent Rigor Primitive, this experiment scheduler enforces methodical control over task execution, thereby maintaining rigor and reproducibility across the experimentation lifecycle.
\if 0
 \textit{Priority-Based Scheduling.} 
Once partitions are created, the scheduler determines their execution order based on their relevance to the experimental objectives. 
The architect, acting as the central coordinator, continuously tracks the progress of all partitions, dynamically updating their priorities as new progress is made or new insights are gained.  
For example, if early results from a partition indicate the need to explore a new parameter range, the scheduler dynamically adjusts priorities to accommodate these new tasks. \todo{the example is optional, unless we do have this.}
% The scheduler also considers resource availability and resource affinity when deciding which system resources (e.g., GPUs, CPUs, dataset) are best suited for executing each partition. 
\fi
Overall, this adaptive scheduling strategy enables large-scale experimentation by improving resource efficiency while adhering to methodical experimental procedures.

% \todo{miss the fine-grained component assignment.}

% \amber{the discussion of parallelization here? otherwise we need to show something in the eval.}
% \zy{if possible, for eval, just add all sub-tasks exec time, vs. time with parallelization enabled (this could also be simulated with task-DAG).}
 


% % Below is random thoughts to give some details:

% \subsection{Inter-Agent Rigor Primitive: Methodical Control}
% \label{subsec:inter-agent-primitive}
% This primitive ensures methodical control over the experimentation lifecycle by managing the plethora of dynamic interactions between our agents. It does so through two key mechanisms: (1) strict control flow to enforce forward experimental progress and (2) dynamic priority-based scheduling to optimize efficiency.


% \noindent\textbf{Control Flow Enforcement.}
% Control flow determines the permissible transitions for a partition based on its current state, ensuring tasks progress methodically within the experiment.
% (1) The system intercepts when an agent completes its assigned task (e.g., a technician finishing an experiment for their partition) and logs this progress through plan updates. (2) It evaluates the current state of the partition by examining what has been completed so far, using recorded deliverables and progress updates. (3) Based on this information, strict control flow transitions are applied to determine the next permissible actions for the partition.
% For example, a partition just completed by a technician cannot be marked as done or forwarded to the architect directly. Instead, it must first pass through the intra-agent rigor primitive for validation.  
% As another example, we disallow termination of the experiment until all plans and their associated partitions are completed or explicitly removed by the architect.  
% By enforcing these transitions, the system maintains rigorous oversight and provides stronger guarantees for the integrity of the experimental process. 

% \noindent\textbf{Dynamic Scheduling.}
% Dynamic scheduling determines which partitions be executed based on the transitions allowed by the control flow mechanism, and the ranked priorities currently assigned by the architect for each partition. 
% To achieve this, the system maintains significant knowledge of the current experimental state, including: (1) plans and partitions currently being executed, (2) technicians and their assigned tasks, (3) any partitions that have been inserted, modified, marked for redo, or removed by the supervisor.
% Once an agent is selected, we provide it with context information relevant to its next task, including metadata about tasks completed by previous agents, ensuring continuity and informed decision-making.
% Metadata is packaged and sent to individual agents, tailored to their specific tasks. This includes inputs needed for the next step, responses from previous agents, and deliverables transformed into actionable items. For example, results from technician are consolidated into a convenient file format for the architect's review or passed as context to another agent to advance the workflow.
% While the current implementation does not yet utilize parallel technicians, the mechanism is designed to support such scalability in the future. 
% By prioritizing tasks, managing transitions, and packaging relevant information, dynamic scheduling ensures efficient resource utilization and seamless progression in experimental workflows.

\subsection{Experiment Knowledge Module - Interpretability}
% \lee{why did we drop the term, "experiment knowledge manager" and introduce a new term, "agent-experimenter interface"?}
\label{subsec:interface}

\begin{figure}
    \centering
    \includegraphics[width=1\linewidth]{figures/time-machine.png}
    \caption{Simplified partial snapshot of an example Time Machine.}
    \label{fig:time-machine}
\end{figure}



\if 0
Interpretability is fundamental to experimentation—not only for scientific accountability but also for effective experiment management. Without it, \sys may struggle to trace outcomes, refine hypotheses, or diagnose failures. In complex, multi-step workflows, it provides real-time visibility, enabling informed decision-making, efficient troubleshooting, and adaptability as new insights emerge. Without it, experimentation becomes a black box, leading to inefficiencies, untraceable errors, and lost progress.
A naive approach would be to delegate experimental knowledge management entirely to LLM-based agents. 
However, LLMs by themselves are ill-suited for this task due to their 
% lack of structured memory, deterministic recall, and systematic decision rules. 
hallucinatory nature, and proneness to inconsistent recall and forgetting.
% Their responses are probabilistic rather than rule-based, meaning they may omit, misinterpret, or hallucinate details when handling experimental data.
Unlike databases, they do not inherently track provenance, making it difficult to reconstruct how conclusions were reached. 
Overall, this can lead to error propagation, and inefficiencies in long-term context management, omit, misinterpret, or hallucinate details when handling experimental data.
especially in long-running experiments.

Scientific rigor requires tracking how experimental data is acquired, processed, and utilized at every stage. Since LLMs do not maintain verifiable audit trails of their reasoning, relying on them alone would compromise transparency, reproducibility, and reliability. Instead, structured mechanisms—such as experiment knowledge managers and modular validation layers—are essential for enforcing interpretability, ensuring that experimental processes remain auditable, accountable, and scientifically rigorous.
\fi

Interpretability is fundamental to experimentation—not only for scientific accountability but also for effective experiment management. 
% Without it, \sys may struggle to trace outcomes, refine hypotheses, or diagnose failures. 
% In complex, long-running experiments,
Specifically, all other components within \sys require this for real-time visibility, enabling informed decision-making, efficient troubleshooting, and adaptability as new insights emerge. 
% —especially at scale, where the volume of experimental data can be massive.
% , making it even harder to track provenance and ensure consistency.
A naive approach would be to delegate experimental knowledge management entirely to LLM-based agents. 
However, LLMs alone are ill-suited for this task for two reasons:
(1) \textit{Inconsistent Reads}: LLMs have inconsistent recall and are prone to forgetting~\cite{xu2024knowledge}. Without a structured and verifiable record of experimental progress, they may retrieve outdated, irrelevant, or hallucinated information, leading to misinterpretations, flawed conclusions, and compounding errors over time. 
(2) \textit{Inconsistent Writes}: LLMs tend to hallucinate, particularly when managing large-scale experimental data. This lack of structured control risks corrupting experimental records, propagating inaccuracies, and ultimately compromising the integrity of the experimentation process.  
Unlike databases, LLMs do not inherently track provenance~\cite{hoque2024hallmark}, making it difficult to reconstruct how conclusions were reached.
\if 0
However, LLMs alone are ill-suited for this task due to their hallucinatory tendencies, inconsistent recall, and susceptibility to forgetting. 
Unlike databases, they do not inherently track provenance, making it difficult to reconstruct how conclusions were reached. 
As the amount of experiment data grows, this can lead to error propagation and lost progress.
To address these challenges, our Experiment Knowledge Module integrates two mechanisms that we discuss below: 
\fi
We address these two challenges in turn: 

\if 0
The Agent-Experimenter Interface ensures consistent and well-structured experiment logging, which is essential for the interpretability of \sys. This component serves two critical purposes:
(1) \textit{Facilitating Agent Collaboration:} Shared progress and structured knowledge are vital for ensuring seamless coordination, particularly in long-ranging or large-scale experiments.
(2) \textit{Ensuring Transparency for Reproducibility:} Providing researchers with transparent access to experimental progress and outcomes is crucial for reproducing results and validating experimental integrity.
We now describe its two key components to support these:
\fi


% \noindent\textbf{Structured Knowledge .}
\noindent\textbf{Structured Knowledge Reads.}
\if 0
This mechanism manages and organizes experimental progress by transforming the experimental plan and process into a structured and enriched format. 
The structured approach enhances collaboration among agents and simplifies the interpretation of experimental workflows by researchers.
The transformation begins with formatting and enriching experimental plans. Experimental plans, typically written in natural language or loosely structured formats, are restructured into an enriched format containing critical metadata. 
This metadata includes information such as the experimental setups, the execution status, and the produced results as shown in \textcolor{red}{an example json}. 
This enriched format provides a unified representation of the experimental state and facilitates downstream operations, such as validation by aforementioned rigor primitive.
This structured management system not only streamlines collaboration between agents but also enhances the interpretability and scalability of \sys.
\fi
This mechanism organizes experimental progress in a structured format. 
The process begins by restructuring new experimental plans that were written by our Architect into an enriched format with critical metadata—such as setups, execution status, and results. 
Subsequent modifications to any part of the plan are recorded as a time machine (Fig.~\ref{fig:time-machine}) for experimental progression, maintaining a structured, DAG-like history of changes. This historical record captures hypotheses tested, variable changes, and the reasoning behind key decisions. By preserving this evolution, \sys can reconstruct past states, trace decision rationales, and diagnose issues with greater precision. 
% This structured approach not only enhances interpretability but also mitigates the risk of lost context, allowing experiments to scale without sacrificing reliability.

% By enforcing structure, this system improves traceability, reduces ambiguity, and scales efficiently as experiments grow in complexity.

% \todo{think about 'partition'}
% \yiming{This seems to be access control, why is this called ``tiered''?} 
\noindent\textbf{Tiered Write Access.}
To maintain experimental integrity and minimize the risk of errors, the interface enforces a tiered write access policy that restricts and validates updates made to the experimental plan. This ensures that our other components can only modify the portions of the plan they are responsible for, while all changes undergo rigorous validation.
Our LLM-based Architect and Technicians are granted fine-grained write permissions tailored to their roles. For example, Technicians are permitted to append experimental results to their assigned partitions but cannot modify unrelated sections of the plan. Similarly, architects have broader write access, including the ability to create or remove entire partitions, but their modifications are still constrained to specific attributes, such as updating variable values or marking partitions for re-execution.
Every write operation is validated before being committed to the knowledge bank. 
This process ensures proper structuring of inputs and enforces semantic integrity (e.g., that result file paths are valid). 
% This validation process includes checks for correct structuring of inputs and various semantic checks (e.g., that a results filepath is valid). 
If errors are detected, the system returns concise error messages, enabling agents to quickly identify and resolve issues. 
Through this, \sys enhances robustness and error resistance in collaboration.


 \begin{table*}[t]
\centering
\caption{Experimentation benchmark overview.  }
\begin{tabular}{c|ccc|c|c}
\toprule
\multirow{2}{*}{\textbf{Domain}} & \multicolumn{3}{c|}{\textbf{Complexity Dist.}} & \multirow{2}{*}{\textbf{Description}}                                                                                                                                              & \multirow{2}{*}{\textbf{Sources}}                                                                                   \\
                                 & Easy            & Med.            & Hard          &                                                                                                                                                                                    &                                                                                                                     \\ \hline  
                                \midrule
LLM Reasoning                    & 4               & 5               & 7               & \begin{tabular}[c]{@{}c@{}}Investigates strategies for scaling test-time \\ computation in LLMs, focusing on \\ balancing accuracy, latency, and cost.\end{tabular}                & \begin{tabular}[c]{@{}c@{}}Research papers: \\ ~\cite{brown2024large}, \\ ~\cite{jin2024impact}.\end{tabular} \\ \hline
Vector Indexing                  & 6               & 6               & 3               & \begin{tabular}[c]{@{}c@{}}Examines efficient vector indexing methods \\ for similarity search, analyzing its trade-offs \\ in retrieval recall, memory, and latency.\end{tabular} & \begin{tabular}[c]{@{}c@{}}Open-source project: \\ Faiss~\cite{douze2024faiss} \end{tabular}                                               \\ \hline
Cloud Computing                  & 2               & 4               & 2               & \begin{tabular}[c]{@{}c@{}}Optimize distributed setups, \\ resource allocation, and cost-performance \\ trade-offs in cloud environments.\end{tabular}                             & \begin{tabular}[c]{@{}c@{}}Cloud platforms: \\ Amazon Web Services\end{tabular}                                     \\ \hline
ML Training                      & 3               & 3               & 1               & \begin{tabular}[c]{@{}c@{}}Optimize ML training pipelines, \\ including hyperparameter tuning \\ and model architecture search.\end{tabular}                      & \begin{tabular}[c]{@{}c@{}}Open-source benchmark: \\ ~\cite{huang2310mlagentbench}, \\ \cite{hong2024metagpt} \end{tabular}   \\ 
\bottomrule
 
\end{tabular}
\label{table:benchmark-overview}
\end{table*}


% \subsection{Agent-Experimenter Interface: Interpretability} 

% Our interface allows our agents to reliably store, track, and maintain experimental progress, reducing errors and enhancing scalability, particularly in long-ranging or large-scale experimentation.
% We now describe its two key mechanisms:



% \noindent\textbf{Structured Knowledge Management.}
% \todo{need a top-down fashion to introduce the responsiblity. I will do this, but still concern about the order of these 3 components.}
% New experimental plans crafted by the architect are restructured into an \textit{enriched} format containing information useful to all agents and rigor primitives (as described later) as the experiment progresses. To enable fine-grained scheduling and efficient execution, new plans are first broken down into smaller partitions, each representing a manageable subset of the overall experiment (our basic experimental unit) defined by distinct independent variables. 
% Each partition is then enriched with various metadata, including details valuable to the architect, such as whether the partition has completed execution correctly, the location of its experimental setup, and the location of its produced results. 
% All metadata and results are stored in a centralized experiment knowledge bank, which organizes information into distinct ``message banks'' for categories such as results, logs, and variables.
% Experimental setups are further organized as sequences of programs, culminating in a main callable program that produces results in a predefined file format, ensuring consistency and reproducibility across workflows.
% The enriched experimental plans are stored and maintained in the experiment knowledge bank, which serves as a central repository for structured knowledge. 
% This approach simplifies decision-making for controllers by exposing an intuitive interface while ensuring the underlying knowledge is internally validated and restructured for operational use. 
% By combining modularity, fine-grained tracking, and centralized storage, the system provides a scalable and interpretable framework for managing complex experiments.
 

% \noindent\textbf{Tiered Write Access.} 
% % basically: (1) accessed controlled writes, and (2) each such write is validated for correctness and otherwise concise errors are returned.
% All changes made by agents and rigor primitives require updates to some underlying enriched experimental plan, whether it involves technicians producing experimental data for their partitions or architects marking a partition for redo. To enable these updates, agents are granted write access to the plan. However, based on our observation that coarse-grained writes pose significant risks for mistakes, we enforce fine-grained write access to protect experimental integrity and minimize errors during experimentation.
% Agents are granted write permissions only for the specific partitions they are responsible for. Within these partitions, they can only modify specific attributes of the plan (e.g., a technician appending experimental data) using a set of distinct interface tools, ensuring that changes remain localized and relevant. 
% Similarly, the architect is granted broader write access, including the ability to remove entire plans/partitions. However, for modifications, the architect is still restricted to fine-grained changes to specific attributes within the plan (e.g., adding new variable values to a partition or marking a partition for redo).
% Additionally, all write operations undergo strict validation before being committed. This includes checks for valid plan IDs and the proper structuring of outputs. If errors are detected, the system returns concise error messages to help agents quickly identify and resolve issues. 
% By scoping write permissions and enforcing rigorous validation, this mechanism ensures a robust and error-resistant workflow, particularly for long-running, large-scale experiments.










% for agents: only provide info for partitions they are in charge of, as decide by the controller
% for verifiers: same
% for architect: all information for read, but only a little for write

% \subsection{Intra-Agent Rigor Primitive: Enhancing Validity}
% \label{subsec:intra-agent-primitive}

% This primitive comprises extensible validation policies that enhance the validity of experimental setups and results. Since unexpected outcomes or new insights often require iterative refinements to hypotheses, methods, or designs, it ensures that these refinements are grounded in a valid and reliable experimental setup. Moreover, it provides a pathway for a reproducible experiment. 
% While we envision incorporating additional validators in the future, here we focus on the two currently implemented in \sys.

% \noindent\textbf{Experimental Setup Validator}
% This is implemented as an AI agent that inspects the experiment setup's structure, logic, and content—without executing it—to ensure that the setup is methodologically sound, logically consistent, and aligned with the goals of the experimental plan. 
% Its validation process involves the following steps: (1) retrieving and analyzing the experimental plan to confirm the setup aligns with the experimental question and includes all specified variables (independent, dependent, and constant); (2) inspecting the setup structure by starting with the main script, tracing dependencies (including nested or recursive scripts), and verifying proper handling of inputs, outputs, and integration; (3) ensuring the setup produces genuine outputs, not placeholders or mock data, by checking for hardcoded values, incomplete variables, or placeholder tokens; (4) validating that all variables defined for the current partition are explicitly and effectively utilized, including in nested scripts; (5) inspecting results files to confirm it corresponds to the required variable values and adheres to the experimental plan; and finally (6) recording its findings with detailed explanations for any issues identified.
 

% \noindent\textbf{Execution Validator}
% If the previous step passes without errors, this validator is applied to reinforce reproducibility by verifying that the setup produces consistent outputs across repeated executions, a cornerstone of scientific rigor. This validator attempts to rerun the experimental setup in a clean environment, validating reliability by ensuring scripts are robust, free from hidden dependencies or placeholders, and confirming the legitimacy of results by ensuring they are complete, accurate, aligned with the experimental plan, and free from errors such as missing or corrupted files. Its process involves the following steps: (1) executing the experimental setup and ensuring it runs without errors; (2) validating results by confirming the generation of the specified results file and verifying its accuracy; (3) performing consistency checks by re-executing the workflow multiple times and comparing results to detect any anomalies or deviations that could indicate flaws; and (4) capturing and logging any errors encountered during execution, including script failures or missing outputs, while providing detailed feedback through verifier log messages. By complementing the structural validation performed by the previous validator, the Execution Validator ensures that workflows are not only logically valid but also functional, consistent, and fully reproducible.





\section{Experimentation Benchmark}
 \label{sec:benchmark}


\if 0
What to highlight?

- Experimental questions are different from other regular benchmark questions, so we can't use them. 
- They are different in the following ways: these questions are typically long-running complex experiments. For cloud, this may even span connecting to remote machines and managing them (running workloads/experiments on a remote machine). For llm reasoning 2:
- These experiments are complex because they stress test different components of experiments, first, we have the design space that is... 
- Then, we have the relationship complexity that... this is difficult because... 

The benchmark is great for two reasons: (1) the way we structure the questions as an experiment, and (2) the way we construct in terms of complexity levels

\fi

We design a novel benchmark to stress test \sys's ability to automate experiments while enforcing rigor in the face of real-world challenges. 
As shown in Table~\ref{table:benchmark-overview} (with full details in App.~\ref{appendix:benchmark-details}), our benchmark consists of 46 tasks across 4 domains within computer science.
Our tasks are derived directly from \textbf{real-world influential research papers} and use-cases within \textbf{popular open-source projects}.
We will open-source our benchmark to enable follow-up research. 
% These tasks are carefully designed with varying levels of experimental complexity, allowing for a comprehensive assessment of \sys’s ability to handle demanding experimentation scenarios.
% It evaluates how well \sys maintains structured, hypothesis-driven exploration and prevents error propagation throughout the experimental process.
% It differs from standard benchmarks in two key ways:
% It consists of two main features: 

\subsection{Experiment-Centric Task Design}
Instead of treating tasks as isolated problems with fixed solutions, we structure each task as a full experimental process. This means that tasks require hypothesis formation, iterative refinement, and rigorous validation, mirroring real-world experiment workflows rather than one-shot problem-solving.

The process begins with distilling high-level contributions from research papers (e.g., theoretical insights or empirical findings), or core system behaviors from open-source projects (e.g., the interplay between configuration parameters and performance). 
These insights are then translated into testable questions framed with explicit configurations, metrics, and expected outcomes.
% Ground truth data is constructed using published empirical results for research papers or official benchmarks provided by open-source projects. 
Ground truth data is derived from published results or official benchmarks provided by open-source projects.
We use these findings to design tasks with three key components:

\noindent\textbf{1. Experiment Formulation:} 
Each task specifies the (a) Experiment Question (e.g., optimizing performance, identifying relationships); (b) Practical constraints (e.g., resource budgets); (c) High-level Setup Requirements - Contextual details such as datasets, and experimental environments.
% , or computational tools. 
This framing ensures that tasks are open-ended, requiring iterative exploration rather than one-shot solutions.

\noindent\textbf{2. Experimental Context:} To ensure agents correctly interpret and execute tasks, the benchmark provides detailed context for each question. This includes: (a) Domain Knowledge – Background information essential for interpreting the problem.
(b) Starter Code \& Tools – Predefined scaffolding to simulate real-world research workflows.
% This ensures that agents operate under realistic conditions where understanding and iterating on a problem is as important as solving it.

\noindent\textbf{3. Ground Truth:} 
% To assess an agent’s ability to conduct rigorous experimentation, we define ground truth in two key areas:
This is defined in two key areas:
(a) \textit{Experimental Design}: Does the agent correctly formulate the experiment, identifying relevant variables and methodologies? 
(b) \textit{Result Analysis:} 
  Does the agent correctly interpret findings, and justify its conclusions? We outline the expected outcomes or acceptable solution ranges.
\if 0 
To assess an agent’s ability to conduct rigorous experimentation, we define ground truth in three key areas:

Experimental Design Validity – Does the agent correctly formulate the experiment, identifying relevant variables, constraints, and methodologies?
Execution Soundness – Does the agent systematically explore the search space, perform well-structured trials, and generate meaningful intermediate results?
Result Analysis & Iteration – Does the agent correctly interpret findings, refine its approach based on evidence, and justify its conclusions with logical reasoning?
\fi

\if 0
\begin{packedenumerate}
    \item \textbf{Questions:}
    The question outlines the objectives (e.g., optimizing latency or accuracy), practical constraints (e.g., resource limits), and necessary contextual requirements, (e.g., dataset or other experimental setups) to guide the agent toward meaningful outcomes.

    \item \textbf{Context:}  
    To ensure agents correctly interpret and execute tasks, the benchmark provides detailed context for each question. This includes problem formulations, domain knowledge, and starter code.
 

    \item \textbf{Ground Truth:} 
    Comprehensive ground truth is provided for evaluating each critical step of experimentation:
   1). \textit{Experimental Design}: Specifies the key variables, parameters, or setups essential for answering the question.
  2). \textit{Experiment Execution}: Defines the expected search space along with the intermediate results.
  3). \textit{Result Analysis:} Outlines the expected outcomes or acceptable solution ranges, ensuring the agent’s conclusions are accurate, logically derived, and aligned with the problem objectives.  
\end{packedenumerate}
\fi


\begin{table*}[]
\caption{Main benchmark results in terms of four metrics introduced in \S\ref{sec:experiments}. We aggregate and average the success rate among all tasks within each domain. 
The final row presents the weighted average, computed based on the number of tasks in each domain.
% Lastly, we weighted average the overall success rate by the number of task in each domain.
% The average score is the weighted average. 
% The success rate for each task is averaged across 5 trials. 
%\textcolor{red}{mark the number with background color. bold the best results?}
}
\begin{tabular}{c|cccc|cccc|cccc}
 \toprule
 \multicolumn{1}{l}{} & \multicolumn{4}{|c}{Curie}      & \multicolumn{4}{|c}{OpenHands}  & \multicolumn{4}{|c}{Microsoft Magentic-One} \\
\multicolumn{1}{l|}{} & Des. & Exec. & Align. & Con.  & Des.                        & Exec.                         & Align.                        & Con.                          & Des.    & Exec.    & Align.    & Con.    \\ \hline                 
LLM Reason.          & 98.3   & 83.3  & 76.7   & 44.9 & 86.7 &  24.6 & 36.7 & 14.2 & 72.0      & 9.3     & 14      & 6.7     \\
Vector DB            & 97.8   & 71.7  & 77.2   & 25.6 & 85.0                         & 48.3                         & 52.3                         & 11.7                         & 85.0      & 6.4      & 63.6      & 0.0     \\
Cloud Comp.          & 100.0  & 92.7  & 96.9   & 32.3 & 96.9                         & 25.2                         & 49.2                         & 5.0                          & 95.0     & 6.3      & 33.8      & 0.0     \\
ML Training          & 95.2   & 66.7  & 39.3   & 41.7 & 63.1                         & 24.3                         & 16.7                         & 5.7                          & 90.0      & 2.9      & 25.7      & 0.0     \\ \midrule
Weighted Avg.              & 97.9   & 78.1  & 73.4   & 36.1 & 83.6                         & 32.4                         & 40.2                         & 10.5             & 82.9      & 6.8      & 35.2      & 2.3    \\
\bottomrule
\end{tabular}
\label{table:main-results}
\end{table*}

\subsection{Experimental Complexity}
% We construct realistic and meaningful tasks by directly deriving them from influential \textbf{research papers} and popular \textbf{open-source projects}, ensuring that our benchmark reflects genuine challenges encountered in scientific inquiry and system optimization. 

% While our benchmark is grounded in authentic research contexts, the agent frameworks may yield conclusions that diverge from the expected ground truth, which provides valuable opportunities to explore and analyze deviations. \amber{add section reference.}
\if 0
Experimental research is rarely a one-size-fits-all process; different problems require varying degrees of complexity and iteration. Our benchmark is designed to reflect this reality by structuring tasks into a hierarchical complexity framework, ensuring that agents are evaluated on their ability to handle increasingly sophisticated experimentation scenarios.
Tasks are designed to test how well an agent navigates multi-step experimentation, adapts to unexpected results, and maintains structured records over long-term iterative processes. 
This ensures that the benchmark evaluates not just problem-solving ability but the capacity to manage and execute rigorous, scalable experimentation.
Unlike standard benchmarks that categorize tasks solely by a single overall difficulty metric (e.g., easy, medium, hard), our benchmark structures complexity along experiment-driven dimensions:
\fi 

Experimental research varies in complexity across different dimensions. Our benchmark reflects this by structuring tasks into a hierarchical framework, assessing an agent’s ability to handle increasingly sophisticated experimentation tasks. 
% Tasks test multi-step reasoning, adaptation to unexpected results, and structured record-keeping over iterative processes.
Unlike standard benchmarks that classify tasks by a single difficulty metric (e.g., easy, medium, hard), ours structures complexity along experiment-driven dimensions (detailed definitions in App.~\ref{app:complex}):

\noindent\textit{1). Design Complexity:} The complexity of structuring an experiment (e.g., requiring hypothesis refinement), including defining the scope of exploration, selecting key variables, and structuring parameter spaces—ranging from discrete to continuous and from sparse to dense configurations.

\noindent\textit{2). Experiment Setup Complexity:} The difficulty of initializing and configuring the experimental environment, from simple predefined setups to intricate dependencies requiring multi-step configuration.

\noindent\textit{3). Relationship Complexity:} The interactions between variables and outcomes, from simple linear dependencies to complex non-monotonic relationships.

\noindent\textit{4). Experiment Goal Complexity:} The number of competing objectives and trade-offs involved, from single-metric optimization to multi-objective balancing under constraints.
% and are subject to change as AI agents improve in their experimentation capabilities over time.

\if 0
\begin{packeditemize}
    \item \textbf{Design Complexity:} The size and structure of the variable configurations, ranging from discrete to continuous, and from sparse to dense parameter spaces.
    \item \textbf{Experiment Setup Complexity:} The difficulty of initializing and configuring the experimental environment, from straightforward setups to intricate dependencies.
    \item \textbf{Relationship Complexity:} The interactions between variables and outcomes, from simple linear dependencies to complex non-monotonic relationships.
    \item \textbf{Experiment Goal Complexity:} The number and trade-offs of objectives, such as optimizing single metrics or navigating multi-objective challenges.
\end{packeditemize}
\fi


% These dimensions enable the evaluation of frameworks across various real-world challenges, with detailed definitions of difficulty levels provided in App.~\ref{app:complex}. 


% Table~\ref{tab:experiment-complexities}.

% \begin{table*}[t]
% \label{table:benchmark-sample}
% \centering
% \caption{Example Questions Across Complexity Dimensions for LLM Reasoning}
% \begin{tabular}{|p{4.3cm}|p{11cm}|}
% \hline
% \textbf{Complexity Dimension} & \textbf{Example Question} \\ \hline
% Search Space Complexity & How does the number of samples impact the success rate? \\ \hline
% Relationship Complexity & What is the mathematical relationship between the number of samples and the success rate (e.g., quadratic, log-linear)? \\ \hline
% Experiment Goal Complexity & To achieve a success rate of 90\% while ensuring response latency remains under 10ms per output token, what is the optimal model type (e.g., GPT-4o, GPT-4o-Mini) and number of samples? \\ \hline
% Experiment Setup Complexity & What is the relationship between the number of samples and success rate across different datasets (Math, Code, etc.)? \\ \hline
% \end{tabular}
% \end{table*}


% \subsection{Workloads}
% Our benchmark workloads span across 4 different topics within computer science, focusing on LLM reasoning, vector indexing, cloud computing and ML training, as summarized in Table~\ref{table:benchmark-overview}.
% These tasks are derived from real-world applications and research challenges to reflect practical use cases. 



% \paragraph{LLM Reasoning.}  
% TODO: what is reasoning, why we need it. scaling test time compute.
% TODO: why we carious about the setups (number of steps, number of generated samples).


% In this workload, we investigate key questions such as: How do the number of generated samples or reasoning steps impact response quality?
% What configurations (e.g., model type and reasoning steps) provide the best trade-offs between accuracy and cost?


% % Answering these questions offers actionable insights for [ LLM reasoning ], ensuring [quality] and cost-efficiency in real-world use cases.

% \paragraph{Vector Indexing.}  
% Vector indexing involves building data structures to efficiently index and retrieve high-dimensional vector embeddings, which are essential for similarity search in tasks like retrieval augmented generation (RAG) and recommendation systems. 
% This workload focuses on algorithm-system co-optimization, where various index mechanisms and configurations directly affect retrieval accuracy, memory usage, and response time.  

% This workload examines questions such as: Which indexing mechanisms (e.g., tree-based or graph-based) perform best for different datasets?
% How do configuration parameters like index size and search depth impact performance?
% What trade-offs exist between retrieval accuracy and resource constraints?
% These questions are critical for understanding the system behavior and deploying scalable, high-performance indexing solutions in production environments.
 

% \paragraph{Cloud Computing.}
% TODO. 
% - Inherently difficult: remote setup unlike in other domains. Interact with cloud infra with complex dependencies. Cite IaC-Eval.
% - Relevance: very much needed search for best instance type leads to cost savings, but very particular to specific use-cases. 

% Cloud computing experimentation investigates the interplay between system configurations, resource utilization, and cost optimization in distributed environments. Unlike other domains, tasks in this workload involve remote setups with complex dependencies, requiring interaction with cloud infrastructure through tools such as Infrastructure-as-Code (IaC).

% Key questions include:

% How do different instance types or configurations impact cost and performance?
% What are the optimal resource allocation strategies for specific use cases?
% By addressing these inherently challenging tasks, this workload highlights the importance of efficient resource management and cost-saving strategies in real-world cloud deployments.

% \paragraph{ML Training.}
% cite MLAgentBench.

% ML training workloads focus on optimizing model training processes across diverse setups, including hyperparameter tuning, resource allocation, and distributed training strategies. These tasks emphasize the interplay between training efficiency, model performance, and resource constraints.

% Sample questions include:

% What are the optimal configurations for distributed training across heterogeneous resources?
% How does the choice of hyperparameters affect convergence time and resource consumption?
% What trade-offs exist between training cost and final model accuracy?
% By addressing these questions, the workload enables frameworks to explore scalable and efficient training methodologies essential for modern machine learning applications.

% Please add the following required packages to your document preamble:
% \usepackage[table,xcdraw]{xcolor}
% Beamer presentation requires \usepackage{colortbl} instead of \usepackage[table,xcdraw]{xcolor}


 
\if 0
\subsection{Evaluation}  
The goal of our evaluation is to assess how rigorously each framework performs the experimentation process, as rigor is the foundation of reliable scientific research.
To achieve this, we evaluate every critical step in the experimentation pipeline, ensuring that each contributes to producing accurate and reproducible conclusions.
% The evaluation utilizes the framework's outputs, including logs of the experimentation process, intermediate results, and final conclusions, to provide a transparent and systematic assessment.

 
% The framework’s performance is measured across three key aspects of the experimentation process:  
% \begin{itemize}  
%     \item \textbf{Experiment Design:} Whether the framework accurately identifies key variables relevant to the research question and generates correct, executable code to set up experiments.  
%     \item \textbf{Experiment Execution:} Whether the framework explores the necessary search space comprehensively, executes all required experiments, and produces valid results.  
%     \item \textbf{Result Analysis:} Whether the framework analyzes experimental data correctly, draws logical conclusions, and provides insights aligned with the documented results.  
% \end{itemize}  

To evaluate each of these steps, we employ the LLM as a judge, which compares the framework's outputs against the ground truth. 
This evaluation ensures that key variables are captured, generated code is correct, the search space is sufficiently explored, and all conclusions are consistent with the ground truth. 
By leveraging an LLM for this purpose, the pipeline provides an efficient and scalable method to verify rigor and identify areas where the framework may require improvement.  

Pat notes: we consutrct this by providing baseline golden truth answers for each of the setup process: search space, setup requirements, log requirements, and so on.. We then pass relevant snippets to each of these validators..
furthermore, we integrated our setup verifier within our pipeline to also perform the analysis. 
Or maybe we can say: we backport everything from our validators, and instead ask these validators to validate the final logs, rather than the progression. 

\amber{do we need to talk about how the eval pipeline is constructed?}

\amber{we might add efficiency (time, cost) as one of the metrics.}


\fi

%%%%%%%%%%%%%%%%%%%%%%%%%%%%%%%%%%%%%%%%%%%%%%%%%%%%%%%%%%%%%%%%%%%%%%%%%%%%%%%%%%%%%%%%%%%%%%%%%%%%%%

%%%%%%%%%%%%%%%%%%%%%%%%%%%%%%%%%%%%%%%%%%%%%%
\begin{table*}[t]
\setlength{\tabcolsep}{3pt}
\centering
\renewcommand{\arraystretch}{1.1}
\tabcolsep=0.2cm
\begin{adjustbox}{max width=\textwidth}  % Set the maximum width to text width
\begin{tabular}{c| cccc ||  c| cc cc}
\toprule
General & \multicolumn{3}{c}{Preference} & Accuracy & Supervised & \multicolumn{3}{c}{Preference} & Accuracy \\ 
LLMs & PrefHit & PrefRecall & Reward & BLEU & Alignment & PrefHit & PrefRecall & Reward & BLEU \\ 
\midrule
GPT-J & 0.2572 & 0.6268 & 0.2410 & 0.0923 & Llama2-7B & 0.2029 & 0.803 & 0.0933 & 0.0947 \\
Pythia-2.8B & 0.3370 & 0.6449 & 0.1716 & 0.1355 & SFT & 0.2428 & 0.8125 & 0.1738 & 0.1364 \\
Qwen2-7B & 0.2790 & 0.8179 & 0.1593 & 0.2530 & Slic & 0.2464 & 0.6171 & 0.1700 & 0.1400 \\
Qwen2-57B & 0.3086 & 0.6481 & 0.6854 & 0.2568 & RRHF & 0.3297 & 0.8234 & 0.2263 & 0.1504 \\
Qwen2-72B & 0.3212 & 0.5555 & 0.6901 & 0.2286 & DPO-BT & 0.2500 & 0.8125 & 0.1728 & 0.1363 \\ 
StarCoder2-15B & 0.2464 & 0.6292 & 0.2962 & 0.1159 & DPO-PT & 0.2572 & 0.8067 & 0.1700 & 0.1348 \\
ChatGLM4-9B & 0.2246 & 0.6099 & 0.1686 & 0.1529 & PRO & 0.3025 & 0.6605 & 0.1802 & 0.1197 \\ 
Llama3-8B & 0.2826 & 0.6425 & 0.2458 & 0.1723 & \textbf{\shortname}* & \textbf{0.3659} & \textbf{0.8279} & \textbf{0.2301} & \textbf{0.1412} \\ 
\bottomrule
\end{tabular}
\end{adjustbox}
\caption{Main results on the StaCoCoQA. The left shows the performance of general LLMs, while the right presents the performance of the fine-tuned LLaMA2-7B across various strong benchmarks for preference alignment. Our method SeAdpra is highlighted in \textbf{bold}.}
\label{main}
\vspace{-0.2cm}
\end{table*}
%%%%%%%%%%%%%%%%%%%%%%%%%%%%%%%%%%%%%%%%%%%%%%%%%%%%%%%%%%%%%%%%%%%%%%%%%%%%%%%%%%%%%%%%%%%%%%%%%%%%
\begin{table}[h]
\centering
\renewcommand{\arraystretch}{1.02}
% \tabcolsep=0.1cm
\begin{adjustbox}{width=0.48\textwidth} % Adjust table width
\begin{tabularx}{0.495\textwidth}{p{1.2cm} p{0.7cm} p{0.95cm}p{0.95cm}p{0.7cm}p{0.7cm}}
     \toprule
    \multirow{2}{*}{\small \textbf{Dataset}} & \multirow{2}{*}{\small Model} & \multicolumn{2}{c}{\small Preference} & \multicolumn{2}{c}{\small Acc } \\ 
    & & \small \textit{PrefHit} & \small \textit{PrefRec} & \small \textit{Reward} & \small \textit{Rouge} \\ 
    \midrule
    \multirow{2}{*}{\small \textbf{Academia}}   & \small PRO & 33.78 & 59.56 & 69.94 & 9.84 \\ 
                                & \small \textbf{Ours} & 36.44 & 60.89 & 70.17 & 10.69 \\ 
    \midrule
    \multirow{2}{*}{\small \textbf{Chemistry}}  & \small PRO & 36.31 & 63.39 & 69.15 & 11.16 \\ 
                                & \small \textbf{Ours} & 38.69 & 64.68 & 69.31 & 12.27 \\ 
    \midrule
    \multirow{2}{*}{\small \textbf{Cooking}}    & \small PRO & 35.29 & 58.32 & 69.87 & 12.13 \\ 
                                & \small \textbf{Ours} & 38.50 & 60.01 & 69.93 & 13.73 \\ 
    \midrule
    \multirow{2}{*}{\small \textbf{Math}}       & \small PRO & 30.00 & 56.50 & 69.06 & 13.50 \\ 
                                & \small \textbf{Ours} & 32.00 & 58.54 & 69.21 & 14.45 \\ 
    \midrule
    \multirow{2}{*}{\small \textbf{Music}}      & \small PRO & 34.33 & 60.22 & 70.29 & 13.05 \\ 
                                & \small \textbf{Ours} & 37.00 & 60.61 & 70.84 & 13.82 \\ 
    \midrule
    \multirow{2}{*}{\small \textbf{Politics}}   & \small PRO & 41.77 & 66.10 & 69.52 & 9.31 \\ 
                                & \small \textbf{Ours} & 42.19 & 66.03 & 69.74 & 9.38 \\ 
    \midrule
    \multirow{2}{*}{\small \textbf{Code}} & \small PRO & 26.00 & 51.13 & 69.17 & 12.44 \\ 
                                & \small \textbf{Ours} & 27.00 & 51.77 & 69.46 & 13.33 \\ 
    \midrule
    \multirow{2}{*}{\small \textbf{Security}}   & \small PRO & 23.62 & 49.23 & 70.13 & 10.63 \\ 
                                & \small \textbf{Ours} & 25.20 & 49.24 & 70.92 & 10.98 \\ 
    \midrule
    \multirow{2}{*}{\small \textbf{Mean}}       & \small PRO & 32.64 & 58.05 & 69.64 & 11.51 \\ 
                                & \small \textbf{Ours} & \textbf{34.25} & \textbf{58.98} & \textbf{69.88} & \textbf{12.33} \\ 
    \bottomrule
\end{tabularx}
\end{adjustbox}
\caption{Main results (\%) on eight publicly available and popular CoQA datasets, comparing the strong list-wise benchmark PRO and \textbf{ours with bold}.}
\label{public}
\end{table}



%%%%%%%%%%%%%%%%%%%%%%%%%%%%%%%%%%%%%%%%%%%%%%%%%%%%%
\begin{table}[h]
\centering
\renewcommand{\arraystretch}{1.02}
\begin{tabularx}{0.48\textwidth}{p{1.45cm} p{0.56cm} p{0.6cm} p{0.6cm} p{0.50cm} p{0.45cm} X}
\toprule
\multirow{2}{*}{Method} & \multicolumn{3}{c}{Preference \((\uparrow)\)} & \multicolumn{3}{c}{Accuracy \((\uparrow)\)} \\ \cmidrule{2-4} \cmidrule{5-7}
& \small PrefHit & \small PrefRec & \small Reward & \small CoSim & \small BLEU & \small Rouge \\ \midrule
\small{SeAdpra} & \textbf{34.8} & \textbf{82.5} & \textbf{22.3} & \textbf{69.1} & \textbf{17.4} & \textbf{21.8} \\ 
\small{-w/o PerAl} & \underline{30.4} & 83.0 & 18.7 & 68.8 & \underline{12.6} & 21.0 \\
\small{-w/o PerCo} & 32.6 & 82.3 & \underline{24.2} & 69.3 & 16.4 & 21.0 \\
\small{-w/o \(\Delta_{Se}\)} & 31.2 & 82.8 & 18.6 & 68.3 & \underline{12.4} & 20.9 \\
\small{-w/o \(\Delta_{Po}\)} & \underline{29.4} & 82.2 & 22.1 & 69.0 & 16.6 & 21.4 \\
\small{\(PerCo_{Se}\)} & 30.9 & 83.5 & 15.6 & 67.6 & \underline{9.9} & 19.6 \\
\small{\(PerCo_{Po}\)} & \underline{30.3} & 82.7 & 20.5 & 68.9 & 14.4 & 20.1 \\ 
\bottomrule
\end{tabularx}
\caption{Ablation Results (\%). \(PerCo_{Se}\) or \(PerCo_{Po}\) only employs Single-APDF in Perceptual Comparison, replacing \(\Delta_{M}\) with \(\Delta_{Se}\) or \(\Delta_{Po}\). The bold represents the overall effect. The underlining highlights the most significant metric for each component's impact.}
\label{ablation}
% \vspace{-0.2cm}
\end{table}

\subsection{Dataset}

% These CoQA datasets contain questions and answers from the Stack Overflow data dump\footnote{https://archive.org/details/stackexchange}, intended for training preference models. 

Due to the additional challenges that programming QA presents for LLMs and the lack of high-quality, authentic multi-answer code preference datasets, we turned to StackExchange \footnote{https://archive.org/details/stackexchange}, a platform with forums that are accompanied by rich question-answering metadata. Based on this, we constructed a large-scale programming QA dataset in real-time (as of May 2024), called StaCoCoQA. It contains over 60,738 programming directories, as shown in Table~\ref{tab:stacocoqa_tags}, and 9,978,474 entries, with partial data statistics displayed in Figure~\ref{fig:dataset}. The data format of StaCoCoQA is presented in Table~\ref{fig::stacocoqa}.

The initial dataset \(D_I\) contains 24,101,803 entries, and is processed by the following steps:
(1) Select entries with "Questioner-picked answer" pairs to represent the preferences of the questioners, resulting in 12,260,106 entries in the \(D_Q\).
(2) Select data where the question includes at least one code block to focus on specific-domain programming QA, resulting in 9,978,474 entries in the dataset \(D_C\).
(3) All HTML tags were cleaned using BeautifulSoup \footnote{https://beautiful-soup-4.readthedocs.io/en/latest/} to ensure that the model is not affected by overly complex and meaningless content.
(4) Control the quality of the dataset by considering factors such as the time the question was posted, the size of the response pool, the difference between the highest and lowest votes within a pool, the votes for each response, the token-level length of the question and the answers, which yields varying sizes: 3K, 8K, 18K, 29K, and 64K. 
The controlled creation time variable and the data details after each processing step are shown in Table~\ref{tab:statistics}.

To further validate the effectiveness of SeAdpra, we also select eight popular topic CoQA datasets\footnote{https://huggingface.co/datasets/HuggingFaceH4/stack-exchange-preferences}, which have been filtered to meet specific criteria for preference models \cite{askell2021general}. Their detailed data information is provided in Table~\ref{domain}.
% Examples of some control variables are shown in Table~\ref{tab:statistics}.
% \noindent\textbf{Baselines}. 
% Following the DPO \cite{rafailov2024direct}, we evaluated several existing approaches aligned with human preference, including GPT-J \cite{gpt-j} and Pythia-2.8B \cite{biderman2023pythia}.  
% Next, we assessed StarCoder2 \cite{lozhkov2024starcoder}, which has demonstrated strong performance in code generation, alongside several general-purpose LLMs: Qwen2 \cite{qwen2}, ChatGLM4 \cite{wang2023cogvlm, glm2024chatglm} and LLaMA serials \cite{touvron2023llama,llama3modelcard}.
% Finally, we fine-tuned LLaMA2-7B on the StaCoCoQA and compared its performance with other strong baselines for supervised learning in preference alignment, including SFT, RRHF \cite{yuan2024rrhf}, Silc \cite{zhao2023slic}, DPO, and PRO \cite{song2024preference}.
%%%%%%%%%%%%%%%%%%%%%%%%%%%%%%%%%%%%%%%%%%%%%%%%%%%%%%%%%%%%%%%%%%%%%%%%%%%%%%%%%%%%%%%%%%%%%%%%%%%%%%%%%%%%%%%%%%%%%%%%%%%%%%%%%%

% For preference evaluation, traditional win-rate assessments are costly and not scalable. For instance, when an existing model \(M_A\) is evaluated against comparison methods \((M_B, M_C, M_D)\) in terms of win rates, upgrading model \(M_A\) would necessitate a reevaluation of its win rates against other models. Furthermore, if a new comparison method \(M_E\) is introduced, the win rates of model \(M_A\) against \(M_E\) would also need to be reassessed. Whether AI or humans are employed as evaluation mediators, binary preference between preferred and non-preferred choices or to score the inference results of the modified model, the costs of this process are substantial. 
% Therefore, from an economic perspective, we propose a novel list preference evaluation method. We utilize manually ranking results as the gold standard for assessing human preferences, to calculate the Hit and Recall, referred to as PrefHit and PrefRecall, respectively. Regardless of whether improving model \(M_A\) or expanding comparison method \(M_E\), only the calculation of PrefHit and PrefRecall for the modified model is required, eliminating the need for human evaluation. 
% We also employ a professional reward model\footnote{https://huggingface.co/OpenAssistant/reward-model-deberta-v3-large}
% for evaluation, denoted as the Reward metric.

% \subsection{Baseline} 
% Following the DPO \cite{rafailov2024direct}, we evaluated several existing approaches aligned with human preference, including GPT-J \cite{gpt-j} and Pythia-2.8B \cite{biderman2023pythia}.  
% Next, we assessed StarCoder2 \cite{lozhkov2024starcoder}, which has demonstrated strong performance in code generation, alongside several general-purpose LLMs: Qwen2 \cite{qwen2}, ChatGLM4 \cite{wang2023cogvlm, glm2024chatglm} and LLaMA serials \cite{touvron2023llama,llama3modelcard}.
% Finally, we fine-tuned LLaMA2-7B on the StaCoCoQA and compared its performance with other strong baselines for supervised learning in preference alignment, including SFT, RRHF \cite{yuan2024rrhf}, Silc \cite{zhao2023slic}, DPO, and PRO \cite{song2024preference}.
\subsection{Evaluation Metrics}
\label{sec: metric}
For preference evaluation, we design PrefHit and PrefRecall, adhering to the "CSTC" criterion outlined in Appendix \ref{sec::cstc}, which overcome the limitations of existing evaluation methods, as detailed in Appendix \ref{metric::mot}.
In addition, we demonstrate the effectiveness of thees new evaluation from two main aspects: 1) consistency with traditional metrics, and 2) applicability in different application scenarios in Appendix \ref{metric::ana}.
Following the previous \cite{song2024preference}, we also employ a professional reward.
% Following the previous \cite{song2024preference}, we also employ a professional reward model\footnote{https://huggingface.co/OpenAssistant/reward-model-deberta-v3-large} \cite{song2024preference}, denoted as the Reward.

For accuracy evaluation, we alternately employ BLEU \cite{papineni2002bleu}, RougeL \cite{lin2004rouge}, and CoSim. Similar to codebertscore \cite{zhou2023codebertscore}, CoSim not only focuses on the semantics of the code but also considers structural matching.
Additionally, the implementation details of SeAdpra are described in detail in the Appendix \ref{sec::imp}.
\subsection{Main Results}
We compared the performance of \shortname with general LLMs and strong preference alignment benchmarks on the StaCoCoQA dataset, as shown in Table~\ref{main}. Additionally, we compared SeAdpra with the strongly supervised alignment model PRO \cite{song2024preference} on eight publicly available CoQA datasets, as presented in Table~\ref{public} and Figure~\ref{fig::public}.

\textbf{Larger Model Parameters, Higher Preference.}
Firstly, the Qwen2 series has adopted DPO \cite{rafailov2024direct} in post-training, resulting in a significant enhancement in Reward.
In a horizontal comparison, the performance of Qwen2-7B and LLaMA2-7B in terms of PrefHit is comparable.
Gradually increasing the parameter size of Qwen2 \cite{qwen2} and LLaMA leads to higher PrefHit and Reward.
Additionally, general LLMs continue to demonstrate strong capabilities of programming understanding and generation preference datasets, contributing to high BLEU scores.
These findings indicate that increasing parameter size can significantly improve alignment.

\textbf{List-wise Ranking Outperforms Pair-wise Comparison.}
Intuitively, list-wise DPO-PT surpasses pair-wise DPO-{BT} on PrefHit. Other list-wise methods, such as RRHF, PRO, and our \shortname, also undoubtedly surpass the pair-wise Slic.

\textbf{Both Parameter Size and Alignment Strategies are Effective.}
Compared to other models, Pythia-2.8B achieved impressive results with significantly fewer parameters .
Effective alignment strategies can balance the performance differences brought by parameter size. For example, LLaMA2-7B with PRO achieves results close to Qwen2-57B in PrefHit. Moreover, LLaMA2-7B combined with our method SeAdpra has already far exceeded the PrefHit of Qwen2-57B.

\textbf{Rather not Higher Reward, Higher PrefHit.}
It is evident that Reward and PrefHit are not always positively correlated, indicating that models do not always accurately learn human preferences and cannot fully replace real human evaluation. Therefore, relying solely on a single public reward model is not sufficiently comprehensive when assessing preference alignment.

% In conclusion, during ensuring precise alignment, SeAdpra will focuse on PrefHit@1, even though the trade-off between PrefHit and PrefRecall is a common issue and increasing recall may sometimes lead to a decrease in hit rate. The positive correlation between Reward and BLEU, indicates that improving the quality of the generated text typically enhances the Reward. 
% Most importantly, evaluating preferences solely based on reward is clearly insufficient, as a high reward does not necessarily correspond to a high PrefHit or PrefRecall.
%%%%%%%%%%%%%%%%%%%%%%%%%%%%%%%%%%%%%%%%%%%
%%%%%%%%%%%%
\begin{figure}
  \centering
  \begin{subfigure}{0.49\linewidth}
    \includegraphics[width=\linewidth]{latex/pic/hit.png}
    \caption{The PrefHit}
    \label{scale:hit}
  \end{subfigure}
  \begin{subfigure}{0.49\linewidth}
    \includegraphics[width=\linewidth]{latex/pic/Recall.png}
    \caption{The PrefRecall}
    \label{scale:recall}
  \end{subfigure}
  \medskip
  \begin{subfigure}{0.48\linewidth}
    \includegraphics[width=\linewidth]{latex/pic/reward.png}
    \caption{The Reward}
    \label{scale:reward}
  \end{subfigure}
  \begin{subfigure}{0.48\linewidth}
    \includegraphics[width=\linewidth]{latex/pic/bleu.png}
    \caption{The BLEU}
    \label{scale:bleu}
  \end{subfigure}
  \caption{The performance with Confidence Interval (CI) of our SeAdpra and PRO at different data scales.}
  \label{fig:scale}
  % \vspace{-0.2cm}
\end{figure}
%%%%%%%%%%%%%%%%%%%%%%%%%%%%%%%%%%%%%%%%%%%%%%%%%%%%%%%%%%%%%%%%%%%%%%%%%%%%%%%%%%%%%%%%%%%%%%%%%%%%%%%%%%%%%%%%

\subsection{Ablation Study}

In this section, we discuss the effectiveness of each component of SeAdpra and its impact on various metrics. The results are presented in Table \ref{ablation}.

\textbf{Perceptual Comparison} aims to prevent the model from relying solely on linguistic probability ordering while neglecting the significance of APDF. Removing this Reward will significantly increase the margin, but PrefHit will decrease, which may hinder the model's ability to compare and learn the preference differences between responses.

\textbf{Perceptual Alignment} seeks to align with the optimal responses; removing it will lead to a significant decrease in PrefHit, while the Reward and accuracy metrics like CoSim will significantly increase, as it tends to favor preference over accuracy.

\textbf{Semantic Perceptual Distance} plays a crucial role in maintaining semantic accuracy in alignment learning. Removing it leads to a significant decrease in BLEU and Rouge. Since sacrificing accuracy recalls more possibilities, PrefHit decreases while PrefRecall increases. Moreover, eliminating both Semantic Perceptual Distance and Perceptual Alignment in \(PerCo_{Po}\) further increases PrefRecall, while the other metrics decline again, consistent with previous observations.


\textbf{Popularity Perceptual Distance} is most closely associated with PrefHit. Eliminating it causes PrefHit to drop to its lowest value, indicating that the popularity attribute is an extremely important factor in code communities.

% In summary, each module has a varying impact on preference and accuracy, but all outperform their respective foundation models and other baselines, as shown in Table \ref{main}, proving their effectiveness.


\subsection{Analysis and Discussion}

\textbf{SeAdpra adept at high-quality data rather than large-scale data.}
In StaCoCoQA, we tested PRO and SeAdpra across different data scales, and the results are shown in Figure~\ref{fig:scale}.
Since we rely on the popularity and clarity of questions and answers to filter data, a larger data scale often results in more pronounced deterioration in data quality. In Figure~\ref{scale:hit}, SeAdpra is highly sensitive to data quality in PrefHit, whereas PRO demonstrates improved performance with larger-scale data. Their performance on Prefrecall is consistent. In the native reward model of PRO, as depicted in Figure~\ref{scale:reward}, the reward fluctuations are minimal, while SeAdpra shows remarkable improvement.

\textbf{SeAdpra is relatively insensitive to ranking length.} 
We assessed SeAdpra's performance on different ranking lengths, as shown in Figure 6a. Unlike PRO, which varied with increasing ranking length, SeAdpra shows no significant differences across different lengths. There is a slight increase in performance on PrefHit and PrefRecall. Additionally, SeAdpra performs better at odd lengths compared to even lengths, which is an interesting phenomenon warranting further investigation.


\textbf{Balance Preference and Accuracy.} 
We analyzed the effect of control weights for Perceptual Comparisons in the optimization objective on preference and accuracy, with the findings presented in Figure~\ref{para:weight}.
When \( \alpha \) is greater than 0.05, the trends in PrefHit and BLEU are consistent, indicating that preference and accuracy can be optimized in tandem. However, when \( \alpha \) is 0.01, PrefHit is highest, but BLEU drops sharply.
Additionally, as \( \alpha \) changes, the variations in PrefHit and Reward, which are related to preference, are consistent with each other, reflecting their unified relationship in the optimization. Similarly, the variations in Recall and BLEU, which are related to accuracy, are also consistent, indicating a strong correlation between generation quality and comprehensiveness. 

%%%%%%%%%%%%%%%%%%%%%%%%%%%%%%%%%%%%%%%%%%%%%%%%%%%%%%%%%%%%%%%%%%%%%%%%%%%%%%%%%
\begin{figure}
  \centering
  \begin{subfigure}{0.475\linewidth}
    \includegraphics[width=\linewidth]{latex/pic/Rank1.png}
    \caption{Ranking length}
    \label{para:rank}
  \end{subfigure}
  \begin{subfigure}{0.475\linewidth}
    \includegraphics[width=\linewidth]{latex/pic/weights1.png}
    \caption{The \(\alpha\) in \(Loss\)}
    \label{para:weight}
  \end{subfigure}
  \caption{Parameters Analysis. Results of experiments on different ranking lengths and the weight \(\alpha\) in \(Loss\).}
  \label{fig:para}
  % \vspace{-0.2cm}
\end{figure}
%%%%%%%%%%%%%%%%%%%%%%%%%%%%%%%%%%%%%%%%%%%%
\begin{figure*}
  \centering
  \begin{subfigure}{0.305\linewidth}
    \includegraphics[width=\linewidth]{latex/pic/se2.pdf}
    \caption{The \(\Delta_{Se}\)}
    \label{visual:se}
  \end{subfigure}
  \begin{subfigure}{0.305\linewidth}
    \includegraphics[width=\linewidth]{latex/pic/po2.pdf}
    \caption{The \(\Delta_{Po}\)}
    \label{visual:po}
  \end{subfigure}
  \begin{subfigure}{0.305\linewidth}
    \includegraphics[width=\linewidth]{latex/pic/sv2.pdf}
    \caption{The \(\Delta_{M}\)}
    \label{visual:sv}
  \end{subfigure}
  \caption{The Visualization of Attribute-Perceptual Distance Factors (APDF) matrix of five responses. The blue represents the response with the highest APDF, and SeAdpra aligns with the fifth response corresponding to the maximum Multi-APDF in (c). The green represents the second response that is next best to the red one.}
  \label{visual}
  % \vspace{-0.2cm}
\end{figure*}
%%%%%%%%%%%%%%%%%%%%%%%%%%%%%%%%%%%%%%%%%
\textbf{Single-APDF Matrix Cannot Predict the Optimal Response.} We randomly selected a pair with a golden label and visualized its specific iteration in Figure~\ref{visual}.
It can be observed that the optimal response in a Single-APDF matrix is not necessarily the same as that in the Multi-APDF matrix.
Specifically, the optimal response in the Semantic Perceptual Factor matrix \(\Delta_{Se}\) is the fifth response in Figure~\ref{visual:se}, while in the Popularity Perceptual Factor matrix \(\Delta_{Po}\) (Figure~\ref{visual:po}), it is the third response. Ultimately, in the Multiple Perceptual Distance Factor matrix \(\Delta_{M}\), the third response is slightly inferior to the fifth response (0.037 vs. 0.038) in Figure~\ref{visual:sv}, and this result aligns with the golden label.
More key findings regarding the ADPF are described in Figure \ref{fig::hot1} and Figure \ref{fig::hot2}.
\section{Conclusion}
In this work, we propose a simple yet effective approach, called SMILE, for graph few-shot learning with fewer tasks. Specifically, we introduce a novel dual-level mixup strategy, including within-task and across-task mixup, for enriching the diversity of nodes within each task and the diversity of tasks. Also, we incorporate the degree-based prior information to learn expressive node embeddings. Theoretically, we prove that SMILE effectively enhances the model's generalization performance. Empirically, we conduct extensive experiments on multiple benchmarks and the results suggest that SMILE significantly outperforms other baselines, including both in-domain and cross-domain few-shot settings.
\section*{Impact Statement}
We introduce \sys, an AI agent framework designed to ensure methodical control, execution reliability, and structured knowledge management throughout the experimentation lifecycle.
We introduce a novel experimentation benchmark, spanning four key domains in computer science, to evaluate the reliability and effectiveness of AI agents in conducting scientific research. Our empirical results demonstrate that \sys achieves higher conclusion accuracy and execution reliability, significantly outperforming state-of-the-art AI agents.


\sys has broad implications across multiple scientific disciplines, including machine learning, cloud computing, and database systems, where rigorous experimentation is essential. Beyond computer science, our framework has the potential to accelerate research in materials science, physics, and biomedical research, where complex experimental setups and iterative hypothesis testing are critical for discovery. By automating experimental workflows with built-in validation, \sys can enhance research productivity, reduce human error, and facilitate large-scale scientific exploration.

Ensuring transparency, fairness, and reproducibility in AI-driven scientific research is paramount. \sys explicitly enforces structured documentation and interpretability, making experimental processes auditable and traceable. However, over-reliance on AI for scientific discovery raises concerns regarding bias in automated decision-making and the need for human oversight. We advocate for hybrid human-AI collaboration, where AI assists researchers rather than replacing critical scientific judgment.

\sys lays the foundation for trustworthy AI-driven scientific experimentation, opening avenues for self-improving agents that refine methodologies through continual learning. Future research could explore domain-specific adaptations, enabling AI to automate rigorous experimentation in disciplines such as drug discovery, materials engineering, and high-energy physics. By bridging AI and the scientific method, \sys has the potential to shape the next generation of AI-powered research methodologies, driving scientific discovery at an unprecedented scale.









% In the unusual situation where you want a paper to appear in the
% references without citing it in the main text, use \nocite
\nocite{langley00}

% \bibliography{example_paper}
% \bibliographystyle{abbrv}

\begin{small}
\bibliography{curie}
\bibliographystyle{icml2025}
\end{small}


%%%%%%%%%%%%%%%%%%%%%%%%%%%%%%%%%%%%%%%%%%%%%%%%%%%%%%%%%%%%%%%%%%%%%%%%%%%%%%%
%%%%%%%%%%%%%%%%%%%%%%%%%%%%%%%%%%%%%%%%%%%%%%%%%%%%%%%%%%%%%%%%%%%%%%%%%%%%%%%
% APPENDIX
%%%%%%%%%%%%%%%%%%%%%%%%%%%%%%%%%%%%%%%%%%%%%%%%%%%%%%%%%%%%%%%%%%%%%%%%%%%%%%%
%%%%%%%%%%%%%%%%%%%%%%%%%%%%%%%%%%%%%%%%%%%%%%%%%%%%%%%%%%%%%%%%%%%%%%%%%%%%%%%
\newpage
\appendix
\onecolumn
\subsection{Lloyd-Max Algorithm}
\label{subsec:Lloyd-Max}
For a given quantization bitwidth $B$ and an operand $\bm{X}$, the Lloyd-Max algorithm finds $2^B$ quantization levels $\{\hat{x}_i\}_{i=1}^{2^B}$ such that quantizing $\bm{X}$ by rounding each scalar in $\bm{X}$ to the nearest quantization level minimizes the quantization MSE. 

The algorithm starts with an initial guess of quantization levels and then iteratively computes quantization thresholds $\{\tau_i\}_{i=1}^{2^B-1}$ and updates quantization levels $\{\hat{x}_i\}_{i=1}^{2^B}$. Specifically, at iteration $n$, thresholds are set to the midpoints of the previous iteration's levels:
\begin{align*}
    \tau_i^{(n)}=\frac{\hat{x}_i^{(n-1)}+\hat{x}_{i+1}^{(n-1)}}2 \text{ for } i=1\ldots 2^B-1
\end{align*}
Subsequently, the quantization levels are re-computed as conditional means of the data regions defined by the new thresholds:
\begin{align*}
    \hat{x}_i^{(n)}=\mathbb{E}\left[ \bm{X} \big| \bm{X}\in [\tau_{i-1}^{(n)},\tau_i^{(n)}] \right] \text{ for } i=1\ldots 2^B
\end{align*}
where to satisfy boundary conditions we have $\tau_0=-\infty$ and $\tau_{2^B}=\infty$. The algorithm iterates the above steps until convergence.

Figure \ref{fig:lm_quant} compares the quantization levels of a $7$-bit floating point (E3M3) quantizer (left) to a $7$-bit Lloyd-Max quantizer (right) when quantizing a layer of weights from the GPT3-126M model at a per-tensor granularity. As shown, the Lloyd-Max quantizer achieves substantially lower quantization MSE. Further, Table \ref{tab:FP7_vs_LM7} shows the superior perplexity achieved by Lloyd-Max quantizers for bitwidths of $7$, $6$ and $5$. The difference between the quantizers is clear at 5 bits, where per-tensor FP quantization incurs a drastic and unacceptable increase in perplexity, while Lloyd-Max quantization incurs a much smaller increase. Nevertheless, we note that even the optimal Lloyd-Max quantizer incurs a notable ($\sim 1.5$) increase in perplexity due to the coarse granularity of quantization. 

\begin{figure}[h]
  \centering
  \includegraphics[width=0.7\linewidth]{sections/figures/LM7_FP7.pdf}
  \caption{\small Quantization levels and the corresponding quantization MSE of Floating Point (left) vs Lloyd-Max (right) Quantizers for a layer of weights in the GPT3-126M model.}
  \label{fig:lm_quant}
\end{figure}

\begin{table}[h]\scriptsize
\begin{center}
\caption{\label{tab:FP7_vs_LM7} \small Comparing perplexity (lower is better) achieved by floating point quantizers and Lloyd-Max quantizers on a GPT3-126M model for the Wikitext-103 dataset.}
\begin{tabular}{c|cc|c}
\hline
 \multirow{2}{*}{\textbf{Bitwidth}} & \multicolumn{2}{|c|}{\textbf{Floating-Point Quantizer}} & \textbf{Lloyd-Max Quantizer} \\
 & Best Format & Wikitext-103 Perplexity & Wikitext-103 Perplexity \\
\hline
7 & E3M3 & 18.32 & 18.27 \\
6 & E3M2 & 19.07 & 18.51 \\
5 & E4M0 & 43.89 & 19.71 \\
\hline
\end{tabular}
\end{center}
\end{table}

\subsection{Proof of Local Optimality of LO-BCQ}
\label{subsec:lobcq_opt_proof}
For a given block $\bm{b}_j$, the quantization MSE during LO-BCQ can be empirically evaluated as $\frac{1}{L_b}\lVert \bm{b}_j- \bm{\hat{b}}_j\rVert^2_2$ where $\bm{\hat{b}}_j$ is computed from equation (\ref{eq:clustered_quantization_definition}) as $C_{f(\bm{b}_j)}(\bm{b}_j)$. Further, for a given block cluster $\mathcal{B}_i$, we compute the quantization MSE as $\frac{1}{|\mathcal{B}_{i}|}\sum_{\bm{b} \in \mathcal{B}_{i}} \frac{1}{L_b}\lVert \bm{b}- C_i^{(n)}(\bm{b})\rVert^2_2$. Therefore, at the end of iteration $n$, we evaluate the overall quantization MSE $J^{(n)}$ for a given operand $\bm{X}$ composed of $N_c$ block clusters as:
\begin{align*}
    \label{eq:mse_iter_n}
    J^{(n)} = \frac{1}{N_c} \sum_{i=1}^{N_c} \frac{1}{|\mathcal{B}_{i}^{(n)}|}\sum_{\bm{v} \in \mathcal{B}_{i}^{(n)}} \frac{1}{L_b}\lVert \bm{b}- B_i^{(n)}(\bm{b})\rVert^2_2
\end{align*}

At the end of iteration $n$, the codebooks are updated from $\mathcal{C}^{(n-1)}$ to $\mathcal{C}^{(n)}$. However, the mapping of a given vector $\bm{b}_j$ to quantizers $\mathcal{C}^{(n)}$ remains as  $f^{(n)}(\bm{b}_j)$. At the next iteration, during the vector clustering step, $f^{(n+1)}(\bm{b}_j)$ finds new mapping of $\bm{b}_j$ to updated codebooks $\mathcal{C}^{(n)}$ such that the quantization MSE over the candidate codebooks is minimized. Therefore, we obtain the following result for $\bm{b}_j$:
\begin{align*}
\frac{1}{L_b}\lVert \bm{b}_j - C_{f^{(n+1)}(\bm{b}_j)}^{(n)}(\bm{b}_j)\rVert^2_2 \le \frac{1}{L_b}\lVert \bm{b}_j - C_{f^{(n)}(\bm{b}_j)}^{(n)}(\bm{b}_j)\rVert^2_2
\end{align*}

That is, quantizing $\bm{b}_j$ at the end of the block clustering step of iteration $n+1$ results in lower quantization MSE compared to quantizing at the end of iteration $n$. Since this is true for all $\bm{b} \in \bm{X}$, we assert the following:
\begin{equation}
\begin{split}
\label{eq:mse_ineq_1}
    \tilde{J}^{(n+1)} &= \frac{1}{N_c} \sum_{i=1}^{N_c} \frac{1}{|\mathcal{B}_{i}^{(n+1)}|}\sum_{\bm{b} \in \mathcal{B}_{i}^{(n+1)}} \frac{1}{L_b}\lVert \bm{b} - C_i^{(n)}(b)\rVert^2_2 \le J^{(n)}
\end{split}
\end{equation}
where $\tilde{J}^{(n+1)}$ is the the quantization MSE after the vector clustering step at iteration $n+1$.

Next, during the codebook update step (\ref{eq:quantizers_update}) at iteration $n+1$, the per-cluster codebooks $\mathcal{C}^{(n)}$ are updated to $\mathcal{C}^{(n+1)}$ by invoking the Lloyd-Max algorithm \citep{Lloyd}. We know that for any given value distribution, the Lloyd-Max algorithm minimizes the quantization MSE. Therefore, for a given vector cluster $\mathcal{B}_i$ we obtain the following result:

\begin{equation}
    \frac{1}{|\mathcal{B}_{i}^{(n+1)}|}\sum_{\bm{b} \in \mathcal{B}_{i}^{(n+1)}} \frac{1}{L_b}\lVert \bm{b}- C_i^{(n+1)}(\bm{b})\rVert^2_2 \le \frac{1}{|\mathcal{B}_{i}^{(n+1)}|}\sum_{\bm{b} \in \mathcal{B}_{i}^{(n+1)}} \frac{1}{L_b}\lVert \bm{b}- C_i^{(n)}(\bm{b})\rVert^2_2
\end{equation}

The above equation states that quantizing the given block cluster $\mathcal{B}_i$ after updating the associated codebook from $C_i^{(n)}$ to $C_i^{(n+1)}$ results in lower quantization MSE. Since this is true for all the block clusters, we derive the following result: 
\begin{equation}
\begin{split}
\label{eq:mse_ineq_2}
     J^{(n+1)} &= \frac{1}{N_c} \sum_{i=1}^{N_c} \frac{1}{|\mathcal{B}_{i}^{(n+1)}|}\sum_{\bm{b} \in \mathcal{B}_{i}^{(n+1)}} \frac{1}{L_b}\lVert \bm{b}- C_i^{(n+1)}(\bm{b})\rVert^2_2  \le \tilde{J}^{(n+1)}   
\end{split}
\end{equation}

Following (\ref{eq:mse_ineq_1}) and (\ref{eq:mse_ineq_2}), we find that the quantization MSE is non-increasing for each iteration, that is, $J^{(1)} \ge J^{(2)} \ge J^{(3)} \ge \ldots \ge J^{(M)}$ where $M$ is the maximum number of iterations. 
%Therefore, we can say that if the algorithm converges, then it must be that it has converged to a local minimum. 
\hfill $\blacksquare$


\begin{figure}
    \begin{center}
    \includegraphics[width=0.5\textwidth]{sections//figures/mse_vs_iter.pdf}
    \end{center}
    \caption{\small NMSE vs iterations during LO-BCQ compared to other block quantization proposals}
    \label{fig:nmse_vs_iter}
\end{figure}

Figure \ref{fig:nmse_vs_iter} shows the empirical convergence of LO-BCQ across several block lengths and number of codebooks. Also, the MSE achieved by LO-BCQ is compared to baselines such as MXFP and VSQ. As shown, LO-BCQ converges to a lower MSE than the baselines. Further, we achieve better convergence for larger number of codebooks ($N_c$) and for a smaller block length ($L_b$), both of which increase the bitwidth of BCQ (see Eq \ref{eq:bitwidth_bcq}).


\subsection{Additional Accuracy Results}
%Table \ref{tab:lobcq_config} lists the various LOBCQ configurations and their corresponding bitwidths.
\begin{table}
\setlength{\tabcolsep}{4.75pt}
\begin{center}
\caption{\label{tab:lobcq_config} Various LO-BCQ configurations and their bitwidths.}
\begin{tabular}{|c||c|c|c|c||c|c||c|} 
\hline
 & \multicolumn{4}{|c||}{$L_b=8$} & \multicolumn{2}{|c||}{$L_b=4$} & $L_b=2$ \\
 \hline
 \backslashbox{$L_A$\kern-1em}{\kern-1em$N_c$} & 2 & 4 & 8 & 16 & 2 & 4 & 2 \\
 \hline
 64 & 4.25 & 4.375 & 4.5 & 4.625 & 4.375 & 4.625 & 4.625\\
 \hline
 32 & 4.375 & 4.5 & 4.625& 4.75 & 4.5 & 4.75 & 4.75 \\
 \hline
 16 & 4.625 & 4.75& 4.875 & 5 & 4.75 & 5 & 5 \\
 \hline
\end{tabular}
\end{center}
\end{table}

%\subsection{Perplexity achieved by various LO-BCQ configurations on Wikitext-103 dataset}

\begin{table} \centering
\begin{tabular}{|c||c|c|c|c||c|c||c|} 
\hline
 $L_b \rightarrow$& \multicolumn{4}{c||}{8} & \multicolumn{2}{c||}{4} & 2\\
 \hline
 \backslashbox{$L_A$\kern-1em}{\kern-1em$N_c$} & 2 & 4 & 8 & 16 & 2 & 4 & 2  \\
 %$N_c \rightarrow$ & 2 & 4 & 8 & 16 & 2 & 4 & 2 \\
 \hline
 \hline
 \multicolumn{8}{c}{GPT3-1.3B (FP32 PPL = 9.98)} \\ 
 \hline
 \hline
 64 & 10.40 & 10.23 & 10.17 & 10.15 &  10.28 & 10.18 & 10.19 \\
 \hline
 32 & 10.25 & 10.20 & 10.15 & 10.12 &  10.23 & 10.17 & 10.17 \\
 \hline
 16 & 10.22 & 10.16 & 10.10 & 10.09 &  10.21 & 10.14 & 10.16 \\
 \hline
  \hline
 \multicolumn{8}{c}{GPT3-8B (FP32 PPL = 7.38)} \\ 
 \hline
 \hline
 64 & 7.61 & 7.52 & 7.48 &  7.47 &  7.55 &  7.49 & 7.50 \\
 \hline
 32 & 7.52 & 7.50 & 7.46 &  7.45 &  7.52 &  7.48 & 7.48  \\
 \hline
 16 & 7.51 & 7.48 & 7.44 &  7.44 &  7.51 &  7.49 & 7.47  \\
 \hline
\end{tabular}
\caption{\label{tab:ppl_gpt3_abalation} Wikitext-103 perplexity across GPT3-1.3B and 8B models.}
\end{table}

\begin{table} \centering
\begin{tabular}{|c||c|c|c|c||} 
\hline
 $L_b \rightarrow$& \multicolumn{4}{c||}{8}\\
 \hline
 \backslashbox{$L_A$\kern-1em}{\kern-1em$N_c$} & 2 & 4 & 8 & 16 \\
 %$N_c \rightarrow$ & 2 & 4 & 8 & 16 & 2 & 4 & 2 \\
 \hline
 \hline
 \multicolumn{5}{|c|}{Llama2-7B (FP32 PPL = 5.06)} \\ 
 \hline
 \hline
 64 & 5.31 & 5.26 & 5.19 & 5.18  \\
 \hline
 32 & 5.23 & 5.25 & 5.18 & 5.15  \\
 \hline
 16 & 5.23 & 5.19 & 5.16 & 5.14  \\
 \hline
 \multicolumn{5}{|c|}{Nemotron4-15B (FP32 PPL = 5.87)} \\ 
 \hline
 \hline
 64  & 6.3 & 6.20 & 6.13 & 6.08  \\
 \hline
 32  & 6.24 & 6.12 & 6.07 & 6.03  \\
 \hline
 16  & 6.12 & 6.14 & 6.04 & 6.02  \\
 \hline
 \multicolumn{5}{|c|}{Nemotron4-340B (FP32 PPL = 3.48)} \\ 
 \hline
 \hline
 64 & 3.67 & 3.62 & 3.60 & 3.59 \\
 \hline
 32 & 3.63 & 3.61 & 3.59 & 3.56 \\
 \hline
 16 & 3.61 & 3.58 & 3.57 & 3.55 \\
 \hline
\end{tabular}
\caption{\label{tab:ppl_llama7B_nemo15B} Wikitext-103 perplexity compared to FP32 baseline in Llama2-7B and Nemotron4-15B, 340B models}
\end{table}

%\subsection{Perplexity achieved by various LO-BCQ configurations on MMLU dataset}


\begin{table} \centering
\begin{tabular}{|c||c|c|c|c||c|c|c|c|} 
\hline
 $L_b \rightarrow$& \multicolumn{4}{c||}{8} & \multicolumn{4}{c||}{8}\\
 \hline
 \backslashbox{$L_A$\kern-1em}{\kern-1em$N_c$} & 2 & 4 & 8 & 16 & 2 & 4 & 8 & 16  \\
 %$N_c \rightarrow$ & 2 & 4 & 8 & 16 & 2 & 4 & 2 \\
 \hline
 \hline
 \multicolumn{5}{|c|}{Llama2-7B (FP32 Accuracy = 45.8\%)} & \multicolumn{4}{|c|}{Llama2-70B (FP32 Accuracy = 69.12\%)} \\ 
 \hline
 \hline
 64 & 43.9 & 43.4 & 43.9 & 44.9 & 68.07 & 68.27 & 68.17 & 68.75 \\
 \hline
 32 & 44.5 & 43.8 & 44.9 & 44.5 & 68.37 & 68.51 & 68.35 & 68.27  \\
 \hline
 16 & 43.9 & 42.7 & 44.9 & 45 & 68.12 & 68.77 & 68.31 & 68.59  \\
 \hline
 \hline
 \multicolumn{5}{|c|}{GPT3-22B (FP32 Accuracy = 38.75\%)} & \multicolumn{4}{|c|}{Nemotron4-15B (FP32 Accuracy = 64.3\%)} \\ 
 \hline
 \hline
 64 & 36.71 & 38.85 & 38.13 & 38.92 & 63.17 & 62.36 & 63.72 & 64.09 \\
 \hline
 32 & 37.95 & 38.69 & 39.45 & 38.34 & 64.05 & 62.30 & 63.8 & 64.33  \\
 \hline
 16 & 38.88 & 38.80 & 38.31 & 38.92 & 63.22 & 63.51 & 63.93 & 64.43  \\
 \hline
\end{tabular}
\caption{\label{tab:mmlu_abalation} Accuracy on MMLU dataset across GPT3-22B, Llama2-7B, 70B and Nemotron4-15B models.}
\end{table}


%\subsection{Perplexity achieved by various LO-BCQ configurations on LM evaluation harness}

\begin{table} \centering
\begin{tabular}{|c||c|c|c|c||c|c|c|c|} 
\hline
 $L_b \rightarrow$& \multicolumn{4}{c||}{8} & \multicolumn{4}{c||}{8}\\
 \hline
 \backslashbox{$L_A$\kern-1em}{\kern-1em$N_c$} & 2 & 4 & 8 & 16 & 2 & 4 & 8 & 16  \\
 %$N_c \rightarrow$ & 2 & 4 & 8 & 16 & 2 & 4 & 2 \\
 \hline
 \hline
 \multicolumn{5}{|c|}{Race (FP32 Accuracy = 37.51\%)} & \multicolumn{4}{|c|}{Boolq (FP32 Accuracy = 64.62\%)} \\ 
 \hline
 \hline
 64 & 36.94 & 37.13 & 36.27 & 37.13 & 63.73 & 62.26 & 63.49 & 63.36 \\
 \hline
 32 & 37.03 & 36.36 & 36.08 & 37.03 & 62.54 & 63.51 & 63.49 & 63.55  \\
 \hline
 16 & 37.03 & 37.03 & 36.46 & 37.03 & 61.1 & 63.79 & 63.58 & 63.33  \\
 \hline
 \hline
 \multicolumn{5}{|c|}{Winogrande (FP32 Accuracy = 58.01\%)} & \multicolumn{4}{|c|}{Piqa (FP32 Accuracy = 74.21\%)} \\ 
 \hline
 \hline
 64 & 58.17 & 57.22 & 57.85 & 58.33 & 73.01 & 73.07 & 73.07 & 72.80 \\
 \hline
 32 & 59.12 & 58.09 & 57.85 & 58.41 & 73.01 & 73.94 & 72.74 & 73.18  \\
 \hline
 16 & 57.93 & 58.88 & 57.93 & 58.56 & 73.94 & 72.80 & 73.01 & 73.94  \\
 \hline
\end{tabular}
\caption{\label{tab:mmlu_abalation} Accuracy on LM evaluation harness tasks on GPT3-1.3B model.}
\end{table}

\begin{table} \centering
\begin{tabular}{|c||c|c|c|c||c|c|c|c|} 
\hline
 $L_b \rightarrow$& \multicolumn{4}{c||}{8} & \multicolumn{4}{c||}{8}\\
 \hline
 \backslashbox{$L_A$\kern-1em}{\kern-1em$N_c$} & 2 & 4 & 8 & 16 & 2 & 4 & 8 & 16  \\
 %$N_c \rightarrow$ & 2 & 4 & 8 & 16 & 2 & 4 & 2 \\
 \hline
 \hline
 \multicolumn{5}{|c|}{Race (FP32 Accuracy = 41.34\%)} & \multicolumn{4}{|c|}{Boolq (FP32 Accuracy = 68.32\%)} \\ 
 \hline
 \hline
 64 & 40.48 & 40.10 & 39.43 & 39.90 & 69.20 & 68.41 & 69.45 & 68.56 \\
 \hline
 32 & 39.52 & 39.52 & 40.77 & 39.62 & 68.32 & 67.43 & 68.17 & 69.30  \\
 \hline
 16 & 39.81 & 39.71 & 39.90 & 40.38 & 68.10 & 66.33 & 69.51 & 69.42  \\
 \hline
 \hline
 \multicolumn{5}{|c|}{Winogrande (FP32 Accuracy = 67.88\%)} & \multicolumn{4}{|c|}{Piqa (FP32 Accuracy = 78.78\%)} \\ 
 \hline
 \hline
 64 & 66.85 & 66.61 & 67.72 & 67.88 & 77.31 & 77.42 & 77.75 & 77.64 \\
 \hline
 32 & 67.25 & 67.72 & 67.72 & 67.00 & 77.31 & 77.04 & 77.80 & 77.37  \\
 \hline
 16 & 68.11 & 68.90 & 67.88 & 67.48 & 77.37 & 78.13 & 78.13 & 77.69  \\
 \hline
\end{tabular}
\caption{\label{tab:mmlu_abalation} Accuracy on LM evaluation harness tasks on GPT3-8B model.}
\end{table}

\begin{table} \centering
\begin{tabular}{|c||c|c|c|c||c|c|c|c|} 
\hline
 $L_b \rightarrow$& \multicolumn{4}{c||}{8} & \multicolumn{4}{c||}{8}\\
 \hline
 \backslashbox{$L_A$\kern-1em}{\kern-1em$N_c$} & 2 & 4 & 8 & 16 & 2 & 4 & 8 & 16  \\
 %$N_c \rightarrow$ & 2 & 4 & 8 & 16 & 2 & 4 & 2 \\
 \hline
 \hline
 \multicolumn{5}{|c|}{Race (FP32 Accuracy = 40.67\%)} & \multicolumn{4}{|c|}{Boolq (FP32 Accuracy = 76.54\%)} \\ 
 \hline
 \hline
 64 & 40.48 & 40.10 & 39.43 & 39.90 & 75.41 & 75.11 & 77.09 & 75.66 \\
 \hline
 32 & 39.52 & 39.52 & 40.77 & 39.62 & 76.02 & 76.02 & 75.96 & 75.35  \\
 \hline
 16 & 39.81 & 39.71 & 39.90 & 40.38 & 75.05 & 73.82 & 75.72 & 76.09  \\
 \hline
 \hline
 \multicolumn{5}{|c|}{Winogrande (FP32 Accuracy = 70.64\%)} & \multicolumn{4}{|c|}{Piqa (FP32 Accuracy = 79.16\%)} \\ 
 \hline
 \hline
 64 & 69.14 & 70.17 & 70.17 & 70.56 & 78.24 & 79.00 & 78.62 & 78.73 \\
 \hline
 32 & 70.96 & 69.69 & 71.27 & 69.30 & 78.56 & 79.49 & 79.16 & 78.89  \\
 \hline
 16 & 71.03 & 69.53 & 69.69 & 70.40 & 78.13 & 79.16 & 79.00 & 79.00  \\
 \hline
\end{tabular}
\caption{\label{tab:mmlu_abalation} Accuracy on LM evaluation harness tasks on GPT3-22B model.}
\end{table}

\begin{table} \centering
\begin{tabular}{|c||c|c|c|c||c|c|c|c|} 
\hline
 $L_b \rightarrow$& \multicolumn{4}{c||}{8} & \multicolumn{4}{c||}{8}\\
 \hline
 \backslashbox{$L_A$\kern-1em}{\kern-1em$N_c$} & 2 & 4 & 8 & 16 & 2 & 4 & 8 & 16  \\
 %$N_c \rightarrow$ & 2 & 4 & 8 & 16 & 2 & 4 & 2 \\
 \hline
 \hline
 \multicolumn{5}{|c|}{Race (FP32 Accuracy = 44.4\%)} & \multicolumn{4}{|c|}{Boolq (FP32 Accuracy = 79.29\%)} \\ 
 \hline
 \hline
 64 & 42.49 & 42.51 & 42.58 & 43.45 & 77.58 & 77.37 & 77.43 & 78.1 \\
 \hline
 32 & 43.35 & 42.49 & 43.64 & 43.73 & 77.86 & 75.32 & 77.28 & 77.86  \\
 \hline
 16 & 44.21 & 44.21 & 43.64 & 42.97 & 78.65 & 77 & 76.94 & 77.98  \\
 \hline
 \hline
 \multicolumn{5}{|c|}{Winogrande (FP32 Accuracy = 69.38\%)} & \multicolumn{4}{|c|}{Piqa (FP32 Accuracy = 78.07\%)} \\ 
 \hline
 \hline
 64 & 68.9 & 68.43 & 69.77 & 68.19 & 77.09 & 76.82 & 77.09 & 77.86 \\
 \hline
 32 & 69.38 & 68.51 & 68.82 & 68.90 & 78.07 & 76.71 & 78.07 & 77.86  \\
 \hline
 16 & 69.53 & 67.09 & 69.38 & 68.90 & 77.37 & 77.8 & 77.91 & 77.69  \\
 \hline
\end{tabular}
\caption{\label{tab:mmlu_abalation} Accuracy on LM evaluation harness tasks on Llama2-7B model.}
\end{table}

\begin{table} \centering
\begin{tabular}{|c||c|c|c|c||c|c|c|c|} 
\hline
 $L_b \rightarrow$& \multicolumn{4}{c||}{8} & \multicolumn{4}{c||}{8}\\
 \hline
 \backslashbox{$L_A$\kern-1em}{\kern-1em$N_c$} & 2 & 4 & 8 & 16 & 2 & 4 & 8 & 16  \\
 %$N_c \rightarrow$ & 2 & 4 & 8 & 16 & 2 & 4 & 2 \\
 \hline
 \hline
 \multicolumn{5}{|c|}{Race (FP32 Accuracy = 48.8\%)} & \multicolumn{4}{|c|}{Boolq (FP32 Accuracy = 85.23\%)} \\ 
 \hline
 \hline
 64 & 49.00 & 49.00 & 49.28 & 48.71 & 82.82 & 84.28 & 84.03 & 84.25 \\
 \hline
 32 & 49.57 & 48.52 & 48.33 & 49.28 & 83.85 & 84.46 & 84.31 & 84.93  \\
 \hline
 16 & 49.85 & 49.09 & 49.28 & 48.99 & 85.11 & 84.46 & 84.61 & 83.94  \\
 \hline
 \hline
 \multicolumn{5}{|c|}{Winogrande (FP32 Accuracy = 79.95\%)} & \multicolumn{4}{|c|}{Piqa (FP32 Accuracy = 81.56\%)} \\ 
 \hline
 \hline
 64 & 78.77 & 78.45 & 78.37 & 79.16 & 81.45 & 80.69 & 81.45 & 81.5 \\
 \hline
 32 & 78.45 & 79.01 & 78.69 & 80.66 & 81.56 & 80.58 & 81.18 & 81.34  \\
 \hline
 16 & 79.95 & 79.56 & 79.79 & 79.72 & 81.28 & 81.66 & 81.28 & 80.96  \\
 \hline
\end{tabular}
\caption{\label{tab:mmlu_abalation} Accuracy on LM evaluation harness tasks on Llama2-70B model.}
\end{table}

%\section{MSE Studies}
%\textcolor{red}{TODO}


\subsection{Number Formats and Quantization Method}
\label{subsec:numFormats_quantMethod}
\subsubsection{Integer Format}
An $n$-bit signed integer (INT) is typically represented with a 2s-complement format \citep{yao2022zeroquant,xiao2023smoothquant,dai2021vsq}, where the most significant bit denotes the sign.

\subsubsection{Floating Point Format}
An $n$-bit signed floating point (FP) number $x$ comprises of a 1-bit sign ($x_{\mathrm{sign}}$), $B_m$-bit mantissa ($x_{\mathrm{mant}}$) and $B_e$-bit exponent ($x_{\mathrm{exp}}$) such that $B_m+B_e=n-1$. The associated constant exponent bias ($E_{\mathrm{bias}}$) is computed as $(2^{{B_e}-1}-1)$. We denote this format as $E_{B_e}M_{B_m}$.  

\subsubsection{Quantization Scheme}
\label{subsec:quant_method}
A quantization scheme dictates how a given unquantized tensor is converted to its quantized representation. We consider FP formats for the purpose of illustration. Given an unquantized tensor $\bm{X}$ and an FP format $E_{B_e}M_{B_m}$, we first, we compute the quantization scale factor $s_X$ that maps the maximum absolute value of $\bm{X}$ to the maximum quantization level of the $E_{B_e}M_{B_m}$ format as follows:
\begin{align}
\label{eq:sf}
    s_X = \frac{\mathrm{max}(|\bm{X}|)}{\mathrm{max}(E_{B_e}M_{B_m})}
\end{align}
In the above equation, $|\cdot|$ denotes the absolute value function.

Next, we scale $\bm{X}$ by $s_X$ and quantize it to $\hat{\bm{X}}$ by rounding it to the nearest quantization level of $E_{B_e}M_{B_m}$ as:

\begin{align}
\label{eq:tensor_quant}
    \hat{\bm{X}} = \text{round-to-nearest}\left(\frac{\bm{X}}{s_X}, E_{B_e}M_{B_m}\right)
\end{align}

We perform dynamic max-scaled quantization \citep{wu2020integer}, where the scale factor $s$ for activations is dynamically computed during runtime.

\subsection{Vector Scaled Quantization}
\begin{wrapfigure}{r}{0.35\linewidth}
  \centering
  \includegraphics[width=\linewidth]{sections/figures/vsquant.jpg}
  \caption{\small Vectorwise decomposition for per-vector scaled quantization (VSQ \citep{dai2021vsq}).}
  \label{fig:vsquant}
\end{wrapfigure}
During VSQ \citep{dai2021vsq}, the operand tensors are decomposed into 1D vectors in a hardware friendly manner as shown in Figure \ref{fig:vsquant}. Since the decomposed tensors are used as operands in matrix multiplications during inference, it is beneficial to perform this decomposition along the reduction dimension of the multiplication. The vectorwise quantization is performed similar to tensorwise quantization described in Equations \ref{eq:sf} and \ref{eq:tensor_quant}, where a scale factor $s_v$ is required for each vector $\bm{v}$ that maps the maximum absolute value of that vector to the maximum quantization level. While smaller vector lengths can lead to larger accuracy gains, the associated memory and computational overheads due to the per-vector scale factors increases. To alleviate these overheads, VSQ \citep{dai2021vsq} proposed a second level quantization of the per-vector scale factors to unsigned integers, while MX \citep{rouhani2023shared} quantizes them to integer powers of 2 (denoted as $2^{INT}$).

\subsubsection{MX Format}
The MX format proposed in \citep{rouhani2023microscaling} introduces the concept of sub-block shifting. For every two scalar elements of $b$-bits each, there is a shared exponent bit. The value of this exponent bit is determined through an empirical analysis that targets minimizing quantization MSE. We note that the FP format $E_{1}M_{b}$ is strictly better than MX from an accuracy perspective since it allocates a dedicated exponent bit to each scalar as opposed to sharing it across two scalars. Therefore, we conservatively bound the accuracy of a $b+2$-bit signed MX format with that of a $E_{1}M_{b}$ format in our comparisons. For instance, we use E1M2 format as a proxy for MX4.

\begin{figure}
    \centering
    \includegraphics[width=1\linewidth]{sections//figures/BlockFormats.pdf}
    \caption{\small Comparing LO-BCQ to MX format.}
    \label{fig:block_formats}
\end{figure}

Figure \ref{fig:block_formats} compares our $4$-bit LO-BCQ block format to MX \citep{rouhani2023microscaling}. As shown, both LO-BCQ and MX decompose a given operand tensor into block arrays and each block array into blocks. Similar to MX, we find that per-block quantization ($L_b < L_A$) leads to better accuracy due to increased flexibility. While MX achieves this through per-block $1$-bit micro-scales, we associate a dedicated codebook to each block through a per-block codebook selector. Further, MX quantizes the per-block array scale-factor to E8M0 format without per-tensor scaling. In contrast during LO-BCQ, we find that per-tensor scaling combined with quantization of per-block array scale-factor to E4M3 format results in superior inference accuracy across models. 
 

\end{document}


% This document was modified from the file originally made available by
% Pat Langley and Andrea Danyluk for ICML-2K. This version was created
% by Iain Murray in 2018, and modified by Alexandre Bouchard in
% 2019 and 2021 and by Csaba Szepesvari, Gang Niu and Sivan Sabato in 2022.
% Modified again in 2023 and 2024 by Sivan Sabato and Jonathan Scarlett.
% Previous contributors include Dan Roy, Lise Getoor and Tobias
% Scheffer, which was slightly modified from the 2010 version by
% Thorsten Joachims & Johannes Fuernkranz, slightly modified from the
% 2009 version by Kiri Wagstaff and Sam Roweis's 2008 version, which is
% slightly modified from Prasad Tadepalli's 2007 version which is a
% lightly changed version of the previous year's version by Andrew
% Moore, which was in turn edited from those of Kristian Kersting and
% Codrina Lauth. Alex Smola contributed to the algorithmic style files.
