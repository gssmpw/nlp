\section{Conclusion and Future Work}
% Poor experimental design and common flaws: https://smcclatchy.github.io/exp-design/03-common-flaws/ 

% Types of experiments: https://smcclatchy.github.io/exp-design/04-types-of-experiments/ (some don't have to be for testing hypothesis, i.e., Exploratory experiments and pilot studies. Could be a pilot study to get prelim data.. on the basis of which we formulate a hypothesis). 

% We don't consider randomization for now?  https://smcclatchy.github.io/exp-design/05-randomized-designs/


% Statistical inference: we don't consider for now due to our area (i.e., CS) but we plan on bringing this in later. There exist some work on this now too

We introduced \sys, an AI agent framework designed to automate and enhance the rigor of scientific experimentation. 
Central to its design is the Experimental Rigor Engine, which enforces methodical control, reliability, and interpretability.
To assess \sys's effectiveness, we developed a new Experimentation Benchmark featuring real-world research-level challenges. Our empirical evaluation, comparing \sys against state-of-the-art AI agents, demonstrated its capability to automate rigorous experimentation.
% achieving a 3.4$\times$ improvement in arriving at the correct conclusion for our experimental tasks. 

% future work here
\if 0
While \sys represents a significant step toward rigorous and automated experimentation, several open research challenges remain.
% First, a key direction is enabling end-to-end scientific discovery, where the system not only conducts experiments but also generates novel research questions and hypotheses based on observed results. 
% Second, while \sys focuses on single-hypothesis experiments, extending it to support multi-hypothesis reasoning—where new hypotheses emerge dynamically—poses challenges in validation and evaluation.
For instance, adapting \sys for interdisciplinary research requires accommodating domain-specific methodologies, uncertainty control, and extended time scales, such as long-term biological studies~\cite{plant1}.
Moreover, enabling knowledge reuse~\cite{agentworkflowmemory} across experiments could enhance efficiency and further accelerate discovery.
We hope \sys inspires further advancements toward fully autonomous and rigorous experimentation in the era of AI agent-driven scientific research.
\fi

We hope \sys inspires further advancements toward fully autonomous and rigorous experimentation in the era of AI agent-driven scientific research.
Several open research challenges remain:
% First, a key direction is enabling end-to-end scientific discovery, where the system not only conducts experiments but also generates novel research questions and hypotheses based on observed results. 
% Second, while \sys focuses on single-hypothesis experiments, extending it to support multi-hypothesis reasoning—where new hypotheses emerge dynamically—poses challenges in validation and evaluation.
For instance, adapting \sys for interdisciplinary research requires accommodating domain-specific methodologies, uncertainty control, and extended time scales, such as long-term biological studies~\cite{plant1}.
Moreover, enabling knowledge reuse~\cite{agentworkflowmemory} across experiments could enhance efficiency and further accelerate discovery.

\if 0
- Ultra Long-running experiments: e.g., determining plant growth https://www.khanacademy.org/science/biology/intro-to-biology/science-of-biology/a/experiments-and-observations 

- multi-hypothesis experiments? (typically these will be more high-level or general questions with room for multiple possible hypothesis, i.e., we run some experiment based on current hypo, get back some observation/data, and another question appears in our head so we create another hypothesis that complements or we use instead going forward. Our claim is that our framework is applicable for multi hypo experiments, but we choose not to focus on these types of questions for now since they are more open-ended and harder to evaluate 

- Statistical testing: quantify uncertain, distinguish real differences. We don't need this since we're currently focusing on CS questions which have much lower stochasticity/uncertainty

- non-experimental forms of hypothesis testing: observing nature, or a computer program simulated model https://www.khanacademy.org/science/biology/intro-to-biology/science-of-biology/a/experiments-and-observations 

- Knowledge reuse: cite workflow paper 
\fi