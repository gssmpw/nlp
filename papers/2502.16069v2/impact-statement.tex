\section*{Impact Statement}
We introduce \sys, an AI agent framework designed to ensure methodical control, execution reliability, and structured knowledge management throughout the experimentation lifecycle.
We introduce a novel experimentation benchmark, spanning four key domains in computer science, to evaluate the reliability and effectiveness of AI agents in conducting scientific research. Our empirical results demonstrate that \sys achieves higher conclusion accuracy and execution reliability, significantly outperforming state-of-the-art AI agents.


\sys has broad implications across multiple scientific disciplines, including machine learning, cloud computing, and database systems, where rigorous experimentation is essential. Beyond computer science, our framework has the potential to accelerate research in materials science, physics, and biomedical research, where complex experimental setups and iterative hypothesis testing are critical for discovery. By automating experimental workflows with built-in validation, \sys can enhance research productivity, reduce human error, and facilitate large-scale scientific exploration.

Ensuring transparency, fairness, and reproducibility in AI-driven scientific research is paramount. \sys explicitly enforces structured documentation and interpretability, making experimental processes auditable and traceable. However, over-reliance on AI for scientific discovery raises concerns regarding bias in automated decision-making and the need for human oversight. We advocate for hybrid human-AI collaboration, where AI assists researchers rather than replacing critical scientific judgment.

\sys lays the foundation for trustworthy AI-driven scientific experimentation, opening avenues for self-improving agents that refine methodologies through continual learning. Future research could explore domain-specific adaptations, enabling AI to automate rigorous experimentation in disciplines such as drug discovery, materials engineering, and high-energy physics. By bridging AI and the scientific method, \sys has the potential to shape the next generation of AI-powered research methodologies, driving scientific discovery at an unprecedented scale.








