\begin{abstract}
\begin{figure*}[t]
    \centering
    \includegraphics[width=\textwidth]{fig/ASAP_pipeline-crop.pdf}
    \vspace{-2mm}
    \caption{Overview of \method. (a) \textbf{Motion Tracking Pre-training and Real Trajectory Collection}: With the humanoid motions retargeted from human videos, we pre-train multiple motion tracking policies to roll out real-world trajectories. (b) \textbf{Delta Action Model Training}: Based on the real-world rollout data, we train the delta action model by minimizing the discrepancy between simulation state $s_t$ and real-world state $s^r_t$. (c) \textbf{Policy Fine-tuning}: We freeze the delta action model, incorporate it into the simulator to align the real-world physics and then fine-tune the pre-trained motion tracking policy. (d) \textbf{Real-World Deployment}: Finally, we deploy the fine-tuned policy directly in the real world without delta action model.
    % \guanya{we need to cite this figure frequently in the paper. (c) is confusing. add ``update''}
    % \zi{The arrows of $s_{t+1}$ are different from Fig.3 (Should be pointing to Policy?)}
    }
    \label{fig:ASAP}
    \vspace{-4mm}
\end{figure*}
Humanoid robots hold the potential for unparalleled versatility for performing human-like, whole-body skills. However, achieving agile and coordinated whole-body motions remains a significant challenge due to the dynamics mismatch between simulation and the real world. Existing approaches, such as system identification (SysID) and domain randomization (DR) methods, often rely on labor-intensive parameter tuning or result in overly conservative policies that sacrifice agility. In this paper, we present \method ({\color{Methodred}A}ligning {\color{Methodred}S}imulation and Re{\color{Methodred}a}l {\color{Methodred}P}hysics), a two-stage framework designed to tackle the dynamics mismatch and enable agile humanoid whole-body skills. 
In the first stage, we pre-train motion tracking policies in simulation using retargeted human motion data. In the second stage, we deploy the policies in the real world and collect real-world data to train a delta (residual) action model that compensates for the dynamics mismatch. Then \method fine-tunes pre-trained policies with the delta action model integrated into the simulator to align effectively with real-world dynamics. We evaluate \method across three transfer scenarios—IsaacGym to IsaacSim, IsaacGym to Genesis, and IsaacGym to the real-world Unitree G1 humanoid robot. Our approach significantly improves agility and whole-body coordination across various dynamic motions, reducing tracking error compared to SysID, DR, and delta dynamics learning baselines. 
\method enables highly agile motions that were previously difficult to achieve, demonstrating the potential of delta action learning in bridging simulation and real-world dynamics. These results suggest a promising sim-to-real direction for developing more expressive and agile humanoids.
\end{abstract}

\IEEEpeerreviewmaketitle
% \vspace{-0.2cm}

% \section{Storyline}

\jiawei{Some phrases I will use:
\begin{itemize}
    \item Dynamics mismatch
    \item Pre-training?(For first-phase deepmimic training)
    \item Post-training?(For second-phase delta A training)
    \item Whole-body skills, Whole-body coordination, whole-body expressiveness
    \item Expressive and agile
\end{itemize}}

\jiawei{Intro}

\begin{enumerate}
    \item Humaoid robots receive attention, because general-purpose, and ``human-like" skills, which makes humanoid stands out.
    \item However, these skills are often limited to locomotion, quasi-static loco-manipulation, and table-top manipulation, lacking agile whole-body skills.
    \item In comparison, in humanoid character control community people have done impressive whole-body skills. The gap is the \textbf{dynamics mismatch} between simulation and real world physics.
    \item Exisiting methods solve it by system identification or sim2real RL. Sysid limitation: laborsome, not scalable; sim2real: over-conservative, can not be too agile.
    \item Key idea: learning residual actions to compensate the sim2real dynamics mismatch.
    \item Our methods achieve this by a two stage framework. First stage: learning motion trakcing policy; Second stage: align physics in sim and real. Collect data, replay, train motion tracking.
    \item We tested our framework on ...
    \item Contribution bullets 1) We propose \method, a two-stage framework that aligns sim and real dynamics by learning residual policy. 2) We show that, through our methods, our policy can achieve super agile skills. 3) Extensive results in simulation and real-world experiment demonstrate that our methods can reduce tracking error.
\end{enumerate}

\jiawei{Related works}

\begin{enumerate}
    \item Learning-based whole body control for humanoid
    \begin{itemize}
        \item Lot of humanoid works recently.
        \item Character animation community achieves more impressive demos.
        \item The gap: dynamics mismatch, which our work aim to adress.
    \end{itemize}
    
    \item System Identification \jiawei{Old version, not precise in my opinion}
    \begin{enumerate}
        \item Explicit Parameter Identification: Using real world performance to calibrate physics parameters, like mass, friction, inertia, and characterastic of motors(Yuxiang's works).
        \item Implicit robot state online sysid: RMA, DATT, history encoder kinds of things.
        \item Our work: A general framework for learning dynamics mismatch.
    \end{enumerate}
    \item Offline and Online System Identification methods
    \begin{enumerate}
        \item Offline: Using pre-collected data, calibrate simulators and robot models
        \item Online: RMA, DATT, history-encoder
        \item Ours: A general framework that combines offline and online sysid(Offline: Collecting data and train delta a, and enable the policy to do online system identification with delta A.)
    \end{enumerate}
    \item Residual Learning for Robotics
    \begin{enumerate}
        \item Residual Policy Learning
        \item Residual Dynamics Learning
        \item Ours: residual policy learning as dynamics?
    \end{enumerate}
\end{enumerate}


\jiawei{Methods1: Learning Motion Tracking Policy}

\begin{enumerate}
    \item Motion Retarget from human to humanoid
    \item Training motion tracking policy
    \item Designs for Sim2Real
        \item A2C
        \item Reward Curriculum
        \item Termination Curriculum
\end{enumerate}

\jiawei{Methods2: Learning Residual Action}

\begin{enumerate}
    \item Rollout pre-trained policy in another simulator / real-world, and record data
    \item Replay data in original environment, and the tracking error represents the dynamics mismatch. Therefore, minimizing the tracking error results in compensating dynamics mismatch.
    \item Training Delta Action to minimizing the tracking error.
    \item Different recipes for training Delta A
        \item Open loop vs Close Loop
        \item Horizons
        \item Regularization
\end{enumerate}

\jiawei{Experiments}

\begin{itemize}
    \item \textbf{Q1}: Can \method outperform other baseline methods(SysID, Domain Randomization) when learning to match multiple simulators dynamics? 
    \item \textbf{Q2}: Can \method learn to match real physics when deployed on real humanoid robots?
    \item \textbf{Q3}: What is the best recipe for training \method?
    \item \textbf{Q4}: Why does \method works? (Some visualization / analysis)
\end{itemize}
\section{Introduction}


\begin{figure}[t]
\centering
\includegraphics[width=0.6\columnwidth]{figures/evaluation_desiderata_V5.pdf}
\vspace{-0.5cm}
\caption{\systemName is a platform for conducting realistic evaluations of code LLMs, collecting human preferences of coding models with real users, real tasks, and in realistic environments, aimed at addressing the limitations of existing evaluations.
}
\label{fig:motivation}
\end{figure}

\begin{figure*}[t]
\centering
\includegraphics[width=\textwidth]{figures/system_design_v2.png}
\caption{We introduce \systemName, a VSCode extension to collect human preferences of code directly in a developer's IDE. \systemName enables developers to use code completions from various models. The system comprises a) the interface in the user's IDE which presents paired completions to users (left), b) a sampling strategy that picks model pairs to reduce latency (right, top), and c) a prompting scheme that allows diverse LLMs to perform code completions with high fidelity.
Users can select between the top completion (green box) using \texttt{tab} or the bottom completion (blue box) using \texttt{shift+tab}.}
\label{fig:overview}
\end{figure*}

As model capabilities improve, large language models (LLMs) are increasingly integrated into user environments and workflows.
For example, software developers code with AI in integrated developer environments (IDEs)~\citep{peng2023impact}, doctors rely on notes generated through ambient listening~\citep{oberst2024science}, and lawyers consider case evidence identified by electronic discovery systems~\citep{yang2024beyond}.
Increasing deployment of models in productivity tools demands evaluation that more closely reflects real-world circumstances~\citep{hutchinson2022evaluation, saxon2024benchmarks, kapoor2024ai}.
While newer benchmarks and live platforms incorporate human feedback to capture real-world usage, they almost exclusively focus on evaluating LLMs in chat conversations~\citep{zheng2023judging,dubois2023alpacafarm,chiang2024chatbot, kirk2024the}.
Model evaluation must move beyond chat-based interactions and into specialized user environments.



 

In this work, we focus on evaluating LLM-based coding assistants. 
Despite the popularity of these tools---millions of developers use Github Copilot~\citep{Copilot}---existing
evaluations of the coding capabilities of new models exhibit multiple limitations (Figure~\ref{fig:motivation}, bottom).
Traditional ML benchmarks evaluate LLM capabilities by measuring how well a model can complete static, interview-style coding tasks~\citep{chen2021evaluating,austin2021program,jain2024livecodebench, white2024livebench} and lack \emph{real users}. 
User studies recruit real users to evaluate the effectiveness of LLMs as coding assistants, but are often limited to simple programming tasks as opposed to \emph{real tasks}~\citep{vaithilingam2022expectation,ross2023programmer, mozannar2024realhumaneval}.
Recent efforts to collect human feedback such as Chatbot Arena~\citep{chiang2024chatbot} are still removed from a \emph{realistic environment}, resulting in users and data that deviate from typical software development processes.
We introduce \systemName to address these limitations (Figure~\ref{fig:motivation}, top), and we describe our three main contributions below.


\textbf{We deploy \systemName in-the-wild to collect human preferences on code.} 
\systemName is a Visual Studio Code extension, collecting preferences directly in a developer's IDE within their actual workflow (Figure~\ref{fig:overview}).
\systemName provides developers with code completions, akin to the type of support provided by Github Copilot~\citep{Copilot}. 
Over the past 3 months, \systemName has served over~\completions suggestions from 10 state-of-the-art LLMs, 
gathering \sampleCount~votes from \userCount~users.
To collect user preferences,
\systemName presents a novel interface that shows users paired code completions from two different LLMs, which are determined based on a sampling strategy that aims to 
mitigate latency while preserving coverage across model comparisons.
Additionally, we devise a prompting scheme that allows a diverse set of models to perform code completions with high fidelity.
See Section~\ref{sec:system} and Section~\ref{sec:deployment} for details about system design and deployment respectively.



\textbf{We construct a leaderboard of user preferences and find notable differences from existing static benchmarks and human preference leaderboards.}
In general, we observe that smaller models seem to overperform in static benchmarks compared to our leaderboard, while performance among larger models is mixed (Section~\ref{sec:leaderboard_calculation}).
We attribute these differences to the fact that \systemName is exposed to users and tasks that differ drastically from code evaluations in the past. 
Our data spans 103 programming languages and 24 natural languages as well as a variety of real-world applications and code structures, while static benchmarks tend to focus on a specific programming and natural language and task (e.g. coding competition problems).
Additionally, while all of \systemName interactions contain code contexts and the majority involve infilling tasks, a much smaller fraction of Chatbot Arena's coding tasks contain code context, with infilling tasks appearing even more rarely. 
We analyze our data in depth in Section~\ref{subsec:comparison}.



\textbf{We derive new insights into user preferences of code by analyzing \systemName's diverse and distinct data distribution.}
We compare user preferences across different stratifications of input data (e.g., common versus rare languages) and observe which affect observed preferences most (Section~\ref{sec:analysis}).
For example, while user preferences stay relatively consistent across various programming languages, they differ drastically between different task categories (e.g. frontend/backend versus algorithm design).
We also observe variations in user preference due to different features related to code structure 
(e.g., context length and completion patterns).
We open-source \systemName and release a curated subset of code contexts.
Altogether, our results highlight the necessity of model evaluation in realistic and domain-specific settings.












\section{Pre-training: Learning Agile Humanoid Skills}
\label{sec:deepmimic}

\subsection{Data Generation: Retargeting Human Video Data}
To track expressive and agile motions, we collect a video dataset of human movements and retarget it to robot motions, creating imitation goals for motion-tracking policies, as shown in \Cref{fig:data_processing} and \Cref{fig:ASAP} (a).

\paragraph{Transforming Human Video to SMPL Motions}

We begin by recording videos (see \Cref{fig:data_processing} (a) and \Cref{fig:action_noise}) of humans performing expressive and agile motions. Using TRAM~\cite{wang2025tram}, we reconstruct 3D motions from videos. TRAM estimates the global trajectory of the human motions in SMPL parameter format~\cite{loper2023smpl}, which includes global root translation, orientation, body poses, and shape parameters, as shown in \Cref{fig:data_processing} (b). The resulting motions are denoted as ${\mathcal{D}}_{\text{SMPL}}$. 





\paragraph{Simulation-based Data Cleaning}

Since the reconstruction process can introduce noise and errors~\cite{he2024learning}, some estimated motions may not be physically feasible, making them unsuitable for motion tracking in the real world. To address this, we employ a ``sim-to-data'' cleaning procedure. Specifically, we use MaskedMimic~\cite{tessler2024maskedmimic}, a physics-based motion tracker, to imitate the SMPL motions from TRAM in IsaacGym simulator~\cite{makoviychuk2021isaac}. The motions (\Cref{fig:data_processing} (c)) that pass this simulation-based validation are saved as the cleaned dataset ${\mathcal{D}}_{\text{SMPL}}^{\text{Cleaned}}$.

\paragraph{Retargeting SMPL Motions to Robot Motions}

With the cleaned dataset ${\mathcal{D}}_{\text{SMPL}}^{\text{Cleaned}}$ in SMPL format, we retarget the motions into robot motions following the shape-and-motion two-stage retargeting process~\cite{he2024learning}. Since the SMPL parameters estimated by TRAM represent various human body shapes, we first optimize the shape parameter $\boldsymbol{\beta}^{\prime}$ to approximate a humanoid shape. By selecting 12 body links with correspondences between humans and humanoids, we perform gradient descent on $\boldsymbol{\beta}^{\prime}$ to minimize joint distances in the rest pose. Using the optimized shape $\boldsymbol{\beta}^{\prime}$ along with the original translation $\boldsymbol{p}$ and pose $\boldsymbol{\theta}$, we apply gradient descent to further minimize the distances of the body links. This process ensures accurate motion retargeting and produces the cleaned robot trajectory dataset ${\mathcal{D}}_{\text{Robot}}^{\text{Cleaned}}$, as shown in \Cref{fig:data_processing} (d). 
% \guanya{no dynamics or RL to get ? Do we need a figure to show a-b-c?}

\subsection{Phase-based Motion Tracking Policy Training}

We formulate the motion-tracking problem as a goal-conditioned reinforcement learning (RL) task, where the policy $\pi$ is trained to track the retargeted robot movement trajectories in the dataset ${\mathcal{D}}_{\text{Robot}}^{\text{Cleaned}}$. Inspired by ~\cite{peng2018deepmimic}, the state $s_t$ includes the robot’s proprioception $s_t^{\mathrm{p}}$ and a time phase variable $\phi \in [0,1]$, where $\phi=0$ represents the start of a motion and $\phi=1$ represents the end. This time phase variable alone $\phi$ is proven to be sufficient to serve as the goal state $\boldsymbol{s}_t^{\mathrm{g}}$ for single-motion tracking~\cite{peng2018deepmimic}. 
The proprioception $s_t^{\mathrm{p}}$ is defined as $s_t^{\mathrm{p}} \triangleq \left[\dofposhist, \dofvelhist,  \rootangvelhist, \gravityhist, \actionhist \right]$, with 5-step history of joint position $\dofpos\in\mathbb{R}^{23}$, joint velocity $\dofvel\in\mathbb{R}^{23}$, root angular velocity $\rootangvel\in\mathbb{R}^3$, root projected gravity $\gravity\in\mathbb{R}^3$, and last action $\actionprev\in\mathbb{R}^{23}$.
Using the agent’s proprioception $s_t^{\mathrm{p}}$ and the goal state $\boldsymbol{s}_t^{\mathrm{g}}$, we define the reward as  $r_t=\mathcal{R}\left(s_t^{\mathrm{p}}, s_t^{\mathrm{g}}\right)$, which is used for policy optimization.  The specific reward terms can be found in~\Cref{tab:deepmimic_reward}. The action $\boldsymbol{a}_t \in \mathbb{R}^{23}$ corresponds to the target joint positions and is passed to a PD controller that actuates the robot’s degrees of freedom. To optimize the policy, we use the proximal policy optimization (PPO)~\cite{schulman2017proximal}, aiming to maximize the cumulative discounted reward $\mathbb{E}\left[\sum_{t=1}^T \gamma^{t-1} r_t\right]$. We identify several design choices that are crucial for achieving stable policy training:

\paragraph{Asymmetric Actor-Critic Training}

Real-world humanoid control is inherently a partially observable Markov decision process (POMDP), where certain task-relevant properties that are readily available in simulation become unobservable in real-world scenarios. However, these missing properties can significantly facilitate policy training in simulation. To bridge this gap, we employ an asymmetric actor-critic framework, where the critic network has access to privileged information such as the global positions of the reference motion and the root linear velocity, while the actor network relies solely on proprioceptive inputs and a time-phase variable. This design not only enhances phase-based motion tracking during training but also enables a simple, phase-driven motion goal for sim-to-real transfer. Crucially, because the actor does not depend on position-based motion targets, our approach eliminates the need for odometry during real-world deployment—overcoming a well-documented challenge in prior work on humanoid robots~\cite{he2024learning,he2024omnih2o}.

% \TODO{\tairan{Check what else is known for critic}}
\paragraph{Termination Curriculum of Tracking Tolerance}
Training a policy to track agile motions in simulation is challenging, as certain motions can be too difficult for the policy to learn effectively. For instance, when imitating a jumping motion, the policy often fails early in training and learns to remain on the ground to avoid landing penalties. To mitigate this issue, we introduce a termination curriculum that progressively refines the motion error tolerance throughout training, guiding the policy toward improved tracking performance. Initially, we set a generous termination threshold of 1.5m, meaning the episode terminates if the robot deviates from the reference motion by this margin. As training progresses, we gradually tighten this threshold to 0.3m, incrementally increasing the tracking demand on the policy. This curriculum allows the policy to first develop basic balancing skills before progressively enforcing stricter motion tracking, ultimately enabling successful execution of high-dynamic behaviors.

% \guanya{a bit more details? termination condition curriculum}
% Specifically, we: \TODO{\tairan{Describe the details of the termination curriculum}}

\paragraph{Reference State Initialization}
Task initialization plays a crucial role in RL training. We find that naively initializing episodes at the start of the reference motion leads to policy failure. For example, in Cristiano Ronaldo's jumping training, starting the episode from the beginning forces the policy to learn sequentially. However, a successful backflip requires mastering the landing first—if the policy cannot land correctly, it will struggle to complete the full motion from takeoff. To address this, we adopt the Reference State Initialization (RSI) framework~\cite{peng2018deepmimic}. Specifically, we randomly sample time-phase variables between 0 and 1, which effectively randomizes the starting point of the reference motion for the policy to track. We then initialize the robot’s state based on the corresponding reference motion at that phase, including root position and orientation, root linear and angular velocities and joint positions and velocities. This initialization strategy significantly improves motion tracking training, particularly for agile whole-body motions, by allowing the policy to learn different motion phases in parallel rather than being constrained to a strictly sequential learning process.

\paragraph{Reward Terms}
We define the reward function $r_t$ with the sum of three terms: 1) penalty, 2) regularization, and 3) task rewards. A detailed summary of these components is provided in \Cref{tab:deepmimic_reward}.
\begin{table}[!h]
    \centering
    \small % Reduce font size
    \setlength{\tabcolsep}{2pt} % Adjust column spacing
    % \renewcommand{\arraystretch}{1.3} % Increase row spacing
    \vspace{-5mm}
    \caption{Reward Terms for Pretraining}
    \vspace{-2mm}
    \label{tab:deepmimic_reward}
    \resizebox{0.8\columnwidth}{!}{ % Ensure the table fits within the column width
    \begin{tabular}{cccc}
        \toprule
        Term & Weight & Term & Weight \\
        \midrule
        \multicolumn{4}{c}{Penalty} \\
        \midrule
        DoF position limits & $-10.0$ & DoF velocity limits & $-5.0$ \\
        Torque limits & $-5.0$ & Termination & $-200.0$ \\
        \midrule
        \multicolumn{4}{c}{Regularization} \\
        \midrule
        Torques & $-1 \times 10^{-6}$ & Action rate & $-0.5$ \\
        Feet orientation & $-2.0$ & Feet heading & $-0.1$ \\
        Slippage & $-1.0$ &  \\
        \midrule
        \multicolumn{4}{c}{Task Reward} \\
        \midrule
        Body position & $1.0$ & VR 3-point & $1.6$ \\
        Body position (feet) & $2.1$ & Body rotation & $0.5$ \\
        Body angular velocity & $0.5$ & Body velocity & $0.5$ \\
        DoF position & $0.75$ & DoF velocity & $0.5$ \\
        \bottomrule
    \end{tabular}
    \vspace{-7mm}
    } % End of resizebox
\end{table}

\paragraph{Domain Randomizations}
To improve the robustness of the pre-trained policy in \Cref{fig:ASAP} (a), we utilized basic domain randomization techniques listed in \Cref{tab:deepmimic_DR}.
% \begin{table}[h]
    \centering
    \setlength{\tabcolsep}{3pt} % Reduce column spacing
    \caption{Domain Randomizations}
    \begin{tabular}{cc}
        \toprule
        Term & Value \\
        \midrule
        \multicolumn{2}{c}{Dynamics Randomization} \\
        \midrule
        Friction & $\mathcal{U}(0.2,1.1)$ \\
        P Gain & $\mathcal{U}(0.925,1.05) \times$ default \\
        Control delay & $\mathcal{U}(20,40) \mathrm{ms}$ \\
        \midrule
        \multicolumn{2}{c}{External Perturbation} \\
        \midrule
        Push robot & interval $= 10 s, v_{x y} = 0.5 \mathrm{~m} / \mathrm{s}$ \\
        \bottomrule
    \end{tabular}
    \label{tab:deepmimic_DR}
\end{table}
\section{Post-training: Training Delta Action Model and Fine-tuning Motion Tracking Policy}
% \zi{The subtitle is a little bit strange. Should be Post-training \textbf{AND} Fine-tuning? Or just: ASAP: Training Delta Dynamics and Fine-tuning Motion Tracking Policy}
The policy trained in the first stage can track the reference motion in the real-world but does not achieve high motion quality. Thus, during the second stage, as shown in ~\Cref{fig:ASAP}~(b) and (c), we leverage real-world data rolled out by the pre-trained policy to train a delta action model, followed by policy refinement through dynamics compensation using this learned delta action model.

\subsection{Data Collection}
We deploy the pretrained policy in the real world to perform whole-body motion tracking tasks (as depicted in~\Cref{fig:data-collect}) and record the resulting trajectories, denoted as $\mathcal{D}^\text{r} = \{s^\text{r}_0, a^\text{r}_0, \dots, s^\text{r}_T, a^\text{r}_T\}$, as illustrated in~\Cref{fig:ASAP}~(a). At each timestep $t$, we use a motion capture device and onboard sensors to record the state: 
$
s_t = [p^\text{base}_t, v_t^\text{base}, \alpha^\text{base}_t, \omega^\text{base}_t, q_t, \dot{q}_t],
$
where $p^\text{base}_t \in \mathbb{R}^3$ represents the robot base 3D position, $v_t^\text{base} \in \mathbb{R}^3$ is base linear velocity, $\alpha^\text{base}_t \in \mathbb{R}^4$ is the robot base orientation represented as a quaternion, $\omega^\text{base}_t \in \mathbb{R}^3$ is the base angular velocity, $q_t \in \mathbb{R}^{23}$ is the vector of joint positions, and $\dot{q}_t \in \mathbb{R}^{23}$ represents joint velocities.


\begin{figure*}[t]
    \centering
    \includegraphics[width=0.9\textwidth]{fig/baselines-crop.pdf}
    \vspace{-1mm}
    \caption{Baselines of \method. (a) Model-free RL training. (b) System ID from real to sim using real-world data. (c) Learning delta dynamics model using real-world data. (d) Our proposed method, learning delta action model using real-world data. }
    \label{fig:baselines}
    \vspace{-4mm}
\end{figure*}

\subsection{Training Delta Action Model}
\label{sec:train-delta-action-model}

Due to the sim-to-real gap, when we replay the real-world trajectories in simulation, the resulting simulated trajectory will likely deviate significantly from real-world recorded trajectories. This discrepancy is a valuable learning signal for learning the mismatch between simulation and real-world physics. We leverage an RL-based delta/residual action model to compensate for the sim-to-real physics gap.

As illustrated in~\Cref{fig:ASAP} (b), the delta action model is defined as $\Delta a_t = \pi^\Delta_\theta(s_t, a_t)$, where the policy $\pi^\Delta_\theta$ learns to output corrective actions based on the current state $s_t$ and the action $a_t$. These corrective actions ($\Delta a_t$) are added to the real-world recorded actions ($a^r_t$) to account for discrepancies between simulation and real-world dynamics.

The RL environment incorporates this delta action model by modifying the simulator dynamics as follows: $ s_{t+1} = f^\text{sim}(s_t, a^r_t + \Delta a_t)$ where $f^\text{sim}$ represents the simulator's dynamics, $a^r_t$ is the reference action recorded from real-world rollouts, and $\Delta a_t$ introduces corrections learned by the delta action model. 




\begin{table}[htp]
    \centering
    \vspace{-2mm}
    \small % Reduce font size
    \setlength{\tabcolsep}{2pt} % Adjust column spacing
    % \renewcommand{\arraystretch}{1.3} % Increase row spacing
    \caption{Reward Terms for Delta Action Learning}
    \vspace{-2mm}
    \label{tab:deltaA_FT_reward}
    \resizebox{0.8\columnwidth}{!}{ % Ensure the table fits within the column width
    \begin{tabular}{cccc}
        \toprule
        Term & Weight & Term & Weight \\
        \midrule
        \multicolumn{4}{c}{Penalty} \\
        \midrule
        DoF position limits & $-10.0$ & DoF velocity limits & $-5.0$ \\
        Torque limits & $-0.1$ & Termination & $-200.0$ \\
        \midrule
        \multicolumn{4}{c}{Regularization} \\
        \midrule
        Action rate & $-0.01$ & Action norm & $-0.2$ \\
        \midrule
        \multicolumn{4}{c}{Task Reward} \\
        \midrule
        Body position & $1.0$ & VR 3-point & $1.0$ \\
        Body position (feet) & $1.0$ & Body rotation & $0.5$ \\
        Body angular velocity & $0.5$ & Body velocity & $0.5$ \\
        DoF position & $0.5$ & DoF velocity & $0.5$ \\
        \bottomrule
    \end{tabular}
    } % End of resizebox
    \vspace{-2mm}
\end{table}

During each RL step: 
\begin{enumerate}
    \item The robot is initialized at the real-world state $s^r_t$.
    \item  A reward signal is computed to minimize the discrepancy between the simulated state $s_{t+1}$ and the recorded real-world state $s^r_{t+1}$, with an additional action magnitude regularization term $\exp(-\lVert a_t \rVert) - 1)$, as specified in~\Cref{tab:deltaA_FT_reward}. The workflow is illustrated in \Cref{fig:ASAP}~(b).
    \item PPO is used to train the delta action policy $\pi^\Delta_\theta$, learning corrected $\Delta a_t$ to match simulation and the real world.
\end{enumerate}



By learning the delta action model, the simulator can accurately reproduce real-world failures. For example, consider a scenario where the simulated robot can jump because its motor strength is overestimated, but the real-world robot cannot jump due to weaker motors. The delta action model $\pi^\Delta_\theta$ will learn to reduce the intensity of lower-body actions, simulating the motor limitations of the real-world robot. This allows the simulator to replicate the real-world dynamics and enables the policy to be fine-tuned to handle these limitations effectively.

\subsection{Fine-tuning Motion Tracking Policy under New Dynamics}
With the learned delta action model $\pi^\Delta (s_t, a_t)$, we can reconstruct the simulation environment with 
$$
s_{t+1} = f^{\text{\method}}(s_t, a_t) = f^\text{sim}(s_t, a_t + \pi^\Delta(s_t, a_t)),
$$
As shown in~\Cref{fig:ASAP} (c), we keep the $\pi^\Delta$ model parameters frozen, and fine-tune the pretrained policy with the same reward summarized in~\Cref{tab:deepmimic_reward}. 






\subsection{Policy Deployment}
Finally, we deploy the fine-tuned policy without delta action model in the real world as shown in \Cref{fig:ASAP}~(d). The fine-tuned policy shows enhanced real-world motion tracking performance compared to the pre-trained policy. Quantitative improvements will be discussed in \Cref{sec:EXP1}.
\section{Experiments}
\label{sec:experiments}
The experiments are designed to address two key research questions.
First, \textbf{RQ1} evaluates whether the average $L_2$-norm of the counterfactual perturbation vectors ($\overline{||\perturb||}$) decreases as the model overfits the data, thereby providing further empirical validation for our hypothesis.
Second, \textbf{RQ2} evaluates the ability of the proposed counterfactual regularized loss, as defined in (\ref{eq:regularized_loss2}), to mitigate overfitting when compared to existing regularization techniques.

% The experiments are designed to address three key research questions. First, \textbf{RQ1} investigates whether the mean perturbation vector norm decreases as the model overfits the data, aiming to further validate our intuition. Second, \textbf{RQ2} explores whether the mean perturbation vector norm can be effectively leveraged as a regularization term during training, offering insights into its potential role in mitigating overfitting. Finally, \textbf{RQ3} examines whether our counterfactual regularizer enables the model to achieve superior performance compared to existing regularization methods, thus highlighting its practical advantage.

\subsection{Experimental Setup}
\textbf{\textit{Datasets, Models, and Tasks.}}
The experiments are conducted on three datasets: \textit{Water Potability}~\cite{kadiwal2020waterpotability}, \textit{Phomene}~\cite{phomene}, and \textit{CIFAR-10}~\cite{krizhevsky2009learning}. For \textit{Water Potability} and \textit{Phomene}, we randomly select $80\%$ of the samples for the training set, and the remaining $20\%$ for the test set, \textit{CIFAR-10} comes already split. Furthermore, we consider the following models: Logistic Regression, Multi-Layer Perceptron (MLP) with 100 and 30 neurons on each hidden layer, and PreactResNet-18~\cite{he2016cvecvv} as a Convolutional Neural Network (CNN) architecture.
We focus on binary classification tasks and leave the extension to multiclass scenarios for future work. However, for datasets that are inherently multiclass, we transform the problem into a binary classification task by selecting two classes, aligning with our assumption.

\smallskip
\noindent\textbf{\textit{Evaluation Measures.}} To characterize the degree of overfitting, we use the test loss, as it serves as a reliable indicator of the model's generalization capability to unseen data. Additionally, we evaluate the predictive performance of each model using the test accuracy.

\smallskip
\noindent\textbf{\textit{Baselines.}} We compare CF-Reg with the following regularization techniques: L1 (``Lasso''), L2 (``Ridge''), and Dropout.

\smallskip
\noindent\textbf{\textit{Configurations.}}
For each model, we adopt specific configurations as follows.
\begin{itemize}
\item \textit{Logistic Regression:} To induce overfitting in the model, we artificially increase the dimensionality of the data beyond the number of training samples by applying a polynomial feature expansion. This approach ensures that the model has enough capacity to overfit the training data, allowing us to analyze the impact of our counterfactual regularizer. The degree of the polynomial is chosen as the smallest degree that makes the number of features greater than the number of data.
\item \textit{Neural Networks (MLP and CNN):} To take advantage of the closed-form solution for computing the optimal perturbation vector as defined in (\ref{eq:opt-delta}), we use a local linear approximation of the neural network models. Hence, given an instance $\inst_i$, we consider the (optimal) counterfactual not with respect to $\model$ but with respect to:
\begin{equation}
\label{eq:taylor}
    \model^{lin}(\inst) = \model(\inst_i) + \nabla_{\inst}\model(\inst_i)(\inst - \inst_i),
\end{equation}
where $\model^{lin}$ represents the first-order Taylor approximation of $\model$ at $\inst_i$.
Note that this step is unnecessary for Logistic Regression, as it is inherently a linear model.
\end{itemize}

\smallskip
\noindent \textbf{\textit{Implementation Details.}} We run all experiments on a machine equipped with an AMD Ryzen 9 7900 12-Core Processor and an NVIDIA GeForce RTX 4090 GPU. Our implementation is based on the PyTorch Lightning framework. We use stochastic gradient descent as the optimizer with a learning rate of $\eta = 0.001$ and no weight decay. We use a batch size of $128$. The training and test steps are conducted for $6000$ epochs on the \textit{Water Potability} and \textit{Phoneme} datasets, while for the \textit{CIFAR-10} dataset, they are performed for $200$ epochs.
Finally, the contribution $w_i^{\varepsilon}$ of each training point $\inst_i$ is uniformly set as $w_i^{\varepsilon} = 1~\forall i\in \{1,\ldots,m\}$.

The source code implementation for our experiments is available at the following GitHub repository: \url{https://anonymous.4open.science/r/COCE-80B4/README.md} 

\subsection{RQ1: Counterfactual Perturbation vs. Overfitting}
To address \textbf{RQ1}, we analyze the relationship between the test loss and the average $L_2$-norm of the counterfactual perturbation vectors ($\overline{||\perturb||}$) over training epochs.

In particular, Figure~\ref{fig:delta_loss_epochs} depicts the evolution of $\overline{||\perturb||}$ alongside the test loss for an MLP trained \textit{without} regularization on the \textit{Water Potability} dataset. 
\begin{figure}[ht]
    \centering
    \includegraphics[width=0.85\linewidth]{img/delta_loss_epochs.png}
    \caption{The average counterfactual perturbation vector $\overline{||\perturb||}$ (left $y$-axis) and the cross-entropy test loss (right $y$-axis) over training epochs ($x$-axis) for an MLP trained on the \textit{Water Potability} dataset \textit{without} regularization.}
    \label{fig:delta_loss_epochs}
\end{figure}

The plot shows a clear trend as the model starts to overfit the data (evidenced by an increase in test loss). 
Notably, $\overline{||\perturb||}$ begins to decrease, which aligns with the hypothesis that the average distance to the optimal counterfactual example gets smaller as the model's decision boundary becomes increasingly adherent to the training data.

It is worth noting that this trend is heavily influenced by the choice of the counterfactual generator model. In particular, the relationship between $\overline{||\perturb||}$ and the degree of overfitting may become even more pronounced when leveraging more accurate counterfactual generators. However, these models often come at the cost of higher computational complexity, and their exploration is left to future work.

Nonetheless, we expect that $\overline{||\perturb||}$ will eventually stabilize at a plateau, as the average $L_2$-norm of the optimal counterfactual perturbations cannot vanish to zero.

% Additionally, the choice of employing the score-based counterfactual explanation framework to generate counterfactuals was driven to promote computational efficiency.

% Future enhancements to the framework may involve adopting models capable of generating more precise counterfactuals. While such approaches may yield to performance improvements, they are likely to come at the cost of increased computational complexity.


\subsection{RQ2: Counterfactual Regularization Performance}
To answer \textbf{RQ2}, we evaluate the effectiveness of the proposed counterfactual regularization (CF-Reg) by comparing its performance against existing baselines: unregularized training loss (No-Reg), L1 regularization (L1-Reg), L2 regularization (L2-Reg), and Dropout.
Specifically, for each model and dataset combination, Table~\ref{tab:regularization_comparison} presents the mean value and standard deviation of test accuracy achieved by each method across 5 random initialization. 

The table illustrates that our regularization technique consistently delivers better results than existing methods across all evaluated scenarios, except for one case -- i.e., Logistic Regression on the \textit{Phomene} dataset. 
However, this setting exhibits an unusual pattern, as the highest model accuracy is achieved without any regularization. Even in this case, CF-Reg still surpasses other regularization baselines.

From the results above, we derive the following key insights. First, CF-Reg proves to be effective across various model types, ranging from simple linear models (Logistic Regression) to deep architectures like MLPs and CNNs, and across diverse datasets, including both tabular and image data. 
Second, CF-Reg's strong performance on the \textit{Water} dataset with Logistic Regression suggests that its benefits may be more pronounced when applied to simpler models. However, the unexpected outcome on the \textit{Phoneme} dataset calls for further investigation into this phenomenon.


\begin{table*}[h!]
    \centering
    \caption{Mean value and standard deviation of test accuracy across 5 random initializations for different model, dataset, and regularization method. The best results are highlighted in \textbf{bold}.}
    \label{tab:regularization_comparison}
    \begin{tabular}{|c|c|c|c|c|c|c|}
        \hline
        \textbf{Model} & \textbf{Dataset} & \textbf{No-Reg} & \textbf{L1-Reg} & \textbf{L2-Reg} & \textbf{Dropout} & \textbf{CF-Reg (ours)} \\ \hline
        Logistic Regression   & \textit{Water}   & $0.6595 \pm 0.0038$   & $0.6729 \pm 0.0056$   & $0.6756 \pm 0.0046$  & N/A    & $\mathbf{0.6918 \pm 0.0036}$                     \\ \hline
        MLP   & \textit{Water}   & $0.6756 \pm 0.0042$   & $0.6790 \pm 0.0058$   & $0.6790 \pm 0.0023$  & $0.6750 \pm 0.0036$    & $\mathbf{0.6802 \pm 0.0046}$                    \\ \hline
%        MLP   & \textit{Adult}   & $0.8404 \pm 0.0010$   & $\mathbf{0.8495 \pm 0.0007}$   & $0.8489 \pm 0.0014$  & $\mathbf{0.8495 \pm 0.0016}$     & $0.8449 \pm 0.0019$                    \\ \hline
        Logistic Regression   & \textit{Phomene}   & $\mathbf{0.8148 \pm 0.0020}$   & $0.8041 \pm 0.0028$   & $0.7835 \pm 0.0176$  & N/A    & $0.8098 \pm 0.0055$                     \\ \hline
        MLP   & \textit{Phomene}   & $0.8677 \pm 0.0033$   & $0.8374 \pm 0.0080$   & $0.8673 \pm 0.0045$  & $0.8672 \pm 0.0042$     & $\mathbf{0.8718 \pm 0.0040}$                    \\ \hline
        CNN   & \textit{CIFAR-10} & $0.6670 \pm 0.0233$   & $0.6229 \pm 0.0850$   & $0.7348 \pm 0.0365$   & N/A    & $\mathbf{0.7427 \pm 0.0571}$                     \\ \hline
    \end{tabular}
\end{table*}

\begin{table*}[htb!]
    \centering
    \caption{Hyperparameter configurations utilized for the generation of Table \ref{tab:regularization_comparison}. For our regularization the hyperparameters are reported as $\mathbf{\alpha/\beta}$.}
    \label{tab:performance_parameters}
    \begin{tabular}{|c|c|c|c|c|c|c|}
        \hline
        \textbf{Model} & \textbf{Dataset} & \textbf{No-Reg} & \textbf{L1-Reg} & \textbf{L2-Reg} & \textbf{Dropout} & \textbf{CF-Reg (ours)} \\ \hline
        Logistic Regression   & \textit{Water}   & N/A   & $0.0093$   & $0.6927$  & N/A    & $0.3791/1.0355$                     \\ \hline
        MLP   & \textit{Water}   & N/A   & $0.0007$   & $0.0022$  & $0.0002$    & $0.2567/1.9775$                    \\ \hline
        Logistic Regression   &
        \textit{Phomene}   & N/A   & $0.0097$   & $0.7979$  & N/A    & $0.0571/1.8516$                     \\ \hline
        MLP   & \textit{Phomene}   & N/A   & $0.0007$   & $4.24\cdot10^{-5}$  & $0.0015$    & $0.0516/2.2700$                    \\ \hline
       % MLP   & \textit{Adult}   & N/A   & $0.0018$   & $0.0018$  & $0.0601$     & $0.0764/2.2068$                    \\ \hline
        CNN   & \textit{CIFAR-10} & N/A   & $0.0050$   & $0.0864$ & N/A    & $0.3018/
        2.1502$                     \\ \hline
    \end{tabular}
\end{table*}

\begin{table*}[htb!]
    \centering
    \caption{Mean value and standard deviation of training time across 5 different runs. The reported time (in seconds) corresponds to the generation of each entry in Table \ref{tab:regularization_comparison}. Times are }
    \label{tab:times}
    \begin{tabular}{|c|c|c|c|c|c|c|}
        \hline
        \textbf{Model} & \textbf{Dataset} & \textbf{No-Reg} & \textbf{L1-Reg} & \textbf{L2-Reg} & \textbf{Dropout} & \textbf{CF-Reg (ours)} \\ \hline
        Logistic Regression   & \textit{Water}   & $222.98 \pm 1.07$   & $239.94 \pm 2.59$   & $241.60 \pm 1.88$  & N/A    & $251.50 \pm 1.93$                     \\ \hline
        MLP   & \textit{Water}   & $225.71 \pm 3.85$   & $250.13 \pm 4.44$   & $255.78 \pm 2.38$  & $237.83 \pm 3.45$    & $266.48 \pm 3.46$                    \\ \hline
        Logistic Regression   & \textit{Phomene}   & $266.39 \pm 0.82$ & $367.52 \pm 6.85$   & $361.69 \pm 4.04$  & N/A   & $310.48 \pm 0.76$                    \\ \hline
        MLP   &
        \textit{Phomene} & $335.62 \pm 1.77$   & $390.86 \pm 2.11$   & $393.96 \pm 1.95$ & $363.51 \pm 5.07$    & $403.14 \pm 1.92$                     \\ \hline
       % MLP   & \textit{Adult}   & N/A   & $0.0018$   & $0.0018$  & $0.0601$     & $0.0764/2.2068$                    \\ \hline
        CNN   & \textit{CIFAR-10} & $370.09 \pm 0.18$   & $395.71 \pm 0.55$   & $401.38 \pm 0.16$ & N/A    & $1287.8 \pm 0.26$                     \\ \hline
    \end{tabular}
\end{table*}

\subsection{Feasibility of our Method}
A crucial requirement for any regularization technique is that it should impose minimal impact on the overall training process.
In this respect, CF-Reg introduces an overhead that depends on the time required to find the optimal counterfactual example for each training instance. 
As such, the more sophisticated the counterfactual generator model probed during training the higher would be the time required. However, a more advanced counterfactual generator might provide a more effective regularization. We discuss this trade-off in more details in Section~\ref{sec:discussion}.

Table~\ref{tab:times} presents the average training time ($\pm$ standard deviation) for each model and dataset combination listed in Table~\ref{tab:regularization_comparison}.
We can observe that the higher accuracy achieved by CF-Reg using the score-based counterfactual generator comes with only minimal overhead. However, when applied to deep neural networks with many hidden layers, such as \textit{PreactResNet-18}, the forward derivative computation required for the linearization of the network introduces a more noticeable computational cost, explaining the longer training times in the table.

\subsection{Hyperparameter Sensitivity Analysis}
The proposed counterfactual regularization technique relies on two key hyperparameters: $\alpha$ and $\beta$. The former is intrinsic to the loss formulation defined in (\ref{eq:cf-train}), while the latter is closely tied to the choice of the score-based counterfactual explanation method used.

Figure~\ref{fig:test_alpha_beta} illustrates how the test accuracy of an MLP trained on the \textit{Water Potability} dataset changes for different combinations of $\alpha$ and $\beta$.

\begin{figure}[ht]
    \centering
    \includegraphics[width=0.85\linewidth]{img/test_acc_alpha_beta.png}
    \caption{The test accuracy of an MLP trained on the \textit{Water Potability} dataset, evaluated while varying the weight of our counterfactual regularizer ($\alpha$) for different values of $\beta$.}
    \label{fig:test_alpha_beta}
\end{figure}

We observe that, for a fixed $\beta$, increasing the weight of our counterfactual regularizer ($\alpha$) can slightly improve test accuracy until a sudden drop is noticed for $\alpha > 0.1$.
This behavior was expected, as the impact of our penalty, like any regularization term, can be disruptive if not properly controlled.

Moreover, this finding further demonstrates that our regularization method, CF-Reg, is inherently data-driven. Therefore, it requires specific fine-tuning based on the combination of the model and dataset at hand.
\section{EXTENSIVE STUDIES AND ANALYSES}

In this section, we aim to thoroughly analyze \method by addressing three central research questions:
\begin{itemize}
    \item \textbf{Q4}: How to best train the delta action model of \method?
    \item \textbf{Q5}: How to best use the delta action model of \method?
    \item \textbf{Q6}: Why and how does \method work?
\end{itemize}

\begin{figure*}[t]
    \centering
    \includegraphics[width=1.0\linewidth]{fig/exp_ablate_deltaA.pdf}
     \vspace{-4mm}
    \caption{Analysis of dataset size, training horizon, and action norm on the performance of $\pi^\Delta$. (a) \textbf{Dataset Size}: Mean Per Joint Position Error (MPJPE) is evaluated for both in-distribution (green) and out-of-distribution (blue) scenarios. Increasing dataset size leads to enhanced generalization, evidenced by decreasing errors in out-of-distribution evaluations. Closed-loop MPJPE (red bars) also shows improvement with larger datasets. (b) \textbf{Training Horizon}: Open-loop MPJPE (heatmap) improves across evaluation points as training horizons increase, achieving the lowest error at 1.5s. However, closed-loop MPJPE (red bars) shows a sweet spot at a training horizon of 1.0s, beyond which no further improvements are observed. The red dashed line represents the pretrained baseline without $\pi^\Delta$ fine-tuning. (c) \textbf{Action Norm}: The action norm weight significantly influences performance. Both open-loop and closed-loop MPJPE decrease as the weight increases up to 0.1, achieving the lowest error. However, further increases in the action norm weight result in degradation of open-loop performance, highlighting the trade-off between action smoothness and policy flexibility.}
    \label{fig:deltaA_ablation}
\end{figure*}

\subsection{Key Factors in Training Delta Action Models}
\label{sec:VA}
To Answer \textbf{Q4} (\textit{How to best train the delta action model of \method}). 
we conduct a systematic study on key factors influencing the performance of the delta action model. 
Specifically, we investigate the impact of dataset size, training horizon, and action norm weight, evaluating their effects on both open-loop and closed-loop performance. Our analysis uncovers the essential principles for effectively training a high-performing delta action model.

\paragraph{Dataset Size} We analyze the impact of dataset size on the training and generalization of $\pi^\Delta$. Simulation data is collected in Isaac Sim, and $\pi^\Delta$ is trained in Isaac Gym. Open-loop performance is assessed on both in-distribution (training) and out-of-distribution (unseen) trajectories, while closed-loop performance is evaluated using the fine-tuned policy in Isaac Sim. As shown in~\Cref{fig:deltaA_ablation}~(a), increasing the dataset size improves $\pi^\Delta$’s generalization, evidenced by reduced errors in out-of-distribution evaluations. However, the improvement in closed-loop performance saturates, with a marginal decrease of only $0.65\%$ when scaling from $4300$ to $43000$ samples, suggesting limited additional benefit from larger datasets.

\paragraph{Training Horizon} The rollout horizon plays a crucial role in learning $\pi^\Delta$. As shown in~\Cref{fig:deltaA_ablation}~(b), longer training horizons generally improve open-loop performance, with a horizon of 1.5s achieving the lowest errors across evaluation points at 0.25s, 0.5s, and 1.0s. However, this trend does not consistently extend to closed-loop performance. The best closed-loop results are observed at a training horizon of 1.0s, indicating that excessively long horizons do not provide additional benefits for fine-tuned policy.

\paragraph{Action Norm Weight} Training $\pi^\Delta$ incorporates an action norm reward to balance dynamics alignment and minimal correction. As illustrated in~\Cref{fig:deltaA_ablation}~(c), both open-loop and closed-loop errors decrease as the action norm weight increases, reaching the lowest error at a weight of $0.1$. However, further increasing the action norm weight causes open-loop errors to rise, likely due to the minimal action norm reward dominates in the delta action RL training. This highlights the importance of carefully tuning the action norm weight to achieve optimal performance.



\subsection{Different Usage of Delta Action Model}
To answer \textbf{Q5} \textit(How to best use the delta action model of \method?), we compare multiple strategies: fixed-point iteration, gradient-based optimization, and reinforcement learning (RL). Given a learned delta policy \(\pi^\Delta\) such that:
\[
f^\text{sim}(s, a + \pi^\Delta(s, a)) \approx f^\text{real}(s, a),
\]
and a nominal policy \(\hat{\pi}(s)\) that performs well in simulation, the goal is to fine-tune \(\hat{\pi}(s)\) for real-world deployment.

A simple approach is one-step dynamics matching, which leads to the relationship:
\[
\pi(s) = \hat{\pi}(s) - \pi^\Delta(s, \pi(s)).
\]
We consider two RL-free methods: fixed-point iteration and gradient-based optimization. Fixed-point iteration refines \(\hat\pi(s)\) iteratively, while gradient-based optimization minimizes a loss function to achieve a better estimate. These methods are compared against RL fine-tuning, which adapts \(\hat\pi(s)\) using reinforcement learning in simulation. The detailed derivation of these two baselines is summarized in \Cref{sec:appendix_more_deltaA_usage}.

Our experiments in \Cref{fig: use_deltaA} show that RL fine-tuning achieves the lowest tracking error during deployment, outperforming training-free methods. 
Both RL-free approaches are myopic and suffer from out-of-distribution issues, limiting their real-world applicability (more discussions in \Cref{sec:appendix_more_deltaA_usage}). 




\begin{figure}[htp]
    \centering
    \includegraphics[width=0.85\linewidth]{fig/exp_use_deltaA.pdf}
    \vspace{-2mm}
    \caption{MPJPE comparison over timesteps for fine-tuning methods using delta actionmodel. RL Fine-Tuning achieves the lowest error, while Fixed-Point Iteration and Gradient Search perform worse than the baseline (Before DeltaA) showing the highest error.}
    \label{fig: use_deltaA}
    \vspace{-4mm}
\end{figure}


\begin{figure}[t]
    \centering
    \includegraphics[width=0.9\linewidth]{fig/exp_ablate_noise.pdf}
    \vspace{-2mm}
    \caption{MPJPE vs. Noise Level for policies fine-tuned with random action noise. Policies with noise levels $\beta \in [0.025, 0.2]$ show improved performance compared to no fine-tuning. Delta action achieves better tracking precision (126 MPJPE) compared to the best action noise (173 MPJPE).}
    \label{fig:action_noise}
    \vspace{-4mm}
\end{figure}




\subsection{Does \method Fine-Tuning Outperform Random Action Noise Fine-Tuning?}
To answer \textbf{Q6} (\textit{How does \method work?}), we validate \method finetuning is better than injecting random-action-noise-based finetuning. And we visualize the average magnitude of the delta action model for each joint.

Random torque noise~\cite{rfi} is a widely used domain randomization technique for legged robots. To determine whether delta action facilitates fine-tuning of pre-trained policies toward real-world dynamics rather than merely enhancing robustness through random action noise, we analyze its impact. Specifically, we assess the effect of applying random action noise during policy fine-tuning in Isaac Gym by modifying the environment dynamics as $s_{t+1} = f^\text{sim}(s_t, a_t + \beta \delta_a)$, where $\delta_a \sim \mathcal{U}[0, 1]$, and deploy it in Genesis. We conduct an ablation study to examine the influence of the noise magnitude, $\beta$, varying from $0.025$ to $0.4$. As shown in~\Cref{fig:action_noise}, within the constrained range of $\beta \in [0.025, 0.2]$, policies fine-tuned with action noise outperform those without fine-tuning in terms of global tracking error (MPJPE). However, the performance of the action noise approach (MPJPE of $150$) does not match the precision achieved by \method (MPJPE of $126$). Furthermore, we visualize the average output of $\pi^\Delta$ learned from IsaacSim data in~\Cref{fig:vis_deltaA_magnitude}, which reveals non-uniform discrepancies across joints. For example, in the G1 humanoid robot under our experimental setup, lower-body motors exhibit a larger dynamics gap compared to upper-body joints. Within the lower-body, the ankle and knee joints show the most pronounced discrepancies. Additionally, asymmetries between the left and right body motors further highlight the complexity. Such structured discrepancies cannot be effectively captured by merely adding uniform action noise.
These findings, along with the results in~\Cref{fig:ASAP_openloop_curves}, demonstrate that delta action not only enhances policy robustness but also enables effective adaptation to real-world dynamics, outperforming naive randomization strategies.

\begin{figure}[t]
    \centering
    \includegraphics[width=1.0\linewidth]{fig/vis_magnitude.pdf}
    \vspace{-2mm}
    \caption{Visualization of IsaacGym-to-IsaacSim $\pi^\Delta$ output magnitude. We compute the average absolute value of each joint over the 4300-episode dataset. Larger red dots indicate higher values. The results suggest that lower-body motors exhibit a larger discrepancy compared to upper-body joints, with the most significant gap observed in the ankle pitch joint of the G1 humanoid.}
    \label{fig:vis_deltaA_magnitude}
    \vspace{-4mm}
\end{figure}



\putsec{related}{Related Work}

\noindent \textbf{Efficient Radiance Field Rendering.}
%
The introduction of Neural Radiance Fields (NeRF)~\cite{mil:sri20} has
generated significant interest in efficient 3D scene representation and
rendering for radiance fields.
%
Over the past years, there has been a large amount of research aimed at
accelerating NeRFs through algorithmic or software
optimizations~\cite{mul:eva22,fri:yu22,che:fun23,sun:sun22}, and the
development of hardware
accelerators~\cite{lee:cho23,li:li23,son:wen23,mub:kan23,fen:liu24}.
%
The state-of-the-art method, 3D Gaussian splatting~\cite{ker:kop23}, has
further fueled interest in accelerating radiance field
rendering~\cite{rad:ste24,lee:lee24,nie:stu24,lee:rho24,ham:mel24} as it
employs rasterization primitives that can be rendered much faster than NeRFs.
%
However, previous research focused on software graphics rendering on
programmable cores or building dedicated hardware accelerators. In contrast,
\name{} investigates the potential of efficient radiance field rendering while
utilizing fixed-function units in graphics hardware.
%
To our knowledge, this is the first work that assesses the performance
implications of rendering Gaussian-based radiance fields on the hardware
graphics pipeline with software and hardware optimizations.

%%%%%%%%%%%%%%%%%%%%%%%%%%%%%%%%%%%%%%%%%%%%%%%%%%%%%%%%%%%%%%%%%%%%%%%%%%
\myparagraph{Enhancing Graphics Rendering Hardware.}
%
The performance advantage of executing graphics rendering on either
programmable shader cores or fixed-function units varies depending on the
rendering methods and hardware designs.
%
Previous studies have explored the performance implication of graphics hardware
design by developing simulation infrastructures for graphics
workloads~\cite{bar:gon06,gub:aam19,tin:sax23,arn:par13}.
%
Additionally, several studies have aimed to improve the performance of
special-purpose hardware such as ray tracing units in graphics
hardware~\cite{cho:now23,liu:cha21} and proposed hardware accelerators for
graphics applications~\cite{lu:hua17,ram:gri09}.
%
In contrast to these works, which primarily evaluate traditional graphics
workloads, our work focuses on improving the performance of volume rendering
workloads, such as Gaussian splatting, which require blending a huge number of
fragments per pixel.

%%%%%%%%%%%%%%%%%%%%%%%%%%%%%%%%%%%%%%%%%%%%%%%%%%%%%%%%%%%%%%%%%%%%%%%%%%
%
In the context of multi-sample anti-aliasing, prior work proposed reducing the
amount of redundant shading by merging fragments from adjacent triangles in a
mesh at the quad granularity~\cite{fat:bou10}.
%
While both our work and quad-fragment merging (QFM)~\cite{fat:bou10} aim to
reduce operations by merging quads, our proposed technique differs from QFM in
many aspects.
%
Our method aims to blend \emph{overlapping primitives} along the depth
direction and applies to quads from any primitive. In contrast, QFM merges quad
fragments from small (e.g., pixel-sized) triangles that \emph{share} an edge
(i.e., \emph{connected}, \emph{non-overlapping} triangles).
%
As such, QFM is not applicable to the scenes consisting of a number of
unconnected transparent triangles, such as those in 3D Gaussian splatting.
%
In addition, our method computes the \emph{exact} color for each pixel by
offloading blending operations from ROPs to shader units, whereas QFM
\emph{approximates} pixel colors by using the color from one triangle when
multiple triangles are merged into a single quad.



\section{Conclusion}
In this work, we propose a simple yet effective approach, called SMILE, for graph few-shot learning with fewer tasks. Specifically, we introduce a novel dual-level mixup strategy, including within-task and across-task mixup, for enriching the diversity of nodes within each task and the diversity of tasks. Also, we incorporate the degree-based prior information to learn expressive node embeddings. Theoretically, we prove that SMILE effectively enhances the model's generalization performance. Empirically, we conduct extensive experiments on multiple benchmarks and the results suggest that SMILE significantly outperforms other baselines, including both in-domain and cross-domain few-shot settings.

\section*{Limitations and Ethical Considerations}

\noindent\textbf{Limitations.} The primary limitation of our work is that it extends only the dataset provided by MUSE and employs DeepSeek-v3 for question generation. 
To mitigate this generalization risk, we have released our code and the generated audit suite, allowing researchers to utilize our framework to create additional audit datasets and evaluate their quality. Meanwhile, this is also our future work to extend our framework to other benchmarks.

\noindent\textbf{Ethical Considerations.} Machine unlearning can be employed to mitigate risks associated with LLMs in terms of privacy, security, bias, and copyright. Our work is dedicated to providing a comprehensive evaluation framework to help researchers better understand the unlearning effectiveness of LLMs, which we believe will have a positive impact on society.



