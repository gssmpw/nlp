\begin{abstract}
\begin{figure*}[t]
    \centering
    \includegraphics[width=\textwidth]{fig/ASAP_pipeline-crop.pdf}
    \vspace{-2mm}
    \caption{Overview of \method. (a) \textbf{Motion Tracking Pre-training and Real Trajectory Collection}: With the humanoid motions retargeted from human videos, we pre-train multiple motion tracking policies to roll out real-world trajectories. (b) \textbf{Delta Action Model Training}: Based on the real-world rollout data, we train the delta action model by minimizing the discrepancy between simulation state $s_t$ and real-world state $s^r_t$. (c) \textbf{Policy Fine-tuning}: We freeze the delta action model, incorporate it into the simulator to align the real-world physics and then fine-tune the pre-trained motion tracking policy. (d) \textbf{Real-World Deployment}: Finally, we deploy the fine-tuned policy directly in the real world without delta action model.
    % \guanya{we need to cite this figure frequently in the paper. (c) is confusing. add ``update''}
    % \zi{The arrows of $s_{t+1}$ are different from Fig.3 (Should be pointing to Policy?)}
    }
    \label{fig:ASAP}
    \vspace{-4mm}
\end{figure*}
Humanoid robots hold the potential for unparalleled versatility for performing human-like, whole-body skills. However, achieving agile and coordinated whole-body motions remains a significant challenge due to the dynamics mismatch between simulation and the real world. Existing approaches, such as system identification (SysID) and domain randomization (DR) methods, often rely on labor-intensive parameter tuning or result in overly conservative policies that sacrifice agility. In this paper, we present \method ({\color{Methodred}A}ligning {\color{Methodred}S}imulation and Re{\color{Methodred}a}l {\color{Methodred}P}hysics), a two-stage framework designed to tackle the dynamics mismatch and enable agile humanoid whole-body skills. 
In the first stage, we pre-train motion tracking policies in simulation using retargeted human motion data. In the second stage, we deploy the policies in the real world and collect real-world data to train a delta (residual) action model that compensates for the dynamics mismatch. Then \method fine-tunes pre-trained policies with the delta action model integrated into the simulator to align effectively with real-world dynamics. We evaluate \method across three transfer scenarios—IsaacGym to IsaacSim, IsaacGym to Genesis, and IsaacGym to the real-world Unitree G1 humanoid robot. Our approach significantly improves agility and whole-body coordination across various dynamic motions, reducing tracking error compared to SysID, DR, and delta dynamics learning baselines. 
\method enables highly agile motions that were previously difficult to achieve, demonstrating the potential of delta action learning in bridging simulation and real-world dynamics. These results suggest a promising sim-to-real direction for developing more expressive and agile humanoids.
\end{abstract}

\IEEEpeerreviewmaketitle
% \vspace{-0.2cm}

% \section{Storyline}

\jiawei{Some phrases I will use:
\begin{itemize}
    \item Dynamics mismatch
    \item Pre-training?(For first-phase deepmimic training)
    \item Post-training?(For second-phase delta A training)
    \item Whole-body skills, Whole-body coordination, whole-body expressiveness
    \item Expressive and agile
\end{itemize}}

\jiawei{Intro}

\begin{enumerate}
    \item Humaoid robots receive attention, because general-purpose, and ``human-like" skills, which makes humanoid stands out.
    \item However, these skills are often limited to locomotion, quasi-static loco-manipulation, and table-top manipulation, lacking agile whole-body skills.
    \item In comparison, in humanoid character control community people have done impressive whole-body skills. The gap is the \textbf{dynamics mismatch} between simulation and real world physics.
    \item Exisiting methods solve it by system identification or sim2real RL. Sysid limitation: laborsome, not scalable; sim2real: over-conservative, can not be too agile.
    \item Key idea: learning residual actions to compensate the sim2real dynamics mismatch.
    \item Our methods achieve this by a two stage framework. First stage: learning motion trakcing policy; Second stage: align physics in sim and real. Collect data, replay, train motion tracking.
    \item We tested our framework on ...
    \item Contribution bullets 1) We propose \method, a two-stage framework that aligns sim and real dynamics by learning residual policy. 2) We show that, through our methods, our policy can achieve super agile skills. 3) Extensive results in simulation and real-world experiment demonstrate that our methods can reduce tracking error.
\end{enumerate}

\jiawei{Related works}

\begin{enumerate}
    \item Learning-based whole body control for humanoid
    \begin{itemize}
        \item Lot of humanoid works recently.
        \item Character animation community achieves more impressive demos.
        \item The gap: dynamics mismatch, which our work aim to adress.
    \end{itemize}
    
    \item System Identification \jiawei{Old version, not precise in my opinion}
    \begin{enumerate}
        \item Explicit Parameter Identification: Using real world performance to calibrate physics parameters, like mass, friction, inertia, and characterastic of motors(Yuxiang's works).
        \item Implicit robot state online sysid: RMA, DATT, history encoder kinds of things.
        \item Our work: A general framework for learning dynamics mismatch.
    \end{enumerate}
    \item Offline and Online System Identification methods
    \begin{enumerate}
        \item Offline: Using pre-collected data, calibrate simulators and robot models
        \item Online: RMA, DATT, history-encoder
        \item Ours: A general framework that combines offline and online sysid(Offline: Collecting data and train delta a, and enable the policy to do online system identification with delta A.)
    \end{enumerate}
    \item Residual Learning for Robotics
    \begin{enumerate}
        \item Residual Policy Learning
        \item Residual Dynamics Learning
        \item Ours: residual policy learning as dynamics?
    \end{enumerate}
\end{enumerate}


\jiawei{Methods1: Learning Motion Tracking Policy}

\begin{enumerate}
    \item Motion Retarget from human to humanoid
    \item Training motion tracking policy
    \item Designs for Sim2Real
        \item A2C
        \item Reward Curriculum
        \item Termination Curriculum
\end{enumerate}

\jiawei{Methods2: Learning Residual Action}

\begin{enumerate}
    \item Rollout pre-trained policy in another simulator / real-world, and record data
    \item Replay data in original environment, and the tracking error represents the dynamics mismatch. Therefore, minimizing the tracking error results in compensating dynamics mismatch.
    \item Training Delta Action to minimizing the tracking error.
    \item Different recipes for training Delta A
        \item Open loop vs Close Loop
        \item Horizons
        \item Regularization
\end{enumerate}

\jiawei{Experiments}

\begin{itemize}
    \item \textbf{Q1}: Can \method outperform other baseline methods(SysID, Domain Randomization) when learning to match multiple simulators dynamics? 
    \item \textbf{Q2}: Can \method learn to match real physics when deployed on real humanoid robots?
    \item \textbf{Q3}: What is the best recipe for training \method?
    \item \textbf{Q4}: Why does \method works? (Some visualization / analysis)
\end{itemize}
\section{Introduction}
\label{sec:intro}

\begin{figure*}[tb]
    \centering
    \includegraphics[width=0.848\linewidth]{figs/circuitnn.pdf} 
    \caption{Illustration of differentiable CircuitNN. CircuitNN is designed based on differentiable NAND gates. After DAS is guided by PI and PO pairs of the truth table, CircuitNN can get the precise circuit architecture logic equivalent to the truth table.}
    \label{fig:circuitnn}
\end{figure*}

% 1. Describe the importance of logic synthesis
% 2. Existing Problems
% (a) Neural Architecture Search: Unstable, Predefined Setting, etc.
% (b) Circuit Generation: Probabilistic Model, Logic Equivalence

With the rapid advancement of technology, the scale of integrated circuits (ICs) has expanded exponentially. 
This expansion has introduced significant challenges in chip manufacturing, particularly concerning power and area metrics.
A primary objective in IC design is achieving the same circuit function with fewer transistors, thereby reducing power usage and area occupancy.

Logic synthesis~\cite{hachtel2005logicsynth}, a critical step in electronic design automation (EDA), transforms behavioral-level circuit designs into optimized gate-level circuits, ultimately yielding the final IC layout. 
The primary goal of logic synthesis is to identify the physical implementation with the fewest gates for a given circuit function. 
This task constitutes a challenging NP-hard combinatorial optimization problem. 
Current logic synthesis tools~\cite{brayton2010abc, wolf2013yosys} rely on human-designed heuristics, often leading to sub-optimal outcomes.

Differentiable architecture search (DAS) techniques~\cite{liu2018darts, chu2020darts} offer novel perspectives on addressing challenges in this problem.
Circuit functions can be represented through truth tables, which map binary inputs to their corresponding outputs. 
Truth tables provide a precise representation of input-output relationships, ensuring the design of functionally equivalent circuits.
Inspired by this, researchers~\cite{deepmind2024ai4sys, wang2024tnet} have begun exploring the application of DAS to synthesize circuits directly from truth tables.
Specifically, \citet{deepmind2024ai4sys} proposed CircuitNN, a framework that learns differentiable connection structures with logic gates, enabling the automatic generation of logic circuits from truth tables.
This approach significantly reduces the complexity of traditional circuit generation. 
Building on this, \citet{wang2024tnet} introduced T-Net, a triangle-shaped variant of CircuitNN, incorporating regularization techniques to enhance the efficiency of DAS.

Despite these advancements, several challenges remain. 
The computational complexity of DAS grows quadratically with the number of gates, posing scalability issues.
Although triangle-shaped architecture~\cite{wang2024tnet} partially mitigates this problem, redundancy persists. 
%Additionally, DAS is susceptible to converging to local optima, limiting the ability to search architectures that satisfy the given truth tables~\cite{liu2018darts}. 
%Furthermore, hyperparameters (network depth and layer width) require extensive searches, introducing complexity and prolonging the synthesis process. 
Additionally, DAS is susceptible to converging to local optima~\cite{liu2018darts} and hyperparameters (network depth and layer width) require extensive searches. 
The challenges arise from the vast search space in DAS. 
% Even with predefined settings for CircuitNN, finding a configuration that meets the truth table requires extensive trial and error during the DAS process. 
Intuitively, limiting the search space through predefined parameters (network depth, gates per layer, and connection probabilities) can significantly reduce the complexity.

Recent advances~\cite{openai2023gpt4, abramson2024alphafold3, esser2024sd3, li2024mar} in conditional generative models have demonstrated remarkable performance across language, vision, and graph generation tasks. 
Motivated by these developments, we propose a novel approach to circuit generation that generates preliminary circuit structures to guide DAS in generating refined circuits matching specified truth tables. 
Firstly, we introduce CircuitVQ, a tokenizer with a discrete codebook for circuit tokenization. 
Built upon our Circuit AutoEncoder framework~\cite{hou2022graphmae,li2023maskgae,wu2025mgvga}, CircuitVQ is trained through a circuit reconstruction task. 
Specifically, the CircuitVQ encoder encodes input circuits into discrete tokens using a learnable codebook, while the decoder reconstructs the circuit adjacency matrix based on these tokens.
Subsequently, the CircuitVQ encoder serves as a circuit tokenizer for CircuitAR pretraining, which employs a masked autoregressive modeling paradigm~\cite{chang2022maskgit, li2023mage}. 
In this process, the discrete codes function as supervision signals. 
After training, CircuitAR can generate discrete tokens progressively, which can be decoded into initial circuit structures by the decoder of the CircuitVQ. 
These prior insights can guide DAS in producing refined circuits that match the target truth tables precisely.

Our key contributions can be summarized as follows:
\begin{itemize}
\item We introduce CircuitVQ, a circuit tokenizer that facilitates graph autoregressive modeling for circuit generation, based on our Circuit AutoEncoder framework;
\item Develop CircuitAR, a model trained using masked autoregressive modeling, which generates initial circuit structures conditioned on given truth tables;
\item Propose a refinement framework that integrates differentiable architecture search to produce functionally equivalent circuits guided by target truth tables;
\item Comprehensive experiments demonstrating the scalability and capability emergence of our CircuitAR and the superior performance of the proposed circuit generation approach.
\end{itemize}

% Motivation
% (a) Diffusion (Vision, Graph), Autoregressive (Language, Vision)
% (b) Circuit Generation for Predefined Setting
% (c) Neural Architecture Search for Strict Logic Equivalence

% Contribution
% (a) Circuit Tokenizer (new transformer arch, training strategy)
% (b) CircuitAR (train and gen strategies, post-ar strategy)
% (c) Extensive Evaluation including BitD (Bit Distance) for Scalability








\section{Pre-training: Learning Agile Humanoid Skills}
\label{sec:deepmimic}

\subsection{Data Generation: Retargeting Human Video Data}
To track expressive and agile motions, we collect a video dataset of human movements and retarget it to robot motions, creating imitation goals for motion-tracking policies, as shown in \Cref{fig:data_processing} and \Cref{fig:ASAP} (a).

\paragraph{Transforming Human Video to SMPL Motions}

We begin by recording videos (see \Cref{fig:data_processing} (a) and \Cref{fig:action_noise}) of humans performing expressive and agile motions. Using TRAM~\cite{wang2025tram}, we reconstruct 3D motions from videos. TRAM estimates the global trajectory of the human motions in SMPL parameter format~\cite{loper2023smpl}, which includes global root translation, orientation, body poses, and shape parameters, as shown in \Cref{fig:data_processing} (b). The resulting motions are denoted as ${\mathcal{D}}_{\text{SMPL}}$. 





\paragraph{Simulation-based Data Cleaning}

Since the reconstruction process can introduce noise and errors~\cite{he2024learning}, some estimated motions may not be physically feasible, making them unsuitable for motion tracking in the real world. To address this, we employ a ``sim-to-data'' cleaning procedure. Specifically, we use MaskedMimic~\cite{tessler2024maskedmimic}, a physics-based motion tracker, to imitate the SMPL motions from TRAM in IsaacGym simulator~\cite{makoviychuk2021isaac}. The motions (\Cref{fig:data_processing} (c)) that pass this simulation-based validation are saved as the cleaned dataset ${\mathcal{D}}_{\text{SMPL}}^{\text{Cleaned}}$.

\paragraph{Retargeting SMPL Motions to Robot Motions}

With the cleaned dataset ${\mathcal{D}}_{\text{SMPL}}^{\text{Cleaned}}$ in SMPL format, we retarget the motions into robot motions following the shape-and-motion two-stage retargeting process~\cite{he2024learning}. Since the SMPL parameters estimated by TRAM represent various human body shapes, we first optimize the shape parameter $\boldsymbol{\beta}^{\prime}$ to approximate a humanoid shape. By selecting 12 body links with correspondences between humans and humanoids, we perform gradient descent on $\boldsymbol{\beta}^{\prime}$ to minimize joint distances in the rest pose. Using the optimized shape $\boldsymbol{\beta}^{\prime}$ along with the original translation $\boldsymbol{p}$ and pose $\boldsymbol{\theta}$, we apply gradient descent to further minimize the distances of the body links. This process ensures accurate motion retargeting and produces the cleaned robot trajectory dataset ${\mathcal{D}}_{\text{Robot}}^{\text{Cleaned}}$, as shown in \Cref{fig:data_processing} (d). 
% \guanya{no dynamics or RL to get ? Do we need a figure to show a-b-c?}

\subsection{Phase-based Motion Tracking Policy Training}

We formulate the motion-tracking problem as a goal-conditioned reinforcement learning (RL) task, where the policy $\pi$ is trained to track the retargeted robot movement trajectories in the dataset ${\mathcal{D}}_{\text{Robot}}^{\text{Cleaned}}$. Inspired by ~\cite{peng2018deepmimic}, the state $s_t$ includes the robot’s proprioception $s_t^{\mathrm{p}}$ and a time phase variable $\phi \in [0,1]$, where $\phi=0$ represents the start of a motion and $\phi=1$ represents the end. This time phase variable alone $\phi$ is proven to be sufficient to serve as the goal state $\boldsymbol{s}_t^{\mathrm{g}}$ for single-motion tracking~\cite{peng2018deepmimic}. 
The proprioception $s_t^{\mathrm{p}}$ is defined as $s_t^{\mathrm{p}} \triangleq \left[\dofposhist, \dofvelhist,  \rootangvelhist, \gravityhist, \actionhist \right]$, with 5-step history of joint position $\dofpos\in\mathbb{R}^{23}$, joint velocity $\dofvel\in\mathbb{R}^{23}$, root angular velocity $\rootangvel\in\mathbb{R}^3$, root projected gravity $\gravity\in\mathbb{R}^3$, and last action $\actionprev\in\mathbb{R}^{23}$.
Using the agent’s proprioception $s_t^{\mathrm{p}}$ and the goal state $\boldsymbol{s}_t^{\mathrm{g}}$, we define the reward as  $r_t=\mathcal{R}\left(s_t^{\mathrm{p}}, s_t^{\mathrm{g}}\right)$, which is used for policy optimization.  The specific reward terms can be found in~\Cref{tab:deepmimic_reward}. The action $\boldsymbol{a}_t \in \mathbb{R}^{23}$ corresponds to the target joint positions and is passed to a PD controller that actuates the robot’s degrees of freedom. To optimize the policy, we use the proximal policy optimization (PPO)~\cite{schulman2017proximal}, aiming to maximize the cumulative discounted reward $\mathbb{E}\left[\sum_{t=1}^T \gamma^{t-1} r_t\right]$. We identify several design choices that are crucial for achieving stable policy training:

\paragraph{Asymmetric Actor-Critic Training}

Real-world humanoid control is inherently a partially observable Markov decision process (POMDP), where certain task-relevant properties that are readily available in simulation become unobservable in real-world scenarios. However, these missing properties can significantly facilitate policy training in simulation. To bridge this gap, we employ an asymmetric actor-critic framework, where the critic network has access to privileged information such as the global positions of the reference motion and the root linear velocity, while the actor network relies solely on proprioceptive inputs and a time-phase variable. This design not only enhances phase-based motion tracking during training but also enables a simple, phase-driven motion goal for sim-to-real transfer. Crucially, because the actor does not depend on position-based motion targets, our approach eliminates the need for odometry during real-world deployment—overcoming a well-documented challenge in prior work on humanoid robots~\cite{he2024learning,he2024omnih2o}.

% \TODO{\tairan{Check what else is known for critic}}
\paragraph{Termination Curriculum of Tracking Tolerance}
Training a policy to track agile motions in simulation is challenging, as certain motions can be too difficult for the policy to learn effectively. For instance, when imitating a jumping motion, the policy often fails early in training and learns to remain on the ground to avoid landing penalties. To mitigate this issue, we introduce a termination curriculum that progressively refines the motion error tolerance throughout training, guiding the policy toward improved tracking performance. Initially, we set a generous termination threshold of 1.5m, meaning the episode terminates if the robot deviates from the reference motion by this margin. As training progresses, we gradually tighten this threshold to 0.3m, incrementally increasing the tracking demand on the policy. This curriculum allows the policy to first develop basic balancing skills before progressively enforcing stricter motion tracking, ultimately enabling successful execution of high-dynamic behaviors.

% \guanya{a bit more details? termination condition curriculum}
% Specifically, we: \TODO{\tairan{Describe the details of the termination curriculum}}

\paragraph{Reference State Initialization}
Task initialization plays a crucial role in RL training. We find that naively initializing episodes at the start of the reference motion leads to policy failure. For example, in Cristiano Ronaldo's jumping training, starting the episode from the beginning forces the policy to learn sequentially. However, a successful backflip requires mastering the landing first—if the policy cannot land correctly, it will struggle to complete the full motion from takeoff. To address this, we adopt the Reference State Initialization (RSI) framework~\cite{peng2018deepmimic}. Specifically, we randomly sample time-phase variables between 0 and 1, which effectively randomizes the starting point of the reference motion for the policy to track. We then initialize the robot’s state based on the corresponding reference motion at that phase, including root position and orientation, root linear and angular velocities and joint positions and velocities. This initialization strategy significantly improves motion tracking training, particularly for agile whole-body motions, by allowing the policy to learn different motion phases in parallel rather than being constrained to a strictly sequential learning process.

\paragraph{Reward Terms}
We define the reward function $r_t$ with the sum of three terms: 1) penalty, 2) regularization, and 3) task rewards. A detailed summary of these components is provided in \Cref{tab:deepmimic_reward}.
\begin{table}[!h]
    \centering
    \small % Reduce font size
    \setlength{\tabcolsep}{2pt} % Adjust column spacing
    % \renewcommand{\arraystretch}{1.3} % Increase row spacing
    \vspace{-5mm}
    \caption{Reward Terms for Pretraining}
    \vspace{-2mm}
    \label{tab:deepmimic_reward}
    \resizebox{0.8\columnwidth}{!}{ % Ensure the table fits within the column width
    \begin{tabular}{cccc}
        \toprule
        Term & Weight & Term & Weight \\
        \midrule
        \multicolumn{4}{c}{Penalty} \\
        \midrule
        DoF position limits & $-10.0$ & DoF velocity limits & $-5.0$ \\
        Torque limits & $-5.0$ & Termination & $-200.0$ \\
        \midrule
        \multicolumn{4}{c}{Regularization} \\
        \midrule
        Torques & $-1 \times 10^{-6}$ & Action rate & $-0.5$ \\
        Feet orientation & $-2.0$ & Feet heading & $-0.1$ \\
        Slippage & $-1.0$ &  \\
        \midrule
        \multicolumn{4}{c}{Task Reward} \\
        \midrule
        Body position & $1.0$ & VR 3-point & $1.6$ \\
        Body position (feet) & $2.1$ & Body rotation & $0.5$ \\
        Body angular velocity & $0.5$ & Body velocity & $0.5$ \\
        DoF position & $0.75$ & DoF velocity & $0.5$ \\
        \bottomrule
    \end{tabular}
    \vspace{-7mm}
    } % End of resizebox
\end{table}

\paragraph{Domain Randomizations}
To improve the robustness of the pre-trained policy in \Cref{fig:ASAP} (a), we utilized basic domain randomization techniques listed in \Cref{tab:deepmimic_DR}.
% \begin{table}[h]
    \centering
    \setlength{\tabcolsep}{3pt} % Reduce column spacing
    \caption{Domain Randomizations}
    \begin{tabular}{cc}
        \toprule
        Term & Value \\
        \midrule
        \multicolumn{2}{c}{Dynamics Randomization} \\
        \midrule
        Friction & $\mathcal{U}(0.2,1.1)$ \\
        P Gain & $\mathcal{U}(0.925,1.05) \times$ default \\
        Control delay & $\mathcal{U}(20,40) \mathrm{ms}$ \\
        \midrule
        \multicolumn{2}{c}{External Perturbation} \\
        \midrule
        Push robot & interval $= 10 s, v_{x y} = 0.5 \mathrm{~m} / \mathrm{s}$ \\
        \bottomrule
    \end{tabular}
    \label{tab:deepmimic_DR}
\end{table}
\section{Post-training: Training Delta Action Model and Fine-tuning Motion Tracking Policy}
% \zi{The subtitle is a little bit strange. Should be Post-training \textbf{AND} Fine-tuning? Or just: ASAP: Training Delta Dynamics and Fine-tuning Motion Tracking Policy}
The policy trained in the first stage can track the reference motion in the real-world but does not achieve high motion quality. Thus, during the second stage, as shown in ~\Cref{fig:ASAP}~(b) and (c), we leverage real-world data rolled out by the pre-trained policy to train a delta action model, followed by policy refinement through dynamics compensation using this learned delta action model.

\subsection{Data Collection}
We deploy the pretrained policy in the real world to perform whole-body motion tracking tasks (as depicted in~\Cref{fig:data-collect}) and record the resulting trajectories, denoted as $\mathcal{D}^\text{r} = \{s^\text{r}_0, a^\text{r}_0, \dots, s^\text{r}_T, a^\text{r}_T\}$, as illustrated in~\Cref{fig:ASAP}~(a). At each timestep $t$, we use a motion capture device and onboard sensors to record the state: 
$
s_t = [p^\text{base}_t, v_t^\text{base}, \alpha^\text{base}_t, \omega^\text{base}_t, q_t, \dot{q}_t],
$
where $p^\text{base}_t \in \mathbb{R}^3$ represents the robot base 3D position, $v_t^\text{base} \in \mathbb{R}^3$ is base linear velocity, $\alpha^\text{base}_t \in \mathbb{R}^4$ is the robot base orientation represented as a quaternion, $\omega^\text{base}_t \in \mathbb{R}^3$ is the base angular velocity, $q_t \in \mathbb{R}^{23}$ is the vector of joint positions, and $\dot{q}_t \in \mathbb{R}^{23}$ represents joint velocities.


\begin{figure*}[t]
    \centering
    \includegraphics[width=0.9\textwidth]{fig/baselines-crop.pdf}
    \vspace{-1mm}
    \caption{Baselines of \method. (a) Model-free RL training. (b) System ID from real to sim using real-world data. (c) Learning delta dynamics model using real-world data. (d) Our proposed method, learning delta action model using real-world data. }
    \label{fig:baselines}
    \vspace{-4mm}
\end{figure*}

\subsection{Training Delta Action Model}
\label{sec:train-delta-action-model}

Due to the sim-to-real gap, when we replay the real-world trajectories in simulation, the resulting simulated trajectory will likely deviate significantly from real-world recorded trajectories. This discrepancy is a valuable learning signal for learning the mismatch between simulation and real-world physics. We leverage an RL-based delta/residual action model to compensate for the sim-to-real physics gap.

As illustrated in~\Cref{fig:ASAP} (b), the delta action model is defined as $\Delta a_t = \pi^\Delta_\theta(s_t, a_t)$, where the policy $\pi^\Delta_\theta$ learns to output corrective actions based on the current state $s_t$ and the action $a_t$. These corrective actions ($\Delta a_t$) are added to the real-world recorded actions ($a^r_t$) to account for discrepancies between simulation and real-world dynamics.

The RL environment incorporates this delta action model by modifying the simulator dynamics as follows: $ s_{t+1} = f^\text{sim}(s_t, a^r_t + \Delta a_t)$ where $f^\text{sim}$ represents the simulator's dynamics, $a^r_t$ is the reference action recorded from real-world rollouts, and $\Delta a_t$ introduces corrections learned by the delta action model. 




\begin{table}[htp]
    \centering
    \vspace{-2mm}
    \small % Reduce font size
    \setlength{\tabcolsep}{2pt} % Adjust column spacing
    % \renewcommand{\arraystretch}{1.3} % Increase row spacing
    \caption{Reward Terms for Delta Action Learning}
    \vspace{-2mm}
    \label{tab:deltaA_FT_reward}
    \resizebox{0.8\columnwidth}{!}{ % Ensure the table fits within the column width
    \begin{tabular}{cccc}
        \toprule
        Term & Weight & Term & Weight \\
        \midrule
        \multicolumn{4}{c}{Penalty} \\
        \midrule
        DoF position limits & $-10.0$ & DoF velocity limits & $-5.0$ \\
        Torque limits & $-0.1$ & Termination & $-200.0$ \\
        \midrule
        \multicolumn{4}{c}{Regularization} \\
        \midrule
        Action rate & $-0.01$ & Action norm & $-0.2$ \\
        \midrule
        \multicolumn{4}{c}{Task Reward} \\
        \midrule
        Body position & $1.0$ & VR 3-point & $1.0$ \\
        Body position (feet) & $1.0$ & Body rotation & $0.5$ \\
        Body angular velocity & $0.5$ & Body velocity & $0.5$ \\
        DoF position & $0.5$ & DoF velocity & $0.5$ \\
        \bottomrule
    \end{tabular}
    } % End of resizebox
    \vspace{-2mm}
\end{table}

During each RL step: 
\begin{enumerate}
    \item The robot is initialized at the real-world state $s^r_t$.
    \item  A reward signal is computed to minimize the discrepancy between the simulated state $s_{t+1}$ and the recorded real-world state $s^r_{t+1}$, with an additional action magnitude regularization term $\exp(-\lVert a_t \rVert) - 1)$, as specified in~\Cref{tab:deltaA_FT_reward}. The workflow is illustrated in \Cref{fig:ASAP}~(b).
    \item PPO is used to train the delta action policy $\pi^\Delta_\theta$, learning corrected $\Delta a_t$ to match simulation and the real world.
\end{enumerate}



By learning the delta action model, the simulator can accurately reproduce real-world failures. For example, consider a scenario where the simulated robot can jump because its motor strength is overestimated, but the real-world robot cannot jump due to weaker motors. The delta action model $\pi^\Delta_\theta$ will learn to reduce the intensity of lower-body actions, simulating the motor limitations of the real-world robot. This allows the simulator to replicate the real-world dynamics and enables the policy to be fine-tuned to handle these limitations effectively.

\subsection{Fine-tuning Motion Tracking Policy under New Dynamics}
With the learned delta action model $\pi^\Delta (s_t, a_t)$, we can reconstruct the simulation environment with 
$$
s_{t+1} = f^{\text{\method}}(s_t, a_t) = f^\text{sim}(s_t, a_t + \pi^\Delta(s_t, a_t)),
$$
As shown in~\Cref{fig:ASAP} (c), we keep the $\pi^\Delta$ model parameters frozen, and fine-tune the pretrained policy with the same reward summarized in~\Cref{tab:deepmimic_reward}. 






\subsection{Policy Deployment}
Finally, we deploy the fine-tuned policy without delta action model in the real world as shown in \Cref{fig:ASAP}~(d). The fine-tuned policy shows enhanced real-world motion tracking performance compared to the pre-trained policy. Quantitative improvements will be discussed in \Cref{sec:EXP1}.
\section{Experiments: Planning outperforms Heuristics}
\label{sec:experiment}

We begin our empirical demonstrations by showcasing the effectiveness of our planning framework on both synthetic and real datasets. We focus on the simplest planning algorithm, 1-step lookaheads (Algorithm~\ref{alg:complete}), and show that even basic planning can hold great promise. 
We illustrate our framework using two uncertainty quantification modules---GPs and 
\ensembles/ \ensembleplus. 

Throughout this section, we focus on evaluating the mean squared error of 
a regression model $\model$,  and develop adaptive policies that minimize uncertainty on $g(f)$ defined in~\eqref{eqn:l2-g-f}.
When GPs provide a valid model of uncertainty, 
our experiments show that our planning framework significantly outperforms other baselines. 
We further demonstrate that our conceptual framework extends to deep learning-based uncertainty quantification methods such as  \ensembleplus while highlighting computational challenges that need to be resolved in order to scale our ideas. 
For simplicity, we assume a naive predictor, i.e., $\psi(\cdot) \equiv 0$. However, we emphasize that this problem is just as complex as if we were using a sophisticated model $\psi(.)$. The performance gap between the algorithms 
primarily depends
on the level  of uncertainty in our prior beliefs.

To evaluate the performance of our algorithm, we benchmark it against several baselines. 
%Active learning baselines use an acquisition function $\ac$ to select points that have the highest   function value: $X\opt_t \in \argmax_{X \in \xpoolj{t}} \ac({X})$ at every step $t$. These methods may also need an UQ module, which we simply use the same UQ module as in our algorithm, and it  outputs $V(X)$ that measures the the uncertainty of each point $X \in \xpoolj{t}$.
Our first set of baselines are from active learning~\citep{AggarwalKoGuHaPh14}:
\\ % \noindent\textbf{Active Learning Heuristics:} 
\textbf{(1)} 
\textsf{Uncertainty Sampling (Static):}  In this approach, we query the samples for which the model is least certain about. Specifically, we estimate the variance of the latent output $f(X)$ for each $X \in \xpool$ using the UQ module and select the top-$K$ points with the highest uncertainty. \\
\textbf{(2)} \textsf{Uncertainty Sampling (Sequential):} This is a greedy heuristic that sequentially selects the points with the highest uncertainty within a batch, while updating the posterior beliefs using pseudo labels from the current posterior state. Unlike \textsf{Uncertainty Sampling (Static)}, this method takes into account the information gained from each point within batch, and hence tries to diversify the selected points within a batch. 

 
We also compare our approach to the  \textbf{(3)} \textsf{Random Sampling}, which selects each batch uniformly at random from the pool. Additionally, we compare solving the planning problem using  \textsf{REINFORCE}-based policy gradients with   $\mathsf{Smoothed\text{-}Autodiff}$ policy gradients.\footnote{Our code repository is available at
  \url{https://github.com/namkoong-lab/adaptive-labeling}.}
%Detailed experimental setups are provided in Section \ref{sec:details-experiments}.

%We repeat all experiments with 10 random seeds.




\begin{figure}[t]
\centering
\begin{minipage}[b]{0.49\textwidth}
\centering
\includegraphics[width=\textwidth, height=5cm]{figures/original_scale/Var_of_l_2_loss.pdf}
\caption{(Synthetic data) Variance of mean squared loss evaluated through the posterior belief $\mu_t$ at each horizon $t$. This is the objective that policy gradient methods like \textsf{REINFORCE} and $\ouralgo$ optimizes. 1-step lookaheads are surprisingly effective even in long horizons.}
\label{fig:var-l2-sim}
\end{minipage}
\hfill
\begin{minipage}[b]{0.49\textwidth}
\centering \includegraphics[width=\textwidth, height=5cm]{figures/original_scale/Error_of_estimated_model_l_2_loss.pdf}
\caption{(Synthetic data) Error between MSE calculated based on collected data $\mc{D}^{0:T}$ vs. population oracle MSE over $\mc{D}_{\rm eval} \sim P_X$. Reducing uncertainty over posteriors directly leads to better OOD evaluations. 1-step lookaheads significantly outperform active learning heuristics in small horizons.}
\label{fig:mean-l2-sim}
\end{minipage}
%\caption{Simulated data for GPs}
%\label{fig:both_plots}
\end{figure}

\subsection{Planning with Gaussian processes}
\label{sec:experiment-plan-GP}
We now briefly describe the data generation process for the GP experiments,  deferring a more detailed discussion of the dataset generation to Section~\ref{sec:details-experiments}. 
We use both the synthetic data and the real data to test our methodology.
For the \emph{simulated data},  we construct a setting where the general population is distributed across \emph{51 non-overlapping clusters} while the initial labeled data $\dtrain$ just comes from one cluster. In contrast, both $\dpool \defeq (\xpool,\ypool),\deval \defeq (\xeval,\yeval)$ are generated   from all the clusters. 
We begin with a low-dimensional scenario, generating a one-dimensional regression setting using a GP. %Gaussian Process (GP).
Although the data-generating process is not known to the algorithms,  we assume that the GP hyperparameters are known to all the algorithms
to ensure fair comparisons. This can be viewed as a setting where our prior is well-specified, allowing us to isolate the effects
of different policy optimization approaches
 without any concerns about the misspecified priors. We select $10$ batches, each of size $K=5$ across $T = 10$ time horizons.

To examine the robustness of our method against the distributional assumptions made  in the simulated case, we then move to a real dataset where the correct prior is not known. We simulate selection bias from the eICU dataset~\citep{PollardJoRaCeMaBa18}, which contains real-world patient data with in-hospital mortality outcomes. 
We conduct a $k$-means clustering to generate 51 clusters and then select data from those clusters. We view this to be a credible replication of practice, as severe distribution shifts are common due to selection bias in clinical labels.  To convert the binary mortality labels into a regression setting, we train a  random forest classifier and fit a GP on predicted scores, which serves as the UQ module for all the algorithms. As before, the task is to select 10 batches, each consisting of 5 samples, across 10 time horizons.

 In Figures~\ref{fig:var-l2-sim} and~\ref{fig:mean-l2-sim}, we present results for the simulated data. 
Figure~\ref{fig:var-l2-sim} shows the variance of $\ell_2$ loss, and Figure~\ref{fig:mean-l2-sim} presents the error in the estimated $\ell_2$ loss using $\mu_t$ (relative to true $\ell_2$ loss, that is unknown to the algorithm). 
As we can see from these plots, our method one-step lookahead  gives substantial improvements  over active learning baselines and random sampling. In addition,
compared to the one-step lookahead planning approach using \textsf{REINFORCE}-based policy gradients, 
we observe that $\mathsf{Smoothed\text{-}Autodiff}$-based policy gradients provide significantly more robust performance over all horizons.

In Figures~\ref{fig:var-l2-real}~and~\ref{fig:mean-l2-real}, we observe similar findings on the eICU data. We see that planning policies (\textsf{REINFORCE} and $\mathsf{Smoothed\text{-}Autodiff}$) consistently outperform other heuristics by a large margin.  Active learning baselines perform poorly in these small-horizon batched problems and can sometimes be even worse than the random search baselines.  Overall, our results show the importance of careful planning in adaptive labeling for reliable model evaluation. 

We offer some intuition as to why one-step lookahead planning may outperform other heuristic algorithms. 
 First,  \textsf{Uncertainty sampling (Static)} while myopically selects the
 top-$K$ inputs with the highest uncertainty, it fails to consider 
the overlap in information content among the ``best” instances; see \citep{AggarwalKoGuHaPh14} for more details. 
In other words,  it might acquire points from the same region with high uncertainty while failing to induce diversity among the batch.
Although \textsf{Uncertainty Sampling (Sequential)} somewhat addresses the issue of information overlap, a significant drawback of 
this algorithm
is the disconnect between the objective we aim to optimize and the algorithm. For example, it might sample from a region with high uncertainty but very low density. 

\begin{figure}[t]
\centering
\begin{minipage}[b]{0.48\textwidth}
\centering
\includegraphics[width=\textwidth, height=5cm]{figures/original_scale/Var_of_l_2_loss_real.pdf}
\caption{(Real-world eICU data) Variance of mean squared loss evaluated through the posterior belief $\mu_t$ at each horizon $t$. Even 1-step lookaheads are extremely effective planners, and auto-differentiation-based pathwise policy gradients provide a reliable optimization algorithm based on low-variance gradient estimates.}
\label{fig:var-l2-real}
\end{minipage}
\hfill
\begin{minipage}[b]{0.48\textwidth}
\centering \includegraphics[width=\textwidth, height=5cm]{figures/original_scale/Error_of_estimated_model_l_2_loss_real.pdf}
\caption{(Real-world eICU data) Error between MSE calculated based on collected data $\mc{D}^{0:T}$ vs. population oracle MSE over $\mc{D}_{\rm eval} \sim P_X$. Reducing uncertainty over posteriors directly leads to better OOD evaluations. Our method significantly outperforms active learning-based heuristics, and random sampling.}
\label{fig:mean-l2-real}
\end{minipage}
%\caption{Real data for GPs}
\end{figure}
 
%\vspace{-1.5cm}
% \begin{wrapfigure}{r}{.32\columnwidth}
%   \vspace{-.5cm} 
%   \centering
% \includegraphics[scale=.29]{figures/Var of l2l_2 loss.pdf}
%   \vspace{-0.2cm}
%   \caption{Results of GP}
% \label{fig:var-l2-gp}
%   \vspace{-0.1cm}
% \end{wrapfigure}


% Attempts have been made  in the past to address these  drawbacks heuristically  (see \citep{AggarwalKoGuHaPh14}). We give a unified computational framework while approaching the problem in a more principled manner and solving it more optimally.




\subsection{Planning with  neural network-based uncertainty quantification methods ($\ensembleplus$)}


We now provide a proof-of-concept that shows the generalizability of our conceptual framework  to the deep learning-based UQ modules, specifically focusing on $\ensembleplus$ due to their previously observed superior performance~\citep{OsbandWenAsDwIbLuRo23}. Recall that implementing our framework with deep learning-based UQ modules  requires us to retrain the model across multiple possible random actions $\bm{a}(\theta)$ sampled from the current policy $\pi_\theta$.
This requires significant computational resources, in sharp contrast to the GPs where the posteriors are in closed form and can be readily updated and differentiated. 

Due to the computational constraints, we test $\ensembleplus$ on a toy setting to demonstrate the generalizability of our framework. We consider a setting where the general population consists of four clusters, while the initial labeled data only comes from one cluster. Again we generate data using GPs.  The task is to select a batch of 2 points in one horizon. We detail the $\ensembleplus$ architecture in Section \ref{sec:details-experiments}, and we assume prior uncertainty to be large (depends on the scaling of the prior generating functions). 
The results are summarized in the Table~\ref{tab:UQ_ensemble}.

% \begin{table}[H]
% \vspace{-10pt}
% \caption{Performance under \ensembleplus as UQ module}
%     \centering
%     \begin{tabular}{|m{3cm}|m{2.5cm}|m{2cm}|} 
%     \hline
%       Algorithm   & Variance of $\loss_2$ loss estimate & Error of $\loss_2$ loss estimate  \\ \hline Random Sampling 
%          & $1710.9 \pm 1352.1$ & $8.67\pm6.62$ 
%       \\ \hline \ouralgo & $1.30 \pm 0.68$ & $0.91\pm0.25$ \\ \hline
%     \end{tabular}
%     \label{tab:UQ_ensemble}
%     %\vspace{-10pt}
% \end{table}




\begin{table}[h]
\vspace{-10pt}
\caption{Performance under \ensembleplus as the UQ module}
\centering
\begin{tabular}{|l|l|l|}
\hline
Algorithm   & Variance of $\loss_2$ loss estimate & Error of $\loss_2$ loss estimate  \\
\hline
\textsf{Random sampling} & 7129.8 $\pm$ 1027.0 & 136.2 $\pm$ 8.28 \\ \hline
\textsf{Uncertainty sampling (Static)} & 10852 $\pm$ 0.0 & 162.156 $\pm$ 0.0 \\ \hline
\textsf{Uncertainty sampling (Sequential)} & 8585.5 $\pm$ 898.9 & 144 $\pm$ 6.93 \\ \hline
\textsf{REINFORCE} & 1697.1 $\pm$ 0.0 & 45.27 $\pm$ 0.0 \\ \hline
\ouralgo & 1697.1 $\pm$ 0.0 & 45.27 $\pm$ 0.0 \\ \hline
\end{tabular}
%\caption{Comparison of different algorithms based on variance   and   error in $\ell_2$ loss estimation with Ensemble $+$ as the UQ module. Our results demonstrate that {\ouralgo} and REINFORCE outperformthe other active learning based heuristics, confirming the benefits of our MDP formulation for the adaptive labeling problem, as also demonstrated in Section 4.\\
%\footnotesize{Experimental details: We use Gaussian Processes as our data generating process, GP parameters are the same as in Section D.3.  The task is to select a batch of 2 points along one horizon.The marginal distribution $p_X$ has 4 \textit{non-overlapping} clusters. Initial data comes from one cluster, while pool and evaluation points comes from all the clusters. We have $20$ initial labeled data points, $10$ pool points, and $252$ evaluation points.  Training procedures are similar to the one in Section D.3.} }
\label{tab:UQ_ensemble}
\end{table}



% We faced  issues in scaling up these experiments which will be our focus in the future. 





% \begin{itemize}
%     \item Posteriors should be consistent. Two dimensions: even with less training,  
%     \item the inference should be  fast enough
% \end{itemize}


% Potential research directions for uncertainty quantification

% In this section we consider a simple setting We consider a simpler setting and 


% For synthetic dataset generation, we use ...... For real datasets, we use ...... We compare our methodolgy to several baselines ()    This Section is structured as follows:
% \begin{itemize}
%     \item \textbf{GPs, square loss objective} (Section \ref{}): 
%     %the broad aim of the experiments  in this section is to isolate the performance of our methodology without any concerns for the inefficiencies induced due to a mis-specified prior or imperfect posterior inference. To accomplish this we generate synthetic datasets using GPs (detailed later). We use the well specified prior (GPs - with same hyperparameter setting) as our UQ module.   
%      As GPs provide differentaible posterior inference - any errors induced due to imperfect posterior updates are also isolated. We note that under this setting
%      \item In Section\ref{} we demonstrate why our methodology performs better than other baselines - by devising various synthetic experiments ()
%     \item  \textbf{UQ Benchmarking }(Section \ref{}): Before diving into the experiments using $\ensembleplus$ and ENNs,  we showcase our benchmarking experiments in Section \ref{}. We use real datasets We observe that ENNs perform better
%      \item \textbf{Ensemble $+$}, objective: recall, accuracy
%     \item \textbf{ENN}, objective: recall, accuracy
% \end{itemize}




% In Section {}, we test 
% \subsection{Experimental details}

% \begin{itemize}
%     \item UQ methodologies - GPs, ENNs
%     \item Objectives - Recall,  ATE
%     \item Datasets - ATE-synthetic datasets, Recall-synthetic, real datasets
%     \item Baselines - 
%     \begin{itemize}
%         \item Random sampling
%         \item Active learning - Uncertainty based sampling - In regression setting almost all of the 
%         \item Myopic greedy - Greedy Batch based sampling
%         \item Policy Gradient
%     \end{itemize}
    
% \end{itemize}

% \subsection{Experiments}
%     \begin{itemize}
%     \item GPs with square loss
%     \item Benchmarking ENN
%         \item ENNs with ATE
%         \item ENNs with Recall
%     \end{itemize}

% \subsection{Benefits over other algorithms - intuition and experiments}

%Active learning - Myopic greedy / Don't rely on the objective rather some entropy version.


%%% Local Variables:
%%% mode: latex
%%% TeX-master: "main"
%%% End:

\section{EXTENSIVE STUDIES AND ANALYSES}

In this section, we aim to thoroughly analyze \method by addressing three central research questions:
\begin{itemize}
    \item \textbf{Q4}: How to best train the delta action model of \method?
    \item \textbf{Q5}: How to best use the delta action model of \method?
    \item \textbf{Q6}: Why and how does \method work?
\end{itemize}

\begin{figure*}[t]
    \centering
    \includegraphics[width=1.0\linewidth]{fig/exp_ablate_deltaA.pdf}
     \vspace{-4mm}
    \caption{Analysis of dataset size, training horizon, and action norm on the performance of $\pi^\Delta$. (a) \textbf{Dataset Size}: Mean Per Joint Position Error (MPJPE) is evaluated for both in-distribution (green) and out-of-distribution (blue) scenarios. Increasing dataset size leads to enhanced generalization, evidenced by decreasing errors in out-of-distribution evaluations. Closed-loop MPJPE (red bars) also shows improvement with larger datasets. (b) \textbf{Training Horizon}: Open-loop MPJPE (heatmap) improves across evaluation points as training horizons increase, achieving the lowest error at 1.5s. However, closed-loop MPJPE (red bars) shows a sweet spot at a training horizon of 1.0s, beyond which no further improvements are observed. The red dashed line represents the pretrained baseline without $\pi^\Delta$ fine-tuning. (c) \textbf{Action Norm}: The action norm weight significantly influences performance. Both open-loop and closed-loop MPJPE decrease as the weight increases up to 0.1, achieving the lowest error. However, further increases in the action norm weight result in degradation of open-loop performance, highlighting the trade-off between action smoothness and policy flexibility.}
    \label{fig:deltaA_ablation}
\end{figure*}

\subsection{Key Factors in Training Delta Action Models}
\label{sec:VA}
To Answer \textbf{Q4} (\textit{How to best train the delta action model of \method}). 
we conduct a systematic study on key factors influencing the performance of the delta action model. 
Specifically, we investigate the impact of dataset size, training horizon, and action norm weight, evaluating their effects on both open-loop and closed-loop performance. Our analysis uncovers the essential principles for effectively training a high-performing delta action model.

\paragraph{Dataset Size} We analyze the impact of dataset size on the training and generalization of $\pi^\Delta$. Simulation data is collected in Isaac Sim, and $\pi^\Delta$ is trained in Isaac Gym. Open-loop performance is assessed on both in-distribution (training) and out-of-distribution (unseen) trajectories, while closed-loop performance is evaluated using the fine-tuned policy in Isaac Sim. As shown in~\Cref{fig:deltaA_ablation}~(a), increasing the dataset size improves $\pi^\Delta$’s generalization, evidenced by reduced errors in out-of-distribution evaluations. However, the improvement in closed-loop performance saturates, with a marginal decrease of only $0.65\%$ when scaling from $4300$ to $43000$ samples, suggesting limited additional benefit from larger datasets.

\paragraph{Training Horizon} The rollout horizon plays a crucial role in learning $\pi^\Delta$. As shown in~\Cref{fig:deltaA_ablation}~(b), longer training horizons generally improve open-loop performance, with a horizon of 1.5s achieving the lowest errors across evaluation points at 0.25s, 0.5s, and 1.0s. However, this trend does not consistently extend to closed-loop performance. The best closed-loop results are observed at a training horizon of 1.0s, indicating that excessively long horizons do not provide additional benefits for fine-tuned policy.

\paragraph{Action Norm Weight} Training $\pi^\Delta$ incorporates an action norm reward to balance dynamics alignment and minimal correction. As illustrated in~\Cref{fig:deltaA_ablation}~(c), both open-loop and closed-loop errors decrease as the action norm weight increases, reaching the lowest error at a weight of $0.1$. However, further increasing the action norm weight causes open-loop errors to rise, likely due to the minimal action norm reward dominates in the delta action RL training. This highlights the importance of carefully tuning the action norm weight to achieve optimal performance.



\subsection{Different Usage of Delta Action Model}
To answer \textbf{Q5} \textit(How to best use the delta action model of \method?), we compare multiple strategies: fixed-point iteration, gradient-based optimization, and reinforcement learning (RL). Given a learned delta policy \(\pi^\Delta\) such that:
\[
f^\text{sim}(s, a + \pi^\Delta(s, a)) \approx f^\text{real}(s, a),
\]
and a nominal policy \(\hat{\pi}(s)\) that performs well in simulation, the goal is to fine-tune \(\hat{\pi}(s)\) for real-world deployment.

A simple approach is one-step dynamics matching, which leads to the relationship:
\[
\pi(s) = \hat{\pi}(s) - \pi^\Delta(s, \pi(s)).
\]
We consider two RL-free methods: fixed-point iteration and gradient-based optimization. Fixed-point iteration refines \(\hat\pi(s)\) iteratively, while gradient-based optimization minimizes a loss function to achieve a better estimate. These methods are compared against RL fine-tuning, which adapts \(\hat\pi(s)\) using reinforcement learning in simulation. The detailed derivation of these two baselines is summarized in \Cref{sec:appendix_more_deltaA_usage}.

Our experiments in \Cref{fig: use_deltaA} show that RL fine-tuning achieves the lowest tracking error during deployment, outperforming training-free methods. 
Both RL-free approaches are myopic and suffer from out-of-distribution issues, limiting their real-world applicability (more discussions in \Cref{sec:appendix_more_deltaA_usage}). 




\begin{figure}[htp]
    \centering
    \includegraphics[width=0.85\linewidth]{fig/exp_use_deltaA.pdf}
    \vspace{-2mm}
    \caption{MPJPE comparison over timesteps for fine-tuning methods using delta actionmodel. RL Fine-Tuning achieves the lowest error, while Fixed-Point Iteration and Gradient Search perform worse than the baseline (Before DeltaA) showing the highest error.}
    \label{fig: use_deltaA}
    \vspace{-4mm}
\end{figure}


\begin{figure}[t]
    \centering
    \includegraphics[width=0.9\linewidth]{fig/exp_ablate_noise.pdf}
    \vspace{-2mm}
    \caption{MPJPE vs. Noise Level for policies fine-tuned with random action noise. Policies with noise levels $\beta \in [0.025, 0.2]$ show improved performance compared to no fine-tuning. Delta action achieves better tracking precision (126 MPJPE) compared to the best action noise (173 MPJPE).}
    \label{fig:action_noise}
    \vspace{-4mm}
\end{figure}




\subsection{Does \method Fine-Tuning Outperform Random Action Noise Fine-Tuning?}
To answer \textbf{Q6} (\textit{How does \method work?}), we validate \method finetuning is better than injecting random-action-noise-based finetuning. And we visualize the average magnitude of the delta action model for each joint.

Random torque noise~\cite{rfi} is a widely used domain randomization technique for legged robots. To determine whether delta action facilitates fine-tuning of pre-trained policies toward real-world dynamics rather than merely enhancing robustness through random action noise, we analyze its impact. Specifically, we assess the effect of applying random action noise during policy fine-tuning in Isaac Gym by modifying the environment dynamics as $s_{t+1} = f^\text{sim}(s_t, a_t + \beta \delta_a)$, where $\delta_a \sim \mathcal{U}[0, 1]$, and deploy it in Genesis. We conduct an ablation study to examine the influence of the noise magnitude, $\beta$, varying from $0.025$ to $0.4$. As shown in~\Cref{fig:action_noise}, within the constrained range of $\beta \in [0.025, 0.2]$, policies fine-tuned with action noise outperform those without fine-tuning in terms of global tracking error (MPJPE). However, the performance of the action noise approach (MPJPE of $150$) does not match the precision achieved by \method (MPJPE of $126$). Furthermore, we visualize the average output of $\pi^\Delta$ learned from IsaacSim data in~\Cref{fig:vis_deltaA_magnitude}, which reveals non-uniform discrepancies across joints. For example, in the G1 humanoid robot under our experimental setup, lower-body motors exhibit a larger dynamics gap compared to upper-body joints. Within the lower-body, the ankle and knee joints show the most pronounced discrepancies. Additionally, asymmetries between the left and right body motors further highlight the complexity. Such structured discrepancies cannot be effectively captured by merely adding uniform action noise.
These findings, along with the results in~\Cref{fig:ASAP_openloop_curves}, demonstrate that delta action not only enhances policy robustness but also enables effective adaptation to real-world dynamics, outperforming naive randomization strategies.

\begin{figure}[t]
    \centering
    \includegraphics[width=1.0\linewidth]{fig/vis_magnitude.pdf}
    \vspace{-2mm}
    \caption{Visualization of IsaacGym-to-IsaacSim $\pi^\Delta$ output magnitude. We compute the average absolute value of each joint over the 4300-episode dataset. Larger red dots indicate higher values. The results suggest that lower-body motors exhibit a larger discrepancy compared to upper-body joints, with the most significant gap observed in the ankle pitch joint of the G1 humanoid.}
    \label{fig:vis_deltaA_magnitude}
    \vspace{-4mm}
\end{figure}




\section{Related Work} \label{sec:related}

% \textbf{Adversarial Attack}
\textbf{Attacks on SLAM.} 
%With the rise of machine learning, 
The robustness of computer vision systems is being actively investigated. With the emergence of adversarial images in the digital domain by adding optimized noise directly to images~\cite{szegedy2013intriguing,carlini2017towards}, researchers find that such attacks also exist physically in the real world \cite{eykholt2018robust,song2018physical,zhao2019seeing}. To fill the gap between attacks in the digital and physical worlds, recent studies have demonstrated that attacks on real-world computer vision systems are practical \cite{eykholt2018robust,li2019adversarial,man2020ghostimage,sharif2016accessorize,zhao2019seeing,zhou2018invisible}. However, attacks on traditional computer vision methods such as SLAM are relatively less explored. \cite{yoshida2022adversarial} proposes an attack against the scan matching algorithm in LiDAR-based SLAM, while most SLAMs in AR/VR devices rely on different sensors like RGB/depth cameras and IMUs. \cite{ikram2022perceptual} and \cite{chen2024adversary} mislead visual SLAM by poisoning the images with special patterns, and \cite{wang2021can} causes the camera to fail using infrared light. In our work, we demonstrate attacks on Visual-Inertial SLAM (VI-SLAM) by perturbing the IMU readings, rather than cameras, and showing its impact on XR user experience. 

\textbf{Acoustic Injection Attacks.} Among various physical attacks, acoustic injection attacks are attractive due to their low cost. Son~\etal~\cite{son2015rocking} were the first to introduce acoustic attacks on MEMS gyroscopes, demonstrating how these attacks could lead to sensor denial-of-service and result in drone crashes. WALNUT~\cite{trippel2017walnut} expanded on this by developing output biasing and control attacks that enable precise manipulation of MEMS accelerometer outputs using modulated sound waves. Wang et al.~\cite{wang2017sonic} demonstrated a sonic gun, showcasing the vulnerability of various smart devices (\eg drones and self-balancing vehicles) to acoustic attacks. Tu et al. \cite{tu2018injected} designed side-swing and switching attacks to alter the outputs of MEMS gyroscopes and accelerometers. Furthermore, Ji et al. \cite{ji2021poltergeist} fool the object detectors by applying acoustic attack to the image stabilizers commonly used in modern cameras. However, none of the existing works study the relationship between the acoustic injections and SLAM outputs on recent XR devices. 

% \zijian{Do we need one session about security in AR/VR?}
% \yicheng{TODO}
%\jiasi{cite the AIVR paper (UMass Amherst?) paper is we have not already. They add IMU perturbation but w/o SLAM, iirc} \yicheng{Cited}

\textbf{XR Security and Privacy.} 
%Security and privacy concerns in XR systems have gained significant attention. 
For single-user XR systems, researchers have demonstrated various side-channel attacks to extract sensitive information (\eg keystrokes) through video feeds~\cite{ling2019know}, head movements~\cite{nair2023unique, slocum2023going}, architectural hints~\cite{zhang2023its,shang2020arspy}, power usage~\cite{li2024dangers}, and EM side-channel leakages~\cite{al2021vr}. In multi-user XR systems, Su et al.~\cite{su2024remote} use avatar motion data to infer keystrokes in shared VR environments. Slocum et al.~\cite{slocum2024doesn} reveal vulnerabilities in the shared state frameworks of multi-user AR. Similarly, Lebeck et al.~\cite{lebeck2017securing} highlight risks like deceptive virtual objects and emphasize access control for managing shared physical and virtual spaces. Ruth et al.~\cite{ruth2019secure} further propose a secure multi-user AR framework focusing on content sharing and permissions.
Chandio et al.~\cite{chandio2024stealthy} %introduced a multi-modal spatiotemporal attack that 
simultaneously manipulated visual and inertial sensors to disrupt XR pose estimation. However, their study evaluated the attack using offline datasets and assumed the attacker's capability to manipulate IMU data streams through acoustic means, without real experiments. Ours is the first to demonstrate acoustic injection attacks on recent XR devices, like the Hololens 2, in the real world.
 



\section*{Conclusion}
This paper aims to enhance our understanding of the computational complexity of computing various Shapley value variants. We found that for various ML models --- including decision trees, regression tree ensembles, weighted automata, and linear regression --- both local and global interventional and baseline SHAP can be computed in polynomial time under HMM modeled distributions. This extends popular algorithms, such as TreeSHAP, beyond their empirical distributional scope. We also establish strict complexity gaps between the various SHAP variants (baseline, interventional, and conditional) and prove the intractability of computing SHAP for tree ensembles and neural networks in simplified scenarios. Overall, we present SHAP as a versatile framework whose complexity depends on four key factors: \begin{inparaenum}[(i)] \item model type, \item SHAP variant, \item distribution modeling approach, \item and local vs. global explanations\end{inparaenum}. We believe this perspective provides deeper insight into the computational complexity of SHAP, paving the way for future work.




%We believe that our framework provides a more intricate understanding of SHAP computation complexity across different models, distributions, and variants, paving the way for further research.

Our work opens promising directions for future research. First, expanding our computational analysis to other SHAP-related metrics, such as asymmetric SHAP~\citep{frye20} and SAGE~\citep{covert2020understanding}, would be valuable. Additionally, we aim to explore more expressive distribution classes and relaxed assumptions beyond those in Section \ref{sec:tractable} while maintaining tractable SHAP computation. Finally, when exact computation is intractable (Section \ref{sec:intractable}), investigating the approximability of SHAP metrics through approximation and parameterized complexity theory~\citep{downey2012parameterized} is an important direction.

%Our work opens several promising avenues for future research on the computational properties of explainable AI methods, with a particular focus on SHAP. First, it would be interesting to broaden the computational analysis conducted in this work to include other popular SHAP-related metrics in the literature, such as asymmetric SHAP \cite{frye20} and SAGE \cite{covert2020understanding}. Also, in the future, we aim to explore more expressive distribution classes and relaxed distributional assumptions—extending beyond those examined in Section \ref{sec:tractable} —that still yield tractable SHAP computation. Finally, when exact computation proves intractable (Section \ref{sec:intractable}), it is worthwhile to theoretically investigate the question of the approximability of computing the SHAP metrics across various configurations, through the lens of approximation and parametrized complexity theory \cite{arora2009computational}.

%This paper aims to deepen our understanding of the computational complexity involved in obtaining different Shapley value variants. We found that for a variety of ML models, including decision trees, tree ensembles for regression, weighted automata, and linear regression models — computing both local and global interventional and baseline SHAP can be done in polynomial time when distributions are modeled by HMMs. This extends the distributional scope of popular algorithms like TreeSHAP, which is limited to empirical distributions. Additionally, we demonstrate a strict complexity gap between SHAP variants, showing that interventional and baseline SHAP can be strictly easier to compute than conditional SHAP. Despite these positive results, we uncovered intractability for various SHAP variants in neural networks and tree ensembles. Finally, we provided generalized complexity relations across SHAP variants. We believe that our framework offers a deeper understanding of the complexity involved in computing SHAP across various variants, models, distributions, as well as in both local and global computations, laying the groundwork for future research.

One limitation of this study is that it only evaluated LLaVA as the target Vision Language Model (VLM), which may limit the generalizability of the findings to other models. Additionally, the alignment of visual attention heatmaps for non-existing objects was not assessed, indicating that further analysis is needed in this area. 

Moreover, the experiments were conducted solely using the MSCOCO dataset, and future work should expand the evaluation to include additional datasets to ensure the robustness and broader applicability of the results. Furthermore, since datasets that contain both questions and corresponding answers alongside matching segmentation data, which can be used to evaluate object hallucination, are scarce, it may be necessary to develop such datasets.




