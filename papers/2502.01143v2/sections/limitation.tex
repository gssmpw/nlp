\section{Limitations}
\label{sec:limitations}
While \method demonstrates promising results in bridging the sim-to-real gap for agile humanoid control, our framework has several real-world limitations that highlights critical challenges in scaling agile humanoid control to real-world:
\begin{itemize}
    \item \textbf{Hardware Constraints}: Agile whole-body motions exert significant stress on robots, leading to motor overheating and hardware failure during data collection. Two Unitree G1 robots were broken to some extent during our experiments. This bottleneck limits the scale and diversity of real-world motion sequences that can be safely collected.
    \item \textbf{Dependence on Motion Capture Systems}: Our pipeline requires MoCap setup to record real-world trajectories. This introduces practical deployment barriers in unstructured environments where MoCap setups are unavailable.
    \item \textbf{Data-Hungry Delta Action Training}: While reducing the delta action model to 4 DoF ankle joints improved sample efficiency, training the full 23 DoF model remains impractical for real-world deployment due to the large demand of required motion clips (e.g., $>$ 400 episodes in simulation for the 23 DoF delta action training).
\end{itemize}
Future directions could focus on developing damage-aware policy to mitigate hardware risks, leveraging MoCap-free alignment to eliminate the reliance on MoCap, and exploring adaptation techniques for delta action models to achieve sample-efficient few-shot alignment.