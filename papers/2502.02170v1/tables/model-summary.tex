\begin{table*}[ht!]
\centering
\renewcommand{\arraystretch}{1.5} % Increases row spacing
\caption{Summary of link prediction models for next cell prediction}

\begin{tabular}{|p{0.2cm}|p{1cm}|p{2.35cm}|p{2.35cm}|p{4.75cm}|p{4.75cm}|}
\cline{1-6}

\textbf{} & \textbf{Name} & \textbf{Input} & \textbf{Output} & \textbf{Strengths} & \textbf{Weaknesses} \\ \hline

\multirow{2}{*}{\raisebox{-1\height}{\rotatebox{90}{\textbf{Autoencoders}}}}

& \textbf{GAE} & 
Node features, edge features and edge indexes. & 
Predicted existence or non-existence of edges. & 
Its architecture is highly adaptable. Also, it is simpler and more efficient than other models.  &  
Unsupervised and deterministic model with a binary output, so no further analysis is allowed. \\ \cline{2-6}

& \textbf{VGAE} & 
Node features, edge features and edge indexes. & 
Predicted probabilities for each edge. & 
Captures uncertainty and variability in the data. Improves the adaptability in networks with changing user behavior. & 
Unsupervised model. Requires an additional process to determine an optimal threshold for the probabilities obtained. \\ \cline{2-6}

\hline
\multirow{2}{*}{\raisebox{-0.95\height}{\rotatebox{90}{\textbf{Subgraph-based}}}}

& \textbf{SEAL} & 
Enclosing subgraphs and node features. & 
Binary classification (link or no link between two nodes). & 
Focuses on local subgraphs, improving short-range predictions; supervised learning enhances performance. & 
Subgraph extraction increases computational cost; requires labeled data. \\ \cline{2-6}

& \textbf{ComplEx} & 
Node pairs and relation types (for multi-relational graphs). & 
Predicted probabilities of links between nodes. & 
Handles multi-relational graphs; captures asymmetric relationships effectively. & 
Limited structural information beyond embeddings; computationally intensive for large-scale graphs. \\ \hline

\end{tabular}
\label{table:models}
\vspace*{-3ex}%
\end{table*}
