\section{Related Work}
With Theorem~\ref{thm:stability} in place we can now provide the appropriate context for our result. 
Stability bounds of the form we describe have a long history. For instance, a qualitative implication of the stability bound we derive in Corollary~\ref{cor:onesample} is that   the convergence of $\widehat{Q}$ to $Q$ in the $W_2$ metric implies the convergence of the OT map $\widehat{T}$ to $T_0$ in the $L^2(P)$ metric.
Qualitative results of this type are well-studied (for instance, see~\citet[Exercise 2.17]{villani2003}, 
\citet[Proposition 1.7.11]{panaretos2020},
and
\cite{segers2022}), and hold under significantly weaker conditions than the ones we impose. On the other hand, qualitative results do not typically yield sharp statistical rates on the risk in~\eqref{eqn:risk}.

To our knowledge, 
the earliest quantitative stability result
was derived by~\cite{gigli2011}
(who attributed it to Ambrosio)
 under the strong 
convexity 
assumption~\ref{ass:strong-convexity}
(see also Theorem~3.2 of \cite{ambrosio2019} and Theorem~3.5 of
\cite{li2021quantitative}).
Gigli showed that if  
$P\in \calP_{2,\mathrm{ac}}(\bbR^d)$, $\hat T$ is a   transport map from $P$ to $\hat Q$,
and $T_0=\nabla\varphi_0$ is the OT map from $P$ to $Q$, with the potential $\varphi_0$ satisfying condition~\ref{ass:strong-convexity}, then:
\begin{align*}
    \|T_0 - \hat T\|_{L^2(P)}^2 \lesssim \mathbb{E}_{X\sim P}\|\hat T(X) - X\|_2^2 - \mathbb{E}_{X \sim P} \|T_0(X) - X\|_2^2.
\end{align*}
This result suggests that the excess transport cost of a sub-optimal transport map grows in proportion to the $L^2(P)$ distance of the map from the OT map. This result is not directly useful for the statistical analysis of transport map estimates since it requires~$\hat T$ to be a valid transport map between $P$ and $\hat Q$, a condition that is not always satisfied by estimates of the map.
This deficiency inspired \citet{hutter2021} (see their Proposition~10) to develop the quadratic growth bound on the semi-dual functional $\semidualpq$, showing:
\begin{align*}
 \|\nabla \varphi - T_0\|_{L^2(P)}^2 \lesssim \semidualpq(\varphi) - \semidualpq(\varphi_0) \lesssim \|\nabla \varphi - T_0\|_{L^2(P)}^2,
\end{align*}
under the assumptions that $\varphi, \varphi_0$ are both smooth and strongly convex. \citet{vacher2021convex, muzellec2021} and \citet{makkuva2020} 
observed that the proof of \cite{hutter2021} also applies when $\varphi$ (alone) is smooth and strongly convex.
 
While the work of \citet{hutter2021} provides a semi-dual growth bound useful for the analysis of dual estimators, the works of \citet{ghosal2022,deb2021} and \citet{manole2024plugin} provide stability bounds useful for the analysis of plugin estimators. We have already discussed the results of \citet{manole2024plugin}, and highlighted how Theorem~\ref{thm:stability} provides useful improvements to their results. The stability bounds of~\citet{ghosal2022,deb2021} are quantitatively weaker than those in~\citet{manole2024plugin}, and replace the upper bound in~\eqref{eqn:mainstability}, by an empirical process term involving the estimated measures $\widehat{P}, \widehat{Q}$ and the corresponding Brenier potential $\varphi_{\widehat{P},\widehat{Q}}$. Analyzing this term to obtain risk bounds then necessitates strong assumptions to reason about the regularity of the Brenier potential $\varphi_{\widehat{P},\widehat{Q}}$ (akin to the torus assumption made by \citet{manole2024plugin}). 
A na\"{i}ve analysis of the empirical process term has a further deficiency which can result in sub-optimal rates of convergence for transport map estimates. In rough terms, the sharper upper bounds of Theorem~\ref{thm:stability} behave as the \emph{square} of an empirical process, yielding much faster rates of convergence.

Thus far, we have discussed
quantitative stability bounds
when one of conditions~\ref{ass:strong-convexity}
and~\ref{ass:smoothness} hold. 
In contrast, the works of \citet{berman2021convergence,merigot2020,delalande2023,gallouet2022strong}, and \citet{letrouit2024}  provide various stability bounds under variations of the target measure, which hold without smoothness assumptions on the optimal transport maps involved. That is, 
given a sufficiently regular
set $\Omega \subseteq \bbR^d$
and measure $P \in \calP_2(\Omega)$,
they derive bounds of the form:
\begin{equation}
\label{eq:merigot}
\|\hat T - T_0\|_{L^2(P)} \lesssim  W_2^\alpha(\hat Q,Q),
\end{equation}
for any measures $\hat Q,Q \in \calP(\Omega)$
satisfying appropriate tail conditions,
and with corresponding H\"older exponents $\alpha\in[0,1]$
which can be taken as high as $\alpha=1/6$.
Conversely, in this level
of generality, it
has been known since the work
of~\cite{gigli2011}
that the exponent $\alpha$ cannot be made greater than $1/2$, 
and it remains an open question to determine
whether this threshold can be achieved.
Our stability bounds show that 
the much more favorable H\"older exponent $\alpha=1$
is achievable when one of the optimal
transport maps is smooth. 

Beyond the statistical applications
that we have in mind, 
bounds of type~\eqref{eq:merigot} are highly sought after, as
the quantity $\|\hat T-T_0\|_{L^2(P)}$
is itself a metric between
$\hat Q$ and $Q$, which can be viewed as a proxy
for the Wasserstein distance. 
This quantity is sometimes
known as the linearized Wasserstein distance~\citep{wang2013linear},
and is widely-used in applications
due to its Hilbertian structure, and  its favorable computational properties~(e.g.\,\cite{cai2020linearized}). 
Corollary~\ref{cor:onesample} provides 
compelling motivation for the use of 
linearized Wasserstein distance, by showing
that it is in fact equivalent to the
original Wasserstein distance
under appropriate smoothness assumptions.