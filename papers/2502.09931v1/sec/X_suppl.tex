\clearpage
\setcounter{page}{1}
\maketitlesupplementary

\section{Dataset Descriptions}
\label{appendix_dataset_descriptions}

\begin{table}[h]
    \centering
    \scriptsize
    \setlength\tabcolsep{2pt} % default value: 6pt
    \begin{tabular}{c|c|cccc}
    \hline
    Segmentation Task &  Dataset & Resolutions       & Train & Test \\
    \hline
    Multi-organ Segmentation      & Synapse   & 512 $\times$ 512        & 18 Scans & 12 Scans \\ 
    \hline
    Skin Cancer Segmentation      & ISIC2018   & Variable          & 1868     & 261 \\ 
    \hline
    COVID19 Infection Segmentation & COVID19-1   & 512 $\times$ 512  & 643      & 383 \\
    \hline
    Breast Cancer Segmentation    & BUSI    & Variable          & 324      & 161 \\
    \hline
    \multicolumn{1}{c|}{\multirow{2}{*}{Polyp Segmentation}} & CVC-ClinicDB    & 384 $\times$ 288          & 490  & 62  \\
      & Kvasir-SEG & Variable           & 800 & 100 \\
    \hline
    \end{tabular}
    \caption{Details of the medical segmentation \textit{seen} clinical settings used in our experiments.}
    \label{tab:seen_clinical_dataset}
\end{table}

\begin{table}[h]
    \centering
    \scriptsize
    \setlength\tabcolsep{3pt} % default value: 6pt
    \begin{tabular}{c|c|ccc}
    \hline
    Segmentation Task &  Dataset  & Resolutions & Test \\
    \hline
    \multicolumn{1}{c|}{\multirow{2}{*}{Multi-organ Segmentation}} & AMOS-CT & Variable & 100 Scans \\
     & AMOS-MRI & Variable & 20 Scans \\
    \hline
    Skin Cancer Segmentation     & PH2   & 767 $\times$ 576  & 200  \\
    \hline
    COVID19 Infection Segmentation & COVID19-2   & 512 $\times$ 512  & 2535 \\
    \hline
    Breast Cancer Segmentation    & STU    & Variable          & 42	  \\
    \hline
    \multicolumn{1}{c|}{\multirow{3}{*}{Polyp Segmentation}} & CVC-300    & 574 $\times$ 500  & 60   \\
      & CVC-ColonDB    & 574 $\times$ 500  & 380  \\
      & ETIS    & 1255 $\times$ 966 & 196  \\
    \hline
    \end{tabular}
    \caption{Details of the medical segmentation \textit{unseen} clinical settings used in our experiments. Note that these datasets are used only for testing.}
    \label{tab:unseen_clinical_dataset}
\end{table}

In this section, we describe the training and testing datasets used in this paper. Tab. \ref{tab:seen_clinical_dataset} and Tab. \ref{tab:unseen_clinical_dataset} summarize the seen and unseen clinical setting datasets, respectively. We want to clarify that Tab. \ref{tab:unseen_clinical_dataset} datasets are used only to evaluated the generalizability of each model.

\begin{itemize}
    \item \textit{Multi-organ Segmentation:} The Synapse multi-organ segmentation dataset \cite{Synapse_dataset} is a widely recognized benchmark in the medical imaging community, specifically designed for the task of abdominal organ segmentation. The dataset consists of 30 abdominal CT scans, encompassing a total of 3,779 axial contrast-enhanced images. Each image is annotated by medical experts to identify and delineate eight different abdominal organs: the aorta, gallbladder, spleen, left kidney, right kidney, liver, pancreas, and stomach. The AMOS dataset \cite{ji2022amos} is an advanced and comprehensive benchmark dataset designed for the multi-organ segmentation. The dataset includes both Computed Tomography (CT) and Magnetic Resonance Imaging (MRI) scans, providing a diverse range of imaging data. We apply same preprocessing steps as the Synapse to ensure consistency. For evaluation, we separated into AMOS-CT (100 Scans) and AMOS-MRI (20 Scans), and the labels were aligned with those used in the Synapse. We want to clarify that AMOS-CT/MRI datasets are used only to evaluated the generalizability of each model. \\
    
    \item \textit{Breast Ultrasound Segmentation:} The BUSI dataset \cite{al2020dataset} comprises 780 images from 600 female patients, including 133 normal cases, 437 benign cases, and 210 malignant tumors. In contrast, the STU dataset \cite{zhuang2019rdau} includes only 42 breast ultrasound images collected by Shantou University. Due to the limited number of images in the STU dataset, it is primarily used to evaluate the generalizability of models across different datasets. \\

    \item \textit{Skin Lesion Segmentation:} The ISIC 2018 dataset \cite{gutman2016skin} comprises 2,594 images of varying sizes. We randomly selected 1,868 images for training and 261 images for testing. Additionally, the PH2 dataset \cite{mendoncca2013ph} was used to evaluate the domain generalizability of each model. \\

    
    \item \textit{COVID19 Infection Segmentation:} The COVID19-1 dataset \cite{ma_jun_2020_3757476} contains 1,277 high-quality CT images. For our experiments, we randomly divided the dataset into 643 training images and 383 test images. To assess the domain generalizability of our models, we utilized the COVID19-2 dataset\footnote{https://www.kaggle.com/datasets/piyushsamant11/pidata-new-names} solely for testing purposes. \\
     
    \item \textit{Polyp Segmentation:} To train and evaluate our proposed model, we utilized five benchmark datasets: CVC-ColonDB \cite{tajbakhsh2015automated}, ETIS \cite{silva2014toward}, Kvasir \cite{jha2020kvasir}, CVC-300 \cite{vazquez2017benchmark}, and CVC-ClinicDB \cite{bernal2015wm}. For training, we adopted the same dataset as the latest image polyp segmentation method, comprising 800 samples from Kvasir and 490 samples from CVC-ClinicDB. The remaining images from these datasets, along with the other three datasets, were used exclusively for testing.
\end{itemize}

\begin{figure*}[t]
    \centering
    \includegraphics[width=\textwidth]{figure/SkipConnectionEngineeringScheme}
    \caption{Comparison of skip connection frameworks scheme. Note that our unique approach (\textbf{TransGUNet}) incorporates \textit{ACS-GNN} with \textit{EFS-based spatial attention}.}
    \label{fig:SkipConnectionEngineeringScheme}
\end{figure*}

\section{Intuitiveness and Design Principle of TransGUNet}
We want to clarify that the design of TransGUNet is carefully considered rather than an ad-hoc decision, as follows: a) Medical images contain diverse anatomical structures, making it essential to flexibly capture both local details and global context. To address this, we transform cross-scale feature maps into a graph and apply efficient node-level attention. This design choice improves the model’s generalization across unseen clinical settings, as demonstrated in the "Ablation Study on ACS-GNN" section. b) Medical images segmentation models often produce uninformative feature maps (Figure 7 in the Appendix) due to various noise and irrelevant background artifacts, which can degrade the quality of spatial attention maps and reduce segmentation performance. To address this, we incorporate EFS-based spatial attention, prioritizing low-entropy feature maps to enhance the quality of spatial attention map. This novel approach also enhances generalization ability across unseen clinical settings, as shown in the "Ablation Study on EFS-based Attention" section.

\section{Technical Novelty of TransGUNet}
\noindent \textbf{PVT-GCASCADE (WACV2024) vs TransGUNet.} Like our approach, PVT-GCASCADE utilizes GNN; however, it does not consider cross-scale information, which is limited to medical images with more diverse lesion sizes. Additionally, unlike PVT-GCASCADE, it reduces the influence of uninformative feature maps with UFS to enable reliable spatial attention.

\noindent \textbf{CFATransGNet (CBM2024) vs TransGUNet.} TransGUNet focuses on efficiently reducing the semantic gap between the encoder and decoder using the GNN, which is not commonly used in skip connections. This suggests the potential for further development of GNN-based skip connection engineering beyond Transformer-based skip connection engineering (CFATransUNet).

\noindent \textbf{MADGNet (CVPR2024) vs TransGUNet.} While MADGNet employs Multi-Frequency and Multi-Scale Attention mechanisms to enhance feature extraction, TransGUNet's use of ACS-GNN allows for more robust integration of cross-scale features and adaptive learning of node importance. This novel approach not only improves segmentation accuracy but also addresses the limitations of traditional convolutional methods by leveraging the strengths of GNNs in handling diverse and noisy medical data.

\begin{table}[t]
    \centering
    \scriptsize
    \setlength\tabcolsep{5.0pt} % default value: 6pt
    % \renewcommand{\arraystretch}{0.9} % Tighter
    \begin{tabular}{c|ccc}
    \hline
    Method & Parameters (M) & FLOPs (G) & Inference Time (ms) \\
    \hline
    UNet        & 8.2 & 23.7 & 10.1 \\ 
    UNet++      & 34.9 & 197.8 & 22.9	\\
    CENet       & 18.0 & 9.2 & 10.5 \\ 
    TransUNet   & 53.4 & 21.8 & 93.4 \\
    nnUNet      & - & - & - \\
    MSRFNet     & 21.5 & 156.6 & 73.8 \\
    DCSAUNet    & 25.9 & 9.2 & 24.3 \\
    M2SNet      & 25.3 & 12.8 & 32.1 \\
    ViGUNet     & - & - & - \\
    PVT-GCASCADE & 25.4 & 7.9 & 17.4 \\ 
    CFATransUNet & 64.6 & 32.9 & 36.0 \\
    MADGNet     & 29.8 & 16.8 & 24.0 \\
    \hline
    \textbf{TransGUNet (Ours)} & 25.0 & 10.0 & 19.4 \\
    \hline
    \end{tabular}
    \caption{The number of parameters (M), FLOPs (G), and Inference Time (ms) of different models.}
    \label{tab:efficiency_analysis}
\end{table}

\section{Broader Impact in Artificial Intelligence}

TransGUNet’s superior performance in medical image segmentation has the potential to reliable medical diagnostics and treatment planning. By providing accurate and reliable segmentation of complex anatomical structures, it enables healthcare professionals to make more informed decisions, leading to improved patient outcomes. The model’s effective integration of Graph Neural Networks (GNNs) and Transformer architectures demonstrates a hybrid approach that addresses the limitations of traditional convolutional neural networks (CNNs), enhancing robustness and adaptability across various AI tasks. Furthermore, TransGUNet highlights the importance of developing efficient AI models that do not compromise on performance. Techniques like entropy-driven feature selection (EFS) and attentional cross-scale GNNs optimize resource use, setting a new benchmark for efficiency in AI-driven medical applications. This balance between high performance and computational efficiency can guide future research in creating powerful and resource-conscious AI models.

\section{More Detailed Ablation Study on TransGUNet}
In this section, we perform a more detailed ablation study on TransGUNet.

\subsection{Ablation Study on Backbone in TransGUNet}

\begin{table}[h]
    \centering
    \scriptsize
    \setlength\tabcolsep{2.0pt} % default value: 6pt
    % \renewcommand{\arraystretch}{0.9} % Tighter
    \begin{tabular}{c|c|cc|cc|c|c}
    \hline
    Network & \multicolumn{1}{c|}{\multirow{2}{*}{Backbone}} & \multicolumn{2}{c|}{\textit{Seen}}  & \multicolumn{2}{c|}{\textit{Unseen}} & \multicolumn{1}{c|}{\multirow{2}{*}{Param (M)}}  & \multicolumn{1}{c}{\multirow{2}{*}{FLOPs (G)}} \\ \cline{3-6}
    Type    & & DSC & mIoU & DSC & mIoU & & \\
    \hline
    \multicolumn{1}{c|}{\multirow{3}{*}{CNN}} & ResNet50 & 85.9 & 79.1 & 67.3 & 60.5 & 25.1M & 18.0G \\
     & Res2Net50 & 86.0 & 79.1 & 70.6 & 62.8 & 25.2M & 18.8G \\
     & ResNeSt50 & \textit{86.6} & 79.7 & 73.3 & 64.9 & 26.9M & 20.4G \\
     \hline
    \multicolumn{1}{c|}{\multirow{3}{*}{Transformer}} & ViT-B-16 & 83.6 & 75.9 & 64.7 & 56.3 & 53.4M & 18.4G \\
     & PVT-v2-b2 & \textit{86.6} & \textit{80.0} & \textit{77.0} & \textit{68.3} & 26.3M & 10.5G \\
     & \textbf{P2T-Small \tiny{(Ours)}} & \textbf{\underline{87.3}} & \textbf{\underline{80.6}} & \textbf{\underline{78.6}} & \textbf{\underline{69.7}} & 25.0M & 10.0G \\
     \hline
    \end{tabular}
    \caption{Quantitative results for each \textit{Seen} and \textit{Unseen} datasets according to backbone network.}
    \label{tab:ablation_backbone_networks}
\end{table}

In this section, we conduct an ablation study to evaluate the impact of different backbone models on the performance of TransGUNet. This experiment uses several popular CNN and Transformer architectures, including ResNet50 \cite{he2016deep}, Res2Net \cite{gao2019res2net}, ResNeSt50 \cite{zhang2022resnest}, ViT-B-16 \cite{dosovitskiy2021an}, PVT-v2-b2 \cite{wang2022pvt}, and P2T-Small \cite{wu2022p2t}. Notably, only the backbone network was changed, while all other architectural settings remained consistent with those in the main experiment. We reported the \textit{mean} performance for each \textit{seen} and \textit{unseen} clinical settings in Tab. \ref{tab:ablation_backbone_networks}. The datasets used in the \textit{seen} and \textit{unseen} clinical settings are the same as Tab. \ref{tab:comparison_sota_in_domain} and Tab. \ref{tab:comparison_sota_out_domain}, respectively. For convenience, we denote $( \cdot, \cdot )$ as the performance improvement gap between TransGUNet and other models for seen and unseen clinical settings.

ResNeSt50, as utilized in MADGNet, achieves comparable or even superior performance. Furthermore, using PVT-v2-b2, as implemented in PVT-GCASCADE, yields higher performance in both seen and unseen clinical settings. Specifically, TransGUNet with PVT-v2-b2 outperforms PVT-GCASCADE by \textbf{\underline{(0.4\%, 2.2\%)}} and \textbf{\underline{(0.7\%, 2.2\%)}} in DSC and mIoU, respectively, demonstrating its robustness and effectiveness according to the backbone type.

\subsection{Ablation Study on ECA Kernel Size in ACS-GNN}

\begin{table}[h]
    \centering
    \scriptsize
    \setlength\tabcolsep{4.0pt} % default value: 6pt
    % \renewcommand{\arraystretch}{0.9} % Tighter
    \begin{tabular}{c|cc|cc|c|c}
    \hline
    \multicolumn{1}{c|}{\multirow{2}{*}{ECA Kernel Size $k$}} & \multicolumn{2}{c|}{\textit{Seen}}  & \multicolumn{2}{c|}{\textit{Unseen}} & \multicolumn{1}{c|}{\multirow{2}{*}{Param (M)}}  & \multicolumn{1}{c}{\multirow{2}{*}{FLOPs (G)}} \\ \cline{2-5}
       & DSC & mIoU & DSC & mIoU & & \\ 
    \hline
    $k = 3$ \textbf{(Ours)}                 & \textbf{\underline{87.3}} & \textbf{\underline{80.6}} & \textbf{\underline{78.6}} & \textbf{\underline{69.7}} & 25.0M & 10.0G \\
    $k = 5$ & 87.5 & 80.7 & 78.3 & 69.6 & 25.0M & 10.0G \\
    $k = 7$ & 87.4 & 80.5 & 78.4 & 69.5 & 25.0M & 10.0G \\
    \hline
    \end{tabular}
    \caption{Quantitative results for each \textit{Seen} and \textit{Unseen} datasets according to various ECA kernel size $k$.}
    \label{tab:ablation_eca_kernel_size}
\end{table}

In this section, we conduct an ablation study to compare the performance according to various ECA kernel size $k = \{ 3, 5, 7 \}$. In the main manuscript, we used $k = 3$ as the ECA kernel size. We reported the \textit{mean} performance for each \textit{seen} and \textit{unseen} clinical settings in Tab. \ref{tab:ablation_eca_kernel_size}. The datasets used in the \textit{seen} and \textit{unseen} clinical settings are the same as Tab. \ref{tab:comparison_sota_in_domain} and Tab. \ref{tab:comparison_sota_out_domain}, respectively. 

As listed in Tab. \ref{tab:ablation_eca_kernel_size}, we observed that while increasing the kernel size for $k = 3$ to $k = 7$, it does not provide meaningful statistical improvement. Consequently, we choose $k = 3$ as our model basic configuration in ECA kernel size.

\subsection{Ablation Study on Repetition Time of ACS-GNN with EFS-based Spatial Attention}

\begin{table}[h]
    \centering
    \scriptsize
    \setlength\tabcolsep{2.5pt} % default value: 6pt
    % \renewcommand{\arraystretch}{0.9} % Tighter
    \begin{tabular}{c|cc|cc|c|c}
    \hline
    \multicolumn{1}{c|}{\multirow{2}{*}{Repetition Time $G$}} & \multicolumn{2}{c|}{\textit{Seen}}  & \multicolumn{2}{c|}{\textit{Unseen}} & \multicolumn{1}{c|}{\multirow{2}{*}{Param (M)}}  & \multicolumn{1}{c}{\multirow{2}{*}{FLOPs (G)}} \\ \cline{2-5}
     & DSC & mIoU & DSC & mIoU & & \\ 
    \hline
     $G = 3$ & 86.7 & 80.0 & 77.4 & 68.6 & 26.6M & 12.4G \\
     $G = 5$ & 86.4 & 79.7 & 77.2 & 68.3 & 28.1M & 14.7G \\
     $G = 7$ & 86.3 & 80.0 & 77.1 & 68.3 & 29.6M & 17.0G \\
     $G = 9$ & 86.3 & 79.7 & 77.1 & 68.2 & 31.1M & 19.4G \\
    \hline
    % \multicolumn{1}{c|}{\multirow{4}{*}{\cmark}} & $G = 3$ & - & - & - & - & 25.0M & 12.4G \\
    %  & $G = 5$ & - & - & - & - & 25.0M & 14.7G \\
    %  & $G = 7$ & - & - & - & - & 25.0M & 17.0G \\
    %  & $G = 9$ & - & - & - & - & 25.0M & 19.4G \\
    % \hline
    $G = 1$ \textbf{(Ours)} & \textbf{\underline{87.3}} & \textbf{\underline{80.6}} & \textbf{\underline{78.6}} & \textbf{\underline{69.7}} & 25.0M & 10.0G \\
    \hline
    \end{tabular}
    \caption{Quantitative results for each \textit{Seen} and \textit{Unseen} datasets according to repetition time $G$ of ACS-GNN with EFS-based spatial attention.}
    \label{tab:ablation_repetition_time}
\end{table}

In this section, we conduct an ablation study to compare the performance according to repetition time $G \in \{ 1, 3, 5, 7, 9 \}$ of ACS-GNN with EFS-based spatial attention. In the main manuscript, we used $G = 1$ as the default repetition time. We reported the \textit{mean} performance for each \textit{seen} and \textit{unseen} clinical settings in Tab. \ref{tab:ablation_repetition_time}. The datasets used in the \textit{seen} and \textit{unseen} clinical settings are the same as Tab. \ref{tab:comparison_sota_in_domain} and Tab. \ref{tab:comparison_sota_out_domain}, respectively. 

The ablation study reveals that repeating the ACS-GNN module with EFS-based spatial attention leads to overfitting. Despite the increased capacity of the model, performance decreased on both seen and unseen datasets. This result indicates that the model is overfitting to the training data.

\section{Metrics Descriptions}
\label{appendix_metric_descriptions}

In this section, we describe the metrics used in this paper. For convenience, we denote $TP, FP$, and $FN$ as the number of samples of true positive, false positive, and false negative between two binary masks $A$ and $B$.  

\begin{itemize}
    \item The \textit{Mean Dice Similarity Coefficient (DSC)} \cite{milletari2016v} measures the similarity between two samples and is widely used to assess the performance of segmentation tasks, such as image segmentation or object detection. \textbf{\underline{Higher is better}}. For given two binary masks $A$ and $B$, DSC is defined as follows:
    \begin{equation}
        \textbf{DSC}(A, B) = \frac{2 \times | A \cap B |}{| A \cup B |} = \frac{2 \times TP}{2 \times TP + FP + FN}
    \end{equation}

    \item The \textit{Mean Intersection over Union (mIoU)} measures the ratio of the intersection area to the union area between predicted and ground truth masks in segmentation tasks. \textbf{\underline{Higher is better}}. For given two binary masks $A$ and $B$, mIoU is defined as follows:
    \begin{equation}
        \textbf{mIoU}(A, B) = \frac{A \cap B}{A \cup B} = \frac{TP}{TP + FP + FN}
    \end{equation}

    \item The \textit{Mean Weighted F-Measure} $F_{\beta}^{\omega}$ \cite{margolin2014evaluate} is a metric that combines weighted precision $Precision^{\omega}$ (Measure of exactness) and weighted recall $Recall^{\omega}$ (Measure of completeness) into a single value by calculating the harmonic mean. $\beta$ signifies the effectiveness of detection with respect to a user who attaches $\beta$ times as much importance to $Recall^{\omega}$ as to $Precision^{\omega}$. \textbf{\underline{Higher is better}}. $F_{\beta}^{\omega}$ is defined as follows:
    \begin{equation}
        F_{\beta}^{\omega} = (1 + \beta^{2}) \cdot \frac{Precision^{\omega} \cdot Recall^{\omega}}{\beta^{2} \cdot Precision^{\omega} + Recall^{\omega}}
    \end{equation}

    \item The \textit{Mean S-Measure} $S_{\alpha}$ \cite{fan2017structure} is used to evaluate the quality of image segmentation, specifically focusing on the structural similarity between the region-aware $S_{o}$ and object-aware $S_{r}$. \textbf{\underline{Higher is better}}. $S_{\alpha}$ is defined as follows:
    \begin{equation}
        S_{\alpha} = \gamma S_{o} + (1 - \gamma) S_{r}
    \end{equation}
    
    \item \textit{Mean E-Measure} $E_{\phi}^{max}$ \cite{fan2018enhanced} assesses the edge accuracy in edge detection or segmentation tasks. It evaluates how well the predicted edges align with the ground truth edges using foreground map $FM$. \textbf{\underline{Higher is better}}. $E_{\phi}^{max}$ is defined as follows: \\
    \begin{equation}
        E_{\phi}^{max} = \frac{1}{H \times W} \sum_{h = 1}^{H} \sum_{w = 1}^{W} \phi FM(h, w)
    \end{equation}

    \item \textit{Mean Absolute Error (MAE)} calculates the average absolute differences between predicted and ground truth values. \textbf{\underline{Lower is better}}. For given two binary masks $A$ and $B$, MAE is defined as follows: \\
    \begin{equation}
        \textbf{MAE}(A, B) = \frac{1}{H \times W} \sum_{h = 1}^{H} \sum_{w = 1}^{W} \left| A(h, w) - B(h, w) \right|
    \end{equation}

    \item \textit{Hausdorff Distance 95 (HD95)} \cite{celaya2023generalized} is the 95th percentile of the Hausdorff Distance, which means it excludes the top 5\% of the most extreme distances for more robust to noise and outliers. It is calculated by first computing all pairwise distances between the points in the predicted and ground truth segmentations, sorting these distances, and then taking the 95th percentile value. Note that the Hausdorff distance ranges from 0, indicating no difference when two sets are identical, to infinity, as the maximum distance between two sets can grow indefinitely. \textbf{\underline{Lower is better}}. For given two binary masks $A$ and $B$ and euclidean distance $d(\cdot, \cdot)$, HD is defined as follows:
    \begin{dmath}
        \textbf{HD}(A, B) = \textbf{max} \left( \textbf{max}_{a \in A} \textbf{min}_{b \in B} d(a, b), \textbf{max}_{b \in B} \textbf{min}_{a \in A} d(a, b) \right)
    \end{dmath}
\end{itemize}

\section{More Qualtative and Quantitative Results}
In this section, we provide the quantitative results with various metrics in Tab. \ref{tab:comparison_sota_dermatoscopy_other_metrics}, \ref{tab:comparison_sota_radiology_other_metrics}, \ref{tab:comparison_sota_ultrasound_other_metrics}, \ref{tab:comparison_sota_colonoscopy_other_metrics} for binary segmentation. Additionally, we also provide the quantitative results in Tab. \ref{tab:comparison_sota_multiorgan_dsc_metrics}, \ref{tab:comparison_sota_multiorgan_mIoU_metrics}, \ref{tab:comparison_sota_multiorgan_HD95_metrics} with each organ for multi-organ segmentation. For all tables, \textbf{\underline{Bold}} and \textit{italic} are the first and second best performance results, respectively. We also present more various qualitative results on datasets in Fig. \ref{fig:Sup_QualitativeResults_Dermatoscopy}, \ref{fig:Sup_QualitativeResults_Radiology}, \ref{fig:Sup_QualitativeResults_Ultrasound}, \ref{fig:Sup_QualitativeResults_Colonoscopy}.

\begin{table}[h]
    \centering
    \scriptsize
    \setlength\tabcolsep{2pt} % default value: 6pt
    \renewcommand{\arraystretch}{0.9} % Tighter
    \begin{tabular}{c|cccccc}
    \hline
    \multicolumn{1}{c|}{\multirow{2}{*}{Method}} & \multicolumn{6}{c}{ISIC2018 $\Rightarrow$ ISIC2018} \\ \cline{2-7}
     & DSC \scriptsize{$\uparrow$} & mIoU \scriptsize{$\uparrow$} & $F_{\beta}^{w}$ \scriptsize{$\uparrow$}  & $S_{\alpha}$ \scriptsize{$\uparrow$} & $E_{\phi}^{max}$ \scriptsize{$\uparrow$} & MAE \scriptsize{$\downarrow$} \\
     \hline
     UNet \tiny{(MICCAI2016)}       & 86.9 & 80.2 & 87.9 & 80.4 & 91.3 & 4.7 \\
     UNet++ \tiny{(DLMIA2018)}      & 87.8 & 80.5 & 86.5 & 80.5 & 92.0 & 4.5 \\
     CENet \tiny{(TMI2019)}         & 89.1 & 82.1 & 88.1 & 81.3 & 93.0 & 4.3 \\
     TransUNet \tiny{(arxiv2021)}   & 87.3 & 81.2 & 88.6 & 80.8 & 91.9 & 4.2 \\
     MSRFNet \tiny{(BHI2022)}       & 88.2 & 81.3 & 86.9 & 80.7 & 92.0 & 4.7 \\
     DCSAUNet \tiny{(CBM2023)}      & 89.0 & 82.0 & 87.8 & 81.4 & 92.9 & 4.4 \\
     M2SNet \tiny{(arxiv2023)}      & 89.2 & 83.4 & \textit{90.0} & 82.0 & 93.8 & 3.7 \\
     PVT-GCASCADE \tiny{(WACV2024)} & \textit{90.3} & 83.5 & 88.9 & 82.0 & 93.9 & 3.7 \\
     CFATransUNet \tiny{(CBM2024)}  & 90.1 & 83.5 & 89.0 & \textit{82.1} & \textit{94.1} & \textit{3.5} \\
     MADGNet \tiny{(CVPR2024)}      & 90.2 & \textit{83.7} & 89.2 & 82.0 & \textit{94.1} & 3.6 \\
     \hline
     \multicolumn{1}{c|}{\multirow{2}{*}{\textbf{TransGUNet \tiny{(Ours)}}}}     & \textbf{\underline{91.1}} & \textbf{\underline{84.8}} & \textbf{\underline{90.1}} & \textbf{\underline{82.6}} & \textbf{\underline{94.4}} & \textbf{\underline{3.4}} \\ \cline{2-7}
     & \textbf{+0.8} & \textbf{+1.1} & \textbf{+0.1} & \textbf{+0.5} & \textbf{+0.3} & \textbf{+0.2} \\
    \hline
    \multicolumn{1}{c|}{\multirow{2}{*}{Method}} & \multicolumn{6}{c}{ISIC2018 $\Rightarrow$ PH2} \\ \cline{2-7}
     & DSC \scriptsize{$\uparrow$} & mIoU \scriptsize{$\uparrow$} & $F_{\beta}^{w}$ \scriptsize{$\uparrow$}  & $S_{\alpha}$ \scriptsize{$\uparrow$} & $E_{\phi}^{max}$ \scriptsize{$\uparrow$} & MAE \scriptsize{$\downarrow$} \\
     \hline
     UNet \tiny{(MICCAI2016)}       & 90.3 & 83.5 & 88.4 & 74.8 & 90.8 & 6.9 \\
     UNet++ \tiny{(DLMIA2018)}      & 88.0 & 80.1 & 85.7 & 73.2 & 89.2 & 7.9 \\
     CENet \tiny{(TMI2019)}         & 90.5 & 83.3 & 87.3 & \textbf{\underline{78.1}} & 91.5 & 6.0 \\
     TransUNet \tiny{(arxiv2021)}   & 89.5 & 82.1 & 86.9 & 74.3 & 90.3 & 6.7 \\
     MSRFNet \tiny{(BHI2022)}       & 90.5 & 83.5 & 87.5 & 75.0 & 91.4 & 6.0 \\
     DCSAUNet \tiny{(CBM2023)}      & 89.0 & 81.5 & 85.7 & 74.0 & 90.2 & 6.9 \\
     M2SNet \tiny{(arxiv2023)}      & 90.7 & 83.5 & 87.6 & 75.5 & 92.0 & 5.9 \\
     PVT-GCASCADE \tiny{(WACV2024)} & \textit{91.5} & 84.9 & 88.6 & 76.3 & 92.7 & 5.3 \\
     CFATransUNet \tiny{(CBM2024)}  & \textit{91.5} & \textit{85.0} & \textit{88.7} & 76.3 & 92.6 & 5.3 \\
     MADGNet \tiny{(CVPR2024)}      & 91.3 & 84.6 & 88.4 & 76.2 & \textit{92.8} & \textit{5.1} \\
     \hline
     \multicolumn{1}{c|}{\multirow{2}{*}{\textbf{TransGUNet \tiny{(Ours)}}}}     & \textbf{\underline{91.7}} & \textbf{\underline{85.2}} & \textbf{\underline{88.9}} & \textit{76.6} & \textbf{\underline{93.1}} & \textbf{\underline{5.0}} \\ \cline{2-7}
     & \textbf{+0.2} & \textbf{+0.2} & \textbf{+0.2} & \textbf{-1.5} & \textbf{+0.3} & \textbf{+0.1} \\
    \hline
    \end{tabular}
    \caption{Segmentation results on \textbf{Skin Lesion Segmentation}. We train each model on ISIC2018 \cite{gutman2016skin} train dataset and evaluate on ISIC2018 \cite{gutman2016skin} and PH2 \cite{mendoncca2013ph} test datasets.}
    \label{tab:comparison_sota_dermatoscopy_other_metrics}
\end{table}

\begin{table}[h]
    \centering
    \scriptsize
    \setlength\tabcolsep{2pt} % default value: 6pt
    \renewcommand{\arraystretch}{0.9} % Tighter
    \begin{tabular}{c|cccccc}
    \hline
    \multicolumn{1}{c|}{\multirow{2}{*}{Method}} & \multicolumn{6}{c}{COVID19-1 $\Rightarrow$ COVID19-1} \\ \cline{2-7}
     & DSC \scriptsize{$\uparrow$} & mIoU \scriptsize{$\uparrow$} & $F_{\beta}^{w}$ \scriptsize{$\uparrow$}  & $S_{\alpha}$ \scriptsize{$\uparrow$} & $E_{\phi}^{max}$ \scriptsize{$\uparrow$} & MAE \scriptsize{$\downarrow$} \\
     \hline
     UNet \tiny{(MICCAI2016)}       & 47.7 & 38.6 & 36.1 & 69.6 & 62.7 & 2.1 \\
     UNet++ \tiny{(DLMIA2018)}      & 65.6 & 57.1 & 54.4 & 78.8 & 73.2 & 1.3 \\
     CENet \tiny{(TMI2019)}         & 76.3 & 69.2 & 64.4 & 83.2 & 76.6 & 0.6 \\
     TransUNet \tiny{(arxiv2021)}   & 75.6 & 68.8 & 63.4 & 82.7 & 75.5 & 0.7 \\
     MSRFNet \tiny{(BHI2022)}       & 75.2 & 68.0 & 63.4 & 82.7 & 76.3 & 0.8 \\
     DCSAUNet \tiny{(CBM2023)}      & 75.3 & 68.2 & 63.1 & 83.0 & 77.3 & 0.7 \\
     M2SNet \tiny{(arxiv2023)}      & 81.7 & 74.7 & \textit{68.3} & 85.7 & 80.1 & 0.6 \\
     PVT-GCASCADE \tiny{(WACV2024)} & 82.3 & 74.8 & 68.1 & 85.8 & 80.1 & \textit{0.5} \\
     CFATransUNet \tiny{(CBM2024)}  & 80.4 & 73.6 & \textit{68.3} & 84.8 & 79.2 & \textit{0.5} \\
     MADGNet \tiny{(CVPR2024)}      & \textit{83.7} & \textit{76.8} & \textbf{\underline{70.2}} & \textit{86.3} & \textbf{\underline{81.5}} & \textit{0.5} \\
     \hline
     \multicolumn{1}{c|}{\multirow{2}{*}{\textbf{TransGUNet \tiny{(Ours)}}}}     & \textbf{\underline{84.0}} & \textbf{\underline{77.0}} & \textbf{\underline{70.2}} & \textbf{\underline{86.6}} & \textit{81.2} & \textbf{\underline{0.4}} \\ \cline{2-7}
     & \textbf{+0.3} & \textbf{+0.2} & \textbf{+0.0} & \textbf{+0.3} & \textbf{-0.3} & \textbf{+0.1} \\
    \hline
    \multicolumn{1}{c|}{\multirow{2}{*}{Method}} & \multicolumn{6}{c}{COVID19-1 $\Rightarrow$ COVID19-2} \\ \cline{2-7}
     & DSC \scriptsize{$\uparrow$} & mIoU \scriptsize{$\uparrow$} & $F_{\beta}^{w}$ \scriptsize{$\uparrow$}  & $S_{\alpha}$ \scriptsize{$\uparrow$} & $E_{\phi}^{max}$ \scriptsize{$\uparrow$} & MAE \scriptsize{$\downarrow$} \\
     \hline
     UNet \tiny{(MICCAI2016)}       & 47.1 & 37.7 & 46.7 & 68.7 & 68.6 & 1.0 \\
     UNet++ \tiny{(DLMIA2018)}      & 50.5 & 40.9 & 50.6 & 69.8 & 75.7 & 1.0 \\
     CENet \tiny{(TMI2019)}         & 60.1 & 49.9 & 61.1 & 73.4 & 80.1 & 1.1 \\
     TransUNet \tiny{(arxiv2021)}   & 56.9 & 48.0 & 58.0 & 72.5 & 79.7 & \textbf{\underline{0.8}} \\
     MSRFNet \tiny{(BHI2022)}       & 58.3 & 48.4 & 59.1 & 72.7 & 79.8 & 1.0 \\
     DCSAUNet \tiny{(CBM2023)}      & 52.4 & 44.0 & 52.0 & 71.3 & 76.3 & 1.0 \\
     M2SNet \tiny{(arxiv2023)}      & 68.6 & 58.9 & 68.5 & 76.9 & 86.1 & 1.1 \\
     PVT-GCASCADE \tiny{(WACV2024)} & 71.0 & 60.4 & 70.0 & 77.8 & 87.9 & 1.2 \\
     CFATransUNet \tiny{(CBM2024)}  & 65.7 & 56.2 & 67.0 & 75.1 & 83.0 & 1.2 \\
     MADGNet \tiny{(CVPR2024)}      & \textit{72.2} & \textbf{\underline{62.6}} & \textbf{\underline{72.3}} & \textit{78.2} & \textit{88.1} & 1.0 \\
     \hline
     \multicolumn{1}{c|}{\multirow{2}{*}{\textbf{TransGUNet \tiny{(Ours)}}}}     & \textbf{\underline{73.0}} & \textit{62.4} & \textit{72.0} & \textbf{\underline{78.7}} & \textbf{\underline{89.5}} & \textit{0.9} \\ \cline{2-7}
     & \textbf{+0.8} & \textbf{-0.2} & \textbf{-0.3} & \textbf{+0.5} & \textbf{+1.4} & \textbf{-0.1} \\
    \hline
    \end{tabular}
    \caption{Segmentation results on \textbf{COVID19 Infection Segmentation}. We train each model on COVID19-1 \cite{ma_jun_2020_3757476} train dataset and evaluate on COVID19-1 \cite{ma_jun_2020_3757476} and COVID19-2 test datasets.}
    \label{tab:comparison_sota_radiology_other_metrics}
\end{table}

\begin{table}[h]
    \centering
    \scriptsize
    \setlength\tabcolsep{2pt} % default value: 6pt
    \renewcommand{\arraystretch}{0.9} % Tighter
    \begin{tabular}{c|cccccc}
    \hline
    \multicolumn{1}{c|}{\multirow{2}{*}{Method}} & \multicolumn{6}{c}{BUSI $\Rightarrow$ BUSI} \\ \cline{2-7}
     & DSC \scriptsize{$\uparrow$} & mIoU \scriptsize{$\uparrow$} & $F_{\beta}^{w}$ \scriptsize{$\uparrow$}  & $S_{\alpha}$ \scriptsize{$\uparrow$} & $E_{\phi}^{max}$ \scriptsize{$\uparrow$} & MAE \scriptsize{$\downarrow$} \\
     \hline
     UNet \tiny{(MICCAI2016)}       & 69.5 & 60.2 & 67.2 & 76.9 & 83.2 & 4.8 \\
     UNet++ \tiny{(DLMIA2018)}      & 71.3 & 62.3 & 68.9 & 78.1 & 84.4 & 4.8 \\
     CENet \tiny{(TMI2019)}         & 79.7 & 71.5 & 78.1 & 82.8 & 91.1 & 3.9 \\
     TransUNet \tiny{(arxiv2021)}   & 75.5 & 68.4 & 73.8 & 79.8 & 88.6 & 4.2 \\
     MSRFNet \tiny{(BHI2022)}       & 76.6 & 68.1 & 75.1 & 80.9 & 88.5 & 4.2 \\
     DCSAUNet \tiny{(CBM2023)}      & 73.7 & 65.0 & 71.5 & 79.6 & 86.0 & 4.6 \\
     M2SNet \tiny{(arxiv2023)}      & 80.4 & 72.5 & 78.7 & 83.0 & 91.2 & 4.1 \\
     PVT-GCASCADE \tiny{(WACV2024)} & \textit{82.0} & \textit{73.6} & \textit{80.2} & 83.7 & \textit{92.1} & 3.8 \\
     CFATransUNet \tiny{(CBM2024)}  & 80.6 & 72.8 & 79.5 & 83.1 & 91.1 & 4.1 \\
     MADGNet \tiny{(CVPR2024)}      & 81.3 & 73.4 & 79.5 & \textit{83.8} & 91.7 & \textbf{\underline{3.6}} \\
     \hline
     \multicolumn{1}{c|}{\multirow{2}{*}{\textbf{TransGUNet \tiny{(Ours)}}}}     & \textbf{\underline{82.7}} & \textbf{\underline{74.7}} & \textbf{\underline{81.0}} & \textbf{\underline{84.3}} & \textbf{\underline{93.0}} & \textit{3.7} \\ \cline{2-7}
     & \textbf{+1.0} & \textbf{+1.1} & \textbf{+0.8} & \textbf{+0.5} & \textbf{+0.9} & \textbf{-0.1} \\
    \hline
    \multicolumn{1}{c|}{\multirow{2}{*}{Method}} & \multicolumn{6}{c}{BUSI $\Rightarrow$ STU} \\ \cline{2-7}
     & DSC \scriptsize{$\uparrow$} & mIoU \scriptsize{$\uparrow$} & $F_{\beta}^{w}$ \scriptsize{$\uparrow$}  & $S_{\alpha}$ \scriptsize{$\uparrow$} & $E_{\phi}^{max}$ \scriptsize{$\uparrow$} & MAE \scriptsize{$\downarrow$} \\
     \hline
     UNet \tiny{(MICCAI2016)}       & 71.6 & 61.6 & 71.6 & 76.1 & 82.4 & 5.2 \\
     UNet++ \tiny{(DLMIA2018)}      & 77.0 & 68.0 & 76.3 & 79.8 & 86.7 & 4.4 \\
     CENet \tiny{(TMI2019)}         & 86.0 & 77.2 & 84.2 & 84.5 & 93.7 & 2.8 \\
     TransUNet \tiny{(arxiv2021)}   & 41.4 & 32.1 & 40.8 & 60.2 & 58.1 & 9.7 \\
     MSRFNet \tiny{(BHI2022)}       & 84.0 & 75.2 & 75.1 & 83.5 & 92.2 & 3.1 \\
     DCSAUNet \tiny{(CBM2023)}      & 86.1 & 76.5 & 82.7 & 84.9 & 94.7 & 3.2 \\
     M2SNet \tiny{(arxiv2023)}      & 79.4 & 69.3 & 76.4 & 81.3 & 90.7 & 4.3 \\
     PVT-GCASCADE \tiny{(WACV2024)} & 86.4 & 76.6 & 82.2 & 84.3 & 84.2 & 3.1 \\
     CFATransUNet \tiny{(CBM2024)}  & \textit{87.9} & \textit{79.2} & \textit{85.3} & \textit{85.7} & \textit{95.4} & \textbf{\underline{2.6}} \\
     MADGNet \tiny{(CVPR2024)}      & \textbf{\underline{88.4}} & \textbf{\underline{79.9}} & \textbf{\underline{86.4}} & \textbf{\underline{86.2}} & \textbf{\underline{95.9}} & \textbf{\underline{2.6}} \\
     \hline
     \multicolumn{1}{c|}{\multirow{2}{*}{\textbf{TransGUNet \tiny{(Ours)}}}}     & 87.4 & 78.2 & 84.1 & 85.4 & 94.9 & \textit{2.7} \\ \cline{2-7}
     & \textbf{-1.0} & \textbf{-1.7} & \textbf{-2.3} & \textbf{-0.8} & \textbf{-1.0} & \textbf{-0.1} \\
    \hline
    \end{tabular}
    \caption{Segmentation results on \textbf{Breast Tumor Segmentation}. We train each model on BUSI \cite{al2020dataset} train dataset and evaluate on BUSI \cite{al2020dataset} and STU \cite{zhuang2019rdau} test datasets.}
    \label{tab:comparison_sota_ultrasound_other_metrics}
\end{table}

\begin{table}[h]
    \centering
    \scriptsize
    \setlength\tabcolsep{2pt} % default value: 6pt
    \renewcommand{\arraystretch}{0.75} % Tighter
    \begin{tabular}{c|cccccc}
    \hline
    \multicolumn{1}{c|}{\multirow{2}{*}{Method}} & \multicolumn{6}{c}{CVC-ClinicDB + Kvasir-SEG $\Rightarrow$ CVC-ClinicDB} \\ \cline{2-7}
     & DSC \scriptsize{$\uparrow$} & mIoU \scriptsize{$\uparrow$} & $F_{\beta}^{w}$ \scriptsize{$\uparrow$}  & $S_{\alpha}$ \scriptsize{$\uparrow$} & $E_{\phi}^{max}$ \scriptsize{$\uparrow$} & MAE \scriptsize{$\downarrow$} \\
     \hline
     UNet \tiny{(MICCAI2016)}       & 76.5 & 69.1 & 75.1 & 83.0 & 86.4 & 2.7 \\
     UNet++ \tiny{(DLMIA2018)}      & 79.7 & 73.6 & 79.4 & 85.1 & 88.3 & 2.2 \\
     CENet \tiny{(TMI2019)}         & 89.4 & 84.0 & 89.1 & 89.8 & 96.0 & 1.1 \\
     TransUNet \tiny{(arxiv2021)}   & 87.4 & 82.4 & 87.2 & 88.5 & 95.2 & 1.3 \\
     MSRFNet \tiny{(BHI2022)}       & 83.2 & 76.5 & 81.9 & 86.4 & 91.3 & 1.7 \\
     DCSAUNet \tiny{(CBM2023)}      & 80.6 & 73.7 & 79.6 & 84.9 & 89.9 & 2.4 \\
     M2SNet \tiny{(arxiv2023)}      & \textit{92.8} & \textit{88.2} & \textit{92.3} & 91.4 & \textit{97.7} & \textbf{\underline{0.7}} \\
     PVT-GCASCADE \tiny{(WACV2024)} & 92.2 & 87.6 & 91.6 & \textit{91.8} & 97.0 & \textit{0.8} \\
     CFATransUNet \tiny{(CBM2024)}  & 91.0 & 86.2 & 90.8 & 90.7 & 97.0 & \textit{0.8} \\
     MADGNet \tiny{(CVPR2024)}      & \textbf{\underline{93.9}} & \textbf{\underline{89.5}} & \textbf{\underline{93.6}} & \textbf{\underline{92.2}} & \textbf{\underline{98.5}} & \textbf{\underline{0.7}} \\
     \hline
     \multicolumn{1}{c|}{\multirow{2}{*}{\textbf{TransGUNet \tiny{(Ours)}}}}     & 92.3 & 87.7 & 91.8 & 91.6 & 97.1 & \textbf{\underline{0.7}} \\ \cline{2-7}
     & \textbf{-1.6} & \textbf{-1.8} & \textbf{-1.8} & \textbf{-0.6} & \textbf{-1.4} & \textbf{+0.0} \\
    \hline
    \multicolumn{1}{c|}{\multirow{2}{*}{Method}} & \multicolumn{6}{c}{CVC-ClinicDB + Kvasir-SEG $\Rightarrow$ Kvasir-SEG} \\ \cline{2-7}
    & DSC \scriptsize{$\uparrow$} & mIoU \scriptsize{$\uparrow$} & $F_{\beta}^{w}$ \scriptsize{$\uparrow$}  & $S_{\alpha}$ \scriptsize{$\uparrow$} & $E_{\phi}^{max}$ \scriptsize{$\uparrow$} & MAE \scriptsize{$\downarrow$} \\
    \hline
     UNet \tiny{(MICCAI2016)}       & 80.5 & 72.6 & 78.2 & 79.9 & 88.2 & 5.2 \\
     UNet++ \tiny{(DLMIA2018)}      & 84.3 & 77.4 & 83.1 & 82.1 & 90.5 & 4.6 \\
     CENet \tiny{(TMI2019)}         & 89.5 & 83.9 & 88.9 & 85.3 & 94.1 & 3.0 \\
     TransUNet \tiny{(arxiv2021)}   & 86.4 & 80.1 & 85.4 & 83.0 & 92.1 & 4.0 \\
     MSRFNet \tiny{(BHI2022)}       & 86.1 & 79.3 & 84.9 & 82.8 & 92.0 & 4.0 \\
     DCSAUNet \tiny{(CBM2023)}      & 82.6 & 75.2 & 80.7 & 81.3 & 90.1 & 4.9 \\
     M2SNet \tiny{(arxiv2023)}      & 90.2 & 85.1 & 89.4 & 85.6 & 94.6 & 2.7 \\
     PVT-GCASCADE \tiny{(WACV2024)} & 91.6 & 86.8 & 91.0 & 86.4 & \textit{96.3} & 2.4 \\
     CFATransUNet \tiny{(CBM2024)}  & \textit{92.1} & \textit{87.2} & \textit{91.7} & \textit{86.6} & 96.0 & \textit{2.2} \\
     MADGNet \tiny{(CVPR2024)}      & 90.7 & 85.3 & 89.9 & 85.6 & 94.7 & 3.1 \\
     \hline
     \multicolumn{1}{c|}{\multirow{2}{*}{\textbf{TransGUNet \tiny{(Ours)}}}}     & \textbf{\underline{93.1}} & \textbf{\underline{88.4}} & \textbf{\underline{92.6}} & \textbf{\underline{87.1}} & \textbf{\underline{96.6}} & \textbf{\underline{2.0}} \\ \cline{2-7}
     & \textbf{+1.0} & \textbf{+1.2} & \textbf{+0.9} & \textbf{+0.5} & \textbf{+0.3} & \textbf{+0.2} \\
    \hline
    \multicolumn{1}{c|}{\multirow{2}{*}{Method}} & \multicolumn{6}{c}{CVC-ClinicDB + Kvasir-SEG $\Rightarrow$ CVC-300} \\ \cline{2-7}
    & DSC \scriptsize{$\uparrow$} & mIoU \scriptsize{$\uparrow$} & $F_{\beta}^{w}$ \scriptsize{$\uparrow$}  & $S_{\alpha}$ \scriptsize{$\uparrow$} & $E_{\phi}^{max}$ \scriptsize{$\uparrow$} & MAE \scriptsize{$\downarrow$} \\
    \hline
     UNet \tiny{(MICCAI2016)}       & 66.1 & 58.5 & 65.0 & 79.7 & 80.0 & 1.7 \\
     UNet++ \tiny{(DLMIA2018)}      & 64.4 & 58.4 & 63.7 & 79.5 & 77.4 & 1.8 \\
     CENet \tiny{(TMI2019)}         & 85.4 & 78.2 & 84.2 & 90.2 & 94.0 & 0.8 \\
     TransUNet \tiny{(arxiv2021)}   & 85.0 & 77.3 & 83.1 & 89.4 & 95.2 & 1.1 \\
     MSRFNet \tiny{(BHI2022)}       & 72.3 & 65.4 & 71.2 & 83.5 & 84.6 & 1.4 \\
     DCSAUNet \tiny{(CBM2023)}      & 68.9 & 59.8 & 66.3 & 81.1 & 83.8 & 2.0 \\
     M2SNet \tiny{(arxiv2023)}      & \textit{89.8} & \textbf{\underline{83.2}} & \textbf{\underline{88.3}} & \textit{93.0} & \textbf{\underline{97.0}} & \textbf{\underline{0.6}} \\
     PVT-GCASCADE \tiny{(WACV2024)} & 88.2 & 81.0 & 85.9 & 92.0 & 95.6 & 0.9 \\
     CFATransUNet \tiny{(CBM2024)}  & 89.1 & 82.4 & 87.4 & 92.5 & 96.6 & \textit{0.7} \\
     MADGNet \tiny{(CVPR2024)}      & 87.4 & 79.9 & 84.5 & 92.0 & 94.7 & 0.9 \\
     \hline
     \multicolumn{1}{c|}{\multirow{2}{*}{\textbf{TransGUNet \tiny{(Ours)}}}}     & \textbf{\underline{90.0}} & \textit{83.1} & \textit{88.0} & \textbf{\underline{93.2}} & \textit{96.8} & \textit{0.7} \\ \cline{2-7}
     & \textbf{+0.2} & \textbf{-0.1} & \textbf{-0.3} & \textbf{+0.2} & \textbf{+0.2} & \textbf{-0.1} \\
    \hline
    \multicolumn{1}{c|}{\multirow{2}{*}{Method}} & \multicolumn{6}{c}{CVC-ClinicDB + Kvasir-SEG $\Rightarrow$ CVC-ColonDB} \\ \cline{2-7}
    & DSC \scriptsize{$\uparrow$} & mIoU \scriptsize{$\uparrow$} & $F_{\beta}^{w}$ \scriptsize{$\uparrow$}  & $S_{\alpha}$ \scriptsize{$\uparrow$} & $E_{\phi}^{max}$ \scriptsize{$\uparrow$} & MAE \scriptsize{$\downarrow$} \\
    \hline
     UNet \tiny{(MICCAI2016)}       & 56.8 & 49.0 & 55.9 & 72.6 & 73.9 & 5.1 \\
     UNet++ \tiny{(DLMIA2018)}      & 57.5 & 50.2 & 56.6 & 73.3 & 73.9 & 5.0 \\
     CENet \tiny{(TMI2019)}         & 65.9 & 59.2 & 65.8 & 77.7 & 79.5 & 4.0 \\
     TransUNet \tiny{(arxiv2021)}   & 63.7 & 58.4 & 62.8 & 75.8 & 79.3 & 4.8 \\
     MSRFNet \tiny{(BHI2022)}       & 61.5 & 54.8 & 60.8 & 75.4 & 76.1 & 4.5 \\
     DCSAUNet \tiny{(CBM2023)}      & 57.8 & 49.3 & 54.9 & 73.3 & 76.0 & 5.8 \\
     M2SNet \tiny{(arxiv2023)}      & 75.8 & 68.5 & 73.7 & 84.2 & 86.9 & 3.8 \\
     PVT-GCASCADE \tiny{(WACV2024)} & \textit{79.5} & \textit{71.6} & \textit{77.7} & \textit{84.5} & \textit{89.6} & 3.8 \\
     CFATransUNet \tiny{(CBM2024)}  & 78.0 & 70.3 & 77.3 & 83.7 & 88.7 & 3.5 \\
     MADGNet \tiny{(CVPR2024)}      & 77.5 & 69.7 & 76.2 & 83.3 & 88.0 & \textit{3.2} \\
     \hline
     \multicolumn{1}{c|}{\multirow{2}{*}{\textbf{TransGUNet \tiny{(Ours)}}}}     & \textbf{\underline{82.0}} & \textbf{\underline{74.1}} & \textbf{\underline{80.5}} & \textbf{\underline{85.6}} & \textbf{\underline{92.0}} & \textbf{\underline{3.0}} \\ \cline{2-7}
     & \textbf{+2.5} & \textbf{+2.5} & \textbf{+2.8} & \textbf{+1.1} & \textbf{+2.4} & \textbf{+0.2} \\
    \hline
    \multicolumn{1}{c|}{\multirow{2}{*}{Method}} & \multicolumn{6}{c}{CVC-ClinicDB + Kvasir-SEG $\Rightarrow$ ETIS} \\ \cline{2-7}
    & DSC \scriptsize{$\uparrow$} & mIoU \scriptsize{$\uparrow$} & $F_{\beta}^{w}$ \scriptsize{$\uparrow$}  & $S_{\alpha}$ \scriptsize{$\uparrow$} & $E_{\phi}^{max}$ \scriptsize{$\uparrow$} & MAE \scriptsize{$\downarrow$} \\
    \hline
     UNet \tiny{(MICCAI2016)}       & 41.6 & 35.4 & 39.5 & 67.2 & 61.7 & 2.7 \\
     UNet++ \tiny{(DLMIA2018)}      & 39.1 & 34.0 & 38.3 & 65.8 & 59.3 & 2.7 \\
     CENet \tiny{(TMI2019)}         & 57.0 & 51.4 & 56.0 & 74.9 & 73.8 & 2.2 \\
     TransUNet \tiny{(arxiv2021)}   & 50.1 & 44.0 & 48.8 & 70.7 & 68.7 & 2.6 \\
     MSRFNet \tiny{(BHI2022)}       & 38.3 & 33.7 & 36.9 & 66.0 & 58.4 & 3.6 \\
     DCSAUNet \tiny{(CBM2023)}      & 43.0 & 36.1 & 40.5 & 67.9 & 69.3 & 4.1 \\
     M2SNet \tiny{(arxiv2023)}      & 74.9 & 67.8 & 71.2 & 84.6 & 87.2 & 1.7 \\
     PVT-GCASCADE \tiny{(WACV2024)} & \textit{79.5} & \textit{71.6} & \textit{76.6} & \textit{86.3} & \textit{90.0} & 1.7 \\
     CFATransUNet \tiny{(CBM2024)}  & 77.0 & 69.5 & 75.0 & 84.4 & 88.6 & \textit{1.5} \\
     MADGNet \tiny{(CVPR2024)}      & 77.0 & 69.7 & 75.3 & 84.6 & 88.4 & 1.6 \\
     \hline
     \multicolumn{1}{c|}{\multirow{2}{*}{\textbf{TransGUNet \tiny{(Ours)}}}}     & \textbf{\underline{81.3}} & \textbf{\underline{73.1}} & \textbf{\underline{76.8}} & \textbf{\underline{87.6}} & \textbf{\underline{91.5}} & \textbf{\underline{1.4}} \\ \cline{2-7}
     & \textbf{+1.8} & \textbf{+1.5} & \textbf{+0.2} & \textbf{+1.3} & \textbf{+1.5} & \textbf{+0.1} \\
     \hline
    \end{tabular}
    \caption{Segmentation results on \textbf{Polyp Segmentation}. We train each model on CVC-ClinicDB \cite{bernal2015wm} + Kvasir-SEG \cite{jha2020kvasir} train dataset and evaluate on CVC-ClinicDB \cite{bernal2015wm}, Kvasir-SEG \cite{jha2020kvasir}, CVC-300 \cite{vazquez2017benchmark}, CVC-ColonDB \cite{tajbakhsh2015automated}, and ETIS \cite{silva2014toward} test datasets.}
    \label{tab:comparison_sota_colonoscopy_other_metrics}
\end{table}

\begin{table*}[t]
    \centering
    \scriptsize
    \setlength\tabcolsep{9pt} % default value: 6pt
    \renewcommand{\arraystretch}{0.6} % Tighter
    \begin{tabular}{c|c|cccccccc}
    \hline
    \multicolumn{1}{c|}{\multirow{2}{*}{Method}} & \multicolumn{9}{c}{Synapse $\Rightarrow$ Synapse} \\ \cline{2-10}
     & Average & Aorta & Gallbladder & Left Kidney & Right Kidney & Liver & Pancreas & Spleen & Stomach \\
    \hline
    UNet \tiny{(MICCAI2016)}     & 69.8 & 84.3 & 44.1 & 73.0 & 71.9 & 91.8 & 46.0 & 79.2 & 68.0 \\
    UNet++ \tiny{(DLMIA2018)}    & 79.3 & 87.4 & 64.8 & 81.2 & 77.9 & 94.3 & 60.8 & \textbf{\underline{89.6}} & 78.7 \\
    CENet \tiny{(TMI2019)}       & 75.2 & 81.3 & 55.1 & 80.3 & 77.1 & 94.0 & 47.8 & 87.2 & 78.4 \\
    TransUNet \tiny{(arxiv2021)} & 77.5 & \textbf{\underline{87.2}} & 63.1 & 81.9 & 77.0 & 94.1 & 55.9 & 85.1 & 75.6 \\
    MSRFNet \tiny{(BHI2022)}     & 77.2 & 87.6 & 58.3 & 82.8 & 73.6 & 94.6 & 57.3 & 88.3 & 75.4 \\
    DCSAUNet \tiny{(CBM2023)}    & 71.0 & 81.4 & 51.9 & 75.1 & 68.8 & 92.7 & 45.4 & 84.6 & 68.0 \\
    M2SNet \tiny{(arxiv2023)}    & 77.1 & 84.9 & 55.6 & 78.2 & 76.1 & 94.9 & 57.4 & 89.1 & 80.3 \\
    PVT-GCASCADE \tiny{(WACV2024)} & 78.1 & 85.1 & 57.0 & 82.2 & 79.3 & 94.9 & 55.2 & 88.7 & 82.7 \\
    CFATransUNet \tiny{(CBM2024)} & \textit{80.5} & 85.8 & \textbf{\underline{65.9}} & \textit{85.8} & \textit{81.7} & \textit{95.2} & 59.4 & 89.0 & 81.4 \\
    MADGNet \tiny{(CVPR2024)}    & 79.3 & 85.2 & 59.9 & 85.4 & 78.0 & 94.5 & \textbf{\underline{60.5}} & 88.2 & \textit{83.0} \\
    \hline
    \multicolumn{1}{c|}{\multirow{2}{*}{\textbf{TransGUNet \tiny{(Ours)}}}} & \textbf{\underline{80.9}} & \textit{86.6} & \textit{61.0} & \textbf{\underline{87.0}} & \textbf{\underline{83.8}} & \textbf{\underline{95.3}} & \textit{59.7} & \textit{89.4} & \textbf{\underline{84.6}} \\ \cline{2-10}
    & \textbf{+0.4} & \textbf{-0.6} & \textbf{-4.9} & \textbf{+1.2} & \textbf{+2.1} & \textbf{+0.1} & \textbf{-0.8} & \textbf{-0.2} & \textbf{+1.6} \\
    \hline
    \multicolumn{1}{c|}{\multirow{2}{*}{Method}} & \multicolumn{9}{c}{Synapse $\Rightarrow$ AMOS-CT} \\ \cline{2-10}
     & Average & Aorta & Gallbladder & Left Kidney & Right Kidney & Liver & Pancreas & Spleen & Stomach \\
    \hline
    UNet \tiny{(MICCAI2016)}     & 56.3 & 62.8 & 43.2 & 50.3 & 55.3 & 84.6 & 26.8 & 68.9 & 58.8 \\
    UNet++ \tiny{(DLMIA2018)}    & 67.5 & 67.0 & 62.7 & 65.2 & 65.9 & 88.2 & 42.3 & 78.1 & 70.4 \\
    CENet \tiny{(TMI2019)}       & 67.9 & 72.4 & 52.0 & 69.5 & 72.3 & 89.3 & 40.8 & 78.9 & 68.2 \\
    TransUNet \tiny{(arxiv2021)} & 68.3 & 75.4 & 60.4 & 65.3 & 68.5 & 90.3 & 38.2 & 78.6 & 69.8 \\
    MSRFNet \tiny{(BHI2022)}     & 61.8 & 70.7 & 54.1 & 57.8 & 54.6 & 87.4 & 31.7 & 75.8 & 62.1 \\
    DCSAUNet \tiny{(CBM2023)}    & 45.7 & 43.4 & 30.6 & 32.5 & 39.5 & 85.1 & 18.7 & 67.6 & 48.0 \\
    M2SNet \tiny{(arxiv2023)}    & 69.6 & 74.4 & 60.9 & 66.7 & 69.5 & 90.8 & 41.8 & 79.6 & 73.3 \\
    PVT-GCASCADE \tiny{(WACV2024)} & 69.3 & 70.5 & 56.6 & 64.6 & 71.9 & \textit{91.7} & 42.0 & 82.0 & 75.4 \\
    CFATransUNet \tiny{(CBM2024)} & 68.0 & 75.0 & 51.8 & 67.5 & 74.5 & 88.2 & 43.6 & 78.2 & 65.5 \\
    MADGNet \tiny{(CVPR2024)}    & \textit{74.9} & \textit{79.4} & \textit{63.3} & \textit{77.6} & \textit{75.6} & 90.7 & \textbf{\underline{51.9}} & \textit{83.2} & \textit{77.3} \\
    \hline
    \multicolumn{1}{c|}{\multirow{2}{*}{\textbf{TransGUNet \tiny{(Ours)}}}} & \textbf{\underline{76.5}} & \textbf{\underline{81.0}} & \textbf{\underline{63.4}} & \textbf{\underline{79.1}} & \textbf{\underline{82.3}} & \textbf{\underline{91.9}} & \textit{51.4} & \textbf{\underline{84.7}} & \textbf{\underline{78.3}} \\ \cline{2-10}
    & \textbf{+1.6} & \textbf{+1.6} & \textbf{+0.1} & \textbf{+1.5} & \textbf{+6.7} & \textbf{+0.2} & \textbf{-0.5} & \textbf{+1.5} & \textbf{+1.0} \\
    \hline
    \multicolumn{1}{c|}{\multirow{2}{*}{Method}} & \multicolumn{9}{c}{Synapse $\Rightarrow$ AMOS-MRI} \\ \cline{2-10}
     & Average & Aorta & Gallbladder & Left Kidney & Right Kidney & Liver & Pancreas & Spleen & Stomach \\
    \hline
    UNet \tiny{(MICCAI2016)}     & 8.3 & 6.1 & 5.6 & 3.0 & 14.0 & 26.1 & 1.2 & 4.9 & 5.1 \\
    UNet++ \tiny{(DLMIA2018)}    & 6.0 & 9.7 & 6.7 & 7.9 & 2.7 & 7.1 & 1.5 & 4.6 & 7.6 \\
    CENet \tiny{(TMI2019)}       & 14.5 & 13.4 & 6.2 & 21.8 & 34.7 & 14.6 & \textit{5.7} & 9.5 & 9.8 \\
    TransUNet \tiny{(arxiv2021)} & 9.1 & 10.2 & 8.0 & 18.9 & 21.1 & 7.5 & 1.9 & 2.7 & 2.8 \\
    MSRFNet \tiny{(BHI2022)}     & 6.5 & 2.6 & 6.9 & 3.7 & 3.0 & 26.9 & 1.4 & 2.4 & 4.9 \\
    DCSAUNet \tiny{(CBM2023)}    & 1.7 & 0.9 & 1.7 & 0.0 & 0.8 & 7.8 & 0.0 & 1.8 & 0.4 \\
    M2SNet \tiny{(arxiv2023)}    & 22.0 & 29.1 & 8.4 & 35.0 & 34.5 & 29.0 & 5.1 & 18.5 & 16.7 \\
    PVT-GCASCADE \tiny{(WACV2024)} & 32.8 & 26.8 & 11.6 & \textit{44.1} & \textit{47.8} & \textit{64.3} & 3.9 & 38.3 & 25.5 \\
    CFATransUNet \tiny{(CBM2024)} & \textit{35.8} & \textit{30.0} & \textit{16.9} & 41.8 & 47.7 & 58.6 & 5.4 & \textit{57.0} & \textit{29.3} \\
    MADGNet \tiny{(CVPR2024)}    & 14.8 & 19.5 & 6.5 & 32.3 & 16.0 & 11.5 & 2.3 & 13.4 & 17.2 \\
    \hline
    \multicolumn{1}{c|}{\multirow{2}{*}{\textbf{TransGUNet \tiny{(Ours)}}}} & \textbf{\underline{47.2}} & \textbf{\underline{52.9}} & \textbf{\underline{20.5}} & \textbf{\underline{51.5}} & \textbf{\underline{59.9}} & \textbf{\underline{73.3}} & \textbf{\underline{17.0}} & \textbf{\underline{63.7}} & \textbf{\underline{38.8}} \\ \cline{2-10}
    & \textbf{+11.4} & \textbf{+22.9} & \textbf{+3.9} & \textbf{+7.4} & \textbf{+12.1} & \textbf{+9.0} & \textbf{+11.3} & \textbf{+6.7} & \textbf{+9.5} \\
    \hline
    \end{tabular}
    \caption{Segmentation results on \textbf{Multi-organ Segmentation} with \textit{DSC}. We train each model on Synapse \cite{Synapse_dataset} train dataset and evaluate on Synapse \cite{Synapse_dataset} and AMOS-CT/MRI \cite{ji2022amos} test datasets.}
    \label{tab:comparison_sota_multiorgan_dsc_metrics}
\end{table*}

\begin{table*}[t]
    \centering
    \scriptsize
    \setlength\tabcolsep{9pt} % default value: 6pt
    \renewcommand{\arraystretch}{0.6} % Tighter
    \begin{tabular}{c|c|cccccccc}
    \hline
    \multicolumn{1}{c|}{\multirow{2}{*}{Method}} & \multicolumn{9}{c}{Synapse $\Rightarrow$ Synapse} \\ \cline{2-10}
     & Average & Aorta & Gallbladder & Left Kidney & Right Kidney & Liver & Pancreas & Spleen & Stomach \\
    \hline
    UNet \tiny{(MICCAI2016)}     & 58.9 & 73.1 & 35.3 & 62.5 & 60.8 & 85.1 & 32.4 & 68.2 & 54.1 \\
    UNet++ \tiny{(DLMIA2018)}    & 69.8 & \textbf{\underline{78.0}} & \textbf{\underline{53.3}} & 72.6 & 69.2 & 89.5 & \textit{46.4} & \textbf{\underline{82.7}} & 67.0 \\
    CENet \tiny{(TMI2019)}       & 64.6 & 68.9 & 43.4 & 71.1 & 67.2 & 88.9 & 33.1 & 78.4 & 65.9 \\
    TransUNet \tiny{(arxiv2021)} & 67.2 & 71.8 & 46.3 & 73.2 & 67.6 & 89.4 & \textbf{\underline{48.0}} & 80.3 & 60.7 \\
    MSRFNet \tiny{(BHI2022)}     & 67.6 & 78.1 & 47.7 & 74.1 & 63.8 & 89.9 & 43.2 & 80.9 & 62.8 \\
    DCSAUNet \tiny{(CBM2023)}    & 59.8 & 68.8 & 41.8 & 64.4 & 57.4 & 86.5 & 32.2 & 73.9 & 53.2 \\
    M2SNet \tiny{(arxiv2023)}    & 67.5 & 74.1 & 45.5 & 70.4 & 67.9 & 90.3 & 42.0 & 81.2 & 68.7 \\
    PVT-GCASCADE \tiny{(WACV2024)} & 68.9 & 75.0 & 47.5 & 74.0 & 70.1 & 90.4 & 39.9 & 80.5 & 73.5 \\
    CFATransUNet \tiny{(CBM2024)} & \textit{70.4} & 76.2 & 47.7 & 75.4 & \textit{73.0} & \textit{90.9} & \textbf{\underline{48.0}} & 80.8 & 71.3 \\
    MADGNet \tiny{(CVPR2024)}    & 69.8 & 73.0 & \textit{48.7} & \textit{75.8} & 71.9 & 90.4 & 45.7 & 80.2 & \textit{72.9} \\
    \hline
    \multicolumn{1}{c|}{\multirow{2}{*}{\textbf{TransGUNet \tiny{(Ours)}}}} & \textbf{\underline{71.4}} & \textit{76.4} & \textit{48.7} & \textbf{\underline{79.4}} & \textbf{\underline{76.0}} & \textbf{\underline{91.1}} & 43.9 & \textit{81.6} & \textbf{\underline{74.0}} \\ \cline{2-10}
    & \textbf{+0.9} & \textbf{-1.6} & \textbf{-4.6} & \textbf{+3.6} & \textbf{+3.0} & \textbf{+0.2} & \textbf{-4.1} & \textbf{-1.1} & \textbf{+1.1} \\
    \hline
    \multicolumn{1}{c|}{\multirow{2}{*}{Method}} & \multicolumn{9}{c}{Synapse $\Rightarrow$ AMOS-CT} \\ \cline{2-10}
     & Average & Aorta & Gallbladder & Left Kidney & Right Kidney & Liver & Pancreas & Spleen & Stomach \\
    \hline
    UNet \tiny{(MICCAI2016)}     & 44.8 & 49.3 & 33.2 & 39.0 & 42.3 & 74.9 & 17.2 & 57.0 & 45.5 \\
    UNet++ \tiny{(DLMIA2018)}    & 56.6 & 54.4 & \textit{51.9} & 54.4 & 54.7 & 80.2 & 30.5 & 68.3 & 58.1 \\
    CENet \tiny{(TMI2019)}       & 56.5 & 59.7 & 40.0 & 58.5 & 61.1 & 81.9 & 27.3 & 68.7 & 54.7 \\
    TransUNet \tiny{(arxiv2021)} & 57.7 & 63.6 & 49.0 & 55.0 & 57.4 & 83.3 & 26.9 & 69.2 & 57.5 \\
    MSRFNet \tiny{(BHI2022)}     & 51.3 & 58.4 & 43.7 & 47.3 & 44.2 & 79.1 & 22.5 & 65.4 & 49.7 \\
    DCSAUNet \tiny{(CBM2023)}    & 36.3 & 32.4 & 22.9 & 25.0 & 31.4 & 75.4 & 11.8 & 56.4 & 34.9 \\
    M2SNet \tiny{(arxiv2023)}    & 58.5 & 62.0 & 49.0 & 55.7 & 58.2 & 84.2 & 28.6 & 69.9 & 60.6 \\
    PVT-GCASCADE \tiny{(WACV2024)} & 58.5 & 57.5 & 45.0 & 54.3 & 60.9 & \textit{85.6} & 28.9 & 72.8 & 62.9 \\
    CFATransUNet \tiny{(CBM2024)} & 56.7 & 62.4 & 40.9 & 56.0 & 63.5 & 80.2 & 30.4 & 67.7 & 52.5 \\
    MADGNet \tiny{(CVPR2024)}    & \textit{64.4} & \textit{68.2} & \textbf{\underline{52.0}} & \textit{68.3} & \textit{65.5} & 84.1 & \textbf{\underline{37.6}} & \textit{74.2} & \textit{65.6} \\
    \hline
    \multicolumn{1}{c|}{\multirow{2}{*}{\textbf{TransGUNet \tiny{(Ours)}}}} & \textbf{\underline{66.2}} & \textbf{\underline{69.8}} & 50.6 & \textbf{\underline{70.1}} & \textbf{\underline{73.3}} & \textbf{\underline{86.0}} & \textit{36.8} & \textbf{\underline{76.1}} & \textbf{\underline{66.5}} \\ \cline{2-10}
    & \textbf{+1.7} & \textbf{+1.6} & \textbf{-1.4} & \textbf{+1.8} & \textbf{+7.8} & \textbf{+0.4} & \textbf{-0.8} & \textbf{+1.9} & \textbf{+0.9} \\
    \hline
    \multicolumn{1}{c|}{\multirow{2}{*}{Method}} & \multicolumn{9}{c}{Synapse $\Rightarrow$ AMOS-MRI} \\ \cline{2-10}
     & Average & Aorta & Gallbladder & Left Kidney & Right Kidney & Liver & Pancreas & Spleen & Stomach \\
    \hline
    UNet \tiny{(MICCAI2016)}     & 5.3 & 3.5 & 5.3 & 1.7 & 8.8 & 16.6 & 0.6 & 3.2 & 2.8 \\
    UNet++ \tiny{(DLMIA2018)}    & 3.8 & 5.9 & 6.0 & 5.0 & 1.6 & 3.8 & 0.8 & 2.9 & 4.2 \\
    CENet \tiny{(TMI2019)}       & 9.0 & 8.1 & 5.6 & 14.1 & 22.3 & 8.0 & \textit{3.1} & 5.5 & 5.3 \\
    TransUNet \tiny{(arxiv2021)} & 5.8 & 5.9 & 6.9 & 12.2 & 13.3 & 4.1 & 1.0 & 1.8 & 1.5 \\
    MSRFNet \tiny{(BHI2022)}     & 4.2 & 2.6 & 6.9 & 3.7 & 3.0 & 8.5 & 1.4 & 2.4 & 4.9 \\
    DCSAUNet \tiny{(CBM2023)}    & 1.1 & 0.5 & 1.7 & 0.0 & 0.4 & 4.8 & 0.0 & 1.0 & 0.2 \\
    M2SNet \tiny{(arxiv2023)}    & 14.7 & 20.2 & 7.0 & 24.7 & 24.1 & 17.8 & 2.8 & 11.4 & 9.8 \\
    PVT-GCASCADE \tiny{(WACV2024)} & 24.3 & 17.4 & 9.4 & \textit{33.6} & 35.8 & \textit{50.4} & 2.2 & 29.6 & 15.9 \\
    CFATransUNet \tiny{(CBM2024)} & \textit{25.9} & \textit{21.1} & \textit{13.1} & 31.0 & \textit{36.1} & 42.6 & 2.9 & \textit{41.9} & \textit{18.5} \\
    MADGNet \tiny{(CVPR2024)}    & 9.8 & 12.5 & 5.8 & 22.1 & 10.3 & 6.5 & 1.2 & 9.5 & 10.2 \\
    \hline
    \multicolumn{1}{c|}{\multirow{2}{*}{\textbf{TransGUNet \tiny{(Ours)}}}} & \textbf{\underline{35.6}} & \textbf{\underline{39.0}} & \textbf{\underline{14.7}} & \textbf{\underline{40.5}} & \textbf{\underline{46.1}} & \textbf{\underline{59.5}} & \textbf{\underline{9.5}} & \textbf{\underline{49.3}} & \textbf{\underline{25.8}} \\ \cline{2-10}
    & \textbf{+9.6} & \textbf{+17.9} & \textbf{+1.6} & \textbf{+6.9} & \textbf{+10.0} & \textbf{+9.1} & \textbf{+6.4} & \textbf{+7.4} & \textbf{+7.3} \\
    \hline
    \end{tabular}
    \caption{Segmentation results on \textbf{Multi-organ Segmentation} with \textit{mIoU}. We train each model on Synapse \cite{Synapse_dataset} train dataset and evaluate on Synapse \cite{Synapse_dataset} and AMOS-CT/MRI \cite{ji2022amos} test datasets.}
    \label{tab:comparison_sota_multiorgan_mIoU_metrics}
\end{table*}

\begin{table*}
    \centering
    \scriptsize
    \setlength\tabcolsep{9pt} % default value: 6pt
    \renewcommand{\arraystretch}{0.6} % Tighter
    \begin{tabular}{c|c|cccccccc}
    \hline
    \multicolumn{1}{c|}{\multirow{2}{*}{Method}} & \multicolumn{9}{c}{Synapse $\Rightarrow$ Synapse} \\ \cline{2-10}
     & Average & Aorta & Gallbladder & Left Kidney & Right Kidney & Liver & Pancreas & Spleen & Stomach \\
    \hline
    UNet \tiny{(MICCAI2016)}     & 43.8 & 15.1 & 46.8 & 45.6 & 90.6 & 28.3 & 21.5 & 76.6 & 26.2 \\
    UNet++ \tiny{(DLMIA2018)}    & 34.9 & 4.3 & 52.5 & 51.2 & 62.2 & 20.4 & 11.4 & 61.5 & 15.6 \\
    CENet \tiny{(TMI2019)}       & 39.8 & 13.5 & 86.5 & 53.8 & 86.8 & 16.4 & 13.8 & 21.6 & 25.9 \\
    TransUNet \tiny{(arxiv2021)} & 34.3 & 10.7 & 53.8 & 37.0 & 76.5 & \textit{15.0} & \textbf{\underline{8.1}} & 55.3 & 17.7 \\
    MSRFNet \tiny{(BHI2022)}     & 38.2 & 8.8 & 56.1 & 59.5 & 66.1 & 26.8 & 13.3 & 36.2 & 38.7 \\
    DCSAUNet \tiny{(CBM2023)}    & 38.6 & 10.4 & 80.0 & 54.9 & 49.4 & 26.0 & 15.7 & 52.4 & 20.2 \\
    M2SNet \tiny{(arxiv2023)}    & 25.4 & \textit{5.7} & \textit{27.7} & 49.6 & 48.0 & 22.4 & 10.9 & \textit{25.7} & 13.5 \\
    PVT-GCASCADE \tiny{(WACV2024)} & \textbf{\underline{18.8}} & \textbf{\underline{5.6}} & \textbf{\underline{20.2}} & 25.5 & \textit{36.0} & 17.8 & 12.9 & \textit{20.8} & \textit{11.6} \\
    CFATransUNet \tiny{(CBM2024)} & \textbf{\underline{18.8}} & \textbf{\underline{5.6}} & 29.0 & \textit{16.0} & \textbf{\underline{34.6}} & 23.1 & 10.8 & \textbf{\underline{20.1}} & \textbf{\underline{10.8}} \\
    MADGNet \tiny{(CVPR2024)}    & 24.9 & \textit{5.7} & 29.8 & 40.4 & 44.1 & \textbf{\underline{14.0}} & 13.2 & 35.0 & 17.0 \\
    \hline
    \multicolumn{1}{c|}{\multirow{2}{*}{\textbf{TransGUNet \tiny{(Ours)}}}} & \textit{23.4} & 7.1 & 33.6 & \textbf{\underline{14.6}} & 48.3 & 22.4 & \textit{10.5} & 39.8 & \textbf{\underline{10.8}} \\ \cline{2-10}
    & \textbf{-4.6} & \textbf{-1.5} & \textbf{-13.4} & \textbf{+1.4} & \textbf{-13.7} & \textbf{-8.4} & \textbf{-2.4} & \textbf{-19.7} & \textbf{+0.0} \\
    \hline
    \multicolumn{1}{c|}{\multirow{2}{*}{Method}} & \multicolumn{9}{c}{Synapse $\Rightarrow$ AMOS-CT} \\ \cline{2-10}
     & Average & Aorta & Gallbladder & Left Kidney & Right Kidney & Liver & Pancreas & Spleen & Stomach \\
    \hline
    UNet \tiny{(MICCAI2016)}     & 72.1 & 39.5 & 83.5 & 94.1 & 132.4 & 36.8 & 24.5 & 129.7 & 36.5 \\
    UNet++ \tiny{(DLMIA2018)}    & 43.9 & 19.9 & 55.0 & 73.4 & \textit{45.7} & 25.5 & \textit{16.9} & 90.5 & 24.0 \\
    CENet \tiny{(TMI2019)}       & 52.2 & 13.5 & 97.5 & 53.3 & 106.2 & 24.1 & 27.0 & 55.5 & 40.1 \\
    TransUNet \tiny{(arxiv2021)} & 39.5 & 15.4 & 46.2 & \textit{46.1} & 102.3 & 19.6 & 21.6 & 40.1 & 25.0 \\
    MSRFNet \tiny{(BHI2022)}     & 47.0 & 20.2 & 31.8 & 96.6 & 49.2 & 22.6 & 26.3 & 90.9 & 38.4 \\
    DCSAUNet \tiny{(CBM2023)}    & 50.5 & 28.7 & 92.1 & 63.1 & 82.9 & 23.4 & 34.6 & 44.7 & 34.5 \\
    M2SNet \tiny{(arxiv2023)}    & 41.2 & \textbf{\underline{10.0}} & \textbf{\underline{30.8}} & 73.7 & 72.9 & 22.0 & 20.3 & 69.8 & 29.7 \\
    PVT-GCASCADE \tiny{(WACV2024)} & \textit{27.5} & 12.8 & 35.9 & 46.5 & \textbf{\underline{40.2}} & \textit{13.9} & 21.9 & \textit{27.7} & \textit{20.9} \\
    CFATransUNet \tiny{(CBM2024)} & 46.3 & 14.9 & 81.0 & 84.3 & 54.1 & 27.1 & 24.7 & 59.0 & 24.9 \\
    MADGNet \tiny{(CVPR2024)}    & 31.2 & \textbf{\underline{10.0}} & 34.7 & 58.5 & 55.7 & 16.8 & 18.6 & 29.2 & 26.2 \\
    \hline
    \multicolumn{1}{c|}{\multirow{2}{*}{\textbf{TransGUNet \tiny{(Ours)}}}} & \textbf{\underline{24.4}} & \textit{12.5} & \textit{31.1} & \textbf{\underline{29.6}} & 46.3 & \textbf{\underline{13.5}} & \textbf{\underline{15.7}} & \textbf{\underline{27.1}} & \textbf{\underline{19.1}} \\ \cline{2-10}
    & \textbf{+1.8} & \textbf{-2.5} & \textbf{-0.3} & \textbf{+16.5} & \textbf{-6.1} & \textbf{+0.4} & \textbf{+1.2} & \textbf{+0.6} & \textbf{+1.8} \\
    \hline
    \multicolumn{1}{c|}{\multirow{2}{*}{Method}} & \multicolumn{9}{c}{Synapse $\Rightarrow$ AMOS-MRI} \\ \cline{2-10}
     & Average & Aorta & Gallbladder & Left Kidney & Right Kidney & Liver & Pancreas & Spleen & Stomach \\
    \hline
    UNet \tiny{(MICCAI2016)}     & 190.3 & 144.3 & 198.4 & 143.0 & 187.3 & 198.4 & 105.9 & 322.0 & 223.2 \\
    UNet++ \tiny{(DLMIA2018)}    & 169.3 & 68.9 & 248.7 & \textit{109.0} & 134.2 & 207.7 & 113.0 & 264.1 & 209.0 \\
    CENet \tiny{(TMI2019)}       & 182.8 & 95.9 & 242.6 & 171.4 & 143.1 & 216.3 & 157.0 & 267.1 & 168.9 \\
    TransUNet \tiny{(arxiv2021)} & 162.3 & 100.6 & 256.8 & 114.3 & 160.9 & 199.3 & 116.4 & 237.4 & 112.8 \\
    MSRFNet \tiny{(BHI2022)}     & 166.3 & 102.4 & 166.1 & 144.0 & 137.0 & 178.6 & 136.9 & 210.9 & 254.4 \\
    DCSAUNet \tiny{(CBM2023)}    & 147.2 & 124.1 & \textbf{\underline{108.7}} & 230.1 & 183.4 & 97.1 & 86.4 & 194.8 & 152.6 \\
    M2SNet \tiny{(arxiv2023)}    & 131.7 & \textit{55.4} & 148.8 & 130.7 & 136.9 & 153.0 & 90.0 & 203.3 & 135.6 \\
    PVT-GCASCADE \tiny{(WACV2024)} & \textit{110.5} & 62.4 & \textit{115.4} & 127.4 & 141.0 & \textbf{\underline{91.0}} & 74.5 & \textit{150.0} & 122.1 \\
    CFATransUNet \tiny{(CBM2024)} & 115.6 & 69.2 & 146.2 & 156.3 & \textit{137.0} & 108.3 & \textbf{\underline{57.4}} & 171.3 & \textbf{\underline{79.2}} \\
    MADGNet \tiny{(CVPR2024)}    & 128.2 & 71.7 & 121.0 & 148.4 & 144.6 & 149.9 & 79.9 & 199.5 & 110.7 \\
    \hline
    \multicolumn{1}{c|}{\multirow{2}{*}{\textbf{TransGUNet \tiny{(Ours)}}}} & \textbf{\underline{98.7}} & \textbf{\underline{45.2}} & 123.4 & \textbf{\underline{105.2}} & \textbf{\underline{136.8}} & \textit{91.2} & \textit{60.7} & \textbf{\underline{142.0}} & \textit{84.9} \\ \cline{2-10}
    & \textbf{+11.8} & \textbf{+10.2} & \textbf{-14.7} & \textbf{+3.8} & \textbf{+0.2} & \textbf{-0.2} & \textbf{-3.3} & \textbf{+8.0} & \textbf{-5.7} \\
    \hline
    \end{tabular}
    \caption{Segmentation results on \textbf{Multi-organ Segmentation} with \textit{HD95}. We train each model on Synapse \cite{Synapse_dataset} train dataset and evaluate on Synapse \cite{Synapse_dataset} and AMOS-CT/MRI \cite{ji2022amos} test datasets.}
    \label{tab:comparison_sota_multiorgan_HD95_metrics}
\end{table*}

\begin{figure*}[hbtp]
    \centering
    \includegraphics[width=\textwidth]{figure/Sup_QualitativeResults_Dermatoscopy}
    \caption{Qualitative comparison of other methods and \textbf{TransGUNet} on \textbf{\underline{Skin Lesion Segmentation}} \cite{gutman2016skin, mendoncca2013ph}. (a) Input images with ground truth. (b) UNet. (c) UNet++. (d) CENet. (e) TransUNet. (f) MSRFNet. (g) DCSAUNet. (h) M2SNet. (i) PVT-GCASCADE. (j) CFATransUNet. (k) MADGNet. (l) \textbf{TransGUNet (Ours)}. \textbf{Green} and \textbf{Red} lines denote the boundaries of the ground truth and prediction, respectively.}
    \label{fig:Sup_QualitativeResults_Dermatoscopy}
\end{figure*}

\begin{figure*}[hbtp]
    \centering
    \includegraphics[width=\textwidth]{figure/Sup_QualitativeResults_Radiology}
    \caption{Qualitative comparison of other methods and \textbf{TransGUNet} on \textbf{\underline{COVID19 Infection Segmentation}} \cite{ma_jun_2020_3757476}. (a) Input images with ground truth. (b) UNet. (c) UNet++. (d) CENet. (e) TransUNet. (f) MSRFNet. (g) DCSAUNet. (h) M2SNet. (i) PVT-GCASCADE. (j) CFATransUNet. (k) MADGNet. (l) \textbf{TransGUNet (Ours)}. \textbf{Green} and \textbf{Red} lines denote the boundaries of the ground truth and prediction, respectively.}
    \label{fig:Sup_QualitativeResults_Radiology}
\end{figure*}

\begin{figure*}[hbtp]
    \centering
    \includegraphics[width=\textwidth]{figure/Sup_QualitativeResults_Ultrasound}
    \caption{Qualitative comparison of other methods and \textbf{TransGUNet} on \textbf{\underline{Breast Tumor Segmentation}} \cite{al2020dataset, zhuang2019rdau}. (a) Input images with ground truth. (b) UNet. (c) UNet++. (d) CENet. (e) TransUNet. (f) MSRFNet. (g) DCSAUNet. (h) M2SNet. (i) PVT-GCASCADE. (j) CFATransUNet. (k) MADGNet. (l) \textbf{TransGUNet (Ours)}. \textbf{Green} and \textbf{Red} lines denote the boundaries of the ground truth and prediction, respectively.}
    \label{fig:Sup_QualitativeResults_Ultrasound}
\end{figure*}

\begin{figure*}[hbtp]
    \centering
    \includegraphics[width=\textwidth]{figure/Sup_QualitativeResults_Colonoscopy}
    \caption{Qualitative comparison of other methods and \textbf{TransGUNet} on \textbf{\underline{Polyp Segmentation}} \cite{bernal2015wm, jha2020kvasir, vazquez2017benchmark, tajbakhsh2015automated, silva2014toward}. (a) Input images with ground truth. (b) UNet. (c) UNet++. (d) CENet. (e) TransUNet. (f) MSRFNet. (g) DCSAUNet. (h) M2SNet. (i) PVT-GCASCADE. (j) CFATransUNet. (k) MADGNet. (l) \textbf{TransGUNet (Ours)}. \textbf{Green} and \textbf{Red} lines denote the boundaries of the ground truth and prediction, respectively.}
    \label{fig:Sup_QualitativeResults_Colonoscopy}
\end{figure*}