\section{Related Works and Motivation}
Researchers in the wireless community have extensively studied various critical aspects of IoT systems, including energy harvesting, maximizing bit transmission, and the utilization of RIS panels ____. Specifically, ____ discussed IoT systems, while ____ emphasized the significance and potential deployment of RIS panels.
For instance, ____ proposed a novel framework to optimize the achievable rate for IoT data transmission from the base station (BS) to the sensors. In ____, the authors suggested that harvesting energy from RF sources is the best option for low-power IoT sensors. Furthermore, ____ designed an energy harvesting model for low-power IoT sensors to explore environmental monitoring systems. In ____, a new concept was introduced to power IoT sensors by harvesting energy from nearby smart devices like smartphones. Lastly, ____ investigated IoT systems where energy is shared among the sensors.


The authors in ____ designed a cognitive radio network that maximizes resource allocation, including efficient information and energy transfer. To achieve this, they considered a secondary user with an unspecified backlog of traffic and an error-free sensing function. They calculated the energy region of a wireless system and looked at optimizing the region among the nodes in an opportunistic mobile network, considering a transmitter, a decoder, and an energy harvesting device. In ____, the authors expanded on this model to maximize system throughput for secondary users using a stochastic approach to consider transmission power. They assumed that users could harvest energy for each other if they were close. Additionally, ____ focused on studying the features of the BS RF signals and microcontrollers to configure the status of the RF systems for b-IoT sensors that use harvested energy to convert external energy sources, such as BS RF signals, into DC energy.
The RIS panel has been proposed in various wireless network domains, such as IoT sensors, cognitive radio networks, device-to-device (D2D) communications, unmanned aerial vehicles (UAVs), etc.

The possible deployment of the RIS panels integrated into IoT systems significantly enhances the system performance ____.
For example, with the help of the RIS panel, a link between a BS and a user was considered to maximize the throughput by increasing the elements of the RIS panel in ____.
They also focused on maximizing energy efficiency in RIS-assisted wireless networks.
The RIS-aided future wireless network was proposed in ____ for indoor and outdoor scenarios.
The authors considered a single RIS panel to maximize the system performance matrices.
The authors in ____ studied relay-based wireless networks with the help of the RIS panel.
In their proposed networks, the RIS panel enhanced the link between transmitter and receiver in the first hop, while the other hop was enhanced by relay.
The required channel state information estimation was considered in ____ by embedding RIS active sensors.
The authors in ____ used the Fisher-Snedecor \textit{F} model to estimate the channel in the RIS panel-assisted wireless networks.
The mean square error for RIS panel-assisted IoT systems was investigated in ____ to minimize the mean square error in their model.
The authors in ____ investigated a rate maximization problem with the help of the RIS panel and UAV.
While energy harvesting from RF sources has been widely studied, most research is limited to decoding information with a non-linear energy harvesting model ____.
A framework for RIS panel-assisted low-power IoT systems was studied to investigate the back-scatter communications in ____.
Unfortunately, none of the works in ____ considered RIS modules and the optimal number of microcontrollers to enhance the performance of their models.