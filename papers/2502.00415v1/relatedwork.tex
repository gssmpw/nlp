\section{Background and Related Work}
\label{sec:2}

Recent advances in LLMs have spurred a wave of research into their applicability to diverse financial tasks, including fundamental analysis, alpha discovery, and portfolio decision-making. This section surveys closely related work in four main areas: (i) LLM-based fundamental analysis, (ii) advanced methods in LLM-driven investment analysis, (iii) retrieval-augmented techniques, (iv) the significance of SEC filings and earnings conference calls in fundamental research, and (v) the impact of the macroeconomic environment on stock analysis.

\subsection{LLM-Based Fundamental Analysis}
A growing body of literature investigates how LLMs can replicate or surpass human analysts’ capabilities for parsing and interpreting financial statements. For instance, \cite{kim2024financial} demonstrate that GPT-4 can execute ratio analysis and detect trends via Chain-of-Thought (CoT) prompting \cite{wei2022chain}, yielding interpretable explanations and confidence assessments for binary earnings forecasts. Similarly, \cite{cheng2024gpt} employ GPT-4 to generate high-return factors grounded in economic reasoning, thereby laying a foundation for quantitative investment models. Both studies highlight LLMs’ ability to extract structured insights, such as financial ratios and performance patterns, directly from extensive textual documents.

\subsection{Advanced Methods in LLM-Driven Investment Analysis}
Beyond processing financial disclosures, LLMs have also been employed to generate alpha signals and optimize trading strategies. \cite{wang2023alpha} introduce Alpha-GPT, which couples human expertise with automated alpha discovery to refine trading signals. Similarly, TradingGPT \cite{li2023tradinggpt} adopts a multi-agent, layered memory architecture for collaborative decision-making—though its evaluation results are limited. Meanwhile, \cite{tan2023large} apply sentiment analysis, model ensembles, and in-context learning to predict returns in the Chinese equity market, achieving promising accuracy. More recently, \cite{papasotiriou2024ai} demonstrate that GPT-4, leveraged through in-context learning, can produce stock ratings (e.g., buy, hold, sell) from fundamental reports and news data—outperforming human analysts in certain scenarios.

\subsection{Retrieval-Augmented Techniques}
RAG \cite{lewis2020retrieval} has emerged as one of the most prevalent applications of LLMs in production systems \cite{menlo_ventures_2024}, allowing models to incorporate extensive corpora beyond their internal parameters and input context. This approach is particularly valuable for finance, where multi-faceted data—regulatory filings, market news, economic reports—can be vast and continually updated. Recent research focuses on advanced chunking, query expansion, and re-ranking algorithms to mitigate context loss when processing large documents \cite{setty2024improving, yepes2024financial}, though optimal methodologies may vary depending on data size, structure, and recency requirements. For instance, in stock analysis, the date-aware document retrieval becomes essential yet is often overlooked in standard similarity searches. Although a few recent works propose RAG pipelines tailored to financial tasks \cite{arslan2024business, zhang2023enhancing}, there remains a gap in comprehensive, domain-specific solutions optimized for financial analytics.

\subsection{Importance of Filings and Earnings Calls in Fundamental Research}
A substantial body of empirical evidence underscores the critical role of SEC filings (e.g., 10-K and 10-Q) and earnings conference calls in shaping market outcomes and guiding investment decisions. Studies by \cite{loughran2011liability, eugene1992relationship} report that changes in language complexity, disclosure content, and tonal shifts within filings predict returns, risk profiles, and management quality. \cite{campbell2008search, mayew2015md} emphasize the importance of footnote analysis for identifying hidden risks, while \cite{dikolli2019cfo} demonstrate how readability and clarity can serve as proxies for managerial competence and earnings transparency.

Earnings conference calls exert a similarly influential role in price discovery. \cite{frankel1999empirical} find that trading volumes and volatility spike during these events, especially in Q\&A sessions where spontaneous managerial insights can move markets. \cite{mayew2012power} show that  the tone of calls offer predictive power regarding a firm’s future performance, while \cite{price2012earnings} reveal how the qualitative tone of calls influences both subsequent returns and analyst revisions. \cite{li2008annual, larcker2012detecting} note that these qualitative cues provide additional signals beyond quantitative metrics, and may even reveal deceptive statements. Finally, \cite{mayew2015md} document how analysts with direct access to earnings calls can generate more precise forecasts. Together, these findings establish filings and conference calls as indispensable avenues for uncovering deeper insights into a firm’s performance and strategy.

Emerging research highlights transformative potential of LLMs in financial disclosures and analysis. For instance tools like ChatReport \cite{ni2023chatreport} and XBRL-Agent \cite{han2024xbrl} show LLMs can democratize analysis of dense reports through automated extraction of sustainability metrics and financial concepts, though challenges persist in numerical accuracy and hallucination mitigation. \cite{cook2023evaluating} validate LLMs’ viability in parsing earnings call sentiment, while \cite{goldsack2024facts} reveal their capacity to generate multi-perspective analytical reports approaching human quality. These advances suggest LLMs could reshape fundamental analysis workflows, but require careful governance to preserve informational integrity.

\subsection{Macroeconomic environment impact in stock analysis}
While fundamental metrics and firm-specific disclosures remain critical, macroeconomic indicators (e.g., GDP growth, inflation rates, interest rates), central bank policies, geopolitical factors, and trade agreements between nations provide a broader context that can significantly influence investment outcomes \cite{kwon2024large}. Fluctuations in these external conditions can affect corporate earnings, valuation models, and overall market sentiment—ultimately impacting both short- and long-term trading strategies.

Expert analysis from leading financial institutions plays a crucial role in interpreting these complex macroeconomic relationships. Research and opinionated reports from investment banks and central banks provide valuable insights into emerging trends, policy implications, and potential market impacts that may not be immediately apparent in quantitative data alone \cite{abaidoo2023inflation}. These expert opinions are particularly valuable when analyzing interconnected global markets where local expertise and institutional knowledge become essential for understanding market dynamics.

Incorporating macro-level context and expert insights alongside firm-level data can lead to more robust and adaptive models, particularly when combined with LLM-based frameworks capable of integrating multiple data streams. Notably, macroeconomic forces often vary in their impact across different stocks and sectors. For example, US tariffs on imported goods from China can weigh heavily on industries reliant on specific commodities or products \cite{nvidia2022q3}. However, many existing quantitative and LLM-based stock-analysis models typically overlook these broader economic factors and expert interpretations, revealing a gap in current approaches to investment research.