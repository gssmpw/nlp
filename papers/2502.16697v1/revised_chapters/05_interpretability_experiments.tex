\begin{figure*}[!t]
   \centering
    \includegraphics[trim={0cm 0cm 0cm 0cm},clip,width = \textwidth]{figures/figure5_spatial_explanation_comparison.pdf}
    \caption{Comparison of different explainability tools for the DR staging task on OCTA images. GradCAM and integrated gradients are state-of-the-art approaches that previous works have applied to DR staging. 
    Sample \textbf{a)} describes a sample with clinical diagnosis \textit{NPDR} that the CNN and graph classification model correctly identified. Sample \textbf{b)} describes a sample with clinical diagnosis \textit{PDR} that the CNN and graph classification model correctly identified. Compared to the traditional spatial explanations, our framework (rightmost column) allows the precise localization of critical vessel segments and intercapillary areas.}
    \label{fig:spatial_explanations}
\end{figure*}

To illustrate the human-understandable explanations of our interpretability framework, we analyze two samples from the test set. Figure \ref{fig:full_explanations} \textbf{a)} and \textbf{b)} depict the spatial and feature-wise explanations for two samples that were correctly classified as \textit{NPDR} and \textit{PDR}.

\subsection{Qualitative Results}

In both samples, we highlight the twenty most important nodes in their associated image region. 
In \textbf{a)}, we show that large intercapillary areas and some small capillaries are most influential to the model's decision. Conversely, in Figure \ref{fig:full_explanations} \textbf{b)}, a combination of large and small vessels, intercapillary areas, and the FAZ are identified as the most important nodes. 



Beyond spatial explanations, our method enables us to attribute the elevated importance of specific nodes to high-level concepts, such as vessel curvature or intercapillary area size. 
The bar plots in Figure \ref{fig:full_explanations} indicate which features of the highlighted nodes are decisive for the predictions. The arrows in the right-most column of Figure \ref{fig:full_explanations} indicate how the respective feature deviates from the average.

In sample \textbf{a)}, the increased number of neighboring intercapillary areas and the larger perimeter of the intercapillary areas are the most influential characteristics. This observation aligns with earlier studies reporting a relation between the size of the largest intercapillary areas and DR progression \cite{schottenhamml2016automatic}. The study points out that the emergence of large intercapillary areas is caused by vascular dropout, which results in the merging of two areas. The number of neighboring intercapillary areas is, therefore, also expected to increase. Furthermore, the reduced number of neighbors at the smaller branching point and the increased deviation of the roundness are important vessel characteristics. 

In sample \textbf{b)}, the high curvature and the diameter variability are the predominant vessel characteristics for the prediction. These characteristics can be related to intraretinal microvascular abnormalities (IRMA) such as vessel dilations and to the neovascularization of tiny vessel loops \cite{ishibazawa2016characteristics}.
The most influential properties of the intercapillary areas are an increased area size and a large number of neighboring areas. Interestingly, the FAZ also has a strong influence on the prediction through a decrease in solidity, defined as its area divided by the area of its convex hull. This explanation is also related to previous studies reporting that peripheral disruptions of the FAZ, which can result in a decrease of solidity, are linked to DR \cite{fernandez2022retinal}. Numerous studies identified the FAZ circularity, a closely related concept, as predictive for DR \cite{sun2021optical}.

\subsection{Comparison to CNN Explanations}
We adopt explainability approaches previously employed by Ryu et al. \cite{ryu2021deep} and Heisler et al. \cite{heisler2020ensemble} and use grad-CAM to create model explanations. Additionally, we utilize integrated gradients, initially proposed in the DR classification setting \cite{sundararajan2017axiomatic}, to highlight finer structures than grad-CAM. Moreover, we experiment with an alternative integrated gradients approach for CNNs, using a baseline input that accounts for the expected lower image intensity around the FAZ. This is achieved by averaging the normalized image intensities across the training set, resulting in the elimination of highlighted areas within the FAZ caused by noise.

Notably, the baselines, as depicted in Figure \ref{fig:spatial_explanations}, are limited to spatial explanations. In contrast to our framework, they do not offer feature-wise attributions towards interpretable concepts (see Figure \ref{fig:full_explanations}). Therefore, our comparison is limited to these spatial explanations. 

Grad-CAM explanations encompass a broad area, spanning multiple vessels and intercapillary areas. Therefore, it is infeasible to identify the exact entities that influence the prediction. Furthermore, it is impossible to identify which characteristics of the respective image regions cause their elevated importance. An analysis of the average grad-CAM explanation revealed a consistent focus on the FAZ, which is also visible in Figure \ref{fig:spatial_explanations}. 

\begin{figure}[!ht]
   \centering
    \includegraphics[trim={0cm 0cm 0cm 0cm},clip,width = \linewidth]{figures/figure6_global_importance.pdf}
    \caption{Quantitative evaluation of \textbf{a)}, the most important characteristics for correct identification of the \textit{PDR} stage in the test set, using our proposed graph classification model. The red and blue colors represent characteristics of the \textcolor{ves_spatial}{vessels} and \textcolor{ica_spatial}{intercapillary areas}, respectively. \textbf{b)} Violin plots of the average curvature across all vessel segments in a graph according to their clinical label. \textbf{c)} Violin plots of the average area in no. of pixels across all intercapillary areas in a graph according to their clinical label.}
    \vspace{-0.3cm}
    \label{fig:global_importance}
\end{figure}


In contrast, explanations using integrated gradients are more precise in location, sometimes even highlighting individual vessel segments. However, in most cases, only a few connected pixels are highlighted, and many of these clusters are dispersed across the image. We observe, across all images, that a significant portion of the highlighted pixels is in the FAZ without any notable deviations between samples. Adopting an average OCTA image as the baseline sample mitigates this behavior. However, this approach leads to vessels close to the FAZ being highlighted, which are not highlighted in the zero image approach. This effect arises from the newly introduced stronger contrast to the average pixel intensity in the FAZ region. Baseline images strongly affect the created explanations, and ideally, multiple baselines are considered to obtain in-depth explanations \cite{sturmfels2020visualizing}.  



\subsection{Global Feature Importance}
Beyond examining model reasoning on single samples (Figure \ref{fig:full_explanations}), we investigate global trends of influential features for decision-making. In Figure \ref{fig:global_importance} \textbf{a)}, we show the relative importance of the most influential characteristics for correctly classifying \textit{PDR}. We identify vessel curvature and the size of the intercapillary areas as the most important features for the respective node types. Interestingly, the FAZ's properties are less important than the vessel and intercapillary area characteristics. Moreover, the distributions of the average curvature (\ref{fig:global_importance} \textbf{b)}) and average intercapillary area size (\ref{fig:global_importance} \textbf{c)}) show that globally, an increase of curvature and area size is directly associated with the \textit{PDR} and \textit{NPDR} stage. Previous studies found that vessel tortuosity is linked to the \textit{NPDR}, but interestingly not the \textit{PDR} stage \cite{lee2018quantification}. This distinction may arise from the difference in the average curvature, where vessel segments are weighted equally, and the vessel tortuosity, where vessels are weighted according to their length. The increase in intercapillary area size was linked to DR progression earlier \cite{schottenhamml2016automatic}. Moreover, in the case of the loss of separating vessels, it is directly related to VD, which has been associated with DR in numerous studies \cite{sun2021optical}.