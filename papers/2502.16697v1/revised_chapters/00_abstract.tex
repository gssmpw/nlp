\begin{abstract}
Interpretability is crucial to enhance trust in machine learning models for medical diagnostics. However, most state-of-the-art image classifiers based on neural networks are not interpretable. As a result, clinicians often resort to known biomarkers for diagnosis, although biomarker-based classification typically performs worse than large neural networks. This work proposes a method that surpasses the performance of established machine learning models while simultaneously improving prediction interpretability for diabetic retinopathy staging from optical coherence tomography angiography (OCTA) images. Our method is based on a novel biology-informed heterogeneous graph representation that models retinal vessel segments, intercapillary areas, and the foveal avascular zone (FAZ) in a human-interpretable way. This graph representation allows us to frame diabetic retinopathy staging as a graph-level classification task, which we solve using an efficient graph neural network. We benchmark our method against well-established baselines, including classical biomarker-based classifiers, convolutional neural networks (CNNs), and vision transformers. 
Our model outperforms all baselines on two datasets. Crucially, we use our biology-informed graph to provide explanations of unprecedented detail. Our approach surpasses existing methods in precisely localizing and identifying critical vessels or intercapillary areas. In addition, we give informative and human-interpretable attributions to critical characteristics. Our work contributes to the development of clinical decision-support tools in ophthalmology.
\end{abstract}