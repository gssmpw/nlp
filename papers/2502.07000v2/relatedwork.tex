\section{Related work}
Stochastic linear search was independently considered by Bellman and Beck in the 1960's (cf. \cite{beck1964linear,bellman1963optimal}). Work in linear search dates back to the 1970s with Beck and Newman who introduced the linear search problem and proposed an algorithm with optimal competitive ratio $9$~\cite{beck1970linearsearch}.
Since then, many variants of the linear search problem have been studied and the competitive ratios of various algorithms have been analyzed~\cite{gal_search_games,baeza_yates,group_search,search_plane}. % ,chases_escapes}.
%\EK{We should add some additional citations on cow-path and group search. This should give more context.}

The multimodal linear search problem we study is a generalization of linear search (when the number of modes $p=1$, the two problems are equivalent).
Despite the vast literature in search theory, work on multimodal search is sparse.
A predator-prey model was introduced in~\cite{knoppien1985predators} where the predator has two modes of searching: a fast mode where the predator has a low probability of detecting the prey when encountered and a slow mode where the predator has a high probability of detecting the prey.
Similarly in~\cite{benichou2011intermittent}, the authors study problems where the searcher moves at a faster speed at which the searcher cannot detect the target and a slower speed at which the searcher can detect the target (with certainty).
The Beachcomber problem~\cite{beachcomber} extends this idea to a multi-agent setting where searchers must cooperate to find the target in as little time as possible.
Another key similarity between these studies and ours is that, depending on the strategy the searcher uses, it may not detect the target when passing through its location.
This is reminiscent of work in search with uncertain detection where the searcher may miss the target with some probability~\cite{gal_search_games,probabilistic_faults}.  % gal_search_games Section 8.6

To the best of our knowledge, this work is the first to consider the multimodal linear search problem.