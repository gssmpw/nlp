\section{Related work}
\label{sec:related_work}
\looseness=-1
Boltzmann generators (BGs) \citep{noe2019boltzmann} have been applied to both free energy estimation \citep{wirnsberger2020targeted, rizzi2023multimap, schebek2024efficient} and molecular sampling. Initially, BGs relied on system-specific representations, such as internal coordinates, to achieve relevant sampling efficiencies \citep{noe2019boltzmann, kohler2021smooth, midgley2022flow, kohler2023rigid, dibak2021temperature}. However, these representations are generally not transferable across different systems, leading to the development of BGs in Cartesian coordinates \citep{klein2023equivariant, midgley2023se,klein2024transferable}. While this improves transferability, they are currently limited in scalability, struggling to extend beyond dipeptides. Scaling to larger systems typically requires sacrificing exact sampling from the target distribution  \citep{jing2022torsional,abdin2023pepflow,jing2024alphafold,lewis2024scalable}, which often includes coarse-graining. 
An alternative to direct sampling from $\mu_{\text{target}}(x)$ is to generate samples iteratively by learning large steps in time \citep{schreiner2023implicit, fu2023simulate, klein2023timewarp, diez2024boltzmann, jing2024generative, daigavane2024jamun}.