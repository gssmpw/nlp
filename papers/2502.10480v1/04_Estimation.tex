\section{Pose Estimation}
A discrete multiplicative extended Kalman filter (MEKF) pose estimator is leveraged to estimate the state of the TPODS module \cite{TPODS_GNC24}. The goal of the estimator is to combine the measurements from the rate gyroscope and UWB radar, along with the knowledge about system dynamics, sensor models, and the current control input to compute the best guess of the current relative position, velocity and orientation of the TPODS module.

A caret sign represents an estimated value of each quantity. It is assumed that unbiased measurements for the angular velocity $\boldsymbol{\omega}$ are available with corresponding Gaussian measurement noise, and a quaternion $\boldsymbol{q}$ tracks the attitude of a TPODS module relative to the mothership. 
\begin{equation}
\boldsymbol{\hat{\omega}} = \boldsymbol{z}_{gyro}
\end{equation}
The estimator predicts a stochastic state vector,
\begin{equation}
\boldsymbol{\zeta} = (\boldsymbol{\rho_{ij}},\skew{-8}\dot{\boldsymbol{\rho_{ij}}},\boldsymbol{\delta q})
\end{equation}
where $\boldsymbol{\delta q}$ is a three-parameter vector representing errors in the estimated attitude. The selected structure of the pose estimator offers two distinct advantages. 
\begin{enumerate}
    \item Firstly, instead of having a single state vector with nine individual states, the estimator is divided into two tandem structures. One for tracking six translational states and another for three attitude error parameters. As the computational cost of the extended Kalman filter is proportional to the cube of filter states \cite{filter_compute}, the tandem structure helps in lowering the total computational cost of the algorithm.
    \item Secondly, instead of tracking the relative attitude via a quaternion, the three attitude error parameters mitigate some of the adverse effects of the quaternion normalization constraint, as errors remain very close to zero for each propagation and update step \cite{crassidis2011optimal}. The estimated attitude is recovered with a reference quaternion from the attitude error parameters with $\mathbf{\delta q} = \mathbf{q}\otimes\mathbf{\hat{q}}^{-1}$.
\end{enumerate}
