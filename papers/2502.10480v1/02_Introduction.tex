\section{Introduction}
The transforming proximity operations and docking system (TPODS) is a conceptual 1U CubeSat module, developed at the Land, Air and Space Robotics (LASR) laboratory of Texas A\&M University \cite{TPODS_system,TPODS_estm,TPODS_GNC24}. The overall objective of the TPODS module is to enable servicing of a tumbling resident space object (RSO). The TPODS modules are stowed in a mothership, which has the necessary sensor suite to locate and approach an RSO. Once in the vicinity, the mothership analyzes the tumbling motion of the RSO and deploys multiple TPODS towards the object. TPODS then leverage non-adhesive attachment mechanisms to firmly affix with the RSO. 

This deployment strategy imparts a specific momentum change, resulting in significant reduction in the rotation rate of the RSO \cite{GNC_24_down}. However, for most practical applications, additional momentum transfer is required to completely detumble the object \cite{TPODS_detumble}. To perform a powered de-tumbling operation in a fuel efficient manner, it is often required to relocate the TPODS modules from their initial position on the body to achieve a better momentum lever. Since the RSO is still under substantial tumbling, the relocation process can be challenging, particularly with the uncertainty in pose estimates of each agent. If the uncertainties are not considered during the motion planning of relocation, it can result in catastrophic consequences due to the proximity of modules and the RSO. 

Once a stable rotation of the RSO is achieved, the TPODS modules can be rearranged to form various scaffolding structures to enable docking of a more capable servicing vehicle. Figure~\ref{fig:scafolding} presents one such example workflow. Since the TPODS modules now have to maneuver in a highly dynamical environment, a safety focused motion planning approach is necessary. Although more accurate pose determination via monocular vision sensors is available at shorter TPODS-to-TPODS distances, for the majority of the scaffolding generation process, the pose of each agent is driven by relative ranging and has significant associated uncertainties \cite{ICRA25,Ali_GNC24}. 

\begin{figure}[t!]
     \begin{subfigure}[b]{0.32\textwidth}
        \centering
         \includegraphics[width=\textwidth]{Figures/scafolding_start.png}
         \caption{Initial Position}\label{fig:1a}
     \end{subfigure}   
     \begin{subfigure}[b]{0.33\textwidth}
        \centering
         \includegraphics[width=\textwidth]{Figures/scafolding_1.png}
         \caption{Structure 1}\label{fig:1b}
     \end{subfigure}
     \centering
     \begin{subfigure}[b]{0.31\textwidth}
        \centering
         \includegraphics[width=\textwidth]{Figures/scafolding_2.png}
         \caption{Structure 2}\label{fig:1c}
    \end{subfigure}
    \caption{Scaffolding generation to enable servicing of RSO}
    \label{fig:scafolding}
\end{figure}

In considering safety of these autonomous systems, encoding notions of safety directly into an existing controller can be very useful. A popular approach to assuring safety in this manner is through the use of \textit{control barrier functions (CBFs)} \cite{ames_2017,ames2019control}, which provide sufficient conditions for forward invariance of safe sets. For control affine dynamics, these sufficient conditions become linear in control and often produce \textit{sets} of safe controls rather than a single one. Thus, a safe control signal can be found which both lies in such a set, and extremizes some cost function. Typically, the cost function is designed to minimize the deviation between some nominal or legacy controller, and the safe control signal. Because for affine dynamics these safe sets of controls are convex, the optimization problem can be solved very efficiently, making this solution especially appealing for spacecraft proximity operations where compute is scarce. Recognizing this fact, researchers have utilized CBF-based controllers for spacecraft docking \cite{dunlap2021comparing,Breeden_2022_docking}, spacecraft inspection \cite{dunlap2023RTA_inspection,vanWijk2024JAIS,dunlap2024run,hibbard_guaranteeing_2022}, safe reorientation \cite{breeden_attitude}, and generating safe trajectories in the presence of disturbances \cite{breeden_robust_2023,vanWijk_DRbCBF_24}.

The main contributions of the manuscript are threefold. First, we develop a guidance algorithm for relocation of multiple TPODS agents which use a multiplicative extended Kalman filter (MEKF) based estimator for state estimation. A differentiable collision detection and avoidance approach that considers the shape of the RSO is implemented to prevent collisions between TPODS agents and the RSO. Second, we design constraints enforced by CBFs to ensure multi-agent system safety informed by the MEKF, resulting in the safe operation of TPODS in close proximity. Lastly, a hybrid approach of TPODS-RSO and TPODS-TPODS collision avoidance is proposed and the efficacy of the approach is demonstrated with extensive simulation analysis of a pragmatic scenario of simultaneous relocation of two modules on a tumbling RSO.

% 1) ekf with sensor fusion + guidance
% 2) safety multi agent
% 3) sim and hardware demo

% {\color{red} David finish:
% \begin{itemize}
%     \item Main contributions
%     \item Organization of paper
% \end{itemize}}