\section{Hybrid approach and Simulation Results}
The objective of this study is to simultaneously relocate a set of TPODS modules from their current positions on a tumbling body to positions more conducive to the detumbling operation. Since the position of the RSO is dynamic, each TPODS has to accurately predict the future position of the RSO and plan safe trajectories to avoid the RSO as well as other TPODS modules. The effectiveness of the proposed approach will be a key enabler for such highly dynamic and complex autonomous operations. 

\subsection{TPODS-RSO Collision avoidance with DCOL}
The TPODS is commanded to follow a respective reference trajectory, generated using analysis presented in Figure~\ref{fig:ref_traj_a}. The differential collision detection and avoidance routine for convex polytopes summarized in Figure~\ref{fig:MPC_flow} is implemented for the ellipsoidal body and results are presented in Figure~\ref{fig:DCOL_CA} for an example reference trajectory. The TPODS module is commanded to maintain an inflation factor of $1.10$ through the motion. From Figure~\ref{fig:inflation_all}, we observe a few instances of the inflation factor being slightly lower than the target (red dotted line), as the limit is a soft target. However, the inflation factor still stays well above the actual collision event, identified as an inflation factor of $1$. The deviation of the actual trajectory from the reference ensures that the TPODS maintains a safe separation from the RSO. The collision avoidance can be switched off at the final stage of the motion to allow for relocation on the RSO. 

\begin{figure}[!t]
\centerline{\includegraphics[width=1\textwidth]{Figures/DCOL_CA.eps}}
\caption{TPODS-RSO Collision avoidance with DCOL}
\label{fig:DCOL_CA}
\end{figure}

\begin{figure}[b!]
    \centerline{\includegraphics[width=1\textwidth]{Figures/inflation_factor5.png}}
     \centering
    \caption{Inflation factor for collision avoidance using DCOL and CBF}
    \label{fig:inflation_all}
\end{figure}

\subsection{TPODS-RSO Collision avoidance with CBF}
While the CBF approach was shown to be effective in preventing head-on TPODS-TPODS collisions, the CBF approach was found to be suboptimal for TPODS-RSO collision avoidance. For a large set of the initial conditions, the myopic nature of CBFs caused it to generate control signals which were over-reactive, resulting in larger deviations from the reference trajectory when compared with the DCOL approach. \Cref{fig:inflation_all} plots the inflation factor using the CBF to assure safety. In the case examined, the CBF approach sees a much larger inflation factor than DCOL, corresponding to that overreaction.

\subsection{Hybrid Approach for TPODS-TPODS and TPODS-RSO Collision Avoidance}
As discussed in previous sections, the collision avoidance approach based on 
differential polytopes performs well while avoiding stationary obstacles but fails to avoid head-on collisions. In contrast, the CBF-based collision avoidance approach successfully navigates around head-on collisions but results in drastic corrections when approaching a stationary target. Hence, none of the collision avoidance approaches are sufficient to enable safe relocation of TPODS when applied in isolation. Consequently a hybrid approach, shown in Figure~\ref{fig:arch_flow}, that switches between CBF-based collision avoidance and DCOL is proposed and validated in this paper. First, the \eqref{eq:hocbf-qp} solves the optimization problem for a safe control signal which satisfies the HOCBFs derived from \eqref{eq:koz} and \eqref{eq:h_tpods_tpods} for each agent. If the inter-TPOD collision avoidance constraint $h_{\rm ca}$ is active (meaning a TPODS-TPODS collision is imminent), then the resulting safe control is used for each TPODS agent. Otherwise, the produced control signal is discarded, and the DCOL framework is used to prevent any TPODS-RSO collisions.

\subsection{Accounting for Uncertainty}

Standard CBF and HOCBF approaches assume perfect state information is available at all times -- an assumption that cannot be made for most real world problems. For the optimal relocation application, the autonomous TPODS agents can only access a best estimate of the true states via the MEKF. As such, measures need to be taken to robustify the safety conditions against uncertainty in state information. For this application, the position uncertainties are of upmost importance, as the state constraints are defined only in terms of these variables. Therefore, we modify the constraints in \eqref{eq:koz} and \eqref{eq:h_tpods_tpods} to be adaptive based on the uncertainty information given by the MEKF's covariance matrix, similar to \cite{vanWijk_FTRTA}. Denoting the posterior covariance for agent $k$ at any instant in time with $\boldsymbol{P}^{+}_{xx,k}$, consider a position uncertainty buffer, $\eta_k$, defined by
\begin{align*}
    \eta_k \triangleq \xi \norm{\sqrt{\texttt{diag}\{ \boldsymbol{P}^{+}_{xx,k} \}}(1:3)}
\end{align*}
where $\xi \in \mathbb{R}_{>0}$ is a constant, tunable term and the $\texttt{diag}\{ \cdot \}$ operator returns a vector containing the diagonal terms of an inputted square matrix. The buffer is a scalar term which captures an uncertainty radius around the best estimate of the state. Therefore, we modified constraints by inflating the effective radius of the TPOD geometry by this additional $\eta_k$ distance. Using $\hat{\boldsymbol{x}}_k^{\mathcal{B}}$ to denote the estimated state of agent $k$, the modified keep-out-zone constraint is written as  
\begin{align} \label{eq:koz_adaptive}
    \hat{h}_{\rm koz}(\hat{\boldsymbol{x}}_k^{\mathcal{B}}) \triangleq \frac{\hat{x}^2}{(a+r_{\rm s}+\eta_k)^2} + \frac{\hat{y}^2}{(b+r_{\rm s}+\eta_k)^2} + \frac{\hat{z}^2}{(c+r_{\rm s}+\eta_k)^2} - 1 \geq 0
\end{align}
Similarly, the inter-agent collision avoidance constraint uses the inflated effective radius for each agent and thus the TPODS-TPODS constraint can be written as
\begin{align} \label{eq:h_tpods_tpods_a}
    \hat{h}_{{\rm ca},ij}(\hat{\boldsymbol{x}}_{ij}^{\mathcal{B}}) \triangleq (\hat{x}_i - \hat{x}_j)^2 + (\hat{y}_i - \hat{y}_j)^2 + (\hat{z}_i - \hat{z}_j)^2 - (2r_{\rm s} + \eta_i + \eta_j)^2 \geq 0
\end{align}
It should be noted that enforcing these constraints using estimated states rather than true states no longer retains the safety guarantees offered by CBFs. Instead, we can only claim that using the uncertainty-based approach will result in fewer collisions than if the original constraints \eqref{eq:koz} and \eqref{eq:h_tpods_tpods} were used with estimated states. A detailed animation of the proposed approach in safe relocation of two TPODS in the vicinity of a tumbling RSO can be found here : \url{https://youtu.be/DSrAHj5wXGg}.

\subsection{Monte Carlo Simulations}

To verify the effectiveness of the proposed solution in solving the relocation task safely, a Monte Carlo simulation was performed with $500$ different sets of initial conditions. \Cref{fig:MC_hvals} plots the constraint values, \eqref{eq:koz} and \eqref{eq:h_tpods_tpods}, for each agent using the true state information and the estimated states. From a visual inspection, it is clear that there are very few cases where the value of $h_{{\rm koz},1}$, $h_{{\rm koz},2}$, or $h_{{\rm ca},12}$ decreased below $0$. Indeed, in \Cref{tab:MC_collisions} we can see that there were at most $5$ violations for any particular safety constraint, and that $97.6\%$ of the trials had no safety violations. Additionally, because the $h$ functions overapproximate the TPODS geometry, minor violations (i.e., small negative values) may not indicate that a true collision has occurred. \Cref{fig:MC_traj.eps} plots all $500$ runs in the RSO position space, showing the general trend of the agents altering their trajectories to avoid collisions.

\begin{table}
    \centering
    \begin{tabular}{|c|c|c|c|c|c|c|c|}
    \hline
        Type & Value & Type & Value & Type & Value & Type & Value \\
        \hline
        $h_{{\rm ca},12}$ & -0.0173 & $h_{{\rm koz},1}$ & -0.0021 & $h_{{\rm koz},2}$ & -0.0040 & $h_{{\rm koz},1}$ & -0.0031\\
        $h_{{\rm koz},1}$ & -0.0079 & $h_{{\rm ca},12}$ & -0.0062 & $h_{{\rm koz},2}$ & -0.0087 & $h_{{\rm koz},2}$ & -0.0054\\
        $h_{{\rm koz},1}$ & -0.0031 & $h_{{\rm ca},12}$ & -0.0063 & $h_{{\rm koz},1}$ & -0.0195 & $h_{{\rm ca},12}$ & -0.0032\\
        \hline
    \end{tabular}
    % \begin{tabular}{|c|c|}
    % \hline
    % \textbf{Constraint} & \textbf{Violations} \\ \hline
    % $h_{{\rm koz},1}$   &   5                            \\ \hline
    % $h_{{\rm koz},2}$   &   3                            \\ \hline
    % $h_{{\rm ca},12}$        &   4                            \\ \hline
    % \end{tabular}
    \caption{Constraint violations for $500$ run Monte Carlo simulation}
    \label{tab:MC_collisions}
\end{table}

\begin{figure}[h!]
    \centering
     \begin{subfigure}[b]{0.325\textwidth}
        \centering
         \includegraphics[width=\textwidth]{Figures/MC_hkoz1.eps}
         \caption{}\label{fig:MC_hkoz1}
     \end{subfigure}   
     \begin{subfigure}[b]{0.325\textwidth}
        \centering
         \includegraphics[width=\textwidth]{Figures/MC_hkoz2.eps}
         \caption{}\label{fig:MC_hkoz2.eps}
     \end{subfigure}
     \begin{subfigure}[b]{0.325\textwidth}
        \centering
         \includegraphics[width=\textwidth]{Figures/MC_hinter.eps}
         \caption{}\label{fig:MC_hinter.eps}
     \end{subfigure}
    \caption{Safety criterion for $500$ run Monte Carlo simulation}
    \label{fig:MC_hvals}
\end{figure}

\begin{figure}[!t]
\centerline{\includegraphics[width=1\textwidth]{Figures/MC_traj.eps}}
\caption{Trajectories for $500$ run Monte Carlo simulation}
\label{fig:MC_traj.eps}
\end{figure}
% \begin{figure}[ht!]
%      \begin{subfigure}[b]{0.32\textwidth}
%          \includegraphics[width=\textwidth]{Figures/scf_1.png}
%          \caption{Initial Position}\label{fig:1a}
%      \end{subfigure}   
%      \hfill
%      \begin{subfigure}[b]{0.2\textwidth}
%          \vfill   
%          \includegraphics[width=\textwidth]{Figures/scf_2.png}
%          \caption{Inflated Position}\label{fig:1b}
%      \end{subfigure}
%      \hfill
%      \begin{subfigure}[b]{0.18\textwidth}
%         \vfill
%          \includegraphics[width=\textwidth]{Figures/scf_3.png}
%          \caption{Desired Structure}\label{fig:1c}
%     \end{subfigure}
%     \caption{Scaffolding generation with control barrier functions}
%     \label{fig:scaf_sBC}
% \end{figure}

% \subsection{Scaffolding Generation}
% The CBFs can be further extended to enable the creation of various scaffolding structures. The desired scaffolding configuration can be inflated to generate a set of intermediate target positions for each TPODS as seen in Figure~\ref{fig:scaf_sBC}. The safety distances can be set to conservative values for the relocation motion. Once the TPODS are in their intermediate positions, the subsequent execution of the final docking stage with reduced keep-out distances can result in safe scaffolding generation. 
