\section{System Dynamics}

\begin{figure}[!t]
\centerline{\includegraphics[width=0.6\textwidth]{Figures/coupling.pdf}}
\caption{TPODS module (right) and mothership satellite (left)}
\label{fig:TPODS_coupling}
\end{figure}

The TPODS modules use Ultra Wide-Band (UWB) radar in two-way ranging mode to measure the distance to stationary anchors. Since the anchors and UWB sensor are not mounted at respective centers of mass, the rotational and translational motion of the UWB sensor relative to the stationary anchors are coupled. For the analysis presented in this paper, a scenario consisting of two CubeSat agents and a relatively stationary mothership satellite is considered \cite{TPODS_GNC24}. The objective is to relocate the two TPODS modules from a specified location to a target location on a tumbling RSO. As shown in Fig.~\ref{fig:TPODS_coupling}, the mothership is equipped with multiple UWB transceivers, located at positions $\boldsymbol{P_0^j}$. Each TPODS agent is also equipped with a UWB transceiver (located at positions $\boldsymbol{P_1^i}$) as well as a gyro and monocular camera. Since the UWB radar range sensors measure time-of-flight (TOF) \cite{OG_fusion}, the measurements are available for the relative distance $\boldsymbol{\rho_{ij}}$. Hence, the pose estimation algorithm needs to accurately predict the motion of the vector $\boldsymbol{\rho_{ij}}$ in the reference frame $\boldsymbol{\hat{c}}$, affixed to the mothership. The translation and rotational motion governing equations \cite{ALFRIEND2010227} are given by 
\begin{align}
\boldsymbol{\ddot{\rho}_{ij}} &= \boldsymbol{\ddot{\rho}}  + \boldsymbol{\dot{\omega}} \times \boldsymbol{P_1^i} + \boldsymbol{\omega} \times \left(\boldsymbol{\omega} \times \boldsymbol{P_1^i}\right)  \label{eqn:coupled_EOM} \\
\boldsymbol{I_0 \dot{\omega}} &= \boldsymbol{I_0 D I_1^{-1}} \left[ \boldsymbol{N_1-D^T \omega} \times\ \boldsymbol{I_1 D^T \omega} \right]   \label{eqn:coupled_EOM_att}
\end{align}
where $\boldsymbol{D}$ is the rotation matrix that transforms vectors from TPODS reference frame $\boldsymbol{\hat{d}}$ to the mothership frame $\boldsymbol{\hat{c}}$, $\boldsymbol{\omega}$ is the rotational angular velocity of TPODS relative to the mothership, $\boldsymbol{I_0}$ and $\boldsymbol{I_1}$ are respective inertia tensors for the mothership and TPODS, and $\boldsymbol{N_1}$ is the external torque applied to the TPODS. The CubeSat module has ten nozzles arranged in an `X' configuration on each opposite face to enable 6-degree-of-freedom (DOF) motion \cite{TPODS_GNC24}.