\section{Optimal Relocation}\label{sec:optimal_reclocation}

\begin{figure}[!b]
\centerline{\includegraphics[width=\textwidth]{Figures/arch_flow.png}}
\caption{Architecture for safe module relocation on a tumbling RSO}
\label{fig:arch_flow}
\end{figure}

An essential step for ensuring efficient and successful detumbling of the RSO involves repositioning the TPODS module on the RSO. Following deployment, the TPODS are expected to group around the central part of the RSO. However, for optimal detumbling, it's advantageous to distribute these modules to generate a larger moment arm. This necessitates precise maneuvering of TPODS modules near the rotating RSO. Furthermore, the challenges are exacerbated by the uncertainties in the pose information of each TPODS module.

In the initial stage of the algorithm depicted in Figure~\ref{fig:arch_flow}, the focus is on designing fuel-efficient relocation trajectories. These paths are constrained, partly due to the presence of the RSO, which introduces an additional state inequality constraint to prevent trajectory intersections with the RSO. When conducting dynamic analysis in the inertial frame, the RSO's rotational movement causes this inequality constraint to vary over time. As a result, the equations of motion for the TPODS module are developed in a reference frame fixed to the RSO \cite{Parikh2021}. This approach allows the state inequality constraint to be expressed as one or more ellipsoidal restricted areas.

\begin{figure}[t!]
     \begin{subfigure}[b]{0.49\textwidth}
        \centering
         \includegraphics[width=\textwidth]{Figures/ref_traj.eps}
         \caption{}\label{fig:ref_traj_a}
     \end{subfigure}   
     \begin{subfigure}[b]{0.49\textwidth}
        \centering
         \includegraphics[width=\textwidth]{Figures/col_traj.eps}
         \caption{}\label{fig:ref_traj_b}
     \end{subfigure}
     \centering
    \caption{Reference trajectories for optimal relocation and collision of TPODS}
\end{figure}

A two-point boundary value problem (TPBVP) is formulated in the RSO attached frame of reference for the optimal relocation of a TPODS module \cite{Parikh2021}. The allotted time for the relocation maneuver is $300\ {\rm s}$ and each thruster is constrained to produce a unidirectional thrust of $25\ {\rm mN}$ \cite{TPODS_GNC24}. The RSO is approximated as a single ellipsoidal keep-out-constraint (KOC) with parameters $a=2,b=0.5\ $and$\ c=0.5$. The resulting TPBVP is solved using forward-shooting and the results are presented in Figure~\ref{fig:ref_traj_a}. In particular, the start and end points of these trajectories are selectively chosen to uncover some of the challenges mentioned in the next subsection.

\subsection{Collision Instances}
As evident from Figure~\ref{fig:ref_traj_a}, some of the reference trajectories are very close to the RSO. Since TPODS modules have finite dimensions and the reference trajectories are generated using a point-mass approximation, there are instances of collision between the TPODS modules and the RSO if no evasive action is taken. Similarly, as presented in Figure~\ref{fig:ref_traj_b}, a few trajectories also result in a close pass or a head-on collision between TPODS. Therefore collision avoidance maneuvers are essential to prevent any module from colliding with the RSO or another module during the relocation process.