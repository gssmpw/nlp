\section{Related Work}
% % \subsection{State-of-the-art Simulators}
% % Several open-source simulators have been introduced to enhance the design and integration of specialized hardware accelerators:

% % \textbf{Gem5-Aladdin} integrates the Aladdin accelerator modeling framework with the Gem5 system simulator. This combination provides a comprehensive tool for co-designing and simulating both accelerators and system architectures, making it particularly powerful for early-stage design space exploration. The simulator supports pre-RTL simulations, enabling designers to evaluate accelerator performance and its interaction with the system before committing to detailed RTL designs \cite{gem5-aladdin}.

% % \textbf{Gem5-Salam} leverages the LLVM compiler infrastructure to offer a system architecture for modeling and simulating custom hardware accelerators. By incorporating LLVM-based methodologies, Gem5-Salam can simultaneously simulate and estimate the performance and area of accelerator designs. This approach facilitates a deeper understanding of how various accelerator architectures impact overall system performance \cite{gem5-salam}.

% % \textbf{Gem5-RTL} extends the capabilities of the Gem5 simulator by directly integrating RTL designs, allowing for detailed performance and area assessments. This framework supports the evaluation of accelerators' impact on system resources through precise power and area measurements, which are critical for making informed design decisions. However, translating RTL to C++ and executing it within the main simulation thread can lead to significant increases in both compilation and simulation times, potentially hindering rapid prototyping \cite{gem5-rtl}.

% % \textbf{Gem5-X} addresses scalability challenges by providing a framework capable of simulating many-core heterogeneous systems with more than 256 cores. It incorporates interfaces for advanced memory technologies like scratchpad memory and High Bandwidth Memory (HBM), making it well-suited for exploring the performance implications of integrating various accelerators in large-scale systems \cite{gem5-x}.


% %\subsection{Limitations of Existing Simulators}
% Despite the advancements offered by these simulators, several limitations hinder their effectiveness in practical, real-world system integration:

% \textbf{Simplistic Interconnection Models}: Most existing simulators rely on basic interconnection models such as buses or shared memory. These models do not accurately capture the physical constraints and performance implications of integrating external accelerators, such as GPUs and Application-Specific Integrated Circuits (ASICs). This simplification can lead to unrealistic simulation results, limiting the applicability of these tools for designing complex, high-performance systems \cite{gem5-aladdin}.

% %\textbf{Compilation and Simulation Overhead}: For instance, Gem5-RTL’s method of translating RTL to C++ and executing it within the main simulation thread significantly increases both compilation and simulation times. This overhead hampers rapid prototyping and iterative design processes, which are critical in early-stage development. The inability to capture full-system dynamics, particularly in terms of complex memory hierarchies, further limits its usefulness under heavy computational workloads \cite{gem5-rtl}.

% \textbf{Limited Real-World System Integration}: Current simulators often lack support for practical interfaces and integration methodologies required for real-world system design. For example, the absence of realistic interfaces like PCIe for accelerator simulation limits the ability to accurately model and evaluate the performance and integration of external accelerators. This gap between simulation capabilities and real-world requirements can lead to suboptimal design decisions \cite{gem5-salam}.

% \textbf{Scalability Issues}: While Gem5-X addresses some scalability concerns by supporting many-core systems, the complexity of integrating and managing diverse accelerators in large-scale heterogeneous systems remains a significant challenge. The simulator’s ability to accurately represent the performance and interaction of various accelerators within such systems is crucial but not fully realized in existing tools \cite{gem5-x}.

\vspace{-0.5cm}