\section{Related Work}
\label{sec:rw}
% A SLR on the security of SDN software is currently missing from the research landscape. While existing studies have explored aspects of SDN security, none have taken an SLR approach to the subject. This gap requires a comprehensive analysis that incorporates well-established publications and offers a broader perspective.
A SLR focusing exclusively on the security of SDN software is currently absent from the research landscape. While existing studies have investigated various aspects of SDN security, none have employed an SLR approach. This gap necessitates a comprehensive analysis synthesizing established research and providing a broader SDN software security perspective.

\begin{itemize}
    % \item \textbf{Bhardwaj et al.\cite{9505089}} conducted a seminal study on SDN controller bugs, a critical area due to their central role and growing complexity. Their meticulous analysis of over 500 bugs in three SDN controllers (FAUCET, ONOS, CORD) established a comprehensive bug taxonomy. This work enhances understanding of bug characteristics and operational impact, aiding in evaluating existing fault-tolerance frameworks. However, their focus on three specific controllers limits generalizability to the wider SDN ecosystem. Our SLR addresses this gap by encompassing a broader range of SDN controllers and software components.
    \item \textbf{\cite{9505089}} conducted a foundational study on SDN controller bugs, an area of critical importance due to controllers' central role and increasing complexity. Their rigorous analysis of over 500 bugs across three controllers (FAUCET, ONOS, CORD) resulted in a detailed bug taxonomy. This work deepens our understanding of bug characteristics and their operational impact, which aids in evaluating the efficacy of current fault-tolerance frameworks. However, their focus on three specific controllers limits the generalizability of their findings to the broader SDN ecosystem. Our SLR addresses this limitation by encompassing a wider range of SDN controllers and software components.

    % \item \textbf{Yahui Li et al.\cite{8453007}} presented a survey on network verification and testing using formal methods. This comprehensive exploration highlights various techniques (formal modeling, verification, testing) to build reliable and secure SDN systems. The study identifies limitations in existing work, particularly regarding post-detection activities like vulnerability localization and automated patching. These points align with aspects of our SLR that explore vulnerability fixing and localization.
    \item \textbf{\cite{8453007}} presented a comprehensive survey on network verification and testing using formal methods. This work explores diverse techniques (formal modeling, verification, testing) for enhancing the reliability and security of SDN systems. The study identifies shortcomings in existing research, particularly concerning post-detection activities such as vulnerability localization and automated patching. These points align with aspects of our SLR that delve into vulnerability mitigation and remediation strategies.
    
    % \item \textbf{Fonseca et al.\cite{7959044}} investigated fault management in SDN networks, acknowledging both the benefits and risks of centralized control. They examined existing fault detection, mitigation, and recovery methods, emphasizing the need to balance resilience with performance. Their proposal for integrating these methods with network operations and emerging technologies like machine learning aligns with our exploration of these challenges in our SLR.
    \item \textbf{\cite{7959044}} investigated fault management in SDN networks, recognizing both the advantages and vulnerabilities introduced by centralized control. They examined existing fault detection, mitigation, and recovery methods, emphasizing the need to balance resilience with performance. Their suggestion to integrate these methods with network operations and emerging technologies like machine learning resonates with our SLR's exploration of these challenges in the context of SDN security.
    
    % \item \textbf{Yoshiaki Hori et al.\cite{10.1007/978-3-319-49106-6_22}} focused on security vulnerabilities within the OpenFlow protocol, a core component of SDN. Their systematic threat analysis, incorporating previous work, resulted in a checksheet combining risk assessment and countermeasures specifically for OpenFlow. This work strengthens our understanding of known threats while identifying new ones. However, their reliance on static analysis and predefined threat models limits the exploration of adaptive security measures and the potential of machine learning for dynamic threat mitigation.
    \item \textbf{\cite{10.1007/978-3-319-49106-6_22}} concentrated on security vulnerabilities within the OpenFlow protocol, a fundamental element of SDN. Their systematic threat analysis, incorporating previous findings, culminated in a checksheet that combines risk assessment with countermeasures specifically for OpenFlow. This work strengthens our understanding of known threats and uncovers new ones. However, their reliance on static analysis and predefined threat models limits the exploration of adaptive security measures and the potential of machine learning for dynamic threat mitigation.
\end{itemize}
% Our SLR stands out by exclusively focusing on the security of SDN software. These studies complement our work, providing a richer understanding of the overall SDN security research.
Our SLR distinguishes itself by focusing solely on the security of SDN software. The studies mentioned above complement our work by providing a richer context for understanding the