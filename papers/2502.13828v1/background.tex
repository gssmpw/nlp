\section{Background software-defined networking}\label{sec:bckg}
Towards offering flexibility and efficiency to network administrators, SDN, designed in a software-centric manner, leverages software to program and manage networks agilely. The core principle of SDN lies in its distinct layered architecture, illustrated in Figure\ref{fig_sdn}. These layers include a control plane, a data plane, and an application plane. 

\begin{figure}[ht!]
  \centering
  \includegraphics[width=0.8\linewidth]{sdn.png}
  \caption{A typical SDN architecture.}
    \label{fig_sdn}
\end{figure}

\subsubsection{Control Plane (CP)}\label{sec:c}
It is the key entity in the SDN architecture. The CP represents the centralized SDN controller software acting as the SDN's brain\citep{ol1}. Various open-source and commercial ventures have built controllers for SDN adoption. OpenDaylight (ODL)\footnote{https://www.opendaylight.org/}, Open Network Operating System (ONOS)\footnote{https://opennetworking.org/onos/}, Ryu\footnote{https://ryu-sdn.org/}, and Floodlight\footnote{https://floodlight.atlassian.net/wiki/spaces/floodlightcontroller/overview} are among the well-known distributions.  
Given the importance of CP in SDN implementations, bugs in its software will lead to critical failures. The CP is also a sweet spot for attackers who wish to compromise an entire SDN-based network deployment.

\subsubsection{Application Plane (AP)}\label{sec:a}
The AP layer hosts the SDN applications. These are control programs designed to implement network control logic and policies. The AP interacts with the control plane via an open API (Northbound API). Examples of applications include firewalls, load balancers, etc.

\subsubsection{Data Plane (DP)}
The DP hosts a network equipment set that forwards data following the intelligence plans from the SDN controller. These network elements include switches, routers, sensor nodes, etc. In the context of SDN, the separation between the CP and the DP requires the DP to be remotely accessible for software-based control via an open vendor-agnostic Southbound interface.

\subsubsection{Application Programming Interfaces (APIs)}\label{sec:apis}
Several interfaces, as illustrated in Figure\ref{fig_sdn}, are available for SDN functioning:
The {\em Northbound Interface} sits between controllers and applications. This interface is generally realized through REST APIs of SDN controllers\citep{ref5}. It gives a common interface for application developers to develop and deploy network applications.
The {\em  Southbound Interface} sits between the programmable control plane (e.g., controller) and data plane. This interface is standardized. It thus uses the OpenFlow\citep{ref7} standardized programmable interface.

The {\em East-West Interface} sits between several controllers within (intra-domain) or across (inter-domain) several networks. This interface provides a communication channel to enable scalability, interoperability, and employability capabilities of SDN networks\citep{ref8}.