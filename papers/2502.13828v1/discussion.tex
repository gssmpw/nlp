\section{Discussion}\label{sec:disc}
This section discusses key aspects of software security in SDN, emphasizing trends observed in our SLR and outlining challenges the research community needs to address.

\subsection{Analysis trends}
Our first research question highlighted that software security in SDN primarily focuses on vulnerability detection, mitigation, and hardening. This emphasis underscores the importance of software security within SDN for current users and researchers. Previous studies have demonstrated how vulnerabilities identified in SDN software can be exploited to launch attacks\citep{9152642}. 

Figure\ref{fig:trends} reveals a trend in research priorities. Over 50\% of studies concentrate on vulnerability detection, while a significant portion of research overlooks fixing, exploitation, and localization. This imbalance suggests prioritizing identifying vulnerabilities over addressing their root causes or understanding their practical implications. The under-exploration of localization, a key aspect of effective fixing, presents a substantial research gap that warrants further investigation.
% Figure\ref{fig:trends} illustrates the evolution of research objectives over the years, revealing that more than 50\% of research efforts are focused on detection, to the detriment of vulnerability remediation and exploitation, possibly reflecting a priority placed on identifying vulnerabilities before addressing their mitigation or potential use. However, the lack of attention to locating vulnerabilities highlights a research gap that could have significant implications for SDN security. The research community must balance its efforts between vulnerability detection, remediation, and localization to effectively strengthen software security in SDNs.

\begin{figure}[ht!]
    \centering
    \begin{adjustbox}{width=\linewidth,center}
        \subcaptionbox*{vulnerability detection}{%
            \includegraphics[width=0.32\textwidth]{dt.png}
        }
        \subcaptionbox*{vulnerability exploitation}{%
            \includegraphics[width=0.32\textwidth]{exp.png}
        }
        \subcaptionbox*{vulnerability fixing}{%
            \includegraphics[width=0.32\textwidth]{fx.png}
        }
        \end{adjustbox}

        \begin{adjustbox}{width=\linewidth,center}
        \subcaptionbox*{vulnerability localisation}{%
            \includegraphics[width=0.32\textwidth]{lz.png}
        }
        \subcaptionbox*{vulnerability categorization}{%
            \includegraphics[width=0.32\textwidth]{cz.png}
        }
        \subcaptionbox*{hardening}{%
            \includegraphics[width=0.32\textwidth]{hr.png}
        }
         \end{adjustbox}
         
        \begin{adjustbox}{width=0.32\linewidth,center}
        \subcaptionbox*{Mitigation}{%
            \includegraphics[width=0.32\textwidth]{mg.png}
        }
        
        \end{adjustbox}
    \caption{The trend of objectives research}
    \label{fig:trends}
\end{figure}

As shown in Figure\ref{fig:trends_app}, the prevalence of static analysis over dynamic testing indicates a prioritization of methods that can rapidly detect vulnerabilities early in the development lifecycle without the overhead of establishing runtime environments.

% Figure\ref{fig:trends_app} reveals that static analysis predominates over dynamic testing in the studies examined. This trend suggests a strong preference for scanning methods that focus on examining source code or configurations without executing the software, possibly because of their ability to detect vulnerabilities without requiring a complex runtime environment.
\begin{figure}[ht!]
    \centering
    \begin{adjustbox}{width=0.64\linewidth,center}
        \subcaptionbox*{static analysis}{%
            \includegraphics[width=0.32\textwidth]{stc.png}
        }
        \subcaptionbox*{dynamic testing}{%
            \includegraphics[width=0.32\textwidth]{dyn.png}
        }
        \end{adjustbox}
    \caption{Testing and analysis approach}
    \label{fig:trends_app}
\end{figure}

Figure\ref{fig:trends_type} illustrates the different types of tests used in our study. Black-box testing is widely used in the analyzed studies, while grey-box and white-box tests are less common.

\begin{figure}[ht!]
    \centering
    \begin{adjustbox}{width=\linewidth,center}
        \subcaptionbox*{black-box}{%
            \includegraphics[width=0.32\textwidth]{blk.png}
        }
        \subcaptionbox*{grey-box}{%
            \includegraphics[width=0.32\textwidth]{gry.png}
        }
        \subcaptionbox*{white-box}{%
            \includegraphics[width=0.32\textwidth]{wb.png}
        }
        \end{adjustbox}
    \caption{The trend of testing types}
    \label{fig:trends_type}
\end{figure}

We found that fuzzing and model checking are the predominant testing and analysis methodologies, collectively accounting for over 42\% of identified approaches, as illustrated in Figure\ref{fig:trends_meth}. The prevalence of fuzzing underscores its value in uncovering unexpected errors by injecting random data, while the adoption of model checking emphasizes the importance of formally verifying specific system properties. This combined emphasis on exploratory and formal methods indicates a growing recognition of the need for comprehensive and multifaceted approaches to identify complex vulnerabilities in the SDN.

\begin{figure}[ht!]
    \centering
    \begin{adjustbox}{width=\linewidth,center}
    \centering
        \subcaptionbox*{fuzzing testing}{%
            \includegraphics[width=0.32\textwidth]{fz.png}
        }
        \subcaptionbox*{model-based testing}{%
            \includegraphics[width=0.32\textwidth]{mb.png}
        }
        \subcaptionbox*{symbolic execution}{%
            \includegraphics[width=0.32\textwidth]{se.png}
        } 
         \end{adjustbox}

        \begin{adjustbox}{width=\linewidth,center}
        \subcaptionbox*{theorem proving}{%
            \includegraphics[width=0.32\textwidth]{tp.png}
        }
        \subcaptionbox*{model checking}{%
            \includegraphics[width=0.32\textwidth]{mc.png}
        }
        \subcaptionbox*{machine learning}{%
            \includegraphics[width=0.32\textwidth]{ml.png}
        }
        \end{adjustbox}
        
        \begin{adjustbox}{width=\linewidth,center}
        \subcaptionbox*{differential checking}{%
            \includegraphics[width=0.32\textwidth]{dc.png}
        }
        \subcaptionbox*{Natural Language Processing}{%
            \includegraphics[width=0.32\textwidth]{nlp.png}
        }
        \subcaptionbox*{control flow analyzer}{%
            \includegraphics[width=0.32\textwidth]{cfa.png}
        }
        \end{adjustbox}

        \begin{adjustbox}{width=0.32\linewidth,center}
        \subcaptionbox*{data Flow analyzer}{%
            \includegraphics[width=0.32\textwidth]{dfa.png}
        }
        \end{adjustbox}
    \caption{The trend of methodologies}
    \label{fig:trends_meth}
\end{figure}


Although vulnerability detection and testing strategies remain key for SDN security, there has been a decrease in related research and publications in recent years. This trend may indicate a shift in research focus or a potential saturation of existing approaches, prompting a reassessment of research priorities and exploration of novel methodologies or related areas.

\subsection{Future Challenges}
Our discussion highlights the significant contributions of each selected publication. Some authors have raised open questions and challenges for the future to stimulate further research. We summarize these key challenges below:
\subsubsection{Automatic Controller-Specific Dependency Analysis} The current implementation of tools like AudiSDN relies on manual analysis to build dependency trees for different SDN controllers. Automating this process would involve developing program analysis modules capable of extracting unique elements and dependencies from various controller implementations, enhancing scalability, and reducing manual effort.
\subsubsection{AI-Enhanced Fuzzing for SDN Systems} Integrating AI with the fuzzing method can improve SDN security testing. AI can predict potential system failures, generating test cases that uncover hidden vulnerabilities. Additionally, AI-driven bug models can enable proactive mitigation strategies, enhancing SDN networks' overall resilience and security. Further research is needed to explore and refine these AI-enhanced fuzzing methods.

\subsubsection{Vulnerability Analysis Tools} The diverse landscape of SDN controllers, utilizing OpenFlow, APIs, or other protocols, necessitates flexible analysis tools. Current tools are often limited to specific controller types, hindering comprehensive vulnerability assessment. Future research should prioritize the development of more tools capable of analyzing source code from various controllers. This adaptability would provide a more complete picture of emerging vulnerabilities, ensuring the security of the entire SDN ecosystem.
