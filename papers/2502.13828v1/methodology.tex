\section{Methodology}\label{sec:meth}
Our review is conducted under systematic literature review (SLR) commonly used in software engineering research\citep{ref1}. An SLR involves identifying, evaluating, synthesizing, and analyzing existing research on a specific topic. This approach facilitates a comprehensive understanding of current knowledge and existing research gaps. Our methodology is inspired by a previous study\citep{ref2}, which is widely used and accepted.

Figure\ref{fig_pr} illustrates the key steps involved in our SLR process:
\begin{list}{}{}
\item{\textbf{Define Scope and Formulate Research Questions:} We define the scope of our research, focusing specifically on software security within SDN. We then formulate clear research questions to guide our search for relevant publications (detailed in Section \ref{sec:rqs}).}
\item{\textbf{Construct Search Query:} Based on our research questions, we enumerate a comprehensive set of keywords and their synonyms to maximize the breadth of our search (Table \ref{tab:t2}). We combine these keywords into a structured search string using Boolean operators, enabling efficient retrieval of relevant publications from major digital libraries (IEEE Xplore, ACM Digital Library, Springer, and ScienceDirect).}
\item{\textbf{Screen and Select Publications:} The initial set of retrieved publications undergoes a multi-step filtering process:
\begin{enumerate}
    \item Preliminary Filtering (Criteria 2, 3 \& 4): An initial filter removes irrelevant publications such as duplicated and non-English publications. 
    \item Screening (Criteria 5): We screen each article's title, abstract, and conclusions based on predefined exclusion criteria to assess relevance.
    \item Skimming (Criteria 6 \& 1): Finally, we perform a detailed analysis of the shortlisted publications, thoroughly examining each article about the research questions.
\end{enumerate}}
\item{\textbf{Apply Snowballing:} We employ forward and backward snowballing. This involves examining the reference lists of included studies and identifying publications that cite them to ensure comprehensiveness. We apply the same exclusion criteria to these newly identified publications.}
\item{\textbf{Quality Assessment:} We establish quality assessment criteria to ensure the inclusion of only the most relevant and impactful research.}
\item{\textbf{Data Extraction:} %We extract information pertinent to our research questions from each publication using a structured protocol.
We put a structured data extraction protocol to gather information relevant to our research questions from each publication.}
\item{\textbf{Report Findings:} The final step involves synthesizing and reporting the findings from the SLR to the research community, contributing to the broader knowledge base in SDN software security.}
\end{list}
\begin{figure}[ht!]
  \centering
  \includegraphics[width=\linewidth]{process1.png}
  \caption{Summary of our SLR process.}
    \label{fig_pr}
 
\end{figure}

\subsection{Research Question}\label{sec:rqs}
Our research questions (RQs) are the foundation of our SLR, guiding our exploration of key areas on software security in SDN. We formulate them using the PICOC criteria (Population, Intervention, Comparison, Outcomes, and Context) as outlined by\cite{kitchenham2007guidelines} (see Table\ref{tab:t1}).
\begin{table}[ht!]
\centering
\caption{Overview of PICOC
\label{tab:t1}}
%\begin{adjustbox}{width=\linewidth,center}
\begin{tabular}{cl}
\toprule
\textbf{Population}   & Software security of SDN                                                                                                      \\ \midrule
\textbf{Intervention} & \begin{tabular}[c]{@{}l@{}}Software, software application, defect, bug, vulnerability, fix, localize\end{tabular}          \\ 
\textbf{Comparison}   & n/a                                                                                                                           \\ 
\textbf{Outcomes}     & \begin{tabular}[c]{@{}l@{}}Testing and analysis strategies, bug taxonomy, and software security\\ trends in SDN\end{tabular} \\ 
\textbf{Context}      & Research on industry and academic                                                                                             \\ 
\bottomrule
\end{tabular}
%\end{adjustbox}
\end{table}

\begin{list}{}{}
\item \textbf{RQ1: What are the current trends in the literature on SDN software security?} As a programmable network, SDN faces security challenges due to its software-based nature. Understanding these challenges is important for identifying the focus areas of researchers. This research question aims to reveal the trends in the security of SDN software .
\item\textbf{RQ2: What testing and analysis strategies are currently employed for SDN software?} This research question explores the testing techniques and analysis methods used to identify security bugs within SDN software, including its code, applications, APIs, and other components. 

%This research question investigates testing techniques and analysis approaches to discover security bugs in the code, applications, APIs, and other components. The goal is to establish best practices for ensuring security and assessing software maturity in SDN environments.}
\item\textbf{RQ3: What is the taxonomy of vulnerabilities in SDN software?} This question focuses on the approaches used to classify vulnerabilities. It also examines existing frameworks or models used to classify vulnerabilities, the criteria used to determine the severity of vulnerabilities, and how these classifications influence the development of mitigation strategies and security policies within the SDN ecosystem.
\end{list}


\subsection{Construct research a query}
We construct the search string using keywords derived from our research questions. We then classify these keywords into three groups based on a manual review of relevant publications (Table\ref{tab:t2}).
Our approach utilizes Boolean operators to create individual strings (denoted as Ci, where i represents the group number) for each category. These strings combine keywords within a group using the OR operator (e.g., $C1 = k1$ OR $k2$ OR $k3 ...$ OR $kn$). The final search string is formed by combining these category-specific strings using the AND operator ($finalString = C1$ AND $C2$ AND $C3$). This final string is then used for our systematic literature search.

\begin{table}[ht!]
\centering
\caption{Search keywords}
\label{tab:t2}
%\begin{adjustbox}{width=\linewidth,center}
\begin{tabular}{cl}
\toprule
\textbf{Category} & \textbf{Keywords} \\ 
\midrule
1             & Software-Defined Network*, SDN  \\ 
2             & \begin{tabular}[c]{@{}l@{}}Test*, Analys*, Analyz*, Assess*, Verification,  Application, \\ Taxonomy, Detect*, Classification, categor* \end{tabular}\\ 
3             & \begin{tabular}[c]{@{}l@{}}Security, Vulnerab*, Threat, Bug*, Issue, Weak*, Mistake, Fault \end{tabular} \\ 
\bottomrule
\end{tabular}
%\end{adjustbox}
\end{table}


% \subsubsection*{Repository search}
%\verb|Search in repositories :| 
To maximize the retrieval of relevant publications, we target four most popular repositories: IEEE Xplore\footnote{https://ieeexplore.ieee.org/Xplore/home.jsp}, ACM Digital Library\footnote{https://dl.acm.org/}, ScienceDirect\footnote{https://www.sciencedirect.com/}, and Springer\footnote{https://www.springer.com/gp}. We employ advanced search functionalities within each repository, primarily focusing on publication titles and abstracts.
While our core search string remains consistent, we adapt it to accommodate the specific limitations of each repository. For instance, IEEE Xplore restricts keywords to 25, and ScienceDirect limits Boolean operators to 8. Without documented limitations for ACM Digital Library and Springer, we adopt the strictest criteria (25 keywords and eight operators) to ensure search efficiency across all platforms.

\subsection{Exclusion and Inclusion Criteria}
After the initial search, we obtained a comprehensive list of publications from the selected repositories. However, using broad keywords to maximize coverage retrieves many irrelevant Publications. To ensure a focused and reliable set of primary studies for our SLR, we apply the following criteria:
\begin{description} 
% \end{description}[label=\alph*)]
\item \textbf{Inclusion Criteria :}
\begin{enumerate}
    \item Publications are included if they focus on:
    \begin{itemize}
        \item Identification of vulnerabilities in SDN software.
        \item Test strategies and analysis methods specific to SDN software environments.
    \end{itemize}
\end{enumerate}
\item \textbf{Exclusion Criteria :}
\begin{enumerate}[start=2]
    \item We restrict our selection to English-language publications.
    \item We develop a Python script to identify and remove duplicates based on title, abstract, author names, and publication year. 
    \item We focus on SDN software security, excluding document types such as books, theses, and BookSections. We prioritize journal publications and conference proceedings.
    \item We employ a two-step filtering process. First, a script analyzes titles and abstracts to remove entries that do not match our search criteria. Second, we manually review the remaining titles and abstracts to exclude publications unrelated to SDN software security.
    \item We conduct a full-text review of the remaining publications to ensure that each study focuses on SDN software security.
\end{enumerate}
\end{description} 
% By applying these rigorous exclusion criteria, we systematically refine our initial dataset to encompass only studies directly relevant to SDN software security. Table\ref{tab:t3} summarizes this exclusion process.
Applying these exclusion and inclusion criteria, the initial dataset is systematically refined to encompass only studies directly relevant to SDN software security.
% We systematically refine our initial dataset by applying these rigorous exclusion and inclusion criteria. Table\ref{tab:t3} summarizes this exclusion and inclusion process.
\subsection{Snowballing}
A snowballing stage is incorporated into the publication selection process to ensure a comprehensive search. This method examines the references of publications meeting the inclusion and exclusion criteria (backward snowballing) and the publications citing them (forward snowballing). This iterative approach identified one additional relevant publication, resulting in 60 publications for our primary publication.

\subsection{Quality Assessment}
To ensure the relevance of the primary publication, we performed a quality assessment of each selected publication. This assessment focused on the following key questions:
\begin{itemize} 
\item Does the paper address at least one research question? 
\item Is the paper motivated? 
\item Are the research objectives clearly defined?
\item Does the paper provide a detailed description of the procedures used? 
\item Are the results described clearly and interpreted in the context of the objectives?
\item Does the paper discuss the key contributions? 
\end{itemize}
Publications that did not adequately meet these criteria were excluded from our SLR. Of the publications selected, two were excluded based on this quality assessment.
\subsection{Data Extraction}
A structured extraction form was defined and applied to all primary publications for consistent and accurate data collection. The extraction process focused on identifying the following information :
\begin{itemize} 
\item Software Security Issues in SDN: The specific vulnerabilities, threats, and risks associated with software components within SDN environments.
\item Vulnerability Management: The approaches employed to address software vulnerabilities, including fixing, detection, categorization, configuration hardening, and other relevant aspects.
\item Security Assessment Methods: The techniques employed to evaluate software security in SDN, such as dynamic testing, static analysis, and other applicable methods.
\end{itemize}
After extraction, the data form was reviewed to ensure the quality of the information and later synthesized into findings to report the review.

\subsection{Final primary publications selection}\label{sec:pps}
The systematic selection process, summarized in Figure\ref{fig_pr}, yielded 58 primary publications for our SLR. Table\ref{tab:flep} (see Appendix) summarizes these publications. Analysis of this final set reveals that 77.6\% were published in conference proceedings, while the remaining 22.4\% were published in journals.

\paragraph{\textbf{Data Availability} Our repository containing the data for our SLR can be found at \url{https://github.com/moustaphaeh/software_security_in_sdn.git}.}


