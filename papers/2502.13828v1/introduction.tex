\section{Introduction}
Our information-driven economy relies heavily on data networks as its technical foundation. However, traditional network architectures and technical evolution processes cannot support the complexity and pace of innovation required by emerging applications such as virtualized computing/cloud computing, the Internet of Things, ubiquitous mobile computing, and Big Data analysis\citep{mad1}. Indeed, the industry standard of deploying purpose-built, fixed-function hardware (e.g., routers, switches, firewalls, load balancers) to implement standardized protocols no longer aligns with the economic demands of modern virtualized computing\citep{mad1}. These barriers to innovation result in long design and development cycles for new network services, along with significant capital and operating expenditures for deploying and operating new network functions.

In response to this challenge, the industry has developed a novel initiative called ‘Software Defined Networking’ (SDN). SDN enables the network to be programmed by centralizing all its intelligence in the ‘controller’ software. This technology has been widely adopted in actual network deployments on the Internet. North America dominated the global software-defined networking market with a major share of over 35\%\citep{mad4}.

Despite the promising features of SDN, its adoption is challenged by security concerns, notably concerning attacks targeting the SDN\citep{mad3}. A large number of research works have addressed security issues related to SDN's network\citep{10157809} and architectural\citep{7593247}\citep{10226193} aspects. Generally, they conclude by emphasizing how the centralization of controllers and the high intensity of traffic make monitoring and detecting suspicious activities challenging.
In contrast, our work focuses on the software side of SDN, which has received substantially less attention regarding security analysis. Modern SDN controllers are complex software systems comprising millions of lines of code, with core networking functionalities implemented in software and deployed on the controllers. SDN applications also use the SDN controller API to request or configure network services. Like any other software, SDN software is susceptible to vulnerabilities (also called "bugs" in this paper). Recent studies by key players such as Google\citep{10.1145/2934872.2934891}, Facebook\citep{10.1145/3230543.3230546}, and Microsoft\citep{10.1145/3132747.3132759} indicate that software bugs account for 30\% of outages in their SDN deployments. Given the increasing evidence from industry and open-source bug analysis, a more systematic and detailed study of software vulnerabilities within the SDN ecosystem is needed.

This paper conducts a systematic literature review (SLR) focusing on research works that target software security in SDN and offer a comprehensive overview of the state of the art. After carefully identifying all related research publications, we perform a trend analysis and provide a detailed overview of SDN software security's critical aspects, such as vulnerability management characteristics, testing and analysis strategies, and some empirical results based on the vulnerability taxonomy. Finally, we summarise the current limitations of SDN software security and indicate potential new research directions.

The rest of the paper follows this organization: Section\ref{sec:bckg} explains the necessary background on SDN. Section\ref{sec:meth} describes the methodology of this SLR, including an overview and detailed review processes of our approach. In Section\ref{sec:tx}, we present our taxonomy of research. Next, we present results and discuss the trends observed and the challenges that the community should attempt to address in the following two sections, Sections\ref{sec:rslt} and \ref{sec:disc}. We then list the threats to the validity of this SLR in Section\ref{sec:ttv}. Section\ref{sec:rw} compares this document with studies in the literature, and finally, we conclude this SLR in Section\ref{sec:con}.

%The rest of the paper follows this organization: Section\ref{sec:bckg} explains the necessary background on the security of SDN Software. Section\ref{sec:meth} describes the methodology of this SLR, including an overview and detailed review processes of our approach. In Section\ref{sec:pps}, we present the results of our selected primary publications. Next, we present our taxonomy of research and corresponding results in the following two sections, Sections\ref{sec:tx} and \ref{sec:rslt}. We then discuss the trends observed and the challenges that the community should attempt to address in Section\ref{sec:disc} and list the threats to the validity of this SLR in Section\ref{sec:ttv}. Section\ref{sec:rw} compares this document with studies in the literature, and finally, we conclude this SLR in Section\ref{sec:con}.