%%%%%%%% ICML 2025 EXAMPLE LATEX SUBMISSION FILE %%%%%%%%%%%%%%%%%

\documentclass{article}

\usepackage{ifthen}
\newboolean{public_version}
\setboolean{public_version}{true}

% Recommended, but optional, packages for figures and better typesetting:
\usepackage{microtype}
\usepackage{graphicx}
%\usepackage{subfigure}
\usepackage{booktabs} % for professional tables

% hyperref makes hyperlinks in the resulting PDF.
% If your build breaks (sometimes temporarily if a hyperlink spans a page)
% please comment out the following usepackage line and replace
% \usepackage{icml2025} with \usepackage[nohyperref]{icml2025} above.
\usepackage{hyperref}


% Attempt to make hyperref and algorithmic work together better:
\newcommand{\theHalgorithm}{\arabic{algorithm}}

% Use the following line for the initial blind version submitted for review:
%\usepackage{icml2025} \setboolean{public_version}{false}

% If accepted, instead use the following line for the camera-ready submission:
%\usepackage[accepted]{icml2025}

\usepackage[accepted]{icml2025v2}
%\usepackage[accepted]{shmicml2025}

% For theorems and such
\usepackage{amsmath}
\usepackage{amssymb}
\usepackage{mathtools}
\usepackage{amsthm}
\usepackage{bbm}

\usepackage{comment}
\definecolor{shadecolor}{RGB}{176,224,230}
%\usepackage[table]{xcolor}
%\usepackage[table,xcdraw]{xcolor}
\usepackage{tcolorbox}
\usepackage{appendix}
\usepackage{enumitem}


\usepackage{subcaption}
\usepackage{pifont}

\newcommand{\ck}[1]{\textcolor{cyan}{CK: #1}}
\newcommand{\jg}[1]{\textcolor{red}{JG: #1}}
\newcommand{\yh}[1]{\textcolor{blue}{YH: #1}}
\newcommand{\jc}[1]{\textcolor{orange}{JC: #1}}

% if you use cleveref..
\usepackage[capitalize,noabbrev]{cleveref}

%%%%%%%%%%%%%%%%%%%%%%%%%%%%%%%%
% THEOREMS
%%%%%%%%%%%%%%%%%%%%%%%%%%%%%%%%
\theoremstyle{plain}
\newtheorem{theorem}{Theorem}[section]
\newtheorem{proposition}[theorem]{Proposition}
\newtheorem{lemma}[theorem]{Lemma}
\newtheorem{corollary}[theorem]{Corollary}
\theoremstyle{definition}
\newtheorem{definition}[theorem]{Definition}
\newtheorem{assumption}[theorem]{Assumption}
\theoremstyle{remark}
\newtheorem{remark}[theorem]{Remark}

% Todonotes is useful during development; simply uncomment the next line
%    and comment out the line below the next line to turn off comments
%\usepackage[disable,textsize=tiny]{todonotes}
\usepackage[textsize=tiny]{todonotes}


% The \icmltitle you define below is probably too long as a header.
% Therefore, a short form for the running title is supplied here:
\icmltitlerunning{Controlling Neural Collapse Enhances Out-of-Distribution Detection and Transfer Learning}

\begin{document}

\twocolumn[
\icmltitle{Controlling Neural Collapse Enhances \\
           Out-of-Distribution Detection and Transfer Learning}

% It is OKAY to include author information, even for blind
% submissions: the style file will automatically remove it for you
% unless you've provided the [accepted] option to the icml2025
% package.

% List of affiliations: The first argument should be a (short)
% identifier you will use later to specify author affiliations
% Academic affiliations should list Department, University, City, Region, Country
% Industry affiliations should list Company, City, Region, Country

% You can specify symbols, otherwise they are numbered in order.
% Ideally, you should not use this facility. Affiliations will be numbered
% in order of appearance and this is the preferred way.
\icmlsetsymbol{equal}{*}

\begin{icmlauthorlist}
\icmlauthor{Md Yousuf Harun}{rit}
\icmlauthor{Jhair Gallardo}{rit}
\icmlauthor{Christopher Kanan}{roc}
%\icmlauthor{}{sch}
%\icmlauthor{}{sch}
\end{icmlauthorlist}

\icmlaffiliation{rit}{Rochester Institute of Technology}
\icmlaffiliation{roc}{University of Rochester}

\icmlcorrespondingauthor{Md Yousuf Harun}{mh1023@rit.edu}

% You may provide any keywords that you
% find helpful for describing your paper; these are used to populate
% the "keywords" metadata in the PDF but will not be shown in the document
\icmlkeywords{Neural Collapse, Out-of-Distribution Detection, Out-of-Distribution Detection, Transfer Learning, ICML}

\vskip 0.3in
]

% this must go after the closing bracket ] following \twocolumn[ ...

% This command actually creates the footnote in the first column
% listing the affiliations and the copyright notice.
% The command takes one argument, which is text to display at the start of the footnote.
% The \icmlEqualContribution command is standard text for equal contribution.
% Remove it (just {}) if you do not need this facility.

\ifthenelse{\boolean{public_version}}{
\printAffiliationsAndNotice{}  % leave blank if no need to mention equal contribution
%\printAffiliationsAndNotice{\icmlEqualContribution} % otherwise use the standard text.
}





%%%%%%%%%%%%%%%%%%%



%%%%%%%%%%%%%%%%


Large language model (LLM)-based agents have shown promise in tackling complex tasks by interacting dynamically with the environment. 
Existing work primarily focuses on behavior cloning from expert demonstrations and preference learning through exploratory trajectory sampling. However, these methods often struggle in long-horizon tasks, where suboptimal actions accumulate step by step, causing agents to deviate from correct task trajectories.
To address this, we highlight the importance of \textit{timely calibration} and the need to automatically construct calibration trajectories for training agents. We propose \textbf{S}tep-Level \textbf{T}raj\textbf{e}ctory \textbf{Ca}libration (\textbf{\model}), a novel framework for LLM agent learning. 
Specifically, \model identifies suboptimal actions through a step-level reward comparison during exploration. It constructs calibrated trajectories using LLM-driven reflection, enabling agents to learn from improved decision-making processes. These calibrated trajectories, together with successful trajectory data, are utilized for reinforced training.
Extensive experiments demonstrate that \model significantly outperforms existing methods. Further analysis highlights that step-level calibration enables agents to complete tasks with greater robustness. 
Our code and data are available at \url{https://github.com/WangHanLinHenry/STeCa}.
%\section{Introduction}

Despite the remarkable capabilities of large language models (LLMs)~\cite{DBLP:conf/emnlp/QinZ0CYY23,DBLP:journals/corr/abs-2307-09288}, they often inevitably exhibit hallucinations due to incorrect or outdated knowledge embedded in their parameters~\cite{DBLP:journals/corr/abs-2309-01219, DBLP:journals/corr/abs-2302-12813, DBLP:journals/csur/JiLFYSXIBMF23}.
Given the significant time and expense required to retrain LLMs, there has been growing interest in \emph{model editing} (a.k.a., \emph{knowledge editing})~\cite{DBLP:conf/iclr/SinitsinPPPB20, DBLP:journals/corr/abs-2012-00363, DBLP:conf/acl/DaiDHSCW22, DBLP:conf/icml/MitchellLBMF22, DBLP:conf/nips/MengBAB22, DBLP:conf/iclr/MengSABB23, DBLP:conf/emnlp/YaoWT0LDC023, DBLP:conf/emnlp/ZhongWMPC23, DBLP:conf/icml/MaL0G24, DBLP:journals/corr/abs-2401-04700}, 
which aims to update the knowledge of LLMs cost-effectively.
Some existing methods of model editing achieve this by modifying model parameters, which can be generally divided into two categories~\cite{DBLP:journals/corr/abs-2308-07269, DBLP:conf/emnlp/YaoWT0LDC023}.
Specifically, one type is based on \emph{Meta-Learning}~\cite{DBLP:conf/emnlp/CaoAT21, DBLP:conf/acl/DaiDHSCW22}, while the other is based on \emph{Locate-then-Edit}~\cite{DBLP:conf/acl/DaiDHSCW22, DBLP:conf/nips/MengBAB22, DBLP:conf/iclr/MengSABB23}. This paper primarily focuses on the latter.

\begin{figure}[t]
  \centering
  \includegraphics[width=0.48\textwidth]{figures/demonstration.pdf}
  \vspace{-4mm}
  \caption{(a) Comparison of regular model editing and EAC. EAC compresses the editing information into the dimensions where the editing anchors are located. Here, we utilize the gradients generated during training and the magnitude of the updated knowledge vector to identify anchors. (b) Comparison of general downstream task performance before editing, after regular editing, and after constrained editing by EAC.}
  \vspace{-3mm}
  \label{demo}
\end{figure}

\emph{Sequential} model editing~\cite{DBLP:conf/emnlp/YaoWT0LDC023} can expedite the continual learning of LLMs where a series of consecutive edits are conducted.
This is very important in real-world scenarios because new knowledge continually appears, requiring the model to retain previous knowledge while conducting new edits. 
Some studies have experimentally revealed that in sequential editing, existing methods lead to a decrease in the general abilities of the model across downstream tasks~\cite{DBLP:journals/corr/abs-2401-04700, DBLP:conf/acl/GuptaRA24, DBLP:conf/acl/Yang0MLYC24, DBLP:conf/acl/HuC00024}. 
Besides, \citet{ma2024perturbation} have performed a theoretical analysis to elucidate the bottleneck of the general abilities during sequential editing.
However, previous work has not introduced an effective method that maintains editing performance while preserving general abilities in sequential editing.
This impacts model scalability and presents major challenges for continuous learning in LLMs.

In this paper, a statistical analysis is first conducted to help understand how the model is affected during sequential editing using two popular editing methods, including ROME~\cite{DBLP:conf/nips/MengBAB22} and MEMIT~\cite{DBLP:conf/iclr/MengSABB23}.
Matrix norms, particularly the L1 norm, have been shown to be effective indicators of matrix properties such as sparsity, stability, and conditioning, as evidenced by several theoretical works~\cite{kahan2013tutorial}. In our analysis of matrix norms, we observe significant deviations in the parameter matrix after sequential editing.
Besides, the semantic differences between the facts before and after editing are also visualized, and we find that the differences become larger as the deviation of the parameter matrix after editing increases.
Therefore, we assume that each edit during sequential editing not only updates the editing fact as expected but also unintentionally introduces non-trivial noise that can cause the edited model to deviate from its original semantics space.
Furthermore, the accumulation of non-trivial noise can amplify the negative impact on the general abilities of LLMs.

Inspired by these findings, a framework termed \textbf{E}diting \textbf{A}nchor \textbf{C}ompression (EAC) is proposed to constrain the deviation of the parameter matrix during sequential editing by reducing the norm of the update matrix at each step. 
As shown in Figure~\ref{demo}, EAC first selects a subset of dimension with a high product of gradient and magnitude values, namely editing anchors, that are considered crucial for encoding the new relation through a weighted gradient saliency map.
Retraining is then performed on the dimensions where these important editing anchors are located, effectively compressing the editing information.
By compressing information only in certain dimensions and leaving other dimensions unmodified, the deviation of the parameter matrix after editing is constrained. 
To further regulate changes in the L1 norm of the edited matrix to constrain the deviation, we incorporate a scored elastic net ~\cite{zou2005regularization} into the retraining process, optimizing the previously selected editing anchors.

To validate the effectiveness of the proposed EAC, experiments of applying EAC to \textbf{two popular editing methods} including ROME and MEMIT are conducted.
In addition, \textbf{three LLMs of varying sizes} including GPT2-XL~\cite{radford2019language}, LLaMA-3 (8B)~\cite{llama3} and LLaMA-2 (13B)~\cite{DBLP:journals/corr/abs-2307-09288} and \textbf{four representative tasks} including 
natural language inference~\cite{DBLP:conf/mlcw/DaganGM05}, 
summarization~\cite{gliwa-etal-2019-samsum},
open-domain question-answering~\cite{DBLP:journals/tacl/KwiatkowskiPRCP19},  
and sentiment analysis~\cite{DBLP:conf/emnlp/SocherPWCMNP13} are selected to extensively demonstrate the impact of model editing on the general abilities of LLMs. 
Experimental results demonstrate that in sequential editing, EAC can effectively preserve over 70\% of the general abilities of the model across downstream tasks and better retain the edited knowledge.

In summary, our contributions to this paper are three-fold:
(1) This paper statistically elucidates how deviations in the parameter matrix after editing are responsible for the decreased general abilities of the model across downstream tasks after sequential editing.
(2) A framework termed EAC is proposed, which ultimately aims to constrain the deviation of the parameter matrix after editing by compressing the editing information into editing anchors. 
(3) It is discovered that on models like GPT2-XL and LLaMA-3 (8B), EAC significantly preserves over 70\% of the general abilities across downstream tasks and retains the edited knowledge better.
\section{Introduction}
\label{sec:intro}


\begin{figure}[t]
    \centering
    \includegraphics[width = 0.99\linewidth]{images/nc_ood_detect_transfer_corr_vgg_updated.png}
  \caption{In this paper, we show that there is a close inverse relationship between OOD detection and generalization with respect to the degree of representation collapse in DNN layers. This plot illustrates this relationship for VGG17 pretrained on ImageNet-100 using four OOD datasets, where we measure collapse and OOD performance for various layers. For OOD detection, there is a strong positive Pearson correlation ($R=0.77$) with the degree of neural collapse (NC1) in a DNN layer, whereas for OOD generalization, there is a strong negative correlation ($R=-0.60$). We rigorously examine this inverse relationship and propose a method to control NC at different layers. %This suggests that stronger neural collapse improves OOD detection, while weaker neural collapse enhances OOD generalization. $R$ denotes the Pearson correlation coefficient.
  %Neural collapse metric NC1 (lower values indicate stronger neural collapse) positively correlates with OOD detection error and negatively correlates with OOD transfer error. This implies the stronger the neural collapse, the lower the OOD detection error and vice-versa. And, the weaker the neural collapse, the lower the OOD transfer error. For this, we analyze different layers of VGG17 networks which are pre-trained on the ImageNet-100 (ID) dataset, and evaluated on four OOD datasets. %e.g., ImageNet-R-200, Flowers-102, NINCO-64, and STL-10. 
  %$R$ denotes the Pearson correlation coefficient.
  } 
  \label{fig:vis_abstract}
  \vspace{-0.21in}
\end{figure}



% \begin{figure}[t]
%     \centering
%     \includegraphics[width = 0.99\linewidth]{images/bar_plot_ood_nc_summary.png}
%   \caption{\textbf{Controlling neural collapse (NC) enhances OOD transfer at encoder and OOD detection at projector.} The encoder achieves higher OOD transfer (indicated by lower error averaged across 8 OOD datasets) while decreasing NC (higher scores indicate lower NC). On the contrary, the projector achieves higher OOD detection (indicated by lower FPR95 averaged across same 8 OOD datasets) while increasing NC (lower scores indicate higher NC). Therefore, NC shows a \emph{linear} relationship with OOD transfer and OOD detection. In nutshell, encoder is a good OOD generalizer but a bad OOD detector whereas opposite is true for projector. All values are percentages.} 
%   \label{fig:vis_abstract}
% \end{figure}




%\begin{figure}[t]
%    \centering
%    \includegraphics[width = 0.99\linewidth]{images/vgg17_tunnel_koleo.png}
%  \caption{\textbf{Mitigating the tunnel effect leads to improved generalization.} The tunnel effect causes impaired OOD generalization for linear probes trained on embeddings from later layers in overparameterized DNNs. Typically, a supervised learning (SL) model suffers from the tunnel effect, as shown by \textcolor{blue}{blue} curves, where OOD accuracy significantly degrades in top layers (9-16). Whereas, using mitigation approach such as KoLeo regularization mitigates the tunnel effect and improves OOD performance, as shown by \textcolor{orange}{orange} curves. In this comparison, SL achieves 15.14\% average accuracy over 8 OOD datasets whereas SL+KoLeo (27.15\%) improves average accuracy by absolute 12\%.} 
%  \label{fig:tunnel_effect_koleo}
%\end{figure}

Out-of-distribution (OOD) detection and OOD generalization are two fundamental challenges in deep learning. OOD detection enables deep neural networks (DNNs) to reject unfamiliar inputs, preventing overconfident mispredictions, while OOD generalization allows DNNs to transfer their knowledge to new distributions. For applications like open-world learning, where a DNN continuously encounters new concepts, both capabilities are essential: OOD detection enables new concepts to be detected, while OOD generalization facilitates forward transfer to improve learning of these new concepts. Despite their importance, these tasks have primarily been studied in isolation. Here, we empirically and theoretically demonstrate a link between both tasks and neural collapse (NC), as illustrated in Fig.~\ref{fig:vis_abstract}.


NC is a phenomenon where DNNs develop compact and structured class representations~\cite{papyan2020prevalence}. While NC was first identified in the final hidden layer, later work has found that it occurs to varying degrees in the last $K$ DNN layers~\cite{rangamani2023feature,harun2024what,sukenikneural2024}. NC has a major impact on both OOD detection and generalization. Strong NC improves OOD detection by forming tightly clustered class features that enhance separation between in-distribution (ID) and OOD data~\cite{haas2023linking, wu2024pursuing, ming2022poem}. Conversely, NC impairs OOD generalization by reducing feature diversity, making it harder to transfer knowledge to novel distributions~\cite{kothapalli2023neural, masarczyk2023tunnel,harun2024what}. However, past work has considered NC in the context of either OOD detection or OOD generalization \textit{individually}, leaving open the question of how NC affects both tasks \textit{simultaneously}. To the best of our knowledge, no prior work has theoretically or empirically examined this relationship.

Here, we establish that the NC exhibited by a DNN layer has an \textbf{inverse relationship} with OOD detection and OOD generalization: \textit{stronger NC improves OOD detection but degrades generalization, while weaker NC enhances generalization at the cost of detection performance}. This trade-off suggests that a single feature space cannot effectively optimize both tasks, motivating the need for a novel approach. %Furthermore, we extend prior work by analyzing NC across \textit{multiple layers} rather than just the final hidden layer, providing new insights into how its extent varies throughout a DNN and how it affects OOD performance.

We propose a framework that strategically controls NC at different DNN layers to optimize both OOD detection and OOD generalization. We introduce entropy regularization to mitigate NC in the encoder, improving feature diversity and enhancing generalization. Simultaneously, we leverage a fixed Simplex Equiangular Tight Frame (ETF) projector to induce NC in the classification layer, improving feature compactness and enhancing detection. This design enables our DNNs to \textit{decouple representations} for detection and generalization, optimizing both objectives simultaneously.

\textbf{Our key contributions are as follows:}
\begin{enumerate}[noitemsep, nolistsep, leftmargin=*]
    \item We present the first unified study linking \textit{Neural Collapse} to both OOD detection and OOD generalization, empirically demonstrating their inverse relationship and extending analyses of NC beyond the final hidden layer.
    \item We develop a theoretical framework explaining how \textbf{entropy regularization mitigates NC} for OOD generalization and how a \textbf{fixed Simplex ETF projector enforces NC} for OOD detection.

    \item In extensive experiments on diverse OOD datasets and DNN architectures, we demonstrate the efficacy of our method compared to baselines. 
\end{enumerate}





%\ck{Point out that stuff like background class regularization and outlier exposure and similar techniques are bad for our use-case, since they would impair OOD generalization and require us to know the distribution of all OOD samples which is intractable. Specifically cite this paper among others pointing out that the approach doesn't make sense for us: \url{https://arxiv.org/pdf/2405.17816}}


\section{Background}
\label{sec:background}
\subsection{OOD Detection}
\vspace{-0.5em}

OOD detection methods aim to separate ID and OOD samples by leveraging the differences between their feature representations. Most existing OOD detection methods are \emph{post-hoc}, meaning they apply a scoring function to a model trained exclusively on ID data, without modifying the training process~\cite{salehi2022a}. These methods inherently rely on the properties of the learned feature space to distinguish ID from OOD samples.

Post-hoc detection techniques can be broadly categorized based on the source of their confidence estimates. Density-based methods model the ID distribution probabilistically and classify low-density test points as OOD~\cite{lee2018simple, zisselman2020deep, choi2018waic, jiang2023diverse}. More commonly, confidence-based approaches estimate OOD likelihood using model outputs~\cite{hendrycks2016baseline, liang2017enhancing, liu2020energy}, feature statistics~\cite{sun2021react, zhu2022boosting, sun2022out}, or gradient-based information~\cite{huang2021importance, wu2024low, lee2023probing, igoe2022useful}.

Since post-hoc methods depend on the representations learned during ID training, their effectiveness is fundamentally constrained by the quality of those features~\cite{roady2020open}. Highly compact, well-separated ID representations generally improve OOD detection by reducing feature overlap with OOD samples. For example, \citet{haas2023linking} demonstrated that $L_2$ normalization of penultimate-layer features induces NC, enhancing ID-OOD separability. Similarly, \citet{wu2024pursuing} introduced a regularization loss that enforces orthogonality between ID and OOD representations, leveraging NC-like properties to improve detection.

Another representation learning approach is to learn representations explicitly tailored for OOD detection by incorporating OOD samples during training~\cite{wu2024pursuing, bai2023feed, katz2022training, ming2022poem}. These methods encourage models to assign lower confidence~\cite{hendrycks2018deep} or higher energy~\cite{liu2020energy} to OOD inputs. However, this approach presents significant challenges, as the space of possible OOD data is essentially infinite, making it impractical to represent all potential OOD variations. Moreover, strong OOD detection performance often comes at the cost of degraded OOD generalization~\cite{zhang2024best}, as representations optimized for separability may lack the diversity needed for adaptation to novel distributions.



\subsection{Transfer Learning and OOD Generalization}
\vspace{-0.5em}

Transfer learning and OOD generalization methods focus on learning features that remain effective across distribution shifts. Robust transfer is particularly important in open-world learning scenarios, where models must not only adapt to new distributions but also improve sample efficiency over time, a key requirement for continual learning. To facilitate generalization, techniques such as feature alignment~\cite{li2018domain, ahuja2021invariance, zhao2020domain, ming2024hypo}, ensemble/meta-learning~\cite{balaji2018metareg, li2018metalearning, li2019episodic, bui2021exploiting}, robust optimization~\cite{rame2022fishr, cha2021swad, krueger2021out, shi2021gradient}, data augmentation~\cite{nam2021reducing, nuriel2021permuted, zhou2020learning}, and feature disentanglement~\cite{zhang2022towards} have been proposed.

Key properties of learned features significantly impact generalization to unseen distributions. Studies examining factors that affect OOD generalization emphasize that feature diversity is essential for robustness~\cite{masarczyk2023tunnel, kornblith2021better, fang2024does, ramanujan2024connection, kolesnikov2020big, vishniakov2024convnet}. Notably, recent work~\cite{kothapalli2023neural, masarczyk2023tunnel, harun2024what} suggests that progressive feature compression in deeper layers, linked to NC emergence, can hinder generalization by reducing representation expressivity. %These findings indicate that controlling the degree of NC may be crucial for balancing generalization and transferability.

\subsection{Neural Collapse}
\label{sec:nc_background}
\vspace{-0.5em}

As noted earlier, NC arises when class features become tightly clustered, often converging toward a Simplex ETF~\cite{papyan2020prevalence, kothapalli2023neural, zhu2021geometric, han2022neural}. Initially, NC was studied primarily in the final hidden layer, but later work demonstrated that NC manifests to varying degrees in earlier layers as well~\cite{rangamani2023feature, harun2024what}. In image classification experiments, \citet{harun2024what} showed that the degree of intermediate NC is heavily influenced by the properties of the training data, including the number of ID classes, image resolution, and the use of augmentations.

NC can be characterized by four main properties:
\begin{enumerate}[noitemsep, nolistsep, leftmargin=*]
    \item \textbf{Feature Collapse} ($\mathcal{NC}1$): Features within each class concentrate around a single mean, exhibiting minimal intra-class variability.
    \item \textbf{Simplex ETF Structure} ($\mathcal{NC}2$): When centered at the global mean, class means lie on a hypersphere with maximal pairwise distances, forming a Simplex ETF.
    \item \textbf{Self-Duality} ($\mathcal{NC}3$): The last-layer classifiers align tightly with their corresponding class means, creating a nearly self-dual configuration.
    \item \textbf{Nearest Class Mean Decision} ($\mathcal{NC}4$): Classification behaves like a nearest-centroid scheme, assigning classes based on proximity to class means.
\end{enumerate}

While NC's structured representations can aid OOD detection by ensuring strong class separability~\cite{haas2023linking, wu2024pursuing}, the same compression may limit the feature diversity needed for generalization. One proposed explanation is the \emph{Tunnel Effect Hypothesis}~\cite{masarczyk2023tunnel}, which suggests that as features become increasingly compressed in deeper layers, generalization to unseen distributions is impeded. 

%Although prior works have studied NC independently in the contexts of OOD detection~\cite{haas2023linking, wu2024pursuing} and OOD generalization~\cite{kothapalli2023neural, masarczyk2023tunnel, harun2024what}, their impact on both tasks \emph{simultaneously} remains an open question. Our work is the first to establish a concrete connection between NC, OOD detection, and OOD generalization, offering new theoretical insights and empirical validation.


\vspace{-5pt}
\section{Method}
\label{sec:method}
\section{Overview}

\revision{In this section, we first explain the foundational concept of Hausdorff distance-based penetration depth algorithms, which are essential for understanding our method (Sec.~\ref{sec:preliminary}).
We then provide a brief overview of our proposed RT-based penetration depth algorithm (Sec.~\ref{subsec:algo_overview}).}



\section{Preliminaries }
\label{sec:Preliminaries}

% Before we introduce our method, we first overview the important basics of 3D dynamic human modeling with Gaussian splatting. Then, we discuss the diffusion-based 3d generation techniques, and how they can be applied to human modeling.
% \ZY{I stopp here. TBC.}
% \subsection{Dynamic human modeling with Gaussian splatting}
\subsection{3D Gaussian Splatting}
3D Gaussian splatting~\cite{kerbl3Dgaussians} is an explicit scene representation that allows high-quality real-time rendering. The given scene is represented by a set of static 3D Gaussians, which are parameterized as follows: Gaussian center $x\in {\mathbb{R}^3}$, color $c\in {\mathbb{R}^3}$, opacity $\alpha\in {\mathbb{R}}$, spatial rotation in the form of quaternion $q\in {\mathbb{R}^4}$, and scaling factor $s\in {\mathbb{R}^3}$. Given these properties, the rendering process is represented as:
\begin{equation}
  I = Splatting(x, c, s, \alpha, q, r),
  \label{eq:splattingGA}
\end{equation}
where $I$ is the rendered image, $r$ is a set of query rays crossing the scene, and $Splatting(\cdot)$ is a differentiable rendering process. We refer readers to Kerbl et al.'s paper~\cite{kerbl3Dgaussians} for the details of Gaussian splatting. 



% \ZY{I would suggest move this part to the method part.}
% GaissianAvatar is a dynamic human generation model based on Gaussian splitting. Given a sequence of RGB images, this method utilizes fitted SMPLs and sampled points on its surface to obtain a pose-dependent feature map by a pose encoder. The pose-dependent features and a geometry feature are fed in a Gaussian decoder, which is employed to establish a functional mapping from the underlying geometry of the human form to diverse attributes of 3D Gaussians on the canonical surfaces. The parameter prediction process is articulated as follows:
% \begin{equation}
%   (\Delta x,c,s)=G_{\theta}(S+P),
%   \label{eq:gaussiandecoder}
% \end{equation}
%  where $G_{\theta}$ represents the Gaussian decoder, and $(S+P)$ is the multiplication of geometry feature S and pose feature P. Instead of optimizing all attributes of Gaussian, this decoder predicts 3D positional offset $\Delta{x} \in {\mathbb{R}^3}$, color $c\in\mathbb{R}^3$, and 3D scaling factor $ s\in\mathbb{R}^3$. To enhance geometry reconstruction accuracy, the opacity $\alpha$ and 3D rotation $q$ are set to fixed values of $1$ and $(1,0,0,0)$ respectively.
 
%  To render the canonical avatar in observation space, we seamlessly combine the Linear Blend Skinning function with the Gaussian Splatting~\cite{kerbl3Dgaussians} rendering process: 
% \begin{equation}
%   I_{\theta}=Splatting(x_o,Q,d),
%   \label{eq:splatting}
% \end{equation}
% \begin{equation}
%   x_o = T_{lbs}(x_c,p,w),
%   \label{eq:LBS}
% \end{equation}
% where $I_{\theta}$ represents the final rendered image, and the canonical Gaussian position $x_c$ is the sum of the initial position $x$ and the predicted offset $\Delta x$. The LBS function $T_{lbs}$ applies the SMPL skeleton pose $p$ and blending weights $w$ to deform $x_c$ into observation space as $x_o$. $Q$ denotes the remaining attributes of the Gaussians. With the rendering process, they can now reposition these canonical 3D Gaussians into the observation space.



\subsection{Score Distillation Sampling}
Score Distillation Sampling (SDS)~\cite{poole2022dreamfusion} builds a bridge between diffusion models and 3D representations. In SDS, the noised input is denoised in one time-step, and the difference between added noise and predicted noise is considered SDS loss, expressed as:

% \begin{equation}
%   \mathcal{L}_{SDS}(I_{\Phi}) \triangleq E_{t,\epsilon}[w(t)(\epsilon_{\phi}(z_t,y,t)-\epsilon)\frac{\partial I_{\Phi}}{\partial\Phi}],
%   \label{eq:SDSObserv}
% \end{equation}
\begin{equation}
    \mathcal{L}_{\text{SDS}}(I_{\Phi}) \triangleq \mathbb{E}_{t,\epsilon} \left[ w(t) \left( \epsilon_{\phi}(z_t, y, t) - \epsilon \right) \frac{\partial I_{\Phi}}{\partial \Phi} \right],
  \label{eq:SDSObservGA}
\end{equation}
where the input $I_{\Phi}$ represents a rendered image from a 3D representation, such as 3D Gaussians, with optimizable parameters $\Phi$. $\epsilon_{\phi}$ corresponds to the predicted noise of diffusion networks, which is produced by incorporating the noise image $z_t$ as input and conditioning it with a text or image $y$ at timestep $t$. The noise image $z_t$ is derived by introducing noise $\epsilon$ into $I_{\Phi}$ at timestep $t$. The loss is weighted by the diffusion scheduler $w(t)$. 
% \vspace{-3mm}

\subsection{Overview of the RTPD Algorithm}\label{subsec:algo_overview}
Fig.~\ref{fig:Overview} presents an overview of our RTPD algorithm.
It is grounded in the Hausdorff distance-based penetration depth calculation method (Sec.~\ref{sec:preliminary}).
%, similar to that of Tang et al.~\shortcite{SIG09HIST}.
The process consists of two primary phases: penetration surface extraction and Hausdorff distance calculation.
We leverage the RTX platform's capabilities to accelerate both of these steps.

\begin{figure*}[t]
    \centering
    \includegraphics[width=0.8\textwidth]{Image/overview.pdf}
    \caption{The overview of RT-based penetration depth calculation algorithm overview}
    \label{fig:Overview}
\end{figure*}

The penetration surface extraction phase focuses on identifying the overlapped region between two objects.
\revision{The penetration surface is defined as a set of polygons from one object, where at least one of its vertices lies within the other object. 
Note that in our work, we focus on triangles rather than general polygons, as they are processed most efficiently on the RTX platform.}
To facilitate this extraction, we introduce a ray-tracing-based \revision{Point-in-Polyhedron} test (RT-PIP), significantly accelerated through the use of RT cores (Sec.~\ref{sec:RT-PIP}).
This test capitalizes on the ray-surface intersection capabilities of the RTX platform.
%
Initially, a Geometry Acceleration Structure (GAS) is generated for each object, as required by the RTX platform.
The RT-PIP module takes the GAS of one object (e.g., $GAS_{A}$) and the point set of the other object (e.g., $P_{B}$).
It outputs a set of points (e.g., $P_{\partial B}$) representing the penetration region, indicating their location inside the opposing object.
Subsequently, a penetration surface (e.g., $\partial B$) is constructed using this point set (e.g., $P_{\partial B}$) (Sec.~\ref{subsec:surfaceGen}).
%
The generated penetration surfaces (e.g., $\partial A$ and $\partial B$) are then forwarded to the next step. 

The Hausdorff distance calculation phase utilizes the ray-surface intersection test of the RTX platform (Sec.~\ref{sec:RT-Hausdorff}) to compute the Hausdorff distance between two objects.
We introduce a novel Ray-Tracing-based Hausdorff DISTance algorithm, RT-HDIST.
It begins by generating GAS for the two penetration surfaces, $P_{\partial A}$ and $P_{\partial B}$, derived from the preceding step.
RT-HDIST processes the GAS of a penetration surface (e.g., $GAS_{\partial A}$) alongside the point set of the other penetration surface (e.g., $P_{\partial B}$) to compute the penetration depth between them.
The algorithm operates bidirectionally, considering both directions ($\partial A \to \partial B$ and $\partial B \to \partial A$).
The final penetration depth between the two objects, A and B, is determined by selecting the larger value from these two directional computations.

%In the Hausdorff distance calculation step, we compute the Hausdorff distance between given two objects using a ray-surface-intersection test. (Sec.~\ref{sec:RT-Hausdorff}) Initially, we construct the GAS for both $\partial A$ and $\partial B$ to utilize the RT-core effectively. The RT-based Hausdorff distance algorithms then determine the Hausdorff distance by processing the GAS of one object (e.g. $GAS_{\partial A}$) and set of the vertices of the other (e.g. $P_{\partial B}$). Following the Hausdorff distance definition (Eq.~\ref{equation:hausdorff_definition}), we compute the Hausdorff distance to both directions ($\partial A \to \partial B$) and ($\partial B \to \partial A$). As a result, the bigger one is the final Hausdorff distance, and also it is the penetration depth between input object $A$ and $B$.


%the proposed RT-based penetration depth calculation pipeline.
%Our proposed methods adopt Tang's Hausdorff-based penetration depth methods~\cite{SIG09HIST}. The pipeline is divided into the penetration surface extraction step and the Hausdorff distance calculation between the penetration surface steps. However, since Tang's approach is not suitable for the RT platform in detail, we modified and applied it with appropriate methods.

%The penetration surface extraction step is extracting overlapped surfaces on other objects. To utilize the RT core, we use the ray-intersection-based PIP(Point-In-Polygon) algorithms instead of collision detection between two objects which Tang et al.~\cite{SIG09HIST} used. (Sec.~\ref{sec:RT-PIP})
%RT core-based PIP test uses a ray-surface intersection test. For purpose this, we generate the GAS(Geometry Acceleration Structure) for each object. RT core-based PIP test takes the GAS of one object (e.g. $GAS_{A}$) and a set of vertex of another one (e.g. $P_{B}$). Then this computes the penetrated vertex set of another one (e.g. $P_{\partial B}$). To calculate the Hausdorff distance, these vertex sets change to objects constructed by penetrated surface (e.g. $\partial B$). Finally, the two generated overlapped surface objects $\partial A$ and $\partial B$ are used in the Hausdorff distance calculation step.

Our goal is to increase the robustness of T2I models, particularly with rare or unseen concepts, which they struggle to generate. To do so, we investigate a retrieval-augmented generation approach, through which we dynamically select images that can provide the model with missing visual cues. Importantly, we focus on models that were not trained for RAG, and show that existing image conditioning tools can be leveraged to support RAG post-hoc.
As depicted in \cref{fig:overview}, given a text prompt and a T2I generative model, we start by generating an image with the given prompt. Then, we query a VLM with the image, and ask it to decide if the image matches the prompt. If it does not, we aim to retrieve images representing the concepts that are missing from the image, and provide them as additional context to the model to guide it toward better alignment with the prompt.
In the following sections, we describe our method by answering key questions:
(1) How do we know which images to retrieve? 
(2) How can we retrieve the required images? 
and (3) How can we use the retrieved images for unknown concept generation?
By answering these questions, we achieve our goal of generating new concepts that the model struggles to generate on its own.

\vspace{-3pt}
\subsection{Which images to retrieve?}
The amount of images we can pass to a model is limited, hence we need to decide which images to pass as references to guide the generation of a base model. As T2I models are already capable of generating many concepts successfully, an efficient strategy would be passing only concepts they struggle to generate as references, and not all the concepts in a prompt.
To find the challenging concepts,
we utilize a VLM and apply a step-by-step method, as depicted in the bottom part of \cref{fig:overview}. First, we generate an initial image with a T2I model. Then, we provide the VLM with the initial prompt and image, and ask it if they match. If not, we ask the VLM to identify missing concepts and
focus on content and style, since these are easy to convey through visual cues.
As demonstrated in \cref{tab:ablations}, empirical experiments show that image retrieval from detailed image captions yields better results than retrieval from brief, generic concept descriptions.
Therefore, after identifying the missing concepts, we ask the VLM to suggest detailed image captions for images that describe each of the concepts. 

\vspace{-4pt}
\subsubsection{Error Handling}
\label{subsec:err_hand}

The VLM may sometimes fail to identify the missing concepts in an image, and will respond that it is ``unable to respond''. In these rare cases, we allow up to 3 query repetitions, while increasing the query temperature in each repetition. Increasing the temperature allows for more diverse responses by encouraging the model to sample less probable words.
In most cases, using our suggested step-by-step method yields better results than retrieving images directly from the given prompt (see 
\cref{subsec:ablations}).
However, if the VLM still fails to identify the missing concepts after multiple attempts, we fall back to retrieving images directly from the prompt, as it usually means the VLM does not know what is the meaning of the prompt.

The used prompts can be found in \cref{app:prompts}.
Next, we turn to retrieve images based on the acquired image captions.

\vspace{-3pt}
\subsection{How to retrieve the required images?}

Given $n$ image captions, our goal is to retrieve the images that are most similar to these captions from a dataset. 
To retrieve images matching a given image caption, we compare the caption to all the images in the dataset using a text-image similarity metric and retrieve the top $k$ most similar images.
Text-to-image retrieval is an active research field~\cite{radford2021learning, zhai2023sigmoid, ray2024cola, vendrowinquire}, where no single method is perfect.
Retrieval is especially hard when the dataset does not contain an exact match to the query \cite{biswas2024efficient} or when the task is fine-grained retrieval, that depends on subtle details~\cite{wei2022fine}.
Hence, a common retrieval workflow is to first retrieve image candidates using pre-computed embeddings, and then re-rank the retrieved candidates using a different, often more expensive but accurate, method \cite{vendrowinquire}.
Following this workflow, we experimented with cosine similarity over different embeddings, and with multiple re-ranking methods of reference candidates.
Although re-ranking sometimes yields better results compared to simply using cosine similarity between CLIP~\cite{radford2021learning} embeddings, the difference was not significant in most of our experiments. Therefore, for simplicity, we use cosine similarity between CLIP embeddings as our similarity metric (see \cref{tab:sim_metrics}, \cref{subsec:ablations} for more details about our experiments with different similarity metrics).

\vspace{-3pt}
\subsection{How to use the retrieved images?}
Putting it all together, after retrieving relevant images, all that is left to do is to use them as context so they are beneficial for the model.
We experimented with two types of models; models that are trained to receive images as input in addition to text and have ICL capabilities (e.g., OmniGen~\cite{xiao2024omnigen}), and T2I models augmented with an image encoder in post-training (e.g., SDXL~\cite{podellsdxl} with IP-adapter~\cite{ye2023ip}).
As the first model type has ICL capabilities, we can supply the retrieved images as examples that it can learn from, by adjusting the original prompt.
Although the second model type lacks true ICL capabilities, it offers image-based control functionalities, which we can leverage for applying RAG over it with our method.
Hence, for both model types, we augment the input prompt to contain a reference of the retrieved images as examples.
Formally, given a prompt $p$, $n$ concepts, and $k$ compatible images for each concept, we use the following template to create a new prompt:
``According to these examples of 
$\mathord{<}c_1\mathord{>:<}img_{1,1}\mathord{>}, ... , \mathord{<}img_{1,k}\mathord{>}, ... , \mathord{<}c_n\mathord{>:<}img_{n,1}\mathord{>}, ... , $
$\mathord{<}img_{n,k}\mathord{>}$,
generate $\mathord{<}p\mathord{>}$'', 
where $c_i$ for $i\in{[1,n]}$ is a compatible image caption of the image $\mathord{<}img_{i,j}\mathord{>},  j\in{[1,k]}$. 

This prompt allows models to learn missing concepts from the images, guiding them to generate the required result. 

\textbf{Personalized Generation}: 
For models that support multiple input images, we can apply our method for personalized generation as well, to generate rare concept combinations with personal concepts. In this case, we use one image for personal content, and 1+ other reference images for missing concepts. For example, given an image of a specific cat, we can generate diverse images of it, ranging from a mug featuring the cat to a lego of it or atypical situations like the cat writing code or teaching a classroom of dogs (\cref{fig:personalization}).
\vspace{-2pt}
\begin{figure}[htp]
  \centering
   \includegraphics[width=\linewidth]{Assets/personalization.pdf}
   \caption{\textbf{Personalized generation example.}
   \emph{ImageRAG} can work in parallel with personalization methods and enhance their capabilities. For example, although OmniGen can generate images of a subject based on an image, it struggles to generate some concepts. Using references retrieved by our method, it can generate the required result.
}
   \label{fig:personalization}\vspace{-10pt}
\end{figure}



\section{Experiments}
\seclabel{experiments}
Our experiments are designed to test a) the extent to which open loop execution is an issue for precise mobile manipulation tasks, b) how effective are blind proprioceptive correction techniques, c) do object detectors and point trackers perform reliably enough in wrist camera images for reliable control, d) is occlusion by the end-effector an issue and how effectively can it be mitigated through the use of video in-painting models, and e) how does our proposed \name methodology compare to large-scale imitation learning? 


\subsection{Tasks and Experimental Setup}
We work with the Stretch RE2 robot. Stretch RE2 is a commodity mobile manipulator with a 5DOF arm mounted on top of a non-holomonic base. We upgrade the robot to use the Dex Wrist 3, which has an eye-in-hand RGB-D camera (Intel D405). 
We consider 3 task families for a total
of 6 different tasks: a) holding a knob to pull open a cabinet or drawer, b) holding a
handle to pull open a cabinet, and c) pushing on objects (light buttons, books
in a book shelf, and light switches). Our focus is on generalization. {\it
Therefore, we exclusively test on previously unseen instances, not used during
development in any way.} 
\figref{tasks} shows the instances that we test on. 

All tasks involve some precise manipulation, followed by execution of a motion
primitive. {\bf For the pushing tasks}, the precise motion is to get the
end-effector exactly at the indicated point and the motion primitive is to push
in the direction perpendicular to the surface and retract the end-effector 
upon contact. The robot is positioned such
that the target position is within the field of view of the wrist camera. A user
selects the point of pushing via a mouse click on the wrist camera image. The
goal is to push at the indicated location. Success is determined by whether the
push results in the desired outcome (light turns on / off or book gets pushed in). 
The original rubber gripper bends upon contact, we use a rigid known tool
that sticks out a bit. We take the geometry of the tool into account while servoing.

{\bf For the opening articulated object tasks}, the precise manipulation is grasping the
knob / handle, while the motion primitive is the whole-body motion that opens
the cupboard. Computing and executing this full body motion is difficult. We
adopt the modular approach to opening articulated objects (MOSART) from Gupta \etal~\cite{gupta2024opening} and invoke it
after the gripper has been placed around the knob / handle. The whole tasks 
starts out with the robot about 1.5m way from the target object, with the 
target object in view
from robot's head mounted camera. We use MOSART to compute articulation
parameters and convey the robot to a pre-grasp
location with the target handle in view of the wrist camera. At this point,
\name (or baseline) is used to center the gripper around the knob / handle, 
before resuming MOSART: extending the gripper till contact, close the gripper, and play rest of the predicted motion plan. Success is 
determined by whether the cabinet opens by more than $60^\circ$
or the drawer is pulled out by more than $24cm$, similar to the criteria used in \cite{gupta2024opening}.


For the precise manipulation part, all baselines consume the current and
previous RGB-D images from the wrist camera and output full body motor
commands.

% % Please add the following required packages to your document preamble:
% % \usepackage{graphicx}
% \begin{table*}[!ht]
% \centering
% \caption{}
% \label{tab:my-table}
% \resizebox{\textwidth}{!}{%
% \begin{tabular}{lcccccc}
% \toprule
%  & \multicolumn{2}{c}{ours} & \multicolumn{2}{c}{Gurobi} & \multicolumn{2}{c}{MOSEK} \\
%  & \multicolumn{1}{l}{time (s)} & \multicolumn{1}{l}{optimality gap (\%)} & \multicolumn{1}{l}{time (s)} & \multicolumn{1}{l}{optimality gap (\%)} & \multicolumn{1}{l}{time (s)} & \multicolumn{1}{l}{optimality gap (\%)} \\ \hline
% \begin{tabular}[c]{@{}l@{}}Linear Regression\\ Synthetic \\ (n=16000, p=16000)\end{tabular} & 57 & 0.0 & 3351 & - & 2148 & - \\ \hline
% \begin{tabular}[c]{@{}l@{}}Linear Regression\\ Cancer Drug Response\\ (n=822, p=2300)\end{tabular} & 47 & 0.0 & 1800 & 0.31 & 212 & 0.0 \\ \hline
% \begin{tabular}[c]{@{}l@{}}Logistic Regression\\ Synthetic\\ (n=16000, p=16000)\end{tabular} & 271 & 0.0 & N/A & N/A & 1800 & - \\ \hline
% \begin{tabular}[c]{@{}l@{}}Logistic Regression\\ Dorothea\\ (n=1150, p=91598)\end{tabular} & 62 & 0.0 & N/A & N/A & 600 & 0.0 \\
% \bottomrule
% \end{tabular}%
% }
% \end{table*}

% Please add the following required packages to your document preamble:
% \usepackage{multirow}
% \usepackage{graphicx}
\begin{table*}[]
\centering
\caption{Certifying optimality on large-scale and real-world datasets.}
\vspace{2mm}
\label{tab:my-table}
\resizebox{\textwidth}{!}{%
\begin{tabular}{llcccccc}
\toprule
 &  & \multicolumn{2}{c}{ours} & \multicolumn{2}{c}{Gurobi} & \multicolumn{2}{c}{MOSEK} \\
 &  & time (s) & opt. gap (\%) & time (s) & opt. gap (\%) & time (s) & opt. gap (\%) \\ \hline
\multirow{2}{*}{Linear Regression} & \begin{tabular}[c]{@{}l@{}}synthetic ($k=10, M=2$)\\ (n=16k, p=16k, seed=0)\end{tabular} & 79 & 0.0 & 1800 & - & 1915 & - \\ \cline{2-8}
 & \begin{tabular}[c]{@{}l@{}}Cancer Drug Response ($k=5, M=5$)\\ (n=822, p=2300)\end{tabular} & 41 & 0.0 & 1800 & 0.89 & 188 & 0.0 \\ \hline
\multirow{2}{*}{Logistic Regression} & \begin{tabular}[c]{@{}l@{}}Synthetic ($k=10, M=2$)\\ (n=16k, p=16k, seed=0)\end{tabular} & 626 & 0.0 & N/A & N/A & 2446 & - \\ \cline{2-8}
 & \begin{tabular}[c]{@{}l@{}}DOROTHEA ($k=15, M=2$)\\ (n=1150, p=91598)\end{tabular} & 91 & 0.0 & N/A & N/A & 634 & 0.0 \\
 \bottomrule
\end{tabular}%
}
% \vspace{-3mm}
\end{table*}

\begin{figure*}
\insertW{1.0}{figures/figure_6_cropped_brighten.pdf}
\caption{{\bf Comparison of \name with the open loop (eye-in-hand) baseline} for opening a cabinet with a knob. Slight errors in getting to the target cause the end-effector to slip off, leading to failure for the baseline, where as our method is able to successfully complete the task.}
\figlabel{rollout}
\end{figure*}

\begin{table}
\setlength{\tabcolsep}{8pt}
  \centering
  \resizebox{\linewidth}{!}{
  \begin{tabular}{lcccg}
  \toprule
                              & \multicolumn{2}{c}{\bf Knobs} & \bf Handle & \bf \multirow{2}{*}{\bf Total} \\
                              \cmidrule(lr){2-3} \cmidrule(lr){4-4}
                              & \bf Cabinets & \bf Drawer & \bf Cabinets & \\
  \midrule
  RUM~\cite{etukuru2024robot}  & 0/3    & 1/4         & 1/3         & 2/10 \\
  \name (Ours) & 2/3    & 2/4         & 3/3     &  7/10 \\
  \bottomrule
  \end{tabular}}
  \caption{Comparison of \name \vs RUM~\cite{etukuru2024robot}, a recent large-scale end-to-end imitation learning method trained on 1200 demos for opening cabinets and 525 demos for opening drawers across 40 different environments. Our evaluation spans objects from three environments across two buildings.}
  \tablelabel{rum}
\end{table}

\subsection{Baselines}
We compare against three other methods for the precise manipulation part of
these tasks. 
\subsubsection{Open Loop (Eye-in-Hand)} To assess the precision requirements of
the tasks and to set it in context with the manipulation capabilities of the
robot platform, this baseline uses open loop execution starting from estimates
for the 3D target position from the first wrist camera image.
\subsubsection{MOSART~\cite{gupta2024opening}}
The recent modular system for opening cabinets and drawers~\cite{gupta2024opening}
reports impressive performance with open-loop control (using the head camera from 1.5m away), combined with proprioception-based feedback to 
compensate for errors in perception and control when interacting with handles. 
We test if such correction is also sufficient for interacting with knobs. Note 
that such correction is not possible for the smaller buttons and pliable books.

\subsubsection{\name (no inpainting)} To understand how much of an issue
occlusion due to the end-effector is during manipulation, we ablate the use of
inpainting. %

\subsubsection{Robot Utility Models (RUM)~\cite{etukuru2024robot}}
For the opening articulated object tasks, we also compare to Robot Utility Models (RUM), 
a closed-loop imitation learning method recently proposed by Etukuru et al. \cite{etukuru2024robot}.
RUM is trained on a substantial dataset comprising expert demonstrations, including 
1,200 instances of cabinet opening and 525 of drawer opening, gathered from roughly 
40 different environments.
This dataset stands as the most extensive imitation 
learning dataset for articulated object manipulation to date, establishing RUM as a 
strong baseline for our evaluation.

Similar to our method, we use MOSART to compute articulation
parameters and convey the robot to a pre-grasp location
with the target handle in view of the wrist camera.
One of the assumptions of RUM is a good view of the handle.
To benefit RUM, we try out three different heights of the wrist camera,
and \textit{report the best result for RUM.}

\begin{figure*}
\insertW{1.0}{figures/figure_9_cropped_brighten.pdf}
\caption{{\bf \name \vs open loop (eye-in-hand) baseline for pushing on user-clicked points}. Slight errors in getting to the target cause failure, where as \name successfully turns the lights off. Note the quality of CoTracker's track ({\color{blue} blue dot}).}
\figlabel{rollout_v2}
\end{figure*}

\begin{figure*}
\insertW{1.0}{figures/figure_5_v2_cropped_brighten.pdf}
\caption{{\bf Comparison of \name with and without inpainting}. Erroneous detection without inpainting causes execution to fail, where as with inpainting the target is correctly detected leading to a successful grasp and a successful execution.}
\figlabel{rollouts2}
\end{figure*}


\subsection{Results}
\tableref{results} presents results from our experiments. 
Our training-free approach \name successfully 
solves over 85\% of task instances that we test on.
As noted, all these
tests were conducted on unseen object instances in unseen
environments that were not used for development in any way. We discuss our key
experimental findings below.

\subsubsection{Closing the loop is necessary for these precise tasks} 
While the proprioception-based strategies proposed in MOSART~\cite{gupta2024opening}
work out for handles, they are inadequate for targets like knobs and just
don't work for tasks like pushing buttons. Using estimates from the wrist
camera is better, but open loop execution still fails for knobs and pushing
buttons. 

\subsubsection{Vision models work reasonably well even on wrist camera images}
Inpainting works well on wrist camera images (see \figref{occlusion} and \figref{inpainting}).
Closing the loop using feedback from vision detectors and point trackers on
wrist camera images also work well, particularly when we use in-painted images.
See some examples detections and point tracks in \figref{rollout} and \figref{rollout_v2}. 
Detic~\cite{zhou2022detecting} was able to reliably detect the knobs and
handles and CoTracker~\cite{karaev2023cotracker} was able to successfully track
the point of interaction letting us solve 24/28 task instances.

\subsubsection{Erroneous detections without inpainting hamper performance on 
handles and our end-effector out-painting strategy effectively mitigates it} 
As shown in \figref{rollouts2}, presence of the end-effector caused the object
detector to miss fire leading to failed execution. Our out painting approach
mitigates this issue leading to a higher success rate than the 
approach without out-painting. Interestingly, CoTracker~\cite{karaev2023cotracker} is quite robust
to occlusion (possibly because it tracks multiple points) and doesn't benefit
from in-painting. 


\subsubsection{Closed-loop imitation learning struggles on novel objects}
As presented in \tableref{rum}, \name significantly outperforms RUM in a paired evaluation on unseen objects across three novel environments. A common failure mode of RUM is its inability to grasp the object's handle, even when it approaches it closely.
Another failure mode we observe is RUM misidentifying keyholes or cabinet edges as handles, also resulting in failed grasp attempts.
These result demonstrate that a modular approach that leverages the broad generalization capabilities of vision foundation models is able to generalize much better than an end-to-end imitation learning approach trained on 1000+ demonstrations, which must learn all aspects of the task from scratch.



\section{Discussion}

% Shift from findings to discussion
This study on robotic art explores human-machine relationships in creative processes.
It first contributes as an empirical description of artistic creativity in robotic art practice---an unconventional use of robots---examined through the artists' perspectives on their creative experiences. Our analysis reveals three facets of creativity in robotic art practices: the \textit{social}, \textit{material}, and \textit{temporal}. Creativity emerges from the co-constitution between artists, robots, audience, and environment in spatial-temporal dimensions, as illustrated in \autoref{PracticeDiagram}. Acknowledging the audience as an important actor reflects the social dimension in that creativity does not stem from the artists but from their interactions with the audience. Robots are the major material and technological actants characterizing creative practices, shaping the conditions for creativity to emerge. The axis of the temporal process signifies that the practice is situated within a time continuum, and the actors/actants and their relations shift over time. In this way, temporality constitutes another dimension of creativity in robotic art.

Accordingly, as the second contribution, this study outlines implications for \textit{socially informed}, \textit{material-attentive}, and \textit{process-oriented} creation with computing systems\footnote{For the sake of clarity, we mean ``creation with computing systems'' by three types of scenarios: human creator(s) create computing system(s) as the final artifact(s) (e.g., robots are artworks themselves); human creator(s) use computing system(s) to create the artifact(s) (e.g., robots create artworks as human planned); and human creator(s) and system(s) work in tandem to produce the artifact(s) (e.g., human-robot co-creation).} to facilitate creation practices. These insights can inform related HCI research on media/art creation, crafting, digital fabrication, and tangible computing.
In each following subsection, we present each implication with a grounding in corresponding findings from our study and relevant literature in HCI and adjacent fields on art, creativity, and creation.

\begin{figure*}[htbp]
    \centering
    \includegraphics[width=0.88\textwidth]{Writings/figure/PracticeDiagram.pdf}
    \caption{Actors/actants in robotic art practice and their interactive relations. Robotic art practice unfolds primarily in two spaces: the creation space where interactions happen mainly between artists and robots, and the exhibition space where interactions mostly involve audiences and robots. The two spaces constitute the ENVIRONMENT plane. Within the plane, directed arrows between the actors indicate the types of interaction. For example, the \textit{Design} arrow indicates that the artist designs the robot(s), and the \textit{Revise} arrow indicates that the robot(s) make the artist revise artistic ideas. All the actors/actants may also intra-act with the ENVIRONMENT. The actors/actants and their interactive relations may differ at different times along the axis of TEMPORAL PROCESS that is orthogonal to the plane.}
    \Description{This figure shows the actors/actants in robotic art practice and their interactive relations.}
    \label{PracticeDiagram}
\end{figure*}

\subsection{Socially Informed Creation}

% Introduce social aspect of distributed creativity
The sociality of creativity means that creativity is distributed among different human actors, commonly within the creators or between the creators and the audience. Glăveanu’s ethnographic study on Easter egg decoration in northern Romania~\cite{glaveanu_distributed_2014} showed that artisans anticipate how others might appreciate their work and adjust their creative decisions accordingly. Even in the absence of direct interaction, the audience’s potential responses become part of the creative process, as artisans imagine feedback and predict reactions. In this sense, the sociologist Katherine Giuffre argues that ``\textit{creative individuals are embedded within specific network contexts so that creativity itself, rather than being an individual personality characteristic is, instead, a collective phenomenon}''~\cite[p. 1]{giuffre2012collective}.

% Recall findings about audience feedback
We found that the practice of robotic art manifests this sociality as it involves, particularly artists and audiences. 
Our artists take audiences' reactions to their artwork as feedback and then revise the robots' functions and aesthetics accordingly. 
For example, as shown earlier, Robert added a protective fuse onto his robot because he expected that children would squeeze the springs together and cause a short circuit; Alex's enthusiasm and attention to the audience's imagination about his robots led him to new aesthetic designs of both the robots and the scene layouts. The artists may directly ask about the audience's judgment of quality but they often receive feedback just by observing the audience's reactions or sometimes by learning from the audience's imagination about the robots.
% Recall findings about audience's sociocultural expectations and codes
Meanwhile, our findings reveal that audience reception is not an individual matter but is often associated with their sociocultural codes, including shared cultural norms, beliefs, expectations, and aesthetic values. The audience can be seen as representatives of these broader cultural codes. For example, Mark and Robert observed that the animist tendency in some East Asian societies is associated with higher acceptance of and interest among the audience in intelligence and agency of robots and non-human entities. Such sociocultural contexts influence not only how audiences interpret the work but also how artists anticipate and respond to these perspectives in their creative process.

% Situate in HCI literature
A creative process, including creation and reception, is essentially a social activity. The second wave of creativity research in psychology has argued for creativity's dependency on sociocultural settings and group dynamics~\cite{sawyer2024explaining}. Recent discussions from creativity-support and social computing researchers also called for more attention to the social aspect of creativity~\cite{kato2023special, fischer2005beyond, fischer2009creativity}. There is a clear need to consider the audience when producing creative content. For instance, researchers studying video-creation support have examined audience preferences to inform system designs that align with these preferences~\cite{wang2024podreels}. Such work highlights how creative activities extend beyond individual creators to co-creators and heterogeneous audiences. Some HCI researchers conceptualize creativity as by large a socially constructed concept, perceived and determined by social groups~\cite{fischer2009creativity}. 
Prior HCI work examined the social aspects between art creators. For example, creators and performers in music and dance form social relationships through artifacts, making the final work a collaborative outcome~\cite{hsueh2019deconstructing}. There is also a system designed to support collaborative creation between artists~\cite{striner2022co}. However, the social creative process between creators and audience is less articulated in HCI. Jeon et al.'s work~\cite{jeon2019rituals} stands as an exception, suggesting that professional interactive art can involve evaluation with the audience in the creation stage. 
Another relevant approach in HCI involves enabling the general public to participate in co-creation alongside professional creators. ~\citet{matarasso2019restless}, for instance, promoted ``participatory art'' as ``\textit{the creation of an artwork by professional artists and non-professional artists working together}'' with non-professional artists referring to the general public engaged in the art-making process. Similarly, socially inclusive community-based art also considers target communities' perception of the artwork during creation~\cite{clark2016situated, clarke2014socially}. But like participatory design~\cite{schuler1993participatory}, these art projects aim for social justice more than creativity in the work~\cite{murray2024designing}, let alone that direct participation in art creation is not always feasible. Our findings suggest that feedback from the audience can lead to creative ideas, as well as that the feedback can be generative and remain low-effort for the audience.

Unlike conventional design feedback---which is typically expected to be specific, justified, and actionable~\cite{yen2024give, krishna2021ready}---the feedback that resonates with our artists is often implicit, creative, and generative. Such feedback may include audiences' imaginations stimulated by the work, personal and societal reflections, and even emotions, facial expressions, micro-actions, and observable behaviors following the art experience. Our artists gathered this implicit feedback not by posing evaluative questions, as commonly done in typical design processes (e.g., usability testing, think-aloud protocols), which seek to elicit clear, relatively structured responses. Instead, they closely observe the audience's reactions and interpret their subjective perceptions. This form of implicit feedback, while indirect, can lead to more creative ideas by embracing open, multifaceted interpretations of the work~\cite{sengers2006staying}. Computing systems for creation should better incorporate implicit feedback in addition to explicit ones from the audience into the creation process. Implicit feedback can be indirect, creative, inspirational, and heuristic about functions and aesthetics. A hypothetical instance of such design can be a system that helps creators perceive audiences' implicit reactions and perceptions and variously interpret them, for further iteration.

% Recall findings about audience interacting with robots as a performative art
Moreover, as seen in Robert and Daniel's experiences, the audience may participate in robotic live performances by interacting with the robots, who may change actions accordingly, triggering a loop of simultaneous mutual influence that makes the work performative and improvisational.
% Situate in HCI
HCI researchers explored performative and improvisational creation with machines, focusing on developing and evaluating systems with performative capabilities, including music improvisation with robots~\cite{hoffman2010shimon}, dance with virtual agents~\cite{jacob2015viewpoints, triebus2023precious}, and narrative theatre~\cite{magerko2011employing, piplica2012full}. \citet{kang2018intermodulation} discussed the improvisational nature of interactions between humans and computers and argued that an HCI researcher-designers' improvisation with the environment facilitates the emergence of creativity and knowledge. Designs of computing systems for creation can leverage performativity in service of creative experience. One possible direction could be to allow the audience to embed themselves in and interact with elements of static artwork in a virtual space, turning the exhibition into an improvisational on-site creation~\cite{zhou2023painterly}.
% Our new implication different from current discussion on perf and impr
While interactions with machines during performance are mostly physical or embodied, we posit that they can also be a \textit{symbolic engagement}. Alex's audience projected themselves and their personalities onto his robots, which established a symbolic relevance, generating creative imaginations. During exhibitions, East Asian audiences carried the animist views shaped by their sociocultural backgrounds, and robots, through the performance, were successful in symbolically matching the views, stimulating aesthetic satisfaction. Symbolic engagement resonates with what ~\citet{nam2014interactive} called the ``reference'' of the interactive installation performance to participants' sociocultural conditions.
As such, we propose that designers of computing systems for creation may consider establishing symbolic engagement between the produced artifacts and the audience as a way to enhance perceived creativity or enrich the creative experience. One example is an interactive installation, \textit{Boundary Functions}~\cite{snibbe1998}, which encourages viewers to reflect on their personal spaces while interacting with the installation and others. Another example is \textit{Blendie}, a voice-controlled blender that requires a user to ``speak'' the machine's language to use it. This interaction builds a symbolic connection between the user and the device, transforming the act of blending into a novel experience~\cite{dobson2004blendie}.


\subsection{Material-Attentive Creation}

% Intro paragraph to the importance of materiality for creative activities with machines and the end goal of this discussion--- design suggestions
The theory of distributed creativity by Glaveanu claims that creativity distributes across humans and materials, so the creation practice itself is inevitably shaped by objects~\cite{glaveanu_distributed_2014}. In his case of Easter egg decoration, materials are not passive objects but active participants in artistic creation; e.g., the egg decorators face challenges from color pigments not matching the shell, wax not melted at the desired temperature, to eggs that break at the last step of decoration; hence, materials often go against the decorators' intentions and influence future creative pathways~\cite{glaveanu_distributed_2014}.
Materials manifest specific properties, which afford certain uses of the materials while constraining others~\cite{leonardi2012materiality}. Our findings highlight the critical role of materiality in artistic practice, showing that artists intentionally arrange materials to enhance the creative values of their work.

% Materiality aspect One: physicality and embodiment
% Embodiment or physicality fascilitates creative interaction with machines
Robotic art relies on the material properties of robots and other objects. An apparent property of most materials is their physicality~\cite{leonardi2012materiality}, meaning they possess a tangible presence that enables interaction with other physical entities. Here, we consider physicality and embodiment interchangeable as computational creativity researchers have conceptualized~\cite{guckelsberger2021embodiment}.
% Recall findings on embodiment's value in making art
Our findings support both the conceptual and operational contributions of embodiment for creative activities. For the conceptual aspect, the embodied presence of robotic systems supports creative thinking for our artists, exemplary in Linda's case where she found new art ideas around the difference between human and robot bodies through bodily engagement with robots. 
For the operational aspect, the embodied nature of robotic artworks and their creation processes exhibit original aesthetics that are based on physics much different from disembodied works, e.g., embodied drawings by David's non-industrial robotic arms are dynamic due to physical movements and thus artistically pleasant, which is hard to replicate in simulated programs.

% References: embodied interaction, embodied cognition theories, tangible computing
These findings on embodiment of robotic art (Section \ref{f:emb}) closely relate to HCI's attention on embodied interaction as a way to leverage human bodies and environmental objects to expand disembodied user experiences. 
For example, as~\citet{hollan2000distributed} explained, a blind person's cane and a cell biologist's microscope as embodied materials are part of the distributed system of cognitive control, showing that cognition is distributed and embodied. 
Similarly, theories of embodied interaction in HCI explicate how bodily interactions shape perception, experience, and cognition~\cite{marshall2013introduction, antle2011workshop, antle2009body}, backed up by the framework of 4E cognition (embodied, embedded, enactive, and extended)~\cite{wheeler2005reconstructing, newen20184E}. 
Prior works suggest that creative activities with interactive machines rely on similar embodied cognitive mechanisms ~\cite{guckelsberger2021embodiment, malinin2019radical}, which are operationalized by tangible computing~\cite{hornecker2011role}. 
% References: embodiment's consequence in creation
As related to robots in creation, HCI researchers show that physicality or embodiment of robots in creation may lead to some beneficial outcomes, such as curiosity from the audience, feelings of co-presence, body engagement, and mutuality, which are hard to simulate through computer programs~\cite{dell2022ah, hoggenmueller2020woodie}. Embodied robotic motions convey emotional expressions and social cues that potentially enrich and facilitate creation activities like drawings~\cite{ariccia2022make, grinberg2023implicit, dietz2017human, santos2021motions}. Guckelsberger et al.~\cite{guckelsberger2021embodiment} showed in their review that embodiment-related constraints (e.g., the physical limitations of a moving robotic arm) can also stimulate creativity. These constraints push creators to develop new and useful movements, echoing the broader principle that encountering obstacles in forms or materials can lead to generative processes. This phenomenon is similarly observed in activities such as art and digital fabrication~\cite{devendorf2015being, hirsch2023nothing}. In co-drawing with robots, physical touch and textures of drawing materials made the artists prefer tangible mediums (e.g., pencils) than digital tools (e.g., tablets) that fall short in these respects~\cite{jansen2021exploring}.

% Transit to materiality aspect two
% Materiality aspect Two: malfunction as manifestation of unique materiality of robots
% Intro to materials of robots
Materiality plays a crucial role in the embodiment of robots, as the choice of materials fundamentally shapes the physical forms and properties. This focus on materials extends to art practices, where robots made with soft materials introduce new aesthetics and sensory experiences~\cite{jorgensen2019constructing, belling2021rhythm}, and the use of plants and soil in robotic printing creates unique visual effects~\cite{harmon2022living}. Following Leonardi's ~\cite{leonardi2012materiality} conceptualization of materiality, we refer to the materials of robots as encompassing physical and digital components---including the shell, hardware, mechanical parts, software, programs, data, and controllers---each significant to the artist's intent. ~\citet{nam2023dreams} found that the material constraints of robots can limit creative expression but simultaneously stimulate creativity when artists push the boundaries.

%-----maybe here the real "malfuction" start ------------------
% Move to introduce malfunctions as unique materiality

Even carefully designed, digital and mechanical components in robots are prone to errors or bugs in everyday runs, causing malfunctions or unexpected consequences. This reflects the unique materiality of robots as complex computing systems. From an engineering perspective, errors signal unreliability and must be eliminated, driving advancements in robotics---where error detection and recovery are central~\cite{gini1987monitoring}---as well as in digital fabrication, which prioritizes precision over creative exploration~\cite{yildirim2020digital}. % Recall findings on embracing malfunctions
However, material failures and accidents are inevitable, exemplifying what has been called the `craftsmanship of risk'~\cite{glaveanu_distributed_2014} in material art. For our artists, these risks are often creatively utilized and incorporated into their work: these moments of breakdown---whether physical or digital---become resources for new creative expression. Errors are anticipated and intentionally designed into the process and work of our artists. In some cases, such as for Alex, the entire concept of one of his works is machine errors.

% Situate in literature
Reports on how artists view errors within engineering and creation processes are dispersed throughout HCI literature. ~\citet{nam2023dreams} showed that the accumulation of ``contingency'' and ``accidents''---unexpected, serendipitous, and emergent events during art creation like errors---meaningfully constituted the final presentation of the artwork. Song and Paulos's concept of ``unmaking'' highlighted the values of material failures in enabling new aesthetics and creativity~\cite{song2021unmaking}. Kang et al.~\cite{kang2022electronicists, kang2023lady} introduced the notion of an ``error-engaged studio'' for design research in which errors in creative processes are identified, accommodated, and leveraged for their creative potential. Collectively, these works advocate for reframing errors from something to avoid to something to embrace and recognize. We want to push this further by arguing that errors can be intended and be part or sometimes entire of the design. Several artists, including participants from our study, have been deliberately seeking errors to formulate their designs. Roboticist Damith Herath recounted when he mistakenly programmed a motion sequence of a robotic arm, his collaborator, robotic artist Stelac responded with ``[W]e need to make more mistakes;'' as many mistakes were made, the initial pointless movements became beautiful, rendering the robot ``alive'' and ``seductive'' \cite{herath2016robots}. Similarly, AI artists sometimes look for program glitches to generate unusual styles and content~\cite{chang2023prompt}. Therefore, creators may not only passively accept errors but can actively seek and utilize them. Errors can be integral to the design itself---errors can \textit{be designed into} an artifact, and the design/idea of the artifact can be all about errors.

Thus, to focus on material-attentive creation---considering the creative arrangement of materials---we suggest exploring the embodiment and materiality of creation materials, objects, and environments to recognize their creative potential. %This perspective aligns with insights from professional digital fabrication practitioners, who advocate for systems that integrate support for machine settings and material properties~\cite{hirsch2023nothing}.
Specifically, we propose using a design method/probe that enables creators to realize both the conceptual and operational contributions of materiality. This approach may build on the material probe developed by~\citet{jung2010material}, which calls for exploring the materiality of digital artifacts. A material-attentive probe would enable creators to engage with diverse materials, objects, and environments through embodied interaction, encouraging them to speculate on material preferences and limitations, and to compare and contrast material qualities---insights that can inform creative decisions.
To accommodate, seek, and actively harness the creative potential of errors, we propose embracing failures, glitches, randomness, and malfunctions in computing systems as critical design materials---elements that creators can intentionally control and manipulate. By doing so, we can begin to systematically approach errors. For instance, as part of the design process, we may document how to replicate these errors and changes, allowing creators to explore them further at their discretion. This could include intentionally inducing errors or random changes to influence the creative process or outcomes.

\subsection{Process-Oriented Creation}

% Introduce the key idea: process itself embeds creative value and can be pursued as the goal of creation
As shown in our findings, the creation process itself embeds creative values and meanings, and experiencing the process can be pursued as the goal of creation with computing systems.
% Recall findings
For the robotic artists in our study, artistic values were often placed on the creation process rather than the outcome.  For example, in Alex's robotic live drawing performance, the drawing process is more important than the drawn pattern on canvas. Techniques used, decisions made, or stimuli received by robots during creation or exhibition reflect artistic ideas and nuanced thinking, as seen in Sophie's exploration of interactive decision-making in robotic drawing.

% Situate in HCI lit
Previous HCI work has touched on the value of the process of creation. ~\citet{bremers2024designing} shared a vignette where a robotic pen plotter simultaneously imitates the creator's drawing, serving as a material presence rather than a pragmatic co-creator; here the focus of the work is no longer the outcome but the process of drawing itself. ~\citet{devendorf2015reimagining} concluded that performative actions of digital fabrication systems, rather than the fabricated products themselves, convey artistic meanings tied to histories, public spaces, time, environments, audiences, and gestures. This emphasis on process is particularly significant for media such as improvisational theatre, where the creation itself is an integral part of the final work~\cite{o2011knowledge}. ~\citet{davis2016empirically} named their improvisational co-drawing robotic agents as ``casual creators,'' who are meant to creatively engage users and provide enjoyable creative experiences rather than necessarily helping users make a higher quality product. Shifting the focus from product to process and experiences \textit{in} creation may generate alternative creative meanings.

% Findings about process extends beyond creation
Our artists pointed out that even a ``finished'' artwork in an exhibition is not truly finished. A crack in Daniel's robotic artwork introduced a new artistic meaning, ultimately subverting the entire work. As the properties of the work change over time---whether due to the artist's intent, material characteristics, or environmental factors---the artwork evolves, revealing new aesthetics and meanings. % Situate in HCI lit
Based on these observations, we argue that creation processes should not be regarded as one-shot transactions, as creative artifacts, particularly physical ones, continue to change and generate artistic values. For instance, material wear and destruction bring unique aesthetics, often contrasting with the original form ~\cite{zoran2013hybrid}, and are seen as signs of mature use~\cite{giaccardi2014growing}.
Changes such as material failure, destruction, decay, and deformation---what~\citet{song2021unmaking} referred to as ``unmaking,'' a process that occurs after making---meaningfully transforms the original objects. Similarly, through Broken Probes, a process of assembling fractured objects, ~\citet{ikemiya2014broken} demonstrated that personal connections, reminiscence, and reflections related to material wear and breakage add new values to the objects. Drawing from Japanese philosophy Wabi-Sabi, ~\citet{tsaknaki2016expanding} reflected on the creeds of `Nothing lasts,' `Nothing is finished,' and `Nothing is perfect' and pointed to the impermanence, incompleteness, and imperfection of artifacts as a resource that designers, producers, and users can utilize to achieve long-term, improving, and richer interactive experience~\cite{tsaknaki2016things}. Insights from this study contribute to this line of thought by showing how robotic artists appreciate the aesthetics and meanings of temporal changes after the creation phase.

The findings underscore the need to reconceptualize creation as encompassing more than just the process aimed at producing a final product; it also includes what we term \textit{post-creation}. Distinct from repair, maintenance, or recycle, \textit{post-creation} entails anticipating and managing how an artifact evolves after its ``completion'' in the conventional sense. Specifically, we encourage creators to anticipate and strategically engage with the post-creation phase, considering potential changes to the artifact and their consequences for interactions with human users. For instance, during the creation process, creators may focus on possible material changes the artifact might undergo post-creation, allowing them to either mitigate or creatively exploit these potential changes. This expanded view of creation invites us to trace post-creation developments and to plan how our creative intentions can be embedded in its potential degradation, transformation, or evolution over time.

% A conclusion paragraph
We categorize the design implications into three aspects, but we do not suggest that a computing system must implement all simultaneously, nor that each aspect should be considered in isolation. Social interactions, such as those between artists and audiences, already presume the presence of material actants like robots, and these interactions inform future arrangements of materials. Thus the social and material aspects can be entangled and mutually constitutive as seen in sociomaterial practices~\cite{orlikowski2007sociomaterial, cheatle2015digital, rosner2012material}. The temporal aspect is orthogonal to the other aspects because both social interactions and material manifestations unfold and shift in a temporal continuum.

\section{Conclusion }
This paper introduces the Latent Radiance Field (LRF), which to our knowledge, is the first work to construct radiance field representations directly in the 2D latent space for 3D reconstruction. We present a novel framework for incorporating 3D awareness into 2D representation learning, featuring a correspondence-aware autoencoding method and a VAE-Radiance Field (VAE-RF) alignment strategy to bridge the domain gap between the 2D latent space and the natural 3D space, thereby significantly enhancing the visual quality of our LRF.
Future work will focus on incorporating our method with more compact 3D representations, efficient NVS, few-shot NVS in latent space, as well as exploring its application with potential 3D latent diffusion models.



%%Acknowledgments paragraph
\ifthenelse{\boolean{public_version}}{
\paragraph{Acknowledgments.} This work was partly supported by NSF awards \#2326491 and \#2317706. 
%This work was partly supported by NSF awards \#2326491, \#2125362, and \#2317706. 
The views and conclusions contained herein are those of the authors and should not be interpreted as representing any sponsor's official policies or endorsements.
We thank Junyu Chen, Helia Dinh, and Shikhar Srivastava for their feedback and comments on the manuscript.
}



\section*{Impact Statement}


This paper aims to contribute to the advancement of the field of Machine Learning. While our work has the potential to influence various societal domains, we do not identify any specific societal impacts that require particular emphasis at this time.



\bibliography{main}
\bibliographystyle{icml2025}


%%%%%%%%%%%%%%%%%%%%%%%%%%%%%%%%%%%%%%%%%%%%%%%%%%%%%%%%%%%%%%%%%%%%%%%%%%%%%%%
%%%%%%%%%%%%%%%%%%%%%%%%%%%%%%%%%%%%%%%%%%%%%%%%%%%%%%%%%%%%%%%%%%%%%%%%%%%%%%%
% APPENDIX
%%%%%%%%%%%%%%%%%%%%%%%%%%%%%%%%%%%%%%%%%%%%%%%%%%%%%%%%%%%%%%%%%%%%%%%%%%%%%%%
%%%%%%%%%%%%%%%%%%%%%%%%%%%%%%%%%%%%%%%%%%%%%%%%%%%%%%%%%%%%%%%%%%%%%%%%%%%%%%%
%\newpage
%\appendix
%\onecolumn
%\section{You \emph{can} have an appendix here.}

%%%%%%%%%%%%%%%%%%%%%%%%%%%%%%%%%%%%%%%%%%%%%%%%%%%%%%%%%%%%%%%%%%%%%%%%%%%%%%%
%%%%%%%%%%%%%%%%%%%%%%%%%%%%%%%%%%%%%%%%%%%%%%%%%%%%%%%%%%%%%%%%%%%%%%%%%%%%%%%

% WARNING: do not forget to delete the supplementary pages from your submission 
\newpage

\onecolumn

\appendix

\section{Supplementary Material}

This is extra material for the submission titled
"\system: Pessimistic Cardinality Estimation Using
  $\ell_p$-Norms of Degree Sequences."
This material is organized as follows.
Section~\ref{subsec:proof:th:lpbase:eq:lptdb} gives the proof for Theorem~\ref{th:lpbase:eq:lptdb}.
Section~\ref{subsec:lptd} describes  a third optimized algorithm for estimating arbitrary conjunctive queries, which uses hypertree decompositions of the queries. This is not yet implemented in \system.
Section~\ref{subsec:lpflow:details} gives more details on the \lpflow optimization and the proof for  Theorem~\ref{th:lpbase:eq:lpflow}.
\nop{Section~\ref{app:predicates-examples} gives examples on how \system accommodates selection predicates.
Section~\ref{app:further-experiments} details further experiments.
}

%%%%%%%%%%%%%%%%%%%%%%%%%%%%%%%%%%%%%%%%%%%%%%%%%%%%%%%%%%%%%%%%%%%%%%%%%%%%%%%%%%%%%%%%%%%%
\subsection{Proof of Theorem~\ref{th:lpbase:eq:lptdb}}
\label{subsec:proof:th:lpbase:eq:lptdb}
For simplicity of presentation, we assume here that the
query $Q$ is connected.
Denote by $b_{\text{base}}$ and $b_{\text{Berge}}$ the values of
\lpbase and \lptdb.  The inequality
$b_{\text{base}}\leq b_{\text{Berge}}$ follows from two observations:
\begin{itemize}
\item For any acyclic query $Q$ and any polymatroid $h$, the
  inequality $E_Q \geq h(X_1\cdots X_n)$ is a Shannon inequality.
  This is a well known
  inequality~\cite{DBLP:journals/tse/Lee87,DBLP:conf/sigmod/KenigMPSS20}
  (which we review in Lemma~\ref{lemma:tony:lee} below).
\item Any feasible solution to \lpbase can be converted to a feasible
  solution of \lptdb by simply ``forgetting'' the terms $h(U)$ that do
  not occur in \lptdb.
\end{itemize}

For the converse, $b_{\text{Berge}}\leq b_{\text{base}}$, we will
prove that every feasible solution $h$ to \lptdb can be extended to a
feasible solution to \lpbase (by defining $h(U)$ for all terms $h(U)$
that did not appear in \lptdb), such that $E_Q = h(X_1\cdots X_n)$.
We will actually prove something stronger: that $h$ can be extended to
a \emph{normal polymatroid}.

\begin{definition} \label{def:normal}
  A set function $h : 2^{\set{X_1,\ldots,X_n}} \rightarrow \R$ is a \emph{normal
    polymatroid} if $h(\emptyset)=0$ and it satisfies:
%
  \begin{align}
    \forall U \subseteq \set{X_1,\ldots,X_n}: \sum_{W\subseteq U}(-1)^{|W|+1}h(W) \geq &0 \label{eq:normal}
  \end{align}
\end{definition}
% 
It is known that every normal polymatroid is an entropic vector, and
every entropic vector is a polymatroid, but none of the converse
holds.

The inequality $b_{\text{Berge}}\leq b_{\text{base}}$ follows from two lemmas:

\begin{lemma} \label{lemma:normal:extension:of:one:bag} Let
  $V = \set{X_1, \ldots, X_n}$, let $a_1, \ldots, a_n, A$ be $n+1$
  non-negative numbers such that:
%
  \begin{align}
    a_1 + \cdots + a_n \geq & A \ \ \ \ \text{and}\ \ \ \ \ a_i \leq A, \forall i=1,n \label{eq:a:a}
  \end{align}
%
  Then there exists a normal polymatroid $h : 2^V \rightarrow \R$ such
  that $h(X_i)=a_i$ for all $i=1,n$ and $h(V) = A$.
\end{lemma}

\begin{lemma}[Stitching Lemma] \label{lemma:stich} Let
  $V_1, V_2$ be two sets of variables, $Z \defeq V_1 \cap V_2$. Let
  $h_1:2^{V_1} \rightarrow \R$, $h_2: 2^{V_2} \rightarrow \R$ be two
  normal polymatroids that agree on their common variables $Z$: in
  other words there exists $h : 2^Z \rightarrow \R$ such that
  $\forall U \subseteq Z$, $h_1(U)=h(U)=h_2(U)$.  Define the following
  function $h' : 2^{V_1\cup V_2} \rightarrow \R$:
%
  \begin{align}
    h'(U) \defeq & h_1(U \cap V_1|U \cap Z) + h_2(U \cap V_2|U \cap Z) + h(U \cap Z) \label{eq:stich}
  \end{align}
%
  Then $h'$ is a normal polymatroid that agrees with $h_1$ on $V_1$ and
  with $h_2$ on $V_2$, and, furthermore, satisfies:
%
  \begin{align}
    h'(V_1 \cup V_2) = & h'(V_1) + h'(V_2) - h'(V_1\cap V_2) \label{eq:independent}
  \end{align}
  %
  In essence, this says that $V_1, V_2$ are independent conditioned on
  $V_1 \cap V_2$.
\end{lemma}


Notice that $h'$ can be written equivalently as
$h'(U) = h_1(U\cap V_1)+h_2(U\cap V_2) - h(U\cap Z)$.  While each term
is a normal polymatroid, it is not obvious that $h'$ is too, because
of the difference operation.  In fact, if $h_1, h_2, h$ are
polymatroids, then $h'$ is not a polymatroid in general.


The two lemmas prove Theorem~\ref{th:lpbase:eq:lptdb}, by showing that
$b_{\text{Berge}}\leq b_{\text{base}}$, as follows.  Consider any
feasible solution $h$ to \lptdb.  Consider first a single atom
$R_j(V_j)$ of $Q$: $h$ is only defined on all its variables and on the
entire set $V_j$.  By Lemma~\ref{lemma:normal:extension:of:one:bag},
we can extend $h$ to a normal polymatroid
$h: 2^{V_j} \rightarrow \Rp$.  We do this separately for each
$j=1,m$.  Next, we stitch these polymatroids together in order to
construct a polymatroid on all variables,
$h:2^{\set{X_1,\ldots, X_n}}\rightarrow \Rp$, and for this purpose we
use the Stitching Lemma~\ref{lemma:stich}.  Notice that the Lemma is
stronger than what we need, since in our case the intersection
$V_1 \cap V_2$ always has size 1 (since $Q$ is Berge-acyclic): we need
the stronger version for our third algorithm described in Sec.~\ref{subsec:lptd}.  By
using the conditional independence equality~\eqref{eq:independent}, we
can prove that $E_Q = h(X_1\cdots X_n)$, which completes the proof of
Theorem~\ref{th:lpbase:eq:lptdb}.

It remains to prove the two lemmas.

\begin{proof}[Proof of Lemma~\ref{lemma:normal:extension:of:one:bag}]
  We briefly review an alternative definition of normal polymatroids
  from~\cite{DBLP:conf/lics/Suciu23}.  For any $U \subseteq V$, the
  step function at $U$ is $h^U$ defined as:
%
\begin{align}
\forall X \subseteq V: &&  h^U(X) \defeq &
                  \begin{cases}
                    1 & \mbox{if $U\cap X\neq \emptyset$}\\
                    0 & \mbox{otherwise}
                  \end{cases} \label{eq:step:function}
\end{align}
%
When $U=\emptyset$, then $h^U \equiv 0$, so we will assume
w.l.o.g. that $U \neq \emptyset$.  A function $h : 2^V \rightarrow \R$
is a normal polymatroid iff it is a non-negative linear combination of
step functions:
%
\begin{align}
  h = & \sum_{U \subseteq V, U \neq \emptyset} c_U h^U
\end{align}
%
where $c_U \geq 0$ for all $U$.

We prove the lemma by induction on $n$, the number of variables in
$V$.  If $n=1$ then the lemma holds trivially because we define
$h(X_1) \defeq a_1$, so assume $n \geq 2$.  Rename variables such that
$a_1 \geq a_2 \geq \cdots \geq a_n$, and let $k \leq n$ be the
smallest number such that $a_1 + \cdots + a_k \geq A$: such $k$ must
exist by assumption of the lemma.  We prove the lemma in two cases.

{\bf Case 1}: $k=n$.  Let $\delta \defeq \sum_{i=1,n}a_i - A$ be the
excess of the inequality~\eqref{eq:a:a}: notice that $a_1 \geq \delta$
and $a_n \geq \delta$.  We define $h$ as follows:
  %
\begin{align*}
  h = & (a_1-\delta)h^{X_1}+\delta h^{X_1,X_n}+\sum_{i=2,n-1}a_i h^{X_i} + (a_n-\delta)h^{X_n}
\end{align*}
  %
By construction, $h$ is a normal polymatroid, and one can check by
direct calculation that $h(X_i)=a_i$ for all $i$ and $h(V)=A$.

{\bf Case 2}: $k < n$.  We prove by induction on $m=k,k+1,\ldots,n$
that there exists a normal polymatroid
$h : 2^{\set{X_1, \ldots, X_m}} \rightarrow \R$ s.t. $h(X_i)=a_i$ for
$i=1,m$ and $h(X_1\ldots X_m)=A$.  The claim holds for $m=k$ by Case
1.  Assuming it holds for $m-1$, let
$h' : 2^{\set{X_1,\ldots,X_{m-1}}}\rightarrow \R$ be such that
$h'(X_i)=a_i$ for $i=1,m-1$ and $h'(X_1\cdots X_{m-1})=A$.  We show
that we can extend it to $X_m$.  For that we first represent $h'$ over
the basis of step functions:
  %
\begin{align*}
  h' = & \sum_{U \subseteq \set{X_1,\ldots,X_{m-1}}, U\neq \emptyset}c_U  h^U
\end{align*}
  %
for some coefficients $c_U \geq 0$, and note that
$\sum_U c_U = h'(X_1\ldots X_{m-1})=A$.  Define $h$ as follows:
  %
  \begin{align*}
    h = & \sum_{U \subseteq \set{X_1,\ldots,X_{m-1}}, U\neq \emptyset}c_U\left(1-\frac{a_m}{A}\right) h^U+ \sum_{U \subseteq \set{X_1,\ldots,X_{m-1}}, U\neq \emptyset}c_U\frac{a_m}{A} h^{U\cup\set{X_m}}
  \end{align*}
  %
  By assumption of the lemma $a_m \leq A$, which implies that all
  coefficients above are $\geq 0$, hence $h$ is a normal
  polymatroid. Furthermore, by direct calculations we check that, for
  $i<m$, $h(X_i) = h'(X_i) = a_i$ (because
  $h^{U\cup \set{X_m}}(X_i)=h^U(X_i)$), and, for $i<m$,
  $h(X_m) = a_m$, because $h^U(X_m) = 0$ and
  $h^{U\cup\set{X_m}}(X_m)=1$, and the claim follows from
  $\sum_U c_U = h'(X_1\cdots X_{m-1})=A$.  Finally, we also have
  $h(X_1\ldots X_m) = \sum_U c_U = A$, as required.
\end{proof}

Finally, we prove the Stitching Lemma~\ref{lemma:stich}.  For that we
need two propositions.

\begin{proposition} \label{prop:technical:1} Let $h : 2^V \rightarrow \R$
  be a normal polymatroid, and $V_0 \supseteq V$ a superset of
  variables.  Define $h': 2^{V_0}\rightarrow \R$ by
  $h'(U) \defeq h(U \cap V)$ for all $U \subseteq V_0$.  Then $h'$ is
  a normal polymatroid.  In other words, $h'$ extends $h$ to $V_0$ by
  simply ignoring the additional variables.
\end{proposition}

\begin{proof}
  We verify condition~\eqref{eq:normal} directly.  When
  $U \subseteq V$, then $h'(W)=h(W)$ for all $W \subseteq U$ and the
  condition holds because $h$ is a normal polymatroid.  When
  $U\not\subseteq V$, then we claim that
  $\sum_{W\subseteq U}(-1)^{|W|+1}h(W \cap V)=0$.  Indeed, fix a variable
  $X_i \in U$, $X_i \not\in V$, and pair every subset $W \subseteq U$
  that does not contain $X_i$ with $W' \defeq W \cup \set{X_i}$.  Then
  $h(W\cap V)=h(W'\cap V)$ and the two terms corresponding to $W$ and
  $W'$ in~\eqref{eq:normal} cancel out, proving that the
  expression~\eqref{eq:normal} is $=0$.
\end{proof}

\begin{proposition} \label{prop:technical:2} Let $h: 2^V \rightarrow \R$ be
  a normal polymatroid, and $Z\subseteq V$ a subset of variables.
  Define the following set functions $h', h'' : 2^V\rightarrow \R$:
%
  \begin{align}
\forall U \subseteq V: &&    h'(U) \defeq & h(U\cap Z) &  h''(U) \defeq & h(U|U\cap Z) \label{eq:hprime:normal}
  \end{align}
%
  (Recall that $h(B|A) = h(AB)-h(A)$.)  Then both $h', h''$ are normal
  polymatroids.
\end{proposition}

\begin{proof}
  We consider two cases as above.  When $U\subseteq Z$, then for all
  $W \subseteq U$ we have $h'(W)=h(W)$, and $h''(W)=0$:
  condition~\eqref{eq:normal} holds for $h'$ because it holds for $h$,
  and it holds for $h''$ trivially since it is $=0$.  No consider
  $U\not\subseteq Z$.  Then we claim that the
  expression~\eqref{eq:normal} for $h'$ is $0$:
%
  \begin{align*}
    \sum_{W \subseteq U} (-1)^{|W|+1}h'(W)=& \sum_{W \subseteq U} (-1)^{|W|+1}h(W\cap Z)=0
  \end{align*}
%
  We use the same argument as in the previous lemma: pick a variable
  $X_i$ s.t. $X_i \in U$ and $X_i \not\in Z$ and pair each set
  $W \subseteq U$ that does not contain $X_i$ with the set
  $W' \defeq W \cup \set{X_i}$.  Then $h(W\cap Z)=h(W'\cap Z)$ and two
  terms for $W$ and $W'$ cancel out.  Finally,
  condition~\eqref{eq:normal} for $h''$ follows similarly:
%
  \begin{align}
    \sum_{W\subseteq U}(-1)^{|W|+1}h''(W) = & \sum_{W\subseteq U}(-1)^{|W|+1}h(W|W\cap Z) =\underbrace{\sum_{W\subseteq U}(-1)^{|W|+1}h(W)}_{\geq 0}-\underbrace{\sum_{W\subseteq U}(-1)^{|W|+1}h(W\cap Z)}_{=0}
  \end{align}
%
  The first term is $\geq 0$ because $h$ is a normal polymatroid, and
  the second term is $=0$, as we have seen.
\end{proof}

Finally, we prove the Stitching Lemma~\ref{lemma:stich}.

\begin{proof}[Proof of Lemma~\ref{lemma:stich}] We first use the two
  propositions to show that $h'$ from Eq.~\eqref{eq:stich} is a normal polymatroid.  Define two
  helper functions $h'_1 : 2^{V_1} \rightarrow \R$ and
  $h'_2 : 2^{V_2} \rightarrow \R$:
%
  \begin{align*}
\forall U \subseteq V_1:\   h'_1(U) \defeq & h_1(U|U \cap Z) & 
\forall U \subseteq V_2:\   h'_2(U) \defeq & h_2(U|U \cap Z)
  \end{align*}
%
  By Lemma~\ref{prop:technical:2}, both $h'_1, h'_2$ are normal
  polymatroids.  Next, we extend $h'_1, h'_2, h$ to the entire set
  $V_1 \cup V_2$ by defining:
  \begin{align*}
  \forall U \subseteq V_1 \cup V_2:&& h''_1(U) \defeq &h'_1(U\cap V_1)  & h''_2(U) \defeq &h'_2(U\cap V_2)  & h''(U) \defeq & h(U\cap Z)
  \end{align*}
%
  By Lemma~\ref{prop:technical:1} each of them is a normal
  polymatroid.  Since $h'$ in the corollary is their sum, it is also a
  normal polymatroid.

  We check that it agrees with $h_1$ on $V_1$.  For any $U \subseteq
  V_1$, we have $h_2(U\cap V_2|U \cap Z) = 0$ therefore:
%
  \begin{align*}
    h'(U) = & h_1(U \cap V_1|U \cap Z) + h(U \cap Z)= h_1(U|U\cap Z)+h_1(U\cap Z) = h_1(U)
  \end{align*}
%
  Similarly, $h'$ agrees with $h_2$ on $V_2$.  Finally,
  condition~\eqref{eq:independent} follows by setting
  $U:= V_1 \cup V_2$ in~\eqref{eq:stich} and applying the definition
  of conditional: $h(B|A)=h(B)-h(A)$ when $A \subseteq B$.
\end{proof}


%%%%%%%%%%%%%%%%%%%%%%%%%%%%%%%%%%%%%%%%%%%%%%%%%%%%%%%%%%%%%%%%%%%%%%%%%%%%%%%%%%%%%%%%%%%%
\subsection{\lptd: Using Hypertree Decomposition}
\label{subsec:lptd}

The \lptdb algorithm is strictly limited by two requirements: $Q$
needs to be Berge-acyclic, and all statistics need to be full.  We
describe here a generalization of \lptdb, called \lptd, which drops
these two limitations.  When the restrictions of \lptdb are met, then
\lptd is slightly less efficient, however, its advantage is that it
can work on any query and constraints, without any restrictions.


A \emph{Hypertree Decomposition} of a full conjunctive query $Q$
 is a pair $(T,\chi)$, where $T$ is a tree and $\chi:
 \nodes(T)\rightarrow 2^V$ satisfying the following:
%
 \begin{itemize}
 \item For every variable $X_i$, the set
   $\setof{t \in \nodes(T)}{X_i \in \chi(t)}$ is connected.  This is
   called the \emph{running intersection property}.
 \item For every atom $R_j(V_j)$ of $Q$, $\exists t \in \nodes(T)$
   s.t. $V_j \subseteq \chi(t)$.
 \end{itemize}
%
 Each set $\chi(t) \subseteq V$ is called a \emph{bag}.  The
 \emph{width} of the tree $T$ is defined as
 $w(T) \defeq \max_{t \in \nodes(T)}|\chi(t)|$. We review a lemma by
 Lee~\cite{DBLP:journals/tse/Lee87}:

 \begin{lemma} \label{lemma:tony:lee} ~\cite{DBLP:journals/tse/Lee87}
   Let $(T,\chi:\nodes(T)\rightarrow 2^V)$ have the running
   intersection property and let $h : 2^V \rightarrow \R$ be a set
   function.  Define:
%
  \begin{align}
    E_{T,h} \defeq & \sum_{t \in \nodes(T)}h(\chi(t))-\sum_{(t_1,t_2)\in\edges(T)}h(\chi(t_1)\cap\chi(t_2)) \label{eq:et}
  \end{align}
%
  (1) If $h$ is a polymatroid (i.e. it satisfies the basic Shannon
  inequalities), then $E_{T,h} \geq h(V)$. (2) Suppose $h$ is the
  entropic vector defined by a uniform probability distribution on a
  relation $R(X_1, \ldots, X_n)$.  Then, $E_{T,h}=h(V)$ if for every
  $(t_1,t_2)\in \edges(T)$, the join dependency
  $R = \Pi_{V_1}(R) \Join \Pi_{V_2}(R)$ holds, where
  $V_1, V_2\subseteq V$ are the variables occurring on the two
  connected components of $T$ obtained by removing the edge
  $(t_1,t_2)$.
\end{lemma}

For a simple illustration, consider the 3-way join $J_3$
(Eq.~\eqref{eq:j3}).  Its tree decomposition $T$ has 3 bags
$XY, YZ, ZU$, and $E_{T,h} = h(XY)+h(YZ)+h(ZU)-h(Y)-h(Z)$; one can
check that $E_{T,h} \geq h(XYZU)$ using two applications of
submodularity.


Our new linear program, called \lptd, is constructed from a
hypertree decomposition $(T,\chi)$ of the query as follows:

\smallskip

\noindent {\bf The Real-valued Variables} are all expressions of the
form $h(U)$ for $U \subseteq \chi(t)$, $t \in \nodes(T)$.  The total
number of real-valued variable is $\sum_{t \in \nodes(T)}
2^{|\chi(t)|}$, i.e. it is exponential in the tree-width of the query.

\smallskip

\noindent {\bf The Objective} is to maximize $E_{T,h}$
(Eq.~\eqref{eq:et}), subject to the following constraints.

\smallskip

\noindent {\bf Statistics Constraints:}
All statistics constraints Eq.~\eqref{eq:h:p} of \lpbase.
Since we don't have numerical variable $h(U)$ for all $U$, we must
check that~\eqref{eq:h:p} uses only available numerical
variables.  This holds, because each statistics refers to some atom
$R_j(V_j)$, and there exists of some bag such that
$V_j \subseteq \chi(t)$, therefore we have numerical variables $h(U)$
for all $U \subseteq V_j$.

\smallskip

\noindent {\bf Normality Constraints:}
For each bag $\chi(t)$, \lptd contains all constraints of the
form~\eqref{eq:normal}.  In other words, the restriction of $h$ to
$\chi(t)$ is normal.


We prove:

\begin{theorem} \label{th:lptd} \lpbase and \lptd compute the same
  value.
\end{theorem}

The theorem holds only when all degree constraints used in the
statistics are \emph{simple}, as we assumed throughout this paper.
For a simple illustration, the \lptd for $J_3$ consists of 7 numerical
variables \\
$h(X),h(Y),h(Z),h(U),h(XY),h(YZ),h(ZU)$ (we omit
$h(\emptyset)=0$) and the following Normality Constraints:
%
\begin{align*}
  h(X)+h(Y)-h(XY)\geq & 0 & h(Y)+h(Z)-h(YZ)\geq & 0 \\
  h(Z)+h(U)-h(ZU) \geq & 0
\end{align*}



To compare \lptd and \lptdb, assume that the query $Q$ is Berge-acyclic
and all statistics are on simple and full degree constraints.  The
difference is that, for each atom $R_j(V_j)$, \lptdb has only $1+|V_j|$
real-valued variables and only $1+|V_j|$ additivity constraints, while
\lptdb has $2^{|V_j|}$ variables and normality constraints.

In the remainder of this section we prove Theorem~\ref{th:lptd}.

Denote by $b_{\text{base}}$ and $b_{\text{td}}$ the optimal solutions
of \lpbase and \lptd respectively.  We will prove that
$b_{\text{base}}=b_{\text{td}}$.

First, we claim that $b_{\text{base}}\leq b_{\text{td}}$.  It is known
from~\cite{DBLP:journals/pacmmod/KhamisNOS24} that \lpbase has an
optimal solution $h^*$ that is a normal polymatroid; thus
$b_{\text{base}}=h^*(V)$ (recall that $V$ is the set of all
variables), where $h^*$ is normal.  Then $h^*$ is also a feasible
solution to \lptd, therefore its optimal value is at least as large as
the value given by $h^*$, in other words
$b_{\text{td}}\geq E_{T,h^*}$.  By Lemma~\ref{lemma:tony:lee}, we have
$E_{T,h^*}\geq h^*(V) = b_{\text{base}}$, which proves the claim.


Second, we prove that $b_{\text{td}}\leq b_{\text{base}}$ by using the Stitching Lemma~\ref{lemma:stich}.
Let $h^*$ be an optimal solution to \lptd, thus
$b_{\text{td}}=E_{T,h^*}$.  The function $h^*$ is defined only on
subsets of the bags $\chi(t)$, $t \in \nodes(T)$, and on each such
subset, it is a normal polymatroid.  We extend it to a normal
polymatroid defined on all variables
$V = \bigcup_{t \in \nodes(T)}\chi(t)$ by repeatedly applying the
Stitching Lemma~\ref{lemma:stich}. Condition~\eqref{eq:independent} of
the corollary implies that this extension satisfies
$h^*(V)=E_{T,h^*}$.  Thus, $h^*$ is a normal polymatroid, and, hence,
a feasible solution to \lpbase.  It follows that the optimal value of
\lpbase is at least as large as that given by $h^*$, in other words
$b_{\text{base}} \geq h^*(V)$.  This completes the proof of the claim.

%%%%%%%%%%%%%%%%%%%%%%%%%%%%%%%%%%%%%%%%%%%%%%%%%%%%%%%%%%%%%%%%%%%%%%%%%%%%%%%%%%%%%%%%%%%%

\subsection{\lpflow: Missing Details from Section~\ref{subsec:lpflow}}
\label{subsec:lpflow:details}

Section~\ref{subsec:lpflow} gives the high-level idea of the \lpflow algorithm
using an example. We give here a more formal description of the algorithm
and prove Theorem~\ref{th:lpbase:eq:lpflow}.
The input to \lpflow is an arbitrary conjunctive query $Q$ of the form Eq.~\eqref{eq:cq}
(not necessarily a full query) and a set of statistics on the input database consisting of $\ell_p$-norms
of {\em simple} degree sequences, i.e. statistics of the form $\lp{\degree_{R_j}(V|U)}_p$ where $|U|\leq 1$.
For the purpose of describing \lpflow,
we construct a flow network $G=(\nodes,\edges)$ that is defined as follows:
(Recall that $\vars(Q) =\{X_1, \ldots, X_n\}$ is the set of variables of the query.)
\begin{itemize}
    \item The set of nodes $\nodes\subseteq 2^{\vars(Q)}$ consists of the following nodes:
    \begin{itemize}
        \item The node $\emptyset$, which is the source node of the flow network.
        \item A node $\{X_i\}$ for every variable $X_i \in \vars(Q)$.
        \item A node $UV$ for every statistics $\lp{\degree_{R_j}(V|U)}_p$. 
    \end{itemize}
    \item The set of edges $\edges$ consists of two types of edges:
    \begin{itemize}
        \item {\bf Forward edges:} These are edges of the form $(a, b)$
        where $a, b \in \nodes$ and $a \subset b$. Each such edge $(a, b)$ has a finite capacity $c_{a, b}$. In particular, for every statistics
        $\lp{\degree_{R_j}(V|U)}_p$, we have two forward edges:
        One edge $(\emptyset, U)$ and another $(U, UV)$. (Recall that $|U|\leq 1$.)
        \item {\bf Backward edges:} These are edges of the form $(a, b)$
        where $a, b \in \nodes$ and $b \subset a$. Each such edge $(a, b)$ has an infinite
        capacity $\infty$.
        In particular, for every statistics $\lp{\degree_{R_j}(V|U)}_p$ and
        every variable $X_i \in UV$, we have a backward edge $(UV, \{X_i\})$.
    \end{itemize}
\end{itemize}
We are now ready to describe the linear program for \lpflow. Recall that $V_0$ is the set of \groupby variables in the query $Q$ from Eq.~\eqref{eq:cq}.

\smallskip

\noindent {\bf The Real-valued Variables} are of two types:
\begin{itemize}
    \item Every statistics $\lp{\degree_{R_j}(V|U)}_p$ has an associated {\em non-negative}
    variable $w_{U,V,p}$.
    \item For every \groupby variable $X_i \in V_0$, we have a flow variable $f_{a, b; X_i}$
    for every edge $(a, b) \in \edges$.
\end{itemize}

\smallskip

\noindent {\bf The Objective} is to minimize the following sum over all available statistics
$\lp{\degree_{R_j}(V|U)}_p$:
\begin{align}
    \sum w_{U,V,p} \cdot \log \lp{\degree_{R_j}(V|U)}_p
    \label{eq:lpflow:objective}
\end{align}

\smallskip

\noindent {\bf The Constraints} are of two types:
\begin{itemize}
    \item {\em Flow conservation constraints:} For every \groupby variable $X_i \in V_0$,
    the variables $f_{a, b;X_i}$ must define a valid flow from the source node
    $\emptyset$ to the sink node $\{X_i\}$ that has a capacity $\geq 1$.
    This means that for every node $c \in \nodes - \{\emptyset\}$, we must have:
    \begin{align}
        \sum_a f_{a, c;X_i} -\sum_b f_{c, b; X_i} \geq 1, & \quad\quad\text{ if $c =\{X_i\}$}\label{eq:flow:conservation:1}\\
        \sum_a f_{a, c;X_i} -\sum_b f_{c, b; X_i} \geq 0, & \quad\quad\text{ otherwise}
        \label{eq:flow:conservation:2}
    \end{align}
    \item {\em Flow capacity constraints:} For every \groupby variable $X_i \in V_0$
    and every {\em forward} edge $(a, b)$, the flow variable $f_{a, b;X_i}$ must satisfy:
    \begin{align}
        f_{a, b;X_i} \leq c_{a, b} \label{eq:flow:capacity}
    \end{align}
    where $c_{a, b}$ is the {\em capacity} of the forward edge $(a, b)$.
    (Recall that backward edges have infinite capacity.)
    The capacity variables $c_{a, b}$ are determined by the statistics constraints.
    In particular, every statistics $\lp{\degree_{R_j}(V|U)}_p$ contributes a capacity
    of $w_{U,V,p}$ to $c_{U,UV}$ and a capacity of $\frac{w_{U,V,p}}{p}$ to $c_{\emptyset,U}$.
    Formally,
    \begin{align}
        c_{\emptyset,U} &\defeq
        \sum_{p} w_{\emptyset,U,p}
        +\sum_{V,p} \frac{w_{U,V,p}}{p}\label{eq:lpflow:c}\\
        c_{U, UV} &\defeq
        \sum_{p} w_{U,V,p} &\text{ if $U \neq \emptyset$}\nonumber
    \end{align}
\end{itemize}
\smallskip



We are now ready to prove Theorem~\ref{th:lpbase:eq:lpflow}, which says that the linear programs
for \lpbase and \lpflow have the same optimal value.
To that end, we first write the dual LP for \lpbase.
For every statistics constraint of the form Eq.~\eqref{eq:h:p}, we introduce a dual variable $w_{U,V,p}$. The dual of \lpbase is equivalent to:
\begin{align}
    \min\quad &\sum w_{U,V,p} \cdot \log \lp{\degree_R(V|U)}_p\label{eq:lpbase:dual}\\
    \text{s.t.}\quad& \text{The following is a valid Shannon inequality:}\nonumber\\
    &h(V_0) \leq \sum w_{U,V,p} \left(\frac{1}{p}h(U)+h(V|U)\right)\label{eq:lpflow:shannon}\\
    & w_{U,V,p} \geq 0\nonumber
\end{align}
Inequality~\eqref{eq:lpflow:shannon} satisfies the property that for every $h(V|U)$
on the RHS, we have $|U| \leq 1$. In order to check that such an inequality is a valid
Shannon inequality, we rely on a key result from~\cite{DBLP:journals/corr/abs-2211-08381}.
In particular,~\cite{DBLP:journals/corr/abs-2211-08381} is concerned
with Shannon inequalities of the following form.
Let $\calX=\{X_1, \ldots, X_n\}$ be a set of variables, and $\calC$
be a set of distinct pairs $(U, V)$ where $U, V\subseteq \calX$, $U \cap V = \emptyset$
and $|U| \leq 1$.
For every pair $(U,V)\in\calC$, let $c_{U,UV}$ be a non-negative constant.
Moreover, let $V_0$ be a subset of $\calX$.
Consider the following inequality:
\begin{align}
    h(V_0) \leq \sum_{(U,V)\in\calC} c_{U, UV} h(V|U)
    \label{eq:lpflow:shannon:general}
\end{align}
\cite{DBLP:journals/corr/abs-2211-08381} describes a reduction from
the problem of checking whether Eq.~\eqref{eq:lpflow:shannon:general} is a valid Shannon inequality to a collection of $|V_0|$ network flow problems.
These flow problems are over the same network $G=(\nodes, \edges)$,
which is similar to the flow network described above for \lpflow. In particular,
\begin{itemize}
    \item The nodes are $\emptyset$, $\{X_i\}$ for each variable $X_i \in \calX$, and $UV$ for each $(U,V)\in\calC$.
    \item The edges have two types:
    \begin{itemize}
        \item Forward edges: For each $(U,V)\in\calC$, we have a forward edge $(U,UV)$ with capacity $c_{U,UV}$.
        \item Backward edges: For each $(U,V)\in\calC$ and each variable $X_i \in UV$,
        we have a backward edge $(UV, \{X_i\})$ with infinite capacity.
    \end{itemize}
\end{itemize}
\begin{lemma}[\cite{DBLP:journals/corr/abs-2211-08381}]
    Inequality~\eqref{eq:lpflow:shannon:general} is a valid Shannon inequality if and only if
    for each variable $X_i \in V_0$, there exists a flow $\left(f_{a, b;X_i}\right)_{(a, b)\in\edges}$ from the source node $\emptyset$
    to the sink node $\{X_i\}$ with capacity at least $1$.
    In particular, the flow variables $\left(f_{a, b;X_i}\right)_{(a, b)\in\edges}$ must satisfy the flow conservation constraints~\eqref{eq:flow:conservation:1} and~\eqref{eq:flow:conservation:2} and the flow capacity constraints~\eqref{eq:flow:capacity}.
\end{lemma}
Using the above lemma, we can prove Theorem~\ref{th:lpbase:eq:lpflow} as follows.
Take inequality~\eqref{eq:lpflow:shannon} and group together identical conditionals
on the RHS in order to convert it into the form of Eq.~\eqref{eq:lpflow:shannon:general}.
The coefficients $c_{U,UV}$ of the resulting Eq.~\eqref{eq:lpflow:shannon:general}
will be identical to those defined by Eq.~\eqref{eq:lpflow:c}.
Then, we can apply the lemma to check the validity of Eq.~\eqref{eq:lpflow:shannon}.
But now, the dual LP~\eqref{eq:lpbase:dual} for \lpbase is equivalent to the linear program for \lpflow.

%%%%%%%%%%%%%%%%%%%%%%%%%%%%%%%%%%%%%%%%%%%%%%%%%%%%%%%%%%%%%%%%%%%%%%%%%%%%%%%%%%%%%%%%%%%%
\nop{
\subsection{Examples: Handling predicates in \system}
\label{app:predicates-examples}

We show how \system handles predicates in the following examples.
Figure~\ref{fig:predicate_example} (left) shows a simple example of a relation $R(X,A,B)$,
where $X$ is a join attribute and $A$ and $B$ are predicate attributes.
The degree sequence for the relation $R$ with respect to the join attribute $X$ is $\deg_{R}(* | X) = (3,2)$. The $\ell_1$ and $\ell_{\infty}$-norm of the degree sequence $(3,2)$ are $3+2 = 5$ and $\max(3,2) = 3$, respectively.

For queries with predicates on attributes $A$ and $B$, using the $\ell_p$-norms of the degree sequence for the entire relation can lead to overestimation, since only a subset of tuples satisfy the predicates.
We show that how to compute tighter bounds of the $\ell_p$-norms for queries with predicates following our discussion in Section~\ref{sec:histograms}.




\begin{figure}[h]
  \centering
  \begin{minipage}[b]{0.50\textwidth}
      \centering
      $R$ \\[0.2em]
      \begin{tabular}{|c|c|c|}
          \hline
          $X$ & $A$ & $B$\\
          \hline
          1 & 1 & 0\\
          1 & 1 & 8\\
          1 & 2 & 13\\
          2 & 3 & 22\\
          2 & 4 & 40\\
          \hline
      \end{tabular}
  \end{minipage}
  % \hfill
  % ex 2
  \begin{minipage}[b]{0.16\textwidth}
      \centering
      $T$ \\[0.2em]
      \begin{tabular}{|c|c|}
          \hline
          $TID$ & $SID$ \\
          \hline
          1 & 1\\
          2 & 2\\
          3 & 2\\
          4 & 3\\
          5 & 3\\
          \hline
      \end{tabular}
  \end{minipage}
  \begin{minipage}[b]{0.16\textwidth}
    \centering
    $S$ \\[0.2em]
    \begin{tabular}{|c|c|}
        \hline
        $SID$ & $A$ \\
        \hline
        1 & 1\\
        2 & 1\\
        3 & 2\\
        \hline
    \end{tabular}
  \end{minipage}
  \begin{minipage}[b]{0.16\textwidth}
    \centering
    $TS$ \\[0.2em]
    \begin{tabular}{|c|c|c|}
        \hline
        $TID$ & $SID$ & $A$ \\
        \hline
        1 & 1 & 1\\
        2 & 2 & 1\\
        3 & 2 & 1\\
        4 & 3 & 2\\
        5 & 3 & 2\\
        \hline
    \end{tabular}
  \end{minipage}
  \caption{Left: relation $R(X,A,B)$, where $X$ is the join attribute and $A$ and $B$ are predicate attributes. Right: Right: relations $T(TID, SID)$ and $S(SID, A)$ where $SID$ is a foreign key in $T$ and the primary key in $S$, and $TS$ is the join of $T$ and $S$.}
  \label{fig:predicate_example}
\end{figure}

\paragraph{Equality Predicate.} 
Consider an equality predicate on $A$.
The $A$-values in $R$, sorted by frequency in descending order, are $(1,2,3,4)$. 
For this example, we consider the most common value $A=1$ as the only MCV for $A$.
We fetch the tuples satisfying $A=1$, which are the first two tuples, and get the degree sequence $\deg_{R}(* | X, A=1) = (2)$.
The $\ell_p$-norm of the degree sequence is $2$ for any $p\geq 1$.

We also compute one degree sequence for all non-MCVs of $A$, i.e., $A\in\{2,3,4\}$.
We fetch the last three tuples in $R$ and compute the degree sequence $\deg_{R}(* | X, A\in \{2, 3, 4\}) = (2, 1)$.
For an arbitrary non-MCV, there is at most one distinct $X$-value in $R$ associated with it,
so we take the top value of $\deg_{R}(* | X, A\in \{2, 3, 4\})$ as the degree sequence, i.e., $(2)$, for all non-MCVs of $A$.
The $\ell_p$-norm of the degree sequence is $2$ for any $p\geq 1$.

An alternative and more accurate, yet more expensive approach is to compute the degree sequence for each non-MCV of $A$: $\deg_{R}(* | X, A=2) = (1)$, $\deg_{R}(* | X, A=3) = (1)$, and $\deg_{R}(* | X, A=4) = (1)$, and then compute the maximum of their $\ell_p$-norms.
The $\ell_p$-norms of these degree sequences are $1$, so the maximum of the $\ell_p$-norms is $1$, for any $p\geq 1$.

To estimate for a query with the predicate $A=1$, we use the $\ell_p$-norms for the MCV of $A$, i.e., $\ell_p = 2$. For a query with the predicate $A=2$, where $2$ is a non-MCV of $A$, we use the $\ell_p$-norms for non-MCVs of $A$, i.e., $\ell_p = 2$ or, if we use the more accurate alternative, $\ell_p = 1$.

\paragraph{Range Predicate.}
Consider a range predicate on $B$
The domain of $B$ is $[0, 40]$. We create a hierarchy of histograms with $2^k, 2^{k-1}, \ldots, 2^0$ buckets.
For this example, we use $k=2$, which means creating $4$ buckets for the bottom layer, $2$ buckets for the next layer, and one bucket for the entire domain.
The buckets for the layers are $\{[0, 10), [10, 20), [20, 30), [30, 40]\}$,  $\{[0, 20), [20, 40]\}$, and $\{[0,40]\}$.

We first construct the $\ell_p$-norms within each bucket $[s, e)$.
For this, we fetch the tuples where $B$ is in the bucket
and compute the degree sequence $\deg_{R}(* | X, B \in [s, e))$ and several $\ell_p$-norms on this degree sequence.
The degree sequences for the buckets in the layers are $((2), (1), (1), (1))$, then $((3), (2))$, and finally $(5)$. The $\ell_p$-norms within each bucket 


thus their $\ell_1$ and $\ell_{\infty}$-norms are 
$\{(\ell_1=\ell_{\infty}=2), (\ell_1=\ell_{\infty}=1), (\ell_1=\ell_{\infty}=1), (\ell_1=\ell_{\infty}=1)\}$ and $\{(\ell_1=\ell_{\infty}=3), (\ell_1=\ell_{\infty}=2)\}$, respectively.

To estimate for the range predicate $5 \leq B \leq 18$, we first find the smallest bucket that covers the range, which is the bucket $[0, 20)$, and use the corresponding $\ell_p$-norms: $\ell_1 = 2$ and $\ell_{\infty} = 2$.


\paragraph{Conjunction and Disjunction of Predicates.}
We show how \system handles the conjunction and disjunction of predicates on $A$ and $B$.
Consider the predicates $A = 0$ and $5 \leq B \leq 18$.
We fetch the $\ell_p$-norms for the two predicates as discussed in the previous examples: $\ell_1 = 1$ and $\ell_{\infty} = 1$ for $A = 1$, and $\ell_1 = 2$ and $\ell_{\infty} = 2$ for $5 \leq B \leq 18$.

For the conjunction of the predicates, we take the minimum of these $\ell_p$-norms to estimate the query. The result is $\ell_1 = \min(1, 2) = 1$ and $\ell_{\infty} = \min(1, 2) = 1$.

For the disjunction of the predicates, we take the sum of the $\ell_1$-norms and the maximum of the $\ell_{\infty}$-norms, which results in $\ell_1 = 1+2 = 3$ and $\ell_{\infty} = \max(1, 2) = 2$.



\paragraph{Optimization 1: Predicate Propagation via FK-PK Joins.}
Consider two relations $T(TID, SID)$ and $S(SID, A)$ in Figure~\ref{fig:predicate_example} (right),
 where $SID$ is a foreign key in $T$ and a primary key in $S$, and $A$ is an equality predicate attribute.
We compute the $\ell_p$-norms for predicates on $A$ in $S$ as discussed in the previous examples: $\ell_1 = 2$ and $\ell_{\infty} = 1$ for the MCV $A=1$, and $\ell_1 = 1$ and $\ell_{\infty} = 1$ for non-MCVs of $A$.

For relation $R$, we apply the optimization to propagate the predicate on $A$ from $S$ to $T$:
We precompute the join results for the FK-PK join $TS(TID, SID, A) = T(TID, SID) \wedge S(SID, A)$ (Figure~\ref{fig:predicate_example} (right)). The size of the join results is bounded by the size of the FK relation $T$.
For this example, we consider only one MCV of $A$, so the only MCV of $A$ is $A=1$.
We fetch the tuples satisfying $A=1$ in $TS$, which are the first two tuples in $TS$, and compute the degree sequence $\deg_{TS}(* | TID, A=1) = (1,1,1)$.
The $\ell_p$-norms of the degree sequence are $\ell_1 = 1+1+1 = 3$ and $\ell_{\infty} = 1$.
Regarding the non-MCVs of $A$, there is only one non-MCV of $A$, which is $A=2$. We compute the degree sequence for $A=2$ in $TS$, i.e., $\deg_{TS}(* | TID, A=2) = (1,1)$ and the $\ell_p$-norms for the degree sequence are $\ell_1 = 1+1 = 2$ and $\ell_{\infty} = 1$.

Consider a query with predicate $A=1$.
For relation $S$, we use the $\ell_p$-norms for the MCV $A=1$ in $S$, i.e., $\ell_1 = 2$ and $\ell_{\infty} = 1$.
For relation $T$, we use the $\ell_p$-norms for the MCV $A=1$ in the join result $TS$, i.e., $\ell_1 = 3$ and $\ell_{\infty} = 1$.

\paragraph{Optimization 2: Compute $\ell_p$-norms for Prefixes of the Degree Sequence.}
Consider two relations $R(X,A)$ and $S(X,B)$ where $X$ is a join attribute, and their degree sequences are $(100, 99, \ldots, 2, 1)$ and $(2, 1)$, respectively.
This means that there are $100$ distinct $X$-values in $R$ and $2$ distinct $X$-values in $S$, which is significantly mis-calibrated.
When the two relations are joined, at most two $X$-values appear in the join results.
If we use the $\ell_p$-norms of the whole degree sequence for $R$, which are $\ell_1 = 5050$ and $\ell_{\infty} = 100$, the estimation can be significantly overestimated.

We reduce overestimation by computing the $\ell_p$-norms for prefixes of the degree sequence for $R$.
We compute the $\ell_p$-norms for the top-$2^i$ values of the degree sequence for $R$ for $i>0$. 
For example, for $i=1$, we compute the $\ell_p$-norms for the top-$2$ values of the degree sequence, which are $(100, 99)$: $\ell_1 = 100+99 = 199$ and $\ell_{\infty} = 100$.
For the join of $R$ and $S$, since there are at most two $X$-values in the join results, we can use these $\ell_p$-norms for the estimation.
}

%%%%%%%%%%%%%%%%%%%%%%%%%%%%%%%%%%%%%%%%%%%%%%%%%%%%%%%%%%%%%%%%%%%%%%%%%%%%%%%%%%%%%%%%%%%%



% moved to the main body
\nop{
%%%%%%%%%%%%%%%%%%%%%%%%%%%%%%
\begin{figure*}[t]
    \centering
    \begin{minipage}[b]{0.48\textwidth}
        \centering
        \includegraphics[width=\textwidth]{experiments/overall_runtime.pdf}
    \end{minipage}
    \hfill
    \begin{minipage}[b]{0.48\textwidth}
        \centering
        \includegraphics[width=\textwidth]{experiments/relative_runtime.pdf}
    \end{minipage}
    \caption{Left: Overall evaluation time of all queries in a benchmark for \psql when using estimates for all subqueries from \system, \safebound, \dbx, \psql and true cardinalities. Right: Relative evaluation times compared to the baseline evaluation time obtained when using true cardinalities.}
    \label{fig:runtime}
\end{figure*}
%%%%%%%%%%%%%%%%%%%%%%%%%%%%%%


\subsection{Further Experiments}
\label{app:further-experiments}

We complement the experiments in Section~\ref{sec:experiments} with further experiments that cannot be accommodated in the main body due to lack of space.

\subsubsection{Estimation Errors}

Fig.~\ref{fig:estimates-STATS} shows that the accuracy of the estimators decreases with the number of relations per query (shown for STATS, a similar trend also holds for JOBlight and JOBrange): The traditional estimators underestimate more, whereas the pessimistic estimators overestimate more. \neurocard starts with a large overestimation for a join of two relations and decreases its estimation as we increase the number of relations; the other ML-based estimators follow this trend but at a smaller scale.


\subsubsection{Optimization Improvements}
Fig.~\ref{fig:improvements-optimizations} shows the improvements to the estimation accuracy brought by each of the two optimizations discussed in Sec.~\ref{sec:histograms}, when taken in isolation.

The left figure shows that,  when propagating predicates from the primary-key relation to the foreign-key relations, the estimation error can improve by over an order of magnitude in the worst case (corresponding to the upper dots in the plot) and by roughly 5x in the median case (corresponding to the red line in the boxplots). 

The right figure shows that,  when using prefix degree sequences for the degree sequences of relations without predicates, the estimation error can improve by up to 50\% for JOBlight queries, up to 65\% for JOBrange queries and up to 10\% for STATS queries. The improvement is measured as the division of (i) the difference between the estimation error without this optimization and the estimation error with this optimization and (2) the the estimation error without this optimization.




\subsubsection{Evaluation Times}




Fig.~\ref{fig:runtime} (left) shows the aggregated \psql evaluation time of all queries in JOBlight, JOBrange, and STATS when using estimates for all sub-queries from \system, \safebound, \dbx, \psql, and true cardinalities (left).
Fig.~\ref{fig:runtime} (right)  shows the relative evaluation times compared to the baseline evaluation time obtained when using true cardinalities. 
We have two observations. First, overestimation can be beneficial for performance of expensive queries, which has been discussed in Section~\ref{sec:experiments}. Second, overestimation can be detrimental for performance of less expensive queries in some cases.

The first observation is reflected in the overall evaluation times, which are dominated by the most expensive queries in the benchmark (some of which are listed in Fig.~\ref{fig:most-expensive-queries}). 
Traditional approaches lead to higher evaluation times for the expensive queries, and therefore to higher overall evaluation times, while the pessimistic approaches lead to lower evaluation times for those expensive queries. Overall, the  evaluation times for the pessimistic approaches are about the same (JOBlight and STATS) or lower (JOBrange) than the baseline evaluation times.
The second observation is reflected in the relative evaluation times for the JOB benchmarks. 
The boxplots for the traditional approaches are lower than those for the pessimistic approaches, indicating that the traditional approaches perform better for the less expensive queries in the benchmarks.

\factorjoin has both high overall evaluation time and high relative evaluation time.
It estimates very accurately for the queries in STATS, thus has similar evaluation time to the baseline evaluation time. For the queries in JOBlight and JOBrange, it mostly overestimates, which leads to lower evaluation times for the expensive queries. However, the overestimations are significant, which makes it perform worse than the pessimistic approaches for the less expensive queries, as shown in the right plot of Fig.~\ref{fig:runtime}. This leads to the high overall evaluation time of \factorjoin.

}

\nop{
\subsection{About Traditional Estimators}

% PostgreSQL
\subsubsection{\psql}
\psql uses four types of statistics to estimate cardinalities:
\begin{itemize}
    \item cardinality of each relation (row count), and the number of pages per relation
    \item number of distinct values for each attribute
    \item MCVs for each attribute and their relative frequencies, i.e., the estimated fraction of rows that have the MCV as the respective attribute value
    \item histogram if the values in each attribute and relative frequency of each histogram bucket
\end{itemize}

The first statistic, the cardinality of each relation, is very accurate. The other statistics, however, are estimated based on a sample of the data. The default sampling size is 30,000 rows. We found that those statistics are often times inaccurate. For example, for some join attributes of the IMDB dataset, the domain size was underestimated by 70\%.

While the statistics mentioned above are computed by default, \psql can be instructed to compute multivariate statistics capturing correlations between attributes of the same relation via the \texttt{CREATE STATISTICS} command.

\paragraph{Selectivity of Equality Predicates and Join Conditions}

\psql uses the concept of {\em selectivity} of a (filter or join) condition to estimate the cardinality of a query output. If the predicate value is a MCV, then \psql considers the relative frequency of this value as its selectivity. If the value is not an MCV, then \psql either use histograms to estimate the frequency of the value in the relation, or it falls back to a default estimate, such as assuming a uniform distribution. In the latter case, the selectivity is assumed to be the estimated domain size of the attribute divided by cardinality of relation. Correlations of attributes across relations are not considered. For join conditions, \psql assumes that the join attributes are independent.

Consider the following join of the two relation $R(A,C)$ and $S(B,C)$ with two equality predicates on the non-join attributes.
\begin{verbatim}
    SELECT * FROM R, S WHERE R.A = 5 AND S.B = 10 AND R.C = S.C;
\end{verbatim}
The estimated cardinality of this query is:
\begin{align*}
    |R| * |S| * \sel{R.A = 5} * \sel{S.B = 10} * \sel{R.C = S.C}.
\end{align*}

\paragraph{Range Conditions}

\subsubsection{\duckdb}
% DuckDB
}



\end{document}

