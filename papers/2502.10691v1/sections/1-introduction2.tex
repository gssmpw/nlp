\section{Introduction}
\label{sec:intro}


\begin{figure}[t]
    \centering
    \includegraphics[width = 0.99\linewidth]{images/nc_ood_detect_transfer_corr_vgg_updated.png}
  \caption{In this paper, we show that there is a close inverse relationship between OOD detection and generalization with respect to the degree of representation collapse in DNN layers. This plot illustrates this relationship for VGG17 pretrained on ImageNet-100 using four OOD datasets, where we measure collapse and OOD performance for various layers. For OOD detection, there is a strong positive Pearson correlation ($R=0.77$) with the degree of neural collapse (NC1) in a DNN layer, whereas for OOD generalization, there is a strong negative correlation ($R=-0.60$). We rigorously examine this inverse relationship and propose a method to control NC at different layers. %This suggests that stronger neural collapse improves OOD detection, while weaker neural collapse enhances OOD generalization. $R$ denotes the Pearson correlation coefficient.
  %Neural collapse metric NC1 (lower values indicate stronger neural collapse) positively correlates with OOD detection error and negatively correlates with OOD transfer error. This implies the stronger the neural collapse, the lower the OOD detection error and vice-versa. And, the weaker the neural collapse, the lower the OOD transfer error. For this, we analyze different layers of VGG17 networks which are pre-trained on the ImageNet-100 (ID) dataset, and evaluated on four OOD datasets. %e.g., ImageNet-R-200, Flowers-102, NINCO-64, and STL-10. 
  %$R$ denotes the Pearson correlation coefficient.
  } 
  \label{fig:vis_abstract}
  \vspace{-0.21in}
\end{figure}



% \begin{figure}[t]
%     \centering
%     \includegraphics[width = 0.99\linewidth]{images/bar_plot_ood_nc_summary.png}
%   \caption{\textbf{Controlling neural collapse (NC) enhances OOD transfer at encoder and OOD detection at projector.} The encoder achieves higher OOD transfer (indicated by lower error averaged across 8 OOD datasets) while decreasing NC (higher scores indicate lower NC). On the contrary, the projector achieves higher OOD detection (indicated by lower FPR95 averaged across same 8 OOD datasets) while increasing NC (lower scores indicate higher NC). Therefore, NC shows a \emph{linear} relationship with OOD transfer and OOD detection. In nutshell, encoder is a good OOD generalizer but a bad OOD detector whereas opposite is true for projector. All values are percentages.} 
%   \label{fig:vis_abstract}
% \end{figure}




%\begin{figure}[t]
%    \centering
%    \includegraphics[width = 0.99\linewidth]{images/vgg17_tunnel_koleo.png}
%  \caption{\textbf{Mitigating the tunnel effect leads to improved generalization.} The tunnel effect causes impaired OOD generalization for linear probes trained on embeddings from later layers in overparameterized DNNs. Typically, a supervised learning (SL) model suffers from the tunnel effect, as shown by \textcolor{blue}{blue} curves, where OOD accuracy significantly degrades in top layers (9-16). Whereas, using mitigation approach such as KoLeo regularization mitigates the tunnel effect and improves OOD performance, as shown by \textcolor{orange}{orange} curves. In this comparison, SL achieves 15.14\% average accuracy over 8 OOD datasets whereas SL+KoLeo (27.15\%) improves average accuracy by absolute 12\%.} 
%  \label{fig:tunnel_effect_koleo}
%\end{figure}

Out-of-distribution (OOD) detection and OOD generalization are two fundamental challenges in deep learning. OOD detection enables deep neural networks (DNNs) to reject unfamiliar inputs, preventing overconfident mispredictions, while OOD generalization allows DNNs to transfer their knowledge to new distributions. For applications like open-world learning, where a DNN continuously encounters new concepts, both capabilities are essential: OOD detection enables new concepts to be detected, while OOD generalization facilitates forward transfer to improve learning of these new concepts. Despite their importance, these tasks have primarily been studied in isolation. Here, we empirically and theoretically demonstrate a link between both tasks and neural collapse (NC), as illustrated in Fig.~\ref{fig:vis_abstract}.


NC is a phenomenon where DNNs develop compact and structured class representations~\cite{papyan2020prevalence}. While NC was first identified in the final hidden layer, later work has found that it occurs to varying degrees in the last $K$ DNN layers~\cite{rangamani2023feature,harun2024what,sukenikneural2024}. NC has a major impact on both OOD detection and generalization. Strong NC improves OOD detection by forming tightly clustered class features that enhance separation between in-distribution (ID) and OOD data~\cite{haas2023linking, wu2024pursuing, ming2022poem}. Conversely, NC impairs OOD generalization by reducing feature diversity, making it harder to transfer knowledge to novel distributions~\cite{kothapalli2023neural, masarczyk2023tunnel,harun2024what}. However, past work has considered NC in the context of either OOD detection or OOD generalization \textit{individually}, leaving open the question of how NC affects both tasks \textit{simultaneously}. To the best of our knowledge, no prior work has theoretically or empirically examined this relationship.

Here, we establish that the NC exhibited by a DNN layer has an \textbf{inverse relationship} with OOD detection and OOD generalization: \textit{stronger NC improves OOD detection but degrades generalization, while weaker NC enhances generalization at the cost of detection performance}. This trade-off suggests that a single feature space cannot effectively optimize both tasks, motivating the need for a novel approach. %Furthermore, we extend prior work by analyzing NC across \textit{multiple layers} rather than just the final hidden layer, providing new insights into how its extent varies throughout a DNN and how it affects OOD performance.

We propose a framework that strategically controls NC at different DNN layers to optimize both OOD detection and OOD generalization. We introduce entropy regularization to mitigate NC in the encoder, improving feature diversity and enhancing generalization. Simultaneously, we leverage a fixed Simplex Equiangular Tight Frame (ETF) projector to induce NC in the classification layer, improving feature compactness and enhancing detection. This design enables our DNNs to \textit{decouple representations} for detection and generalization, optimizing both objectives simultaneously.

\textbf{Our key contributions are as follows:}
\begin{enumerate}[noitemsep, nolistsep, leftmargin=*]
    \item We present the first unified study linking \textit{Neural Collapse} to both OOD detection and OOD generalization, empirically demonstrating their inverse relationship and extending analyses of NC beyond the final hidden layer.
    \item We develop a theoretical framework explaining how \textbf{entropy regularization mitigates NC} for OOD generalization and how a \textbf{fixed Simplex ETF projector enforces NC} for OOD detection.

    \item In extensive experiments on diverse OOD datasets and DNN architectures, we demonstrate the efficacy of our method compared to baselines. 
\end{enumerate}


