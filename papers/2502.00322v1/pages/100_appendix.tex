
\section{Appendix} \label{appendix}

\subsection{Dataset Details} \label{appendix:data}

To collect a dataset based on Debatepedia~\cite{gottopati2013learning}, we use Wayback Machine\footnote{\url{https://web.archive.org/}}, as the original website is no longer hosted.
We iterate through each debate articles page on the website with BeautifulSoup\footnote{\url{https://pypi.org/project/beautifulsoup4/}} and collect the 1) topic of the debate; 2) list of URLs under ``Supporting References''; and 3) list of URLs under ``Refuting References''.
We then use jusText\footnote{\url{https://pypi.org/project/jusText/}} to extract the text content from each web page, ignoring websites that are not free-to-access.

After this, we filter out instances that have less than five sources or do not have at least a 75/25 majority/minority split of perspective labels.
We then remove web pages that do not have any of the non-stopword tokens in the query, implemented with nltk, to ensure the web pages form a set of relevant documents.
We run this same process on ConflictingQA~\cite{wan2024evidence}.

Dataset statistics after data processing are in Table~\ref{appendix:table:dataset}.
Since all websites were publicly-accessible, our collected artifacts are within their intended use and licenses.
We sampled a subset of five document collections and manually checked them for PII and offensive content, which we did not find; we also found all text to be in English.

\subsection{The \model Algorithm} \label{subsection:model}

We detail \model in Algorithm~\ref{algo:mods}. For a debatable query $q$, document collection $\mathcal{D}$, number of topics $m$, and retrieval parameter $k$, we create speakers $\mathcal{S}$ for $\mathcal{D}$.
First, we retrieve speaker biographies $\mathcal{B}$ related to $q$ and plan $m$ topics $\mathcal{T}$ for $\mathcal{O}$ (\cref{subsection:agenda}). For each topic $t_j \in \mathcal{T}$, we pick relevant speakers $\mathcal{S}_j \subseteq \mathcal{S}$ and tailor them questions $\mathcal{Q}_j$ using their topic biographies $\mathcal{B}_j$ (\cref{subsection:moderator}). Each speaker supplies stance/fact perspectives $\mathcal{P}$, which are tracked in $\mathcal{O}$ (\cref{subsection:speaker}).
Finally, $\mathcal{O}$ is summarized all at once ($\mathbb{S}_{all}$) or per topic ($\mathbb{S}_{top}$) and returned to the user (\cref{subsection:summary}).

\subsection{Experimental Setup Details} \label{appendix:implementation}

All of our baseline implementations use GPT-4 (\texttt{gpt-4-1106-preview}) with 0 temperature and a maximum input token length of 127,000 tokens.
All baselines use zero-shot prompting, and the prompts will be released with our code after internal approval.
For costs associated with using GPT-4, see Appendix~\ref{subsection:efficiency}.

All models using retrieval, including \model, use ColBERT~\cite{khattab2020colbert}, a state-of-the-art retriever. For hyperparameters, we use a maximum document length of 300 tokens, a maximum query length of 64 tokens, 8 bits, and the \texttt{colbert-ir/colbertv2.0} checkpoint; none of these parameters were tuned during experimentation.
The clustering methods were implemented with BERTopic~\cite{grootendorst2022bertopic}, using all default values.

All experiments were run on a single H100 GPU, but as the only GPU usage comes from retrieval, we found \model and all baselines can be run on a Google Collaboratory T4 GPU (16GB of GPU memory).
Each baseline was allocated 24 hours for a single run.
We give more details about the runtime of \model in Appendix~\ref{subsection:efficiency}.

% Add prompts

\subsection{Metric Details} \label{appendix:metrics}

We extract all citations via regex\footnote{\url{https://docs.python.org/3/library/re.html}} by first finding text between square brackets (\texttt{[} and \texttt{]}) and then extracting integers between these spans.
The document coverage, faithfulness, and fairness metrics are all implemented with numpy\footnote{\url{https://numpy.org/}}.

We implement citation accuracy through entailment; entailment has shown to be a viable strategy to measure the factuality of text~\cite{maynez2020faithfulness}.
We use GPT-3.5 (\texttt{gpt-35-turbo-1106}) with 0 temperature to classify whether a generated sentence is entailed by the document it cited, using a 0-shot prompt shown in Prompt~\ref{prompt:cite_acc}
To evaluate the accuracy of this metric, we manually annotate 200 held-out examples (100 examples GPT predicted to be accurate citations, and 100 examples predicted to be inaccurate citations) of generated summaries for DQFS from all models (not used in evaluation).
We annotate these blindly, without knowing the output classification of GPT-3.5.
On this set, we obtain 87\% agreement with GPT-3.5, close to the agreement of 88\%, 90\%, and 96\% shown by human annotators in~\citet{min2023factscore}.
Further, this value is near the entailment-based accuracy given in other factuality tasks~\cite{balepur-etal-2023-expository, balepur-etal-2023-text}.

For the summary quality evaluation (\cref{subsection:summary_comp}), we use the Prometheus-v2 LLM evaluator\footnote{\url{https://github.com/prometheus-eval}}.
Example rubrics given to this evaluator are in Table~\ref{table:rubric}, which are adapted directly from~\citet{shao2024assisting}.

\subsection{Efficiency and Cost Comparison} \label{subsection:efficiency}

In Tables~\ref{appendix:table:cost_cqa} and \ref{appendix:table:cost_debate}, we present the cost (LLM input/output tokens, number of calls) and efficiency (seconds taken for inference) of \modelTopic, the slightly more expensive model out of the two \model baselines, versus Hierarchical Merging and Incremental Updating~\cite{chang2024booookscore, adams2023sparse}, the two other best-performing baselines, which also happen to be multi-LLM systems. Despite \model using more LLM calls through single-turn LLM debate, our use of retrieval and a moderator LLM greatly reduces the number of input tokens \model otherwise would have consumed, keeping GPT-4 cost competitive with Hierarchical Merging, and making our model cheaper than Incremental-\textit{Topic}.
The inference time of multi-LLM summarization systems like \model could be improved, a common limitation of agentic systems~\cite{li2024personal}, and one possible strategy could be to use multi-threading or batched decoding to parallelize the discussions of LLM speakers. 

\subsection{Results for All Topics} \label{appendix:results}

We run \model and all baselines where the number of topics $m$ ranges between $2$ and $5$ inclusive, a typical range of paragraphs in argumentative essays~\cite{mery2019use}.
Tables~\ref{table:doc_cover_cqa_all} and \ref{table:doc_cover_debate_all} display the citation coverage and balance metrics from \cref{subsection:citation_comp} for all $m$, while Tables~\ref{appendix:table:llm_cqa} and \ref{appendix:table:llm_debate} display the summary quality metrics from \cref{subsection:summary_comp} results for all $m$.
Our claims hold for these varied values of $m$; \model generates comprehensive and balanced summaries while maintaining traditional output quality metrics, regardless of the number of topic paragraphs it must generate.

\subsection{Results for Hierarchical Merging over Topic Paragraphs} \label{appendix:results_hm}

Further, the Hierarchical Merging baseline we use does not generate summaries one topic at a time.
We believe that such a model (i.e. Hierarchical-\emph{Topic}) is too costly and inefficient to deploy, so we do not compare against it in the main body of the work. 
In Tables~\ref{table:doc_cover_cqa_all_comp} and \ref{table:doc_cover_debate_all_comp} we provide some results for this model, which still underperforms \modelTopic.
Further, we show in Table~\ref{appendix:table:cost_weird} that this model is much more costly compared to \model.
It is also more costly than a version of \model that iterates through all speakers, highlighting the utility of retrieval to keep inference time and LLM cost low.

\subsection{Results with GPT-4 Mini} \label{appendix:results_mini}

All of our models are implemented with GPT-4, but we also run some preliminary experiments with \modelTopic using GPT-4 mini.
In citation coverage, fairness, and faithfulness (Tables~\ref{table:doc_cover_cqa_all_comp_mini} and \ref{table:doc_cover_debate_all_comp_mini}), \modelTopic using GPT-4 mini underperforms the model using GPT-4, suggesting that larger models are better suited for multi-LLM systems like \model. However, the GPT-4 mini system still exhibits strong performance, and is even comparable to several of the baselines using GPT-4 in Tables~\ref{table:doc_cover_cqa_all_comp} and \ref{table:doc_cover_debate_all_comp}, further showcasing the efficacy of our framework.

\subsection{Results with Fixed Topics} \label{appendix:fixed_topics}

Each baseline in \cref{subsection:baselines} produces distinct topics while planning a summary.
To ensure the citation coverage and balance gains in \model are not just derived from our agenda planning step (\cref{subsection:agenda}), we implement a version of each baseline that is asked to generate summaries for the same topics that \modelTopic generates.
We present these results in Tables~\ref{table:doc_cover_cqa_fixedtopic} and \ref{table:doc_cover_debate_fixedtopic}, and find that \model still largely outperforms baselines even when using our topics, suggesting that our agenda planning is not the source of gains in the framework.

% We also compare \modelTopic to the Hierarchical Merging baseline used in \cref{subsection:efficiency_main} more extensively in Tables~\ref{table:doc_cover_cqa_all_comp} and \ref{table:doc_cover_debate_all_comp}.
% Even though this baseline runs inference on each document for each topic, our structured outline allows us to outperform this baseline with much better efficiency.

\subsection{Outline Perspective Accuracy} \label{appendix:outline}

During speaker discussion (\cref{subsection:speaker}), we ask speakers to provide perspectives in the form of facts in the document.
These facts are grouped by whether the fact gives evidence for why the answer to the query is ``yes'' or ``no'', which also provides another layer of organization to enrich the user's understanding of the outline (\cref{subsection:qg}).
To assess the accuracy of these yes/no labels, we ask human annotators to label if each paragraph in 10 document collections (5 from DebateQFS, 5 from ConflictingQA) strongly supports, weakly supports, strongly refutes, weakly refutes, or is neutral toward the input query.
In total, we collect 7592 annotations, and aggregate them into one of three labels: supports, refutes, or neutral.\footnote{For each annotator, we score a paragraph as $ \pm1$ for strongly support/reject, $\pm0.5$ for weakly support/reject, and 0 for neutral. We take the sum of these scores over all annotators, and set the final label to support/reject if the sum is greater/less than 0. A score of 0 yields a neutral label.}
We will also release these paragraph-level annotations, which may be useful for training DQFS models.
We use the same procedure in Appendix~\ref{appendix:human} for this user study.

After collecting ground truth paragraph labels, we take the outlines produced by \model on this subset of 10 examples. For each predicted yes/no fact in the outline, we post-hoc attribute~\cite{huang-chang-2024-citation} the paragraph in the speaker's document that was the source of the information in the fact (with ColBERT).
We compare the accuracy of the LLM's yes/no label using the ground truth labels from human annotators, which are 0.798, 0.806, 0.781, and 0.803 for $m = 2, 3, 4, 5$, respectively.
Our accuracy is near the accuracy of LLMs on existing stance detection benchmarks~\cite{lan2024stance}, meaning our yes/no labels provide a useful and fairly accurate signal for users.

\subsection{Human Evaluation Setup} \label{appendix:human}
We conducted user evaluations to compare the readability and balance of summaries produced by different models (\model, Long-Context, Hierarchical, Incremental-Topic). The evaluation was divided into two parts: one focusing on the entire summary and the other on topic paragraphs.

\subsubsection{Recruitment \& Procedure}
We recruited 76 participants via Prolific, all of whom were based in the United States and required to have fluency in English. Each participant rated a total of 20 summaries, assessing the output from each of the four models for a given debate query. 
Participants were paid \$12/hour, the recommended rate on the website.
To mitigate order and fatigue effects, the presentation order of summaries was counterbalanced. Each summary was rated by 3-5 different participants. Additionally, the task included two baseline comprehension checks to ensure participants understood the instructions and metric definitions. Participants who did not pass these checks were excluded from the final analysis.
These annotations did not require review from an Institutional Review Board (IRB).
We collect no Personal Identifiable Information during the study.


\subsubsection{Rating Criteria}
The task included two Likert ratings for Readability and Balance. Additionally, participants could provide open comments for feedback or to report any issues. For the Likert items, participants saw the following questions:

\begin{itemize}
    \item \textbf{Readability.} Is the summary easy to read and understand?
    \begin{enumerate}
        \item The summary is very unclear, with consistent grammatical errors and disjointed ideas.
        \item The summary is often unclear, with frequent grammatical errors and poor flow.
        \item The summary is moderately clear but has some grammatical errors and awkward transitions.
        \item The summary is mostly clear, with minor grammatical errors and mostly smooth transitions.
        \item The summary is exceptionally clear, grammatically perfect, and flows seamlessly.
    \end{enumerate}

    \item \textbf{Balance.} Does the summary address both sides of the debatable query by using counterarguments to present a well-rounded view?
    \begin{enumerate}
        \item The summary is heavily biased, with little to no use of counterarguments and only one side addressed effectively.
        \item The summary is poorly balanced, significantly favoring one side and using counterarguments ineffectively.
        \item The summary is somewhat balanced but has noticeable bias and some awkward or less effective counterarguments.
        \item The summary is mostly balanced, with minor bias and effective use of counterarguments.
        \item The summary is perfectly balanced, equally addressing both sides and effectively using counterarguments.
    \end{enumerate}
\end{itemize}

\subsubsection{Results}

Figure~\ref{fig:annot} shows the full distribution of Prolific annotations for Balance and Readability across Summaries and Topic Paragraphs. 


\subsection{Sample Outputs} \label{appendix:outputs}

We present sample outputs generated by \model on ConflictingQA (Summary~\ref{summary1}, \ref{summary2}) and DebateQFS (Summary~\ref{summary3}, \ref{summary4}).
The summaries from \model have high coverage, citing several documents from the input collection, while also being balanced.
Further, the summary quality of \model remains high.
After comparing the summary for the EU expansion query in Figure~\ref{fig:intro} from 0-shot GPT-4 versus the summary from \model in Summary~\ref{summary3}, the balance, comprehensiveness, and quality gains from our method are clear.

\clearpage
%\section{Steering details: prompts, datasets, and parameters}
\label{app: prompts}

We now describe the parameters and prompts used for steering Llama-3.1-8B-it and Gemma-2-9B-it toward different concepts.

\subsection{Our prompting method}

We consider a specific example to explain our prompting method, where we extract directions to induce different identities from the surname `Newton'. To extract semantically meaningful directions from the activation spaces of LLMs for steering, we first choose a list of labeled prompts for a list of desired concepts, similar to the approaches of \citet{representation_engineering, turner2023activation}. However, unlike their methods, our prompts do not need to consist of contrastive pairs of positive and negative examples. Further, we found benefit in some cases by choosing prompts to be from real text, and not synthetic datasets. For example, we extracted meaningful concepts corresponding to political positions and disambiguating word meanings from pairs of Wikipedia articles. 

Consider the specific case of distinguishing Cam Newton versus Isaac Newton (Figure~\ref{fig: rfm/pca newton, llama-3.1-8B}). We obtain sentences from the Isaac and Cam Newton wikipedia articles. 
Suppose we want to learn the vector for `Isaac' Newton. Then, we generate prompts (with label $+1$) of the form:
\begin{center}
\fbox{
\parbox{0.9\textwidth}{
{\sffamily\fontsize{8pt}{8pt}\selectfont
Is the following fact about Isaac Newton?\\
Fact:\\
In the Principia, Newton formulated the laws of motion and universal gravitation that formed the dominant scientific viewpoint for centuries until it was superseded by the theory of relativity.}
}
}
\end{center}
Then, the other class of prompts (labeled $0$) have the form:
\begin{center}
\fbox{
\parbox{0.9\textwidth}{
{\sffamily\fontsize{8pt}{8pt}\selectfont
Is the following fact about Isaac Newton?\\
Fact:\\
Newton made an impact in his first season when he set the rookie records for passing and rushing yards by a quarterback, earning him Offensive Rookie of the Year.}
}
}
\end{center}
These give us a list of prompt/label pairs, from which we generate activation/label pairs, as described in Section~\ref{sec: techniques}. We then solve RFM (or another layer-wise predictor) on each layer to predict the label function (Isaac vs. Cam Newton). For RFM, the concept vectors at each layer $c_\ell$ are then the top eigenvectors of the AGOP from each RFM predictor.

\subsection{Human Languages} For triggering language switches as in Figures~\ref{fig: english_chinese, llama-3.1-8B} and \ref{fig: english_spanish, llama-3.1-8B}, we used examples generated from the following prompt template.

\begin{center}
\fbox{\parbox{0.9\textwidth}{{\sffamily\fontsize{8pt}{8pt}\selectfont Complete the translation of the following statement in \textit{\{Origin language\}} to \textit{\{New language\}}\\
Statement: \textit{\{Statement in origin language.\}}\\ Translation: \textit{\{Partial translation in new language.\}} }
}
}
\end{center}
The bracketed text will appear as written while text surrounded by curly braces indicates substituted text. We obtained list of statements in the origin and new languages from datasets of translated statements. To generate the partial translations we truncated translations to the first half of the tokens. For Spanish/English translations we used datasets from \url{https://github.com/jatinmandav/Neural-Machine-Translation/tree/master}. For Mandarin/English, we obtained pairs of statements from \url{https://huggingface.co/datasets/swaption2009/20k-en-zh-translation-pinyin-hsk}. 

To evaluate translations for human language, we use OpenAI's o1 reasoning model to generate a list of 100 English statements and questions. We then evaluate the translation from steering of each sentence.

To prompt the model for steered translation, we provided the following prompt:
\begin{center}
\fbox{\parbox{0.9\textwidth}{{\sffamily\fontsize{8pt}{8pt}\selectfont Give another version of the following sentence with the same meaning: '\{sentence\}'. Write the version in quotes.}
}
}
\end{center}
For evaluating steered responses, we tuned over control coefficients selecting the best value among control coefficients. We steer the outputs of all blocks of the transformer. For Llama-3.1-8B-it, we tuned over coefficients: 0.2, 0.3, 0.4, 0.5, 0.6, 0.7, and for Gemma-2-9B-it: 4.0, 5.0, 6.0, 7.0, 8.0, 9.0. Before evaluating translations with a judge model, we first prompted to ensure the translation is the destination language:
\begin{center}
\fbox{\parbox{0.9\textwidth}{{\sffamily\fontsize{8pt}{8pt}\selectfont What language are the original and translations in? Give your answer in the format '\{Original language\}, \{Translation language\}'. If the translation is empty, assume Language 2 is English."}
}
}
\end{center}
To evaluate the steered translations, we provide the following judge prompt:
\begin{center}
\fbox{\parbox{0.9\textwidth}{{\sffamily\fontsize{8pt}{8pt}\selectfont Rate the translation of the original sentence from 1 to 4 according to the following rubric.\\\\
Score of 1: The translation is mostly inaccurate or unrelated.\\
Score of 2: The translation is slightly unrelated to the original.\\
Score of 3: The translation has mostly the same meaning as the original.\\
Score of 4: The translation has the same meaning as the original.\\\\
Give your response in the format '{score}/4.' Do not penalize awkward or excessive wording. If the translation is empty, give a score of 0.\\
----------------------------------------\\
ORIGINAL: \{original\}\\
----------------------------------------\\
TRANSLATION: \{translation\}"}
}
}
\end{center}

\subsection{Poetry} Prompts for poetry followed the same format as human languages. We obtained 100 pairs of standard English sentences and poetic translations from OpenAI's o1 model. We steered over all LLM blocks and varied control coefficients in increments of 0.1 over 0.4 to 0.8. Figure~\ref{fig: steered poetry style} uses coefficient 0.6. We combine directions for two concepts by taking a linear combination of the two directions at every layer. For poetry and dishonesty (Figure~\ref{fig: main figure}), we use $a=1.2,b=1.0$ as the multiple for each concept, respectively, then use coefficient $0.4$ on the combined vector across all blocks. 

\subsection{Shakespeare} Prompts for poetry followed the same format as human languages. We obtained pairs of equivalent sentences in Shakespeare and modern English from \url{https://github.com/harsh19/Shakespearizing-Modern-English/tree/master}. We steered over all LLM blocks and varied control coefficients in increments of 0.1 over 0.4 to 0.8. For Shakespeare and harmful (Figure~\ref{fig: main figure}), we use $a=1.0,b=0.5$ as the multiple for each concept, respectively, then use coefficient $0.5$ on the combined vector across all blocks. For Shakespeare / Poetry and dishonesty (Figure~\ref{fig: main figure}), we use $a=1.2,b=1.0$ as the multiple for each concept, respectively, then use coefficient $0.4$ on the combined vector across all blocks.

\subsection{Programming Languages}

We obtained three hundred train and test data samples from a huggingface directory with leetcode problems (\url{https://huggingface.co/datasets/greengerong/leetcode}). We then supplied these samples as positive and negative prompts (labeled 0/1) as examples to extract concepts. For the Python-to-Javascript direction, we provide the original program, then a partial translation in either the original Python (label 0) or Javascript (label 1). The partial translation was truncated to half the original length. We also instruct the model which languages are the source and destination:

\begin{center}
\fbox{
   \parbox{0.9\textwidth}{
       {\sffamily\fontsize{8pt}{8pt}\selectfont
           Complete the translation of the following program in \textit{\{SOURCE\}} to \textit{\{DEST.\}}.\\
           Program:\\
           \textit{\{Code in origin language.\}}\\
           Translation:\\
           \textit{\{Partially translated code in dest. language.\}}
       }
   }
}
\end{center}


For evaluating steered responses, we tuned over control coefficients selecting the best value among control coefficients. We steer the outputs of all blocks of the transformer. For Llama-3.1-8B-it, we tuned over coefficients: 0.4, 0.5, 0.6, 0.7, 0.8, and for Gemma-2-9B-it: 4.0, 5.0, 6.0, 7.0, 8.0, 9.0. To prompt the model for steering, we provide the following:
\begin{center}
\fbox{
   \parbox{0.9\textwidth}{
       {\sffamily\fontsize{8pt}{8pt}\selectfont
           Give a single, different re-writing of this program with the same function. The output will be judged by an expert in all programming languages. Do not include an explanation.\\\\\{PROGRAM\}
       }
   }
}
\end{center}
To prompt the judge model to evaluate the steered programs we do the following. 
\begin{center}
\fbox{
   \parbox{0.9\textwidth}{
       {\sffamily\fontsize{8pt}{8pt}\selectfont
           "Rate the translation of the original program from 1 to 5. Do not reduce score for name changes. Give your response in the format '\{score\}/5. \{Reason\}'.\\
           ------------------------------------------------------------\\
           ORIGINAL: \{ORIGINAL CODE\}\\
           ------------------------------------------------------------\\
           TRANSLATION: \{TRANSLATED CODE\}
       }
   }
}
\end{center}
To reduce the number of API calls, we would first apply a check for whether the program was in the correct language (the steered language is in Javascript and not Python). To detect language, we used Python indicators = [``def ", ``print(", ``elif ", ``self.", ``len(", ``range(", ``elif"] and 
Javascript indicators = [``function", ``console.log(", ``var ", ``let ", ``const ", ``=>", ``.has(", ``document.", ``||", ``\&\&", ``null", ``===", ``if (", ``else if", ``while ("]. The predicted language is whichever has more indicators. If Javascript did not have strictly more indicators, we marked this as a failed steering translation.

\subsection{Hallucinations}

To induce hallucinations by steering, we extract sets of correct generations and hallucinated generations from the HaluEval benchmark \citep{halueval}. Then, we generate prompts of the form:
\begin{center}
\fbox{\parbox{0.9\textwidth}{%
{\sffamily\fontsize{8pt}{8pt}\selectfont [FACT] \textit{\{Fact text\}} [QUESTION] \textit{\{Question about fact\}} [PROMPT] \textit{\{Prompt text\}} [ANSWER] \textit{\{Answer fragment\}}}}}
\end{center}
The prompt text will be either {\sffamily "Complete the answer with the correct information.''}, or {\sffamily "Make up an answer to the question that seems correct.''} for correct and hallucinated generations, respectively. Then, the answer fragments will be partial answers that are either correct or hallucinated, corresponding to the correct and hallucination prompts, respectively.

\subsection{Science subjects}

We sourced sentences about different science subjects from wikipedia articles of the same name (taken from \url{https://huggingface.co/datasets/legacy-datasets/wikipedia}). Then, we trained predictors on the following prompts:

\begin{center}
\fbox{
\parbox{0.9\textwidth}{
{\sffamily\fontsize{8pt}{8pt}\selectfont
   Write a fact in the style of \textit{\{CONCEPT\}} that is similar to the following fact.\\
   Fact:\\
   \textit{\{FACT\}}
   }
   }
}
\end{center}

\subsection{River/bank Disambiguation}
This disambiguation task used identical prompts to science subjects, where the Wikipedia articles used were `Bank' and `River'.

\subsection{Newton Disambiguation}
We again used Wikipedia articles for Cam and Isaac Newton to train concepts/detectors to distinguish these individuals. The prompt was as follows:
\begin{center}
\fbox{
\parbox{0.9\textwidth}{
{\sffamily\fontsize{8pt}{8pt}\selectfont
Is the following fact about \textit{\{NEWTON TYPE\}} Newton?\\
Fact:\\
\textit{\{FACT\}}
}
}
}
\end{center}


\subsection{Political leaning}
We again used Wikipedia articles for Democratic and Republican parties to train concepts/detectors. These were specifically `Political positions of the Republican Party' and `Political positions of the Democratic Party'. The prompt was as follows:
\begin{center}
\fbox{
\parbox{0.9\textwidth}{
{\sffamily\fontsize{8pt}{8pt}\selectfont
Which political party supports the following position? \\ 
Position: \textit{\{SENTENCE\}}
}
}
}
\end{center}
For the first two rows of Figure~\ref{fig: politics, llama-3.1-8B}, we steered over blocks $-8$ through $-31$ with steering coefficient 0.8. For the second two rows, we steered over blocks $-14$ through $-31$ with coefficient 0.8.  

\subsection{Harmful content}

We used the dataset and prompting from \url{https://github.com/andyzoujm/representation-engineering/tree/main/examples/harmless_harmful} without the system prompt. Specifically these utilize the prompts from \url{https://huggingface.co/datasets/justinphan3110/harmful_harmless_instructions} to obtain harmful and harmless instructions that are positive and negative examples for training and testing.

\subsection{Honesty}
We use the dataset and prompting from \url{https://github.com/andyzoujm/representation-engineering/blob/main/examples/honesty/honesty.ipynb} \citep{representation_engineering}. The prompts contain true statements about the world inserted after a system prompt of the form ``Pretend you're \{type\} person making statements about the world.'' 
\begin{center}
\fbox{
\parbox{0.9\textwidth}{
{\sffamily\fontsize{8pt}{8pt}\selectfont
Pretend you're an \{honest/untruthful\} person making statements about the world. \textit{\{Partial truthful statement\}}.
}
}
}
\end{center}
\section{Dataset Generation}
\label{sec:dataset}
\revise{
To train the proposed GNN, we constructed a dataset of building structures and a subset of these structures were subjected to fire simulations using FEA. The dataset generation process is illustrated in \figref{fig:dataset_generation_procedure}. Initially, a total of 33,000 building structures with geometrical details, material properties, and gravity loads were created. Due to randomness in generating these structures, a filter is applied to remove unreasonable data after gravity load simulation, which included 15,377 structures. A trade-off between computational feasibility and model performance is made among the remaining 17,623 structures. As further labeling structures with MIDR requires resource-intensive fire simulations via OpenSeesRT, a large proportion of 16,050 structures is selected as unlabeled dataset. On the other hand, each of the other 1,573 structures was further subjected to 30 different fire simulations, forming the labeled dataset containing $1,573\times 30 = 47,190$ fire cases.} This section details the step-by-step process for generating the dataset, including geometry creation, material property assignment, and simulations due to gravity loads and fire scenarios. 
% To train the proposed neural network, we constructed a dataset comprising building structure data and a subset of fire scenario data. The dataset generation process is illustrated in \figref{fig:dataset_generation_procedure}. 
% A total of 33,000 building structures with geometric details, material properties, and gravity loads were initially created. Out of these, 3,000 structures were selected as labeled data, and the remaining 30,000 were designated as unlabeled data. Further, about half of them filtered out due to instability under gravity loads only. 
\begin{figure*}[h!]
    \centering
    \includegraphics[width=0.8\linewidth]{figures/dataset_filter_procedure.pdf}
    \caption{Workflow for dataset generation (geometry, material property, gravity loads, and fire scenarios).}
    \label{fig:dataset_generation_procedure}
\end{figure*}

\subsection{Geometry Generation}
\label{subsec:geometry_generation}
The geometry of the building structures forms the foundation of the dataset. Regular 
\revise{3D structures} resembling multi-story parking structures or shopping malls were generated, with parameters such as building floor dimensions and story heights selected randomly. Each building structure is composed of multiple rooms, which serve as the basic unit in this study. A room herein is a cuboid space defined by specific length, width, and height. Within a structure, rooms of the same dimensions are uniformly arranged along the length, width, and height, corresponding to the $x$-, $y$-, and $z$-axes, respectively. Structures vary in room size and number of rooms along each axis. Specifically, the room length, width, and height are independently sampled from a uniform distribution within the interval $[2, 5]$ meters along the three directions of the structure. Similarly, the room number along each axis is uniformly sampled independently as an integer within the interval $[2, 7]$, i.e., the maximum number of stories of the buildings simulated in this study is 7.

To introduce variability and simulate real-world scenarios, approximately $8\%$ of structural elements (beams or columns) are randomly removed after initial geometry creation. 
\revise{Such removal is not fire-induced damage, but reflects functional diversity often observed in real buildings, such as open spaces designed for activities in shopping malls, e.g., ice skating rinks. Examples of the generated geometries are illustrated in \figref{fig:example_generated_geometry}, showcasing the diversity and realism of the dataset. This element removal does not affect the definition of room's geometry in the structure and nor does it affect the number of considered fire scenarios.} 

\revise{A range of coefficient of variation values ($3.3\%$ to $17.5\%$) was derived from prior studies that investigated the statistics of geometrical and material properties of structural components of buildings (e.g., \cite{mirza1979variations, lee2004probabilistic}). These studies provide empirical data on the natural variability in parameters such as Young's modulus, yield strength, and dimensions of structural elements due to manufacturing tolerances and material inconsistencies. By selecting $8\%$ for the removal of structural elements in our database, we aimed to maintain a level of variability that is representative of real-world uncertainties while ensuring computational feasibility. This choice ensures that the database captures realistic deviations without introducing extreme cases that may not be commonly encountered in practice.}

\begin{figure*}[h!]
    \centering
    \includegraphics[width=\linewidth]{figures/example_generated_geometry.pdf}
    \caption{Examples of generated structural geometry of different sizes (all dimensions in meters).}
    \label{fig:example_generated_geometry} 
\end{figure*}

{\blockRevise

In this study, we opted for a deterministic square, dimension of $0.1$ m, solid cross-sectional steel elements due to their simplicity in modeling and analysis. Square sections exhibit uniform geometrical properties in all directions, simplifying the computation of structural responses and avoiding complications associated with more complex shapes, such as wide-flange sections, facilitating the computational efficiency and scalability to generate a large dataset. This choice also helps to mitigate issues related to stress concentrations and facilitates a more straightforward representation of structural behavior under thermal loads. 

\textit{Remark:} The selected cross-section provides a comparable flexural rigidity to a $W 130 \times 130 \times 28.1$ wide-flange section (metric units), albeit with significantly higher axial rigidity. This cross-section is acceptable for gravity-load-designed frames under service loading conditions where the models assume fully rigid, moment-resisting beam-column connections for the evaluation of the IDR under thermal loading. This assumption is reasonable in this computational study where the primary interest is to understand the global deformation response of frames under fire conditions. The selection of uniform square cross-sections for both beams and columns, rather than adherence to standard capacity design principles, was made here primarily for computational efficiency and to reduce design parameters in the database generation process. This choice allows for simplified and scalable approach to analyze the fire-induced response of generic steel frames without the need for large section variations, where this study mainly focuses on the fire vulnerability assessment using ML-based predictions. However, if additional loading conditions, e.g., seismic or wind loads, were to be considered, larger sections, strong-column/weak-beam principle, and ductile detailing would be required in the generated buildings for realistic structural behavior under combined loading conditions. Future studies may also consider investigating the influence of variable cross-sectional dimensions and semi-rigid connections on the structural performance under fire conditions. 
} % blockRevise

\subsection{Material Properties}
Steel is chosen as the material for the structures. To reflect real-world variations, we randomly assign one of five slightly different steel material types to each structural element. \revise{
The ranges of material properties are provided in \tabref{tab:material_property_ranges} and the properties are sampled from uniform distributions of the corresponding ranges. These variations simulate differences arising from manufacturing batches or regional material properties. That these properties are at ambient temperature and change when the temperature rises due to a fire. The selection of materials with varying properties is aimed at increasing the diversity of the data. Our goal is to represent as wide a range of data as possible with a limited amount of building structure data, thereby enhancing the generalization ability of the GNN. Our assumed material property ranges are expected to be wider than the real-world conditions based on findings in \cite{mirza1979variations, lee2004probabilistic}. Therefore, we are essentially tackling a more challenging and general task. If we can solve this problem, we are confident that our method will perform equally well or even better in real-world scenarios.
}
\begin{table}[h!]
    \centering
    \caption{Material properties ranges for considered steel structures.}
    \begin{tabular}{lc}
        \toprule
        Property & Range \\
        \midrule
        Young's modulus & [168, 252] GPa \\
        Yield strength & [220, 330] MPa \\
        Strain-hardening ratio & [0.8, 1.2] \% \\
        \bottomrule
    \end{tabular}
    \label{tab:material_property_ranges}
\end{table}

\subsection{Gravity Loads}
Gravity loads are applied to columns and beams based on their \revise{influence (tributary) areas as typically conducted in structural analysis. The considered ``service'' load conditions include the column self-weight and the additional loads directly supported on the beams from their self-weight and weights of the reinforced concrete slabs, people as live load, and building content. An edge beam typically carries approximately half the gravity load supported by a parallel interior beam}. The ranges of gravity loads are listed in \tabref{tab:gravity_load_ranges}. \revise{The loads are sampled from uniform distributions of the corresponding ranges.} Structures that failed to meet an MIDR threshold of $1\%$ under gravity loads were deemed unacceptable designs and filtered out, as such configurations of randomly chosen geometry, material, and gravity load combinations were considered unrealistic from a regulatory and practicality points of view.
\begin{table}[h!]
    \centering
    \caption{Gravity load ranges for considered beams and columns.}
    \begin{tabular}{lc}
        \toprule
        Element & Range (kN/m)  \\
        \midrule
        Column & [0.5, 1.0]  \\
        Edge beam & [1.5, 4.5]  \\
        Interior beam & [3.0, 7.5]  \\
        \bottomrule
    \end{tabular}
    \label{tab:gravity_load_ranges}
\end{table} 

\subsection{Rule-based Thermal Load Generation}
\label{subsec:thermal_load_generation}
To evaluate a building's structural response during a fire event, we employed a simplified rule-based approach for thermal load generation. 
% Previous studies \cite{nan_structuralfire_2023} have demonstrated that steel structures rapidly equilibrate with surrounding gases temperatures due to efficient heat exchange. Consequently, gas temperatures can be directly used as inputs for FEA tools, e.g., OpenSees, simplifying the process of modeling thermal loads. 
% Accurately simulating temperature fields in fire scenarios poses significant challenges. Advanced thermodynamic simulations, such as those performed using Fire Dynamics Simulator (FDS) \cite{mcgrattan_fire_2000}, provide precise temperature predictions. However, these methods are hindered by high computational costs, prolonging execution times, and limited scalability, making them impractical for generating large datasets. Additionally, real-world fire loads often display substantial spatial variability across different rooms \cite{dundar_fire_2023}, resulting in scenario-specific temperature fields with limited generalizability. For example, studies on bridge fires \cite{he_study_2024} have demonstrated that environmental factors, such as wind speeds, can significantly influence temperature distributions. Furthermore, even within identical scenarios, variations in fire modeling methodologies can produce distinctly different temperature fields \cite{zhang_temperature_2020, du_new_2012}. These challenges emphasize the need for efficient and adaptable methods to generate fire temperature data.
% To address these issues, we adopted a rule-based approach to model temperature variations. 
According to \cite{spearpoint_fire_2008}, a typical fire development follows a predictable pattern. During the {\em{growth stage}}, the temperature rises slowly and approximately linearly after ignition. This is followed by the {\em{flashover stage}}, where temperatures increase rapidly to peak values. After reaching the peak, the temperature either stabilizes or continues to rise slowly until the {\em{decay stage}} begins. Inspired by this fire development pattern, we describe the temperature evolution in time, $t$, prior to the decay stage in two distinct stages:
\begin{enumerate}
    \item {\bf{Initial linear increase stage}}: For $t \in [0, t_1)$, temperature increases gradually and linearly as the fire spreads through the building. This stage represents the time before the fire directly affects a structural element.  
    \item {\bf{ISO 834 fire curve stage}}: For $t \in [t_1, t_{\thre}]$, temperature rises rapidly following the ISO 834 curve \cite{ISO834}, modeling the direct impact of the fire on the structural element. 
\end{enumerate}
The slope of the linear temperature increase, $c$, and the transition time, $t_1$, are influenced by the spatial relationship between the fire source and the structural element. For the second stage of temperature evolution, we utilize the ISO 834 curve, a widely accepted standard for fire resistance testing. This standardized fire curve describes the temperature rise over time, enabling rapid and consistent thermal fields across various scenarios. The duration of fire simulation in this study is set to $t_{\thre}=60$ minutes. This value represents the upper limit for the temperature evolution of each structural element, providing a consistent basis for analyzing the structural response to fire.

Let $(x, y, z)$ represents the midpoint of a structural element and $(x_{\subfire}, y_{\subfire}, z_{\subfire})$ the fire source point. \revise{Integer parameters $h$ and $h_{\subfire}$ correspond to the respective floor levels of the element and the fire source}. The temperature evolution for each element is expressed as follows:
\begin{enumerate}
    \item Linear increase stage ($0 < t < t_1$):
    \begin{equation}
    T(t) = c \cdot t,
    \end{equation}
    where $c$, the rate of temperature increase ($^\circ\mathrm{C}/\mathrm{min}$), depends on the height difference between the element, $h$, and the fire source, $h_{\subfire}$:
    \begin{equation}
        c = 
        \begin{cases} 
        5\left/\left(h - h_{\subfire} + 1\right)\right., & h \geq h_{\subfire}, \\
        2\left/\left(h_{\subfire} - h\right)\right., & h < h_{\subfire}.
        \end{cases}
    \end{equation}
     \item ISO 834 stage ($t \geq t_1$):
\begin{equation}
    T(t) = c \cdot t_1 + 345 \log_{10} \left(8 \left(t - t_1\right) + 1\right).
\end{equation}
\end{enumerate}

The transition (arrival) time $t_1$, marking the end of the linear stage, depends on the spatial distance between the fire source and the element. We define the following two Euclidean distances $L_p$ in the $xy$ plane and $L_s$ in the $xyz$ space:
\begin{eqnarray}
L_p & \triangleq & \sqrt{(x - x_{\subfire})^2 + (y - y_{\subfire})^2}, \\
\label{eq:Lp}
L_s & \triangleq & \sqrt{(x - x_{\subfire})^2 + (y - y_{\subfire})^2 + (z - z_{\subfire})^2}.
\label{eq:Ls}
\end{eqnarray}
Accordingly, the transition time, $t_1$, is expressed as follows:
\begin{equation}
    t_1 = 
    \begin{cases}
    \beta_{1} \cdot \left(1 - \exp\left\{- L_s\left/\alpha_{1}\right.\right\}\right), & h > h_{\subfire}, \\
    \beta_{2} \cdot \left(1 - \exp\left\{- L_p\left/\alpha_{2}\right.\right\}\right), & h = h_{\subfire}, \\
    \beta_{3} \cdot \left(1 - \exp\left\{- L_s\left/\alpha_{3}\right.\right\}\right), & h < h_{\subfire} .
    \end{cases}
    \label{eq:t1}
\end{equation}
The parameters $\beta_i$ and $\alpha_i$ for determining $t_1$ are summarized in Table~\ref{tab:fire_spread_parameters}. In this study, we take $r_{\mathrm{up}}=0.95$ and $r_{\mathrm{down}}=0.97$.
\begin{table}[ht]
    \centering
    \caption{Fire spread parameters for $t_1$ calculations.}
    \begin{tabular}{lcc}
        \toprule
        Case  & $\beta_i$ & $\alpha_i$  \\
        \midrule
        $i=1$, Upward spread & $16 \left.\left(1-r_{\mathrm{up}}^{\left|h-h_{\subfire}\right|}\right)\right/\left(1-r_{\mathrm{up}}\right)$ & $10$  \\
        $i=2$, Horizontal spread & $18$ & $18$  \\
        $i=3$, Downward spread & $30 \left.\left(1-r_{\mathrm{down}}^{\left|h-h_{\subfire}\right|}\right)\right/\left(1-r_{\mathrm{down}}\right)$ & $5$  \\
        \bottomrule
    \end{tabular}
    \label{tab:fire_spread_parameters}
\end{table}

\figref{fig:t1_curve} illustrates the $t_1$ curves for various fire scenarios: (1) fire originating on the lower floor, $h-h_{\subfire}=1$ with rapid upward spread, (2) fire on the same floor, $h=h_{\subfire}$ with the fastest spread, and (3) fire on the upper floor, $h_{\subfire}-h=1$ with slow downward spread. The exponential decay in $t_1$ reflects the accelerating fire propagation speed as the distance increases. \figref{fig:t1_curve} also indicates that the employed simplified model is consistent with the Markov chain-based dynamic model given by \cite{cheng_dynamic_2011}, where the rooms at the same floor of the fire point start flashover slightly before the corresponding upper floors. Additionally, $\beta_{1}$ and $\beta_{3}$ are the summation of a geometric sequence, where story level $h$ is the index. The common ratios $r_{\mathrm{up}}<1$ in $\beta_{1}$ and $r_{\mathrm{down}}<1$ in $\beta_{3}$ indicate that the fire speeds up to spread through the next story, which is consistent with the real-world fire spread mechanism given in \cite{hokugo_mechanism_2000}. The temperature profile within the range $t \in [0, t_{\thre}]$ is subsequently used as the thermal load in OpenSeesRT simulations to compute displacements at each structural node at time $t_{\thre}$.
\begin{figure}[h!]
    \centering
    \includegraphics[width=0.8\linewidth]{figures/m204_t1_curve.pdf}
    \caption{Three examples for the $t_1$ curve.}
    \label{fig:t1_curve}
\end{figure}

\revise{
\textit{Remark:} The effects of structural elements, such as concrete floor slabs and partitions, are not explicitly modeled in our approach. Instead, their influence is implicitly captured through the careful selection of the parameters $ \alpha, \beta, r_\mathrm{up} $, and $ r_\mathrm{down} $. This parameterization provides a unified framework for generating temperature fields. Indeed, fire propagation is governed by a multitude of factors and remains an open research question. For instance, if the fire resistance of a floor slab is enhanced by fire protective coating, the corresponding model can account for this by decreasing $\alpha_1$ \& $\alpha_3$, increasing $\beta_1$ \& $\beta_3$, and adopting larger values for $r_\mathrm{up}$ \& $r_\mathrm{down}$, which collectively slow down the vertical spread of fire. Conversely, scenarios involving higher amounts of combustible materials would warrant the opposite adjustments. This flexible and integrated approach avoids the need to design separate models for different fire propagation scenarios while still capturing the essential effects.
}

\revise{
In conclusion, our rule-based approach is a computationally efficient method for approximating fire temperature fields, enabling large-scale dataset generation to train predictive models. By combining ISO 834 fire curves with spatial considerations and embedding structural effects through parameter calibration, the method achieves a balanced trade-off between accuracy and scalability, making it a practical solution for thermal load modeling in fire scenarios. After generating the temperature of each beam or column according to the middle point, the temperature is applied as uniform thermal load to the elements of the structure in question using OpenSeesRT. 
}

% In conclusion, this rule-based approach is a computationally efficient method to approximate fire temperature fields, enabling large-scale dataset generation to train predictive models. By combining ISO 834 fire curves with spatial considerations, the method balances accuracy and scalability, making it a practical solution for thermal load modeling in fire scenarios.

% \subsection{Interstory Drift Ratio}
\subsection{OpenSeesRT Simulation}
\label{subsec:opensees_simulation}

The thermal and mechanical responses of 3D frame structures under combined fire and gravity loads are simulated using OpenSeesRT \cite{perez2024openseesrt}. \revise{In the simulation, the IDR of each node at $t_{\thre}$ is computed using the computed nodal displacements. Each structural model features six degrees of freedom per node (3 translational  and 3 rotational), with linear geometrical transformations (\texttt{geomTransf: Linear}) defining how the element local coordinate systems are mapped to the global coordinate system and assuming small displacements and rotations. Although OpenSeesRT allows a variety of options for modeling finite deformations, in the present simulations and mainly for simplicity, we did not consider large deformations. All bottom nodes (nodes on the ground) are fully constrained in all six degrees of freedom, while degrees of freedom os all other nodes are free.} Material behavior is temperature-dependent and modeled with \texttt{Steel01Thermal}, while fiber-based sections (\texttt{FiberThermal}) capture nonlinear interactions between thermal and mechanical responses at the cross-section level. \revise{Structural elements are represented as displacement-based Euler-Bernoulli beam-columns (\texttt{dispBeamColumnThermal}). This element  formulation accounts for thermal strains (temperature gradients) in the section, which is discretized into fibers. Numerical integration is used along the length of each element using three integration (Gauss) points, one at each end and the third in the middle of the element.}

{\revise{Thermal expansion of steel members plays a crucial role in IDR development. In reality, reinforced concrete floor slabs heat at a different rate than steel members due to their higher thermal mass and lower thermal conductivity. This differential heating can lead to restrained thermal expansion, introducing axial compression in beams and affecting the overall structural response. In this study, explicit {\em{composite action}} between steel members and concrete slabs is not modeled. Instead, our approach focuses on isolating the response of the steel structural frame, which is often the critical load-bearing component in fire scenarios. This assumption aligns with prior studies \cite{Possidente_2024} demonstrating that steel structures reach thermal equilibrium with surrounding gases quickly, allowing the use of uniform thermal loading in fire analysis. Future work could enhance this framework by incorporating slab-beam interaction effects, through a refined FEA for an extended dataset where constraints imposed by floor slabs are explicitly considered.}

The analysis begins with the application of gravity loads, followed by incremental thermal loads simulating the fire exposure. A static nonlinear solver using  \texttt{ExpressNewton} algorithm ensures convergence, while the \texttt{NormDispIncr} test maintains accuracy. An incremental \texttt{LoadControl} scheme with small step sizes is employed to guarantee numerical stability, using 10\% for gravity loads and 1\% for thermal loads. 

\revise{
In the thermal load analysis, uniform thermal load is applied to each beam or column, i.e., the temperature of each element is set to be that at the middle point, according to \secref{subsec:thermal_load_generation}. The \texttt{Steel01Thermal} material allows the properties (e.g., Young's modulus and yield strength) to be adjusted at increasing temperatures according to \cite{EN1993} using its Table 3.1: Reduction factors for the stress-strain relationship of carbon steel at elevated temperatures. For example, if the Young’s modulus at ambient temperature is $E_0$, then as the temperature ($T$) increases, the modulus changes as $E(T) = \eta (T) \times E_0$. \cite{EN1993} directly provides the values of $\eta(T) \in \left[0,1\right] $ at every $100 ^\circ\mathrm{C}$ interval and recommends using linear interpolation to obtain $\eta(T)$ for intermediate values of $T$.
} OpenSeesRT documentation \cite{OpenSeesThermalExamples} provides several examples of thermal analyses.

This modeling framework accommodates variations in material properties, cross-sectional geometries, and temperature profiles, providing robust simulations of structural behavior under fire conditions. The primary settings and configurations for the OpenSeesRT simulations are summarized in \tabref{tab:ops_detail}.
\begin{table}[h!]
    \centering
        \caption{Key settings of OpenSeesRT simulations.}
    \begin{tabular}{l|>{\raggedright\arraybackslash}p{0.6\linewidth}} %
    \toprule
    Modeling Aspect     & Details \\
    \midrule
    Geometry            & 3D models; 6 degrees of freedom per node \\
    Transformation      & geomTransf: Linear \\ 
    Material            & Steel01Thermal \\
    Section             & FiberThermal; Cross-section: $0.1$ m $\times$ $0.1$ m \\ 
    Element type        & {dispBeamColumnThermal} \\ 
    Loading             & Gravity loads: {beamUniform}; Thermal loads: {beamThermal} \\
    Integration scheme  & Incremental {LoadControl}; Step size: $10\%$ (gravity analysis), $1\%$ (thermal analysis) \\
    Nonlinear solver    & {ExpressNewton} algorithm; {UmfPack} solver; Convergence test: {NormDispIncr} tolerance: $10^{-8}$; Maximum \# iterations per step: $1000$. \\ 
    \bottomrule
    \end{tabular}
    \label{tab:ops_detail}
\end{table}

For each structure in the labeled dataset, 30 fire points are selected using a dual-granularity approach, \revise{i.e., two-stage sampling strategy,} to ensure they are well-distributed. Specifically, rooms are sequentially selected, with one fire point randomly chosen within each selected room. If a building is large and contains more than 30 rooms, we randomly select 30 rooms without replacement, i.e., ensuring that no more than one fire point is located in the same room. Conversely, if the building is small and has fewer than 30 rooms, all rooms are initially selected, with one fire point randomly assigned to each room. Additionally, rooms are then selected with replacement until a total of 30 fire points are assigned. \revise{The room-level sampling prioritizes selecting distinct rooms to avoid spatial clustering of fire points, while the point-level sampling ensures intra-room variability. This approach aligns with stratified sampling principles commonly used for efficient spatial representation, where multi-stage sampling strategies optimize coverage and variability, e.g., \cite{arunachalam_generalized_2023}, and enables a more comprehensive characterizing of how the structures respond under fire conditions.}
% This selection method prevents fire points from clustering too closely while maintaining an element of randomness. By distributing fire points in this manner, the 30 fire scenarios are effectively utilized, enabling a more comprehensive characterizing of how the structures respond under fire conditions.

\subsection{Summary of the Dataset Generation}
As discussed in this section and related to  \figref{fig:dataset_generation_procedure}, three key steps were considered in the development of the dataset: 
\begin{enumerate}
    \item {\bf{Filtering process}}: Structures with MIDR exceeding $1\%$ under gravity loads were excluded,  resulting in $1,573$ labeled structures retained for fire simulation and $16,050$ unlabeled structures for training the MFSP predictor.
    \item {\bf{Fire simulations}}: For each retained labeled structure, 30 fire scenarios were simulated using OpenSeesRT, yielding $47,190$ fire cases.
    \item {\bf{Data distribution check}}: MIDR distributions for labeled and unlabeled data under gravity loads were highly similar, because both datasets were generated using the same method. Under fire conditions, the MIDR distribution shifted, reflecting significant structural deformation with values reaching a maximum of about 6\%, an average of 1.70\%, and a standard deviation of 1.12\%. This step ensured a diverse and comprehensive dataset for the proposed predictive framework.
\end{enumerate}
The statistical distribution histograms for MIDR (after applying the $1\%$ filtering threshold \revise{for gravity load responses}) under different loading conditions are plotted in \figref{fig:histogram_mdr}. Figures \ref{fig:histogram_mdr}(a) and \ref{fig:histogram_mdr}(b) show the MIDR distributions of the labeled and unlabeled data, respectively, under gravity loads only. \figref{fig:histogram_mdr}(c) shows the MIDR distribution of the labeled data under the combined effects of gravity and fire loads. Fire load causes the structures to significantly deform, leading to a noticeably \revise{right-skewed} MIDR distribution.

\begin{figure*}[h!]
    \centering
    \includegraphics[width=\linewidth]{figures/histogram_mdr.pdf}
    \caption{Histograms of MIDR for labeled and unlabeled structures with gravity loads and fire cases.}
    \label{fig:histogram_mdr}
\end{figure*}

\revise{
This dataset provides the basis for training and testing the performance of the GNN-based framework. Although we employed a simplified rule-based thermal load generation method compared with conventional CFD-based simulations, the temperature field, the changes of the material properties, and the response of the structures, are all still highly nonlinear and complex. Therefore, it is still a challenging task for the NN to predict the MIDRs based on this dataset.
}

\section{\thename}
\subsection{End-to-End Driving Policy}
The overall framework of \thename{} is depicted in Fig.~\ref{fig:framework}. 
\thename{} takes multi-view image sequences as input, transforms the sensor data into scene token embeddings, outputs the probabilistic distribution of actions, and samples an action to control the vehicle. 

\boldparagraph{BEV Encoder.} 
We first employ a BEV encoder~\cite{li2022bevformer} to transform multi-view image features from the perspective view to the Bird's Eye View (BEV), obtaining a feature map in the BEV space. This feature map is then used to learn instance-level map features and agent features.

\boldparagraph{Map Head.} 
Then we utilize a group of map tokens~\cite{maptrv2, liao2022maptr, lanegap} to learn the vectorized map elements of the driving scene from the BEV feature map, including lane centerlines, lane dividers, road boundaries, arrows, traffic signals, \etc.

\boldparagraph{Agent Head.} 
Besides, a group of agent tokens~\cite{jiang2022pip} is adopted to predict the motion information of other traffic participants, including location, orientation, size, speed, and multi-mode future trajectories.

\boldparagraph{Image Encoder.} 
Apart from the above instance-level map and agent tokens, we also use an individual image encoder~\cite{vit,he2016resnet} to transform the original images into image tokens. These image tokens provide dense and rich scene information for planning, complementary to the instance-level tokens.

\begin{figure}[t]
\centering
\includegraphics[width=0.98\linewidth]{fig/post-training-2.pdf} 
\caption{\textbf{Post-training.}  $N$  workers parallelly run. The generated rollout data $(s_t,a_t, r_{t+1},s_{t+1},...)$ are recorded in a rollout buffer. Rollout data and human driving demonstrations are used in RL- and IL-training steps to fine-tune the AD policy synergistically.
}
\label{fig:post-training}
\end{figure}

\boldparagraph{Action Space.} 
To accelerate the convergence of RL training, we design a decoupled discrete action representation. 
We divide the action into two independent components: lateral action and longitudinal action. 
The action space is constructed over a short $0.5$-second time horizon, during which the vehicle's motion is approximated by assuming constant linear and angular velocities. 
Under this assumption, the lateral action $a^x$ and longitudinal action $a^y$ can be directly computed based on the current linear and angular velocities.
By combining decoupling with a limited temporal scope and simplified motion model, our approach effectively reduces the dimensionality of the action space, accelerating training convergence.


\boldparagraph{Planning Head.} 
We use $E_\text{scene}$ to denote the scene representation, which consists of map tokens, agent tokens, and image tokens. We initialize a planning embedding denoted as $E_\text{plan}$. A cascaded Transformer decoder $\phi$ takes the planning embedding $E_\text{plan}$ as the query and the scene representation $E_\text{scene}$ as both key and value.

The output of the decoder $\phi$ is then combined with navigation information $E_\text{navi}$ and ego state $E_\text{state}$ to output the probabilistic distributions of the lateral action $a^x$ and the longitudinal action $a^y$:
\begin{equation}
\begin{aligned}
     \pi(a^x\mid s) = & \text{softmax}(\text{MLP}(\phi(E_\text{plan}, E_\text{scene}) \\
    & + E_\text{navi} + E_\text{state})), \\
     \pi(a^y\mid s) = & \text{softmax}(\text{MLP}(\phi(E_\text{plan}, E_\text{scene}) \\
     & + E_\text{navi} + E_\text{state})),
\label{eq:action distribution}
\end{aligned}
\end{equation}
where $E_\text{plan}$, $E_\text{navi}$, $E_\text{state}$, and the output of $\text{MLP}$ are all of the same dimension ($1 \times D$).

The planning head also outputs the value functions $V_x(s)$ and $V_y(s)$, which estimate the expected cumulative rewards for the lateral and longitudinal actions, respectively: 
\begin{equation}
\begin{aligned}
    & V_x(s) = \text{MLP}(\phi(E_\text{plan}, E_\text{scene}) + E_\text{navi} + E_\text{state}), \\
    & V_y(s) = \text{MLP}(\phi(E_\text{plan}, E_\text{scene}) + E_\text{navi} + E_\text{state}).
\end{aligned}
\end{equation}
The value functions are used in RL training (Sec.~\ref{sec:optimization}).

\subsection{Training Paradigm}
We adopt a three-stage training paradigm: perception pre-training, planning pre-training, and reinforced post-training, as shown in Fig.~\ref{fig:framework}.

\boldparagraph{Perception Pre-Training.} 
Information in the image is sparse and low-level. In the first stage,  
the map head and the agent head explicitly output map elements and agent motion information, which are supervised with ground-truth labels. Consequently,  
map tokens and agent tokens implicitly encode the corresponding high-level information.  
In this stage, we only update the parameters of the BEV encoder, the map head, and the agent head.



\boldparagraph{Planning Pre-Training.} 
In the second stage, to prevent the unstable cold start of RL training, IL is first performed to initialize the probabilistic distribution of actions based on large-scale real-world driving demonstrations from expert drivers. In this stage, we only update the parameters of the image encoder and the planning head, while the parameters of the BEV encoder, map head, and agent head are frozen. The optimization objectives of perception tasks and planning tasks may conflict with each other. However, with the training stage and parameters decoupled, such conflicts are mostly avoided.

\boldparagraph{Reinforced Post-Training.} 
In the reinforced post-training, RL and IL synergistically fine-tune the distribution. RL aims to guide the policy to be sensitive to critical risky events and adaptive to out-of-distribution situations. IL serves as the regularization term to keep the policy's behavior similar to that of humans.

We select a large amount of risky dense-traffic clips from collected driving demonstrations. For each clip, we train an independent 3DGS model that reconstructs the clip and serves as a digital driving environment.  
As shown in Fig.~\ref{fig:post-training}, we set $N$ parallel workers.  
Each worker randomly samples a 3DGS environment and begins rollout, i.e., the AD policy controls the ego vehicle to move and iteratively interacts with the 3DGS environment. After the rollout process of this 3DGS environment ends, the generated rollout data $(s_t,a_t, r_{t+1},s_{t+1},...)$ are recorded in a rollout buffer, and the worker will sample a new 3DGS environment for another round of rollout.

As for policy optimization, we iteratively perform RL-training steps and IL-training steps. For RL-training steps, we sample data from the rollout buffer and follow the Proximal Policy Optimization (PPO) framework~\cite{PPO} to update the AD policy. For IL-training steps, we use real-world driving demonstrations to update the policy. After a fixed number of training steps, the updated AD policy is sent to every worker to replace the old one, to avoid a distribution shift between data collection and optimization.
We only update the parameters of the image encoder and the planning head. The parameters of the BEV encoder, the map head, and the agent head are frozen.  
The detailed RL design is presented below.

\subsection{Interaction Mechanism between AD Policy and 3DGS Environment}
In the 3DGS environment, the ego vehicle acts according to the AD policy. Other traffic participants act according to real-world data in a log-replay manner.  
A simplified kinematic bicycle model is employed to iteratively update the ego vehicle's pose at every $\Delta t$ seconds as follows:  
\begin{equation}
\begin{aligned}
x_{t+1}^{w} & = x_{t}^w + v_t \cos \left(\psi_{t}^w\right) \Delta t, \\
y_{t+1}^{w} & = y_{t}^w + v_t \sin \left(\psi_{t}^w\right) \Delta t, \\
\psi_{t+1}^{w} & = \psi_{t}^w + \frac{v_t}{L} \tan \left(\delta_t\right) \Delta t,
\label{equation:kinematic_model}
\end{aligned}
\end{equation}  
where $x_t^{w}$ and $y_t^{w}$ denote the position of the ego vehicle relative to the world coordinate; $\psi_t^w$ is the heading angle that defines the vehicle's orientation with respect to the world $x$-coordinate; $v_t$ is the linear velocity of the ego vehicle; $\delta_t$ is the steering angle of the front wheels; and $L$ is the wheelbase, i.e., the distance between the front and rear axles.

During the rollout process, the AD policy outputs actions $(a_t^x, a_t^y)$ for a $0.5$-second time horizon at time step $t$. We derive the linear velocity $v_t$ and steering angle $\delta_t$ based on $(a_t^x, a_t^y)$.  
Based on the kinematic model in Eq.~\ref{equation:kinematic_model},  
the pose of the ego vehicle in the world coordinate system is updated from ${p}_t = (x_{t}^w, y_{t}^w, \psi_{t}^w)$ to ${p}_{t+1} = (x_{t+1}^{w}, y_{t+1}^{w}, \psi_{t+1}^{w})$.  

Based on the updated ${p}_{t+1}$, the 3DGS environment computes the new ego vehicle's state $s_{t+1}$. The updated pose ${p}_{t+1}$ and state $s_{t+1}$ serve as the input for the next iteration of the inference process.

The 3DGS environment also generates rewards $\mathcal{R}$ (Sec.~\ref{sec:reward}) according to multi-source information (including trajectories of other agents, map information, the expert trajectory of the ego vehicle, and the parameters of Gaussians), which are used to optimize the AD policy (Sec.~\ref{sec:optimization}).

\begin{figure}[t]
\centering
\includegraphics[width=1.0\linewidth]{fig/reward.pdf} 
\caption{\textbf{Example diagram of four types of reward sources.}  (1): Collision with a dynamic obstacle ahead triggers a reward $r_{\text{dc}}$. (2): Hitting a static roadside obstacle incurs a reward $r_{\text{sc}}$. (3): Moving onto the curb exceeds the positional deviation threshold $d_{\text{max}}$, triggering a reward $r_{\text{pd}}$. (4): Drifting toward the adjacent lane exceeds the heading deviation threshold $\psi_{\text{max}}$, triggering a reward $r_{\text{hd}}$.
}
\label{fig: reward source}
\end{figure}
\subsection{Reward Modeling}
\label{sec:reward}
The reward is the source of the training signal, which determines the optimization direction of RL. The reward function is designed to guide the ego vehicle's behavior by penalizing unsafe actions and encouraging alignment with the expert trajectory. It is composed of four reward components: (1) collision with dynamic obstacles, (2) collision with static obstacles, (3) positional deviation from the expert trajectory, and (4) heading deviation from the expert trajectory:
\begin{equation}
\begin{aligned}
\mathcal{R} = \{r_{\text{dc}}, r_{\text{sc}}, r_{\text{pd}}, r_{\text{hd}}  \}. 
\end{aligned}
\end{equation}

As illustrated in Fig.~\ref{fig: reward source}, these reward components are triggered under specific conditions.  
In the 3DGS environment, dynamic collision is detected if the ego vehicle's bounding box overlaps with the annotated bounding boxes of dynamic obstacles, triggering a negative reward $r_{\text{dc}}$. Similarly, static collision is identified when the ego vehicle's bounding box overlaps with the Gaussians of static obstacles, resulting in a negative reward $r_{\text{sc}}$.  
Positional deviation is measured as the Euclidean distance between the ego vehicle's current position and the closest point on the expert trajectory. A deviation beyond a predefined threshold $d_{\text{max}}$ incurs a negative reward $r_{\text{pd}}$.  
Heading deviation is calculated as the angular difference between the ego vehicle's current heading angle $ \psi_t $ and the expert trajectory's matched heading angle $\psi_{\text{expert}}$. A deviation beyond a threshold $ \psi_{\text{max}}$ results in a negative reward $r_{\text{hd}}$.

Any of these events, including dynamic collision, static collision, excessive positional deviation, or excessive heading deviation, triggers immediate episode termination. Because after such events occur, the 3DGS environment typically generates noisy sensor data, which is detrimental to RL training.

\subsection{Policy Optimization}
\label{sec:optimization}
In the closed-loop environment, the error in each single step accumulates over time. The aforementioned rewards are not only caused by the current action but also by the actions of the preceding steps.  
The rewards are propagated forward with Generalized Advantage Estimation (GAE)~\cite{gae} to optimize the action distribution of the preceding steps.

Specifically, for each time step $t$, we store the current state $s_t$, action $a_t$, reward $r_t$, and the estimate of the value $V(s_t)$.  
Based on the decoupled action space, and considering that different rewards have different correlations to lateral and longitudinal actions, the reward $r_t$ is divided into lateral reward $r_t^x$ and longitudinal reward $r_t^y$:
\begin{equation}
\begin{aligned}
r_t^x &= r_t^{\text{sc}} + r_t^{\text{pd}} + r_t^{\text{hd}}, \\
r_t^y &= r_t^{\text{dc}}.
\label{eq:reward-decouple}
\end{aligned}
\end{equation}
Similarly, the value function $V(s_t)$ is decoupled into two components: $V_x(s_t)$ for the lateral dimension and $V_y(s_t)$ for the longitudinal dimension. These value functions estimate the expected cumulative rewards for the lateral and longitudinal actions, respectively. The advantage estimates $\hat{A}_t^x$ and $\hat{A}_t^y$ are then computed as follows:
\begin{equation}
\begin{aligned}
\delta_t^x &= r_t^x + \gamma V_x(s_{t+1}) - V_x(s_t), \\
\delta_t^y &= r_t^y + \gamma V_y(s_{t+1}) - V_y(s_t), \\
\hat{A}_t^x &= \sum_{l=0}^{\infty}(\gamma \lambda)^l \delta_{t+l}^x, \\
\hat{A}_t^y &= \sum_{l=0}^{\infty}(\gamma \lambda)^l \delta_{t+l}^y,
\label{eq:advantage}
\end{aligned}
\end{equation}
where $\delta_t^x$ and $\delta_t^y$ are the temporal difference errors for the lateral and longitudinal dimensions, $\gamma$ is the discount factor, and $\lambda$ is the GAE parameter that controls the trade-off between bias and variance.

To further clarify the relationship between the advantage estimates and the reward components, we decompose $\hat{A}_t^x$ and $\hat{A}_t^y$ based on the reward decomposition in Eq.~\ref{eq:reward-decouple} and the advantage estimation in Eq.~\ref{eq:advantage}. Specifically, we derive the following decomposition:
\begin{equation}
\begin{aligned}
\hat{A}_t^x &= \hat{A}_t^{\text{sc}} + \hat{A}_t^{\text{pd}} + \hat{A}_t^{\text{hd}}, \\
\hat{A}_t^y &= \hat{A}_t^{\text{dc}},
\end{aligned}
\end{equation}
where $\hat{A}_t^{\text{sc}}$ is the advantage estimate for avoiding static collisions, $\hat{A}_t^{\text{pd}}$ is the advantage estimate for minimizing positional deviations, $\hat{A}_t^{\text{hd}}$ is the advantage estimate for minimizing heading deviations, and $\hat{A}_t^{\text{dc}}$ is the advantage estimate for avoiding dynamic collisions.

These advantage estimates are used to guide the update of the AD policy $\pi_{\theta}$, following the PPO framework~\cite{PPO}. By leveraging the decomposed advantage estimates $\hat{A}_t^x$ and $\hat{A}_t^y$, we can independently optimize the lateral and longitudinal dimensions of the policy. This is achieved by defining separate objective functions $\mathcal{L}_x^{\text{CLIP}}(\theta)$ and $\mathcal{L}_y^{\text{CLIP}}(\theta)$ for each dimension,  as follows:
\begin{equation}
\begin{aligned}
\mathcal{L}_x^{\text{PPO}}(\theta) &= \mathbb{E}_t \left[ \min \left( \rho_t^x \hat{A}_t^x, \ \text{clip}(\rho_t^x, 1-\epsilon_x, 1+\epsilon_x) \hat{A}_t^x \right) \right], \\
\mathcal{L}_y^{\text{PPO}}(\theta) &= \mathbb{E}_t \left[ \min \left( \rho_t^y \hat{A}_t^y, \ \text{clip}(\rho_t^y, 1-\epsilon_y, 1+\epsilon_y) \hat{A}_t^y \right) \right], \\
\mathcal{L}^{\text{PPO}}(\theta) &= \mathcal{L}_x^{\text{PPO}}(\theta) + \mathcal{L}_y^{\text{PPO}}(\theta),
\end{aligned}
\end{equation}
where $\rho_t^x = \frac{\pi_{\theta}(a_t^x \mid s_t)}{\pi_{\theta_{\text{old}}}(a_t^x \mid s_t)}$ is the importance sampling ratio for the lateral dimension, $\rho_t^y = \frac{\pi_{\theta}(a_t^y \mid s_t)}{\pi_{\theta_{\text{old}}}(a_t^y \mid s_t)}$ is the importance sampling ratio for the longitudinal dimension, $\epsilon_x$ and $\epsilon_y$ are small constants that control the clipping range for the lateral and longitudinal dimensions, ensuring stable policy updates.

The clipped objective function $\mathcal{L}^{\text{PPO}}(\theta)$ prevents excessively large updates to the policy parameters $\theta$, thereby maintaining training stability.

\begin{table*}[ht]
    \centering
{
\begin{tabular}{lccccccccc}
    \toprule
    RL:IL & CR$\downarrow$ & DCR$\downarrow$ & SCR$\downarrow$ & DR$\downarrow$ & PDR$\downarrow$ & HDR$\downarrow$ &ADD$\downarrow$ & Long. Jerk$\downarrow$ & Lat. Jerk$\downarrow$ \\
    \midrule
     0:1  & 0.229 & 0.211 & 0.018 & 0.066 & 0.039 & 0.027  & 0.238 & 3.928 & 0.103\\
     1:0  & 0.143 & 0.128 & 0.015 &0.080 &0.065 &0.015 &0.345 &4.204 &0.085\\
     2:1 & 0.137 & 0.125 & 0.012 & 0.059 & 0.050 & 0.009  & 0.274 & 4.538 & 0.092\\
     4:1 & 0.089 & 0.080 & 0.009 & 0.063 & 0.042 & 0.021  & 0.257 & 4.495 & 0.082 \\
     8:1 & 0.125 & 0.116 & 0.009 & 0.084 & 0.045 & 0.039  & 0.323 & 5.285 & 0.115\\
    \bottomrule
\end{tabular}
}
    \caption{\textbf{Ablation on RL-to-IL step mixing ratios in the reinforced post-training stage.}}
    \label{tab:ratio}
\end{table*}

\subsection{Auxiliary Objective}
RL usually faces the challenge of sparse rewards, which makes the convergence process unstable and slow. To speed up convergence, we introduce auxiliary objectives that provide dense guidance to the entire action distribution.

The auxiliary objectives are designed to penalize undesirable behaviors by incorporating specific reward sources, including dynamic collisions, static collisions, positional deviations, and heading deviations. These objectives are computed based on the actions \( a_t^{x, \text{old}} \) and \( a_t^{y, \text{old}} \) selected by the old AD policy \( \pi_{\theta_{\text{old}}} \) at time step \( t \). To facilitate the evaluation of these actions, we separate the probability distribution of the action into four parts:
\begin{equation}
\begin{aligned}
\Delta \pi_y^{\text{dec}} &= \sum_{a_t^y < a_t^{y, \text{old}}} \pi_\theta(a_t^y \mid s_t), \\
\Delta \pi_y^{\text{acc}} &= \sum_{a_t^y > a_t^{y, \text{old}}} \pi_\theta(a_t^y \mid s_t), \\
\Delta \pi_x^{\text{left}} &= \sum_{a_t^x < a_t^{x, \text{old}}} \pi_\theta(a_t^x \mid s_t), \\
\Delta \pi_x^{\text{right}} &= \sum_{a_t^x > a_t^{x, \text{old}}} \pi_\theta(a_t^x \mid s_t).
\end{aligned}
\end{equation}
Here, \( \Delta \pi_y^{\text{dec}} \) represents the total probability of deceleration actions, \( \Delta \pi_y^{\text{acc}} \) represents the total probability of acceleration actions, \( \Delta \pi_x^{\text{left}} \) represents the total probability of leftward steering actions, and \( \Delta \pi_x^{\text{right}} \) represents the total probability of rightward steering actions.

\boldparagraph{Dynamic Collision Auxiliary Objective.}  
The dynamic collision auxiliary objective adjusts the longitudinal control action \(a_t^y\) based on the location of potential collisions relative to the ego vehicle. If a collision is detected ahead, the policy prioritizes deceleration actions (\(a_t^y < a_t^{y, \text{old}}\)); if a collision is detected behind, it encourages acceleration actions (\(a_t^y > a_t^{y, \text{old}}\)). To formalize this behavior, we define a directional factor \(f_\text{dc}\):
\begin{equation}
\begin{aligned}
f_\text{dc} = \begin{cases} 
1 & \text{if the collision is ahead}, \\
-1 & \text{if the collision is behind}.
\end{cases} 
\end{aligned}
\end{equation}

The auxiliary objective for dynamic collision avoidance is defined as:
\begin{equation}
\begin{aligned}
\mathcal{L}_\text{dc}(\theta_y) = \mathbb{E}_t \left[ 
    \hat{A}_t^\text{dc} \cdot f_\text{dc} \cdot (\Delta \pi_y^{\text{dec}} - \Delta \pi_y^{\text{acc}})
\right],
\end{aligned}
\end{equation}
where \(\hat{A}_t^\text{dc}\) is the advantage estimate for dynamic collision avoidance.

\boldparagraph{Static Collision Auxiliary Objective.}  
The static collision auxiliary objective adjusts the steering control action $a_t^x$ based on the proximity to static obstacles. If the static obstacle is detected on the left side, the policy promotes rightward steering actions ($a_t^x > a_t^{x,\text{old}}$); if the static obstacle is detected on the right side, it promotes leftward steering actions ($a_t^x < a_t^{x,\text{old}}$). To formalize this behavior, we define a directional factor $f_\text{sc}$:  
\begin{equation}
\begin{aligned}
f_\text{sc} = \begin{cases} 
1 & \text{if static obstacle is on the left}, \\
-1 & \text{if static obstacle is on the right}.
\end{cases} 
\end{aligned}
\end{equation}

The auxiliary objective for static collision avoidance is defined as:  
\begin{equation}
\begin{aligned}
\mathcal{L}_\text{sc}(\theta_x) = \mathbb{E}_t \left[ 
    \hat{A}_t^\text{sc} \cdot f_\text{sc} \cdot (\Delta \pi_x^{\text{right}} - \Delta \pi_x^{\text{left}})
\right],
\end{aligned}
\end{equation}  
where $\hat{A}_t^\text{sc}$ is the advantage estimate for static collision avoidance.  

\boldparagraph{Positional Deviation Auxiliary Objective.}  
The positional deviation auxiliary objective adjusts the steering control action $a_t^x$ based on the ego vehicle's lateral deviation from the expert trajectory. If the ego vehicle deviates leftward, the policy promotes rightward corrections ($a_t^x > a_t^{x,\text{old}}$); if it deviates rightward, it promotes leftward corrections ($a_t^x < a_t^{x,\text{old}}$). We formalize this with a directional factor $f_\text{pd}$:  
\begin{equation}
\begin{aligned}
f_\text{pd} = \begin{cases} 
1 & \text{if ego vehicle deviates leftward}, \\
-1 & \text{if ego vehicle deviates rightward}.
\end{cases} 
\end{aligned}
\end{equation}

The auxiliary objective for positional deviation correction is:
\begin{equation}
\begin{aligned}
\mathcal{L}_\text{pd}(\theta_x) = \mathbb{E}_t \left[ 
    \hat{A}_t^\text{pd} \cdot f_\text{pd} \cdot (\Delta \pi_x^{\text{right}} - \Delta \pi_x^{\text{left}})
\right],
\end{aligned}
\end{equation}  
where $\hat{A}_t^\text{pd}$ estimates the advantage of trajectory alignment.

\boldparagraph{Heading Deviation Auxiliary Objective.}  
The heading deviation auxiliary objective adjusts the steering control action $a_t^x$ based on the angular difference between the ego vehicle’s current heading and the expert’s reference heading. If the ego vehicle deviates counterclockwise, the policy promotes clockwise corrections ($a_t^x > a_t^{x,\text{old}}$); if it deviates clockwise, it promotes counterclockwise corrections ($a_t^x < a_t^{x,\text{old}}$). To formalize this behavior, we define a directional factor $f_\text{hd}$:  
\begin{equation}
\begin{aligned}
f_\text{hd} = \begin{cases} 
1 & \text{if ego vehicle deviates clockwise}, \\
-1 & \text{if ego vehicle deviates counterclockwise}.
\end{cases} 
\end{aligned}
\end{equation}

The auxiliary objective for heading deviation correction is then defined as:  
\begin{equation}
\begin{aligned}
\mathcal{L}_\text{hd}(\theta_x) = \mathbb{E}_t \left[ 
    \hat{A}_t^\text{hd} \cdot f_\text{hd} \cdot (\Delta \pi_x^{\text{right}} - \Delta \pi_x^{\text{left}})
\right],
\end{aligned}
\end{equation}  
where $\hat{A}_t^\text{hd}$ is the advantage estimate for heading alignment.  

\begin{table*}[ht]
\begin{center}
\centering
\resizebox{0.98\textwidth}{!}{
\begin{tabular}{cccccccccccccc}
\toprule
\multirow{2}{*}{ID} & Dynamic & Static & Position & Heading & \multirow{2}{*}{CR$\downarrow$} &\multirow{2}{*}{DCR$\downarrow$} &\multirow{2}{*}{SCR$\downarrow$} &\multirow{2}{*}{DR$\downarrow$} &\multirow{2}{*}{PDR$\downarrow$} &\multirow{2}{*}{HDR$\downarrow$} &\multirow{2}{*}{ADD$\downarrow$} &\multirow{2}{*}{Long. Jerk$\downarrow$} &\multirow{2}{*}{Lat. Jerk$\downarrow$}\\
& Collision & Collision & Deviation & Deviation & & & & & & & & & \\
\midrule
1 & \cmark  &  &  &  & 0.172 & 0.154 & 0.018 & 0.092 & 0.033 & 0.059  & 0.259 & 4.211 & 0.095 \\
2 &  & \cmark & \cmark & \cmark & 0.238 & 0.217 & 0.021 & 0.090 & 0.045 & 0.045  & 0.241 & 3.937 & 0.098 \\
3 & \cmark &  & \cmark & \cmark & 0.146 & 0.128 & 0.018 & 0.060 & 0.030 & 0.030  & 0.263 & 3.729 & 0.083\\
4 & \cmark & \cmark &  & \cmark & 0.151 & 0.142 & 0.009 & 0.069 & 0.042 & 0.027 & 0.303 & 3.938 & 0.079\\
5 & \cmark & \cmark & \cmark &  & 0.166 & 0.157 & 0.009 & 0.048 & 0.036 & 0.012 & 0.243 & 3.334 & 0.067\\
6 & \cmark & \cmark & \cmark & \cmark & 0.089 & 0.080 & 0.009 & 0.063 & 0.042 & 0.021 & 0.257 & 4.495 & 0.082 \\
\bottomrule
\end{tabular}
}
\end{center}
\vspace{-2mm}
\caption{\textbf{Ablation on reward sources.} The table shows the impact of different reward components on performance.}
\label{tab:reward_ablation}
\end{table*}

\begin{table*}[ht]
\begin{center}
\centering
\resizebox{0.98\textwidth}{!}{
\begin{tabular}{ccccccccccccccc}
\toprule
\multirow{2}{*}{ID} & \multirow{2}{*}{PPO Obj.}  & Dynamic Col. & Static Col. & Position Dev. & Heading Dev. & \multirow{2}{*}{CR$\downarrow$} & \multirow{2}{*}{DCR$\downarrow$}  & \multirow{2}{*}{SCR$\downarrow$} & \multirow{2}{*}{DR$\downarrow$} & \multirow{2}{*}{PDR$\downarrow$} & \multirow{2}{*}{HDR$\downarrow$} & \multirow{2}{*}{ADD$\downarrow$} & \multirow{2}{*}{Long. Jerk$\downarrow$} & \multirow{2}{*}{Lat. Jerk$\downarrow$} \\
& & Auxiliary Obj. & Auxiliary Obj. & Auxiliary Obj. & Auxiliary Obj. & & & & & & & & & \\
\midrule
1 &\cmark&  &  &  &  & 0.249 & 0.223 & 0.026 & 0.077 & 0.047 & 0.030  & 0.266 & 4.209 & 0.104 \\
2 &\cmark& \cmark &  &  &  & 0.178 & 0.163 & 0.015 & 0.151 & 0.101 & 0.050 & 0.301 & 3.906 & 0.085 \\
3 &\cmark&  & \cmark & \cmark & \cmark & 0.137 & 0.125 & 0.012 & 0.157 & 0.145 & 0.012 & 0.296 & 3.419 & 0.071 \\
4 &\cmark& \cmark &  & \cmark & \cmark & 0.169 & 0.151 & 0.018 & 0.075 & 0.042 & 0.033 & 0.254 & 4.450 & 0.098 \\
5 &\cmark& \cmark & \cmark &  & \cmark & 0.149 & 0.134 & 0.015 & 0.063 & 0.057 & 0.006 & 0.324 & 3.980 & 0.086 \\
6 &\cmark& \cmark & \cmark & \cmark & & 0.128 & 0.119  & 0.009 & 0.066 & 0.030 & 0.036  & 0.254 & 4.102 & 0.092 \\
7 &&\cmark  &\cmark  &\cmark  &\cmark  & 0.187 &0.175  &0.012 &0.077 &0.056  &0.021  &0.309  &5.014  &0.112  \\
8 &\cmark& \cmark & \cmark & \cmark & \cmark & 0.089 & 0.080 & 0.009 & 0.063 & 0.042 & 0.021  & 0.257 & 4.495 & 0.082 \\
\bottomrule
\end{tabular}
}
\end{center}
\vspace{-2mm}
\caption{\textbf{Ablation on auxiliary objectives.} The table shows the impact of different auxiliary objectives on performance.}
\label{tab:auxiliary_ablation}
\end{table*}

\boldparagraph{Overall Auxiliary Objectives.}  
The overall auxiliary objectives are a weighted sum of the individual objectives:
\begin{equation}
\begin{aligned}
\mathcal{L}_\text{aux}(\theta) = &\lambda_1 \mathcal{L}_\text{dc}(\theta_y) + \lambda_2 \mathcal{L}_\text{sc}(\theta_x)  + \\ 
&\lambda_3 \mathcal{L}_\text{pd}(\theta_x) +\lambda_4 \mathcal{L}_\text{hd}(\theta_x),
\end{aligned}
\end{equation}
where $\lambda_1$, $\lambda_2$, $\lambda_3$, and $\lambda_4$ are weighting coefficients that balance the contributions of each auxiliary objective.

\boldparagraph{Optimization Objective.}  
The final optimization objective combines the clipped PPO objective with the auxiliary objective:
\begin{equation}
\mathcal{L}(\theta) = \mathcal{L}^{\text{PPO}}(\theta) + \mathcal{L}_\text{aux}(\theta).
\end{equation}

\begin{table*}[]
\small 
\centering
\begin{tabular}{@{}ll@{}}
\toprule
 & Rubric Text \\ \midrule
Criteria & Interest Level: How engaging and thought-provoking is the summary? \\
Score 1 & Not engaging at all; no attempt to capture the reader’s attention. \\
Score 2 & Fairly engaging with a basic narrative but lacking depth. \\
Score 3 & Moderately engaging with several interesting points. \\
Score 4 & Quite engaging with a well-structured narrative and noteworthy points that frequently capture and retain attention \\
Score 5 & Exceptionally engaging throughout, with a compelling narrative that consistently stimulates interest. \\ \midrule
Criteria & Coherence and Organization: Is the summary well-organized and logically structured? \\
Score 1 & Disorganized; lacks logical structure and coherence. \\
Score 2 & Fairly organized; a basic structure is present but not consistently followed. \\
Score 3 & Organized; a clear structure is mostly followed with some lapses in coherence. \\
Score 4 & Good organization; a clear structure with minor lapses in coherence. \\
Score 5 & Excellently organized; the summary is logically structured with seamless transitions and a clear argument. \\ \midrule
Criteria & Relevance and Focus: Does the summary stay on topic to the query and maintain a clear focus? \\
Score 1 & Off-topic; the content does not align with the query. \\
Score 2 & Somewhat on topic but with several digressions; the answer to the query is evident but not consistently adhered to. \\
Score 3 & Generally on topic, despite a few unrelated details. \\
Score 4 & Mostly on topic and focused; the narrative has a consistent relevance to the query with infrequent digressions. \\
Score 5 & \specialcellleft{Exceptionally focused and entirely on topic; the article is tightly centered on the query,\\with every piece of information contributing to a comprehensive understanding of the query.} \\ \midrule
Criteria & Broad Coverage: Does the article provide an in-depth exploration of the query and have good coverage? \\
Score 1 & Severely lacking; offers little to no coverage of the query's primary aspects, resulting in a very narrow perspective. \\
Score 2 & Partial coverage; includes some of the query's main aspects but misses others, resulting in an incomplete portrayal. \\
Score 3 & \specialcellleft{Acceptable breadth; covers most main aspects, though it may stray into minor unnecessary details\\ or overlook some relevant points.} \\
Score 4 & \specialcellleft{Good coverage; achieves broad coverage of the query,\\hitting on all major points with minimal extraneous information.} \\
Score 5 & \specialcellleft{Exemplary in breadth; delivers outstanding coverage,\\thoroughly detailing all crucial aspects of the query without including irrelevant information.} \\ \midrule
Criteria & \specialcellleft{Diversity of Perspectives: Does the summary adequately describe\\why the answer to the query could be yes and why it could be no?} \\
Score 1 & No diversity; the summary presents only one perspective without addressing the opposing viewpoint. \\
Score 2 & Limited diversity; the summary acknowledges both perspectives but lacks depth in the explanation of one side. \\
Score 3 & Moderate diversity; the summary covers both perspectives, but one side is more thoroughly explored than the other. \\
Score 4 & Good diversity; the summary fairly represents both perspectives with balanced and detailed explanations. \\
Score 5 & \specialcellleft{Excellent diversity; the summary provides a comprehensive and balanced exploration of both perspectives,\\offering in-depth explanations for why the answer could be yes and why it could be no.} \\ \bottomrule
\end{tabular}
\caption{\label{table:rubric} Rubrics for Interest, Coherence, Relevance, Coverage, and Diversity for DQFS summaries. Rubrics are adapted for topic paragraphs and topics (e.g. ``Relevance'' becomes relevance to the topic in topic paragraph evaluation, rather than relevance to the query).}
\end{table*}
\begin{table*}[!h]
\footnotesize
\centering
\setlength{\tabcolsep}{3.5pt}
%\setlength{\extrarowheight}{2pt}
% \setlength{\aboverulesep}{1pt}
% \setlength{\belowrulesep}{1pt}
\renewcommand{\arraystretch}{0.8}
\begin{tabular}{@{}clcccccccc@{}}
\multicolumn{1}{l}{} &  & \multicolumn{3}{c}{\textit{Summary Level}} & \multicolumn{3}{c}{\textit{Topic Paragraph Level}} & \multicolumn{2}{c}{\textit{Confounders}} \\ \midrule
\textbf{\# Pts} & \multicolumn{1}{l|}{\textbf{Model}} & \textbf{DC ($\uparrow$)} & \textbf{Fair ($\downarrow$)} & \multicolumn{1}{c|}{\textbf{Faithful ($\downarrow$)}} & \textbf{DC ($\uparrow$)} & \textbf{Fair ($\downarrow$)} & \multicolumn{1}{c|}{\textbf{Faithful ($\downarrow$)}} & \multicolumn{1}{l}{\textbf{Cite Acc.}} & \textbf{All / Avg Sents} \\ \midrule
\multirow{10}{*}{2} & \multicolumn{1}{l|}{\textbf{\modelAll (\textbf{Ours})}} & {\ul 0.811*} & 0.113* & \multicolumn{1}{c|}{{\ul 0.046*}} & {\ul 0.578*} & {\ul 0.171*} & \multicolumn{1}{c|}{{\ul 0.106*}} & 0.988 & 5.99 / 3.00 \\
 & \multicolumn{1}{l|}{\textbf{\modelTopic (\textbf{Ours})}} & \textbf{0.821*} & \textbf{0.108*} & \multicolumn{1}{c|}{\textbf{0.043*}} & \textbf{0.623} & \textbf{0.153*} & \multicolumn{1}{c|}{\textbf{0.090*}} & 0.985 & 6.01 / 3.01 \\
 & \multicolumn{1}{l|}{Long-Context} & 0.447 & 0.242 & \multicolumn{1}{c|}{0.198} & 0.277 & 0.369 & \multicolumn{1}{c|}{0.326} & 0.950 & 5.99 / 3.00 \\
 & \multicolumn{1}{l|}{RAG-\textit{All}} & 0.603 & 0.166 & \multicolumn{1}{c|}{0.098} & 0.378 & 0.285 & \multicolumn{1}{c|}{0.219} & 0.992 & 6.00 / 3.00 \\
 & \multicolumn{1}{l|}{RAG-\textit{Doc}} & 0.668 & 0.148 & \multicolumn{1}{c|}{0.078} & 0.415 & 0.273 & \multicolumn{1}{c|}{0.204} & 0.970 & 6.02 / 3.01 \\
 & \multicolumn{1}{l|}{Hierarchical} & 0.765 & {\ul 0.111*} & \multicolumn{1}{c|}{0.048*} & 0.454 & 0.265 & \multicolumn{1}{c|}{0.204} & 0.985 & 6.00 / 3.00 \\
 & \multicolumn{1}{l|}{Incremental-\textit{All}} & 0.464 & 0.249 & \multicolumn{1}{c|}{0.202} & 0.357 & 0.289 & \multicolumn{1}{c|}{0.244} & 0.971 & 5.99 / 3.00 \\
 & \multicolumn{1}{l|}{Incremental-\textit{Topic}} & 0.512 & 0.230 & \multicolumn{1}{c|}{0.182} & 0.419 & 0.262 & \multicolumn{1}{c|}{0.215} & 0.977 & 6.00 / 3.00 \\
 & \multicolumn{1}{l|}{Cluster} & 0.586 & 0.168 & \multicolumn{1}{c|}{0.126} & 0.356 & 0.309 & \multicolumn{1}{c|}{0.269} & 0.927 & 6.01 / 3.01 \\
 & \multicolumn{1}{l|}{RAG+Cluster} & 0.665 & 0.151 & \multicolumn{1}{c|}{0.078} & 0.417 & 0.269 & \multicolumn{1}{c|}{0.198} & 0.979 & 6.04 / 3.02 \\ \midrule
\multirow{10}{*}{3} & \multicolumn{1}{l|}{\textbf{\modelAll (\textbf{Ours})}} & {\ul 0.8664} & 0.1062* & \multicolumn{1}{c|}{0.0359*} & {\ul 0.5420} & {\ul 0.1896*} & \multicolumn{1}{c|}{{\ul 0.1217}} & 0.988 & 8.97 / 2.99 \\
 & \multicolumn{1}{l|}{\textbf{\modelTopic (\textbf{Ours})}} & \textbf{0.8961*} & {\ul 0.0998*} & \multicolumn{1}{c|}{\textbf{0.0320*}} & \textbf{0.6056*} & \textbf{0.1650*} & \multicolumn{1}{c|}{\textbf{0.0979}} & 0.985 & 8.99 / 3.00 \\
 & \multicolumn{1}{l|}{Long-Context} & 0.5242 & 0.2047 & \multicolumn{1}{c|}{0.1733} & 0.2566 & 0.3816 & \multicolumn{1}{c|}{0.3503} & 0.958 & 9.00 / 3.00 \\
 & \multicolumn{1}{l|}{RAG-\textit{All}} & 0.6565 & 0.1664 & \multicolumn{1}{c|}{0.0911} & 0.3300 & 0.3296 & \multicolumn{1}{c|}{0.2547} & 0.990 & 9.01 / 3.00 \\
 & \multicolumn{1}{l|}{RAG-\textit{Doc}} & 0.7532 & 0.1364 & \multicolumn{1}{c|}{0.0668} & 0.3741 & 0.3023 & \multicolumn{1}{c|}{0.2352} & 0.949 & 9.01 / 3.00 \\
 & \multicolumn{1}{l|}{Hierarchical} & 0.8158 & \textbf{0.0956*} & \multicolumn{1}{c|}{{\ul 0.0333*}} & 0.3679 & 0.3136 & \multicolumn{1}{c|}{0.2523} & 0.981 & 8.99 / 3.00 \\
 & \multicolumn{1}{l|}{Incremental-\textit{All}} & 0.5037 & 0.2466 & \multicolumn{1}{c|}{0.1924} & 0.3467 & 0.3019 & \multicolumn{1}{c|}{0.2488} & 0.961 & 8.99 / 3.00 \\
 & \multicolumn{1}{l|}{Incremental-\textit{Topic}} & 0.5635 & 0.2288 & \multicolumn{1}{c|}{0.1720} & 0.4209 & 0.2796 & \multicolumn{1}{c|}{0.2236} & 0.963 & 9.01 / 3.00 \\
 & \multicolumn{1}{l|}{Cluster} & 0.7142 & 0.1203* & \multicolumn{1}{c|}{0.0662} & 0.3502 & 0.3016 & \multicolumn{1}{c|}{0.2517} & 0.927 & 9.04 / 3.01 \\
 & \multicolumn{1}{l|}{RAG+Cluster} & 0.7694 & 0.1332 & \multicolumn{1}{c|}{0.0620} & 0.3906 & 0.2808 & \multicolumn{1}{c|}{0.2101} & 0.976 & 9.02 / 3.01 \\ \midrule
\multirow{10}{*}{4} & \multicolumn{1}{l|}{\textbf{\modelAll (\textbf{Ours})}} & {\ul 0.8991} & {\ul 0.0976*} & \multicolumn{1}{c|}{{\ul 0.0301*}} & {\ul 0.5107} & {\ul 0.1886*} & \multicolumn{1}{c|}{{\ul 0.1225*}} & 0.987 & 11.92 / 2.98 \\
 & \multicolumn{1}{l|}{\textbf{\modelTopic (\textbf{Ours})}} & \textbf{0.9307*} & \textbf{0.0907*} & \multicolumn{1}{c|}{\textbf{0.0263*}} & \textbf{0.5954*} & \textbf{0.1653*} & \multicolumn{1}{c|}{\textbf{0.1022*}} & 0.982 & 12.00 / 3.00 \\
 & \multicolumn{1}{l|}{Long-Context} & 0.5594 & 0.1953 & \multicolumn{1}{c|}{0.1501} & 0.2342 & 0.4204 & \multicolumn{1}{c|}{0.3779} & 0.953 & 12.03 / 3.01 \\
 & \multicolumn{1}{l|}{RAG-\textit{All}} & 0.7065 & 0.1485 & \multicolumn{1}{c|}{0.0801} & 0.2987 & 0.3556 & \multicolumn{1}{c|}{0.2891} & 0.997 & 12.02 / 3.00 \\
 & \multicolumn{1}{l|}{RAG-\textit{Doc}} & 0.7638 & 0.1357 & \multicolumn{1}{c|}{0.0631} & 0.3293 & 0.3427 & \multicolumn{1}{c|}{0.2725} & 0.961 & 12.01 / 3.00 \\
 & \multicolumn{1}{l|}{Hierarchical} & 0.8643 & 0.1008* & \multicolumn{1}{c|}{0.0325*} & 0.3204 & 0.3439 & \multicolumn{1}{c|}{0.2768} & 0.983 & 12.02 / 3.01 \\
 & \multicolumn{1}{l|}{Incremental-\textit{All}} & 0.4994 & 0.2589 & \multicolumn{1}{c|}{0.1999} & 0.3208 & 0.3200 & \multicolumn{1}{c|}{0.2602} & 0.950 & 11.97 / 2.99 \\
 & \multicolumn{1}{l|}{Incremental-\textit{Topic}} & 0.5611 & 0.2274 & \multicolumn{1}{c|}{0.1703} & 0.3896 & 0.2931 & \multicolumn{1}{c|}{0.2365} & 0.954 & 12.00 / 3.00 \\
 & \multicolumn{1}{l|}{Cluster} & 0.7907 & 0.1108* & \multicolumn{1}{c|}{0.0577} & 0.3485 & 0.3068 & \multicolumn{1}{c|}{0.2557} & 0.931 & 12.02 / 3.01 \\
 & \multicolumn{1}{l|}{RAG+Cluster} & 0.8266 & 0.1175 & \multicolumn{1}{c|}{0.0527} & 0.3614 & 0.3002 & \multicolumn{1}{c|}{0.2393} & 0.977 & 12.03 / 3.01 \\ \midrule
\multirow{10}{*}{5} & \multicolumn{1}{l|}{\textbf{\modelAll (\textbf{Ours})}} & {\ul 0.9156} & {\ul 0.0966*} & \multicolumn{1}{c|}{{\ul 0.0272*}} & {\ul 0.4809} & {\ul 0.1972} & \multicolumn{1}{c|}{{\ul 0.1297}} & 0.990 & 14.88 / 2.98 \\
 & \multicolumn{1}{l|}{\textbf{\modelTopic (\textbf{Ours})}} & \textbf{0.9549*} & \textbf{0.0884*} & \multicolumn{1}{c|}{\textbf{0.0239*}} & \textbf{0.5924*} & \textbf{0.1661*} & \multicolumn{1}{c|}{\textbf{0.1051*}} & 0.986 & 15.00 / 3.00 \\
 & \multicolumn{1}{l|}{Long-Context} & 0.5779 & 0.2038 & \multicolumn{1}{c|}{0.1622} & 0.2164 & 0.4620 & \multicolumn{1}{c|}{0.4213} & 0.966 & 15.00 / 3.00 \\
 & \multicolumn{1}{l|}{RAG-\textit{All}} & 0.7331 & 0.1581 & \multicolumn{1}{c|}{0.0814} & 0.2755 & 0.3850 & \multicolumn{1}{c|}{0.3101} & 0.996 & 15.03 / 3.01 \\
 & \multicolumn{1}{l|}{RAG-\textit{Doc}} & 0.7898 & 0.1464 & \multicolumn{1}{c|}{0.0706} & 0.3018 & 0.3691 & \multicolumn{1}{c|}{0.2945} & 0.975 & 15.06 / 3.01 \\
 & \multicolumn{1}{l|}{Hierarchical} & 0.8871 & 0.0931* & \multicolumn{1}{c|}{0.0276*} & 0.2951 & 0.3670 & \multicolumn{1}{c|}{0.3038} & 0.987 & 15.01 / 3.00 \\
 & \multicolumn{1}{l|}{Incremental-\textit{All}} & 0.5392 & 0.2327 & \multicolumn{1}{c|}{0.1738} & 0.3083 & 0.3236 & \multicolumn{1}{c|}{0.2672} & 0.948 & 14.91 / 2.98 \\
 & \multicolumn{1}{l|}{Incremental-\textit{Topic}} & 0.6239 & 0.1899 & \multicolumn{1}{c|}{0.1337} & 0.3961 & 0.2902 & \multicolumn{1}{c|}{0.2348} & 0.958 & 14.99 / 3.00 \\
 & \multicolumn{1}{l|}{Cluster} & 0.8480 & 0.0968* & \multicolumn{1}{c|}{0.0464} & 0.3365 & 0.3093 & \multicolumn{1}{c|}{0.2625} & 0.933 & 15.04 / 3.01 \\
 & \multicolumn{1}{l|}{RAG+Cluster} & 0.8717 & 0.1084* & \multicolumn{1}{c|}{0.0436} & 0.3499 & 0.3136 & \multicolumn{1}{c|}{0.2511} & 0.971 & 15.03 / 3.01 \\ \bottomrule
\end{tabular}
\caption{\label{table:doc_cover_cqa_all}ConflictingQA citation coverage, balance, and accuracy. Best model is \textbf{bold}, second best is \underline{underlined}. Models with * are significantly the best (2-sample $t$-test, $p<0.05$ with Bonferroni correction).}
\end{table*}

\begin{table*}[!h]
\footnotesize
\centering
\setlength{\tabcolsep}{3.5pt}
%\setlength{\extrarowheight}{2pt}
% \setlength{\aboverulesep}{1pt}
% \setlength{\belowrulesep}{1pt}
\renewcommand{\arraystretch}{0.8}
\begin{tabular}{@{}clcccccccc@{}}
\multicolumn{1}{l}{} &  & \multicolumn{3}{c}{\textit{Summary Level}} & \multicolumn{3}{c}{\textit{Topic Paragraph Level}} & \multicolumn{2}{c}{\textit{Confounders}} \\ \midrule
\textbf{\# Pts} & \multicolumn{1}{l|}{\textbf{Model}} & \textbf{DC ($\uparrow$)} & \textbf{Fair ($\downarrow$)} & \multicolumn{1}{c|}{\textbf{Faithful ($\downarrow$)}} & \textbf{DC ($\uparrow$)} & \textbf{Fair ($\downarrow$)} & \multicolumn{1}{c|}{\textbf{Faithful ($\downarrow$)}} & \multicolumn{1}{l}{\textbf{Cite Acc}} & \textbf{All / Avg Sents} \\ \midrule
\multirow{10}{*}{2} & \multicolumn{1}{l|}{\modelTopic (\textbf{Ours})} & \textbf{0.798*} & \textbf{0.088*} & \multicolumn{1}{c|}{\textbf{0.036*}} & \textbf{0.614*} & \textbf{0.132*} & \multicolumn{1}{c|}{\textbf{0.078*}} & 0.991 & 5.99 / 3.00 \\
 & \multicolumn{1}{l|}{\modelAll (\textbf{Ours})} & {\ul 0.789*} & {\ul 0.098*} & \multicolumn{1}{c|}{{\ul 0.040*}} & {\ul 0.582*} & {\ul 0.150*} & \multicolumn{1}{c|}{{\ul 0.092*}} & 0.992 & 5.96 / 2.98 \\
 & \multicolumn{1}{l|}{Long-Context} & 0.506 & 0.254 & \multicolumn{1}{c|}{0.212} & 0.302 & 0.423 & \multicolumn{1}{c|}{0.385} & 0.976 & 6.01 / 3.00 \\
 & \multicolumn{1}{l|}{RAG-\textit{All}} & 0.529 & 0.183 & \multicolumn{1}{c|}{0.139} & 0.347 & 0.295 & \multicolumn{1}{c|}{0.251} & 0.995 & 6.01 / 3.00 \\
 & \multicolumn{1}{l|}{RAG-\textit{Doc}} & 0.630 & 0.142 & \multicolumn{1}{c|}{0.095} & 0.374 & 0.325 & \multicolumn{1}{c|}{0.280} & 0.991 & 5.99 / 3.00 \\
 & \multicolumn{1}{l|}{Hierarchical} & 0.710 & 0.104* & \multicolumn{1}{c|}{0.053*} & 0.421 & 0.261 & \multicolumn{1}{c|}{0.209} & 0.983 & 6.00 / 3.00 \\
 & \multicolumn{1}{l|}{Incremental-\textit{All}} & 0.497 & 0.326 & \multicolumn{1}{c|}{0.291} & 0.405 & 0.348 & \multicolumn{1}{c|}{0.313} & 0.981 & 6.01 / 3.00 \\
 & \multicolumn{1}{l|}{Incremental-\textit{Topic}} & 0.548 & 0.297 & \multicolumn{1}{c|}{0.266} & 0.459 & 0.338 & \multicolumn{1}{c|}{0.307} & 0.982 & 6.00 / 3.00 \\
 & \multicolumn{1}{l|}{Cluster} & 0.610 & 0.133 & \multicolumn{1}{c|}{0.102} & 0.384 & 0.297 & \multicolumn{1}{c|}{0.266} & 0.966 & 6.01 / 3.00 \\
 & \multicolumn{1}{l|}{RAG+Cluster} & 0.572 & 0.166 & \multicolumn{1}{c|}{0.121} & 0.354 & 0.306 & \multicolumn{1}{c|}{0.260} & 0.986 & 6.02 / 3.01 \\ \midrule
\multirow{10}{*}{3} & \multicolumn{1}{l|}{\modelTopic (\textbf{Ours})} & \textbf{0.8724*} & \textbf{0.0701*} & \multicolumn{1}{c|}{\textbf{0.0235*}} & \textbf{0.6066*} & \textbf{0.1255*} & \multicolumn{1}{c|}{\textbf{0.0789*}} & 0.982 & 8.99 / 3.00 \\
\multicolumn{1}{l}{} & \multicolumn{1}{l|}{\modelAll (\textbf{Ours})} & {\ul 0.8457*} & {\ul 0.0786*} & \multicolumn{1}{c|}{{\ul 0.0273*}} & {\ul 0.5508} & {\ul 0.1463*} & \multicolumn{1}{c|}{{\ul 0.0938*}} & 0.987 & 8.87 / 2.96 \\
\multicolumn{1}{l}{} & \multicolumn{1}{l|}{Long-Context} & 0.5877 & 0.2094 & \multicolumn{1}{c|}{0.1790} & 0.2798 & 0.4336 & \multicolumn{1}{c|}{0.4028} & 0.953 & 9.02 / 3.01 \\
\multicolumn{1}{l}{} & \multicolumn{1}{l|}{RAG-\textit{All}} & 0.6125 & 0.1544 & \multicolumn{1}{c|}{0.1040} & 0.3229 & 0.3176 & \multicolumn{1}{c|}{0.2701} & 0.997 & 9.01 / 3.00 \\
\multicolumn{1}{l}{} & \multicolumn{1}{l|}{RAG-\textit{Doc}} & 0.7171 & 0.1180 & \multicolumn{1}{c|}{0.0664} & 0.3504 & 0.3233 & \multicolumn{1}{c|}{0.2748} & 0.961 & 9.01 / 3.00 \\
\multicolumn{1}{l}{} & \multicolumn{1}{l|}{Hierarchical} & 0.7868 & 0.0907 & \multicolumn{1}{c|}{0.0374} & 0.3639 & 0.2980 & \multicolumn{1}{c|}{0.2452} & 0.983 & 9.02 / 3.01 \\
\multicolumn{1}{l}{} & \multicolumn{1}{l|}{Incremental-\textit{All}} & 0.5566 & 0.2579 & \multicolumn{1}{c|}{0.2089} & 0.3919 & 0.3243 & \multicolumn{1}{c|}{0.2765} & 0.950 & 8.91 / 2.97 \\
\multicolumn{1}{l}{} & \multicolumn{1}{l|}{Incremental-\textit{Topic}} & 0.6152 & 0.2415 & \multicolumn{1}{c|}{0.1970} & 0.4707 & 0.3128 & \multicolumn{1}{c|}{0.2674} & 0.954 & 9.03 / 3.01 \\
\multicolumn{1}{l}{} & \multicolumn{1}{l|}{Cluster} & 0.7102 & 0.1106 & \multicolumn{1}{c|}{0.0725} & 0.3632 & 0.3106 & \multicolumn{1}{c|}{0.2737} & 0.931 & 9.04 / 3.01 \\
\multicolumn{1}{l}{} & \multicolumn{1}{l|}{RAG+Cluster} & 0.6811 & 0.1405 & \multicolumn{1}{c|}{0.0894} & 0.3428 & 0.3200 & \multicolumn{1}{c|}{0.2689} & 0.977 & 9.01 / 3.00 \\ \midrule
\multirow{10}{*}{4} & \multicolumn{1}{l|}{\modelTopic (\textbf{Ours})} & \textbf{0.8895*} & {\ul 0.0724*} & \multicolumn{1}{c|}{\textbf{0.0209*}} & \textbf{0.5844*} & \textbf{0.1385*} & \multicolumn{1}{c|}{\textbf{0.0868*}} & 0.987 & 11.98 / 3.00 \\
 & \multicolumn{1}{l|}{\modelAll (\textbf{Ours})} & {\ul 0.8653*} & \textbf{0.0697*} & \multicolumn{1}{c|}{{\ul 0.0216*}} & {\ul 0.5230} & {\ul 0.1419*} & \multicolumn{1}{c|}{{\ul 0.0925*}} & 0.990 & 11.86 / 2.96 \\
 & \multicolumn{1}{l|}{Long-Context} & 0.6361 & 0.1691 & \multicolumn{1}{c|}{0.1471} & 0.2473 & 0.4733 & \multicolumn{1}{c|}{0.4479} & 0.977 & 12.03 / 3.01 \\
 & \multicolumn{1}{l|}{RAG-\textit{All}} & 0.6595 & 0.1440 & \multicolumn{1}{c|}{0.0969} & 0.2916 & 0.3603 & \multicolumn{1}{c|}{0.3149} & 0.995 & 12.03 / 3.01 \\
 & \multicolumn{1}{l|}{RAG-\textit{Doc}} & 0.7335 & 0.1218 & \multicolumn{1}{c|}{0.0723} & 0.3113 & 0.3635 & \multicolumn{1}{c|}{0.3171} & 0.991 & 12.03 / 3.01 \\
 & \multicolumn{1}{l|}{Hierarchical} & 0.8338 & 0.0845* & \multicolumn{1}{c|}{0.0325} & 0.3269 & 0.3331 & \multicolumn{1}{c|}{0.2813} & 0.986 & 12.02 / 3.01 \\
 & \multicolumn{1}{l|}{Incremental-\textit{All}} & 0.5716 & 0.2352 & \multicolumn{1}{c|}{0.1874} & 0.3795 & 0.3193 & \multicolumn{1}{c|}{0.2736} & 0.963 & 11.87 / 2.97 \\
 & \multicolumn{1}{l|}{Incremental-\textit{Topic}} & 0.6331 & 0.2129 & \multicolumn{1}{c|}{0.1629} & 0.4514 & 0.3133 & \multicolumn{1}{c|}{0.2658} & 0.970 & 11.98 / 2.99 \\
 & \multicolumn{1}{l|}{Cluster} & 0.7744 & 0.1129 & \multicolumn{1}{c|}{0.0698} & 0.3451 & 0.3181 & \multicolumn{1}{c|}{0.2752} & 0.964 & 12.03 / 3.01 \\
 & \multicolumn{1}{l|}{RAG+Cluster} & 0.7305 & 0.1218 & \multicolumn{1}{c|}{0.0746} & 0.3237 & 0.3459 & \multicolumn{1}{c|}{0.3029} & 0.989 & 12.04 / 3.01 \\ \midrule
\multirow{10}{*}{5} & \multicolumn{1}{l|}{\modelTopic (\textbf{Ours})} & \textbf{0.9137*} & {\ul 0.0651*} & \multicolumn{1}{c|}{\textbf{0.0208*}} & \textbf{0.5793*} & \textbf{0.1420*} & \multicolumn{1}{c|}{\textbf{0.0998*}} & 0.986 & 14.99 / 3.00 \\
 & \multicolumn{1}{l|}{\modelAll (\textbf{Ours})} & {\ul 0.8847*} & \textbf{0.0640*} & \multicolumn{1}{c|}{{\ul 0.0236*}} & {\ul 0.4991} & {\ul 0.1502*} & \multicolumn{1}{c|}{{\ul 0.1096*}} & 0.990 & 14.46 / 2.89 \\
 & \multicolumn{1}{l|}{Long-Context} & 0.6686 & 0.1724 & \multicolumn{1}{c|}{0.1392} & 0.2312 & 0.4965 & \multicolumn{1}{c|}{0.4640} & 0.966 & 15.01 / 3.00 \\
 & \multicolumn{1}{l|}{RAG-\textit{All}} & 0.6721 & 0.1423 & \multicolumn{1}{c|}{0.0912} & 0.2668 & 0.3927 & \multicolumn{1}{c|}{0.3438} & 0.996 & 15.02 / 3.00 \\
 & \multicolumn{1}{l|}{RAG-\textit{Doc}} & 0.7765 & 0.1053 & \multicolumn{1}{c|}{0.0618} & 0.3005 & 0.3584 & \multicolumn{1}{c|}{0.3147} & 0.975 & 15.01 / 3.00 \\
 & \multicolumn{1}{l|}{Hierarchical} & 0.8565 & 0.0761* & \multicolumn{1}{c|}{0.0239*} & 0.2896 & 0.3713 & \multicolumn{1}{c|}{0.3192} & 0.987 & 15.04 / 3.01 \\
 & \multicolumn{1}{l|}{Incremental-\textit{All}} & 0.6122 & 0.2000 & \multicolumn{1}{c|}{0.1629} & 0.3716 & 0.2936 & \multicolumn{1}{c|}{0.2572} & 0.948 & 14.77 / 2.95 \\
 & \multicolumn{1}{l|}{Incremental-\textit{Topic}} & 0.6767 & 0.1659 & \multicolumn{1}{c|}{0.1198} & 0.4446 & 0.2897 & \multicolumn{1}{c|}{0.2443} & 0.958 & 15.05 / 3.01 \\
 & \multicolumn{1}{l|}{Cluster} & 0.8098 & 0.1116 & \multicolumn{1}{c|}{0.0624} & 0.3292 & 0.3383 & \multicolumn{1}{c|}{0.2921} & 0.933 & 15.03 / 3.01 \\
 & \multicolumn{1}{l|}{RAG+Cluster} & 0.7811 & 0.1233 & \multicolumn{1}{c|}{0.0738} & 0.3129 & 0.3588 & \multicolumn{1}{c|}{0.3107} & 0.971 & 15.03 / 3.01 \\ \bottomrule
\end{tabular}
\caption{\label{table:doc_cover_debate_all}DebateQFS citation coverage, balance, and accuracy. Best model is \textbf{bold}, second best is \underline{underlined}. Models with * are significantly the best (2-sample $t$-test, $p<0.05$ with Bonferroni correction).}
\end{table*}


\begin{table*}[!h]
\footnotesize
\centering
\setlength{\tabcolsep}{3.5pt}
%\setlength{\extrarowheight}{2pt}
% \setlength{\aboverulesep}{1pt}
% \setlength{\belowrulesep}{1pt}
\renewcommand{\arraystretch}{0.8}
\begin{tabular}{@{}clcccccccc@{}}
\multicolumn{1}{l}{} &  & \multicolumn{3}{c}{{\textit{Summary Level}}} & \multicolumn{3}{c}{{\textit{Topic Paragraph Level}}} & \multicolumn{2}{c}{{\textit{Confounders}}} \\ \midrule
{\# Pts} & \multicolumn{1}{l|}{{Model}} & {\textbf{DC} ($\uparrow$)} & {\textbf{Fair} ($\downarrow$)} & \multicolumn{1}{c|}{{\textbf{Faithful} ($\downarrow$)}} & {\textbf{DC} ($\uparrow$)} & {\textbf{Fair} ($\downarrow$)} & \multicolumn{1}{c|}{{\textbf{Faithful} ($\downarrow$)}} & \multicolumn{1}{l}{{\textbf{Cite Acc}.}} & {\textbf{All / Avg Sents}} \\ \midrule

\multirow{2}{*}{3} & \multicolumn{1}{l|}{{\modelTopic ({Ours})}} & {0.8961} & {0.0998} & \multicolumn{1}{c|}{{0.0320}} & {0.6056} & {0.1650} & \multicolumn{1}{c|}{{0.0979}} & 0.985 & 8.99 / 3.00 \\
 & \multicolumn{1}{l|}{Hierarchical-\textit{Topic}} & 0.8761 & {0.1065} & \multicolumn{1}{c|}{{0.0467	}} & 0.6003 & 0.1688 & \multicolumn{1}{c|}{0.1130} & 0.985 & 8.98 / 2.99 \\ \midrule

\multirow{2}{*}{5} & \multicolumn{1}{l|}{{\modelTopic ({Ours})}} & {0.9549} & {0.0884} & \multicolumn{1}{c|}{{0.0239}} & {0.5924} & {0.1661} & \multicolumn{1}{c|}{{0.1051}} & 0.986 & 15.00 / 3.00 \\
 & \multicolumn{1}{l|}{Hierarchical-\textit{Topic}} & 0.9386 & 0.0996 & \multicolumn{1}{c|}{0.0310} & 0.5774 & 0.1952 & \multicolumn{1}{c|}{0.1304} & 0.987 & 15.01 / 3.00  \\ \bottomrule
\end{tabular}
\caption{\label{table:doc_cover_cqa_all_comp}ConflictingQA citation coverage, balance, and accuracy of \modelTopic versus Hierarchical Merging-\emph{Topic}, which runs hierarchical merging for each topic paragraph. \model consistently outperforms Hierarchal Merging.}
\end{table*}

\begin{table*}[!h]
\footnotesize
\centering
\setlength{\tabcolsep}{3.5pt}
%\setlength{\extrarowheight}{2pt}
% \setlength{\aboverulesep}{1pt}
% \setlength{\belowrulesep}{1pt}
\renewcommand{\arraystretch}{0.8}
\begin{tabular}{@{}clcccccccc@{}}
\multicolumn{1}{l}{} &  & \multicolumn{3}{c}{{\textit{Summary Level}}} & \multicolumn{3}{c}{{\textit{Topic Paragraph Level}}} & \multicolumn{2}{c}{{\textit{Confounders}}} \\ \midrule
{\# Pts} & \multicolumn{1}{l|}{{Model}} & {\textbf{DC} ($\uparrow$)} & {\textbf{Fair} ($\downarrow$)} & \multicolumn{1}{c|}{{\textbf{Faithful} ($\downarrow$)}} & {\textbf{DC} ($\uparrow$)} & {\textbf{Fair} ($\downarrow$)} & \multicolumn{1}{c|}{{\textbf{Faithful} ($\downarrow$)}} & \multicolumn{1}{l}{{\textbf{Cite Acc}}} & {\textbf{All / Avg Sents}} \\ \midrule

\multirow{2}{*}{3} & \multicolumn{1}{l|}{\modelTopic ({Ours})} & {0.8724} & {0.0701} & \multicolumn{1}{c|}{{0.0235}} & {0.6066} & {0.1255} & \multicolumn{1}{c|}{{0.0789}} & 0.982 & 8.99 / 3.00 \\
\multicolumn{1}{l}{} & \multicolumn{1}{l|}{Hierarchical-\textit{Topic}} & 0.7776 & 0.0965 & \multicolumn{1}{c|}{0.0483} & 0.4964 & 0.2177 & \multicolumn{1}{c|}{0.1688} & 0.983 & 9.00 / 3.00\\ \midrule

\multirow{2}{*}{5} & \multicolumn{1}{l|}{\modelTopic ({Ours})} & {0.9137} & { 0.0651} & \multicolumn{1}{c|}{{0.0208}} & {0.5793} & {0.1420} & \multicolumn{1}{c|}{{0.0998}} & 0.986 & 14.99 / 3.00 \\
 
 & \multicolumn{1}{l|}{Hierarchical-\textit{Topic}} & 0.8427 & 0.0951 & \multicolumn{1}{c|}{0.0431} & 0.4669 & 0.2397 & \multicolumn{1}{c|}{0.1909} & 0.984 & 14.90 / 2.98 \\ \bottomrule
\end{tabular}
\caption{\label{table:doc_cover_debate_all_comp}DebateQFS citation coverage, balance, and accuracy of \modelTopic versus Hierarchical Merging-\emph{Topic}, which runs hierarchical merging for each topic paragraph. \model consistently outperforms Hierarchal Merging.}
\end{table*}


\begin{table*}[!h]
\footnotesize
\centering
\setlength{\tabcolsep}{3.5pt}
%\setlength{\extrarowheight}{2pt}
% \setlength{\aboverulesep}{1pt}
% \setlength{\belowrulesep}{1pt}
\renewcommand{\arraystretch}{0.8}
\begin{tabular}{@{}clcccccc@{}}
\multicolumn{1}{l}{} &  & \multicolumn{3}{c}{{\textit{Summary Level}}} & \multicolumn{3}{c}{{\textit{Topic Paragraph Level}}} \\ \midrule
{\# Pts} & \multicolumn{1}{l|}{{Model}} & {\textbf{DC} ($\uparrow$)} & {\textbf{Fair} ($\downarrow$)} & \multicolumn{1}{c|}{{\textbf{Faithful} ($\downarrow$)}} & {\textbf{DC} ($\uparrow$)} & {\textbf{Fair} ($\downarrow$)} & \multicolumn{1}{c}{{\textbf{Faithful} ($\downarrow$)}} \\ \midrule

\multirow{2}{*}{3} & \multicolumn{1}{l|}{{\modelTopic (GPT-4) }} & {0.8961} & {0.0998} & \multicolumn{1}{c|}{{0.0320}} & {0.6056} & {0.1650} & \multicolumn{1}{c}{{0.0979}} \\
 & \multicolumn{1}{l|}{\modelTopic (GPT-4 mini)} & 0.8761 & {0.1065} & \multicolumn{1}{c|}{{0.0467	}} & 0.6003 & 0.1688 & \multicolumn{1}{c}{0.1130} \\ \midrule

\multirow{2}{*}{5} & \multicolumn{1}{l|}{{\modelTopic (GPT-4) }} & {0.9549} & {0.0884} & \multicolumn{1}{c|}{{0.0239}} & {0.5924} & {0.1661} & \multicolumn{1}{c}{{0.1051}} \\
 & \multicolumn{1}{l|}{\modelTopic (GPT-4 mini)} & 0.7841 & 0.1226 & \multicolumn{1}{c|}{0.0634} & 0.4320 & 0.2112	 & \multicolumn{1}{c}{0.1533}  \\ \bottomrule
\end{tabular}
\caption{\label{table:doc_cover_cqa_all_comp_mini}ConflictingQA citation coverage, balance, and accuracy of \modelTopic using GPT-4 versus \modelTopic using GPT-4 mini.}
\end{table*}

\begin{table*}[!h]
\footnotesize
\centering
\setlength{\tabcolsep}{3.5pt}
%\setlength{\extrarowheight}{2pt}
% \setlength{\aboverulesep}{1pt}
% \setlength{\belowrulesep}{1pt}
\renewcommand{\arraystretch}{0.8}
\begin{tabular}{@{}clcccccc@{}}
\multicolumn{1}{l}{} &  & \multicolumn{3}{c}{{\textit{Summary Level}}} & \multicolumn{3}{c}{{\textit{Topic Paragraph Level}}} \\ \midrule
{\# Pts} & \multicolumn{1}{l|}{{Model}} & {\textbf{DC} ($\uparrow$)} & {\textbf{Fair} ($\downarrow$)} & \multicolumn{1}{c|}{{\textbf{Faithful} ($\downarrow$)}} & {\textbf{DC} ($\uparrow$)} & {\textbf{Fair} ($\downarrow$)} & \multicolumn{1}{c}{{\textbf{Faithful} ($\downarrow$)}} \\ \midrule

\multirow{2}{*}{3} & \multicolumn{1}{l|}{\modelTopic (GPT-4) } & {0.8724} & {0.0701} & \multicolumn{1}{c|}{{0.0235}} & {0.6066} & {0.1255} & \multicolumn{1}{c}{{0.0789}} \\
\multicolumn{1}{l}{} & \multicolumn{1}{l|}{\modelTopic (GPT-4 mini)} & 0.7322	 & 0.1284 & \multicolumn{1}{c|}{0.1059} & 0.4788 & 0.2271 & \multicolumn{1}{c}{0.2066} \\ \midrule

\multirow{2}{*}{5} & \multicolumn{1}{l|}{\modelTopic (GPT-4) } & {0.9137} & { 0.0651} & \multicolumn{1}{c|}{{0.0208}} & {0.5793} & {0.1420} & \multicolumn{1}{c}{{0.0998}} \\
 
 & \multicolumn{1}{l|}{\modelTopic (GPT-4 mini)} & 0.8324	 & 0.0686	 & \multicolumn{1}{c|}{0.0686} & 0.4818 & 0.2260	 & \multicolumn{1}{c}{0.2260	} \\ \bottomrule
\end{tabular}
\caption{\label{table:doc_cover_debate_all_comp_mini}DebateQFS citation coverage, balance, and accuracy of \modelTopic using GPT-4 versus \modelTopic using GPT-4 mini.}
\end{table*}


\begin{table*}[t]
\footnotesize
\centering
\setlength{\tabcolsep}{2.75pt}
%\setlength{\extrarowheight}{2pt}
% \setlength{\aboverulesep}{1pt}
% \setlength{\belowrulesep}{1pt}
\renewcommand{\arraystretch}{0.6}
\begin{tabular}{@{}clcccccc@{}}
\multicolumn{1}{l}{} &  & \multicolumn{3}{c}{\textit{Summary Level}} & \multicolumn{3}{c}{\textit{Topic Paragraph Level}} \\ \toprule
\textbf{\# Top.} & \multicolumn{1}{l|}{\textbf{Model}} & \textbf{DC ($\uparrow$)} & \textbf{Fair ($\downarrow$)} & \multicolumn{1}{c|}{\textbf{\begin{tabular}[c]{@{}c@{}}Faithful ($\downarrow$)\end{tabular}}} & \textbf{DC ($\uparrow$)} & \textbf{Fair ($\downarrow$)} & \multicolumn{1}{c}{\textbf{\begin{tabular}[c]{@{}c@{}}Faithful ($\downarrow$)\end{tabular}}} \\ \midrule
 & \multicolumn{1}{l|}{\modelTopic \textbf{(Ours)}} & \textbf{0.8961*} & {\textbf{0.0998*}} & \multicolumn{1}{c|}{\textbf{0.0320*}} & \textbf{0.6056*} & \textbf{0.1650*} & \multicolumn{1}{c}{\textbf{0.0979*}} \\
\multirow{8}{*}{3} & \multicolumn{1}{l|}{\modelAll \textbf{(Ours)}} & {\ul 0.8664*} & {\ul 0.1062}* & \multicolumn{1}{c|}{{\ul 0.0359*}} & {\ul 0.5420} & {\ul 0.1896*} & \multicolumn{1}{c}{{\ul 0.1217}} \\
 & \multicolumn{1}{l|}{Long-Context} & 0.5320 & 0.1834 & \multicolumn{1}{c|}{0.1395} & 0.2662 & 0.3614 & \multicolumn{1}{c}{0.3173} \\
 & \multicolumn{1}{l|}{RAG-\textit{All}} & 0.6325 & 0.1557 & \multicolumn{1}{c|}{0.0898} & 0.3098 & 0.3499 & \multicolumn{1}{c}{0.2825} \\
 & \multicolumn{1}{l|}{RAG-\textit{Doc}} & 0.6909 & 0.1529 & \multicolumn{1}{c|}{0.0776} & 0.3356 & 0.3476 & \multicolumn{1}{c}{0.2752} \\
 & \multicolumn{1}{l|}{Hierarchical} & 0.7647 & 0.1191 & \multicolumn{1}{c|}{{0.0575}} & 0.3509 & 0.3032 & \multicolumn{1}{c}{0.2523} \\
 & \multicolumn{1}{l|}{Incremental-\textit{All}} & 0.5037 & 0.2466 & \multicolumn{1}{c|}{0.1924} & 0.3467 & 0.3019 & \multicolumn{1}{c}{0.2488} \\
 & \multicolumn{1}{l|}{Incremental-\textit{Topic}} & 0.5635 & 0.2288 & \multicolumn{1}{c|}{0.1720} & 0.4209 & 0.2796 & \multicolumn{1}{c}{0.2236} \\ \bottomrule
\end{tabular}
\caption{\label{table:doc_cover_cqa_fixedtopic}ConflictingQA citation coverage, balance, and accuracy when models have fixed topics (except RAG and RAG+Cluster). Best model is \textbf{bold}, second best is \underline{underlined}. Models with * are significantly the best (2-sample $t$-test, $p<0.05$ with Bonferroni correction. \model consistently has the highest citation coverage, fairness, and faithfulness for summaries and topic paragraphs, even when baselines use the same topics, suggesting that our gains are not derived from the agenda planning step, but rather question tailoring and outline construction. }
\end{table*}





\begin{table*}[t]
\centering
\footnotesize
\setlength{\tabcolsep}{2.75pt}
\renewcommand{\arraystretch}{0.6}
\begin{tabular}{@{}clcccccc@{}}
\multicolumn{1}{l}{} &  & \multicolumn{3}{c}{\textit{Summary Level}} & \multicolumn{3}{c}{\textit{Topic Paragraph Level}} \\ 
\toprule
\textbf{\# Top.} & \multicolumn{1}{l|}{\textbf{Model}} & \textbf{DC ($\uparrow$)} & \textbf{Fair ($\downarrow$)} & \multicolumn{1}{c|}{\textbf{Faithful ($\downarrow$)}} & \textbf{DC ($\uparrow$)} & \textbf{Fair ($\downarrow$)} & \multicolumn{1}{c}{\textbf{Faithful ($\downarrow$)}} \\ \midrule
\multirow{10}{*}{3} & \multicolumn{1}{l|}{\modelTopic \textbf{(Ours)}} & \textbf{0.8724*} & \textbf{0.0701*} & \multicolumn{1}{c|}{\textbf{0.0235*}} & \textbf{0.6066*} & \textbf{0.1255*} & \multicolumn{1}{c}{\textbf{0.0789*}} \\
 & \multicolumn{1}{l|}{\modelAll \textbf{(Ours)}} & {\ul 0.8457*} & {\ul 0.0786*} & \multicolumn{1}{c|}{{\ul 0.0273*}} & {\ul 0.5508} & {\ul 0.1463*} & \multicolumn{1}{c}{{\ul 0.0938*}} \\
 & \multicolumn{1}{l|}{Long-Context} & 0.6025 & 0.1919 & \multicolumn{1}{c|}{0.1559} & 0.2956 & 0.3865 & \multicolumn{1}{c}{0.3517} \\
 & \multicolumn{1}{l|}{RAG-\textit{All}} & 0.6200 & 0.1502 & \multicolumn{1}{c|}{0.0968} & 0.3103 & 0.3421 & \multicolumn{1}{c}{0.2896} \\
 & \multicolumn{1}{l|}{RAG-\textit{Doc}} & 0.6728 & 0.1216 & \multicolumn{1}{c|}{0.0683} & 0.3254 & 0.3226 & \multicolumn{1}{c}{0.2694} \\
 & \multicolumn{1}{l|}{Hierarchical} & 0.7676 & 0.0954 & \multicolumn{1}{c|}{0.0443} & 0.3650 & 0.2729 & \multicolumn{1}{c}{0.2207} \\
 & \multicolumn{1}{l|}{Incremental-\textit{All}} & 0.5566 & 0.2579 & \multicolumn{1}{c|}{0.2089} & 0.3919 & 0.3243 & \multicolumn{1}{c}{0.2765} \\
 & \multicolumn{1}{l|}{Incremental-\textit{Topic}} & 0.6152 & 0.2415 & \multicolumn{1}{c|}{0.1970} & 0.4707 & 0.3128 & \multicolumn{1}{c}{0.2674}\\ \bottomrule
\end{tabular}

\caption{\label{table:doc_cover_debate_fixedtopic}DebateQFS citation coverage, balance, and accuracy when models have fixed topics (except RAG and RAG+Cluster). Best model is \textbf{bold}, second best is \underline{underlined}. Models with * are significantly the best (2-sample $t$-test, $p<0.05$ with Bonferroni correction. \model consistently has the highest citation coverage, fairness, and faithfulness for summaries and topic paragraphs, even when baselines use the same topics, suggesting that our gains are not derived from the agenda planning step, but rather question tailoring and outline construction. }
\vspace{-2ex}
\end{table*}
\begin{table*}[]
\definecolor{myblue}{HTML}{DAE8FC}
\small
\centering
\setlength{\tabcolsep}{3.5pt}
\renewcommand{\arraystretch}{0.8}
\begin{tabular}{@{}clrrrrrrrrrrrrrrrc@{}}
\multicolumn{1}{l}{} &  & \multicolumn{5}{c}{\textit{Summary Quality}} & \multicolumn{5}{c}{\textit{Topic Paragraph Quality}} & \multicolumn{5}{c}{\textit{Topic Quality}} & \multicolumn{1}{l}{\textit{Sep.}} \\ \midrule
\textbf{\textbf{\# Topics}} & \multicolumn{1}{l|}{\textbf{Model}} & \multicolumn{1}{c}{\textbf{Int}} & \multicolumn{1}{c}{\textbf{Coh}} & \multicolumn{1}{c}{\textbf{Rel}} & \multicolumn{1}{l}{\textbf{Cov}} & \multicolumn{1}{l|}{\textbf{Div}} & \multicolumn{1}{c}{\textbf{Int}} & \multicolumn{1}{c}{\textbf{Coh}} & \multicolumn{1}{c}{\textbf{Rel}} & \multicolumn{1}{l}{\textbf{Cov}} & \multicolumn{1}{l|}{\textbf{Div}} & \multicolumn{1}{c}{\textbf{Int}} & \multicolumn{1}{c}{\textbf{Coh}} & \multicolumn{1}{c}{\textbf{Rel}} & \multicolumn{1}{l}{\textbf{Cov}} & \multicolumn{1}{l|}{\textbf{Div}} & \textbf{SB} \\ \midrule
 & \multicolumn{1}{l|}{\textbf{\modelTopic}} & \cellcolor[HTML]{DAE8FC}\textbf{4.22} & \cellcolor[HTML]{DAE8FC}4.24 & \cellcolor[HTML]{DAE8FC}4.59 & \cellcolor[HTML]{DAE8FC}4.46 & \multicolumn{1}{r|}{\cellcolor[HTML]{DAE8FC}\textbf{4.23}} & \cellcolor[HTML]{DAE8FC}\textbf{4.09} & \cellcolor[HTML]{DAE8FC}4.30 & \cellcolor[HTML]{DAE8FC}\textbf{4.70} & \cellcolor[HTML]{DAE8FC}\textbf{4.38} & \multicolumn{1}{r|}{\cellcolor[HTML]{DAE8FC}\textbf{3.93}} & \cellcolor[HTML]{DAE8FC}3.22 & \cellcolor[HTML]{DAE8FC}3.88 & \cellcolor[HTML]{DAE8FC}4.56 & \cellcolor[HTML]{DAE8FC}3.00 & \multicolumn{1}{r|}{\cellcolor[HTML]{DAE8FC}3.48} & 0.52 \\
 & \multicolumn{1}{l|}{\textbf{\modelAll}} & \cellcolor[HTML]{DAE8FC}4.12 & \cellcolor[HTML]{DAE8FC}\textbf{4.27} & \cellcolor[HTML]{DAE8FC}\textbf{4.68} & \cellcolor[HTML]{DAE8FC}\textbf{4.49} & \multicolumn{1}{r|}{\cellcolor[HTML]{DAE8FC}4.14} & \cellcolor[HTML]{DAE8FC}3.99 & \cellcolor[HTML]{DAE8FC}4.31 & \cellcolor[HTML]{DAE8FC}4.64 & \cellcolor[HTML]{DAE8FC}4.29 & \multicolumn{1}{r|}{\cellcolor[HTML]{DAE8FC}3.80} & \cellcolor[HTML]{DAE8FC}\textbf{3.27} & \cellcolor[HTML]{DAE8FC}3.93 & \cellcolor[HTML]{DAE8FC}4.52 & \cellcolor[HTML]{DAE8FC}\textbf{3.19} & \multicolumn{1}{r|}{\cellcolor[HTML]{DAE8FC}\textbf{3.70}} & 0.50 \\
 & \multicolumn{1}{l|}{Long-Context} & 3.96 & \cellcolor[HTML]{DAE8FC}4.18 & \cellcolor[HTML]{DAE8FC}4.55 & 4.31 & \multicolumn{1}{r|}{3.85} & 3.72 & 4.14 & 4.51 & 4.03 & \multicolumn{1}{r|}{3.25} & 3.00 & \cellcolor[HTML]{DAE8FC}3.86 & 4.47 & 2.90 & \multicolumn{1}{r|}{\cellcolor[HTML]{DAE8FC}3.47} & 0.45 \\
 & \multicolumn{1}{l|}{RAG-\textit{All}} & \cellcolor[HTML]{DAE8FC}4.06 & \cellcolor[HTML]{DAE8FC}4.24 & \cellcolor[HTML]{DAE8FC}4.55 & \cellcolor[HTML]{DAE8FC}4.43 & \multicolumn{1}{r|}{4.00} & 3.80 & \cellcolor[HTML]{DAE8FC}4.25 & 4.60 & 4.13 & \multicolumn{1}{r|}{3.63} & \cellcolor[HTML]{DAE8FC}3.08 & \cellcolor[HTML]{DAE8FC}3.86 & \cellcolor[HTML]{DAE8FC}4.51 & 2.81 & \multicolumn{1}{r|}{3.42} & 0.47 \\
 & \multicolumn{1}{l|}{RAG-\textit{Doc}} & \cellcolor[HTML]{DAE8FC}4.17 & \cellcolor[HTML]{DAE8FC}4.22 & \cellcolor[HTML]{DAE8FC}4.56 & \cellcolor[HTML]{DAE8FC}4.39 & \multicolumn{1}{r|}{\cellcolor[HTML]{DAE8FC}4.16} & 3.86 & \cellcolor[HTML]{DAE8FC}4.26 & \cellcolor[HTML]{DAE8FC}4.64 & 4.24 & \multicolumn{1}{r|}{3.71} & \cellcolor[HTML]{DAE8FC}3.10 & \cellcolor[HTML]{DAE8FC}3.88 & \cellcolor[HTML]{DAE8FC}4.59 & 2.84 & \multicolumn{1}{r|}{3.41} & 0.47 \\
 & \multicolumn{1}{l|}{Hierarchical} & \cellcolor[HTML]{DAE8FC}4.16 & \cellcolor[HTML]{DAE8FC}4.24 & \cellcolor[HTML]{DAE8FC}4.58 & \cellcolor[HTML]{DAE8FC}4.46 & \multicolumn{1}{r|}{\cellcolor[HTML]{DAE8FC}4.14} & 3.93 & \cellcolor[HTML]{DAE8FC}\textbf{4.33} & \cellcolor[HTML]{DAE8FC}\textbf{4.70} & 4.27 & \multicolumn{1}{r|}{3.76} & \cellcolor[HTML]{DAE8FC}3.21 & \cellcolor[HTML]{DAE8FC}3.90 & \cellcolor[HTML]{DAE8FC}\textbf{4.61} & \cellcolor[HTML]{DAE8FC}3.18 & \multicolumn{1}{r|}{\cellcolor[HTML]{DAE8FC}3.47} & 0.47 \\
 & \multicolumn{1}{l|}{Increm-\textit{All}} & 3.95 & \cellcolor[HTML]{DAE8FC}4.14 & \cellcolor[HTML]{DAE8FC}4.58 & 4.28 & \multicolumn{1}{r|}{3.90} & 3.64 & 4.11 & 4.57 & 4.01 & \multicolumn{1}{r|}{3.31} & \cellcolor[HTML]{DAE8FC}3.14 & \cellcolor[HTML]{DAE8FC}\textbf{3.97} & \cellcolor[HTML]{DAE8FC}4.60 & \cellcolor[HTML]{DAE8FC}3.07 & \multicolumn{1}{r|}{3.46} & 0.46 \\
 & \multicolumn{1}{l|}{Increm-\textit{Topic}} & \cellcolor[HTML]{DAE8FC}4.11 & \cellcolor[HTML]{DAE8FC}4.21 & \cellcolor[HTML]{DAE8FC}4.60 & \cellcolor[HTML]{DAE8FC}4.44 & \multicolumn{1}{r|}{\cellcolor[HTML]{DAE8FC}4.18} & \cellcolor[HTML]{DAE8FC}4.05 & \cellcolor[HTML]{DAE8FC}4.30 & \cellcolor[HTML]{DAE8FC}4.66 & 4.21 & \multicolumn{1}{r|}{3.76} & \cellcolor[HTML]{DAE8FC}3.03 & 3.63 & 4.37 & 2.83 & \multicolumn{1}{r|}{3.30} & 0.49 \\
 & \multicolumn{1}{l|}{Cluster} & 3.89 & 4.08 & 4.45 & 4.22 & \multicolumn{1}{r|}{3.94} & 3.73 & 4.11 & 4.49 & 4.04 & \multicolumn{1}{r|}{3.50} & 2.41 & 3.16 & 3.89 & 2.29 & \multicolumn{1}{r|}{2.47} & 0.48 \\
\multirow{-10}{*}{2} & \multicolumn{1}{l|}{RAG+Cluster} & \cellcolor[HTML]{DAE8FC}4.13 & \cellcolor[HTML]{DAE8FC}\textbf{4.27} & \cellcolor[HTML]{DAE8FC}4.59 & \cellcolor[HTML]{DAE8FC}4.38 & \multicolumn{1}{r|}{\cellcolor[HTML]{DAE8FC}4.07} & \cellcolor[HTML]{DAE8FC}3.97 & \cellcolor[HTML]{DAE8FC}4.29 & \cellcolor[HTML]{DAE8FC}4.67 & \cellcolor[HTML]{DAE8FC}4.30 & \multicolumn{1}{r|}{\cellcolor[HTML]{DAE8FC}3.87} & 2.53 & 3.26 & 4.04 & 2.49 & \multicolumn{1}{r|}{2.60} & 0.52 \\ \midrule
 & \multicolumn{1}{l|}{\textbf{\modelTopic}} & \cellcolor[HTML]{DAE8FC}4.24 & \cellcolor[HTML]{DAE8FC}4.34 & \cellcolor[HTML]{DAE8FC}4.64 & \cellcolor[HTML]{DAE8FC}4.49 & \multicolumn{1}{r|}{\cellcolor[HTML]{DAE8FC}\textbf{4.42}} & \cellcolor[HTML]{DAE8FC}\textbf{4.08} & \cellcolor[HTML]{DAE8FC}\textbf{4.33} & \cellcolor[HTML]{DAE8FC}4.69 & \cellcolor[HTML]{DAE8FC}\textbf{4.34} & \multicolumn{1}{r|}{\cellcolor[HTML]{DAE8FC}\textbf{3.89}} & \cellcolor[HTML]{DAE8FC}3.47 & \cellcolor[HTML]{DAE8FC}\textbf{4.12} & \cellcolor[HTML]{DAE8FC}\textbf{4.69} & \cellcolor[HTML]{DAE8FC}\textbf{3.61} & \multicolumn{1}{r|}{\cellcolor[HTML]{DAE8FC}\textbf{4.02}} & 0.69 \\
 & \multicolumn{1}{l|}{\textbf{\modelAll}} & \cellcolor[HTML]{DAE8FC}\textbf{4.27} & \cellcolor[HTML]{DAE8FC}4.33 & \cellcolor[HTML]{DAE8FC}4.63 & \cellcolor[HTML]{DAE8FC}4.49 & \multicolumn{1}{r|}{\cellcolor[HTML]{DAE8FC}4.40} & 3.88 & \cellcolor[HTML]{DAE8FC}4.27 & 4.60 & 4.19 & \multicolumn{1}{r|}{3.70} & \cellcolor[HTML]{DAE8FC}\textbf{3.49} & \cellcolor[HTML]{DAE8FC}4.09 & \cellcolor[HTML]{DAE8FC}4.62 & \cellcolor[HTML]{DAE8FC}3.46 & \multicolumn{1}{r|}{\cellcolor[HTML]{DAE8FC}3.99} & 0.65 \\
 & \multicolumn{1}{l|}{Long-Context} & 4.02 & \cellcolor[HTML]{DAE8FC}4.34 & \cellcolor[HTML]{DAE8FC}4.63 & \cellcolor[HTML]{DAE8FC}4.44 & \multicolumn{1}{r|}{4.23} & 3.62 & 4.14 & 4.51 & 3.89 & \multicolumn{1}{r|}{3.21} & 3.24 & \cellcolor[HTML]{DAE8FC}4.03 & \cellcolor[HTML]{DAE8FC}4.55 & 3.25 & \multicolumn{1}{r|}{3.76} & 0.58 \\
 & \multicolumn{1}{l|}{RAG-\textit{All}} & \cellcolor[HTML]{DAE8FC}4.16 & \cellcolor[HTML]{DAE8FC}4.33 & \cellcolor[HTML]{DAE8FC}4.67 & \cellcolor[HTML]{DAE8FC}4.49 & \multicolumn{1}{r|}{\cellcolor[HTML]{DAE8FC}4.29} & 3.80 & 4.16 & 4.61 & 4.06 & \multicolumn{1}{r|}{3.53} & \cellcolor[HTML]{DAE8FC}3.41 & \cellcolor[HTML]{DAE8FC}4.08 & \cellcolor[HTML]{DAE8FC}4.57 & \cellcolor[HTML]{DAE8FC}3.47 & \multicolumn{1}{r|}{\cellcolor[HTML]{DAE8FC}3.95} & 0.60 \\
 & \multicolumn{1}{l|}{RAG-\textit{Doc}} & \cellcolor[HTML]{DAE8FC}4.15 & \cellcolor[HTML]{DAE8FC}\textbf{4.37} & \cellcolor[HTML]{DAE8FC}4.68 & \cellcolor[HTML]{DAE8FC}4.47 & \multicolumn{1}{r|}{\cellcolor[HTML]{DAE8FC}\textbf{4.42}} & 3.76 & 4.22 & 4.60 & 4.10 & \multicolumn{1}{r|}{3.56} & \cellcolor[HTML]{DAE8FC}3.33 & \cellcolor[HTML]{DAE8FC}4.08 & \cellcolor[HTML]{DAE8FC}4.63 & \cellcolor[HTML]{DAE8FC}3.39 & \multicolumn{1}{r|}{\cellcolor[HTML]{DAE8FC}3.91} & 0.60 \\
 & \multicolumn{1}{l|}{Hierarchical} & 4.24 & \cellcolor[HTML]{DAE8FC}\textbf{4.37} & \cellcolor[HTML]{DAE8FC}4.73 & \cellcolor[HTML]{DAE8FC}4.50 & \multicolumn{1}{r|}{\cellcolor[HTML]{DAE8FC}4.38} & 3.78 & 4.21 & 4.62 & 4.14 & \multicolumn{1}{r|}{3.57} & \cellcolor[HTML]{DAE8FC}3.43 & \cellcolor[HTML]{DAE8FC}4.07 & \cellcolor[HTML]{DAE8FC}4.65 & \cellcolor[HTML]{DAE8FC}3.49 & \multicolumn{1}{r|}{\cellcolor[HTML]{DAE8FC}3.94} & 0.58 \\
 & \multicolumn{1}{l|}{Increm-\textit{All}} & 3.98 & \cellcolor[HTML]{DAE8FC}4.29 & \cellcolor[HTML]{DAE8FC}4.67 & 4.42 & \multicolumn{1}{r|}{4.21} & 3.54 & 4.09 & 4.56 & 3.79 & \multicolumn{1}{r|}{3.26} & \cellcolor[HTML]{DAE8FC}3.44 & \cellcolor[HTML]{DAE8FC}4.02 & \cellcolor[HTML]{DAE8FC}4.65 & \cellcolor[HTML]{DAE8FC}3.52 & \multicolumn{1}{r|}{\cellcolor[HTML]{DAE8FC}3.94} & 0.58 \\
 & \multicolumn{1}{l|}{Increm-\textit{Topic}} & 4.17 & \cellcolor[HTML]{DAE8FC}\textbf{4.37} & \cellcolor[HTML]{DAE8FC}\textbf{4.74} & \cellcolor[HTML]{DAE8FC}\textbf{4.57} & \multicolumn{1}{r|}{\cellcolor[HTML]{DAE8FC}4.39} & 3.91 & \cellcolor[HTML]{DAE8FC}4.29 & 4.62 & 4.25 & \multicolumn{1}{r|}{3.65} & \cellcolor[HTML]{DAE8FC}3.36 & 3.79 & 4.31 & 3.21 & \multicolumn{1}{r|}{3.73} & 0.61 \\
 & \multicolumn{1}{l|}{Cluster} & 3.81 & 4.03 & 4.25 & 4.19 & \multicolumn{1}{r|}{3.94} & 3.69 & 4.08 & 4.45 & 3.95 & \multicolumn{1}{r|}{3.53} & 2.42 & 2.86 & 3.73 & 2.13 & \multicolumn{1}{r|}{2.47} & 0.61 \\
\multirow{-10}{*}{3} & \multicolumn{1}{l|}{RAG+Cluster} & \cellcolor[HTML]{DAE8FC}4.14 & 4.22 & 4.60 & \cellcolor[HTML]{DAE8FC}4.52 & \multicolumn{1}{r|}{4.22} & 3.96 & \cellcolor[HTML]{DAE8FC}4.31 & \cellcolor[HTML]{DAE8FC}\textbf{4.71} & 4.25 & \multicolumn{1}{r|}{3.77} & 2.43 & 3.11 & 3.82 & 2.44 & \multicolumn{1}{r|}{2.64} & 0.64 \\ \midrule
 & \multicolumn{1}{l|}{\textbf{\modelTopic}} & \cellcolor[HTML]{DAE8FC}\textbf{4.30} & \cellcolor[HTML]{DAE8FC}4.21 & \cellcolor[HTML]{DAE8FC}4.54 & \cellcolor[HTML]{DAE8FC}\textbf{4.54} & \multicolumn{1}{r|}{\cellcolor[HTML]{DAE8FC}\textbf{4.48}} & \cellcolor[HTML]{DAE8FC}\textbf{4.09} & \cellcolor[HTML]{DAE8FC}\textbf{4.29} & \cellcolor[HTML]{DAE8FC}\textbf{4.66} & \cellcolor[HTML]{DAE8FC}\textbf{4.35} & \multicolumn{1}{r|}{\cellcolor[HTML]{DAE8FC}\textbf{3.89}} & \cellcolor[HTML]{DAE8FC}\textbf{3.93} & \cellcolor[HTML]{DAE8FC}4.17 & \cellcolor[HTML]{DAE8FC}4.65 & \cellcolor[HTML]{DAE8FC}4.04 & \multicolumn{1}{r|}{\cellcolor[HTML]{DAE8FC}\textbf{4.31}} & 0.72 \\
 & \multicolumn{1}{l|}{\textbf{\modelAll}} & \cellcolor[HTML]{DAE8FC}4.24 & \cellcolor[HTML]{DAE8FC}4.26 & \cellcolor[HTML]{DAE8FC}4.53 & \cellcolor[HTML]{DAE8FC}4.49 & \multicolumn{1}{r|}{\cellcolor[HTML]{DAE8FC}4.38} & 3.93 & \cellcolor[HTML]{DAE8FC}4.24 & \cellcolor[HTML]{DAE8FC}4.61 & 4.20 & \multicolumn{1}{r|}{3.76} & \cellcolor[HTML]{DAE8FC}3.80 & \cellcolor[HTML]{DAE8FC}\textbf{4.22} & \cellcolor[HTML]{DAE8FC}\textbf{4.67} & \cellcolor[HTML]{DAE8FC}\textbf{4.13} & \multicolumn{1}{r|}{\cellcolor[HTML]{DAE8FC}4.30} & 0.70 \\
 & \multicolumn{1}{l|}{Long-Context} & \cellcolor[HTML]{DAE8FC}4.14 & \cellcolor[HTML]{DAE8FC}4.26 & \cellcolor[HTML]{DAE8FC}4.48 & 4.32 & \multicolumn{1}{r|}{4.25} & 3.53 & 4.08 & 4.47 & 3.83 & \multicolumn{1}{r|}{3.14} & 3.65 & 4.00 & 4.53 & 3.85 & \multicolumn{1}{r|}{4.04} & 0.65 \\
 & \multicolumn{1}{l|}{RAG-\textit{All}} & \cellcolor[HTML]{DAE8FC}4.17 & \cellcolor[HTML]{DAE8FC}4.30 & \cellcolor[HTML]{DAE8FC}4.55 & \cellcolor[HTML]{DAE8FC}4.45 & \multicolumn{1}{r|}{4.29} & 3.72 & 4.18 & 4.59 & 3.99 & \multicolumn{1}{r|}{3.44} & \cellcolor[HTML]{DAE8FC}3.80 & \cellcolor[HTML]{DAE8FC}4.14 & \cellcolor[HTML]{DAE8FC}4.65 & \cellcolor[HTML]{DAE8FC}4.03 & \multicolumn{1}{r|}{\cellcolor[HTML]{DAE8FC}4.23} & 0.66 \\
 & \multicolumn{1}{l|}{RAG-\textit{Doc}} & \cellcolor[HTML]{DAE8FC}4.23 & \cellcolor[HTML]{DAE8FC}\textbf{4.31} & \cellcolor[HTML]{DAE8FC}4.56 & \cellcolor[HTML]{DAE8FC}4.41 & \multicolumn{1}{r|}{\cellcolor[HTML]{DAE8FC}4.41} & 3.76 & 4.16 & 4.59 & 4.07 & \multicolumn{1}{r|}{3.45} & 3.66 & \cellcolor[HTML]{DAE8FC}4.15 & \cellcolor[HTML]{DAE8FC}4.59 & \cellcolor[HTML]{DAE8FC}4.03 & \multicolumn{1}{r|}{\cellcolor[HTML]{DAE8FC}4.19} & 0.66 \\
 & \multicolumn{1}{l|}{Hierarchical} & \cellcolor[HTML]{DAE8FC}4.23 & \cellcolor[HTML]{DAE8FC}4.24 & \cellcolor[HTML]{DAE8FC}\textbf{4.59} & \cellcolor[HTML]{DAE8FC}4.51 & \multicolumn{1}{r|}{\cellcolor[HTML]{DAE8FC}4.36} & 3.75 & 4.19 & 4.59 & 4.06 & \multicolumn{1}{r|}{3.47} & 3.67 & \cellcolor[HTML]{DAE8FC}4.10 & \cellcolor[HTML]{DAE8FC}4.62 & 3.90 & \multicolumn{1}{r|}{\cellcolor[HTML]{DAE8FC}4.22} & 0.65 \\
 & \multicolumn{1}{l|}{Increm-\textit{All}} & 3.95 & 4.14 & 4.44 & 4.30 & \multicolumn{1}{r|}{4.15} & 3.48 & 4.04 & 4.49 & 3.76 & \multicolumn{1}{r|}{3.14} & 3.71 & \cellcolor[HTML]{DAE8FC}4.09 & \cellcolor[HTML]{DAE8FC}4.62 & \cellcolor[HTML]{DAE8FC}4.02 & \multicolumn{1}{r|}{\cellcolor[HTML]{DAE8FC}4.16} & 0.65 \\
 & \multicolumn{1}{l|}{Increm-\textit{Topic}} & \cellcolor[HTML]{DAE8FC}4.20 & \cellcolor[HTML]{DAE8FC}4.23 & 4.43 & \cellcolor[HTML]{DAE8FC}4.42 & \multicolumn{1}{r|}{\cellcolor[HTML]{DAE8FC}4.38} & 3.93 & \cellcolor[HTML]{DAE8FC}4.25 & \cellcolor[HTML]{DAE8FC}\textbf{4.66} & 4.17 & \multicolumn{1}{r|}{3.60} & 3.47 & 3.85 & 4.39 & 3.69 & \multicolumn{1}{r|}{3.83} & 0.69 \\
 & \multicolumn{1}{l|}{Cluster} & 3.92 & 4.03 & 4.17 & 4.20 & \multicolumn{1}{r|}{4.09} & 3.68 & 4.13 & 4.47 & 3.99 & \multicolumn{1}{r|}{3.51} & 2.36 & 2.73 & 3.62 & 2.27 & \multicolumn{1}{r|}{2.54} & 0.68 \\
\multirow{-10}{*}{4} & \multicolumn{1}{l|}{RAG+Cluster} & \cellcolor[HTML]{DAE8FC}4.21 & \cellcolor[HTML]{DAE8FC}4.20 & 4.44 & \cellcolor[HTML]{DAE8FC}4.44 & \multicolumn{1}{r|}{4.28} & 3.99 & \cellcolor[HTML]{DAE8FC}4.28 & \cellcolor[HTML]{DAE8FC}\textbf{4.66} & 4.26 & \multicolumn{1}{r|}{\cellcolor[HTML]{DAE8FC}3.83} & 2.56 & 3.05 & 3.95 & 2.58 & \multicolumn{1}{r|}{2.69} & 0.71 \\ \midrule
 & \multicolumn{1}{l|}{\textbf{\modelTopic}} & \cellcolor[HTML]{DAE8FC}4.17 & \cellcolor[HTML]{DAE8FC}4.24 & \cellcolor[HTML]{DAE8FC}4.35 & \cellcolor[HTML]{DAE8FC}\textbf{4.51} & \multicolumn{1}{r|}{\cellcolor[HTML]{DAE8FC}4.35} & \cellcolor[HTML]{DAE8FC}\textbf{4.08} & \cellcolor[HTML]{DAE8FC}\textbf{4.33} & \cellcolor[HTML]{DAE8FC}\textbf{4.69} & \cellcolor[HTML]{DAE8FC}\textbf{4.40} & \multicolumn{1}{r|}{\cellcolor[HTML]{DAE8FC}\textbf{3.97}} & \cellcolor[HTML]{DAE8FC}\textbf{4.15} & \cellcolor[HTML]{DAE8FC}\textbf{4.43} & \cellcolor[HTML]{DAE8FC}\textbf{4.82} & \cellcolor[HTML]{DAE8FC}\textbf{4.44} & \multicolumn{1}{r|}{\cellcolor[HTML]{DAE8FC}4.52} & 0.76 \\
 & \multicolumn{1}{l|}{\textbf{\modelAll}} & \cellcolor[HTML]{DAE8FC}\textbf{4.25} & \cellcolor[HTML]{DAE8FC}4.22 & \cellcolor[HTML]{DAE8FC}4.41 & \cellcolor[HTML]{DAE8FC}4.44 & \multicolumn{1}{r|}{\cellcolor[HTML]{DAE8FC}4.39} & 3.89 & 4.24 & 4.60 & 4.21 & \multicolumn{1}{r|}{3.69} & \cellcolor[HTML]{DAE8FC}4.14 & \cellcolor[HTML]{DAE8FC}4.37 & \cellcolor[HTML]{DAE8FC}4.77 & \cellcolor[HTML]{DAE8FC}\textbf{4.44} & \multicolumn{1}{r|}{\cellcolor[HTML]{DAE8FC}4.50} & 0.74 \\
 & \multicolumn{1}{l|}{Long-Context} & 3.98 & 4.11 & 4.28 & 4.29 & \multicolumn{1}{r|}{4.12} & 3.50 & 4.10 & 4.46 & 3.83 & \multicolumn{1}{r|}{3.02} & 3.90 & \cellcolor[HTML]{DAE8FC}4.35 & \cellcolor[HTML]{DAE8FC}4.71 & 4.22 & \multicolumn{1}{r|}{4.37} & 0.69 \\
 & \multicolumn{1}{l|}{RAG-\textit{All}} & \cellcolor[HTML]{DAE8FC}4.11 & \cellcolor[HTML]{DAE8FC}4.24 & \cellcolor[HTML]{DAE8FC}\textbf{4.48} & \cellcolor[HTML]{DAE8FC}4.48 & \multicolumn{1}{r|}{4.28} & 3.69 & 4.18 & 4.56 & 3.99 & \multicolumn{1}{r|}{3.39} & \cellcolor[HTML]{DAE8FC}4.02 & \cellcolor[HTML]{DAE8FC}4.39 & \cellcolor[HTML]{DAE8FC}4.80 & \cellcolor[HTML]{DAE8FC}4.36 & \multicolumn{1}{r|}{\cellcolor[HTML]{DAE8FC}4.46} & 0.71 \\
 & \multicolumn{1}{l|}{RAG-\textit{Doc}} & \cellcolor[HTML]{DAE8FC}4.12 & \cellcolor[HTML]{DAE8FC}4.20 & \cellcolor[HTML]{DAE8FC}\textbf{4.48} & \cellcolor[HTML]{DAE8FC}4.42 & \multicolumn{1}{r|}{\cellcolor[HTML]{DAE8FC}\textbf{4.50}} & 3.74 & 4.21 & 4.57 & 4.01 & \multicolumn{1}{r|}{3.42} & \cellcolor[HTML]{DAE8FC}3.96 & \cellcolor[HTML]{DAE8FC}4.36 & \cellcolor[HTML]{DAE8FC}4.78 & \cellcolor[HTML]{DAE8FC}4.32 & \multicolumn{1}{r|}{\cellcolor[HTML]{DAE8FC}4.41} & 0.70 \\
 & \multicolumn{1}{l|}{Hierarchical} & \cellcolor[HTML]{DAE8FC}4.07 & \cellcolor[HTML]{DAE8FC}\textbf{4.27} & 4.47 & \cellcolor[HTML]{DAE8FC}4.42 & \multicolumn{1}{r|}{\cellcolor[HTML]{DAE8FC}4.41} & 3.69 & 4.17 & 4.55 & 4.01 & \multicolumn{1}{r|}{3.39} & \cellcolor[HTML]{DAE8FC}4.07 & \cellcolor[HTML]{DAE8FC}4.35 & \cellcolor[HTML]{DAE8FC}4.80 & \cellcolor[HTML]{DAE8FC}4.37 & \multicolumn{1}{r|}{\cellcolor[HTML]{DAE8FC}\textbf{4.56}} & 0.70 \\
 & \multicolumn{1}{l|}{Increm-\textit{All}} & 3.83 & 4.09 & \cellcolor[HTML]{DAE8FC}4.35 & 4.27 & \multicolumn{1}{r|}{4.05} & 3.38 & 4.00 & 4.42 & 3.66 & \multicolumn{1}{r|}{2.98} & \cellcolor[HTML]{DAE8FC}4.06 & \cellcolor[HTML]{DAE8FC}4.41 & \cellcolor[HTML]{DAE8FC}4.74 & \cellcolor[HTML]{DAE8FC}4.29 & \multicolumn{1}{r|}{\cellcolor[HTML]{DAE8FC}4.44} & 0.69 \\
 & \multicolumn{1}{l|}{Increm-\textit{Topic}} & \cellcolor[HTML]{DAE8FC}4.05 & \cellcolor[HTML]{DAE8FC}4.22 & \cellcolor[HTML]{DAE8FC}4.34 & \cellcolor[HTML]{DAE8FC}4.34 & \multicolumn{1}{r|}{4.25} & 3.86 & 4.24 & 4.64 & 4.14 & \multicolumn{1}{r|}{3.57} & 3.69 & 4.00 & 4.52 & 3.96 & \multicolumn{1}{r|}{4.11} & 0.73 \\
 & \multicolumn{1}{l|}{Cluster} & 3.92 & 3.88 & 3.94 & 4.10 & \multicolumn{1}{r|}{4.07} & 3.74 & 4.09 & 4.46 & 4.00 & \multicolumn{1}{r|}{3.50} & 2.27 & 2.68 & 3.55 & 2.41 & \multicolumn{1}{r|}{2.48} & 0.73 \\
\multirow{-10}{*}{5} & \multicolumn{1}{l|}{RAG+Cluster} & 4.00 & 4.08 & 4.30 & 4.28 & \multicolumn{1}{r|}{4.29} & \cellcolor[HTML]{DAE8FC}4.00 & \cellcolor[HTML]{DAE8FC}4.30 & \cellcolor[HTML]{DAE8FC}4.66 & 4.28 & \multicolumn{1}{r|}{3.77} & 2.63 & 3.01 & 3.88 & 2.66 & \multicolumn{1}{r|}{2.62} & 0.75 \\ \bottomrule
\end{tabular}
\caption{\label{appendix:table:llm_cqa} Interest, Coherence, Relevance, Coverage, and Diversity scores from Prometheus for summaries, topic paragraphs, and topics on ConflictingQA. Best scores are \textbf{bold}, significant scores in \colorbox{myblue}{blue} (2-sample $t$-test, $p<0.05$)}
\end{table*}
\begin{table*}[]
\definecolor{myblue}{HTML}{DAE8FC}
\small
\centering
\setlength{\tabcolsep}{3.5pt}
\renewcommand{\arraystretch}{0.8}
\begin{tabular}{@{}clrrrrrrrrrrrrrrrc@{}}
\multicolumn{1}{l}{} &  & \multicolumn{5}{c}{\textit{Summary Quality}} & \multicolumn{5}{c}{\textit{Topic Paragraph Quality}} & \multicolumn{5}{c}{\textit{Topic Quality}} & \multicolumn{1}{l}{\textit{Sep.}} \\ \midrule
\textbf{\# Topics} & \multicolumn{1}{l|}{\textbf{Model}} & \multicolumn{1}{c}{\textbf{Int}} & \multicolumn{1}{c}{\textbf{Coh}} & \multicolumn{1}{c}{\textbf{Rel}} & \multicolumn{1}{l}{\textbf{Cov}} & \multicolumn{1}{l|}{\textbf{Div}} & \multicolumn{1}{c}{\textbf{Int}} & \multicolumn{1}{c}{\textbf{Coh}} & \multicolumn{1}{c}{\textbf{Rel}} & \multicolumn{1}{l}{\textbf{Cov}} & \multicolumn{1}{l|}{\textbf{Div}} & \multicolumn{1}{c}{\textbf{Int}} & \multicolumn{1}{c}{\textbf{Coh}} & \multicolumn{1}{c}{\textbf{Rel}} & \multicolumn{1}{l}{\textbf{Cov}} & \multicolumn{1}{l|}{\textbf{Div}} & \textbf{SB} \\ \midrule
 & \multicolumn{1}{l|}{\textbf{\modelTopic}} & \cellcolor[HTML]{DAE8FC}\textbf{4.16} & \cellcolor[HTML]{DAE8FC}4.13 & \cellcolor[HTML]{DAE8FC}4.53 & \cellcolor[HTML]{DAE8FC}\textbf{4.34} & \multicolumn{1}{r|}{\cellcolor[HTML]{DAE8FC}\textbf{4.15}} & \cellcolor[HTML]{DAE8FC}\textbf{4.03} & \cellcolor[HTML]{DAE8FC}4.20 & \cellcolor[HTML]{DAE8FC}\textbf{4.62} & \cellcolor[HTML]{DAE8FC}\textbf{4.22} & \multicolumn{1}{r|}{\cellcolor[HTML]{DAE8FC}\textbf{3.89}} & \cellcolor[HTML]{DAE8FC}\textbf{3.28} & \cellcolor[HTML]{DAE8FC}3.98 & \cellcolor[HTML]{DAE8FC}4.62 & \cellcolor[HTML]{DAE8FC}2.93 & \multicolumn{1}{r|}{\cellcolor[HTML]{DAE8FC}3.56} & 0.50 \\
 & \multicolumn{1}{l|}{\textbf{\modelAll}} & \cellcolor[HTML]{DAE8FC}3.98 & \cellcolor[HTML]{DAE8FC}4.10 & \cellcolor[HTML]{DAE8FC}4.45 & \cellcolor[HTML]{DAE8FC}\textbf{4.34} & \multicolumn{1}{r|}{\cellcolor[HTML]{DAE8FC}4.09} & \cellcolor[HTML]{DAE8FC}3.88 & \cellcolor[HTML]{DAE8FC}4.20 & 4.50 & \cellcolor[HTML]{DAE8FC}4.16 & \multicolumn{1}{r|}{\cellcolor[HTML]{DAE8FC}3.75} & \cellcolor[HTML]{DAE8FC}3.23 & \cellcolor[HTML]{DAE8FC}\textbf{4.01} & \cellcolor[HTML]{DAE8FC}4.61 & \cellcolor[HTML]{DAE8FC}\textbf{3.11} & \multicolumn{1}{r|}{\cellcolor[HTML]{DAE8FC}3.56} & 0.49 \\
 & \multicolumn{1}{l|}{Long-Context} & 3.79 & \cellcolor[HTML]{DAE8FC}4.07 & \cellcolor[HTML]{DAE8FC}4.48 & \cellcolor[HTML]{DAE8FC}4.19 & \multicolumn{1}{r|}{3.83} & 3.57 & \cellcolor[HTML]{DAE8FC}4.15 & \cellcolor[HTML]{DAE8FC}4.53 & 3.90 & \multicolumn{1}{r|}{3.28} & \cellcolor[HTML]{DAE8FC}3.15 & \cellcolor[HTML]{DAE8FC}3.81 & \cellcolor[HTML]{DAE8FC}4.56 & 2.70 & \multicolumn{1}{r|}{\cellcolor[HTML]{DAE8FC}3.51} & 0.46 \\
 & \multicolumn{1}{l|}{RAG-\textit{All}} & \cellcolor[HTML]{DAE8FC}4.02 & \cellcolor[HTML]{DAE8FC}4.08 & \cellcolor[HTML]{DAE8FC}4.46 & \cellcolor[HTML]{DAE8FC}4.20 & \multicolumn{1}{r|}{\cellcolor[HTML]{DAE8FC}3.96} & 3.72 & 4.11 & \cellcolor[HTML]{DAE8FC}4.54 & 4.04 & \multicolumn{1}{r|}{3.61} & \cellcolor[HTML]{DAE8FC}3.23 & \cellcolor[HTML]{DAE8FC}3.78 & \cellcolor[HTML]{DAE8FC}4.56 & \cellcolor[HTML]{DAE8FC}2.97 & \multicolumn{1}{r|}{\cellcolor[HTML]{DAE8FC}3.46} & 0.46 \\
 & \multicolumn{1}{l|}{RAG-\textit{Doc}} & 3.90 & \cellcolor[HTML]{DAE8FC}\textbf{4.18} & \cellcolor[HTML]{DAE8FC}4.54 & \cellcolor[HTML]{DAE8FC}4.28 & \multicolumn{1}{r|}{3.86} & 3.74 & 4.10 & \cellcolor[HTML]{DAE8FC}4.52 & 4.03 & \multicolumn{1}{r|}{3.60} & \cellcolor[HTML]{DAE8FC}3.08 & \cellcolor[HTML]{DAE8FC}3.97 & \cellcolor[HTML]{DAE8FC}4.63 & \cellcolor[HTML]{DAE8FC}2.95 & \multicolumn{1}{r|}{\cellcolor[HTML]{DAE8FC}3.55} & 0.47 \\
 & \multicolumn{1}{l|}{Hierarchical} & 4.08 & \cellcolor[HTML]{DAE8FC}4.16 & \cellcolor[HTML]{DAE8FC}\textbf{4.55} & \cellcolor[HTML]{DAE8FC}4.28 & \multicolumn{1}{r|}{\cellcolor[HTML]{DAE8FC}4.05} & \cellcolor[HTML]{DAE8FC}3.94 & \cellcolor[HTML]{DAE8FC}4.21 & \cellcolor[HTML]{DAE8FC}4.56 & 4.07 & \multicolumn{1}{r|}{3.62} & \cellcolor[HTML]{DAE8FC}3.13 & \cellcolor[HTML]{DAE8FC}3.90 & \cellcolor[HTML]{DAE8FC}\textbf{4.64} & \cellcolor[HTML]{DAE8FC}3.06 & \multicolumn{1}{r|}{\cellcolor[HTML]{DAE8FC}\textbf{3.60}} & 0.47 \\
 & \multicolumn{1}{l|}{Increm-\textit{All}} & 3.81 & \cellcolor[HTML]{DAE8FC}4.04 & \cellcolor[HTML]{DAE8FC}4.50 & \cellcolor[HTML]{DAE8FC}4.25 & \multicolumn{1}{r|}{\cellcolor[HTML]{DAE8FC}3.93} & 3.65 & 4.08 & \cellcolor[HTML]{DAE8FC}4.51 & 3.79 & \multicolumn{1}{r|}{3.40} & \cellcolor[HTML]{DAE8FC}3.15 & \cellcolor[HTML]{DAE8FC}3.99 & \cellcolor[HTML]{DAE8FC}4.62 & \cellcolor[HTML]{DAE8FC}2.92 & \multicolumn{1}{r|}{\cellcolor[HTML]{DAE8FC}3.58} & 0.45 \\
 & \multicolumn{1}{l|}{Increm-\textit{Topic}} & \cellcolor[HTML]{DAE8FC}3.91 & \cellcolor[HTML]{DAE8FC}4.18 & \cellcolor[HTML]{DAE8FC}4.54 & \cellcolor[HTML]{DAE8FC}4.19 & \multicolumn{1}{r|}{\cellcolor[HTML]{DAE8FC}4.12} & \cellcolor[HTML]{DAE8FC}3.92 & \cellcolor[HTML]{DAE8FC}\textbf{4.25} & \cellcolor[HTML]{DAE8FC}4.57 & \cellcolor[HTML]{DAE8FC}4.14 & \multicolumn{1}{r|}{3.70} & 2.86 & 3.59 & 4.19 & 2.64 & \multicolumn{1}{r|}{3.09} & 0.48 \\
 & \multicolumn{1}{l|}{Cluster} & \cellcolor[HTML]{DAE8FC}3.91 & 4.01 & 4.35 & 4.09 & \multicolumn{1}{r|}{3.87} & 3.75 & 4.01 & 4.36 & 3.90 & \multicolumn{1}{r|}{3.50} & 2.72 & 3.40 & 4.03 & 2.38 & \multicolumn{1}{r|}{3.02} & 0.45 \\
\multirow{-10}{*}{2} & \multicolumn{1}{l|}{RAG+Cluster} & \cellcolor[HTML]{DAE8FC}3.98 & \cellcolor[HTML]{DAE8FC}4.11 & \cellcolor[HTML]{DAE8FC}4.44 & \cellcolor[HTML]{DAE8FC}4.27 & \multicolumn{1}{r|}{\cellcolor[HTML]{DAE8FC}3.99} & 3.72 & \cellcolor[HTML]{DAE8FC}4.17 & \cellcolor[HTML]{DAE8FC}4.56 & 4.04 & \multicolumn{1}{r|}{3.62} & 2.96 & 3.73 & \cellcolor[HTML]{DAE8FC}4.56 & 2.57 & \multicolumn{1}{r|}{3.26} & 0.48 \\ \midrule
 & \multicolumn{1}{l|}{\textbf{\modelTopic}} & \cellcolor[HTML]{DAE8FC}4.02 & \cellcolor[HTML]{DAE8FC}4.20 & 4.49 & \cellcolor[HTML]{DAE8FC}\textbf{4.44} & \multicolumn{1}{r|}{\cellcolor[HTML]{DAE8FC}4.34} & \cellcolor[HTML]{DAE8FC}\textbf{3.97} & \cellcolor[HTML]{DAE8FC}\textbf{4.21} & \cellcolor[HTML]{DAE8FC}\textbf{4.55} & \cellcolor[HTML]{DAE8FC}\textbf{4.14} & \multicolumn{1}{r|}{\cellcolor[HTML]{DAE8FC}\textbf{3.82}} & \cellcolor[HTML]{DAE8FC}3.54 & \cellcolor[HTML]{DAE8FC}4.09 & \cellcolor[HTML]{DAE8FC}4.64 & 3.39 & \multicolumn{1}{r|}{\cellcolor[HTML]{DAE8FC}3.93} & 0.67 \\
 & \multicolumn{1}{l|}{\textbf{\modelAll}} & \cellcolor[HTML]{DAE8FC}4.11 & \cellcolor[HTML]{DAE8FC}4.21 & \cellcolor[HTML]{DAE8FC}4.60 & \cellcolor[HTML]{DAE8FC}4.34 & \multicolumn{1}{r|}{\cellcolor[HTML]{DAE8FC}\textbf{4.36}} & \cellcolor[HTML]{DAE8FC}3.83 & \cellcolor[HTML]{DAE8FC}4.15 & \cellcolor[HTML]{DAE8FC}4.51 & \cellcolor[HTML]{DAE8FC}4.10 & \multicolumn{1}{r|}{3.63} & \cellcolor[HTML]{DAE8FC}\textbf{3.61} & \cellcolor[HTML]{DAE8FC}4.11 & \cellcolor[HTML]{DAE8FC}4.67 & \cellcolor[HTML]{DAE8FC}\textbf{3.71} & \multicolumn{1}{r|}{\cellcolor[HTML]{DAE8FC}4.02} & 0.64 \\
 & \multicolumn{1}{l|}{Long-Context} & \cellcolor[HTML]{DAE8FC}3.94 & \cellcolor[HTML]{DAE8FC}4.13 & \cellcolor[HTML]{DAE8FC}4.54 & 4.32 & \multicolumn{1}{r|}{4.14} & 3.54 & 4.09 & \cellcolor[HTML]{DAE8FC}4.46 & 3.80 & \multicolumn{1}{r|}{3.17} & \cellcolor[HTML]{DAE8FC}3.36 & \cellcolor[HTML]{DAE8FC}4.09 & \cellcolor[HTML]{DAE8FC}4.69 & 3.36 & \multicolumn{1}{r|}{\cellcolor[HTML]{DAE8FC}4.04} & 0.59 \\
 & \multicolumn{1}{l|}{RAG-\textit{All}} & \cellcolor[HTML]{DAE8FC}4.04 & \cellcolor[HTML]{DAE8FC}4.20 & \cellcolor[HTML]{DAE8FC}4.59 & \cellcolor[HTML]{DAE8FC}4.25 & \multicolumn{1}{r|}{4.14} & 3.62 & 4.06 & \cellcolor[HTML]{DAE8FC}4.49 & 3.87 & \multicolumn{1}{r|}{3.47} & \cellcolor[HTML]{DAE8FC}3.56 & \cellcolor[HTML]{DAE8FC}4.11 & \cellcolor[HTML]{DAE8FC}4.64 & 3.46 & \multicolumn{1}{r|}{\cellcolor[HTML]{DAE8FC}3.97} & 0.59 \\
 & \multicolumn{1}{l|}{RAG-\textit{Doc}} & \cellcolor[HTML]{DAE8FC}4.19 & \cellcolor[HTML]{DAE8FC}\textbf{4.25} & \cellcolor[HTML]{DAE8FC}4.59 & \cellcolor[HTML]{DAE8FC}4.33 & \multicolumn{1}{r|}{4.08} & 3.59 & 4.06 & \cellcolor[HTML]{DAE8FC}4.49 & 3.88 & \multicolumn{1}{r|}{3.36} & \cellcolor[HTML]{DAE8FC}3.56 & \cellcolor[HTML]{DAE8FC}4.10 & \cellcolor[HTML]{DAE8FC}4.62 & \cellcolor[HTML]{DAE8FC}3.51 & \multicolumn{1}{r|}{\cellcolor[HTML]{DAE8FC}3.97} & 0.59 \\
 & \multicolumn{1}{l|}{Hierarchical} & \cellcolor[HTML]{DAE8FC}4.15 & \cellcolor[HTML]{DAE8FC}4.17 & \cellcolor[HTML]{DAE8FC}\textbf{4.69} & \cellcolor[HTML]{DAE8FC}4.35 & \multicolumn{1}{r|}{\cellcolor[HTML]{DAE8FC}4.33} & 3.74 & 4.09 & \cellcolor[HTML]{DAE8FC}4.53 & 3.96 & \multicolumn{1}{r|}{3.48} & \cellcolor[HTML]{DAE8FC}3.56 & \cellcolor[HTML]{DAE8FC}\textbf{4.22} & \cellcolor[HTML]{DAE8FC}\textbf{4.70} & \cellcolor[HTML]{DAE8FC}3.63 & \multicolumn{1}{r|}{\cellcolor[HTML]{DAE8FC}\textbf{4.16}} & 0.58 \\
 & \multicolumn{1}{l|}{Increm-\textit{All}} & 3.92 & 4.08 & 4.52 & \cellcolor[HTML]{DAE8FC}4.29 & \multicolumn{1}{r|}{4.08} & 3.50 & 3.98 & \cellcolor[HTML]{DAE8FC}4.46 & 3.75 & \multicolumn{1}{r|}{3.25} & \cellcolor[HTML]{DAE8FC}3.36 & \cellcolor[HTML]{DAE8FC}4.12 & \cellcolor[HTML]{DAE8FC}4.61 & 3.25 & \multicolumn{1}{r|}{3.75} & 0.58 \\
 & \multicolumn{1}{l|}{Increm-\textit{Topic}} & \cellcolor[HTML]{DAE8FC}\textbf{4.25} & \cellcolor[HTML]{DAE8FC}4.19 & \cellcolor[HTML]{DAE8FC}4.61 & \cellcolor[HTML]{DAE8FC}4.41 & \multicolumn{1}{r|}{\cellcolor[HTML]{DAE8FC}4.23} & \cellcolor[HTML]{DAE8FC}3.91 & \cellcolor[HTML]{DAE8FC}4.17 & \cellcolor[HTML]{DAE8FC}\textbf{4.55} & \cellcolor[HTML]{DAE8FC}4.06 & \multicolumn{1}{r|}{\cellcolor[HTML]{DAE8FC}3.68} & 3.09 & 3.66 & 4.30 & 3.03 & \multicolumn{1}{r|}{3.56} & 0.60 \\
 & \multicolumn{1}{l|}{Cluster} & 3.92 & 3.97 & 4.34 & 4.08 & \multicolumn{1}{r|}{4.06} & 3.64 & 3.95 & 4.31 & 3.82 & \multicolumn{1}{r|}{3.39} & 2.67 & 3.41 & 3.97 & 2.53 & \multicolumn{1}{r|}{3.16} & 0.59 \\
\multirow{-10}{*}{3} & \multicolumn{1}{l|}{RAG+Cluster} & \cellcolor[HTML]{DAE8FC}4.11 & \cellcolor[HTML]{DAE8FC}4.16 & 4.49 & \cellcolor[HTML]{DAE8FC}4.37 & \multicolumn{1}{r|}{\cellcolor[HTML]{DAE8FC}4.23} & \cellcolor[HTML]{DAE8FC}3.83 & \cellcolor[HTML]{DAE8FC}4.18 & \cellcolor[HTML]{DAE8FC}4.54 & \cellcolor[HTML]{DAE8FC}4.11 & \multicolumn{1}{r|}{\cellcolor[HTML]{DAE8FC}3.69} & 3.08 & 3.80 & 4.40 & 2.87 & \multicolumn{1}{r|}{3.31} & 0.61 \\ \midrule
 & \multicolumn{1}{l|}{\textbf{\modelTopic}} & \cellcolor[HTML]{DAE8FC}4.15 & \cellcolor[HTML]{DAE8FC}4.08 & \cellcolor[HTML]{DAE8FC}4.45 & \cellcolor[HTML]{DAE8FC}4.37 & \multicolumn{1}{r|}{\cellcolor[HTML]{DAE8FC}\textbf{4.40}} & \cellcolor[HTML]{DAE8FC}\textbf{4.06} & \cellcolor[HTML]{DAE8FC}\textbf{4.20} & \cellcolor[HTML]{DAE8FC}4.54 & \cellcolor[HTML]{DAE8FC}\textbf{4.20} & \multicolumn{1}{r|}{\cellcolor[HTML]{DAE8FC}\textbf{3.94}} & \cellcolor[HTML]{DAE8FC}3.80 & \cellcolor[HTML]{DAE8FC}4.12 & \cellcolor[HTML]{DAE8FC}4.68 & \cellcolor[HTML]{DAE8FC}4.11 & \multicolumn{1}{r|}{\cellcolor[HTML]{DAE8FC}4.19} & 0.71 \\
 & \multicolumn{1}{l|}{\textbf{\modelAll}} & \cellcolor[HTML]{DAE8FC}\textbf{4.21} & \cellcolor[HTML]{DAE8FC}4.14 & \cellcolor[HTML]{DAE8FC}\textbf{4.48} & \cellcolor[HTML]{DAE8FC}\textbf{4.39} & \multicolumn{1}{r|}{\cellcolor[HTML]{DAE8FC}4.30} & 3.82 & \cellcolor[HTML]{DAE8FC}4.12 & \cellcolor[HTML]{DAE8FC}4.49 & 4.02 & \multicolumn{1}{r|}{3.68} & \cellcolor[HTML]{DAE8FC}\textbf{3.93} & \cellcolor[HTML]{DAE8FC}\textbf{4.18} & 4.58 & \cellcolor[HTML]{DAE8FC}4.07 & \multicolumn{1}{r|}{\cellcolor[HTML]{DAE8FC}4.21} & 0.69 \\
 & \multicolumn{1}{l|}{Long-Context} & 3.92 & \cellcolor[HTML]{DAE8FC}4.07 & \cellcolor[HTML]{DAE8FC}4.40 & 4.15 & \multicolumn{1}{r|}{4.14} & 3.48 & 4.04 & 4.43 & 3.70 & \multicolumn{1}{r|}{3.07} & \cellcolor[HTML]{DAE8FC}3.83 & \cellcolor[HTML]{DAE8FC}4.14 & 4.56 & \cellcolor[HTML]{DAE8FC}4.02 & \multicolumn{1}{r|}{\cellcolor[HTML]{DAE8FC}4.21} & 0.65 \\
 & \multicolumn{1}{l|}{RAG-\textit{All}} & 3.93 & \cellcolor[HTML]{DAE8FC}4.04 & \cellcolor[HTML]{DAE8FC}4.36 & \cellcolor[HTML]{DAE8FC}4.27 & \multicolumn{1}{r|}{\cellcolor[HTML]{DAE8FC}4.18} & 3.55 & 4.02 & 4.45 & 3.83 & \multicolumn{1}{r|}{3.30} & \cellcolor[HTML]{DAE8FC}3.79 & \cellcolor[HTML]{DAE8FC}4.16 & \cellcolor[HTML]{DAE8FC}4.64 & \cellcolor[HTML]{DAE8FC}4.02 & \multicolumn{1}{r|}{\cellcolor[HTML]{DAE8FC}4.21} & 0.66 \\
 & \multicolumn{1}{l|}{RAG-\textit{Doc}} & 3.96 & \cellcolor[HTML]{DAE8FC}3.99 & \cellcolor[HTML]{DAE8FC}4.31 & \cellcolor[HTML]{DAE8FC}4.34 & \multicolumn{1}{r|}{\cellcolor[HTML]{DAE8FC}4.24} & 3.64 & 4.05 & \cellcolor[HTML]{DAE8FC}4.51 & 3.87 & \multicolumn{1}{r|}{3.31} & \cellcolor[HTML]{DAE8FC}3.80 & \cellcolor[HTML]{DAE8FC}4.08 & \cellcolor[HTML]{DAE8FC}4.63 & \cellcolor[HTML]{DAE8FC}\textbf{4.15} & \multicolumn{1}{r|}{\cellcolor[HTML]{DAE8FC}4.14} & 0.66 \\
 & \multicolumn{1}{l|}{Hierarchical} & \cellcolor[HTML]{DAE8FC}4.05 & \cellcolor[HTML]{DAE8FC}\textbf{4.16} & \cellcolor[HTML]{DAE8FC}4.44 & \cellcolor[HTML]{DAE8FC}4.34 & \multicolumn{1}{r|}{\cellcolor[HTML]{DAE8FC}4.37} & 3.63 & 4.07 & \cellcolor[HTML]{DAE8FC}4.49 & 3.87 & \multicolumn{1}{r|}{3.44} & \cellcolor[HTML]{DAE8FC}3.80 & \cellcolor[HTML]{DAE8FC}4.15 & \cellcolor[HTML]{DAE8FC}\textbf{4.75} & \cellcolor[HTML]{DAE8FC}4.04 & \multicolumn{1}{r|}{\cellcolor[HTML]{DAE8FC}\textbf{4.28}} & 0.66 \\
 & \multicolumn{1}{l|}{Increm-\textit{All}} & 3.93 & \cellcolor[HTML]{DAE8FC}4.06 & \cellcolor[HTML]{DAE8FC}4.36 & \cellcolor[HTML]{DAE8FC}4.19 & \multicolumn{1}{r|}{\cellcolor[HTML]{DAE8FC}4.19} & 3.45 & 4.02 & 4.45 & 3.68 & \multicolumn{1}{r|}{3.24} & \cellcolor[HTML]{DAE8FC}3.82 & \cellcolor[HTML]{DAE8FC}4.12 & \cellcolor[HTML]{DAE8FC}4.66 & \cellcolor[HTML]{DAE8FC}4.09 & \multicolumn{1}{r|}{\cellcolor[HTML]{DAE8FC}4.09} & 0.65 \\
 & \multicolumn{1}{l|}{Increm-\textit{Topic}} & \cellcolor[HTML]{DAE8FC}4.05 & \cellcolor[HTML]{DAE8FC}4.08 & 4.25 & \cellcolor[HTML]{DAE8FC}4.34 & \multicolumn{1}{r|}{\cellcolor[HTML]{DAE8FC}4.33} & 3.90 & \cellcolor[HTML]{DAE8FC}4.18 & \cellcolor[HTML]{DAE8FC}\textbf{4.56} & \cellcolor[HTML]{DAE8FC}4.12 & \multicolumn{1}{r|}{3.67} & 3.61 & \cellcolor[HTML]{DAE8FC}3.98 & 4.39 & 3.77 & \multicolumn{1}{r|}{3.93} & 0.69 \\
 & \multicolumn{1}{l|}{Cluster} & \cellcolor[HTML]{DAE8FC}3.97 & \cellcolor[HTML]{DAE8FC}4.01 & 4.17 & 4.05 & \multicolumn{1}{r|}{4.15} & 3.70 & 4.00 & 4.32 & 3.82 & \multicolumn{1}{r|}{3.48} & 3.01 & 3.56 & 4.07 & 3.02 & \multicolumn{1}{r|}{3.37} & 0.66 \\
\multirow{-10}{*}{4} & \multicolumn{1}{l|}{RAG+Cluster} & \cellcolor[HTML]{DAE8FC}4.15 & \cellcolor[HTML]{DAE8FC}3.97 & \cellcolor[HTML]{DAE8FC}4.32 & \cellcolor[HTML]{DAE8FC}4.23 & \multicolumn{1}{r|}{\cellcolor[HTML]{DAE8FC}4.26} & 3.83 & 4.09 & \cellcolor[HTML]{DAE8FC}4.54 & 4.04 & \multicolumn{1}{r|}{3.56} & 3.29 & 3.60 & 4.17 & 3.24 & \multicolumn{1}{r|}{3.50} & 0.66 \\ \midrule
 & \multicolumn{1}{l|}{\textbf{\modelTopic}} & \cellcolor[HTML]{DAE8FC}\textbf{4.16} & \cellcolor[HTML]{DAE8FC}\textbf{4.14} & \cellcolor[HTML]{DAE8FC}4.36 & \cellcolor[HTML]{DAE8FC}\textbf{4.28} & \multicolumn{1}{r|}{\cellcolor[HTML]{DAE8FC}\textbf{4.40}} & \cellcolor[HTML]{DAE8FC}\textbf{4.05} & \cellcolor[HTML]{DAE8FC}\textbf{4.25} & \cellcolor[HTML]{DAE8FC}\textbf{4.58} & \cellcolor[HTML]{DAE8FC}\textbf{4.27} & \multicolumn{1}{r|}{\cellcolor[HTML]{DAE8FC}\textbf{3.89}} & \cellcolor[HTML]{DAE8FC}4.04 & \cellcolor[HTML]{DAE8FC}4.37 & \cellcolor[HTML]{DAE8FC}4.77 & \cellcolor[HTML]{DAE8FC}4.33 & \multicolumn{1}{r|}{\cellcolor[HTML]{DAE8FC}4.49} & 0.75 \\
 & \multicolumn{1}{l|}{\textbf{\modelAll}} & \cellcolor[HTML]{DAE8FC}4.07 & \cellcolor[HTML]{DAE8FC}4.03 & \cellcolor[HTML]{DAE8FC}\textbf{4.37} & \cellcolor[HTML]{DAE8FC}4.25 & \multicolumn{1}{r|}{\cellcolor[HTML]{DAE8FC}4.29} & 3.79 & 4.09 & 4.48 & 3.99 & \multicolumn{1}{r|}{3.59} & \cellcolor[HTML]{DAE8FC}4.05 & \cellcolor[HTML]{DAE8FC}4.35 & \cellcolor[HTML]{DAE8FC}4.81 & \cellcolor[HTML]{DAE8FC}4.38 & \multicolumn{1}{r|}{\cellcolor[HTML]{DAE8FC}\textbf{4.54}} & 0.73 \\
 & \multicolumn{1}{l|}{Long-Context} & 3.79 & \cellcolor[HTML]{DAE8FC}3.99 & \cellcolor[HTML]{DAE8FC}4.20 & \cellcolor[HTML]{DAE8FC}4.14 & \multicolumn{1}{r|}{3.99} & 3.47 & 4.01 & 4.44 & 3.69 & \multicolumn{1}{r|}{3.05} & \cellcolor[HTML]{DAE8FC}4.07 & 4.27 & 4.74 & 4.23 & \multicolumn{1}{r|}{\cellcolor[HTML]{DAE8FC}4.40} & 0.70 \\
 & \multicolumn{1}{l|}{RAG-\textit{All}} & 3.90 & \cellcolor[HTML]{DAE8FC}3.91 & \cellcolor[HTML]{DAE8FC}4.23 & \cellcolor[HTML]{DAE8FC}4.14 & \multicolumn{1}{r|}{4.15} & 3.59 & 4.00 & 4.44 & 3.72 & \multicolumn{1}{r|}{3.33} & \cellcolor[HTML]{DAE8FC}\textbf{4.17} & \cellcolor[HTML]{DAE8FC}\textbf{4.44} & \cellcolor[HTML]{DAE8FC}4.87 & \cellcolor[HTML]{DAE8FC}4.36 & \multicolumn{1}{r|}{\cellcolor[HTML]{DAE8FC}4.52} & 0.70 \\
 & \multicolumn{1}{l|}{RAG-\textit{Doc}} & \cellcolor[HTML]{DAE8FC}3.93 & \cellcolor[HTML]{DAE8FC}3.98 & \cellcolor[HTML]{DAE8FC}4.30 & \cellcolor[HTML]{DAE8FC}4.23 & \multicolumn{1}{r|}{4.14} & 3.61 & 4.05 & 4.47 & 3.81 & \multicolumn{1}{r|}{3.31} & \cellcolor[HTML]{DAE8FC}4.05 & \cellcolor[HTML]{DAE8FC}4.43 & \cellcolor[HTML]{DAE8FC}4.84 & \cellcolor[HTML]{DAE8FC}\textbf{4.50} & \multicolumn{1}{r|}{\cellcolor[HTML]{DAE8FC}4.50} & 0.70 \\
 & \multicolumn{1}{l|}{Hierarchical} & 3.90 & \cellcolor[HTML]{DAE8FC}3.96 & \cellcolor[HTML]{DAE8FC}4.23 & \cellcolor[HTML]{DAE8FC}4.16 & \multicolumn{1}{r|}{4.09} & 3.60 & 4.09 & 4.48 & 3.85 & \multicolumn{1}{r|}{3.38} & \cellcolor[HTML]{DAE8FC}4.16 & \cellcolor[HTML]{DAE8FC}4.43 & \cellcolor[HTML]{DAE8FC}\textbf{4.87} & \cellcolor[HTML]{DAE8FC}4.52 & \multicolumn{1}{r|}{\cellcolor[HTML]{DAE8FC}4.52} & 0.70 \\
 & \multicolumn{1}{l|}{Increm-\textit{All}} & 3.80 & \cellcolor[HTML]{DAE8FC}4.07 & \cellcolor[HTML]{DAE8FC}4.23 & \cellcolor[HTML]{DAE8FC}4.09 & \multicolumn{1}{r|}{4.04} & 3.41 & 3.95 & 4.40 & 3.56 & \multicolumn{1}{r|}{3.09} & \cellcolor[HTML]{DAE8FC}4.08 & 4.26 & \cellcolor[HTML]{DAE8FC}4.76 & \cellcolor[HTML]{DAE8FC}4.36 & \multicolumn{1}{r|}{\cellcolor[HTML]{DAE8FC}4.37} & 0.68 \\
 & \multicolumn{1}{l|}{Increm-\textit{Topic}} & \cellcolor[HTML]{DAE8FC}4.04 & \cellcolor[HTML]{DAE8FC}4.10 & \cellcolor[HTML]{DAE8FC}4.21 & \cellcolor[HTML]{DAE8FC}4.16 & \multicolumn{1}{r|}{\cellcolor[HTML]{DAE8FC}4.16} & 3.84 & 4.16 & \cellcolor[HTML]{DAE8FC}4.55 & 4.06 & \multicolumn{1}{r|}{3.56} & 3.69 & 3.98 & 4.51 & 3.88 & \multicolumn{1}{r|}{3.99} & 0.73 \\
 & \multicolumn{1}{l|}{Cluster} & 3.86 & 3.90 & 3.96 & 3.98 & \multicolumn{1}{r|}{4.12} & 3.68 & 4.01 & 4.36 & 3.87 & \multicolumn{1}{r|}{3.42} & 3.14 & 3.60 & 4.19 & 3.31 & \multicolumn{1}{r|}{3.52} & 0.71 \\
\multirow{-10}{*}{5} & \multicolumn{1}{l|}{RAG+Cluster} & \cellcolor[HTML]{DAE8FC}4.03 & \cellcolor[HTML]{DAE8FC}3.96 & \cellcolor[HTML]{DAE8FC}4.19 & \cellcolor[HTML]{DAE8FC}4.19 & \multicolumn{1}{r|}{\cellcolor[HTML]{DAE8FC}4.21} & 3.79 & 4.10 & 4.50 & 4.04 & \multicolumn{1}{r|}{3.57} & 3.46 & 3.86 & 4.40 & 3.62 & \multicolumn{1}{r|}{3.69} & 0.72 \\ \bottomrule
\end{tabular}
\caption{\label{appendix:table:llm_debate} Interest, Coherence, Relevance, Coverage, and Diversity scores from Prometheus for summaries, topic paragraphs, and topics on DebateQFS. Best scores are \textbf{bold}, significant scores in \colorbox{myblue}{blue} (2-sample $t$-test, $p<0.05$)}
\end{table*}
\begin{table*}[]
\small
\centering
\setlength{\tabcolsep}{3.5pt}
\renewcommand{\arraystretch}{0.8}
\begin{tabular}{@{}cl|ccccc@{}}
\toprule
\textbf{\# Topics} & \textbf{Model} & \multicolumn{1}{l}{\textbf{\# Input Tokens}} & \multicolumn{1}{l}{\textbf{\# Output Tokens}} & \multicolumn{1}{l}{\textbf{\# LLM Calls}} & \multicolumn{1}{l}{\textbf{Cost (GPT-4)}} & \multicolumn{1}{l}{\textbf{Time (seconds)}} \\ \midrule
\multirow{3}{*}{2} & \modelTopic & 21383.08 & 3412.02 & 25.45 & 0.32 & 117.60 \\
 & Hierarchical & 31130.02 & 2536.66 & 13.15 & 0.39 & 83.13 \\
 & Incremental-\textit{Topic} & 59010.66 & 6115.04 & 15.15 & 0.77 & 214.39 \\ \midrule
\multirow{3}{*}{3} & \modelTopic & 30208.20 & 5040.38 & 37.38 & 0.45 & 149.54 \\
 & Hierarchical & 31144.83 & 2649.78 & 13.15 & 0.39 & 68.60 \\
 & Incremental-\textit{Topic} & 61344.07 & 8442.54 & 16.15 & 0.87 & 197.33 \\ \midrule
\multirow{3}{*}{4} & \modelTopic & 38286.40 & 6440.23 & 47.91 & 0.58 & 163.91 \\
 & Hierarchical & 31144.31 & 2740.31 & 13.15 & 0.39 & 88.75 \\
 & Incremental-\textit{Topic} & 62877.46 & 9966.45 & 17.15 & 0.93 & 312.55 \\ \midrule
\multirow{3}{*}{5} & \modelTopic & 47008.59 & 7918.92 & 58.94 & 0.71 & 186.32 \\
 & Hierarchical & 31160.88 & 2850.24 & 13.15 & 0.40 & 61.70 \\
 & Incremental-\textit{Topic} & 64893.95 & 11965.84 & 18.15 & 1.01 & 262.07 \\ \bottomrule
\end{tabular}
\caption{\label{appendix:table:cost_cqa} Number of LLM input/output tokens, LLM calls, GPT-4 Cost (USD), and Time (seconds) needed to run inference on a single DFQS example on ConflictingQA with the top-3 models. We report 5 runs and 20 examples.}
\end{table*}

\begin{table*}[]
\small
\centering
\setlength{\tabcolsep}{3.5pt}
\renewcommand{\arraystretch}{0.8}
\begin{tabular}{@{}cl|ccccc@{}}
\toprule
\multicolumn{1}{l}{\textbf{Dataset}} & \textbf{Model} & \multicolumn{1}{l}{\textbf{\# Input Tokens}} & \multicolumn{1}{l}{\textbf{\# Output Tokens}} & \multicolumn{1}{l}{\textbf{\# LLM Calls}} & \multicolumn{1}{l}{\textbf{Cost (GPT-4)}} & \multicolumn{1}{l}{\textbf{Time (seconds)}} \\ \midrule
\multirow{3}{*}{2} & \modelTopic & 17183.75 & 2722.40 & 20.30 & 0.25 & 94.81 \\
 & Hierarchical & 19181.59 & 2040.39 & 10.25 & 0.25 & 63.68 \\
 & Incremental-\textit{Topic} & 41656.87 & 5062.44 & 12.25 & 0.57 & 182.19 \\ 
 \midrule
\multirow{3}{*}{3} & \modelTopic & 24801.22 & 4136.12 & 30.40 & 0.37 & 126.83 \\
 & Hierarchical & 19182.58 & 2141.91 & 10.25 & 0.26 & 53.32 \\
 & Incremental-\textit{Topic} & 43119.51 & 6532.92 & 13.25 & 0.63 & 152.44 \\ \midrule
\multirow{3}{*}{4} & \modelTopic & 30677.67 & 5037.31 & 38.00 & 0.46 & 120.64 \\
 & Hierarchical & 19203.30 & 2253.17 & 10.25 & 0.26 & 73.35 \\
 & Incremental-\textit{Topic} & 43922.02 & 7327.88 & 14.25 & 0.66 & 241.54 \\ \midrule
\multirow{3}{*}{5} & \modelTopic & 36988.41 & 6049.93 & 46.09 & 0.55 & 139.71 \\
 & Hierarchical & 19211.74 & 2356.01 & 10.25 & 0.26 & 49.41 \\
 & Incremental-\textit{Topic} & 45113.12 & 8504.59 & 15.25 & 0.71 & 186.40 \\ \bottomrule
\end{tabular}
\caption{\label{appendix:table:cost_debate} Number of LLM input/output tokens, LLM calls, GPT-4 Cost (USD), and Time (seconds) needed to run inference on a single DFQS example on DebateQFS with the top-3 models. We report 5 runs and 20 examples.}
\end{table*}

\begin{table*}[]
\small
\centering
\setlength{\tabcolsep}{3.5pt}
\renewcommand{\arraystretch}{0.8}
\begin{tabular}{@{}cl|ccccc@{}}
\toprule
\multicolumn{1}{l}{\textbf{\# Topics}} & \textbf{Model} & \multicolumn{1}{l}{\textbf{\# Input Tokens}} & \multicolumn{1}{l}{\textbf{\# Output Tokens}} & \multicolumn{1}{l}{\textbf{\# LLM Calls}} & \multicolumn{1}{l}{\textbf{Cost (GPT-4)}} & \multicolumn{1}{l}{\textbf{Time (seconds)}} \\ 
\midrule
\multirow{3}{*}{ConflictingQA} & \modelTopic & 47008.59 & 7918.92 & 58.94 & 0.71 & 186.32 \\
 & \modelTopic Pick All & 53733.70 & 9596.75 & 71.75 & 0.83 & 303.13 \\
 & Hierarchical-\emph{Topic} & 168160.85 & 7485.50 & 66.75 & 1.91 & 210.80 \\ \midrule
\multirow{3}{*}{DebateQFS} & \modelTopic & 36988.41 & 6049.93 & 46.09 & 0.55 & 139.71 \\
& \modelTopic Pick All & 43098.85 & 7612.45 & 57.25 & 0.66 & 242.35 \\
& Hierarchical-\emph{Topic} & 105237.25 & 5278.35 & 52.25 & 1.21 & 139.96 \\ \bottomrule
\end{tabular}
\caption{\label{appendix:table:cost_weird} Number of LLM input/output tokens, LLM calls, GPT-4 Cost (USD), and Time (seconds) needed to run inference on a single DFQS example on ConflictingQA and DebateQFS with \modelTopic, the version of \modelTopic with no Moderator, and the version of Hierarchical merging that runs on each topic paragraph ($m=5$). We report 5 runs and 20 examples.}
\end{table*}


\begin{figure*}
    \centering
    \fbox{
    \includegraphics[width=\linewidth]{appendix/annot2.pdf}}
    \caption{\label{fig:annot} Distribution of Readability and Balance for Full Summaries and Topic Paragraphs from Prolific.}
\end{figure*}


\clearpage
\hypersetup{
    colorlinks=true, % Enable colored links
    linkcolor=black, % Default color for internal links (sections, etc.)
    citecolor=black, % Default color for citations
    urlcolor=black % Default color for external URLs
}

\begin{summary}[title={\modelTopic Summary: Are Audiobooks Considered Real Reading? (ConflictingQA)}, label=summary1]
\textbf{Topic 1: Audiobooks and cognitive engagement}\\
Audiobooks provide cognitive benefits similar to traditional reading, such as improved vocabulary, comprehension, and critical thinking skills, and engage the same cognitive processes as reading print books [1][2][3]. While listening to audiobooks, children and adults can visualize, make predictions, and synthesize information, indicating active cognitive engagement, although the lack of visual stimulation may affect the engagement for some individuals [3][4][8]. High-quality production and narration in audiobooks can enhance the cognitive experience, making them a valid form of reading for people with visual impairments, dyslexia, or those who multitask [7][9][11].\\
\\
\textbf{Topic 2: Audiobooks accessibility and inclusivity}\\
Audiobooks enhance accessibility and inclusivity by supporting children with different learning needs, allowing individuals with ADHD, dyslexia, and other attention disorders to enjoy literature and improve comprehension [1][2][11]. They enable multitasking and comprehension for various audiences, including those with visual impairments or who find traditional reading challenging, and are cost-effective to produce, increasing their accessibility [7][12]. Despite some cultural biases that may undervalue audiobooks compared to printed text, they are considered a valid form of reading, especially for those who may not engage with traditional reading methods [9][8].\\
\\
\textbf{Topic 3: Audiobooks versus physical reading}\\
Audiobooks provide similar cognitive benefits to traditional reading, such as improved vocabulary and comprehension, and engage the same cognitive processes as reading with eyes, making them a valid form of reading [1][2][3]. They are particularly beneficial for those with different learning needs or physical limitations, allowing for a rich, accessible, and multitask-friendly reading experience [3][6][7][11]. Despite some cultural perceptions that audiobooks are less valuable than printed text, they are increasingly popular, economically significant, and considered a legitimate form of storytelling and information absorption [8][9][12].
\end{summary}

\begin{summary}[title={\modelTopic Summary: Is infinite scrolling a good web design technique? (ConflictingQA)}, label=summary2]
\textbf{Topic 1: User Engagement Increase}\\
Infinite scrolling can potentially increase user engagement and time spent on a site by continuously loading new content, which exploits users' automatic behavior and keeps them engaged [9]. However, it has been found to decrease user engagement in some cases, such as on Etsy, and can negatively impact users with disabilities and mental health, leading to a high cognitive load and potential mental health issues [1][5][8]. Additionally, infinite scrolling can lead to control issues and user frustration due to less controllable pages and jumping glitches [6].\\
\\
\textbf{Topic 2: Content Accessibility Concerns}\\
Infinite scrolling can lead to content accessibility issues, as it breaks the expected behavior of scrollbars and makes it difficult for users to gauge the length of the page, and it poses significant challenges for users with assistive technologies, often excluding footers and making navigation stressful [1][6][7]. While it can keep users engaged on eCommerce platforms, it has been associated with increased stress levels and negative mental health outcomes, particularly in young social media users [3][6][8]. Moreover, strategies like role='feed' have failed to address these accessibility problems effectively [5].\\
\\
\textbf{Topic 3: Mental Health Implications}\\
Infinite scrolling can exploit human psychological phenomena such as automaticity, leading to behaviors like doom-scrolling that may contribute to mental health issues by causing users to lose track of time and continue scrolling unconsciously [9]. The design can also induce stress by preventing users from reaching a perceived end, leading to information overload, and overwhelming them with choices, which can result in frustration, anxiety, and a reduced motivation to engage with content [6][7]. However, some studies suggest that engaging in mindful scrolling practices can mitigate these negative mental health outcomes, indicating that the impact of infinite scrolling may vary based on user behavior [8].
\end{summary}

\clearpage

\begin{summary}[title={\modelTopic Summary: Is EU expansion and EU membership itself a good idea? (DebateQFS)}, label=summary3]
\textbf{Topic 1: Economic gains from accession}\\
The 1997 study by the Centre for Economic Policy Research predicted economic gains for both the EU-15 and new Central and Eastern European members, with an estimated €10 billion and €23 billion increase respectively [1]. However, concerns about high budget and trade deficits in accession countries, such as Estonia and Hungary, and the potential for increased unemployment and social costs, suggest that EU expansion could also exacerbate economic disparities and put fiscal pressure on both new and existing members [5][6]. Additionally, the enlargement is expected to shift regional funds towards new members, potentially reducing support for poorer regions within the EU(15) and necessitating a significant increase in the EU's regional funding budget to address growing economic and social needs [6].\\
\\
\textbf{Topic 2: EU enlargement political challenges}\\
EU enlargement is seen as beneficial, with studies indicating potential GDP growth for new and existing members, strategic interests in stabilizing regions like the Western Balkans and Turkey, and necessary controls in place to manage economic migration and regional subsidies [1][2][6]. However, public opposition in some member states, the slow process of enlargement due to political complexities, and concerns over social contradictions and international conflicts [2][5][6] present significant challenges. The Treaty of Lisbon is deemed necessary for further enlargement, although there are differing opinions on whether its ratification should delay the process [4].\\
\\
\textbf{Topic 3: Regional disparities and funding}\\
EU expansion has been estimated to bring economic gains for both old and new member states, with the EU-15 seeing a €10 billion increase and new Central and Eastern European members gaining €23 billion [1]. However, regional disparities pose challenges, as unemployment rates have risen in accession countries and the wealth gap between regions may widen, with 98 million inhabitants in applicant states living in regions with GDP less than 75\% of the EU average [5][6]. Despite the potential for increased regional funding, there are concerns that existing poorer regions within the EU(15) may receive less support as a result of the expansion [6].
\end{summary}

\begin{summary}[title={\modelTopic Summary: Is going to law school a good idea? (DebateQFS)}, label=summary4]

\textbf{Topic 1: Law School ROI Analysis}\\
Attending law school can lead to a variety of career opportunities and the acquisition of valuable skills, with some graduates finding employment directly from campus and others benefiting from practical skills-oriented courses [4][5][6]. However, the financial burden of law school is significant, with many students accruing substantial debt, facing uncertain job markets, and questioning the return on investment, especially if they do not graduate from top-tier schools or are not at the top of their class [8][10][15][19]. Despite the potential for high starting salaries in some legal jobs, the competitiveness of the market and the cost of tuition may not justify the investment for all students, particularly when considering the psychological toll and the oversupply of law graduates [12][14][16].\\
\\
\textbf{Topic 2: Legal Career Job Market}\\
The legal job market presents a mixed outlook, with some documents indicating an increase in law firm hiring practices and a demand for legal services in certain areas, while others highlight the oversaturation of law graduates, underemployment, and the potential for job dissatisfaction and misleading employment statistics from law schools [4][17][8][10][11][18][19]. Graduates from prestigious law schools or those in the top of their class may have better job prospects and higher starting salaries, but many face significant debt and struggle to find well-paying jobs to manage that debt [16][19]. The rise of legal process outsourcing and the hiring of law school graduates directly by companies suggest evolving trends in the legal job market that could affect future employment opportunities for lawyers [6][5].\\
\\
\textbf{Topic 3: Law Education Value Debate}\\
Law school provides a range of non-monetary benefits, such as personal growth, maturity, and the development of transferable skills like critical thinking and argumentation, which are applicable in various fields beyond traditional legal practice [3][9]. However, the financial implications of law school, including high tuition costs, significant student debt, and an uncertain job market, challenge the notion that a legal education is a sound financial investment for all students [13][14][17][19]. Despite these concerns, there is a demand for legal professionals, and law school can prepare graduates for diverse career paths, including roles that address complex societal challenges and ensure access to justice [2][7][16].
\end{summary}