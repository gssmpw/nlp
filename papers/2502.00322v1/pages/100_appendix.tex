
\section{Appendix} \label{appendix}

\subsection{Dataset Details} \label{appendix:data}

To collect a dataset based on Debatepedia~\cite{gottopati2013learning}, we use Wayback Machine\footnote{\url{https://web.archive.org/}}, as the original website is no longer hosted.
We iterate through each debate articles page on the website with BeautifulSoup\footnote{\url{https://pypi.org/project/beautifulsoup4/}} and collect the 1) topic of the debate; 2) list of URLs under ``Supporting References''; and 3) list of URLs under ``Refuting References''.
We then use jusText\footnote{\url{https://pypi.org/project/jusText/}} to extract the text content from each web page, ignoring websites that are not free-to-access.

After this, we filter out instances that have less than five sources or do not have at least a 75/25 majority/minority split of perspective labels.
We then remove web pages that do not have any of the non-stopword tokens in the query, implemented with nltk, to ensure the web pages form a set of relevant documents.
We run this same process on ConflictingQA~\cite{wan2024evidence}.

Dataset statistics after data processing are in Table~\ref{appendix:table:dataset}.
Since all websites were publicly-accessible, our collected artifacts are within their intended use and licenses.
We sampled a subset of five document collections and manually checked them for PII and offensive content, which we did not find; we also found all text to be in English.

\subsection{The \model Algorithm} \label{subsection:model}

We detail \model in Algorithm~\ref{algo:mods}. For a debatable query $q$, document collection $\mathcal{D}$, number of topics $m$, and retrieval parameter $k$, we create speakers $\mathcal{S}$ for $\mathcal{D}$.
First, we retrieve speaker biographies $\mathcal{B}$ related to $q$ and plan $m$ topics $\mathcal{T}$ for $\mathcal{O}$ (\cref{subsection:agenda}). For each topic $t_j \in \mathcal{T}$, we pick relevant speakers $\mathcal{S}_j \subseteq \mathcal{S}$ and tailor them questions $\mathcal{Q}_j$ using their topic biographies $\mathcal{B}_j$ (\cref{subsection:moderator}). Each speaker supplies stance/fact perspectives $\mathcal{P}$, which are tracked in $\mathcal{O}$ (\cref{subsection:speaker}).
Finally, $\mathcal{O}$ is summarized all at once ($\mathbb{S}_{all}$) or per topic ($\mathbb{S}_{top}$) and returned to the user (\cref{subsection:summary}).

\subsection{Experimental Setup Details} \label{appendix:implementation}

All of our baseline implementations use GPT-4 (\texttt{gpt-4-1106-preview}) with 0 temperature and a maximum input token length of 127,000 tokens.
All baselines use zero-shot prompting, and the prompts will be released with our code after internal approval.
For costs associated with using GPT-4, see Appendix~\ref{subsection:efficiency}.

All models using retrieval, including \model, use ColBERT~\cite{khattab2020colbert}, a state-of-the-art retriever. For hyperparameters, we use a maximum document length of 300 tokens, a maximum query length of 64 tokens, 8 bits, and the \texttt{colbert-ir/colbertv2.0} checkpoint; none of these parameters were tuned during experimentation.
The clustering methods were implemented with BERTopic~\cite{grootendorst2022bertopic}, using all default values.

All experiments were run on a single H100 GPU, but as the only GPU usage comes from retrieval, we found \model and all baselines can be run on a Google Collaboratory T4 GPU (16GB of GPU memory).
Each baseline was allocated 24 hours for a single run.
We give more details about the runtime of \model in Appendix~\ref{subsection:efficiency}.

% Add prompts

\subsection{Metric Details} \label{appendix:metrics}

We extract all citations via regex\footnote{\url{https://docs.python.org/3/library/re.html}} by first finding text between square brackets (\texttt{[} and \texttt{]}) and then extracting integers between these spans.
The document coverage, faithfulness, and fairness metrics are all implemented with numpy\footnote{\url{https://numpy.org/}}.

We implement citation accuracy through entailment; entailment has shown to be a viable strategy to measure the factuality of text~\cite{maynez2020faithfulness}.
We use GPT-3.5 (\texttt{gpt-35-turbo-1106}) with 0 temperature to classify whether a generated sentence is entailed by the document it cited, using a 0-shot prompt shown in Prompt~\ref{prompt:cite_acc}
To evaluate the accuracy of this metric, we manually annotate 200 held-out examples (100 examples GPT predicted to be accurate citations, and 100 examples predicted to be inaccurate citations) of generated summaries for DQFS from all models (not used in evaluation).
We annotate these blindly, without knowing the output classification of GPT-3.5.
On this set, we obtain 87\% agreement with GPT-3.5, close to the agreement of 88\%, 90\%, and 96\% shown by human annotators in~\citet{min2023factscore}.
Further, this value is near the entailment-based accuracy given in other factuality tasks~\cite{balepur-etal-2023-expository, balepur-etal-2023-text}.

For the summary quality evaluation (\cref{subsection:summary_comp}), we use the Prometheus-v2 LLM evaluator\footnote{\url{https://github.com/prometheus-eval}}.
Example rubrics given to this evaluator are in Table~\ref{table:rubric}, which are adapted directly from~\citet{shao2024assisting}.

\subsection{Efficiency and Cost Comparison} \label{subsection:efficiency}

In Tables~\ref{appendix:table:cost_cqa} and \ref{appendix:table:cost_debate}, we present the cost (LLM input/output tokens, number of calls) and efficiency (seconds taken for inference) of \modelTopic, the slightly more expensive model out of the two \model baselines, versus Hierarchical Merging and Incremental Updating~\cite{chang2024booookscore, adams2023sparse}, the two other best-performing baselines, which also happen to be multi-LLM systems. Despite \model using more LLM calls through single-turn LLM debate, our use of retrieval and a moderator LLM greatly reduces the number of input tokens \model otherwise would have consumed, keeping GPT-4 cost competitive with Hierarchical Merging, and making our model cheaper than Incremental-\textit{Topic}.
The inference time of multi-LLM summarization systems like \model could be improved, a common limitation of agentic systems~\cite{li2024personal}, and one possible strategy could be to use multi-threading or batched decoding to parallelize the discussions of LLM speakers. 

\subsection{Results for All Topics} \label{appendix:results}

We run \model and all baselines where the number of topics $m$ ranges between $2$ and $5$ inclusive, a typical range of paragraphs in argumentative essays~\cite{mery2019use}.
Tables~\ref{table:doc_cover_cqa_all} and \ref{table:doc_cover_debate_all} display the citation coverage and balance metrics from \cref{subsection:citation_comp} for all $m$, while Tables~\ref{appendix:table:llm_cqa} and \ref{appendix:table:llm_debate} display the summary quality metrics from \cref{subsection:summary_comp} results for all $m$.
Our claims hold for these varied values of $m$; \model generates comprehensive and balanced summaries while maintaining traditional output quality metrics, regardless of the number of topic paragraphs it must generate.

\subsection{Results for Hierarchical Merging over Topic Paragraphs} \label{appendix:results_hm}

Further, the Hierarchical Merging baseline we use does not generate summaries one topic at a time.
We believe that such a model (i.e. Hierarchical-\emph{Topic}) is too costly and inefficient to deploy, so we do not compare against it in the main body of the work. 
In Tables~\ref{table:doc_cover_cqa_all_comp} and \ref{table:doc_cover_debate_all_comp} we provide some results for this model, which still underperforms \modelTopic.
Further, we show in Table~\ref{appendix:table:cost_weird} that this model is much more costly compared to \model.
It is also more costly than a version of \model that iterates through all speakers, highlighting the utility of retrieval to keep inference time and LLM cost low.

\subsection{Results with GPT-4 Mini} \label{appendix:results_mini}

All of our models are implemented with GPT-4, but we also run some preliminary experiments with \modelTopic using GPT-4 mini.
In citation coverage, fairness, and faithfulness (Tables~\ref{table:doc_cover_cqa_all_comp_mini} and \ref{table:doc_cover_debate_all_comp_mini}), \modelTopic using GPT-4 mini underperforms the model using GPT-4, suggesting that larger models are better suited for multi-LLM systems like \model. However, the GPT-4 mini system still exhibits strong performance, and is even comparable to several of the baselines using GPT-4 in Tables~\ref{table:doc_cover_cqa_all_comp} and \ref{table:doc_cover_debate_all_comp}, further showcasing the efficacy of our framework.

\subsection{Results with Fixed Topics} \label{appendix:fixed_topics}

Each baseline in \cref{subsection:baselines} produces distinct topics while planning a summary.
To ensure the citation coverage and balance gains in \model are not just derived from our agenda planning step (\cref{subsection:agenda}), we implement a version of each baseline that is asked to generate summaries for the same topics that \modelTopic generates.
We present these results in Tables~\ref{table:doc_cover_cqa_fixedtopic} and \ref{table:doc_cover_debate_fixedtopic}, and find that \model still largely outperforms baselines even when using our topics, suggesting that our agenda planning is not the source of gains in the framework.

% We also compare \modelTopic to the Hierarchical Merging baseline used in \cref{subsection:efficiency_main} more extensively in Tables~\ref{table:doc_cover_cqa_all_comp} and \ref{table:doc_cover_debate_all_comp}.
% Even though this baseline runs inference on each document for each topic, our structured outline allows us to outperform this baseline with much better efficiency.

\subsection{Outline Perspective Accuracy} \label{appendix:outline}

During speaker discussion (\cref{subsection:speaker}), we ask speakers to provide perspectives in the form of facts in the document.
These facts are grouped by whether the fact gives evidence for why the answer to the query is ``yes'' or ``no'', which also provides another layer of organization to enrich the user's understanding of the outline (\cref{subsection:qg}).
To assess the accuracy of these yes/no labels, we ask human annotators to label if each paragraph in 10 document collections (5 from DebateQFS, 5 from ConflictingQA) strongly supports, weakly supports, strongly refutes, weakly refutes, or is neutral toward the input query.
In total, we collect 7592 annotations, and aggregate them into one of three labels: supports, refutes, or neutral.\footnote{For each annotator, we score a paragraph as $ \pm1$ for strongly support/reject, $\pm0.5$ for weakly support/reject, and 0 for neutral. We take the sum of these scores over all annotators, and set the final label to support/reject if the sum is greater/less than 0. A score of 0 yields a neutral label.}
We will also release these paragraph-level annotations, which may be useful for training DQFS models.
We use the same procedure in Appendix~\ref{appendix:human} for this user study.

After collecting ground truth paragraph labels, we take the outlines produced by \model on this subset of 10 examples. For each predicted yes/no fact in the outline, we post-hoc attribute~\cite{huang-chang-2024-citation} the paragraph in the speaker's document that was the source of the information in the fact (with ColBERT).
We compare the accuracy of the LLM's yes/no label using the ground truth labels from human annotators, which are 0.798, 0.806, 0.781, and 0.803 for $m = 2, 3, 4, 5$, respectively.
Our accuracy is near the accuracy of LLMs on existing stance detection benchmarks~\cite{lan2024stance}, meaning our yes/no labels provide a useful and fairly accurate signal for users.

\subsection{Human Evaluation Setup} \label{appendix:human}
We conducted user evaluations to compare the readability and balance of summaries produced by different models (\model, Long-Context, Hierarchical, Incremental-Topic). The evaluation was divided into two parts: one focusing on the entire summary and the other on topic paragraphs.

\subsubsection{Recruitment \& Procedure}
We recruited 76 participants via Prolific, all of whom were based in the United States and required to have fluency in English. Each participant rated a total of 20 summaries, assessing the output from each of the four models for a given debate query. 
Participants were paid \$12/hour, the recommended rate on the website.
To mitigate order and fatigue effects, the presentation order of summaries was counterbalanced. Each summary was rated by 3-5 different participants. Additionally, the task included two baseline comprehension checks to ensure participants understood the instructions and metric definitions. Participants who did not pass these checks were excluded from the final analysis.
These annotations did not require review from an Institutional Review Board (IRB).
We collect no Personal Identifiable Information during the study.


\subsubsection{Rating Criteria}
The task included two Likert ratings for Readability and Balance. Additionally, participants could provide open comments for feedback or to report any issues. For the Likert items, participants saw the following questions:

\begin{itemize}
    \item \textbf{Readability.} Is the summary easy to read and understand?
    \begin{enumerate}
        \item The summary is very unclear, with consistent grammatical errors and disjointed ideas.
        \item The summary is often unclear, with frequent grammatical errors and poor flow.
        \item The summary is moderately clear but has some grammatical errors and awkward transitions.
        \item The summary is mostly clear, with minor grammatical errors and mostly smooth transitions.
        \item The summary is exceptionally clear, grammatically perfect, and flows seamlessly.
    \end{enumerate}

    \item \textbf{Balance.} Does the summary address both sides of the debatable query by using counterarguments to present a well-rounded view?
    \begin{enumerate}
        \item The summary is heavily biased, with little to no use of counterarguments and only one side addressed effectively.
        \item The summary is poorly balanced, significantly favoring one side and using counterarguments ineffectively.
        \item The summary is somewhat balanced but has noticeable bias and some awkward or less effective counterarguments.
        \item The summary is mostly balanced, with minor bias and effective use of counterarguments.
        \item The summary is perfectly balanced, equally addressing both sides and effectively using counterarguments.
    \end{enumerate}
\end{itemize}

\subsubsection{Results}

Figure~\ref{fig:annot} shows the full distribution of Prolific annotations for Balance and Readability across Summaries and Topic Paragraphs. 


\subsection{Sample Outputs} \label{appendix:outputs}

We present sample outputs generated by \model on ConflictingQA (Summary~\ref{summary1}, \ref{summary2}) and DebateQFS (Summary~\ref{summary3}, \ref{summary4}).
The summaries from \model have high coverage, citing several documents from the input collection, while also being balanced.
Further, the summary quality of \model remains high.
After comparing the summary for the EU expansion query in Figure~\ref{fig:intro} from 0-shot GPT-4 versus the summary from \model in Summary~\ref{summary3}, the balance, comprehensiveness, and quality gains from our method are clear.

\clearpage
%\begin{lstlisting}[title={Sampling Responses During Training/Inference}]
Please reason step by step, and put your final answer within 
\boxed{}. 
Problem: {problem} 
\end{lstlisting}

\begin{lstlisting}[title={Verification Refinement}]
You are a math teacher. I will give you a math problem and an answer. 
Verify the answer's correctness without step-by-step solving. Use alternative verification methods. 
Question: {problem}
Answer: {answer}
Verification:
\end{lstlisting}

\begin{lstlisting}[title={Verification Collection}]
Refine this verification text to read as a natural self-check within a solution. Maintain logical flow and professionalism.
Key Requirements:
1. Avoid phrases like "without solving step-by-step" or "as a math teacher".
2. Treat the answer as your own prior solution.
3. Conclude with EXACTLY one of:
Therefore, the answer is correct.
Therefore, the answer is incorrect.
Therefore, the answer cannot be verified.
Original text: {verification}
\end{lstlisting}

\section{Dataset}
\label{sec:dataset}

\subsection{Data Collection}

To analyze political discussions on Discord, we followed the methodology in \cite{singh2024Cross-Platform}, collecting messages from politically-oriented public servers in compliance with Discord's platform policies.

Using Discord's Discovery feature, we employed a web scraper to extract server invitation links, names, and descriptions, focusing on public servers accessible without participation. Invitation links were used to access data via the Discord API. To ensure relevance, we filtered servers using keywords related to the 2024 U.S. elections (e.g., Trump, Kamala, MAGA), as outlined in \cite{balasubramanian2024publicdatasettrackingsocial}. This resulted in 302 server links, further narrowed to 81 English-speaking, politics-focused servers based on their names and descriptions.

Public messages were retrieved from these servers using the Discord API, collecting metadata such as \textit{content}, \textit{user ID}, \textit{username}, \textit{timestamp}, \textit{bot flag}, \textit{mentions}, and \textit{interactions}. Through this process, we gathered \textbf{33,373,229 messages} from \textbf{82,109 users} across \textbf{81 servers}, including \textbf{1,912,750 messages} from \textbf{633 bots}. Data collection occurred between November 13th and 15th, covering messages sent from January 1st to November 12th, just after the 2024 U.S. election.

\subsection{Characterizing the Political Spectrum}
\label{sec:timeline}

A key aspect of our research is distinguishing between Republican- and Democratic-aligned Discord servers. To categorize their political alignment, we relied on server names and self-descriptions, which often include rules, community guidelines, and references to key ideologies or figures. Each server's name and description were manually reviewed based on predefined, objective criteria, focusing on explicit political themes or mentions of prominent figures. This process allowed us to classify servers into three categories, ensuring a systematic and unbiased alignment determination.

\begin{itemize}
    \item \textbf{Republican-aligned}: Servers referencing Republican and right-wing and ideologies, movements, or figures (e.g., MAGA, Conservative, Traditional, Trump).  
    \item \textbf{Democratic-aligned}: Servers mentioning Democratic and left-wing ideologies, movements, or figures (e.g., Progressive, Liberal, Socialist, Biden, Kamala).  
    \item \textbf{Unaligned}: Servers with no defined spectrum and ideologies or opened to general political debate from all orientations.
\end{itemize}

To ensure the reliability and consistency of our classification, three independent reviewers assessed the classification following the specified set of criteria. The inter-rater agreement of their classifications was evaluated using Fleiss' Kappa \cite{fleiss1971measuring}, with a resulting Kappa value of \( 0.8191 \), indicating an almost perfect agreement among the reviewers. Disagreements were resolved by adopting the majority classification, as there were no instances where a server received different classifications from all three reviewers. This process guaranteed the consistency and accuracy of the final categorization.

Through this process, we identified \textbf{7 Republican-aligned servers}, \textbf{9 Democratic-aligned servers}, and \textbf{65 unaligned servers}.

Table \ref{tab:statistics} shows the statistics of the collected data. Notably, while Democratic- and Republican-aligned servers had a comparable number of user messages, users in the latter servers were significantly more active, posting more than double the number of messages per user compared to their Democratic counterparts. 
This suggests that, in our sample, Democratic-aligned servers attract more users, but these users were less engaged in text-based discussions. Additionally, around 10\% of the messages across all server categories were posted by bots. 

\subsection{Temporal Data} 

Throughout this paper, we refer to the election candidates using the names adopted by their respective campaigns: \textit{Kamala}, \textit{Biden}, and \textit{Trump}. To examine how the content of text messages evolves based on the political alignment of servers, we divided the 2024 election year into three periods: \textbf{Biden vs Trump} (January 1 to July 21), \textbf{Kamala vs Trump} (July 21 to September 20), and the \textbf{Voting Period} (after September 20). These periods reflect key phases of the election: the early campaign dominated by Biden and Trump, the shift in dynamics with Kamala Harris replacing Joe Biden as the Democratic candidate, and the final voting stage focused on electoral outcomes and their implications. This segmentation enables an analysis of how discourse responds to pivotal electoral moments.

Figure \ref{fig:line-plot} illustrates the distribution of messages over time, highlighting trends in total messages volume and mentions of each candidate. Prior to Biden's withdrawal on July 21, mentions of Biden and Trump were relatively balanced. However, following Kamala's entry into the race, mentions of Trump surged significantly, a trend further amplified by an assassination attempt on him, solidifying his dominance in the discourse. The only instance where Trump’s mentions were exceeded occurred during the first debate, as concerns about Biden’s age and cognitive abilities temporarily shifted the focus. In the final stages of the election, mentions of all three candidates rose, with Trump’s mentions peaking as he emerged as the victor.
\begin{algorithm}[h!]
\caption{Gait-Net-augmented Sequential CMPC}
\label{alg:gaitMPC}
\begin{algorithmic}[1]
\Require $\mathbf q, \: \dot{\mathbf q}, \: \mathbf q^\text{cmd}, \: \dot{\mathbf q}^\text{cmd}$
\State \textbf{intialize} $\bm x_0 = f_\text{j2m}(\mathbf q, \: \dot{\mathbf q}), \: \bm u^0 =\bm u_\text{IG}, \: dt^0 = 0.05$ 
\State $\{ \mathbf q^\text{ref},\:\dot{\mathbf q}^\text{ref},\:\bm p_f^\text{ref}\} = f_\text{ref} \big(\mathbf q, \: \dot{\mathbf q}, \: \mathbf q^\text{cmd}, \: \dot{\mathbf q}^\text{cmd} \big)$
\State $\bm x^\text{ref} = f_\text{j2m}(\mathbf q^\text{ref},\:\dot{\mathbf q}^\text{ref},\:\bm p_f^\text{ref})$
\State $ j = 0$ 
\While{$j \leq j_\text{max} \:\text{and}\: \bm \eta \leq \delta \bm u  $} 
\State $\delta \bm u^{j} = \texttt{cmpc}(\bm x^\text{ref},\:\bm p_f^\text{ref},\:\bm p_c^\text{ref},\: \bm x_0,\: dt^j, \: \bm u^j)$
\State $\bm u^{j+1} = \bm u^j + \delta \bm u^j$ 
\State $dt^{j+1} = \Pi_\text{GN}(\mathbf q, \: \dot{\mathbf q},\: \bm p_f^{j})$
\State $\{ \bm x^\text{ref},\:\bm p_f^\text{ref}\}= f_\text{IK}(\bm p_f^{j},\:\bm p_c^{j},\: dt^{j+1})$
\State $j=j+1$
\EndWhile \\
\Return $\bm u^{j+1} $
\end{algorithmic}
\end{algorithm}
\begin{table*}[]
\small 
\centering
\begin{tabular}{@{}ll@{}}
\toprule
 & Rubric Text \\ \midrule
Criteria & Interest Level: How engaging and thought-provoking is the summary? \\
Score 1 & Not engaging at all; no attempt to capture the reader’s attention. \\
Score 2 & Fairly engaging with a basic narrative but lacking depth. \\
Score 3 & Moderately engaging with several interesting points. \\
Score 4 & Quite engaging with a well-structured narrative and noteworthy points that frequently capture and retain attention \\
Score 5 & Exceptionally engaging throughout, with a compelling narrative that consistently stimulates interest. \\ \midrule
Criteria & Coherence and Organization: Is the summary well-organized and logically structured? \\
Score 1 & Disorganized; lacks logical structure and coherence. \\
Score 2 & Fairly organized; a basic structure is present but not consistently followed. \\
Score 3 & Organized; a clear structure is mostly followed with some lapses in coherence. \\
Score 4 & Good organization; a clear structure with minor lapses in coherence. \\
Score 5 & Excellently organized; the summary is logically structured with seamless transitions and a clear argument. \\ \midrule
Criteria & Relevance and Focus: Does the summary stay on topic to the query and maintain a clear focus? \\
Score 1 & Off-topic; the content does not align with the query. \\
Score 2 & Somewhat on topic but with several digressions; the answer to the query is evident but not consistently adhered to. \\
Score 3 & Generally on topic, despite a few unrelated details. \\
Score 4 & Mostly on topic and focused; the narrative has a consistent relevance to the query with infrequent digressions. \\
Score 5 & \specialcellleft{Exceptionally focused and entirely on topic; the article is tightly centered on the query,\\with every piece of information contributing to a comprehensive understanding of the query.} \\ \midrule
Criteria & Broad Coverage: Does the article provide an in-depth exploration of the query and have good coverage? \\
Score 1 & Severely lacking; offers little to no coverage of the query's primary aspects, resulting in a very narrow perspective. \\
Score 2 & Partial coverage; includes some of the query's main aspects but misses others, resulting in an incomplete portrayal. \\
Score 3 & \specialcellleft{Acceptable breadth; covers most main aspects, though it may stray into minor unnecessary details\\ or overlook some relevant points.} \\
Score 4 & \specialcellleft{Good coverage; achieves broad coverage of the query,\\hitting on all major points with minimal extraneous information.} \\
Score 5 & \specialcellleft{Exemplary in breadth; delivers outstanding coverage,\\thoroughly detailing all crucial aspects of the query without including irrelevant information.} \\ \midrule
Criteria & \specialcellleft{Diversity of Perspectives: Does the summary adequately describe\\why the answer to the query could be yes and why it could be no?} \\
Score 1 & No diversity; the summary presents only one perspective without addressing the opposing viewpoint. \\
Score 2 & Limited diversity; the summary acknowledges both perspectives but lacks depth in the explanation of one side. \\
Score 3 & Moderate diversity; the summary covers both perspectives, but one side is more thoroughly explored than the other. \\
Score 4 & Good diversity; the summary fairly represents both perspectives with balanced and detailed explanations. \\
Score 5 & \specialcellleft{Excellent diversity; the summary provides a comprehensive and balanced exploration of both perspectives,\\offering in-depth explanations for why the answer could be yes and why it could be no.} \\ \bottomrule
\end{tabular}
\caption{\label{table:rubric} Rubrics for Interest, Coherence, Relevance, Coverage, and Diversity for DQFS summaries. Rubrics are adapted for topic paragraphs and topics (e.g. ``Relevance'' becomes relevance to the topic in topic paragraph evaluation, rather than relevance to the query).}
\end{table*}
\begin{table*}[!h]
\footnotesize
\centering
\setlength{\tabcolsep}{3.5pt}
%\setlength{\extrarowheight}{2pt}
% \setlength{\aboverulesep}{1pt}
% \setlength{\belowrulesep}{1pt}
\renewcommand{\arraystretch}{0.8}
\begin{tabular}{@{}clcccccccc@{}}
\multicolumn{1}{l}{} &  & \multicolumn{3}{c}{\textit{Summary Level}} & \multicolumn{3}{c}{\textit{Topic Paragraph Level}} & \multicolumn{2}{c}{\textit{Confounders}} \\ \midrule
\textbf{\# Pts} & \multicolumn{1}{l|}{\textbf{Model}} & \textbf{DC ($\uparrow$)} & \textbf{Fair ($\downarrow$)} & \multicolumn{1}{c|}{\textbf{Faithful ($\downarrow$)}} & \textbf{DC ($\uparrow$)} & \textbf{Fair ($\downarrow$)} & \multicolumn{1}{c|}{\textbf{Faithful ($\downarrow$)}} & \multicolumn{1}{l}{\textbf{Cite Acc.}} & \textbf{All / Avg Sents} \\ \midrule
\multirow{10}{*}{2} & \multicolumn{1}{l|}{\textbf{\modelAll (\textbf{Ours})}} & {\ul 0.811*} & 0.113* & \multicolumn{1}{c|}{{\ul 0.046*}} & {\ul 0.578*} & {\ul 0.171*} & \multicolumn{1}{c|}{{\ul 0.106*}} & 0.988 & 5.99 / 3.00 \\
 & \multicolumn{1}{l|}{\textbf{\modelTopic (\textbf{Ours})}} & \textbf{0.821*} & \textbf{0.108*} & \multicolumn{1}{c|}{\textbf{0.043*}} & \textbf{0.623} & \textbf{0.153*} & \multicolumn{1}{c|}{\textbf{0.090*}} & 0.985 & 6.01 / 3.01 \\
 & \multicolumn{1}{l|}{Long-Context} & 0.447 & 0.242 & \multicolumn{1}{c|}{0.198} & 0.277 & 0.369 & \multicolumn{1}{c|}{0.326} & 0.950 & 5.99 / 3.00 \\
 & \multicolumn{1}{l|}{RAG-\textit{All}} & 0.603 & 0.166 & \multicolumn{1}{c|}{0.098} & 0.378 & 0.285 & \multicolumn{1}{c|}{0.219} & 0.992 & 6.00 / 3.00 \\
 & \multicolumn{1}{l|}{RAG-\textit{Doc}} & 0.668 & 0.148 & \multicolumn{1}{c|}{0.078} & 0.415 & 0.273 & \multicolumn{1}{c|}{0.204} & 0.970 & 6.02 / 3.01 \\
 & \multicolumn{1}{l|}{Hierarchical} & 0.765 & {\ul 0.111*} & \multicolumn{1}{c|}{0.048*} & 0.454 & 0.265 & \multicolumn{1}{c|}{0.204} & 0.985 & 6.00 / 3.00 \\
 & \multicolumn{1}{l|}{Incremental-\textit{All}} & 0.464 & 0.249 & \multicolumn{1}{c|}{0.202} & 0.357 & 0.289 & \multicolumn{1}{c|}{0.244} & 0.971 & 5.99 / 3.00 \\
 & \multicolumn{1}{l|}{Incremental-\textit{Topic}} & 0.512 & 0.230 & \multicolumn{1}{c|}{0.182} & 0.419 & 0.262 & \multicolumn{1}{c|}{0.215} & 0.977 & 6.00 / 3.00 \\
 & \multicolumn{1}{l|}{Cluster} & 0.586 & 0.168 & \multicolumn{1}{c|}{0.126} & 0.356 & 0.309 & \multicolumn{1}{c|}{0.269} & 0.927 & 6.01 / 3.01 \\
 & \multicolumn{1}{l|}{RAG+Cluster} & 0.665 & 0.151 & \multicolumn{1}{c|}{0.078} & 0.417 & 0.269 & \multicolumn{1}{c|}{0.198} & 0.979 & 6.04 / 3.02 \\ \midrule
\multirow{10}{*}{3} & \multicolumn{1}{l|}{\textbf{\modelAll (\textbf{Ours})}} & {\ul 0.8664} & 0.1062* & \multicolumn{1}{c|}{0.0359*} & {\ul 0.5420} & {\ul 0.1896*} & \multicolumn{1}{c|}{{\ul 0.1217}} & 0.988 & 8.97 / 2.99 \\
 & \multicolumn{1}{l|}{\textbf{\modelTopic (\textbf{Ours})}} & \textbf{0.8961*} & {\ul 0.0998*} & \multicolumn{1}{c|}{\textbf{0.0320*}} & \textbf{0.6056*} & \textbf{0.1650*} & \multicolumn{1}{c|}{\textbf{0.0979}} & 0.985 & 8.99 / 3.00 \\
 & \multicolumn{1}{l|}{Long-Context} & 0.5242 & 0.2047 & \multicolumn{1}{c|}{0.1733} & 0.2566 & 0.3816 & \multicolumn{1}{c|}{0.3503} & 0.958 & 9.00 / 3.00 \\
 & \multicolumn{1}{l|}{RAG-\textit{All}} & 0.6565 & 0.1664 & \multicolumn{1}{c|}{0.0911} & 0.3300 & 0.3296 & \multicolumn{1}{c|}{0.2547} & 0.990 & 9.01 / 3.00 \\
 & \multicolumn{1}{l|}{RAG-\textit{Doc}} & 0.7532 & 0.1364 & \multicolumn{1}{c|}{0.0668} & 0.3741 & 0.3023 & \multicolumn{1}{c|}{0.2352} & 0.949 & 9.01 / 3.00 \\
 & \multicolumn{1}{l|}{Hierarchical} & 0.8158 & \textbf{0.0956*} & \multicolumn{1}{c|}{{\ul 0.0333*}} & 0.3679 & 0.3136 & \multicolumn{1}{c|}{0.2523} & 0.981 & 8.99 / 3.00 \\
 & \multicolumn{1}{l|}{Incremental-\textit{All}} & 0.5037 & 0.2466 & \multicolumn{1}{c|}{0.1924} & 0.3467 & 0.3019 & \multicolumn{1}{c|}{0.2488} & 0.961 & 8.99 / 3.00 \\
 & \multicolumn{1}{l|}{Incremental-\textit{Topic}} & 0.5635 & 0.2288 & \multicolumn{1}{c|}{0.1720} & 0.4209 & 0.2796 & \multicolumn{1}{c|}{0.2236} & 0.963 & 9.01 / 3.00 \\
 & \multicolumn{1}{l|}{Cluster} & 0.7142 & 0.1203* & \multicolumn{1}{c|}{0.0662} & 0.3502 & 0.3016 & \multicolumn{1}{c|}{0.2517} & 0.927 & 9.04 / 3.01 \\
 & \multicolumn{1}{l|}{RAG+Cluster} & 0.7694 & 0.1332 & \multicolumn{1}{c|}{0.0620} & 0.3906 & 0.2808 & \multicolumn{1}{c|}{0.2101} & 0.976 & 9.02 / 3.01 \\ \midrule
\multirow{10}{*}{4} & \multicolumn{1}{l|}{\textbf{\modelAll (\textbf{Ours})}} & {\ul 0.8991} & {\ul 0.0976*} & \multicolumn{1}{c|}{{\ul 0.0301*}} & {\ul 0.5107} & {\ul 0.1886*} & \multicolumn{1}{c|}{{\ul 0.1225*}} & 0.987 & 11.92 / 2.98 \\
 & \multicolumn{1}{l|}{\textbf{\modelTopic (\textbf{Ours})}} & \textbf{0.9307*} & \textbf{0.0907*} & \multicolumn{1}{c|}{\textbf{0.0263*}} & \textbf{0.5954*} & \textbf{0.1653*} & \multicolumn{1}{c|}{\textbf{0.1022*}} & 0.982 & 12.00 / 3.00 \\
 & \multicolumn{1}{l|}{Long-Context} & 0.5594 & 0.1953 & \multicolumn{1}{c|}{0.1501} & 0.2342 & 0.4204 & \multicolumn{1}{c|}{0.3779} & 0.953 & 12.03 / 3.01 \\
 & \multicolumn{1}{l|}{RAG-\textit{All}} & 0.7065 & 0.1485 & \multicolumn{1}{c|}{0.0801} & 0.2987 & 0.3556 & \multicolumn{1}{c|}{0.2891} & 0.997 & 12.02 / 3.00 \\
 & \multicolumn{1}{l|}{RAG-\textit{Doc}} & 0.7638 & 0.1357 & \multicolumn{1}{c|}{0.0631} & 0.3293 & 0.3427 & \multicolumn{1}{c|}{0.2725} & 0.961 & 12.01 / 3.00 \\
 & \multicolumn{1}{l|}{Hierarchical} & 0.8643 & 0.1008* & \multicolumn{1}{c|}{0.0325*} & 0.3204 & 0.3439 & \multicolumn{1}{c|}{0.2768} & 0.983 & 12.02 / 3.01 \\
 & \multicolumn{1}{l|}{Incremental-\textit{All}} & 0.4994 & 0.2589 & \multicolumn{1}{c|}{0.1999} & 0.3208 & 0.3200 & \multicolumn{1}{c|}{0.2602} & 0.950 & 11.97 / 2.99 \\
 & \multicolumn{1}{l|}{Incremental-\textit{Topic}} & 0.5611 & 0.2274 & \multicolumn{1}{c|}{0.1703} & 0.3896 & 0.2931 & \multicolumn{1}{c|}{0.2365} & 0.954 & 12.00 / 3.00 \\
 & \multicolumn{1}{l|}{Cluster} & 0.7907 & 0.1108* & \multicolumn{1}{c|}{0.0577} & 0.3485 & 0.3068 & \multicolumn{1}{c|}{0.2557} & 0.931 & 12.02 / 3.01 \\
 & \multicolumn{1}{l|}{RAG+Cluster} & 0.8266 & 0.1175 & \multicolumn{1}{c|}{0.0527} & 0.3614 & 0.3002 & \multicolumn{1}{c|}{0.2393} & 0.977 & 12.03 / 3.01 \\ \midrule
\multirow{10}{*}{5} & \multicolumn{1}{l|}{\textbf{\modelAll (\textbf{Ours})}} & {\ul 0.9156} & {\ul 0.0966*} & \multicolumn{1}{c|}{{\ul 0.0272*}} & {\ul 0.4809} & {\ul 0.1972} & \multicolumn{1}{c|}{{\ul 0.1297}} & 0.990 & 14.88 / 2.98 \\
 & \multicolumn{1}{l|}{\textbf{\modelTopic (\textbf{Ours})}} & \textbf{0.9549*} & \textbf{0.0884*} & \multicolumn{1}{c|}{\textbf{0.0239*}} & \textbf{0.5924*} & \textbf{0.1661*} & \multicolumn{1}{c|}{\textbf{0.1051*}} & 0.986 & 15.00 / 3.00 \\
 & \multicolumn{1}{l|}{Long-Context} & 0.5779 & 0.2038 & \multicolumn{1}{c|}{0.1622} & 0.2164 & 0.4620 & \multicolumn{1}{c|}{0.4213} & 0.966 & 15.00 / 3.00 \\
 & \multicolumn{1}{l|}{RAG-\textit{All}} & 0.7331 & 0.1581 & \multicolumn{1}{c|}{0.0814} & 0.2755 & 0.3850 & \multicolumn{1}{c|}{0.3101} & 0.996 & 15.03 / 3.01 \\
 & \multicolumn{1}{l|}{RAG-\textit{Doc}} & 0.7898 & 0.1464 & \multicolumn{1}{c|}{0.0706} & 0.3018 & 0.3691 & \multicolumn{1}{c|}{0.2945} & 0.975 & 15.06 / 3.01 \\
 & \multicolumn{1}{l|}{Hierarchical} & 0.8871 & 0.0931* & \multicolumn{1}{c|}{0.0276*} & 0.2951 & 0.3670 & \multicolumn{1}{c|}{0.3038} & 0.987 & 15.01 / 3.00 \\
 & \multicolumn{1}{l|}{Incremental-\textit{All}} & 0.5392 & 0.2327 & \multicolumn{1}{c|}{0.1738} & 0.3083 & 0.3236 & \multicolumn{1}{c|}{0.2672} & 0.948 & 14.91 / 2.98 \\
 & \multicolumn{1}{l|}{Incremental-\textit{Topic}} & 0.6239 & 0.1899 & \multicolumn{1}{c|}{0.1337} & 0.3961 & 0.2902 & \multicolumn{1}{c|}{0.2348} & 0.958 & 14.99 / 3.00 \\
 & \multicolumn{1}{l|}{Cluster} & 0.8480 & 0.0968* & \multicolumn{1}{c|}{0.0464} & 0.3365 & 0.3093 & \multicolumn{1}{c|}{0.2625} & 0.933 & 15.04 / 3.01 \\
 & \multicolumn{1}{l|}{RAG+Cluster} & 0.8717 & 0.1084* & \multicolumn{1}{c|}{0.0436} & 0.3499 & 0.3136 & \multicolumn{1}{c|}{0.2511} & 0.971 & 15.03 / 3.01 \\ \bottomrule
\end{tabular}
\caption{\label{table:doc_cover_cqa_all}ConflictingQA citation coverage, balance, and accuracy. Best model is \textbf{bold}, second best is \underline{underlined}. Models with * are significantly the best (2-sample $t$-test, $p<0.05$ with Bonferroni correction).}
\end{table*}

\begin{table*}[!h]
\footnotesize
\centering
\setlength{\tabcolsep}{3.5pt}
%\setlength{\extrarowheight}{2pt}
% \setlength{\aboverulesep}{1pt}
% \setlength{\belowrulesep}{1pt}
\renewcommand{\arraystretch}{0.8}
\begin{tabular}{@{}clcccccccc@{}}
\multicolumn{1}{l}{} &  & \multicolumn{3}{c}{\textit{Summary Level}} & \multicolumn{3}{c}{\textit{Topic Paragraph Level}} & \multicolumn{2}{c}{\textit{Confounders}} \\ \midrule
\textbf{\# Pts} & \multicolumn{1}{l|}{\textbf{Model}} & \textbf{DC ($\uparrow$)} & \textbf{Fair ($\downarrow$)} & \multicolumn{1}{c|}{\textbf{Faithful ($\downarrow$)}} & \textbf{DC ($\uparrow$)} & \textbf{Fair ($\downarrow$)} & \multicolumn{1}{c|}{\textbf{Faithful ($\downarrow$)}} & \multicolumn{1}{l}{\textbf{Cite Acc}} & \textbf{All / Avg Sents} \\ \midrule
\multirow{10}{*}{2} & \multicolumn{1}{l|}{\modelTopic (\textbf{Ours})} & \textbf{0.798*} & \textbf{0.088*} & \multicolumn{1}{c|}{\textbf{0.036*}} & \textbf{0.614*} & \textbf{0.132*} & \multicolumn{1}{c|}{\textbf{0.078*}} & 0.991 & 5.99 / 3.00 \\
 & \multicolumn{1}{l|}{\modelAll (\textbf{Ours})} & {\ul 0.789*} & {\ul 0.098*} & \multicolumn{1}{c|}{{\ul 0.040*}} & {\ul 0.582*} & {\ul 0.150*} & \multicolumn{1}{c|}{{\ul 0.092*}} & 0.992 & 5.96 / 2.98 \\
 & \multicolumn{1}{l|}{Long-Context} & 0.506 & 0.254 & \multicolumn{1}{c|}{0.212} & 0.302 & 0.423 & \multicolumn{1}{c|}{0.385} & 0.976 & 6.01 / 3.00 \\
 & \multicolumn{1}{l|}{RAG-\textit{All}} & 0.529 & 0.183 & \multicolumn{1}{c|}{0.139} & 0.347 & 0.295 & \multicolumn{1}{c|}{0.251} & 0.995 & 6.01 / 3.00 \\
 & \multicolumn{1}{l|}{RAG-\textit{Doc}} & 0.630 & 0.142 & \multicolumn{1}{c|}{0.095} & 0.374 & 0.325 & \multicolumn{1}{c|}{0.280} & 0.991 & 5.99 / 3.00 \\
 & \multicolumn{1}{l|}{Hierarchical} & 0.710 & 0.104* & \multicolumn{1}{c|}{0.053*} & 0.421 & 0.261 & \multicolumn{1}{c|}{0.209} & 0.983 & 6.00 / 3.00 \\
 & \multicolumn{1}{l|}{Incremental-\textit{All}} & 0.497 & 0.326 & \multicolumn{1}{c|}{0.291} & 0.405 & 0.348 & \multicolumn{1}{c|}{0.313} & 0.981 & 6.01 / 3.00 \\
 & \multicolumn{1}{l|}{Incremental-\textit{Topic}} & 0.548 & 0.297 & \multicolumn{1}{c|}{0.266} & 0.459 & 0.338 & \multicolumn{1}{c|}{0.307} & 0.982 & 6.00 / 3.00 \\
 & \multicolumn{1}{l|}{Cluster} & 0.610 & 0.133 & \multicolumn{1}{c|}{0.102} & 0.384 & 0.297 & \multicolumn{1}{c|}{0.266} & 0.966 & 6.01 / 3.00 \\
 & \multicolumn{1}{l|}{RAG+Cluster} & 0.572 & 0.166 & \multicolumn{1}{c|}{0.121} & 0.354 & 0.306 & \multicolumn{1}{c|}{0.260} & 0.986 & 6.02 / 3.01 \\ \midrule
\multirow{10}{*}{3} & \multicolumn{1}{l|}{\modelTopic (\textbf{Ours})} & \textbf{0.8724*} & \textbf{0.0701*} & \multicolumn{1}{c|}{\textbf{0.0235*}} & \textbf{0.6066*} & \textbf{0.1255*} & \multicolumn{1}{c|}{\textbf{0.0789*}} & 0.982 & 8.99 / 3.00 \\
\multicolumn{1}{l}{} & \multicolumn{1}{l|}{\modelAll (\textbf{Ours})} & {\ul 0.8457*} & {\ul 0.0786*} & \multicolumn{1}{c|}{{\ul 0.0273*}} & {\ul 0.5508} & {\ul 0.1463*} & \multicolumn{1}{c|}{{\ul 0.0938*}} & 0.987 & 8.87 / 2.96 \\
\multicolumn{1}{l}{} & \multicolumn{1}{l|}{Long-Context} & 0.5877 & 0.2094 & \multicolumn{1}{c|}{0.1790} & 0.2798 & 0.4336 & \multicolumn{1}{c|}{0.4028} & 0.953 & 9.02 / 3.01 \\
\multicolumn{1}{l}{} & \multicolumn{1}{l|}{RAG-\textit{All}} & 0.6125 & 0.1544 & \multicolumn{1}{c|}{0.1040} & 0.3229 & 0.3176 & \multicolumn{1}{c|}{0.2701} & 0.997 & 9.01 / 3.00 \\
\multicolumn{1}{l}{} & \multicolumn{1}{l|}{RAG-\textit{Doc}} & 0.7171 & 0.1180 & \multicolumn{1}{c|}{0.0664} & 0.3504 & 0.3233 & \multicolumn{1}{c|}{0.2748} & 0.961 & 9.01 / 3.00 \\
\multicolumn{1}{l}{} & \multicolumn{1}{l|}{Hierarchical} & 0.7868 & 0.0907 & \multicolumn{1}{c|}{0.0374} & 0.3639 & 0.2980 & \multicolumn{1}{c|}{0.2452} & 0.983 & 9.02 / 3.01 \\
\multicolumn{1}{l}{} & \multicolumn{1}{l|}{Incremental-\textit{All}} & 0.5566 & 0.2579 & \multicolumn{1}{c|}{0.2089} & 0.3919 & 0.3243 & \multicolumn{1}{c|}{0.2765} & 0.950 & 8.91 / 2.97 \\
\multicolumn{1}{l}{} & \multicolumn{1}{l|}{Incremental-\textit{Topic}} & 0.6152 & 0.2415 & \multicolumn{1}{c|}{0.1970} & 0.4707 & 0.3128 & \multicolumn{1}{c|}{0.2674} & 0.954 & 9.03 / 3.01 \\
\multicolumn{1}{l}{} & \multicolumn{1}{l|}{Cluster} & 0.7102 & 0.1106 & \multicolumn{1}{c|}{0.0725} & 0.3632 & 0.3106 & \multicolumn{1}{c|}{0.2737} & 0.931 & 9.04 / 3.01 \\
\multicolumn{1}{l}{} & \multicolumn{1}{l|}{RAG+Cluster} & 0.6811 & 0.1405 & \multicolumn{1}{c|}{0.0894} & 0.3428 & 0.3200 & \multicolumn{1}{c|}{0.2689} & 0.977 & 9.01 / 3.00 \\ \midrule
\multirow{10}{*}{4} & \multicolumn{1}{l|}{\modelTopic (\textbf{Ours})} & \textbf{0.8895*} & {\ul 0.0724*} & \multicolumn{1}{c|}{\textbf{0.0209*}} & \textbf{0.5844*} & \textbf{0.1385*} & \multicolumn{1}{c|}{\textbf{0.0868*}} & 0.987 & 11.98 / 3.00 \\
 & \multicolumn{1}{l|}{\modelAll (\textbf{Ours})} & {\ul 0.8653*} & \textbf{0.0697*} & \multicolumn{1}{c|}{{\ul 0.0216*}} & {\ul 0.5230} & {\ul 0.1419*} & \multicolumn{1}{c|}{{\ul 0.0925*}} & 0.990 & 11.86 / 2.96 \\
 & \multicolumn{1}{l|}{Long-Context} & 0.6361 & 0.1691 & \multicolumn{1}{c|}{0.1471} & 0.2473 & 0.4733 & \multicolumn{1}{c|}{0.4479} & 0.977 & 12.03 / 3.01 \\
 & \multicolumn{1}{l|}{RAG-\textit{All}} & 0.6595 & 0.1440 & \multicolumn{1}{c|}{0.0969} & 0.2916 & 0.3603 & \multicolumn{1}{c|}{0.3149} & 0.995 & 12.03 / 3.01 \\
 & \multicolumn{1}{l|}{RAG-\textit{Doc}} & 0.7335 & 0.1218 & \multicolumn{1}{c|}{0.0723} & 0.3113 & 0.3635 & \multicolumn{1}{c|}{0.3171} & 0.991 & 12.03 / 3.01 \\
 & \multicolumn{1}{l|}{Hierarchical} & 0.8338 & 0.0845* & \multicolumn{1}{c|}{0.0325} & 0.3269 & 0.3331 & \multicolumn{1}{c|}{0.2813} & 0.986 & 12.02 / 3.01 \\
 & \multicolumn{1}{l|}{Incremental-\textit{All}} & 0.5716 & 0.2352 & \multicolumn{1}{c|}{0.1874} & 0.3795 & 0.3193 & \multicolumn{1}{c|}{0.2736} & 0.963 & 11.87 / 2.97 \\
 & \multicolumn{1}{l|}{Incremental-\textit{Topic}} & 0.6331 & 0.2129 & \multicolumn{1}{c|}{0.1629} & 0.4514 & 0.3133 & \multicolumn{1}{c|}{0.2658} & 0.970 & 11.98 / 2.99 \\
 & \multicolumn{1}{l|}{Cluster} & 0.7744 & 0.1129 & \multicolumn{1}{c|}{0.0698} & 0.3451 & 0.3181 & \multicolumn{1}{c|}{0.2752} & 0.964 & 12.03 / 3.01 \\
 & \multicolumn{1}{l|}{RAG+Cluster} & 0.7305 & 0.1218 & \multicolumn{1}{c|}{0.0746} & 0.3237 & 0.3459 & \multicolumn{1}{c|}{0.3029} & 0.989 & 12.04 / 3.01 \\ \midrule
\multirow{10}{*}{5} & \multicolumn{1}{l|}{\modelTopic (\textbf{Ours})} & \textbf{0.9137*} & {\ul 0.0651*} & \multicolumn{1}{c|}{\textbf{0.0208*}} & \textbf{0.5793*} & \textbf{0.1420*} & \multicolumn{1}{c|}{\textbf{0.0998*}} & 0.986 & 14.99 / 3.00 \\
 & \multicolumn{1}{l|}{\modelAll (\textbf{Ours})} & {\ul 0.8847*} & \textbf{0.0640*} & \multicolumn{1}{c|}{{\ul 0.0236*}} & {\ul 0.4991} & {\ul 0.1502*} & \multicolumn{1}{c|}{{\ul 0.1096*}} & 0.990 & 14.46 / 2.89 \\
 & \multicolumn{1}{l|}{Long-Context} & 0.6686 & 0.1724 & \multicolumn{1}{c|}{0.1392} & 0.2312 & 0.4965 & \multicolumn{1}{c|}{0.4640} & 0.966 & 15.01 / 3.00 \\
 & \multicolumn{1}{l|}{RAG-\textit{All}} & 0.6721 & 0.1423 & \multicolumn{1}{c|}{0.0912} & 0.2668 & 0.3927 & \multicolumn{1}{c|}{0.3438} & 0.996 & 15.02 / 3.00 \\
 & \multicolumn{1}{l|}{RAG-\textit{Doc}} & 0.7765 & 0.1053 & \multicolumn{1}{c|}{0.0618} & 0.3005 & 0.3584 & \multicolumn{1}{c|}{0.3147} & 0.975 & 15.01 / 3.00 \\
 & \multicolumn{1}{l|}{Hierarchical} & 0.8565 & 0.0761* & \multicolumn{1}{c|}{0.0239*} & 0.2896 & 0.3713 & \multicolumn{1}{c|}{0.3192} & 0.987 & 15.04 / 3.01 \\
 & \multicolumn{1}{l|}{Incremental-\textit{All}} & 0.6122 & 0.2000 & \multicolumn{1}{c|}{0.1629} & 0.3716 & 0.2936 & \multicolumn{1}{c|}{0.2572} & 0.948 & 14.77 / 2.95 \\
 & \multicolumn{1}{l|}{Incremental-\textit{Topic}} & 0.6767 & 0.1659 & \multicolumn{1}{c|}{0.1198} & 0.4446 & 0.2897 & \multicolumn{1}{c|}{0.2443} & 0.958 & 15.05 / 3.01 \\
 & \multicolumn{1}{l|}{Cluster} & 0.8098 & 0.1116 & \multicolumn{1}{c|}{0.0624} & 0.3292 & 0.3383 & \multicolumn{1}{c|}{0.2921} & 0.933 & 15.03 / 3.01 \\
 & \multicolumn{1}{l|}{RAG+Cluster} & 0.7811 & 0.1233 & \multicolumn{1}{c|}{0.0738} & 0.3129 & 0.3588 & \multicolumn{1}{c|}{0.3107} & 0.971 & 15.03 / 3.01 \\ \bottomrule
\end{tabular}
\caption{\label{table:doc_cover_debate_all}DebateQFS citation coverage, balance, and accuracy. Best model is \textbf{bold}, second best is \underline{underlined}. Models with * are significantly the best (2-sample $t$-test, $p<0.05$ with Bonferroni correction).}
\end{table*}


\begin{table*}[!h]
\footnotesize
\centering
\setlength{\tabcolsep}{3.5pt}
%\setlength{\extrarowheight}{2pt}
% \setlength{\aboverulesep}{1pt}
% \setlength{\belowrulesep}{1pt}
\renewcommand{\arraystretch}{0.8}
\begin{tabular}{@{}clcccccccc@{}}
\multicolumn{1}{l}{} &  & \multicolumn{3}{c}{{\textit{Summary Level}}} & \multicolumn{3}{c}{{\textit{Topic Paragraph Level}}} & \multicolumn{2}{c}{{\textit{Confounders}}} \\ \midrule
{\# Pts} & \multicolumn{1}{l|}{{Model}} & {\textbf{DC} ($\uparrow$)} & {\textbf{Fair} ($\downarrow$)} & \multicolumn{1}{c|}{{\textbf{Faithful} ($\downarrow$)}} & {\textbf{DC} ($\uparrow$)} & {\textbf{Fair} ($\downarrow$)} & \multicolumn{1}{c|}{{\textbf{Faithful} ($\downarrow$)}} & \multicolumn{1}{l}{{\textbf{Cite Acc}.}} & {\textbf{All / Avg Sents}} \\ \midrule

\multirow{2}{*}{3} & \multicolumn{1}{l|}{{\modelTopic ({Ours})}} & {0.8961} & {0.0998} & \multicolumn{1}{c|}{{0.0320}} & {0.6056} & {0.1650} & \multicolumn{1}{c|}{{0.0979}} & 0.985 & 8.99 / 3.00 \\
 & \multicolumn{1}{l|}{Hierarchical-\textit{Topic}} & 0.8761 & {0.1065} & \multicolumn{1}{c|}{{0.0467	}} & 0.6003 & 0.1688 & \multicolumn{1}{c|}{0.1130} & 0.985 & 8.98 / 2.99 \\ \midrule

\multirow{2}{*}{5} & \multicolumn{1}{l|}{{\modelTopic ({Ours})}} & {0.9549} & {0.0884} & \multicolumn{1}{c|}{{0.0239}} & {0.5924} & {0.1661} & \multicolumn{1}{c|}{{0.1051}} & 0.986 & 15.00 / 3.00 \\
 & \multicolumn{1}{l|}{Hierarchical-\textit{Topic}} & 0.9386 & 0.0996 & \multicolumn{1}{c|}{0.0310} & 0.5774 & 0.1952 & \multicolumn{1}{c|}{0.1304} & 0.987 & 15.01 / 3.00  \\ \bottomrule
\end{tabular}
\caption{\label{table:doc_cover_cqa_all_comp}ConflictingQA citation coverage, balance, and accuracy of \modelTopic versus Hierarchical Merging-\emph{Topic}, which runs hierarchical merging for each topic paragraph. \model consistently outperforms Hierarchal Merging.}
\end{table*}

\begin{table*}[!h]
\footnotesize
\centering
\setlength{\tabcolsep}{3.5pt}
%\setlength{\extrarowheight}{2pt}
% \setlength{\aboverulesep}{1pt}
% \setlength{\belowrulesep}{1pt}
\renewcommand{\arraystretch}{0.8}
\begin{tabular}{@{}clcccccccc@{}}
\multicolumn{1}{l}{} &  & \multicolumn{3}{c}{{\textit{Summary Level}}} & \multicolumn{3}{c}{{\textit{Topic Paragraph Level}}} & \multicolumn{2}{c}{{\textit{Confounders}}} \\ \midrule
{\# Pts} & \multicolumn{1}{l|}{{Model}} & {\textbf{DC} ($\uparrow$)} & {\textbf{Fair} ($\downarrow$)} & \multicolumn{1}{c|}{{\textbf{Faithful} ($\downarrow$)}} & {\textbf{DC} ($\uparrow$)} & {\textbf{Fair} ($\downarrow$)} & \multicolumn{1}{c|}{{\textbf{Faithful} ($\downarrow$)}} & \multicolumn{1}{l}{{\textbf{Cite Acc}}} & {\textbf{All / Avg Sents}} \\ \midrule

\multirow{2}{*}{3} & \multicolumn{1}{l|}{\modelTopic ({Ours})} & {0.8724} & {0.0701} & \multicolumn{1}{c|}{{0.0235}} & {0.6066} & {0.1255} & \multicolumn{1}{c|}{{0.0789}} & 0.982 & 8.99 / 3.00 \\
\multicolumn{1}{l}{} & \multicolumn{1}{l|}{Hierarchical-\textit{Topic}} & 0.7776 & 0.0965 & \multicolumn{1}{c|}{0.0483} & 0.4964 & 0.2177 & \multicolumn{1}{c|}{0.1688} & 0.983 & 9.00 / 3.00\\ \midrule

\multirow{2}{*}{5} & \multicolumn{1}{l|}{\modelTopic ({Ours})} & {0.9137} & { 0.0651} & \multicolumn{1}{c|}{{0.0208}} & {0.5793} & {0.1420} & \multicolumn{1}{c|}{{0.0998}} & 0.986 & 14.99 / 3.00 \\
 
 & \multicolumn{1}{l|}{Hierarchical-\textit{Topic}} & 0.8427 & 0.0951 & \multicolumn{1}{c|}{0.0431} & 0.4669 & 0.2397 & \multicolumn{1}{c|}{0.1909} & 0.984 & 14.90 / 2.98 \\ \bottomrule
\end{tabular}
\caption{\label{table:doc_cover_debate_all_comp}DebateQFS citation coverage, balance, and accuracy of \modelTopic versus Hierarchical Merging-\emph{Topic}, which runs hierarchical merging for each topic paragraph. \model consistently outperforms Hierarchal Merging.}
\end{table*}


\begin{table*}[!h]
\footnotesize
\centering
\setlength{\tabcolsep}{3.5pt}
%\setlength{\extrarowheight}{2pt}
% \setlength{\aboverulesep}{1pt}
% \setlength{\belowrulesep}{1pt}
\renewcommand{\arraystretch}{0.8}
\begin{tabular}{@{}clcccccc@{}}
\multicolumn{1}{l}{} &  & \multicolumn{3}{c}{{\textit{Summary Level}}} & \multicolumn{3}{c}{{\textit{Topic Paragraph Level}}} \\ \midrule
{\# Pts} & \multicolumn{1}{l|}{{Model}} & {\textbf{DC} ($\uparrow$)} & {\textbf{Fair} ($\downarrow$)} & \multicolumn{1}{c|}{{\textbf{Faithful} ($\downarrow$)}} & {\textbf{DC} ($\uparrow$)} & {\textbf{Fair} ($\downarrow$)} & \multicolumn{1}{c}{{\textbf{Faithful} ($\downarrow$)}} \\ \midrule

\multirow{2}{*}{3} & \multicolumn{1}{l|}{{\modelTopic (GPT-4) }} & {0.8961} & {0.0998} & \multicolumn{1}{c|}{{0.0320}} & {0.6056} & {0.1650} & \multicolumn{1}{c}{{0.0979}} \\
 & \multicolumn{1}{l|}{\modelTopic (GPT-4 mini)} & 0.8761 & {0.1065} & \multicolumn{1}{c|}{{0.0467	}} & 0.6003 & 0.1688 & \multicolumn{1}{c}{0.1130} \\ \midrule

\multirow{2}{*}{5} & \multicolumn{1}{l|}{{\modelTopic (GPT-4) }} & {0.9549} & {0.0884} & \multicolumn{1}{c|}{{0.0239}} & {0.5924} & {0.1661} & \multicolumn{1}{c}{{0.1051}} \\
 & \multicolumn{1}{l|}{\modelTopic (GPT-4 mini)} & 0.7841 & 0.1226 & \multicolumn{1}{c|}{0.0634} & 0.4320 & 0.2112	 & \multicolumn{1}{c}{0.1533}  \\ \bottomrule
\end{tabular}
\caption{\label{table:doc_cover_cqa_all_comp_mini}ConflictingQA citation coverage, balance, and accuracy of \modelTopic using GPT-4 versus \modelTopic using GPT-4 mini.}
\end{table*}

\begin{table*}[!h]
\footnotesize
\centering
\setlength{\tabcolsep}{3.5pt}
%\setlength{\extrarowheight}{2pt}
% \setlength{\aboverulesep}{1pt}
% \setlength{\belowrulesep}{1pt}
\renewcommand{\arraystretch}{0.8}
\begin{tabular}{@{}clcccccc@{}}
\multicolumn{1}{l}{} &  & \multicolumn{3}{c}{{\textit{Summary Level}}} & \multicolumn{3}{c}{{\textit{Topic Paragraph Level}}} \\ \midrule
{\# Pts} & \multicolumn{1}{l|}{{Model}} & {\textbf{DC} ($\uparrow$)} & {\textbf{Fair} ($\downarrow$)} & \multicolumn{1}{c|}{{\textbf{Faithful} ($\downarrow$)}} & {\textbf{DC} ($\uparrow$)} & {\textbf{Fair} ($\downarrow$)} & \multicolumn{1}{c}{{\textbf{Faithful} ($\downarrow$)}} \\ \midrule

\multirow{2}{*}{3} & \multicolumn{1}{l|}{\modelTopic (GPT-4) } & {0.8724} & {0.0701} & \multicolumn{1}{c|}{{0.0235}} & {0.6066} & {0.1255} & \multicolumn{1}{c}{{0.0789}} \\
\multicolumn{1}{l}{} & \multicolumn{1}{l|}{\modelTopic (GPT-4 mini)} & 0.7322	 & 0.1284 & \multicolumn{1}{c|}{0.1059} & 0.4788 & 0.2271 & \multicolumn{1}{c}{0.2066} \\ \midrule

\multirow{2}{*}{5} & \multicolumn{1}{l|}{\modelTopic (GPT-4) } & {0.9137} & { 0.0651} & \multicolumn{1}{c|}{{0.0208}} & {0.5793} & {0.1420} & \multicolumn{1}{c}{{0.0998}} \\
 
 & \multicolumn{1}{l|}{\modelTopic (GPT-4 mini)} & 0.8324	 & 0.0686	 & \multicolumn{1}{c|}{0.0686} & 0.4818 & 0.2260	 & \multicolumn{1}{c}{0.2260	} \\ \bottomrule
\end{tabular}
\caption{\label{table:doc_cover_debate_all_comp_mini}DebateQFS citation coverage, balance, and accuracy of \modelTopic using GPT-4 versus \modelTopic using GPT-4 mini.}
\end{table*}


\begin{table*}[t]
\footnotesize
\centering
\setlength{\tabcolsep}{2.75pt}
%\setlength{\extrarowheight}{2pt}
% \setlength{\aboverulesep}{1pt}
% \setlength{\belowrulesep}{1pt}
\renewcommand{\arraystretch}{0.6}
\begin{tabular}{@{}clcccccc@{}}
\multicolumn{1}{l}{} &  & \multicolumn{3}{c}{\textit{Summary Level}} & \multicolumn{3}{c}{\textit{Topic Paragraph Level}} \\ \toprule
\textbf{\# Top.} & \multicolumn{1}{l|}{\textbf{Model}} & \textbf{DC ($\uparrow$)} & \textbf{Fair ($\downarrow$)} & \multicolumn{1}{c|}{\textbf{\begin{tabular}[c]{@{}c@{}}Faithful ($\downarrow$)\end{tabular}}} & \textbf{DC ($\uparrow$)} & \textbf{Fair ($\downarrow$)} & \multicolumn{1}{c}{\textbf{\begin{tabular}[c]{@{}c@{}}Faithful ($\downarrow$)\end{tabular}}} \\ \midrule
 & \multicolumn{1}{l|}{\modelTopic \textbf{(Ours)}} & \textbf{0.8961*} & {\textbf{0.0998*}} & \multicolumn{1}{c|}{\textbf{0.0320*}} & \textbf{0.6056*} & \textbf{0.1650*} & \multicolumn{1}{c}{\textbf{0.0979*}} \\
\multirow{8}{*}{3} & \multicolumn{1}{l|}{\modelAll \textbf{(Ours)}} & {\ul 0.8664*} & {\ul 0.1062}* & \multicolumn{1}{c|}{{\ul 0.0359*}} & {\ul 0.5420} & {\ul 0.1896*} & \multicolumn{1}{c}{{\ul 0.1217}} \\
 & \multicolumn{1}{l|}{Long-Context} & 0.5320 & 0.1834 & \multicolumn{1}{c|}{0.1395} & 0.2662 & 0.3614 & \multicolumn{1}{c}{0.3173} \\
 & \multicolumn{1}{l|}{RAG-\textit{All}} & 0.6325 & 0.1557 & \multicolumn{1}{c|}{0.0898} & 0.3098 & 0.3499 & \multicolumn{1}{c}{0.2825} \\
 & \multicolumn{1}{l|}{RAG-\textit{Doc}} & 0.6909 & 0.1529 & \multicolumn{1}{c|}{0.0776} & 0.3356 & 0.3476 & \multicolumn{1}{c}{0.2752} \\
 & \multicolumn{1}{l|}{Hierarchical} & 0.7647 & 0.1191 & \multicolumn{1}{c|}{{0.0575}} & 0.3509 & 0.3032 & \multicolumn{1}{c}{0.2523} \\
 & \multicolumn{1}{l|}{Incremental-\textit{All}} & 0.5037 & 0.2466 & \multicolumn{1}{c|}{0.1924} & 0.3467 & 0.3019 & \multicolumn{1}{c}{0.2488} \\
 & \multicolumn{1}{l|}{Incremental-\textit{Topic}} & 0.5635 & 0.2288 & \multicolumn{1}{c|}{0.1720} & 0.4209 & 0.2796 & \multicolumn{1}{c}{0.2236} \\ \bottomrule
\end{tabular}
\caption{\label{table:doc_cover_cqa_fixedtopic}ConflictingQA citation coverage, balance, and accuracy when models have fixed topics (except RAG and RAG+Cluster). Best model is \textbf{bold}, second best is \underline{underlined}. Models with * are significantly the best (2-sample $t$-test, $p<0.05$ with Bonferroni correction. \model consistently has the highest citation coverage, fairness, and faithfulness for summaries and topic paragraphs, even when baselines use the same topics, suggesting that our gains are not derived from the agenda planning step, but rather question tailoring and outline construction. }
\end{table*}





\begin{table*}[t]
\centering
\footnotesize
\setlength{\tabcolsep}{2.75pt}
\renewcommand{\arraystretch}{0.6}
\begin{tabular}{@{}clcccccc@{}}
\multicolumn{1}{l}{} &  & \multicolumn{3}{c}{\textit{Summary Level}} & \multicolumn{3}{c}{\textit{Topic Paragraph Level}} \\ 
\toprule
\textbf{\# Top.} & \multicolumn{1}{l|}{\textbf{Model}} & \textbf{DC ($\uparrow$)} & \textbf{Fair ($\downarrow$)} & \multicolumn{1}{c|}{\textbf{Faithful ($\downarrow$)}} & \textbf{DC ($\uparrow$)} & \textbf{Fair ($\downarrow$)} & \multicolumn{1}{c}{\textbf{Faithful ($\downarrow$)}} \\ \midrule
\multirow{10}{*}{3} & \multicolumn{1}{l|}{\modelTopic \textbf{(Ours)}} & \textbf{0.8724*} & \textbf{0.0701*} & \multicolumn{1}{c|}{\textbf{0.0235*}} & \textbf{0.6066*} & \textbf{0.1255*} & \multicolumn{1}{c}{\textbf{0.0789*}} \\
 & \multicolumn{1}{l|}{\modelAll \textbf{(Ours)}} & {\ul 0.8457*} & {\ul 0.0786*} & \multicolumn{1}{c|}{{\ul 0.0273*}} & {\ul 0.5508} & {\ul 0.1463*} & \multicolumn{1}{c}{{\ul 0.0938*}} \\
 & \multicolumn{1}{l|}{Long-Context} & 0.6025 & 0.1919 & \multicolumn{1}{c|}{0.1559} & 0.2956 & 0.3865 & \multicolumn{1}{c}{0.3517} \\
 & \multicolumn{1}{l|}{RAG-\textit{All}} & 0.6200 & 0.1502 & \multicolumn{1}{c|}{0.0968} & 0.3103 & 0.3421 & \multicolumn{1}{c}{0.2896} \\
 & \multicolumn{1}{l|}{RAG-\textit{Doc}} & 0.6728 & 0.1216 & \multicolumn{1}{c|}{0.0683} & 0.3254 & 0.3226 & \multicolumn{1}{c}{0.2694} \\
 & \multicolumn{1}{l|}{Hierarchical} & 0.7676 & 0.0954 & \multicolumn{1}{c|}{0.0443} & 0.3650 & 0.2729 & \multicolumn{1}{c}{0.2207} \\
 & \multicolumn{1}{l|}{Incremental-\textit{All}} & 0.5566 & 0.2579 & \multicolumn{1}{c|}{0.2089} & 0.3919 & 0.3243 & \multicolumn{1}{c}{0.2765} \\
 & \multicolumn{1}{l|}{Incremental-\textit{Topic}} & 0.6152 & 0.2415 & \multicolumn{1}{c|}{0.1970} & 0.4707 & 0.3128 & \multicolumn{1}{c}{0.2674}\\ \bottomrule
\end{tabular}

\caption{\label{table:doc_cover_debate_fixedtopic}DebateQFS citation coverage, balance, and accuracy when models have fixed topics (except RAG and RAG+Cluster). Best model is \textbf{bold}, second best is \underline{underlined}. Models with * are significantly the best (2-sample $t$-test, $p<0.05$ with Bonferroni correction. \model consistently has the highest citation coverage, fairness, and faithfulness for summaries and topic paragraphs, even when baselines use the same topics, suggesting that our gains are not derived from the agenda planning step, but rather question tailoring and outline construction. }
\vspace{-2ex}
\end{table*}
\begin{table*}[]
\definecolor{myblue}{HTML}{DAE8FC}
\small
\centering
\setlength{\tabcolsep}{3.5pt}
\renewcommand{\arraystretch}{0.8}
\begin{tabular}{@{}clrrrrrrrrrrrrrrrc@{}}
\multicolumn{1}{l}{} &  & \multicolumn{5}{c}{\textit{Summary Quality}} & \multicolumn{5}{c}{\textit{Topic Paragraph Quality}} & \multicolumn{5}{c}{\textit{Topic Quality}} & \multicolumn{1}{l}{\textit{Sep.}} \\ \midrule
\textbf{\textbf{\# Topics}} & \multicolumn{1}{l|}{\textbf{Model}} & \multicolumn{1}{c}{\textbf{Int}} & \multicolumn{1}{c}{\textbf{Coh}} & \multicolumn{1}{c}{\textbf{Rel}} & \multicolumn{1}{l}{\textbf{Cov}} & \multicolumn{1}{l|}{\textbf{Div}} & \multicolumn{1}{c}{\textbf{Int}} & \multicolumn{1}{c}{\textbf{Coh}} & \multicolumn{1}{c}{\textbf{Rel}} & \multicolumn{1}{l}{\textbf{Cov}} & \multicolumn{1}{l|}{\textbf{Div}} & \multicolumn{1}{c}{\textbf{Int}} & \multicolumn{1}{c}{\textbf{Coh}} & \multicolumn{1}{c}{\textbf{Rel}} & \multicolumn{1}{l}{\textbf{Cov}} & \multicolumn{1}{l|}{\textbf{Div}} & \textbf{SB} \\ \midrule
 & \multicolumn{1}{l|}{\textbf{\modelTopic}} & \cellcolor[HTML]{DAE8FC}\textbf{4.22} & \cellcolor[HTML]{DAE8FC}4.24 & \cellcolor[HTML]{DAE8FC}4.59 & \cellcolor[HTML]{DAE8FC}4.46 & \multicolumn{1}{r|}{\cellcolor[HTML]{DAE8FC}\textbf{4.23}} & \cellcolor[HTML]{DAE8FC}\textbf{4.09} & \cellcolor[HTML]{DAE8FC}4.30 & \cellcolor[HTML]{DAE8FC}\textbf{4.70} & \cellcolor[HTML]{DAE8FC}\textbf{4.38} & \multicolumn{1}{r|}{\cellcolor[HTML]{DAE8FC}\textbf{3.93}} & \cellcolor[HTML]{DAE8FC}3.22 & \cellcolor[HTML]{DAE8FC}3.88 & \cellcolor[HTML]{DAE8FC}4.56 & \cellcolor[HTML]{DAE8FC}3.00 & \multicolumn{1}{r|}{\cellcolor[HTML]{DAE8FC}3.48} & 0.52 \\
 & \multicolumn{1}{l|}{\textbf{\modelAll}} & \cellcolor[HTML]{DAE8FC}4.12 & \cellcolor[HTML]{DAE8FC}\textbf{4.27} & \cellcolor[HTML]{DAE8FC}\textbf{4.68} & \cellcolor[HTML]{DAE8FC}\textbf{4.49} & \multicolumn{1}{r|}{\cellcolor[HTML]{DAE8FC}4.14} & \cellcolor[HTML]{DAE8FC}3.99 & \cellcolor[HTML]{DAE8FC}4.31 & \cellcolor[HTML]{DAE8FC}4.64 & \cellcolor[HTML]{DAE8FC}4.29 & \multicolumn{1}{r|}{\cellcolor[HTML]{DAE8FC}3.80} & \cellcolor[HTML]{DAE8FC}\textbf{3.27} & \cellcolor[HTML]{DAE8FC}3.93 & \cellcolor[HTML]{DAE8FC}4.52 & \cellcolor[HTML]{DAE8FC}\textbf{3.19} & \multicolumn{1}{r|}{\cellcolor[HTML]{DAE8FC}\textbf{3.70}} & 0.50 \\
 & \multicolumn{1}{l|}{Long-Context} & 3.96 & \cellcolor[HTML]{DAE8FC}4.18 & \cellcolor[HTML]{DAE8FC}4.55 & 4.31 & \multicolumn{1}{r|}{3.85} & 3.72 & 4.14 & 4.51 & 4.03 & \multicolumn{1}{r|}{3.25} & 3.00 & \cellcolor[HTML]{DAE8FC}3.86 & 4.47 & 2.90 & \multicolumn{1}{r|}{\cellcolor[HTML]{DAE8FC}3.47} & 0.45 \\
 & \multicolumn{1}{l|}{RAG-\textit{All}} & \cellcolor[HTML]{DAE8FC}4.06 & \cellcolor[HTML]{DAE8FC}4.24 & \cellcolor[HTML]{DAE8FC}4.55 & \cellcolor[HTML]{DAE8FC}4.43 & \multicolumn{1}{r|}{4.00} & 3.80 & \cellcolor[HTML]{DAE8FC}4.25 & 4.60 & 4.13 & \multicolumn{1}{r|}{3.63} & \cellcolor[HTML]{DAE8FC}3.08 & \cellcolor[HTML]{DAE8FC}3.86 & \cellcolor[HTML]{DAE8FC}4.51 & 2.81 & \multicolumn{1}{r|}{3.42} & 0.47 \\
 & \multicolumn{1}{l|}{RAG-\textit{Doc}} & \cellcolor[HTML]{DAE8FC}4.17 & \cellcolor[HTML]{DAE8FC}4.22 & \cellcolor[HTML]{DAE8FC}4.56 & \cellcolor[HTML]{DAE8FC}4.39 & \multicolumn{1}{r|}{\cellcolor[HTML]{DAE8FC}4.16} & 3.86 & \cellcolor[HTML]{DAE8FC}4.26 & \cellcolor[HTML]{DAE8FC}4.64 & 4.24 & \multicolumn{1}{r|}{3.71} & \cellcolor[HTML]{DAE8FC}3.10 & \cellcolor[HTML]{DAE8FC}3.88 & \cellcolor[HTML]{DAE8FC}4.59 & 2.84 & \multicolumn{1}{r|}{3.41} & 0.47 \\
 & \multicolumn{1}{l|}{Hierarchical} & \cellcolor[HTML]{DAE8FC}4.16 & \cellcolor[HTML]{DAE8FC}4.24 & \cellcolor[HTML]{DAE8FC}4.58 & \cellcolor[HTML]{DAE8FC}4.46 & \multicolumn{1}{r|}{\cellcolor[HTML]{DAE8FC}4.14} & 3.93 & \cellcolor[HTML]{DAE8FC}\textbf{4.33} & \cellcolor[HTML]{DAE8FC}\textbf{4.70} & 4.27 & \multicolumn{1}{r|}{3.76} & \cellcolor[HTML]{DAE8FC}3.21 & \cellcolor[HTML]{DAE8FC}3.90 & \cellcolor[HTML]{DAE8FC}\textbf{4.61} & \cellcolor[HTML]{DAE8FC}3.18 & \multicolumn{1}{r|}{\cellcolor[HTML]{DAE8FC}3.47} & 0.47 \\
 & \multicolumn{1}{l|}{Increm-\textit{All}} & 3.95 & \cellcolor[HTML]{DAE8FC}4.14 & \cellcolor[HTML]{DAE8FC}4.58 & 4.28 & \multicolumn{1}{r|}{3.90} & 3.64 & 4.11 & 4.57 & 4.01 & \multicolumn{1}{r|}{3.31} & \cellcolor[HTML]{DAE8FC}3.14 & \cellcolor[HTML]{DAE8FC}\textbf{3.97} & \cellcolor[HTML]{DAE8FC}4.60 & \cellcolor[HTML]{DAE8FC}3.07 & \multicolumn{1}{r|}{3.46} & 0.46 \\
 & \multicolumn{1}{l|}{Increm-\textit{Topic}} & \cellcolor[HTML]{DAE8FC}4.11 & \cellcolor[HTML]{DAE8FC}4.21 & \cellcolor[HTML]{DAE8FC}4.60 & \cellcolor[HTML]{DAE8FC}4.44 & \multicolumn{1}{r|}{\cellcolor[HTML]{DAE8FC}4.18} & \cellcolor[HTML]{DAE8FC}4.05 & \cellcolor[HTML]{DAE8FC}4.30 & \cellcolor[HTML]{DAE8FC}4.66 & 4.21 & \multicolumn{1}{r|}{3.76} & \cellcolor[HTML]{DAE8FC}3.03 & 3.63 & 4.37 & 2.83 & \multicolumn{1}{r|}{3.30} & 0.49 \\
 & \multicolumn{1}{l|}{Cluster} & 3.89 & 4.08 & 4.45 & 4.22 & \multicolumn{1}{r|}{3.94} & 3.73 & 4.11 & 4.49 & 4.04 & \multicolumn{1}{r|}{3.50} & 2.41 & 3.16 & 3.89 & 2.29 & \multicolumn{1}{r|}{2.47} & 0.48 \\
\multirow{-10}{*}{2} & \multicolumn{1}{l|}{RAG+Cluster} & \cellcolor[HTML]{DAE8FC}4.13 & \cellcolor[HTML]{DAE8FC}\textbf{4.27} & \cellcolor[HTML]{DAE8FC}4.59 & \cellcolor[HTML]{DAE8FC}4.38 & \multicolumn{1}{r|}{\cellcolor[HTML]{DAE8FC}4.07} & \cellcolor[HTML]{DAE8FC}3.97 & \cellcolor[HTML]{DAE8FC}4.29 & \cellcolor[HTML]{DAE8FC}4.67 & \cellcolor[HTML]{DAE8FC}4.30 & \multicolumn{1}{r|}{\cellcolor[HTML]{DAE8FC}3.87} & 2.53 & 3.26 & 4.04 & 2.49 & \multicolumn{1}{r|}{2.60} & 0.52 \\ \midrule
 & \multicolumn{1}{l|}{\textbf{\modelTopic}} & \cellcolor[HTML]{DAE8FC}4.24 & \cellcolor[HTML]{DAE8FC}4.34 & \cellcolor[HTML]{DAE8FC}4.64 & \cellcolor[HTML]{DAE8FC}4.49 & \multicolumn{1}{r|}{\cellcolor[HTML]{DAE8FC}\textbf{4.42}} & \cellcolor[HTML]{DAE8FC}\textbf{4.08} & \cellcolor[HTML]{DAE8FC}\textbf{4.33} & \cellcolor[HTML]{DAE8FC}4.69 & \cellcolor[HTML]{DAE8FC}\textbf{4.34} & \multicolumn{1}{r|}{\cellcolor[HTML]{DAE8FC}\textbf{3.89}} & \cellcolor[HTML]{DAE8FC}3.47 & \cellcolor[HTML]{DAE8FC}\textbf{4.12} & \cellcolor[HTML]{DAE8FC}\textbf{4.69} & \cellcolor[HTML]{DAE8FC}\textbf{3.61} & \multicolumn{1}{r|}{\cellcolor[HTML]{DAE8FC}\textbf{4.02}} & 0.69 \\
 & \multicolumn{1}{l|}{\textbf{\modelAll}} & \cellcolor[HTML]{DAE8FC}\textbf{4.27} & \cellcolor[HTML]{DAE8FC}4.33 & \cellcolor[HTML]{DAE8FC}4.63 & \cellcolor[HTML]{DAE8FC}4.49 & \multicolumn{1}{r|}{\cellcolor[HTML]{DAE8FC}4.40} & 3.88 & \cellcolor[HTML]{DAE8FC}4.27 & 4.60 & 4.19 & \multicolumn{1}{r|}{3.70} & \cellcolor[HTML]{DAE8FC}\textbf{3.49} & \cellcolor[HTML]{DAE8FC}4.09 & \cellcolor[HTML]{DAE8FC}4.62 & \cellcolor[HTML]{DAE8FC}3.46 & \multicolumn{1}{r|}{\cellcolor[HTML]{DAE8FC}3.99} & 0.65 \\
 & \multicolumn{1}{l|}{Long-Context} & 4.02 & \cellcolor[HTML]{DAE8FC}4.34 & \cellcolor[HTML]{DAE8FC}4.63 & \cellcolor[HTML]{DAE8FC}4.44 & \multicolumn{1}{r|}{4.23} & 3.62 & 4.14 & 4.51 & 3.89 & \multicolumn{1}{r|}{3.21} & 3.24 & \cellcolor[HTML]{DAE8FC}4.03 & \cellcolor[HTML]{DAE8FC}4.55 & 3.25 & \multicolumn{1}{r|}{3.76} & 0.58 \\
 & \multicolumn{1}{l|}{RAG-\textit{All}} & \cellcolor[HTML]{DAE8FC}4.16 & \cellcolor[HTML]{DAE8FC}4.33 & \cellcolor[HTML]{DAE8FC}4.67 & \cellcolor[HTML]{DAE8FC}4.49 & \multicolumn{1}{r|}{\cellcolor[HTML]{DAE8FC}4.29} & 3.80 & 4.16 & 4.61 & 4.06 & \multicolumn{1}{r|}{3.53} & \cellcolor[HTML]{DAE8FC}3.41 & \cellcolor[HTML]{DAE8FC}4.08 & \cellcolor[HTML]{DAE8FC}4.57 & \cellcolor[HTML]{DAE8FC}3.47 & \multicolumn{1}{r|}{\cellcolor[HTML]{DAE8FC}3.95} & 0.60 \\
 & \multicolumn{1}{l|}{RAG-\textit{Doc}} & \cellcolor[HTML]{DAE8FC}4.15 & \cellcolor[HTML]{DAE8FC}\textbf{4.37} & \cellcolor[HTML]{DAE8FC}4.68 & \cellcolor[HTML]{DAE8FC}4.47 & \multicolumn{1}{r|}{\cellcolor[HTML]{DAE8FC}\textbf{4.42}} & 3.76 & 4.22 & 4.60 & 4.10 & \multicolumn{1}{r|}{3.56} & \cellcolor[HTML]{DAE8FC}3.33 & \cellcolor[HTML]{DAE8FC}4.08 & \cellcolor[HTML]{DAE8FC}4.63 & \cellcolor[HTML]{DAE8FC}3.39 & \multicolumn{1}{r|}{\cellcolor[HTML]{DAE8FC}3.91} & 0.60 \\
 & \multicolumn{1}{l|}{Hierarchical} & 4.24 & \cellcolor[HTML]{DAE8FC}\textbf{4.37} & \cellcolor[HTML]{DAE8FC}4.73 & \cellcolor[HTML]{DAE8FC}4.50 & \multicolumn{1}{r|}{\cellcolor[HTML]{DAE8FC}4.38} & 3.78 & 4.21 & 4.62 & 4.14 & \multicolumn{1}{r|}{3.57} & \cellcolor[HTML]{DAE8FC}3.43 & \cellcolor[HTML]{DAE8FC}4.07 & \cellcolor[HTML]{DAE8FC}4.65 & \cellcolor[HTML]{DAE8FC}3.49 & \multicolumn{1}{r|}{\cellcolor[HTML]{DAE8FC}3.94} & 0.58 \\
 & \multicolumn{1}{l|}{Increm-\textit{All}} & 3.98 & \cellcolor[HTML]{DAE8FC}4.29 & \cellcolor[HTML]{DAE8FC}4.67 & 4.42 & \multicolumn{1}{r|}{4.21} & 3.54 & 4.09 & 4.56 & 3.79 & \multicolumn{1}{r|}{3.26} & \cellcolor[HTML]{DAE8FC}3.44 & \cellcolor[HTML]{DAE8FC}4.02 & \cellcolor[HTML]{DAE8FC}4.65 & \cellcolor[HTML]{DAE8FC}3.52 & \multicolumn{1}{r|}{\cellcolor[HTML]{DAE8FC}3.94} & 0.58 \\
 & \multicolumn{1}{l|}{Increm-\textit{Topic}} & 4.17 & \cellcolor[HTML]{DAE8FC}\textbf{4.37} & \cellcolor[HTML]{DAE8FC}\textbf{4.74} & \cellcolor[HTML]{DAE8FC}\textbf{4.57} & \multicolumn{1}{r|}{\cellcolor[HTML]{DAE8FC}4.39} & 3.91 & \cellcolor[HTML]{DAE8FC}4.29 & 4.62 & 4.25 & \multicolumn{1}{r|}{3.65} & \cellcolor[HTML]{DAE8FC}3.36 & 3.79 & 4.31 & 3.21 & \multicolumn{1}{r|}{3.73} & 0.61 \\
 & \multicolumn{1}{l|}{Cluster} & 3.81 & 4.03 & 4.25 & 4.19 & \multicolumn{1}{r|}{3.94} & 3.69 & 4.08 & 4.45 & 3.95 & \multicolumn{1}{r|}{3.53} & 2.42 & 2.86 & 3.73 & 2.13 & \multicolumn{1}{r|}{2.47} & 0.61 \\
\multirow{-10}{*}{3} & \multicolumn{1}{l|}{RAG+Cluster} & \cellcolor[HTML]{DAE8FC}4.14 & 4.22 & 4.60 & \cellcolor[HTML]{DAE8FC}4.52 & \multicolumn{1}{r|}{4.22} & 3.96 & \cellcolor[HTML]{DAE8FC}4.31 & \cellcolor[HTML]{DAE8FC}\textbf{4.71} & 4.25 & \multicolumn{1}{r|}{3.77} & 2.43 & 3.11 & 3.82 & 2.44 & \multicolumn{1}{r|}{2.64} & 0.64 \\ \midrule
 & \multicolumn{1}{l|}{\textbf{\modelTopic}} & \cellcolor[HTML]{DAE8FC}\textbf{4.30} & \cellcolor[HTML]{DAE8FC}4.21 & \cellcolor[HTML]{DAE8FC}4.54 & \cellcolor[HTML]{DAE8FC}\textbf{4.54} & \multicolumn{1}{r|}{\cellcolor[HTML]{DAE8FC}\textbf{4.48}} & \cellcolor[HTML]{DAE8FC}\textbf{4.09} & \cellcolor[HTML]{DAE8FC}\textbf{4.29} & \cellcolor[HTML]{DAE8FC}\textbf{4.66} & \cellcolor[HTML]{DAE8FC}\textbf{4.35} & \multicolumn{1}{r|}{\cellcolor[HTML]{DAE8FC}\textbf{3.89}} & \cellcolor[HTML]{DAE8FC}\textbf{3.93} & \cellcolor[HTML]{DAE8FC}4.17 & \cellcolor[HTML]{DAE8FC}4.65 & \cellcolor[HTML]{DAE8FC}4.04 & \multicolumn{1}{r|}{\cellcolor[HTML]{DAE8FC}\textbf{4.31}} & 0.72 \\
 & \multicolumn{1}{l|}{\textbf{\modelAll}} & \cellcolor[HTML]{DAE8FC}4.24 & \cellcolor[HTML]{DAE8FC}4.26 & \cellcolor[HTML]{DAE8FC}4.53 & \cellcolor[HTML]{DAE8FC}4.49 & \multicolumn{1}{r|}{\cellcolor[HTML]{DAE8FC}4.38} & 3.93 & \cellcolor[HTML]{DAE8FC}4.24 & \cellcolor[HTML]{DAE8FC}4.61 & 4.20 & \multicolumn{1}{r|}{3.76} & \cellcolor[HTML]{DAE8FC}3.80 & \cellcolor[HTML]{DAE8FC}\textbf{4.22} & \cellcolor[HTML]{DAE8FC}\textbf{4.67} & \cellcolor[HTML]{DAE8FC}\textbf{4.13} & \multicolumn{1}{r|}{\cellcolor[HTML]{DAE8FC}4.30} & 0.70 \\
 & \multicolumn{1}{l|}{Long-Context} & \cellcolor[HTML]{DAE8FC}4.14 & \cellcolor[HTML]{DAE8FC}4.26 & \cellcolor[HTML]{DAE8FC}4.48 & 4.32 & \multicolumn{1}{r|}{4.25} & 3.53 & 4.08 & 4.47 & 3.83 & \multicolumn{1}{r|}{3.14} & 3.65 & 4.00 & 4.53 & 3.85 & \multicolumn{1}{r|}{4.04} & 0.65 \\
 & \multicolumn{1}{l|}{RAG-\textit{All}} & \cellcolor[HTML]{DAE8FC}4.17 & \cellcolor[HTML]{DAE8FC}4.30 & \cellcolor[HTML]{DAE8FC}4.55 & \cellcolor[HTML]{DAE8FC}4.45 & \multicolumn{1}{r|}{4.29} & 3.72 & 4.18 & 4.59 & 3.99 & \multicolumn{1}{r|}{3.44} & \cellcolor[HTML]{DAE8FC}3.80 & \cellcolor[HTML]{DAE8FC}4.14 & \cellcolor[HTML]{DAE8FC}4.65 & \cellcolor[HTML]{DAE8FC}4.03 & \multicolumn{1}{r|}{\cellcolor[HTML]{DAE8FC}4.23} & 0.66 \\
 & \multicolumn{1}{l|}{RAG-\textit{Doc}} & \cellcolor[HTML]{DAE8FC}4.23 & \cellcolor[HTML]{DAE8FC}\textbf{4.31} & \cellcolor[HTML]{DAE8FC}4.56 & \cellcolor[HTML]{DAE8FC}4.41 & \multicolumn{1}{r|}{\cellcolor[HTML]{DAE8FC}4.41} & 3.76 & 4.16 & 4.59 & 4.07 & \multicolumn{1}{r|}{3.45} & 3.66 & \cellcolor[HTML]{DAE8FC}4.15 & \cellcolor[HTML]{DAE8FC}4.59 & \cellcolor[HTML]{DAE8FC}4.03 & \multicolumn{1}{r|}{\cellcolor[HTML]{DAE8FC}4.19} & 0.66 \\
 & \multicolumn{1}{l|}{Hierarchical} & \cellcolor[HTML]{DAE8FC}4.23 & \cellcolor[HTML]{DAE8FC}4.24 & \cellcolor[HTML]{DAE8FC}\textbf{4.59} & \cellcolor[HTML]{DAE8FC}4.51 & \multicolumn{1}{r|}{\cellcolor[HTML]{DAE8FC}4.36} & 3.75 & 4.19 & 4.59 & 4.06 & \multicolumn{1}{r|}{3.47} & 3.67 & \cellcolor[HTML]{DAE8FC}4.10 & \cellcolor[HTML]{DAE8FC}4.62 & 3.90 & \multicolumn{1}{r|}{\cellcolor[HTML]{DAE8FC}4.22} & 0.65 \\
 & \multicolumn{1}{l|}{Increm-\textit{All}} & 3.95 & 4.14 & 4.44 & 4.30 & \multicolumn{1}{r|}{4.15} & 3.48 & 4.04 & 4.49 & 3.76 & \multicolumn{1}{r|}{3.14} & 3.71 & \cellcolor[HTML]{DAE8FC}4.09 & \cellcolor[HTML]{DAE8FC}4.62 & \cellcolor[HTML]{DAE8FC}4.02 & \multicolumn{1}{r|}{\cellcolor[HTML]{DAE8FC}4.16} & 0.65 \\
 & \multicolumn{1}{l|}{Increm-\textit{Topic}} & \cellcolor[HTML]{DAE8FC}4.20 & \cellcolor[HTML]{DAE8FC}4.23 & 4.43 & \cellcolor[HTML]{DAE8FC}4.42 & \multicolumn{1}{r|}{\cellcolor[HTML]{DAE8FC}4.38} & 3.93 & \cellcolor[HTML]{DAE8FC}4.25 & \cellcolor[HTML]{DAE8FC}\textbf{4.66} & 4.17 & \multicolumn{1}{r|}{3.60} & 3.47 & 3.85 & 4.39 & 3.69 & \multicolumn{1}{r|}{3.83} & 0.69 \\
 & \multicolumn{1}{l|}{Cluster} & 3.92 & 4.03 & 4.17 & 4.20 & \multicolumn{1}{r|}{4.09} & 3.68 & 4.13 & 4.47 & 3.99 & \multicolumn{1}{r|}{3.51} & 2.36 & 2.73 & 3.62 & 2.27 & \multicolumn{1}{r|}{2.54} & 0.68 \\
\multirow{-10}{*}{4} & \multicolumn{1}{l|}{RAG+Cluster} & \cellcolor[HTML]{DAE8FC}4.21 & \cellcolor[HTML]{DAE8FC}4.20 & 4.44 & \cellcolor[HTML]{DAE8FC}4.44 & \multicolumn{1}{r|}{4.28} & 3.99 & \cellcolor[HTML]{DAE8FC}4.28 & \cellcolor[HTML]{DAE8FC}\textbf{4.66} & 4.26 & \multicolumn{1}{r|}{\cellcolor[HTML]{DAE8FC}3.83} & 2.56 & 3.05 & 3.95 & 2.58 & \multicolumn{1}{r|}{2.69} & 0.71 \\ \midrule
 & \multicolumn{1}{l|}{\textbf{\modelTopic}} & \cellcolor[HTML]{DAE8FC}4.17 & \cellcolor[HTML]{DAE8FC}4.24 & \cellcolor[HTML]{DAE8FC}4.35 & \cellcolor[HTML]{DAE8FC}\textbf{4.51} & \multicolumn{1}{r|}{\cellcolor[HTML]{DAE8FC}4.35} & \cellcolor[HTML]{DAE8FC}\textbf{4.08} & \cellcolor[HTML]{DAE8FC}\textbf{4.33} & \cellcolor[HTML]{DAE8FC}\textbf{4.69} & \cellcolor[HTML]{DAE8FC}\textbf{4.40} & \multicolumn{1}{r|}{\cellcolor[HTML]{DAE8FC}\textbf{3.97}} & \cellcolor[HTML]{DAE8FC}\textbf{4.15} & \cellcolor[HTML]{DAE8FC}\textbf{4.43} & \cellcolor[HTML]{DAE8FC}\textbf{4.82} & \cellcolor[HTML]{DAE8FC}\textbf{4.44} & \multicolumn{1}{r|}{\cellcolor[HTML]{DAE8FC}4.52} & 0.76 \\
 & \multicolumn{1}{l|}{\textbf{\modelAll}} & \cellcolor[HTML]{DAE8FC}\textbf{4.25} & \cellcolor[HTML]{DAE8FC}4.22 & \cellcolor[HTML]{DAE8FC}4.41 & \cellcolor[HTML]{DAE8FC}4.44 & \multicolumn{1}{r|}{\cellcolor[HTML]{DAE8FC}4.39} & 3.89 & 4.24 & 4.60 & 4.21 & \multicolumn{1}{r|}{3.69} & \cellcolor[HTML]{DAE8FC}4.14 & \cellcolor[HTML]{DAE8FC}4.37 & \cellcolor[HTML]{DAE8FC}4.77 & \cellcolor[HTML]{DAE8FC}\textbf{4.44} & \multicolumn{1}{r|}{\cellcolor[HTML]{DAE8FC}4.50} & 0.74 \\
 & \multicolumn{1}{l|}{Long-Context} & 3.98 & 4.11 & 4.28 & 4.29 & \multicolumn{1}{r|}{4.12} & 3.50 & 4.10 & 4.46 & 3.83 & \multicolumn{1}{r|}{3.02} & 3.90 & \cellcolor[HTML]{DAE8FC}4.35 & \cellcolor[HTML]{DAE8FC}4.71 & 4.22 & \multicolumn{1}{r|}{4.37} & 0.69 \\
 & \multicolumn{1}{l|}{RAG-\textit{All}} & \cellcolor[HTML]{DAE8FC}4.11 & \cellcolor[HTML]{DAE8FC}4.24 & \cellcolor[HTML]{DAE8FC}\textbf{4.48} & \cellcolor[HTML]{DAE8FC}4.48 & \multicolumn{1}{r|}{4.28} & 3.69 & 4.18 & 4.56 & 3.99 & \multicolumn{1}{r|}{3.39} & \cellcolor[HTML]{DAE8FC}4.02 & \cellcolor[HTML]{DAE8FC}4.39 & \cellcolor[HTML]{DAE8FC}4.80 & \cellcolor[HTML]{DAE8FC}4.36 & \multicolumn{1}{r|}{\cellcolor[HTML]{DAE8FC}4.46} & 0.71 \\
 & \multicolumn{1}{l|}{RAG-\textit{Doc}} & \cellcolor[HTML]{DAE8FC}4.12 & \cellcolor[HTML]{DAE8FC}4.20 & \cellcolor[HTML]{DAE8FC}\textbf{4.48} & \cellcolor[HTML]{DAE8FC}4.42 & \multicolumn{1}{r|}{\cellcolor[HTML]{DAE8FC}\textbf{4.50}} & 3.74 & 4.21 & 4.57 & 4.01 & \multicolumn{1}{r|}{3.42} & \cellcolor[HTML]{DAE8FC}3.96 & \cellcolor[HTML]{DAE8FC}4.36 & \cellcolor[HTML]{DAE8FC}4.78 & \cellcolor[HTML]{DAE8FC}4.32 & \multicolumn{1}{r|}{\cellcolor[HTML]{DAE8FC}4.41} & 0.70 \\
 & \multicolumn{1}{l|}{Hierarchical} & \cellcolor[HTML]{DAE8FC}4.07 & \cellcolor[HTML]{DAE8FC}\textbf{4.27} & 4.47 & \cellcolor[HTML]{DAE8FC}4.42 & \multicolumn{1}{r|}{\cellcolor[HTML]{DAE8FC}4.41} & 3.69 & 4.17 & 4.55 & 4.01 & \multicolumn{1}{r|}{3.39} & \cellcolor[HTML]{DAE8FC}4.07 & \cellcolor[HTML]{DAE8FC}4.35 & \cellcolor[HTML]{DAE8FC}4.80 & \cellcolor[HTML]{DAE8FC}4.37 & \multicolumn{1}{r|}{\cellcolor[HTML]{DAE8FC}\textbf{4.56}} & 0.70 \\
 & \multicolumn{1}{l|}{Increm-\textit{All}} & 3.83 & 4.09 & \cellcolor[HTML]{DAE8FC}4.35 & 4.27 & \multicolumn{1}{r|}{4.05} & 3.38 & 4.00 & 4.42 & 3.66 & \multicolumn{1}{r|}{2.98} & \cellcolor[HTML]{DAE8FC}4.06 & \cellcolor[HTML]{DAE8FC}4.41 & \cellcolor[HTML]{DAE8FC}4.74 & \cellcolor[HTML]{DAE8FC}4.29 & \multicolumn{1}{r|}{\cellcolor[HTML]{DAE8FC}4.44} & 0.69 \\
 & \multicolumn{1}{l|}{Increm-\textit{Topic}} & \cellcolor[HTML]{DAE8FC}4.05 & \cellcolor[HTML]{DAE8FC}4.22 & \cellcolor[HTML]{DAE8FC}4.34 & \cellcolor[HTML]{DAE8FC}4.34 & \multicolumn{1}{r|}{4.25} & 3.86 & 4.24 & 4.64 & 4.14 & \multicolumn{1}{r|}{3.57} & 3.69 & 4.00 & 4.52 & 3.96 & \multicolumn{1}{r|}{4.11} & 0.73 \\
 & \multicolumn{1}{l|}{Cluster} & 3.92 & 3.88 & 3.94 & 4.10 & \multicolumn{1}{r|}{4.07} & 3.74 & 4.09 & 4.46 & 4.00 & \multicolumn{1}{r|}{3.50} & 2.27 & 2.68 & 3.55 & 2.41 & \multicolumn{1}{r|}{2.48} & 0.73 \\
\multirow{-10}{*}{5} & \multicolumn{1}{l|}{RAG+Cluster} & 4.00 & 4.08 & 4.30 & 4.28 & \multicolumn{1}{r|}{4.29} & \cellcolor[HTML]{DAE8FC}4.00 & \cellcolor[HTML]{DAE8FC}4.30 & \cellcolor[HTML]{DAE8FC}4.66 & 4.28 & \multicolumn{1}{r|}{3.77} & 2.63 & 3.01 & 3.88 & 2.66 & \multicolumn{1}{r|}{2.62} & 0.75 \\ \bottomrule
\end{tabular}
\caption{\label{appendix:table:llm_cqa} Interest, Coherence, Relevance, Coverage, and Diversity scores from Prometheus for summaries, topic paragraphs, and topics on ConflictingQA. Best scores are \textbf{bold}, significant scores in \colorbox{myblue}{blue} (2-sample $t$-test, $p<0.05$)}
\end{table*}
\begin{table*}[]
\definecolor{myblue}{HTML}{DAE8FC}
\small
\centering
\setlength{\tabcolsep}{3.5pt}
\renewcommand{\arraystretch}{0.8}
\begin{tabular}{@{}clrrrrrrrrrrrrrrrc@{}}
\multicolumn{1}{l}{} &  & \multicolumn{5}{c}{\textit{Summary Quality}} & \multicolumn{5}{c}{\textit{Topic Paragraph Quality}} & \multicolumn{5}{c}{\textit{Topic Quality}} & \multicolumn{1}{l}{\textit{Sep.}} \\ \midrule
\textbf{\# Topics} & \multicolumn{1}{l|}{\textbf{Model}} & \multicolumn{1}{c}{\textbf{Int}} & \multicolumn{1}{c}{\textbf{Coh}} & \multicolumn{1}{c}{\textbf{Rel}} & \multicolumn{1}{l}{\textbf{Cov}} & \multicolumn{1}{l|}{\textbf{Div}} & \multicolumn{1}{c}{\textbf{Int}} & \multicolumn{1}{c}{\textbf{Coh}} & \multicolumn{1}{c}{\textbf{Rel}} & \multicolumn{1}{l}{\textbf{Cov}} & \multicolumn{1}{l|}{\textbf{Div}} & \multicolumn{1}{c}{\textbf{Int}} & \multicolumn{1}{c}{\textbf{Coh}} & \multicolumn{1}{c}{\textbf{Rel}} & \multicolumn{1}{l}{\textbf{Cov}} & \multicolumn{1}{l|}{\textbf{Div}} & \textbf{SB} \\ \midrule
 & \multicolumn{1}{l|}{\textbf{\modelTopic}} & \cellcolor[HTML]{DAE8FC}\textbf{4.16} & \cellcolor[HTML]{DAE8FC}4.13 & \cellcolor[HTML]{DAE8FC}4.53 & \cellcolor[HTML]{DAE8FC}\textbf{4.34} & \multicolumn{1}{r|}{\cellcolor[HTML]{DAE8FC}\textbf{4.15}} & \cellcolor[HTML]{DAE8FC}\textbf{4.03} & \cellcolor[HTML]{DAE8FC}4.20 & \cellcolor[HTML]{DAE8FC}\textbf{4.62} & \cellcolor[HTML]{DAE8FC}\textbf{4.22} & \multicolumn{1}{r|}{\cellcolor[HTML]{DAE8FC}\textbf{3.89}} & \cellcolor[HTML]{DAE8FC}\textbf{3.28} & \cellcolor[HTML]{DAE8FC}3.98 & \cellcolor[HTML]{DAE8FC}4.62 & \cellcolor[HTML]{DAE8FC}2.93 & \multicolumn{1}{r|}{\cellcolor[HTML]{DAE8FC}3.56} & 0.50 \\
 & \multicolumn{1}{l|}{\textbf{\modelAll}} & \cellcolor[HTML]{DAE8FC}3.98 & \cellcolor[HTML]{DAE8FC}4.10 & \cellcolor[HTML]{DAE8FC}4.45 & \cellcolor[HTML]{DAE8FC}\textbf{4.34} & \multicolumn{1}{r|}{\cellcolor[HTML]{DAE8FC}4.09} & \cellcolor[HTML]{DAE8FC}3.88 & \cellcolor[HTML]{DAE8FC}4.20 & 4.50 & \cellcolor[HTML]{DAE8FC}4.16 & \multicolumn{1}{r|}{\cellcolor[HTML]{DAE8FC}3.75} & \cellcolor[HTML]{DAE8FC}3.23 & \cellcolor[HTML]{DAE8FC}\textbf{4.01} & \cellcolor[HTML]{DAE8FC}4.61 & \cellcolor[HTML]{DAE8FC}\textbf{3.11} & \multicolumn{1}{r|}{\cellcolor[HTML]{DAE8FC}3.56} & 0.49 \\
 & \multicolumn{1}{l|}{Long-Context} & 3.79 & \cellcolor[HTML]{DAE8FC}4.07 & \cellcolor[HTML]{DAE8FC}4.48 & \cellcolor[HTML]{DAE8FC}4.19 & \multicolumn{1}{r|}{3.83} & 3.57 & \cellcolor[HTML]{DAE8FC}4.15 & \cellcolor[HTML]{DAE8FC}4.53 & 3.90 & \multicolumn{1}{r|}{3.28} & \cellcolor[HTML]{DAE8FC}3.15 & \cellcolor[HTML]{DAE8FC}3.81 & \cellcolor[HTML]{DAE8FC}4.56 & 2.70 & \multicolumn{1}{r|}{\cellcolor[HTML]{DAE8FC}3.51} & 0.46 \\
 & \multicolumn{1}{l|}{RAG-\textit{All}} & \cellcolor[HTML]{DAE8FC}4.02 & \cellcolor[HTML]{DAE8FC}4.08 & \cellcolor[HTML]{DAE8FC}4.46 & \cellcolor[HTML]{DAE8FC}4.20 & \multicolumn{1}{r|}{\cellcolor[HTML]{DAE8FC}3.96} & 3.72 & 4.11 & \cellcolor[HTML]{DAE8FC}4.54 & 4.04 & \multicolumn{1}{r|}{3.61} & \cellcolor[HTML]{DAE8FC}3.23 & \cellcolor[HTML]{DAE8FC}3.78 & \cellcolor[HTML]{DAE8FC}4.56 & \cellcolor[HTML]{DAE8FC}2.97 & \multicolumn{1}{r|}{\cellcolor[HTML]{DAE8FC}3.46} & 0.46 \\
 & \multicolumn{1}{l|}{RAG-\textit{Doc}} & 3.90 & \cellcolor[HTML]{DAE8FC}\textbf{4.18} & \cellcolor[HTML]{DAE8FC}4.54 & \cellcolor[HTML]{DAE8FC}4.28 & \multicolumn{1}{r|}{3.86} & 3.74 & 4.10 & \cellcolor[HTML]{DAE8FC}4.52 & 4.03 & \multicolumn{1}{r|}{3.60} & \cellcolor[HTML]{DAE8FC}3.08 & \cellcolor[HTML]{DAE8FC}3.97 & \cellcolor[HTML]{DAE8FC}4.63 & \cellcolor[HTML]{DAE8FC}2.95 & \multicolumn{1}{r|}{\cellcolor[HTML]{DAE8FC}3.55} & 0.47 \\
 & \multicolumn{1}{l|}{Hierarchical} & 4.08 & \cellcolor[HTML]{DAE8FC}4.16 & \cellcolor[HTML]{DAE8FC}\textbf{4.55} & \cellcolor[HTML]{DAE8FC}4.28 & \multicolumn{1}{r|}{\cellcolor[HTML]{DAE8FC}4.05} & \cellcolor[HTML]{DAE8FC}3.94 & \cellcolor[HTML]{DAE8FC}4.21 & \cellcolor[HTML]{DAE8FC}4.56 & 4.07 & \multicolumn{1}{r|}{3.62} & \cellcolor[HTML]{DAE8FC}3.13 & \cellcolor[HTML]{DAE8FC}3.90 & \cellcolor[HTML]{DAE8FC}\textbf{4.64} & \cellcolor[HTML]{DAE8FC}3.06 & \multicolumn{1}{r|}{\cellcolor[HTML]{DAE8FC}\textbf{3.60}} & 0.47 \\
 & \multicolumn{1}{l|}{Increm-\textit{All}} & 3.81 & \cellcolor[HTML]{DAE8FC}4.04 & \cellcolor[HTML]{DAE8FC}4.50 & \cellcolor[HTML]{DAE8FC}4.25 & \multicolumn{1}{r|}{\cellcolor[HTML]{DAE8FC}3.93} & 3.65 & 4.08 & \cellcolor[HTML]{DAE8FC}4.51 & 3.79 & \multicolumn{1}{r|}{3.40} & \cellcolor[HTML]{DAE8FC}3.15 & \cellcolor[HTML]{DAE8FC}3.99 & \cellcolor[HTML]{DAE8FC}4.62 & \cellcolor[HTML]{DAE8FC}2.92 & \multicolumn{1}{r|}{\cellcolor[HTML]{DAE8FC}3.58} & 0.45 \\
 & \multicolumn{1}{l|}{Increm-\textit{Topic}} & \cellcolor[HTML]{DAE8FC}3.91 & \cellcolor[HTML]{DAE8FC}4.18 & \cellcolor[HTML]{DAE8FC}4.54 & \cellcolor[HTML]{DAE8FC}4.19 & \multicolumn{1}{r|}{\cellcolor[HTML]{DAE8FC}4.12} & \cellcolor[HTML]{DAE8FC}3.92 & \cellcolor[HTML]{DAE8FC}\textbf{4.25} & \cellcolor[HTML]{DAE8FC}4.57 & \cellcolor[HTML]{DAE8FC}4.14 & \multicolumn{1}{r|}{3.70} & 2.86 & 3.59 & 4.19 & 2.64 & \multicolumn{1}{r|}{3.09} & 0.48 \\
 & \multicolumn{1}{l|}{Cluster} & \cellcolor[HTML]{DAE8FC}3.91 & 4.01 & 4.35 & 4.09 & \multicolumn{1}{r|}{3.87} & 3.75 & 4.01 & 4.36 & 3.90 & \multicolumn{1}{r|}{3.50} & 2.72 & 3.40 & 4.03 & 2.38 & \multicolumn{1}{r|}{3.02} & 0.45 \\
\multirow{-10}{*}{2} & \multicolumn{1}{l|}{RAG+Cluster} & \cellcolor[HTML]{DAE8FC}3.98 & \cellcolor[HTML]{DAE8FC}4.11 & \cellcolor[HTML]{DAE8FC}4.44 & \cellcolor[HTML]{DAE8FC}4.27 & \multicolumn{1}{r|}{\cellcolor[HTML]{DAE8FC}3.99} & 3.72 & \cellcolor[HTML]{DAE8FC}4.17 & \cellcolor[HTML]{DAE8FC}4.56 & 4.04 & \multicolumn{1}{r|}{3.62} & 2.96 & 3.73 & \cellcolor[HTML]{DAE8FC}4.56 & 2.57 & \multicolumn{1}{r|}{3.26} & 0.48 \\ \midrule
 & \multicolumn{1}{l|}{\textbf{\modelTopic}} & \cellcolor[HTML]{DAE8FC}4.02 & \cellcolor[HTML]{DAE8FC}4.20 & 4.49 & \cellcolor[HTML]{DAE8FC}\textbf{4.44} & \multicolumn{1}{r|}{\cellcolor[HTML]{DAE8FC}4.34} & \cellcolor[HTML]{DAE8FC}\textbf{3.97} & \cellcolor[HTML]{DAE8FC}\textbf{4.21} & \cellcolor[HTML]{DAE8FC}\textbf{4.55} & \cellcolor[HTML]{DAE8FC}\textbf{4.14} & \multicolumn{1}{r|}{\cellcolor[HTML]{DAE8FC}\textbf{3.82}} & \cellcolor[HTML]{DAE8FC}3.54 & \cellcolor[HTML]{DAE8FC}4.09 & \cellcolor[HTML]{DAE8FC}4.64 & 3.39 & \multicolumn{1}{r|}{\cellcolor[HTML]{DAE8FC}3.93} & 0.67 \\
 & \multicolumn{1}{l|}{\textbf{\modelAll}} & \cellcolor[HTML]{DAE8FC}4.11 & \cellcolor[HTML]{DAE8FC}4.21 & \cellcolor[HTML]{DAE8FC}4.60 & \cellcolor[HTML]{DAE8FC}4.34 & \multicolumn{1}{r|}{\cellcolor[HTML]{DAE8FC}\textbf{4.36}} & \cellcolor[HTML]{DAE8FC}3.83 & \cellcolor[HTML]{DAE8FC}4.15 & \cellcolor[HTML]{DAE8FC}4.51 & \cellcolor[HTML]{DAE8FC}4.10 & \multicolumn{1}{r|}{3.63} & \cellcolor[HTML]{DAE8FC}\textbf{3.61} & \cellcolor[HTML]{DAE8FC}4.11 & \cellcolor[HTML]{DAE8FC}4.67 & \cellcolor[HTML]{DAE8FC}\textbf{3.71} & \multicolumn{1}{r|}{\cellcolor[HTML]{DAE8FC}4.02} & 0.64 \\
 & \multicolumn{1}{l|}{Long-Context} & \cellcolor[HTML]{DAE8FC}3.94 & \cellcolor[HTML]{DAE8FC}4.13 & \cellcolor[HTML]{DAE8FC}4.54 & 4.32 & \multicolumn{1}{r|}{4.14} & 3.54 & 4.09 & \cellcolor[HTML]{DAE8FC}4.46 & 3.80 & \multicolumn{1}{r|}{3.17} & \cellcolor[HTML]{DAE8FC}3.36 & \cellcolor[HTML]{DAE8FC}4.09 & \cellcolor[HTML]{DAE8FC}4.69 & 3.36 & \multicolumn{1}{r|}{\cellcolor[HTML]{DAE8FC}4.04} & 0.59 \\
 & \multicolumn{1}{l|}{RAG-\textit{All}} & \cellcolor[HTML]{DAE8FC}4.04 & \cellcolor[HTML]{DAE8FC}4.20 & \cellcolor[HTML]{DAE8FC}4.59 & \cellcolor[HTML]{DAE8FC}4.25 & \multicolumn{1}{r|}{4.14} & 3.62 & 4.06 & \cellcolor[HTML]{DAE8FC}4.49 & 3.87 & \multicolumn{1}{r|}{3.47} & \cellcolor[HTML]{DAE8FC}3.56 & \cellcolor[HTML]{DAE8FC}4.11 & \cellcolor[HTML]{DAE8FC}4.64 & 3.46 & \multicolumn{1}{r|}{\cellcolor[HTML]{DAE8FC}3.97} & 0.59 \\
 & \multicolumn{1}{l|}{RAG-\textit{Doc}} & \cellcolor[HTML]{DAE8FC}4.19 & \cellcolor[HTML]{DAE8FC}\textbf{4.25} & \cellcolor[HTML]{DAE8FC}4.59 & \cellcolor[HTML]{DAE8FC}4.33 & \multicolumn{1}{r|}{4.08} & 3.59 & 4.06 & \cellcolor[HTML]{DAE8FC}4.49 & 3.88 & \multicolumn{1}{r|}{3.36} & \cellcolor[HTML]{DAE8FC}3.56 & \cellcolor[HTML]{DAE8FC}4.10 & \cellcolor[HTML]{DAE8FC}4.62 & \cellcolor[HTML]{DAE8FC}3.51 & \multicolumn{1}{r|}{\cellcolor[HTML]{DAE8FC}3.97} & 0.59 \\
 & \multicolumn{1}{l|}{Hierarchical} & \cellcolor[HTML]{DAE8FC}4.15 & \cellcolor[HTML]{DAE8FC}4.17 & \cellcolor[HTML]{DAE8FC}\textbf{4.69} & \cellcolor[HTML]{DAE8FC}4.35 & \multicolumn{1}{r|}{\cellcolor[HTML]{DAE8FC}4.33} & 3.74 & 4.09 & \cellcolor[HTML]{DAE8FC}4.53 & 3.96 & \multicolumn{1}{r|}{3.48} & \cellcolor[HTML]{DAE8FC}3.56 & \cellcolor[HTML]{DAE8FC}\textbf{4.22} & \cellcolor[HTML]{DAE8FC}\textbf{4.70} & \cellcolor[HTML]{DAE8FC}3.63 & \multicolumn{1}{r|}{\cellcolor[HTML]{DAE8FC}\textbf{4.16}} & 0.58 \\
 & \multicolumn{1}{l|}{Increm-\textit{All}} & 3.92 & 4.08 & 4.52 & \cellcolor[HTML]{DAE8FC}4.29 & \multicolumn{1}{r|}{4.08} & 3.50 & 3.98 & \cellcolor[HTML]{DAE8FC}4.46 & 3.75 & \multicolumn{1}{r|}{3.25} & \cellcolor[HTML]{DAE8FC}3.36 & \cellcolor[HTML]{DAE8FC}4.12 & \cellcolor[HTML]{DAE8FC}4.61 & 3.25 & \multicolumn{1}{r|}{3.75} & 0.58 \\
 & \multicolumn{1}{l|}{Increm-\textit{Topic}} & \cellcolor[HTML]{DAE8FC}\textbf{4.25} & \cellcolor[HTML]{DAE8FC}4.19 & \cellcolor[HTML]{DAE8FC}4.61 & \cellcolor[HTML]{DAE8FC}4.41 & \multicolumn{1}{r|}{\cellcolor[HTML]{DAE8FC}4.23} & \cellcolor[HTML]{DAE8FC}3.91 & \cellcolor[HTML]{DAE8FC}4.17 & \cellcolor[HTML]{DAE8FC}\textbf{4.55} & \cellcolor[HTML]{DAE8FC}4.06 & \multicolumn{1}{r|}{\cellcolor[HTML]{DAE8FC}3.68} & 3.09 & 3.66 & 4.30 & 3.03 & \multicolumn{1}{r|}{3.56} & 0.60 \\
 & \multicolumn{1}{l|}{Cluster} & 3.92 & 3.97 & 4.34 & 4.08 & \multicolumn{1}{r|}{4.06} & 3.64 & 3.95 & 4.31 & 3.82 & \multicolumn{1}{r|}{3.39} & 2.67 & 3.41 & 3.97 & 2.53 & \multicolumn{1}{r|}{3.16} & 0.59 \\
\multirow{-10}{*}{3} & \multicolumn{1}{l|}{RAG+Cluster} & \cellcolor[HTML]{DAE8FC}4.11 & \cellcolor[HTML]{DAE8FC}4.16 & 4.49 & \cellcolor[HTML]{DAE8FC}4.37 & \multicolumn{1}{r|}{\cellcolor[HTML]{DAE8FC}4.23} & \cellcolor[HTML]{DAE8FC}3.83 & \cellcolor[HTML]{DAE8FC}4.18 & \cellcolor[HTML]{DAE8FC}4.54 & \cellcolor[HTML]{DAE8FC}4.11 & \multicolumn{1}{r|}{\cellcolor[HTML]{DAE8FC}3.69} & 3.08 & 3.80 & 4.40 & 2.87 & \multicolumn{1}{r|}{3.31} & 0.61 \\ \midrule
 & \multicolumn{1}{l|}{\textbf{\modelTopic}} & \cellcolor[HTML]{DAE8FC}4.15 & \cellcolor[HTML]{DAE8FC}4.08 & \cellcolor[HTML]{DAE8FC}4.45 & \cellcolor[HTML]{DAE8FC}4.37 & \multicolumn{1}{r|}{\cellcolor[HTML]{DAE8FC}\textbf{4.40}} & \cellcolor[HTML]{DAE8FC}\textbf{4.06} & \cellcolor[HTML]{DAE8FC}\textbf{4.20} & \cellcolor[HTML]{DAE8FC}4.54 & \cellcolor[HTML]{DAE8FC}\textbf{4.20} & \multicolumn{1}{r|}{\cellcolor[HTML]{DAE8FC}\textbf{3.94}} & \cellcolor[HTML]{DAE8FC}3.80 & \cellcolor[HTML]{DAE8FC}4.12 & \cellcolor[HTML]{DAE8FC}4.68 & \cellcolor[HTML]{DAE8FC}4.11 & \multicolumn{1}{r|}{\cellcolor[HTML]{DAE8FC}4.19} & 0.71 \\
 & \multicolumn{1}{l|}{\textbf{\modelAll}} & \cellcolor[HTML]{DAE8FC}\textbf{4.21} & \cellcolor[HTML]{DAE8FC}4.14 & \cellcolor[HTML]{DAE8FC}\textbf{4.48} & \cellcolor[HTML]{DAE8FC}\textbf{4.39} & \multicolumn{1}{r|}{\cellcolor[HTML]{DAE8FC}4.30} & 3.82 & \cellcolor[HTML]{DAE8FC}4.12 & \cellcolor[HTML]{DAE8FC}4.49 & 4.02 & \multicolumn{1}{r|}{3.68} & \cellcolor[HTML]{DAE8FC}\textbf{3.93} & \cellcolor[HTML]{DAE8FC}\textbf{4.18} & 4.58 & \cellcolor[HTML]{DAE8FC}4.07 & \multicolumn{1}{r|}{\cellcolor[HTML]{DAE8FC}4.21} & 0.69 \\
 & \multicolumn{1}{l|}{Long-Context} & 3.92 & \cellcolor[HTML]{DAE8FC}4.07 & \cellcolor[HTML]{DAE8FC}4.40 & 4.15 & \multicolumn{1}{r|}{4.14} & 3.48 & 4.04 & 4.43 & 3.70 & \multicolumn{1}{r|}{3.07} & \cellcolor[HTML]{DAE8FC}3.83 & \cellcolor[HTML]{DAE8FC}4.14 & 4.56 & \cellcolor[HTML]{DAE8FC}4.02 & \multicolumn{1}{r|}{\cellcolor[HTML]{DAE8FC}4.21} & 0.65 \\
 & \multicolumn{1}{l|}{RAG-\textit{All}} & 3.93 & \cellcolor[HTML]{DAE8FC}4.04 & \cellcolor[HTML]{DAE8FC}4.36 & \cellcolor[HTML]{DAE8FC}4.27 & \multicolumn{1}{r|}{\cellcolor[HTML]{DAE8FC}4.18} & 3.55 & 4.02 & 4.45 & 3.83 & \multicolumn{1}{r|}{3.30} & \cellcolor[HTML]{DAE8FC}3.79 & \cellcolor[HTML]{DAE8FC}4.16 & \cellcolor[HTML]{DAE8FC}4.64 & \cellcolor[HTML]{DAE8FC}4.02 & \multicolumn{1}{r|}{\cellcolor[HTML]{DAE8FC}4.21} & 0.66 \\
 & \multicolumn{1}{l|}{RAG-\textit{Doc}} & 3.96 & \cellcolor[HTML]{DAE8FC}3.99 & \cellcolor[HTML]{DAE8FC}4.31 & \cellcolor[HTML]{DAE8FC}4.34 & \multicolumn{1}{r|}{\cellcolor[HTML]{DAE8FC}4.24} & 3.64 & 4.05 & \cellcolor[HTML]{DAE8FC}4.51 & 3.87 & \multicolumn{1}{r|}{3.31} & \cellcolor[HTML]{DAE8FC}3.80 & \cellcolor[HTML]{DAE8FC}4.08 & \cellcolor[HTML]{DAE8FC}4.63 & \cellcolor[HTML]{DAE8FC}\textbf{4.15} & \multicolumn{1}{r|}{\cellcolor[HTML]{DAE8FC}4.14} & 0.66 \\
 & \multicolumn{1}{l|}{Hierarchical} & \cellcolor[HTML]{DAE8FC}4.05 & \cellcolor[HTML]{DAE8FC}\textbf{4.16} & \cellcolor[HTML]{DAE8FC}4.44 & \cellcolor[HTML]{DAE8FC}4.34 & \multicolumn{1}{r|}{\cellcolor[HTML]{DAE8FC}4.37} & 3.63 & 4.07 & \cellcolor[HTML]{DAE8FC}4.49 & 3.87 & \multicolumn{1}{r|}{3.44} & \cellcolor[HTML]{DAE8FC}3.80 & \cellcolor[HTML]{DAE8FC}4.15 & \cellcolor[HTML]{DAE8FC}\textbf{4.75} & \cellcolor[HTML]{DAE8FC}4.04 & \multicolumn{1}{r|}{\cellcolor[HTML]{DAE8FC}\textbf{4.28}} & 0.66 \\
 & \multicolumn{1}{l|}{Increm-\textit{All}} & 3.93 & \cellcolor[HTML]{DAE8FC}4.06 & \cellcolor[HTML]{DAE8FC}4.36 & \cellcolor[HTML]{DAE8FC}4.19 & \multicolumn{1}{r|}{\cellcolor[HTML]{DAE8FC}4.19} & 3.45 & 4.02 & 4.45 & 3.68 & \multicolumn{1}{r|}{3.24} & \cellcolor[HTML]{DAE8FC}3.82 & \cellcolor[HTML]{DAE8FC}4.12 & \cellcolor[HTML]{DAE8FC}4.66 & \cellcolor[HTML]{DAE8FC}4.09 & \multicolumn{1}{r|}{\cellcolor[HTML]{DAE8FC}4.09} & 0.65 \\
 & \multicolumn{1}{l|}{Increm-\textit{Topic}} & \cellcolor[HTML]{DAE8FC}4.05 & \cellcolor[HTML]{DAE8FC}4.08 & 4.25 & \cellcolor[HTML]{DAE8FC}4.34 & \multicolumn{1}{r|}{\cellcolor[HTML]{DAE8FC}4.33} & 3.90 & \cellcolor[HTML]{DAE8FC}4.18 & \cellcolor[HTML]{DAE8FC}\textbf{4.56} & \cellcolor[HTML]{DAE8FC}4.12 & \multicolumn{1}{r|}{3.67} & 3.61 & \cellcolor[HTML]{DAE8FC}3.98 & 4.39 & 3.77 & \multicolumn{1}{r|}{3.93} & 0.69 \\
 & \multicolumn{1}{l|}{Cluster} & \cellcolor[HTML]{DAE8FC}3.97 & \cellcolor[HTML]{DAE8FC}4.01 & 4.17 & 4.05 & \multicolumn{1}{r|}{4.15} & 3.70 & 4.00 & 4.32 & 3.82 & \multicolumn{1}{r|}{3.48} & 3.01 & 3.56 & 4.07 & 3.02 & \multicolumn{1}{r|}{3.37} & 0.66 \\
\multirow{-10}{*}{4} & \multicolumn{1}{l|}{RAG+Cluster} & \cellcolor[HTML]{DAE8FC}4.15 & \cellcolor[HTML]{DAE8FC}3.97 & \cellcolor[HTML]{DAE8FC}4.32 & \cellcolor[HTML]{DAE8FC}4.23 & \multicolumn{1}{r|}{\cellcolor[HTML]{DAE8FC}4.26} & 3.83 & 4.09 & \cellcolor[HTML]{DAE8FC}4.54 & 4.04 & \multicolumn{1}{r|}{3.56} & 3.29 & 3.60 & 4.17 & 3.24 & \multicolumn{1}{r|}{3.50} & 0.66 \\ \midrule
 & \multicolumn{1}{l|}{\textbf{\modelTopic}} & \cellcolor[HTML]{DAE8FC}\textbf{4.16} & \cellcolor[HTML]{DAE8FC}\textbf{4.14} & \cellcolor[HTML]{DAE8FC}4.36 & \cellcolor[HTML]{DAE8FC}\textbf{4.28} & \multicolumn{1}{r|}{\cellcolor[HTML]{DAE8FC}\textbf{4.40}} & \cellcolor[HTML]{DAE8FC}\textbf{4.05} & \cellcolor[HTML]{DAE8FC}\textbf{4.25} & \cellcolor[HTML]{DAE8FC}\textbf{4.58} & \cellcolor[HTML]{DAE8FC}\textbf{4.27} & \multicolumn{1}{r|}{\cellcolor[HTML]{DAE8FC}\textbf{3.89}} & \cellcolor[HTML]{DAE8FC}4.04 & \cellcolor[HTML]{DAE8FC}4.37 & \cellcolor[HTML]{DAE8FC}4.77 & \cellcolor[HTML]{DAE8FC}4.33 & \multicolumn{1}{r|}{\cellcolor[HTML]{DAE8FC}4.49} & 0.75 \\
 & \multicolumn{1}{l|}{\textbf{\modelAll}} & \cellcolor[HTML]{DAE8FC}4.07 & \cellcolor[HTML]{DAE8FC}4.03 & \cellcolor[HTML]{DAE8FC}\textbf{4.37} & \cellcolor[HTML]{DAE8FC}4.25 & \multicolumn{1}{r|}{\cellcolor[HTML]{DAE8FC}4.29} & 3.79 & 4.09 & 4.48 & 3.99 & \multicolumn{1}{r|}{3.59} & \cellcolor[HTML]{DAE8FC}4.05 & \cellcolor[HTML]{DAE8FC}4.35 & \cellcolor[HTML]{DAE8FC}4.81 & \cellcolor[HTML]{DAE8FC}4.38 & \multicolumn{1}{r|}{\cellcolor[HTML]{DAE8FC}\textbf{4.54}} & 0.73 \\
 & \multicolumn{1}{l|}{Long-Context} & 3.79 & \cellcolor[HTML]{DAE8FC}3.99 & \cellcolor[HTML]{DAE8FC}4.20 & \cellcolor[HTML]{DAE8FC}4.14 & \multicolumn{1}{r|}{3.99} & 3.47 & 4.01 & 4.44 & 3.69 & \multicolumn{1}{r|}{3.05} & \cellcolor[HTML]{DAE8FC}4.07 & 4.27 & 4.74 & 4.23 & \multicolumn{1}{r|}{\cellcolor[HTML]{DAE8FC}4.40} & 0.70 \\
 & \multicolumn{1}{l|}{RAG-\textit{All}} & 3.90 & \cellcolor[HTML]{DAE8FC}3.91 & \cellcolor[HTML]{DAE8FC}4.23 & \cellcolor[HTML]{DAE8FC}4.14 & \multicolumn{1}{r|}{4.15} & 3.59 & 4.00 & 4.44 & 3.72 & \multicolumn{1}{r|}{3.33} & \cellcolor[HTML]{DAE8FC}\textbf{4.17} & \cellcolor[HTML]{DAE8FC}\textbf{4.44} & \cellcolor[HTML]{DAE8FC}4.87 & \cellcolor[HTML]{DAE8FC}4.36 & \multicolumn{1}{r|}{\cellcolor[HTML]{DAE8FC}4.52} & 0.70 \\
 & \multicolumn{1}{l|}{RAG-\textit{Doc}} & \cellcolor[HTML]{DAE8FC}3.93 & \cellcolor[HTML]{DAE8FC}3.98 & \cellcolor[HTML]{DAE8FC}4.30 & \cellcolor[HTML]{DAE8FC}4.23 & \multicolumn{1}{r|}{4.14} & 3.61 & 4.05 & 4.47 & 3.81 & \multicolumn{1}{r|}{3.31} & \cellcolor[HTML]{DAE8FC}4.05 & \cellcolor[HTML]{DAE8FC}4.43 & \cellcolor[HTML]{DAE8FC}4.84 & \cellcolor[HTML]{DAE8FC}\textbf{4.50} & \multicolumn{1}{r|}{\cellcolor[HTML]{DAE8FC}4.50} & 0.70 \\
 & \multicolumn{1}{l|}{Hierarchical} & 3.90 & \cellcolor[HTML]{DAE8FC}3.96 & \cellcolor[HTML]{DAE8FC}4.23 & \cellcolor[HTML]{DAE8FC}4.16 & \multicolumn{1}{r|}{4.09} & 3.60 & 4.09 & 4.48 & 3.85 & \multicolumn{1}{r|}{3.38} & \cellcolor[HTML]{DAE8FC}4.16 & \cellcolor[HTML]{DAE8FC}4.43 & \cellcolor[HTML]{DAE8FC}\textbf{4.87} & \cellcolor[HTML]{DAE8FC}4.52 & \multicolumn{1}{r|}{\cellcolor[HTML]{DAE8FC}4.52} & 0.70 \\
 & \multicolumn{1}{l|}{Increm-\textit{All}} & 3.80 & \cellcolor[HTML]{DAE8FC}4.07 & \cellcolor[HTML]{DAE8FC}4.23 & \cellcolor[HTML]{DAE8FC}4.09 & \multicolumn{1}{r|}{4.04} & 3.41 & 3.95 & 4.40 & 3.56 & \multicolumn{1}{r|}{3.09} & \cellcolor[HTML]{DAE8FC}4.08 & 4.26 & \cellcolor[HTML]{DAE8FC}4.76 & \cellcolor[HTML]{DAE8FC}4.36 & \multicolumn{1}{r|}{\cellcolor[HTML]{DAE8FC}4.37} & 0.68 \\
 & \multicolumn{1}{l|}{Increm-\textit{Topic}} & \cellcolor[HTML]{DAE8FC}4.04 & \cellcolor[HTML]{DAE8FC}4.10 & \cellcolor[HTML]{DAE8FC}4.21 & \cellcolor[HTML]{DAE8FC}4.16 & \multicolumn{1}{r|}{\cellcolor[HTML]{DAE8FC}4.16} & 3.84 & 4.16 & \cellcolor[HTML]{DAE8FC}4.55 & 4.06 & \multicolumn{1}{r|}{3.56} & 3.69 & 3.98 & 4.51 & 3.88 & \multicolumn{1}{r|}{3.99} & 0.73 \\
 & \multicolumn{1}{l|}{Cluster} & 3.86 & 3.90 & 3.96 & 3.98 & \multicolumn{1}{r|}{4.12} & 3.68 & 4.01 & 4.36 & 3.87 & \multicolumn{1}{r|}{3.42} & 3.14 & 3.60 & 4.19 & 3.31 & \multicolumn{1}{r|}{3.52} & 0.71 \\
\multirow{-10}{*}{5} & \multicolumn{1}{l|}{RAG+Cluster} & \cellcolor[HTML]{DAE8FC}4.03 & \cellcolor[HTML]{DAE8FC}3.96 & \cellcolor[HTML]{DAE8FC}4.19 & \cellcolor[HTML]{DAE8FC}4.19 & \multicolumn{1}{r|}{\cellcolor[HTML]{DAE8FC}4.21} & 3.79 & 4.10 & 4.50 & 4.04 & \multicolumn{1}{r|}{3.57} & 3.46 & 3.86 & 4.40 & 3.62 & \multicolumn{1}{r|}{3.69} & 0.72 \\ \bottomrule
\end{tabular}
\caption{\label{appendix:table:llm_debate} Interest, Coherence, Relevance, Coverage, and Diversity scores from Prometheus for summaries, topic paragraphs, and topics on DebateQFS. Best scores are \textbf{bold}, significant scores in \colorbox{myblue}{blue} (2-sample $t$-test, $p<0.05$)}
\end{table*}
\begin{table*}[]
\small
\centering
\setlength{\tabcolsep}{3.5pt}
\renewcommand{\arraystretch}{0.8}
\begin{tabular}{@{}cl|ccccc@{}}
\toprule
\textbf{\# Topics} & \textbf{Model} & \multicolumn{1}{l}{\textbf{\# Input Tokens}} & \multicolumn{1}{l}{\textbf{\# Output Tokens}} & \multicolumn{1}{l}{\textbf{\# LLM Calls}} & \multicolumn{1}{l}{\textbf{Cost (GPT-4)}} & \multicolumn{1}{l}{\textbf{Time (seconds)}} \\ \midrule
\multirow{3}{*}{2} & \modelTopic & 21383.08 & 3412.02 & 25.45 & 0.32 & 117.60 \\
 & Hierarchical & 31130.02 & 2536.66 & 13.15 & 0.39 & 83.13 \\
 & Incremental-\textit{Topic} & 59010.66 & 6115.04 & 15.15 & 0.77 & 214.39 \\ \midrule
\multirow{3}{*}{3} & \modelTopic & 30208.20 & 5040.38 & 37.38 & 0.45 & 149.54 \\
 & Hierarchical & 31144.83 & 2649.78 & 13.15 & 0.39 & 68.60 \\
 & Incremental-\textit{Topic} & 61344.07 & 8442.54 & 16.15 & 0.87 & 197.33 \\ \midrule
\multirow{3}{*}{4} & \modelTopic & 38286.40 & 6440.23 & 47.91 & 0.58 & 163.91 \\
 & Hierarchical & 31144.31 & 2740.31 & 13.15 & 0.39 & 88.75 \\
 & Incremental-\textit{Topic} & 62877.46 & 9966.45 & 17.15 & 0.93 & 312.55 \\ \midrule
\multirow{3}{*}{5} & \modelTopic & 47008.59 & 7918.92 & 58.94 & 0.71 & 186.32 \\
 & Hierarchical & 31160.88 & 2850.24 & 13.15 & 0.40 & 61.70 \\
 & Incremental-\textit{Topic} & 64893.95 & 11965.84 & 18.15 & 1.01 & 262.07 \\ \bottomrule
\end{tabular}
\caption{\label{appendix:table:cost_cqa} Number of LLM input/output tokens, LLM calls, GPT-4 Cost (USD), and Time (seconds) needed to run inference on a single DFQS example on ConflictingQA with the top-3 models. We report 5 runs and 20 examples.}
\end{table*}

\begin{table*}[]
\small
\centering
\setlength{\tabcolsep}{3.5pt}
\renewcommand{\arraystretch}{0.8}
\begin{tabular}{@{}cl|ccccc@{}}
\toprule
\multicolumn{1}{l}{\textbf{Dataset}} & \textbf{Model} & \multicolumn{1}{l}{\textbf{\# Input Tokens}} & \multicolumn{1}{l}{\textbf{\# Output Tokens}} & \multicolumn{1}{l}{\textbf{\# LLM Calls}} & \multicolumn{1}{l}{\textbf{Cost (GPT-4)}} & \multicolumn{1}{l}{\textbf{Time (seconds)}} \\ \midrule
\multirow{3}{*}{2} & \modelTopic & 17183.75 & 2722.40 & 20.30 & 0.25 & 94.81 \\
 & Hierarchical & 19181.59 & 2040.39 & 10.25 & 0.25 & 63.68 \\
 & Incremental-\textit{Topic} & 41656.87 & 5062.44 & 12.25 & 0.57 & 182.19 \\ 
 \midrule
\multirow{3}{*}{3} & \modelTopic & 24801.22 & 4136.12 & 30.40 & 0.37 & 126.83 \\
 & Hierarchical & 19182.58 & 2141.91 & 10.25 & 0.26 & 53.32 \\
 & Incremental-\textit{Topic} & 43119.51 & 6532.92 & 13.25 & 0.63 & 152.44 \\ \midrule
\multirow{3}{*}{4} & \modelTopic & 30677.67 & 5037.31 & 38.00 & 0.46 & 120.64 \\
 & Hierarchical & 19203.30 & 2253.17 & 10.25 & 0.26 & 73.35 \\
 & Incremental-\textit{Topic} & 43922.02 & 7327.88 & 14.25 & 0.66 & 241.54 \\ \midrule
\multirow{3}{*}{5} & \modelTopic & 36988.41 & 6049.93 & 46.09 & 0.55 & 139.71 \\
 & Hierarchical & 19211.74 & 2356.01 & 10.25 & 0.26 & 49.41 \\
 & Incremental-\textit{Topic} & 45113.12 & 8504.59 & 15.25 & 0.71 & 186.40 \\ \bottomrule
\end{tabular}
\caption{\label{appendix:table:cost_debate} Number of LLM input/output tokens, LLM calls, GPT-4 Cost (USD), and Time (seconds) needed to run inference on a single DFQS example on DebateQFS with the top-3 models. We report 5 runs and 20 examples.}
\end{table*}

\begin{table*}[]
\small
\centering
\setlength{\tabcolsep}{3.5pt}
\renewcommand{\arraystretch}{0.8}
\begin{tabular}{@{}cl|ccccc@{}}
\toprule
\multicolumn{1}{l}{\textbf{\# Topics}} & \textbf{Model} & \multicolumn{1}{l}{\textbf{\# Input Tokens}} & \multicolumn{1}{l}{\textbf{\# Output Tokens}} & \multicolumn{1}{l}{\textbf{\# LLM Calls}} & \multicolumn{1}{l}{\textbf{Cost (GPT-4)}} & \multicolumn{1}{l}{\textbf{Time (seconds)}} \\ 
\midrule
\multirow{3}{*}{ConflictingQA} & \modelTopic & 47008.59 & 7918.92 & 58.94 & 0.71 & 186.32 \\
 & \modelTopic Pick All & 53733.70 & 9596.75 & 71.75 & 0.83 & 303.13 \\
 & Hierarchical-\emph{Topic} & 168160.85 & 7485.50 & 66.75 & 1.91 & 210.80 \\ \midrule
\multirow{3}{*}{DebateQFS} & \modelTopic & 36988.41 & 6049.93 & 46.09 & 0.55 & 139.71 \\
& \modelTopic Pick All & 43098.85 & 7612.45 & 57.25 & 0.66 & 242.35 \\
& Hierarchical-\emph{Topic} & 105237.25 & 5278.35 & 52.25 & 1.21 & 139.96 \\ \bottomrule
\end{tabular}
\caption{\label{appendix:table:cost_weird} Number of LLM input/output tokens, LLM calls, GPT-4 Cost (USD), and Time (seconds) needed to run inference on a single DFQS example on ConflictingQA and DebateQFS with \modelTopic, the version of \modelTopic with no Moderator, and the version of Hierarchical merging that runs on each topic paragraph ($m=5$). We report 5 runs and 20 examples.}
\end{table*}


\begin{figure*}
    \centering
    \fbox{
    \includegraphics[width=\linewidth]{appendix/annot2.pdf}}
    \caption{\label{fig:annot} Distribution of Readability and Balance for Full Summaries and Topic Paragraphs from Prolific.}
\end{figure*}


\clearpage
\hypersetup{
    colorlinks=true, % Enable colored links
    linkcolor=black, % Default color for internal links (sections, etc.)
    citecolor=black, % Default color for citations
    urlcolor=black % Default color for external URLs
}

\begin{summary}[title={\modelTopic Summary: Are Audiobooks Considered Real Reading? (ConflictingQA)}, label=summary1]
\textbf{Topic 1: Audiobooks and cognitive engagement}\\
Audiobooks provide cognitive benefits similar to traditional reading, such as improved vocabulary, comprehension, and critical thinking skills, and engage the same cognitive processes as reading print books [1][2][3]. While listening to audiobooks, children and adults can visualize, make predictions, and synthesize information, indicating active cognitive engagement, although the lack of visual stimulation may affect the engagement for some individuals [3][4][8]. High-quality production and narration in audiobooks can enhance the cognitive experience, making them a valid form of reading for people with visual impairments, dyslexia, or those who multitask [7][9][11].\\
\\
\textbf{Topic 2: Audiobooks accessibility and inclusivity}\\
Audiobooks enhance accessibility and inclusivity by supporting children with different learning needs, allowing individuals with ADHD, dyslexia, and other attention disorders to enjoy literature and improve comprehension [1][2][11]. They enable multitasking and comprehension for various audiences, including those with visual impairments or who find traditional reading challenging, and are cost-effective to produce, increasing their accessibility [7][12]. Despite some cultural biases that may undervalue audiobooks compared to printed text, they are considered a valid form of reading, especially for those who may not engage with traditional reading methods [9][8].\\
\\
\textbf{Topic 3: Audiobooks versus physical reading}\\
Audiobooks provide similar cognitive benefits to traditional reading, such as improved vocabulary and comprehension, and engage the same cognitive processes as reading with eyes, making them a valid form of reading [1][2][3]. They are particularly beneficial for those with different learning needs or physical limitations, allowing for a rich, accessible, and multitask-friendly reading experience [3][6][7][11]. Despite some cultural perceptions that audiobooks are less valuable than printed text, they are increasingly popular, economically significant, and considered a legitimate form of storytelling and information absorption [8][9][12].
\end{summary}

\begin{summary}[title={\modelTopic Summary: Is infinite scrolling a good web design technique? (ConflictingQA)}, label=summary2]
\textbf{Topic 1: User Engagement Increase}\\
Infinite scrolling can potentially increase user engagement and time spent on a site by continuously loading new content, which exploits users' automatic behavior and keeps them engaged [9]. However, it has been found to decrease user engagement in some cases, such as on Etsy, and can negatively impact users with disabilities and mental health, leading to a high cognitive load and potential mental health issues [1][5][8]. Additionally, infinite scrolling can lead to control issues and user frustration due to less controllable pages and jumping glitches [6].\\
\\
\textbf{Topic 2: Content Accessibility Concerns}\\
Infinite scrolling can lead to content accessibility issues, as it breaks the expected behavior of scrollbars and makes it difficult for users to gauge the length of the page, and it poses significant challenges for users with assistive technologies, often excluding footers and making navigation stressful [1][6][7]. While it can keep users engaged on eCommerce platforms, it has been associated with increased stress levels and negative mental health outcomes, particularly in young social media users [3][6][8]. Moreover, strategies like role='feed' have failed to address these accessibility problems effectively [5].\\
\\
\textbf{Topic 3: Mental Health Implications}\\
Infinite scrolling can exploit human psychological phenomena such as automaticity, leading to behaviors like doom-scrolling that may contribute to mental health issues by causing users to lose track of time and continue scrolling unconsciously [9]. The design can also induce stress by preventing users from reaching a perceived end, leading to information overload, and overwhelming them with choices, which can result in frustration, anxiety, and a reduced motivation to engage with content [6][7]. However, some studies suggest that engaging in mindful scrolling practices can mitigate these negative mental health outcomes, indicating that the impact of infinite scrolling may vary based on user behavior [8].
\end{summary}

\clearpage

\begin{summary}[title={\modelTopic Summary: Is EU expansion and EU membership itself a good idea? (DebateQFS)}, label=summary3]
\textbf{Topic 1: Economic gains from accession}\\
The 1997 study by the Centre for Economic Policy Research predicted economic gains for both the EU-15 and new Central and Eastern European members, with an estimated €10 billion and €23 billion increase respectively [1]. However, concerns about high budget and trade deficits in accession countries, such as Estonia and Hungary, and the potential for increased unemployment and social costs, suggest that EU expansion could also exacerbate economic disparities and put fiscal pressure on both new and existing members [5][6]. Additionally, the enlargement is expected to shift regional funds towards new members, potentially reducing support for poorer regions within the EU(15) and necessitating a significant increase in the EU's regional funding budget to address growing economic and social needs [6].\\
\\
\textbf{Topic 2: EU enlargement political challenges}\\
EU enlargement is seen as beneficial, with studies indicating potential GDP growth for new and existing members, strategic interests in stabilizing regions like the Western Balkans and Turkey, and necessary controls in place to manage economic migration and regional subsidies [1][2][6]. However, public opposition in some member states, the slow process of enlargement due to political complexities, and concerns over social contradictions and international conflicts [2][5][6] present significant challenges. The Treaty of Lisbon is deemed necessary for further enlargement, although there are differing opinions on whether its ratification should delay the process [4].\\
\\
\textbf{Topic 3: Regional disparities and funding}\\
EU expansion has been estimated to bring economic gains for both old and new member states, with the EU-15 seeing a €10 billion increase and new Central and Eastern European members gaining €23 billion [1]. However, regional disparities pose challenges, as unemployment rates have risen in accession countries and the wealth gap between regions may widen, with 98 million inhabitants in applicant states living in regions with GDP less than 75\% of the EU average [5][6]. Despite the potential for increased regional funding, there are concerns that existing poorer regions within the EU(15) may receive less support as a result of the expansion [6].
\end{summary}

\begin{summary}[title={\modelTopic Summary: Is going to law school a good idea? (DebateQFS)}, label=summary4]

\textbf{Topic 1: Law School ROI Analysis}\\
Attending law school can lead to a variety of career opportunities and the acquisition of valuable skills, with some graduates finding employment directly from campus and others benefiting from practical skills-oriented courses [4][5][6]. However, the financial burden of law school is significant, with many students accruing substantial debt, facing uncertain job markets, and questioning the return on investment, especially if they do not graduate from top-tier schools or are not at the top of their class [8][10][15][19]. Despite the potential for high starting salaries in some legal jobs, the competitiveness of the market and the cost of tuition may not justify the investment for all students, particularly when considering the psychological toll and the oversupply of law graduates [12][14][16].\\
\\
\textbf{Topic 2: Legal Career Job Market}\\
The legal job market presents a mixed outlook, with some documents indicating an increase in law firm hiring practices and a demand for legal services in certain areas, while others highlight the oversaturation of law graduates, underemployment, and the potential for job dissatisfaction and misleading employment statistics from law schools [4][17][8][10][11][18][19]. Graduates from prestigious law schools or those in the top of their class may have better job prospects and higher starting salaries, but many face significant debt and struggle to find well-paying jobs to manage that debt [16][19]. The rise of legal process outsourcing and the hiring of law school graduates directly by companies suggest evolving trends in the legal job market that could affect future employment opportunities for lawyers [6][5].\\
\\
\textbf{Topic 3: Law Education Value Debate}\\
Law school provides a range of non-monetary benefits, such as personal growth, maturity, and the development of transferable skills like critical thinking and argumentation, which are applicable in various fields beyond traditional legal practice [3][9]. However, the financial implications of law school, including high tuition costs, significant student debt, and an uncertain job market, challenge the notion that a legal education is a sound financial investment for all students [13][14][17][19]. Despite these concerns, there is a demand for legal professionals, and law school can prepare graduates for diverse career paths, including roles that address complex societal challenges and ensure access to justice [2][7][16].
\end{summary}