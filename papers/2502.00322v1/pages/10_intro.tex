\section{Introduction}
\section{Introduction}
\label{sec:intro}

\begin{figure*}[tb]
    \centering
    \includegraphics[width=0.848\linewidth]{figs/circuitnn.pdf} 
    \caption{Illustration of differentiable CircuitNN. CircuitNN is designed based on differentiable NAND gates. After DAS is guided by PI and PO pairs of the truth table, CircuitNN can get the precise circuit architecture logic equivalent to the truth table.}
    \label{fig:circuitnn}
\end{figure*}

% 1. Describe the importance of logic synthesis
% 2. Existing Problems
% (a) Neural Architecture Search: Unstable, Predefined Setting, etc.
% (b) Circuit Generation: Probabilistic Model, Logic Equivalence

With the rapid advancement of technology, the scale of integrated circuits (ICs) has expanded exponentially. 
This expansion has introduced significant challenges in chip manufacturing, particularly concerning power and area metrics.
A primary objective in IC design is achieving the same circuit function with fewer transistors, thereby reducing power usage and area occupancy.

Logic synthesis~\cite{hachtel2005logicsynth}, a critical step in electronic design automation (EDA), transforms behavioral-level circuit designs into optimized gate-level circuits, ultimately yielding the final IC layout. 
The primary goal of logic synthesis is to identify the physical implementation with the fewest gates for a given circuit function. 
This task constitutes a challenging NP-hard combinatorial optimization problem. 
Current logic synthesis tools~\cite{brayton2010abc, wolf2013yosys} rely on human-designed heuristics, often leading to sub-optimal outcomes.

Differentiable architecture search (DAS) techniques~\cite{liu2018darts, chu2020darts} offer novel perspectives on addressing challenges in this problem.
Circuit functions can be represented through truth tables, which map binary inputs to their corresponding outputs. 
Truth tables provide a precise representation of input-output relationships, ensuring the design of functionally equivalent circuits.
Inspired by this, researchers~\cite{deepmind2024ai4sys, wang2024tnet} have begun exploring the application of DAS to synthesize circuits directly from truth tables.
Specifically, \citet{deepmind2024ai4sys} proposed CircuitNN, a framework that learns differentiable connection structures with logic gates, enabling the automatic generation of logic circuits from truth tables.
This approach significantly reduces the complexity of traditional circuit generation. 
Building on this, \citet{wang2024tnet} introduced T-Net, a triangle-shaped variant of CircuitNN, incorporating regularization techniques to enhance the efficiency of DAS.

Despite these advancements, several challenges remain. 
The computational complexity of DAS grows quadratically with the number of gates, posing scalability issues.
Although triangle-shaped architecture~\cite{wang2024tnet} partially mitigates this problem, redundancy persists. 
%Additionally, DAS is susceptible to converging to local optima, limiting the ability to search architectures that satisfy the given truth tables~\cite{liu2018darts}. 
%Furthermore, hyperparameters (network depth and layer width) require extensive searches, introducing complexity and prolonging the synthesis process. 
Additionally, DAS is susceptible to converging to local optima~\cite{liu2018darts} and hyperparameters (network depth and layer width) require extensive searches. 
The challenges arise from the vast search space in DAS. 
% Even with predefined settings for CircuitNN, finding a configuration that meets the truth table requires extensive trial and error during the DAS process. 
Intuitively, limiting the search space through predefined parameters (network depth, gates per layer, and connection probabilities) can significantly reduce the complexity.

Recent advances~\cite{openai2023gpt4, abramson2024alphafold3, esser2024sd3, li2024mar} in conditional generative models have demonstrated remarkable performance across language, vision, and graph generation tasks. 
Motivated by these developments, we propose a novel approach to circuit generation that generates preliminary circuit structures to guide DAS in generating refined circuits matching specified truth tables. 
Firstly, we introduce CircuitVQ, a tokenizer with a discrete codebook for circuit tokenization. 
Built upon our Circuit AutoEncoder framework~\cite{hou2022graphmae,li2023maskgae,wu2025mgvga}, CircuitVQ is trained through a circuit reconstruction task. 
Specifically, the CircuitVQ encoder encodes input circuits into discrete tokens using a learnable codebook, while the decoder reconstructs the circuit adjacency matrix based on these tokens.
Subsequently, the CircuitVQ encoder serves as a circuit tokenizer for CircuitAR pretraining, which employs a masked autoregressive modeling paradigm~\cite{chang2022maskgit, li2023mage}. 
In this process, the discrete codes function as supervision signals. 
After training, CircuitAR can generate discrete tokens progressively, which can be decoded into initial circuit structures by the decoder of the CircuitVQ. 
These prior insights can guide DAS in producing refined circuits that match the target truth tables precisely.

Our key contributions can be summarized as follows:
\begin{itemize}
\item We introduce CircuitVQ, a circuit tokenizer that facilitates graph autoregressive modeling for circuit generation, based on our Circuit AutoEncoder framework;
\item Develop CircuitAR, a model trained using masked autoregressive modeling, which generates initial circuit structures conditioned on given truth tables;
\item Propose a refinement framework that integrates differentiable architecture search to produce functionally equivalent circuits guided by target truth tables;
\item Comprehensive experiments demonstrating the scalability and capability emergence of our CircuitAR and the superior performance of the proposed circuit generation approach.
\end{itemize}

% Motivation
% (a) Diffusion (Vision, Graph), Autoregressive (Language, Vision)
% (b) Circuit Generation for Predefined Setting
% (c) Neural Architecture Search for Strict Logic Equivalence

% Contribution
% (a) Circuit Tokenizer (new transformer arch, training strategy)
% (b) CircuitAR (train and gen strategies, post-ar strategy)
% (c) Extensive Evaluation including BitD (Bit Distance) for Scalability

Query-focused summaries (QFS) give an overview of documents to answer a query~\cite{rosner2008multisum, el2021automatic}.
By combining each document's content useful for answering the query, or their \textbf{perspectives}~\cite{lin2006side}, these summaries can aid decision-making~\cite{hsu2021decision}.
For example, doctors pick treatments based on research paper perspectives~\cite{goff2008patients} and legislators vote based on perspectives in policy reports~\cite{jones1994reconceiving}. 
Past QFS work assumes documents have aligned perspectives~\cite{roy2023review}, but some queries, like ``\emph{Is law school worth it?}'', are debatable, containing opposing perspectives~\cite{wan2024evidence}.
In such cases, it is key to \textit{balance} perspectives from \textit{diverse} sources so users consider all sides before deciding~\cite{dale2015heuristics}.

To address this gap, we propose \textbf{\textit{debatable} QFS (DQFS}).
As input, DQFS uses documents and a debatable query, defined as a yes/no query where documents have opposing, equally-valid\footnote{This is meant to avoid input questions like ``Is the earth flat?'' where ``yes'' and ``no'' are not equally-valid (\cref{subsection:ethics}).} ``yes'' and ``no'' perspectives (Fig~\ref{fig:intro}).
Such queries are broad (\textit{Is law school worth it?}), and decomposing broad concepts into more specific topics (\textit{cost}, \textit{job market}) improves comprehension~\cite{johnson1983mental}.
Thus, DQFS creates a multi-aspect summary, with each paragraph covering one of an input number of topics ($2$ in Fig~\ref{fig:intro}).
The full summary and each paragraph must be \textit{comprehensive} and \textit{balanced}~(\cref{section:task}).
Comprehensive text has perspectives from all documents, while balanced text is not skewed towards the yes or no perspectives; our goals aid informed, unbiased decision-making~\cite{ziems2024measuring}.


While LLMs are deft summarizers~\cite{zhang2024benchmarking}, they cannot directly solve DQFS, as they fail to use diverse sources~\cite{huang-etal-2024-embrace}.
In Figure~\ref{fig:intro}, GPT-4 mainly gives perspectives favoring EU expansion (\textcolor{blue}{\textbf{blue}}), yielding a biased output.
Also, when asked for citations~\cite{huang-chang-2024-citation}, GPT-4 only cites 3/6 (\textcolor{yellowcite}{\textbf{yellow}}), missing half the documents' perspectives.
We intuit this arises since GPT-4 uses one inference step, with all documents in a single prompt.
This can omit document perspectives in certain positions of the prompt~\cite{liu2024lost} or that oppose parametric memory~\cite{jin2024tug}, reducing output coverage and balance.

Multi-LLM summarizers~\cite{chang2024booookscore, adams2023sparse}, which use LLMs to summarize documents individually into intermediate outputs before merging them with another LLM call, are better choices, as they represent documents more equally. 
However, they have two key issues.
\textbf{First}, they use the same topic or query as input to summarize each document, which is subpar if we wish to use retrieval in summarization to reduce LLM costs.
Queries unaligned to a document's unique content and expertise will fail to retrieve all of its most relevant contexts~\cite{sachan2022improving}; this reduces the total number of perspectives in the intermediate output, resulting in lower coverage.
\textbf{Second}, their intermediate outputs are unstructured, free-form texts, which are hard for the LLM to combine into a final output.
Free-form text needs extra reasoning to extract, classify, and compare the texts' perspectives~\cite{barrow2021syntopical}, steps that distract from the final goal of generating a balanced summary.

% A \textit{structured} intermediate output that clearly organizes documents and their perspectives on topics would greatly simplify the final step of synthesizing a balanced, comprehensive summary~\cite{shao2024assisting}.

To solve our issues, we build \textbf{\model} (Fig~\ref{fig:model}), a multi-LLM system using a \textbf{M}ixture \textbf{o}f \textbf{D}ocument \textbf{S}peakers.
Inspired by panel discussions~\cite{doumont2014english}, \model has a \textit{Speaker} LLM for each document that responds to queries using its document, and a \textit{Moderator} LLM that decides when and how speakers respond.
Specifically, \model: 1) plans an agenda of topics for the outline (\cref{subsection:agenda}); 2) picks a subset of speakers with relevant perspectives for each topic and tailors them a query (\cref{subsection:moderator}); and 3) asks each speaker to obtain its document's context relevant to the tailored query and give the context's ``yes'' and ``no'' perspectives for the topic. 
%All steps are efficiently done via~retrieval.

When a speaker supplies its document's perspectives, the topic, document number, tailored query, and perspectives update an outline, tracking the LLM discourse.
After the discussion, the outline is summarized for a DQFS output.
In all, \model frames DQFS as a discussion of document speakers to represent sources equally, tailors queries for speakers to optimize the retrieval of contexts used to find perspectives, and builds a structured outline of document perspectives to simplify the synthesis of a final output---a novel combination that leads to comprehensive and balanced summaries~(\cref{subsection:ablation}).

We compare \model to eight strong baselines~on ConflictingQA~\cite{wan2024evidence} and \textbf{DebateQFS} (\cref{subsection:datasets}), a new dataset for DQFS drawn from the debate community on Debatepedia~\cite{gottopati2013learning}.
To assess summaries, we have models give citations in their outputs (Fig~\ref{fig:intro}), showing the documents the model intends to use~\cite{huang-chang-2024-citation}.
Many works use citations for factuality~\cite{li2024citation}, but
we repurpose them for coverage and balance---measuring the proportion of documents cited and distribution of ground-truth yes/no perspective stances of cited documents (\cref{subsection:metrics}).


\model has the best document coverage and balance in full summaries and topic paragraphs (\cref{subsection:citation_comp}), surpassing SOTA by 38-58\% in paragraphs.
The Prometheus LLM~\cite{kim2024prometheus} ranks \model as one of the best models in summarization quality 28/30 times, the most of any model (\cref{subsection:summary_comp}).
Users also find \model's outputs to be the most balanced, and preserve readability despite using perspectives from more documents (\cref{subsection:human_eval}).
Lastly, analyses show the utility of tailoring queries and building outlines, which improve \model (\cref{subsection:ablation}) and offer rich, structured tools for users (\cref{subsection:qg}). Our contributions are:

\noindent \textbf{1)} We propose \textbf{debatable query-focused summarization}, a new task to help users navigate yes/no queries in documents with opposing perspectives. \\
\noindent \textbf{2)} We design \model, a multi-LLM DQFS system that treats documents as \textbf{individual} \textbf{LLM speakers}, uses a moderator to \textbf{tailor queries} to apt speakers, and tracks speaker perspectives in an \textbf{outline}. \\
\noindent \textbf{3)} We release \textbf{DebateQFS} for DQFS and \textbf{citation metrics} to capture summary coverage and~balance. \\
\noindent \textbf{4)} Experiments show \model \textbf{beats baselines by 38-58\%} in topic paragraph coverage and balance, while annotators find \model's summaries \textbf{maintain readability} and \textbf{better balance perspectives}.