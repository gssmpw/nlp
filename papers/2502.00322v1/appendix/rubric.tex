\begin{table*}[]
\small 
\centering
\begin{tabular}{@{}ll@{}}
\toprule
 & Rubric Text \\ \midrule
Criteria & Interest Level: How engaging and thought-provoking is the summary? \\
Score 1 & Not engaging at all; no attempt to capture the reader’s attention. \\
Score 2 & Fairly engaging with a basic narrative but lacking depth. \\
Score 3 & Moderately engaging with several interesting points. \\
Score 4 & Quite engaging with a well-structured narrative and noteworthy points that frequently capture and retain attention \\
Score 5 & Exceptionally engaging throughout, with a compelling narrative that consistently stimulates interest. \\ \midrule
Criteria & Coherence and Organization: Is the summary well-organized and logically structured? \\
Score 1 & Disorganized; lacks logical structure and coherence. \\
Score 2 & Fairly organized; a basic structure is present but not consistently followed. \\
Score 3 & Organized; a clear structure is mostly followed with some lapses in coherence. \\
Score 4 & Good organization; a clear structure with minor lapses in coherence. \\
Score 5 & Excellently organized; the summary is logically structured with seamless transitions and a clear argument. \\ \midrule
Criteria & Relevance and Focus: Does the summary stay on topic to the query and maintain a clear focus? \\
Score 1 & Off-topic; the content does not align with the query. \\
Score 2 & Somewhat on topic but with several digressions; the answer to the query is evident but not consistently adhered to. \\
Score 3 & Generally on topic, despite a few unrelated details. \\
Score 4 & Mostly on topic and focused; the narrative has a consistent relevance to the query with infrequent digressions. \\
Score 5 & \specialcellleft{Exceptionally focused and entirely on topic; the article is tightly centered on the query,\\with every piece of information contributing to a comprehensive understanding of the query.} \\ \midrule
Criteria & Broad Coverage: Does the article provide an in-depth exploration of the query and have good coverage? \\
Score 1 & Severely lacking; offers little to no coverage of the query's primary aspects, resulting in a very narrow perspective. \\
Score 2 & Partial coverage; includes some of the query's main aspects but misses others, resulting in an incomplete portrayal. \\
Score 3 & \specialcellleft{Acceptable breadth; covers most main aspects, though it may stray into minor unnecessary details\\ or overlook some relevant points.} \\
Score 4 & \specialcellleft{Good coverage; achieves broad coverage of the query,\\hitting on all major points with minimal extraneous information.} \\
Score 5 & \specialcellleft{Exemplary in breadth; delivers outstanding coverage,\\thoroughly detailing all crucial aspects of the query without including irrelevant information.} \\ \midrule
Criteria & \specialcellleft{Diversity of Perspectives: Does the summary adequately describe\\why the answer to the query could be yes and why it could be no?} \\
Score 1 & No diversity; the summary presents only one perspective without addressing the opposing viewpoint. \\
Score 2 & Limited diversity; the summary acknowledges both perspectives but lacks depth in the explanation of one side. \\
Score 3 & Moderate diversity; the summary covers both perspectives, but one side is more thoroughly explored than the other. \\
Score 4 & Good diversity; the summary fairly represents both perspectives with balanced and detailed explanations. \\
Score 5 & \specialcellleft{Excellent diversity; the summary provides a comprehensive and balanced exploration of both perspectives,\\offering in-depth explanations for why the answer could be yes and why it could be no.} \\ \bottomrule
\end{tabular}
\caption{\label{table:rubric} Rubrics for Interest, Coherence, Relevance, Coverage, and Diversity for DQFS summaries. Rubrics are adapted for topic paragraphs and topics (e.g. ``Relevance'' becomes relevance to the topic in topic paragraph evaluation, rather than relevance to the query).}
\end{table*}