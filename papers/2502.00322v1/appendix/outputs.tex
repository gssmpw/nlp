\hypersetup{
    colorlinks=true, % Enable colored links
    linkcolor=black, % Default color for internal links (sections, etc.)
    citecolor=black, % Default color for citations
    urlcolor=black % Default color for external URLs
}

\begin{summary}[title={\modelTopic Summary: Are Audiobooks Considered Real Reading? (ConflictingQA)}, label=summary1]
\textbf{Topic 1: Audiobooks and cognitive engagement}\\
Audiobooks provide cognitive benefits similar to traditional reading, such as improved vocabulary, comprehension, and critical thinking skills, and engage the same cognitive processes as reading print books [1][2][3]. While listening to audiobooks, children and adults can visualize, make predictions, and synthesize information, indicating active cognitive engagement, although the lack of visual stimulation may affect the engagement for some individuals [3][4][8]. High-quality production and narration in audiobooks can enhance the cognitive experience, making them a valid form of reading for people with visual impairments, dyslexia, or those who multitask [7][9][11].\\
\\
\textbf{Topic 2: Audiobooks accessibility and inclusivity}\\
Audiobooks enhance accessibility and inclusivity by supporting children with different learning needs, allowing individuals with ADHD, dyslexia, and other attention disorders to enjoy literature and improve comprehension [1][2][11]. They enable multitasking and comprehension for various audiences, including those with visual impairments or who find traditional reading challenging, and are cost-effective to produce, increasing their accessibility [7][12]. Despite some cultural biases that may undervalue audiobooks compared to printed text, they are considered a valid form of reading, especially for those who may not engage with traditional reading methods [9][8].\\
\\
\textbf{Topic 3: Audiobooks versus physical reading}\\
Audiobooks provide similar cognitive benefits to traditional reading, such as improved vocabulary and comprehension, and engage the same cognitive processes as reading with eyes, making them a valid form of reading [1][2][3]. They are particularly beneficial for those with different learning needs or physical limitations, allowing for a rich, accessible, and multitask-friendly reading experience [3][6][7][11]. Despite some cultural perceptions that audiobooks are less valuable than printed text, they are increasingly popular, economically significant, and considered a legitimate form of storytelling and information absorption [8][9][12].
\end{summary}

\begin{summary}[title={\modelTopic Summary: Is infinite scrolling a good web design technique? (ConflictingQA)}, label=summary2]
\textbf{Topic 1: User Engagement Increase}\\
Infinite scrolling can potentially increase user engagement and time spent on a site by continuously loading new content, which exploits users' automatic behavior and keeps them engaged [9]. However, it has been found to decrease user engagement in some cases, such as on Etsy, and can negatively impact users with disabilities and mental health, leading to a high cognitive load and potential mental health issues [1][5][8]. Additionally, infinite scrolling can lead to control issues and user frustration due to less controllable pages and jumping glitches [6].\\
\\
\textbf{Topic 2: Content Accessibility Concerns}\\
Infinite scrolling can lead to content accessibility issues, as it breaks the expected behavior of scrollbars and makes it difficult for users to gauge the length of the page, and it poses significant challenges for users with assistive technologies, often excluding footers and making navigation stressful [1][6][7]. While it can keep users engaged on eCommerce platforms, it has been associated with increased stress levels and negative mental health outcomes, particularly in young social media users [3][6][8]. Moreover, strategies like role='feed' have failed to address these accessibility problems effectively [5].\\
\\
\textbf{Topic 3: Mental Health Implications}\\
Infinite scrolling can exploit human psychological phenomena such as automaticity, leading to behaviors like doom-scrolling that may contribute to mental health issues by causing users to lose track of time and continue scrolling unconsciously [9]. The design can also induce stress by preventing users from reaching a perceived end, leading to information overload, and overwhelming them with choices, which can result in frustration, anxiety, and a reduced motivation to engage with content [6][7]. However, some studies suggest that engaging in mindful scrolling practices can mitigate these negative mental health outcomes, indicating that the impact of infinite scrolling may vary based on user behavior [8].
\end{summary}

\clearpage

\begin{summary}[title={\modelTopic Summary: Is EU expansion and EU membership itself a good idea? (DebateQFS)}, label=summary3]
\textbf{Topic 1: Economic gains from accession}\\
The 1997 study by the Centre for Economic Policy Research predicted economic gains for both the EU-15 and new Central and Eastern European members, with an estimated €10 billion and €23 billion increase respectively [1]. However, concerns about high budget and trade deficits in accession countries, such as Estonia and Hungary, and the potential for increased unemployment and social costs, suggest that EU expansion could also exacerbate economic disparities and put fiscal pressure on both new and existing members [5][6]. Additionally, the enlargement is expected to shift regional funds towards new members, potentially reducing support for poorer regions within the EU(15) and necessitating a significant increase in the EU's regional funding budget to address growing economic and social needs [6].\\
\\
\textbf{Topic 2: EU enlargement political challenges}\\
EU enlargement is seen as beneficial, with studies indicating potential GDP growth for new and existing members, strategic interests in stabilizing regions like the Western Balkans and Turkey, and necessary controls in place to manage economic migration and regional subsidies [1][2][6]. However, public opposition in some member states, the slow process of enlargement due to political complexities, and concerns over social contradictions and international conflicts [2][5][6] present significant challenges. The Treaty of Lisbon is deemed necessary for further enlargement, although there are differing opinions on whether its ratification should delay the process [4].\\
\\
\textbf{Topic 3: Regional disparities and funding}\\
EU expansion has been estimated to bring economic gains for both old and new member states, with the EU-15 seeing a €10 billion increase and new Central and Eastern European members gaining €23 billion [1]. However, regional disparities pose challenges, as unemployment rates have risen in accession countries and the wealth gap between regions may widen, with 98 million inhabitants in applicant states living in regions with GDP less than 75\% of the EU average [5][6]. Despite the potential for increased regional funding, there are concerns that existing poorer regions within the EU(15) may receive less support as a result of the expansion [6].
\end{summary}

\begin{summary}[title={\modelTopic Summary: Is going to law school a good idea? (DebateQFS)}, label=summary4]

\textbf{Topic 1: Law School ROI Analysis}\\
Attending law school can lead to a variety of career opportunities and the acquisition of valuable skills, with some graduates finding employment directly from campus and others benefiting from practical skills-oriented courses [4][5][6]. However, the financial burden of law school is significant, with many students accruing substantial debt, facing uncertain job markets, and questioning the return on investment, especially if they do not graduate from top-tier schools or are not at the top of their class [8][10][15][19]. Despite the potential for high starting salaries in some legal jobs, the competitiveness of the market and the cost of tuition may not justify the investment for all students, particularly when considering the psychological toll and the oversupply of law graduates [12][14][16].\\
\\
\textbf{Topic 2: Legal Career Job Market}\\
The legal job market presents a mixed outlook, with some documents indicating an increase in law firm hiring practices and a demand for legal services in certain areas, while others highlight the oversaturation of law graduates, underemployment, and the potential for job dissatisfaction and misleading employment statistics from law schools [4][17][8][10][11][18][19]. Graduates from prestigious law schools or those in the top of their class may have better job prospects and higher starting salaries, but many face significant debt and struggle to find well-paying jobs to manage that debt [16][19]. The rise of legal process outsourcing and the hiring of law school graduates directly by companies suggest evolving trends in the legal job market that could affect future employment opportunities for lawyers [6][5].\\
\\
\textbf{Topic 3: Law Education Value Debate}\\
Law school provides a range of non-monetary benefits, such as personal growth, maturity, and the development of transferable skills like critical thinking and argumentation, which are applicable in various fields beyond traditional legal practice [3][9]. However, the financial implications of law school, including high tuition costs, significant student debt, and an uncertain job market, challenge the notion that a legal education is a sound financial investment for all students [13][14][17][19]. Despite these concerns, there is a demand for legal professionals, and law school can prepare graduates for diverse career paths, including roles that address complex societal challenges and ensure access to justice [2][7][16].
\end{summary}