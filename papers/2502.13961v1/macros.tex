% --- Usual set names --- %
\renewcommand{\vec}{\mathbf}
\newcommand{\dR}{\mathbb{R}}
\newcommand{\dZ}[1][]{\ifblank{#1}{\mathbb{Z}}{\mathbb{Z}/#1\mathbb{Z}}}
\newcommand{\dN}{\mathbb{N}}
\newcommand{\dQ}{\mathbb{Q}}
\newcommand{\dC}{\mathbb{C}}
\newcommand{\dF}[1]{\mathbb{F}_#1}
\newcommand{\dE}{\mathbb{E}}
\newcommand{\dP}{\mathbb{P}}
\newcommand{\Id}{\mathbf{I}}
\newcommand{\R}{\mathbb{R}} % Reals
\newcommand{\real}{\mathbb{R}} % Reals
\newcommand{\N}{\mathbb{N}}
\newcommand{\cA}{\mathcal{A}}
\newcommand{\cB}{\mathcal{B}}
\newcommand{\cC}{\mathcal{C}}
\newcommand{\cD}{\mathcal{D}}
\newcommand{\cE}{\mathcal{E}}
\newcommand{\cF}{\mathcal{F}}
\newcommand{\cG}{\mathcal{G}}
\newcommand{\cH}{\mathcal{H}}
\newcommand{\cI}{\mathcal{I}}
\newcommand{\cK}{\mathcal{K}}
\newcommand{\cL}{\mathcal{L}}
\newcommand{\cM}{\mathcal{M}}
\newcommand{\cN}{\mathcal{N}}
\newcommand{\cO}{\mathcal{O}}
\newcommand{\cP}{\mathcal{P}}
\newcommand{\cR}{\mathcal{R}}
\newcommand{\cS}{\mathcal{S}}
\newcommand{\cT}{\mathcal{T}}
\newcommand{\cU}{\mathcal{U}}
\newcommand{\cV}{\mathcal{V}}
\newcommand{\cW}{\mathcal{W}}

\newcommand{\Ea}[1]{\E\left[#1\right]}
\newcommand{\Eb}[2]{\E_{#1}\left[#2\right]}
%\newcommand{\vDelta}{\mathbf{\Delta}}
\def\hR{\widehat{\mathcal{R}}}

\newcommand{\ind}{\mathbf{1}}

\newcommand{\dd}{\mathrm{d}}

% --- Usual set names --- %

\newcommand{\dS}{\mathbb{S}}

\newcommand{\E}{\mathbb{E}}


% --- Conditional expectations and probabilities + sets --- %



% --- Language-dependent macros --- %

\newcommand{\bilingualcommand}[3]{%
	\newcommand{#1}[1][\ ]{%
		##1%
		\iflanguage{english}{\text{#2}}{%
			\iflanguage{french}{\text{#3}}{}%
		}%
		##1%
	}%
}

\bilingualcommand{\where}{where}{où}
\bilingualcommand{\textif}{if}{si}
\bilingualcommand{\textand}{and}{et}
\bilingualcommand{\textiff}{if and only if}{si et seulement si}
\bilingualcommand{\otherwise}{otherwise}{sinon}

% --- Miscellaneous ---%

\newcommand{\eps}{\varepsilon}
\renewcommand{\theenumi}{(\roman{enumi})}
\renewcommand{\labelenumi}{\theenumi}
%%%%% Bold font
\newcommand{\bsA}{{\boldsymbol{\mathsf A}}}
\newcommand{\bG}{{\boldsymbol{G}}}
\newcommand{\bT}{{\boldsymbol{T}}}
\newcommand{\bOne}{{\boldsymbol{1}}}
\newcommand{\bR}{{\boldsymbol{R}}}
\newcommand{\bhR}{{\hat{\boldsymbol{R}}}}
\newcommand{\bg}{{\boldsymbol{g}}}
\newcommand{\bH}{{\boldsymbol{H}}}
\newcommand{\bA}{{\boldsymbol{A}}}
\newcommand{\bC}{{\boldsymbol{C}}}
\newcommand{\bS}{{\boldsymbol{S}}}
\newcommand{\bhS}{{\hat{\boldsymbol{S}}}}
\newcommand{\bU}{{\boldsymbol{U}}}
\newcommand{\bhU}{{\hat{\boldsymbol{U}}}}
\newcommand{\bh}{{\boldsymbol{h}}}
\newcommand{\bW}{{{\boldsymbol{{W}}}}}
\newcommand{\bw}{{\boldsymbol{w}}}
\newcommand{\bQ}{{\boldsymbol{Q}}}
\newcommand{\bV}{{\boldsymbol{V}}}
\newcommand{\bsQ}{{\boldsymbol{\mathsf Q}}}
\newcommand{\bsS}{{\boldsymbol{\mathsf S}}}
\newcommand{\bM}{{\boldsymbol{m}}}
\newcommand{\bhQ}{{\hat{\boldsymbol{Q}}}}
\newcommand{\bhV}{{\hat{\boldsymbol{V}}}}
\newcommand{\bhM}{{\hat{\boldsymbol{m}}}}
\newcommand{\bMM}{{{\boldsymbol{M}}}}
\newcommand{\bhMM}{{\hat{\boldsymbol{M}}}}
\newcommand{\bbb}{{\boldsymbol{b}}}
\newcommand{\bv}{{\boldsymbol{v}}}
\newcommand{\ba}{{\boldsymbol{a}}}
\newcommand{\bbB}{{\boldsymbol{B}}}
\newcommand{\bF}{{\boldsymbol{F}}}
\newcommand{\bZ}{{\boldsymbol{Z}}}
\newcommand{\bY}{{\boldsymbol{Y}}}
\newcommand{\btheta}{{\boldsymbol{\theta}}}
\newcommand{\bTheta}{{\boldsymbol{\Theta}}}
\newcommand{\bSigma}{{\boldsymbol{\Sigma}}}
\newcommand{\bPi}{{\boldsymbol{\Pi}}}
\newcommand{\bDelta}{{\boldsymbol{\Delta}}}
\newcommand{\brho}{{\boldsymbol{\rho}}}
\newcommand{\bphi}{{\boldsymbol{\phi}}}
\newcommand{\bsigma}{{\boldsymbol{\sigma}}}
\newcommand{\blambda}{{\boldsymbol{\lambda}}}
\newcommand{\bomega}{{\boldsymbol{\omega}}}
\newcommand{\bzeta}{{\boldsymbol{\zeta}}}
\newcommand{\bXi}{{\boldsymbol{\Xi}}}
\newcommand{\bmu}{{\boldsymbol{\mu}}}
\newcommand{\bvarphi}{{\boldsymbol{\varphi}}}
\newcommand{\bz}{{\boldsymbol{z}}}
\newcommand{\bx}{{\boldsymbol{x}}}
\newcommand{\bn}{{\boldsymbol{n}}}
\newcommand{\bX}{{\boldsymbol{X}}}
\newcommand{\bI}{{\boldsymbol{I}}}
%\newcommand{\bG}{{\boldsymbol{G}}}
\newcommand{\bB}{{\boldsymbol{B}}}
\newcommand{\by}{{\boldsymbol{y}}}
\newcommand{\bff}{{\boldsymbol{f}}}
\newcommand{\be}{{\boldsymbol{e}}}
\newcommand{\beeta}{{\boldsymbol{\eta}}}
\newcommand{\bxi}{{\boldsymbol{\xi}}}
\newcommand{\bpi}{{\boldsymbol{\pi}}}
\newcommand{\bu}{{\boldsymbol{u}}}
\def\sT{{\mathsf T}}
% --- Autorefs --- %
\def\sectionautorefname{Section}
\def\subsectionautorefname{Subsection}
\endinput

% Bruno
\def\mat#1{\text{#1}}
\def\sign{\text{sign}}

% --- Conditional expectations and probabilities + sets --- %

\providecommand{\given}{}
\DeclarePairedDelimiterXPP{\Pb}[1]{\mathbb{P}}{\lparen}{\rparen}{}{\renewcommand{\given}{\nonscript{}\:\delimsize\vert\nonscript{}\:\mathopen{}} #1}
\DeclarePairedDelimiterXPP{\E}[1]{\mathbb{E}}[]{}{\renewcommand{\given}{\nonscript{}\:\delimsize\vert\nonscript{}\:\mathopen{}} #1}
\DeclarePairedDelimiterX{\Set}[1]\lbrace\rbrace{\renewcommand{\given}{\nonscript{}\:\delimsize\vert\nonscript{}\:\mathopen{}} #1}

% --- Usual math functions --- %
\DeclareMathOperator{\card}{Card}
\DeclareMathOperator{\diag}{diag}
\DeclareMathOperator{\tr}{tr}
\DeclareMathOperator{\Var}{Var}
\DeclareMathOperator{\Cov}{Cov}
\DeclareMathOperator{\im}{im}
\DeclareMathOperator{\vect}{vect}
\DeclareMathOperator{\diam}{diam}
\DeclareMathOperator{\rank}{rank}
\DeclareMathOperator*{\argmax}{arg\,max}
\DeclareMathOperator*{\argmin}{arg\,min}
\DeclarePairedDelimiterX{\norm}[1]\lVert\rVert{\ifblank{#1}{\: \cdot \:}{#1}}
\DeclarePairedDelimiterX{\abs}[1]\lvert\rvert{\ifblank{#1}{\: \cdot \:}{#1}}
\DeclareMathOperator{\Poi}{Poi}
\DeclareMathOperator{\Bin}{Bin}
\DeclareMathOperator{\Ber}{Ber}
\DeclareMathOperator{\Unif}{Unif}
\DeclareMathOperator{\dtv}{d_{TV}}
\DeclareMathOperator{\polylog}{polylog}

% --- Language-dependent macros --- %

\newcommand{\bilingualcommand}[3]{%
	\newcommand{#1}[1][\ ]{%
		##1%
		\iflanguage{english}{\text{#2}}{%
			\iflanguage{french}{\text{#3}}{}%
		}%
		##1%
	}%
}

\bilingualcommand{\where}{where}{où}
\bilingualcommand{\textif}{if}{si}
\bilingualcommand{\textand}{and}{et}
\bilingualcommand{\textiff}{if and only if}{si et seulement si}
\bilingualcommand{\otherwise}{otherwise}{sinon}

% --- Miscellaneous ---%

\newcommand{\eps}{\varepsilon}
\renewcommand{\theenumi}{(\roman{enumi})}
\renewcommand{\labelenumi}{\theenumi}
\newcommand{\quand}{\quad \textand{} \quad}
\newcommand{\qquand}{\qquad \textand{} \qquad}

% --- Theorems --- %

% !TEX root = main.tex



\newcommand\tagthis{\addtocounter{equation}{1}\tag{\theequation}}


% Paired notation: usage explained below using \inp as an example:
% \inp just prints standard sized brackets and \inp* uses \left...\right to scale
% the brackets to enclose the material.
% Often \inp* will produce brackets that are too big, and manual scaling can be
% provided by \[\big], \[\Big], \[\bigg], \[\Bigg]


\providecommand{\refLE}[1]{\ensuremath{\stackrel{(\ref{#1})}{\leq}}}
\providecommand{\refEQ}[1]{\ensuremath{\stackrel{(\ref{#1})}{=}}}
\providecommand{\refGE}[1]{\ensuremath{\stackrel{(\ref{#1})}{\geq}}}
\providecommand{\refID}[1]{\ensuremath{\stackrel{(\ref{#1})}{\equiv}}}

\providecommand{\mgeq}{\succeq}
\providecommand{\mleq}{\preceq}

\providecommand{\divides}{\mid} % a divides b means there exists integer c such that b = ac
\providecommand{\tsum}{\textstyle\sum} % smaller sum symbols---displays as if inline
  % basic sets
  \providecommand{\R}{\mathbb{R}} % Reals
  \providecommand{\real}{\mathbb{R}} % Reals
  \providecommand{\N}{\mathbb{N}} % Naturals

  % random variables
  \makeatletter
  \def\sign{\@ifnextchar*{\@sgnargscaled}{\@ifnextchar[{\sgnargscaleas}{\@ifnextchar{\bgroup}{\@sgnarg}{\sgn} }}}
  \def\@sgnarg#1{\sgn\rbr{#1}}
  \def\@sgnargscaled#1{\sgn\rbr*{#1}}
  \def\@sgnargscaleas[#1]#2{\sgn\rbr[#1]{#2}}
  \makeatother


  % bold vectors
  \providecommand{\0}{\bm{0}}
  \providecommand{\1}{\bm{1}}
  \providecommand{\alphav}{\bm{\alpha}}
  \renewcommand{\aa}{\bm{a}}
  \providecommand{\bb}{\bm{b}}
  \providecommand{\cc}{\bm{c}}
  \providecommand{\dd}{\bm{d}}
  \providecommand{\ee}{\bm{e}}
  \providecommand{\ff}{\bm{f}}
  \let\ggg\gg
  \renewcommand{\gg}{\bm{g}}
  \providecommand{\hh}{\bm{h}}
  \providecommand{\ii}{\bm{i}}
  \providecommand{\jj}{\bm{j}}
  \providecommand{\kk}{\bm{k}}
  \let\lll\ll
  \renewcommand{\ll}{\bm{l}}
  \providecommand{\mm}{\bm{m}}
  \providecommand{\nn}{\bm{n}}
  \providecommand{\oo}{\bm{o}}
  \providecommand{\pp}{\bm{p}}
  \providecommand{\qq}{\bm{q}}
  \providecommand{\rr}{\bm{r}}
  \let\sss\ss
  \renewcommand{\ss}{\bm{s}}
  \providecommand{\tt}{\bm{t}}
  \providecommand{\uu}{\bm{u}}
  \providecommand{\vv}{\bm{v}}
  \providecommand{\ww}{\bm{w}}
  \providecommand{\xx}{\bm{x}}
  \providecommand{\yy}{\bm{y}}
  \providecommand{\ddot}{\dot\dot}
  \providecommand{\dddot}{\dot\dot}
  \providecommand{\zz}{\bm{z}}
  \providecommand{\thth}{\bm{\theta}}
  \newcommand{\bxi}{\boldsymbol{\xi}}

  % tilde vectors
  \providecommand{\txx}{\tilde\xx}
  \providecommand{\tgg}{\tilde\gg}


  % bold matrices
  \providecommand{\mA}{\bm{A}}
  \providecommand{\mB}{\bm{B}}
  \providecommand{\mC}{\bm{C}}
  \providecommand{\mD}{\bm{D}}
  \providecommand{\mE}{\bm{E}}
  \providecommand{\mF}{\bm{F}}
  \providecommand{\mG}{\bm{G}}
  \providecommand{\mH}{\bm{H}}
  \providecommand{\mI}{\bm{I}}
  \providecommand{\mJ}{\bm{J}}
  \providecommand{\mK}{\bm{K}}
  \providecommand{\mL}{\bm{L}}
  \providecommand{\mM}{\bm{M}}
  \providecommand{\mN}{\bm{N}}
  \providecommand{\mO}{\bm{O}}
  \providecommand{\mP}{\bm{P}}
  \providecommand{\mQ}{\bm{Q}}
  \providecommand{\mR}{\bm{R}}
  \providecommand{\mS}{\bm{S}}
  \providecommand{\mT}{\bm{T}}
  \providecommand{\mU}{\bm{U}}
  \providecommand{\mV}{\bm{V}}
  \providecommand{\mW}{\bm{W}}
  \providecommand{\mX}{\bm{X}}
  \providecommand{\mY}{\bm{Y}}
  \providecommand{\mZ}{\bm{Z}}
  \providecommand{\mLambda}{\bm{\Lambda}}

  % calligraphic
  \providecommand{\cA}{\mathcal{A}}
  \providecommand{\cB}{\mathcal{B}}
  \providecommand{\cC}{\mathcal{C}}
  \providecommand{\cD}{\mathcal{D}}
  \providecommand{\cE}{\mathcal{E}}
  \providecommand{\cF}{\mathcal{F}}
  \providecommand{\cG}{\mathcal{G}}
  \providecommand{\cH}{\mathcal{H}}
  \providecommand{\cII}{\mathcal{H}}
  \providecommand{\cJ}{\mathcal{J}}
  \providecommand{\cK}{\mathcal{K}}
  \providecommand{\cL}{\mathcal{L}}
  \providecommand{\cM}{\mathcal{M}}
  \providecommand{\cN}{\mathcal{N}}
  \providecommand{\cO}{\mathcal{O}}
  \providecommand{\cP}{\mathcal{P}}
  \providecommand{\cQ}{\mathcal{Q}}
  \providecommand{\cR}{\mathcal{R}}
  \providecommand{\cS}{\mathcal{S}}
  \providecommand{\cT}{\mathcal{T}}
  \providecommand{\cU}{\mathcal{U}}
  \providecommand{\cV}{\mathcal{V}}
  \providecommand{\cX}{\mathcal{X}}
  \providecommand{\cY}{\mathcal{Y}}
  \providecommand{\cW}{\mathcal{W}}
  \providecommand{\cZ}{\mathcal{Z}}

  \providecommand{\error}{\mathcal{E}}
  \providecommand{\ce}{\mathcal{C}}
  \providecommand{\tce}{\Xi}

%%%%%%%%%%%%%%%%%%%%%%%%%
%%%%%% THEOREMS
%%%%%%%%%%%%%%%%%%%%%%%%%

% Theorems, propositions, observations, corollaries, conjectures
% , and hypotheses all have the same counter.
% Lemmas, claims, remarks, examples and properties have same counter.
% Definitions. notations and Assumptions have same alphabetic counter.



\newcommand{\eg}{e.g.}
\newcommand{\ie}{i.e.}
\newcommand{\iid}{i.i.d.}
\newcommand{\cf}{cf.}


\newcommand{\ti}{\widetilde}
\newcommand{\ov}{\overline}
\newcommand{\set}[2][]{#1 \{ #2 #1 \} }

\newcommand{\tgamma}{\tilde \gamma}
\newcommand{\true}{\texttt{true}}
\newcommand{\false}{\texttt{false}}
% \usepackage[colorlinks=true,linkcolor=blue]{hyperref}
% \usepackage[capitalize,noabbrev]{cleveref}


% Check marks
\newcommand{\cmark}{\ding{51}}%
\newcommand{\xmark}{\ding{55}}%
\newcommand{\yes}{$\checkmark$}%
\newcommand{\no}{$\times$}%

\newcommand{\minus}{\scalebox{0.8}{$-$}}
\newcommand{\plus}{\scalebox{0.6}{$+$}}

% custom item in enumerate with reference
% \makeatletter
% \makeatother


% code to highlight parts of algorithm taken from https://tex.stackexchange.com/questions/386272/how-to-highlight-sections-of-my-code-in-algorithm
%define a marking command
%define a marking command
\newcommand*{\tikzmk}[1]{\tikz[remember picture,overlay,] \node (#1) {};}
%define a boxing command, argument = color of box
\newcommand{\boxit}[1]{\tikz[remember picture,overlay]{\node[yshift=3pt,fill=#1,opacity=.25,fit={($(A)+(-0.2*\linewidth - 3pt,3pt)$)($(B)+(0.75*\linewidth - 5pt,-2pt)$)}] {};}}

\newcommand{\blockfed}[1]{\tikz[remember picture,overlay]{\node[yshift=3pt,fill=#1,opacity=.25,fit={($(A)+(-0.1*\linewidth - 3pt,3pt)$)($(B)+(0.88*\linewidth - 5pt,-2pt)$)}] {};}}

\newcommand{\highlight}[1]{\tikz[remember picture,overlay]{\node[yshift=3pt,fill=#1,opacity=.25,fit={($(A)+(2pt,6pt)$)($(B)+(-3pt,-7pt)$)}] {};}}
%define some colors according to algorithm parts (or any other method you like)

\newcommand{\speedup}[1]{{\color{gray}(\ifdim #1 pt > 0.3pt #1\else $< #1$\fi{}$\times$)}}
% \newcommand{\speedup}[1]{{\color{lightgray} (#1 \times)}}
% \colorlet{client}{cyan!60}



\newcommand{\algopt}{\textsc{Choco-SGD}\xspace} 
\newcommand{\algcons}{\textsc{Choco-Gossip}\xspace} 
\newcommand{\eqncons}{\textsc{Choco-G}\xspace} %

\providecommand{\lin}[1]{\ensuremath{\left\langle #1 \right\rangle}}
\providecommand{\abs}[1]{\left\lvert#1\right\rvert}
\providecommand{\norm}[1]{\left\lVert#1\right\rVert}

\providecommand{\refLE}[1]{\ensuremath{\stackrel{(\ref{#1})}{\leq}}}
\providecommand{\refEQ}[1]{\ensuremath{\stackrel{(\ref{#1})}{=}}}
\providecommand{\refGE}[1]{\ensuremath{\stackrel{(\ref{#1})}{\geq}}}
\providecommand{\refID}[1]{\ensuremath{\stackrel{(\ref{#1})}{\equiv}}}

  \providecommand{\R}{\mathbb{R}} %
  \providecommand{\N}{\mathbb{N}} %
\providecommand{\vDelta}{\mathbb{Delta}}
 
  \providecommand{\Ec}[2]{{\mathbb E}_{#2}\left[#1\right] }%
  \providecommand{\EE}[2]{{\mathbb E}_{#1}\left.#2\right. }  %
  \providecommand{\EEb}[2]{{\mathbb E}_{#1}\left[#2\right] } %
  \providecommand{\prob}[1]{{\rm Pr}\left[#1\right] } 
  \providecommand{\Prob}[2]{{\rm Pr}_{#1}\left[#2\right] } 
  \providecommand{\P}[1]{{\rm Pr}\left.#1\right. }
  \providecommand{\Pb}[1]{{\rm Pr}\left[#1\right] }
  \providecommand{\PP}[2]{{\rm Pr}_{#1}\left[#2\right] }
  \providecommand{\PPb}[2]{{\rm Pr}_{#1}\left[#2\right] }

  \DeclareMathOperator*{\Circ}{\bigcirc}
\newcommand{\Ea}[1]{\E\left[#1\right]}
\newcommand{\Eb}[2]{\E_{#1}\left[#2\right]}
\newcommand{\Vara}[1]{\Var\left[#1\right]}
\newcommand{\Varb}[2]{\Var_{#1}\left[#2\right]}
\newcommand{\lone}[1]{\norm{#1}_1}
\newcommand{\og}{\overline{g}}
\newcommand{\oh}{\overline{H}}
\newcommand{\bhM}{{\hat{\boldsymbol{m}}}}
\newcommand{\bMM}{{{\boldsymbol{M}}}}
\newcommand{\bhMM}{{\hat{\boldsymbol{M}}}}
  \providecommand{\0}{\mathbf{0}}
  \providecommand{\1}{\mathbf{1}}
  \renewcommand{\aa}{\mathbf{a}}
  \providecommand{\bb}{\mathbf{b}}
  \providecommand{\cc}{\mathbf{c}}
  \providecommand{\dd}{\mathbf{d}}
  \providecommand{\ee}{\mathbf{e}}
  \providecommand{\ff}{\mathbf{f}}
  \let\ggg\gg
  \renewcommand{\gg}{\mathbf{g}}
  \providecommand{\hh}{\mathbf{h}}
  \providecommand{\ii}{\mathbf{i}}
  \providecommand{\jj}{\mathbf{j}}
  \providecommand{\erf}{\operatorname{erf}}
  \let\lll\ll
  \renewcommand{\ll}{\mathbf{l}}
  \providecommand{\mm}{\mathbf{m}}
  \providecommand{\nn}{\mathbf{n}}
  \providecommand{\oo}{\mathbf{o}}
  \providecommand{\pp}{\mathbf{p}}
  \providecommand{\qq}{\mathbf{q}}
  \providecommand{\rr}{\mathbf{r}}
  \providecommand{\ss}{\mathbb{s}}
  \providecommand{\tt}{\mathbf{t}}
  \providecommand{\uu}{\mathbf{u}}
  \providecommand{\vv}{\mathbf{v}}
  \providecommand{\ww}{\mathbf{w}}
  \providecommand{\xx}{\mathbf{x}}
  \providecommand{\yy}{\mathbf{y}}
  \providecommand{\zz}{\mathbf{z}}
  
  \providecommand{\mA}{\mathbf{A}}
  \providecommand{\mB}{\mathbf{B}}
  \providecommand{\mC}{\mathbf{C}}
  \providecommand{\mD}{\mathbf{D}}
  \providecommand{\mE}{\mathbf{E}}
  \providecommand{\mF}{\mathbf{F}}
  \providecommand{\mG}{\mathbf{G}}
  \providecommand{\mH}{\mathbf{H}}
  \providecommand{\mI}{\mathbf{I}}
  \providecommand{\mJ}{\mathbf{J}}
  \providecommand{\mK}{\mathbf{K}}
  \providecommand{\mL}{\mathbf{L}}
  \providecommand{\mM}{\mathbf{M}}
  \providecommand{\mN}{\mathbf{N}}
  \providecommand{\mO}{\mathbf{O}}
  \providecommand{\mP}{\mathbf{P}}
  \providecommand{\mQ}{\mathbf{Q}}
  \providecommand{\mR}{\mathbf{R}}
  \providecommand{\mS}{\mathbf{S}}
  \providecommand{\mT}{\mathbf{T}}
  \providecommand{\mU}{\mathbf{U}}
  \providecommand{\mV}{\mathbf{V}}
  \providecommand{\mW}{\mathbf{W}}
  \providecommand{\mX}{\mathbf{X}}
  \providecommand{\mY}{\mathbf{Y}}
  \providecommand{\mZ}{\mathbf{Z}}
  \providecommand{\mLambda}{\mathbf{\Lambda}}
  \providecommand{\mlambda}{\bm{\lambda}}
  \providecommand{\mmu}{\bm{\mu}}
  \providecommand{\cA}{\mathcal{A}}
  \providecommand{\cB}{\mathcal{B}}
  \providecommand{\cC}{\mathcal{C}}
  \providecommand{\cD}{\mathcal{D}}
  \providecommand{\cE}{\mathcal{E}}
  \providecommand{\cF}{\mathcal{F}}
  \providecommand{\cG}{\mathcal{G}}
  \providecommand{\cH}{\mathcal{H}}
  \providecommand{\cJ}{\mathcal{J}}
  \providecommand{\cK}{\mathcal{K}}
  \providecommand{\cL}{\mathcal{L}}
  \providecommand{\cM}{\mathcal{M}}
  \providecommand{\cN}{\mathcal{N}}
  \providecommand{\cO}{\mathcal{O}}
  \providecommand{\cP}{\mathcal{P}}
  \providecommand{\cQ}{\mathcal{Q}}
  \providecommand{\cR}{\mathcal{R}}
  \providecommand{\cS}{\mathcal{S}}
  \providecommand{\cT}{\mathcal{T}}
  \providecommand{\cU}{\mathcal{U}}
  \providecommand{\cV}{\mathcal{V}}
  \providecommand{\cX}{\mathcal{X}}
  \providecommand{\cY}{\mathcal{Y}}
  \providecommand{\cW}{\mathcal{W}}
  \providecommand{\cZ}{\mathcal{Z}}


  \usepackage[textwidth=5cm]{todonotes}
  
\providecommand{\mycomment}[3]{\todo[caption={},size=footnotesize,color=#1!20]{\textbf{#2: }#3}}%
\providecommand{\inlinecomment}[3]{%
  {\color{#1}#2: #3}}%
\newcommand\commenter[2]%
{%
  \expandafter\newcommand\csname i#1\endcsname[1]{\inlinecomment{#2}{#1}{##1}}
  \expandafter\newcommand\csname #1\endcsname[1]{\mycomment{#2}{#1}{##1}}
}
  


