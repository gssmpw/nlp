\section{Background \& Related Work}
\label{sec:background}

\subsection{Background}

\noindent
\textbf{File System Configurations.}   
The configuration methods of file systems are different from that of many applications, {which makes the problem arguably more challenging}. As shown in Figure~\ref{fig:usage}, a typical file system may be configured through a set of utilities at four different stages:
    \vspace{-0.05in}
\begin{itemize}
    \item \textbf{Create}. When creating file systems,  the \texttt{mkfs} utility (e.g., \texttt{mke2fs} for Ext4) generates the initial configurations.
    \item \textbf{Mount}. When mounting file systems, certain configurations can be specified via \texttt{mount} (e.g., `\texttt{-o dax}' to enable the Direct Access or DAX feature~\cite{Ext4_DAX}). 
     \item \textbf{Online}. Many  utilities can change the  configurations of a mounted file system directly by modifying the metadata online (e.g., Ext4 defragmenter \texttt{e4defrag}~\cite{e4defrag},  Windows NTFS checker   \texttt{ChkDsk}~\cite{chkdsk}).       
     \item \textbf{Offline}. Offline utilities can also modify file system images and change  the configurations   (e.g.,   \texttt{resize2fs}~\cite{resize2fs},   \texttt{e2fsck}~\cite{e2fsck})
     
\end{itemize}


 
\begin{figure}[tb]
	\centering
    \includegraphics[width=3.3in]{./configmethods}
   \vspace{-0.1in}
	\caption{  {\bf Methods of Configuring File Systems}. {This figure shows 
	four typical stages to configure a file system: (a) at creation (e.g., \texttt{mke2fs}) or mount time (\texttt{mount}) before usage; (b) via  online utilities (e.g., \texttt{e4defrag}); (c) via offline utilities. }} 
	\label{fig:usage}
\end{figure}

 Note that all the  utilities  have different configuration parameters to control their own behaviors, which will eventually affect the file system state. 
{Moreover, the configuration parameters may affect the behavior of the file system long after the FS image is created, and some configurations cannot be changed later. Also,}
  the validation of parameters may occur at both user level and kernel level.
 For example, the `\texttt{-O inline\_data}' parameter of \texttt{mke2fs} and the `\texttt{-o dax}' of \texttt{mount} are further validated in the \texttt{ext4\_fill\_super}  function of Ext4.
Therefore, we believe it is necessary to consider the file system itself as well as all the associated utilities as an \textit{FS ecosystem} to address the configuration challenge.
For simplicity, we call the  file system  and utilities as \textit{components} within the FS ecosystem.

The multi-stage configuration method  is common among   file systems. As listed in Table~\ref{tab:manyFSandUtilities}, many popular file systems follow  similar modular designs and can be configured via different utilities at different stages.
Therefore, we believe that the multi-component configuration challenge is general.

\begin{table}[t]
	\small
	\begin{center}
		\begin{tabular}{ c | c | c | c | c }
	\textbf{FS (OS)}  & \multicolumn{4}{c}{\textbf{Four  Stages of Configuration}}  \\
			\cline{2-5}
			\textbf{ } & \textbf{Create} & \textbf{Mount} &  \textbf{Online} & \textbf{Offline} \\ 
			\hline
			{Ext4 (Linux)} & \cite{mke2fs} & \cite{mount} & \cite{e4defrag}, \cite{resize2fs} & \cite{e2fsck}, \cite{resize2fs}  \\
			\hline
			{XFS (Linux)} & \cite{mkfsxfs} & \cite{mount} & \cite{xfsfsr}, \cite{xfsgrowfs} & \cite{xfsadmin}, \cite{xfsrepair}  \\
			\hline
			{BtrFS (Linux)} & \cite{mkfsbtrfs} & \cite{mount} & \cite{btrfsbalance}, \cite{btrfsscrub}  & \cite{btrfscheck} \\
			\hline
			{UFS (FreeBSD)} & \cite{newfs} & \cite{mountfreebsd} & \cite{growfs}, \cite{restore} & \cite{dump}, \cite{fsckufs}  \\
			\hline
			{ZFS (FreeBSD)} & \cite{zfs-create} & \cite{zfs-mount} & \cite{zfs-rollback}, \cite{zfsset} & \cite{zfs-destroy} \\
			\hline
			{NTFS (Windows)} & \cite{format} & \cite{mountvol} & \cite{chkdsk}, \cite{defrag}  & \cite{chkdsk}, \cite{shrink} \\
			\hline
			{APFS (MacOS)} & \cite{diskutil} & 
			\cite{mountapfs} &
			\cite{diskutil}  & \cite{diskutil}, \cite{fsckapfs} \\
			\hline
		\end{tabular}
	\end{center}
	\caption{ {\bf Examples of configuration methods for different file systems}. The last four columns  list example  utilities that can affect the file system configuration states.
	}
	\label{tab:manyFSandUtilities}
\end{table}

\smallskip
\noindent
\textbf{FS Test Suites.}  
Practical test suites have been created to ensure the correctness of file systems under various configurations. Unfortunately, 
due to the complexity  of configurations, their coverage in terms of configuration is limited. As shown in Table~\ref{tab:testsuites},
fewer than half of configuration parameters are used in  the standard test suites of Linux file systems (i.e., \texttt{xfstests}~\cite{xfstest}, \texttt{e2fsprogs/tests}~\cite{e2fsprogs-test}) based on our study.
Since each parameter may have a wide range of values representing different states, the total number of 
missed configuration states is much more than the number of unused  parameters, which implies the need for better tool support.


\begin{table}[t]
	\small
	\begin{center}
		\begin{tabular}{ c | c | c | c  }
			\textbf{Test} & \textbf{Target} &  \multicolumn{2}{c}{\textbf{\# of Conf. Param.}}  \\
			\cline{3-4}
			\textbf{Suite} & \textbf{Software} & \textbf{Total} & \textbf{Used}  \\
			\hline
			\texttt{xfstests/ext4} & 	{Ext4} & $>$85 & 29 ($<$ 34.1\%)   \\
			\hline
			\texttt{e2fsprogs} & \texttt{e2fsck} & $>$35 & 6 ($<$ 17.1\%)  \\
		     	\texttt{/tests} & \texttt{resize2fs}  & $>$15 & 7 ($<$ 46.7\%)  \\
			\hline
		\end{tabular}
	\end{center}
 
	\caption{ {\bf Configuration Coverage of Test Suites.} 
	}
 	 \vspace{-0.1in}
	\label{tab:testsuites}
\end{table}


\smallskip
\noindent
\textbf{Configuration Constraints \& Dependencies.}
 Configuration \emph{constraints}  specify the configuration requirements (e.g., data type,  value range) of software~\cite{spex}. Intuitively, such information can help identify important configuration states, and it has proved to be effective for addressing  configuration-related  issues in a wide range of applications~\cite{cdep,Conferr-DSN08,spex,AnEmpiricalStudy, ConfigEverywhere-DongpuJin-ICSE14}. 
Configuration \emph{dependency} is one special type of constraint describing the dependent correlation among parameters~\cite{cdep,spex}, which has shown recently to be critical for addressing complex configuration issues in cloud systems~\cite{cdep}.
For simplicity, we use constraints and dependencies interchangeably in the rest of the paper.
Note that although the basic concepts have been proposed, 
there is limited understanding of them in the context of file systems. 
This paper attempts to fill the gap.


\putsec{related}{Related Work}

\noindent \textbf{Efficient Radiance Field Rendering.}
%
The introduction of Neural Radiance Fields (NeRF)~\cite{mil:sri20} has
generated significant interest in efficient 3D scene representation and
rendering for radiance fields.
%
Over the past years, there has been a large amount of research aimed at
accelerating NeRFs through algorithmic or software
optimizations~\cite{mul:eva22,fri:yu22,che:fun23,sun:sun22}, and the
development of hardware
accelerators~\cite{lee:cho23,li:li23,son:wen23,mub:kan23,fen:liu24}.
%
The state-of-the-art method, 3D Gaussian splatting~\cite{ker:kop23}, has
further fueled interest in accelerating radiance field
rendering~\cite{rad:ste24,lee:lee24,nie:stu24,lee:rho24,ham:mel24} as it
employs rasterization primitives that can be rendered much faster than NeRFs.
%
However, previous research focused on software graphics rendering on
programmable cores or building dedicated hardware accelerators. In contrast,
\name{} investigates the potential of efficient radiance field rendering while
utilizing fixed-function units in graphics hardware.
%
To our knowledge, this is the first work that assesses the performance
implications of rendering Gaussian-based radiance fields on the hardware
graphics pipeline with software and hardware optimizations.

%%%%%%%%%%%%%%%%%%%%%%%%%%%%%%%%%%%%%%%%%%%%%%%%%%%%%%%%%%%%%%%%%%%%%%%%%%
\myparagraph{Enhancing Graphics Rendering Hardware.}
%
The performance advantage of executing graphics rendering on either
programmable shader cores or fixed-function units varies depending on the
rendering methods and hardware designs.
%
Previous studies have explored the performance implication of graphics hardware
design by developing simulation infrastructures for graphics
workloads~\cite{bar:gon06,gub:aam19,tin:sax23,arn:par13}.
%
Additionally, several studies have aimed to improve the performance of
special-purpose hardware such as ray tracing units in graphics
hardware~\cite{cho:now23,liu:cha21} and proposed hardware accelerators for
graphics applications~\cite{lu:hua17,ram:gri09}.
%
In contrast to these works, which primarily evaluate traditional graphics
workloads, our work focuses on improving the performance of volume rendering
workloads, such as Gaussian splatting, which require blending a huge number of
fragments per pixel.

%%%%%%%%%%%%%%%%%%%%%%%%%%%%%%%%%%%%%%%%%%%%%%%%%%%%%%%%%%%%%%%%%%%%%%%%%%
%
In the context of multi-sample anti-aliasing, prior work proposed reducing the
amount of redundant shading by merging fragments from adjacent triangles in a
mesh at the quad granularity~\cite{fat:bou10}.
%
While both our work and quad-fragment merging (QFM)~\cite{fat:bou10} aim to
reduce operations by merging quads, our proposed technique differs from QFM in
many aspects.
%
Our method aims to blend \emph{overlapping primitives} along the depth
direction and applies to quads from any primitive. In contrast, QFM merges quad
fragments from small (e.g., pixel-sized) triangles that \emph{share} an edge
(i.e., \emph{connected}, \emph{non-overlapping} triangles).
%
As such, QFM is not applicable to the scenes consisting of a number of
unconnected transparent triangles, such as those in 3D Gaussian splatting.
%
In addition, our method computes the \emph{exact} color for each pixel by
offloading blending operations from ROPs to shader units, whereas QFM
\emph{approximates} pixel colors by using the color from one triangle when
multiple triangles are merged into a single quad.

