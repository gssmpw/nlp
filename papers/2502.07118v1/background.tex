\section{Background \& Related Work}
\label{sec:background}

\subsection{Background}

\noindent
\textbf{File System Configurations.}   
The configuration methods of file systems are different from that of many applications, {which makes the problem arguably more challenging}. As shown in Figure~\ref{fig:usage}, a typical file system may be configured through a set of utilities at four different stages:
    \vspace{-0.05in}
\begin{itemize}
    \item \textbf{Create}. When creating file systems,  the \texttt{mkfs} utility (e.g., \texttt{mke2fs} for Ext4) generates the initial configurations.
    \item \textbf{Mount}. When mounting file systems, certain configurations can be specified via \texttt{mount} (e.g., `\texttt{-o dax}' to enable the Direct Access or DAX feature~\cite{Ext4_DAX}). 
     \item \textbf{Online}. Many  utilities can change the  configurations of a mounted file system directly by modifying the metadata online (e.g., Ext4 defragmenter \texttt{e4defrag}~\cite{e4defrag},  Windows NTFS checker   \texttt{ChkDsk}~\cite{chkdsk}).       
     \item \textbf{Offline}. Offline utilities can also modify file system images and change  the configurations   (e.g.,   \texttt{resize2fs}~\cite{resize2fs},   \texttt{e2fsck}~\cite{e2fsck})
     
\end{itemize}


 
\begin{figure}[tb]
	\centering
    \includegraphics[width=3.3in]{./configmethods}
   \vspace{-0.1in}
	\caption{  {\bf Methods of Configuring File Systems}. {This figure shows 
	four typical stages to configure a file system: (a) at creation (e.g., \texttt{mke2fs}) or mount time (\texttt{mount}) before usage; (b) via  online utilities (e.g., \texttt{e4defrag}); (c) via offline utilities. }} 
	\label{fig:usage}
\end{figure}

 Note that all the  utilities  have different configuration parameters to control their own behaviors, which will eventually affect the file system state. 
{Moreover, the configuration parameters may affect the behavior of the file system long after the FS image is created, and some configurations cannot be changed later. Also,}
  the validation of parameters may occur at both user level and kernel level.
 For example, the `\texttt{-O inline\_data}' parameter of \texttt{mke2fs} and the `\texttt{-o dax}' of \texttt{mount} are further validated in the \texttt{ext4\_fill\_super}  function of Ext4.
Therefore, we believe it is necessary to consider the file system itself as well as all the associated utilities as an \textit{FS ecosystem} to address the configuration challenge.
For simplicity, we call the  file system  and utilities as \textit{components} within the FS ecosystem.

The multi-stage configuration method  is common among   file systems. As listed in Table~\ref{tab:manyFSandUtilities}, many popular file systems follow  similar modular designs and can be configured via different utilities at different stages.
Therefore, we believe that the multi-component configuration challenge is general.

\begin{table}[t]
	\small
	\begin{center}
		\begin{tabular}{ c | c | c | c | c }
	\textbf{FS (OS)}  & \multicolumn{4}{c}{\textbf{Four  Stages of Configuration}}  \\
			\cline{2-5}
			\textbf{ } & \textbf{Create} & \textbf{Mount} &  \textbf{Online} & \textbf{Offline} \\ 
			\hline
			{Ext4 (Linux)} & \cite{mke2fs} & \cite{mount} & \cite{e4defrag}, \cite{resize2fs} & \cite{e2fsck}, \cite{resize2fs}  \\
			\hline
			{XFS (Linux)} & \cite{mkfsxfs} & \cite{mount} & \cite{xfsfsr}, \cite{xfsgrowfs} & \cite{xfsadmin}, \cite{xfsrepair}  \\
			\hline
			{BtrFS (Linux)} & \cite{mkfsbtrfs} & \cite{mount} & \cite{btrfsbalance}, \cite{btrfsscrub}  & \cite{btrfscheck} \\
			\hline
			{UFS (FreeBSD)} & \cite{newfs} & \cite{mountfreebsd} & \cite{growfs}, \cite{restore} & \cite{dump}, \cite{fsckufs}  \\
			\hline
			{ZFS (FreeBSD)} & \cite{zfs-create} & \cite{zfs-mount} & \cite{zfs-rollback}, \cite{zfsset} & \cite{zfs-destroy} \\
			\hline
			{NTFS (Windows)} & \cite{format} & \cite{mountvol} & \cite{chkdsk}, \cite{defrag}  & \cite{chkdsk}, \cite{shrink} \\
			\hline
			{APFS (MacOS)} & \cite{diskutil} & 
			\cite{mountapfs} &
			\cite{diskutil}  & \cite{diskutil}, \cite{fsckapfs} \\
			\hline
		\end{tabular}
	\end{center}
	\caption{ {\bf Examples of configuration methods for different file systems}. The last four columns  list example  utilities that can affect the file system configuration states.
	}
	\label{tab:manyFSandUtilities}
\end{table}

\smallskip
\noindent
\textbf{FS Test Suites.}  
Practical test suites have been created to ensure the correctness of file systems under various configurations. Unfortunately, 
due to the complexity  of configurations, their coverage in terms of configuration is limited. As shown in Table~\ref{tab:testsuites},
fewer than half of configuration parameters are used in  the standard test suites of Linux file systems (i.e., \texttt{xfstests}~\cite{xfstest}, \texttt{e2fsprogs/tests}~\cite{e2fsprogs-test}) based on our study.
Since each parameter may have a wide range of values representing different states, the total number of 
missed configuration states is much more than the number of unused  parameters, which implies the need for better tool support.


\begin{table}[t]
	\small
	\begin{center}
		\begin{tabular}{ c | c | c | c  }
			\textbf{Test} & \textbf{Target} &  \multicolumn{2}{c}{\textbf{\# of Conf. Param.}}  \\
			\cline{3-4}
			\textbf{Suite} & \textbf{Software} & \textbf{Total} & \textbf{Used}  \\
			\hline
			\texttt{xfstests/ext4} & 	{Ext4} & $>$85 & 29 ($<$ 34.1\%)   \\
			\hline
			\texttt{e2fsprogs} & \texttt{e2fsck} & $>$35 & 6 ($<$ 17.1\%)  \\
		     	\texttt{/tests} & \texttt{resize2fs}  & $>$15 & 7 ($<$ 46.7\%)  \\
			\hline
		\end{tabular}
	\end{center}
 
	\caption{ {\bf Configuration Coverage of Test Suites.} 
	}
 	 \vspace{-0.1in}
	\label{tab:testsuites}
\end{table}


\smallskip
\noindent
\textbf{Configuration Constraints \& Dependencies.}
 Configuration \emph{constraints}  specify the configuration requirements (e.g., data type,  value range) of software~\cite{spex}. Intuitively, such information can help identify important configuration states, and it has proved to be effective for addressing  configuration-related  issues in a wide range of applications~\cite{cdep,Conferr-DSN08,spex,AnEmpiricalStudy, ConfigEverywhere-DongpuJin-ICSE14}. 
Configuration \emph{dependency} is one special type of constraint describing the dependent correlation among parameters~\cite{cdep,spex}, which has shown recently to be critical for addressing complex configuration issues in cloud systems~\cite{cdep}.
For simplicity, we use constraints and dependencies interchangeably in the rest of the paper.
Note that although the basic concepts have been proposed, 
there is limited understanding of them in the context of file systems. 
This paper attempts to fill the gap.



\section{Related Work} \label{sec:related}

% \textbf{Adversarial Attack}
\textbf{Attacks on SLAM.} 
%With the rise of machine learning, 
The robustness of computer vision systems is being actively investigated. With the emergence of adversarial images in the digital domain by adding optimized noise directly to images~\cite{szegedy2013intriguing,carlini2017towards}, researchers find that such attacks also exist physically in the real world \cite{eykholt2018robust,song2018physical,zhao2019seeing}. To fill the gap between attacks in the digital and physical worlds, recent studies have demonstrated that attacks on real-world computer vision systems are practical \cite{eykholt2018robust,li2019adversarial,man2020ghostimage,sharif2016accessorize,zhao2019seeing,zhou2018invisible}. However, attacks on traditional computer vision methods such as SLAM are relatively less explored. \cite{yoshida2022adversarial} proposes an attack against the scan matching algorithm in LiDAR-based SLAM, while most SLAMs in AR/VR devices rely on different sensors like RGB/depth cameras and IMUs. \cite{ikram2022perceptual} and \cite{chen2024adversary} mislead visual SLAM by poisoning the images with special patterns, and \cite{wang2021can} causes the camera to fail using infrared light. In our work, we demonstrate attacks on Visual-Inertial SLAM (VI-SLAM) by perturbing the IMU readings, rather than cameras, and showing its impact on XR user experience. 

\textbf{Acoustic Injection Attacks.} Among various physical attacks, acoustic injection attacks are attractive due to their low cost. Son~\etal~\cite{son2015rocking} were the first to introduce acoustic attacks on MEMS gyroscopes, demonstrating how these attacks could lead to sensor denial-of-service and result in drone crashes. WALNUT~\cite{trippel2017walnut} expanded on this by developing output biasing and control attacks that enable precise manipulation of MEMS accelerometer outputs using modulated sound waves. Wang et al.~\cite{wang2017sonic} demonstrated a sonic gun, showcasing the vulnerability of various smart devices (\eg drones and self-balancing vehicles) to acoustic attacks. Tu et al. \cite{tu2018injected} designed side-swing and switching attacks to alter the outputs of MEMS gyroscopes and accelerometers. Furthermore, Ji et al. \cite{ji2021poltergeist} fool the object detectors by applying acoustic attack to the image stabilizers commonly used in modern cameras. However, none of the existing works study the relationship between the acoustic injections and SLAM outputs on recent XR devices. 

% \zijian{Do we need one session about security in AR/VR?}
% \yicheng{TODO}
%\jiasi{cite the AIVR paper (UMass Amherst?) paper is we have not already. They add IMU perturbation but w/o SLAM, iirc} \yicheng{Cited}

\textbf{XR Security and Privacy.} 
%Security and privacy concerns in XR systems have gained significant attention. 
For single-user XR systems, researchers have demonstrated various side-channel attacks to extract sensitive information (\eg keystrokes) through video feeds~\cite{ling2019know}, head movements~\cite{nair2023unique, slocum2023going}, architectural hints~\cite{zhang2023its,shang2020arspy}, power usage~\cite{li2024dangers}, and EM side-channel leakages~\cite{al2021vr}. In multi-user XR systems, Su et al.~\cite{su2024remote} use avatar motion data to infer keystrokes in shared VR environments. Slocum et al.~\cite{slocum2024doesn} reveal vulnerabilities in the shared state frameworks of multi-user AR. Similarly, Lebeck et al.~\cite{lebeck2017securing} highlight risks like deceptive virtual objects and emphasize access control for managing shared physical and virtual spaces. Ruth et al.~\cite{ruth2019secure} further propose a secure multi-user AR framework focusing on content sharing and permissions.
Chandio et al.~\cite{chandio2024stealthy} %introduced a multi-modal spatiotemporal attack that 
simultaneously manipulated visual and inertial sensors to disrupt XR pose estimation. However, their study evaluated the attack using offline datasets and assumed the attacker's capability to manipulate IMU data streams through acoustic means, without real experiments. Ours is the first to demonstrate acoustic injection attacks on recent XR devices, like the Hololens 2, in the real world.
 

