\section{Discussions}
\label{sec:discussion}

 No study or tool is perfect, and our work is no exception. We discuss lessons learned,  limitations of \prj prototype, and a few promising extensions in this section.

 
\subsection{Lessons Learned}
\noindent
\textbf{Heterogeneity in storage configuration  management.}
Through this comprehensive study, we found that
although many file systems follow similar modular design principles (\S\ref{sec:background}), 
their configuration management may be implemented in various ways. Classic file systems, such as Ext4 and XFS, tend to have a relatively simpler management mechanism. For example, there is usually one main configuration function where most of the checking of configuration parameters are implemented. On the other hand, newer file and storage systems (e.g., ZFS and WireTiger) may have relatively more sophisticated  configuration management to cover a wider configuration space for more advanced features (e.g., snapshot, cloning).
For example, different from Ext4/XFS which  use \texttt{mkfs.ext4}/\texttt{mkfs.xfs} only to create the file system itself, 
ZFS supports creating various types of dataset (e.g.,  file system, volume, snapshot,  pool~\cite{zfs}) via  \texttt{zfs-create}.
All the ZFS datasets have their own sets of parameters; meanwhile, they may also be affected by a same set of shared configuration parameters.  Moreover, ZFS supports user defined configurations and inherited parameters. Instead of keeping user input values in variables (as in Ext4/XFS),  each ZFS parameter and its value is stored as a pair in a separate data structure. To manage such unique features, ZFS configuration subsystem uses various different functions to handle parameter conversion and checking, which involves both general checking for all parameters and  specific checks for a subset of special parameters.

In addition,  heterogeneity has also been observed in terms of configuration documentation. 
Such heterogeneity inevitably contributes to the complexity of configuration-oriented
testing and makes a one-size-fits-all solution almost impossible.
We hope that  our study as well as the plugin-based \prj design can help alleviate the challenge in configuration management and inspire follow up research on handling the growing heterogeneity and complexity.

\vspace{0.05in}
\noindent
\textbf{Need for better configuration design.}
An alternate perspective of the configuration challenge studied and exposed in this work is that we may have too many parameters today. One might argue that  it is perhaps better to reduce the parameters to avoid  vulnerabilities or confusions, instead of adding new configurations for more features. 
Also, one might suggest that (in theory) we can implement every utility functionality in  the file system itself 
to avoid  tricky cross-component configuration dependencies.
Essentially, these are trade-offs of the file system and configuration design that deserve more investigation from the community. We hope that by studying real-world configuration issues and releasing the \textsc{ConfD} prototype, our work can help {identify problematic configuration parameters and further help with the reduction of such parameters to} improve the configuration design  in general.\\

\vspace{-0.15in}
\subsection{Limitations}
\noindent
{\bf Limitations of the multilevel taxonomy}. As briefly mentioned in \S\ref{sec:methodology}, the multilevel configuration dependencies should be interpreted with the study methodology in mind, 
because they are derived from an incomplete set of configuration-related issues from two FS ecosystems. It is likely that there are more complex dependencies in  FS ecosystems, which deserves further investigation. 
 
\vspace{0.05in}
\noindent
{\bf Limitations of the \textsc{ConfD} framework}.
The current prototype 
requires a few user inputs (\S\ref{sec:annotation}) to guide the automated dependency analysis, which we hope to reduce  through more sophisticated state analysis.
The current taint analysis is dependent on configuration variables identified in the source code. We found that the configuration management in newer file and storage systems may not always maintain dedicated configuration variables for individual parameters, which may limit the scope of taint analysis.
Also,  \textsc{ConfD}  can only handle a subset of   LLVM  IR   for taint analysis and it only considers two parameters at a time for CPD and CCD, which may lead to incomplete dependency or false positives. 
We hope to improve these  through more advanced software engineering efforts in the future, which will likely improve the   effectiveness further. 

Similarly, there are limitations in  plugins. For example, 
\texttt{ConfD-handlingCk} only induces at most two violations for one configuration state for simplicity; there may be more issues if we consider more than two.
Also,
\texttt{ConfD-xfstests}   only transforms  a subset of the test suite due to the irregular configuration handling. 
{The failed test cases in \texttt{ConfD-xfstests} requires a lot of time and effort to further analyze the test case and get to the root cause of failure. This is unavoidable since this part requires deep file system internal knowledge to understand the issue. Although we have minimized the failure cases in our extension, it is still time consuming. In the future we would like to further minimize this by using AI to filter out some of the failed test cases to enhance efficiency.}

Despite the limitations, \prj has been effective in analyzing dependencies and exposing configuration-related issues in our experiments, so we believe that it  
will be valuable to the community.


\subsection{Future Work}
\noindent
{\bf Integration with other file systems and tools.}
As mentioned in Table~\ref{tab:manyFSandUtilities}, many file systems 
can be configured through different utilities, which could potentially benefit from the multilevel dependency analysis of \textsc{ConfD} after minor customization (e.g., providing  FS-specific inputs in JSON format~\ref{sec:annotation}).
Also, \textsc{ConfD} is complementary to  other modern tools besides the base tools used in   current plugins. 
For example, FB-HYDRA ~\cite{Yadan2019Hydra}    uses YAML files to store configurations which is compatible with the JSON files used by \textsc{ConfD}. Moreover, it supports a set of plugins called ``Sweepers'' to manipulate the selection of parameters. The dependency-based  state generation in \textsc{ConfD} could be implemented as one special ``Sweeper'' for FB-HYDRA~\cite{Yadan2019Hydra}. 
Similarly, the configurations generated by \textsc{ConfD} could potentially be integrated into CI/CD frameworks~\cite{Jenkins} to enable pipelined configuration-oriented testing and deployment. 

\vspace{0.05in}
\noindent
{\bf Support for other software}.
Configuration dependency is not limited to the  file systems (Ext4, XFS, ZFS) and  the storage engine (WiredTiger) studied in this paper.
For example, NDCTL~\cite{ndctl} is a utility to configure the \texttt{libnvdimm} subsystem in Linux.
We expect that adding NDCTL to the dependency analysis  will likely help address NVM-specific configuration issues more effectively.
Also, researchers   have observed functionality or correctness dependencies between local file systems and other software  
(e.g.,  databases~\cite{Zheng-OSDI14-DB}, distributed storage systems~\cite{Jinrui-ICS18,RedundancyDoesNotImplyFaultTolerance-FAST17,Runzhou-TOS22,Runzhou-PDSW20}), 
many of which are also related to  configurations. 
The  dependencies studied in this work   may serve as a foundation for investigating such configuration-related 
issues beyond file systems.
Also, since LLVM supports compiling a wide set of languages (e.g.,  C++, Rust, Swift) to IR through various frontends~\cite{LLVM},
the core analysis of \textsc{ConfD} is expected to be applicable to software written in other languages as well.  
