\subsubsection{Dependency-agnostic vs. Dependency-guided}
\label{sec:dependency-aware-vs-agnostic}

In this section, we compare the effectiveness of two open-source research prototypes (i.e.,  \texttt{rfsck}~\cite{OmFAST18} and \texttt{gt-hydra}~\cite{hydra-2019}) with and without  \prj support. We focus on the two research prototypes and the corresponding plugins for   comparison   because they provide quantitative metrics to measure the effectiveness straightforwardly. We defer the results of other plugins to the next section. 

\begin{table}[t]
	\small
	\begin{center}
		\begin{tabular}{ c| c | c | c    }
 \textbf{ID} &	\textbf{Symptom of  }  &  \multicolumn{2}{c}{\textbf{Triggered?}}    \\
          \cline{3-4}
          & 	\textbf{Uncorrectable Corruption   }  & {\texttt{\space \space rfsck \space \space}}  & {{\texttt{ConfD-rfsck}}  }    \\
\hline
1  & Unable to mount the FS & N  &  Y  (6) \\
			\hline
2  & Invalid file data  & N  &  Y  (24) \\
			\hline
3  & Truncated file data  & Y (11) &  Y (250)   \\
			\hline
\multicolumn{2}{r|}{\textbf{Total}} & 11  &  280    \\
			\hline
		\end{tabular}
	\end{center}
	\caption{ {\bf Comparison of Corruption Symptoms Triggered.} 
\texttt{ConfD-rfsck} triggered (`Y') more types of corruptions. The counts are in parentheses.
}
	\label{tab:result-rfsck-symptom}
\end{table}



In the first experiment, we applied fault injectors \texttt{rfsck}  and \texttt{ConfD-rfsck} to analyze Ext4 and its checker utility \texttt{e2fsck}. The fault injectors  interrupt the checker operation and examine if the interrupted checker   could lead to uncorrectable corruptions on the file system (i.e., cannot be fixed by another run of checker). They report the number of repaired FS images containing uncorrectable corruptions (i.e., ``uncorrectable image''). Each uncorrectable image implies a vulnerability in the FS ecosystem that could lead to data loss~\cite{OmFAST18}.  

The result of the experiment is summarized in Table~\ref{tab:result-rfsck}. \texttt{rfsck} reports 11 uncorrectable images with the default configuration.  \texttt{ConfD-rfsck} can explore different configuration states and we analyze the reports generated under 25 configuration states for comparison. In 4   out of the 25 states, \texttt{ConfD-rfsck} generates less than 11 uncorrectable images; in   4 states,  \texttt{ConfD-rfsck} generates the same amount of  uncorrectable images (i.e., `$=$ 11'); 
 in the majority states (17), \texttt{ConfD-rfsck}  generates more uncorrectable images (i.e., `$>$ 11'), which suggests  it is more effective in exposing potential vulnerabilities in the FS ecosystem. 
Table~\ref{tab:result-rfsck-symptom} further compares the symptoms of uncorrectable corruptions triggered by \texttt{rfsck}  and \texttt{ConfD-rfsck}. Overall, \texttt{ConfD-rfsck} triggers three different types of symptoms,  while \texttt{rfsck} only triggers one symptom in our experiment.  
 Since different symptoms typically imply different vulnerabilities in metadata protection and/or recovery in the FS ecosystem, the result also suggests that   the dependency-guided configuration states used by  \texttt{ConfD-rfsck} can help improve the effectiveness of \texttt{rfsck}.



\begin{table}[t]
	\small
	\begin{center}
		\begin{tabular}{ c | c | c    }
 	\textbf{Target FS }  &  \multicolumn{2}{c}{\textbf{\# of Issues Reported (in two weeks)}}    \\
          \cline{2-3}
           	\textbf{   }  & {\texttt{\space \space \space gt-hydra \space \space \space}}  & {\texttt{ConfD-gt-hydra}}    \\
\hline
Ext4  & 1  &  17      \\
			\hline
		\end{tabular}
	\end{center}
	\caption{ {\bf Comparison of Two FS Fuzzers.} 
\texttt{ConfD-gt-hydra} reports more hangs given the same fuzzing time.
}
	\label{tab:result-gt-hydra}
\end{table}

In the  second experiment, we applied \texttt{gt-hydra}  and \texttt{ConfD-gt-hydra} to fuzz the Ext4 file system.  
The fuzzers systematically generate various inputs (i.e., FS metadata corruptions and system calls) to explore different code paths in the file system for triggering latent bugs~\cite{hydra-2019}.
We run each  fuzzer continuously for two weeks.  The fuzzers report the number of reliability issues  detected on the target file system within the running period. The issues may include different types depending on the bug checkers used. We use the default \textsc{SymC3} checker which can detect crash inconsistency bugs. Meanwhile, since  the fuzzers are based on the AFL fuzzer ~\cite{afl}, they also report crash and hang issues (detected by AFL) by default. 
Note that the only difference \texttt{ConfD-gt-hydra} introduces is the dependency-guided configurations,
i.e., it does not change the test logic or criteria for reporting issues. Therefore, both the types of issues (e.g., `crash', `hang', `crash inconsistency') and the number of issues reported can be 
used as the metric to evaluate effectiveness.

The result of the fuzzing experiment is summarized in Table~\ref{tab:result-gt-hydra}.
To make the comparison fair, we limit the two fuzzers to the same total execution time (i.e., two weeks each).
We set the \texttt{ConfD-gt-hydra} to switch to a new dependency-guided configuration state every 12 hours, which leads to 28 critical configuration states being explored within two weeks. While each configuration in \texttt{ConfD-gt-hydra} is  explored with only 1/28 of the  time used by \texttt{gt-hydra} for its configuration, the overall result of \texttt{ConfD-gt-hydra} is   better:  \texttt{gt-hydra} only detects 1 issue on Ext4 by the end of the two week period, while \texttt{ConfD-gt-hydra} detects 17 issues in total. 
Interestingly, all issues   reported in the experiment are `hang'. 
This is expected because triggering  more complicated semantic bugs may require multiple weeks. 
 
In summary, the two sets of comparison experiments  above show that  \prj can  amplify the effectiveness of existing FS tools for identifying vulnerabilities quickly, which is particularly valuable for time-consuming methodologies like fault injection or fuzzing. 
Note that in all experiments, we do not randomly generate combinations of configurations. This is because a naive algorithm without any knowledge of inherent dependencies can easily lead to time-wasting configurations, as   will be  demonstrated further  in \S\ref{sec:state-generation-vs-fb-hydra}.