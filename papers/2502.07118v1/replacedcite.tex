\section{Related Work}
\label{sec:related}
\vspace{-0.05in}

\smallskip
\noindent
{\bf Analysis of Software Configurations.} 
Configuration issues have been  studied in many software applications____. 
For example, ConfErr____ manipulates parameters to emulate human errors;  
Ctests____ detects failure-inducing configuration changes.
In general, these works do not analyze deep   dependencies within the  software.
The closest work is {cDEP}____,
which notably observes \textit{inter-component dependencies} in  Hadoop____ and OpenStack____. Unfortunately, their solution is largely inapplicable for file systems. 
This is because their target components share   configuration specifications (e.g., XML)  and libraries____, which makes them equivalent to one single  program in terms of configuration. 
In contrast, 
the configuration dependencies in file systems may cross different programs and the user-kernel boundary, which requires  non-trivial mechanisms to extract.
In addition, cDEP relies on a Java   framework____ 
which cannot handle C-based file systems. 

\smallskip
\noindent
{\bf Reliability of File Systems.} 
Great efforts have been made to improve the reliability of 
file systems____ 
and  their  utilities____.
For example, 
Prabhakaran et al.____ apply fault injection to analyze the failure policies of  file
systems and propose improved designs based on the IRON taxonomy; 
Xu et al.____ and Kim et al.____ use fuzzing to detect file system bugs;  
SQCK ____ and RFSCK____ improve the checker utilities of file systems to avoid inaccurate fixes. 
 While effective for their original  goals, these works do not consider multi-component configuration issues. 
 On the other hand, the configuration dependencies from this work may be integrated with these existing efforts to improve their coverage (see \S\ref{sec:plugins}).  
 Therefore, we view them as complementary. 
 
\smallskip
\noindent
{\bf Configuration Management Tools.} Faced by the increasing  challenge, practitioners have created dedicated frameworks for configuration management 
____. 
For example,  
Facebook HYDRA____ supports managing hierarchical configurations elegantly. 
While helpful for developing new applications, refactoring FS ecosystems to leverage such frameworks would require substantial efforts (if possible at all).
Notably, the framework supports running a program with different compositions of configurations automatically. Nevertheless,  since it does not understand configuration dependencies, it may generate many invalid configuration states (see \S\ref{sec:state-generation-vs-fb-hydra}). This work aims to address such limitations.