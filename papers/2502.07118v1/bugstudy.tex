
\section{Configuration Dependencies in File Systems}
\label{sec:bugstudy}
  \vspace{-0.05in}

 
In this section, we present a study on the Ext4 and XFS ecosystems to understand the potential patterns of configuration issues and guide the design of solutions. 
We discuss the methodology and  findings in \S\ref{sec:methodology}  and \S\ref{sec:findings}, respectively. 

\begin{table*}[t]
	\small
	\begin{center}
		\begin{tabular}{ l | l | c | c | c | c | c }
	\multicolumn{2}{l|}{\textbf{FS Usage Scenarios}} & \textbf{Description} & \textbf{ \# of}  & \multicolumn{3}{c}{\textbf{Multilevel Config. Dependencies}}\\
         \cline{5-7}
		
	\multicolumn{2}{l|}{	 {(key configuration utilities are in bold)}} &   &  \textbf{ Bug} &  \textbf{SD}  &   \textbf{CPD}  &  \textbf{CCD}    \\
					\hline
1 & \textbf{mke2fs} - \textbf{mount} - Ext4 & create \&	mount an Ext4 to use & 13  & 13 (100\%) & 1 (7.7\%) & 13 (100\%) \\
			\hline
2 & \textbf{mke2fs} - \textbf{mount} - Ext4 - \textbf{e4defrag} & online defragmentation & 1 &  1 (100\%)  & -- & -- \\
			\hline
3 & 	\textbf{mke2fs} - mount - Ext4 - umount - \textbf{{resize2fs}}  &  resize an umounted Ext4  & 17 &  17 (100\%)  & -- & 17 (100\%)\\
			\hline
4 &  \textbf{mke2fs} - mount - Ext4 - umount - \textbf{{e2fsck}}  &  check  Ext4 \& fix inconsistencies   & 36 &  36 (100\%) & 4 (11.1\%) & 34 (94.4\%)\\
\hline

5 &  \textbf{mkfs.xfs} - \textbf{mount} - XFS & create \&	mount an XFS to use & 5&5 (100\%)& 2 (40\%)& 5 (100\%)\\
 \hline
6 &  \textbf{mkfs.xfs} - mount - XFS - umount - \textbf{xfs\_repair} & check XFS   \& fix  inconsistencies &6 &6 (100\%) &1 (16.7\%) &6 (100\%)\\
 \hline

\multicolumn{3}{r|}{\textbf{Total}}  & 78 &  78  (100\%) & 8 (10.3\%) & 75 (96.2\%)  \\
\hline
		\end{tabular}
	\end{center}
	\caption{ {\bf Distribution of Configuration Bugs in Six Scenarios.} This table shows  the distribution of 78 configuration bugs in six typical usage  scenarios of file system. The last three columns shows the percentages of bug cases that involve Self-Dependency (SD), Cross-Parameter Dependency (CPD), and Cross-Component Dependency (CCD), respectively. 
}
	\label{tab:bugstudy}
\end{table*}




\begin{table*}[t]
	\small
	\begin{center}
		\begin{tabular}{ c | c | c | c | c  }
		\multicolumn{2}{c|}{	\textbf{Multilevel Config. Dependencies}} & \textbf{Description} & \textbf{Observed?}  &  \textbf{Count}  \\
          \hline
Self Dependency	  &  Data Type  & parameter $P$ must be of a specific data type (e.g., integer)   &   Y  &  44\\
					\cline{2-5}
   (SD)        & Value Range & $P$  must be within a specific  {value range} (e.g., $P$  $<$ 4096)  & Y & 41\\
			\hline
Cross-Parameter      & Control  &  $P1$  of $C1$ can   be enabled iff   $P2$  of $C1$  is enabled/disabled & Y & 5 \\
					\cline{2-5}
  Dependency      &  Value    &  $P1$'s  value depends on   $P2$ 's value (e.g., $P1 < P2$)   &  N & -- \\
             \cline{2-5}
     (CPD) &    Behavioral &     component $C1$'s behavior depends on $P1$ and $P2$ of component $C1$ & Y & 1 \\ 

			\hline
Cross-Component      & Control    & $P1$  of $C1$ can   be enabled iff   $P2$  of $C2$  is enabled/disabled &  Y&  1 \\
					\cline{2-5}
 Dependency    & Value   &$P1$'s  value depends on  $P2$ from another component   &  Y & 1\\
 \cline{2-5}
 (CCD)  & Behavioral   &  component $C1$'s behavior  depends on  $P2$ of  $C2$ &  Y&  75\\
 \hline
 \multicolumn{3}{r|}{\textbf{Total}} & 7/8  & 168  \\

\hline
		\end{tabular}
	\end{center}
	\caption{ {\bf Multilevel Configuration Dependencies.} This table describes the multilevel configuration dependencies observed. $Pn$ means parameter, $Cn$ means component. The last column shows the count of each sub-category of dependency observed. 
}
	\label{tab:dependencies}
\end{table*}

\subsection{Methodology}
\label{sec:methodology}

Our dateset includes two parts: (1) the source code of Ext4 and XFS and
seven important utilities including \texttt{mke2fs}, \texttt{mount}, \texttt{e4defrag}, \texttt{resize2fs}, \texttt{e2fsck}, \texttt{mkfs.xfs}, and \texttt{xfs\_repair}, which are described in Table~\ref{tab:bugstudy};
(2) a set of 78 configuration-related bug patches for the two FS ecosystems, 
which are collected from the commit histories of their source code repositories via a combination of keyword search (e.g., ‘configuration’, ‘parameter’, ‘option’), random sampling, and manual validation. 
Note that the patch collection method is inspired by previous studies of real-world bugs~\cite{Lu-FAST13-FS,lu2008learning,Duo-SYSTOR21}. While time-consuming, it has proved to be valuable for driving system improvements~\cite{Lu-FAST13-FS,lu2008learning}.
On the other hand, similar to previous studies, the findings of our study should be interpreted with the method in
mind. For example, the 78 patches only represent a subset of issues that have been triggered and fixed; there are likely  other configuration-related issues  not yet discovered (see \S\ref{sec:discussion} for further discussion). 

\vspace{-0.05in}
\subsection{Findings}
\label{sec:findings}
Based on the   dataset,  we analyzed  each patch 
and the relevant source code in depth to understand the logic,
which enables us to identify the configuration usage scenarios  as well as  configuration constraints that are critical. We summarize our findings in Table~\ref{tab:bugstudy} and Table~\ref{tab:dependencies} and discuss them below. 

\vspace{0.1in}
\noindent
\textbf{Finding \#1:} \textit{The majority of cases (96.2\%) involve critical  parameters from more than one component}.  
The first column of Table~\ref{tab:bugstudy} shows six typical usage scenarios of file systems which cover all bug cases in our dataset (78 in total). 96.2\% of the bug cases require specific parameters from at least two key utilities (i.e., the utilities in bold in each usage scenario) to manifest.
 
This reflects the complexity of the  issues  and suggests that we cannot only consider one single component.

\noindent
\textbf{Finding \#2:} \textit{There is a hierarchy of configuration dependencies}.
We classify the configuration constraints derived from our dataset into three major categories as follows: 


\begin{itemize}

  \vspace{-0.05in}
    \item \textbf{Self Dependency (SD)} means individual parameters  must satisfy their own  constraints (e.g., data type or value range). For example,  the \texttt{blocksize} parameter of \texttt{mke2fs} has a value range of 1024 - 65536 and must be a power of 2. 
    
    \item \textbf{Cross-Parameter Dependency (CPD)} means multiple parameters of the same component must satisfy relative relation constraints (e.g., two \texttt{mke2fs} parameters \texttt{meta\_bg} and \texttt{resize\_inode} cannot be used together). 
       
    \item \textbf{Cross-Component Dependency (CCD)} means the parameters or behaviors of  one component  depend on the  parameters of another component. 
    Both dependencies   in Figure~\ref{fig:resizebug} belong to this category becasue they involve parameters of \texttt{mke2fs} and the (buggy) behavior of \texttt{resize2fs} depend on them.
    

\end{itemize}

 
As summarized in Table~\ref{tab:dependencies}, 
each major category may contain a couple of sub-categories which describe more specific constraints. 
 Together, these constraints form a hierarchy which we call \textit{multilevel configuration dependencies}.
 Note that we only observed 7 out of 8 sub-categories  in the dataset. We include the  unseen ``Value'' sub-category in CPD based on the literature~\cite{spex} for completeness.

Moreover, among all the dependencies, there is a subset which directly contribute to the manifestation of the bugs in our dataset: the relevant parameters are explicitly mentioned in the bug patches, and modifications to the corresponding functionalities are needed to fix the bugs (i.e., they are related to the root causes). We call this subset of dependencies as \textit{critical dependencies}.
The count of the critical dependencies for each sub-category is shown in the last column of  Table~\ref{tab:dependencies}. 
We are able to derive 168 critical dependencies  manually  in total, which is larger than the number of bug cases. This is because multiple critical dependencies may be needed to trigger a bug. For example, both dependencies in Figure~\ref{fig:resizebug} are critical dependencies for this bug case.

As shown in the last three columns of Table~\ref{tab:bugstudy}, SD  and CCD  are almost always involved in all scenarios (100\% and 96.2\% respectively), while CPD is non-negligible (10.3\%).
This is because SD represents relatively simple constraints which always need to be satisfied first  {to make the target component work} (e.g.,  correct spelling). 
SD is relatively easy to check and has been the focus of previous work~\cite{Conferr-DSN08}.
However, this does not mean that 100\% of the bugs could be avoided if SD is checked or satisfied. 
For example, a bug related to both the \texttt{bigalloc} and \texttt{extent} parameters (i.e., there is a CPD involved) may still occur even if the two parameters are spelled correctly.
In other words, only considering  simple constraints of individual parameters is not enough. 

Interestingly, we observed both CPD and CCD between the DAX feature and other seemingly irrelevant configurations. 
 In one case, a corruption was triggered when
`\texttt{-O inline\_data}' was used in \texttt{mke2fs} and the image was mounted with `\texttt{-o dax}' subsequently. 
In another case, the DAX  feature conflicted with the  `\texttt{has\_journal}' configuration, which may lead to corruptions when changing the journaling mode online.   Such unexpected dependencies implies the complexity of adding the DAX support to the Linux kernel.

\smallskip
\noindent
{\textbf{Finding \#3:} \textit{Configuration parameters are handled in heterogeneous ways in an FS ecosystem}.} 
We identified four major sources of heterogeneity in  FS configurations. First, different parameters may be mapped to different types of variables in the code. For example,   the parameters of Ext4 may be stored  in (at least) four different ways including (i) a local variable, (ii) a global variable, (iii) a bit in a  bitmap   accessed via bit operations, and (iv) directly in the superblock.    
Second, within the superblock, parameters may be kept either in one single field (e.g., \texttt{ s\_log\_block\_size}) or as one member of a compound field. Third, parameters can be   loaded from the superblock either directly or through library calls. Lastly, different components may use different functions for handling configurations  (e.g., \texttt{resize2fs}  uses the ``main'' function, while \texttt{mke2fs} invokes a special function called ``PRS'').
Such heterogeneity makes previous solutions mostly inapplicable.

\smallskip
\noindent
 \textbf{Finding \#4:} \textit{The majority of cases (71.8\%)  do not require configuration-specific workloads to manifest}. 
Interestingly, despite the complexity, many bugs can be triggered without applying configuration-specific workloads. This  suggests that we may re-use existing efforts on stressing file systems~\cite{e2fsprogs-test,xfstest} to analyze configuration-related issues effectively.
