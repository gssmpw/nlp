\section{Introduction}
\label{sec:intro}

File systems (FS), such as Ext4 \cite{ext4} and XFS~\cite{XFS} on Linux-based  operating systems (OS) and NTFS~\cite{ntfs} on Windows OS, play an essential role in modern society.
They directly manage various files on desktops, laptops, and smartphones for numerous end users~\cite{Lu-FAST13-FS}. Moreover,
they often serve as the local storage backend for distributed storage systems (e.g., Lustre~\cite{lustre}, GFS~\cite{GoogleFS}, HopsFS~\cite{Hopfs}, MySQL NDB Cluster~\cite{mysqlndb}) to enable  storage management at scale.

To meet diverse needs, many file systems are designed with a wide range of configuration parameters controllable via  utilities~\cite{e2fsprogs,xfsprogs,btrfsbalance,fsckufs,zfsset,fsckminix,chkdsk,diskutil}, 
which enables   users to tune the systems with different tradeoffs.
For example,   Ext4 contains more than 85 configuration parameters which can be modified through a set of utilities called \texttt{e2fsprogs}~\cite{e2fsprogs}. The combination of the configuration parameters represents over $10^{37}$   configuration states~\cite{Carver}.


While configuration parameters have  improved the system flexibility, they introduce additional complexity for reliability. 
Subtle correctness issues often rely on specific parameters to trigger~\cite{confu, spex};  
consequently, they may elude  intensive testing  and affect  end users negatively.
For example, in December 2020, users of Windows OS observed that the checker utility of  NTFS (i.e., \texttt{ChkDsk}~\cite{chkdsk}) may  destroy NTFS on SSDs~\cite{hothardware20201220,borns20201218}. 
The incident turned out to be configuration-related: 
 two specific parameters must be satisfied to manifest the issue, including the `\texttt{/f}' parameter of \texttt{ChkDsk} and another (unnamed)  parameter in Windows OS~\cite{borns20201221}.
 
 
 \begin{figure}[tb]
	\centering
 \includegraphics[width=2.5in]{./bugexample}
	\caption{  {\bf A Configuration-Related Issue of Ext4}.  { When \texttt{sparse\_super2} feature is enabled and the \texttt{size} parameter of \texttt{resize2fs} is larger than the Ext4 size, expanding the file system results in metadata corruption.  
	}}
	\label{fig:resizebug}
\end{figure}

Similarly, Figure~\ref{fig:resizebug} shows another configuration-related issue involving Ext4 and its \texttt{mke2fs} and \texttt{resize2fs} utilities~\cite{e2fsprogs}.
Two conditions must hold to trigger the bug: 
(1) the \texttt{sparse\_super2} feature is enabled in Ext4 (via \texttt{mke2fs});
(2) the value of the \texttt{size} parameter of ~\texttt{resize2fs} must be larger than the size of Ext4 (i.e., expanding the file system). Once triggered, the bug will corrupt the Ext4 metadata with incorrect free blocks. {The root cause behind the issue was logical: with the specific configuration, the free block count of the last block group of Ext4 was calculated before adding new blocks for expansion.}

Due to the combinatorial explosion of configuration states and the substantial time needed to scrutinize a file system under each 
configuration state,  
it is practically impossible to exhaust all
states 
for thorough testing  today~\cite{
EMClarke-ModelCheckingAndTheStateExplosionProblem}.
Moreover, with more and more heterogeneous devices and advanced features 
 being introduced~\cite{Ext4_DAX,samsungsmartssd,trimsupport}, the  configuration states 
 are expected to grow. 
 Therefore,  effective methods to help improve configuration-related testing and identify critical configuration issues efficiently are much needed. 



\subsection{Limitations of the State of the Art}
There are practical test suites  to ensure the correctness of file systems under different configurations (e.g., \texttt{xfstests}~\cite{xfstest}). 
Unfortunately, their coverage in terms of configuration is limited:
fewer than half of configuration parameters are used based on our study, which reflects the need for better tool support.
Also, configuration-related issues have   emerged in other software systems and have received much attention~\cite{cdep,ctest-OSDI20,Conferr-DSN08,spex,NavigatingMazeBioinfo-MikaelaCashman-ASE18}. But unfortunately, existing efforts mainly focus on relatively simple configuration issues  (e.g., typos~\cite{Conferr-DSN08}) within  one single application, which is  limited for addressing the file system configuration challenge involving multiple programs.

Please refer to \S\ref{sec:background} for more details.  


\subsection{Our Efforts \& Contributions}
This paper presents one of the first steps to address the increasing configuration challenge of file systems. Inspired by a recent study~\cite{cdep} on configuration issues in Hadoop~\cite{Hadoop} and OpenStack~\cite{OpenStack}, we focus on \textit{configuration dependency}, which describes the dependent relations among configuration parameters~\cite{cdep}. Such dependency has been identified as a key source of  complexity caused problems, and capturing the dependency is essential for improving  configuration design and tooling~\cite{cdep,TianyinXu-AnHCIViewofConfigurationProblemsArXiv16,spex}.

While the basic concept of configuration dependency  
has been proposed in the literature (see \S\ref{sec:background}),
the understanding of specific dependency patterns and  implications  in the context of  file  systems is still limited. 
Therefore, we first conducted an empirical study on 78 configuration-related issues in two major file systems (i.e., Ext4 and XFS).
By scrutinizing   real-world   bugs and the relevant source code, 
we answer one important question: What critical configuration dependencies exist in file systems? 

Our study reveals a prevalent pattern called \textit{multilevel configuration dependencies}.
Besides the relatively simple configuration constraints (e.g., value range~\cite{spex}),
there are implicit dependencies among parameters from different utilities of a file system.
The majority  (96.2\%) of  issues in our dataset requires meeting 
such deep configuration dependencies to manifest.
Interestingly, the  workloads applied to the file system do not have to be configuration-specific: 
71.8\% issues only involve generic file system operations. 

Based on the study, we built an extensible framework called \textsc{ConfD} to extract the multilevel configuration dependencies automatically and leverage dependency-guided   configuration states for further analysis. 
One key challenge is how to establish the correlation between parameters specified through different utilities which have different ways of  configuration handling.
We address the challenge by metadata-assisted taint analysis, which leverages the fact that all utilities of a given file system share the same  metadata structures.  
Moreover, based on the   dependencies extracted,  we created a set of plugins to help address  configuration-related issues in file systems from different angles.

 Our experiments show that
\textsc{ConfD}  can extract 154 different configuration dependencies 
with a low false positive rate (8.4\%) for  Ext4 and XFS. 
Moreover, with the dependency guidance, 
the \textsc{ConfD} plugins can  
identify various configuration-related issues, including inaccurate documentations, configuration handling issues, and regression test failures induced by valid configurations.

{This article is an extension of our previous work \cite{mahmud2023confd}. In this version, we explored further the applicability of \textsc{ConfD} on two new storage systems including a modern copy-on-write file system (i.e., ZFS~\cite{zfs}), and a popular NoSQL storage engine (i.e.,  WiredTiger~\cite{wt}) and extracted the configuration dependencies in the new context. Moreover, based on the dependencies extracted from ZFS, we introduced a new plugin to help enhance the regression test suite of ZFS.
In addition, we  optimized the plugin for regression testing of Ext4 and XFS  which reduces the false positive rate  by 68\% on average. 
Through this comprehensive analysis of configuration dependencies of storage systems, 
we hope that our effort can shed light on addressing configuration-related challenges for practical systems in general. 

In summary, this paper makes the following contributions:
\begin{itemize}
    \item Deriving a taxonomy of critical  configuration dependencies of file systems based on  real-world  issues from two major Linux file systems (i.e., Ext4, XFS).
    \item Building the \textsc{ConfD} prototype \footnote{\textsc{ConfD} is on \url{https://github.com/data-storage-lab/ConfD}}
    to  extract configuration dependencies from three file systems (i.e., Ext4, XFS, ZFS) and one database engine (i.e., WiredTiger), and expose relevant issues in storage systems. 
    \item Integrating  with multiple practical tools (e.g.,fault injector~\cite{OmFAST18}, fuzzer~\cite{hydra-2019},  regression test suites~\cite{xfstest,e2fsprogs-test}) to improve their  configuration coverage and effectiveness.
\end{itemize}
}

The rest of the paper is organized as follows: \S\ref{sec:background} introduces the background and related work; \S\ref{sec:bugstudy} presents the empirical study and findings; \S\ref{sec:design-extract} describes the  \textsc{ConfD} framework;
\S\ref{sec:results} shows experimental results; \S\ref{sec:discussion} discusses our observations, limitations and potential extensions;   \S\ref{sec:conclusion} concludes the paper.