\section{Appendix}\label{appendix:foo}

\subsection{Experimental Setup}
\label{app:hyperparameters}
We provide detailed hyperparameters of our experiments in Table \ref{tab:hyperparams}.


\begin{figure*}[!ht]
    \centering
    \includegraphics[width=\linewidth]{rebuttal-figures-src/hyperparams.pdf}
    \vspace{-1.5em}
    \caption{Concept Sliders Comparison \& Hyperparameter analysis: (Left) Impact of PCA directions: SliderSpace with 10 directions matches the FID of 64 Concept Sliders. More directions, upto 40, leads to improved FID. (Right) Effect of LoRA rank: Given a fixed training budget rank-one sliders are efficient than higher rank versions and outperforms Concept Sliders}
    \vspace{-0.3em}
    \label{fig:reb-hyperparam}
\end{figure*}



\subsection{Vision-to-Text}
\label{app:vision_to_text}

In this experiment, we analyze how \alignmodule{} maps visual features to the LLM’s text tokens. To do so, we manually curate a small dataset of image crops, each containing either a single word or a small set of visual text elements. Unlike the processing of high-resolution images described earlier (Section~\ref{sec:vision-encoder}), these image crops are not divided into tiles. Instead, the backbone image encoder processes each crop as a single tile, producing $14 \times 14$ features from the input image. The resulting features pass through the Softmax operation (Equation~\ref{eq:vocab_projection}), yielding a probability distribution over the LLM's text tokens for each feature (region). We examine the decoded text tokens from specific image regions to better understand how visual features are mapped to textual representations.

As shown in Figure~\ref{fig:text-vision-alignment}, white regions in the images tend to assign higher probabilities to punctuation tokens, such as commas or periods. Since punctuation structures written text, while white space separates document components like paragraphs, tables, and sections, \alignmodule{} appears to leverage these implicit patterns to align visual structures with semantically meaningful representations in the LLM’s embedding space.


\begin{figure}[hb]
    \centering
    \includegraphics[width=0.5\linewidth]{figures/token_visualization_top_k.pdf}
    \caption{\textbf{Mapping Visual-to-Text tokens.}  The left column shows the visual input to the model. In contrast, the right column visualizes the decoded tokens on a 14×14 grid, displaying the top k=2 tokens corresponding to the most likely LLM tokens predicted for the respective visual feature in each cell.}
    \label{fig:text-vision-alignment}
\end{figure}


\subsection{Case Studies}
\label{app:case}

In this section, we provide case studies for the experiments in Section~\ref{sec:main_results}. Specifically, we provide examples of our Llama-3.2-3B-\alignmodule{}, and its counterpart model with alternative connectors Llama-3.2-3B-MLP and Llama-3.2-3B-Ovis on three different datasets: KLC~\citep{stanislawek2021kleister}, DocVQA~\citep{docvqa}, and TextVQA~\citep{singh2019towards}. The examples are shown in Figure~\ref{fig:case_klc}, \ref{fig:case_docvqa}, and \ref{fig:case_textvqa}.

\begin{figure*}[htbp]
\centering
\begin{tabular}{c@{\hspace{6em}}c}
\begin{subfigure}[b]{0.32\textwidth}
  \centering
  \includegraphics[width=0.95\linewidth]{figures/case_1.jpg}
  \small
  \begin{tabular}{@{}l@{\hspace{3pt}}p{0.75\linewidth}}
    \textbf{Question:} & What is the value for the charity name?\\
    \textbf{GT:}       & \textit{(Ardingly College Ltd.)} \\
    \textbf{MLP:}      & \textit{(\textcolor{red}{Ardington} College Ltd.)} \xmark \\
    \textbf{Ovis:}     & \textit{(\textcolor{red}{Ardington} College Ltd.)} \xmark \\
    \textbf{\alignmodule{}:} & \textit{(Ardingly College Ltd.)} \cmark \\
  \end{tabular}
  \caption{Positive Example \#1}
\end{subfigure}
&
\begin{subfigure}[b]{0.32\textwidth}
  \centering
  \includegraphics[width=0.95\linewidth]{figures/case_2.jpg}
  \small
  \begin{tabular}{@{}l@{\hspace{3pt}}p{0.70\linewidth}}
    \textbf{Question:} & What is the value for the address postcode?\\
    \textbf{GT:}       & \textit{(SW2 2QP)} \\
    \textbf{MLP:}      & \textit{(\textcolor{red}{SW22 0PQ})} \xmark \\
    \textbf{Ovis:}     & \textit{(\textcolor{red}{SW2 2OP})} \xmark \\
    \textbf{\alignmodule{}:} & \textit{(SW2 2QP)} \cmark \\
  \end{tabular}
  \caption{Positive Example \#2}
\end{subfigure}
\\[0.8em]
\begin{subfigure}[b]{0.32\textwidth}
  \centering
  \includegraphics[width=0.95\linewidth]{figures/case_3.jpg}
  \small
  \begin{tabular}{@{}l@{\hspace{3pt}}p{0.70\linewidth}}
    \textbf{Question:} & What is the value for the charity name?\\
    \textbf{GT:}       & \textit{(Human Appeal)} \\
    \textbf{MLP:}      & \textit{(\textcolor{red}{Humanitarian Agenda})} \xmark \\
    \textbf{Ovis:}     & \textit{(Human Appeal)} \cmark \\
    \textbf{\alignmodule{}:} & \textit{(Human \textcolor{red}{Rightsappeal})} \xmark \\
  \end{tabular}
  \caption{Negative Example \#1}
\end{subfigure}
&
\begin{subfigure}[b]{0.32\textwidth}
  \centering
  \includegraphics[width=0.95\linewidth]{figures/case_4.jpg}
  \small
  \begin{tabular}{@{}l@{\hspace{3pt}}p{0.70\linewidth}}
    \textbf{Question:} & What is the value for the post town address?\\
    \textbf{GT:}       & \textit{(Bishop's Stortford)} \\
    \textbf{MLP:}      & \textit{(Stortford)} \xmark \\
    \textbf{Ovis:}     & \textit{(Bishop's Stortford)} \cmark \\
    \textbf{\alignmodule{}:} & \textit{(Stortford)} \xmark \\
  \end{tabular}
  \caption{Negative Example \#2}
\end{subfigure}
\end{tabular}
\caption{\textbf{Case Study for Connector Comparison on the KLC dataset~\citep{stanislawek2021kleister}.} 
We show four qualitative examples (including two correct and two incorrect examples) comparing Llama-3.2-3B-\alignmodule{} to the same architecture with different connectors, Llama-3.2-3B-MLP and Llama-3.2-3B-Ovis. ``GT'' denotes the ground truth.}
\label{fig:case_klc}
\end{figure*}



\begin{figure*}[htbp]
\centering
\begin{tabular}{c@{\hspace{6em}}c}
\begin{subfigure}[b]{0.35\textwidth}
  \centering
  \includegraphics[width=0.95\linewidth]{figures/case_5.jpg}
  \small
  \begin{tabular}{@{}l@{\hspace{4pt}}p{0.7\linewidth}}
    \textbf{Question:} & What does the afternoon session begin on June 29?\\
    \textbf{GT:}       & \textit{(1:00)} \\
    \textbf{MLP:}      & \textit{(\textcolor{red}{2:45})} \xmark \\
    \textbf{Ovis:}     & \textit{(\textcolor{red}{3:30})} \xmark \\
    \textbf{\alignmodule:} & \textit{(1:00)} \cmark
  \end{tabular}
  \caption{Positive Example \#1}
\end{subfigure}
&
\begin{subfigure}[b]{0.35\textwidth}
  \centering
  \includegraphics[width=0.95\linewidth]{figures/case_6.jpg}
  \small
  \begin{tabular}{@{}l@{\hspace{4pt}}p{0.8\linewidth}}
    \textbf{Question:} & What levels does the second table indicate?\\
    \textbf{GT:}       & \textit{(hematocrit data - Massachusetts)} \\
    \textbf{MLP:}      & \textit{(\textcolor{red}{SATISFACTORY})} \xmark \\
    \textbf{Ovis:}     & \textit{(\textcolor{red}{Females})} \xmark \\
    \textbf{\alignmodule:} & \textit{(hematocrit data - Massachusetts)} \cmark
  \end{tabular}
  \caption{Positive Example \#2}
\end{subfigure}
\\[1em]
\begin{subfigure}[b]{0.35\textwidth}
  \centering
  \includegraphics[width=0.95\linewidth]{figures/case_7.jpg}
  \small
  \begin{tabular}{@{}l@{\hspace{4pt}}p{0.7\linewidth}}
    \textbf{Question:} & What type of policy is described in this document?\\
    \textbf{GT:}       & \textit{(Policy on Document Control)} \\
    \textbf{MLP:}      & \textit{(Policy on Document Control)} \cmark \\
    \textbf{Ovis:}     & \textit{(\textcolor{red}{General Provisions})} \xmark \\
    \textbf{\alignmodule:} & \textit{(\textcolor{red}{Document Control})} \xmark
  \end{tabular}
  \caption{Negative Example \#1}
\end{subfigure}
&
\begin{subfigure}[b]{0.35\textwidth}
  \centering
  \includegraphics[width=0.95\linewidth]{figures/case_8.jpg}
  \small
  \begin{tabular}{@{}l@{\hspace{4pt}}p{0.7\linewidth}}
    \textbf{Question:} & What was the diet fed to the \#1 group?\\
    \textbf{GT:}       & \textit{(basal diet)} \\
    \textbf{MLP:}      & \textit{(basel diet)} \cmark \\
    \textbf{Ovis:}     & \textit{(\textcolor{red}{Whole blood})} \xmark \\
    \textbf{\alignmodule:} & \textit{(\textcolor{red}{control} diet)} \xmark
  \end{tabular}
  \caption{Negative Example \#2}
\end{subfigure}
\end{tabular}
\caption{\textbf{Case Study for Connector Comparison on the DocVQA dataset~\citep{docvqa}.} 
We show four qualitative examples (including two correct and two incorrect examples) comparing Llama-3.2-3B-\alignmodule{} to the same architecture with different connectors, Llama-3.2-3B-MLP and Llama-3.2-3B-Ovis. ``GT'' denotes the ground truth.}
\label{fig:case_docvqa}
\end{figure*}



    
    
    


\begin{figure*}[htbp]  
\centering
\begin{tabular}{c@{\hspace{1em}}c}
\begin{subfigure}[b]{0.40\textwidth} %
  \centering
  \includegraphics[width=0.95\linewidth]{figures/case_9.jpg}
  \small
  \begin{tabular}{@{}l@{\hspace{3pt}}p{0.70\linewidth}}
    \textbf{Question:} & What greeting is written on the letter? \\
    \textbf{GT:}       & \textit{(good bye)} \\
    \textbf{MLP:}      & \textit{(good)} \xmark \\
    \textbf{Ovis:}     & \textit{(good \textcolor{red}{buy})} \xmark \\
    \textbf{\alignmodule:} & \textit{(good bye)} \cmark
  \end{tabular}
  \caption{Positive Example \#1}
\end{subfigure}
&
\begin{subfigure}[b]{0.40\textwidth} %
  \centering
  \includegraphics[width=0.95\linewidth]{figures/case_10.jpg}
  \small
  \begin{tabular}{@{}l@{\hspace{3pt}}p{0.70\linewidth}}
    \textbf{Question:} & What indoor temperature is shown? \\
    \textbf{GT:}       & \textit{(68.4)} \\
    \textbf{MLP:}      & \textit{(\textcolor{red}{68 F})} \xmark \\
    \textbf{Ovis:}     & \textit{(\textcolor{red}{40.0})} \xmark \\
    \textbf{\alignmodule:} & \textit{(68.4)} \cmark
  \end{tabular}
  \caption{Positive Example \#2}
\end{subfigure}
\\[0.8em] %
\begin{subfigure}[b]{0.40\textwidth}
  \centering
  \includegraphics[width=0.95\linewidth]{figures/case_11.jpg}
  \small
  \begin{tabular}{@{}l@{\hspace{3pt}}p{0.70\linewidth}}
    \textbf{Question:} & What type of club is advertised? \\
    \textbf{GT:}       & \textit{(health club)} \\
    \textbf{MLP:}      & \textit{(\textcolor{red}{topnote} health club)} \xmark \\
    \textbf{Ovis:}     & \textit{(health club)} \cmark \\
    \textbf{\alignmodule:} & \textit{(\textcolor{red}{professional passionate personal})} \xmark
  \end{tabular}
  \caption{Negative Example \#1}
\end{subfigure}
&
\begin{subfigure}[b]{0.40\textwidth}
  \centering
  \includegraphics[width=0.95\linewidth]{figures/case_12.jpg}
  \small
  \begin{tabular}{@{}l@{\hspace{3pt}}p{0.70\linewidth}}
    \textbf{Question:} & What credit card is this? \\
    \textbf{GT:}       & \textit{(hadiah plus)} \\
    \textbf{MLP:}      & \textit{(hadiah plus)} \cmark \\
    \textbf{Ovis:}     & \textit{(\textcolor{red}{american big loyalty program})} \xmark \\
    \textbf{\alignmodule:} & \textit{(\textcolor{red}{hadia} plus)} \xmark
  \end{tabular}
  \caption{Negative Example \#2}
\end{subfigure}
\end{tabular}
\caption{\textbf{Case Study for Connector Comparison on the TextVQA dataset~\citep{singh2019towards}.} 
We show four qualitative examples (including two correct and two incorrect examples) comparing Llama-3.2-3B-\alignmodule{} to the same architecture with different connectors, Llama-3.2-3B-MLP and Llama-3.2-3B-Ovis. ``GT'' denotes the ground truth.}
\label{fig:case_textvqa}
\end{figure*}
