\vspace{-0.5em}
\section{Experimental Setup}
\begin{table*}[t]
\centering
\caption{\textbf{Impact of Connector Designs on VLM Performance:} We present the results of experiments evaluating different connector designs for conditioning LLMs on visual features. Our proposed \textbf{\alignmodule} connector is compared against a basic Multi-Layer Perceptron (\textbf{MLP}), the \textbf{Perceiver Resampler}, and \textbf{Ovis}. The results demonstrate that \alignmodule{} consistently outperforms these alternatives across all benchmarks.}
\resizebox{0.9\textwidth}{!}{%
\begin{tabular}{lccccccccccc}

\textbf{Model} &
\rot{\shortstack{\textbf{DocVQA} \\ \textcolor{gray}{\tiny{\textbf{VAL}}}}} & 
 \rot{\shortstack{\textbf{InfoVQA} \\ \textcolor{gray}{\tiny{VAL}}}} &
 \rot{\shortstack{\textbf{DeepForm} \\ \textcolor{gray}{\tiny{TEST}}}} &
 \rot{\shortstack{\textbf{KLC} \\ \textcolor{gray}{\tiny{TEST}}}} &
 \rot{\shortstack{\textbf{WTQ} \\ \textcolor{gray}{\tiny{TEST}}}} &
 \rot{\shortstack{\textbf{TabFact} \\ \textcolor{gray}{\tiny{TEST}}}} &
 \rot{\shortstack{\textbf{ChartQA} \\ \textcolor{gray}{\tiny{TEST}}}} &
 \rot{\shortstack{\textbf{TextVQA} \\ \textcolor{gray}{\tiny{VAL}}}} &
 \rot{\shortstack{\textbf{TableVQA} \\ \textcolor{gray}{\tiny{TEST}}}} &
 \rot{\textbf{Avg. Score}}\\ 

\toprule
Llama-3.2-3B-\textbf{MLP} & 
71.46 & 37.56 & 62.07 & 33.36 & 28.94 & 73.22 & 66.48 & 53.56 & 50.96 & \cellcolor{lightblue}53.06 \\


Llama-3.2-3B-\textbf{Perciever R.} & 
69.08 & 34.13 & 57.08 & 31.75 & 27.95 & 71.93 & 65.16 & 51.33 & 47.76 & \cellcolor{lightblue}50.68 \\


Llama-3.2-3B-\textbf{Ovis} & 
74.68 & 42.11 & 58.02 & 33.50 & 33.13 & 76.67 & 67.92 & 52.60 & 53.93 & \cellcolor{lightblue}54.72 \\


\hline

\rowcolor{lightgray!20}
Llama-3.2-3B-\textbf{\alignmodule} (ours) & 
\textbf{79.63} & \textbf{44.53} & \textbf{63.49} & \textbf{35.25} & \textbf{38.59} & \textbf{78.51} & \textbf{71.88} & \textbf{57.38} & \textbf{60.10} & \cellcolor{lightblue}\textbf{58.81} \\


\bottomrule
\end{tabular}%
}

\label{tab:connectors-ablations}
\vspace{-10px}
\end{table*}

\vspace{-0.4em}

\paragraph{Setup.} We conduct all experiments using 8 nodes of H100 GPUs, totaling 64 GPUs. For model training, we leverage the MS-Swift framework~\citep{zhao2024swiftascalablelightweightinfrastructure} for its flexibility. Additionally, we utilize the DeepSpeed framework~\citep{aminabadi2022deepspeedinferenceenablingefficient}, specifically the ZeRO-3 configuration, to optimize efficient parallel training across multiple nodes. Detailed hyperparameters are outlined in Appendix~\ref{app:hyperparameters}.

\paragraph{Baselines.} 
Our work focuses on architectural innovations, so we ensure that all baselines are trained on the same datasets.
To enable fair comparisons, we evaluate our models against a set of \textbf{Base VLMs} fine-tuned on the same instruction-tuning tasks (Stages 2 and 3) as our models, using the BigDocs-7.5M and BigDocs-DocDownstream datasets. This approach ensures consistent training data, avoiding biases introduced by the \textbf{Instruct} versions of VLMs, which are often trained on undisclosed instruction-tuning datasets. 
Due to the scarcity of recently released publicly available Base VLMs, we primarily compare our model against the following Base VLMs of varying sizes: Qwen2-VL-2B~\citep{wang2024qwen2vlenhancingvisionlanguagemodels}, DocOwl1.5-8B~\citep{hu2024mplugdocowl15unifiedstructure}, and LLama 3.2-11B~\citep{llama3}.

For additional context, we also include results from the Instruct versions of recent VLMs of different sizes: Phi3.5-Vision-4B~\citep{abdin2024phi3technicalreporthighly}, Qwen2-VL-2B and 7B~\citep{wang2024qwen2vlenhancingvisionlanguagemodels}, LLaVA-NeXT-7B~\citep{liu2024llavanext}, InternVL2.5-2B and 8B~\citep{chen2024internvl}, Janus-1.3B~\citep{wu2024janusdecouplingvisualencoding}, DeepSeek-VL2-Tiny~\citep{wu2024deepseekvl2mixtureofexpertsvisionlanguagemodels}, Ovis1.6-Gemma-9B~\citep{ovis}, Llama3.2-11B~\citep{llama3}, DocOwl1.5-8B~\citep{hu2024mplugdocowl15unifiedstructure}, and Pixtral-12B~\citep{agrawal2024pixtral12b}.

\paragraph{Evaluation Benchmarks.} We evaluate our models on a diverse range of document understanding benchmarks that assess the model's capabilities in OCR, chart reasoning, table processing, or form comprehension. In particular, we employ the VLMEvalKit~\citep{duan2024vlmevalkit} framework and report the results on the following popular benchmarks:  DocVQA~\citep{docvqa}, InfoVQA~\citep{mathew2021infographicvqa}, DeepForm~\citep{svetlichnaya2020deepform}, KLC~\citep{stanislawek2021kleister}, WTQ~\citep{pasupat2015compositional}, TabFact~\citep{Chen2020TabFact}, ChartQA~\citep{masry2022chartqa}, TextVQA~\citep{singh2019towards}, %
and TableVQA~\citep{kim2024tablevqa}.
