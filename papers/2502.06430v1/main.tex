\documentclass[sigconf]{acmart}

\usepackage{tabularx}
\usepackage{calc}
\usepackage[capitalize, noabbrev]{cleveref}
\usepackage{subcaption}
\usepackage{listings}
\lstset{
basicstyle=\small\ttfamily,
columns=flexible,
breaklines=true
}

\usepackage{siunitx}
\usepackage{enumitem}

\AtBeginDocument{%
  \providecommand\BibTeX{{%
    \normalfont B\kern-0.5em{\scshape i\kern-0.25em b}\kern-0.8em\TeX}}}

\copyrightyear{2025} 
\acmYear{2025} 
\setcopyright{cc}
\setcctype{by}
\acmConference[CHI '25]{CHI Conference on Human Factors in Computing Systems}{April 26-May 1, 2025}{Yokohama, Japan}
\acmBooktitle{CHI Conference on Human Factors in Computing Systems (CHI '25), April 26-May 1, 2025, Yokohama, Japan}\acmDOI{10.1145/3706598.3713890}
\acmISBN{979-8-4007-1394-1/25/04}





\begin{document}

\title[Content-Driven Local Response: Sentence-Level and Message-Level Mobile Email Replies]{Content-Driven Local Response: Supporting Sentence-Level and Message-Level Mobile Email Replies With and Without AI}


\author{Tim Zindulka}
\authornote{Both authors contributed equally to this research.}
\email{tim.zindulka@uni-bayreuth.de}
\orcid{0009-0009-1972-351X}
\affiliation{%
  \institution{University of Bayreuth}
  \city{Bayreuth}
  \country{Germany}
}

\author{Sven Goller}
\authornotemark[1]
\email{sven.goller@uni-bayreuth.de}
\orcid{0000-0001-5263-5372}
\affiliation{%
  \institution{University of Bayreuth}
  \city{Bayreuth}
  \country{Germany}
}

\author{Florian Lehmann}
\email{florian.lehmann@uni-bayreuth.de}
\orcid{0000-0003-0201-867X}
\affiliation{%
  \institution{University of Bayreuth}
  \city{Bayreuth}
  \country{Germany}
}

\author{Daniel Buschek}
\email{daniel.buschek@uni-bayreuth.de}
\orcid{0000-0002-0013-715X}
\affiliation{%
  \institution{University of Bayreuth}
  \city{Bayreuth}
  \country{Germany}
}

\renewcommand{\shortauthors}{Zindulka and Goller et al.}


\definecolor{TimsColor}{rgb}{0.1,0.5,0.8}
\newcommand{\tim}[1]{\textsf{\textbf{\textcolor{TimsColor}{[Tim: #1]}}}}
\definecolor{SvensColor}{rgb}{0.5,0.8,0.5}
\newcommand{\sven}[1]{\textsf{\textbf{\textcolor{SvensColor}{[Sven: #1]}}}}
\definecolor{FlosColor}{rgb}{0.9,0.1,0.8}
\newcommand{\flo}[1]{\textsf{\textbf{\textcolor{FlosColor}{[Flo: #1]}}}}
\definecolor{DanielsColor}{rgb}{0.9,0.6,0.1}
\newcommand{\daniel}[1]{\textsf{\textbf{\textcolor{DanielsColor}{[Daniel: #1]}}}}


\newcommand{\minsec}[2]{\SI{#1}{\minute} \SI{#2}{\second}}
\newcommand{\mins}[1]{\SI{#1}{\minute}}
\newcommand{\secs}[1]{\SI{#1}{\second}}
\newcommand{\pct}[1]{\ifnum\pdfstrcmp{#1}{X}=0
        X\% 
    \else\SI{#1}{\percent}\fi
}

\newcommand{\lmmci}[5]{$\beta$=#1, SE=#2, CI$_{95\%}$=[#3, #4], p#5}
\newcommand{\posthoc}[2]{#1, p#2}
\newcommand{\artf}[4]{$F$(#1,\,#2)=#3, p#4}
\newcommand{\artc}[3]{$t$(#1)=#2, p#3}
\newcommand{\petasq}[1]{$\eta_p^2$=#1}

\definecolor{deemphColor}{rgb}{0.4,0.4,0.4}
\newcommand{\deemph}[1]{\textcolor{deemphColor}{#1}}

\newcommand{\ivmode}{\textsc{UImode}}
\newcommand{\modeourstxt}{content-driven local response}
\newcommand{\modeoursTxt}{Content-driven local response}
\newcommand{\modeoursTXT}{Content-Driven Local Response}
\newcommand{\modemailtxt}{message-level reply generation}
\newcommand{\modemailTxt}{Message-level reply generation}
\newcommand{\modemanual}{\textsc{NoAI}}
\newcommand{\modeours}{\textsc{CDLR}}
\newcommand{\modemail}{\textsc{MSG}}

\newcommand{\imppass}{improvement pass}
\newcommand{\Imppass}{Improvement pass}



\newcommand{\studyOneN}{17} %
\newcommand{\studyTwoN}{126} %


\newcommand{\lastaccessed}{\textit{last accessed 22.08.2024}}

\newcommand{\oldId}[1]{} %


\newcommand\revision[1]{\textcolor{black}{#1}}

\begin{abstract}

During the early stages of interface design, designers need to produce multiple sketches to explore a design space.  Design tools often fail to support this critical stage, because they insist on specifying more details than necessary. Although recent advances in generative AI have raised hopes of solving this issue, in practice they fail because expressing loose ideas in a prompt is impractical. In this paper, we propose a diffusion-based approach to the low-effort generation of interface sketches. It breaks new ground by allowing flexible control of the generation process via three types of inputs: A) prompts, B) wireframes, and C) visual flows. The designer can provide any combination of these as input at any level of detail, and will get a diverse gallery of low-fidelity solutions in response. The unique benefit is that large design spaces can be explored rapidly with very little effort in input-specification. We present qualitative results for various combinations of input specifications. Additionally, we demonstrate that our model aligns more accurately with these specifications than other models. 

% OLD ABSTRACT
%When sketching Graphical User Interfaces (GUIs), designers need to explore several aspects of visual design simultaneously, such as how to guide the user’s attention to the right aspects of the design while making the intended functionality visible. Although current Large Language Models (LLMs) can generate GUIs, they do not offer the finer level of control necessary for this kind of exploration. To address this, we propose a diffusion-based model with multi-modal conditional generation. In practice, our model optionally takes semantic segmentation, prompt guidance, and flow direction to generate multiple GUIs that are aligned with the input design specifications. It produces multiple examples. We demonstrate that our approach outperforms baseline methods in producing desirable GUIs and meets the desired visual flow.

% Designing visually engaging Graphical User Interfaces (GUIs) is a challenge in HCI research. Effective GUI design must balance visual properties, like color and positioning, with user behaviors to ensure GUIs easy to comprehend and guide attention to critical elements. Modern GUIs, with their complex combinations of text, images, and interactive components, make it difficult to maintain a coherent visual flow during design.
% Although current Large Language Models (LLMs) can generate GUIs, they often lack the fine control necessary for ensuring a coherent visual flow. To address this, we propose a diffusion-based model that effectively handles multi-modal conditional generation. Our model takes semantic segmentation, optional prompt guidance, and ordered viewing elements to generate high-fidelity GUIs that are aligned with the input design specifications.
% We demonstrate that our approach outperforms baseline methods in producing desirable GUIs and meets the desired visual flow. Moreover, a user study involving XX designers indicates that our model enhances the efficiency of the GUI design ideation process and provides designers with greater control compared to existing methods.    



% %%%%%%%%%%%%%%%%%%%%%%%%%%%%%%%%%%%%%%%%%%%%%%%%%%%%%%
% % Writing Clinic Comments:
% %%%%%%%%%%%%%%%%%%%%%%%%%%%%%%%%%%%%%%%%%%%%%%%%%%%%%%
% % Define: Effective UI design
% % Motivate GANs and write in full form.
% % LLMs vs ControlNet vs GANs
% % Say something about the Figma plugin?
% % Write the work is novel or what has been done before
% % What is desirable UI and how to evalutate that?
% % Visual Flow - main theme (center around it)
% % Re-Title: use word Flow!
% % Use ControlNet++ & SPADE for abstract.
% % Write about input/output. 
% % Why better than previous work?
% %%%%%%%%%%%%%%%%%%%%%%%%%%%%%%%%%%%%%%%%%%%%%%%%%%%%%

% % v2:
% % \noindent \textcolor{red}{\textbf{NEW Abstract!} (Post Writing Clinic 1 - 25-Jun)}

% % \noindent \textcolor{red}{----------------------------------------------------------------------}

% % \noindent Designing user interfaces (UIs) is a time-consuming process, particularly for novice designers. 
% % Creating UI designs that are effective in market funneling or any other designer defined goal requires a good understanding of the visual flow to guide users' attention to UI elements in the desired order. 
% % While current Large Language Models (LLMs) can generate UIs from just prompts, they often lack finer pixel-precise control and fail to consider visual flow. 
% % In this work, we present a UI synthesis method that incorporates visual flow alongside prompts and semantic layouts. 
% % Our efficient approach uses a carefully designed Generative Adversarial Network (GAN) optimized for scenarios with limited data, making it more suitable than diffusion-based and large vision-language models.
% % We demonstrate that our method produces more "desirable" UIs according to the well-known contrast, repetition, alignment, and proximity principles of design. 
% % We further validate our method through comprehensive automatic non-reference, human-preference aligned network scoring and subjective human evaluations.
% % Finally, an evaluation with xx non-expert designers using our contributed Figma plugin shows that <method-name> improves the time-efficiency as well as the overall quality of the UI design development cycle.

% % \noindent \textcolor{red}{----------------------------------------------------------------------}


% \noindent \textcolor{blue}{\textbf{NEW Abstract!} (Pre Writing Clinic 9-July)}

% \noindent \textcolor{blue}{----------------------------------------------------------------------}

% \noindent Exploring different graphical user interface (GUI) design ideas is time-consuming, particularly for novice designers. 
% Given the segmentation masks, design requirement as prompt, and/or preferred visual flow, we aim to facilitate creative exploration for GUI design and generate different UI designs for inspiration.
% While current Vision Language Models (VLMs) can generate GUIs from just prompts, they often lack control over visual concepts and flow that are difficult to convey through language during the generation process. 
% In this work, we present FlowGenUI, a semantic map-guided GUI synthesis method that optionally incorporates visual flow information based on the user's choice alongside language prompts. 
% We demonstrate that our model not only creates more realistic GUIs but also creates "predictable" (how users pay attention to and order of looking at GUI elements) GUIs.
% Our approach uses Stable Diffusion (SD), a large paired image-text pretrained diffusion model with a rich latent space that we steer toward realistic GUIs using a trainable copy of SD's encoder for every condition (segmentation masks, prompts, and visual flow). 
% We further provide a semantic typography feature to create custom text-fonts and styles while also alleviating SD's inherent limitations in drawing coherent, meaningful and correct aspect-ratio text. 
% Finally, a subjective evaluation study of XX non-expert and expert designers demonstrates the efficiency and fidelity of our method.


% This process encourages creativity and prevents designers from falling into habitual patterns.


% ------------------------------------------------------------------
% Joongi Why is it important to create realistic GUI?
% I do not see how the Visual Flow given on the left hand side is reflected in the results on the right hand side. 
% I’d avoid making unsubstantiated claims about designers (falling into habitual patterns).
% The UIs you generate do not “align with users’ attention patterns” but rather try to control it (that’s what visual flow means)
% ------------------------------------------------------------------
% Comments - Writing Clinic - 9th July:
% Improve title. More names: FlowGen
% Figure 1: Use an inference time hand-drawn mask
% Figure 1: Show both workflows. Add a designer --> Input.
% Figure 1: Make them more diverse
% ------------------------------------------------------------------
% Designing graphical user interfaces (GUIs) requires human creativity and time. Designers often fall into habitual patterns, which can limit the exploration of new ideas. 
% To address this, we introduce FlowGenUI, a method that facilitates creative exploration and generates diverse GUI designs for inspiration. By using segmentation masks, design requirements as prompts, and/or selected visual flows, our approach enhances control over the visual concepts and flows during the generation process, which current Vision Language Models (VLMs) often lack.
% FlowGenUI uses Stable Diffusion (SD), a largely pretrained text-to-image diffusion model, and guides it to create realistic GUIs. 
% We achieve this by using a trainable copy of SD's encoder for each condition (segmentation masks, prompts, and visual flow). 
% This method enables the creation of more realistic and predictable GUIs that align with users' attention patterns and their preferred order of viewing elements.
% We also offer a semantic typography feature that creates custom text fonts and styles while addressing SD's limitations in generating coherent, meaningful, and correctly aspect-ratio text.
% Our approach's efficiency and fidelity are evaluated through a subjective user study involving XX designers. 
% The results demonstrate the effectiveness of FlowGenUI in generating high-quality GUI designs that meet user requirements and visual expectations.

% ---------------------------------------


%A critical and general issue remains while using such deep generative priors: creating coherent, meaningful and correct aspect-ratio text. 
%We tackle this issue within our framework and additionally provide a semantic typography feature to create custom text-fonts and styles. 


% %Creating UI designs that are effective in market funneling or any other designer-defined goal requires a good understanding of the visual flow to guide users' attention to UI elements in the desired order. 
% %While current largely pre-trained Vision Language Models (VLMs) can generate GUIs from just prompts, they often lack finer or pixel-precise control which can be crucial for many easy-to-understand visual concepts but difficult to convey through language. 
% % However, obtaining such pixe-level labels is an extremely expensive so we
% % For example - overlaying text on images with certain aspect ratios and two equally separated buttons 
% Additionally, all prior GUI generation work fails to consider visual flow information during the generation process. 
% We demonstrate that visual flow-informed generation not only creates more realistic and human-friendly GUIs but also creates "predictable" (how users pay attention to and order of looking at GUI elements) UIs that could be beneficial for designers for tasks like creating effective market funnels.
% In this work, we present a semantic map-guided GUI synthesis method that optionally incorporates visual flow information based on the user's choice alongside language prompts. 
% Our approach uses Stable Diffusion, a large (billions) paired image-text pretrained diffusion model with a rich latent space that we steer toward realistic GUIs using an ensemble of ControlNets. 
% % TODO: Mention it in 1 sentence:
% A critical and general issue remains while using such deep generative priors: creating coherent, meaningful and correct aspect-ratio text. 
% We tackle this issue within our framework and additionally provide a semantic typography feature to create custom text-fonts and styles. 
% To evaluate our method, we demonstrate that our method produces more "desirable" UIs according to the well-known contrast, repetition, alignment, and proximity principles of design. 
% % We further validate our method through comprehensive automatic non-reference and human-preference aligned scores. (TODO: Maybe Unskip if we get UIClip from Jason!)
% % TODO: Re-word this and only keep ideation cycles and time-efficiency.
% Finally, a subjective evaluation study of XX non-expert and expert designers demonstrates the efficiency and fidelity of our method.
% % improves the time-efficiency by quick iterations of the UI design ideation process.
% %Finally, an evaluation with xx non-expert designers using our contributed <method-name> improves the time-efficiency by quick iterations of the UI design ideation cycle.

%\noindent \textcolor{blue}{----------------------------------------------------------------------}


%In an evaluation with xx designers, we found that GenerativeLayout: 1) enhances designers' exploration by expanding the coverage of the design space, 2) reduces the time required for exploration, and 3) maintains a perceived level of control similar to that of manual exploration.



% Present-day graphical user interfaces (GUIs) exhibit diverse arrangements of text, graphics, and interactive elements such as buttons and menus, but representations of GUIs have not kept up. They do not encapsulate both semantic and visuo-spatial relationships among elements. %\color{red} 
% To seize machine learning's potential for GUIs more efficiently, \papername~ exploits graph neural networks to capture individual elements' properties and their semantic—visuo-spatial constraints in a layout. The learned representation demonstrated its effectiveness in multiple tasks, especially generating designs in a challenging GUI autocompletion task, which involved predicting the positions of remaining unplaced elements in a partially completed GUI. The new model's suggestions showed alignment and visual appeal superior to the baseline method and received higher subjective ratings for preference. 
% Furthermore, we demonstrate the practical benefits and efficiency advantages designers perceive when utilizing our model as an autocompletion plug-in.


% Overall pipeline: Maybe drop semantic typography / visual flow?
\end{abstract}

\begin{CCSXML}
<ccs2012>
   <concept>
       <concept_id>10003120.10003121.10011748</concept_id>
       <concept_desc>Human-centered computing~Empirical studies in HCI</concept_desc>
       <concept_significance>500</concept_significance>
       </concept>
   <concept>
       <concept_id>10003120.10003121.10003128.10011753</concept_id>
       <concept_desc>Human-centered computing~Text input</concept_desc>
       <concept_significance>500</concept_significance>
       </concept>
   <concept>
       <concept_id>10010147.10010178.10010179</concept_id>
       <concept_desc>Computing methodologies~Natural language processing</concept_desc>
       <concept_significance>500</concept_significance>
       </concept>
 </ccs2012>
\end{CCSXML}

\ccsdesc[500]{Human-centered computing~Empirical studies in HCI}
\ccsdesc[500]{Human-centered computing~Text input}
\ccsdesc[500]{Computing methodologies~Natural language processing}

\keywords{Writing assistance, Large language models, Human-AI interaction, Email, Mobile text entry}

\begin{teaserfigure}
    \centering
    \includegraphics[width=\textwidth]{figures/teaser}
   \caption{Replying to an email with \textit{\modeoursTXT}: \textit{(1)} In the \textit{local response view}, users can insert responses \textit{(A)} directly while reading the email. \textit{(B)} Tapping on a sentence opens a response widget, \textit{(C)} with a text box where users enter a response or a prompt that affects \textit{(D)} the sentence suggestions below. \textit{(2)} After adding local responses, users go to the \textit{draft view}, to turn their responses into a full reply email. They can do so manually and/or with the help of \textit{(E)} an AI \imppass{} feature, which generates \textit{(F)} a message-level suggestion, displayed with highlighted changes. These AI features are flexible and optional: Users can add local responses without using suggestions. They can also skip directly to the draft view, optionally enter a prompt there, and use the improvement feature to generate a full reply directly. This supports flexible workflows.}
   \label{fig:teaser}
   \Description{This figure illustrates the process of replying to an email using the "Content-Driven Local Response" feature and outlines different flexible workflows with optional AI support.
   The figure is divided into three main sections:
   Local Response View with Sentence-Level AI (Left Panel)
   A) Users can insert responses directly while reading the email.
   B) Tapping on a sentence opens a response widget, which allows users to interact with the email content.
   C) The widget includes a text box where users can enter their own response or a prompt that will influence the sentence suggestions displayed below.
   D) Below the text box, sentence-level AI-generated suggestions are provided based on the entered prompt or context of the email.
   Draft View with Message-Level AI (Middle Panel)
   After entering responses or prompts in the local response view, users can proceed to the draft view.
   (E) In the draft view, users can manually edit the draft or use an AI improvement pass feature to generate a message-level suggestion.
   (F) The AI-generated suggestion is displayed with highlighted changes, allowing users to review and accept the improvements.
   Examples of Flexible Workflows with Optional AI (Bottom Section)
   The figure outlines various workflows users can follow, ranging from full AI-assisted reply generation to fully manual drafting:
   Full reply generation: Users can go directly to the draft view and use the improvement feature to generate a complete reply.
   Partial AI-assisted reply generation: Users can draft partially with sentence-level AI support and then finalise the reply with message-level AI or manual edits.
   Fully manual: Users can choose to draft and send the email without using any AI assistance.
   This figure highlights the flexibility of the system, allowing users to choose between different levels of AI support depending on their preference or the specific requirements of the email task.}
\end{teaserfigure}


\maketitle


% humans are sensitive to the way information is presented.

% introduce framing as the way we address framing. say something about political views and how information is represented.

% in this paper we explore if models show similar sensitivity.

% why is it important/interesting.



% thought - it would be interesting to test it on real world data, but it would be hard to test humans because they come already biased about real world stuff, so we tested artificial.


% LLMs have recently been shown to mimic cognitive biases, typically associated with human behavior~\citep{ malberg2024comprehensive, itzhak-etal-2024-instructed}. This resemblance has significant implications for how we perceive these models and what we can expect from them in real-world interactions and decisionmaking~\citep{eigner2024determinants, echterhoff-etal-2024-cognitive}.

The \textit{framing effect} is a well-known cognitive phenomenon, where different presentations of the same underlying facts affect human perception towards them~\citep{tversky1981framing}.
For example, presenting an economic policy as only creating 50,000 new jobs, versus also reporting that it would cost 2B USD, can dramatically shift public opinion~\cite{sniderman2004structure}. 
%%%%%%%% 图1:  %%%%%%%%%%%%%%%%
\begin{figure}[t]
    \centering
    \includegraphics[width=\columnwidth]{Figs/01.pdf}
    \caption{Performance comparison (Top-1 Acc (\%)) under various open-vocabulary evaluation settings where the video learners except for CLIP are tuned on Kinetics-400~\cite{k400} with frozen text encoders. The satisfying in-context generalizability on UCF101~\cite{UCF101} (a) can be severely affected by static bias when evaluating on out-of-context SCUBA-UCF101~\cite{li2023mitigating} (b) by replacing the video background with other images.}
    \label{fig:teaser}
\end{figure}


Previous research has shown that LLMs exhibit various cognitive biases, including the framing effect~\cite{lore2024strategic,shaikh2024cbeval,malberg2024comprehensive,echterhoff-etal-2024-cognitive}. However, these either rely on synthetic datasets or evaluate LLMs on different data from what humans were tested on. In addition, comparisons between models and humans typically treat human performance as a baseline rather than comparing patterns in human behavior. 
% \gabis{looks good! what do we mean by ``most studies'' or ``rarely'' can we remove those? or we want to say that we don't know of previous work doing both at the same time?}\gili{yeah the main point is that some work has done each separated, but not all of it together. how about now?}

In this work, we evaluate LLMs on real-world data. Rather than measuring model performance in terms of accuracy, we analyze how closely their responses align with human annotations. Furthermore, while previous studies have examined the effect of framing on decision making, we extend this analysis to sentiment analysis, as sentiment perception plays a key explanatory role in decision-making \cite{lerner2015emotion}. 
%Based on this, we argue that examining sentiment shifts in response to reframing can provide deeper insights into the framing effect. \gabis{I don't understand this last claim. Maybe remove and just say we extend to sentiment analysis?}

% Understanding how LLMs respond to framing is crucial, as they are increasingly integrated into real-world applications~\citep{gan2024application, hurlin2024fairness}.
% In some applications, e.g., in virtual companions, framing can be harnessed to produce human-like behavior leading to better engagement.
% In contrast, in other applications, such as financial or legal advice, mitigating the effect of framing can lead to less biased decisions.
% In both cases, a better understanding of the framing effect on LLMs can help develop strategies to mitigate its negative impacts,
% while utilizing its positive aspects. \gabis{$\leftarrow$ reading this again, maybe this isn't the right place for this paragraph. Consider putting in the conclusion? I think that after we said that people have worked on it, we don't necessarily need this here and will shorten the long intro}


% If framing can influence their outputs, this could have significant societal effects,
% from spreading biases in automated decision-making~\citep{ghasemaghaei2024understanding} to reducing public trust in AI-generated content~\citep{afroogh2024trust}. 
% However, framing is not inherently negative -- understanding how it affects LLM outputs can offer valuable insights into both human and machine cognition.
% By systematically investigating the framing effect,


%It is therefore crucial to systematically investigate the framing effect, to better understand and mitigate its impact. \gabis{This paragraph is important - I think that right now it's saying that we don't want models to be influenced by framing (since we want to mitigate its impact, right?) When we talked I think we had a more nuanced position?}




To better understand the framing effect in LLMs in comparison to human behavior,
we introduce the \name{} dataset (Section~\ref{sec:data}), comprising 1,000 statements, constructed through a three-step process, as shown in Figure~\ref{fig:fig1}.
First, we collect a set of real-world statements that express a clear negative or positive sentiment (e.g., ``I won the highest prize'').
%as exemplified in Figure~\ref{fig:fig1} -- ``I won the highest prize'' positive base statement. (2) next,
Second, we \emph{reframe} the text by adding a prefix or suffix with an opposite sentiment (e.g., ``I won the highest prize, \emph{although I lost all my friends on the way}'').
Finally, we collect human annotations by asking different participants
if they consider the reframed statement to be overall positive or negative.
% \gabist{This allows us to quantify the extent of \textit{sentiment shifts}, which is defined as labeling the sentiment aligning with the opposite framing, rather then the base sentiment -- e.g., voting ``negative'' for the statement ``I won the highest prize, although I lost all my friends on the way'', as it aligns with the opposite framing sentiment.}
We choose to annotate Amazon reviews, where sentiment is more robust, compared to e.g., the news domain which introduces confounding variables such as prior political leaning~\cite{druckman2004political}.


%While the implications of framing on sensitive and controversial topics like politics or economics are highly relevant to real-world applications, testing these subjects in a controlled setting is challenging. Such topics can introduce confounding variables, as annotators might rely on their personal beliefs or emotions rather than focusing solely on the framing, particularly when the content is emotionally charged~\cite{druckman2004political}. To balance real-world relevance with experimental reliability, we chose to focus on statements derived from Amazon reviews. These are naturally occurring, sentiment-rich texts that are less likely to trigger strong preexisting biases or emotional reactions. For instance, a review like ``The book was engaging'' can be framed negatively without invoking specific cultural or political associations. 

 In Section~\ref{sec:results}, we evaluate eight state-of-the-art LLMs
 % including \gpt{}~\cite{openai2024gpt4osystemcard}, \llama{}~\cite{dubey2024llama}, \mistral{}~\cite{jiang2023mistral}, \mixtral{}~\cite{mistral2023mixtral}, and \gemma{}~\cite{team2024gemma}, 
on the \name{} dataset and compare them against human annotations. We find  that LLMs are influenced by framing, somewhat similar to human behavior. All models show a \emph{strong} correlation ($r>0.57$) with human behavior.
%All models show a correlation with human responses of more than $0.55$ in Pearson's $r$ \gabis{@Gili check how people report this?}.
Moreover, we find that both humans and LLMs are more influenced by positive reframing rather than negative reframing. We also find that larger models tend to be more correlated with human behavior. Interestingly, \gpt{} shows the lowest correlation with human behavior. This raises questions about how architectural or training differences might influence susceptibility to framing. 
%\gabis{this last finding about \gpt{} stands in opposition to the start of the statement, right? Even though it's probably one of the largest models, it doesn't correlate with humans? If so, better to state this explicitly}

This work contributes to understanding the parallels between LLM and human cognition, offering insights into how cognitive mechanisms such as the framing effect emerge in LLMs.\footnote{\name{} data available at \url{https://huggingface.co/datasets/gililior/WildFrame}\\Code: ~\url{https://github.com/SLAB-NLP/WildFrame-Eval}}

%\gabist{It also raises fundamental philosophical and practical questions -- should LLMs aim to emulate human-like behavior, even when such behavior is susceptible to harmful cognitive biases? or should they strive to deviate from human tendencies to avoid reproducing these pitfalls?}\gabis{$\leftarrow$ also following Itay's comment, maybe this is better in the dicsussion, since we don't address these questions in the paper.} %\gabis{This last statement brings the nuance back, so I think it contradicts the previous parapgraph where we talked about ``mitigating'' the effect of framing. Also, I think it would be nice to discuss this a bit more in depth, maybe in the discussion section.}






\section{Background on Causal Inference}
\label{sec:background-causal} 



 \newtextold{In this section, we 
 %formalize the notion of {\em Average Treatment Effect and understand the 
 review the basic concepts and key assumptions for inferring the effects of an intervention on the outcome on collected datasets without performing randomized controlled experiments. 
We use {\em Pearl's graphical causal model} for {\em observational causal analysis} \cite{pearl2009causal} to define these concepts.}


\par
\paratitle{Causal Inference and Causal DAGs} The primary goal of causal inference is to model causal dependencies between attributes and evaluate how changing one variable (referred to as intervention) would affect the other.
Pearl's Probabilistic Graphical Causal Model \cite{pearl2009causal} can be written as a tuple $(\exo, \edvar, Pr_{\exo}, \psi)$, where $\exo$ is a set of {\em exogenous} variables, $\Pr_{\exo}$ is the joint distribution of \exo, and $\edvar$ is a set of observed {\em endogenous variables}.
Here $\psi$ is a set of structural equations that encode dependencies among variables. The equation for $A \in \edvar$ takes the following form:
%that encode the dependencies among the variables.  These equations are of the form 
$$\psi_{A}: 
\dom(Pa_{\exo}(A)) {\times} \dom(Pa_{\edvar}(A)) \to \dom(A)$$
Here $Pa_{\exo}(A) {\subseteq} {\exo}$ and $Pa_{\edvar}(A) {\subseteq} \edvar \setminus \{A\}$ respectively denote the exogenous and endogenous parents of $A$. A causal relational model is associated with a directed acyclic graph ({\em causal DAG}) $G$, whose nodes are the endogenous variables $\edvar$ and there is a directed edge from $X$ to $O$ if  $X {\in} Pa_{\edvar}(O)$. The causal DAG obfuscates exogenous variables as they are unobserved. %Any given set of values for the exogenous variables completely determines the values of the endogenous variables by the structural equations (we do not need any known closed-form expressions of the structural equations in this work). 
The probability distribution $\Pr_{\exo}$ on exogenous variables $\exo$ induces a probability distribution  
on the endogenous variables $\edvar$ by the structural equations $\psi$.  A causal DAG can be constructed by a domain expert as in the above example, or using existing {\em causal discovery} algorithms~\cite{glymour2019review}. 



\begin{figure}
    \centering
    \small
    \begin{tikzpicture}[node distance=0.6cm and 1cm, every node/.style={minimum size=0.5cm}]
        \tikzset{vertex/.style = {draw, circle, align=center}}

        \node[vertex] (Ethnicity) {\bf\scriptsize{{Ethnicity}}};
        \node[vertex, right=0.3cm of Ethnicity] (Gender) {\bf{\scriptsize{Gender}}};
        \node[vertex, right=0.3cm of Gender] (Age) {\bf{\scriptsize{Age}}};
        \node[vertex, below=0.3cm of Gender] (Role) {\bf{\scriptsize{Role}}};
        \node[vertex, right=0.3cm of Role] (Education) {\bf{\small{\scriptsize{Education}}}};
        \node[vertex, below=0.3cm of Role] (Salary) {\bf{\scriptsize{Salary}}};

        \draw[->] (Ethnicity) -- (Salary);
        \draw[->] (Gender) -- (Role);
        \draw[->] (Age) -- (Role);
         \draw[->] (Education) -- (Role);
           \draw[->] (Education) -- (Salary);
             \draw[->] (Ethnicity) -- (Education);
                \draw[->] (Ethnicity) -- (Role);
             \draw[->] (Gender) -- (Education);
               \draw[->] (Age) -- (Education);
                 \draw[->] (Role) -- (Salary);
        \draw[->] (Gender) to[bend right] (Salary);
        \draw[->] (Age) -- (Salary);
    \end{tikzpicture}
    \caption{Partial causal DAG for the Stack Overflow dataset.}
    \label{fig:causal_DAG}
\end{figure}



 \begin{example}
Figure \ref{fig:causal_DAG} depicts a partial causal DAG for the SO dataset over the attributes in Table \ref{tab:data} as endogenous variables (we use a larger causal DAG with all 20 attributes in our experiments). 
  Given this causal DAG, we can observe that the role that a coder has in their company depends on their education, age gender and ethnicity.
\end{example}
\par


\par
\paratitle{Intervention} In Pearl's model, a treatment $T = t$ (on one or more variables) is considered as an {\em intervention} to a causal DAG by mechanically changing the DAG such that the values of node(s) of $T$ in $G$ are set to the value(s) in $t$, which is denoted by $\doop(T = t)$. Following this operation, the probability distribution of the nodes in the graph changes as the treatment nodes no longer depend on the values of their parents. Pearl's model gives an approach to estimate the new probability distribution by identifying the confounding factors $Z$ described earlier using conditions such as {\em d-separation} and {\em backdoor criteria} \cite{pearl2009causal}, which we do not discuss in this paper.


\par
\paratitle{Average Treatment Effect} The effects of an intervention are often measured by evaluating
% \par
% \paratitle{Causal inference, Treatment, ATE, and CATE}
% \newtextold{One of the primary goals  of {\em causal inference} is to estimate the effect of making a change in terms of a {\em treatment} $T$ (often referred to as an intervention)
% on the outcome $O$. 
% %A variable that is modified is often referred to as the treatment variable $T$ and the metric used to captures 
% The effect of treatment $T$ on outcome $O$ is measured by 
% %is known as 
{\em Conditional Average treatment effect (CATE)}, 
%a {\em treatment variable} $T$ on an outcome variable $O$ (e.g., what is the effect of higher \verb|Education| on \verb|Salary|). 
measuring the effect of an intervention on a subset of records~\cite{rubin1971use,holland1986statistics} by calculating the difference in average outcomes between the group that receives the treatment and the group that does not (called the {\em control} group), providing an estimate of how the intervention by $T$ influences an outcome $O$ for a given subpopulation. 
% Mathematically,
% \begin{equation}
%     %{\small ATE(T,O) = \mathbb{E}[O \mid \doop(T=1)] -      \mathbb{E}[O \mid \doop(T=0)]}
%     {\small ATE(T, O) = \mathbb{E}[O \mid \doop(T=1)] -  
%     \mathbb{E}[O \mid \doop(T=0)]}
% \label{eq:ate}
% \end{equation}
% In our work, where the treatment with maximum effect may vary among different subpopulations, we are interested in computing the \emph{Conditional Average Treatment Effect} (CATE), which measures the effect of a treatment on an outcome on \emph{a subset of input units}~\cite{rubin1971use,holland1986statistics}. 
Given a subset of the records defined by (a vector of) attributes $B$ and their values $b$, 
%g {\in} \Qagg(\db)$ defined by a predicate $G {=} g$ 
we can compute $CATE(T,O \mid B = b)$ as:
{
\begin{eqnarray}    
    %CATE(T,O \mid G=g) = \mathbb{E}[O \mid \doop(T=1)&, G=g] -  \mathbb{E}[O \mid \doop(T=0), G=g] 
   % CATE(T,O \mid B = b) = 
    \mathbb{E}[O \mid \doop(T=1), B = b] -  
    \mathbb{E}[O \mid \doop(T=0), B = b]\label{eq:cate}
\end{eqnarray}
}
Setting $B=\phi$ is equivalent to the ATE estimate.
The above definitions assumes that the treatment assigned to one unit does not affect the outcome of another unit (called the {Stable Unit Treatment Value Assumption (SUTVA)) \cite{rubin2005causal}}\footnote{This assumption does not hold for causal inference on multiple tables and even on a single table where tuples depend on each other.}. 


The ideal way of estimating the ATE and CATE is through {\em randomized controlled experiments}, 
where the population is randomly divided into two groups (treated and control, for binary treatments): 
%treated group that receives the treatment and control group that does not (denoted by 
%{the \em treated} group 
denoted by 
$\doop(T = 1)$ 
%for a binary treatment)  (the {\em control} group, 
and $\doop(T = 0)$ resp.)~\cite{pearl2009causal}.
%\sr{edited up to here, going to read the rest first, this section should not look like causumx}
%\par
%\par
However, randomized experiments cannot always be performed due to ethical or feasibility issues. In these scenarios, observational data is used to estimate the treatment effect, which requires the following additional assumptions. 
% {\em Observational Causal Analysis} still allows sound causal inference under additional assumptions. Randomization in controlled trials mitigates the effect of {\em confounding factors}, i.e., attributes that can affect the treatment assignment and outcome. Suppose we want to understand the causal effect of \verb|Education| on \verb|Salary| from the SO dataset.  %in Example~\ref{ex:running_example}. 
% We no longer apply Eq. (\ref{eq:ate}) since the values of \verb|Education| were not assigned at random in this data, and obtaining higher education largely depends on other attributes like \verb|Gender|, \verb|Age|, and \verb|Country|. 
% Pearl's model provides ways to account for these confounding attributes $Z$ to get an unbiased causal estimate from observational data under the following assumptions ($\independent$ denotes independence):
% \vspace{-2mm}
\newtextold{
The first assumption is called {\em unconfoundedness} or {\em strong ignorability}  \cite{rosenbaum1983central} says that the independence of outcome $O$ and treatment $T$ conditioning on a set of confounder variables  (covariates) $Z$, i.e.,
%\begin{eqnarray}
 $    O \independent T | Z {=} z$.
 %\label{eq:unconfoundedness}
%\end{eqnarray}
The second assumption called {\em overlap or positivity} says that there is a chance of observing individuals in both the treatment and control groups for every combination of covariate values, i.e., 
%\begin{eqnarray}
   $ 0 < Pr(T {=} 1 ~~|~~Z {=} z)< 1 $.
   %\label{eq:overlap}
%\end{eqnarray}
}
%\sg{Is this overlap or positivity? maybe both are the same?} \sr{yeah - same - from Google AI - The overlap assumption, also known as the positivity assumption, is a key assumption in causal inference that states that there is a chance of observing individuals in both the treatment and control groups for every combination of covariate values.}
% The above conditions are known as {\em Strong Ignorability} in Rubin's model \cite{rubin2005causal}.
The unconfoundedness assumption requires that the treatment $T$ and the outcome $O$ be independent when conditioned on a set of variables $Z$. In SO, assuming that only $Z$ =\{\verb|Gender|, \verb|Age|, \verb|Country|\} affects $T = $ \verb|Education|, if we condition on a fixed set of values of $Z$, i.e., consider people of a given gender, from a given country, and at a given age, then $T = $ \verb|Education| and $O = $ \verb|Salary| are independent. For such confounding factors $Z$,  Eq. (\ref{eq:cate}) reduces to the following form 
(positivity 
gives the feasibility of the expectation difference): 
 \vspace{-1mm}
{\small
\begin{flalign}    
% \begin{eqnarray}
   % % & ATE(T,O) = \mathbb{E}_Z \left[\mathbb{E}[O \mid T=1, Z = z] -  
   %  \mathbb{E}[O \mid T=0, Z = z] \right] \label{eq:conf-ate}\\
 & CATE(T,O {\mid} B {=} b) {=} \nonumber
    \mathbb{E}_Z \left[\mathbb{E}[O {\mid} T{=}1, B {=} b, Z {=} z] {-}  
    \mathbb{E}[O {\mid} T{=}0, B {=} b, Z {=} z]\right]\label{eq:conf-cate}
\end{flalign}
% \end{eqnarray}
}
% \vspace{-4mm}
This equation contains conditional probabilities and not $\doop(T = b)$, which can be estimated from an observed data. 
Pearl's model gives a systematic way to find such a $Z$ when a causal DAG is available. 




\section{Concept Development}
We introduce the concept of \modeourstxt. At a high level, we take inspiration from mobile microtasking UIs and integrate a local response interface directly into the incoming email. Overall, our final design is situated in the  unexplored space in between sentence-level and message-level approaches for AI support.


\begin{figure*}
    \centering
    \includegraphics[width=0.5\textwidth]{figures/formative_prototype}
    \caption{Replying to an email with our first prototype: \textit{(1)} In the first screen, users insert responses directly while reading the email. Tapping on a sentence opens a response widget, with a text box where users enter a response or a prompt that affects the sentence suggestions below. \textit{(2)} After adding local responses, users can edit their reply on a second screen, by reordering paragraphs via drag-and-drop, by deleting paragraphs via swiping left-to-right, and by manual editing via the integrated keyboard.  \textit{(3)} On the third screen, users can finalise the reply before sending it.}
    \Description{This Figure shows how to reply to an email with our first prototype: On the left it shows the first screen, where users insert responses directly while reading the email. Tapping on a sentence opens a response widget, with a text box where users enter a response or a prompt that affects the sentence suggestions below. After adding local responses, users can edit their reply on a second screen, which is shown in the centre of the figure. By reordering paragraphs via drag-and-drop, swiping left-to-right to delete paragraphs, and via manually editing using the integrated keyboard user can edit their reply. On the third screen, which is shown on the right, users to finalise the reply, before sending it.
    }
    \label{fig:BA_prototype}
\end{figure*}

\subsection{Design Goals and Rationale} 
With insights from the literature (\cref{sec:related_work}), we designed our system with several goals in mind. For each goal, we briefly mention our approach here, with more details on its final realisation in \cref{sec:implementation}.
\begin{enumerate}[leftmargin=*]
    \item \textbf{Human decides, AI supports:}
    \label{dg:humandecides}
    The user should be able to make all important decisions, while AI supports these. 
    In our design, the user selects the sentences they want to reply to. 
    The system then suggests response sentences, designed to offer a mix of positive, neutral, and negative responses. 
    \item \textbf{The user stays in control:}
    \label{dg:control}
    The user should stay in control of their reply. 
    The AI should not make unnoticed or unwanted adjustments. 
    Our system does not change text unless requested and the user can always edit the reply before sending it.
    \item \textbf{Support mobile microtasking:}
    \label{dg:microtasking}
    The user should be able to leverage microtasking principles for mobile email replies. %
    Our design provides the surrounding email as context while entering reply text and thus shifts from recall to recognition by eliminating the need to remember the email or scroll back to it.%
    \item \textbf{Support diverse workflows -- with and without AI:}
    \label{dg:workflows}
    In each situation, users should be able to answer in their preferred way.
    Our skippable components offer flexibility.
    Even without AI, users get supportive microtasking structure.
    Conversely, users can choose to rely on AI text to respond fast, with little typing. %
\end{enumerate}





\subsection{First Prototype}
We implemented the concept as an Android app with React Native.\footnote{\url{https://reactnative.dev/}}
At this point, the prototype had our sentence-based mode as shown in \cref{fig:BA_prototype}, with three screens.
The first screen (\cref{fig:BA_prototype} left) showed the email for users to select sentences via touch. Each selection triggered a card view that displayed AI suggestions and a text box for entering text (as a manual response or as a prompt to refine the suggestions).
The LLM always generated two positive, two negative, and two neutral answering options.
One positive and one negative suggestion were shown on the first page, if possible. Users could click on the arrows on the sides to access the others.
A second screen (\cref{fig:BA_prototype} centre) supported manual editing of paragraphs, inserting new ones and/or reordering them via drag and drop.
Finally, a third screen showed the result in a standard text editor view for a last check and final edits, if necessary (\cref{fig:BA_prototype} right). %


\subsection{Formative Study}\label{sec:formative_study}
We conducted a first study to understand how users perceive and interact with our concept, and to inform a design iteration. We recruited 17 participants (1 female, 15 male, 1 preferred not to disclose) from our university network. The study followed our institute's regulations, including information on goals, process, data recording, opt-out and consent.

Participants used our prototype on their own phones. They received a tutorial beforehand. The app did not integrate with actual email accounts to preserve privacy. Instead, it simulated to receive two emails per day, for five days. \revision{These emails were quite long, ranging from 140 to 491 words per email (median: 227), to allow participants to test the prototype extensively.} People were asked to respond to the emails in a reasonable time frame. 

After each reply, the app displayed three 5-point Likert items (``The AI tool was helpful'', ``The AI tool helped me reply to the email faster'', ``The AI tool helped me write a better reply'') and space for open feedback. Following the final email, participants completed a questionnaire about their overall experience and demographics.


\subsection{Results}\label{sec:formative_study_results}
The median time of engaging with each email task in our study was 6.9 minutes, including the time taken to enter feedback. 
People accepted 9.17 suggestions per email.
Nearly \pct{80} of accepted suggestions were accepted without making use of the text input for refinement. %
In \pct{30} of emails, participants composed the email entirely with suggestions without edits afterwards.
When they indeed made edits, the most common ones we identified through manual coding were the following: On the first screen, they removed text (25 instances), added information (11), changed details (8), shortened text (6), and added salutations (5) or closing statements (5). 
On the second screen, they reordered or merged sentences and paragraphs (35), shortened text (10), and changed minor details (8). 
They made similar edits on the third screen. %





The Likert results (\cref{fig:likert_items_formative_study} \revision{in \cref{sec:appendix_extra_figures}}) indicated that participants found the AI to be helpful and that it supported them in writing the replies. They felt in control of the email content and found the suggestions to make sense and not be distracting. They generally agreed that the approach helped them remember to address all parts of an email. However, they were more divided on whether they overall preferred the step-by-step process or the traditional one-step approach for replying to emails.

Open in-app feedback was provided by 14 out of 17 participants. Positive aspects mentioned there and in the final questionnaire included ease of use, faster replies, the quality and inspirational potential of AI suggestions, and an overall improved workflow. 

Negative aspects included slow AI response times, quality of suggestions (e.g. too short or not aligning with their input), minor bugs (e.g. failure to load suggestions), and the number of steps (e.g. some suggested to merge the last two screens into one).  

The final questionnaire asked people to reflect on their workflow with our app. They reported different strategies, such as generating custom replies with keywords, reading the entire email before replying, or reviewing generated suggestions first. Some manually merged or adjusted AI-generated text, while others used it as is. 
The final questionnaire also included the System Usability Scale (SUS)~\cite{brooke1996sus}. The mean score was 78.67 (``very good'', \revision{details in \cref{fig:sus_items_formative_study} in \cref{sec:appendix_extra_figures}}).












\subsection{Prototype Iteration}
\label{ssec:protiter}
In summary, the findings from this first study indicated that participants appreciated our concept as it helped them to write fast and high-quality replies with AI, while still feeling in control. It also revealed individual approaches when answering emails and interacting with the suggestions. Based on the study insights, we made the following concrete changes to our prototype:

\begin{itemize}[leftmargin=*]
    \itemsep.2em
    \item \textit{Reduced number of steps:} We removed the second screen (\cref{fig:BA_prototype} centre) %
    and direly offered the third one for free text editing and finalising. Some participants suggested this and they overall made very similar edits across these two screens.
    \item \textit{Added optional improvement pass:} %
    \revision{We added an ``improve email'' button to the final screen to better support users' varying strategies and preferences for answering emails with AI. We observed that some participants manually edited their emails to create transitions between individual paragraphs generated on the first screen. The \imppass{} feature automates this process, adding missing greetings, sign-offs, and correcting grammar and spelling (see \cref{subsec:appendix_improve_email_prompt} for the used prompt)}. %
    \item \textit{Faster suggestion generation:} We switched from GPT-3.5 Turbo\footnote{\url{https://platform.openai.com/docs/models}} to Llama 3 8B Instruct\footnote{\url{https://huggingface.co/meta-llama/Meta-Llama-3-8B-Instruct}}\cite{llama3modelcard}, which we hosted locally to reduce latency and avoid request failures. \revision{While we did not conduct an in-depth evaluation of the models, we compared a set of generations qualitatively and found that the output quality was similar for our use case. Related, we envision that real-world applications could rely on smaller models that can be executed on devices locally to avoid the need to send private email content to a model provider.}%
    \item \textit{Refined suggestions:} We refined our prompting templates to improve suggestions, even with the smaller model. Our new prompts included more context, i.e. all sentence-level replies that were already given up to this point.  
    \item \textit{\revision{Port to a React web app}:} \revision{We ported the app from React Native to a React web app and optimised it. This eased access for participants, as running a React Native prototype required several steps for setup. Instead, a React web app can be accessed via a web browser on any smartphone.}%
\end{itemize}

In the rest of the paper, we always refer to the improved prototype. We next describe it in detail.


\begin{figure*}[t]
    \centering
    \includegraphics[width=\linewidth]{figures/local_response_prompting}
    \caption{The text suggestions in the local response widget are flexible:  \textit{(A)} Users get suggestions without any input.  \textit{(B)} \revision{Suggestions} can be adapted and refined by entering text, for example keywords or a draft snippet. In all cases, suggestions are generated with an LLM based on the text of the incoming email and all local responses that the user has entered so far, \revision{even if responses have been} added to later parts of the email \revision{first}. \revision{In \textit{(C)}, for example, the suggested title of the idea pitch is generated based on the information about the project that the user has already entered in local responses below.} Note that suggestions are paginated, with three pages of two suggestions each.}
    \Description{
    This figure shows screenshots of our user interface displaying an incoming email. The figure is divided into three sections:
    A) Left Section: Email-driven Suggestions without User Input
    The selected sentence in the incoming email reads: "Please feel free to tell me any ideas what we could get her!"
    Below the email, there is an empty text field for optional user input.
    Two AI-generated responses are shown beneath the text field: one suggests a piece of jewellery as a gift, while the other does not offer any ideas.
    B) Centre Section: Prompt-driven Suggestions with User Input
    The same sentence from the email is selected: "Please feel free to tell me any ideas what we could get her!"
    This time, the keywords "balloon ride" are entered into the text field below.
    As a result, the AI-generated suggestions include the idea of a balloon ride in both proposed texts.
    C) Right Section: AI Suggestions Respecting Existing Responses
    One sentence from the incoming email is selected, and the text field below remains empty.
    The AI-generated responses incorporate information from existing responses elsewhere in the email.
    Additionally, all AI suggestions are paginated, with three pages of two suggestions each, as indicated by arrows next to the suggestions.}
    \label{fig:local_response_prompting}
\end{figure*}



\section{Implementation}
\label{sec:implementation}
We implemented a frontend and backend, which preprocessed emails, logged user data, and generated responses. 

\subsection{Frontend}
We implemented our web app with the React\footnote{\url{https://react.dev/}} framework.

\subsubsection{Display of the Incoming Email}
This view matches standard mobile email UIs: It includes the sender's name and picture, the email subject, and the main text body (\cref{fig:teaser} left). 
The user can select sentences in the incoming email by tapping on them (cf. design goal \ref{dg:humandecides}: \revision{Human decides, AI supports}; and goal \ref{dg:microtasking}: \revision{Support mobile microtasking}). This opens the local response widget (\cref{sec:impl_local_response_widget}).
The ``Finalize Reply'' button at the bottom of the UI switches to the next screen (\cref{fig:teaser} centre), which we describe in \cref{sec:impl_finalize}.
In accordance with design goal \ref{dg:workflows} \revision{(Support diverse workflows -- with and without AI)}, no interaction with any sentence or AI feature is required before proceeding to this next screen.



\subsubsection{Local Response Widget}\label{sec:impl_local_response_widget}

This UI widget is inserted into the email text below the user's selected sentence. It comprises of a text field (\cref{fig:teaser} C) and a paginated card view that shows text suggestions (\cref{fig:teaser} D). In the text field, users can enter both manual responses or prompts to refine these suggestions (\cref{fig:local_response_prompting}). 

Concretely, the widget offers six suggestions (2 positive, 2 negative, 2 neutral), showing two at once. The system aims to show one positive and one negative response on the first page, if possible. \revision{This was motivated by findings on positivity bias in AI-generated communication text~\cite{Mieczkowski2021} and to increase the chance of offering a response option fitting to the user's intent (cf.~\cite{Kannan2016smartreply}).} \revision{We realised this by prompting the LLM to do so (see \cref{sec:appendix_sentence_without_input_prompt}). Concretely, the variable ``attribute'' in the prompt template was replaced with \textit{accepting}, \textit{declining}, and \textit{neutral} to generate varying suggestions. In our tests, we observed that this simple prompting approach worked well and that it did not negatively impact generated suggestions in cases where these attributes do not apply (e.g. our ``cat'' example in \cref{fig:teaser}D).} %
Users can navigate through suggestions using the adjacent arrow buttons. They can accept a suggestion by tapping on it.

The widget has two states -- open and collapsed (\cref{fig:teaser} A, B): 
It is collapsed by tapping the currently selected sentence again, by selecting a different sentence, by accepting a suggestion, or by clicking on the check mark in the top right corner. When the text field is empty, the check mark transforms into a trash icon to delete the local reply. Multiple widgets can be in the collapsed state throughout the email but only one widget at a time can be open and in focus. %
A widget's text is shown in the collapsed state. This allows users to keep track of all their local replies so far. Tapping on a collapsed widget opens it again for further editing. 






\subsubsection{Finalising the Reply}\label{sec:impl_finalize}
This view  (\cref{fig:teaser} centre) shows the current state of the reply after the local response step. That is, it displays any text entered in response to individually selected sentences together in a single text field.

Users can manually adjust this text and/or tap the ``Improve'' button to request the AI to enhance the email. 
This \imppass{} feature is realised with a prompt \revision{(see \cref{subsec:appendix_improve_email_prompt})} to the underlying LLM to correct spelling and grammar, refine wording, and add missing salutations or regards while adhering to both the incoming email's content and the existing reply text. 

If no text is entered first, the ``Improve'' button acts as message-level support, generating a reply based on the incoming email's text and the current input on this screen. For example, a user could skip the local response and enter a prompt here, effectively realising a message generation workflow similar to the industry default (\cref{sec:related_work_current_products}). This flexibility contributes to our design goal \ref{dg:workflows} \revision{(Support diverse workflows – with and without AI).}

When the user is satisfied with their reply, the email can be sent by tapping the ``Send Email'' button at the bottom of this screen.


\subsubsection{Improved Email Pop-up}\label{sec:impl_imppass}
The \imppass{} feature does not change the user's text directly, in line with our design goal \ref{dg:control} \revision{(The user stays in control).}
Instead, the new text is shown in a pop-up view with formatting familiar from ``track changes'' in text editors (\cref{fig:teaser} right). 
Users can approve these changes, which updates the text, or discard them (cf. design goal \ref{dg:humandecides}: \revision{Human decides, AI supports}.) Further editing after acceptance and/or requesting improvements repeatedly is possible. 


\subsection{Backend}
Our prototype's backend has three purposes: 
(1) It \textit{hosts the web app} on a Next.js server. 
(2) It provides \textit{email preprocessing}, which handles tasks such as sentence-splitting and making API calls to the LLM. 
(3) It \textit{hosts the LLM}. 

We experimented with several models and APIs and discussed factors such as latency, stability of service, and subjective response quality in meetings with all authors. Based on this, we used the Llama 3 8B Instruct \cite{llama3modelcard} model for the main study. 

Similarly, we iterated over several prompting approaches for the text generation features. Overall, this resulted in a few-shot approach, providing the model with several input-output examples to generate fitting responses. 
As an overview, we use the following prompt templates (details in \cref{sec:appendix_prompts}):

\paragraph{Sentence-level support, without user input:}
We prompted six suggestions for the sentence selected in the email (2 positive, 2 neutral, 2 negative). This balanced the options, following related work~\cite{Kannan2016smartreply}, as the LLM favoured positive responses in our tests.

\paragraph{Sentence-level support, with user input:} 
This was identical to the above case but now it included the user's input in addition to the selected sentence. %
We emphasised alignment with the user's sentiment (e.g. no negative suggestions if the user had entered ``yes'').


\paragraph{Message-level support:}
We prompted the LLM to answer to the whole email, also by taking into account any current user input, if available. A variation of this was also used for the \imppass{} feature (\cref{sec:impl_imppass}). That prompt emphasised improving the current state of the reply while closely adhering to the information provided by the user. %

\section{Method}
\revision{For the main study, we switched from the field study design of the formative study to a more controlled web-based experiment. Our motivation was to scale the study, to compare interaction designs quantitatively, and to introduce task briefings that would allow us to evaluate if people accept even unfitting suggestions for convenience in the study. Thus, we} conducted a within-subject user study with our \revision{iterated} prototype.
The independent variable \ivmode{} had three levels: \modeoursTxt{} (\modeours) -- our proposed design (\cref{sec:implementation}); \modemailtxt{} (\modemail) -- a one-prompt reply generation design close to currently available UIs (\cref{sec:related_work_current_products}); and writing without any AI features (\modemanual). As dependent variables, we logged interaction metrics and collected subjective feedback via questionnaires.

\subsection{Apparatus}

\subsubsection{Web App}
For \modeours, we hosted our prototype (\cref{sec:implementation}) as a web app, with added study information, study logic, and logging.
We integrated a screen for briefings (\cref{sec:method_emails_briefings}) before each email task and one with four Likert items (\cref{sec:procedure_email_tasks}) after each email task (\cref{fig:briefing_and_feedback} \revision{in \cref{sec:appendix_extra_figures}}).
We used a custom study framework to manage counterbalancing and the flow from consent information to prototypes to surveys.

\subsubsection{Comparative Designs}
We implemented two alternatives for the study: For \modemanual, the app only showed a typical drafting view (\cref{fig:baseline_uis_manual} \revision{in \cref{sec:appendix_extra_figures}}). For \modemail, it was designed similar to the typical UI pattern shown in \cref{fig:current_products} -- it offered a text field for (optional) prompting and displayed the generated text (\cref{fig:baseline_uis_msg} \revision{in \cref{sec:appendix_extra_figures}}). Accepting the text with a button inserted it into a draft view for further editing or sending. Rejecting it allowed users to refine their prompt and generate a new draft.

\subsubsection{Incoming Emails and Reply Briefings}\label{sec:method_emails_briefings}
We prepared nine emails, covering an idea pitch contest, a high school reunion, a sales offer, a lunch meeting, a marketing slogan, proofreading for a friend, a sales report deadline, server access, and a gift idea for a retiring coworker. This set was motivated to cover various plausible email topics, with and without (multiple) questions. It also covered various emails lengths, \revision{ranging from 24 to 155 words (median 57).}%

We also prepared a reply briefing for each email. It provided information relevant for \textit{what} to answer, without specifying \textit{how} to write (e.g. tone, structure). For example, for the high school reunion email, the briefing specified that the user was unavailable on a certain date.
These briefings were \textit{not} given to the LLM, which would have simulated unrealistic ``mindreading''. In contrast, our motivation for the briefings was to assess to what extent participants might accept unfitting suggestions for convenience. %
Moreover, the briefings mimic an email workflow where some information is readily available while details may need to be retrieved. For instance, in the high school example, reading the briefing could be seen as similar to checking a calendar app.


\subsection{Participants}
\label{sec:participants}
We recruited 162 participants through the online platform Prolific.\footnote{\url{https://www.prolific.com/}} We excluded 36 participants from our analysis because they either did not complete all tasks (18 participants) or the logs indicated that a technical issue had occurred (18 participants). Our analyses are based on the remaining \studyTwoN{} participants (83 male, 40 female, 1 non-binary, 2 preferred not to disclose). 
Their age ranged from 18 to 72 years (median 32). %
All were proficient in English (91.3\% native speakers).
Their occupations included both professions where frequent email usage is expected (e.g. IT consultant, project manager) and others (e.g. gardener, waiter).
Participants were compensated with about \pounds 10 per hour.

Most participants reported to answer emails at least once a day (\pct{48.41}) or even more than 10 times a day (\pct{21.43}).
Another \pct{16.67} answer emails at least once a week, \pct{7.94} less than once a week, and \pct{5.56} less than once a month.

Most participants (\pct{85.71}) use their smartphone for answering emails. Many also use a laptop (\pct{82.54}) or desktop computer (\pct{53.97}). Some also use a tablet (\pct{19.05}).
They answer emails at home (\pct{84.92}), at the office (\pct{69.84}), and on the go (\pct{53.17}).
Many answer emails for business (\pct{84.92}) and in a private context (\pct{58.73}).

Only \pct{14.29} reported no previous experience with AI.
Many have used ChatGPT (\pct{72.22}). Many have experience with auto-correction (\pct{51.59}), some also with auto-completion (\pct{26.98}), with word or sentence suggestions (\pct{21.42}), and with Smart Reply (\pct{15.87}).
\revision{An overview of all questions and answer options can be found in \cref{sec:appendix_questionnaires}}.


\subsection{Procedure}\label{sec:procedure}
The study was conducted remotely on participants' own smartphones. Access via Prolific was restricted to one person at a time to balance the load for our server.
The sessions were scheduled for 45 minutes and structured as follows:

\subsubsection{Study Intro}
An introduction page explained the study, including information about GDPR compliance, privacy, data collection, and informed consent, in accordance with our institutional regulations. 
In addition, whenever encountering a UI for the first time, our study framework showed an explanation of its features.

\subsubsection{Email Answering Task}\label{sec:procedure_email_tasks}
Participants were asked to reply to nine given emails, three per \ivmode{} (\modeours, \modemail, \modemanual).
We counterbalanced the email topics and also the order of the UIs with a Latin square design to address potential learning or fatigue effects.

For each email task, the briefing (\cref{sec:method_emails_briefings}) was shown at the start and could be accessed again anytime via the information button in the top right corner of the app (\cref{fig:teaser}).
Participants were instructed to ``consider the information in the briefing for answering the email[s]''.


After submitting each email, participants rated four Likert items: ``The app interface was helpful'', ``The app interface helped me reply to the email quickly'', ``The app interface helped me write a good reply'', and ``I was in control of the content of my reply''. 
They could share comments in a text field.


\subsubsection{Final Questionnaire}
This questionnaire was displayed after the final email task.
Participants provided demographics, selected their favourite UI mode, and explained their choice.
They could also leave comments, questions, and feedback. %


\subsection{Qualitative Analysis}
Here we describe our approach to coding open feedback and analysing email texts.

\subsubsection{Coding of Open Feedback}
We followed Grounded Theory~\cite{corbin1990basics} to analyse participants' open feedback. 
In the open coding round, two researchers independently reviewed the data, identifying and labelling sentences that represent specific ideas and principles. 
We then refined these initial codes by merging and clustering related ones, forming (sub-)categories during an axial coding round. 
Our research team discussed emerging themes, leading to synthesised, overarching labels for the clusters and, in some cases, further split categories to capture more nuanced insights from the feedback.
We repeated this process until reaching consensus.

\subsubsection{Analysis of the Email Replies}
\label{sec:quality_m}
Assessing email quality is complex and subjective, as known from studies on people's preferences (cf. \cite{Liu2022aimailperception, Robertson2021cantreply}).
Therefore, we use multiple  quality indicators: 
Formal indicators~\cite{reeves2008emailover50, lewi_jones2014email} include the presence of a \textit{salutation} and a \textit{closing statement} (\cref{sec:results_structure}), and proper \textit{spelling and grammar} (\cref{sec:results_errors}).
We also analysed \textit{briefing conformity}, that is, we checked whether replies covered the key information provided in the briefings (\cref{sec:results_briefing}).
Finally, we share our subjective impressions (\cref{sec:result_quality}).

We employed Binary Coding~\cite{miles2013qualitative} to assess the briefing conformity as well as the formal indicators, except spelling and grammar, which we checked using the language-tool-python\footnote{\url{https://pypi.org/project/language-tool-python/}} library. 
Two researchers coded all email replies independently, assigning a ``1'' if the email met the criteria and a ``0'' if it did not. 
Ambiguous emails were flagged for further review. 
In a second round, these were re-evaluated by the research team until reaching consensus.


\subsection{\revision{Statistical Analysis}}\label{sec:appendix_sigtest}


\revision{To declutter our following report,} \cref{tab:lmm_overview} and \cref{tab:lmm_overview2} summarise the statistical analyses \revision{and results} referred to throughout \cref{sec:results}.
We computed (generalised) linear mixed-effects models (LMMs) in R~\cite{R2020}, using the packages \textit{lme4}~\cite{Bates2015} and \textit{lmerTest}~\cite{Kuznetsova2017}. These models accounted for the individual differences between participants and for differences between the incoming emails via random intercepts. 

The models' fixed effects were \ivmode{} and whether the \imppass{} feature was used in \modeours. 
For the model for briefing conformity, we additionally included a predictor for whether the reply was generated without user input, such that the result was generated fully by the LLM based on the incoming email only. In \modemail, this is done by not entering a prompt for the reply generation. In \modeours, this is done by not providing any local response (manual or suggestion) on screen 1, before using the \imppass{} feature on screen 2. 
Pairwise comparisons were computed with the \textit{emmeans} package with Bonferroni-Holm correction.


For the Likert data, we used rank-aligned repeated measures ANOVA  (ART)~\cite{wobbrock2011art} and ART-C contrasts with Bonferroni-Holm correction for the follow-up analysis~\cite{elkin2021artc}.

We report significance at p~<~0.05. 


\begin{table*}[t!]
\centering
\footnotesize
\newcolumntype{L}{>{\raggedright\arraybackslash}X}
\newcolumntype{P}[1]{>{\raggedright\arraybackslash}p{#1}}
\renewcommand{\arraystretch}{1.4}
\setlength{\tabcolsep}{4pt}
\begin{tabularx}{\linewidth}{lP{2.75em}P{5em}P{22em}P{7em}L}
\toprule
    &
    \textbf{Section} &
    \textbf{Aspect}\newline and model &
    \textbf{Predictors} (baseline: \modemanual) &
    \textbf{Pairwise comparisons} &
    \textbf{Takeaways in words}\newline(only considering sig. results) \\ \midrule
1 &
    \ref{sec:results_time} 
    &
    Completion time\medskip\newline
    \textit{LMM on seconds}
    &
    \modeours{}	$\downarrow$ \newline 
    \deemph{(\lmmci{-4.34}{12.00}{-27.89}{19.21}{=.718})}\medskip\newline 
    \modemail{} $\downarrow^*$ \newline 
    \deemph{(\lmmci{-70.05}{7.72}{-85.20}{-54.90}{<.0001})}\medskip\newline 
    \Imppass{} feature used $\downarrow$ \newline 
    \deemph{(\lmmci{-15.18}{13.23}{-41.14}{10.79}{=.252})}
    &
    \modeours{} vs \modemanual{} \deemph{(\posthoc{-4.34}{=.718})} \medskip\newline 
    \modemail{} vs \modemanual{} \deemph{(\posthoc{-70.05}{<.0001})} \medskip\newline 
    \modeours{} vs \modemail{} \deemph{(\posthoc{65.71}{<.0001})}
    &
    People finished replying faster with \modemailtxt{} than without AI (by 70 seconds on average). \modemailTxt{} was also faster than \modeourstxt{} (by 66 seconds on average).
    \\
    \midrule 
2 &
    \ref{sec:results_speed}
    &
    Writing speed\medskip\newline
    \textit{LMM on characters per second}
    &
    \modeours{} $\uparrow$ \newline 
    \deemph{(\lmmci{.61}{.56}{-.49}{1.71}{=.278})}\medskip\newline 
    \modemail{} $\uparrow^*$ \newline 
    \deemph{(\lmmci{5.16}{.37}{4.43}{5.88}{<.0001})}\medskip\newline 
    \Imppass{} feature used $\uparrow^*$ \newline 
    \deemph{(\lmmci{2.48}{.61}{1.29}{3.68}{<.0001})}
    &
    \modeours{} vs \modemanual{} \deemph{(\posthoc{.61}{=.278})} \medskip\newline 
    \modemail{} vs \modemanual{} \deemph{(\posthoc{5.16}{<.0001})} \medskip\newline 
    \modeours{} vs \modemail{} \deemph{(\posthoc{-4.55}{<.0001})}
    &
    People produced more characters per second with \modemailtxt{} (5.2 chars more per s) and if they used the \imppass{} feature in \modeourstxt{} (2.5 chars more per s).
    \\
    \midrule
3 &
    \ref{sec:results_keystrokes}
    &
    Manual typing\medskip\newline
    \textit{GLMM (Poisson) on keystroke counts}
    &
    \modeours{}	$\downarrow^*$ \newline 
    \deemph{(\lmmci{-.65}{.009}{-.66}{-.63}{<.0001})}\medskip\newline
    \modemail{}	$\downarrow^*$ \newline 
    \deemph{(\lmmci{-.86}{.005}{-.87}{-.85}{<.0001})}\medskip\newline 
    \Imppass{} feature used $\downarrow^*$ \newline 
    \deemph{(\lmmci{-.06}{.011}{-.08}{-.04}{<.0001})}
    &
    \modeours{} vs \modemanual{} \deemph{(\posthoc{-.65}{<.0001})} \medskip\newline 
    \modemail{} vs \modemanual{} \deemph{(\posthoc{-.86}{<.0001})} \medskip\newline 
    \modeours{} vs \modemail{} \deemph{(\posthoc{.21}{<.0001})}
    &
    People needed fewer keystrokes with AI than without it; concretely, even fewer with \modemailtxt{} (\pct{58} decrease) than with \modeourstxt{} (\pct{48} decrease). Using the \imppass{} feature in \modeourstxt{} reduced them further for that UI (\pct{5.9} decrease).
    \\
    \midrule 
4 &
    \ref{sec:results_lengths}
    &
    Reply lengths\medskip\newline
    \textit{GLMM (Poisson) on character counts}
    &
    \modeours{} $\uparrow^*$ \newline 
    \deemph{(\lmmci{.24}{.0060}{.22}{.25}{<.0001})}\medskip\newline
    \modemail{} $\uparrow^*$ \newline 
    \deemph{(\lmmci{.57}{.0037}{.56}{.58}{<.001})}\medskip\newline 
    \Imppass{} feature used $\uparrow^*$ \newline 
    \deemph{(\lmmci{.32}{.0063}{.31}{.33}{<.0001})}
    &
    \modeours{} vs \modemanual{} \deemph{(\posthoc{.24}{<.0001})} \medskip\newline 
    \modemail{} vs \modemanual{} \deemph{(\posthoc{.57}{<.0001})} \medskip\newline 
    \modeours{} vs \modemail{} \deemph{(\posthoc{-.33}{<.0001})}
    &
    People wrote longer replies with AI, even more so with \modemailtxt{} (exp($\beta$)=exp(.57)=1.77 i.e. \pct{77} increase) than with \modeourstxt{} (\pct{27} increase). Using the \imppass{} feature in \modeourstxt{} increased it further for that UI (\pct{38} increase).
    \\
    \midrule 
5 &
    \ref{sec:results_errors}
    &
    Error rates\medskip\newline
    \textit{LMM on errors per character}
    &
    \modeours{}	$\downarrow^*$ \newline 
    \deemph{(\lmmci{-.0011}{.0004}{-.0018}{-.0004}{=.0024})}\medskip\newline 
    \modemail{}	$\downarrow^*$ \newline 
    \deemph{(\lmmci{-.0020}{.0002}{-.0025}{.0015}{<.0001})}\medskip\newline 
    \Imppass{} feature used $\downarrow^*$ \newline 
    \deemph{(\lmmci{-.0012}{.0004}{-.0020}{-.0005}{=.0012})}
    &
    \modeours{} vs \modemanual{} \deemph{(\posthoc{-.0011}{=.0048})} \medskip\newline 
    \modemail{} vs \modemanual{} \deemph{(\posthoc{.-0020}{<.0001})} \medskip\newline 
    \modeours{} vs \modemail{} \deemph{(\posthoc{.0009}{=.0096})}
    &
    People wrote emails with lower error rates with AI than without, even lower with \modemailtxt{} than with \modeourstxt. Using the \imppass{} feature in \modeourstxt{} reduced the error rates further for that UI.
    \\
    \midrule
6 &
    \ref{sec:results_email_similarity}
    &
    Email similarity\medskip\newline
    \textit{LMM on cosine similarity of SBERT embeddings}
    &
    \modeours{} $\uparrow^*$ \newline 
    \deemph{(\lmmci{.09}{.0026}{.09}{.10}{<.0001})}\medskip\newline 
    \modemail{} $\uparrow^*$ \newline 
    \deemph{(\lmmci{.17}{.0023}{.17}{.18}{<.0001})}\medskip\newline 
    \Imppass{} feature used $\uparrow^*$ \newline
    \deemph{(\lmmci{.07}{.0029}{.07}{.08}{<.0001})}
    &
    \modeours{} vs \modemanual{} \deemph{(\posthoc{.09}{<.0001})} \medskip\newline 
    \modemail{} vs \modemanual{} \deemph{(\posthoc{.17}{<.0001})} \medskip\newline 
    \modeours{} vs \modemail{} \deemph{(\posthoc{-.08}{<.0001})}
    &
    People wrote semantically more similar (i.e. less diverse) emails with AI than without, more so with \modemailtxt{} than with \modeourstxt{}. For the latter, using the \imppass{} feature contributed to increasing the similarity of emails.
    \\
    \midrule
7 &
    \ref{sec:results_lexical_diversity}
    &
    Lexical diversity\medskip\newline
    \textit{LMM on the distinct2 metric}
    &
    \modeours{} $\downarrow^*$ \newline 
    \deemph{(\lmmci{-.03}{.0043}{-.04}{-.03}{<.0001})}\medskip\newline 
    \modemail{} $\downarrow^*$ \newline 
    \deemph{(\lmmci{-.02}{.0029}{-.03}{-.01}{<.0001})}\medskip\newline 
    \Imppass{} feature used $\uparrow^*$ \newline
    \deemph{(\lmmci{.01}{.0045}{.003}{.02}{=.0087})}
    &
    \modeours{} vs \modemanual{} \deemph{(\posthoc{-.03}{<.0001})} \medskip\newline 
    \modemail{} vs \modemanual{} \deemph{(\posthoc{-.02}{<.0001})} \medskip\newline 
    \modeours{} vs \modemail{} \deemph{(\posthoc{-.015}{=.0011})}
    &
    People wrote emails with lower lexical diversity (measured as: unique bigrams / number of words) with AI than without it, even more so with \modeourstxt{} than with \modemailtxt{}. Using the \imppass{} feature in \modeourstxt{} closed this gap.
    \\
  \bottomrule
\end{tabularx}
\caption{Overview of significance tests with links to the section, tested measure, predictors, pairwise comparisons, and written interpretation. The arrows indicate if predictors increase ($\uparrow$) or decrease ($\downarrow$) the outcome aspect, with an asterix if these impacts are significant (*).}
\Description{Overview of significance tests with links to the section, tested measure, predictors, pairwise comparisons, and written interpretation. For each statistical test it describes the Section, Aspect and model, Predictors (baseline: NoAI), Pairwise comparisons, and Takeaways in words (only considering sig. results).}
\label{tab:lmm_overview}
\end{table*}




\begin{table*}[t!]
\centering
\footnotesize
\newcolumntype{L}{>{\raggedright\arraybackslash}X}
\newcolumntype{P}[1]{>{\raggedright\arraybackslash}p{#1}}
\renewcommand{\arraystretch}{1.4}
\setlength{\tabcolsep}{4pt}
\begin{tabularx}{\linewidth}{lP{2.75em}P{5em}P{22em}P{7em}L}
\toprule
    &
    \textbf{Section} &
    \textbf{Aspect}\newline and model &
    \textbf{Predictors} (baseline: \modemanual) &
    \textbf{Pairwise comparisons} &
    \textbf{Takeaways in words}\newline(only considering sig. results) \\ \midrule
1 &
    \ref{sec:results_briefing}
    &
    Briefing conformity\medskip\newline
    \textit{GLMM (Binomial) on binary conformity coding}
    &
    \modeours{} $\downarrow^*$ \newline 
    \deemph{(\lmmci{-.77}{.3107}{-1.38}{-.16}{=.013})}\medskip\newline 
    \modemail{} $\downarrow$ \newline 
    \deemph{(\lmmci{-.04}{.2417}{-.52}{.43}{=.857})}\medskip\newline
    \Imppass{} feature used $\uparrow$ \newline
    \deemph{(\lmmci{.25}{.3310}{-.40}{.90}{=.444})}\medskip\newline
    Full reply generated without input $\downarrow^*$ \newline
    \deemph{(\lmmci{-1.70}{.3467}{-2.38}{-1.02}{<.0001})}
    &
    \modeours{} vs \modemanual{} \deemph{(\posthoc{-.77}{=.040})} \medskip\newline 
    \modemail{} vs \modemanual{} \deemph{(\posthoc{-.04}{=.857})} \medskip\newline 
    \modeours{} vs \modemail{} \deemph{(\posthoc{-.73}{=.040})}
    &
    With \modeours, people wrote emails that had a higher chance to miss a key aspect of the study briefing than those written with \modemail{} or manually (\pct{23} of emails missed it for \modeours{} vs \pct{18} for \modemail{} vs \pct{13} for \modemanual). People's prompting behaviour had a larger impact here: Across \modeours{} and \modemail{}, generating a full reply without any own input (83 emails in the data) missed a key aspect of the briefing in half of the cases (\pct{49}).
    \\
    \midrule
2 &
    \ref{sec:results_workflows}
    &
    Skipping local response \medskip\newline
    \textit{GLMM (Binomial) on skipped yes/no}
    &
    Length of incoming email\newline
    (num. standardised words, i.e. characters/5) $\downarrow^*$ \newline
    \deemph{(\lmmci{-.025}{.010}{-.050}{-.004}{=.0171})}
    &
    -
    &
    Each additional word (defined as 5 additional characters) in the incoming email is associated with a \pct{2.46} decreased chance of skipping the local response step in \modeours{}. Skipping is defined as not entering any text on the local response screen of that UI.
    \\
  \bottomrule
\end{tabularx}
\caption{Further significance tests with links to the section, tested measure, predictors, pairwise comparisons, and written interpretation. The arrows indicate if predictors increase ($\uparrow$) or decrease ($\downarrow$) the outcome aspect, with an asterix if these impacts are significant (*).}
\Description{Further significance tests with links to the section, tested measure, predictors, pairwise comparisons, and written interpretation. For each statistical test it describes the Section, Aspect and model, Predictors (baseline: NoAI), Pairwise comparisons, and Takeaways in words (only considering sig. results).}
\label{tab:lmm_overview2}
\end{table*}

\section{Results}\label{sec:results}


\subsection{Analysis of Interaction Logs}\label{sec:results_interaction_logs}
We report our analyses of the interaction data. 

\begin{figure*}[t]
    \centering
    \includegraphics[width=\linewidth]{figures/interaction_logs_boxplots.pdf}
    \caption{Three measures of interaction behaviour: Task completion time \textit{(left)}, manual typing \textit{(centre)}, and writing speed \textit{(right)}. All AI features increased typing speed and reduced the time taken (both sig. for \modemail). They also reduced the number of keystrokes (sig. for \modeours{} and \modemail). If people made use of the optional \imppass{} feature \revision{(impr.)} in \modeours, this contributed to narrowing the gap between the otherwise sentence-level design of \modeours{} and the message-level design of \modemail{} (sig. for manual typing and writing speed). See \cref{sec:results_interaction_logs} for details.}
    \Description{This figure presents three box plots that measure interaction behaviour from the study across three different user interfaces: NoAI (manual mode), CDLR (AI-supported with and without the optional improvement pass feature), and MSG (AI-supported). The three metrics being compared are:
    Task Completion Time (Left):
    This plot shows how long it took participants to complete the task in minutes.
    Both CDLR (with and without improvement) and MSG show reduced task completion times compared to NoAI, with MSG showing the most significant reduction.
    Manual Typing (Center):
    This plot shows the number of keystrokes made by participants during the task.
    Both CDLR and MSG significantly reduced the number of keystrokes compared to NoAI, particularly when participants used the optional improvement pass feature in CDLR.
    Writing Speed (Right):
    This plot shows the writing speed of participants measured in characters per second.
    Both CDLR and MSG increased writing speed compared to NoAI, with MSG again showing the most notable improvement.
    Overall, the figure demonstrates that AI-supported interfaces (CDLR and MSG) led to faster task completion, fewer keystrokes, and increased writing speed. In particular, the optional improvement pass feature in CDLR helped narrow the performance gap between the sentence-level design of CDLR and the message-level design of MSG. For detailed statistical analysis, see the corresponding section of the paper (Section 6.1).}
    \label{fig:interaction_logs_boxplots}
\end{figure*}


\subsubsection{Task Completion Time}\label{sec:results_time}

On average, participants took \mins{3.20} (SD 2.51, median 2.46) to write an email manually.
With \modemail, this decreased to \mins{2.03} (SD 1.90, median 1.44), while \modeours{} reduced it to \mins{2.95} (SD 2.49, median 2.32). For \modeours, using the \imppass{} feature resulted in \mins{2.90} (SD 2.58, median 2.26), while not using it had \mins{3.06} (SD 2.29, median 2.57).
\cref{fig:interaction_logs_boxplots} (left) shows this as box plots.
These differences were significant as follows (\cref{tab:lmm_overview}, row 1): 
Participants finished replying significantly faster with \modemail{} than without AI (-\secs{70}). \modemail{} was also significantly faster than \modeours{} (-\secs{66}).



\subsubsection{Writing Speed}\label{sec:results_speed}

On average, participants wrote 2.05 characters per second without AI (SD 1.57, median 1.78).
Replying with \modemail{} had a mean of 7.21 (SD 7.90, median 4.89), while the mean speed with \modeours{} was 4.38 (SD 5.59, median 2.91). For \modeours, using the \imppass{} feature resulted in 5.03 (SD 6.47, median 3.23), while not using it had 2.91 (SD 2.00, median 2.31).
\cref{fig:interaction_logs_boxplots} (right) shows this as box plots.
These differences were significant as follows (\cref{tab:lmm_overview}, row 2): 
Compared to writing without AI, participants produced significantly more characters per second with \modemail{} (5.2 chars more per s) and if they used the \imppass{} feature in \modeours{} (2.5 chars more per s). The difference between \modemail{} and \modeours{} was also significant.




\subsubsection{Manual Typing}\label{sec:results_keystrokes}
Without AI, participants on average needed 321.69 keystrokes (SD 217.76, median 284), compared to 146.67 (SD 144.97, median 108) with \modemail, and 174.62 (SD 166.57, median 128) with \modeours. For \modeours, using the \imppass{} feature had a mean of 176.0 (SD 173.7, median 126), while not using it had 171.4 (SD 149.9, median 133.5).
\cref{fig:interaction_logs_boxplots} (centre) shows this as box plots.
These differences were significant (\cref{tab:lmm_overview}, row 3): People needed significantly fewer keystrokes with AI features than without them, and even significantly fewer with \modemail{} (\pct{58} decrease) than with \modeours{} (\pct{48} decrease). Using the \imppass{} feature in \modeours{} significantly reduced this further for that UI (\pct{5.9} decrease). In summary, all AI features significantly reduced manual typing.




\subsubsection{Interaction with \modeours}
We logged interactions specific to \modeours.
On average participants tapped on 2.64  (SD 2.89, median 2) sentences per email, that is, on \pct{30.36} (SD \pct{29.59}, median \pct{23.08}) of sentences in each email.
They replied to \pct{87.37} of tapped sentences. In \pct{83.27} of the cases, they did so by accepting a suggestion. %

Suggestions were paginated; most suggestions (\pct{67.15}) were accepted on the first page. Another \pct{19.50} and \pct{13.36} were accepted on the second and third page, respectively.
The majority of accepted suggestions (\pct{80.14}) were generated without an explicit prompt, and most (\pct{92.30}) were not edited afterwards.


\revision{On the first screen, \pct{69.05} of participants accepted a sentence suggestion at least once, and \pct{55.56} manually entered text for at least one local response. %
On the second screen, \pct{83.33} of participants accepted at least one email-level suggestion. Only \pct{1.59} (two participants) did not use any \modeours-specific features.}




On average, \mins{1.67} (\pct{57.23}) were spent on the first screen and \mins{1.25} (\pct{42.77}) on the second (\cref{fig:time_spent_on_screens_barplot}).
Participants used the \imppass{} feature for 287 emails (\pct{75.93}) and accepted an improved email for 274 emails (\pct{72.49}).
When the \imppass{} feature was used at least once, an improved email was requested on average 1.35 times (SD 0.95, median 1.00) and accepted 1.13 times (SD 0.47, median 1.00) per email.
The last accepted improved email was identical to the sent email in \pct{83.94} of all cases.
When participants made changes these had a mean edit distance of 72.73 (SD 97.10, median 37).


\subsubsection{Interaction with Full Email Generation (\modemail)}
With this UI, \pct{71.98} of the first generated replies were accepted; otherwise, a new generation was requested.
Users spent \mins{0.43} (\pct{21.20}) on the incoming email screen, \mins{1.31} (\pct{65.09}) on the generation view and \mins{0.28} (\pct{13.71}) on the editing screen (\cref{fig:time_spent_on_screens_barplot}).

\subsubsection{Workflow Analysis}\label{sec:results_workflows}


\begin{figure}
    \centering
    \includegraphics[width=\minof{\columnwidth}{0.66\textwidth}]{figures/sentence_based_workflow_scatterplot}
    \caption{Analysis of workflows with \modeourstxt: Each point is one email and its position is the state of the drafting process at the moment when the user switched from the first screen (\cref{fig:teaser}.1) to the second (\cref{fig:teaser}.2). Concretely, the x-axis shows normalised time (0-\pct{100}), i.e. temporal progression. The y-axis shows normalised length, i.e. draft progression. Note that y-values >\pct{100} are possible if an intermediate draft is longer than the final version. Colour and marker shape indicate if the \imppass{} feature was used or not. The figure reveals three clusters: \textit{(1) Bottom left} -- here, people skipped to the second screen and used the \imppass{} feature to generate a draft. \textit{(2) Top right} -- mostly drafting on the first screen, with light manual editing on the second. \textit{(3) In between} -- partly drafting on the first screen and finalising it with AI on the second one.}
    \Description{This figure displays a scatter plot analysing workflows with content-driven local responses, focusing on when participants switched from the first screen to the second screen during the email drafting process.
    X-axis (Progress in Time [\% total time]): This axis represents the normalised time (ranging from 0 to 100) indicating the temporal progression of the drafting process.
    Y-axis (Draft Progress [\% final length]): This axis represents the normalised length of the draft at the moment of switching screens. Values greater than 1 are possible if an intermediate draft was longer than the final version.
    Colour Coding: The colour of each point indicates whether the "improvement pass" feature was used:
    Blue points represent emails where the improvement pass feature was used.
    Orange points indicate emails where it was not used.
    The scatter plot reveals three distinct clusters of participant behaviour:
    Bottom Left Cluster:
    Participants in this group quickly moved to the second screen and utilised the improvement pass feature to generate a draft with minimal work on the first screen.
    Top Right Cluster:
    Participants in this group spent most of their time drafting on the first screen, with only light manual editing on the second screen.
    In Between Cluster:
    This group represents participants who partly drafted on the first screen and then finalised the draft with AI assistance on the second screen.
    This analysis highlights the different strategies participants used during the email drafting process, depending on their interaction with the interface and the improvement pass feature.}
    \label{fig:workflow_scatterplot}
\end{figure}

\begin{figure}
    \centering
    \includegraphics[width=\minof{\columnwidth}{0.75\textwidth}]{figures/time_spent_on_screens.pdf}
    \caption{Participants spent their time on different screens and thus different aspects. The figure shows the means for the time spent on screens that focus on reading the incoming email vs on screens that focus on responding (colour). Borders indicate which steps required AI (solid) or offered it optionally (dashed). For \textit{\modemanual}, users read the email, then spend most of the time writing the reply. For \textit{\modeours}, the local response screen (\cref{fig:teaser}.1) enables reading and responding in parallel (striped), followed by responding on the second screen (\cref{fig:teaser}.2), both with optional AI (sentence suggestions, \imppass{}). In contrast, \textit{\modemail} requires AI after the initial reading phase to generate the response, which can then be manually edited.}
    \Description{This figure displays a bar plot analysing times spent on each step in the answering process. The figure shows the means for the time spent on screens that focus on reading the incoming email vs on the screens that focus on responding. For NoAI, users read the email, then spend most of the time writing the reply. For CDLR, the local response screen enables reading and responding in parallel, followed by responding on the second screen, both with optional AI (sentence suggestions, improvement pass). In contrast, MSG requires AI after the initial reading phase to generate the response, which can then be manually edited.}
    \label{fig:time_spent_on_screens_barplot}
\end{figure}

For \modeours, we discovered three main workflows by plotting when people switched from the first screen (\cref{fig:teaser}A) to the second (\cref{fig:teaser}B). \cref{fig:workflow_scatterplot} reveals three clusters: (1) Sometimes people went straight to the second screen and used the \imppass{} feature to create a draft. (2) Alternatively, they spent most of their time drafting on the first screen, with light manual editing on the second. (3) Finally, people partially drafted on the first screen and finished it on the second screen, using AI.
We fitted a GMM\footnote{Gaussian Mixture Model with 3 components using \url{https://scikit-learn.org/}} to estimate the number of emails: Cluster 1 had 136, cluster 2 had 54, and cluster 3 had 188 emails.

We also examined the relationship of the incoming email's length and whether people skipped the local response screen without entering any text. This was significant (\cref{tab:lmm_overview2}, row 2): Each additional word (i.e. 5 additional characters) in the incoming email is associated with a \pct{2.46} decreased chance of skipping the local response step in \modeours{}. 

For \modemail, we found that in most cases (\pct{77.8} of emails) people sent the generated drafts without manually editing them further. When they indeed edited them (\pct{22.2} of emails), the mean edit distance between generated and edited version was 64.87 (SD 67.48, median 44.50). This corresponds to typing about twelve words~\cite{kristensson2014inviscid}.

We also analysed how people prompted with \modemail: In a majority of cases (\pct{82.01} of emails), participants entered a prompt right away. Otherwise, they generated text solely based on the information in the incoming email. In half of those cases (\pct{51.47}), participants did not accept the result and generated another. In comparison, such a regeneration was only needed in \pct{22.9} of the cases where participants entered a prompt. We observed a learning effect for some: \pct{40} of people started entering a prompt if they were not happy with the initial result generated without a prompt. %



\subsection{Perception of Interaction}\label{sec:results_perception}
We analysed participants' perception of the three UIs.


\subsubsection{In-app Questionnaire (Likert Data)}\label{sec:results_in_app}


\begin{figure*}[t]
    \centering
    \includegraphics[width=\linewidth]{figures/inapp_likert_items.pdf}
    \caption{Likert results on perception of the UIs and interaction, rated after each email task. Overall, participants rated the AI-supported UIs higher on speed, quality, and helpfulness compared to the manual mode. However, the latter was rated higher on control.}
    \Description{This figure presents bar charts displaying Likert scale results from participants' perceptions of different UIs and their interactions, rated after completing email tasks. 
    The figure is divided into four sections, each comparing three UIs: NoAI (manual mode), CDLR (AI-supported), and MSG (AI-supported).
    Top-Left Chart: The app interface was helpful
    The NoAI interface received mixed responses, with a significant portion of participants disagreeing or remaining neutral, and fewer strongly agreeing.
    Both CDLR and MSG interfaces were rated more positively, with a larger number of participants agreeing or strongly agreeing that the interfaces were helpful.
    Top-Right Chart: The app interface helped me reply to the email quickly
    The NoAI interface again had a more varied response, with some participants disagreeing or remaining neutral, while others agreed.
    CDLR and MSG interfaces were rated highly for helping users reply quickly, with the majority of participants agreeing or strongly agreeing.
    Bottom-Left Chart: The app interface helped me write a good reply
    Similar to the other charts, the NoAI interface had a mix of responses, with fewer participants strongly agreeing.
    CDLR and MSG interfaces were again rated highly, with most participants agreeing or strongly agreeing that these interfaces helped them write good replies.
    Bottom-Right Chart: I was in control of the content of my reply
    For this aspect, the NoAI interface was rated slightly higher, with more participants strongly agreeing that they felt in control of their reply content.
    Although CDLR and MSG interfaces were also rated positively, there was a slight decrease in the number of participants who strongly agreed compared to the NoAI interface.
    In summary, the figure shows that participants generally rated the AI-supported UIs (CDLR and MSG) higher in terms of speed, quality, and helpfulness. 
    However, the manual mode (NoAI) was rated slightly higher in terms of giving users a sense of control over their replies.}
    \label{fig:inapp_likert_items}
\end{figure*}

Participants rated four Likert items in the app after each email (\cref{fig:inapp_likert_items}). 
We found statistically significant \revision{effects of \ivmode{} on all four items -- speed (\artf{2}{996.11}{433.71}{<.0001}, \petasq{.46}), control (\artf{2}{996.11}{20.60}{<.0001}, \petasq{.04}), quality (\artf{2}{996.13}{466.99}{<.0001}, \petasq{.48}), and helpfulness (\artf{2}{996.1}{430.61}{<.0001}, \petasq{.46}).}
\revision{Concretely,} \modeours{} and \modemail{} were both rated significantly higher than \modemanual{} on speed \revision{(\modeours{}: \artc{996}{23.34}{<.0001}; \modemail{}: \artc{996}{27.25}{<.0001})}, quality \revision{(\modeours{}: \artc{996}{25.72}{<.0001}; \modemail{}: \artc{996}{27.18}{<.0001})}, and helpfulness \revision{(\modeours{}: \artc{996}{24.65}{<.0001}; \modemail{}: \artc{996}{26.14}{<.0001})}. They were both rated significantly lower than \modemanual{} on control \revision{(\modeours{}: \artc{996}{-5.88}{<.0001}; \modemail{}: \artc{996}{-5.17}{<.0001})}. 
The only significant difference between the UIs with AI was that \modemail{} was rated higher on speed than \modeours{} \revision{(\artc{996}{3.91}{=.0001})}. 


\subsubsection{In-App Feedback}
We reviewed the in-app feedback optionally provided after each reply.
For both AI modes it was overwhelmingly positive, such as: ``Im really enjoying this kind of AI help mode.'' (P60\oldId{P1351}, \modemail), ``It made work easy'' (P76\oldId{P1370}, \modemail), ``Very smooth process, good suggestions for each part.'' (P47\oldId{P1329}, \modeours), and ``This made my reply look way better.'' (P81\oldId{P1375}, \modeours).



The negative feedback was less homogeneous.
For \modemail, around half of these critiques highlighted difficulties in getting the AI to incorporate specific information, such as: ``the AI who seemed to resist wanting to offer access to my colleague'' (P25\oldId{P1298}); or ``had a bit of trouble trying to get the AI to properly acknowledge that \$200 was okay [...]''  (P37\oldId{P1312}). 
Related, P53\oldId{P1338} noted that ``Control in replying was lacking, It didn't give me many options to 'add' ideas of my own.''
People found ways to steer the system; P47\oldId{P1329} said that ``I had to adjust the prompt a few times to get the sort of reply that I was looking for, but it did generate a good reply overall and I was satisfied with the end result.''

Notably, issues with including specific information were rarely mentioned for \modeours.
Most of the negative feedback instead concerned the tone: ``This was too wordy for an informal email.'' (P56\oldId{P1341}), %
and ``The ai was helpful but it made the response feel slightly too formal and professional.'' (P49\oldId{P1333}) %

For the manual mode, people ``had no issues, [and] felt able to use the platform freely and there was no technical faults'' (P7\oldId{P1276}), and that ``It was just like normal email.'' (P60\oldId{P1351}). %


\subsubsection{Favourite Reply Support}\label{sec:results_fav_mode}
Only \pct{4} (5 people) preferred \modemanual, %
\pct{49.2} (62 people) favoured \modemail, and \pct{43.7} (55 people) preferred \modeours.
The remaining \pct{3.2} (4 people) did not pick a favourite.

Notably, the high-level code ``Efficiency'' occurred in \pct{56.45} of comments for \modemail{} and \pct{32.73} for \modeours{}.
``Quality'' in \pct{25.82} for \modemail{} and \pct{30.91} for \modeours{}.
``Control'' in \pct{8.07} for \modemail{} and \pct{29.09} for \modeours{}.
``Tailoring'' zero times for \modemail{} and \pct{5.46} for \modeours{}.
The remaining comments were assigned the code ``Others'' (e.g. P105\oldId{P1405} ``just liked the interface'' of \modemail{} and P76\oldId{P1370} favoured \modeours{} because it supported them in being creative).

Two out of the four people who did not select a favourite stated that they liked both depending on the ``context'' (P77\oldId{P1371}).
For instance, P19\oldId{P1290} explained that ``both have different advantages in different situations. Single prompt allows to produce a full email much faster so is handy when you are short of time but still want to respond. Sentence based provides the user the ability to create a much more tailored email which can cover all bases.''

The \pct{4} (5 people) who preferred \modemanual{} said they were ``used to it'' (P4\oldId{P1273}) or ``confident in [their] writing ability'' (P78\oldId{P1372}).
















\subsubsection{Summary}
People perceived AI features as helpful and preferred having them. They were divided about their favourite and perceived meaningful tradeoffs between the two designs with AI on control vs efficiency: While people felt in control with all UIs (\cref{sec:results_in_app}), when reflecting on their favourite, they mentioned control aspects relatively more frequently for \modeours{} than \modemail{} -- and vice versa for efficiency.




\subsection{Analysis of Emails}\label{sec:results_emails}
We analysed the content of the emails. \cref{fig:quality_boxplots} shows four box plots as an overview.

\begin{figure*}[t]
    \centering
    \includegraphics[width=\linewidth]{figures/quality_boxplots.pdf}
    \caption{Four measures of email characteristics from our study. The plots show email length \textit{(left)} and rate of spelling/grammar/punctuation errors \textit{(centre left)}. Moreover, we measured lexical diversity with the distinct-2 score \textit{(centre right)}, which is defined as an email's number of distinct bigrams divided by its number of words (higher = more diverse). Finally, we measure diversity between emails \textit{(right)}, based on the cosine similarity of vector embeddings (higher = less diverse). Overall, all AI features increased reply lengths, decreased error rates, and lowered diversity (all sig.).  \revision{For \modeours{} we further distinguish between emails where the improvement pass feature was used (impr.) and those where it was not (no impr.).}
    See \cref{sec:results_emails} for details.}
    \Description{This figure presents four box plots comparing different characteristics of emails across three user interfaces: NoAI (manual mode), CDLR (AI-supported with and without the optional improvement pass feature), and MSG (AI-supported). The metrics being compared are:
    Email Length (Left):
    This plot shows the length of the emails in characters.
    AI-supported interfaces (CDLR and MSG) led to longer emails compared to the NoAI interface, with MSG producing the longest emails.
    Error Rate (Center Left):
    This plot displays the rate of spelling, grammar, and punctuation errors relative to the number of characters in the email.
    Both CDLR and MSG reduced the error rate compared to NoAI, with slightly better performance for the improvement pass-enabled CDLR and MSG interfaces.
    Diversity within Emails (Center Right):
    The distinct-2 score measures the lexical diversity within individual emails by calculating the number of unique bigrams (word pairs) relative to the total word count. A higher score indicates greater diversity.
    NoAI emails had higher within-email diversity compared to CDLR and MSG, which showed similar scores indicating reduced lexical diversity.
    Diversity between Emails (Right):
    This plot measures the cosine similarity between vector embeddings of emails, with higher scores indicating less diversity between emails.
    Emails generated using CDLR and MSG were more similar to one another, as indicated by higher cosine similarity scores, compared to those written using the NoAI interface.
    Overall, the figure demonstrates that AI-supported interfaces (CDLR and MSG) increased email length, reduced error rates, and resulted in slightly less diversity both within and between emails. For detailed analysis, refer to the corresponding section in the paper (Section 6.2).
    }
    \label{fig:quality_boxplots}
\end{figure*}


\subsubsection{Email Lengths}\label{sec:results_lengths}

On average, emails written without AI were 302.5 characters long (SD 169.7, median 267).
\modemail{} resulted in 536.1 (SD 319.0, median 447.5), while \modeours{} had 483.0 (SD 285.4, median 382). For \modeours, using the \imppass{} feature had a mean length of 523.9 (SD 294.1, median 412.0), while not using it had 390.6 (SD 241.5, median 324.5).
These differences were significant (\cref{tab:lmm_overview}, row 4): People wrote significantly longer replies with the AI features than without them, and significantly more so with \modemail{} (\pct{77} increase) than with \modeours{} (\pct{27} increase). Using the \imppass{} feature in \modeours{} significantly increased this further for that UI (\pct{38} increase). In summary, all AI features significantly increased text lengths.


\subsubsection{Error Rates}\label{sec:results_errors}
We checked grammar and spelling with the language-tool-python\footnote{\url{https://pypi.org/project/language-tool-python/}} library.
Per email, we recorded the minimum of British English and American English spell checking to avoid penalising spelling differences. %
Manual writing had the highest mean error rate of .00375 errors per character.  Both AI versions were about half of that: \modemail{} had .00176 and \modeours{} had .00182. Using the \imppass{} feature in \modeours{} contributed to reducing errors (mean .00147 when using it vs .00260 when not). All these differences were significant (\cref{tab:lmm_overview}, row 5). In summary, all AI features significantly reduced error rates.

\subsubsection{Diversity Across Emails}\label{sec:results_email_similarity}
We analysed the semantic similarity between emails, following related work~\cite{padmakumar2024diversity}. We computed the cosine similarity of the vector embeddings of all pairs of emails written with the same mode and for the same briefing, using the Sentence Transformers library (SBERT\footnote{\url{https://sbert.net}, specifically \url{https://huggingface.co/sentence-transformers/all-MiniLM-L6-v2}})~\cite{reimers2019sbert}.
As expected from the literature, manual emails had the lowest mean pairwise similarity (.582), that is, they were the most diverse. \modemail{} had the highest similarity (.756), followed by \modeours{} (.726). 
Using the \imppass{} feature in \modeours{} contributed to the increase (.676 without using it vs .749 with it). 
All these differences were significant (\cref{tab:lmm_overview}, row 6). In summary, all AI features significantly reduced semantic diversity.


\subsubsection{Diversity Within Emails}\label{sec:results_lexical_diversity}
We analysed lexical diversity, as in related work~\cite{Fu2023sentencevsmessage}, with the distinct-2 metric, defined as the number of distinct bigrams divided by the total number of words.
Our results match the related work: Writing without AI had the highest mean lexical diversity (.950), \modemail{} lowered it to .930, and \modeours{} had .924.
Using the \imppass{} feature in \modeours{} (almost) closes the gap between \modeours{} and \modemail{} (.915 without it vs .927 with it). 
All these differences were significant (\cref{tab:lmm_overview}, row 7). In summary, lexical diversity was significantly affected by all AI features, in the way we would expect from related work~\cite{Fu2023sentencevsmessage}: Sentence-level generations decreased it more than message-level generations.







\subsubsection{Email Structure}\label{sec:results_structure}

Salutations were missing in \pct{8.5} of manually written replies and in \pct{10.6} with \modeours. All replies with \modemail{} had salutations. All replies with \modeours{} and the \imppass{} feature had a salutation.
Similarly, only one email with \modemail{} lacked a closing statement, compared to \pct{14.3} for \modemanual{} and \pct{9.0} for \modeours. Again, when the \imppass{} was used in \modeours, all emails ended with a closing signature.


\subsubsection{Briefing Conformity}\label{sec:results_briefing}
Each email reply task showed a briefing that asked participants to respond with certain information (see \cref{sec:procedure_email_tasks}). This allowed us to analyse if their emails conformed to this or not (see \cref{sec:quality_m}).
This varied across the UIs: With \modeours, \pct{23} of emails missed a key aspect of the study briefing, compared to \pct{18} for \modemail{} and \pct{13} for \modemanual.

The differences between \modeours{} and the other two UIs were significant (\cref{tab:lmm_overview2}, row 1). %
However, people's prompting behaviour had a larger impact here: Across \modeours{} and \modemail{}, generating a full reply without any own input (83 emails in the data) missed a key aspect of the briefing in half of the cases (\pct{49}). We return to this in the discussion (\cref{sec:discussion_methods})



\subsubsection{Subjective Assessment of Quality}\label{sec:result_quality}
All emails were read by (at least) two researchers. %
While length and structure varied (\cref{sec:results_lengths}, \cref{sec:results_structure}), we did not notice ``nonsense'' responses. Even those emails which did not conform to the briefings (\cref{sec:results_briefing}) reflected the general topic and most would have been believable replies. We noticed that sensible replies can vary drastically -- from short responses to elaborate, formal emails. The latter mostly coincided with message-level AI generation. We do not consider this an issue of quality since we did not specify a level of formality to follow. Overall, based on our subjective assessment, we thus concluded that reply quality was suitable across all UIs. 

\section{Discussion}
\label{sec:discussion}
\textsc{WWD} is a socio-technical infrastructure that supports the collection of cultural data, in the form of food, in a bottom-up community-led manner. Community members' needs and experiences actively shaped the architecture of WWD. Our data collection platform was constructed to be compatible with the established digital infrastructure and cultural norms of the communities we worked with. The types of data we collected (e.g., the attributes for each dish) were informed by community members who identified what was important to capture about a dish to accurately represent how the dish is prepared and consumed in their culture. 

Building the \textsc{WWD System} in a bottom-up, community-led manner required an immense amount of labour. Data did not simply flood in once the system architecture was built. Core Organisers and Community Ambassadors engaged in \textit{data work}---a socio-technical process through which data about local cuisines was produced. As many social computing scholars have noted, data work is often overlooked despite its essential role in shaping the epistemology of a dataset and consequently the downstream performance of ML systems~\cite{sambasivan2021everyone,ismailEngagingSolidarityData2018,mollerWhoDoesWork2020,scheuermanProductsPositionalityHow2024}. 

In our discussion, we surface the tensions that occurred during data work. These breakdowns in the data work process help us to reveal deeper structural issues in the AI/ML production pipeline that confound bottom-up, community-led approaches to dataset construction. Communities facing representational harms~\cite{weidinger2021ethical} and disparities in quality of service~\cite{shankar2017allocational, de2019doescvworkallocational} face a catch-22 when participating in efforts to improve dataset coverage: They can shoulder the burden of participation or be excluded from model ontology. New technologies, particularly GenAI tools, have been proposed as a way for communities to preserve their culture representation by participating in efforts to contribute data to model training~\cite{heritage7030070}. However, participation does not necessarily entail improved outcomes for communities~\cite{birhanePowerPeopleOpportunities2022}. We point to a difference in ethical frameworks between communities on the African continent with whom we worked and those of large tech companies that build and control GenAI technologies to illuminate why the promises of participation often fall short. 


\subsection{Tensions in data collection}
 
Our results show that there were tensions around data collection. Specifically, issues surrounding image provenance, the accuracy of information about a dish, and the benefits of participation arose throughout the data collection process. These issues reveal deeper structural problems with the AI/ML pipeline. 
\subsubsection{Establishing a clean bill of data provenance}
Recent efforts in participatory ML research attempt to safeguard the labour and intellectual property of community data contributors by creating dataset licenses restricting the use of the community's dataset, which assign ownership and terms of access and use to these datasets~\cite{birhanePowerPeopleOpportunities2022,longpre2024largelicence}. However, for the license to be effective, the data must have a clean bill of provenance. \textsc{World Wide Dishes} was built to be an open-source dataset with a Creative Commons license that could be used for model evaluation. As a result, the WWD dataset had to fulfil strict requirements for data quality, including ensuring that the dataset creators had a right to the images contained within the dataset.  In other words, the creators of \textsc{WWD} must then be able to claim rightful use and ownership over all the images collected as part of the project. However, many of the images that Contributors submitted during the data collection phase were taken from the Internet and lacked proper licensing. As a result, Community Ambassadors had to engage in extensive consultation and discussion with Contributors to ensure they understood the importance of data provenance in their submissions. Contributions, where the origins of the submitted were unclear, had to be deleted, erasing bits of cultural knowledge from our dataset. Ensuring a clean bill of data provenance was time-intensive and not easily scalable. It was difficult to enforce image upload guidelines in a volunteer effort, resulting in a smaller, less representative, dataset than we would have liked. However, the rigorous process of ensuring a clean bill of data provenance for each submission enabled us to, in good faith, release our dataset as an open-source project. 

The standard of open-sourcing datasets and applying Creative Commons license, while understandable, places massive burdens on small, community-led projects such as \textsc{WWD} to ensure clean bills of data provenance. To be clear, we are not arguing against open-source datasets or Creative Commons licenses, but rather are demonstrating the need to build infrastructures that support and fund the labour needed to verify that data for their projects can be used. As mentioned in~\cref{background}, understanding cultural nuance on a fine-grained, regional scale requires extensive (and non-extractive) consultation with community members who have the capacity to share local expertise. As such, non-exploitative and non-extractive community consultation is an important step in verifying the validity and veracity of cultural information. 

\subsubsection{Verifying cultural information}

Collecting accurate and representative cultural data is exceptionally difficult. Cultures are not bounded by government borders and/or other manufactured systems, but rather extend across larger regions and are often the product of intercultural exchanges~\cite{gupta2008beyondculture}. This makes determining the veracity of a data point in \textsc{WWD} almost impossible without extensive consultation with a community member with local expertise. In \textsc{WWD}, we sought to include as many Contributors as possible to collect a granular representation of cultural data. We accessed Contributors through our community ambassadors who had established relationships and trust with the folks they asked to contribute. Inclusion, however, can be a slippery slope~\cite{epstein2008rise,benjamin2016informed}. 

ML researchers continue to pursue the construction of ever more representative datasets in the name of improving model performance for \textit{everyone}~\cite{luccioni2021everyone, radford2018improvingeveryone}. Often, ML researchers have trouble accessing ``hard-to-reach'' populations, such as the communities we worked with to build \textsc{WWD}. Many recent projects have attempted to solicit engagement from ``hard-to-reach'' populations~\cite{kirk2024prism,ramaswamy2023geodegeographicallydiverseevaluation,singh2024aya_dataset}, yet none of these projects interrogate why these populations might be hard for researchers to access. Drawing on Benjamin's work~\cite{benjamin2016informed}, \textbf{we urge ML researchers to consider how research institutions and industry laboratories may engender distrust within communities that have endured centuries of extractive practices by actors from the Global North.} It is essential that researchers not only endeavour to make participation accessible to members of ``hard-to-reach'' communities but also work towards establishing themselves as trustworthy partners in the research process, in the same way that~\citet{singh2024aya_dataset} do this. 

\subsubsection{Explaining the benefits of participation}

Community Ambassadors wrestled with explaining the benefits of participation in \textsc{WWD} to potential Contributors. Participation was not financially compensated. The research team chose not to make use of professional data centre workers\footnote{Professional data centre workers are those people employed in a centralised manner to perform data collection tasks. Their livelihood is, therefore, connected to the requirement to engage in data contribution, which does not align with \textsc{WWD} goals. Additionally, even had we wished to use data centre workers, we lacked the resources to which a large technology company might have access, such as the ability to engage a business outsourcing company (e.g., Enlabler) to recruit and pay data workers.} because the nature of the data collection process argued for prioritising organic engagement through social networks to collect perspectives from people who do not, and have not, typically contributed to Internet datasets from around the world. We purposely chose a data collection method that would enable the use of social networks and allow us to reach participants other than those employed in a data worker centre, such as older generations and those across a wide socioeconomic range. We also wanted to empower participants to involve their families in the process. 

Although the research team would have preferred to individually compensate each Contributor, because \textsc{WWD} relies on a decentralised, global-scale data collection method, and, crucially, as of the time of data collection, normative standards and infrastructure do not exist to support such a decentralised payment process to effectively \textbf{pay data contributors}, we were \textit{unable} to pay them. The research team explored many possible avenues for paying participants but each time came up against prohibitively expensive and logistically insurmountable barriers. For example, money transfer services such as PayPal~\cite{paypal_countries_2023} and Wise~\cite{wise_usage_2023} were unavailable in many of the regions where \textsc{WWD} operates. These types of services also require that payment recipients have access to digital banking services, which many within our target communities do not. In addition, some of our Core Organisers, who are from the African continent and utilise digital banking services, provided anecdotal evidence of times when their transactions were flagged for seemingly no other reason than their nationality. Infrastructures to support financial remuneration for research participants in the Majority World are simply not commensurate with the many calls from Western researchers to engage participants in these parts of the world. \textbf{Researchers must therefore build the infrastructures to enable equitable participation with communities}; in particular, researchers should investigate how to address breakdowns in participant compensation infrastructures. Other similarly decentralised efforts have remunerated contributors with material items (e.g., sweaters and small gifts)~\cite{singh2024aya_dataset}. Still other researchers point to the limitations of financial compensation for participants and urge researchers to consider what kinds of remuneration would be useful given the context of their research site~\cite{hodge2020relational}.

Despite the lack of extrinsic, financial incentives, the Contributors did exhibit some intrinsic motivation. Contributors shared many different reasons for having participated, such as wanting to make a difference in GenAI outputs, supporting a friend, or contributing to a mission and team they believed in. The majority of data contributions came from the African continent. The authors have speculated why this might be, and have wondered if there is a common focus uniting these Contributors: a central philosophy of ``familyhood'' and unity. This is known by different terms across the continent, including djema’a (Arabic), ubuntu (Zulu), ujamaa (KiSwahili), umuntu (Chichewa), and unhu (Shona). Community Ambassadors also suspected that their positionality as members of the communities from which they were soliciting data contributions further strengthened sentiments of unity among participants who saw \textsc{WWD} as an extension of the growing ``By Africans, for Africans'' movement in ML~\cite{birhane_2024_for_africans}. 

Whilst we can only speculate about why participants engaged in the data contribution process, the authors recognise the responsibility they were given to respect and honour these Contributors and to avoid extractive and exploitative practices. 

\subsection{Participate or be excluded: A catch-22}


Cultural erasure and lack of representation are rooted in deep systemic issues that date back centuries. GenAI, especially T2I models, play an increasingly prominent role in shaping the media ecosystem. However, relying on these models to ``fix'' centuries of intentional cultural erasure overlooks the deeper systemic issues that will likely constrain the efficacy of these technocentric solutions. During the data collecting for \textsc{WWD}, Community Ambassadors often found themselves rationalising the uncompensated nature of data contributions by demonstrating that existing T2I models perform poorly when creating images of local dishes so Contributors should provide accurate data to teach the model what the dish should look like. Regardless, many Contributors, Community Ambassadors recalled, were eager to participate in an African-researcher-led ML effort. 

Through the reflection process, Community Ambassadors shared conflicting feelings about tapping into the shared philosophy of familyhood and unity that they suspected motivated Contributors' participation. On the one hand, local communities were engaging in the dataset creation process through the lens of \textit{Ubuntu} (broadly translated as ``I am because we are'')---an ethical framework that emphasises dignity, reciprocity, and the common good~\cite{ewuoso2019core}. In contrast, the models that would subsequently be trained by these datasets are developed in Western contexts and imbued with utilitarian ethics---a framework that emphasises the best for the greatest number of people~\cite{selbstFairnessAbstractionSociotechnical2019,west2004introduction}. These two distinct, yet interrelated elements of the ML pipeline---dataset production and model development---are therefore produced not only in distinct geographic regions~\cite{sambasivan2021everyone,scheuermanDatasetsHavePolitics2021} but also, in our case, in two distinct ethical frameworks. Contributors who engaged with us out of a sense of \textit{Ubuntu} are unlikely to see their values recognised and preserved in the actual functioning of the downstream T2I model that is optimising for fundamentally different well-being criteria.  

Participation in dataset construction is not a guaranteed way to achieve representational justice in T2I models. Thus, proposing to communities that to avoid being excluded from the future of media representation, they should participate in dataset development, is misleading. This false choice obfuscates (1) the deeper systematic issues that dictate whose culture gets preserved and represented and (2) the disjunction between the value system under which participants may contribute data and that of the models that are then trained on this data. 

Future research efforts should examine how to bring the ethical frameworks of dataset creation and model development into alignment by prioritising local, community ownership over AI. 


\section{Conclusions}
In this paper, we proposed the formalism of distributionally robust DPO,  developed two novel algorithms using this framework,  and established their theoretical guarantees. We also developed efficient approximation techniques that enable scalable implementation of these algorithms as part of the existing LLM alignment pipeline. We showed extensive empirical evaluations that validate the effectiveness of our proposed algorithms in addressing preference distribution shifts in LLM alignment. In future works, we plan to extend our distributionally robust DPO algorithms to address the challenges of reward hacking. We also plan to develop distributionally robust algorithms for other RLHF approaches. 


\begin{acks}
Funded by the Deutsche Forschungsgemeinschaft (DFG, German Research Foundation) -- 525037874.
\end{acks}

\bibliographystyle{ACM-Reference-Format}
\bibliography{bibliography}

\appendix

\section{Appendix}

This appendix lists our prompt templates (\cref{sec:appendix_prompts}), questionnaires (\cref{sec:appendix_questionnaires}) and additional figures (\cref{sec:appendix_extra_figures}).

\subsection{Prompting}\label{sec:appendix_prompts}
Our final prototype used the prompt templates shown below. For a better overview, we shortened them for this appendix by leaving out the concrete few-shot examples and only indicating their position in the templates. See the project repository (link in \cref{sec:conclusion}) for the full prompts including these examples.

Contained variables are defined as follows:

\begin{itemize}
    \item sender: The full name of the sender.
    \item email\_text: Text content of the received email.
    \item existing\_reply: Either `This is the reply you have written so far: ``{existing\_text}'' ' or an empty string if there is no existing text.
    \item attribute: ``accepting'', or ``neutral'', or ``declining'' (two generations for each attribute)
    \item referenced\_text: The sentence that was selected on.
    \item input: Prompt that the user gave.
\end{itemize}

\subsubsection{Sentence-level support, without user input}\label{sec:appendix_sentence_without_input_prompt} %
\begin{lstlisting}
    System: "You are answering an email sentence by sentence. For each given sentence think of a suitable reply. The reply should only answer the selected sentence."
    ... few-shot examples...
    User: "You are Jamie Doe and have received this email from {sender}: '{email_text}'.
    {existing_reply} Formulate a short, {attribute} reply to this selected part of the email: '{referenced_text}'. Only output the short reply in one or two sentences."
\end{lstlisting}

\subsubsection{Sentence-level support, with user input} %
\begin{lstlisting}
    System: "You are answering an email sentence by sentence. For each given sentence think of a suitable reply. The reply should only answer the selected sentence."
    ... few-shot examples...
    User: "You are Jamie Doe and have received this email from {sender}:"{email_text}".
    {existing_reply} Formulate a short reply to this selected part of the email: "{referenced_text}". Incorporate this information into your reply: "{input}".  Only output the short reply in one or two sentences."
\end{lstlisting}

\subsubsection{Improve Email}\label{subsec:appendix_improve_email_prompt} %
\begin{lstlisting}
    System: "You have received an email and have drafted a reply. Now you review your draft and make some final edits to make it sound better. You output the entire improved email at once and nothing else."
    ... few-shot examples...
    User: "You are Jamie Doe and have received this email from {sender}:"{email_text}"
    You have written this reply as an answer:"{existing_reply}"
    You improve this email by fixing any mistakes and adding an email greeting or sign-off if missing. You also make sure to make it sound better but you do not change the content of the email. At last you only output the well formatted email."
\end{lstlisting}

\subsubsection{Message-level reply generation} \mbox{}
\begin{lstlisting}
    System: "You have received an email and are writing a response to it."
    ... few-shot examples...
    You are Jamie Doe and have received this email from {sender}:"{email_text}"
    You answer with a well written email following these instructions: "{input}". You make sure to add a greeting and a sign-off. You do not make anything up that is not mentioned in the instruction. You double check that the email is well formatted."
\end{lstlisting}







\subsection{Questionnaires}\label{sec:appendix_questionnaires}

\subsubsection{Favourite Mode}
\begin{enumerate}
  \item 
    \textbf{Question:} Which mode did you prefer for answering emails? \\
    \textbf{Question Type:} single-choice \\
    \textbf{Answer Options:} 
    \begin{itemize}
      \item Sentence-based suggestions
      \item Single prompt suggestion
      \item Without AI-Support
      \item Depends (please describe) [free-response]
    \end{itemize}
  \item 
    \textbf{Question:} Why did you prefer this mode? \\
    \textbf{Question Type:} free-response

  \item 
    \textbf{Question (Optional):} Did you face any problems, issues or bugs during your participation in this study? \\
    \textbf{Question Type:} free-response
\end{enumerate}


\subsubsection{Demographic Questionnaire}
\begin{enumerate}

\item
  \textbf{Question:} What gender do you identify with? \\
  \textbf{Question Type:} single-choice \\
  \textbf{Answer Options:}
  \begin{itemize}
    \item Woman
    \item Man
    \item Non-Binary
    \item Prefer not to disclose
    \item Prefer to self-describe: [free-response]
  \end{itemize}

\item
  \textbf{Question:} How well do you speak English? \\
  \textbf{Question Type:} single-choice \\
  \textbf{Answer Options:}
  \begin{itemize}
    \item No knowledge of English
    \item Speak poorly (beginner knowledge)
    \item Fairly well (intermediate knowledge)
    \item Well (advanced knowledge)
    \item Very well (proficient in English)
    \item Native speaker
  \end{itemize}

\item
  \textbf{Question:} How old are you? \\
  \textbf{Question Type:} numeric response

\item
  \textbf{Question:} What is your current occupation? \\
  \textbf{Question Type:} free-response

\item
  \textbf{Question:} What is the highest academic level you have achieved? \\
  \textbf{Question Type:} single-choice \\
  \textbf{Answer Options:}
  \begin{itemize}
    \item High School Diploma or equivalent
    \item Bachelor's Degree
    \item Master's Degree
    \item Doctoral Degree
    \item Other: [free-response]
  \end{itemize}

\item
  \textbf{Question:} How often do you reply to emails? \\
  \textbf{Question Type:} single-choice \\
  \textbf{Answer Options:}
  \begin{itemize}
    \item Never
    \item Less than monthly
    \item At least once a month
    \item At least once a week
    \item Daily
    \item More than 10 times a day
    \item Other: [free-response]
  \end{itemize}

\item
  \textbf{Question:} What devices do you use to answer emails on? \\
  \textbf{Question Type:} multiple-select \\
  \textbf{Answer Options:}
  \begin{itemize}
    \item Desktop-PC
    \item Laptop
    \item Tablet
    \item Smartphone
    \item Smartwatch
    \item Other: [free-response]
  \end{itemize}

\item
  \textbf{Question:} Do you have experience with AI writing support (including for email)? \\
  \textbf{Question Type:} multiple-select \\
  \textbf{Answer Options:}
  \begin{itemize}
    \item No, I have no experience writing with AI support
    \item Writing with word- or sentence-suggestions
    \item Writing with auto-correction
    \item Writing with auto-completion
    \item Using ChatGPT (or similar)
    \item Using the smart reply feature
    \item Other: [free-response]
  \end{itemize}

\item
  \textbf{Question:} Where do you usually reply to emails? \\
  \textbf{Question Type:} multiple-select \\
  \textbf{Answer Options:}
  \begin{itemize}
    \item On the go
    \item At home
    \item At the office
    \item Somewhere else: [free-response]
  \end{itemize}

\item
  \textbf{Question:} What context do most of your emails have? \\
  \textbf{Question Type:} multiple-select \\
  \textbf{Answer Options:}
  \begin{itemize}
    \item Business
    \item Private
    \item Other: [free-response]
  \end{itemize}

\end{enumerate}






\subsection{Additional Figures}\label{sec:appendix_extra_figures}
Here we provide additional figures.
\cref{fig:likert_items_formative_study} shows the Likert and \cref{fig:sus_items_formative_study} the SUS \cite{brooke1996sus} results from the formative study (\cref{sec:formative_study}).
The other figures show the UIs used in the study: \cref{fig:briefing_and_feedback} shows the screens for the briefing and in-app feedback. \cref{fig:baseline_uis_manual} shows the UI of the manual mode (\modemanual), and \cref{fig:baseline_uis_msg} shows the UI of the \modemailtxt{} (\modemail). 

\begin{figure*}[h!]
    \centering
    \includegraphics[width=\linewidth]{figures/likert_formative_study.pdf}
    \caption{Likert results from the formative study (\cref{sec:formative_study}). The first three items were asked via an in-app feedback screen after each email, while the others were part of the final questionnaire at the end of the study.}
    \Description{This figure presents the Likert scale results from the formative study, which evaluates participant feedback on various aspects of using the AI tool for email writing. The responses are categorised into five levels of agreement: strongly disagree, disagree, neutral, agree, and strongly agree. The figure indicates that the majority of participants had a positive experience with the AI tool, finding it helpful in writing faster, higher-quality replies with enough variation in suggestions, though some reported mixed feelings about control and distractions.}
    \label{fig:likert_items_formative_study}
\end{figure*}

\begin{figure*}[h!]
    \centering
    \includegraphics[width=\linewidth]{figures/sus_formative_study.pdf}
    \caption{\revision{SUS \cite{brooke1996sus} results from the formative study (\cref{sec:formative_study}).}}
    \Description{This figure presents the SUS Likert scale results from the formative study. The responses are categorised into five levels of agreement: strongly disagree, disagree, neutral, agree, and strongly agree. The figure indicates that the majority of participants had a positive experience with the AI tool.}
    \label{fig:sus_items_formative_study}
\end{figure*}

\begin{figure*}[h!]
    \centering
    \includegraphics[width=0.6\linewidth]{figures/briefing_and_feedback}
    \caption{The study-specific UI screens in our prototype: A screen showing the briefing before each email task \textit{(1)}, and a screen asking users to rate four Likert items after each email task \textit{(2)}.}
    \Description{This figure presents two study-specific UI screens from the prototype used in the research:
    Briefing Screen (Left Panel):
    This screen is shown before each email task.
    It provides a brief description of the task, explaining the scenario or context that the participant needs to respond to.
    The screen includes instructions to "Answer the following email according to this idea" and has a "Continue" button to move forward.
    In-App Feedback Screen (Right Panel):
    This screen appears after each email task, asking participants to rate their experience.
    The specific statements in the example ask about the helpfulness of the interface.
    Participants can choose from five options ranging from "strongly disagree" to "strongly agree" for each statement, and submit their feedback by pressing the "Send Email" button.}
    \label{fig:briefing_and_feedback}
\end{figure*}


\begin{figure*}[h!]
    \centering
    \includegraphics[width=0.6\linewidth]{figures/baseline_uis_manual}
    \caption{The UI design for manual typing (\modemanual) used in the study. It has one screen to show the incoming email \textit{(1)} and one with a text box to type the reply manually \textit{(2)}. This figure shows the state after typing a reply, as an example. As is usual on mobile devices, the keyboard opened from the bottom when tapping on the text field.}
    \Description{This figure illustrates the UI design for manual typing (NoAI) used in the study, showcasing two views:
    Incoming Email View (Left Panel):
    This screen displays the email received by the user.
    In this example, the email is about the subject "What pet to get?" asking for advice on choosing a family pet between a cat or a dog.
    At the bottom of the screen, there is a "Reply" button, allowing the user to initiate a response.
    Manual Reply View (Right Panel):
    Once the "Reply" button is pressed, the user is taken to the reply screen, where they can type their response manually in a text box.
    The "Send Email" button is at the bottom.}
    \label{fig:baseline_uis_manual}
\end{figure*}


\begin{figure*}[h!]
    \centering
    \includegraphics[width=0.8\linewidth]{figures/baseline_uis_msg}
    \caption{The UI design for \modemailtxt{} (\modemail) used in the study. The first screen shows the incoming email \textit{(1)}. On the reply generation screen \textit{(2)}, users can generate a full message suggestion, optionally guided by entering a prompt in the text box at the top. Finally, the manual screen \textit{(3)} allowed users to freely edit the generated draft.}
    \Description{This figure shows the UI design for message-level reply generation (MSG) used in the study, illustrating three stages of the interaction:
    Incoming Email View (Left Panel):
    Similar to the manual UI, this screen displays the incoming email.
    A "Reply" button is available at the bottom.
    Reply Generation View (Middle Panel):
    In this screen, users can generate a full message reply using AI.
    There is an optional prompt field at the top, where users can enter specific keywords or topics to guide the AI's response.
    Once the prompt is entered, users can press the "Generate Reply" button, and the AI will generate a suggested email, displayed in the output field below.
    The user can either discard the generated reply or choose to save and edit it.
    Manual Editing View (Right Panel):
    After generating the AI-suggested reply, users are taken to this screen to manually edit the draft.
    The "Send Email" button at the bottom allows the user to send the edited reply.}
    \label{fig:baseline_uis_msg}
\end{figure*}


\end{document}
