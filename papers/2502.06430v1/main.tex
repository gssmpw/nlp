\documentclass[sigconf]{acmart}

\usepackage{tabularx}
\usepackage{calc}
\usepackage[capitalize, noabbrev]{cleveref}
\usepackage{subcaption}
\usepackage{listings}
\lstset{
basicstyle=\small\ttfamily,
columns=flexible,
breaklines=true
}

\usepackage{siunitx}
\usepackage{enumitem}

\AtBeginDocument{%
  \providecommand\BibTeX{{%
    \normalfont B\kern-0.5em{\scshape i\kern-0.25em b}\kern-0.8em\TeX}}}

\copyrightyear{2025} 
\acmYear{2025} 
\setcopyright{cc}
\setcctype{by}
\acmConference[CHI '25]{CHI Conference on Human Factors in Computing Systems}{April 26-May 1, 2025}{Yokohama, Japan}
\acmBooktitle{CHI Conference on Human Factors in Computing Systems (CHI '25), April 26-May 1, 2025, Yokohama, Japan}\acmDOI{10.1145/3706598.3713890}
\acmISBN{979-8-4007-1394-1/25/04}





\begin{document}

\title[Content-Driven Local Response: Sentence-Level and Message-Level Mobile Email Replies]{Content-Driven Local Response: Supporting Sentence-Level and Message-Level Mobile Email Replies With and Without AI}


\author{Tim Zindulka}
\authornote{Both authors contributed equally to this research.}
\email{tim.zindulka@uni-bayreuth.de}
\orcid{0009-0009-1972-351X}
\affiliation{%
  \institution{University of Bayreuth}
  \city{Bayreuth}
  \country{Germany}
}

\author{Sven Goller}
\authornotemark[1]
\email{sven.goller@uni-bayreuth.de}
\orcid{0000-0001-5263-5372}
\affiliation{%
  \institution{University of Bayreuth}
  \city{Bayreuth}
  \country{Germany}
}

\author{Florian Lehmann}
\email{florian.lehmann@uni-bayreuth.de}
\orcid{0000-0003-0201-867X}
\affiliation{%
  \institution{University of Bayreuth}
  \city{Bayreuth}
  \country{Germany}
}

\author{Daniel Buschek}
\email{daniel.buschek@uni-bayreuth.de}
\orcid{0000-0002-0013-715X}
\affiliation{%
  \institution{University of Bayreuth}
  \city{Bayreuth}
  \country{Germany}
}

\renewcommand{\shortauthors}{Zindulka and Goller et al.}


\definecolor{TimsColor}{rgb}{0.1,0.5,0.8}
\newcommand{\tim}[1]{\textsf{\textbf{\textcolor{TimsColor}{[Tim: #1]}}}}
\definecolor{SvensColor}{rgb}{0.5,0.8,0.5}
\newcommand{\sven}[1]{\textsf{\textbf{\textcolor{SvensColor}{[Sven: #1]}}}}
\definecolor{FlosColor}{rgb}{0.9,0.1,0.8}
\newcommand{\flo}[1]{\textsf{\textbf{\textcolor{FlosColor}{[Flo: #1]}}}}
\definecolor{DanielsColor}{rgb}{0.9,0.6,0.1}
\newcommand{\daniel}[1]{\textsf{\textbf{\textcolor{DanielsColor}{[Daniel: #1]}}}}


\newcommand{\minsec}[2]{\SI{#1}{\minute} \SI{#2}{\second}}
\newcommand{\mins}[1]{\SI{#1}{\minute}}
\newcommand{\secs}[1]{\SI{#1}{\second}}
\newcommand{\pct}[1]{\ifnum\pdfstrcmp{#1}{X}=0
        X\% 
    \else\SI{#1}{\percent}\fi
}

\newcommand{\lmmci}[5]{$\beta$=#1, SE=#2, CI$_{95\%}$=[#3, #4], p#5}
\newcommand{\posthoc}[2]{#1, p#2}
\newcommand{\artf}[4]{$F$(#1,\,#2)=#3, p#4}
\newcommand{\artc}[3]{$t$(#1)=#2, p#3}
\newcommand{\petasq}[1]{$\eta_p^2$=#1}

\definecolor{deemphColor}{rgb}{0.4,0.4,0.4}
\newcommand{\deemph}[1]{\textcolor{deemphColor}{#1}}

\newcommand{\ivmode}{\textsc{UImode}}
\newcommand{\modeourstxt}{content-driven local response}
\newcommand{\modeoursTxt}{Content-driven local response}
\newcommand{\modeoursTXT}{Content-Driven Local Response}
\newcommand{\modemailtxt}{message-level reply generation}
\newcommand{\modemailTxt}{Message-level reply generation}
\newcommand{\modemanual}{\textsc{NoAI}}
\newcommand{\modeours}{\textsc{CDLR}}
\newcommand{\modemail}{\textsc{MSG}}

\newcommand{\imppass}{improvement pass}
\newcommand{\Imppass}{Improvement pass}



\newcommand{\studyOneN}{17} %
\newcommand{\studyTwoN}{126} %


\newcommand{\lastaccessed}{\textit{last accessed 22.08.2024}}

\newcommand{\oldId}[1]{} %


\newcommand\revision[1]{\textcolor{black}{#1}}

\begin{abstract}
Humor is a social binding agent. It is an act of creativity that can provoke emotional reactions on a broad range of topics. Humor has long been thought to be “too human” for AI to generate. However, humans are complex, and humor requires our complex set of skills: cognitive reasoning, social understanding, a broad base of knowledge, creative thinking, and audience understanding. We explore whether giving AI such skills enables it to write humor. We target one audience: Gen Z humor fans. We ask people to rate meme caption humor from three sources: highly upvoted human captions, 2) basic LLMs, and 3) LLMs captions with humor skills. We find that users like LLMs captions with humor skills more than basic LLMs and almost on par with top-rated humor written by people. We discuss how giving AI human-like skills can help it generate communication that resonates with people. 

\end{abstract}

\begin{CCSXML}
<ccs2012>
   <concept>
       <concept_id>10003120.10003121.10011748</concept_id>
       <concept_desc>Human-centered computing~Empirical studies in HCI</concept_desc>
       <concept_significance>500</concept_significance>
       </concept>
   <concept>
       <concept_id>10003120.10003121.10003128.10011753</concept_id>
       <concept_desc>Human-centered computing~Text input</concept_desc>
       <concept_significance>500</concept_significance>
       </concept>
   <concept>
       <concept_id>10010147.10010178.10010179</concept_id>
       <concept_desc>Computing methodologies~Natural language processing</concept_desc>
       <concept_significance>500</concept_significance>
       </concept>
 </ccs2012>
\end{CCSXML}

\ccsdesc[500]{Human-centered computing~Empirical studies in HCI}
\ccsdesc[500]{Human-centered computing~Text input}
\ccsdesc[500]{Computing methodologies~Natural language processing}

\keywords{Writing assistance, Large language models, Human-AI interaction, Email, Mobile text entry}

\begin{teaserfigure}
    \centering
    \includegraphics[width=\textwidth]{figures/teaser}
   \caption{Replying to an email with \textit{\modeoursTXT}: \textit{(1)} In the \textit{local response view}, users can insert responses \textit{(A)} directly while reading the email. \textit{(B)} Tapping on a sentence opens a response widget, \textit{(C)} with a text box where users enter a response or a prompt that affects \textit{(D)} the sentence suggestions below. \textit{(2)} After adding local responses, users go to the \textit{draft view}, to turn their responses into a full reply email. They can do so manually and/or with the help of \textit{(E)} an AI \imppass{} feature, which generates \textit{(F)} a message-level suggestion, displayed with highlighted changes. These AI features are flexible and optional: Users can add local responses without using suggestions. They can also skip directly to the draft view, optionally enter a prompt there, and use the improvement feature to generate a full reply directly. This supports flexible workflows.}
   \label{fig:teaser}
   \Description{This figure illustrates the process of replying to an email using the "Content-Driven Local Response" feature and outlines different flexible workflows with optional AI support.
   The figure is divided into three main sections:
   Local Response View with Sentence-Level AI (Left Panel)
   A) Users can insert responses directly while reading the email.
   B) Tapping on a sentence opens a response widget, which allows users to interact with the email content.
   C) The widget includes a text box where users can enter their own response or a prompt that will influence the sentence suggestions displayed below.
   D) Below the text box, sentence-level AI-generated suggestions are provided based on the entered prompt or context of the email.
   Draft View with Message-Level AI (Middle Panel)
   After entering responses or prompts in the local response view, users can proceed to the draft view.
   (E) In the draft view, users can manually edit the draft or use an AI improvement pass feature to generate a message-level suggestion.
   (F) The AI-generated suggestion is displayed with highlighted changes, allowing users to review and accept the improvements.
   Examples of Flexible Workflows with Optional AI (Bottom Section)
   The figure outlines various workflows users can follow, ranging from full AI-assisted reply generation to fully manual drafting:
   Full reply generation: Users can go directly to the draft view and use the improvement feature to generate a complete reply.
   Partial AI-assisted reply generation: Users can draft partially with sentence-level AI support and then finalise the reply with message-level AI or manual edits.
   Fully manual: Users can choose to draft and send the email without using any AI assistance.
   This figure highlights the flexibility of the system, allowing users to choose between different levels of AI support depending on their preference or the specific requirements of the email task.}
\end{teaserfigure}


\maketitle


The increasing reliance on LLMs for multimodal tasks across far-reaching sectors such as healthcare, finance, and manufacturing underscores the need to assess the accuracy and reliability of the information they generate. Vision-Language Models (VLM) have achieved state-of-the-art (SoTA) performance on Visual Question-Answering (VQA) benchmarks, and these models often utilize Retrieval-Augmented Generation (RAG) to maintain factual accuracy and relevance in a dynamic information environment. However, this has led to uncertainty in the information the LLM bases its answer on, as it may choose between parametric memory and retrieved sources. When models rely on memorized information instead of dynamically retrieving information, they may inadvertently propagate outdated or incorrect information, causing serious legal and ethical risks and undermining trust and reliability in AI systems \citep{huang2023survey}.
% The ability to strike a balance between generalization and specialization in AI systems is therefore crucial for ensuring the safe, reliable use of these technologies in real-world applications.

Despite these concerns, the way that Vision-Language models (VLMs) memorize and retrieve information, particularly in complex multimodal tasks, remains under-explored. Current research often focuses on either the general capabilities of large language models (LLMs) or the specialized retrieval mechanisms in retrieval augmented generation systems (RAG) \citep{incontext_rag,chen_murag_2022,liu_universal_2023}. Particularly in the context of multimodal retrieval and multihop reasoning, few studies analyze the tradeoff between finetuning for specialized tasks and zero-shot prompting for general-purpose vision-language capabilities. A lack of consensus on how to approach this tradeoff motivates the development of measures to quantify reliance on parametric memory, as well as metrics for quantifying the potential performance impact of extending LLMs with RAG systems.

To address this gap, we investigate how multimodal QA models balance accuracy with memorization on the WebQA benchmark. We compare finetuned multimodal systems against zero-shot VLMs, analyzing how retrieval performance influences QA accuracy. In particular, we focus on cases where retrieval fails, allowing us to measure reliance on parametric memory through two proposed metrics---the \ppr (\PPR) which quantifies how much model accuracy is influenced by retrieval quality, contrasting performance in best-case versus worst-case retrieval scenarios, and the \ucr (\UCR) which measures how often correct QA responses are generated when the retriever fails, providing a proxy for memorization.

To enable this analysis, we make several methodological contributions. For the finetuned QA models, we investigate Vision-Transformer (ViT) architectures, which allow for multihop reasoning over multiple sources. To investigate the impact of retrieval performance on trained LMs, we propose a variable-input Fusion-in-Decoder (FiD) model \cite{tanaka_slidevqa_2023, nlvr2}, building upon the VoLTA architecture \citep{pramanick_volta_2023}. For the zero-shot case, we build upon previous research on In-Context Retrieval \citep{incontext_rag} by demonstrating that LLMs such as GPT-4o are capable of performing the final ranking step of the retrieval process. In doing so, we find that GPT-4o, a general-purpose LLM, achieves SoTA performance on the WebQA task, outperforming existing finetuned RAG models by a significant margin (7\% higher accuracy). 

Crucially, our results reveal that while retrieval-augmented models reduce memorization, the training paradigm plays an important role. Finetuned models exhibit higher reliance on parametric memory, whereas zero-shot RAG approaches have lower memorization scores at the cost of accuracy. This suggests that while retrieval modules may mitigate the risks associated with outdated or incorrect information, SoTA performance requires that they be coupled with specialized QA models. Our memorization measures contribute to the development of transparent and reliable AI systems, particularly in applications where the sourcing of up-to-date, factual information is critical.



% We investigate the impact of question complexity on the ability of these models to integrate multiple data sources—such as images, text, and external retrievers—and produce coherent and accurate answers. We also explore whether in-context retrieval can be a viable alternative to traditional retrieval-augmented systems, offering a more streamlined approach to multimodal QA.

% To achieve this, we first compare zero-shot prompting multimodal LLMs with finetuned multimodal systems. We evaluate both types of models on the WebQA benchmark, a dataset designed for complex question answering that requires reasoning across both image and text sources. For the finetuned models, we use a Fusion-in-Decoder (FiD) architecture, which allows for multihop reasoning over multiple sources. Additionally, we introduce the concept of In-Context Retrieval Language Modeling (RLM), where the LLM itself performs retrieval tasks without the need for external retrievers. This method builds upon existing research in in-context learning  and aims to explore the viability of LLMs retrieving relevant sources and generating accurate answers directly from their context window.

% In order to investigate source utilization in finetuned multimodal models and LLMs, three lines of inquiry are established; 
% \begin{itemize}
%     \item Study 1: retrieval vs QA performance on webQA (motivating example, does QA answer correctly even with incorrect sources?)
%     \item Study 2: performance on adversarial examples where parametric knowledge would be incorrect by design
%     \item Study 3: improving performance on adversarial examples by fine-tuning (i.e model robustness)
% \end{itemize}

% Note, there is one weakness in this plan which is tying in the work we've already done. 
% If we added something from adversarial generation to the retrieval experiment (like a combination of study 1 + 3) it would be complete. So for instance we could try fine-tuning the retriever with adversarial examples (and not just the QA model)

% \begin{figure}
%     \centering
%     \includegraphics[width=0.95\linewidth]{figures/segmentation/webqa_segment_infill.png}
%     \caption{Example of the segmentation substitution pipeline from the WebQA task.}
%     % d5c76d760dba11ecb1e81171463288e9
%     \label{fig:seg_sub_pipeline}
% \end{figure}



% Retrieval augmented generation (RAG) with zero-shot prompting and fine-tuning Large Language Models (LLMs) have become the go-to methods for tasks relying on information retrieval and text generation. In many cases the LLMs parametric memory can sufficiently generalize to answer questions without being provided with retrieval mechanisms for out-of-domain knowledge. However, LLMs often hallucinate and provide wrong information in certain scenarios. This problem is amplified even further on open-domain Question Answering (QA) tasks involving multiple modalities. Grounded text generation using retrieved sources \citep{lewis2021retrievalaugmented} has been extensively studied for text-to-text QA tasks, but its application in multimodal settings has not been studied as much.


% Multimodal reasoning and question answering have gained prominence in recent research endeavors, with an increasing emphasis on handling various forms of data, particularly text and images. In this study, we address a specific gap in the existing literature by focusing on the development of a versatile multihop model capable of accommodating varying numbers of input images.

% Our motivation for this research lies in the growing complexity of answering questions using information on the web, where the challenge of navigating the open-domain setting is further complicated by the presence of multiple modalities and sometimes requires reasoning over multiple sources. WebQA is an ideal dataset on which to compare performance of finetuned RAG systems against general purpose LLMs; it is multimodal, with correct answers requiring reasoning over image and text sources. It is multihop, requiring a complex reasoning process over multiple sources. Finally, WebQA questions from different categories can be broken down into subdomains to analyze performance over domains of varying cardinality.

% Motivated by the real-world challenges of building retrieval and question answering (QA) systems, we design and finetune a closed domain, multimodal, multihop QA model, that is capable of reasoning over a varying number of sources taken as input from an external retriever module. This research contributes to the relatively underexplored domain of multihop reasoning across various input sources and modalities. Our goal is to explore the challenges posed by these scenarios and develop strategies that enable QA models to retrieve relevant information, conduct logical or numerical reasoning across diverse modalities, and generate coherent responses in natural language. To our knowledge, this is the first application of the Fusion-in-Decoder (FiD) architecture \cite{tanaka_slidevqa_2023, nlvr2} that is shown to work with a variable number of inputs, enabling multi-hop reasoning over sources.

% In-Context Learning refers to the ability of LLMs to perform any task by simply providing examples in the input prompt \citep{dong2022survey,min2022rethinking}. Inspired by this research, we propose a method to use the LLM itself as a multimodal retriever, potentially eschewing the requirement of a distinct retrieval module, thereby allowing the design of simpler retrieval-augmented QA systems. We dub this method In-Context Retrieval Language Modeling (RLM). To the best of the authors knowledge, In-Content RLM is disparate from other retrieval augmented approaches which utilize external retrieval modules \citep{incontext_rag,chen_murag_2022,liu_universal_2023}. Despite being a natural extension of In-Context learning, In-Context RLM has not yet been studied empirically.

% To expand on our contribution of In-Context Retrieval, this stems from the well-researched in-context learning of LLMs. In-context learning is the ability of a model to perform any task given a sufficient context window \citep{dong2022survey,min2022rethinking}. Such tasks could include retrieval and ranking, but typically, the go-to solution for tasks requiring retrieval has been RAG. To the best of the authors knowledge, In-Context Retrieval is distinct from In-Context Retrieval Augmented Language Modelling (RALM), and despite being a natural extension of In-Context learning, In-Context Retrieval has not yet been shown empirically.

% Finally, we explore the tradeoff between using zero-shot prompting LLMs and the fine-tuning approach. While we find that, overall, GPT-4o obtains SoTA performance on the WebQA task, outperforming the accuracy of existing finetuned RAG approaches by 7\%, finetuned approaches still perform better on more restricted subdomains\footnote{``In-Context RLM" @ \url{https://eval.ai/web/challenges/challenge-page/1255/leaderboard/3168}}. Finally, we validate that GPT-4o is relying on retrieval abilities to solve the task; we find that GPT-4o is capable of retrieving relevant sources in the presence of distractors and furthermore, when GPT-4o fails to retrieve correct sources, it answers incorrectly 75\% of the time, meaning that it is not relying on parametric memory for this task.

% \paragraph{Contributions}
% Based on our experimentation and analysis on the WebQA benchmark, we make the following contributions:
% \begin{itemize}
%     \item Propose a new architecture for multimodal multihop QA that takes variable number of input sources inspired by the Fusion-in-Decoder method.
%     \item Comparison of general purpose LLMs vs specialized models on the WebQA benchmark.
%     \item Observation of In-Context Multimodal Retrieval abilities of GPT-4o and that it does not rely on parametric memory for multimodal QA.
%     \item Analysis of relationship between retrieval and QA task performance.
%     \item Analysis of task and query complexity on the performance of retrieval and QA tasks.
% \end{itemize}
















% Throughout this paper, we will present our methodology, experiments, and findings, emphasizing our approach to multihop reasoning over varying numbers of input images. We believe that our work contributes to a deeper understanding of multimodal reasoning and has the potential to enhance the capabilities of question-answering systems in the intricate, multimodal landscape of web-based information.
\section{Background and Motivation}
\label{sec:background}

We introduce the background on serverless workload serving and motivate the use of runtime resource adaptation to address resource inefficiency in existing serverless platforms.

\subsection{Resource Inefficiency with Early Binding}
% In current serverless platforms, developers are required to specify immutable sizes for their deployed functions.
% Then, providers consider functions' runtime workloads  (e.g., concurrency)  and resource usage to scale out/in their instances.
% Moreover, due to high runtime variability, functions must size their functions for worst-case scenarios.
% This, however, incurs considerable resource inefficiency.
Current serverless workflow platforms (e.g., AWS Step Functions~\cite{aws-step-function} and Azure Durable Functions~\cite{azure-durable-function}) offer the opportunity for developers to build various applications with advanced logic like chaining, branching, and parallel execution.
These applications can be defined by JSON-based structured languages (e.g., Amazon States Language) or other programming languages.
Meanwhile, developers require to specify resource configurations, including memory size, CPU cores, and scaling options, for individual functions---an early-binding approach.
The serverless platform is responsible for monitoring the workload intensity and resource usage at runtime and scaling out/in function instances accordingly.
To account for potential runtime variability, developers must size the functions in their application workflow accounting for the worst case in order to provide SLO guarantees over the end-to-end delay of request processing, e.g., the 99th percentile (P99) of the end-to-end delay must be within a given target. 
After deployment, the function sizes become immutable. The worst case is not representative and over-shoots most of the time, leading to resource inefficiency. 


To verify this claim, we conduct a data-driven analysis with a dataset from Microsoft Azure Functions~\cite{azure-dataset} to explicitly demonstrate the resource inefficiency issue. % , deriving from the worst-case based early bind.
To quantify the inefficiency, we define a metric called \emph{slack}---the margin between the actual execution time and the SLO, which is calculated as $1-l/T$ with $l$ and $T$ representing end-to-end latency and SLO, respectively.
Under certain SLO defined with P99 latency as done by existing works (e.g., \cite{osdi22-orion,mac22-wisefuse}),  we can see from Figure \ref{fig:bg:slack} that more than 60\% function invocations have slacks over 60\%.
Particularly, we analyze slacks of the top 100 most popular functions, whose invocations account for 81.6\% of the total function invocations. % (depicted in Figure~\ref{fig:bg:popular_func}) of overall invocations.
The result shows that only 20\% of the invocations of the popular functions (blue line in Figure~\ref{fig:bg:slack}) have slacks less than 40\%.
This means the majority of requests are processed faster than necessary.
Notably, in DAG-based workloads (i.e., Azure Durable Functions), the resource inefficiency further deteriorates wherein the ratio between the 95th percentile and 50th percentile is by up to three times \cite{mac22-wisefuse}.

% \begin{figure}[t!]
% \centering
% \includegraphics[width=0.25\textwidth]{./figure/motivation/Average_P99_cdf_top=100.pdf}
% \vspace{-0.3cm}
% \caption{Sufficient function slacks in production traces.}
% \label{fig:bg:slack}
% \end{figure}

\subsection{Runtime Dynamics}
\label{sec:bg:worst-case}

The resource inefficiency caused by the large slack can be mainly attributed to the over-provisioning of resources by the developer. This is to ensure that the SLO is guaranteed even in the worst case (i.e., P99). However, normal cases deviate from the worst case significantly due to runtime dynamics. 
In particular, we observe that functions face two major dynamic factors at runtime: varying working sets and inevitable performance interference. These two factors contribute significantly to the variance of the function execution time. 
% Functions face two remarkably dynamic factors at runtime: working sets and performance interference, which lead to considerable variance of execution latency.

\begin{figure*}[!t]
	\centering
	\subfloat[]{
		\includegraphics[width=0.24\textwidth]{./figure/motivation/Average_P99_cdf_top=100.pdf}
		\label{fig:bg:slack}
	}
	\hspace{8mm}
	\subfloat[]{
		\includegraphics[width=0.25\textwidth]{./figure/motivation/function-latency-ml-analyze-varying-worksets.pdf}
		\label{fig:bg:ml-func-latency}
	}
	\hspace{8mm}
	\subfloat[]{
	\includegraphics[width=0.28\textwidth]{./figure/motivation/coresident-perf.pdf}   
	\label{fig:bg:perf-inteference}
	}
	%\vspace{-0.1cm}
	\caption{(a) slacks of function invocations in production traces, (b) function latency variance caused by varying input worksets for functions object detection (OD), question answering (QA), and and text-to-speech (TS), respectively,
 (c) performance interference attributed to co-location of homogeneous function with different dominant resource demands.}
 %\vspace{-0.4cm}
\end{figure*}

%'ml-analyze':{'text-to-speech': 'text-to-speech', 'question-answer': 'question answer',
%                      'object-detection': 'object detection'
\textbf{\textit{Varying working sets.}} 
The working set, i.e., input data like videos, audios, and texts, can have varying sizes.
Taking Microsoft Azure Function Blobs (storage service) as an example, their data size difference can be as high as nine orders of magnitude~\cite{azure-function-blob}.
Such a large difference results in substantial variance of the execution time even for the same function~\cite{socc21-faast,eurosys21-ofc}.
Specifically, we measure the execution time of three functions under different working sets (detailed in \S\ref{exp:setup}).
Figure~\ref{fig:bg:ml-func-latency} illustrates the results, where we can observe a variance of up to 3.8 times in function execution caused by varying working set sizes.

% \begin{figure}[t!]
% \centering
% \includegraphics[width=0.25\textwidth]{././figure/motivation/function-latency-ml-analyze-varying-worksets.pdf}
% \vspace{-0.3cm}
% \caption{Function latency variance caused by varying input worksets}
% \label{fig:bg:ml-func-latency}
% \end{figure}	

\textbf{\textit{Performance interference.}}
% On the other hand, function deployment, which decides when and where to deploy functions, is completely undertaken by providers.
For simplicity and security, commercial serverless platforms, such as Alibaba Function Compute, Microsoft Azure, and AWS Lambda, exclusively deploy function instances belonging to the same tenant, or even belonging to the same function, in the same virtual machine~\cite{socc22-owl,atc18-peek-bench}.
For example, the empirical study in~\cite{socc22-owl} shows that in Alibaba Function Compute 65\% of the virtual machines exclusively deploy instances of the same function.
This co-location of homogeneous function instances, however, can incur severe resource contention on the same resource dimensions, particularly for network bandwidth and memory bandwidth of virtual machines~\cite{sc21-gsight,micro19-faaSprofiler,socc22-owl,atc18-peek-bench}.
To verify this observation, we use a virtual machine to run a function increasing the number of co-located instances from one to six while measuring the execution time of four different functions with resource dominance on different dimensions namely computing, I/O, network, and memory, respectively (detailed in \S\ref{exp:setup}). 
As shown in Figure~\ref{fig:bg:perf-inteference}, the co-location of homogeneous functions leads to substantial resource contention and performance interference, prolonging the function execution time up to 8.1 times. The performance interference is often hard to model and predict.

% this co-residency results in substantial increase of execution latency by up to 8.1 times,leading to considerable variance in function execution time.
% when compared with that with concurrency as one.

%for CPU-, IO-, network- and memory-intensive functions as the concurrency rises from one to six.
%Figure shows that significant performance interference can be observed, . 
%compared with the inclusive deployment (concurrency as one), 
% this exclusive deployment (gray bar) results in substantial increase of execution latency by up to 8.1$\times$ for CPU-, IO-, network- and memory-intensive functions as the concurrency rises from one to six.

% this exclusive deployment (gray bar) results in substantial increase of execution latency by up to 8.1$\times$ for CPU-, IO-, network- and memory-intensive functions as the concurrency rises from one to six.
% As depicted in Figure~\ref{fig:bg:concurrent_latency}, with the concurrency rising  from one to six,  the exclusive deployment results in substantial increase of execution latency by up to 8.1$\times$.
% This significantly magnifies execution latency variance.

% \begin{figure}[t!]
% \centering
% \includegraphics[width=0.25\textwidth]{./figure/motivation/coresident-perf.pdf}
% \vspace{-0.3cm}
% \caption{Performance interference attributed to co-residency of homogeneous function.}
% \label{fig:bg:perf-inteference}
% \end{figure}




\subsection{Runtime Resource Adaptation}
\label{sec:bg:adaptive-allocation}
To tackle the aforementioned resource inefficiency issue, we can adopt a late-binding approach through \emph{runtime resource adaptation}, which resizes functions on the fly based on runtime information (e.g., function slacks), achieving higher resource efficiency without violating SLO. For example, given a workflow as a chain of functions, the resource allocation of the downstream functions can be adjusted when the first function finishes execution. This way, the slack from the first function can be exploited to optimize resource efficiency. 

The idea sounds straightforward and has been considered in some existing works \cite{infocom22-stepconf,middleware20-fifer,socc21-llama,socc21-kraken,middleware20-xanadu}.
However, most of these works make an unrealistic assumption that either the developer performs the adaptation decision with access to runtime information or the serverless platform provider performs the adaptation with domain knowledge of the application workflow. These assumptions render these solutions impractical to deploy in real-world serverless systems. The information barrier between the developer and the provider calls for a new solution. 

We identify the following challenges and opportunities for a full-fledged design for runtime resource adaptation. 

\textbf{\textit{Skewed function execution time distribution.}} 
Resource allocation for a serverless workflow is typically done by leveraging performance profiles of all the functions in the workflow. 
During the offline profiling, the execution time distribution for each function is first obtained by running the function with a variety of sample inputs under different resource conditions. Then, given a time budget, existing approaches typically use P99 of the function execution time as a target and calculate the corresponding resource demands. However, due to the high runtime variability, the distribution of the function execution time is highly skewed where the difference between P50 and P99 can be as high as 100 times~\cite{socc23-huawei-cloud}. This means that if only the function execution time at a single percentile (P50 or P99) is used for resource allocation, there will be significant resource under-provisioning and over-provisioning for most requests at runtime. To address this issue, our idea is to allow for the exploration of the function execution time at diverse percentiles during resource allocation. 


% It is a prerequisite to profile execution latency for adaptive resource allocation.  
% As aforementioned, owing to a variety of unexpected runtime dynamics,  execution latency demonstrates skewed distributions, by up to 100$\times$ between 99\% percentile and 50\% percentile on Huawei cloud serverless~\cite{socc23-huawei-cloud} .
% This makes the current a single statistic (e.g., mean) or 99\% percentile distribution based profiling suffer significant under- and over-estimation.
% To fix this issue, our insight is to \textit{introduce more diverse percentiles to profile execution latency}. 

\textbf{\textit{Dependencies of adaptation decisions.}}
As the function execution progresses, a sub-workflow will be generated by removing the finished function(s) from the workflow. Within each sub-workflow, the resource adaptation decisions for remaining functions are dependent on each other due to the constraint imposed by the end-to-end latency SLO. For example, under-provisioning a function will result in a reduction of the time budget for executing its downstream functions, thus calling for more resources for these downstream functions to avoid SLO violations. Meanwhile, the selection of the percentile for the execution time of each function dictates resource-latency tradeoff for that function. For example, a higher percentile means that more resources will be allocated to ensure that more requests processed by the function will finish within the given time budget. On the contrary, a lower percentile means that more requests will risk SLO violation, but at the benefits of reduced resource consumption. To address such complex dependencies, we propose the following ideas: (1) We introduce two metrics (i.e., the timeout metric and the resilience metric detailed in \S\ref{sec:profilier}) to balance the resource adaptation decisions of the head function of the current sub-workflow and those of the remaining downstream functions. These metrics help us connect the decision making across sub-workflows and avoids sub-optimal adaptation decisions in each sub-workflow. 
(2) We explore lower percentiles for the head function and a high percentile (i.e., P99) for other functions in each sub-workflow. Using lower percentiles maximizes the opportunity for resource optimization since any over-time execution of the head function can later be compensated by resource adaptation in the next round. The high percentile ensures that the resource adaptation is not too radical to cause SLO violations. 

% Each workflow generates multiple sub-workflows as the execution moves forwards. 
% Within sub-workflows, the provisioning is inter-corrected.
% For instance, under-provisioning upstream functions may directly shrink the time budget for downstream functions, which dictates more resources required by the latter against (sub-) SLO violation. 
% This makes sub-workflows generally adopted as the basic unit to make adaptation decisions~\cite{socc21-llama,rtas22-fa2}. 
%  Moreover,  due to the high variance of execution performance, runtime adaptation requires to carry out function by function, i.e.,  discrete adaptation.
%  This, however, can easily lead to a sub-optimal (analyzed in~\S~\ref{sec:synthesizer:generate}).
% Our insight is to \emph{introduce a metric (i.e., resilience detailed in \S~\ref{sec:profilier}) to quantify the inter-correlation as well as a heuristic design (i.e., heavier head explained in \S~\ref{sec:synthesizer:generate})  to calibrate the sub-optimal,  such that resource adaptation can explore higher resource efficiency without SLO guarantee}.

% In particular, latency percentiles (introduced by the profiling)  involves resource adaptation as a new knob.
% Specifically, higher percentile earns  stronger guarantees in SLOs but may be highly prone to resource over-allocation because of its latency over-estimation, impairing resource efficiency.
% In contrast, decreasing percentiles offers the opportunity to explore higher resource efficiency, but suffers the risk of timeout, i.e., execution latency beyond specified time budget, and  may thus incur  SLO violations.
% Here, our insight is to \emph{moderately explore percentiles (detailed in~\S~\ref{sec:synthesizer:generate}), where head functions of  (sub-)workflows can explore lower percentiles because this creates the opportunity to reap higher resource efficiency while possible timeout can be recovered by subsequent functions' re-adaptive allocation.
% On the other head, non head functions maintain percentiles as 99\%}.
% This can well keep the trade-off between opportunities of exploring higher resource efficiency and risks of SLO violations. 
% Additionally, it effectively shrinks the searching space, benefiting the adaptation with higher time-efficiency.


\textbf{\textit{Tight resource adaptation window.}}
Runtime resource adaptation requires to calculate a new resource allocation decision for the remaining sub-workflow immediately when a function finishes execution. Since serverless functions are typically short-lived (less than 1s on average)~\cite{atc18-peek-bench,socc22-owl,atc20-serverless-in-the-wild,socc23-huawei-cloud}, the window for resource adaptation is quite tight. Assuming the serverless platform will perform the runtime adaptation on behalf of the developer since the platform has access to full runtime information, the resource adaptation decision making should be fast without involving complex calculations and logic or exploring a large space. As discussed before, the serverless platform provider does not have domain knowledge of the serverless workflow. Hence, the developer must pass the necessary information to the serverless platform for runtime adaptation decision making. Our idea is to let the developer synthesize critical hints containing resource allocation rules and options, which the serverless platform provider utilizes to perform runtime resource adaptation. The hints should be highly condensed so the serverless platform can make adaptation decisions quickly enough. 


% Apart from highly varying execution performance, serverless functions are also short-living (less than 1s on average)~\cite{atc18-peek-bench,socc22-owl,atc20-serverless-in-the-wild,socc23-huawei-cloud}, so is the window for adaptive allocation. 
% This variance and volatility calls for a well-preparation of hints for all possible runtime situations while promising them compact and straightforward enough for providers to easily take action.

% Here, our insight is to \emph{holistically synthesize hints in an offline manner, and then utilize the discreteness of adaptive allocation in both decision-making and decision-executing (detailed in~\S~\ref{sec:synthesizer:condense}) to fully condense the hints.
% Finally, hints are warped into a simple and compact table.
% Base on that, providers can accomplish the runtime adaption promptly and properly}.

To demonstrate the potential of runtime resource adaptation incorporating all the above ideas, we take a real-world serverless workflow (explained in \S\ref{exp:setup}) as an example, and evaluate its end-to-end latency (denoted by E2E) and resource consumption (CPU cores).
As illustrated in Figure~\ref{fig:bg:size}, the late-binding (blue triangle) reduces the resource consumption by up to 42.2\% compared with existing early-binding solutions (orange circle), while ensuring SLO guarantees. This highlights the significant gains from runtime resource adaptation. 


\begin{figure}[t!]
\centering
\includegraphics[width=0.45\textwidth]{./figure/motivation/size_early_bind_vs_ours.pdf}
%\vspace{-0.1cm}
\caption{Performance comparison between early-binding (left)~\cite{eurosys19-grandslam} and late-binding (runtime resource adaptation), where the CPU consumption (right) is normalized by the optimal obtained with exhaustive search.} 
%\vspace{-0.3cm}
\label{fig:bg:size}
\end{figure}

   
	







\section{Concept Development}
We introduce the concept of \modeourstxt. At a high level, we take inspiration from mobile microtasking UIs and integrate a local response interface directly into the incoming email. Overall, our final design is situated in the  unexplored space in between sentence-level and message-level approaches for AI support.


\begin{figure*}
    \centering
    \includegraphics[width=0.5\textwidth]{figures/formative_prototype}
    \caption{Replying to an email with our first prototype: \textit{(1)} In the first screen, users insert responses directly while reading the email. Tapping on a sentence opens a response widget, with a text box where users enter a response or a prompt that affects the sentence suggestions below. \textit{(2)} After adding local responses, users can edit their reply on a second screen, by reordering paragraphs via drag-and-drop, by deleting paragraphs via swiping left-to-right, and by manual editing via the integrated keyboard.  \textit{(3)} On the third screen, users can finalise the reply before sending it.}
    \Description{This Figure shows how to reply to an email with our first prototype: On the left it shows the first screen, where users insert responses directly while reading the email. Tapping on a sentence opens a response widget, with a text box where users enter a response or a prompt that affects the sentence suggestions below. After adding local responses, users can edit their reply on a second screen, which is shown in the centre of the figure. By reordering paragraphs via drag-and-drop, swiping left-to-right to delete paragraphs, and via manually editing using the integrated keyboard user can edit their reply. On the third screen, which is shown on the right, users to finalise the reply, before sending it.
    }
    \label{fig:BA_prototype}
\end{figure*}

\subsection{Design Goals and Rationale} 
With insights from the literature (\cref{sec:related_work}), we designed our system with several goals in mind. For each goal, we briefly mention our approach here, with more details on its final realisation in \cref{sec:implementation}.
\begin{enumerate}[leftmargin=*]
    \item \textbf{Human decides, AI supports:}
    \label{dg:humandecides}
    The user should be able to make all important decisions, while AI supports these. 
    In our design, the user selects the sentences they want to reply to. 
    The system then suggests response sentences, designed to offer a mix of positive, neutral, and negative responses. 
    \item \textbf{The user stays in control:}
    \label{dg:control}
    The user should stay in control of their reply. 
    The AI should not make unnoticed or unwanted adjustments. 
    Our system does not change text unless requested and the user can always edit the reply before sending it.
    \item \textbf{Support mobile microtasking:}
    \label{dg:microtasking}
    The user should be able to leverage microtasking principles for mobile email replies. %
    Our design provides the surrounding email as context while entering reply text and thus shifts from recall to recognition by eliminating the need to remember the email or scroll back to it.%
    \item \textbf{Support diverse workflows -- with and without AI:}
    \label{dg:workflows}
    In each situation, users should be able to answer in their preferred way.
    Our skippable components offer flexibility.
    Even without AI, users get supportive microtasking structure.
    Conversely, users can choose to rely on AI text to respond fast, with little typing. %
\end{enumerate}





\subsection{First Prototype}
We implemented the concept as an Android app with React Native.\footnote{\url{https://reactnative.dev/}}
At this point, the prototype had our sentence-based mode as shown in \cref{fig:BA_prototype}, with three screens.
The first screen (\cref{fig:BA_prototype} left) showed the email for users to select sentences via touch. Each selection triggered a card view that displayed AI suggestions and a text box for entering text (as a manual response or as a prompt to refine the suggestions).
The LLM always generated two positive, two negative, and two neutral answering options.
One positive and one negative suggestion were shown on the first page, if possible. Users could click on the arrows on the sides to access the others.
A second screen (\cref{fig:BA_prototype} centre) supported manual editing of paragraphs, inserting new ones and/or reordering them via drag and drop.
Finally, a third screen showed the result in a standard text editor view for a last check and final edits, if necessary (\cref{fig:BA_prototype} right). %


\subsection{Formative Study}\label{sec:formative_study}
We conducted a first study to understand how users perceive and interact with our concept, and to inform a design iteration. We recruited 17 participants (1 female, 15 male, 1 preferred not to disclose) from our university network. The study followed our institute's regulations, including information on goals, process, data recording, opt-out and consent.

Participants used our prototype on their own phones. They received a tutorial beforehand. The app did not integrate with actual email accounts to preserve privacy. Instead, it simulated to receive two emails per day, for five days. \revision{These emails were quite long, ranging from 140 to 491 words per email (median: 227), to allow participants to test the prototype extensively.} People were asked to respond to the emails in a reasonable time frame. 

After each reply, the app displayed three 5-point Likert items (``The AI tool was helpful'', ``The AI tool helped me reply to the email faster'', ``The AI tool helped me write a better reply'') and space for open feedback. Following the final email, participants completed a questionnaire about their overall experience and demographics.


\subsection{Results}\label{sec:formative_study_results}
The median time of engaging with each email task in our study was 6.9 minutes, including the time taken to enter feedback. 
People accepted 9.17 suggestions per email.
Nearly \pct{80} of accepted suggestions were accepted without making use of the text input for refinement. %
In \pct{30} of emails, participants composed the email entirely with suggestions without edits afterwards.
When they indeed made edits, the most common ones we identified through manual coding were the following: On the first screen, they removed text (25 instances), added information (11), changed details (8), shortened text (6), and added salutations (5) or closing statements (5). 
On the second screen, they reordered or merged sentences and paragraphs (35), shortened text (10), and changed minor details (8). 
They made similar edits on the third screen. %





The Likert results (\cref{fig:likert_items_formative_study} \revision{in \cref{sec:appendix_extra_figures}}) indicated that participants found the AI to be helpful and that it supported them in writing the replies. They felt in control of the email content and found the suggestions to make sense and not be distracting. They generally agreed that the approach helped them remember to address all parts of an email. However, they were more divided on whether they overall preferred the step-by-step process or the traditional one-step approach for replying to emails.

Open in-app feedback was provided by 14 out of 17 participants. Positive aspects mentioned there and in the final questionnaire included ease of use, faster replies, the quality and inspirational potential of AI suggestions, and an overall improved workflow. 

Negative aspects included slow AI response times, quality of suggestions (e.g. too short or not aligning with their input), minor bugs (e.g. failure to load suggestions), and the number of steps (e.g. some suggested to merge the last two screens into one).  

The final questionnaire asked people to reflect on their workflow with our app. They reported different strategies, such as generating custom replies with keywords, reading the entire email before replying, or reviewing generated suggestions first. Some manually merged or adjusted AI-generated text, while others used it as is. 
The final questionnaire also included the System Usability Scale (SUS)~\cite{brooke1996sus}. The mean score was 78.67 (``very good'', \revision{details in \cref{fig:sus_items_formative_study} in \cref{sec:appendix_extra_figures}}).












\subsection{Prototype Iteration}
\label{ssec:protiter}
In summary, the findings from this first study indicated that participants appreciated our concept as it helped them to write fast and high-quality replies with AI, while still feeling in control. It also revealed individual approaches when answering emails and interacting with the suggestions. Based on the study insights, we made the following concrete changes to our prototype:

\begin{itemize}[leftmargin=*]
    \itemsep.2em
    \item \textit{Reduced number of steps:} We removed the second screen (\cref{fig:BA_prototype} centre) %
    and direly offered the third one for free text editing and finalising. Some participants suggested this and they overall made very similar edits across these two screens.
    \item \textit{Added optional improvement pass:} %
    \revision{We added an ``improve email'' button to the final screen to better support users' varying strategies and preferences for answering emails with AI. We observed that some participants manually edited their emails to create transitions between individual paragraphs generated on the first screen. The \imppass{} feature automates this process, adding missing greetings, sign-offs, and correcting grammar and spelling (see \cref{subsec:appendix_improve_email_prompt} for the used prompt)}. %
    \item \textit{Faster suggestion generation:} We switched from GPT-3.5 Turbo\footnote{\url{https://platform.openai.com/docs/models}} to Llama 3 8B Instruct\footnote{\url{https://huggingface.co/meta-llama/Meta-Llama-3-8B-Instruct}}\cite{llama3modelcard}, which we hosted locally to reduce latency and avoid request failures. \revision{While we did not conduct an in-depth evaluation of the models, we compared a set of generations qualitatively and found that the output quality was similar for our use case. Related, we envision that real-world applications could rely on smaller models that can be executed on devices locally to avoid the need to send private email content to a model provider.}%
    \item \textit{Refined suggestions:} We refined our prompting templates to improve suggestions, even with the smaller model. Our new prompts included more context, i.e. all sentence-level replies that were already given up to this point.  
    \item \textit{\revision{Port to a React web app}:} \revision{We ported the app from React Native to a React web app and optimised it. This eased access for participants, as running a React Native prototype required several steps for setup. Instead, a React web app can be accessed via a web browser on any smartphone.}%
\end{itemize}

In the rest of the paper, we always refer to the improved prototype. We next describe it in detail.


\begin{figure*}[t]
    \centering
    \includegraphics[width=\linewidth]{figures/local_response_prompting}
    \caption{The text suggestions in the local response widget are flexible:  \textit{(A)} Users get suggestions without any input.  \textit{(B)} \revision{Suggestions} can be adapted and refined by entering text, for example keywords or a draft snippet. In all cases, suggestions are generated with an LLM based on the text of the incoming email and all local responses that the user has entered so far, \revision{even if responses have been} added to later parts of the email \revision{first}. \revision{In \textit{(C)}, for example, the suggested title of the idea pitch is generated based on the information about the project that the user has already entered in local responses below.} Note that suggestions are paginated, with three pages of two suggestions each.}
    \Description{
    This figure shows screenshots of our user interface displaying an incoming email. The figure is divided into three sections:
    A) Left Section: Email-driven Suggestions without User Input
    The selected sentence in the incoming email reads: "Please feel free to tell me any ideas what we could get her!"
    Below the email, there is an empty text field for optional user input.
    Two AI-generated responses are shown beneath the text field: one suggests a piece of jewellery as a gift, while the other does not offer any ideas.
    B) Centre Section: Prompt-driven Suggestions with User Input
    The same sentence from the email is selected: "Please feel free to tell me any ideas what we could get her!"
    This time, the keywords "balloon ride" are entered into the text field below.
    As a result, the AI-generated suggestions include the idea of a balloon ride in both proposed texts.
    C) Right Section: AI Suggestions Respecting Existing Responses
    One sentence from the incoming email is selected, and the text field below remains empty.
    The AI-generated responses incorporate information from existing responses elsewhere in the email.
    Additionally, all AI suggestions are paginated, with three pages of two suggestions each, as indicated by arrows next to the suggestions.}
    \label{fig:local_response_prompting}
\end{figure*}



\section{Implementation}
\label{sec:implementation}
We implemented a frontend and backend, which preprocessed emails, logged user data, and generated responses. 

\subsection{Frontend}
We implemented our web app with the React\footnote{\url{https://react.dev/}} framework.

\subsubsection{Display of the Incoming Email}
This view matches standard mobile email UIs: It includes the sender's name and picture, the email subject, and the main text body (\cref{fig:teaser} left). 
The user can select sentences in the incoming email by tapping on them (cf. design goal \ref{dg:humandecides}: \revision{Human decides, AI supports}; and goal \ref{dg:microtasking}: \revision{Support mobile microtasking}). This opens the local response widget (\cref{sec:impl_local_response_widget}).
The ``Finalize Reply'' button at the bottom of the UI switches to the next screen (\cref{fig:teaser} centre), which we describe in \cref{sec:impl_finalize}.
In accordance with design goal \ref{dg:workflows} \revision{(Support diverse workflows -- with and without AI)}, no interaction with any sentence or AI feature is required before proceeding to this next screen.



\subsubsection{Local Response Widget}\label{sec:impl_local_response_widget}

This UI widget is inserted into the email text below the user's selected sentence. It comprises of a text field (\cref{fig:teaser} C) and a paginated card view that shows text suggestions (\cref{fig:teaser} D). In the text field, users can enter both manual responses or prompts to refine these suggestions (\cref{fig:local_response_prompting}). 

Concretely, the widget offers six suggestions (2 positive, 2 negative, 2 neutral), showing two at once. The system aims to show one positive and one negative response on the first page, if possible. \revision{This was motivated by findings on positivity bias in AI-generated communication text~\cite{Mieczkowski2021} and to increase the chance of offering a response option fitting to the user's intent (cf.~\cite{Kannan2016smartreply}).} \revision{We realised this by prompting the LLM to do so (see \cref{sec:appendix_sentence_without_input_prompt}). Concretely, the variable ``attribute'' in the prompt template was replaced with \textit{accepting}, \textit{declining}, and \textit{neutral} to generate varying suggestions. In our tests, we observed that this simple prompting approach worked well and that it did not negatively impact generated suggestions in cases where these attributes do not apply (e.g. our ``cat'' example in \cref{fig:teaser}D).} %
Users can navigate through suggestions using the adjacent arrow buttons. They can accept a suggestion by tapping on it.

The widget has two states -- open and collapsed (\cref{fig:teaser} A, B): 
It is collapsed by tapping the currently selected sentence again, by selecting a different sentence, by accepting a suggestion, or by clicking on the check mark in the top right corner. When the text field is empty, the check mark transforms into a trash icon to delete the local reply. Multiple widgets can be in the collapsed state throughout the email but only one widget at a time can be open and in focus. %
A widget's text is shown in the collapsed state. This allows users to keep track of all their local replies so far. Tapping on a collapsed widget opens it again for further editing. 






\subsubsection{Finalising the Reply}\label{sec:impl_finalize}
This view  (\cref{fig:teaser} centre) shows the current state of the reply after the local response step. That is, it displays any text entered in response to individually selected sentences together in a single text field.

Users can manually adjust this text and/or tap the ``Improve'' button to request the AI to enhance the email. 
This \imppass{} feature is realised with a prompt \revision{(see \cref{subsec:appendix_improve_email_prompt})} to the underlying LLM to correct spelling and grammar, refine wording, and add missing salutations or regards while adhering to both the incoming email's content and the existing reply text. 

If no text is entered first, the ``Improve'' button acts as message-level support, generating a reply based on the incoming email's text and the current input on this screen. For example, a user could skip the local response and enter a prompt here, effectively realising a message generation workflow similar to the industry default (\cref{sec:related_work_current_products}). This flexibility contributes to our design goal \ref{dg:workflows} \revision{(Support diverse workflows – with and without AI).}

When the user is satisfied with their reply, the email can be sent by tapping the ``Send Email'' button at the bottom of this screen.


\subsubsection{Improved Email Pop-up}\label{sec:impl_imppass}
The \imppass{} feature does not change the user's text directly, in line with our design goal \ref{dg:control} \revision{(The user stays in control).}
Instead, the new text is shown in a pop-up view with formatting familiar from ``track changes'' in text editors (\cref{fig:teaser} right). 
Users can approve these changes, which updates the text, or discard them (cf. design goal \ref{dg:humandecides}: \revision{Human decides, AI supports}.) Further editing after acceptance and/or requesting improvements repeatedly is possible. 


\subsection{Backend}
Our prototype's backend has three purposes: 
(1) It \textit{hosts the web app} on a Next.js server. 
(2) It provides \textit{email preprocessing}, which handles tasks such as sentence-splitting and making API calls to the LLM. 
(3) It \textit{hosts the LLM}. 

We experimented with several models and APIs and discussed factors such as latency, stability of service, and subjective response quality in meetings with all authors. Based on this, we used the Llama 3 8B Instruct \cite{llama3modelcard} model for the main study. 

Similarly, we iterated over several prompting approaches for the text generation features. Overall, this resulted in a few-shot approach, providing the model with several input-output examples to generate fitting responses. 
As an overview, we use the following prompt templates (details in \cref{sec:appendix_prompts}):

\paragraph{Sentence-level support, without user input:}
We prompted six suggestions for the sentence selected in the email (2 positive, 2 neutral, 2 negative). This balanced the options, following related work~\cite{Kannan2016smartreply}, as the LLM favoured positive responses in our tests.

\paragraph{Sentence-level support, with user input:} 
This was identical to the above case but now it included the user's input in addition to the selected sentence. %
We emphasised alignment with the user's sentiment (e.g. no negative suggestions if the user had entered ``yes'').


\paragraph{Message-level support:}
We prompted the LLM to answer to the whole email, also by taking into account any current user input, if available. A variation of this was also used for the \imppass{} feature (\cref{sec:impl_imppass}). That prompt emphasised improving the current state of the reply while closely adhering to the information provided by the user. %

\section{Method}
\revision{For the main study, we switched from the field study design of the formative study to a more controlled web-based experiment. Our motivation was to scale the study, to compare interaction designs quantitatively, and to introduce task briefings that would allow us to evaluate if people accept even unfitting suggestions for convenience in the study. Thus, we} conducted a within-subject user study with our \revision{iterated} prototype.
The independent variable \ivmode{} had three levels: \modeoursTxt{} (\modeours) -- our proposed design (\cref{sec:implementation}); \modemailtxt{} (\modemail) -- a one-prompt reply generation design close to currently available UIs (\cref{sec:related_work_current_products}); and writing without any AI features (\modemanual). As dependent variables, we logged interaction metrics and collected subjective feedback via questionnaires.

\subsection{Apparatus}

\subsubsection{Web App}
For \modeours, we hosted our prototype (\cref{sec:implementation}) as a web app, with added study information, study logic, and logging.
We integrated a screen for briefings (\cref{sec:method_emails_briefings}) before each email task and one with four Likert items (\cref{sec:procedure_email_tasks}) after each email task (\cref{fig:briefing_and_feedback} \revision{in \cref{sec:appendix_extra_figures}}).
We used a custom study framework to manage counterbalancing and the flow from consent information to prototypes to surveys.

\subsubsection{Comparative Designs}
We implemented two alternatives for the study: For \modemanual, the app only showed a typical drafting view (\cref{fig:baseline_uis_manual} \revision{in \cref{sec:appendix_extra_figures}}). For \modemail, it was designed similar to the typical UI pattern shown in \cref{fig:current_products} -- it offered a text field for (optional) prompting and displayed the generated text (\cref{fig:baseline_uis_msg} \revision{in \cref{sec:appendix_extra_figures}}). Accepting the text with a button inserted it into a draft view for further editing or sending. Rejecting it allowed users to refine their prompt and generate a new draft.

\subsubsection{Incoming Emails and Reply Briefings}\label{sec:method_emails_briefings}
We prepared nine emails, covering an idea pitch contest, a high school reunion, a sales offer, a lunch meeting, a marketing slogan, proofreading for a friend, a sales report deadline, server access, and a gift idea for a retiring coworker. This set was motivated to cover various plausible email topics, with and without (multiple) questions. It also covered various emails lengths, \revision{ranging from 24 to 155 words (median 57).}%

We also prepared a reply briefing for each email. It provided information relevant for \textit{what} to answer, without specifying \textit{how} to write (e.g. tone, structure). For example, for the high school reunion email, the briefing specified that the user was unavailable on a certain date.
These briefings were \textit{not} given to the LLM, which would have simulated unrealistic ``mindreading''. In contrast, our motivation for the briefings was to assess to what extent participants might accept unfitting suggestions for convenience. %
Moreover, the briefings mimic an email workflow where some information is readily available while details may need to be retrieved. For instance, in the high school example, reading the briefing could be seen as similar to checking a calendar app.


\subsection{Participants}
\label{sec:participants}
We recruited 162 participants through the online platform Prolific.\footnote{\url{https://www.prolific.com/}} We excluded 36 participants from our analysis because they either did not complete all tasks (18 participants) or the logs indicated that a technical issue had occurred (18 participants). Our analyses are based on the remaining \studyTwoN{} participants (83 male, 40 female, 1 non-binary, 2 preferred not to disclose). 
Their age ranged from 18 to 72 years (median 32). %
All were proficient in English (91.3\% native speakers).
Their occupations included both professions where frequent email usage is expected (e.g. IT consultant, project manager) and others (e.g. gardener, waiter).
Participants were compensated with about \pounds 10 per hour.

Most participants reported to answer emails at least once a day (\pct{48.41}) or even more than 10 times a day (\pct{21.43}).
Another \pct{16.67} answer emails at least once a week, \pct{7.94} less than once a week, and \pct{5.56} less than once a month.

Most participants (\pct{85.71}) use their smartphone for answering emails. Many also use a laptop (\pct{82.54}) or desktop computer (\pct{53.97}). Some also use a tablet (\pct{19.05}).
They answer emails at home (\pct{84.92}), at the office (\pct{69.84}), and on the go (\pct{53.17}).
Many answer emails for business (\pct{84.92}) and in a private context (\pct{58.73}).

Only \pct{14.29} reported no previous experience with AI.
Many have used ChatGPT (\pct{72.22}). Many have experience with auto-correction (\pct{51.59}), some also with auto-completion (\pct{26.98}), with word or sentence suggestions (\pct{21.42}), and with Smart Reply (\pct{15.87}).
\revision{An overview of all questions and answer options can be found in \cref{sec:appendix_questionnaires}}.


\subsection{Procedure}\label{sec:procedure}
The study was conducted remotely on participants' own smartphones. Access via Prolific was restricted to one person at a time to balance the load for our server.
The sessions were scheduled for 45 minutes and structured as follows:

\subsubsection{Study Intro}
An introduction page explained the study, including information about GDPR compliance, privacy, data collection, and informed consent, in accordance with our institutional regulations. 
In addition, whenever encountering a UI for the first time, our study framework showed an explanation of its features.

\subsubsection{Email Answering Task}\label{sec:procedure_email_tasks}
Participants were asked to reply to nine given emails, three per \ivmode{} (\modeours, \modemail, \modemanual).
We counterbalanced the email topics and also the order of the UIs with a Latin square design to address potential learning or fatigue effects.

For each email task, the briefing (\cref{sec:method_emails_briefings}) was shown at the start and could be accessed again anytime via the information button in the top right corner of the app (\cref{fig:teaser}).
Participants were instructed to ``consider the information in the briefing for answering the email[s]''.


After submitting each email, participants rated four Likert items: ``The app interface was helpful'', ``The app interface helped me reply to the email quickly'', ``The app interface helped me write a good reply'', and ``I was in control of the content of my reply''. 
They could share comments in a text field.


\subsubsection{Final Questionnaire}
This questionnaire was displayed after the final email task.
Participants provided demographics, selected their favourite UI mode, and explained their choice.
They could also leave comments, questions, and feedback. %


\subsection{Qualitative Analysis}
Here we describe our approach to coding open feedback and analysing email texts.

\subsubsection{Coding of Open Feedback}
We followed Grounded Theory~\cite{corbin1990basics} to analyse participants' open feedback. 
In the open coding round, two researchers independently reviewed the data, identifying and labelling sentences that represent specific ideas and principles. 
We then refined these initial codes by merging and clustering related ones, forming (sub-)categories during an axial coding round. 
Our research team discussed emerging themes, leading to synthesised, overarching labels for the clusters and, in some cases, further split categories to capture more nuanced insights from the feedback.
We repeated this process until reaching consensus.

\subsubsection{Analysis of the Email Replies}
\label{sec:quality_m}
Assessing email quality is complex and subjective, as known from studies on people's preferences (cf. \cite{Liu2022aimailperception, Robertson2021cantreply}).
Therefore, we use multiple  quality indicators: 
Formal indicators~\cite{reeves2008emailover50, lewi_jones2014email} include the presence of a \textit{salutation} and a \textit{closing statement} (\cref{sec:results_structure}), and proper \textit{spelling and grammar} (\cref{sec:results_errors}).
We also analysed \textit{briefing conformity}, that is, we checked whether replies covered the key information provided in the briefings (\cref{sec:results_briefing}).
Finally, we share our subjective impressions (\cref{sec:result_quality}).

We employed Binary Coding~\cite{miles2013qualitative} to assess the briefing conformity as well as the formal indicators, except spelling and grammar, which we checked using the language-tool-python\footnote{\url{https://pypi.org/project/language-tool-python/}} library. 
Two researchers coded all email replies independently, assigning a ``1'' if the email met the criteria and a ``0'' if it did not. 
Ambiguous emails were flagged for further review. 
In a second round, these were re-evaluated by the research team until reaching consensus.


\subsection{\revision{Statistical Analysis}}\label{sec:appendix_sigtest}


\revision{To declutter our following report,} \cref{tab:lmm_overview} and \cref{tab:lmm_overview2} summarise the statistical analyses \revision{and results} referred to throughout \cref{sec:results}.
We computed (generalised) linear mixed-effects models (LMMs) in R~\cite{R2020}, using the packages \textit{lme4}~\cite{Bates2015} and \textit{lmerTest}~\cite{Kuznetsova2017}. These models accounted for the individual differences between participants and for differences between the incoming emails via random intercepts. 

The models' fixed effects were \ivmode{} and whether the \imppass{} feature was used in \modeours. 
For the model for briefing conformity, we additionally included a predictor for whether the reply was generated without user input, such that the result was generated fully by the LLM based on the incoming email only. In \modemail, this is done by not entering a prompt for the reply generation. In \modeours, this is done by not providing any local response (manual or suggestion) on screen 1, before using the \imppass{} feature on screen 2. 
Pairwise comparisons were computed with the \textit{emmeans} package with Bonferroni-Holm correction.


For the Likert data, we used rank-aligned repeated measures ANOVA  (ART)~\cite{wobbrock2011art} and ART-C contrasts with Bonferroni-Holm correction for the follow-up analysis~\cite{elkin2021artc}.

We report significance at p~<~0.05. 


\begin{table*}[t!]
\centering
\footnotesize
\newcolumntype{L}{>{\raggedright\arraybackslash}X}
\newcolumntype{P}[1]{>{\raggedright\arraybackslash}p{#1}}
\renewcommand{\arraystretch}{1.4}
\setlength{\tabcolsep}{4pt}
\begin{tabularx}{\linewidth}{lP{2.75em}P{5em}P{22em}P{7em}L}
\toprule
    &
    \textbf{Section} &
    \textbf{Aspect}\newline and model &
    \textbf{Predictors} (baseline: \modemanual) &
    \textbf{Pairwise comparisons} &
    \textbf{Takeaways in words}\newline(only considering sig. results) \\ \midrule
1 &
    \ref{sec:results_time} 
    &
    Completion time\medskip\newline
    \textit{LMM on seconds}
    &
    \modeours{}	$\downarrow$ \newline 
    \deemph{(\lmmci{-4.34}{12.00}{-27.89}{19.21}{=.718})}\medskip\newline 
    \modemail{} $\downarrow^*$ \newline 
    \deemph{(\lmmci{-70.05}{7.72}{-85.20}{-54.90}{<.0001})}\medskip\newline 
    \Imppass{} feature used $\downarrow$ \newline 
    \deemph{(\lmmci{-15.18}{13.23}{-41.14}{10.79}{=.252})}
    &
    \modeours{} vs \modemanual{} \deemph{(\posthoc{-4.34}{=.718})} \medskip\newline 
    \modemail{} vs \modemanual{} \deemph{(\posthoc{-70.05}{<.0001})} \medskip\newline 
    \modeours{} vs \modemail{} \deemph{(\posthoc{65.71}{<.0001})}
    &
    People finished replying faster with \modemailtxt{} than without AI (by 70 seconds on average). \modemailTxt{} was also faster than \modeourstxt{} (by 66 seconds on average).
    \\
    \midrule 
2 &
    \ref{sec:results_speed}
    &
    Writing speed\medskip\newline
    \textit{LMM on characters per second}
    &
    \modeours{} $\uparrow$ \newline 
    \deemph{(\lmmci{.61}{.56}{-.49}{1.71}{=.278})}\medskip\newline 
    \modemail{} $\uparrow^*$ \newline 
    \deemph{(\lmmci{5.16}{.37}{4.43}{5.88}{<.0001})}\medskip\newline 
    \Imppass{} feature used $\uparrow^*$ \newline 
    \deemph{(\lmmci{2.48}{.61}{1.29}{3.68}{<.0001})}
    &
    \modeours{} vs \modemanual{} \deemph{(\posthoc{.61}{=.278})} \medskip\newline 
    \modemail{} vs \modemanual{} \deemph{(\posthoc{5.16}{<.0001})} \medskip\newline 
    \modeours{} vs \modemail{} \deemph{(\posthoc{-4.55}{<.0001})}
    &
    People produced more characters per second with \modemailtxt{} (5.2 chars more per s) and if they used the \imppass{} feature in \modeourstxt{} (2.5 chars more per s).
    \\
    \midrule
3 &
    \ref{sec:results_keystrokes}
    &
    Manual typing\medskip\newline
    \textit{GLMM (Poisson) on keystroke counts}
    &
    \modeours{}	$\downarrow^*$ \newline 
    \deemph{(\lmmci{-.65}{.009}{-.66}{-.63}{<.0001})}\medskip\newline
    \modemail{}	$\downarrow^*$ \newline 
    \deemph{(\lmmci{-.86}{.005}{-.87}{-.85}{<.0001})}\medskip\newline 
    \Imppass{} feature used $\downarrow^*$ \newline 
    \deemph{(\lmmci{-.06}{.011}{-.08}{-.04}{<.0001})}
    &
    \modeours{} vs \modemanual{} \deemph{(\posthoc{-.65}{<.0001})} \medskip\newline 
    \modemail{} vs \modemanual{} \deemph{(\posthoc{-.86}{<.0001})} \medskip\newline 
    \modeours{} vs \modemail{} \deemph{(\posthoc{.21}{<.0001})}
    &
    People needed fewer keystrokes with AI than without it; concretely, even fewer with \modemailtxt{} (\pct{58} decrease) than with \modeourstxt{} (\pct{48} decrease). Using the \imppass{} feature in \modeourstxt{} reduced them further for that UI (\pct{5.9} decrease).
    \\
    \midrule 
4 &
    \ref{sec:results_lengths}
    &
    Reply lengths\medskip\newline
    \textit{GLMM (Poisson) on character counts}
    &
    \modeours{} $\uparrow^*$ \newline 
    \deemph{(\lmmci{.24}{.0060}{.22}{.25}{<.0001})}\medskip\newline
    \modemail{} $\uparrow^*$ \newline 
    \deemph{(\lmmci{.57}{.0037}{.56}{.58}{<.001})}\medskip\newline 
    \Imppass{} feature used $\uparrow^*$ \newline 
    \deemph{(\lmmci{.32}{.0063}{.31}{.33}{<.0001})}
    &
    \modeours{} vs \modemanual{} \deemph{(\posthoc{.24}{<.0001})} \medskip\newline 
    \modemail{} vs \modemanual{} \deemph{(\posthoc{.57}{<.0001})} \medskip\newline 
    \modeours{} vs \modemail{} \deemph{(\posthoc{-.33}{<.0001})}
    &
    People wrote longer replies with AI, even more so with \modemailtxt{} (exp($\beta$)=exp(.57)=1.77 i.e. \pct{77} increase) than with \modeourstxt{} (\pct{27} increase). Using the \imppass{} feature in \modeourstxt{} increased it further for that UI (\pct{38} increase).
    \\
    \midrule 
5 &
    \ref{sec:results_errors}
    &
    Error rates\medskip\newline
    \textit{LMM on errors per character}
    &
    \modeours{}	$\downarrow^*$ \newline 
    \deemph{(\lmmci{-.0011}{.0004}{-.0018}{-.0004}{=.0024})}\medskip\newline 
    \modemail{}	$\downarrow^*$ \newline 
    \deemph{(\lmmci{-.0020}{.0002}{-.0025}{.0015}{<.0001})}\medskip\newline 
    \Imppass{} feature used $\downarrow^*$ \newline 
    \deemph{(\lmmci{-.0012}{.0004}{-.0020}{-.0005}{=.0012})}
    &
    \modeours{} vs \modemanual{} \deemph{(\posthoc{-.0011}{=.0048})} \medskip\newline 
    \modemail{} vs \modemanual{} \deemph{(\posthoc{.-0020}{<.0001})} \medskip\newline 
    \modeours{} vs \modemail{} \deemph{(\posthoc{.0009}{=.0096})}
    &
    People wrote emails with lower error rates with AI than without, even lower with \modemailtxt{} than with \modeourstxt. Using the \imppass{} feature in \modeourstxt{} reduced the error rates further for that UI.
    \\
    \midrule
6 &
    \ref{sec:results_email_similarity}
    &
    Email similarity\medskip\newline
    \textit{LMM on cosine similarity of SBERT embeddings}
    &
    \modeours{} $\uparrow^*$ \newline 
    \deemph{(\lmmci{.09}{.0026}{.09}{.10}{<.0001})}\medskip\newline 
    \modemail{} $\uparrow^*$ \newline 
    \deemph{(\lmmci{.17}{.0023}{.17}{.18}{<.0001})}\medskip\newline 
    \Imppass{} feature used $\uparrow^*$ \newline
    \deemph{(\lmmci{.07}{.0029}{.07}{.08}{<.0001})}
    &
    \modeours{} vs \modemanual{} \deemph{(\posthoc{.09}{<.0001})} \medskip\newline 
    \modemail{} vs \modemanual{} \deemph{(\posthoc{.17}{<.0001})} \medskip\newline 
    \modeours{} vs \modemail{} \deemph{(\posthoc{-.08}{<.0001})}
    &
    People wrote semantically more similar (i.e. less diverse) emails with AI than without, more so with \modemailtxt{} than with \modeourstxt{}. For the latter, using the \imppass{} feature contributed to increasing the similarity of emails.
    \\
    \midrule
7 &
    \ref{sec:results_lexical_diversity}
    &
    Lexical diversity\medskip\newline
    \textit{LMM on the distinct2 metric}
    &
    \modeours{} $\downarrow^*$ \newline 
    \deemph{(\lmmci{-.03}{.0043}{-.04}{-.03}{<.0001})}\medskip\newline 
    \modemail{} $\downarrow^*$ \newline 
    \deemph{(\lmmci{-.02}{.0029}{-.03}{-.01}{<.0001})}\medskip\newline 
    \Imppass{} feature used $\uparrow^*$ \newline
    \deemph{(\lmmci{.01}{.0045}{.003}{.02}{=.0087})}
    &
    \modeours{} vs \modemanual{} \deemph{(\posthoc{-.03}{<.0001})} \medskip\newline 
    \modemail{} vs \modemanual{} \deemph{(\posthoc{-.02}{<.0001})} \medskip\newline 
    \modeours{} vs \modemail{} \deemph{(\posthoc{-.015}{=.0011})}
    &
    People wrote emails with lower lexical diversity (measured as: unique bigrams / number of words) with AI than without it, even more so with \modeourstxt{} than with \modemailtxt{}. Using the \imppass{} feature in \modeourstxt{} closed this gap.
    \\
  \bottomrule
\end{tabularx}
\caption{Overview of significance tests with links to the section, tested measure, predictors, pairwise comparisons, and written interpretation. The arrows indicate if predictors increase ($\uparrow$) or decrease ($\downarrow$) the outcome aspect, with an asterix if these impacts are significant (*).}
\Description{Overview of significance tests with links to the section, tested measure, predictors, pairwise comparisons, and written interpretation. For each statistical test it describes the Section, Aspect and model, Predictors (baseline: NoAI), Pairwise comparisons, and Takeaways in words (only considering sig. results).}
\label{tab:lmm_overview}
\end{table*}




\begin{table*}[t!]
\centering
\footnotesize
\newcolumntype{L}{>{\raggedright\arraybackslash}X}
\newcolumntype{P}[1]{>{\raggedright\arraybackslash}p{#1}}
\renewcommand{\arraystretch}{1.4}
\setlength{\tabcolsep}{4pt}
\begin{tabularx}{\linewidth}{lP{2.75em}P{5em}P{22em}P{7em}L}
\toprule
    &
    \textbf{Section} &
    \textbf{Aspect}\newline and model &
    \textbf{Predictors} (baseline: \modemanual) &
    \textbf{Pairwise comparisons} &
    \textbf{Takeaways in words}\newline(only considering sig. results) \\ \midrule
1 &
    \ref{sec:results_briefing}
    &
    Briefing conformity\medskip\newline
    \textit{GLMM (Binomial) on binary conformity coding}
    &
    \modeours{} $\downarrow^*$ \newline 
    \deemph{(\lmmci{-.77}{.3107}{-1.38}{-.16}{=.013})}\medskip\newline 
    \modemail{} $\downarrow$ \newline 
    \deemph{(\lmmci{-.04}{.2417}{-.52}{.43}{=.857})}\medskip\newline
    \Imppass{} feature used $\uparrow$ \newline
    \deemph{(\lmmci{.25}{.3310}{-.40}{.90}{=.444})}\medskip\newline
    Full reply generated without input $\downarrow^*$ \newline
    \deemph{(\lmmci{-1.70}{.3467}{-2.38}{-1.02}{<.0001})}
    &
    \modeours{} vs \modemanual{} \deemph{(\posthoc{-.77}{=.040})} \medskip\newline 
    \modemail{} vs \modemanual{} \deemph{(\posthoc{-.04}{=.857})} \medskip\newline 
    \modeours{} vs \modemail{} \deemph{(\posthoc{-.73}{=.040})}
    &
    With \modeours, people wrote emails that had a higher chance to miss a key aspect of the study briefing than those written with \modemail{} or manually (\pct{23} of emails missed it for \modeours{} vs \pct{18} for \modemail{} vs \pct{13} for \modemanual). People's prompting behaviour had a larger impact here: Across \modeours{} and \modemail{}, generating a full reply without any own input (83 emails in the data) missed a key aspect of the briefing in half of the cases (\pct{49}).
    \\
    \midrule
2 &
    \ref{sec:results_workflows}
    &
    Skipping local response \medskip\newline
    \textit{GLMM (Binomial) on skipped yes/no}
    &
    Length of incoming email\newline
    (num. standardised words, i.e. characters/5) $\downarrow^*$ \newline
    \deemph{(\lmmci{-.025}{.010}{-.050}{-.004}{=.0171})}
    &
    -
    &
    Each additional word (defined as 5 additional characters) in the incoming email is associated with a \pct{2.46} decreased chance of skipping the local response step in \modeours{}. Skipping is defined as not entering any text on the local response screen of that UI.
    \\
  \bottomrule
\end{tabularx}
\caption{Further significance tests with links to the section, tested measure, predictors, pairwise comparisons, and written interpretation. The arrows indicate if predictors increase ($\uparrow$) or decrease ($\downarrow$) the outcome aspect, with an asterix if these impacts are significant (*).}
\Description{Further significance tests with links to the section, tested measure, predictors, pairwise comparisons, and written interpretation. For each statistical test it describes the Section, Aspect and model, Predictors (baseline: NoAI), Pairwise comparisons, and Takeaways in words (only considering sig. results).}
\label{tab:lmm_overview2}
\end{table*}

\subsection{Qualitative Results}
After generating a large number of samples (\textgreater$200,000$), we apply quality checks to remove noisy generations, resulting in approximately $35,225$ samples. See \autoref{fig:counterfactual_examples}  for examples of counterfactual image generations and \autoref{fig:perturbed_examples} for parametric and source conflicts.

Two raters independently labeled a subset of $100$ quality-checked generations for each category of conflicts to determine if the new label (\retlabel or \updatedlabel) matches the perturbed image---see label quality ratings in \autoref{tab:dataset_description}. Counterfactuals have a higher quality rating (\textgreater90\%). Parametric (76\%) and source conflicts (82\%) produce more noisy generations which we attribute to the increased difficulty in replacing an object versus removing it. Raters only disagreed on a small fraction of samples (30/300), while a Cohen's Kappa of 0.45 reflects that disagreements happened only on lower quality generations \cite{delgado2019cohen}.
% However, since these disagreements are on the  ~\cite{delgado2019cohen} this results in a Cohen's Kappa of 0.45. 

%The low Kappa score is a consequence of class imbalance ~\cite{delgado2019cohen} which, for instance, can be seen from the high quality rating score (87-93\%) for counterfactual generations.


% \begin{figure*}
%     \centering
%     \vspace{-4mm}
%     \includegraphics[width=0.8\linewidth]{figures/results/finetuning_eval.pdf}
%     \vspace{-2mm}
%     \caption{Evaluation of baseline (-Base) and \segsub finetuned (-Ft) model accuracy on counterfactual and source conflicts (higher is better). Evaluation on original samples from VQAv2, OK-VQA, and WebQA datasets shows that finetuning does not result in performance regression on these tasks (except on WebQA two-image samples). Finetuned models outperform baselines across all types of knowledge conflict.}
%     \vspace{-4mm}
%     \label{fig:finetuning_results}
    
% \end{figure*}


\vspace{-2mm}
\paragraph{Parametric Conflicts}
While Phi3 model does benefit somewhat from finetuning (4\% drop in parametric response rate), Qwen2 and Llava are unaffected. Parametric response rates are low across the board ($\sim$20\%, \autoref{fig:parametric_effect}), showing that baseline models are already robust to conflicts between input sources and parametric memory.

\begin{figure}
    \centering
   
    \includegraphics[width=0.8\linewidth]{figures/results/parametric_plot.pdf}
    % \includegraphics[width=0.9\linewidth]{figures/results/perturbed_1_vs_2_image.pdf}
    \vspace{-4mm}
    \caption{Parametric effect analysis: how often does the model predict the original label for perturbed images? Lower is better, implying a reduced parametric effect.}
    \label{fig:parametric_effect}
    \vspace{-2mm}
\end{figure}


\subsection{Quantitative Results}
In \autoref{fig:finetuning_results} we find that baseline VLMs fail to acknowledge counterfactual conflicts (Counter) and source conflicts (Source). Finetuning mitigates this across every dataset. The resulting finetuned models (-Ft) outperform the baseline models (-Base) on perturbed samples. Finetuning has some benefit on the original samples (Original) for VQA and WebQA counterfactual sources, but a large performance regression is apparent for samples with source conflicts in WebQA.

%\begin{figure}[!ht]
%    \centering
%     \includegraphics[width=0.9\linewidth]{figures/results/perturbed_1_vs_2_image.pdf}
%     \caption{Finetuning on samples generated to induce parametric conflicts improves performance on original datasets for one image questions, but regresses it for the two image multihop case. \todo{update image or caption, currently inconsistent}}
%     \label{fig:one_two_image}
% \end{figure}


%Baseline models are capable of predicting the generated labels for samples generated to induce parametric conflicts (\autoref{fig:one_two_image}). Furthermore, finetuning on these augmented samples does not lead to a substantial improvement in model performance on the original labels. 

%As this perturbation category can have either one or two images per sample, we partition our evaluation into a single image set and a two image set. We find that for finetuned models, there is a gulf in performance between one and two image samples (\autoref{fig:one_two_image}). We attribute this performance gap to the increased difficulty in multihop reasoning. Rather than learning to determine if a two image sample has a valid answer, finetuned models over-predict the \retlabel, indicating a source conflict. Thus, performance is deteriorated for original two image samples from the WebQA dataset.



% As such, training models with images that differ only in the concept in question does not seem to improve performance, at least for the color and shape categories. We qualify this by the observation that questions concerning color and shape attributes are well represented in popular VQA datasets, including all three datasets in this study. As such, adding perturbed samples in these categories will have diminishing returns at best. 

% It is worth noting that these samples are still necessary to scaffold the model in learning the concept of knowledge conflicts so that the model cannot shortcut to predicting that every set of generated samples with 2 images is a conflict.

\vspace{-2mm}
\paragraph{Source Conflicts}
For WebQA samples with source conflicts, the finetuned models have extremely low accuracy on original samples. This is a result of the finetuned models failing to predict the old label and instead overpredicting the \retlabel when presented with two images. %We attribute this to the high amount of noise with knowledge conflict generations achieving a quality rating of only 82\% (\autoref{tab:dataset_description}), and the multihop nature of the knowledge conflict task. 
Interestingly, instead of generating an `acknowledgement' response, baseline models tend to predict one of the two incorrect answers---either the original label (for the unperturbed image) or \updatedlabel (for the perturbed image)---uniformly at random. 
% See \autoref{tab:webqa_conflicts} in the supplementary.


\begin{figure*}
    \centering
    \vspace{-4mm}
    \includegraphics[width=0.8\linewidth]{figures/results/finetuning_eval.pdf}
    \vspace{-2mm}
    \caption{Evaluation of baseline (-Base) and \segsub finetuned (-Ft) model accuracy on counterfactual and source conflicts (higher is better). Evaluation on original samples from VQAv2, OK-VQA, and WebQA datasets shows that finetuning does not result in performance regression on these tasks (except on WebQA two-image samples). Finetuned models outperform baselines across all types of knowledge conflict.}
    \vspace{-4mm}
    \label{fig:finetuning_results}
    
\end{figure*}



\vspace{-2mm}
\paragraph{Counterfactual Conflicts}
Baseline models perform poorly on counterfactual conflicts, with no model achieving more than 30\% accuracy. Since these models are not trained to return the \retlabel, we consider any admission of failure by the model as a \retlabel. These baseline models are sometimes able to determine when an image lacks the information required to answer a question, they are not robust to these samples. Finetuning on enables these models to identify counterfactual conflicts with high accuracy, without degrading performance on the original datasets. Additionally, finetuning provides a 5-10\% performance gain on the original samples from WebQA and VQA datasets.

\begin{figure}
    \centering
    \includegraphics[width=0.9\linewidth]{figures/results/context_scores.pdf}
    \vspace{-3mm}
    \caption{Decreased Accuracy on Counterfactual Conflicts in finetuned VLMs (and GPT4-o-mini) with Increasing Image Contextualization Scores. Baseline unsmoothed data is in the background.}
    \label{fig:context_scores}
    \vspace{-4mm}
\end{figure}


\vspace{-2mm}
\paragraph{Robustness of Counterfactual Conflicts}
We find that finetuned models are robust in detecting randomized counterfactual samples. They are not simply detecting images that have been modified by LaMa to remove objects. The finetuned Qwen2 model predicts \retlabel for 80\% of the randomized counterfactuals sampled from the WebQA dataset. \autoref{tab:natural_counterfactual_results} in the supplementary has further details.

\vspace{-2mm}
\paragraph{Parameter Size}
We find that performance improvements on the evaluation metrics derived from increasing model size have diminishing returns. There exists a gap in performance between SoTA models (i.e. GPT-4o-mini) and the finetuned models (see \autoref{fig:baseline_model_performance} in the supplementary).

\begin{figure}
    \centering
    \begin{subfigure}[b]{0.22\textwidth}
        \centering
        \includegraphics[width=\textwidth]{figures/segmentation/context/baseball_boy.jpeg}  % Replace with your figure file
        \caption{ChatGPT: "There doesn’t appear to be an object clearly visible in his hands."}
        \label{fig:bat_boy}
    \end{subfigure}%
    \hfill
    \begin{subfigure}[b]{0.22\textwidth}
        \centering
        \includegraphics[width=\textwidth]{figures/segmentation/context/bat_removed.png}  % Replace with your figure file
        \caption{ChatGPT: "The batsman in the image is holding a baseball bat as he prepares to swing."}
        \label{fig:bat_batsman}
    \end{subfigure}
    \caption{These counterfactual examples were generated by removing a baseball bat from two different VQA images. When asked 'what is he holding?', ChatGPT only hallucinates in the highly contextualized case (right).}
    \label{fig:baseball_example}
\end{figure}


\paragraph{Image-Question Contextualization}
Intuitively, image-question contextualization relates to contextual cues within an image that provides the models with clues to answer the question, as in \autoref{fig:baseball_example}. We find evidence for a link between image-question contextualization, as approximated by GPT-4o-mini, and accuracy on counterfactual samples. \autoref{fig:context_scores} reveals that models perform poorly in identifying a sample as counterfactual (i.e. lower accuracy of predicting \retlabel) and is more likely to hallucinate on heavily contextualized image question pairs. Interestingly, GPT-4o-mini hallucinates for all of the counterfactual examples given in \autoref{fig:counterfactual_examples}.

For a concrete example, see \autoref{fig:baseball_example}, where both counterfactual examples were generated by removing a baseball bat. Here, a poorly contextualized image question pair features a child standing in a field with the question "what is he holding?" (\autoref{fig:bat_boy}). The only contextual cues as to what the child might have been holding are the generic outdoor setting, and the child's body positioning. Contrasting this in the adjoining sample is a baseball player, adorned in a jersey with his player number printed on the back, in a stadium filled with sporting fans (\autoref{fig:bat_batsman}). ChatGPT recognizes that the child is holding nothing, but hallucinates a bat in the hands of the batsman. Alongside previous works that show a relationship between image context and object detection \citep{beery2018recognition}, these results indicate that contextual cues have a priming effect that induces hallucinations in VLMs for highly contextualized counterfactuals.
% \citep{beer}. 



% \begin{table}[]
%     \centering
%     \begin{tabular}{c|c}
%          &  \\
%          & 
%     \end{tabular}
%     \caption{Caption}
%     \label{tab:qualatitive_examples}
% \end{table}



\section{Discussion}
\subsection{Giving Human-like Skills to AI} 
This study showed that for one form of humor - Gen-Z style Instagram image captioning humor - our AI-written humor was funnier than GPT's native humorsense, and as funny at the top 5 highest rated Instagram captions. We attribute this to a variety of features we added to the system. First, the visual detail extraction was able to find aspects of the image to poke fun at that were often sharper than GPT's joke target and more similar to the Instagram captions' joke target. Second, the narrative extrapolation step allowed the system to broaden its base of relatable joke targets - moving the focus away from making fun of the literal objects in the image, but using them as metaphors for relatable joke targets like relationship disasters, teamwork breakdowns, and the burden of student loans. This opened more creativity possibilities for joke targets. Lastly, we used an LLM-as-judge to rank the outputs accord to Gen-Z humor taste, thus giving the system some notion of the audience. These skills - detail observation, finding analogous and relatable social situations, and modeling the audience through fine tuning - are all considered somewhat ``human.'' Skills like reasoning and chaining are considered more typical of machines. But this shows that machines might be able to approach these more human skills with the right architecture and training.  


% In this paper, we showed that a model of GPT that is enhanced to have 3 human-like skills used in humor 
% \color{red}
% (observation, sense of story, and in-group knowledge) 
% \color{black}
% outperforms standard GPT-4o. Many other researchers have devised prompting techniques and architectures for improving LLM's reasoning capabilities such as reflection, chain-of-thought, and prompt chaining. However, fewer papers have explored how social skills can enhance LLMs communication abilities. Systems like Generative Agents ~\cite{joon_agents}
% and Character.AI \cite{characterAI} do this to great success. In this paper, we gave VLMs a few simple ``skills'' that were relevant to humor generation and showed that overall, it improved AI's ability to write humor. The focus of this paper was the human evaluation to see whether AI could get closer to parity with most upvoted human captions. However, if even a simple system like this can improve humor, perhaps more sophisticated systems could do better. 

There are many ways to improve the skills in HumorSkills. Building and testing a better Gen Z humor ranking would probably improve the filtering of bad captions. More fine-tuning could improve the breadth of Gen Z slang and references. More narratives and conflicts would expand its vocabulary of relatable situations. Finding ways to automatically collect narratives and conflicts to be applied would accelerate this process. Adding new skills would also be future research. Theories of humor abound. With recent advances in LLM's ability to do long chains of logical reasoning in DeepSeek and GPT-4o, it would be interesting to have AI try to analyze the humor and extract it's own theories or techniques for humor.

One of biggest shortcomings of the captions is that some of them are not logical enough to make sense, but are also not illogical enough to be absurd. These sound like mistakes. As future work, one could test whether an AI-based reflection step could think through the logic of a joke and decide whether it actually made sense or not. 
% The current rating system seems to let some of these by. 

% Other papers have tried fine-tuning b

Further testing or ablation studies could help shed light on which skills are most helpful. However, humor ratings have high variance among raters, and the data required to get statistical significance is often quite high. There may not be an effect of each skill individually - they might only work together. 



\subsection{Implications of Machine with Human-like Social Skills}
Human-like social skills - like humor - are often used for human bonding. If AI can write humor as well as the best people, the AI has the potential to both disingenuously create human bonding ~\cite{diresta2024spammersscammersleverageaigenerated,naaman_opinion} and to augment human's ability to bond~\cite{socialglue}. Either way, this has the potential to change the nature of human trust and communication.
In many ways, this is already happening in other domains. 
ChatGPT and Gmail SmartCompose~\cite{smartcompose} can already rewrite emails to sound more polite and we really are.
AI sales and scams can trick people into giving money to what they think are friend or loved ones in need~\cite{ai_scams}. 
AI has successfully been integrated into Gen Z dating apps that suggest messages to send to potential dates based on both a dating profile (for the opening line) or message history (for continued conversation)~\cite{majic2024rizz}. Many apps attempt this, but the quality of the suggested text sets them apart - the apps that generate more human-like texts have millions of active paying users~\cite{majic2024rizz}.
To some, this potential for disingenuousness is horrifying. Although disingenuous portrayals of oneself for dating purposes far precede the invention of generative AI, there is a possibility that AI will amplify this ability. 


As AI for social, cultural, and personally relevant communication improves, we may need a way to discern genuine from disingenuous communication. There are high-tech ways of doing this, such as making a video of oneself (until AI can do that). There are also low-tech ways of doing this, like talking in person. It would be highly ironic if the advancement of AI drives people to abandon technology, because it could not be trusted to be genuine. 

% For people who adopt these products, a common reason is that the bar for texting banter is so high for Gen Z, that help is appreciated, even when the sentiment is genuine. 

% Even at work, polite communication is socially demanded, and with ever-increasing amounts of communication, the emotional labor of even typing simple pleasantries is tiring. Early generative AI applications like Gmail Smart Compose~\cite{smartcompose} were noted for lowering the burden of writing a polite introductory line in emails. Although these lines are perfunctory and don't necessarily need to be genuine, it makes a difference to readers whether they are there or not. Social effort matters.



% \textit{Although hopefully AI will not force civilization back into a barter economy that necessatiate personal interaction to establish trust.} (LYDIA: TOO MUCH?)


% emotional labor. 

% Also with AI friends like Persona aI?



\section{Limitations}
This study targeted only one form of humor for only one audience:  Gen Z humor Instagram captions. This type of humor tends toward absurdities, which can be easier to generate than something that needs to be logically sound. Being illogically surprising is probably easier than being logical and surprising. Future work would have to test whether similar techniques work on other humor tastes. Some of our techniques, like fine-tuning, would likely work generically for all humor types, but other skills might need to be tried.

The caption humor is difficult, but it is more well-defined than other forms of humor. Caption humor only requires a punchline for a given image (the setup). Other forms of humor like standup comedy and popular humor magazines require generating both the setup and the punchline. A future direction is to explore what additional "skills" are needed to generate jokes with both setup and punchline. 

The humor generated here is for a public audience, but most humor made spontaneously is made for friends, and often users insider knowledge about the friends, their background, and their shared experiences. LLMs would likely struggle to make in-joke humor without a source of inside information to train on.

In our baseline captions, we crafted a simple prompt for GPT-4o to write humor. It is possible that with a better prompt or multiple generations, one could generate similar results. However, that is effectively what the system performs. It might be possible that there are prompts that don't employ any humor skills that can also generate jokes funnier than baseline GPT. Future work should test more prompts - both with and without skills to see if there are approaches other than skills that can enhance LLM humor generation.

% ensuring better generations and more consistent 







% In this work, we introduced \segsub, a Segmentation Substitution framework designed to improve the robustness of visual reasoning in VLMs. Through the application of image segmentation and inpainting techniques, we augment VQA datasets with counterfactual samples and knowledge conflicts. These samples test LLMs' abilities to recognize and respond to various types of image-based reasoning challenges. Our experiments demonstrate that while VLMs show resilience to certain perturbations such as feature modifications that lie within their training distribution, they struggle with counterfactual cases and inconsistencies across multiple image sources, especially in multi-hop scenarios.

% Our findings highlight the need for robust VQA models that can navigate diverse visual contexts. We hope our contribution to advancing visual reasoning and model resilience against counterfactual noise will encourage future research in this area. 

% The \segsub framework serves as a tool for strengthening VQA tasks, advancing the study of multimodal reasoning in real-world applications.

We introduce \segsub, a framework designed to improve the robustness of visual reasoning in VLMs. Through the application of image segmentation and inpainting techniques, we augment VQA datasets with parametric, source and counterfactual conflicts. These samples test LLMs' abilities to recognize and respond to various types of image-based reasoning challenges. While our experiments demonstrate VLM resilience to perturbations that lie within their training distribution (i.e. feature modifications that induce parametric conflicts), they struggle with counterfactual cases and conflicts across multiple image sources, especially in multi-hop scenarios. Our findings highlight the need for VQA models that are robust to knowledge conflicts and we hope that our contribution will inspire future research in advancing visual reasoning. 

% Shorter
% We introduced \segsub, a framework that enhances visual reasoning in VLMs by augmenting VQA datasets with counterfactual samples and knowledge conflicts through segmentation and inpainting. Our experiments show that while VLMs are resilient to perturbations within their training distribution, they struggle with counterfactuals and inconsistencies in multi-image contexts, particularly in multi-hop tasks. These results emphasize the need for more robust VQA models capable of handling diverse visual challenges, encouraging further research in counterfactual noise and visual reasoning


% The \segsub framework serves as a tool 

\begin{acks}
Funded by the Deutsche Forschungsgemeinschaft (DFG, German Research Foundation) -- 525037874.
\end{acks}

\bibliographystyle{ACM-Reference-Format}
\bibliography{bibliography}

\appendix

\section{Appendix}

This appendix lists our prompt templates (\cref{sec:appendix_prompts}), questionnaires (\cref{sec:appendix_questionnaires}) and additional figures (\cref{sec:appendix_extra_figures}).

\subsection{Prompting}\label{sec:appendix_prompts}
Our final prototype used the prompt templates shown below. For a better overview, we shortened them for this appendix by leaving out the concrete few-shot examples and only indicating their position in the templates. See the project repository (link in \cref{sec:conclusion}) for the full prompts including these examples.

Contained variables are defined as follows:

\begin{itemize}
    \item sender: The full name of the sender.
    \item email\_text: Text content of the received email.
    \item existing\_reply: Either `This is the reply you have written so far: ``{existing\_text}'' ' or an empty string if there is no existing text.
    \item attribute: ``accepting'', or ``neutral'', or ``declining'' (two generations for each attribute)
    \item referenced\_text: The sentence that was selected on.
    \item input: Prompt that the user gave.
\end{itemize}

\subsubsection{Sentence-level support, without user input}\label{sec:appendix_sentence_without_input_prompt} %
\begin{lstlisting}
    System: "You are answering an email sentence by sentence. For each given sentence think of a suitable reply. The reply should only answer the selected sentence."
    ... few-shot examples...
    User: "You are Jamie Doe and have received this email from {sender}: '{email_text}'.
    {existing_reply} Formulate a short, {attribute} reply to this selected part of the email: '{referenced_text}'. Only output the short reply in one or two sentences."
\end{lstlisting}

\subsubsection{Sentence-level support, with user input} %
\begin{lstlisting}
    System: "You are answering an email sentence by sentence. For each given sentence think of a suitable reply. The reply should only answer the selected sentence."
    ... few-shot examples...
    User: "You are Jamie Doe and have received this email from {sender}:"{email_text}".
    {existing_reply} Formulate a short reply to this selected part of the email: "{referenced_text}". Incorporate this information into your reply: "{input}".  Only output the short reply in one or two sentences."
\end{lstlisting}

\subsubsection{Improve Email}\label{subsec:appendix_improve_email_prompt} %
\begin{lstlisting}
    System: "You have received an email and have drafted a reply. Now you review your draft and make some final edits to make it sound better. You output the entire improved email at once and nothing else."
    ... few-shot examples...
    User: "You are Jamie Doe and have received this email from {sender}:"{email_text}"
    You have written this reply as an answer:"{existing_reply}"
    You improve this email by fixing any mistakes and adding an email greeting or sign-off if missing. You also make sure to make it sound better but you do not change the content of the email. At last you only output the well formatted email."
\end{lstlisting}

\subsubsection{Message-level reply generation} \mbox{}
\begin{lstlisting}
    System: "You have received an email and are writing a response to it."
    ... few-shot examples...
    You are Jamie Doe and have received this email from {sender}:"{email_text}"
    You answer with a well written email following these instructions: "{input}". You make sure to add a greeting and a sign-off. You do not make anything up that is not mentioned in the instruction. You double check that the email is well formatted."
\end{lstlisting}







\subsection{Questionnaires}\label{sec:appendix_questionnaires}

\subsubsection{Favourite Mode}
\begin{enumerate}
  \item 
    \textbf{Question:} Which mode did you prefer for answering emails? \\
    \textbf{Question Type:} single-choice \\
    \textbf{Answer Options:} 
    \begin{itemize}
      \item Sentence-based suggestions
      \item Single prompt suggestion
      \item Without AI-Support
      \item Depends (please describe) [free-response]
    \end{itemize}
  \item 
    \textbf{Question:} Why did you prefer this mode? \\
    \textbf{Question Type:} free-response

  \item 
    \textbf{Question (Optional):} Did you face any problems, issues or bugs during your participation in this study? \\
    \textbf{Question Type:} free-response
\end{enumerate}


\subsubsection{Demographic Questionnaire}
\begin{enumerate}

\item
  \textbf{Question:} What gender do you identify with? \\
  \textbf{Question Type:} single-choice \\
  \textbf{Answer Options:}
  \begin{itemize}
    \item Woman
    \item Man
    \item Non-Binary
    \item Prefer not to disclose
    \item Prefer to self-describe: [free-response]
  \end{itemize}

\item
  \textbf{Question:} How well do you speak English? \\
  \textbf{Question Type:} single-choice \\
  \textbf{Answer Options:}
  \begin{itemize}
    \item No knowledge of English
    \item Speak poorly (beginner knowledge)
    \item Fairly well (intermediate knowledge)
    \item Well (advanced knowledge)
    \item Very well (proficient in English)
    \item Native speaker
  \end{itemize}

\item
  \textbf{Question:} How old are you? \\
  \textbf{Question Type:} numeric response

\item
  \textbf{Question:} What is your current occupation? \\
  \textbf{Question Type:} free-response

\item
  \textbf{Question:} What is the highest academic level you have achieved? \\
  \textbf{Question Type:} single-choice \\
  \textbf{Answer Options:}
  \begin{itemize}
    \item High School Diploma or equivalent
    \item Bachelor's Degree
    \item Master's Degree
    \item Doctoral Degree
    \item Other: [free-response]
  \end{itemize}

\item
  \textbf{Question:} How often do you reply to emails? \\
  \textbf{Question Type:} single-choice \\
  \textbf{Answer Options:}
  \begin{itemize}
    \item Never
    \item Less than monthly
    \item At least once a month
    \item At least once a week
    \item Daily
    \item More than 10 times a day
    \item Other: [free-response]
  \end{itemize}

\item
  \textbf{Question:} What devices do you use to answer emails on? \\
  \textbf{Question Type:} multiple-select \\
  \textbf{Answer Options:}
  \begin{itemize}
    \item Desktop-PC
    \item Laptop
    \item Tablet
    \item Smartphone
    \item Smartwatch
    \item Other: [free-response]
  \end{itemize}

\item
  \textbf{Question:} Do you have experience with AI writing support (including for email)? \\
  \textbf{Question Type:} multiple-select \\
  \textbf{Answer Options:}
  \begin{itemize}
    \item No, I have no experience writing with AI support
    \item Writing with word- or sentence-suggestions
    \item Writing with auto-correction
    \item Writing with auto-completion
    \item Using ChatGPT (or similar)
    \item Using the smart reply feature
    \item Other: [free-response]
  \end{itemize}

\item
  \textbf{Question:} Where do you usually reply to emails? \\
  \textbf{Question Type:} multiple-select \\
  \textbf{Answer Options:}
  \begin{itemize}
    \item On the go
    \item At home
    \item At the office
    \item Somewhere else: [free-response]
  \end{itemize}

\item
  \textbf{Question:} What context do most of your emails have? \\
  \textbf{Question Type:} multiple-select \\
  \textbf{Answer Options:}
  \begin{itemize}
    \item Business
    \item Private
    \item Other: [free-response]
  \end{itemize}

\end{enumerate}






\subsection{Additional Figures}\label{sec:appendix_extra_figures}
Here we provide additional figures.
\cref{fig:likert_items_formative_study} shows the Likert and \cref{fig:sus_items_formative_study} the SUS \cite{brooke1996sus} results from the formative study (\cref{sec:formative_study}).
The other figures show the UIs used in the study: \cref{fig:briefing_and_feedback} shows the screens for the briefing and in-app feedback. \cref{fig:baseline_uis_manual} shows the UI of the manual mode (\modemanual), and \cref{fig:baseline_uis_msg} shows the UI of the \modemailtxt{} (\modemail). 

\begin{figure*}[h!]
    \centering
    \includegraphics[width=\linewidth]{figures/likert_formative_study.pdf}
    \caption{Likert results from the formative study (\cref{sec:formative_study}). The first three items were asked via an in-app feedback screen after each email, while the others were part of the final questionnaire at the end of the study.}
    \Description{This figure presents the Likert scale results from the formative study, which evaluates participant feedback on various aspects of using the AI tool for email writing. The responses are categorised into five levels of agreement: strongly disagree, disagree, neutral, agree, and strongly agree. The figure indicates that the majority of participants had a positive experience with the AI tool, finding it helpful in writing faster, higher-quality replies with enough variation in suggestions, though some reported mixed feelings about control and distractions.}
    \label{fig:likert_items_formative_study}
\end{figure*}

\begin{figure*}[h!]
    \centering
    \includegraphics[width=\linewidth]{figures/sus_formative_study.pdf}
    \caption{\revision{SUS \cite{brooke1996sus} results from the formative study (\cref{sec:formative_study}).}}
    \Description{This figure presents the SUS Likert scale results from the formative study. The responses are categorised into five levels of agreement: strongly disagree, disagree, neutral, agree, and strongly agree. The figure indicates that the majority of participants had a positive experience with the AI tool.}
    \label{fig:sus_items_formative_study}
\end{figure*}

\begin{figure*}[h!]
    \centering
    \includegraphics[width=0.6\linewidth]{figures/briefing_and_feedback}
    \caption{The study-specific UI screens in our prototype: A screen showing the briefing before each email task \textit{(1)}, and a screen asking users to rate four Likert items after each email task \textit{(2)}.}
    \Description{This figure presents two study-specific UI screens from the prototype used in the research:
    Briefing Screen (Left Panel):
    This screen is shown before each email task.
    It provides a brief description of the task, explaining the scenario or context that the participant needs to respond to.
    The screen includes instructions to "Answer the following email according to this idea" and has a "Continue" button to move forward.
    In-App Feedback Screen (Right Panel):
    This screen appears after each email task, asking participants to rate their experience.
    The specific statements in the example ask about the helpfulness of the interface.
    Participants can choose from five options ranging from "strongly disagree" to "strongly agree" for each statement, and submit their feedback by pressing the "Send Email" button.}
    \label{fig:briefing_and_feedback}
\end{figure*}


\begin{figure*}[h!]
    \centering
    \includegraphics[width=0.6\linewidth]{figures/baseline_uis_manual}
    \caption{The UI design for manual typing (\modemanual) used in the study. It has one screen to show the incoming email \textit{(1)} and one with a text box to type the reply manually \textit{(2)}. This figure shows the state after typing a reply, as an example. As is usual on mobile devices, the keyboard opened from the bottom when tapping on the text field.}
    \Description{This figure illustrates the UI design for manual typing (NoAI) used in the study, showcasing two views:
    Incoming Email View (Left Panel):
    This screen displays the email received by the user.
    In this example, the email is about the subject "What pet to get?" asking for advice on choosing a family pet between a cat or a dog.
    At the bottom of the screen, there is a "Reply" button, allowing the user to initiate a response.
    Manual Reply View (Right Panel):
    Once the "Reply" button is pressed, the user is taken to the reply screen, where they can type their response manually in a text box.
    The "Send Email" button is at the bottom.}
    \label{fig:baseline_uis_manual}
\end{figure*}


\begin{figure*}[h!]
    \centering
    \includegraphics[width=0.8\linewidth]{figures/baseline_uis_msg}
    \caption{The UI design for \modemailtxt{} (\modemail) used in the study. The first screen shows the incoming email \textit{(1)}. On the reply generation screen \textit{(2)}, users can generate a full message suggestion, optionally guided by entering a prompt in the text box at the top. Finally, the manual screen \textit{(3)} allowed users to freely edit the generated draft.}
    \Description{This figure shows the UI design for message-level reply generation (MSG) used in the study, illustrating three stages of the interaction:
    Incoming Email View (Left Panel):
    Similar to the manual UI, this screen displays the incoming email.
    A "Reply" button is available at the bottom.
    Reply Generation View (Middle Panel):
    In this screen, users can generate a full message reply using AI.
    There is an optional prompt field at the top, where users can enter specific keywords or topics to guide the AI's response.
    Once the prompt is entered, users can press the "Generate Reply" button, and the AI will generate a suggested email, displayed in the output field below.
    The user can either discard the generated reply or choose to save and edit it.
    Manual Editing View (Right Panel):
    After generating the AI-suggested reply, users are taken to this screen to manually edit the draft.
    The "Send Email" button at the bottom allows the user to send the edited reply.}
    \label{fig:baseline_uis_msg}
\end{figure*}


\end{document}
