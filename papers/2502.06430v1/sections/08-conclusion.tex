\section{Conclusion}\label{sec:conclusion}

We have proposed the concept of \textit{\modeoursTXT{} (\modeours)}, which %
allows users to insert concise responses at selected points in the incoming email as they read it. Selecting text for this purpose doubles as an expression of intent to guide the AI's text suggestions. These local responses can then be further edited manually or again with AI support. This design for the first time combines sentence-level and message-level AI support, while keeping both optional.
The key benefit is flexibility. Involving AI with \modeours{} shares the general characteristics of AI responses but its modularity allows users to make their own choices. %

Participants' comments highlight the distinct strengths of \modeours{} and the involved tradeoffs. They overall appreciated all AI features but when asked to pick a favourite, their reasoning for message-level AI highlighted efficiency in the majority of cases, while for \modeours{} they emphasised quality and control relatively more often.

Future work could explore how people make use of \modeours{} features in real life emailing, with varying social and contextual factors~\cite{Robertson2021cantreply}, and in long-term use.

While recent feature releases in widely used email apps have focused on AI that generates complete emails, our findings with \modeours{} motivate the exploration of more flexible alternatives. Rather than adding AI ``on top'', we motivate designing for users' underlying workflows -- both for those with and without AI at the same time. In this way, flexible UIs can empower users to dynamically adjust their desired degree of AI involvement to manage tradeoffs, such as between responding quickly and retaining control over the outcome.

We release our prototype and \revision{study} material in this project repository to support future research:

\url{https://osf.io/fsxzv/}
