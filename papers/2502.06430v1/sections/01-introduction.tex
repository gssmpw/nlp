
\begin{figure*}[t]
    \centering
    \includegraphics[width=\linewidth]{figures/current_products}
    \caption{A commonly used UI and interaction design for reply generation in mobile email apps: \textit{(A)} Users work with AI in a pop-up, \textit{(B)} on top of the empty draft view. \textit{(C)} They can enter a prompt, \textit{(D)} view the generated reply, and \textit{(E)} accept it with a button. At the time of writing this paper, this pattern appears across Gmail (left), Superhuman (centre), and Outlook (right). It also appears in other email apps, such as Shortwave. Screenshots from: Google's website~\cite{google2024geminiwebsite}, Superhuman's YouTube channel~\cite{superhuman2024video}, The Copilot Connection's YouTube channel~\cite{copilotconnection2023video}.}
    \Description{This figure displays a commonly used UI and interaction design for reply generation in mobile email apps.
    Three UIs are shown: On the left Gmail, Superhuman in the centre, and Outlook on the right.
    Users work with AI in a pop-up window, on top of the empty draft view. They can enter a prompt in a text field, view the generated reply, and accept it with a button closeby. At the time of writing, this pattern appears across Gmail, Superhuman, and Outlook. It also appears in other email apps, such as Shortwave. Screenshots from: Google's website, Superhuman's YouTube channel, The Copilot Connection's YouTube channel.}
    \label{fig:current_products}
\end{figure*}


\section{Introduction}


Emails challenge us to stay responsive and responsible -- anytime, anywhere.
With mobile devices, people wish to stay responsive despite contextual challenges: 
They might lack the time and attention to write detailed replies quickly, especially on small on-screen keyboards and in demanding mobile contexts. 
This motivates new text suggestion features that delegate the writing of replies to the system.
\revision{For example, \textit{Smart Reply}~\cite{Kannan2016smartreply} and \textit{Smart Compose}~\cite{Chen2019smartcompose} suggest short email replies and sentence continuations, respectively. This is especially attractive in mobile contexts, where AI serves to reduce cumbersome typing on a small keyboard~\cite{Quinn2016} or to speed up the reply process~\cite{Kannan2016smartreply}. More broadly, research in AI-mediated communication (AIMC) highlights relevant conceptual dimensions~\cite{Hancock2020aimc}, such as the AI’s magnitude of involvement, optimisation goal, and its autonomy. Operationalised as concrete design decisions, for instance, message-level vs sentence-level suggestions come with different degrees of magnitude and autonomy~\cite{Fu2023sentencevsmessage}, and generally optimise for replying faster.}

\revision{Having different options matters, including those that retain high user control, because} people \revision{also} wish to stay responsible  \revision{and maintain their agency~\cite{Mieczkowski2022thesis}}. Their replies need to adequately consider the complex nature of their professional and private relationships, \revision{which currently} is hard to get right with generated text~\cite{fu2024texttoself, Liu2022aimailperception, Mieczkowski2021, Robertson2021cantreply}. This creates complex tradeoffs for users in navigating the use of AI in their mobile response workflows.



Current mobile email apps offer two approaches to involving AI: (1) sentence completions, which help with manual writing; and (2) reply generations, which seek to entirely replace the need for manual writing. 
Both come with tradeoffs:

Sentence-level suggestions, often sporadic and utility-gated to avoid showing unsuitable text~\cite{Chen2019smartcompose}, still demand considerable manual typing and do not account for situations in which users would prefer to automate the reply process further, for example, when only a short reply is needed (cf.~\cite{Kannan2016smartreply}).

In contrast, message-level support can take important decisions away from the user, pushing them into an editing role if they prefer to respond in a different way compared to what the AI has drafted. Moreover, there are several negative impacts associated with workflows in which the AI generates the first draft, including reducing self-expression and diversity~\cite{Li2024aivalue}.

Today's mobile email apps largely follow the same UI and interaction design for AI integration -- such as \textit{Gmail} with Gemini, \textit{Outlook} with Copilot, and \textit{Superhuman} and \textit{Shortwave}: When replying to an email, users decide as a first step if they want to switch to full reply generation. This brings up a pop-up on top of an empty draft view. Here, users can enter a prompt and check the generated reply, before accepting it with a button to be inserted into the draft. \revision{\cref{fig:current_products} shows this.} 

This design and workflow hides the incoming email while the user is prompting and assessing the quality of the generated reply. At the same time, the draft view is nothing but an empty background to the pop-up.
In summary, there is at least one design contradiction in the current industry standard: UI space is wasted while useful information is omitted.

We address the challenge of AI integration into mobile email replies by taking a step back to reconsider the reply workflow more holistically. This leads to our three guiding research questions:
\begin{enumerate}
    \item How might we redesign the mobile email reply UI for flexible and optional AI involvement?
    \item How do users perceive and interact with this design to reply to emails?
    \item What are the specific advantages and drawbacks of this design, and how do they differ from existing approaches?
\end{enumerate}
To address these questions, we developed a prototype web app (\cref{fig:teaser}), informed by a formative study (N=\studyOneN).
Our design not only bridges the gaps between manual writing, sentence suggestions, and message-level AI. It also supports a new reply workflow that allows users to still see the incoming email while they work on their response, despite the limited screen space of mobile devices. 

Concretely, inspired by concepts from mobile microtasking~\cite{august2020microwriting, iqbal2018playwrite}, we integrate a sentence-based response interface directly into the incoming email. We refer to this as \textit{\modeoursTXT{} (\modeours)} since it allows for responding locally within the content -- both for the user and the AI. Users tap on sentences they wish to address, which brings up a local response widget. They then reply manually or use sentence suggestions that adapt to their input.
Additionally, users can continue their reply process on the level of the whole message, with an AI \imppass{} or by manually editing their reply as a whole.
In this way, each AI feature remains optional, empowering users to dynamically build their own workflows, ranging from manual writing to fully AI-generated responses.

We evaluated our concept in a user study (N=\studyTwoN) with a functional prototype and two comparative designs (manual writing and message generation as in today's email apps). We logged interactions and assessed perception with surveys and in-app feedback.

We found that our system takes a new distinct spot in the design space between sentence-level and message-level support, reflected in participants' subjective feedback and significant differences in interaction metrics. %
People used it in varying workflows with different degrees of AI involvement, while retaining the benefits of reduced typing and errors. 

In summary, we contribute: (1) A novel concept for local sentence-based replies with optional AI support for mobile email communication, (2) its implementation in a functional prototype, and (3) insights from an evaluation with users.

In a broader perspective, our work demonstrates the potential of a new approach to integrating AI capabilities; not by adding them ``on top'' but rather by focusing on improving the UI for users' underlying workflows -- with and without AI. In this way, designers can empower users to dynamically adjust the desired degree of AI involvement. %
