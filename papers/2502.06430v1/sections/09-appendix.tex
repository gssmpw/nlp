\section{Appendix}

This appendix lists our prompt templates (\cref{sec:appendix_prompts}), questionnaires (\cref{sec:appendix_questionnaires}) and additional figures (\cref{sec:appendix_extra_figures}).

\subsection{Prompting}\label{sec:appendix_prompts}
Our final prototype used the prompt templates shown below. For a better overview, we shortened them for this appendix by leaving out the concrete few-shot examples and only indicating their position in the templates. See the project repository (link in \cref{sec:conclusion}) for the full prompts including these examples.

Contained variables are defined as follows:

\begin{itemize}
    \item sender: The full name of the sender.
    \item email\_text: Text content of the received email.
    \item existing\_reply: Either `This is the reply you have written so far: ``{existing\_text}'' ' or an empty string if there is no existing text.
    \item attribute: ``accepting'', or ``neutral'', or ``declining'' (two generations for each attribute)
    \item referenced\_text: The sentence that was selected on.
    \item input: Prompt that the user gave.
\end{itemize}

\subsubsection{Sentence-level support, without user input}\label{sec:appendix_sentence_without_input_prompt} %
\begin{lstlisting}
    System: "You are answering an email sentence by sentence. For each given sentence think of a suitable reply. The reply should only answer the selected sentence."
    ... few-shot examples...
    User: "You are Jamie Doe and have received this email from {sender}: '{email_text}'.
    {existing_reply} Formulate a short, {attribute} reply to this selected part of the email: '{referenced_text}'. Only output the short reply in one or two sentences."
\end{lstlisting}

\subsubsection{Sentence-level support, with user input} %
\begin{lstlisting}
    System: "You are answering an email sentence by sentence. For each given sentence think of a suitable reply. The reply should only answer the selected sentence."
    ... few-shot examples...
    User: "You are Jamie Doe and have received this email from {sender}:"{email_text}".
    {existing_reply} Formulate a short reply to this selected part of the email: "{referenced_text}". Incorporate this information into your reply: "{input}".  Only output the short reply in one or two sentences."
\end{lstlisting}

\subsubsection{Improve Email}\label{subsec:appendix_improve_email_prompt} %
\begin{lstlisting}
    System: "You have received an email and have drafted a reply. Now you review your draft and make some final edits to make it sound better. You output the entire improved email at once and nothing else."
    ... few-shot examples...
    User: "You are Jamie Doe and have received this email from {sender}:"{email_text}"
    You have written this reply as an answer:"{existing_reply}"
    You improve this email by fixing any mistakes and adding an email greeting or sign-off if missing. You also make sure to make it sound better but you do not change the content of the email. At last you only output the well formatted email."
\end{lstlisting}

\subsubsection{Message-level reply generation} \mbox{}
\begin{lstlisting}
    System: "You have received an email and are writing a response to it."
    ... few-shot examples...
    You are Jamie Doe and have received this email from {sender}:"{email_text}"
    You answer with a well written email following these instructions: "{input}". You make sure to add a greeting and a sign-off. You do not make anything up that is not mentioned in the instruction. You double check that the email is well formatted."
\end{lstlisting}







\subsection{Questionnaires}\label{sec:appendix_questionnaires}

\subsubsection{Favourite Mode}
\begin{enumerate}
  \item 
    \textbf{Question:} Which mode did you prefer for answering emails? \\
    \textbf{Question Type:} single-choice \\
    \textbf{Answer Options:} 
    \begin{itemize}
      \item Sentence-based suggestions
      \item Single prompt suggestion
      \item Without AI-Support
      \item Depends (please describe) [free-response]
    \end{itemize}
  \item 
    \textbf{Question:} Why did you prefer this mode? \\
    \textbf{Question Type:} free-response

  \item 
    \textbf{Question (Optional):} Did you face any problems, issues or bugs during your participation in this study? \\
    \textbf{Question Type:} free-response
\end{enumerate}


\subsubsection{Demographic Questionnaire}
\begin{enumerate}

\item
  \textbf{Question:} What gender do you identify with? \\
  \textbf{Question Type:} single-choice \\
  \textbf{Answer Options:}
  \begin{itemize}
    \item Woman
    \item Man
    \item Non-Binary
    \item Prefer not to disclose
    \item Prefer to self-describe: [free-response]
  \end{itemize}

\item
  \textbf{Question:} How well do you speak English? \\
  \textbf{Question Type:} single-choice \\
  \textbf{Answer Options:}
  \begin{itemize}
    \item No knowledge of English
    \item Speak poorly (beginner knowledge)
    \item Fairly well (intermediate knowledge)
    \item Well (advanced knowledge)
    \item Very well (proficient in English)
    \item Native speaker
  \end{itemize}

\item
  \textbf{Question:} How old are you? \\
  \textbf{Question Type:} numeric response

\item
  \textbf{Question:} What is your current occupation? \\
  \textbf{Question Type:} free-response

\item
  \textbf{Question:} What is the highest academic level you have achieved? \\
  \textbf{Question Type:} single-choice \\
  \textbf{Answer Options:}
  \begin{itemize}
    \item High School Diploma or equivalent
    \item Bachelor's Degree
    \item Master's Degree
    \item Doctoral Degree
    \item Other: [free-response]
  \end{itemize}

\item
  \textbf{Question:} How often do you reply to emails? \\
  \textbf{Question Type:} single-choice \\
  \textbf{Answer Options:}
  \begin{itemize}
    \item Never
    \item Less than monthly
    \item At least once a month
    \item At least once a week
    \item Daily
    \item More than 10 times a day
    \item Other: [free-response]
  \end{itemize}

\item
  \textbf{Question:} What devices do you use to answer emails on? \\
  \textbf{Question Type:} multiple-select \\
  \textbf{Answer Options:}
  \begin{itemize}
    \item Desktop-PC
    \item Laptop
    \item Tablet
    \item Smartphone
    \item Smartwatch
    \item Other: [free-response]
  \end{itemize}

\item
  \textbf{Question:} Do you have experience with AI writing support (including for email)? \\
  \textbf{Question Type:} multiple-select \\
  \textbf{Answer Options:}
  \begin{itemize}
    \item No, I have no experience writing with AI support
    \item Writing with word- or sentence-suggestions
    \item Writing with auto-correction
    \item Writing with auto-completion
    \item Using ChatGPT (or similar)
    \item Using the smart reply feature
    \item Other: [free-response]
  \end{itemize}

\item
  \textbf{Question:} Where do you usually reply to emails? \\
  \textbf{Question Type:} multiple-select \\
  \textbf{Answer Options:}
  \begin{itemize}
    \item On the go
    \item At home
    \item At the office
    \item Somewhere else: [free-response]
  \end{itemize}

\item
  \textbf{Question:} What context do most of your emails have? \\
  \textbf{Question Type:} multiple-select \\
  \textbf{Answer Options:}
  \begin{itemize}
    \item Business
    \item Private
    \item Other: [free-response]
  \end{itemize}

\end{enumerate}






\subsection{Additional Figures}\label{sec:appendix_extra_figures}
Here we provide additional figures.
\cref{fig:likert_items_formative_study} shows the Likert and \cref{fig:sus_items_formative_study} the SUS \cite{brooke1996sus} results from the formative study (\cref{sec:formative_study}).
The other figures show the UIs used in the study: \cref{fig:briefing_and_feedback} shows the screens for the briefing and in-app feedback. \cref{fig:baseline_uis_manual} shows the UI of the manual mode (\modemanual), and \cref{fig:baseline_uis_msg} shows the UI of the \modemailtxt{} (\modemail). 

\begin{figure*}[h!]
    \centering
    \includegraphics[width=\linewidth]{figures/likert_formative_study.pdf}
    \caption{Likert results from the formative study (\cref{sec:formative_study}). The first three items were asked via an in-app feedback screen after each email, while the others were part of the final questionnaire at the end of the study.}
    \Description{This figure presents the Likert scale results from the formative study, which evaluates participant feedback on various aspects of using the AI tool for email writing. The responses are categorised into five levels of agreement: strongly disagree, disagree, neutral, agree, and strongly agree. The figure indicates that the majority of participants had a positive experience with the AI tool, finding it helpful in writing faster, higher-quality replies with enough variation in suggestions, though some reported mixed feelings about control and distractions.}
    \label{fig:likert_items_formative_study}
\end{figure*}

\begin{figure*}[h!]
    \centering
    \includegraphics[width=\linewidth]{figures/sus_formative_study.pdf}
    \caption{\revision{SUS \cite{brooke1996sus} results from the formative study (\cref{sec:formative_study}).}}
    \Description{This figure presents the SUS Likert scale results from the formative study. The responses are categorised into five levels of agreement: strongly disagree, disagree, neutral, agree, and strongly agree. The figure indicates that the majority of participants had a positive experience with the AI tool.}
    \label{fig:sus_items_formative_study}
\end{figure*}

\begin{figure*}[h!]
    \centering
    \includegraphics[width=0.6\linewidth]{figures/briefing_and_feedback}
    \caption{The study-specific UI screens in our prototype: A screen showing the briefing before each email task \textit{(1)}, and a screen asking users to rate four Likert items after each email task \textit{(2)}.}
    \Description{This figure presents two study-specific UI screens from the prototype used in the research:
    Briefing Screen (Left Panel):
    This screen is shown before each email task.
    It provides a brief description of the task, explaining the scenario or context that the participant needs to respond to.
    The screen includes instructions to "Answer the following email according to this idea" and has a "Continue" button to move forward.
    In-App Feedback Screen (Right Panel):
    This screen appears after each email task, asking participants to rate their experience.
    The specific statements in the example ask about the helpfulness of the interface.
    Participants can choose from five options ranging from "strongly disagree" to "strongly agree" for each statement, and submit their feedback by pressing the "Send Email" button.}
    \label{fig:briefing_and_feedback}
\end{figure*}


\begin{figure*}[h!]
    \centering
    \includegraphics[width=0.6\linewidth]{figures/baseline_uis_manual}
    \caption{The UI design for manual typing (\modemanual) used in the study. It has one screen to show the incoming email \textit{(1)} and one with a text box to type the reply manually \textit{(2)}. This figure shows the state after typing a reply, as an example. As is usual on mobile devices, the keyboard opened from the bottom when tapping on the text field.}
    \Description{This figure illustrates the UI design for manual typing (NoAI) used in the study, showcasing two views:
    Incoming Email View (Left Panel):
    This screen displays the email received by the user.
    In this example, the email is about the subject "What pet to get?" asking for advice on choosing a family pet between a cat or a dog.
    At the bottom of the screen, there is a "Reply" button, allowing the user to initiate a response.
    Manual Reply View (Right Panel):
    Once the "Reply" button is pressed, the user is taken to the reply screen, where they can type their response manually in a text box.
    The "Send Email" button is at the bottom.}
    \label{fig:baseline_uis_manual}
\end{figure*}


\begin{figure*}[h!]
    \centering
    \includegraphics[width=0.8\linewidth]{figures/baseline_uis_msg}
    \caption{The UI design for \modemailtxt{} (\modemail) used in the study. The first screen shows the incoming email \textit{(1)}. On the reply generation screen \textit{(2)}, users can generate a full message suggestion, optionally guided by entering a prompt in the text box at the top. Finally, the manual screen \textit{(3)} allowed users to freely edit the generated draft.}
    \Description{This figure shows the UI design for message-level reply generation (MSG) used in the study, illustrating three stages of the interaction:
    Incoming Email View (Left Panel):
    Similar to the manual UI, this screen displays the incoming email.
    A "Reply" button is available at the bottom.
    Reply Generation View (Middle Panel):
    In this screen, users can generate a full message reply using AI.
    There is an optional prompt field at the top, where users can enter specific keywords or topics to guide the AI's response.
    Once the prompt is entered, users can press the "Generate Reply" button, and the AI will generate a suggested email, displayed in the output field below.
    The user can either discard the generated reply or choose to save and edit it.
    Manual Editing View (Right Panel):
    After generating the AI-suggested reply, users are taken to this screen to manually edit the draft.
    The "Send Email" button at the bottom allows the user to send the edited reply.}
    \label{fig:baseline_uis_msg}
\end{figure*}
