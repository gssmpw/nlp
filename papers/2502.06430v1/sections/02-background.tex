


\section{Background \& Related Work}\label{sec:related_work}




We relate our concept and design decisions to research on emailing with AI and the impact of text generation.


\subsection{Text Suggestions for Emailing}

There are two main approaches for text suggestions for (mobile) emailing, exemplified by 
Google's \textit{Smart Compose}~\cite{Chen2019smartcompose} and \textit{Smart Reply}~\cite{Kannan2016smartreply}. \textit{Smart Compose} supports manual writing with sentence completions, while \textit{Smart Reply} generates short replies to replace manual writing. In contrast, we explore the design space in between by mixing sentence-based and message-based AI involvement. %



The study by \citet{Fu2023sentencevsmessage} compared sentence-level and message-level text suggestions. They found that suggesting whole emails put users in the role of editors, leading to faster task completion. Sentence suggestions, on the other hand, allowed people to retain a greater sense of agency and write more original content. Their call for further design exploration motivates our work: We offer sentence suggestions in one step, followed by an email-level improvement pass -- all optional. %
As \citet{Fu2023sentencevsmessage} pointed out, designing appropriate AI support for a particular use case requires identifying the ``sweet spot'' for ``suggestion units''. Based on our results, our design provides a new and distinct intermediate option beyond the two existing ``extremes'' (of pure sentence vs. message-based support). %



\subsection{AI Integration in Current Email Apps}\label{sec:related_work_current_products}



Recently, many mobile email apps have added generative AI features, including Google's \textit{Gmail} with Gemini, Microsoft's \textit{Outlook} with Copilot, and the apps by \textit{Superhuman}\footnote{\url{https://superhuman.com/}} and \textit{Shortwave}.\footnote{\url{https://www.shortwave.com/ai-email/a/}}
As shown in \cref{fig:current_products}, they share a common UI and interaction design: When replying to an email, users can open a pop-up to enter a prompt and trigger generation. The result is shown in the pop-up and can be accepted with a button. It is then inserted into the draft view for sending or editing.

This design hides the email while working with AI to respond to it. Hence, the user needs to remember all relevant information in the email while prompting and checking generations. At the same time, the draft view is an empty backdrop to the AI pop-up.
In short: UI space is wasted while useful information is omitted. 

Thus, we diagnose that AI is not optimally integrated for mobile email replies:
It is added (literally) ``on top'' of the existing UI. %
This motivates our approach to take a step back to redesign the UI with users' workflows in mind -- also those without AI. 










\subsection{Learning from Mobile Microtasking}
Microtasking principles provide inspiration for our redesign: Limited screen space~\cite{raptis2013phonesize} and frequent interruptions~\cite{leiva2012backtoapp, oulasvirta2005bursts}, along with input method constraints~\cite{palin2019mobiletyping}, make it difficult to work on text documents on mobile touchscreen devices.
This motivates research on mobile microtasking~\cite{cheng2015breakitdown} to break down larger tasks, such as writing a document, into manageable chunks that can be completed in short, opportune moments on the go, one at a time~\cite{august2020microwriting, iqbal2018playwrite}.

The success of microtasking for mobile writing has not yet been brought to email. Possibly, emailing itself is seen as a microtask, as checking mails is fast (cf.~\cite{bao2011phoneuse, oulasvirta2012habits, zheng2024waitingtime}) and a common productive task while waiting, as shown by \citet{zheng2024waitingtime}. However, productive tasks were less likely with only a phone, highlighting the gap between checking and responding. \revision{It is also likely that short emails do not require microtasking. That said, emailing ``on the go'' may involve interruptions~\cite{oulasvirta2005bursts}, which motivate keeping relevant context information in view, even for shorter emails.}

\revision{Together,} this motivates our redesign of the response workflow with microtasking in mind: Traditionally, users read an email and then type into a separate, empty text box. Instead, we allow users to write short responses in boxes shown directly within the incoming email text.

This acknowledges that writing depends on local context~\cite{salehi2017communicatecontext}, which should be shown in microtasking todos~\cite{august2020microwriting, iqbal2018playwrite}. %
Unlike the related work, we do not assume that users go to a desktop computer to finalise their text. Thus, we add a second UI view where users can finalise their reply, for example by connecting partial responses coherently. Our design offers optional AI support for this.

\subsection{Design Factors for Mobile Emailing with AI}


\citet{Park2019inboxneedfinding} revealed user needs for inboxes that also motivate our design for responding to a single email: Support for attention management, a presentation closer to the lighter interaction of messaging UIs, and breaking up longer emails. In line with this, our concept allows users to respond to longer emails in more lightweight chunks by writing local responses to parts of an email.
Related, \textit{Rambler} by \citet{lin2024rambler} provides text cells to break up text while drafting and editing with text-to-speech input and AI features on tablets. 

We further use insights from related work on the number and type of suggestions~\cite{Buschek2021chi, Dang2023diegetic}: Our design offers multiple suggestions that users can refine. Unlike the related work~\cite{Dang2023diegetic}, users can use the \textit{same} UI element (text box) to either type their own response or enter a prompt to refine the suggestions. This supports dynamic user decisions about AI involvement without the need for separate UI elements for writing and prompting, taking into account the limited space of mobile screens. %

To support emailing for people with dyslexia, \textit{LaMPost}~ \cite{Goodman2022lampost} offers AI rewriting and subject line generation. Similar to \textit{Wordcraft}~\cite{Yuan2022}, which is not email-specific, the design allows users to select text with the mouse to then trigger AI support. We bring this pattern of ``selection for local AI support'' to mobile touchscreens, with two adaptations: (1) selecting whole sentences rather than character-level selection, to account for the limited accuracy of touch; and (2) displaying suggestions directly below the selection, rather than in a sidebar, which would not fit on a mobile screen. 




\subsection{Impact and Perception of Generated Text}

Work in AI-mediated communication (AIMC~\cite{Hancock2020aimc}) reveals the complexity of social issues involved in writing emails with AI-generated text: \citet{Robertson2021cantreply} studied how people edit suggested email texts and found that suggestions can be inappropriate, with respect to intersecting factors, including relationship types and cultural norms. 
Related, \citet{lucy2024onesizefitsall} found that the desired appropriate system behaviour in this context is highly individual. 

Focusing on the receivers instead, \citet{Liu2022aimailperception} studied perception and trust towards emails, including the perceived degrees of AI involvement and interpersonal emphasis. Overall, trust in writers decreased when told about AI involvement, yet the findings also suggested differences between people's statements about AI and actual reactions, indicating an ongoing development of perspectives.

Together, the above results show that it is difficult to provide text generations for emails that entirely avoid problematic suggestions. Thus, \citet{Robertson2021cantreply} called for more studies on designs that explore ``Giving users more control'' beyond one-shot generation, based on their finding that ``a large fraction of [suggested] replies were amended''. This directly motivates our design, which allows users to control -- on a sentence-level -- in the incoming email (1) what to respond to at all, (2) whether to involve AI here, and (3) if so, what additional information to give the AI for that involvement (e.g. by entering keywords to refine the suggestions). %


Related, \citet{Li2024aivalue} studied people’s choices of AI assistance in controlled argumentative and creative writing tasks at the desktop. They report on desirable and undesirable changes on several metrics, depending on whether the ``primary'' writer is the human or the AI. %
All identified negative impacts arise in the ``AI-primary'' case, where ChatGPT provides a first draft. %
Thus, their findings  motivate our exploration beyond such ``AI-primary'' designs for emails. %
Our results extend the bigger picture to mobile text entry and emailing; for example, in line with their results, our proposed design retains more content diversity than the ``AI-primary'' baseline.







