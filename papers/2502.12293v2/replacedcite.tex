\section{Related works}
Reconstructing the interior of an object from its projections has been a long-standing challenge in CT. Traditional reconstruction
techniques, such as filtered back projection (FBP) ____ and algebraic
reconstruction techniques (ART) ____, often perform poorly under angle constraints due to missing information. FBP is an analytic solution for full-angle CT scans but leaves significant artifacts when working with limited angles. Current solutions to the limited-angle problem encompass a wide range of approaches, such as iterative and model-based solutions. This section will focus on iterative, deep learning, and hybrid methods.

\subsection{Iterative methods}
Iterative methods have long been a cornerstone in addressing limited-angle CT challenges. These approaches, such as ART and simultaneous iterative reconstruction technique (SIRT) ____, reconstruct the image by iteratively minimizing the difference between measured and calculated projections. Since the limited-angle problem is ill-posed, regularization strategies such as total variation (TV) ____ and sparsity constraints ____ are often used to incorporate prior information, such as piecewise constant solutions or specific shapes. Increasingly, Deep Image Priors (DIP) ____ have been used to regularize the reconstructions to favor solutions with more "natural" image statistics ____.
These methods are computationally intensive but offer flexibility in integrating priors and physical models of the imaging system.

\subsection{Deep learning methods}
Machine learning has emerged as a tool for tackling limited-angle CT ____.
Neural networks have been used to reconstruct images from limited projection data by training on large datasets of paired sinograms and ground-truth images. However, the scarcity of real-world data in limited-angle scenarios poses a significant challenge for supervised learning.
Sometimes creating synthetic data is possible, which allows researchers to create an arbitrary number of samples with known ground truth images enabling improved supervised models ____.
However, the synthetic data should accurately represent the true images, the imaging system, and noise to generalize to real data. Arguably, creating a synthetic dataset contains as many assumptions as regularization methods in iterative approaches.


\subsection{Hybrid approaches}
Hybrid approaches combine iterative methods and machine learning. These methods aim to leverage the strengths of both approaches to improve reconstruction quality. For instance, iterative methods can provide a good initial estimate, while machine learning techniques can refine this estimate by removing artifacts or enhancing details.

One common hybrid approach is to use one reconstruction algorithm, such as FBP, ART, or SIRT, to create the first reconstruction, and then use a neural network, often a U-net, to remove artifacts or noise from the initial reconstruction ____. The benefit of this approach compared to learning the end-to-end reconstruction is that the initial reconstruction can provide a good starting point for the neural network, the task is simplified, and the input space is fixed to the image size rather than varying with the sinogram size.

Additionally, some hybrid approaches use machine learning to learn regularization terms or hyperparameters that are then incorporated into the iterative reconstruction process ____. For example, a neural network can be trained to estimate the best total variation regularization strength based on the input sinogram, allowing for adaptive regularization that improves reconstruction quality.