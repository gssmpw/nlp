\section{Related Work}
\label{sec:related}

\begin{figure*}[t]
\setlength{\abovecaptionskip}{1.2mm}
\centering
\includegraphics[width=0.98\linewidth]{figs/arch.pdf}
\caption{The training pipeline of Point2RBox-v2. Gaussian overlap loss and Voronoi watershed loss utilize the spatial layout (see Fig. \ref{fig:loss}), while edge loss (see Sec. \ref{sec:method-le}), symmetry-aware learning (see Sec. \ref{sec:method-lss}), and copy-paste (see Sec. \ref{sec:method-cp}) further enhance the method.}
\label{fig:arch}
\vspace{-6pt}
\end{figure*}

\subsection{RBox-supervised Oriented Detection} 

In addition to horizontal detection \citep{zhao2019object,liu2020deep}, oriented object detection (OOD) \citep{yang2018automatic,wen2023comprehensive} has received extensive attention. Representative works include anchor-based detector Rotated RetinaNet \citep{lin2017focal}, anchor-free detector Rotated FCOS \citep{tian2019fcos}, and two-stage solutions, e.g. RoI Transformer \citep{ding2018learning}, Oriented R-CNN \citep{xie2021oriented}, and ReDet \citep{han2021redet}. Some research enhances the detector by exploiting alignment features, e.g. R$^3$Det \citep{yang2021r3det} and S$^2$A-Net \citep{han2022align}. The angle regression may face boundary discontinuity and remedies are developed, including modulated losses \citep{yang2019scrdet, yang2022scrdet++, qian2021rsdet} that alleviate loss jumps, angle coders \citep{yang2020arbitrary, yang2021dense, yang2022arbitrary, yu2023psc} that convert the angle into boundary-free coded data, and Gaussian-based losses \citep{yang2021rethinking, yang2021learning, yang2023detecting, yang2023kfiou, murrugarra2024probabilistic} transforming rotated bounding boxes into Gaussian distributions. RepPoint-based methods \citep{yang2019reppoints, hou2022grep, li2022oriented} provide alternatives that predict a set of points that bounds the spatial extent of an object. LMMRotate \cite{li2025simple} is a new paradigm of OOD based on multimodal language model and performs object localization through autoregressive prediction.

\subsection{Point-supervised Oriented Detection}

Recently, several methods for point-supervised oriented detection have been proposed: \textbf{1)} P2RBox \citep{cao2023p2rbox}, PMHO \citep{zhang2024pmho}, and PointSAM \citep{liu2024pointsam} propose oriented detection with point prompts by employing the zero-shot Point-to-Mask ability of SAM \citep{kirillov2023segment}. \textbf{2)} Point2RBox \citep{yu2024point2rbox} introduces an end-to-end approach based on knowledge combination. \textbf{3)} PointOBB \citep{luo2024pointobb, zhang2025pointobbv3} achieves RBox generation through scale consistency and multiple instance learning. \textbf{4)} PointOBB-v2 \cite{ren2024pointobbv2} learns a class probability map to generate pseudo RBox labels.

Among these methods, P2RBox, PMHO, and PointSAM rely on the SAM model pre-trained on massive labeled datasets, whereas Point2RBox requires one-shot examples for each category. In contrast, PointOBB series do not use many priors, but they necessitate two-stage training. 

\subsection{Other Weakly-supervised Settings} 

Compared to Point-to-RBox, some other settings have been better studied. These methods are potentially applicable to our Point-to-RBox task setting by using a cascade pipeline, such as Point-to-HBox-to-RBox. In our experiment, cascade pipelines powered by state-of-the-art approaches are also compared. Here, representative works are introduced.

\textbf{HBox-to-RBox.} H2RBox \citep{yang2023h2rbox} establishes a paradigm that limits the object to a few candidate angles through geometric constraint from HBoxes, with a self-supervised branch eliminating the undesired results. An enhanced version H2RBox-v2 \citep{yu2023h2rboxv2} is proposed to leverage the reflection symmetry of objects to further boost the accuracy. EIE-Det~\citep{wang2024explicit} uses an explicit equivariance branch for learning rotation consistency, and an implicit equivariance branch for learning position, aspect ratio, and scale consistency. KCR \cite{zhu2023knowledge} combines RBox- and HBox-annotated datasets for transfer learning. Some studies~\citep{iqbal2021leveraging,sun2021oriented} use additional annotated data for training, which are also attractive but less general.

\textbf{Point-to-HBox.} Several related approaches have been developed, including: \textbf{1)} P2BNet \citep{chen2022pointtobox} samples box proposals of different sizes around the labeled point and classify them to achieve point-supervised horizontal object detection. \textbf{2)} PSOD \citep{gao2022weakly} achieves point-supervised salient object detection using an edge detector and adaptive masked flood fill.

\textbf{Point-to-Mask.} Point2Mask \citep{li2023point2mask} is proposed to achieve panoptic segmentation using single point annotation per target. SAM (Segment Anything Model) \citep{kirillov2023segment} produces object masks from input point/HBox prompts. Though RBoxes can be obtained from the segmentation mask by finding the minimum circumscribed rectangle, such a complex pipeline can be less cost-efficient and perform worse \citep{yang2023h2rbox, yu2023h2rboxv2}.
