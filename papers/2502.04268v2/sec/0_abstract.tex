\begin{abstract}
With the rapidly increasing demand for oriented object detection (OOD), recent research involving weakly-supervised detectors for learning OOD from point annotations has gained great attention. 
In this paper, we rethink this challenging task setting with the layout among instances and present Point2RBox-v2. 
At the core are three principles: 
\textbf{1) Gaussian overlap loss.} It learns an upper bound for each instance by treating objects as 2D Gaussian distributions and minimizing their overlap. 
\textbf{2) Voronoi watershed loss.} It learns a lower bound for each instance through watershed on Voronoi tessellation. 
\textbf{3) Consistency loss.} It learns the size/rotation variation between two output sets with respect to an input image and its augmented view. 
Supplemented by a few devised techniques, e.g. edge loss and copy-paste, the detector is further enhanced.
To our best knowledge, Point2RBox-v2 is the first approach to explore the spatial layout among instances for learning point-supervised OOD. Our solution is elegant and lightweight, yet it is expected to give a competitive performance especially in densely packed scenes: 62.61\%/86.15\%/34.71\% on DOTA/HRSC/FAIR1M.
\end{abstract}