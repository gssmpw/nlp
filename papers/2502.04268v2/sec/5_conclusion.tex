\section{Conclusion}
\label{sec:conclusion}

This paper introduces Point2RBox-v2, a point-supervised oriented object detector that effectively leverages the arrangement and layout of instances. We propose the integration of Gaussian overlay and Voronoi tessellation to constrain the size and rotation of instances based on their spatial relationships. Additionally, by incorporating self-supervised consistency loss, edge loss, and copy-paste augmentation, the accuracy of the model is further enhanced.

Experiments yield the following observations: 
\textbf{1)} The integration of Gaussian and Voronoi concepts effectively harnesses the spatial layout of objects, significantly enhancing point-supervised OOD. 
\textbf{2)} Point2RBox-v2 demonstrates exceptional performance in densely packed scenes (see Fig. \ref{fig:vis}), where existing methods struggle. 
\textbf{3)} Our method does not require priors (i.e. pre-trained SAM or one-shot examples) and is applicable to both end-to-end and pseudo-generation modes.
\textbf{4)} It advances the state of the art by a large amount, achieving 62.61\%, 86.15\%, and 34.71\% on the DOTA-v1.0, HRSC, and FAIR1M datasets, respectively.

\textbf{Limitations.} The gap between Point2RBox-v2 and the RBox-supervised OOD is still huge in terms of sparse categories (i.e. BR/SBF) since little constraint can be obtained from the layout between them when the objects are sparse.