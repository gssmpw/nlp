\RequirePackage[hyphens]{url}
\documentclass{article}

% Recommended, but optional, packages for figures and better typesetting:
\usepackage{microtype}
\usepackage{graphicx}
\usepackage{booktabs} % for professional tables

% hyperref makes hyperlinks in the resulting PDF.
% If your build breaks (sometimes temporarily if a hyperlink spans a page)
% please comment out the following usepackage line and replace
% \usepackage{icml2025} with \usepackage[nohyperref]{icml2025} above.
\usepackage{hyperref} 

% Attempt to make hyperref and algorithmic work together better:
\newcommand{\theHalgorithm}{\arabic{algorithm}}

% Use the following line for the initial blind version submitted for review:
%\usepackage{icml2025}
% If accepted, instead use the following line for the camera-ready submission:
\usepackage[accepted]{icml2025}

% For theorems and such
\usepackage{amsmath}
\usepackage{amssymb}
\usepackage{mathtools}
\usepackage{amsthm}

% if you use cleveref..
\usepackage[capitalize,noabbrev]{cleveref}

%%%%%%%%%%%%%%%%%%%%%%%%%%%%%%%%
% THEOREMS
%%%%%%%%%%%%%%%%%%%%%%%%%%%%%%%%
\theoremstyle{plain}
\newtheorem{theorem}{Theorem}[section]
\newtheorem{proposition}[theorem]{Proposition}
\newtheorem{lemma}[theorem]{Lemma}
\newtheorem{corollary}[theorem]{Corollary}
\theoremstyle{definition}
\newtheorem{definition}[theorem]{Definition}
\newtheorem{assumption}[theorem]{Assumption}
\theoremstyle{remark}
\newtheorem{remark}[theorem]{Remark}

% Todonotes is useful during development; simply uncomment the next line
%    and comment out the line below the next line to turn off comments
%\usepackage[disable,textsize=tiny]{todonotes}
%\usepackage[textsize=tiny]{todonotes}

\usepackage[utf8]{inputenc} % allow utf-8 input
\usepackage[T1]{fontenc}    % use 8-bit T1 fonts
\usepackage{microtype}      % microtypography
\usepackage{nicefrac}       % compact symbols for 1/2, etc.
\usepackage{enumitem}

%% Table of contents (needs to come before hyperref!)
\usepackage{titletoc}
\usepackage[page,header]{appendix}

%% Hyperlinks
\usepackage[hyphens]{url}            % simple URL typesetting
\usepackage{hyperref}
\hypersetup{
    colorlinks,
    citecolor=blue,
    filecolor=blue,
    linkcolor=blue,
    urlcolor=blue,
}

%% Figures
\usepackage{graphicx}
\usepackage{chngcntr} % make figure numbering in appendix work
% Suppress weird warnings about pdf page groups
\begingroup
	\expandafter\ifx\csname pdfsuppresswarningpagegroup\endcsname\relax\else\global\pdfsuppresswarningpagegroup=1\relax\fi
\endgroup

%% Tables
\usepackage{makecell}
\usepackage{tabularx}
\usepackage{wrapfig}
\usepackage{booktabs}
\usepackage{multirow}
\usepackage{colortbl}
\usepackage{tablefootnote}
\usepackage{array}
\newcolumntype{P}[1]{>{\centering\arraybackslash}p{#1}}
\usepackage{tikz}
\def\checkmark{\tikz\fill[scale=0.4](0,.35) -- (.25,0) -- (1,.7) -- (.25,.15) -- cycle;}

%%% Math typesetting
\usepackage{amsmath,amssymb,amsfonts,amsthm}
\usepackage{scalerel}
\usepackage{bm}
\DeclareMathOperator*{\argmin}{argmin} % no space, limits underneath in displays
\DeclareMathOperator*{\argmax}{argmax} % no space, limits underneath in displays

%% Small captions
\usepackage{caption}
\usepackage{subcaption}
\captionsetup[table]{font=small,labelfont=small}
\captionsetup[figure]{font=small,labelfont=small}
%\captionsetup[wrapfig]{font=small,labelfont=small}
%\captionsetup[wraptable]{font=small,labelfont=small}

%% Algorithms
%\usepackage{algorithm,algcompatible}
%\usepackage{algorithmicx}
%\usepackage{algpseudocode}

%%% SPACE-REDUCING HACK 1/3
% Reduce spacing around section headers
% \usepackage{titlesec}
% \titlespacing\section{0pt}{0pt plus 2pt minus 0pt}{-2pt plus 2pt minus 0pt}
% \titlespacing\subsection{0pt}{0pt plus 2pt minus 0pt}{-2pt plus 2pt minus 0pt}

% %%% SPACE-REDUCING HACK 2/3
% % Reduce spacing around floats 
% \setlength{\floatsep}{8pt plus 4pt minus 4pt}
% \setlength{\textfloatsep}{8pt plus 4pt minus 3pt}

%% PACKAGES (also see cvpr_header.tex)

\usepackage{graphicx}	
\usepackage{amsmath}	
\usepackage{amssymb}	
\usepackage{booktabs}
\usepackage{times}
\usepackage{microtype}
\usepackage{epsfig}
\usepackage[table,xcdraw,dvipsnames]{xcolor}
\usepackage{caption}
\usepackage{float}
\usepackage{placeins}
\usepackage{color, colortbl}
\usepackage{stfloats}
\usepackage{enumitem}
\usepackage{tabularx}
\usepackage{xstring}
\usepackage{multirow}
\usepackage{xspace}
\usepackage{url}
\usepackage{subcaption}
\usepackage{xcolor}
\usepackage[hang,flushmargin]{footmisc}

% Unfortunately, this package interferes with arxiv's stamp
\ifcamera \usepackage[accsupp]{axessibility} \fi

% added by wang
\usepackage{comment}
\usepackage{booktabs}
\usepackage{multirow}
\usepackage{graphicx}
\usepackage{xfrac}
\usepackage{placeins}
%\usepackage[hidelinks]{hyperref}
\usepackage[T1]{fontenc}
% remove font size warning
\usepackage{lmodern}
\usepackage{type1cm}
%\usepackage{slashbox}
\usepackage{diagbox}
%\usepackage[dvipsnames]{xcolor}
\usepackage{pifont}% http://ctan.org/pkg/pifont

%% MACROS

% \newcommand{\authorname}[1]{{\textcolor{blue}{[Author: #1]}}}
% ...

% \newcommand{\commandname}{string\xspace}
% \definecolor{colorname}{rgb}{0.92,0.49,0.19}

% General

\newcommand{\nbf}[1]{{\noindent \textbf{#1.}}}

\newcommand{\supp}{supplemental material\xspace}
\ifarxiv \renewcommand{\supp}{appendix\xspace} \fi

\newcommand{\todo}[1]{{\textcolor{red}{[TODO: #1]}}}

% Reviewer commands (1 to 5), e.g. \R{1}, \R{2}
\newcommand{\R}[1]{{%
    \textbf{%
        \ifstrequal{#1}{1}{\textcolor{red}{R#1}}{%
        \ifstrequal{#1}{2}{\textcolor{blue}{R#1}}{%
        \ifstrequal{#1}{3}{\textcolor{magenta}{R#1}}{%
        \ifstrequal{#1}{4}{\textcolor{teal}{R#1}}{%
                           \textcolor{cyan}{R#1}%
        }}}}%
    }%
}}

% added by wang
\DeclareMathOperator*{\argmax}{arg\,max}
\DeclareMathOperator*{\argmin}{arg\,min}

\newcommand{\cmark}{\ding{51}}%
\newcommand{\xmark}{\ding{55}}%
\renewcommand{\floatpagefraction}{0.9} % Require 90% full float page before creating one


% The \icmltitle you define below is probably too long as a header.
% Therefore, a short form for the running title is supplied here:
\icmltitlerunning{Learning Hyperparameters via a Data-Emphasized Variational Objective}

\begin{document}
%% Config Table-of-Contents (which appears in the appendix) to NOT show main paper sections
\addtocontents{toc}{\protect\setcounter{tocdepth}{0}}

\twocolumn[
\icmltitle{Learning Hyperparameters via a Data-Emphasized Variational Objective}

% It is OKAY to include author information, even for blind
% submissions: the style file will automatically remove it for you

% List of affiliations: The first argument should be a (short)
% identifier you will use later to specify author affiliations
% Academic affiliations should list Department, University, City, Region, Country
% Industry affiliations should list Company, City, Region, Country

% You can specify symbols, otherwise they are numbered in order.
% Ideally, you should not use this facility. Affiliations will be numbered
% in order of appearance and this is the preferred way.
\icmlsetsymbol{equal}{*}

\begin{icmlauthorlist}
\icmlauthor{Ethan Harvey}{cs}
\icmlauthor{Mikhail Petrov}{meche}
\icmlauthor{Michael C. Hughes}{cs}
\end{icmlauthorlist}

\icmlaffiliation{cs}{Department of Computer Science, Tufts University, Medford, USA}
\icmlaffiliation{meche}{Department of Mechanical Engineering, Tufts University, Medford, USA}

\icmlcorrespondingauthor{Ethan Harvey}{Ethan.Harvey@tufts.edu}

% You may provide any keywords that you
% find helpful for describing your paper; these are used to populate
% the "keywords" metadata in the PDF but will not be shown in the document
\icmlkeywords{Machine Learning, ICML}

\vskip 0.3in
]

% this must go after the closing bracket ] following \twocolumn[ ...

% This command actually creates the footnote in the first column
% listing the affiliations and the copyright notice.
% The command takes one argument, which is text to display at the start of the footnote.
% The \icmlEqualContribution command is standard text for equal contribution.
% Remove it (just {}) if you do not need this facility.

\printAffiliationsAndNotice{}  % leave blank if no need to mention equal contribution
%\printAffiliationsAndNotice{\icmlEqualContribution} % otherwise use the standard text.

\begin{abstract}
%  Runtime Enforcement (RE) is a lightweight method to formally ensure some specified properties on the systems executions. It has attracted more and more interest. Modern systems have real-time requirements - reactive systems. Traditional mechanisms of RE (such as blocking the execution, suppressing or delaying input actions) may not be suitable for such a system. 
% Instead, property enforcement must occur in real-time, allowing the system to continue operating without halting. 

% With this in mind, we propose runtime enforcement of properties defined in Signal Temporal Logic (STL) for reactive systems. STL is a powerful temporal language that can effectively specify signal properties in dense time.
% In this work, we aim to construct a runtime enforcer for a given STL formula that takes a signal as input and outputs a minimally modified signal that satisfies the formula.
% To achieve this, the STL formula to be enforced is first translated into a timed transducer, while the signal to be corrected is encoded as a timed word. The enforcer / enforcement algorithm is then applied to ensure the property is enforced onto the signal. We give timed transducers for the temporal operators "until" and "release".
% Our approach enables effective enforcement of STL properties in reactive systems.


% Title: Runtime Enforcement of CPS using Signal Temporal Logic.

Cyber-Physical Systems (CPSs), especially those involving autonomy, need guarantees of their safety. Runtime Enforcement (RE) is a lightweight method to formally ensure that some specified properties are satisfied over the executions of the system.  Hence, there is recent interest in the RE of CPS. However, existing methods are not designed to tackle specifications suitable for the hybrid dynamics of CPS. With this in mind, we develop runtime enforcement of CPS using properties defined in Signal Temporal Logic (STL).
 
In this work, we aim to construct a runtime enforcer for a given STL formula to minimally modify a signal to satisfy the formula. To achieve this, the STL formula to be enforced is first translated into a timed transducer, while the signal to be corrected is encoded as timed words. We provide timed transducers for the temporal operators \emph{until} and \emph{release} noting that other temporal operators can be expressed using these two. Our approach enables effective enforcement of STL properties for CPS. A case study is provided to illustrate the approach and generate empirical evidence of its suitability for CPS.

\end{abstract}




\keywords{Reactive System, Runtime Enforcement, Signal Temporal Logic, Timed Automata, Timed Transducer}
\section{Introduction}
Safety is a critical consideration in various applications, including robots, autonomous vehicles, smart grids, and transportation control systems~\cite{wolf2017safety}. These safety-critical scenarios demand formal guarantees to ensure that systems operate as expected, as failures may result in severe consequences, such as harm to humans or significant financial costs. Safety verification refers to the task of determining whether a system satisfies a given safety specification over a specified period~\cite{guiochet2017safety, vicentini2019safety}. 
Conventional safe set specifications primarily focus on spatial requirements, ensuring that the system state never enters an unsafe region~\cite{prajna2004safety}. However, as the complexity of autonomous systems increases, many real-world tasks require specifications that are not only spatial but also temporal in nature. For instance, a mobile robot needs to pass Area A before entering Area B. In this paper, we focus on safety verification under Signal Temporal Logic (STL) specification, which uses both boolean and temporal logic operators to formulate constraints for continuous-valued systems~\cite{maler2004monitoring}. 

Real-world systems are subject to various types of uncertainty. It is essential for safety verification algorithms to account for disturbances. Many existing approaches model these uncertainties as bounded disturbances and employ worst-case analysis to guarantee the satisfaction of safety specifications. Examples of such approaches for safe set specification include Hamilton–Jacobi Reachability (HJ Reachability)~\cite{bansal2017hamilton}, reachability analysis, and barrier certificates~\cite{prajna2004safety}. For STL specifications, methods such as HJ Reachability~\cite{chen2018signal} and reachability analysis~\cite{roehm2016stl, lercher2024using, kochdumper2024fully} have been employed to formally verify STL satisfaction under bounded disturbance inputs.


In many practical situations, disturbances are better modeled as stochastic noise, which provides a more realistic representation, as in the case of sensor noise. When considering stochastic disturbances, the aforementioned deterministic methods are not applicable or tend to be overly conservative, as they focus on worst-case scenarios that rarely occur in practice. To better account for stochastic disturbances, we adopt a probabilistic setting, where the goal is to ensure the safety specification is satisfied with high probability, e.g., greater than 99.9\%. 
For safe set specifications, several methods have been proposed to verify stochastic systems, including martingale-based approaches~\cite{steinhardt2012finite, santoyo2021barrier}, risk estimation~\cite{frey2020collision}, and sampling-based methods~\cite{janson2017monte}. Our recent paper significantly reduces the conservativeness of the verification algorithms for safe set specifications~\cite{liu2024safety}.
For STL specifications, most existing approaches are limited to handling the probability constraint for a single trajectory satisfying the STL specification~\cite{sadigh2016safe, farahani2018shrinking, yang2023distributed, vlahakis2024probabilistic, kordabad2024control}. Very few studies have focused on STL verification under both bounded and stochastic disturbances. In \cite{salamati2021data}, a method is proposed to address this problem for linear systems under Gaussian noise. In this work, we focus on the problem of STL verification for nonlinear systems under both bounded and stochastic disturbances. 

In this work, we present a novel framework for verifying the probabilistic STL satisfaction of discrete-time nonlinear stochastic systems. To the best of our knowledge, this is the first approach capable of addressing this problem for nonlinear systems under both bounded and stochastic disturbances. Given a desired probability requirement, our method first erodes the superlevel set of the predicates in an STL formula to get a tighter STL formula. If the deterministic system is verified to satisfy the tighter STL formula, then the stochastic system is guaranteed to satisfy the original STL formula with the specified probability constraint. As a result, the stochastic verification problem is transformed into a deterministic one. The depth of erosion is determined by the sharp probabilistic bound proposed in our previous work~\cite{liu2024probabilistic}, which helps reduce the conservativeness of the verification result, especially when the probability tolerance is low and the time horizon is long. Our method does not rely on restrictive assumptions, such as linear system dynamics or affine predicates, which is common in previous work~\cite{vlahakis2024probabilistic}. This broader applicability makes our approach suitable for real-world applications.



\textit{Notations.}
% \textit{Vectors, matrices, and probability.} 
Denote by $\real$ and $\n$ the sets of real numbers and nonnegative integers, and define $\n_{[a,b]}=\setb{a, a+1, \dots, b}$ where $a,b\in \n$ and $a<b$. Given a vector sequence $\{x_t\}$, define $\boldsymbol{x}_{[t_1,t_2]} = (x_{t_1}, x_{t_1+1}, \dots, x_{t_2}) = [x\tran_{t_1}, x\tran_{t_1+1}, \dots, x\tran_{t_2}]\tran$ , $t\in\n_{[t_1,t_2]}$. When $X_t$ are random vectors, $\boldsymbol{X}_{[t_1,t_2]}$ is a random process. We use $\bP$ to denote probability. A random vector $X \sim \mathcal{N}(\mu, \Sigma)$ follows a multivariate Gaussian distribution with mean $\mu$ and covariance $\Sigma$.
Given a vector $x\in \real^n$, $\|x\|$ denotes the euclidean norm and $\|x\|_P = \sqrt{x\tran P x}$, where $P\in\real^{n\times n}$ is a positive definite matrix.
% \textit{Sets.} 
The $n$ dimensional ball with radius $r$ and center $y$ is denoted by $\BB^n(r, y)=\setb{x\in \real^n : \|x-y\| \leq r}$. Denote the complement of set $A$ as $\setcomp{A}$ and $-B = \setb{-y: \forall y\in B}$. Given sets $A$ and $B$, define the Minkowski sum of $A$ and $B$ by $A\oplus B = \setb{x+y: x\in A,~ y\in B}$, and the Minkowski difference or Pontryargin difference of $A$ and $B$ by $A\ominus B=\setb{x:x+y\in A, \forall y\in B}$ \cite{kolmanovsky1998theory}. The Minkowski sum and difference satisfy the relation $(A\ominus B)\oplus B \subseteq A$.






\section{Contributions}
\label{sec:contributions}

\subsection{Problem setup: ELBo for hyperparameter selection}

Consider a generic template of a probabilistic model for observed data $\{y_i \}_{i=1}^N$, governed by high-dimensional parameters $\theta \in \mathbb{R}^D$. We assume a typical factorization structure:
\begin{align}
    p_{\eta}( y_{1:N}, \theta) = p_{\eta}(\theta ) \cdot  \textstyle \prod_{i=1}^N p_{\eta}( y_i | \theta).
\end{align}
This template is instantiated by specifying a concrete likelihood $p_{\eta}(y_i | \theta)$ and prior $p_{\eta}(\theta)$. The subscript of each indicates possible dependence on hyperparameter vector $\eta$.

The Bayesian approach to fit this model to data involves estimating the posterior distribution $p_{\eta}( \theta | y_{1:N})$, yet this is typically intractable. Instead, a popular approach is to pursue an approximate posterior via variational methods~\citep{blei2017variational,jordan1999introduction}. We first select an ``easy-to-use'' family of distributions $q_r(\theta)$ over the parameter $\theta$. Each concrete value of parameter $r$ defines a specific distribution over $\theta$. We then pose an optimization problem: find the variational parameter $r$ that makes $q_r(\theta)$ as close as possible to the true (intractable) posterior. % in Kullback-Leibler (KL) divergence.

A tractable way to do this is to estimate $r$ by maximizing an objective function known as the \emph{evidence lower bound} (``ELBo'', \citep{blei2017variational}), denoted as $J_{\text{ELBo}}$ and defined for our model $p$ and approximate posterior $q$ as $\JELBO :=$
\begin{align}
    \label{eq:elbo_objective_generic}
    \mathbb{E}_{q_{r}(\theta)} \left[ \sum_{i=1}^{N} \log p_{\eta}(y_i | \theta) \right] - \mathbb{KL}(q_{r}(\theta) \| p_{\eta}(\theta)).
\end{align}
This objective is a function of data $y_{1:N}$, variational parameters $r$, and hyperparameters $\eta$. Maximizing $\JELBO$ can be shown equivalent to finding the specific $q$ that is ``closest'' to the true posterior in the sense of Kullback-Leibler (KL) divergence. Further, as the name of the objective suggests, we can show mathematically that $\JELBO( y_{1:N}, r, \eta ) \leq \log \int_{\theta} p_{\eta}( y_{1:N}, \theta) d\theta$. That is, the ELBo is a lower bound on the log marginal likelihood or \emph{evidence}. The evidence is itself a function of $\eta$.

The bound relation underlying the ELBo suggests its potential utility for data-driven selection of hyperparameters $\eta$. 
Indeed, past work has used the ELBo to select hyperparameters in several contexts~\citep{uedaBayesianModelSearch2002,damianouDeepGaussianProcesses2013}.
Recent theoretical results~\citep{cherief2019consistency} argue that using ELBo for model selection enjoys strong theoretical guarantees on quality, even under misspecification.
Unfortunately, practical efforts remain dominated by grid search ~\citep{osawa2019practical,shwartz2022pre} rather than gradient-of-ELBo learning of hyperparameters.

In a similar line of work, \citet{immer2021scalable} use Laplace's approximation to estimate the marginal likelihood for hyperparameter selection.
While effective, this approach relies on expensive approximations of the Hessian, which are at least $\mathcal{O}(D^2N)$, and an inner loop to learn hyperparameters.

\subsection{Contribution: Data-emphasized ELBo}

Consider applications of very flexible models where dataset size $N$ is limited relative to the parameter size $D$.
In such cases, where $D \gg N$, classical arguments from statistical learning hardly favor large models due to the risk of overfitting; despite this, well-trained flexible models often enjoy practical success~\citep{sharif2014cnn,xuhong2018explicit}.
Our work offers two core contributions to improve hyperparameter tuning in such applications. 

First, we point out that straightforward use of the ELBo in the $D \gg N$ regime can be \emph{overly conservative} in hyperparameter selection. Selected hyperparameters often prefer simpler models with far worse performance at the target prediction task. 
See our own brief demo in Fig.~\ref{fig:elbo_comparison}, as well as our two later case studies.
This finding is corroborated by \citet{blundell2015weight}, who reportedly tried to learn hyperparameters of Bayesian neural nets via gradients of the ELBo, but found it ``not be useful, and yield worse results.''

Second, we suggest a modified ELBo objective that can overcome this issue by better emphasizing the data likelihood. Concretely, we suggest this \emph{data-emphasized} ELBo objective $J_{\text{DE ELBo}} :=$
\begin{align}
    \label{eq:de_elbo_objective}
    \kappa \cdot \mathbb{E}_{q_r(\theta)} \left[ \sum_{i=1}^{N} \log p_{\eta}(y_i | \theta) \right] 
    - \mathbb{KL}(q_r(\theta) \| p_{\eta}(\theta) ).
\end{align}
where we have introduced a scaling factor $\kappa$ on the likelihood term. 
When using the standard ELBo ($\kappa = 1$) in our target applications, we find the KL term comparing distributions over the high-dimensional random variable $\theta$ dominates the overall objective. By setting $\kappa > 1$, our approach emphasizes data fit more and overcomes this imbalance.
Concretely, in all our case studies we recommend setting $\kappa = D/N$ to achieve an improved balance between the expected likelihood and KL terms.

This objective with varying $\kappa$ has been used in previous work on \emph{tempering} the variational ELBo~\citep{mandt2016variational,aitchison2021statistical,pitas2024fine}. This past work was not directly motivated by concerns about hyperparameter tuning as we are. Instead, they sought to improve the posterior's downstream generalization performance or counter misspecification in the probabilistic model. See Sec.~\ref{sec:related_work} for a thorough discussion.

\textbf{Justification.} Beyond later empirical success across two case studies, we offer two arguments suggesting this revised objective as suitable for hyperparameter selection.
One justification that applies to all models in our framework views this approach as acting as if we are modeling $\kappa N$ \emph{i.i.d.} data instances, and we just happen to observe $\kappa$ copies of the size-$N$ dataset $y_{1:N}$. Under these assumptions, the ELBo of this replicated data is exactly the DE ELBo of the original data. Another justification applies only when each $y_i$ is a discrete random variable, such as a 1-of-C class label. In such cases, the log likelihood upweighted by $\kappa > 1$ is a log PMF and thus always negative ($\mathbb{E}_{q}[\log p( y_i | \theta)] \leq 0$).
Thus, the function in Eq.~\eqref{eq:de_elbo_objective} is still a valid lower bound on the evidence $\log p( y_{1:N} )$. While the bound may be ``loose'' in an absolute sense, what matters more is which hyperparameter values $\eta$ it favors.

\begin{figure}[t!]
  \centering
  \includegraphics[width=\linewidth]{images/elbo_comparison.pdf}
  \caption{Model selection comparison between the ELBo (left) and our \emph{data-emphasized ELBo} (DE ELBo, right) for two different approximate posteriors $q$ defined in Case Study B: ResNet-50s with $D$ in millions trained on CIFAR-10 with $N=1000$. Each panel varies hyperparameters $\eta = \lambda = \tau$.
  \textbf{Takeaway: When $D \gg N$, the ELBo prefers simpler $q$ (\pink{pink}), while our DE ELBo favors $q$ that produce higher test accuracy (\purple{purple}).}}
  \label{fig:elbo_comparison}
\end{figure}

\textbf{Demo: Improved selection with $\kappa=D/N$.}
In Fig.~\ref{fig:elbo_comparison}, we compare $\kappa=1$ (left panel) and $\kappa=D/N$ (right) for models defined later in Case Study B: deep neural net classifiers originally trained on ImageNet that are transfered to the CIFAR-10 classification task. In each plot, we compare two possible $q$, colored \emph{pink} and \emph{purple}, across a range of $\eta$ values. The pink version fixes neural net parameters and variational parameters to a solution favored by ELBo, with known test-set accuracy 28.6\%. The purple version instead optimizes these parameters to other values via the \emph{data-emphasized ELBo}, with known test-set accuracy of 87.3\%. We see the poor-accuracy model is strongly favored by the conventional ELBo ($\kappa{=}1$), while our DE ELBo ($\kappa{=}D/N$) prefers the more accurate model.

\input{sec_related_work}
\begin{figure*}[!t]
\begin{center}
	\includegraphics[width=\textwidth]{images/regression_classification_demo.pdf}
\begin{subfigure}{1.3246in}
	\subcaption{\makecell{RFF ($\gamma{=}1.0, \nu{=}1.0$)\\\baseline}}
\end{subfigure}
\begin{subfigure}{1.3246in}
	\subcaption{\makecell{RFF ($\gamma{=}\sqrt{20}, \nu{=}1.0$)\\\baseline}}
\end{subfigure}
\begin{subfigure}{1.3246in}
	\subcaption{\makecell{RFF (learned $\gamma,\nu$)\\ELBo}}
\end{subfigure}
\begin{subfigure}{1.3246in}
	\subcaption{\makecell{RFF (learned $\gamma,\nu$)\\DE ELBo (ours)}}
\end{subfigure}
\begin{subfigure}{1.3246in}
	\subcaption{\makecell{GP (learned $\gamma,\nu$)\\Marginal likelihood}}
\end{subfigure}
\end{center}
\vspace{-6pt}
\caption{
\textbf{Case Study A: Demo of hyperparameter sensitivity and selection for RFF models.}
The first four columns use the RFF regression/classification model in Sec.~\ref{sec:caseA_model}, varying estimation and selection techniques.
The last column shows the reference fit of a GP, a gold-standard for this toy data but less scalable.
Columns a and b show MAP point estimation, with fixed $\gamma,\nu$ and grid search (GS) for $\tau$ (separate GD run for each value).
Columns c and d use variational methods, \emph{learning} $\gamma,\nu,\tau$ in one run of GD.
For the regression dataset, we plot the mean function of $y$ over samples from $q$.
Our DE ELBo objective enables the best approximation of the GP.
}%endcaption
\label{fig:caseA_regression_classification_demo}
\end{figure*}

\section{Case Study A: Random Fourier features}

To motivate this first case study, recall the Gaussian process (GP) \citep{rasmussen2006gaussian} as a flexible model for regression or classification.
A common choice for GPs is to select a \emph{radial basis function} (RBF) kernel $k(x, x') = \nu^2 \exp \left(- \frac{\|x - x'\|_2^2}{2\gamma^2} \right)$, where $\gamma > 0$ determines the length-scale and $\nu > 0$ the output-scale. 
This GP model allows decision function complexity to elegantly grow with $N$, the number of training instances.  However, a primary downside to GPs is scalability, with classic fitting algorithms scaling cubically with $N$ \citep{rasmussen2006gaussian}. As a possible remedy, we consider a \emph{random Fourier feature} (RFF) \citep{rahimi2007random} approach, which posits an alternative weight-space model that scales \emph{linearly} in $N$ during fitting yet approximates a GP. The approximation quality increases with a user-controlled size parameter $R$, yet $R$ increases the overall parameter dimension $D$. 

GPs are notoriously sensitive to hyperparameters \citep{rasmussen2006Varying}. RFFs are also sensitive, yet tuning hyperparameters well has been largely overlooked. \citet{rahimi2007random} just fix $\gamma = 1,\nu=1$. When \citet{liu2020simple,liu2023simple} used RFFs to build distance-aware neural net classifiers, they recommend $\gamma = 2$ (though released code sets $\gamma = \sqrt{20}$) and tune $\nu$ on a heldout validation set via expensive search. 

We intend to show how fitting this RFF weight-space model via a data-emphasized ELBo yields effective and affordable hyperparameter learning of $\gamma, \nu$ in the $D \gg N$ regime. Better hyperparameter learning translates to notable gains in downstream prediction quality, better matching the ideal GP on small datasets (as in Fig.~\ref{fig:caseA_regression_classification_demo}) but scaling better for large $N$. Our work here could be used as an efficient (\emph{linear in $N$}) approximation of GPs on large datasets, or as a drop-in modification of \citeauthor{liu2020simple}'s distance-aware neural nets to enable even better downstream performance.

\subsection{Model A definition}
\label{sec:caseA_model}

We assume a training set of $N$ pairs $x_i, y_i$ of feature vector $x_i \in \mathbb{R}^H$ and corresponding class label $y_i \in \{1, 2, \ldots C\}$.
We define a simple linear regression/classification model using RFFs of user-defined size $R$.
First, draw random values that define RFF weights $A \in \mathbb{R}^{H \times R}$ and $b \in \mathbb{R}^R$; once drawn they are fixed throughout training and prediction.
Next, use fixed $A,b$ and a cosine function to non-linearly map each $x_i$ to a representation vector $\phi(x_i) \in \mathbb{R}^{R}$. Formally, these two steps are:
\begin{align}
    A_{h,r} &\sim \mathcal{N}(0, 1),
    ~b_{r} \sim \text{Unif}(0, 2\pi)
    \text{~for all~} r,h. \\
    \phi(x_i) &= \nu \sqrt{\frac{2}{R}} \cos\left(\frac{1}{\gamma} A^T x_i + b\right).
    \label{eq:phi}
\end{align}
To complete the model, a linear layer with weights $V \in \mathbb{R}^{C \times R}$ maps $\phi(x_i)$ to predicted regression targets or logit-probabilities over $C$ classes. 

\textbf{Contribution: RFF for arbitrary length-scale.} 
Our featurization in Eq.~\eqref{eq:phi} generalizes the classic construction of RFFs by \citet{rahimi2007random} to any length-scale $\gamma > 0$ and output-scale $\nu > 0$.
In the appendix, we prove that our RFF construction allows a Monte Carlo approximation of the RBF kernel. That is, for any pair of feature vectors we have $\phi(x_i)^T \phi(x_j) \approx k(x_i, x_j)$, where $k$ is an RBF kernel whose $\gamma,\nu$ values match those used to construct $\phi$ in Eq.~\eqref{eq:phi}. The quality of this approximation increases with $R$, typically $100 < R < 10000$. Past work has proven this when $\gamma{=}1,\nu{=}1$ \citep{rahimi2007random}; however, to the best of our knowledge our use of RFF with an arbitrary $\gamma$ is potentially novel.
We later show how to \emph{learn} $\gamma,\nu$ with substantial practical utility.

\textbf{Point estimation view.} To fit the RFF model to observed data, an  empirical risk minimization strategy would find the value of weights $V$ that minimizes the loss function $L(V) := $
\begin{align}
  \label{eq:rff_loss_with_l2_penalty}
  \sum_{i=1}^N \ell(y_i, V\phi(x_i) ) + \frac{\tau^{-1}}{2} || \text{vec}(V) ||_2 ^2,
\end{align}
where $\ell$ is an appropriate loss function for prediction quality (e.g., mean squared error, cross entropy). The sum-of-squares (L2) penalty on $V$ encourages lower magnitude weights to avoid overfitting on the high-dimensional features. Ultimately, model quality is impacted by 3 key hyperparameters: length-scale $\gamma > 0$, output-scale $\nu > 0$, and L2-penalty strength $\tau > 0$. None of these can be effectively set using the training loss alone.

\textbf{Bayesian view.}
A Bayesian interpretation of the RFF problem assumes a joint probabilistic model $p(V, y_{1:N}) = p(V) \prod_{i} p(y_i | V)$, with factors for the classification case
\begin{align}
    \label{eq:rff_joint_pdf_model}
    p(V) &= \mathcal{N}( \text{vec}(V) | 0, \tau I), \\
    p(y_i | V) &= \text{CatPMF}( y_i | \textsc{sm}( V\phi(x_i) ) ).
\end{align}
For regression, instead of the softmax function $\textsc{sm}$ we'd just model $y_i$ as normally distributed with mean $V \phi(x_i)$. 
In either case, $\tau > 0$ is a hyperparameter controlling under/over-fitting.
%Under this model, each feature $x_i$ is a known fixed value, not a random variable; we leave fixed quantities out of conditioning notation for simplicity.
Maximum a-posteriori (MAP) estimation of $V$ recovers the objective in Eq.~\eqref{eq:rff_loss_with_l2_penalty} when we set $\ell$ to $- \log p(y_i | V)$. To fit this model into our general framework, we have $\theta = \{V\}$ and $\eta = \{\gamma, \nu, \tau\}$.

\subsection{Variational methods for Model A}
\label{sec:caseA_variational}

To apply the general variational recipe described in Sec.~\ref{sec:contributions} to model A, we first select an approximate posterior over parameter $V$. For simplicity, we chose a Normal with unknown mean and isotropic covariance.
\begin{align}
    q(V) &= \mathcal{N}( \text{vec}(V) | \text{vec}(\bar{V}), \bar{\sigma}^2 I).
\end{align}
Here, the free parameters that define $q$ are $r = \{\bar{V}, \bar{\sigma} \}$.

\subsection{Implementation for Model A}

\textbf{Learning $\tau$.}
To learn the prior precision $\tau$, we make use of closed-form expressions for the KL divergence between two Gaussians. We can solve for the optimal $\tau$ given other parameters and hyperparameters (see App.~\ref{sec:learning_lambda_tau}).

\textbf{Learning $\nu,\gamma$.}
To learn other hyperparameters, no such closed-form update exists. We employ first-order gradient descent as the gradient of each ELBo-based objective $J$ is easily computed with respect to $\nu$ and $\gamma$ via automatic differentiation. We reparameterize to handle the positivity constraints with the invertible softplus mapping, so that all gradient descent is done on unconstrained parameters.

\textbf{Training.}
For this case study A, our toy datasets are small enough that we perform gradient descent without minibatching. At each step of our ELBo or DE ELBo methods, we update variational parameters $r$ and hyperparameters $\gamma,\nu$ via a gradient step and then update $\tau$ to its closed-form optimal value. We run for a specified number of iterations, verifying convergence by manual inspection.

\textbf{Competitors.}
We compare to a baseline that performs MAP point-estimation of $V$, with a separate gradient descent run at each candidate $\tau$ configuration in a fixed grid and fix $\gamma,\nu$ to previously recommended values.
Given only small toy data, we pick the $\tau$ that delivers best train-set likelihood.
\subsection{Evaluation for Model A}

In Fig.~\ref{fig:caseA_regression_classification_demo}, we compare the decision functions of different modeling pipelines on a simple univariate regression dataset inspired by Fig.~2.2 in \citet{rasmussen2006gaussian} (top row) and the \emph{two moons} classification task (bottom).
The goal here is to illustrate sensitivity to hyperparameters $\eta = \{\gamma,\nu,\tau\}$ and the effective learning of $\eta$ enabled by our approach. 
Both dataset sizes are rather small ($N=5$ on top, $N=20$ on bottom).
For all RFFs, we set $R=1024$, so the effective dimension is $D=1024$. %, thus $D \gg N$.

For reference, the last column of the figure shows an ideal GP with hyperparameters tuned to optimize marginal likelihood (exact in the regression case, approximate for classification via a Dirichlet-based GP \citep{milios2018dirichlet}).
The GP is a ``gold standard'' for these toy tasks, yet the GP does not scale well to larger $N$.
While RFF models are in the $D \gg N$ regime we intend for our DE ELBo, the GP's natural formulation is not.

\textbf{Results.}
First, results in columns 1-2 of Fig.~\ref{fig:caseA_regression_classification_demo} show that hyperparameters matter. Past work on RFFs \cite{liu2020simple} recommended $\gamma = \sqrt{20}$, yet we see here that different $\gamma$ are preferred across the top and bottom datasets.

Next, we find the posteriors and hyperparameters estimated with our DE ELBo (column 4) result in better approximations of the gold-standard GP (column 5). Using the conventional ELBo (column 3), the chosen hyperparameters produce decision functions that \emph{underfit}, because the KL term dominates the unweighted likelihood when $D \gg N$.

\section{Case Study B: Transfer learning}

For case study B, we investigate transfer learning of image classifiers using informative priors~\citep{xuhong2018explicit,shwartz2022pre}.

\subsection{Model B definition}

Consider training a neural network classifier composed of two parts. First, a backbone encoder $f$ with weights $w \in \mathbb{R}^D$, non-linearly maps input vector $x_i$ to a representation vector $z_i \in \mathbb{R}^H$ (which includes an ``always one'' feature to handle the need for a bias/intercept). Second, a linear-decision-boundary classifier head with weights $V \in \mathbb{R}^{C \times H}$, leads to probabilistic predictions over $C$ possible classes. We wish to find values of these parameters that produce good classification decisions on a provided \emph{target task} dataset of $N$ pairs $x_i, y_i$ of features $x_i$ and corresponding class labels $y_i \in \{1, 2, \ldots C\}$. For transfer learning, we assume the backbone weights $w$ have been pretrained to high-quality values $\mu$ on a source task.

%\textbf{Related work on Bayesian neural nets.}
%Variational inference for neural networks has a long history, dating back to work by \citet{hinton1993keeping} and \citet{graves2011practical}. Modern efforts include MOdel Priors Extracted from Deterministic DNN (MOPED) \cite{krishnan2019efficient}, which focuses on using informative priors to enable scalable variational inference, not to select regularization hyperparameters.

\textbf{Deep learning view.}
Typical approaches to transfer learning in the deep learning tradition (e.g., baselines in \citet{xuhong2018explicit}) would pursue empirical risk minimization with an L2-penalty on weight magnitudes for regularization, training to minimize the loss function $L(w, V) := $
\begin{align}
  \label{eq:loss_with_l2_penalty}
  \sum_{i=1}^N \ell(y_i, V f_w( x_i) ) + \frac{\alpha}{2} || w ||_2 ^2 + \frac{\beta}{2} || \text{vec}(V) ||_2 ^2 
\end{align}
where $\ell$ represents a cross-entropy loss indicating agreement with the true label $y_i$ (one of $C$ categories), while the L2-penalty on weights $w,V$ encourages their magnitude to \emph{decay} toward zero, and this regularization is thus often referred to as ``weight decay''.
The key hyperparameters $\alpha \geq 0$, $\beta \geq 0$ encode the strength of the L2 penalty, with higher values yielding simpler representations and simpler decision boundaries.
Model training would thus consist of solving $w^*, V^* \gets \arg\!\min_{w,V} L(w,V)$ via stochastic gradient descent, given fixed hyperparameters $\alpha, \beta$. 
In turn, the values of $\alpha,\beta$ would be selected via grid search seeking to optimize $\ell$ or error on a validation set. 

\textbf{Bayesian view.} Bayesian interpretation of this neural classification problem would define a joint probabilistic model $p(w, V, y_{1:N})$ decomposed into factors $p(w)p(V) \prod_{i} p(y_i | w, V)$ defined as:
\begin{align}
    \label{eq:joint_pdf_model}
    p(w) &= \mathcal{N}( w | \mu_p, \lambda \Sigma_p), \\
    p(V) &= \mathcal{N}( \text{vec}(V) | 0, \tau I), \\
    p(y_i | w, V) &= \text{CatPMF}( y_i | \textsc{sm}( V f_w(x_i) ) ).
\end{align}
Here, $\lambda > 0, \tau > 0$ are  hyperparameters controlling under/over-fitting, $\mu_p, \Sigma_p$ represent \emph{a priori} knowledge of the mean and covariance of backbone weights $w$ (see paragraph below), and $\textsc{sm}$ is the softmax function.
% MCH: cut for space
% Note $x_i$ is a known fixed value, not a random variable in the model; we leave such fixed quantities out of probabilistic conditioning notation for simplicity.
Pursing MAP estimation of both $w$ and $V$ recovers the objective in Eq.~\eqref{eq:loss_with_l2_penalty} when we set
$\alpha = \frac{1}{\lambda}, \beta = \frac{1}{\tau}, \mu_p=0, \Sigma_p=I$, and $\ell$ to $- \log p(y_i | w, V)$. To fit this model into our general framework, we have $\theta = \{w, V\}$ and $\eta = \{\lambda, \tau\}$.

\textbf{Need for validation set and grid search.}
Selecting $\alpha,\beta$ (or equivalently $\lambda,\tau$) to directly minimize Eq.~\eqref{eq:loss_with_l2_penalty} on the training set alone is not a coherent way to guard against over-fitting. Regardless of data content or weight parameter values, we would select $\alpha^* = 0,\beta^* = 0$ to minimize $L$ as a function of $\alpha,\beta$ and thus enforce no penalty on weight magnitudes at all. Carving out a validation set for selecting these hyperparameters is thus critical to avoid over-fitting when point estimating $w,V$.

\input{fig_ConvNeXt_Tiny_computational_time_comparison}

\setlength{\tabcolsep}{4pt}
\setlength{\columnsep}{6pt} % Default is 18.06749pt
\begin{wraptable}[7]{R}{117.43875pt}
    \caption{Possible priors.}
    \label{tab:transfer_learning_methods}
    \centering
    \footnotesize
    \begin{tabular}{lcc}
        \hline
        \bfseries Method & \bfseries $p(w)$ & Init. \\
        \hline
        L2-zero & $\mathcal{N}(0, \lambda I)$ & $\mu$ \\
        L2-SP & $\mathcal{N}(\mu, \lambda I)$ & $\mu$ \\
        PTYL & $\mathcal{N}(\mu, \lambda \Sigma)$ & $\mu$ \\
        \hline
    \end{tabular}
\end{wraptable}
\setlength{\tabcolsep}{6pt}
\textbf{Backbone priors.}
Several recent transfer learning approaches correspond to specific settings of the mean and covariance $\mu_p, \Sigma_p$ of the backbone prior $p(w)$. Let $\mu$ represent a specific pretrained vector of backbone weights that performs well on a source task.
Setting $\mu_p{=}0, \Sigma_p{=}I$ recovers a conventional approach to transfer learning, which we call L2-zero, where regularization pushes backbone weights to zero. The pretrained value $\mu$ only informs the initial value of backbone weights $w$ before SGD~\citep{xuhong2018explicit,harvey2024transfer}.  Instead, setting the prior mean as $\mu_p = \mu$ along with $\Sigma_p = I$ recovers \emph{L2 starting point} (L2-SP) regularization~\citep{xuhong2018explicit}, also suggested by \citet{chelba2006adaptation}. Further setting $\Sigma_p$ to the estimated covariance matrix $\Sigma$ of a Gaussian approximation of the posterior over backbone weights for the source task recovers the ``Pre-Train Your Loss'' (PTYL) method~\citep{shwartz2022pre}.

\textbf{Need to specify a search space.}
Selecting $\alpha,\beta$ (or equivalently $\lambda,\tau$) via grid search requires practitioners to specify a grid of candidate values spanning a finite range.
For the PTYL method, the optimal search space for these key hyperparameters is still unclear.
For the same prior and the same datasets, the search space has varied between works: PTYL's creators recommended large values from 1e0 to 1e10~\citep{shwartz2022pre}.
Later works search far smaller values (1e-5 to 1e-3)~\citep{rudner2024finetuning}.

\subsection{Variational methods for Model B}

To apply the general variational recipe described in Sec.~\ref{sec:contributions} to model B, we first select an approximate posterior over parameters $w, V$. For simplicity, we chose a factorized Normal with unknown means and isotropic covariance.
\begin{align}
    q(w, V) &= q(w)q(V), \\
    q(w) &= \mathcal{N}( w | \bar{w}, \bar{\sigma}^2 I), \\
    q(V) &= \mathcal{N}( \text{vec}(V) | \text{vec}(\bar{V}), \bar{\sigma}^2 I).
\end{align}
Here, the free parameters that define $q$ are $r = \{\bar{w}, \bar{V}, \bar{\sigma} \}$. 

To fit to data, we estimate $r$ and $\eta$ values that optimize the ELBo or the DE ELBo. To evaluate the KL term in each expression, we use the closed-form available because both prior and $q$ are Gaussian. To evaluate the expected log likelihood term, we use Monte Carlo averaging of $S$ samples from $q$ \citep{xu2019variance,mohamed2020monte}. We further use the \emph{reparameterization trick} to obtain gradient estimates of $\nabla_r J$ \citep{blundell2015weight}.
We find using \emph{just one sample} per training step is sufficient and most efficient.

\setlength{\tabcolsep}{2pt}
\begin{table*}[t!]
  \caption{Computational time comparison between methods for transfer learning with informative priors using \emph{\baselineLong} and our \emph{data-emphasized ELBo} (DE ELBo) for CIFAR-10 $N=50000$. See App.~\ref{sec:classifier_details} for search space details. Runtime measured on four Intel Xeon Gold 6226R CPUs (2.90 GHz) and one NVIDIA A100 GPU (40 GB).}
  \label{tab:computational_time_comparison}
  \centering
  \scriptsize
  \begin{tabular}{llcccccccc}
    \hline
    & & & & \multicolumn{2}{c}{ResNet-50} & \multicolumn{2}{c}{ViT-B/16} & \multicolumn{2}{c}{ConvNeXt-Tiny} \\
    \bfseries Model & \bfseries Method & \bfseries lr search space & \bfseries $\lambda,\tau$ search space & \bfseries Avg. SGD runtime & \bfseries Total GS time & \bfseries Avg. SGD runtime & \bfseries Total GS time & \bfseries Avg. SGD runtime & \bfseries Total GS time \\
    \hline
    \rowcolor{bright-gray} L2-SP & \baseline & 4 & 36 & 36 mins. 37 secs. & \phantom{0}88 hrs. 29 mins. & 2 hrs. 16 mins. & 328 hrs. 56 mins. & 1 hrs. 15 mins. & 181 hrs. 53 mins. \\    
    & DE ELBo & 4 & n/a & 32 mins. 44 secs. & \phantom{00}2 hrs. 10 mins. & 2 hrs. 22 mins. & \phantom{00}9 hrs. 29 mins. & 1 hrs. \phantom{0}6 mins. & \phantom{00}4 hrs. \phantom{0}2  mins. \\
    \rowcolor{bright-gray} PTYL & \baseline & 4 & 60 & 37 mins. \phantom{0}2 secs. & 148 hrs. 43 mins. & n/a & n/a & n/a & n/a \\    
    & DE ELBo & 4 & n/a & 35 mins. 55 secs. & \phantom{00}2 hrs. 23 mins. & n/a & n/a & n/a & n/a \\
    \hline
  \end{tabular}
\end{table*}
\setlength{\tabcolsep}{6pt}

\setlength{\tabcolsep}{2pt}
\begin{table*}[t!]
  \caption{Accuracy on CIFAR-10, Pet-37, and Flower-102 test sets for different probabilistic models, methods, and backbones. We report mean (min-max) over 3 separately-sampled training sets. The \emph{\baselineLong} baseline requires 24 different SGD runs for L2-zero, 144 for L2-SP, and 240 for PTYL, while our \emph{data-emphasized ELBo} (DE ELBo) requires 4 different SGD runs.
  }%end caption
  \label{tab:acc_subset}
  \centering
  \scriptsize
  \begin{tabular}{llcccccccc}
    \hline
    & & \multicolumn{4}{c}{CIFAR-10} & \multicolumn{2}{c}{Pet-37} & \multicolumn{2}{c}{Flower-102} \\
    \bfseries Model & \bfseries Method & $N =$ {\bfseries 100 (10/cl.)} & \bfseries 1000 (100/cl.) & \bfseries 10000 (1k/cl.) & \bfseries 50000 (5k/cl.) & \bfseries 370 (10/cl.) & \bfseries 3441(93/cl.) & \bfseries 510 (5/cl.) & \bfseries 1020 (10/cl.) \\
    \hline
    \multicolumn{10}{c}{ResNet-50} \\
    %\rowcolor{bright-gray} Linear probing & \baseline & 60.1 {\tiny(59.3-61.2)} & 74.5 {\tiny(74.2-74.8)} & 81.4 {\tiny(81.3-81.4)} & 83.2 {\tiny(83.1-83.2)} & 85.9 {\tiny(85.2-86.5)} & 90.8 {\tiny(90.8-90.8)} & 80.6 {\tiny(79.5-81.5)} & 87.9 {\tiny(87.9-88.1)} \\
    %& DE ELBo & 58.4 {\tiny(57.8-58.9)} & 73.6 {\tiny(73.2-74.1)} & 81.6 {\tiny(81.5-81.8)} & 83.2 {\tiny(83.1-83.3)} & 85.6 {\tiny(85.0-85.9)} & 90.7 {\tiny(90.6-90.9)} & 80.8 {\tiny(79.6-81.6)} & 88.0 {\tiny(87.8-88.2)} \\
    \rowcolor{bright-gray} L2-zero & \baseline & 67.7 {\tiny(66.0-69.2)} & 87.7 {\tiny(87.2-88.2)} & 94.6 {\tiny(93.8-95.1)} & 97.0 {\tiny(96.9-97.1)} & 87.6 {\tiny(86.7-88.4)} & 93.1 {\tiny(92.9-93.4)} & 86.4 {\tiny(86.0-86.7)} & 92.8 {\tiny(92.4-93.2)} \\
    & DE ELBo & 62.0 {\tiny(60.7-63.4)} & 87.4 {\tiny(87.0-87.8)} & 91.3 {\tiny(90.9-91.7)} & 93.0 {\tiny(92.9-93.4)} & 83.3 {\tiny(83.2-83.3)} & 91.6 {\tiny(91.6-91.7)} & 81.2 {\tiny(81.0-81.4)} & 89.9 {\tiny(89.3-90.5)} \\
    \rowcolor{bright-gray} L2-SP & \baseline & 69.0 {\tiny(65.9-72.3)} & 88.7 {\tiny(88.3-89.4)} & 95.4 {\tiny(95.3-95.6)} & 97.3 {\tiny(97.2-97.4)} & 88.0 {\tiny(87.1-88.8)} & 92.9 {\tiny(92.6-93.1)} & 86.1 {\tiny(85.3-86.7)} & 93.2 {\tiny(93.1-93.3)} \\
    & DE ELBo & 69.9 {\tiny(67.4-71.9)} & 88.5 {\tiny(86.9-89.5)} & 95.1 {\tiny(95.0-95.2)} & 96.8 {\tiny(96.7-96.9)} & 87.5 {\tiny(87.2-87.8)} & 92.9 {\tiny(92.7-93.0)} & 84.5 {\tiny(84.5-84.6)} & 91.8 {\tiny(91.6-92.1)} \\
    %\rowcolor{bright-gray} PTYL (SSL) & \baseline & 57.5 {\tiny(56.1-58.6)} & 78.4 {\tiny(77.8-79.0)} & 90.6 {\tiny(90.1-90.8)} & 96.6 {\tiny(96.5-96.6)} & 58.3 {\tiny(55.6-60.7)} & 86.3 {\tiny(85.9-86.8)} & 81.9 {\tiny(80.9-82.8)} & 89.7 {\tiny(89.2-90.1)} \\
    %& DE ELBo & 60.2 {\tiny(59.5-60.7)} & 78.1 {\tiny(77.5-78.8)} & 90.6 {\tiny(90.3-90.8)} & 96.7 {\tiny(96.6-96.7)} & 56.6 {\tiny(56.1-57.2)} & 80.1 {\tiny(80.0-80.3)} & 76.8 {\tiny(76.5-77.2)} & 84.9 {\tiny(84.6-85.1)} \\
    \rowcolor{bright-gray} PTYL & \baseline & 70.1 {\tiny(69.2-71.4)} & 89.8 {\tiny(89.5-90.3)} & 95.6 {\tiny(95.5-95.8)} & 97.0 {\tiny(96.8-97.2)} & 87.9 {\tiny(87.5-88.2)} & 93.0 {\tiny(92.8-93.2)} & 86.3 {\tiny(85.8-86.6)} & 92.9 {\tiny(92.6-93.1)} \\
    & DE ELBo & 70.0 {\tiny(67.9-72.1)} & 89.2 {\tiny(89.0-89.5)} & 95.1 {\tiny(94.9-95.4)} & 96.9 {\tiny(96.8-97.0)} & 87.5 {\tiny(87.3-87.8)} & 92.9 {\tiny(92.7-93.0)} & 84.6 {\tiny(84.5-84.6)} & 91.8 {\tiny(91.7-92.0)} \\
    \hline
    \multicolumn{10}{c}{ViT-B/16} \\
    \rowcolor{bright-gray} L2-SP & \baseline & 86.5 {\tiny(85.3-87.7)} & 93.8 {\tiny(93.3-94.0)} & 97.2 {\tiny(97.1-97.2)} & 98.2 {\tiny(98.2-98.2)} & 88.1 {\tiny(87.1-89.4)} & 93.3 {\tiny(93.2-93.5)} & 85.8 {\tiny(84.9-86.6)} & 91.4 {\tiny(88.3-93.0)} \\
    & DE ELBo & 88.5 {\tiny(87.1-89.8)} & 94.1 {\tiny(94.0-94.2)} & 97.4 {\tiny(97.3-97.6)} & 98.2 {\tiny(98.1-98.3)} & 88.9 {\tiny(88.3-89.5)} & 93.6 {\tiny(93.5-93.7)} & 80.5 {\tiny(78.0-84.0)} & 89.2 {\tiny(88.5-89.6)} \\
    \hline
    \multicolumn{10}{c}{ConNeXt-Tiny} \\
    \rowcolor{bright-gray} L2-SP & \baseline & 85.4 {\tiny(84.3-86.4)} & 94.1 {\tiny(93.9-94.3)} & 96.9 {\tiny(96.7-97.0)} & 97.9 {\tiny(97.8-98.0)} & 89.1 {\tiny(87.4-90.4)} & 94.3 {\tiny(94.2-94.5)} & 88.7 {\tiny(88.3-89.3)} & 94.2 {\tiny(92.3-95.7)} \\
    & DE ELBo & 83.9 {\tiny(82.9-85.3)} & 94.5 {\tiny(94.4-94.6)} & 97.1 {\tiny(97.0-97.1)} & 97.8 {\tiny(97.8-97.8)} & 90.7 {\tiny(89.7-91.5)} & 94.3 {\tiny(94.2-94.5)} & 88.0 {\tiny(86.8-89.0)} & 94.1 {\tiny(93.9-94.2)} \\    
    \hline
  \end{tabular}
\end{table*}
\setlength{\tabcolsep}{6pt}


\subsection{Implementation for Model B}

\textbf{Learning $\lambda,\tau$.}
To learn the prior precisions $\lambda, \tau$, we make use of the closed-form expression for the KL term in our ELBo objectives.
For example, setting $\nabla_\lambda J = 0$ and solving for $\lambda$, we get 
\begin{align}
    \lambda^* = \frac{1}{D} \Big[ \bar{\sigma}^2 \Tr(\Sigma_p^{-1}) + (\mu_p-\bar{w})^T \Sigma_p^{-1} (\mu_p-\bar{w}) \Big]
    \label{eq:lambda_update}
\end{align}
with assurances of a local maximum of $\nabla_\lambda J$ via a second derivative test (see App.~\ref{sec:second_derivative_test}). Similar updates can be derived for $\tau$ (see App.~\ref{sec:learning_lambda_tau}).

\textbf{Training.}
For this case study B, we use stochastic gradient descent (SGD) with batches of 128 to fit to large datasets.
At each step of the algorithm, we update variational parameters $r$ via a gradient step, and then update $\lambda, \tau$ to its closed-form optimal values as in Eq.~\eqref{eq:lambda_update}.
%We run until a specified maximum number of iterations is reached.
Each run of our method and the baseline depends on the adequate selection of learning rate. All runs search over 4 candidate values and select the best according to either the DE ELBo (ours) or validation-set likelihood (baseline).

\textbf{Estimating ELBo.}
Given a fit $q$, we estimate the expected log likelihood term of the ELBo or DE ELBo for final selection decisions by averaging over 10 samples from $q$. 

\textbf{Evaluating accuracy.}
To evaluate classification accuracy for a $q$, we find that just plugging in the estimated posterior mean $\bar{w}, \bar{V}$ to make predictions on heldout images achieves similar accuracy to averaging over 10 posterior samples without the added computational cost.

\textbf{Competitors.}
We compare to a baseline that simply performs MAP point-estimation of $w,V$. This executes a separate SGD run at each candidate $\lambda, \tau$ configuration in a fixed grid (see App.~\ref{sec:classifier_details}).
We select the best according to the validation-set likelihood.
This ``grid search'' baseline is representative of cutting-edge work in transfer learning~\citep{shwartz2022pre,harvey2024transfer}.

Both our method and the MAP baseline can be implemented with any of the 3 settings of the backbone prior in Tab.~\ref{tab:transfer_learning_methods}. For the PTYL method \citep{shwartz2022pre}, we use released code from \citet{harvey2024transfer}, which fixes a key issue in the original implementation so that the learned low-rank covariance $\Sigma_p$ is properly scaled.
We use the Woodbury matrix identity \citep{woodbury1950inverting}, trace properties, and the matrix determinant lemma to compute the trace of the inverse, squared Mahalanobis distance, and log determinant of low-rank covariance matrix $\Sigma_p$ for the KL term. %See App.~\ref{sec:low-rank} for details.

\subsection{Results for Model B}

Across several possible priors for transfer learning, backbone architectures, and datasets, our findings are:

\textbf{The runtime of our DE ELBO is affordable, and avoids the extreme runtime costs of grid search.}
In Tab.~\ref{tab:computational_time_comparison}, we show that an individual SGD run of our DE ELBo has comparable cost to one SGD run of standard MAP estimation. However, the cumulative cost of grid search needed to select $\lambda,\tau$ for the MAP baseline is far higher than our approach: for PTYL the recommended grid search costs over 148 hours; our approach delivers in under 3 hours.

\textbf{Our DE ELBo achieves heldout accuracy comparable to existing approaches with far less compute time.}
In Fig.~\ref{fig:ConvNeXt_Tiny_computational_time_comparison}, we compare test set accuracy on CIFAR-10~\citep{krizhevsky2010cifar}, Oxford-IIIT Pet~\citep{parkhi2012cats}, and Oxford Flower~\citep{nilsback2008automated} images over training time for L2-SP with \baseline\ and our DE ELBo.
Our DE ELBo achieves comparable heldout accuracy with far less compute time.
When training data is limited in size (Pet-37 $N=370$ in Fig.~\ref{fig:ConvNeXt_Tiny_computational_time_comparison}), our approach can perform even better than grid search.
In Tab.~\ref{tab:acc_subset}, we report test-set accuracies for both expensive grid search and our (faster) DE ELBo. See App.~\ref{sec:results} for more results.
\section{Discussion and Conclusion}

We proposed an alternative to grid search: directly learning hyperparameters that control model complexity on the full training set via model selection techniques based on the ELBo.
We showed that a modified ELBo that upweights the influence of the data likelihood relative to the prior improves model selection with limited training data or an overparameterized model.
Experiments for transfer learning examine 3 architectures (ResNet, ViT, and ConvNexT) and 3 distinct datasets (CIFAR-10, Pet-37, and Flower-102), each at several sizes. 
Results showed our DE ELBo achieves heldout accuracy comparable to existing approaches with far less compute time.

\textbf{Learning hyperparameters lets practitioners focus on other aspects that could improve accuracy more.}
Our DE ELBo reduces compute time by directly learning hyperparameters. With dozens of hours of saved time, practitioners can focus on other efforts that improve target task performance, such as better data augmentation or trying other pretrained weights.
For example, on Pet-37 $N=370$, we found that L2-zero using weights pre-trained with supervised learning on ImageNet resulted in a gain of 32.4 percentage points in accuracy over an alternative \emph{self-supervised} pretraining (see Tab.~\ref{tab:acc}).

Our DE ELBo would benefit from further exploration of how $\kappa$ values impact model performance.
Beyond saving practioners valuable time, we hope our work sparks interest in theoretical understanding of how modified ELBos can form a compelling basis for hyperparameter learning.

\section*{Acknowledgments}
Authors EH and MCH gratefully acknowledge support in part from the Alzheimer’s Drug Discovery Foundation and the National Institutes of Health (grant \# 1R01NS134859-01). MCH is also supported in part by the U.S. National Science Foundation (NSF) via grant IIS \# 2338962. We are thankful for computing infrastructure support provided by Research Technology Services at Tufts University, with hardware funded in part by NSF award OAC CC* \# 2018149. We would like to thank Tim G. J. Rudner for helpful comments on an earlier draft of this paper.

\section*{Impact Statement}
This paper presents work whose goal is to advance the field of Machine Learning.
%There are many potential societal consequences of our work, none which we feel must be specifically highlighted here.
Our work has the potential to reduce the energy consumption required for machine learning research and applications.


\nocite{hinton1993keeping,graves2011practical,krishnan2019efficient,he2016deep,dosovitskiy2021vit,liu2022convnet}

\bibliography{main}
\bibliographystyle{icml2025}

\newpage

\appendix
\onecolumn

%% Config Table-of-Contents to track the sections of the appendix
\addtocontents{toc}{\protect\setcounter{tocdepth}{2}}

\counterwithin{table}{section}
\setcounter{table}{0}
\counterwithin{figure}{section}
\setcounter{figure}{0}
\counterwithin{algorithm}{section}
\setcounter{algorithm}{0}

%% Use ONE counter for all figs and tables to give unique identifiers in supplement
\makeatletter 
\let\c@table\c@figure
\let\c@lstlisting\c@figure
\let\c@algorithm\c@figure
\makeatother

%\makesupplementtitle
\begin{center}
    {\large Appendix of} \\
    {\large Learning Hyperparameters via a Data-Emphasized Variational Objective}
\end{center}

%\tableofcontents

\section{Lenth-Scale and Output-Scale}
\label{sec:length-scale_and_output-scale}
% AGW cites a blogpost https://proceedings.mlr.press/v176/wilson22a/wilson22a.pdf
RFFs are typically used with an RBF kernel where $\gamma = 1, \nu = 1$. Below we show that our RFF construction allows a Monte Carlo approximation of the RBF kernel for any $\gamma > 0, \nu > 0$. Our proof extends base case material found in the blogpost \emph{Random Fourier Features}: {\footnotesize \url{https://gregorygundersen.com/blog/2019/12/23/random-fourier-features/}}.

Recall that $x \in \mathbb{R}^H$, $A \in \mathbb{R}^{H \times R}$ and $b \in \mathbb{R}^{R}$. Fill $b$ so each entry is $b_r {\sim} \text{Unif}(0, 2\pi)$. Fill $A$ so each entry is $A_{hr} {\sim} \mathcal{N}(0,1)$.
\begin{align}
    \phi(x)^T\phi(x') &= \frac{\nu^2}{R} \sum_{r=1}^R 2\cos \left( \frac{1}{\gamma} A_{:,r}^T x + b_r \right) \cos \left( \frac{1}{\gamma} A_{:,r}^T x' + b_r \right) \\
    &= \frac{\nu^2}{R} \sum_{r=1}^R \cos \left( \frac{1}{\gamma} A_{:,r}^T (x+x') + 2b_r \right) + \cos \left( \frac{1}{\gamma} A_{:,r}^T (x-x') \right) && \text{Sum of angles formula} \\
    &= \frac{\nu^2}{R} \sum_{r=1}^R \cos \left( \frac{1}{\gamma} A_{:,r}^T (x-x') \right) && \text{\makecell[lt]{Expectation of $\cos$ over its period is\\zero}} \\
    &= \frac{\nu^2}{R} \sum_{r=1}^R \cos \left( \frac{1}{\gamma} A_{:,r}^T (x-x') \right) + i\sin \left( \frac{1}{\gamma} A_{:,r}^T (x-x') \right) && \text{\makecell[lt]{Add imaginary part, which is zero\\in expectation, as $\mathbb{E}[\sin(a)] = 0$\\for any zero-mean Gaussian r.v. $a$}} \\
    &= \frac{\nu^2}{R} \sum_{r=1}^R \exp \left( i \frac{1}{\gamma} A_{:,r}^T (x-x') \right) && \text{Euler's formula}
\end{align}
The above sum over $R$ samples is an unbiased estimate of the expectation of a complex exponential with respect to a vector $\omega \in \mathbb{R}^H$ where each entry is drawn $\omega_h \sim \mathcal{N}(0, \frac{1}{\gamma^2})$ for all $h$.
Using the reparametrization trick, we can sample a draw $\omega$ as $\frac{1}{\gamma} A_{:,r}$ where $A_{h,r} \sim \mathcal{N}(0, 1)$ for all $h$. Then we have:
\begin{align}
    \frac{\nu^2}{R} \sum_{r=1}^R \exp \left( i \frac{1}{\gamma} A_{:,r}^T (x-x') \right) \approx \nu^2\mathbb{E}_{\mathcal{N}(\omega | 0, \frac{1}{\gamma^2} I_H)}[\exp(i\omega^T (x-x'))].
\end{align}
In the remainder, we define a vector $\Delta = x - x'$ for simplicity, where $\Delta \in \mathbb{R}^H$.
Next, we show that this expectation of a complex exponential is equivalent to the RBF kernel $k(\Delta)$ whose lengthscale $\gamma$ and outputscale $\nu$ values match those used to construct $\phi$.
\begin{align}
    & \nu^2\mathbb{E}_{p(\omega)}[\exp(i\omega^T\Delta)] && p(\omega) = \textstyle \mathcal{N}(\omega | 0, \frac{1}{\gamma^2} I_H)
    \\
    & = \nu^2 \int p\left(\omega\right) \exp\left(i\omega^T\Delta\right) d\omega && \text{Expectation as integral}
    \\
    &= \nu^2 \int \left(\frac{\gamma^2}{2\pi}\right)^{H/2} \exp\left(- \frac{\gamma^2\omega^T\omega}{2} \right) \exp\left(i\omega^T\Delta\right) d\omega && \text{Definition of $p(\omega)$} \\
    &= \nu^2 \left(\frac{\gamma^2}{2\pi}\right)^{H/2} \int \exp\left(- \frac{\gamma^2\omega^T\omega}{2} + i\omega^T\Delta\right) d\omega && \text{Product rule of exponents} \\
    &= \nu^2 \left(\frac{\gamma^2}{2\pi}\right)^{H/2} \int \exp\left(- \frac{\gamma^2\omega^T\omega}{2} + i\omega^T\Delta + \frac{\Delta^T\Delta}{2\gamma^2} - \frac{\Delta^T\Delta}{2\gamma^2}\right) d\omega && \text{Add and subtract same term} \\
    &= \nu^2 \left(\frac{\gamma^2}{2\pi}\right)^{H/2} \exp\left( - \frac{\Delta^T\Delta}{2\gamma^2} \right) \int \exp\left(- \frac{\gamma^2}{2} \left(\omega - \frac{i}{\gamma^2}\Delta\right)^T \left(\omega - \frac{i}{\gamma^2}\Delta\right) \right) d\omega && \text{Completing the square} \\
    &= \nu^2 \left(\frac{\gamma^2}{2\pi}\right)^{H/2} \exp\left( - \frac{\Delta^T\Delta}{2\gamma^2} \right) \left(\frac{2\pi}{\gamma^2}\right)^{H/2} && \text{\makecell[lt]{Translation invariance of\\Gaussian integral}} \\
    &= \nu^2 \exp\left( - \frac{\Delta^T\Delta}{2\gamma^2} \right) \\
    &= k(\Delta) && \text{By definition of RBF kernel}
\end{align}

\section{Importance Weighted Evidence}
\label{sec:importance_weighted_evidence}

There are two possible explanations for why our \emph{data-emphasized ELBo} favors $q$ that produce higher test accuracy: 1)~our \emph{data-emphasized} objective addresses looseness in the ELBo and 2)~our \emph{data-emphasized} objective addresses a misspecified prior.
To demonstrate the latter is true, we use importance weighted evidence (IWE) \citep{burda2016importance} to approximate the logarithm of evidence.
IWE is a better estimator of the logarithm of evidence than the ELBo and a should approach the logarithm of evidence as the number of samples goes to infinity.
\begin{align}
    \log p_{\eta}(y_{1:N}) = \log \mathbb{E}_{q_{r}(\theta)} \left[ \frac{p_{\eta}(y_{1:N} | \theta) p_{\eta}(\theta)}{q_{r}(\theta)} \right] \geq \mathbb{E}_{q_{r}(\theta)} \left[ \log \frac{p_{\eta}(y_{1:N} | \theta) p_{\eta}(\theta)}{q_{r}(\theta)} \right] = \JELBO
\end{align}
\begin{align}
    J_{\text{IWE}} := \log \mathbb{E}_{q_{r}(\theta)} \left[ \exp \left( \sum_{i=1}^{N} \log p_{\eta}(y_i | \theta) + \log p_{\eta}(\theta) - \log q_{r}(\theta) \right) \right]
\end{align}
However, just like the ELBo, when $D \gg N$ the IWE prefers simpler $q$, while our \emph{data-emphasized IWE} (DE IWE) favors $q$ that produce higher test accuracy.
This suggests that our \emph{data-emphasized} objective addresses a misspecified prior rather than looseness in the ELBo.
\begin{align}
    J_{\text{DE IWE}} := \log \mathbb{E}_{q_{r}(\theta)} \left[ \exp \left( \kappa \cdot \sum_{i=1}^{N} \log p_{\eta}(y_i | \theta) + \log p_{\eta}(\theta) - \log q_{r}(\theta) \right) \right]
\end{align}

\begin{figure}[htbp!]
  \centering
  \includegraphics[width=234.8775pt]{images/iwe_comparison.pdf}
  \caption{Model selection comparison between the importance weighted evidence (IWE, left) and our \emph{data-emphasized IWE} (DE IWE, right) for two different approximate posteriors $q$ defined in Case Study B: ResNet-50s with $D$ in millions trained on CIFAR-10 with $N=1000$. Each panel varies hyperparameters $\eta = \lambda = \tau$.
  \textbf{Takeaway: When $D \gg N$, the IWE prefers simpler $q$ (\cyan{cyan}), while our DE IWE favors $q$ that produce higher test accuracy (\blue{blue}).}}
  \label{fig:iwe_comparison}
\end{figure}

\section{Classification}
\label{sec:classification}

\subsection{Dataset details}
\label{sec:dataset_details}
We include experiments on CIFAR-10 \citep{krizhevsky2010cifar}, Oxford-IIIT Pet \citep{parkhi2012cats}, and Oxford Flower \citep{nilsback2008automated}.
For all experiments, we randomly select $N$ training images, stratifying by class to ensure balanced class frequencies.

We use the same preprocessing steps for all datasets.
For each distinct training set size $N$, we compute the mean and standard deviation of each channel to normalize images.
During fine-tuning we resize the images to $256 \times 256$ pixels, randomly crop images to $224 \times 224$, and randomly flip images horizontal with probability 0.5.
At test time, we resize the images to $256 \times 256$ pixels and center crop to $224 \times 224$.

\subsection{Classifier details}
\label{sec:classifier_details}

We use SGD with a Nesterov momentum parameter of 0.9 and batch size of 128 for optimization.
We train for 6,000 steps using a cosine annealing learning rate \citep{loshchilov2016sgdr}.

For \baseline, we select the initial learning rate from $\{0.1, 0.01, 0.001, 0.0001\}$. 
For linear probing and L2-zero, we select the regularization strength from $\{0.01, 0.001, 0.0001, 1\text{e-}5, 1\text{e-}6, 0.0\}$.
For L2-SP we select $\frac{\alpha}{N}$ from $\{0.01, 0.001, 0.0001, 1\text{e-}5, 1\text{e-}6, 0.0\}$ and $\frac{\beta}{N}$ from $\{0.01, 0.001, 0.0001, 1\text{e-}5, 1\text{e-}6, 0.0\}$.
For PTYL, we select $\lambda$ from 10 logarithmically spaced values between 1e0 to 1e9 and $\frac{1}{\tau N}$ from $\{0.01, 0.001, 0.0001, 1\text{e-}5, 1\text{e-}6, 0.0\}$.

While tuning hyperparameters, we hold out 1/5 of the training set for validation, ensuring balanced class frequencies between sets.

After selecting the optimal hyperparameters from the validation set NLL, we retrain the model using the selected hyperparameters on the combined set of all $N$ images (merging training and validation). 
All results report performance on the task in question's predefined test set.

For DE ELBo, we select the initial learning rate from $\{0.1, 0.01, 0.001, 0.0001\}$ and learn optimal $\lambda, \tau$ values.

\section{Learning $\lambda, \tau$}
\label{sec:learning_lambda_tau}

In our particular model in Eq.~\eqref{eq:joint_pdf_model}, the KL divergence between two Gaussians \citep{murphy2022Example} simplifies for the backbone KL term as:
\begin{align}
-\mathbb{KL}(q(w) \| p(w)) = -\frac{1}{2} \left[ \frac{\bar{\sigma}^2}{\lambda} \Tr (\Sigma_p^{-1}) + \frac{1}{\lambda} (\mu_p-\bar{w})^T\Sigma_p^{-1}(\mu_p-\bar{w}) - D + \log \left( \frac{\lambda^{D}\det(\Sigma_p)}{\bar{\sigma}^{2D}} \right) \right].
\end{align}

\textbf{Closed-form updates.} To find an optimal $\lambda$ value with respect to the $\JELBO$, notice that of the 3 additive terms in Eq.~\eqref{eq:de_elbo_objective}, only the KL term between $q(w)$ and $p(w)$ involves $\lambda$. We solve for $\lambda$ by taking the gradient of the KL term with respect to $\lambda$, setting to zero, and solving, with assurances of a local maximum of $\JELBO$ via a second derivative test (see App.~\ref{sec:second_derivative_test}). The gradient is
\begin{align}
    \nabla_\lambda -\mathbb{KL}(q(w) \| p(w)) = -\frac{1}{2} \left[ - \frac{\bar{\sigma}^2}{\lambda^2} \Tr(\Sigma_p^{-1}) - \frac{1}{\lambda^2} (\mu_p-\bar{w})^T \Sigma_p^{-1} (\mu_p-\bar{w}) + \frac{D}{\lambda} \right].
\end{align}
Setting $\nabla_\lambda -\mathbb{KL}(q(w) \| p(w)) = 0$ and solving for $\lambda$, we get 
\begin{align}
    \lambda^* = \frac{1}{D} \Big[ \bar{\sigma}^2 \Tr(\Sigma_p^{-1}) + (\mu_p-\bar{w})^T \Sigma_p^{-1} (\mu_p-\bar{w}) \Big].    
\end{align}

In our particular model in Eq.~(\ref{eq:joint_pdf_model}), the KL divergence between two Gaussians \citep{murphy2022Example} simplifies for the classifier head KL term as:
\begin{align}
    -\mathbb{KL}(q(V) \| p(V)) = -\frac{1}{2} \left[ \frac{\bar{\sigma}^2}{\tau} D + \frac{1}{\tau} || \text{vec}(\bar{V}) ||_2^2 - D + \log \left( \frac{\tau^{D}}{\bar{\sigma}^{2D}} \right)\right].
\end{align}
\textbf{Closed-form updates}
To find an optimal $\tau$ value with respect to the $\JELBO$, notice that of the 3 additive terms in Eq.~\eqref{eq:de_elbo_objective}, only the KL term between $q(V)$ and $p(V)$ involves $\tau$. We solve for $\tau$ by taking the gradient of the KL term with respect to $\tau$, setting to zero, and solving, with assurances of a local maximum of $\JELBO$ via a second derivative test (see App.~\ref{sec:second_derivative_test}). The gradient is
\begin{align}
    \nabla_\tau -\mathbb{KL}(q(V) \| p(V)) = -\frac{1}{2} \left[ - \frac{\bar{\sigma}^2}{\tau^2} D - \frac{1}{\tau^2} || \text{vec}(\bar{V}) ||_2^2 + \frac{D}{\tau} \right].
\end{align}
Setting $\nabla_\tau -\mathbb{KL}(q(V) \| p(V)) = 0$ and solving for $\tau$, we get 
\begin{align}
    \tau^* = \bar{\sigma}^2 + \frac{1}{D} || \text{vec}(\bar{V}) ||_2^2.
\end{align}

\section{Second Derivative Test}
\label{sec:second_derivative_test}

The second derivative is
\begin{align}
    \nabla^2_\lambda -\mathbb{KL}(q(w) \| p(w)) &= -\frac{1}{2} \left[ \frac{2\bar{\sigma}^2}{\lambda^3} \Tr(\Sigma_p^{-1}) + \frac{2}{\lambda^3} (\mu_p - \bar{w})^T \Sigma_p^{-1} (\mu_p - \bar{w}) - \frac{D}{\lambda^2} \right] \\
    &= -\frac{1}{2} \left[ \frac{2D}{\lambda^3} \frac{1}{D} \left( \bar{\sigma}^2 \Tr(\Sigma_p^{-1}) + (\mu_p - \bar{w})^T \Sigma_p^{-1} (\mu_p - \bar{w}) \right) - \frac{D}{\lambda^2} \right] \\
    &= -\frac{1}{2} \left[ \frac{2D}{\lambda^3} \lambda^* - \frac{D}{\lambda^2} \right].
\end{align}
Plugging in $\lambda^*$ and simplifying, we get
\begin{align}
    \nabla^2_\lambda -\mathbb{KL}(q(w) \| p(w | \lambda^*)) &= -\frac{D}{2}  \frac{1}{\lambda^{*2}}
\end{align}
This expression is always negative, indicating that $\lambda^*$ is a local maximum of $\JELBO$.

The second derivative is
\begin{align}
    \nabla_\tau^2 -\mathbb{KL}(q(V) \| p(V)) &= -\frac{1}{2} \left[ \frac{2\bar{\sigma}^2}{\tau^3} D + \frac{2}{\tau^3} || \text{vec}(\bar{V}) ||_2^2 - \frac{D}{\tau^2} \right] \\
    &= -\frac{1}{2} \left[  \frac{2D}{\tau^3}  \left( \bar{\sigma}^2 + \frac{1}{D} || \text{vec}(\bar{V}) ||_2^2 \right) - \frac{D}{\tau^2} \right] \\ 
    &= -\frac{1}{2} \left[  \frac{2D}{\tau^3} \tau^*  - \frac{D}{\tau^2} \right].
\end{align}
Plugging in $\tau^*$ and simplifying, we get
\begin{align}
    \nabla^2_\tau -\mathbb{KL}(q(V) \| p(V | \tau^*)) &= -\frac{D}{2}  \frac{1}{\tau^{*2}}.
\end{align}
This expression is always negative, indicating that $\tau^*$ is a local maximum of $\JELBO$.

\section{Low-Rank $\Sigma_p$}
\label{sec:low-rank}

The PTYL method \citep{shwartz2022pre} uses Stochastic Weight Averaging-Gaussian (SWAG) \citep{maddox2019simple} to approximate the posterior distribution $p(w|\mathcal{D}_S)$ of the source data $\mathcal{D}_S$ with a Gaussian distribution $\mathcal{N}(\mu, \Sigma)$ where $\mu$ is the learned mean and $\Sigma = \frac{1}{2}(\Sigma_{\text{diag}} + \Sigma_{\text{LR}})$ is a representation of a covariance matrix with both diagonal and \emph{low-rank} components.
The LR covariance has the form $\Sigma_{\textrm{LR}} = \frac{1}{K-1} Q Q^T$, where $Q \in \mathbb{R}^{D \times K}$.

We use the Woodbury matrix identity \cite{woodbury1950inverting}, trace properties, and the matrix determinant lemma to compute the trace of the inverse, squared Mahalanobis distance, and log determinant of the low-rank covariance matrix for the KL term.

The trace and log determinant of the low-rank covariance matrix can be calculated once and used during training.
Like in the PTYL method, the squared Mahalanobis distance needs to be re-evaluated every iteration of gradient descent.

\subsection{Trace of the inverse}
We compute the trace of the inverse of the low-rank covariance matrix using the Woodbury matrix identity and trace properties.
\begin{align*}
\Tr (\Sigma_p^{-1}) &= \Tr( (A + UCV )^{-1} ) \\
&= \Tr (A^{-1} - A^{-1}U(C^{-1} + VA^{-1}U)^{-1}VA^{-1}) && \text{Woodbury matrix identity} \\
&= \Tr (A^{-1}) - \Tr(A^{-1}U(C^{-1} + VA^{-1}U)^{-1}VA^{-1}) && \Tr(A+B) = \Tr(A) + \Tr(B) \\
&= \Tr (A^{-1}) - \Tr((C^{-1} + VA^{-1}U)^{-1}VA^{-1}A^{-1}U) && \Tr(AB) = \Tr(BA)
\end{align*}
where $A=\frac{1}{2}\Sigma_{\text{diag}}$, $C = I_K$, $U = \frac{1}{\sqrt{2K-2}}Q$, and $V=\frac{1}{\sqrt{2K-2}}Q^T$.
The last trace property, lets us compute the trace of the inverse of the low-rank covariance matrix without having to store a $D \times D$ covariance matrix.

\subsection{Squared Mahalanobis distance}
We compute the squared Mahalanobis distance $(\mu_p-\bar{w})^T\Sigma_p^{-1}(\mu_p-\bar{w})$ by distributing the mean difference vector into the Woodbury matrix identity.
\begin{align*}
\Sigma_p^{-1} &= (A + UCV )^{-1} \\
&= (A^{-1} - A^{-1}U(C^{-1} + VA^{-1}U)^{-1}VA^{-1}) && \text{Woodbury matrix identity}
\end{align*}

\subsection{Log determinant}
We compute the log determinant of the low-rank covariance matrix using the matrix determinant lemma.
\begin{align*}
\log \det(\Sigma_p) &= \log \det(A + UV) \\
&= \log (\det(I_K + VA^{-1}U) \det(A))  && \text{Matrix determinant lemma}
\end{align*}

\section{Computational Time Comparison}
\label{sec:computational_time_comparison}

\input{fig_ViT_B_16_computational_time_comparison}
\begin{figure*}[htbp!]
  \centering
  \includegraphics[width=\linewidth]{images/ResNet_50_computational_time_comparison.pdf}
  \caption{Test-set accuracy on CIFAR-10, Pet-37, and Flower-102 over total runtime for L2-SP with \emph{\baselineLong} and our \emph{data-emphasized ELBo} (DE ELBo) using ResNet-50. We run each method on 3 separate training sets of size $N$ (3 different marker styles).
  \textbf{Takeaway: Our DE ELBo achieves as good or better performance at small dataset sizes and similar performance at large dataset sizes with far less compute time.} To make the blue curves, we did the full grid search once (markers). Then, at each given shorter compute time, we subsampled a fraction of all hyperparameter configurations with that runtime and chose the best via validation NLL. Averaging this over 500 subsamples at each runtime created each blue line.
  }%endcaption
  \label{fig:ResNet_50_computational_time_comparison}
\end{figure*}
\section{Results}
\label{sec:results}

\input{tab_acc}
\setlength{\tabcolsep}{2pt}
\begin{table*}[htbp!]
  \caption{NLL on CIFAR-10, Pet-37, and Flower-102 test sets for different probabilistic models, methods, and backbones. We report mean (min-max) over 3 separately-sampled training sets. For each separately-sampled training set and training set size, the \emph{\baselineLong} baseline requires 24 different SGD runs for L2-zero, 144 for L2-SP, and 240 for PTYL. Our \emph{data-emphasized ELBo} (DE ELBo) requires 4 different SGD runs (one for each initial learning rate) and learns optimal $\lambda,\tau$ values. See App.~\ref{sec:classifier_details} for hyperparameter search space details.
  }%end caption
  \label{tab:nll}
  \centering
  \scriptsize
  \begin{tabular}{llcccccccc}
    \hline
    & & \multicolumn{4}{c}{CIFAR-10} & \multicolumn{2}{c}{Pet-37} & \multicolumn{2}{c}{Flower-102} \\
    \bfseries Model & \bfseries Method & $N =$ {\bfseries 100 (10/cl.)} & \bfseries 1000 (100/cl.) & \bfseries 10000 (1k/cl.) & \bfseries 50000 (5k/cl.) & \bfseries 370 (10/cl.) & \bfseries 3441(93/cl.) & \bfseries 510 (5/cl.) & \bfseries 1020 (10/cl.) \\
    \hline
    \multicolumn{10}{c}{ResNet-50} \\
    \rowcolor{bright-gray} Linear probing & \baseline & 1.22 {\tiny(1.20-1.25)} & 0.75 {\tiny(0.75-0.76)} & 0.55 {\tiny(0.54-0.55)} & 0.48 {\tiny(0.48-0.48)} & 0.46 {\tiny(0.42-0.52)} & 0.30 {\tiny(0.30-0.30)} & 0.83 {\tiny(0.81-0.86)} & 0.53 {\tiny(0.53-0.53)} \\
    & DE ELBo & 1.89 {\tiny(1.87-1.90)} & 1.06 {\tiny(1.03-1.09)} & 0.55 {\tiny(0.54-0.55)} & 0.48 {\tiny(0.48-0.48)} & 0.51 {\tiny(0.46-0.55)} & 0.39 {\tiny(0.39-0.39)} & 0.83 {\tiny(0.81-0.86)} & 0.53 {\tiny(0.53-0.53)} \\
    \rowcolor{bright-gray} L2-zero & \baseline & 0.97 {\tiny(0.94-1.03)} & 0.41 {\tiny(0.39-0.44)} & 0.19 {\tiny(0.19-0.20)} & 0.10 {\tiny(0.10-0.10)} & 0.42 {\tiny(0.37-0.47)} & 0.24 {\tiny(0.24-0.25)} & 0.65 {\tiny(0.63-0.68)} & 0.35 {\tiny(0.32-0.38)} \\
    & DE ELBo & 2.71 {\tiny(2.51-2.85)} & 0.53 {\tiny(0.50-0.55)} & 0.28 {\tiny(0.27-0.29)} & 0.28 {\tiny(0.27-0.29)} & 0.81 {\tiny(0.78-0.84)} & 0.27 {\tiny(0.26-0.28)} & 0.86 {\tiny(0.79-0.91)} & 0.45 {\tiny(0.42-0.46)} \\
    \rowcolor{bright-gray} L2-SP & \baseline & 0.94 {\tiny(0.85-1.02)} & 0.40 {\tiny(0.39-0.41)} & 0.15 {\tiny(0.15-0.16)} & 0.09 {\tiny(0.09-0.09)} & 0.41 {\tiny(0.36-0.45)} & 0.27 {\tiny(0.24-0.30)} & 0.65 {\tiny(0.63-0.68)} & 0.33 {\tiny(0.32-0.35)} \\
    & DE ELBo & 1.01 {\tiny(0.92-1.12)} & 0.44 {\tiny(0.42-0.46}) & 0.24 {\tiny(0.23-0.24)} & 0.12 {\tiny(0.12-0.12)} & 0.51 {\tiny(0.49-0.53)} & 0.24 {\tiny(0.24-0.24)} & 0.71 {\tiny(0.67-0.74)} & 0.38 {\tiny(0.38-0.39)} \\
    \rowcolor{bright-gray} PTYL (SSL) & \baseline & 1.34 {\tiny(1.20-1.41)} & 0.50 {\tiny(0.49-0.52)} & 0.23 {\tiny(0.22-0.24)} & 0.11 {\tiny(0.11-0.11)} & 1.64 {\tiny(1.55-1.68)} & 0.54 {\tiny(0.50-0.56)} & 0.84 {\tiny(0.81-0.86)} & 0.49 {\tiny(0.46-0.51)} \\
    & DE ELBo & 1.36 {\tiny(1.33-1.43)} & 0.76 {\tiny(0.74-0.78)} & 0.29 {\tiny(0.28-0.31)} & 0.12 {\tiny(0.12-0.12)} & 2.07 {\tiny(2.04-2.08)} & 0.68 {\tiny(0.67-0.69)} & 1.10 {\tiny(1.05-1.15)} & 0.72 {\tiny(0.70-0.74)} \\
    \rowcolor{bright-gray} PTYL & \baseline &  0.90 {\tiny(0.87-0.94)} & 0.36 {\tiny(0.34-0.37)} & 0.15 {\tiny(0.14-0.16)} & 0.10 {\tiny(0.10-0.10)} & 0.42 {\tiny(0.37-0.46)} & 0.26 {\tiny(0.21-0.31)} & 0.65 {\tiny(0.63-0.68)} & 0.33 {\tiny(0.32-0.34)} \\
    & DE ELBo & 1.00 {\tiny(0.92-1.08)} & 0.46 {\tiny(0.46-0.47)} & 0.23 {\tiny(0.22-0.24)} & 0.12 {\tiny(0.12-0.12)} & 0.51 {\tiny(0.49-0.53)} & 0.24 {\tiny(0.24-0.24)} & 0.71 {\tiny(0.67-0.74)} & 0.38 {\tiny(0.38-0.39)}  \\
    \hline
    \multicolumn{10}{c}{ViT-B/16} \\
    \rowcolor{bright-gray} L2-SP & \baseline & 0.46 {\tiny(0.41-0.51)} & 0.25 {\tiny(0.20-0.28)} & 0.10 {\tiny(0.10-0.10)} & 0.06 {\tiny(0.06-0.06)} & 0.42 {\tiny(0.38-0.47)} & 0.25 {\tiny(0.24-0.29)} & 0.69 {\tiny(0.64-0.72)} & 0.42 {\tiny(0.34-0.54)} \\
    & DE ELBo & 0.36 {\tiny(0.32-0.41)} & 0.24 {\tiny(0.23-0.26)} & 0.12 {\tiny(0.12-0.13)} & 0.07 {\tiny(0.07-0.07)} & 0.40 {\tiny(0.38-0.42)} & 0.30 {\tiny(0.30-0.31)} & 0.90 {\tiny(0.73-1.03)} & 0.51 {\tiny(0.48-0.55)} \\
    \hline
    \multicolumn{10}{c}{ConNeXt-Tiny} \\
    \rowcolor{bright-gray} L2-SP & \baseline & 0.46 {\tiny(0.44-0.50)} & 0.22 {\tiny(0.19-0.24)} & 0.11 {\tiny(0.10-0.12)} & 0.07 {\tiny(0.07-0.07)} & 0.38 {\tiny(0.30-0.46)} & 0.18 {\tiny(0.18-0.18)} & 0.54 {\tiny(0.51-0.56)} & 0.28 {\tiny(0.23-0.38)} \\    
    & DE ELBo & 0.54 {\tiny(0.48-0.59)} & 0.23 {\tiny(0.22-0.24)} & 0.12 {\tiny(0.11-0.12)} & 0.07 {\tiny(0.07-0.07)} & 0.32 {\tiny(0.27-0.36)} & 0.22 {\tiny(0.22-0.22)} & 0.58 {\tiny(0.53-0.64)} & 0.30 {\tiny(0.29-0.31)} \\    
    \hline
  \end{tabular}
\end{table*}
\setlength{\tabcolsep}{6pt}


\subsection{Changes to Training Performance from Running/Computed Mean and Variance in Batch Normalization Layers}
During training, batch normalization layers keep running estimates of its computed mean and variance, which are then used for normalization during evaluation.
We find that when using batch normalization, the ELBo and accuracy on the train set can change between \texttt{train()} and \texttt{eval()} mode (see Fig.~\ref{fig:train_eval_comparison}).
We recommend using the computed mean and variance to evaluate the ELBo on the training set for model selection since this mode is used for normalization during evaluation.

\begin{figure}[htbp!]
  \centering
  \includegraphics[width=234.8775pt]{images/train_eval_comparison.pdf}
  \caption{Negative \emph{data-emphasized ELBo} (DE ELBo) and accuracy on train set for two ResNet-50s trained on CIFAR-10 $N=1000$. We use \emph{just one sample} every training step to approximate the expected log likelihood. \textbf{Takeaway: We find that when using batch normalization, the ELBo and accuracy on the train set can change between} \texttt{train()} \textbf{and} \texttt{eval()} \textbf{mode.}}
  \label{fig:train_eval_comparison}
\end{figure}


\end{document}