\section{Second Derivative Test}
\label{sec:second_derivative_test}

The second derivative is
\begin{align}
    \nabla^2_\lambda -\mathbb{KL}(q(w) \| p(w)) &= -\frac{1}{2} \left[ \frac{2\bar{\sigma}^2}{\lambda^3} \Tr(\Sigma_p^{-1}) + \frac{2}{\lambda^3} (\mu_p - \bar{w})^T \Sigma_p^{-1} (\mu_p - \bar{w}) - \frac{D}{\lambda^2} \right] \\
    &= -\frac{1}{2} \left[ \frac{2D}{\lambda^3} \frac{1}{D} \left( \bar{\sigma}^2 \Tr(\Sigma_p^{-1}) + (\mu_p - \bar{w})^T \Sigma_p^{-1} (\mu_p - \bar{w}) \right) - \frac{D}{\lambda^2} \right] \\
    &= -\frac{1}{2} \left[ \frac{2D}{\lambda^3} \lambda^* - \frac{D}{\lambda^2} \right].
\end{align}
Plugging in $\lambda^*$ and simplifying, we get
\begin{align}
    \nabla^2_\lambda -\mathbb{KL}(q(w) \| p(w | \lambda^*)) &= -\frac{D}{2}  \frac{1}{\lambda^{*2}}
\end{align}
This expression is always negative, indicating that $\lambda^*$ is a local maximum of $\JELBO$.

The second derivative is
\begin{align}
    \nabla_\tau^2 -\mathbb{KL}(q(V) \| p(V)) &= -\frac{1}{2} \left[ \frac{2\bar{\sigma}^2}{\tau^3} D + \frac{2}{\tau^3} || \text{vec}(\bar{V}) ||_2^2 - \frac{D}{\tau^2} \right] \\
    &= -\frac{1}{2} \left[  \frac{2D}{\tau^3}  \left( \bar{\sigma}^2 + \frac{1}{D} || \text{vec}(\bar{V}) ||_2^2 \right) - \frac{D}{\tau^2} \right] \\ 
    &= -\frac{1}{2} \left[  \frac{2D}{\tau^3} \tau^*  - \frac{D}{\tau^2} \right].
\end{align}
Plugging in $\tau^*$ and simplifying, we get
\begin{align}
    \nabla^2_\tau -\mathbb{KL}(q(V) \| p(V | \tau^*)) &= -\frac{D}{2}  \frac{1}{\tau^{*2}}.
\end{align}
This expression is always negative, indicating that $\tau^*$ is a local maximum of $\JELBO$.
