\section{Lenth-Scale and Output-Scale}
\label{sec:length-scale_and_output-scale}
% AGW cites a blogpost https://proceedings.mlr.press/v176/wilson22a/wilson22a.pdf
RFFs are typically used with an RBF kernel where $\gamma = 1, \nu = 1$. Below we show that our RFF construction allows a Monte Carlo approximation of the RBF kernel for any $\gamma > 0, \nu > 0$. Our proof extends base case material found in the blogpost \emph{Random Fourier Features}: {\footnotesize \url{https://gregorygundersen.com/blog/2019/12/23/random-fourier-features/}}.

Recall that $x \in \mathbb{R}^H$, $A \in \mathbb{R}^{H \times R}$ and $b \in \mathbb{R}^{R}$. Fill $b$ so each entry is $b_r {\sim} \text{Unif}(0, 2\pi)$. Fill $A$ so each entry is $A_{hr} {\sim} \mathcal{N}(0,1)$.
\begin{align}
    \phi(x)^T\phi(x') &= \frac{\nu^2}{R} \sum_{r=1}^R 2\cos \left( \frac{1}{\gamma} A_{:,r}^T x + b_r \right) \cos \left( \frac{1}{\gamma} A_{:,r}^T x' + b_r \right) \\
    &= \frac{\nu^2}{R} \sum_{r=1}^R \cos \left( \frac{1}{\gamma} A_{:,r}^T (x+x') + 2b_r \right) + \cos \left( \frac{1}{\gamma} A_{:,r}^T (x-x') \right) && \text{Sum of angles formula} \\
    &= \frac{\nu^2}{R} \sum_{r=1}^R \cos \left( \frac{1}{\gamma} A_{:,r}^T (x-x') \right) && \text{\makecell[lt]{Expectation of $\cos$ over its period is\\zero}} \\
    &= \frac{\nu^2}{R} \sum_{r=1}^R \cos \left( \frac{1}{\gamma} A_{:,r}^T (x-x') \right) + i\sin \left( \frac{1}{\gamma} A_{:,r}^T (x-x') \right) && \text{\makecell[lt]{Add imaginary part, which is zero\\in expectation, as $\mathbb{E}[\sin(a)] = 0$\\for any zero-mean Gaussian r.v. $a$}} \\
    &= \frac{\nu^2}{R} \sum_{r=1}^R \exp \left( i \frac{1}{\gamma} A_{:,r}^T (x-x') \right) && \text{Euler's formula}
\end{align}
The above sum over $R$ samples is an unbiased estimate of the expectation of a complex exponential with respect to a vector $\omega \in \mathbb{R}^H$ where each entry is drawn $\omega_h \sim \mathcal{N}(0, \frac{1}{\gamma^2})$ for all $h$.
Using the reparametrization trick, we can sample a draw $\omega$ as $\frac{1}{\gamma} A_{:,r}$ where $A_{h,r} \sim \mathcal{N}(0, 1)$ for all $h$. Then we have:
\begin{align}
    \frac{\nu^2}{R} \sum_{r=1}^R \exp \left( i \frac{1}{\gamma} A_{:,r}^T (x-x') \right) \approx \nu^2\mathbb{E}_{\mathcal{N}(\omega | 0, \frac{1}{\gamma^2} I_H)}[\exp(i\omega^T (x-x'))].
\end{align}
In the remainder, we define a vector $\Delta = x - x'$ for simplicity, where $\Delta \in \mathbb{R}^H$.
Next, we show that this expectation of a complex exponential is equivalent to the RBF kernel $k(\Delta)$ whose lengthscale $\gamma$ and outputscale $\nu$ values match those used to construct $\phi$.
\begin{align}
    & \nu^2\mathbb{E}_{p(\omega)}[\exp(i\omega^T\Delta)] && p(\omega) = \textstyle \mathcal{N}(\omega | 0, \frac{1}{\gamma^2} I_H)
    \\
    & = \nu^2 \int p\left(\omega\right) \exp\left(i\omega^T\Delta\right) d\omega && \text{Expectation as integral}
    \\
    &= \nu^2 \int \left(\frac{\gamma^2}{2\pi}\right)^{H/2} \exp\left(- \frac{\gamma^2\omega^T\omega}{2} \right) \exp\left(i\omega^T\Delta\right) d\omega && \text{Definition of $p(\omega)$} \\
    &= \nu^2 \left(\frac{\gamma^2}{2\pi}\right)^{H/2} \int \exp\left(- \frac{\gamma^2\omega^T\omega}{2} + i\omega^T\Delta\right) d\omega && \text{Product rule of exponents} \\
    &= \nu^2 \left(\frac{\gamma^2}{2\pi}\right)^{H/2} \int \exp\left(- \frac{\gamma^2\omega^T\omega}{2} + i\omega^T\Delta + \frac{\Delta^T\Delta}{2\gamma^2} - \frac{\Delta^T\Delta}{2\gamma^2}\right) d\omega && \text{Add and subtract same term} \\
    &= \nu^2 \left(\frac{\gamma^2}{2\pi}\right)^{H/2} \exp\left( - \frac{\Delta^T\Delta}{2\gamma^2} \right) \int \exp\left(- \frac{\gamma^2}{2} \left(\omega - \frac{i}{\gamma^2}\Delta\right)^T \left(\omega - \frac{i}{\gamma^2}\Delta\right) \right) d\omega && \text{Completing the square} \\
    &= \nu^2 \left(\frac{\gamma^2}{2\pi}\right)^{H/2} \exp\left( - \frac{\Delta^T\Delta}{2\gamma^2} \right) \left(\frac{2\pi}{\gamma^2}\right)^{H/2} && \text{\makecell[lt]{Translation invariance of\\Gaussian integral}} \\
    &= \nu^2 \exp\left( - \frac{\Delta^T\Delta}{2\gamma^2} \right) \\
    &= k(\Delta) && \text{By definition of RBF kernel}
\end{align}
