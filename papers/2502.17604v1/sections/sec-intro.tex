\section{Introduction}
% In the contemporary landscape of technological advancements, substantial progress has been observed in the realm of AI and large language models (LLMs)~\cite{devlin2018bert}, notably the generative pre-trained transformer (GPT) models~\cite{brown2020language} and Google PaLM2~\cite{anil2023palm}. These models have reached a stage where they are increasingly being integrated into practical applications~\cite{https://doi.org/10.48550/arxiv.2302.03202,brown2020language}. A particularly intriguing application of AI and LLMs is in the generation of source code including smart contracts from natural language, a development that holds the promise of transforming the programming process. Examples include ChatGPT and related systems like Github CoPilot~\cite{10.1145/3511861.3511863}, which are capable of translating natural language instructions from programmers into source code in various programming languages. Another area of applications is applying LLMs for modeling and predicting time series data. A major advantage of LLM based time series data prediction over all the traditional approaches is that it can support multi-modal prediction, integrating time series data with other unstructured data types. This new capability may open a new frontier in financial modeling. 


In today's technological era, significant strides have been made in the field of artificial intelligence and large language models (LLMs), such as the advancements seen with GPT models~\cite{brown2020language} and Google's PaLM2~\cite{anil2023palm}. These models are now progressively utilized in real-world applications~\cite{https://doi.org/10.48550/arxiv.2302.03202,brown2020language}. An exciting application of AI and LLMs is their use in generating source code, including smart contracts, directly from natural language descriptions. This application, exemplified by tools like ChatGPT and Github CoPilot~\cite{10.1145/3511861.3511863}, translates programmers' instructions into code across various languages, potentially revolutionizing the programming landscape. Furthermore, LLMs are being applied in forecasting and modeling time series data, outperforming traditional methods by enabling multi-modal predictions that incorporate both time series and other forms of unstructured data. This innovation could greatly enhance financial modeling techniques.


Amidst this backdrop, the increasing use of decentralized ledgers and smart contracts, particularly in the financial sector, is noteworthy. For example, Uniswap's smart contracts managed transactions worth approximately \$7.17 billion daily in 2021~\footnote{https://decrypt.co/63280/uniswap-trading-volume-exploded-7-
billion-heres-why}. Given the growing importance of smart contracts in various contexts, such as Confidential Computing \cite{10.1145/3505253.3505259,10174906}, Decentralized Serverless Functions~\cite{10174945}, and Event-based Transactions \cite{10.1145/3464298.3493401,9461133}, it becomes crucial to investigate whether AI and LLMs can be used effectively for finance modeling and predictions in a distributed and safe way while augmenting a smart contract workflow. 

In this research, we explore the feasibility of executing an AI agent or machine learning model through a smart contract on a blockchain. Our focus is on developing a system that integrates with the current smart contract workflows seamlessly, without necessitating disruptive changes to the existing infrastructure. We aim to adhere to established standards while utilizing AI-driven financial models to produce numerical, textual, and multi-modal outputs. The core of our investigation is to determine whether our proposed system can function within well-established frameworks and offer an innovative approach to interacting with AI models, including LLMs, directly on-chain.

In our study, we introduce a framework designed to enable decentralized on-device AI model agnostic inferences for smart contracts. These smart contracts act as triggers for invoking the AI inference engine via a WebAssembly runtime~\cite{haas2017bringing}. Our research primarily seeks to address the following questions:


\begin{tcolorbox}[colback=red!5!white,colframe=red!75!black]
  \textbf{RQ1: Can smart contracts utilize AI and LLM inferences with reasonable performance?} We try to see if using existing smart contract platforms, we can build a workflow that is AI model and LLM agnostic. 

\textbf{RQ2: Can we get inference locally in a secure way?} We try to run models locally to see if the smart contracts can execute and invoke the models directly on the device and get the inferences back.

\textbf{RQ3: Can we utilize AI accelerators such as GPU for fast inference while executing a smart contract?} Most of the AI and LLM models are compute-intensive and though we can run some of them in CPU, running them using GPU is desirable for reasonable inference throughput. We evaluate if our framework is capable of doing this.

\textbf{RQ4: What are the security implications?} We explore the security implications of such a framework its benefits over traditional methods and pitfalls. Here we show how using our framework inherently mitigates some covert channel security attacks, which are possible for native inferences in the same machine.
\end{tcolorbox}


To address the above, we propose \underline{W}ASM-Powered \underline{I}nterchain \underline{C}ommunication for \underline{A}I Enabled \underline{S}mart Contracts or \sln{} in short. Which uses \cite{kwon2019cosmos} for the blockchain environment to execute a smart contract and uses our reference implementation of AI inferences with open-source models. 

In summary, the contributions of this work are listed below:
\begin{tcolorbox}[colback=green!5!white,colframe=green!75!black]
\begin{itemize}
    \item We propose \sln{} which provides a way to get AI and LLM invocation and response based on a smart contract execution, a new capability for smart contracts that may enable and power emerging application scenarios in the blockchain and Web3 domain. 
    \item We demonstrate the viability of AI and LLM inferences based on smart contracts on local models.
    \item We explore the potential security benefits and considerations for the proposed system.
    \item We demonstrate that our framework is an AI model and underlying inference engine agnostics and can be adapted for other future systems.
\end{itemize}
\end{tcolorbox}


To the best of our knowledge, we are the first to propose and evaluate a system to (1) run AI and machine learning inference based on smart contract execution on a device and (2) show that it is possible to build such a system in heterogeneous systems with varying CPU/GPU and scalability.