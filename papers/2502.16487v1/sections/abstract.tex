









\begin{abstract}
    Automating scientific research is considered the final frontier of science. Recently, several papers claim autonomous research agents can generate novel research ideas. Amidst the prevailing optimism, we document a critical concern: a considerable fraction of such research documents are smartly plagiarized. Unlike past efforts where experts evaluate the novelty and feasibility of research ideas, we request $13$ experts to operate under a different situational logic: to identify similarities between LLM-generated research documents and existing work. Concerningly, the experts identify $24\%$ of the $50$ evaluated research documents to be either paraphrased (with one-to-one methodological mapping), or significantly borrowed from existing work. These reported instances are cross-verified by authors of the source papers. Problematically, these LLM-generated research documents do not acknowledge original sources, and bypass inbuilt plagiarism detectors. Lastly, through controlled experiments we show that automated plagiarism detectors are inadequate at catching deliberately plagiarized ideas from an LLM. We recommend a careful assessment of LLM-generated research, and discuss the implications of our findings on research and academic publishing.\footnote{Our code, along with  expert-provided scores and explanations for each proposal, is available at: \url{https://anonymous.4open.science/r/AI-Papers-Plagiarism-ECCA}}
\end{abstract}

