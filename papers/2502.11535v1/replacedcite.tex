\section{Related Works}
\subsection{Effects of Center of Mass on Human Grasping Behavior}
        In human grasping, aligning the hand's CoM with the object's CoM enhances grasp stability by increasing contact area, stabilizing force distribution, and reducing rotational moments. Studies have shown that humans predict an object's CoM and adjust their contact points accordingly.
        For instance, Lukos et al. compared contact point selection in grasping when the CoM position was known versus unknown, showing that contact points are adapted based on a predictable CoM____.
        Furthermore, Desanghere \& Marotta investigated how CoM estimation through visual information influences the selection of grasp positions, reporting that fixation locations are sensitive to CoM and affect grasp locations____.
        
        These findings highlight the crucial role of CoM alignment in human grasping behavior. Motivated by this principle, our study explores how to systematically incorporate CoM alignment into computational grasp planning, particularly within our surface fitting-based approach.


    \subsection{Computational Modeling of CoM on Grasp Pose Optimization}
        While CoM alignment has been shown to contribute to grasp stability in human grasping behavior, how it is incorporated into computational models remains an open question.
        Biomechanical studies have demonstrated that CoM alignment is achieved as a result of optimizing joint positions and contact points____.
        Specifically, as finger joint configurations adapt to the object surface, stable contact is ensured, and CoM alignment is implicitly achieved in the process.
        Moreover, computational biology has proposed grasp pose prediction models that explicitly consider CoM alignment, such as Klein et al. ____, who demonstrated that variations in an object's CoM influence grasp location choices. These findings suggest that grasp poses predicted through CoM-aware optimization can explain many aspects of human grasping behavior and that CoM alignment is likely an essential factor in grasp planning. Building upon these insights, this study integrates CoM alignment into a surface fitting-based grasp planning algorithm, enabling it to account not only for geometric compatibility but also for contact stability.



    \subsection{Grasp Planning via Iterative Surface Fitting}
        Optimal grasping has been shown to result in a posture where the hand and the object's surface align____.
        This alignment ensures that the robot hand establishes stable contact with the object by properly configuring the contact points.
        From this perspective, grasp planning can be interpreted as a surface fitting problem, where recent studies have applied point cloud processing techniques from Computer Vision to optimize the geometric compatibility between the object and the robot hand____. 
        Unlike traditional grasp pose prediction models studied in biomechanics, this approach does not require the object's shape to be mathematically well-defined.
        As a result, it enables a more flexible grasp planning framework that is not constrained by predefined geometric models and can be applied to a wide variety of object shapes____.
        However, existing surface fitting algorithms evaluate geometric compatibility alone without considering contact point configuration, which can result in unstable grasps where contact fails to occur or the distribution of contact points is imbalanced.
        
        In this study, we build upon flexible surface fitting algorithms and newly integrate CoM alignment to propose a grasp planning algorithm that ensures the proper establishment of contact points.
        This enables a grasp planning framework that considers both geometric compatibility and contact stability.





%%%%%%%%%%%%%%%%%%%%%%%%%%%%%%%%%%