%%
%% This is file `sample-manuscript.tex',
%% generated with the docstrip utility.
%%
%% The original source files were:
%%
%% samples.dtx  (with options: `manuscript')
%% 
%% IMPORTANT NOTICE:
%% 
%% For the copyright see the source file.
%% 
%% Any modified versions of this file must be renamed
%% with new filenames distinct from sample-manuscript.tex.
%% 
%% For distribution of the original source see the terms
%% for copying and modification in the file samples.dtx.
%% 
%% This generated file may be distributed as long as the
%% original source files, as listed above, are part of the
%% same distribution. (The sources need not necessarily be
%% in the same archive or directory.)
%%
%% The first command in your LaTeX source must be the \documentclass command.
%%%% Small single column format, used for CIE, CSUR, DTRAP, JACM, JDIQ, JEA, JERIC, JETC, PACMCGIT, TAAS, TACCESS, TACO, TALG, TALLIP (formerly TALIP), TCPS, TDSCI, TEAC, TECS, TELO, THRI, TIIS, TIOT, TISSEC, TIST, TKDD, TMIS, TOCE, TOCHI, TOCL, TOCS, TOCT, TODAES, TODS, TOIS, TOIT, TOMACS, TOMM (formerly TOMCCAP), TOMPECS, TOMS, TOPC, TOPLAS, TOPS, TOS, TOSEM, TOSN, TQC, TRETS, TSAS, TSC, TSLP, TWEB.
%\documentclass[acmsmall]{acmart}

%%%% Large single column format, used for IMWUT, JOCCH, PACMPL, POMACS, TAP, PACMHCI
% \documentclass[acmlarge,screen]{acmart}

%%%% Large double column format, used for TOG
% \documentclass[acmtog, authorversion]{acmart}

%%%% Generic manuscript mode, required for submission
%%%% and peer review
%\documentclass[manuscript,screen,review]{acmart}

%\documentclass[manuscript]{acmart}
% I use it for manuscript
% \documentclass[manuscript,review]{acmart}

% \documentclass[manuscript,review]{acmart}
\documentclass[sigconf]{acmart}


% \documentclass[sigconf]{acmart}
% I use it for TAPS submussion
% \documentclass[sigconf]{acmart}
\usepackage{color}
% \usepackage[dvipsnames]{xcolor}
% added 2021/5/22
\usepackage{pifont}
\usepackage{subfigure} 
\usepackage{float}

\usepackage{booktabs}
\usepackage{multirow}

\usepackage[toc,page]{appendix} 
% \usepackage{chngpage}
\usepackage{tabularx}
\usepackage{url}
% \usepackage[inkscapelatex=false]{svg}
\usepackage{enumitem}
\setlist[itemize]{leftmargin=*}

\newcommand{\mxj}[1]{{\color{red} #1}}
\newcommand{\xm}[1]{{\color{blue} #1}}
% \newcommand{\ms}[1]{{\color{teal} #1}}
\newcommand{\ms}[1]{{\color{black} #1}}
\newcommand{\my}[1]{{\color{orange} [Ming: #1]}}


\newcommand{\SM}[1]{\textcolor{blue}{\textbf{*Shuai*}: #1}}
\newcommand{\JW}[1]{\textcolor{purple}{\textbf{*Junling*}: #1}}
\newcommand{\AW}[1]{\textcolor{red}{\textbf{*April*}: #1}}

%\usepackage[T1]{fontenc}
%\usepackage[skip=1ex]{caption}
%\usepackage{booktabs,tabularx}
%\usepackage{siunitx}
%\newcolumntype{C}{>{\centering\arraybackslash}X}
%\newcommand{\mcx}[2]{\multicolumn{#1}{>{\hsize=\dimexpr#1\hsize
%                                        + #1\tabcolsep + #1\tabcolsep\relax}C}{#2}}
%\newcommand{\mcone}[1]{\multicolumn{1}{C}{#1}}



\usepackage{color, colortbl}
\usepackage{listings}
\usepackage[ruled]{algorithm2e}
%\usepackage{algorithm}
%\usepackage{algorithmic}


\definecolor{color1}{HTML}{CAEEFB}
\definecolor{color2}{HTML}{D9F2D0}
\definecolor{color3}{HTML}{FBE3D6}
\definecolor{darkblue}{RGB}{31,78,121}

\definecolor{orange}{RGB}{255,69,0}
\setlength{\fboxsep}{0.5pt} % 设置背景框的内边距




%%
%% \BibTeX command to typeset BibTeX logo in the docs
\AtBeginDocument{%
  \providecommand\BibTeX{{%
    \normalfont B\kern-0.5em{\scshape i\kern-0.25em b}\kern-0.8em\TeX}}}


\copyrightyear{2025}
\acmYear{2025}
\setcopyright{acmlicensed}\acmConference[CHI '25]{CHI Conference on Human Factors in Computing Systems}{April 26-May 1, 2025}{Yokohama, Japan}
\acmBooktitle{CHI Conference on Human Factors in Computing Systems (CHI '25), April 26-May 1, 2025, Yokohama, Japan}
\acmDOI{10.1145/3706598.3713748}
\acmISBN{979-8-4007-1394-1/25/04}

\begin{document}

%%
%% The ``title" command has an optional parameter,
%% allowing the author to define a ``short title" to be used in page headers.
%\title{TransClass: Assisting Transparent Online Class via Integrating Adaptable User Interface with Student Learning Status Detection Algorithm}
%\title{ClassBridger: Bridging Student Learning Status with Teachers in Synchronous Online Class  \mxj{ClassBridger: An Adaptable System for Bridging Communication of Learning Status between Students and Teachers in Synchronous Online Classes} \xm{ClassBridger: An Adaptable System to Assist Communication of Student Learning Status to Teachers in Synchronous Online Classes}}
%TeacherLens


% \title{Beyond Fixed Recommendation: Designing Human-AI Deliberation to Improve AI-Assisted Decision-Making}
% \title{``Let's Deliberate over the Conflicts, AI!'' Designing Large Language Models Powered Human-AI Deliberation to Improve AI-Assisted Decision-Making}

% \title{"Let's Resolve Conflicts": A Human-AI Deliberation Framework for AI-Assisted Decision-Making}

\title{DBox: Scaffolding Algorithmic Programming Learning through Learner-LLM Co-Decomposition}


%Who Is Better at This Case? Calibrating Humans' Trust \xm{I feel that we should not put trust calibration as the focus -- we basically fail so. Instead, it should be promoting complementary performance or something like that...} Based on Human-AI Capability in AI-Assisted Decision-Making

%An Adaptable System to Assist Communication of Student Learning Status to Instructors in Synchronous Online Classes}

%\mxj{Modeling and Exploring the Effects of Adaptive Robot Engagement Expression when Learning from Human Demonstrations}
%Investigating the Effects of Robot Learning Engagement on Human Instructors
%%
%% The ``author" command and its associated commands are used to define
%% the authors and their affiliations.
%% Of note is the shared affiliation of the first two authors, and the
%% ``authornote" and ``authornotemark" commands
%% used to denote shared contribution to the research.





%%
%% By default, the full list of authors will be used in the page
%% headers. Often, this list is too long, and will overlap
%% other information printed in the page headers. This command allows
%% the author to define a more concise list
%% of authors' names for this purpose.
% \renewcommand{\shortauthors}{Anonymous, et al.}



% \author{Shuai Ma}
% \orcid{0000-0002-7658-292X}
% \affiliation{
%   \institution{The Hong Kong University of Science and Technology}
%   \city{Hong Kong}
%   \country{China}
% }
% \email{shuai.ma@connect.ust.hk}

\author{Shuai Ma}
\authornote{Work done during the first author's PhD studies at The Hong Kong University of Science and Technology.}
\orcid{0000-0002-7658-292X}
% \affiliation{
%   \institution{The Hong Kong University of Science and Technology}
%   \city{Hong Kong}
%   \country{China}
% }
\affiliation{
  \institution{Aalto University}
  \city{Espoo}
  \country{Finland}
}
\email{shuai.ma@aalto.fi}

\author{Junling Wang}
\affiliation{
  \department{}
  \institution{ETH Zürich}
  \city{Zürich}
  \country{Switzerland}
}
\email{junling.wang@ai.ethz.ch}


\author{Yuanhao Zhang}
\affiliation{
  \department{}
  \institution{The Hong Kong University of Science and Technology}
  \city{Hong Kong}
  \country{China}
}
\email{yzhangiy@connect.ust.hk}


\author{Xiaojuan Ma}
\affiliation{
  \department{}
  \institution{The Hong Kong University of Science and Technology}
  \city{Hong Kong}
  \country{China}
}
\email{mxj@cse.ust.hk}

\author{April Yi Wang}
\affiliation{
  \department{}
  \institution{ETH Zürich}
  \city{Zürich}
  \country{Switzerland}
}
\email{april.wang@inf.ethz.ch}


\renewcommand{\shortauthors}{Shuai Ma, et al.}





            
%%
%% The abstract is a short summary of the work to be presented in the
%% article.
\begin{abstract}

% In AI-assisted decision-making, AI often offers fixed suggestions disregarding human thoughts. In such a paradigm, humans are found to rarely trigger analytical thinking and face difficulties in deciding whether to adopt AI recommendations when disagreements occur.
% To tackle this challenge, we propose \emph{Human-AI Deliberation}, a novel framework promoting human reflection and discussion on conflicting human-AI opinions in decision-making. To enable deliberation, we design \emph{Deliberative AI} which leverages LLMs to bridge the domain-specific model and humans. To understand the effects of human-AI deliberation, we implemented our framework in a graduate admission task and conducted a between-subjects study (N=110). Results show that \emph{Deliberative AI} outperforms conventional AI assistants in improving humans' cognitive engagement and task performance. Based on a mixed-method analysis of participants' behaviors (dialogue dynamics and opinion changes), perceptions of AI, experiences, and open-ended feedback, we further derive a set of practical implications for designing better AI systems that can deliberate with humans.

% In AI-assisted decision-making, traditional AI systems often present fixed suggestions that users passively accept or reject as a whole, limiting opportunities for nuanced communication, particularly in cases of disagreement. To address this, we propose a novel Human-AI Deliberation Framework, inspired by theories of human deliberation, that supports dimension-level opinion elicitation, deliberative discussion, and decision updates between humans and AI. Central to this approach is Deliberative AI, which leverages large language models (LLMs) to facilitate flexible, conversational interactions and precise information exchange with domain-specific models. In a mixed-methods user study, we found that Deliberative AI outperforms conventional explainable AI (XAI) systems by enhancing appropriate human reliance and improving task performance. Based on in-depth analysis of participant behavior, perceptions, user experience, and open-ended feedback, we offer key design implications for future AI-assisted decision-making systems.


% In AI-assisted decision-making, traditional AI systems often present fixed suggestions that users passively accept or reject in their entirety, limiting opportunities for nuanced communication, especially in cases of disagreement. To address this limitation, we propose a \ms{\emph{Human-AI Deliberation} approach}, inspired by theories of human deliberation, which enables dimension-level opinion elicitation, deliberative discussion, and iterative decision updates between humans and AI. Central to this approach is a new AI assistant, \emph{Deliberative AI}, which leverages large language models (LLMs) to facilitate flexible, conversational interactions and enable precise information exchange with domain-specific models. In a mixed-methods user study, we found that \emph{Deliberative AI} outperforms conventional explainable AI (XAI) systems by promoting appropriate human reliance and improving task performance. Drawing from an in-depth analysis of participant perceptions, user experience, and open-ended feedback, we discuss key findings, potential concerns, and the generalizability of this framework for future AI-assisted decision-making systems.
Decomposition is a fundamental skill in algorithmic programming, requiring learners to break down complex problems into smaller, manageable parts. However, current self-study methods, such as browsing reference solutions or using LLM assistants, often provide excessive or generic assistance that misaligns with learners' decomposition strategies, hindering independent problem-solving and critical thinking. To address this, we introduce Decomposition Box (DBox), an interactive LLM-based system that scaffolds and adapts to learners' personalized construction of a step tree through a \emph{``learner-LLM co-decomposition''} approach, providing tailored support at an appropriate level. A within-subjects study (N=24) found that compared to the baseline, DBox significantly improved learning gains, cognitive engagement, and critical thinking. Learners also reported a stronger sense of achievement and found the assistance appropriate and helpful for learning. Additionally, we examined DBox's impact on cognitive load, identified usage patterns, and analyzed learners' strategies for managing system errors. We conclude with design implications for future AI-powered tools to better support algorithmic programming education.

\end{abstract}

\begin{CCSXML}
<ccs2012>
    <concept>
        <concept_id>10003120.10003121.10011748</concept_id>
        <concept_desc>Human-centered computing~Empirical studies in HCI</concept_desc>
        <concept_significance>500</concept_significance>
    </concept>
    % <concept>
    %     <concept_id>10010405.10010489.10010490</concept_id>
    %     <concept_desc>Applied computing~Computer-assisted instruction</concept_desc>
    %     <concept_significance>500</concept_significance>
    % </concept>
   <concept>
       <concept_id>10003120.10003121.10003129</concept_id>
       <concept_desc>Human-centered computing~Interactive systems and tools</concept_desc>
       <concept_significance>500</concept_significance>
   </concept>
 </ccs2012>
\end{CCSXML}

\ccsdesc[500]{Human-centered computing~Empirical studies in HCI}
\ccsdesc[500]{Human-centered computing~Interactive systems and tools}
% \ccsdesc[500]{Applied computing~Computer-assisted instruction}


\keywords{Programming Learning, Self-Paced Learning, Large Language Models, AI for Coding, Human-AI Collaboration}

\maketitle


\section{Introduction}
\label{section:introduction}

% redirection is unique and important in VR
Virtual Reality (VR) systems enable users to embody virtual avatars by mirroring their physical movements and aligning their perspective with virtual avatars' in real time. 
As the head-mounted displays (HMDs) block direct visual access to the physical world, users primarily rely on visual feedback from the virtual environment and integrate it with proprioceptive cues to control the avatar’s movements and interact within the VR space.
Since human perception is heavily influenced by visual input~\cite{gibson1933adaptation}, 
VR systems have the unique capability to control users' perception of the virtual environment and avatars by manipulating the visual information presented to them.
Leveraging this, various redirection techniques have been proposed to enable novel VR interactions, 
such as redirecting users' walking paths~\cite{razzaque2005redirected, suma2012impossible, steinicke2009estimation},
modifying reaching movements~\cite{gonzalez2022model, azmandian2016haptic, cheng2017sparse, feick2021visuo},
and conveying haptic information through visual feedback to create pseudo-haptic effects~\cite{samad2019pseudo, dominjon2005influence, lecuyer2009simulating}.
Such redirection techniques enable these interactions by manipulating the alignment between users' physical movements and their virtual avatar's actions.

% % what is hand/arm redirection, motivation of study arm-offset
% \change{\yj{i don't understand the purpose of this paragraph}
% These illusion-based techniques provide users with unique experiences in virtual environments that differ from the physical world yet maintain an immersive experience. 
% A key example is hand redirection, which shifts the virtual hand’s position away from the real hand as the user moves to enhance ergonomics during interaction~\cite{feuchtner2018ownershift, wentzel2020improving} and improve interaction performance~\cite{montano2017erg, poupyrev1996go}. 
% To increase the realism of virtual movements and strengthen the user’s sense of embodiment, hand redirection techniques often incorporate a complete virtual arm or full body alongside the redirected virtual hand, using inverse kinematics~\cite{hartfill2021analysis, ponton2024stretch} or adjustments to the virtual arm's movement as well~\cite{li2022modeling, feick2024impact}.
% }

% noticeability, motivation of predicting a probability, not a classification
However, these redirection techniques are most effective when the manipulation remains undetected~\cite{gonzalez2017model, li2022modeling}. 
If the redirection becomes too large, the user may not mitigate the conflict between the visual sensory input (redirected virtual movement) and their proprioception (actual physical movement), potentially leading to a loss of embodiment with the virtual avatar and making it difficult for the user to accurately control virtual movements to complete interaction tasks~\cite{li2022modeling, wentzel2020improving, feuchtner2018ownershift}. 
While proprioception is not absolute, users only have a general sense of their physical movements and the likelihood that they notice the redirection is probabilistic. 
This probability of detecting the redirection is referred to as \textbf{noticeability}~\cite{li2022modeling, zenner2024beyond, zenner2023detectability} and is typically estimated based on the frequency with which users detect the manipulation across multiple trials.

% version B
% Prior research has explored factors influencing the noticeability of redirected motion, including the redirection's magnitude~\cite{wentzel2020improving, poupyrev1996go}, direction~\cite{li2022modeling, feuchtner2018ownershift}, and the visual characteristics of the virtual avatar~\cite{ogawa2020effect, feick2024impact}.
% While these factors focus on the avatars, the surrounding virtual environment can also influence the users' behavior and in turn affect the noticeability of redirection.
% One such prominent external influence is through the visual channel - the users' visual attention is constantly distracted by complex visual effects and events in practical VR scenarios.
% Although some prior studies have explored how to leverage user blindness caused by visual distractions to redirect users' virtual hand~\cite{zenner2023detectability}, there remains a gap in understanding how to quantify the noticeability of redirection under visual distractions.

% visual stimuli and gaze behavior
Prior research has explored factors influencing the noticeability of redirected motion, including the redirection's magnitude~\cite{wentzel2020improving, poupyrev1996go}, direction~\cite{li2022modeling, feuchtner2018ownershift}, and the visual characteristics of the virtual avatar~\cite{ogawa2020effect, feick2024impact}.
While these factors focus on the avatars, the surrounding virtual environment can also influence the users' behavior and in turn affect the noticeability of redirection.
This, however, remains underexplored.
One such prominent external influence is through the visual channel - the users' visual attention is constantly distracted by complex visual effects and events in practical VR scenarios.
We thus want to investigate how \textbf{visual stimuli in the virtual environment} affect the noticeability of redirection.
With this, we hope to complement existing works that focus on avatars by incorporating environmental visual influences to enable more accurate control over the noticeability of redirected motions in practical VR scenarios.
% However, in realistic VR applications, the virtual environment often contains complex visual effects beyond the virtual avatar itself. 
% We argue that these visual effects can \textbf{distract users’ visual attention and thus affect the noticeability of redirection offsets}, while current research has yet taken into account.
% For instance, in a VR boxing scenario, a user’s visual attention is likely focused on their opponent rather than on their virtual body, leading to a lower noticeability of redirection offsets on their virtual movements. 
% Conversely, when reaching for an object in the center of their field of view, the user’s attention is more concentrated on the virtual hand’s movement and position to ensure successful interaction, resulting in a higher noticeability of offsets.

Since each visual event is a complex choreography of many underlying factors (type of visual effect, location, duration, etc.), it is extremely difficult to quantify or parameterize visual stimuli.
Furthermore, individuals respond differently to even the same visual events.
Prior neuroscience studies revealed that factors like age, gender, and personality can influence how quickly someone reacts to visual events~\cite{gillon2024responses, gale1997human}. 
Therefore, aiming to model visual stimuli in a way that is generalizable and applicable to different stimuli and users, we propose to use users' \textbf{gaze behavior} as an indicator of how they respond to visual stimuli.
In this paper, we used various gaze behaviors, including gaze location, saccades~\cite{krejtz2018eye}, fixations~\cite{perkhofer2019using}, and the Index of Pupil Activity (IPA)~\cite{duchowski2018index}.
These behaviors indicate both where users are looking and their cognitive activity, as looking at something does not necessarily mean they are attending to it.
Our goal is to investigate how these gaze behaviors stimulated by various visual stimuli relate to the noticeability of redirection.
With this, we contribute a model that allows designers and content creators to adjust the redirection in real-time responding to dynamic visual events in VR.

To achieve this, we conducted user studies to collect users' noticeability of redirection under various visual stimuli.
To simulate realistic VR scenarios, we adopted a dual-task design in which the participants performed redirected movements while monitoring the visual stimuli.
Specifically, participants' primary task was to report if they noticed an offset between the avatar's movement and their own, while their secondary task was to monitor and report the visual stimuli.
As realistic virtual environments often contain complex visual effects, we started with simple and controlled visual stimulus to manage the influencing factors.

% first user study, confirmation study
% collect data under no visual stimuli, different basic visual stimuli
We first conducted a confirmation study (N=16) to test whether applying visual stimuli (opacity-based) actually affects their noticeability of redirection. 
The results showed that participants were significantly less likely to detect the redirection when visual stimuli was presented $(F_{(1,15)}=5.90,~p=0.03)$.
Furthermore, by analyzing the collected gaze data, results revealed a correlation between the proposed gaze behaviors and the noticeability results $(r=-0.43)$, confirming that the gaze behaviors could be leveraged to compute the noticeability.

% data collection study
We then conducted a data collection study to obtain more accurate noticeability results through repeated measurements to better model the relationship between visual stimuli-triggered gaze behaviors and noticeability of redirection.
With the collected data, we analyzed various numerical features from the gaze behaviors to identify the most effective ones. 
We tested combinations of these features to determine the most effective one for predicting noticeability under visual stimuli.
Using the selected features, our regression model achieved a mean squared error (MSE) of 0.011 through leave-one-user-out cross-validation. 
Furthermore, we developed both a binary and a three-class classification model to categorize noticeability, which achieved an accuracy of 91.74\% and 85.62\%, respectively.

% evaluation study
To evaluate the generalizability of the regression model, we conducted an evaluation study (N=24) to test whether the model could accurately predict noticeability with new visual stimuli (color- and scale-based animations).
Specifically, we evaluated whether the model's predictions aligned with participants' responses under these unseen stimuli.
The results showed that our model accurately estimated the noticeability, achieving mean squared errors (MSE) of 0.014 and 0.012 for the color- and scale-based visual stimili, respectively, compared to participants' responses.
Since the tested visual stimuli data were not included in the training, the results suggested that the extracted gaze behavior features capture a generalizable pattern and can effectively indicate the corresponding impact on the noticeability of redirection.

% application
Based on our model, we implemented an adaptive redirection technique and demonstrated it through two applications: adaptive VR action game and opportunistic rendering.
We conducted a proof-of-concept user study (N=8) to compare our adaptive redirection technique with a static redirection, evaluating the usability and benefits of our adaptive redirection technique.
The results indicated that participants experienced less physical demand and stronger sense of embodiment and agency when using the adaptive redirection technique. 
These results demonstrated the effectiveness and usability of our model.

In summary, we make the following contributions.
% 
\begin{itemize}
    \item 
    We propose to use users' gaze behavior as a medium to quantify how visual stimuli influences the noticebility of redirection. 
    Through two user studies, we confirm that visual stimuli significantly influences noticeability and identify key gaze behavior features that are closely related to this impact.
    \item 
    We build a regression model that takes the user's gaze behavioral data as input, then computes the noticeability of redirection.
    Through an evaluation study, we verify that our model can estimate the noticeability with new participants under unseen visual stimuli.
    These findings suggest that the extracted gaze behavior features effectively capture the influence of visual stimuli on noticeability and can generalize across different users and visual stimuli.
    \item 
    We develop an adaptive redirection technique based on our regression model and implement two applications with it.
    With a proof-of-concept study, we demonstrate the effectiveness and potential usability of our regression model on real-world use cases.

\end{itemize}

% \delete{
% Virtual Reality (VR) allows the user to embody a virtual avatar by mirroring their physical movements through the avatar.
% As the user's visual access to the physical world is blocked in tasks involving motion control, they heavily rely on the visual representation of the avatar's motions to guide their proprioception.
% Similar to real-world experiences, the user is able to resolve conflicts between different sensory inputs (e.g., vision and motor control) through multisensory integration, which is essential for mitigating the sensory noise that commonly arises.
% However, it also enables unique manipulations in VR, as the system can intentionally modify the avatar's movements in relation to the user's motions to achieve specific functional outcomes,
% for example, 
% % the manipulations on the avatar's movements can 
% enabling novel interaction techniques of redirected walking~\cite{razzaque2005redirected}, redirected reaching~\cite{gonzalez2022model}, and pseudo haptics~\cite{samad2019pseudo}.
% With small adjustments to the avatar's movements, the user can maintain their sense of embodiment, due to their ability to resolve the perceptual differences.
% % However, a large mismatch between the user and avatar's movements can result in the user losing their sense of embodiment, due to an inability to resolve the perceptual differences.
% }

% \delete{
% However, multisensory integration can break when the manipulation is so intense that the user is aware of the existence of the motion offset and no longer maintains the sense of embodiment.
% Prior research studied the intensity threshold of the offset applied on the avatar's hand, beyond which the embodiment will break~\cite{li2022modeling}. 
% Studies also investigated the user's sensitivity to the offsets over time~\cite{kohm2022sensitivity}.
% Based on the findings, we argue that one crucial factor that affects to what extent the user notices the offset (i.e., \textit{noticeability}) that remains under-explored is whether the user directs their visual attention towards or away from the virtual avatar.
% Related work (e.g., Mise-unseen~\cite{marwecki2019mise}) has showcased applications where adjustments in the environment can be made in an unnoticeable manner when they happen in the area out of the user's visual field.
% We hypothesize that directing the user's visual attention away from the avatar's body, while still partially keeping the avatar within the user's field-of-view, can reduce the noticeability of the offset.
% Therefore, we conduct two user studies and implement a regression model to systematically investigate this effect.
% }

% \delete{
% In the first user study (N = 16), we test whether drawing the user's visual attention away from their body impacts the possibility of them noticing an offset that we apply to their arm motion in VR.
% We adopt a dual-task design to enable the alteration of the user's visual attention and a yes/no paradigm to measure the noticeability of motion offset. 
% The primary task for the user is to perform an arm motion and report when they perceive an offset between the avatar's virtual arm and their real arm.
% In the secondary task, we randomly render a visual animation of a ball turning from transparent to red and becoming transparent again and ask them to monitor and report when it appears.
% We control the strength of the visual stimuli by changing the duration and location of the animation.
% % By changing the time duration and location of the visual animation, we control the strengths of attraction to the users.
% As a result, we found significant differences in the noticeability of the offsets $(F_{(1,15)}=5.90,~p=0.03)$ between conditions with and without visual stimuli.
% Based on further analysis, we also identified the behavioral patterns of the user's gaze (including pupil dilation, fixations, and saccades) to be correlated with the noticeability results $(r=-0.43)$ and they may potentially serve as indicators of noticeability.
% }

% \delete{
% To further investigate how visual attention influences the noticeability, we conduct a data collection study (N = 12) and build a regression model based on the data.
% The regression model is able to calculate the noticeability of the offset applied on the user's arm under various visual stimuli based on their gaze behaviors.
% Our leave-one-out cross-validation results show that the proposed method was able to achieve a mean-squared error (MSE) of 0.012 in the probability regression task.
% }

% \delete{
% To verify the feasibility and extendability of the regression model, we conduct an evaluation study where we test new visual animations based on adjustments on scale and color and invite 24 new participants to attend the study.
% Results show that the proposed method can accurately estimate the noticeability with an MSE of 0.014 and 0.012 in the conditions of the color- and scale-based visual effects.
% Since these animations were not included in the dataset that the regression model was built on, the study demonstrates that the gaze behavioral features we extracted from the data capture a generalizable pattern of the user's visual attention and can indicate the corresponding impact on the noticeability of the offset.
% }

% \delete{
% Finally, we demonstrate applications that can benefit from the noticeability prediction model, including adaptive motion offsets and opportunistic rendering, considering the user's visual attention. 
% We conclude with discussions of our work's limitations and future research directions.
% }

% \delete{
% In summary, we make the following contributions.
% }
% % 
% \begin{itemize}
%     \item 
%     \delete{
%     We quantify the effects of the user's visual attention directed away by stimuli on their noticeability of an offset applied to the avatar's arm motion with respect to the user's physical arm. 
%     Through two user studies, we identified gaze behavioral features that are indicative of the changes in noticeability.
%     }
%     \item 
%     \delete{We build a regression model that takes the user's gaze behavioral data and the offset applied to the arm motion as input, then computes the probability of the user noticing the offset.
%     Through an evaluation study, we verified that the model needs no information about the source attracting the user's visual attention and can be generalizable in different scenarios.
%     }
%     \item 
%     \delete{We demonstrate two applications that potentially benefit from the regression model, including adaptive motion offsets and opportunistic rendering.
%     }

% \end{itemize}

\begin{comment}
However, users will lose the sense of embodiment to the virtual avatars if they notice the offset between the virtual and physical movements.
To address this, researchers have been exploring the noticing threshold of offsets with various magnitudes and proposing various redirection techniques that maintain the sense of embodiment~\cite{}.

However, when users embody virtual avatars to explore virtual environments, they encounter various visual effects and content that can attract their attention~\cite{}.
During this, the user may notice an offset when he observes the virtual movement carefully while ignoring it when the virtual contents attract his attention from the movements.
Therefore, static offset thresholds are not appropriate in dynamic scenarios.

Past research has proposed dynamic mapping techniques that adapted to users' state, such as hand moving speed~\cite{frees2007prism} or ergonomically comfortable poses~\cite{montano2017erg}, but not considering the influence of virtual content.
More specifically, PRISM~\cite{frees2007prism} proposed adjusting the C/D ratio with a non-linear mapping according to users' hand moving speed, but it might not be optimal for various virtual scenarios.
While Erg-O~\cite{montano2017erg} redirected users' virtual hands according to the virtual target's relative position to reduce physical fatigue, neglecting the change of virtual environments. 

Therefore, how to design redirection techniques in various scenarios with different visual attractions remains unknown.
To address this, we investigate how visual attention affects the noticing probability of movement offsets.
Based on our experiments, we implement a computational model that automatically computes the noticing probability of offsets under certain visual attractions.
VR application designers and developers can easily leverage our model to design redirection techniques maintaining the sense of embodiment adapt to the user's visual attention.
We implement a dynamic redirection technique with our model and demonstrate that it effectively reduces the target reaching time without reducing the sense of embodiment compared to static redirection techniques.

% Need to be refined
This paper offers the following contributions.
\begin{itemize}
    \item We investigate how visual attractions affect the noticing probability of redirection offsets.
    \item We construct a computational model to predict the noticing probability of an offset with a given visual background.
    \item We implement a dynamic redirection technique adapting to the visual background. We evaluate the technique and develop three applications to demonstrate the benefits. 
\end{itemize}



First, we conducted a controlled experiment to understand how users perceived the movement offset while subjected to various distractions.
Since hand redirection is one of the most frequently used redirections in VR interactions, we focused on the dynamic arm movements and manually added angular offsets to the' elbow joint~\cite{li2022modeling, gonzalez2022model, zenner2019estimating}. 
We employed flashing spheres in the user's field of view as distractions to attract users' visual attention.
Participants were instructed to report the appearing location of the spheres while simultaneously performing the arm movements and reporting if they perceived an offset during the movement. 
(\zhipeng{Add the results of data collection. Analyze the influence of the distance between the gaze map and the offset.}
We measured the visual attraction's magnitude with the gaze distribution on it.
Results showed that stronger distractions made it harder for users to notice the offset.)
\zhipeng{Need to rewrite. Not sure to use gaze distribution or a metric obtained from the visual content.}
Secondly, we constructed a computational model to predict the noticing probability of offsets with given visual content.
We analyzed the data from the user studies to measure the influence of visual attractions on the noticing probability of offsets.
We built a statistical model to predict the offset's noticing probability with a given visual content.
Based on the model, we implement a dynamic redirection technique to adjust the redirection offset adapted to the user's current field of view.
We evaluated the technique in a target selection task compared to no hand redirection and static hand redirection.
\zhipeng{Add the results of the evaluation.}
Results showed that the dynamic hand redirection technique significantly reduced the target selection time with similar accuracy and a comparable sense of embodiment.
Finally, we implemented three applications to demonstrate the potential benefits of the visual attention adapted dynamic redirection technique.
\end{comment}

% This one modifies arm length, not redirection
% \citeauthor{mcintosh2020iteratively} proposed an adaptation method to iteratively change the virtual avatar arm's length based on the primary tasks' performance~\cite{mcintosh2020iteratively}.



% \zhipeng{TO ADD: what is redirection}
% Redirection enables novel interactions in Virtual Reality, including redirected walking, haptic redirection, and pseudo haptics by introducing an offset to users' movement.
% \zhipeng{TO ADD: extend this sentence}
% The price of this is that users' immersiveness and embodiment in VR can be compromised when they notice the offset and perceive the virtual movement not as theirs~\cite{}.
% \zhipeng{TO ADD: extend this sentence, elaborate how the virtual environment attracts users' attention}
% Meanwhile, the visual content in the virtual environment is abundant and consistently captures users' attention, making it harder to notice the offset~\cite{}.
% While previous studies explored the noticing threshold of the offsets and optimized the redirection techniques to maintain the sense of embodiment~\cite{}, the influence of visual content on the probability of perceiving offsets remains unknown.  
% Therefore, we propose to investigate how users perceive the redirection offset when they are facing various visual attractions.


% We conducted a user study to understand how users notice the shift with visual attractions.
% We used a color-changing ball to attract the user's attention while instructing users to perform different poses with their arms and observe it meanwhile.
% \zhipeng{(Which one should be the primary task? Observe the ball should be the primary one, but if the primary task is too simple, users might allocate more attention on the secondary task and this makes the secondary task primary.)}
% \zhipeng{(We need a good and reasonable dual-task design in which users care about both their pose and the visual content, at least in the evaluation study. And we need to be able to control the visual content's magnitude and saliency maybe?)}
% We controlled the shift magnitude and direction, the user's pose, the ball's size, and the color range.
% We set the ball's color-changing interval as the independent factor.
% We collect the user's response to each shift and the color-changing times.
% Based on the collected data, we constructed a statistical model to describe the influence of visual attraction on the noticing probability.
% \zhipeng{(Are we actually controlling the attention allocation? How do we measure the attracting effect? We need uniform metrics, otherwise it is also hard for others to use our knowledge.)}
% \zhipeng{(Try to use eye gaze? The eye gaze distribution in the last five seconds to decide the attention allocation? Basically constructing a model with eye gaze distribution and noticing probability. But the user's head is moving, so the eye gaze distribution is not aligned well with the current view.)}

% \zhipeng{Saliency and EMD}
% \zhipeng{Gaze is more than just a point: Rethinking visual attention
% analysis using peripheral vision-based gaze mapping}

% Evaluation study(ideal case): based on the visual content, adjusting the redirection magnitude dynamically.

% \zhipeng{(The risk is our model's effect is trivial.)}

% Applications:
% Playing Lego while watching demo videos, we can accelerate the reaching process of bricks, and forbid the redirection during the manipulation.

% Beat saber again: but not make a lot of sense? Difficult game has complicated visual effects, while allows larger shift, but do not need large shift with high difficulty



\section{Related Work}
\label{lit_review}

\begin{highlight}
{

Our research builds upon {\em (i)} Assessing Web Accessibility, {\em (ii)} End-User Accessibility Repair, and {\em (iii)} Developer Tools for Accessibility.

\subsection{Assessing Web Accessibility}
From the earliest attempts to set standards and guidelines, web accessibility has been shaped by a complex interplay of technical challenges, legal imperatives, and educational campaigns. Over the past 25 years, stakeholders have sought to improve digital inclusion by establishing foundational standards~\cite{chisholm2001web, caldwell2008web}, enforcing legal obligations~\cite{sierkowski2002achieving, yesilada2012understanding}, and promoting a broader culture of accessibility awareness among developers~\cite{sloan2006contextual, martin2022landscape, pandey2023blending}. 
Despite these longstanding efforts, systemic accessibility issues persist. According to the 2024 WebAIM Million report~\cite{webaim2024}, 95.9\% of the top one million home pages contained detectable WCAG violations, averaging nearly 57 errors per page. 
These errors take many forms: low color contrast makes the interface difficult for individuals with color deficiency or low vision to read text; missing alternative text leaves users relying on screen readers without crucial visual context; and unlabeled form inputs or empty links and buttons hinder people who navigate with assistive technologies from completing basic tasks. 
Together, these accessibility issues not only limit user access to critical online resources such as healthcare, education, and employment but also result in significant legal risks and lost opportunities for businesses to engage diverse audiences. Addressing these pervasive issues requires systematic methods to identify, measure, and prioritize accessibility barriers, which is the first step toward achieving meaningful improvements.

Prior research has introduced methods blending automation and human evaluation to assess web accessibility. Hybrid approaches like SAMBA combine automated tools with expert reviews to measure the severity and impact of barriers, enhancing evaluation reliability~\cite{brajnik2007samba}. Quantitative metrics, such as Failure Rate and Unified Web Evaluation Methodology, support large-scale monitoring and comparative analysis, enabling cost-effective insights~\cite{vigo2007quantitative, martins2024large}. However, automated tools alone often detect less than half of WCAG violations and generate false positives, emphasizing the need for human interpretation~\cite{freire2008evaluation, vigo2013benchmarking}. Recent progress with large pretrained models like Large Language Models (LLMs)~\cite{dubey2024llama,bai2023qwen} and Large Multimodal Models (LMMs)~\cite{liu2024visual, bai2023qwenvl} offers a promising step forward, automating complex checks like non-text content evaluation and link purposes, achieving higher detection rates than traditional tools~\cite{lopez2024turning, delnevo2024interaction}. Yet, these large models face challenges, including dependence on training data, limited contextual judgment, and the inability to simulate real user experiences. These limitations underscore the necessity of combining models with human oversight for reliable, user-centered evaluations~\cite{brajnik2007samba, vigo2013benchmarking, delnevo2024interaction}. 

Our work builds on these prior efforts and recent advancements by leveraging the capabilities of large pretrained models while addressing their limitations through a developer-centric approach. CodeA11y integrates LLM-powered accessibility assessments, tailored accessibility-aware system prompts, and a dedicated accessibility checker directly into GitHub Copilot---one of the most widely used coding assistants. Unlike standalone evaluation tools, CodeA11y actively supports developers throughout the coding process by reinforcing accessibility best practices, prompting critical manual validations, and embedding accessibility considerations into existing workflows.
% This pervasive shortfall reflects the difficulty of scaling traditional approaches---such as manual audits and automated tools---that either demand immense human effort or lack the nuanced understanding needed to capture real-world user experiences. 
%
% In response, a new wave of AI-driven methods, many powered by large language models (LLMs), is emerging to bridge these accessibility detection and assessment gaps. Early explorations, such as those by Morillo et al.~\cite{morillo2020system}, introduced AI-assisted recommendations capable of automatic corrections, illustrating how computational intelligence can tackle the repetitive, common errors that plague large swaths of the web. Building on this foundation, Huang et al.~\cite{huang2024access} proposed ACCESS, a prompt-engineering framework that streamlines the identification and remediation of accessibility violations, while López-Gil et al.~\cite{lopez2024turning} demonstrated how LLMs can help apply WCAG success criteria more consistently---reducing the reliance on manual effort. Beyond these direct interventions, recent work has also begun integrating user experiences more seamlessly into the evaluation process. For example, Huq et al.~\cite{huq2024automated} translate user transcripts and corresponding issues into actionable test reports, ensuring that accessibility improvements align more closely with authentic user needs.
% However, as these AI-driven solutions evolve, researchers caution against uncritical adoption. Othman et al.~\cite{othman2023fostering} highlight that while LLMs can accelerate remediation, they may also introduce biases or encourage over-reliance on automated processes. Similarly, Delnevo et al.~\cite{delnevo2024interaction} emphasize the importance of contextual understanding and adaptability, pointing to the current limitations of LLM-based systems in serving the full spectrum of user needs. 
% In contrast to this backdrop, our work introduces and evaluates CodeA11y, an LLM-augmented extension for GitHub Copilot that not only mitigates these challenges by providing more consistent guidance and manual validation prompts, but also aligns AI-driven assistance with developers’ workflows, ultimately contributing toward more sustainable propulsion for building accessible web.

% Broader implications of inaccessibility—legal compliance, ethical concerns, and user experience
% A Historical Review of Web Accessibility Using WAVE
% "I tend to view ads almost like a pestilence": On the Accessibility Implications of Mobile Ads for Blind Users

% In the research domain, several methods have been developed to assess and enhance web accessibility. These include incorporating feedback into developer tools~\cite{adesigner, takagi2003accessibility, bigham2010accessibility} and automating the creation of accessibility tests and reports for user interfaces~\cite{swearngin2024towards, taeb2024axnav}. 

% Prior work has also studied accessibility scanners as another avenue of AI to improve web development practices~\cite{}.
% However, a persistent challenge is that developers need to be aware of these tools to utilize them effectively. With recent advancements in LLMs, developers might now build accessible websites with less effort using AI assistants. However, the impact of these assistants on the accessibility of their generated code remains unclear. This study aims to investigate these effects.

\subsection{End-user Accessibility Repair}
In addition to detecting accessibility errors and measuring web accessibility, significant research has focused on fixing these problems.
Since end-users are often the first to notice accessibility problems and have a strong incentive to address them, systems have been developed to help them report or fix these problems.

Collaborative, or social accessibility~\cite{takagi2009collaborative,sato2010social}, enabled these end-user contributions to be scaled through crowd-sourcing.
AccessMonkey~\cite{bigham2007accessmonkey} and Accessibility Commons~\cite{kawanaka2008accessibility} were two examples of repositories that store accessibility-related scripts and metadata, respectively.
Other work has developed browser extensions that leverage crowd-sourced databases to automatically correct reading order, alt-text, color contrast, and interaction-related issues~\cite{sato2009s,huang2015can}.

One drawback of collaborative accessibility approaches is that they cannot fix problems for an ``unseen'' web page on-demand, so many projects aim to automatically detect and improve interfaces without the need for an external source of fixes.
A large body of research has focused on making specific web media (e.g., images~\cite{gleason2019making,guinness2018caption, twitterally, gleason2020making, lee2021image}, design~\cite{potluri2019ai,li2019editing, peng2022diffscriber, peng2023slide}, and videos~\cite{pavel2020rescribe,peng2021say,peng2021slidecho,huh2023avscript}) accessible through a combination of machine learning (ML) and user-provided fixes.
Other work has focused on applying more general fixes across all websites.

Opportunity accessibility addressed a common accessibility problem of most websites: by default, content is often hard to see for people with visual impairments, and many users, especially older adults, do not know how to adjust or enable content zooming~\cite{bigham2014making}.
To this end, a browser script (\texttt{oppaccess.js}) was developed that automatically adjusted the browser's content zoom to maximally enlarge content without introducing adverse side-effects (\textit{e.g.,} content overlap).
While \texttt{oppaccess.js} primarily targeted zoom-related accessibility, recent work aimed to enable larger types of changes, by using LLMs to modify the source code of web pages based on user questions or directives~\cite{li2023using}.

Several efforts have been focused on improving access to desktop and mobile applications, which present additional challenges due to the unavailability of app source code (\textit{e.g.,} HTML).
Prefab is an approach that allows graphical UIs to be modified at runtime by detecting existing UI widgets, then replacing them~\cite{dixon2010prefab}.
Interaction Proxies used these runtime modification strategies to ``repair'' Android apps by replacing inaccessible widgets with improved alternatives~\cite{zhang2017interaction, zhang2018robust}.
The widget detection strategies used by these systems previously relied on a combination of heuristics and system metadata (\textit{e.g.,} the view hierarchy), which are incomplete or missing in the accessible apps.
To this end, ML has been employed to better localize~\cite{chen2020object} and repair UI elements~\cite{chen2020unblind,zhang2021screen,wu2023webui,peng2025dreamstruct}.

In general, end-user solutions to repairing application accessibility are limited due to the lack of underlying code and knowledge of the semantics of the intended content.

\subsection{Developer Tools for Accessibility}
Ultimately, the best solution for ensuring an accessible experience lies with front-end developers. Many efforts have focused on building adequate tooling and support to help developers with ensuring that their UI code complies with accessibility standards.

Numerous automated accessibility testing tools have been created to help developers identify accessibility issues in their code: i) static analysis tools, such as IBM Equal Access Accessibility Checker~\cite{ibm2024toolkit} or Microsoft Accessibility Insights~\cite{accessibilityinsights2024}, scan the UI code's compliance with predefined rules derived from accessibility guidelines; and ii) dynamic or runtime accessibility scanners, such as Chrome Devtools~\cite{chromedevtools2024} or axe-Core Accessibility Engine~\cite{deque2024axe}, perform real-time testing on user interfaces to detect interaction issues not identifiable from the code structure. While these tools greatly reduce the manual effort required for accessibility testing, they are often criticized for their limited coverage. Thus, experts often recommend manually testing with assistive technologies to uncover more complex interaction issues. Prior studies have created accessibility crawlers that either assist in developer testing~\cite{swearngin2024towards,taeb2024axnav} or simulate how assistive technologies interact with UIs~\cite{10.1145/3411764.3445455, 10.1145/3551349.3556905, 10.1145/3544548.3580679}.

Similar to end-user accessibility repair, research has focused on generating fixes to remediate accessibility issues in the UI source code. Initial attempts developed heuristic-based algorithms for fixing specific issues, for instance, by replacing text or background color attributes~\cite{10.1145/3611643.3616329}. More recent work has suggested that the code-understanding capabilities of LLMs allow them to suggest more targeted fixes.
For example, a study demonstrated that prompting ChatGPT to fix identified WCAG compliance issues in source code could automatically resolve a significant number of them~\cite{othman2023fostering}. Researchers have sought to leverage this capability by employing a multi-agent LLM architecture to automatically identify and localize issues in source code and suggest potential code fixes~\cite{mehralian2024automated}.

While the approaches mentioned above focus on assessing UI accessibility of already-authored code (\textit{i.e.,} fixing existing code), there is potential for more proactive approaches.
For example, LLMs are often used by developers to generate UI source code from natural language descriptions or tab completions~\cite{chen2021evaluating,GitHubCopilot,lozhkov2024starcoder,hui2024qwen2,roziere2023code,zheng2023codegeex}, but LLMs frequently produce inaccessible code by default~\cite{10.1145/3677846.3677854,mowar2024tab}, leading to inaccessible output when used by developers without sufficient awareness of accessibility knowledge.
The primary focus of this paper is to design a more accessibility-aware coding assistant that both produces more accessible code without manual intervention (\textit{e.g.,} specific user prompting) and gradually enables developers to implement and improve accessibility of automatically-generated code through IDE UI modifications (\textit{e.g.}, reminder notifications).

}
\end{highlight}



% Work related to this paper includes {\em (i)} Web Accessibility and {\em (ii)} Developer Practices in AI-Assisted Programming.

% \ipstart{Web Accessibility: Practice, Evaluation, and Improvements} Substantial efforts have been made to set accessibility standards~\cite{chisholm2001web, caldwell2008web}, establish legal requirements~\cite{sierkowski2002achieving, yesilada2012understanding}, and promote education and advocacy among developers~\cite{sloan2006contextual, martin2022landscape, pandey2023blending}. In the research domain, several methods have been developed to assess and enhance web accessibility. These include incorporating feedback into developer tools~\cite{adesigner, takagi2003accessibility, bigham2010accessibility} and automating the creation of accessibility tests and reports for user interfaces~\cite{swearngin2024towards, taeb2024axnav}. 
% % Prior work has also studied accessibility scanners as another avenue of AI to improve web development practices~\cite{}.
% However, a persistent challenge is that developers need to be aware of these tools to utilize them effectively. With recent advancements in LLMs, developers might now build accessible websites with less effort using AI assistants. However, the impact of these assistants on the accessibility of their generated code remains unclear. This study aims to investigate these effects.

% \ipstart{Developer Practices in AI-Assisted Programming}
% Recent usability research on AI-assisted development has examined the interaction strategies of developers while using AI coding assistants~\cite{barke2023grounded}.
% They observed developers interacted with these assistants in two modes -- 1) \textit{acceleration mode}: associated with shorter completions and 2) \textit{exploration mode}: associated with long completions.
% Liang {\em et al.} \cite{liang2024large} found that developers are driven to use AI assistants to reduce their keystrokes, finish tasks faster, and recall the syntax of programming languages. On the other hand, developers' reason for rejecting autocomplete suggestions was the need for more consideration of appropriate software requirements. This is because primary research on code generation models has mainly focused on functional correctness while often sidelining non-functional requirements such as latency, maintainability, and security~\cite{singhal2024nofuneval}. Consequently, there have been increasing concerns about the security implications of AI-generated code~\cite{sandoval2023lost}. Similarly, this study focuses on the effectiveness and uptake of code suggestions among developers in mitigating accessibility-related vulnerabilities. 


% ============================= additional rw ============================================
% - Paulina Morillo, Diego Chicaiza-Herrera, and Diego Vallejo-Huanga. 2020. System of Recommendation and Automatic Correction of Web Accessibility Using Artificial Intelligence. In Advances in Usability and User Experience, Tareq Ahram and Christianne Falcão (Eds.). Springer International Publishing, Cham, 479–489
% - Juan-Miguel López-Gil and Juanan Pereira. 2024. Turning manual web accessibility success criteria into automatic: an LLM-based approach. Universal Access in the Information Society (2024). https://doi.org/10.1007/s10209-024-01108-z
% - s
% - Calista Huang, Alyssa Ma, Suchir Vyasamudri, Eugenie Puype, Sayem Kamal, Juan Belza Garcia, Salar Cheema, and Michael Lutz. 2024. ACCESS: Prompt Engineering for Automated Web Accessibility Violation Corrections. arXiv:2401.16450 [cs.HC] https://arxiv.org/abs/2401.16450
% - Syed Fatiul Huq, Mahan Tafreshipour, Kate Kalcevich, and Sam Malek. 2025. Automated Generation of Accessibility Test Reports from Recorded User Transcripts. In Proceedings of the 47th International Conference on Software Engineering (ICSE) (Ottawa, Ontario, Canada). IEEE. https://ics.uci.edu/~seal/publications/2025_ICSE_reca11.pdf To appear in IEEE Xplore
% - Achraf Othman, Amira Dhouib, and Aljazi Nasser Al Jabor. 2023. Fostering websites accessibility: A case study on the use of the Large Language Models ChatGPT for automatic remediation. In Proceedings of the 16th International Conference on PErvasive Technologies Related to Assistive Environments (Corfu, Greece) (PETRA ’23). Association for Computing Machinery, New York, NY, USA, 707–713. https://doi.org/10.1145/3594806.3596542
% - Zsuzsanna B. Palmer and Sushil K. Oswal. 0. Constructing Websites with Generative AI Tools: The Accessibility of Their Workflows and Products for Users With Disabilities. Journal of Business and Technical Communication 0, 0 (0), 10506519241280644. https://doi.org/10.1177/10506519241280644
% ============================= additional rw ============================================


\section{Formative Study}
In addition to traditional programming support tools like LeetCode, search engines, and Q\&A sites like Stack Overflow, AI-assisted tools such as ChatGPT and GitHub Copilot have further enriched these resources. 
However, it remains unclear whether these tools effectively enhance algorithmic programming learning and whether challenges persist despite their availability.
To explore this, we conducted a formative study using contextual inquiry and interviews to understand learner's needs and obstacles. 
Our study focuses on students who have completed foundational computer science courses and are working to improve their algorithmic problem-solving skills. 
%We are not focusing on K-12 students or novice learners just beginning to understand programming logic. Instead, our goal is to support students who already possess basic programming knowledge but seek to improve their problem-solving abilities with algorithms. 

%LeetCode\footnote{https://leetcode.com/}, a widely-used platform for honing algorithmic skills, exemplifies this focus.

% \subsection{Materials}
% To ground participants' feedback in recent experiences, we provided unrestricted access to various representative support tools, including Leetcode, Search Engines, ChatGPT, Github Copilot, Stake Overflow, etc. A detailed description of these tools can be seen in Sec xxx in Appendix.


%Participants had unrestricted access to these tools to explore the challenges in algorithmic programming learning, despite the wide availability of learning support resources.



% \section{Formative Study}


% With the proliferation of programming support tools, students have access to extensive resources ranging from online platforms like LeetCode, to search engines such as Google and Bing, and Q\&A sites like Stack Overflow. Additionally, innovations in large language models (LLMs) have introduced tools like ChatGPT and GitHub Copilot. Despite this wealth of options, questions persist about their actual effectiveness in the programming learning process. Are these tools genuinely aiding learning? Do students still encounter challenges in algorithmic programming despite these resources? What specific issues arise? To address these questions, we conducted a formative study using contextual inquiry and interviews to better understand the needs and hurdles faced by learners.


% \subsection{Scope of Our Target Users}
% Before embarking on our formative study, it is crucial to define the scope of our target users precisely. We are not focusing on K-12 students learning basic programming logic or novice learners in introductory computer science (CS1). Our target group comprises students who have already acquired the foundational knowledge from CS1 and are now aiming to improve their ability to apply algorithms in solving real-world problems. Our goal is to aid these students in effectively utilizing algorithms to tackle specific challenges, rather than teaching the basic concepts and principles. A prime example of this focus is LeetCode\footnote{https://leetcode.com/}, an online platform extensively used by students and professionals to hone their algorithmic skills.

% \subsection{Materials} Despite our participants' prior experience with algorithm programming, we required them to re-engage with the process to ensure their feedback was based on the most current experiences. We facilitated this by providing access to various representative support tools and having participants familiarize themselves with their latest features. We combined contextual inquiry with post-hoc interviews to collect their insights. The materials prepared for their practice in algorithm programming included:

% \begin{itemize} \item \textbf{Programming Learning Platform: LeetCode} - LeetCode features a comprehensive problem bank with over 1,000 questions, each comprising a problem description, an editor, and a results area with test cases. We selected problems that offer extensive solution resources, allowing participants to access rich references through the \emph{editorial panel} (official solutions) or the \emph{solution panel} (community-contributed solutions). \item \textbf{Search Engines: Google and Bing} - Participants can freely search anything related to the programming problem. \item \textbf{Conversational LLM Tool: ChatGPT\footnote{https://chat.openai.com/}} - Participants were provided with accounts to access ChatGPT with GPT-4o model, enabling them to ask programming questions, seek code assistance, get auto-completion suggestions, and address syntax errors using natural language via an accessible web interface. \item \textbf{Pair Programming Tool: GitHub Copilot\footnote{https://github.com/features/copilot}} - Integrated with code editors such as Visual Studio Code, GitHub Copilot offers real-time coding suggestions including snippets, functions, and complete blocks, thus enhancing the coding process. We provided participants with GitHub Codespaces\footnote{https://github.com/features/codespaces} integrated with Copilot for their use. \item \textbf{Programming Q\&A Platform: Stack Overflow} - A community-driven platform where developers exchange coding-related queries and solutions, Stack Overflow serves as a crucial resource for resolving technical problems and acquiring new programming skills. \end{itemize}

% To thoroughly investigate the challenges learners still face despite these resources, we granted participants unrestricted access to all provided tools.

\subsection{Study Procedure}
We recruited 15 university students (5F, 10M, aged 18-29), including 10 undergraduates and 5 graduate students. 
Most (11) were computer science majors, with the rest from related fields such as data science and electrical engineering. 
All participants had prior experience with LeetCode or similar platforms, 14 had used ChatGPT for programming tasks, and eight had experience with GitHub Copilot or similar tools. 
Each session lasted 40 minutes with \$10 compensation.

During the study, participants used tools like LeetCode, search engines, ChatGPT, and GitHub Copilot to solve a randomly assigned algorithmic problem for 20 minutes, during which we observed their tool usage and conducted contextual inquiries as notable behaviors arose. Afterward, we reviewed their activities and conducted a semi-structured interview to discuss their tool choices, usage, assistance received, and perceptions of the tools, including any benefits or drawbacks they experienced.

\subsection{Data Analysis and Results}
We conducted inductive thematic analysis \cite{hsieh2005three} on interviews and contextual inquiry data. Recorded sessions were transcribed and manually reviewed and corrected by the first author. 
Two authors then independently coded the data and discussed to finalize themes and categorizations. 
The analysis revealed four key challenges students face with existing assistance tools during algorithmic programming learning.

% \subsubsection{\textbf{Challenge 1: Excessive Help Hindering Active Learning}} Students expressed concerns about how easily accessible solutions on platforms like LeetCode and ChatGPT hinder independent problem-solving. Most of the time, learners only need minimal guidance to overcome specific hurdles, but these tools frequently provide complete solutions, which can prematurely reveal answers and prevent independent thinking. As P1 noted, ``\emph{I just wanted help with the step I'm stuck on, but the solution panel [in Leetcode] showed everything, making it hard to ignore what I didn't need.}'' Similarly, P3 shared, ``\emph{Even if I rewrote the code after seeing the solution, it felt uninteresting and unfulfilling.}'' Participants also found ChatGPT provided too much assistance. P5 explained, ``\emph{ChatGPT often directly showed the complete solution and code. Once I saw the solution, it was hard to ignore, and it felt like I was not really improving my programming skills.}''

% \subsubsection{\textbf{Challenge 2: Difficulty in Utilizing ChatGPT for Effective Learning}} Participants found it challenging to use ChatGPT as a learning tool, despite its accessibility and coding capabilities. Many felt it wasn't designed to support active learning. P10 shared, ``\emph{Even though I just asked GPT about the specific problem I was facing, ChatGPT went straight to showing the full solution and code after answering my question, which deprives me of the chance to solve it on my own. Writing the code after seeing the solution feels like cheating.}'' P12 noted the effort required to make ChatGPT useful for learning: ``\emph{To get it to help me learn rather than just solve the problem, I have to carefully craft prompts. This is cumbersome, so I often just copy the problem description directly into ChatGPT. }''

\ms{
\subsubsection{\textbf{Challenge 1: Excessive Help Hindering Active Learning}}
Students expressed concerns that platforms like LeetCode and ChatGPT provide solutions that are too easily accessible, hindering their ability to engage in independent problem-solving. 
While learners typically need minimal guidance to overcome specific hurdles, these tools often present complete solutions prematurely, making the learning experience ``\emph{uninteresting}'' and ``\emph{unfulfilling}'' (P3). As P1 noted, ``\emph{I just wanted help with the step I'm stuck on, but the solution panel [in Leetcode] showed everything, making it hard to ignore what I didn't need.}'' Participants also found it difficult to use ChatGPT as a learning tool, as it often provided full solutions rather than encouraging active engagement. P5 explained, ``\emph{ChatGPT often directly showed the complete solution and code, even though I just asked GPT about the specific problem I was facing. Once I saw the solution, it was hard to ignore it. It felt like I was not really improving my programming skills.}'' Additionally, participants like P12 noted that making ChatGPT useful for active learning required significant effort to ``carefully craft prompt'', as a result they ``\emph{just copy the problem description directly into ChatGPT}''.
}


% \subsection{Results}
% Through our qualitative analysis, we identified several challenges that students face when using existing programming assistance tools to learn algorithm programming.

% \subsubsection{Challenge 1: Receiving Excessive Help That Prevents Learners from Active Learning}
% Students found it too easy to access complete solutions. While LeetCode's editorial and solution features often provide high-quality solutions, organized progressively from intuition to steps to final code, they frequently reveal the entire solution too quickly, limiting students' ability to think independently. The lack of targeted assistance for specific difficulties means that students, regardless of where they get stuck—whether at the beginning, middle, or end of the problem-solving process—are often presented with the full answer. In fact, learners often need just simple guidance. Sometimes students are stuck on a nuanced aspect and only need a little guidance to proceed. However, the current methods provide the entire correct procedure, often including the complete solution and code. When receiving such help, students don't feel that they are the ones solving the problem. They doubt whether they have really mastered the problem and question the effectiveness of their practice and learning. For instance, P1 stated, "I just want to know how to get past the step I'm stuck on, but the solution panel shows all the steps. I try to focus only on what I need, but it's hard not to see everything at once." P3 mentioned, "I want to solve the problem on my own. Even if I go back to the editor and write the code after seeing the answer, it feels uninteresting and gives me no sense of achievement."

% Similar issues arose with tools like ChatGPT. Participants noted that ChatGPT often provided too much help, which impeded their learning process. For example, P5 shared, "When I described the problem and the difficulties I was facing to ChatGPT, it immediately told me how to solve my issue and provided the next steps, even offering complete code. I tried to tell myself not to look at the answers it gave, but I couldn't resist. After seeing the solution, when I returned to the editor to write my own code, I felt like my programming skills weren't being properly exercised."










% (ZPD) Overwhelming Answers from GPT: Asking GPT often results in receiving too much information, including both the thought process and the code. Unless participants carefully crafted their instructions to GPT, they found it exhausting to manage, as they didn't want to simultaneously think about solving the problem and how to better prompt GPT.



% \subsubsection{Challenge 3: Provided Solutions or How to Decompose the Solution May Not Align with the Approach that Learners are Exploring}
% When learners view solutions from Leetcode's solution panel or editorial panel, the solutions provided were not what learners wanted to use. From a high level, learners couldn't find a solution that aligned with their approach. From a detailed level, even the general idea was similar (e.g., both the learners and the provided solutions adopt a binary search algorithm), but the detailed steps and detailed approach differed significantly. Students preferred to continue solving the problem in their own way rather than adopting someone else's method. Moreover, students wanted to construct their own mental model of the problem-solving structure, with the tools serving as aids and guides rather than taking away their control by providing another solution. For example, P5 said, "I want to follow my own idea. The solution's approach is different from mine. I just got stuck and need help specific to my approach. I don't want to adapt my thinking to someone else's solution, even if my solution is not the best one."


% \subsubsection{Challenge 3: Misalignment Between Provided Solutions and Learners' Approaches} Learners often find that the solutions available on LeetCode's solution or editorial panels do not match their intended approach. At a high level, they may struggle to find a solution that aligns with their strategy. Even when the overarching concept is similar, such as both using a binary search algorithm, the specifics of the approach can differ greatly. Students generally prefer to pursue their solution paths, aiming to develop their mental model of the problem-solving process, with tools serving more as aids than prescriptive solutions. For instance, P5 stated, "I prefer to stick to my own ideas. The provided solution takes a different approach, and I just need help tailored to my method. I don't want to adjust my thinking to fit someone else's solution, even if it might be superior."

\ms{
\subsubsection{\textbf{Challenge 2: Misalignment Between Provided Solutions and Learners' Own Problem-Solving Approaches}}
Students often struggled when provided solutions failed to align with their intended approaches or existing code. 
While learners preferred developing their own strategies, the solutions offered by platforms like LeetCode, search engines, or ChatGPT were often prescriptive that mismatched their efforts. 
Even when algorithms matched, differences in specifics made integration challenging, especially with partial or incorrect code. 
Rather than starting over, learners wanted help tailored to their approach and context. As P5 explained, ``\emph{I prefer sticking to my own ideas. The provided solution takes a different approach, and I just need help tailored to my method. I don’t want to change my thinking to fit someone else's solution, even if it's better.}'' Similarly, P8 noted the frustration of aligning external solutions with her own work: ``\emph{I don't want to scrap my code and start anew, so I try to align the solution with my code line by line. It’s tedious to figure out which parts of the solution relate to what I’ve written, where my errors began, and what I missed.}''
}

% \subsubsection{\textbf{Challenge 3: Misalignment Between Provided Solutions and Learners' Approaches}} Students often found that solutions on LeetCode's solution or editorial panels did not align with their intended approach. Even if the overall algorithm used was consistent, the specifics might vary greatly. Learners preferred to follow their own strategies and develop their own problem-solving mental models, using tools as aids rather than prescriptive solutions. As P5 explained, ``\emph{I prefer sticking to my own ideas. The provided solution takes a different approach, and I just need help tailored to my method. I don’t want to change my thinking to fit someone else's solution, even if it's better.}''

% \subsubsection{\textbf{Challenge 4: Disconnect Between Existing Code and Provided Solutions}}

% All 15 participants initially attempted to solve the problem on their own, often producing incomplete or incorrect code. When seeking help—from LeetCode, search engines, Stake Overflow, or ChatGPT—they found the guidance rarely aligned with their existing code. This made it difficult to integrate the newly received assistance with their previous problem-solving efforts. Existing tools overlooked the context of their code, leaving students feeling like they had to start from scratch. Students preferred to build on their initial attempts, but aligning the provided solutions with their code was time-consuming. As P8 noted, ``\emph{I don’t want to scrap my code and start anew, so I try to align the solution with my code line by line. It’s tedious to figure out which parts of the solution relate to what I’ve written, where my errors began, and what I missed.}''






% \subsubsection{Challenge 4: Lack of a Close Connection Between Learners' Existing Code and the Solution}

% From our observations, regardless of their progress in solving the problem, all 15 students typically chose to attempt solving it on their own first. They would try writing some code in the editor, even though this code was often incomplete or incorrect. When they encountered difficulties and sought help—whether from LeetCode's built-in solution, search engine results, or ChatGPT—the solutions provided were disconnected from the students' existing attempts. The students struggled to clearly relate these solutions to the code they had already written. Even when students pasted their incomplete code into a ChatGPT conversation, the responses often failed to maintain this connection. As a result, students felt as though they were starting over in a new editor, rather than continuing to build upon their existing work. They preferred to return to their editor and improve their unfinished code, but this required significant effort to painstakingly match the provided solution with their existing code. For instance, P8 said: "I don't want to delete my code and start over, so I have to compare the solution line by line with my code. I need to figure out which part of the feedback or solution corresponds to the code I’ve already written, which parts I got right, where I started to go wrong, and what I missed in my previous attempt. It’s extremely time-consuming."


% \subsubsection{Challenge 4: Disconnect Between Existing Code and Provided Solutions}

% Our study revealed that all 15 participants initially attempted to solve the assigned problem by themselves, writing some code that was often incomplete or incorrect. When they sought help—whether from LeetCode’s solutions, search engine queries, or ChatGPT—the guidance they received generally did not connect well with their existing code. This made it difficult for students to integrate new solutions with their prior efforts effectively. Responses, particularly from ChatGPT, often did not account for the context of the students' existing code, making it feel as though they were starting afresh rather than building on what they had already written. Students expressed a preference to continue refining their initial attempts, but aligning the provided solutions with their code proved to be laborious and time-consuming. For example, P8 commented: "I don’t want to scrap my code and start anew, so I try to align the solution with my code line by line. It’s challenging to discern which parts of the solution or feedback relate to what I’ve written, identify my correct segments, detect where errors began, and understand what I missed. This process is incredibly tedious."





% \subsubsection{Challenge 5: Lack of a Structured Problem-Solving Process}

% Students expressed a need for a structured approach to problem-solving, one that would allow them to break down the problem from a broad overview into finer details. When tackling more complex algorithmic problems, students often split the problem-solving process into several manageable steps. These might include handling initialization and edge cases, managing loops within arrays, and processing return values. This reflects a key aspect of computational thinking [ref]. However, the assistance provided by current tools is usually presented as a large block of text or code. Even well-organized solutions that use bullet points to outline the steps often get buried in long paragraphs, lacking a clear, structured visual presentation (see Appendix A). For example, P13 said: "When I read through the solution, it tells me there are six steps, but the specific reasoning and code that follow are not organized according to these steps. It just informs me upfront that there are six steps. The goals, actions, and corresponding code for each step are all mixed together. It’s challenging to build a clear, structured thought process from such a lengthy solution." Furthermore, six students found that while using ChatGPT, they did not simply throw the entire problem at GPT but instead engaged in a step-by-step interaction, asking specific questions as they encountered difficulties. However, they discovered that resolving a single issue often required multiple rounds of conversation, which were then followed by discussions of another challenge. This made it easy for them to lose track within the lengthy conversation. P5, for instance, complained: "I tried asking GPT some questions, and it provided good answers, but after asking several questions, I realized I had engaged in pages of conversation. When I tried to tackle a specific problem, I had to scroll back through the conversation to find what I had discussed with GPT. After asking GPT for help, I still felt confused. I seemed to have received answers to each of my questions, but I couldn’t organize them into a coherent thought process."



\subsubsection{\textbf{Challenge 3: Lack of a Structured Problem-Solving Approach}}

Students highlighted the need for a structured method to break down complex algorithmic problems into manageable steps, such as initialization, handling edge cases, loops, and return values. Current tools often present solutions as large blocks of text or code without clear visual structure. As P13 noted when using LeetCode’s solution panel, ``\emph{The solution outlines six steps, but the content is very disorganized and the text is unclear, making it hard to follow a structured process.}'' Some participants preferred step-by-step interactions with ChatGPT, asking specific questions as they progressed. However, resolving issues often required lengthy, multiple rounds of conversations, making it difficult to stay organized. Learners frequently had to scroll back and forth to find points related to specific steps. As P15 (using ChatGPT) shared, ``\emph{I had a detailed conversation with GPT spanning several pages. Scrolling back to revisit earlier points was cumbersome, and despite getting answers, it was hard to organize them into a coherent thought process}.''




% \subsubsection{Challenge 5: Lack of a Structured Problem-Solving Approach}

% Students highlighted the need for a more structured approach to problem-solving that would help them decompose complex algorithmic challenges into manageable steps. Such steps often include handling initialization and edge cases, managing loops within arrays, and processing return values, reflecting essential elements of computational thinking. However, current tools typically present assistance as large blocks of text or code, with well-intended solutions embedded in lengthy paragraphs lacking clear, structured visual guidance. For example, P13 noted, "The solution outlines six steps, but it fails to organize the reasoning and code within these steps effectively; it all blends together, making it hard to follow a structured thought process."

% Furthermore, interactions with ChatGPT revealed that students preferred a step-by-step engagement, asking specific questions as they progressed through problems. However, resolving issues often required multiple rounds of conversation, leading to lengthy discussions that made it easy to lose track. P5 expressed frustration: "I engaged in a detailed conversation with GPT that spanned several pages. Trying to address specific issues became cumbersome as I had to scroll back to revisit earlier parts of the conversation. Despite receiving answers to my questions, organizing them into a coherent thought process remained challenging."


% \subsubsection{Challenge 6: Lack of Fine-Grained Evaluation of Learners' Thoughts}

% Currently, the run code feature on LeetCode executes the entire code and only indicates success when the student's code is entirely correct. Regardless of how close the student is to the correct solution or how many errors remain, any deviation from correctness results in the same "wrong" feedback. When students encounter a problem and seek help, they want to immediately verify whether they understand the solution. However, the current code verification method cannot provide fine-grained feedback on whether specific steps are correct, preventing students from breaking down the problem into smaller cycles of attempting, evaluating, receiving feedback, and revising. Additionally, students cannot track their progress in implementing their thought process. For instance, P8 mentioned: "I want to know whether this particular step is correct. Only if this step is correct will I know how to proceed. If this step is wrong, I might need to start over. But I can’t get any feedback on this step, so I have to assume it’s correct and keep going, only to realize at the end that this step was actually wrong. I should have been told earlier." P9 added: "This problem is challenging. I need to manage every step and scenario correctly, but I can’t tell how many subcases I’ve handled correctly since there’s no progress indicator."

% Moreover, students want to know whether their thought process is correct. Before writing code, students usually have an initial idea of how to solve the problem. Ten students expressed a desire for a feature that could help them assess whether their approach is correct, which would boost their confidence in writing code or prompt them to adjust their approach sooner. For example, P1 noted: "A correct thought process is the foundation of correct code. When I have an uncertain idea, I don’t want to spend time implementing it until I know it’s correct. Only if my approach is correct do I feel that continuing to write code is meaningful. But currently, I can’t find any support to check the correctness of my approach."

\subsubsection{\textbf{Challenge 4: Insufficient Fine-Grained Feedback on Learners' Coding Progress}}

LeetCode's ``run code'' button evaluates the entire solution and only provides feedback on the overall correctness. This all-or-nothing approach overlooks partial correctness and fails to highlight specific errors, making it difficult for students to track their progress or verify their understanding step by step. For example, P8 noted, ``\emph{I need to know if a step is correct before I can decide how to proceed. Without step-by-step feedback, I can only keep doing it until I complete all the steps.}'' Additionally, students wanted early validation of their thought process before committing to code. Ten participants expressed the need for a feature to confirm if their approach was on the right track. As P1 stated, ``\emph{The correctness of my initial thought process is crucial. I hesitate to implement code until I'm confident it's correct. There's no tool to validate my approach early on.}''




% However, the current help resources are primarily presented in large blocks of text or code, without the visual structure that could help users clarify their thoughts. 



% Interactive Engagement with Help Information: Students wanted the ability to interact with the help information. Currently, LeetCode does not support such interactivity, and while search engines and ChatGPT allow for question-and-answer interactions, students desired more ways to actively engage with the help information to better obtain targeted assistance.



% (Personalization) Lack of Trial-and-Error Process: Students wanted to experiment with solving a particular step, but the current evaluation feedback mechanism only runs the entire code at once. Often, students just wanted to know if their approach for a specific step was correct. "I want to know if my approach to this step is right; if it is, I’ll have a good idea of what to do next, but I can’t find any mechanism that gives feedback on the correctness of my steps," one participant said.

% Time-Consuming Description of Issues to GPT: Getting targeted help by describing difficulties to GPT was time-consuming. Students not only needed to describe the problem but also their current thought process, incomplete code, the specific step where they were stuck, and the issues they were encountering. Students felt this was too time-consuming, so they tended to paste the problem description into GPT rather than taking the time to articulate their thoughts and problems clearly.




% Natural Language as a Thought Process: Students needed to first form their thoughts in natural language as it serves as a good medium before the idea fully takes shape.




% Low-Quality Guidance for Some Issues: Some participants encountered programming problems for which the solution panel only provided direct solution code without a step-by-step high-level explanation. This made it difficult for them to understand and hindered their ability to try coding on their own after grasping the thought process.





% \subsubsection{Design Goals}
% Based on the user needs identified in the formative study, we established the following design goals:

% \begin{itemize} \item \textbf{Goal 1: Scaffolding to Encourage Learners' Active Learning and Independent Thinking}:
% Addressing the first and second challanges, our tool should incorporate scaffolding functions aligned with the Zone of Proximal Development (ZPD). The system should not overwhelm students by presenting all answers at once but rather provide just the right amount of support based on the specific challenges they face. The assistance offered by the system should primarily foster independent thinking, providing appropriate guidance while still requiring students to engage in their own problem-solving to arrive at the answers. Our system should encourage students to first form their own ideas, then offer feedback that guides them in refining and improving their thought processes.

% \item \textbf{Goal 2: Personalization to Individual's Solution}:
% Addressing the third challange, our system should respect each student's unique problem-solving approach, rather than imposing a one-size-fits-all solution. The system should understand how students approach the problem, how they decompose the solution into steps and sub-steps, and where they encounter difficulties. While preserving the students' original thought processes as much as possible, the system should provide personalized feedback and guidance tailored to the specific issues each student encounters.

% \item \textbf{Goal 3: Connection and Structured Solution}:
% Addressing the fourth and fifth challanges, our system should create a clear visual mapping between the assistance provided and the student's existing code, allowing students to easily identify how each part of their thought process corresponds to specific problems and feedback. Additionally, the interface design should align with the student's step-by-step problem-solving mental model, offering a structured visual representation that helps students gradually build a complete mental model of the solution. This design will help students maintain a clear overview of the problem-solving process and avoid losing track of their progress.

% \item \textbf{Goal 4: Fine-Grained Evaluation and Feedback}:
% Addressing the sixth challange, our system should be capable of verifying the correctness of a student's thought process, whether it is expressed through code, pseudocode, or even natural language. Furthermore, the system should provide feedback on the accuracy of each individual step, helping students monitor the correctness of their thought process in real time. This will enable students to establish a flexible cycle of attempting, evaluating, receiving feedback, and revising their approach. \end{itemize}




\subsection{Design Goals} \label{designgoal} Based on the challenges identified in our formative study, we established the following design goals for our tool:

\begin{itemize}
    \item \textbf{D1: Scaffolding for Active Learning and Independent Thinking}: To address Challenge 1, our system should provide scaffolding support. Instead of presenting complete solutions, it should offer tailored support that encourages active problem-solving and independent thinking.
    
    \item \textbf{D2: Personalization to Individual Problem-Solving Styles}: In response to Challenge 2, our system should adapt to each student's unique problem-solving approach by analyzing how they break down problems and offering personalized feedback that preserves and enhances learners' own strategies.

    \item \textbf{D3: Connection and Structured Solution Presentation}: To address Challenge 2\&3, our system should visually connect the system-generated guidance to students’ existing codes and solutions, promoting a structured problem-solving approach that aligns with their mental models.

    \item \textbf{D4: Fine-Grained Evaluation and Feedback}: Addressing Challenge 4, our system should evaluate the correctness of students' thought processes—whether conveyed through code, pseudocode, or natural language—and provide detailed, step-by-step feedback to support continuous improvement.
\end{itemize}



% \subsubsection{Design Goals} Based on the challenges identified in our formative study, we have established specific design goals for our tool:

% \begin{itemize} \item \textbf{Goal 1: Scaffolding for Active Learning and Independent Thinking}: In response to the first and second challenges, our system needs to implement scaffolding aligned with the Zone of Proximal Development (ZPD). It aims to avoid overwhelming students by presenting complete solutions upfront, instead providing the right level of support tailored to their specific needs. This approach encourages students to engage actively in problem-solving, forming their own ideas and receiving guidance that helps refine their thought processes.

% \item \textbf{Goal 2: Personalization to Individual Problem-Solving Styles}: Addressing the third challenge, our system needs to respect and adapt to each student's unique problem-solving approach. It will analyze how students break down problems into steps and sub-steps and where they encounter difficulties, offering personalized feedback that preserves and enhances their original strategies.

% \item \textbf{Goal 3: Connection and Structured Solution Presentation}: In response to the fourth and fifth challenges, our system needs to ensure that assistance provided is visually connected to the student’s existing code. This will help students relate specific parts of the solution to their ongoing work, facilitating a structured problem-solving approach that mirrors their mental model and promotes a clear understanding of the entire process.

% \item \textbf{Goal 4: Fine-Grained Evaluation and Feedback}: Addressing the sixth challenge, our system should be able to evaluate the correctness of a student's (incomplete) thought process—whether conveyed through code, pseudocode, or natural language—and provide detailed feedback on each step. This feature will support a continuous cycle of attempting, evaluating, and revising, enabling students to track and improve their approach in real-time. \end{itemize}
\section{Decomposition Box}





% Feature 1: Inferring Users' Thought Process from Existing Code
% Feature 2: Independent Problem Decomposition through Step Trees and Natural Language.
% Feature 3: Step Tree Node Status Evaluation with Preservation of Original Structure
% Feature 4: Progressive Hint for Idea Formation
% Feature 5: Converting the Step Tree into Comments
% Feature 6: Validating Code Implementation against the Step Tree
% Feature 7: Progressive Hint for Idea Implementation


% Challenge 1: Excessive Help Hindering Active Learning
% Challenge 2: Difficulty in Utilizing ChatGPT for Effective Learning
% Challenge 3: Misalignment Between Provided Solutions and Learners’ Approaches
% Challenge 4: Disconnect Between Existing Code and Provided Solutions.
% Challenge 5: Lack of a Structured Problem-Solving Approach
% Challenge 6: Insufficient Fine-Grained Feedback on Learner Progress


% Goal 1: Scaffolding for Active Learning and Independent Thinking
% Goal 2: Personalization to Individual Problem-Solving Styles
% Goal 3: Connection and Structured Solution Presentation
% Goal 4: Fine-Grained Evaluation and Feedback

% \renewcommand{\arraystretch}{1.2}
% \begin{table*}[tp]  

% \centering  
% \fontsize{8}{8}\selectfont  

% \caption{A mapping between the identified challenges from our formative, the design goals, and the features of Decomposition Box}\label{table:mapping}
% \label{intention_themes}
% \begin{tabular}{m{5cm}<{\centering}m{4.5cm}<{\centering}m{4.5cm}<{\centering}}
% \toprule
% \textbf{Challenge}&\textbf{Design Goal}&\textbf{DBox Feature}\\
% \midrule

% Challenge 1: Excessive Help Hindering Active Learning & \multirow{3}*{\shortstack{Goal 1: Scaffolding for Active \\Learning and Independent Thinking}}& \multirow{3}*{\shortstack{Feature 1\&2 (Stage 1), \\Feature 4\&7 (Stage 1\&2)}}\\

% Challenge 2: Difficulty in Utilizing ChatGPT for Effective Learning &  & \\


% \midrule
% Challenge 3: Misalignment Between Provided Solutions and Learners’ Approaches & Goal 2: Personalization to Individual Problem-Solving Styles & Feature 1 (Stage 1), Feature 3 (Stage 1)\\


% \midrule
% Challenge 4: Disconnect Between Existing Code and Provided Solutions & \multirow{3}*{\shortstack{Goal 3: Connection and Structured\\ Solution Presentation}} & \multirow{3}*{\shortstack{Feature 1 (Stage 1), Feature 2 (Stage 1),\\ Feature 5 (Stage 2), Feature 6 (Stage 2)}}\\

% Challenge 5: Lack of a Structured Problem-Solving Approach &  & \\

% \midrule
% Challenge 6: Insufficient Fine-Grained Feedback on Learner Progress & Goal 4: Fine-Grained Evaluation and Feedback & Feature 3 (Stage 1), Feature 6 (Stage 2)\\

% \bottomrule
% \end{tabular}
% \end{table*}



% Based on our design goals, we developed Decomposition Box (DBox) accordingly. 
% % Table \ref{table:mapping} shows the mapping between challenges identified in our formative study, the design goals, and DBox’s features. 
% Figure \ref{fig:interface} shows the interface.

\subsection{Overview} \label{overview}




\begin{figure*}[htbp]
	\centering 
	\includegraphics[width=\linewidth]{figures/UI_new.pdf}
	\caption{The interface of Decomposition Box. The top row displays the full interface in the solution formation stage (the solution implementation stage is similar, but with different status indicators). The middle row demonstrates a learner's solution formation stage showing basic DBox features. The bottom row illustrates a learner's solution implementation stage. An overview of the DBox interface and workflow is provided in Sec. \ref{overview}, and an illustrative example is described in Sec. \ref{illustrative}. To save space, the second row omits the problem description and editor area, and the third row excludes the problem description area.}
	\label{fig:interface}
        \Description{}
\end{figure*}

As shown in Figure \ref{fig:interface} (top row), DBox's interface has three main parts. The Problem Description (Figure \ref{fig:interface}.A)) and Solution Code Editor (Figure \ref{fig:interface}.B) are similar to the LeetCode platform.
The Interactive Step Tree Widget (Figure \ref{fig:interface}.D) enables users to refine their thought process and receive feedback via an interactive step tree. 
Three buttons—``From Editor to Step Tree'', ``Check Match'', and ``Copy to Comments''—connect the editor and step tree. 
The ``Check Step Tree'' button provides feedback on the step tree's status, categorized into five types (Figure \ref{fig:interface}.D). 
Clicking ``Hint'' offers progressive guidance based on learners' existing attempts (Figure \ref{fig:interface}.E). 
Hovering over steps shows buttons for editing the step tree (Figure \ref{fig:interface}.F).
% \begin{itemize}
%     \item Problem Description (Figure \ref{fig:interface} A): Displays the current problem, with a "Problem List" button , similar to traditional platforms.
%     \item Code Editor (Figure \ref{fig:interface} B): Allows users to write and run code. Output and test case results are shown in Figure \ref{fig:interface} C.
%     \item Interactive Step Tree Widget (Figure \ref{fig:interface} D): Users refine their thought process and receive feedback via an interactive step tree. Three buttons—“From Editor to Step Tree”, “Check Match”, and “Copy to Comments”—connect the editor and step tree. The “Check Step Tree” button provides feedback on the step tree's status, categorized into five types (Figure \ref{fig:interface} D). Clicking ``Hint'' offers progressive guidance based on learners' existing attempts (Figure \ref{fig:interface} E). Hovering over steps shows buttons for editing the step tree (Figure \ref{fig:interface} F).
% \end{itemize}
Figure \ref{fig:workflow} shows two key stages in DBox's workflow:
\begin{itemize}
    \item \textbf{Solution Formation}: 
    The step tree starts as an empty box where students can freely add steps in either coding mode (directly writing code) or description mode (building a step tree using natural language). 
They can evaluate their progress with the ``From Editor to Step Tree'' or ``Check Step Tree'' buttons. The tree contains steps and substeps labeled as \emph{Correct}, \emph{Incorrect}, \emph{System Generated}, \emph{Missing}, or \emph{Can be Divided}. 
Layouts adjust dynamically based on the hierarchy. 
Steps that can be further divided are marked with dashed outlines, serving as a reminder, though students can decide whether further division is necessary. 
    \item \textbf{Solution Implementation}: Students can convert the step tree into comments with “Copy to Comments” or verify alignment by clicking “Check Match”. Nodes in the step tree are labeled as Implemented, Incorrectly Implemented, or To Be Coded. In this stage, DBox also offers progressive hints. When all nodes are implemented, students can test their solution by clicking “Run” button against the provided test cases.
\end{itemize}






\begin{figure*}[htbp]
	\centering 
	\includegraphics[width=\linewidth]{figures/pipeline.pdf}
	\caption{The DBox workflow supports learners through solution formation and implementation stages. During solution formation, (A) students can input ideas by either coding or using natural language to build a step tree. (B) By clicking ``From Editor to Step Tree'' or ``Check Step Tree'', (C) DBox renders the step tree and identifies node statuses (e.g., correct, incorrect, missing). Students can iteratively refine their code or step tree, receiving progressive hints, (D) until the step tree is fully correct. In the solution implementation stage, (E) students can convert the step tree into code comments or (F) check the alignment between their code and the step tree. Each node displays one of three statuses, and students can refine their work with ongoing hints until (G) all nodes are marked as ``implemented''.
 Finally, students can test if their code passes all test cases.}
	\label{fig:workflow}
        \Description{}
\end{figure*}



% Based on our design goals, we developed Decomposition Box (DBox). Table \ref{table:mapping} shows a mapping between the identified challenges from our formative study, the design goals, and the features of DBox. Figure \ref{fig:interface} shows the interface. Below, we provide an overview of the interface, followed by a detailed explanation of the key features and backend design.

% \subsection{Overview}

% As shown in Figure \ref{fig:interface} (top row), the interface of DBox is strategically divided into three distinct sections, from left to right, designed to enhance the user experience in algorithm programming:

% \begin{itemize}
%     \item Left Section - Problem Description (Figure \ref{fig:interface} (A)): This area allows users to view the problem they are currently working on. By clicking the ``Problem List'' button, users can easily switch between different problems they wish to practice. This setup is similar to traditional programming exercise platforms, providing a familiar layout for users.
%     \item Middle Section - Code Editor (Figure \ref{fig:interface} (B)): Central to the user interface, this section features an integrated CodeMirror editor where users can write and edit code in their chosen programming language. The ``Run'' button allows users to execute their code, with the output and any error messages (such as failed test cases or syntax errors) displayed (Figure \ref{fig:interface} C). If the code passes all test cases, it is marked as accepted, providing immediate feedback on the correctness of the solution.
%     \item Right Section - Decomposition Box (Figure \ref{fig:interface} (D)): This core area is where users can refine their thought processes and receive specific feedback and reminders related to their coding approach. Positioned between the code editor and this section are three interactive buttons that facilitate seamless communication between the editor and the Decomposition Box, enhancing the integration of feedback into the coding process.
% \end{itemize}

% As shown in Figure \ref{fig:workflow}, DBox is designed to support two key stages in students' algorithmic programming practice: \textbf{Idea Formation} and \textbf{Idea Implementation}. In the Idea Formation stage, students can input their thoughts in two modes (Figure \ref{fig:workflow} (A)): one is the coding mode, where they directly write code, and the other is the description mode, where they build a step tree interactively using natural language to describe each node. Based on the selected input mode, students can click different buttons to check the status of their step tree (Figure \ref{fig:workflow} (B)). The ``From Editor to Step Tree'' button converts incomplete code into a step tree, while the ``Check Step Tree'' button evaluates the status of each node in the tree. The step tree can include steps, sub-steps, and even sub-sub-steps, with DBox identifying the status of each node, which could be one of five types: Correct, Incorrect, AI Generated, Missing, or Divisible (Figure \ref{fig:workflow} (C)). Students can refine their code or step tree and click the corresponding buttons to check the updated status of the tree. Throughout this process, DBox provides progressive hints, divided into three levels. After several refinements, the step tree becomes fully correct (Figure \ref{fig:workflow} (D)), allowing students to proceed to the Idea Implementation stage.

% At this stage, students can use the ``Check Match'' button to verify whether the code in the editor aligns with the nodes in the step tree, or they can click the ``Copy to Comments'' button to convert the step tree into comments in the editor to assist with coding (Figure \ref{fig:workflow} (E)). After clicking ``Check Match,'' the nodes in the step tree will be labeled with one of three statuses: Implemented, Incorrectly Implemented, or To Be Coded (Figure \ref{fig:workflow} (F)). DBox continues to offer progressive hints during this phase as well. After modifying the code based on feedback, the step tree will eventually display all nodes as implemented (Figure \ref{fig:workflow} (G)). At this point, the student can click the ``Run'' button to validate the solution against the test cases.





% The interface is crafted to be easily integrated as a plugin into existing programming tools and platforms, such as online environments like LeetCode or offline editors like VSCode, offering flexibility and adaptability across different learning and development settings.






% \subsection{Overview of DBox Interface}
% A learner named Alice wants to practice her algorithmic programming using a problem ``Search in Rotated Sorted Array". After reading the problem description, Alice comes up with some initial thoughts in her mind. DBox provides two input modes, one is through directly writing codes, the other is building a step tree via natural languege description at the right-side of the interface. (1) Alice chooses to add two steps, Step 1 and Step 2, to construct an initial step tree. (2) After clicking the ``Check Step Tree'' button, DBox identifies Step 1 as correct and Step 2 as incorrect, also indicating a missing Step 3. Alice then clicks the hint button on Step 2 to access general and detailed hints. (3) If Alice makes two consecutive errors at Step 2, a new hint triggers: ``reveal step.'' (4) Clicking this button reveals a crucial sub-step (Step2-3) and leaves Step 2-1 and Step 2-2 for Alice to complete. (5) Once Alice completes these, she clicks ``Check Step Tree'' again and finds all of Step 2 correct. (6) Alice also correctly constructs Step 3. \textbf{The third row displays Alice's process of implementing the solution}. (8) First, Alice clicks the ``Copy to Comments'' button, and DBox converts the step tree into code comments, inserting them at the corresponding positions in the editor. (9) After writing some code, Alice uses the ``Check Match'' button to identify steps that are not correctly implemented, noting that Step 2-3 and Step 3 are incomplete. (10) Guided by DBox, Alice writes the corresponding code. Upon clicking ``Check Match'' again, all steps turn green to indicate they are implemented correctly. When hovering over a step, the corresponding code line is highlighted. Finally, Alice hits the Run button, passes all test cases, and successfully solves the problem. It's important to note that this figure only shows just one of many possible interactions. To save space, the second row of images displays only the step tree on the right side of the interface, while the third row shows both the middle editor and the step tree.






% \subsection{Overview of DBox Interface}
% Alice, a learner, is practicing algorithmic programming with the problem ``Search in Rotated Sorted Array.'' After reading the problem description, she formulates some initial ideas in her mind. DBox offers two input modes: directly writing code or constructing a step tree using natural language descriptions on the right side of the interface. (1) Alice opts to build a step tree and adds two initial steps, Step 1 and Step 2. (2) Upon clicking the ``Check Step Tree'' button, DBox identifies Step 1 as correct, flags Step 2 as incorrect, and highlights a missing Step 3. Alice clicks the hint button on Step 2 to access both general and detailed hints. (3) If she makes two consecutive errors in Step 2, DBox triggers an additional hint: ``reveal step.'' (4) Clicking this reveals a crucial sub-step (Step 2-3 in this case), while leaving Step 2-1 and Step 2-2 for Alice to complete. (5) After completing these sub-steps, (6) Alice checks the step tree again and finds all of Step 2 marked as correct. (7) She then constructs Step 3 successfully. Now, all nodes in the step tree are correct. (8) Alice then clicks the ``Copy to Comments'' button, and DBox converts the step tree into code comments, automatically inserting them into the editor. (9) After writing some code, she uses the ``Check Match'' button, which highlights steps that are not properly implemented, indicating that Step 2-3 and Step 3 are incomplete. (10) Following DBox's guidance, Alice writes the necessary code, and upon rechecking, all steps turn green, indicating correct implementation. Note that when Alice hovers over a step, the corresponding line of code is highlighted. Finally, she clicks the Run button, passes all test cases, and successfully solves the problem.



\subsection{Target Users and A System Walkthrough} \label{illustrative}
\ms{DBox is designed for learners who understand basic algorithm concepts but struggle to apply them to solve practical problems. Using a scaffolding approach, DBox emphasizes independent thinking by offering only essential support. It assumes students are motivated, self-regulated, and actively engaging with the tool to improve their decomposition skills. If a student is less motivated or prefers a quicker solution, they may bypass DBox to search for answers online.}
Next, we present an example walkthrough (Figure \ref{fig:interface}) of such a self-regulated student Alice: 
% usage example from our pilot study, as shown in Figure \ref{fig:interface}. While this illustrates one specific interaction, users can follow various workflows based on their preferences. Below is a detailed walkthrough of the scenario.

Alice, a learner tackling the ``Search in Rotated Sorted Array'' problem, begins by organizing her thoughts in the solution formation stage. DBox offers two options: she can either start coding or build a step tree using natural language. She opts for the latter and adds two initial steps (Figure \ref{fig:interface}.1). To check her progress, Alice clicks ``Check Step Tree'' button. DBox flags Step 1 as correct, Step 2 as incorrect, and highlights a missing Step 3 (Figure \ref{fig:interface}.2). She clicks the hint button on Step 2, receiving general and detailed guidance, but after another failed attempt, DBox offers another option for revealing a substep (Figure \ref{fig:interface}.3). Alice clicks ``Reveal (Sub)Step'', uncovering a sub-step 2-3 while leaving sub-steps 2-1 and 2-2 for her to solve (Figure \ref{fig:interface}.4). Inspired by the hints, Alice figures out how to break down and fills in these sub-steps (Figure \ref{fig:interface}.5). After checking again, Step 2 is marked correct (Figure \ref{fig:interface}.6). Alice adds the missing Step 3 (Figure \ref{fig:interface}.7), and finally, after checking, all steps turn to correct (Figure \ref{fig:interface}.8).

Next, Alice moves to the solution implementation stage. She clicks ``Copy to Comments'', and DBox converts her step tree into code comments (Figure \ref{fig:interface}.9). As Alice writes her code, she uses the ``Check Match'' button to identify incorrectly implemented and unimplemented steps. Step 2 is identified as unimplemented and Step 3 is identified as incorrectly implemented (Figure \ref{fig:interface}.10). Following DBox's guidance, she revises the code, and after another check, all steps turn to be correctly implemented (Figure \ref{fig:interface}.11). Satisfied with her progress, Alice clicks ``Run'' and successfully passes all test cases, solving the problem.

Note that we have presented only a simple walkthrough here, whereas the steps in a student's actual problem-solving process are more complex and dynamic (as shown later in Sec. \ref{actual_use}). Next, we introduce the specific features aligned with the four design goals as described in Sec. \ref{designgoal}.
% \textbf{D1}: Scaffolding for Active Learning and Independent Thinking; \textbf{D2}: Personalization to Individual Problem-Solving Styles; \textbf{D3}: Connection and Structured Solution Presentation; and \textbf{D4}: Fine-Grained Evaluation and Feedback. Each feature is tailored to different stages of the user's algorithmic programming journey.




% \subsection{An Illustrative Example} \label{illustrative}
% We now present a usage example observed during our user study, as shown in Figure \ref{fig:workflow}. While this illustrates one specific interaction, users can follow various workflows depending on their preferences. To conserve space, the second row in Figure \ref{fig:workflow} omits the problem description and editor, while the third row excludes the problem description. Below, we provide a detailed walkthrough of the scenario depicted.

% Alice, a algorithmic programming learner, is tackling the ``Search in Rotated Sorted Array'' problem. After reading the description, she starts to map out her approach. DBox offers two ways to proceed: Alice can either dive straight into coding or take a more structured route by building a step tree through natural language descriptions. She chooses the latter, organizing her thoughts by adding two initial steps—Step 1 and Step 2—to the step tree (Figure \ref{fig:workflow} (1)). Eager to check her progress, Alice clicks the ``Check Step Tree'' button. DBox instantly provides feedback: Step 1 is correct, but Step 2 is flagged as incorrect, with a missing Step 3 also highlighted (Figure \ref{fig:workflow} (2)). Alice clicks on the hint button for Step 2, receiving both general and detailed guidance. But she still experiences one more failed attempt on Step 2. At this time, DBox suggests a new hint strategy: revealing part of the solution (Figure \ref{fig:workflow} (3)). She clicks ``reveal step,'' uncovering a crucial sub-step (Step 2-3), while leaving her to figure out sub-steps 2-1 and 2-2 on her own (Figure \ref{fig:workflow} (4)). Alice continues working through the sub-steps (Figure \ref{fig:workflow} (5)). Finally, Step 2 is fully correct (Figure \ref{fig:workflow} (6)), and she successfully adds the missing Step 3. Now, the entire step tree is complete (Figure \ref{fig:workflow} (7)).

% With the structure in place, Alice moves on to the coding phase. She clicks the ``Copy to Comments'' button, and DBox seamlessly transforms her step tree into code comments, which are automatically inserted into the editor (Figure \ref{fig:workflow} (8)). She starts writing her code, and as she works, the ``Check Match'' button becomes her guide, highlighting which steps are incorrectly implemented. It's clear that Step 2-3 and Step 3 need further attention ("to be coded") (Figure \ref{fig:workflow} (9)). Following DBox's prompts, Alice revises the code, and after another check, all steps turn green, signaling success. As she hovers over each step in the tree, the corresponding line of code is highlighted, helping her stay aligned. Satisfied with her progress, Alice hits the Run button. Her code passes all test cases, and she successfully solves the problem (Figure \ref{fig:workflow} (10)). 



% Next, we will explore the specific features tailored to each stage of the user's journey in algorithmic programming.





% \subsection{Stage 1: Idea Formation}

% \subsubsection{Feature 1: Inferring Users' Thought Process from Existing Code.}
% When users click the “From Editor to Step Tree” button, DBox analyzes the current problem and the user’s incomplete code to infer their thought process, which is then displayed as a step tree on the right side. It’s important to note that this feature uses the user’s existing code as the primary input and does not take into account any pre-existing step tree, which will be overwritten. Once the step tree is generated, hovering over any step node highlights the corresponding code lines in the editor, helping users easily connect the step tree with their existing code. This feature is particularly useful when users have written some code and are stuck or when they want to check for errors in their existing code.

\subsection{Stage 1: Solution Formation}
\subsubsection{Two Input Modes (D1, D2, D3)}
DBox offers users the flexibility to develop their solutions through two distinct input modes: by writing code directly or by constructing a step tree using natural language descriptions, without needing to start with code. In the latter mode, users begin with a blank step tree and can click ``Add'' to insert nodes or ``Split'' to create sub-steps for more granular detail. Each node contains a text input field where users can articulate their thought process. Steps and sub-steps can be rearranged or deleted, allowing learners to iteratively and interactively refine and structure their mental model.

\subsubsection{Inferring Users' Thought Process from Existing Code (D1, D3)}
The ``From Editor to Step Tree'' function in DBox infers a learner’s intended solution and thought process based on their incomplete code. When activated, the system analyzes the code and problem, presenting the inferred steps as a tree on the right-hand side of the interface. Hovering over each node highlights the corresponding lines in the code editor, linking the inferred steps directly to the code. This feature assists users in diagnosing errors and identifying potential issues, especially when they are unsure how to proceed.


% \subsubsection{Feature 1: Inferring Users' Thought Process from Existing Code} The ``From Editor to Step Tree'' button in DBox leverages the user’s existing, incomplete code to infer their thought process. When activated, the system analyzes the problem and code, displaying the inferred steps as a tree on the right side of the interface. It's important to note that this feature prioritizes the current code over any pre-existing step tree, which will be overwritten. As the step tree populates, hovering over any node will highlight the corresponding lines in the code editor, linking conceptual steps directly to the code. This feature is invaluable for users who are stuck or wish to identify errors in their existing code, enhancing their ability to diagnose and resolve coding issues efficiently.



% \subsubsection{Feature 2: Decomposing A Problem via Step Tree via Natural Language Description.}
% In addition to generating a step tree from the user’s existing code, we allow users to develop their thought process directly through natural language descriptions, without needing to write code initially. We use an interactive visual step tree to help users organize their problem-solving ideas. Initially, the step tree area is blank. Users can click the Add button to create a node representing a step. They can add as many steps as needed, dividing the space of the step tree area. For each step node, users can click the Split button to add sub-steps. Each step and sub-step includes a blank text input area where users can describe their thought process in simple natural language. Users can also delete or rearrange any step or sub-step as they wish. Such a step tree is useful for helping learners build a structured mental model of the problem-solving strategy.


% \subsubsection{Feature 2: Independent Problem Decomposition through Step Trees and Natural Language}

% Our tool enhances problem-solving by allowing users to independently construct a step tree using natural language descriptions, without initially requiring code. The step tree area starts blank, and users can click the ``Add'' button to insert nodes representing individual steps, organizing their thought process visually. For more detailed breakdowns, the ``Split'' button enables users to add sub-steps under each main step. Each node in the step tree features a blank text input area where users can articulate their thought process in straightforward natural language. Users have the flexibility to delete or rearrange steps and sub-steps as needed, facilitating the development of a structured mental model for tackling complex problems. This method supports learners in systematically building and refining their problem-solving strategies.


% \subsubsection{Feature 2: Independent Problem Decomposition through Step Trees and Natural Language}

% DBox allows users to independently construct a step tree using natural language descriptions, without requiring initial code. Starting with a blank step tree, users can click ``Add'' to insert nodes representing steps, and ``Split'' to add sub-steps for more detailed breakdowns. Each node includes a text input area for users to articulate their thought process. Steps and sub-steps can be deleted or rearranged, helping users build a structured mental model. This feature supports learners in interactively developing and refining their problem-solving strategies.





% \subsubsection{Feature 2: Step Tree Node Status Evaluation.}
% Our tool provides a fine-grained evaluation of the user's thought process. For each step or sub-step, the system categorizes it into one of the following states: (1) Correct, indicating that the step is appropriate for the current problem-solving approach; (2) Incorrect, indicating an error in the thought process or specific details; (3) Missing, indicating a step that is necessary for the complete solution but is absent from the user’s incomplete code; (4) Can Be Further Divided, indicating that the step is complex and can be broken down into sub-steps; and (5) AI Suggested, where the system offers a description of the step in a blue box if the user triggers the most detailed hint level. Users have complete freedom to decide whether to further divide steps, ensuring flexibility in their problem-solving approach. 

% We’ve made a special design choice here: except when the system identifies missing steps and adds a blank missing node, the structure and content of the step tree remain as constructed by the user. The system does not override the user’s current step tree with what GPT considers to be the correct steps and descriptions. Instead, we provide feedback and guidance on the status of each node in the step tree, ensuring that users can continue to advance along their problem-solving approach.

% \subsubsection{Feature 3: Step Tree Node Status Evaluation with Preservation of Original Structure} Once the learner clicks ``From Editor to Step Tree'' or ``Check Step Tree'' button, DBox will conduct a detailed evaluation of each node in the user's step tree, categorizing each step or sub-step into one of five states: (1) \textbf{Correct}: The step is suitable for the current problem-solving approach. (2) \textbf{Incorrect}: There is an error in the thought process or specific details. (3) \textbf{Missing}: A necessary step is absent from the user's code. (4) \textbf{Can Be Further Divided}: The step is complex and could be broken into smaller, more manageable sub-steps. (5) \textbf{System Generated}: The system provides a step description in a blue box if the user activates the most detailed hint level. Users retain complete control over whether to sub-divide steps, allowing them to tailor their problem-solving approach flexibly. Importantly, except for adding a blank node when steps are identified as missing, the system preserves the user’s original step tree without replacing it with AI-determined correct steps. Feedback and guidance are provided on the status of each node, supporting users as they refine and advance their problem-solving strategies.


\subsubsection{Step Tree Node Status Evaluation with Preservation of Original Structure (D2, D4)}
When the learner clicks ``From Editor to Step Tree'' or ``Check Step Tree'', DBox evaluates each node, assigning one of five statuses:
(1) \textbf{Correct}: The step aligns with the learner's intended approach.
(2) \textbf{Incorrect}: Errors are identified in the step.
(3) \textbf{Missing}: A required step is absent.
(4) \textbf{Can Be Divided}: The step is complex and can be broken into sub-steps, indicated by dashed borders. Users decide whether to subdivide. This status can coexist with other statuses.
(5) \textbf{System Generated}: Step content is created by the system. This status is triggered only when the learner requests to reveal a (sub)step after repeated failures.
During the ``Check Step Tree'' process, DBox preserves the original step tree (both structure and contents), only adding blank nodes for missing steps, ensuring scaffolding while respecting the learner's thought process.





% \subsubsection{Feature 4: Progressive Hint.}
% We’ve designed a detailed scaffolding process that provides only the necessary guidance, encouraging users to think independently before offering more specific hints as needed. The first level is a hint presented as a question. When users know a step is incorrect or missing but still lack ideas, they can click the C button to receive initial guidance. This guidance does not directly reveal the answer but steers users in the right direction, such as: “Before you convert the string variable to an array, what should you do first?” The second level offers more specific guidance, including more concrete clues, if the user still struggles after the first hint. The third level of guidance is triggered if a user repeatedly fails to correct their thought process for a specific step. This level provides an option to view the AI-suggested correct steps. Although this third level is triggered, it is not displayed by default; users must click the View button to see it. The third-level feedback can appear in two scenarios: (1) If the step has no sub-steps, the tool directly presents the correct description of the step; (2) If the step includes sub-steps, the tool highlights one key sub-step and marks other sub-steps as missing, reminding the user to complete the remaining sub-steps. It is important to note that users can choose not to view any of these three levels of feedback to solve the problem independently.


% We also provide multi-level guidance for users who incorrectly implement or have not yet implemented their thought process, aiming to help them complete their code independently as much as possible. The first level is a simple hint that offers basic guidance, such as: “How should you correctly update variable A?” or “Consider updating the index before the loop ends.” The second level is a pseudocode hint, offering a more specific guide if the user is still unsure after viewing the basic hint. At this point, the user only needs to convert the pseudocode into actual code. The third level of guidance is triggered if the user incorrectly implements a step twice in a row. This level provides an option to view the correct code for that step. Although the third level is triggered, it is not displayed by default; users must click the View button to see it. It’s important to note that users can choose not to view any of these three levels of feedback to independently implement their ideas.

% \subsubsection{Feature 4: Progressive Hint for Idea Formation}

% We have implemented a scaffolding feature that progressively delivers guidance, fostering independent problem-solving while providing specific hints as needed. This feature is structured into three levels. (1) \textbf{Initial Hint (Question-Based)}: This is the first level of assistance where users receive a hint framed as a question, prompting them to think about the next step without giving away the solution. For instance, if a user is uncertain about the initial steps, they might see a hint like, “Before you convert the string variable to an array, what should you do first?” (2) \textbf{Detailed Guidance}: If the user continues to struggle, a second, more specific hint is provided. This could include more direct clues that guide the user closer to the solution but still require them to apply their reasoning. (3) \textbf{Recommendation for A (Sub)Step}: Triggered by repeated difficulties in correcting a specific step, this level offers the option to view the AI-suggested correct steps. For steps without sub-steps, it directly presents the correct description. For steps that include sub-steps, it highlights one key sub-step and marks the remaining as missing, prompting users to complete them. This level is optional and can be accessed by clicking a ``Reveal Step'' button, allowing users to decide whether to see the full solution or continue working independently. These layers of hints are designed to support users in building their problem-solving skills gradually while enabling them to maintain control over their learning process.

\subsubsection{Progressive Hints for Solution Formation (D1)}

DBox provides progressive hints to scaffold learners' problem-solving in three levels: (1) \textbf{General Hint} (Question-Based): Prompts learners' critical thinking without revealing solutions, e.g., ``Before converting the string to an array, what should you do first?'' (2) \textbf{Detailed Hint}: Offers more specific clues while requiring reasoning, e.g., ``Think about how you can traverse each character in the string.'' (3) \textbf{Reveal (Sub)Step} ((Sub)Step Recommendation): For repeated errors, the AI can suggest a substep within a larger step when users click the ``Reveal (Sub)Step'' button. This reveals one key substep while leaving the remaining steps for the learner to complete. Notably, students can choose not to trigger this hint. These progressive hints support problem-solving development while allowing learners to maintain independence and control.

Once the step tree is complete and all nodes are correct, learners proceed to the solution implementation stage.


% \subsection{Stage 2: Idea Implementation} After constructing a complete and correct step tree, users advance to the idea implementation stage.

% \subsubsection{Feature 5: Converting the Step Tree into Comments} This feature transforms each node of the step tree into code comments. By clicking ``Copy to Comments", these comments are pasted directly into the corresponding section of the editor, guiding students to systematically complete their code implementation.

% \subsubsection{Feature 6: Validating Code Implementation against the Step Tree} The tool evaluates the alignment between the user’s code and the problem-solving approach defined in the step tree. Using the ``Check Match'' button, the step tree's status updates to reflect: (1) \textbf{Implemented}: The code accurately implements the described step. (2) \textbf{Incorrectly Implemented}: There is an error in how the step is coded.
% (3) \textbf{To be Coded}: The step has yet to be coded. Highlighting in the editor indicates the correlation of code lines with the step tree, showing whether each step is correctly or incorrectly matched.

% \subsubsection{Feature 7: Progressive Hint for Idea Implementation.} Additionally, for users who have incorrectly implemented or have yet to implement their thought process, multi-level guidance is available. (1) \textbf{Basic Hint}: Offers straightforward suggestions to nudge the user in the right direction, like “How should you correctly update variable A?” or “Consider updating the index before the loop ends.” (2) \textbf{Pseudocode Hint}: If the user remains unsure after the basic hint, this level provides a pseudocode hint, clarifying what needs to be done, which the user then translates into actual code. (3) \textbf{Recommended Implementation}: This is activated if a user misimplements a step twice consecutively. It allows them to view the recommended code for that step by clicking a ``View'' button, although, like the previous levels, viewing is optional.

% Note that DBox does not teach specific algorithm concepts or knowledge, as our target users are learners who have a basic understanding of algorithms but want to improve their ability to apply algorithms to solve real-world problems.


\subsection{Stage 2: Solution Implementation}

\subsubsection{Converting the Step Tree into Comments (D3)} This feature converts each node of the step tree into code comments. When students click ``Copy to Comments'', the system intelligently inserts these comments into the appropriate sections of the code editor. This guides learners to implement their solutions within the corresponding parts of their code, ensuring a smooth transition from planning to coding while reinforcing their structured approach.

\subsubsection{Validating Code Implementation against the Step Tree (D3, D4)} The ``Check Match'' button evaluates the alignment between the code and the step tree. Steps are categorized and color-coded as: (1) \textbf{Implemented}, (2) \textbf{Incorrectly Implemented}, and (3) \textbf{To Be Coded}. Hovering over a step highlights the corresponding lines in the code, providing a direct mapping between the step tree and the code to help users efficiently debug their implementation.


\begin{figure*}[htbp]
	\centering 
	\includegraphics[width=\linewidth]{figures/processing_new.pdf}
	\caption{An illustration of DBox's data processing workflow highlights its core function—creating a step tree with node statuses from student inputs. The LLM processes learners' incomplete code or a step tree they’ve constructed. It outputs a structured JSON object containing steps, sub-steps (and sub-sub-steps, etc.), each with several attributes. Then the JSON object is rendered to the interface, preserving the original structure and only adding nodes for any missing steps. Each node keeps the student's original input, without directly revealing the correct solution. DBox encodes the status of each step with colors and provides progressive hints.}
 
	\label{fig:processing}
        \Description{}
\end{figure*}

\subsubsection{Progressive Hints for Solution Implementation (D1)}
For steps that are incorrectly implemented or yet to be coded, multi-level hints are available: (1) \textbf{General Hint}: Shows simple thought-provoking prompts/suggestions, e.g., ``How should you correctly iterate until the second last character?'' (2) \textbf{Detailed Hint} (Pseudocode): Provides simplified pseudocode to guide the user. (3) \textbf{Reveal Code} (Recommended Implementation): This option is activated only after two failed attempts. Clicking the ``Reveal Code'' button displays the recommended code implementation for the specific step.




\subsection{Backend Design}
DBox's backend is primarily powered by Large Language Models (the GPT-4o model specifically), with four distinct interactions corresponding to four buttons in the interface:
\begin{itemize}
    \item \textbf{From Editor to Step Tree}: This button sends the problem description and the user’s code to the LLM, which generates a step tree with nodes labeled as correct, incorrect, missing, or divisible.
    \item \textbf{Check Step Tree}: Clicking this button inputs the problem description and the user-constructed step tree into the LLM, which returns a labeled step tree with node statuses such as correct, incorrect, missing, or divisible.
    \item \textbf{Copy to Comments}: This button sends the problem description, current step tree, and user’s code to the LLM, generating a mapping of step tree nodes to corresponding lines of code.
    \item \textbf{Check Match}: Pressing this button sends the problem description, step tree, and user’s code to the LLM, which outputs a tree categorizing nodes as implemented, incorrectly implemented, or to be coded.
\end{itemize}

% To further illustrate, we present the data processing workflow behind two core functions: ``From Editor to Step Tree'' and ``Check Step Tree.'' Figure \ref{fig:processing} shows the workflow, where DBox processes student inputs in two modes: code or a constructed step tree. The LLM adapts based on the input. If only code is provided, it generates steps and substeps for the step tree, evaluates node status, and provides hints. If the student supplies a step tree, the LLM assesses the status of each node without altering the structure, only adding steps where needed. The LLM outputs a structured JSON format that organizes steps, substeps, and subdivisions. Each node includes the student’s input, its status (correctness, completeness, potential for subdivision), a general hint, a detailed hint, and the correct content. The step tree is \emph{selectively} displayed on the interface, primarily \emph{retaining the originally student-created structure} while highlighting missing steps. Node statuses are encoded in colors, with corresponding hints accessible through designated buttons.

To illustrate, we present the data processing workflow for two core functions: ``From Editor to Step Tree'' and ``Check Step Tree''. Figure \ref{fig:processing} shows how DBox processes student inputs in two modes: coding mode and language description mode (step tree input). Prompts adapt based on the input type. When only code is provided (\textcolor{darkblue}{dark blue} lines in Figure \ref{fig:processing}), the system first calls \colorbox{darkblue}{\textcolor{white}{[prompt from code]}} to generate a step tree, mapping each code line to a corresponding (sub)step based on the code’s meaning. It then uses the generated step tree to invoke \colorbox{orange}{\textcolor{white}{[prompt from step tree]}} (\textcolor{orange}{orange} lines) to evaluate node statuses, add missing nodes, and generate multi-level hints for incorrect or missing nodes. If the input is a natural language step tree, the LLM directly calls \colorbox{orange}{\textcolor{white}{[prompt from step tree]}} while preserving the original structure. 

The LLM outputs a JSON containing steps, sub-steps, and subdivisions, each with attributes like student original input, status (e.g., correctness, completeness), LLM-validated content, and hints. The JSON is rendered conditionally, preserving the student’s original structure while highlighting missing or incorrect steps. Even if a student's input differs from what the LLM considers correct, the original content is preserved and marked with a status color. Hints are provided and triggered progressively, offering ``just the right'' level of guidance.



\ms{
\subsection{Implementation}
For the front-end, we used native HTML, JavaScript, and jQuery. On the back-end, we deployed the application with the Flask\footnote{https://flask.palletsprojects.com/en/stable/} framework on our university's server. The code editor utilized CodeMirror\footnote{https://codemirror.net/} integrated with pyodide.js\footnote{https://pyodide.org/en/stable/} for running Python code. We employed OpenAI's GPT-4o model with a temperature of 0.8 to maintain flexibility during scaffolding. To align better with the front-end's step tree, we set the response format by setting the parameter response\_format={``type'': ``json\_object''}, restricting the LLM's output. Prompts designed with the chain-of-thought (CoT) technique \cite{wei2022chain} are detailed in the supplementary materials.
}

\section{Technical Evaluation}
\label{techeval}






In DBox, the key feature is that learners construct a step tree using either code or natural language descriptions, while the LLM evaluates each step and provides necessary feedback. To assess whether effective prompt engineering enables the GPT-4o model (hereafter referred to as GPT or LLM) to accurately determine node statuses (i.e., correct, incorrect, or missing), we conducted a preliminary technical evaluation. Detailed prompts used are provided in the supplementary material.






\subsection{Dataset Creation} 
We created a dataset of learners' authentic thought errors to evaluate LLMs' ability to recognize the status of the thought process.
Based on GPT-4's performance on coding tasks~\cite{finnie2023my, savelka2023thrilled}, we selected 25 easy-level LeetCode problems covering various algorithms and data structures problems (e.g., dynamic programming, sorting, greedy algorithms).

To capture natural variations, we recruited five computer science students from a local university to manually create the step trees for five randomly selected problems (25 in total). 
Using correct code samples as references, annotators constructed a step tree based on their solution, including steps, substeps, and sub-substeps. 
They described each node in their own words and linked it to the relevant code.
After collecting the annotated step trees, we manually created various types of errors to simulate common student misconceptions in coding \cite{qian2017students}.
An algorithmic programming expert created seven error types for each problem (e.g., missing steps, incorrect step order, logical errors, and syntax errors), which were reviewed by another expert, resulting in 175 error-laden step trees (25 problems x 7 error types).
% \subsubsection{Procedure} We briefed participants on the annotation task's purpose and process, assigning 25 randomly selected problems among five students, with each solving five problems. Participants constructed a step tree for each problem based on their chosen solution approach, facilitated by providing correct code samples for various possible solutions. A step tree comprises steps, substeps, and even sub-substeps. Participants were tasked with constructing this tree according to their thought process, describing each node in their own words, and associating each node with the relevant code. We organized the problems, reference codes, and step tree components into a Google spreadsheet for annotation.

% \subsubsection{Error Generation} Once we collected the annotated step trees, we proceeded to generate potential errors reflecting possible thought process missteps. Since DBox handles both direct code inputs and natural language descriptions, we generated errors for both modalities.

% Drawing from common student misconceptions identified in previous studies \cite{qian2017students}, we defined three error types for natural language descriptions: missing steps, incorrect step order, and logical step errors. For code-based errors, we identified four types: missing steps, incorrect step order, logical step errors, and syntax errors. An expert in algorithmic programming constructed these errors for each problem's step tree, creating seven types of errors per problem, reviewed by another expert. This resulted in 175 error-laden step trees (25 problems x 7 error types).


% \subsection{Analysis Approach} For each error type, we categorized the steps into two parts: the correct part, consisting of steps with a correct status (labeled as 1), and the incorrect/missing part, consisting of steps with an incorrect or missing status (labeled as 0). To simplify the analysis and maintain a rigorous evaluation, we did not detail each step or substep within a task due to their potential complexity. Instead, we adopted a straightforward evaluation method: if all steps in the correct part are predicted as correct by GPT, the prediction is classified as 1; any incorrect prediction in this section classifies the overall prediction as 0. Conversely, if all steps in the incorrect/missing part are identified as incorrect or missing, the prediction is marked as 0; any deviation from this results in a prediction classified as 1. This method enables us to calculate classification metrics such as accuracy, F1 score, precision, recall, specificity, false positive rate, and false negative rate effectively.

% We removed the status fields from 175 error-containing step trees and input them into the LLMs for status prediction. We then compared the LLMs' predictions against the expert-annotated ground truth. For instances where LLM predictions were inaccurate, two authors independently performed open coding of GPT’s outputs to identify error themes, causes, and the scenarios in which they occurred. After a detailed discussion and analysis, we consolidated our findings into thematic insights.

\subsection{Analysis Approach} We divide the steps into two parts based on the expert annotations: the correct part (1) or the incorrect/missing part (0). To calculate the performance of GPT, we use a very strict evaluation method: if all steps in the correct part are determined to be correct, the prediction of this part is marked as 1; otherwise, the prediction is 0. Similarly, if all steps in the incorrect/missing part are determined to be incorrect/missing, the prediction of this part is 0; otherwise, the prediction is 1. This approach enabled calculation of accuracy, F1 score, precision, recall, specificity, false positive rate, and false negative rate.
We removed status fields from 175 error-containing step trees and input them into GPT for prediction. The results were compared against expert-annotated ground truth. For incorrect predictions, two authors independently coded GPT's outputs to identify error themes and causes, later consolidated through discussion.



% \subsection{Results}
% As demonstrated in Table \ref{tab:technicalresult}, \textbf{GPT more effectively identifies students' thought processes when expressed directly through code rather than natural language descriptions}. This difference is attributed to the extensive code-based training data for GPT \cite{liu2024your}, contrasted with the relatively sparse data on natural language descriptions of thought processes. Additionally, GPT employs specialized post-evaluation mechanisms for code handling \cite{achiam2023gpt}, and students' descriptions often contain imprecise or non-standard terminology, leading to potential ambiguities and misunderstandings.

% For sequence change errors, GPT's accuracy is markedly lower from natural language descriptions, recording only 70\% accuracy and a 72\% F1 score. Misclassifications include 36\% of incorrect steps judged as correct (False Positive Rate, FPR = 0.36) and 24\% of correct steps judged as incorrect (False Negative Rate, FNR = 0.24). In contrast, code-based evaluations yield perfect accuracy and F1 scores of 100\%. For logical errors evaluated from natural language, accuracy dips to 88\% with an F1 score of 87\%, FPR of 8\%, and FNR of 16\%. However, when evaluated from code, these figures improve dramatically to 98\% accuracy and F1, with a minimal FPR of 4\% and an FNR of 0. Similarly, for missing steps or substeps, natural language evaluations show only 86\% accuracy and F1 score, with an FPR of 16\% and an FNR of 12\%. Code-based evaluations show better results at 92\% accuracy and F1 score, with both FPR and FNR at 8\%. Syntax error identifications maintain a steady 90\% accuracy and F1 score.

% Moreover, our analysis of GPT's output reveals \textbf{occasional modifications in the structure and content of the step tree}. Despite instructions to maintain the original structure and add nodes only for missing steps, discrepancies occur in how GPT segments steps and sometimes alters the students' original input.

% Additionally, a significant finding is that \textbf{GPT sometimes inaccurately judges students' approaches as incorrect if they deviate from commonly recognized solutions}. This occurred in five out of 25 tasks, suggesting that GPT's training may bias it towards optimal solutions commonly represented in its data. For instance, GPT judged sorting an array to find a missing number as unnecessary, incorrectly marking a divergent student approach as incorrect.

% In conclusion, while GPT excels in processing code-based inputs, it struggles with natural language, especially in identifying sequence change errors. Given the infrequency of such errors, GPT-4's capabilities are generally sufficient for assessing and guiding students' thought processes. As LLM technology advances, its potential for accurately understanding and guiding users' thought processes is expected to become more reliable and practical.


\begin{table*}[hbpt]
% the environment \color{blue} change all cell color
	\centering
	\caption{The technical evaluation of GPT-4o assesses its ability to identify the status of learners' steps. \textbf{Precision} refers to the proportion of steps correctly predicted as correct by GPT. \textbf{TPR} (True Positive Rate) measures the proportion of truly correct steps that GPT identifies correctly. \textbf{TNR} (True Negative Rate) reflects the proportion of truly incorrect/missing steps that GPT correctly predicts. \textbf{FPR} (False Positive Rate) indicates the proportion of incorrect/missing steps that GPT incorrectly predicts as correct. \textbf{FNR} (False Negative Rate) represents the proportion of correct steps that GPT incorrectly predicts as incorrect/missing.}
% 	~\glcomment{To check whether we should conduct such analysis for explanation effect. Or do ablation-like, compare with explanation without explanation?(with explanation, with DKE vs without DKE or with explanation, with DKE vs without DKE, with explanation)}}
	\label{tab:technicalresult}%
	\begin{small}
	\begin{tabular}{c | c c c c c c c}
	    \hline
	   % Hypothesis&	\multicolumn{4}{c|}{\textbf{H1}}& \multicolumn{4}{c}{\textbf{H2}} \\
	    % \textbf{Participants}&	\multicolumn{5}{c|}{\textbf{All}}& Post-hoc results\\
	    % \hline
	    \textbf{Error Type}&Accuracy&F1&Precision&TPR/Recall&TNR/Specificity&FPR&FNR\\
	    \hline
 \hline
\multicolumn{8}{l}{\textbf{Identify step/substep status from learners' natural language-based step descriptions}}\\
\hline    
	    \textbf{Sequence Changed}&0.70&0.72&0.68&0.76&0.64&0.36&0.24\\
	
 \rowcolor{gray!15}\textbf{Logical Error}&0.88&0.87&0.91&0.84&0.92&0.08&0.16\\
	    \textbf{Missing}&0.86&0.86&0.85&0.88&0.84&0.16&0.12 \\

 \hline
\multicolumn{8}{l}{\textbf{Identify step/substep status from learners' codes}}\\
\hline
    \rowcolor{gray!15}\textbf{Sequence Changed}&1.00&1.00&1.00&1.00&1.00&0.00&0.00\\

	\textbf{Logical Error}&0.98&0.98&0.96&1.00&0.96&0.04&0.00\\

 
    \rowcolor{gray!15}\textbf{Missing}&0.92&0.92&0.92&0.92&0.92&0.08&0.08\\


    \textbf{Syntax Error}&0.90&0.90&0.88&0.92&0.88&0.12&0.08\\
    % \textbf{TiA-Trust}& 3.93& .140& $3.00 \pm 0.81$& $3.22 \pm 0.87$& $3.11 \pm 0.85$& -\\
	    \hline
	\end{tabular}%
	\end{small}
\end{table*}

\subsection{Results}
As shown in Table \ref{tab:technicalresult}, \textbf{GPT more accurately identifies students' thought processes when expressed through code rather than natural language}. This difference may stem from GPT's extensive code-based training data \cite{liu2024your} and specialized code-handling mechanisms \cite{achiam2023gpt}, while natural language descriptions often include imprecise or non-standard terminology, leading to ambiguities \cite{liu2023wants}.

\textbf{GPT sometimes identifies incorrect steps as correct (false positives) or correct steps as incorrect (false negatives)}. For \emph{sequence change errors}, GPT’s accuracy drops to 70\% with an F1 score of 72\% from natural language descriptions, with a False Positive Rate (FPR) of 36\% and a False Negative Rate (FNR) of 24\%. In contrast, code-based evaluations achieve 100\% accuracy and F1 scores. For \emph{logical errors} from natural language, GPT’s accuracy is 88\% (F1 score 87\%, FPR 8\%, FNR 16\%), compared to 98\% accuracy and F1 scores from code-based evaluations (FPR 4\%, FNR 0\%). \emph{Missing step error} evaluations from natural language yield 86\% accuracy (FPR 16\%, FNR 12\%), improving to 92\% from code inputs (FPR/FNR 8\%). \emph{Syntax error} identification remains steady at 90\% accuracy and F1 score.

Additionally, we find that \textbf{GPT occasionally alters the structure and content of the step tree}, despite instructions to only add missing steps. It sometimes modifies how steps are segmented or misinterprets the student's original input. Another key finding is that \textbf{GPT sometimes incorrectly judges non-standard approaches as wrong}. In five out of 25 tasks, GPT mistakenly flagged correct solutions as incorrect simply because they deviated from the common approaches in its training data. For example, it marked a sorting-based solution as incorrect, even though it was correct, albeit not the most optimal approach.

\ms{In summary, GPT demonstrates strong capabilities in processing code-based inputs but faces challenges with natural language, particularly in detecting sequence changes. Our analysis of participants' logs from the user study revealed that most errors encountered were logical errors or missing steps, while errors involving sequence changes were relatively uncommon. This suggests that GPT is generally effective in evaluating students' thought processes during algorithmic programming learning. We recommend that researchers considering GPT for supporting learners in natural language programming carefully evaluate its limitations and conduct technical assessments to determine its suitability for their specific scenarios. We hope this technical evaluation serves as a valuable reference for similar future research.}




% \subsection{Results}
% As shown in Table \ref{tab:technicalresult}, overall, \textbf{GPT is more effective at identifying students' thought processes when they express their ideas directly through code rather than through natural language descriptions}. One reason for this is that GPT's training data includes a substantial amount of code \cite{liu2024your}, whereas there is likely much less data on verbal descriptions of thought processes. Second, GPT employs some special post-evaluation mechanisms when dealing with code \cite{achiam2023gpt}. Thirdly, students' natural language descriptions of specific steps in their thought processes are often unclear, using imprecise or non-standard terms that can lead to ambiguity and misunderstanding. For instance, we found that students' natural language descriptions are sometimes ambiguous and imprecise, which causes GPT to perform worse when recognizing steps compared to recognizing code. Additionally, when students provide overly simplistic descriptions of steps, such as failing to explicitly consider edge cases, GPT is more likely to incorrectly judge the status as wrong. These issues are also common in human instructional communication—informal language can indeed lead to misinterpretation.

% Specifically, when identifying sequence change errors, GPT performs poorly when judging from natural language descriptions. The accuracy is only 70\%, and the F1 score is 72\%. Among the incorrect steps, 36\% were wrongly identified as correct (FPR = 0.36), and among the correct steps, 24\% were misjudged as incorrect (FNR = 0.24). In contrast, when judging from code, both accuracy and F1 are 100\%, with no errors observed. For identifying logical errors, if judged from natural language descriptions, the accuracy is only 88\% and the F1 score is 87\%. Among the incorrect steps, 8\% were misidentified as correct (FPR = 0.08), and among the correct steps, 16\% were misjudged as incorrect (FNR = 0.16). When judged from code, the accuracy and F1 score reach 98\%, with only 4\% of the incorrect steps being misidentified as correct (FPR = 0.04), and none of the correct steps being judged as incorrect (FNR = 0). When identifying missing steps or substeps, the accuracy and F1 score are only 86\% when judged from natural language descriptions. Among the incorrect steps, 16\% were wrongly identified as correct (FPR = 0.16), and among the correct steps, 12\% were misjudged as incorrect (FNR = 0.12). When judged from code, the accuracy and F1 score are 92\%, with lower FPR and FNR (both 0.08). When identifying syntax errors, the accuracy and F1 score are both 90\%.

% Additionally, based on our analysis of GPT's returned step tree, we identified two other issues. First, \textbf{GPT occasionally changes the structure and content of the step tree}. When identifying the status from a student's natural language description of steps, we prompt GPT to evaluate the status of each node based on the student's step tree, allowing GPT to add nodes only in cases of missing steps while preserving the original structure of the step tree. Furthermore, we require GPT to retain the original input for each node so that we can match each node in GPT's returned step tree with the nodes in the UI. However, we found that GPT does not always strictly adhere to the student's input step tree. Specifically, there are instances where the step tree returned by GPT differs in how it segments steps compared to the student's segmentation. Additionally, GPT sometimes alters the student's original input.

% Second, one of our core objectives is to enable GPT to provide personalized support to students, continuing to develop the solution along the student's thought process. However, \textbf{there are instances where, if the student does not adopt the same approach as GPT, GPT will judge the student's approach as incorrect}. This occurred five times out of 25 tasks. We speculate that some tasks may have widely recognized optimal solutions, and GPT's training data is dominated by these solutions, leading GPT to classify non-optimal approaches used by students as incorrect. For example, in a problem where students are asked to find the missing number in an array, some students start by sorting the array, and GPT responds by saying that sorting is unnecessary, judging the step as incorrect.

% In summary, GPT performs well when determining the status of step trees based on code. However, due to limitations in the training data and the inaccuracies in users' language, GPT performs less effectively when judging the status of step trees based on natural language descriptions, particularly when identifying sequence change errors. That said, since sequence changes occur very infrequently, we believe that the current capabilities of GPT-4 are sufficient to assess students' thought processes and provide guidance. As LLMs continue to improve in the future, using them to understand and identify users' thought processes should become a straightforward and practical application.







% \subsubsection{RQ1: Can our approach generate meaningful steps based on the current code solution provided by the users?}

% \begin{itemize}
%     \item If users' solution is correct, can our approach generate correct steps to map each code unit?
%     \item If users' solution is correct but incomplete, can our approach identify the correct steps as well as missing steps?
%     \item If users' solution contains incorrect code lines, can our approach identify the correct steps as well as missing steps?
% \end{itemize}
\section{User Study}




We conducted a user study to evaluate the effects of DBox, focusing on three questions:
\begin{itemize}
    \item \textbf{Q1}: How does DBox support algorithmic programming learning?
    \item \textbf{Q2}: How does DBox affect learners' perceptions and user experience?
    \item \textbf{Q3}: How do learners interact with DBox and perceive the usefulness of different features?
\end{itemize}

% \ms{
% \subsection{Target User Group}
% DBox targets highly motivated, self-regulated students. Using a scaffolding approach, it prioritizes learners' independent thinking by providing only essential support. DBox assumes students actively engage with the tool to improve decomposition skills rather than passively completing tasks. However, we acknowledge that users seeking shortcuts may bypass this by searching for answers online.
% }




\subsection{Conditions}
We conducted a within-subjects design to control for individual differences in programming abilities. Participants experienced two conditions in a randomly assigned order:


\begin{itemize}
    \item \textbf{DBox}: Participants solved problems using the proposed DBox.
    \item \textbf{Baseline}: Participants freely used any available tools (e.g., ChatGPT, Copilot, search engines, LeetCode, QA platforms) to reflect their real-world learning habits, with no restrictions on tool usage or combinations.
\end{itemize}






\subsection{Task and Materials}
In this experiment, participants solve problems from two distinct algorithm types. Each type includes a learning problem, where participants use DBox or baseline tools, and a test problem, solved independently without assistance.

We selected problems from the LeetCode problem bank based on several criteria: First, all problems were of medium difficulty, with an acceptance rate between 40\% and 50\% to ensure sufficient challenge. Second, GPT performs well on these problems. Third, the two algorithm types are distinctly different to avoid learning effects. Finally, the learning and test problems within each algorithm type require similar programming skills to avoid unfair comparisons due to differences in additional coding skills needed for each problem. Based on these criteria, we chose two algorithm types: Greedy and Binary Search. For Greedy, we selected ``Jump Game''\footnote{https://leetcode.com/problems/jump-game/description/} and ``Jump Game II''\footnote{https://leetcode.com/problems/jump-game-ii/description/}; for Binary Search, we selected ``Search in Rotated Sorted Array''\footnote{https://leetcode.com/problems/search-in-rotated-sorted-array/description/} and ``Search in Rotated Sorted Array II''\footnote{https://leetcode.com/problems/search-in-rotated-sorted-array-ii/description/}.

\ms{To help participants become familiar with or recall the algorithms used in the study, we provide them with lecture materials prior to the start of the study. The lecture materials for the two types of algorithms were sourced from GeeksforGeeks\footnote{Greedy: https://www.geeksforgeeks.org/introduction-to-greedy-algorithm-data-structures-and-algorithm-tutorials/}\footnote{Binary Search: https://www.geeksforgeeks.org/binary-search/}. These materials include an introduction to each algorithm, illustrated figures, and practical examples.}


\begin{figure*}[htbp]
	\centering 
	\includegraphics[width=0.9\linewidth]{figures/procedure.pdf}
	\caption{The procedure of our user study. \ms{To avoid learning effects, we used a counterbalanced design with four combinations: (1) DBox-Type1 → Baseline-Type2, (2) Baseline-Type1 → DBox-Type2, (3) DBox-Type2 → Baseline-Type1, and (4) Baseline-Type2 → DBox-Type1. For each combination, six participants were randomly assigned.}}
	\label{procedure}
        \Description{}
\end{figure*}

\subsection{Procedure}
\ms{As shown in Figure \ref{procedure}, obtaining participants' consent, we begin by explaining the study's objectives and procedure. We then familiarize participants with the algorithms they will be practicing using the provided lecture materials. Note that the lecture material was designed to help participants recap the key concepts of these two algorithms. Although the two problems are labeled as Binary Search and Greedy on the LeetCode platform, participants were not restricted to using these two specific approaches to solve the problems}. Afterwards, we administer a pre-test problem to assess their expertise. We also have participants rate their confidence in solving the problem on a 7-point Likert scale following \cite{jin2024teach}. \ms{After participants completed the task or indicated they could not proceed, the two authors (as experiment operators) assessed their solutions against pre-verified answers (with multiple solutions). Participants who solved the problem correctly were excluded, as it suggested higher expertise in the tested problem type. Following \cite{jin2024teach}, we also excluded participants who rated their confidence at 6 or above, as high self-confidence likely indicates less need for additional support. While self-reported confidence may not always align with actual ability, this criterion helps focus the study on the intended user group for our tool. We then provided participants with an interactive tutorial to familiarize them with DBox. The tutorial guides them through each view, button, and functionality of the tool. After the tutorial, they can explore the tool by solving an exercise problem, different from the two problem types in the main study.

Next, we assigned experimental conditions and problem types to participants with a counterbalanced design.} Within each problem type, one problem is randomly assigned in the learning session and the other in the testing session. In the learning session, participants use either DBox or baseline tools, during which they must successfully pass all test cases to proceed. They then complete an in-task survey about their perceptions and experience with the tool they just used. In the test session, participants solve problems without any tool assistance. After completing both problem types, they fill out a post-task survey, followed by a semi-structured interview.





\subsection{Participants}

We conducted a power analysis using G*Power \cite{faul2009statistical} for our two-condition within-subjects design. Assuming an effect size of $f = 0.6$ (moderate), $\alpha = 0.05$, and power of 0.8, we estimated a required sample size of 24 participants.

After IRB approval, we recruited 24 participants via emails and social media at local universities (10 female, 14 male, average age 23.5, SD = 1.7). The group included 16 undergraduates and eight graduate students, with majors in computer science (17), data science (3), electrical engineering (3), and mathematics (1). Most participants (18) coded weekly, six coded monthly, and 23 had used platforms like LeetCode. The 90-minute study compensated participants with 20 USD, equating to 13 USD/hour.






\subsection{Measurements}
Our measurements are summarized in Table \ref{tab:measurement}. To address \textbf{Q1} (effects on learning outcomes), we assessed correctness in the testing session \cite{kazemitabaar2023studying}, perceived learning gain \cite{zhou2021does}, confidence in solving similar problems \cite{hendriana2018role}, improvement in algorithmic thinking \cite{yaugci2019valid}, and self-efficacy \cite{tsai2019improving, yildiz2018digital}. For \textbf{Q2} (effects on perceptions and user experience), we measured cognitive engagement \cite{pitterson2016measuring}, critical thinking \cite{kamin2001measuring}, sense of achievement \cite{wiedenbeck2004factors}, feeling of cheating \cite{kazemitabaar2023studying}, perceived help appropriateness, usefulness \cite{davis1989perceived}, mental demand, effort, frustration \cite{hart2006nasa}, ease of use, satisfaction \cite{bangor2008empirical}, and future use intention \cite{holden2010technology}. For \textbf{Q3} (usage patterns and perceptions of DBox), we analyzed usage logs (e.g., clicks, edits, help-seeking), post-task feature ratings, and conducted semi-structured interviews to delve into participants' underlying reasons behind their perceptions, usage patterns, and reactions to AI errors. All questionnaires used a 7-point Likert scale.

% Most questions, aside from those specific to DBox features, were adapted from established sources such as the NASA Task Load Index \cite{hart2006nasa}, System Usability Scale \cite{brooke2013sus}, and Technology Acceptance Model \cite{holden2010technology}.








% \begin{itemize}
%     \item \textbf{Correctness Score}: The accuracy of learners' solutions in the test tasks. Following the approach used in \cite{kazemitabaar2023studying}, two independent researchers graded each submitted solution using a straightforward and consistent rubric, deducting 25\% for each major issue or missing component in the final submission, resulting in scores of 0\%, 25\%, 50\%, 75\%, or 100\%. The two graders fully agreed on 87.5\% of the submissions. For the remaining 12.5\% where disagreements arose, the graders discussed to resolve conflicts.
%     \item \textbf{Perceived Learning Gain}: "I have learned how to solve this type of problem." 
%     \item \textbf{Confidence in Solving Similar Problems}: "After solving this problem with the tool's help, I feel confident in tackling similar problems." 
%     \item \textbf{Algorithmic Thinking Improvement}: "This tool improved my ability to break down complex problems into smaller, manageable parts." 
%     \item \textbf{Self-Efficacy}: "I have mastered the problem-solving skills necessary for this type of problem." 
% \end{itemize}

% To address \textbf{RQ2: How does Decomposition Box affect learners’ perceptions and user experience?}, we measured the following factors:
% \begin{itemize}
%     \item \textbf{Cognitive Engagement}: "I was cognitively engaged in the programming exercises." 
%     \item \textbf{Critical Thinking}: "The learning process challenged me to think critically." 
%     \item \textbf{Sense of Achievement}: "I feel a sense of accomplishment/achievement when I complete the programming task." 
%     \item \textbf{Sense of Cheating}: "Using this tool feels like cheating." 
%     \item \textbf{Perceived Appropriateness of Help}: "I felt I received the right amount of help when needed—neither too much nor too little." 
%     \item \textbf{Perceived Usefulness}: "This tool is useful for learning how to solve specific problems." 
%     \item \textbf{Mental Demand}: "How mentally demanding was the task?" 
%     \item \textbf{Effort}: "How hard did you have to work to achieve your level of performance?" 
%     \item \textbf{Frustration}: "How insecure, discouraged, irritated, stressed, and annoyed were you?" 
%     \item \textbf{Ease of Use}: "I find this tool easy to use for learning algorithms." 
%     \item \textbf{Satisfaction}: "I am satisfied with the overall learning experience using this tool." 
%     \item \textbf{Future Use}: "I would like to use this tool in my future programming learning." 
% \end{itemize}




% 1. I have learned how to solve this type of problems.
% 2. I was cognitively engaged in the programming exercises.
% 3. The learning process challenged me to think critically.
% 4. After solving this problem with the help, I feel confident in dealing with similar problems.
% 5. I feel a sense of accomplishment/achievement when I complete the programming task.
% 6. Using this tool feels like cheating
% 7. I feel this task was NOT done by myself.
% 8. I felt I received the right amount of help when needed, which is not too much or too few.
% 9. This tool is useful for learning how to solve certain problems. 
% 10. I have mastered the problem-solving skills necessary for this problem.
% 11. This tool improved my ability to break down complex problems into smaller, manageable parts. 
% 12. How mentally demanding was the task?
% 13. How hard did you have to work to accomplish your level of performance?
% 14. How insecure, discouraged, irritated, stressed, and annoyed were you?
% 15. I find this tool easy to use for learning algorithm.
% 16. I am satisfied with the overall learning experience of using this tool.
% 17. I would like to use this tool in my future programming learning.



\renewcommand{\arraystretch}{1.5}
\begin{table*}[htp]  

\centering  
\fontsize{8}{8}\selectfont  

\caption{Measurements used in our user study. For the questionnaire items (within the quotation marks), a 7-point Likert scale was used, with 1 indicating ``Strongly disagree/Very low'' and 7 indicating ``Strongly agree/Very high''.}\label{tab:measurement}

\begin{tabular}{m{0.5cm}<{\centering}m{3cm}<{\centering}m{10.5cm}}
\toprule
\textbf{}&\textbf{Metrics}&\textbf{Detailed Meaning and Questions}\\ \hline\hline


\multirow{6}*{\shortstack{\textbf{Q1}}}&Correctness Score & The correctness of learners' test task solutions was evaluated using a consistent rubric from \cite{kazemitabaar2023studying}. Two authors independently graded submissions, deducting 25\% for each major issue or missing component, yielding scores of 0\%, 25\%, 50\%, 75\%, or 100\%. They agreed on 87.5\% of submissions, resolving disagreements through discussion for the rest.\\
\cline{2-3}
&\cellcolor{gray!15}Perceived Learning Gain &\cellcolor{gray!15} "I have learned how to solve this type of problem." \\ 
\cline{2-3}
&Confidence in Solving Similar Problems & "After solving this problem with the tool's help, I feel confident in tackling similar problems." \\ 
\cline{2-3}
&\cellcolor{gray!15}Perceived Algorithmic Thinking Improvement & \cellcolor{gray!15}"This tool improved my ability to break down complex problems into smaller, manageable parts." \\ 
\cline{2-3}
&Self-Efficacy & "I have mastered the problem-solving skills necessary for this type of problem." \\ 
\hline

\multirow{12}*{\shortstack{\textbf{Q2}}}&\cellcolor{gray!15}Cognitive Engagement &\cellcolor{gray!15} "I was cognitively engaged in the programming exercises." \\
\cline{2-3}
&Critical Thinking & "The learning process challenged me to think critically." \\
\cline{2-3}
&\cellcolor{gray!15}Sense of Achievement &\cellcolor{gray!15} "I feel a sense of accomplishment/achievement when I complete the programming task." \\
\cline{2-3}
&Sense of Cheating & "Using this tool feels like cheating." \\
\cline{2-3}
&\cellcolor{gray!15}Perceived Appropriateness of Help &\cellcolor{gray!15} "I felt I received the right amount of help when needed—neither too much nor too little." \\
\cline{2-3}
&Perceived Usefulness & "This tool is useful for learning how to solve specific problems." \\
\cline{2-3}
&\cellcolor{gray!15}Mental Demand &\cellcolor{gray!15} "How mentally demanding was the task?" \\
\cline{2-3}
&Effort & "How hard did you have to work to achieve your level of performance?" \\
\cline{2-3}
&\cellcolor{gray!15}Frustration &\cellcolor{gray!15} "How insecure, discouraged, irritated, stressed, and annoyed were you?" \\
\cline{2-3}
&Ease of Use & "I find this tool easy to use for learning algorithms." \\
\cline{2-3}
&\cellcolor{gray!15}Satisfaction &\cellcolor{gray!15} "I am satisfied with the overall learning experience using this tool." \\
\cline{2-3}
&Future Use & "I would like to use this tool in my future programming learning." \\
\hline


\multirow{6}*{\shortstack{\textbf{Q3}}}&\cellcolor{gray!15}Button Clicking &\cellcolor{gray!15} (with timestamp) From Editor to Step Tree, Check Step Tree, Check Match, From Step Tree to Comments, Run Code\\
\cline{2-3}
&Editing & (with timestamp) Code edit, step tree edit\\
\cline{2-3}
&\cellcolor{gray!15}Help-Seeking &\cellcolor{gray!15} (with timestamp) see general hint, see detailed hint, and reveal step/code\\
\cline{2-3}
&Usefulness Rating &Participants' ratings on the usefulness of various features in DBox using a 7-point Likert scale\\
\cline{2-3}
&\cellcolor{gray!15}Interviews &\cellcolor{gray!15} Participants' detailed reasons for their perceptions, reactions to AI errors, and self-reported usage patterns, etc.\\

\bottomrule
\end{tabular}
\end{table*}




\subsection{Data Analysis}
\ms{To eliminate the unfair comparison caused by the learning effect of participants using both tools to solve the same type of problem, we selected two distinct problem types. We implemented a randomization procedure to ensure that each participant used either DBox or the baseline tool in a random order, with a randomly assigned problem type for each tool. As a result, each participant used only one tool to solve one problem.

For the quantitative analysis, we employed a linear mixed effects model to analyze the data. The dependent variables (DVs) were our outcome measures (e.g., scores or questionnaire ratings). First, we analyzed the main effect of the two different tools (the coefficient and p-value were reported based on this analysis). Then, we examined the interaction effects between the learning tool and problem type (\emph{Tool*Problem Type}), as well as the interaction effect between the learning tool and the order of tool usage (\emph{Tool*Order}). The fixed effects in the models included the learning tool, problem type, and the order of tool usage, while the random effect accounted for individual differences between participants.

}

For the qualitative analysis of our semi-structured interview data, grounded in the designed questions, we conducted a thematic analysis \cite{hsieh2005three}. Two authors independently coded the data, developed a codebook, and reached a consensus through discussion. In the results, we present key themes supported by representative participant quotes.


% For the quantitative analysis of the questionnaire data and correctness scores, we first conducted a Shapiro–Wilk test to assess normality. Since most of the questionnaire data did not follow a normal distribution, we applied the Wilcoxon signed-rank test, reporting the test statistic ($Z$), significance levels ($p$), and effect sizes ($r$). Additionally, descriptive statistics were used to summarize participants' usage logs.




\section{Results}
In this section, we examine how DBox supports learners in algorithmic programming and how they interact with the tool. \ms{We compared DBox with the baseline tool and analyzed the interaction effects between tool and problem type, as well as the ordering effect (e.g., DBox first or Baseline first). Overall, we didn't find significant interaction or ordering effects for most metrics. Therefore, we report only the differences between the two tools unless notable interactions or ordering effects were observed, which are analyzed in detail.}



\subsection{How does Decomposition Box help with algorithmic programming learning?}

We first compared the correctness scores of participants' test session submissions under both DBox and baseline conditions. As shown in Figure \ref{RQ1} (a), participants using DBox achieved significantly higher correctness scores than those in the baseline condition ($Coef.$=0.198, $p$<0.05), suggesting that practicing with DBox better prepared learners to transfer their skills to similar algorithmic challenges.

This finding aligns with participants' subjective perceptions. Figure \ref{RQ1} (b) shows that learners in the DBox condition reported significantly higher perceived learning gains ($Coef.$=2.250, $p$<0.001), higher confidence in solving similar problems ($Coef.$=2.333, $p$<0.001), more improvements in algorithmic thinking ($Coef.$=3.875, $p$<0.001), and greater self-efficacy ($Coef.$=3.042, $p$<0.001) compared to the baseline condition.



Interview analysis further highlighted that participants felt solving tasks independently during the learning session enhanced their perceived learning gains. Overcoming challenges on their own also boosted their confidence. In contrast, baseline participants felt their algorithmic thinking was underdeveloped due to easy access to complete solutions (e.g., via search, ChatGPT, or Copilot), leading to lower perceived learning and confidence. As P5 noted, ``\emph{Even though I couldn’t write the full solution, the tool [DBox] encouraged me to break down the problem. I started with what I knew, and the tool guided me through the rest. Decomposing the problem helped me structure my approach, and as I saw the step tree fill in correctly, I felt my algorithmic thinking improve, and I gained confidence in solving the problem.}''




% \subsection{RQ1: How does Decomposition Box help with algorithmic programming learning?}
% First, we compared the objective correctness scores of the codes submitted by participants during the test sessions under both the DBox and baseline conditions. As shown in Figure \ref{RQ1} (a), participants in the DBox condition achieved significantly higher correctness scores than those in the baseline condition ($p<0.05$), indicating that practicing algorithmic problems with DBox better equipped learners to transfer their practiced skills to similar algorithmic challenges.

% This result is consistent with participants' subjective perceptions of their learning outcomes. As illustrated in Figure \ref{RQ1} (b), students in the DBox condition reported significantly higher perceived learning gains, confidence in solving similar problems, improvement in algorithmic thinking, and self-efficacy compared to the baseline condition (all $p<0.001$).

% Our interview analysis further revealed that participants felt that independently solving tasks during the learning session enhanced their perceived learning gains. Successfully overcoming challenges on their own also boosted their confidence in tackling similar problems. In contrast, participants in the baseline condition expressed that their algorithmic thinking was insufficiently exercised since they could easily access complete answers (e.g., by looking them up, asking ChatGPT, or using Copilot). This led to lower perceived learning gains and reduced confidence in facing similar problems. For example, as P5 remarked, ``\emph{Even though I couldn't write the full solution, the tool encouraged me to break down the problem. I just started by writing out the steps I knew, and the tool guided me to complete my thought process. Decomposing the problem helped me form a structured approach, and as I saw the step tree gradually fill in and become correct, I felt my algorithmic thinking improved, and I gained the skills to solve the problem.}''




% First, we compared the objective correctness scores of the code submitted by participants during the test sessions in the DBox and baseline conditions. As shown in Figure \ref{RQ1}(a), participants in the DBox condition achieved significantly higher correctness scores than those in the baseline condition ($p<0.05$). This indicates that after practicing algorithmic problems with DBox, learners were better able to transfer the skills they had developed to similar algorithmic challenges.

% This result aligns with the students' subjective perceptions of their learning outcomes. As illustrated in Figure \ref{RQ1} (b), compared to the baseline condition, students in the DBox condition reported significantly higher perceived learning gains, confidence in solving similar problems, improvement in algorithmic thinking, and self-efficacy (all $p$<.001).

% Our analysis of the interview data revealed that participants felt that independently solving the tasks during the learning session enhanced their perceived learning gains. The success achieved through their own efforts also boosted their confidence in tackling similar problems. In contrast, in the baseline condition, participants felt that their algorithmic thinking was not adequately exercised, nor did they learn much problem-solving skills, as they could easily access complete answers (whether by looking them up, asking ChatGPT, or using Copilot). This led them to feel lower learning gains and less confidence when facing similar problems. For example, as P5 mentioned, "Although I knew I couldn't write the full solution, this tool encouraged me to break down the problem. I just needed to start by writing out the steps I knew, and the tool would guide me to complete my thought process. On one hand, decomposing the problem helped me begin to form a structured approach. As I watched the step tree gradually fill in and turn correct, I felt that my algorithmic thinking improved, and I acquired the skills needed to solve this problem."



\subsection{How does Decomposition Box affect learners’ perceptions and user experience?}



\begin{figure*}[htbp]
	\centering 
	\includegraphics[width=0.95\linewidth]{figures/RQ1_new.pdf}
	\caption{Effects on participants' learning outcomes: (a) Participants' correctness scores during the testing session, where they solved the problem independently. (b) Participants' self-reported metrics on their learning outcomes.}
	\label{RQ1}
        \Description{}
\end{figure*}


\begin{figure*}[htbp]
	\centering 
	\includegraphics[width=0.95\linewidth]{figures/perception.pdf}
	\caption{Participants' perceptions of the two conditions in their learning process.}
	\label{perception}
        \Description{}
\end{figure*}


\begin{figure*}[htbp]
	\centering 
	\includegraphics[width=0.95\linewidth]{figures/UX_new.pdf}
	\caption{Participants' cognitive load and user experience.}
	\label{UX}
        \Description{}
\end{figure*}


\subsubsection{Effects on Learners' Perceptions}

As shown in Figure \ref{perception}, participants in the DBox condition reported significantly higher cognitive engagement ($Coef.$=2.625, $p$<0.001), greater critical thinking ($Coef.$=3.375, $p$<0.001), and a stronger sense of achievement ($Coef.$=3.375, $p$<0.001) compared to the baseline condition. Conversely, those in the baseline condition felt their problem-solving process resembled more ``cheating'' ($Coef.$=-4.250, $p$<0.001). DBox was also rated as providing more appropriate assistance ($Coef.$=2.625, $p$<0.001) and being more useful for programming learning ($Coef.$=2.792, $p$<0.001). Interviews supported these results, with 19 participants noting that DBox allowed them to independently develop their thought process while offering just enough feedback to guide them. This stimulated high engagement in problem-solving, as they critically analyzed both the overall solution and each step. As P1 noted, ``\emph{When I hit a block, unlike other tools (referring to the baseline), DBox didn’t give me the answer outright, which forced me to think through the problem myself. Even with help, I still had to do most of the thinking.}''

In contrast, baseline participants using tools like ChatGPT, Copilot, or LeetCode often \textbf{bypassed independent thinking, focusing on comparing or copying provided answers}. Twelve out of 24 compared answers while coding, and eight simply copied solutions, leading to a lack of achievement and a sense of ``cheating''. As P16 (using ChatGPT) admitted, ``\emph{I tried to convert its provided code into my own, but I didn’t feel like it was truly my solution; there was no sense of achievement.}'' Besides, \textbf{excessive help in the baseline tools led participants to feel it was unhelpful for learning}. For example, P23 (who used LeetCode’s built-in solution) shared, ``\emph{I was stuck on a small part, but the solution showed the entire answer immediately. I memorized it, but later, writing the code felt more like repetition than actual learning.}'' P15 (using ChatGPT) added, ``\emph{ChatGPT explained the problem and gave the full code. While its solution seemed right, I realized I was just judging its correctness rather than improving my programming skills.}''

\ms{Moreover, we found an interaction effect between tool and problem type ($Coef.$=1.250, $p$<0.05) in perceived usefulness. Post-hoc analysis showed that DBox outperformed the baseline in both Binary Search ($t=6.159$, $p<0.001$) and Greedy problems ($t=9.273$, $p<0.001$). With DBox, there was no significant difference in perceived usefulness between the two problems ($t=0.294$, $p=0.771$). However, with the baseline, perceived usefulness was lower for the Greedy problem compared to Binary Search ($t=-2.755$, $p<0.05$). We found no significant interaction effects on correctness scores, with participants showing similar performance across the two problems using either DBox or the baseline. This suggests that the lower perceived usefulness of the baseline for the Greedy problem was not due to the problem being inherently more difficult. A likely explanation is that the Greedy problem requires higher planning and decomposition skills (i.e., breaking a complex problem into subproblems solvable by a greedy algorithm) and the baseline did not provide scaffolding to support this, leading to lower perceived usefulness.
}





% \subsection{RQ2: How does Decomposition Box affect learners’ perceptions and user experience?}

% \subsubsection{Effects on Learners' Perceptions}

% As shown in Figure \ref{perception}, results indicate that participants in the DBox condition exhibited significantly higher cognitive engagement ($p$<0.001) and employed significantly more critical thinking ($p$<0.001) compared to those in the baseline condition. They also reported a greater sense of achievement ($p$<0.001). In contrast, participants in the baseline condition felt that their problem-solving process resembled more ``cheating'' ($p$<0.001). Additionally, they found the assistance provided by DBox to be more appropriate ($p$<0.001) and considered DBox to be more beneficial for learning programming ($p$<0.001).

% Our interviews with participants supported these findings. Nineteen participants mentioned that DBox allowed them to independently develop their thought processes, providing only necessary feedback. This approach required them to fully engage in problem-solving, critically analyzing both the overall correctness and each individual step of their solutions. As P1 noted, ``\emph{When I hit a mental block, unlike the other tool (referring to the baseline), I couldn’t just get the answer directly, which forced me to think through the problem on my own. Even when I received help, the tool didn’t give me the answer outright; I still had to do most of the thinking myself.}''

% In contrast, participants in the baseline condition, who used tools like ChatGPT, Copilot, or searched for answers online (e.g., on LeetCode), were less inclined to think independently, focusing instead on understanding the provided answers. This required much less engagement and critical thinking. We observed that many participants (12 out of 24) in the baseline condition directly compared the given answers while writing code in the editor. In some cases (8 out of 24), they simply copied the provided code into the editor. This approach left users feeling that, while they may have technically solved the problem, they didn’t experience a sense of achievement and instead felt as though they had cheated. For instance, P16 (who used ChatGPT in the baseline condition) admitted, ``\emph{Even though I passed the test cases, it doesn’t feel like I solved the problem—it feels like ChatGPT did it. Even though I tried to convert ChatGPT’s code into my own, I still don’t feel like it was really my solution; there’s no sense of achievement at all.}''

% Participants also expressed frustration with the excessive assistance provided by existing tools, which they felt hindered their learning. P2 (using Copilot in the baseline condition) commented, ``\emph{I just typed a variable, and Copilot instantly suggested the entire code. My curiosity made me look at it, but sometimes, after seeing it, I realized I hadn’t had the chance to think through the problem myself. Other times, the code it suggested was completely different from what I had in mind, so I ended up spending time trying to understand it. The experience was frustrating and detrimental to my learning.}'' Similarly, P23 (who used LeetCode’s built-in solution panel in the baseline condition) shared, ``\emph{I only struggled with a certain part, but when I checked the solution, it gave me the full answer immediately. After seeing how to solve the part I was stuck on, I memorized the entire solution. Later, when I went back to the editor to write the code, it felt more like a process of repetition rather than actual learning.}'' P15 (using ChatGPT in the baseline) echoed this sentiment: ``\emph{ChatGPT is incredibly powerful—it explained the problem-solving approach and provided the full code. After reading it, I felt its solution was correct, so I went with it. But I realized that this wasn’t my thought process; I was just judging whether ChatGPT’s answer was correct. I don’t feel like my programming skills improved.}''




\begin{figure*}[htbp]
	\centering 
	\includegraphics[width=0.96\linewidth]{figures/usageNEW.pdf}
	\caption{Participants' three distinct types of system usage, each represented by a different color. We analyzed participants' interactions, including code editing, step tree editing, help-seeking, and five types of button clicks.}
	\label{usage}
        \Description{}
\end{figure*}



\subsubsection{Effects on Learners' User Experience}
\label{Effects_on_UX}
As shown in Figure \ref{UX}, participants in the DBox condition found the learning process more mentally demanding ($Coef.$=1.125, $p$<0.01) and reported putting in more effort ($Coef.$=1.625, $p$<0.001) than those in the baseline condition. This aligns with our expectations, as DBox requires learners to independently construct a step tree rather than merely providing solutions. Interestingly, there were no significant differences in frustration levels between the two conditions ($Coef.$=-0.375, $p$=0.305), suggesting that while DBox encouraged independent thinking and led to some failed attempts, the process of building the step tree did not cause excessive frustration. 
\ms{However, we found a significant ordering effect on frustration ($Coef.$=1.917, $p$<0.01). 
Post-hoc analysis showed no significant difference between the two tools when DBox was used first ($t$=1.186, $p$=0.248), but participants reported significantly higher frustration with the baseline when it was used first ($t$=2.491, $p$<0.05). 
One possible explanation is the ``learning effect'' of soliciting assistance -- using DBox first better prepared participants to request targeted assistance from the baseline. 
This aligns how participants expressed frustration with the unsolicited assistance from the baseline. 
For example, P2 (using Copilot) commented, ``\emph{I typed a variable and hit [Tab], and Copilot suggested the entire code. However, the suggestion was different from what I had in mind, and I ended up spending time trying to understand it, which was frustrating.}''}



We expected participants to find DBox less easy to use due to its more complex operations, but participants' perceived ease of use ($Coef.$=0.750, $p$=0.063) did not differ significantly between conditions. Interviews revealed that in the baseline condition, participants often had to switch between the editor and solution pages or spend time crafting precise prompts for ChatGPT. As P6 noted, ``\emph{I had to explain the problem and my understanding to ChatGPT, which was quite complex. I didn't just want to copy its answer, so I constantly did line-by-line comparison between ChatGPT-provided code and my code.}'' Moreover, participants reported significantly higher satisfaction ($Coef.$=3.125, $p$<0.001) and a greater willingness to use DBox for future programming learning ($Coef.$=2.042, $p$<0.001).

% In the exit survey on tool preference (DBox, Baseline, or Neither), 22 participants chose DBox for algorithmic programming learning, while two selected "Neither," and none preferred the Baseline.




% Participants gave their perceived difference between DBox and baseline tools in the interviews. Many preferred DBox for its structured, guided learning approach, which encouraged independent thinking and active engagement (P1-3, 5-13, 15-24). Unlike ChatGPT’s unstructured responses, DBox broke problems into steps, fostering better understanding and computational thinking (P2-9, 15, 17, 19, 20, 23). DBox also provided a greater sense of accomplishment by allowing incremental problem-solving rather than just giving solutions (P7-10, 18, 23). Additionally, DBox offered targeted advice on specific steps, while ChatGPT struggled in this area (P1, 3, 5-7, 9-10, 17). Participants using Copilot in the baseline condition expressed concerns about its negative impact on learning, noting it as more of a productivity tool that hindered independent thinking (P15, 19, 23). Although some appreciated its efficiency in reducing syntax errors, many criticized it for failing to promote deeper understanding or coding skill development (P21, 24).




% \subsubsection{Effects on Learners' User Experience}

% As shown in Figure \ref{UX}, participants in the DBox condition perceived the learning process as significantly more mentally demanding ($p$<0.01) and reported putting in significantly more effort ($p$<0.001) compared to the baseline condition. This aligns with our expectations, as DBox scaffolds learners to solve problems by having learners independently construct a step tree rather than simply presenting solutions. Furthermore, there were no significant differences in frustration levels between the two conditions, suggesting that while DBox required learners to work through the problem step by step with independent thinking, this analytical process did not lead to frustration.

% Surprisingly, there was no significant difference in perceived ease of use between the two conditions. We initially expected participants to find DBox less user-friendly, as baseline tools required only simple retrieval or inputting questions into ChatGPT, whereas DBox involved more complex operations to generate and refine the step tree. However, interviews revealed that participants approached both tools with a learning mindset. When using baseline tools, they often had to switch between the editor and the solution page to compare code, or spend time crafting precise questions for ChatGPT. As P6 (using ChatGPT in baseline) described, ``\emph{I had to explain the problem, my understanding, and the issue I was facing to ChatGPT, which was quite complicated. And I didn’t want to just paste ChatGPT’s answer into the editor, so I had to switch back and forth, comparing the differences line by line.}''

% Additionally, participants reported significantly higher satisfaction ($p$<0.001) and a greater willingness to use DBox for future programming learning ($p$<0.001). In the exit survey, we also gathered participants' preferences for learning algorithmic programming (they could choose between DBox, the baseline, or neither). Twenty-two participants preferred using DBox, while two chose "neither." These two participants expressed a preference for practicing programming without any assistance, even though DBox only provides minimal help. They preferred to solve problems entirely on their own.

% To explore these results further, we analyzed interview data to understand participants' perceptions of the differences between DBox and baseline tools. In the baseline condition, 18 participants used ChatGPT, and 10 used Copilot (4 of whom used both). The analysis revealed that DBox was generally preferred over ChatGPT for providing a more structured and guided learning experience that encouraged independent thinking and active engagement (P1-3, P5-13, P15-19, P22-24). Unlike GPT, which often required significant effort in crafting effective prompts and could overwhelm users with complete answers upfront, DBox broke down problems into steps, requiring users to think through each step, which fostered a better understanding and stronger coding habits (P2-4, P 5-9, P15, P17, P19, P20, P23). Participants appreciated DBox’s stability and reliability, noting that it offered more consistent and trustworthy feedback, in contrast to ChatGPT, which could occasionally be inaccurate or less dependable (P3, P17, P19). DBox also enhanced the sense of accomplishment, as solving problems incrementally felt more rewarding than simply receiving answers from ChatGPT (P7-10, P18, P23). The tool’s structured interaction design was seen as conducive to systematically learning complex concepts, while ChatGPT's unstructured approach made it harder to follow (P1, P4, P10-12, P22). Although ChatGPT was recognized for its versatility and ability to handle a wide range of tasks, this flexibility often required more effort to manage, making DBox a better fit for learning and problem-solving tasks (P1, P10, P21). Participants also noted that ChatGPT’s interactivity was limited, as it struggled to provide targeted advice on specific steps, whereas DBox offered a more focused and interactive learning experience (P1, P3, P5-7, P9-10, P17).

% For those who used Copilot in the baseline condition, many expressed concerns about its negative impacts on learning, viewing it more as a productivity tool than an educational aid. Participants like P15, P19, and P23 found Copilot intrusive and frustrating, as it often provided suggestions that hindered independent thinking and learning by offering too much unsolicited help. They noted that it encouraged laziness, similar to ChatGPT, but required even less thought (P3, P11, P15, P17). While Copilot was seen as useful for reducing repetitive work and improving efficiency in development contexts where quick code completion was needed (P1, P15, P16, P21), it lacked the ability to guide users through the thought process, making it less suitable for learning and more appropriate for productivity tasks (P17, P20, P24). Some participants, like P21 and P24, acknowledged its efficiency in writing code and reducing syntax errors, but criticized it for not helping with understanding or developing problem-solving skills. Overall, Copilot was viewed as enhancing coding efficiency but unsuitable for learning or teaching, as it did not promote deep thinking or the development of coding skills.




\subsection{How do learners interact with Decomposition Box and perceive the usefulness of different features?}
\label{actual_use}

\subsubsection{Learners' Overall Usage Patterns}

During the user study, we tracked participants' interactions with DBox, focusing on key actions such as code editing, step tree editing, help-seeking, and five main button clicks (e.g., From Editor to Step Tree, Check Step Tree, Check Match, Copy to Comments, and Run Code). These interactions are visualized in Figure \ref{usage}, and we analyzed the patterns in combination with interview data.

Students adopted varying approaches when using DBox. Eleven began by constructing the step tree interactively, while thirteen started by writing code directly. We identified three distinct usage patterns:

\colorbox{color2}{\textbf{Type 1: Building the step tree before writing code}.} Some participants (P6, 8, 9, 14, 16, 18, and 21) focused on constructing the step tree first, iteratively checking and refining it before moving on to code implementation. This approach enabled them to write code efficiently once the structure was finalized. As P21 noted, ``\emph{I used natural language to express my thoughts and verify them, and after a few iterations, I recognized valid ideas and wrote the code myself.}'' P6 added, ``\emph{Writing code from scratch is more challenging for complex problems, so I prefer starting with the step tree.}''

\colorbox{color3}{\textbf{Type 2: Using DBox for verification}.} Some participants (P7, 12, 17, 22, and 24) used DBox primarily to verify their code. They wrote code first, then used the ``From Editor to Step Tree'' feature to check correctness and get hints. P17 explained, ``\emph{I usually solve problems by writing code first. Describing each step in natural language doesn’t feel natural for me.}'' Similarly, P22 stated, ``\emph{I know the general direction, so I write code first and use the tool to verify correctness or catch edge cases.}''

\colorbox{color1}{\textbf{Type 3: Flexible switching between the two modes.}} Students like P1-5, 10, 11, 13, 15, 19, 20, and 23 alternated between coding and step tree construction, adjusting their approach based on confidence, familiarity with specific steps, and real-time coding challenges. For example, P19 stated, ``\emph{If I'm confident in certain steps, I code first and then convert it to steps for verification. If unsure, I verify my thought process before coding.}'' P2 added, ``\emph{For familiar problems, I code first and refine it with the step tree. For new problems, I outline my thoughts and break down the steps to ensure accuracy before coding.}'' P10 shared, ``\emph{Initially, I felt confident, but when I got stuck, I refined my understanding of a step in the step tree before continuing with the code.}''





\begin{figure*}[htbp]
	\centering 
	\includegraphics[width=\linewidth]{figures/finalsurvey3.pdf}
	\caption{Participants rated the features of DBox based on their firsthand experience during the experiment. In the questionnaire, they provided feedback from a first-person perspective (e.g., ``I think [feature] is useful'', rated on a 7-point Likert scale).}
	\label{finalsurvey}
        \Description{}
\end{figure*}


\subsubsection{Hint Usage and Problem-Solving Approach}
\label{hintusage}
\ms{We tracked hint usage frequency among all 24 participants, recording a total of 164 hints triggered. Only 32 instances (19.5\%) involved the ``reveal sub-step'' feature, showing that students mainly relied on simpler hints and did not exploit the system by repeatedly making errors. This feature reveals only one sub-step from the user's incorrect or missing step—leaving the rest for them to solve independently—and can only be triggered after repeated struggle. This approach maximally preserves the students' independent problem-solving process. It strikes a balance between fostering independent thinking and preventing students from becoming permanently stuck, which could otherwise lead to frustration or a loss of motivation to learn.

We did a qualitative analysis which revealed varied problem-solving approaches adopted by participants in the problem-solving processes. For the ``Can Jump'' problem, tackled by 12 participants, we observed three approaches: Greedy (7 participants), Dynamic Programming (3), and Recursion (2). The other 12 participants addressed the ``Search in Rotated Sorted Array'' problem, using Binary Search (8), Two Pointers (2), Binary Tree (1), and Divide and Conquer (1). We observed that participants using the same approach still exhibited differences in reasoning and coding styles. Despite these variations, DBox effectively adapted its support to align with their individual styles and reasoning processes.
}



% \subsubsection{Learners' Usage Patterns}

% During the user study, we logged participants' interactions with DBox, focusing on key actions such as code editing, step tree editing, help-seeking, and five types of button clicks (e.g., From Editor to Step Tree, Check Step Tree, Check Match, From Step Tree to Comment, and Run Code). These interactions are visualized in Figure \ref{usage}. We then analyzed their usage patterns in combination with the interview data.

% Overall, students demonstrated varying approaches in the initial stages of using the tool. Eleven students began by interactively constructing the step tree (describing their thought process in natural language), while thirteen students started directly by writing code. We identified three distinct types of system usage throughout this process:

% \textbf{Type 1: Building the step tree before writing code} (indicated by a green background). Some students, such as P6, P8, P9, P14, P16, P18, and P21, primarily focused on constructing the step tree. They frequently checked the status of the step tree and refined it based on feedback. Only in the final steps did they begin writing code. These students typically invested significant effort into building the step tree, allowing them to quickly write the corresponding code once the structure was complete. For example, P21 mentioned, "At first, I had no clear idea, so I used natural language to explore my thoughts and see if they were correct. After a few iterations, I started recognizing valid ideas and stopped relying on the tool, trying to write the code myself." P6 added, "I prefer to write down my thoughts first and then check the steps. Writing code from scratch is more challenging for complex problems."

% \textbf{Type 2: Using DBox for verification purposes} (indicated by a blue background). Other students, such as P7, P12, P17, P22, and P24, primarily used DBox to verify the correctness of their code. Their main interactions took place in the code editor, where they wrote code first and then used the "From Editor to Step Tree" feature to generate the step tree, focusing on checking correctness and receiving hints for each step. As P17 explained, "I usually start by writing code to solve problems. Describing my thought process in natural language for each step isn’t natural for me—I'm more comfortable using code for step-by-step thinking." Similarly, P22 noted, "I know the general direction and key points of this problem, so I write the code first and use the tool to check if it's correct or if I’ve missed any edge cases."

% \textbf{Type 3: Flexible switching between coding and step tree building} (indicated by an orange background). Some students, such as P1-P5, P10, P11, P13, P15, P19, P20, and P23, demonstrated a flexible approach by switching between code editing and step tree construction as needed. They adapted their interaction style based on the problem at hand, seamlessly transitioning between both modes to support their problem-solving process. For example, P13 initially tried writing code but, after encountering issues, switched to building the step tree, later returning to coding after refining the steps. Some students noted that their use of DBox depended on their confidence in the solution and the completeness of their thought process. P19 mentioned, "If I’m confident, I’ll write the code first and then convert it into steps to check. If I’m unsure, I do the reverse—I check my thought process first and then write the code." Others believed that their approach depended on their familiarity with the problem. As P2 explained, "If it’s a problem I’m familiar with, I stick to my usual pattern of writing code first, converting it into steps, and then refining and checking. For new problems, I start by outlining my thoughts, gradually breaking down the steps to ensure accuracy before moving on to coding."


% Through our contextual inquiry and interviews, we analyzed the usage patterns of the students.





% \subsubsection{Learners' Usage Pattern of ChatGPT}
% The user responses reveal a range of interactions with GPT, highlighting both positive and negative aspects of its use in learning and problem-solving. Many users rely on a direct copy-paste approach, either using GPT's output as their final solution or comparing it with their own work (P3, P4, P15, P17, P19, P20, P21, P23, P24). Some users, however, take a more reflective approach by verifying GPT's output against their own understanding, which can increase their confidence (P2, P3, P15, P16, P23). Despite this, there is a noticeable tendency for users to experience a decrease in independent thinking and a reduced sense of accomplishment, as they often bypass deeper engagement in favor of quicker solutions provided by GPT (P2, P15, P24). This reliance can lead to mental laziness, as users might become less inclined to solve problems on their own (P2, P7, P8, P10, P14). On the other hand, some users actively seek to understand GPT's logic and use it to enhance their learning, particularly when GPT offers step-by-step explanations (P1, P16, P18). However, challenges remain, such as the inconsistency in the level of detail provided by GPT in teaching scenarios (P17), the difficulty in crafting effective prompts (P1, P24), and the overall complexity of the learning process, which can still be cumbersome even with GPT’s assistance (P1).




% \subsubsection{Learners' Reactions to System Errors}
% As noted in our technical evaluation, the backend of DBox, powered by a GPT-4 model, is not flawless and occasionally makes judgment errors. For instance, it might incorrectly mark a correct step as wrong or vice versa. We were particularly interested in exploring whether learners could identify these AI errors and how these mistakes influenced their use of DBox. We also wanted to understand their perceptions and responses to such errors. Among our 24 participants, 16 encountered instances where the system made incorrect judgments. Based on their feedback, we identified several key insights.

% First, within this learning context, most students were able to recognize system errors after one or more attempts. While they might sometimes struggle to generate new ideas, they possess the ability to verify existing ones. Moreover, because DBox’s role is primarily evaluative—assessing the code or step trees constructed by the students rather than generating or recommending content—students had a certain degree of confidence in their own work. This reduced the likelihood of over-reliance on the system, a common issue with traditional recommendation systems. As a result, system errors generally did not have a significant impact.

% However, there were some negative effects. For example, when a correct step was wrongly judged as incorrect, students had to revisit their work, which could disrupt their thought process. In some cases, they made several revisions before realizing the error was with the system, not their solution. The extent to which these errors affected students depended largely on their confidence level. When students were uncertain, they were more easily influenced by the system. As P15 noted, ``\emph{If I know what the hint is saying and I know it's wrong, I just ignore the message. But if I don't fully understand it and I am not sure, I might try following the hint.}''

% Students responded to these errors in a variety of ways. The first common response was to \textbf{ignore the error}. Five participants mentioned that when they recognized the system had misjudged a step, they simply moved on. The second approach was to \textbf{focus on the content of the hint rather than the status indicator}. Even when the system's judgment was wrong, the hints could still provide useful guidance. As P4 explained, ``\emph{I focus on what the hint is trying to say. If it aligns with what I wrote, I ignore the error and continue using the hint for guidance, disregarding the status indicator.}'' The third response involved \textbf{running the code to verify the system’s judgment}. As P2 shared, ``\emph{I run the code. If I'm correct, I stick to my approach and ignore the system. If I'm wrong, I reconsider whether the system’s judgment might be valid.}'' The fourth response was to \textbf{recheck the step}. For example, P3 said, ``\emph{I rethink my approach and evaluate it again. Even if the system’s judgment is wrong, the process helps me eliminate potential errors.}'' Similarly, P23 noted, ``\emph{I get confused and recheck a few times. If it's still wrong, I just move on and continue writing my own code.}''

% Overall, while system errors cannot be entirely avoided at the current stage, the design of our system enables learners to quickly recognize when an error may have occurred. Additionally, learners demonstrated flexibility in employing various strategies to address these errors. Future work should focus on minimizing the impact of system errors on the learning process.



\subsubsection{Learners' Reactions to System Errors}
\ms{DBox, powered by the GPT-4o model, occasionally misjudged step statuses. During our user study, 24 participants triggered the ``check'' function (``Check Step Tree'' and ``From Editor to Step Tree'') 208 times, with 16 participants encountering 18 system errors (an 8.7\% error rate). Fourteen participants faced one error each, while two experienced two errors. These errors can be categorized into four types:

\begin{itemize}[leftmargin=0em] % Adjust left margin for a cleaner look

    \item \textbf{Type-1 (11 occurrences):} Misjudging correct steps as incorrect.  
    \begin{itemize}
        \item \textbf{Example:} A participant using a greedy approach stated, 
        \textit{``Use a greedy approach to minimize jumps.''}  
        GPT flagged this as incorrect due to insufficient detail on range expansion and jump counter updates. However, missing details does not necessarily mean the solution is incorrect.
    \end{itemize}

    \item \textbf{Type-2 (2 occurrences):} Flagging unnecessary steps as missing.  
    \begin{itemize}
        \item \textbf{Example:} GPT incorrectly required a check for single-element arrays, though the solution worked without it.  
    \end{itemize}

    \item \textbf{Type-3 (3 occurrences):} Overlooking subtle mistakes in seemingly correct steps.  
    \begin{itemize}
        \item \textbf{Example:} A participant's wrote a step ``Iterate through the array. If at any index \texttt{i}, \texttt{maxReach < i}, return false'', leading to errors with unreachable indices.  
        GPT failed to detect this error.  
    \end{itemize}

    \item \textbf{Type-4 (2 occurrences):} Missing crucial steps while DBox marking solutions as complete.  
    \begin{itemize}
        \item \textbf{Example:} A participant omitted a final return statement (\texttt{return true}), GPT still judged the solution as correct.  
    \end{itemize}

\end{itemize}
}






% DBox, powered by the GPT-4o model, occasionally misjudges the correctness of steps. Among the 24 participants, 16 experienced minor or significant system errors during the study。在遇到过system error的参与者中,其中14参与者只遇到了一次错误,2个参与者遇到两次错误。经过定性分析,在这全部18次错误中,第一种错误(11次)是把参与者写的一个正确的步骤误判为错误,第二种错误(2次)是误判参与者有missing step(实际上步骤是完整的),第三种错误(3次)是把参与者写的一个错误的步骤判断为正确,第四种错误(2次)是误判参与者步骤是完整的(实际上缺少一个步骤导致不能通过测试)。经过定性分析,我们发现,在11次第一种错误中,有8次是因为参与者对步骤的描述过于简单缺乏关键信息,有3次是因为参与者步骤中所使用的解题方法比较小众,不是所谓的常见解法。在2次Type 2错误中,GPT认为参与者缺少了一个步骤,但实际上这个步骤是不必要的。在3次Type 3错误中,参与者对步骤的描述中大部分是正确的,隐藏着小错误,GPT没有识别到。在2次Type 4错误中,参与者没有写处理函数返回值的步骤,但是GPT没有识别出来。





We then analyzed whether participants successfully identified the system errors and how they reacted to the errors. We found that all the 16 students who encountered system errors recognized these errors after one or more attempts. Since DBox evaluates rather than generates content, students remained confident in verifying their own work, which minimized over-reliance on the system. However, incorrect judgments, particularly when correct steps were flagged as wrong, could disrupt their thought process. The impact of these errors largely depended on the student's confidence. As P15 noted, “\emph{If I know it's wrong, I ignore the message. If I'm unsure, I might follow the hint.}''

Students adopted various strategies to deal with system errors. (1) \textbf{Ignoring the error} (8 participants): Some simply moved on after recognizing a misjudgment. (2) \textbf{Focusing on the hint, not the status} (6 participants): Even if the status evaluation was incorrect, participants often found the hints useful. \ms{For example, participants wrote very brief steps, which GPT flagged as incorrect. The hints prompted them to consider important details, and despite the incorrect judgment, participants found the hints very useful.} (3) \textbf{Running the code to verify} (12 participants): Many used the ``Run'' button to test their hypotheses. As P2 explained, ``\emph{If my code works, I stick with my approach. If it fails, I reconsider the system's judgment.}'' (4) \textbf{Rechecking the step} (10 participants): Some participants revisited a step to refresh its status. As P3 mentioned, ``\emph{Even if the system is wrong, rechecking helps me identify potential issues.}''

Though system errors are unavoidable at the current stage, DBox’s design allows learners to quickly recognize and manage them. Future improvements should aim to minimize disruptions caused by system inaccuracies.


% As noted in our technical evaluation, DBox's backend, powered by the GPT-4o model, is not flawless and occasionally makes errors in judgment, such as incorrectly marking a correct step as wrong or vice versa. We were particularly interested in how learners identified these AI errors, how they influenced DBox's use, and how students responded to them. Among our 24 participants, 16 encountered system errors, providing several key insights.

% Most students were able to recognize system errors after one or more attempts. While they sometimes struggled to generate new ideas, they were generally confident in verifying their existing work. Since DBox primarily evaluates rather than generates content, students maintained a level of trust in their own solutions, reducing over-reliance on the system—an issue common with recommendation tools. As a result, system errors typically did not significantly impact their learning experience.

% However, incorrect judgments could disrupt students' thought processes, particularly when correct steps were wrongly flagged as incorrect. In some cases, students made unnecessary revisions before realizing the error was with the system, not their solution. The extent of disruption depended on their confidence. As P15 noted, ``\emph{If I know what the hint is saying and I know it's wrong, I just ignore the message. But if I don't fully understand it and I am not sure, I might try following the hint.}''

% Students responded to system errors in various ways. The most common responses included:

% 1. \textbf{Ignoring the error}. Five participants mentioned simply moving on when they recognized a misjudgment.
% 2. \textbf{Focusing on the hint, not the status}. Even when the system’s judgment was wrong, the hints were often useful. As P4 explained, ``\emph{I focus on the hint and ignore the error if it aligns with my approach.}''
% 3. \textbf{Running the code to verify the system’s judgment}. As P2 shared, ``\emph{If my code runs correctly, I trust my approach and ignore the system. If it fails, I reconsider the system’s judgment.}''
% 4. \textbf{Rechecking the step}. P3 noted, ``\emph{I rethink my approach and recheck the step. Even if the system is wrong, the process helps me clarify potential errors.}''

% While system errors are inevitable at this stage, DBox's design allows learners to recognize and address them quickly. Participants demonstrated flexibility in employing strategies to mitigate the impact of these errors. Future work should focus on minimizing the disruption caused by system inaccuracies.







\subsubsection{Learners' Ratings of Different Features of DBox}
DBox offers a range of features designed to enhance algorithmic programming learning, most of which participants found helpful, as illustrated in Figure \ref{finalsurvey}. Students particularly appreciated the flexibility to either write code directly or build a step tree, and they valued DBox's ability to infer their thought process from the code. The interactive step tree and the fine-grained correctness assessment were also well-received, although two participants found it somewhat cumbersome. The progressive, multi-level hint system was praised by 22 participants. Additionally, the feature that checks the alignment between the step tree and the code implementation was seen as highly beneficial. However, some participants felt that the ability to paste the step tree as comments into the editor was unnecessary, since the step tree and code editor could already be viewed side by side. This feature could be even more useful if DBox were developed as a plugin in the future.

% Interviews revealed additional benefits. Participants appreciated the structured step-by-step guidance (P1-4, P6-8, P10, P12, P15, P16, P18, P19, P23) and found breaking problems into smaller steps helped identify issues in logic or code (P1-6, P7-12, P15, P16, P22, P24). This method also improved problem-solving habits, such as commenting and step planning (P19). Targeted hints effectively guided thought processes without overwhelming users (P3-5, P17, P20, P21, P23). Participants liked how DBox customized feedback based on progress, making learning more efficient and personalized (P4, P13, P14, P20, P21, P24). Overall, DBox supported logical thinking by providing incremental guidance and encouraging independent problem-solving.




% DBox includes several features designed to enhance programming learning, and we wanted to understand which ones students found most helpful. Figure \ref{finalsurvey} shows participants' ratings for each feature. Overall, most features were considered useful. Participants especially appreciated DBox's flexibility, allowing them to solve problems either by writing code or constructing a step tree. They also valued the system's ability to deduce their thought process from the code and provide relevant feedback. The interactive step tree and its correctness assessment were well-received, though two participants found working with the step tree cumbersome. Additionally, 22 participants praised the multi-level hint system. However, many found the feature that pastes the step tree into the editor as comments unnecessary, as they could easily compare the step tree and editor side by side. This feature might become more useful if DBox were developed as a plugin.

% Interviews revealed further benefits. Many participants appreciated the step-by-step guidance, which provided a structured and controlled learning experience (P1-4, P6-8, P10, P12, P15, P16, P18, P19, P23). The ability to break down problems into smaller, manageable steps was highlighted as a key feature, helping users identify specific issues in their logic or code (P1-6, P7-12, P15, P16, P22, P24). This approach made error detection easier and promoted better problem-solving habits, like commenting and step planning (P19). Targeted hints were seen as helpful for guiding thought processes without overwhelming users, allowing them to focus on areas needing improvement (P3-5, P17, P20, P21, P23). Participants also appreciated how the system customized feedback based on their progress and previous inputs, making learning more efficient and personalized (P4, P13, P14, P20, P21, P24). Overall, participants felt DBox effectively supported logical thinking and learning by offering incremental guidance and encouraging independent error correction.







% \subsubsection{Existing Limitations and Suggested Improvements}
% Participants identified several areas for improvement in DBox. Three participants expressed frustration when uncommon solutions were incorrectly flagged as wrong. Speed and accuracy issues with GPT-4's processing were also raised, along with concerns about GPT's misinterpreting of their vague inputs (four participants). Two participants suggested more support for initial steps and examples to help struggling users get started and emphasized the need for logical consistency across steps. Three participants expressed interest in auto-completion features and integrating conversational capabilities for them to ask follow-up questions.




% Participants identified several areas where the tool could be improved:

% \begin{itemize}
%     \item \textbf{Hint Quality and Clarity}: Some participants found the hints confusing or not particularly useful (P3, P15, P17, P19, P22, P23). They suggested making the hints more specific and clearer, with explanations for why certain steps are necessary and how to achieve specific outcomes (P19, P23). More targeted hints that directly address users' needs, rather than offering generic advice, were also recommended (P15, P17).
    
%     \item \textbf{Handling Non-Optimal Solutions}: Participants noted frustration when the tool incorrectly marked non-optimal solutions as wrong (P16). They recommended distinguishing between incorrect and non-optimal solutions and providing hints to guide users toward optimization, rather than flagging the solution as incorrect.

%     \item \textbf{Flexibility in Step Judgment}: Some participants requested more flexibility in the tool's judgment, allowing for different solution paths and thought processes (P19, P21). They wanted the tool to accommodate various approaches rather than adhering to a single correct method.

%     \item \textbf{Speed and Accuracy}: Several participants found the tool slow due to GPT-4’s processing time, which hindered their workflow (P3, P4). Faster processing would improve the overall experience. Additionally, there were concerns about the tool misinterpreting users' inputs, especially with minimal code or vague descriptions (P1, P2, P3, P20).

%     \item \textbf{Support for Initial Steps and Examples}: Some participants, particularly those struggling to start, felt the tool lacked adequate support for getting started (P18, P24). They suggested providing more detailed initial steps and examples to help build momentum.

%     \item \textbf{Logical Consistency and Conceptual Guidance}: Participants expressed a desire for the tool to not only provide step-by-step guidance but also provide a high-level integrated prompt for the entire problem and connect these steps into a coherent, logical framework (P6, P23). They wanted the tool to explain the underlying principles and ensure each step was logically connected.

%     \item \textbf{More Auto-completion}: While the tool focuses on having users create the step tree and write code themselves, some participants (P1, P15, P23) expressed interest in intelligent auto-completion features to help them focus on learning key concepts.

%     \item \textbf{Integrating Conversational Features}: Some participants (P2, P11, P15) wanted the ability to ask follow-up questions or describe difficulties. They suggested integrating a conversational feature where users could interact with GPT based on their current step tree or a specific step—without GPT revealing the answer but guiding them instead.
    
% \end{itemize}















\section{Discussion}
\omniUIST is capable of tracking a passive tool with an accuracy of roughly 6.9 mm and, at the same time, deliver a maximum force of up to 2 N to the tool. This is enabled by our novel gradient-based approach in 3D position reconstruction that accounts for the force exerted by the electromagnet. 

Over extended periods of time, \omniUIST can comfortably produce a force of 0.615 N without the risk of overheating. In our applications, we show that \omniUIST has the potential for a wide range of usage scenarios, specifically to enrich AR and VR interactions.

\omniUIST is, however, not limited to spatial applications. We believe that \omniUIST can be a valuable addition to desktop interfaces, e.g., navigating through video editing tools or gaming. We plan to broaden \omniUIST's usage scenarios in the future.

The overall tracking performance of \omniUIST suffices for interactive applications such as the ones shown in this paper. The accuracy could be improved by adding more Hall sensors, or optimizing their placement further (e.g., placing them on the outer hull of the device).
Furthermore, a spherical tip on the passive tool that more closely resembles the dipole in our magnetic model could further improve \omniUIST's accuracy. We believe, however, that the design of \omniUIST represents a good balance of cost and complexity of manufacturing, and accuracy.

Our current implementation of \omniUIST and the accompanying tracking and actuation algorithms assumes the presence of a single passive tool. Our method, however, potentially generalizes to tracking multiple passive tools by accounting for the presence of multiple permanent magnets. This poses another interesting challenge: the magnets of multiple tools will interact with each other, i.e., attract and repel each other.The electromagnet will also jointly interact with those tools, leading to challenges in terms of computation and convergence. We believe that our gradient-based optimization can account for such interactions and plan to investigate this in the future.

In developing and testing our applications, we found that \omniUIST's current frame rate of 40 Hz suffices for many interactive scenarios. The frame rate is a trade-off between speed and accuracy. In our tests, decreasing the desired accuracy in our optimization doubled the frame rate, while resulting in errors in the 3D position estimation of more than 1 cm, however. Finding the sweet spot for this trade-off depends on the application. While our applications worked well with 40 Hz and the current accuracy, more intricate actions such as high-precision sculpting might benefit from higher frame rates \textit{and} precision.
Reducing the latency of several system components (e.g., sensor latency, convergence time) is another interesting direction of future research. 

Furthermore, the control strategy we used was fairly naïve, as it only takes the current tool position into account. A model predictive strategy could account for future states, user intent, and optimize to reduce heating. We will explore in the next chapter how model predictive approaches can be used for haptic systems.

Overall, the main benefits of \omniUIST lie in the high accuracy and large force it can produce. It does so without mechanically moving parts, which would be subject to wear.
Such wear is not the case for our device, because it is exclusively based on electromagnetic force. We believe that different form factors of \omniUIST (e.g., body-mounted, larger size) can present interesting directions of future research. \add{A body-mounted version could be interesting for VR applications in which the user moves in 3D space. The larger size could result in more discernible points.}

Additionally, the influence of strength on user perception and factors such as just-noticeable-difference will allow us to characterize the benefits and challenges of \omniUIST, and electromagnetic haptic devices in general.
We believe that \omniUIST opens interesting directions for future research in terms of novel devices, and magnetic actuation and tracking.



\begin{acks}
This research was supported by the Dieter Schwarz Stiftung Foundation, ETH Foundation, and in part by the EdUHK-HKUST Joint Centre for Artificial Intelligence (JC\_AI) research scheme: Grant No. FB454.
\end{acks}


\bibliographystyle{ACM-Reference-Format}
\bibliography{sample-base}


\end{document}