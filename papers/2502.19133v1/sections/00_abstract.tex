\begin{abstract}


% Algorithmic programming is a crucial skill that computer science students must master. While there are plenty of supportive tools available to help students when they encounter difficulties—such as search engines, the solution panel on platforms like LeetCode, ChatGPT, GitHub Copilot, and various Q\&A forums—these tools often provide generic answers or excessive help. This can hinder students' independent problem-solving and development of critical thinking skills.

% To address this issue, we first conducted a formative study to identify six challenges students still face in learning algorithmic programming, even with the abundance of supportive tools available today. Based on the design goals derived from these challenges, we developed a novel tool called Decomposition Box (DBox) using scaffolding pedagogy. DBox helps students interactively construct a step tree either by coding or by describing their thought processes in natural language. In this co-creation process, students take the lead, while DBox provides personalized and adaptive assistance at the step level only when necessary. Specifically, DBox offers fine-grained guidance and feedback throughout both the idea formation and implementation stages, and it scaffolds the problem-solving process with multi-level hints.

% Through a user study involving 24 participants, we found that DBox led to significantly higher learning gains compared to existing supportive tools (baseline). Students demonstrated stronger cognitive engagement and critical thinking, felt a greater sense of achievement in solving problems, and perceived the assistance provided by DBox to be more appropriate and useful for their learning. We also explored the impact of DBox on students' cognitive load and user experience. Additionally, we summarized students' usage patterns and their strategies for dealing with system errors.

% Based on our key findings, we offer in-depth discussions on how LLMs can assist students in learning and provide design recommendations for future tools aimed at supporting algorithmic programming education.


% Decomposition is a fundamental skill in computational thinking and algorithmic programming, involving breaking down a problem into smaller, manageable parts.
% Current self-study methods, such as browsing reference solutions or using LLM assistants, often provide excessive help or generic answers that fail to align with learners' decomposition strategies. This over-assistance can impede independent problem-solving and critical thinking. To address this, we introduce Decomposition Box (DBox), an LLM-based interactive system that provides the right amount of assistance, encouraging students to develop their own decomposition processes. A within-subjects study (N=24) revealed that DBox led to significantly higher learning gains, cognitive engagement, and critical thinking. Learners also reported a stronger sense of achievement and found the assistance appropriate and useful for learning. Additionally, we analyzed DBox's impact on learners' cognitive load, identified usage patterns, and examined strategies for managing system errors. We conclude with design recommendations for future tools to support algorithmic programming education.



% Decomposition is a fundamental skill for learners in algorithmic programming, requiring the breakdown of complex problems into smaller, manageable parts. However, current self-study methods, such as browsing reference solutions or using LLM assistants, often provide excessive or generic assistance that does not align with learners' decomposition strategies. This misaligned support can hinder independent problem-solving and critical thinking. To address this, we introduce Decomposition Box (DBox), an LLM-based interactive system that scaffolds learners' construction of a step tree by offering just the right level of assistance and promoting independent thinking. A within-subjects study (N=24) revealed that DBox significantly enhanced learning gains, cognitive engagement, and critical thinking. Learners also reported a stronger sense of achievement and found the assistance appropriate and useful for learning. Additionally, we examined DBox's impact on cognitive load, identified usage patterns, and explored strategies for managing system errors. We conclude with design implications for future tools to better support algorithmic programming education.

Decomposition is a fundamental skill in algorithmic programming, requiring learners to break down complex problems into smaller, manageable parts. However, current self-study methods, such as browsing reference solutions or using LLM assistants, often provide excessive or generic assistance that misaligns with learners' decomposition strategies, hindering independent problem-solving and critical thinking. To address this, we introduce the Decomposition Box (DBox), an interactive LLM-based system that scaffolds and adapts to learners' personalized construction of a step tree through a ``learner-LLM co-decomposition'' approach, providing tailored support at an appropriate level. A within-subjects study (N=24) showed that DBox significantly improved learning gains, cognitive engagement, and critical thinking. Learners also reported a stronger sense of achievement and found the assistance appropriate and helpful for learning. Additionally, we examined DBox's impact on cognitive load, identified usage patterns, and analyzed learners' strategies for managing system errors. We conclude with design implications for future AI-empowered tools to better support algorithmic programming education.

\end{abstract}