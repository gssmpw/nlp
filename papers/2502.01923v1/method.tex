\section{Method}
\label{method}

\subsection{Context of Study}
We conducted this research in the context of SENG302, a 300-level two-semester software engineering group project at the University of Canterbury. Students work in teams of 6-8 members to develop a software application while following the Scrum framework. SENG302 is divided into seven Sprints, which are each 2-3 term weeks long, although some sprints span term breaks. Students attend regular formal sessions, including the Sprint Planning, Daily Scrums (which occur bi-weekly in this case), Sprint Review, demonstration, and Sprint Retrospective sessions. In 2023, students submitted peer-feedback for each of their teammates and a self-reflection each Sprint. These discussed what each person did well, what they could do better, and actions they could take to improve. Peer-feedback also included ratings on a Likert scale from one to five for each teammates' contribution towards the project through skills like testing and communication. Communicating technical knowledge and opinions, and with teammates in a professional environment is included in the course learning outcomes, so students are expected to improve this skill during the year.

In 2023, there were 77 students enrolled in the course, who were split into 10 teams. 
% 19 students identified as women, and 58 identified as men. Students were between 19 and 31 years old, with most of the cohort 20 or 21. 
All students had prior knowledge of programming concepts, including introductory software engineering and databases courses.

% 47 students were enrolled in software engineering, while 30 were enrolled in computer science degrees. 

To apply the analysis techniques we identified from prior work to this cohort, we collected data from four sources. From Slack, we collected messages sent in public channels in each team's workspace. We retrieved code contribution and merge request (MR) data from a locally hosted instance of GitLab. We collected work logs and peer-feedback from the project management tool, Scrumboard~\cite{minish}, which is developed locally. 

Finally, we took teams' Sprint results in the form of delivered story points per sprint, and the overall team performance score following the course marking schedule, e.g., knowledge of Scrum, quality of code, documentation.

%We wrote Python scripts for data processing with Pandas, NetworkX, SciPy, and matplotlib. We also developed network visualisations in the project management tool.

We received ethics approval from the University of Canterbury Human Research Ethics Committee to undertake this research.


\subsection{Socio-Technical Congruence}
We followed the method for calculating STC scores as proposed by Cataldo et al.~\cite{cataldoIdentificationCoordinationRequirements2006}, illustrated in Fig.~\ref{fig:socio-tech-flowchart}. 

\begin{enumerate}
\item The task assignments matrix maps students to the MRs they committed to. We retrieved students' commit histories, and the commits in each MR, and combined these to construct this matrix.

\item The task dependencies matrix maps each MR to other MRs that have any files in common. %We omitted MRs from the test branch to the production branch, as they would add noise to the dependencies. While including these MRs would ensure we consider commits made directly to the test branch, these are rare, and these MRs typically contain entire stories or multiple stories of functionality. Including this many commits in the same MR would create task dependencies between all files, leading to every team member having a coordination requirement with every other team member, which would be less accurate.

\item We combined the task assignments and task dependencies matrices to produce a coordination requirements matrix, which reflects which teammates a student is expected to need to communicate with in a given week. For example, if student X contributed to MR X, and student Y contributed to MR Y, and MR X and MR Y have files in common, then student X and student Y have a need to coordinate.

\item We used communication utterances from public channels in each team's Slack workspace to construct the actual coordination matrix. We considered communication to have occurred from student X to student Y when student X sent a message in a thread started by student Y. 

\item We calculated each students' STC score for each week based on the communication requirements matrix and actual communication matrix. For each student, we created a list where each value represents whether they communicated with another teammate as expected or not. Each value is one if there was a communication requirement that was fulfilled, zero if there was a communication requirement that was not fulfilled, and null if there was no communication requirement. A student's STC score for the week is the mean of these values. Finally, we obtained each team's mean STC score for the year by taking the mean of the team members' STC scores for that week, excluding null values.
\end{enumerate}
% We retrieved students' MRs and commits via the GitLab API, where the repositories were hosted. We used these data to construct the task dependency matrix, which represents the dependencies between MRs. 

% Then, we used students' commit histories to construct a matrix containing MRs each student contributed commits to. We omit MRs from the test branch to the production branch, as they would add noise to the task dependencies. Including these MRs would ensure we consider commits made directly to the test branch. However, all commits being included in the same MR would create task dependencies between all files, leading to every team member having a coordination requirement with every other team member. 

% We combined these two matrices to produce a coordination requirements matrix, which reflects which teammates a student is expected to need to communicate with in a given week. For example, if student X contributed to MR X, and student Y contributed to MR Y, and MR X and MR Y have files in common, then student X and student Y have a need to coordinate. 

% Following this, we retrieved communication utterances from an export of public channels in each team's Slack workspace, and used these utterances to construct the matrix that represents actual communication between students. We considered communication to have occurred from student X to student Y when student X sent a message in a thread started by student Y. Finally, we calculated each students' congruence score for each week based on the communication requirements matrix and actual communication matrix. For each student, we created a list where each value represents whether they communicated with another teammate as expected or not. Each value is one if there was a communication requirement that was fulfilled, zero if there was a communication requirement that was not fulfilled, and null if there was no communication requirement. A student's congruence score for the week is the mean of these values. Finally, we obtained each team's mean congruence score for the year by taking the mean of the team members' congruence scores for that week, excluding null values.

%First, we omit MRs from dev to main, as the benefit of including them would be outweighed by the noise they would add to the task dependencies. Including these commits would ensure we consider commits made directly to dev. However, all commits being included in the same MR would create task dependencies between all files, leading to every team member having a coordination requirement with every other team member.

\begin{figure}
    \centering
    \includegraphics[width=1\linewidth]{figures/socio_tech_flowchart.png}
    \caption{The process for calculating socio-technical congruence}
    \label{fig:socio-tech-flowchart}
\end{figure}

\subsection{Triad Census}
We constructed undirected networks for each team, with each network representing their written communication on Slack during a given Sprint. Each node in a network is a student, and each edge represents communication from one student to another. If student X sent a message in a thread created by student Y for the given Sprint, then there is an edge between student X and student Y. We then found the triad census for each week and Sprint, which involved finding the number of each triad type in the network for the given time period. In an undirected network, there are four possible triad types, which are determined by the number of edges between nodes in the triad: zero, one, two, or three edges. For example, the network in Fig.~\ref{fig:census_example} has four nodes, and four distinct triads: ABC, ABD, ACD, and BCD. These triads have one, two, three, and two edges, respectively, so there are zero triads with zero edges, one triad with one edge, two triads with two edges, and one triad with three edges. Therefore, the triad census is (0, 1, 2, 1).

\begin{figure}
    \centering
    \includegraphics[width=0.4\linewidth]{figures/census_example.png}
    \caption{An example network}
    \label{fig:census_example}
\end{figure}

As teams were different sizes, the nominal count of triads differed between teams' networks, so we converted the triad census to the relative triad census. We did this by finding the relative frequency of each triad type for each Sprint, and the mean relative frequency of each type of triad for each week of each Sprint. For our example in Fig.~\ref{fig:census_example}, the relative frequency of triads in the census (0, 1, 2, 1) is (0, 0.25, 0.5, 0.25).

We then calculated the Pearson correlation for each team's triad census for each Sprint with the percentage of committed story points the team passed in that Sprint, as well as the teams' performance score.
