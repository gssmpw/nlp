\section{Introduction}
Communication and balanced contributions between team members are widely recognised as key components of effective teamwork, and are crucial for the success of software engineering teams~\cite{curtisFieldStudySoftware1988,hoeglTeamworkQualitySuccess2001, mensahImpactTeamworkQuality2024}. However, teams are often ineffective at these skills~\cite{krautCoordinationSoftwareDevelopment1995}, leading to reduced productivity and effectiveness~\cite{r.noelExploringCollaborativeWriting2018,strodeTeamworkEffectivenessModel2022a}, which is detrimental for project success. In particular, teams identify a lack of knowledge sharing within the team and communication breakdowns as sources of issues, especially in times of stress~\cite{salasHowCanYou1997}.

There are often differences between members' participation in team discussions. More vocal team members tend to dominate discussions, causing the team to miss knowledge from quieter team members that could be used to get the best solutions~\cite{hoeglTeamworkQualitySuccess2001}. Knowledgeable team members who are also skilled communicators tend to become ``communication focal points'', and gather knowledge through the process of educating their teams about the application domain~\cite{omalleyAnalysisSocialNetworks2008}. These team members are recognised by their teammates as being exceptional contributors to their teams' success~\cite{curtisFieldStudySoftware1988}. However, this behaviour leads to reduced performance at the team level~\cite{sparroweSocialNetworksPerformance2001}. In well performing teams, members tend to know other members' expertise~\cite{r.noelExploringCollaborativeWriting2018}, and so are able to coordinate knowledge without depending on these ``communication focal points'' to broker information. These teams with better coordination skills show not only better performance, but also higher team member satisfaction than their counterparts~\cite{mensahImpactTeamworkQuality2024,seersTeammemberExchangeQuality1989}, indicating that improving communication and coordination in teams provides benefits at both the individual and team levels. 

To identify communication behaviours that can negatively affect team productivity, we can use social network analysis~\cite{otteSocialNetworkAnalysis2002}. Social networks represent relationships between entities, where each node in the network is an entity, and each edge between two nodes represents a relationship between them. Through analysing the characteristics of these networks and identifying communication patterns that occur in less and more productive teams, we can attempt to address negative patterns and foster positive patterns. 

While there is extensive research on the characteristics of effective teams and the importance of communication~\cite{mathieuTeamEffectiveness199720072008}, to our knowledge there is little work regarding \textit{how} we can improve social networks in team discussions based on social network analysis. We propose that social network analysis can be used to identify areas of improvement in teams' communication, and these improvement areas can inform interventions that help teams strengthen their communication, leading to better performance. In the context of software engineering education, this is valuable for helping students identify weaknesses and track their progress. It may also provide educators with early indicators of teams that need additional support to develop strong communication.  
% However, Acuña et al.~\cite{acunaHowPersonalityTeam2009} indicate that task allocation can be used to increase performance and team satisfaction. We propose that social network analysis can be used to identify areas of improvement in student teams' communication, and these improvement areas can inform task allocation interventions that improve teams' performance. 


% and we hypothesise that task allocation interventions can be used to nudge team members towards behaviours that strengthen team communication and coordination. Such behaviours may include encouraging quieter team members to contribute to key product features so they have knowledge that needs to be shared with the rest of the team, or improving their self-confidence in their role and technical knowledge through pair programming with each of their teammates~\cite{hughesRemotePairProgramming2020,mcdowellPairProgrammingImproves2006}, so contributing to team discussions is less daunting and they have more opportunities to engage in transferring knowledge.

% We aim to assist teams in developing communication networks that foster knowledge sharing and effective coordination. To do this, we will first identify areas for improvement in communication between team members. Then, we will employ task allocation interventions to encourage behaviours that improve these areas. We will conduct this research in the context of an Agile software engineering group project course, with the intended outcome of improved intra-team communication and coordination. An immediate benefit of the proposed work will be improving education in software engineering group projects through targeting communication and teamwork-related learning outcomes.

In this paper, we first present prior work that informs our analysis of team communication in Section~\ref{rel_work}, followed by the method we used to apply these analysis techniques in Section~\ref{method}. In Section~\ref{results}, we discuss our results, and in Section~\ref{sec:discussion}, we explain the implications and limitations of our findings. Finally, Section~\ref{conclusion} concludes and discusses applications of the results in future work.