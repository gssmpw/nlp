\begin{abstract}
Students working in teams in software engineering group project often communicate ineffectively, which reduces the quality of deliverables, and is therefore detrimental for project success. An important step towards addressing areas of improvement is identifying which changes to communication will improve team performance the most. We applied two different communication analysis techniques, triad census and socio-technical congruence, to data gathered from a two-semester software engineering group project. Triad census uses the presence of edges between groups of three nodes as a measure of network structure, while socio-technical congruence compares the fit of a team's communication to their technical dependencies. Our findings suggest that each team's triad census for a given sprint is promising as a predictor of the percentage of story points they pass, which is closely linked to project success. Meanwhile, socio-technical congruence is inadequate as the sole metric for predicting project success in this context. We discuss these findings, and their potential applications improve communication in a software engineering group project.
\end{abstract}
