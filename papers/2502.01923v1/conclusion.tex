\section{Conclusions and Future Work}
\label{conclusion}
% Notably, there is a moderate negative correlation between the frequency of triads with one edge and the score the team received for the Sprint, indicating that teams with many people communicating in pairs and not engaging in threads with others in the team tend to receive worse scores in the team assessment.
We evaluated the suitability of STC and triad census for identifying communication areas of improvement in teams of software engineering students. Our results indicate that triad census can be used to predict team performance, while STC does not indicate performance in this context.

We intend to use these findings to inform the design of features that we will add to Scrumboard to help students identify and improve where they can improve their contributions to the team. We will then evaluate the effects of these features in the context of SENG302. To do this, we will first implement a set of features that assist students in reflecting on their contributions, like a visualisation of their teams' social network, or their work consistency. Then, we will run a brief pilot study with past students, and improve the features based on their feedback. Finally, we will deploy the features for semester two of the course, and analyse the effects on students and their delivered products compared to previous cohorts, considering students' characteristics and interactions with the features. 

\section{Data Availability}
We have provided anonymised data for the Slack exports, average peer-feedback ratings for each sprint, scores and points for each team by sprint, and total time each team spent pair programming. Note that the peer-feedback ratings omit sprint 1, as we did not use these due to the teams being split into sub-teams.