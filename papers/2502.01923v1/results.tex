\section{Results}
\label{results}
\subsection{Socio-technical congruence}
We plotted each of the teams' mean STC scores for each week of the year, as shown in Fig.~\ref{fig:congruence_over_year}. Sprint 1 is excluded, as teams are split in half and work on separate projects for this initial Sprint for on-boarding purposes. In Fig.~\ref{fig:congruence_over_year}, dotted and solid vertical lines indicate Sprint start and end weeks, respectively. Note that the gap mid-year is due to exam and break period, when students do not work on the project.

\begin{figure}[ht]
    \centering
    \includegraphics[width=1\linewidth]{figures/cong_score_team_100.png}
    \caption{Mean STC scores for a team for each week of the project (Sprint 2 - Sprint 5)}
    \label{fig:congruence_over_year}
\end{figure}

Plotting the line of best fit on each team's graph showed that five out of ten teams' STC scores tended to increase over the year, while the other five decreased over the year. A Mann-Whitney U Test comparing the set of teams whose STC scores increased over the year with those who decreased over the year showed no significant difference in the number of stories passed ($U = 15.00$, $p = .68$). This aligns with MacKeller's findings that the scores of student teams in a semester-long group project did not converge~\cite{b.k.mackellarAnalyzingCoordinationStudents2013}. 

% However, MacKellar found that analysing individual students' congruence scores may be valuable for identifying students who act as knowledge brokers.

We increased the granularity of analysis by then taking the mean STC score for each team for each Sprint. The Pearson correlation coefficients for STC scores with the percentage of story points teams passed each Sprint mean rating team members gave their peers for communication in the peer-feedback for that Sprint, are shown in Table~\ref{tbl:pearson}. 

\begin{table}
    \caption{Pearson Correlation Coefficients}
    \begin{center}
         \label{tbl:pearson}
        \begin{tabular}{|p{0.31\columnwidth} | p{0.3\columnwidth} | p{0.22\columnwidth}|} 
            \hline &\textbf{Mean Sprint STC Score}&\textbf{\% Story Points Passed}
            \\ \hline
            \textbf{\% Story Points Passed} & 0.109 & -
            \\ \hline
            \textbf{Mean Peer Feedback Communication Score} & 0.156 & 0.377$^{*}$
            \\ \hline
            \multicolumn{3}{l}{$^{*}$p < .05}
            \\ \multicolumn{3}{l}{$^{**}$p < .01}
        \end{tabular}
    \end{center}
\end{table}

The correlation between the percentage of story points passed in a Sprint and the mean peer-feedback scores for communication is significant, with a p-value of 0.003. This indicates that teams pass a lower proportion of their committed stories in Sprints where team members perceive their teammates' communication as being less effective.

We cannot intervene on a Sprint that has finished, so the correlation between the mean rating students give each other and the percentage of points they deliver is not as useful as a metric that can inform interventions during a given Sprint. However, it reinforces the need for an intervention to mitigate communication issues before the product quality is impaired. Still, a correlation between the mean rating of communication in Sprint \textit{n}, and the proportion of stories passed in Sprint \textit{n+1} would be useful. While the Pearson correlation coefficient for this is insignificant ($r = 0.26$, $p = .063$), the correlation between the mean communication score and the final team score is significant and moderate ($r = 0.33$, $p < .019$).

% \begin{itemize}
%     \item extremes not necessarily more productive
% \end{itemize}

% One of the teams did not use merge requests for the first half of the year, so we excluded this team from the analysis of congruence scores over the year.

% To investigate whether teams use in-person communication as an alternative to written online communication, we plotted the time they spent pair programming each week over their mean weekly congruence scores, as shown in figure~\ref{fig:congruence_and_pairing}. From this, we observe that weeks with low congruence scores are sometimes filled by higher pair programming, indicating that teams may use this as an alternative coordination technique. This may be pair programming between those who need to coordinate, or an indicator that the team is co-locating more, and thus uses informal communication during their co-locations to resolve coordination issues. 

% It is also notable that teams engage in less pair programming during the first week of each Sprint, likely due to the week being partially devoted to planning, and the Sprint deadline being further away than in other weeks. This indicates that students may use pair programming more in times of stress, or when they have implemented more of each feature.
% \begin{figure}
%     \centering
%     \includegraphics[width=1\linewidth]{images/Screenshot 2024-09-13 at 9.51.09 AM.png}
%     \caption{Time spent pair programming each week and mean congruence scores for a SENG302 team}
%     \label{fig:congruence_and_pairing}
% \end{figure}


\subsection{Triad census}
After calculating the triad census for each team and each Sprint, we correlated the relative frequency of the number of edges in each triad with the percentage of story points passed and performance score the team received in each Sprint.
Tables~\ref{tbl:triad_census} and \ref{tbl:triad_census_avg} show the results of this, with significant results denoted by asterisks. We note that the only significant correlation is a moderate negative correlation between the score a team receives for a given Sprint, and the relative frequency of triads with one edge. Teams with many of these triads have many pairs of members where neither member is communicating with other members of the team. This may be indicative of pair silos, where members form knowledge silos by working with only one teammate.

\begin{table}
\caption{Pearson Correlations Between Relative Frequencies of Triad Types and Sprint Performance}
    \begin{center}
         \label{tbl:triad_census}
        \begin{tabular}{|p{0.15\columnwidth} | p{0.37\columnwidth} | p{0.2\columnwidth}|} 
            \hline
            & \textbf{\% Passed Story Points} & \textbf{Team Score} 
            \\ \hline
            0 edges & 0.02 & -0.01
            \\ \hline
            1 edge & -0.22 & -0.30$^{*}$
            \\ \hline
            2 edges & -0.01 & -0.04
            \\ \hline
            3 edges & 0.12 & 0.19
            \\ \hline
        \end{tabular}  
    \end{center}
\end{table}

\begin{table}
    \caption{Pearson Correlations Between Mean Weekly Relative Frequencies of Triad Types and Sprint Performance}
    \begin{center}
        \label{tbl:triad_census_avg}
        \begin{tabular}{|p{0.15\columnwidth} | p{0.37\columnwidth} | p{0.2\columnwidth}|} 
            \hline
            & \textbf{\% Passed Story Points} & \textbf{Team Score} 
            \\ \hline
            0 edges & -0.20 & -0.23
            \\ \hline
            1 edge & 0.13 & 0.25
            \\ \hline
            2 edges & 0.10 & 0.08
            \\ \hline
            3 edges & 0.28 & 0.20
            \\ \hline
        \end{tabular}
    \end{center}
\end{table}

% Other header options:
% - Correlations Between Average Weekly Relative Frequencies of Triad Types and Sprint Performance
% - Correlations Between Average Weekly Frequencies of Triad Types and Sprint Performance
% - Correlations Between Average Weekly Triad Frequencies and Sprint Performance


We then looked for anomalies in teams' communication methods, where we identified two teams that did not follow the norms based on the data presented in Table~\ref{tbl:team_stats}. We note that team G had the highest mean STC score for the year, and yet passed the least stories of any team. This may be an indicator of low quality communication. The second anomalous team, Team J, engaged in little written communication relative to the dependencies between team members' work, as shown by their low mean STC score for the year. However, their work logs reported that they engaged in more hours of pair programming than any other team, indicating more in-person communication.

\begin{table}
    \caption{Team summaries}
    \begin{center}
        \label{tbl:team_stats}
        \begin{tabular}{| p{0.06\columnwidth} | p{0.26\columnwidth} | p{0.23\columnwidth} | p{0.2\columnwidth} |} 
            \hline 
            \textbf{Team}&\textbf{Pair Programming Hours}&\textbf{Mean STC Score}&\textbf{Stories Passed}
            \\ \hline
            A & 169 & 0.18 & 46
            \\ \hline
            B & 180 & 0.28 & 35
            \\ \hline
            C & 261 & 0.19 & 28
            \\ \hline
            D & 292 & 0.23 & 46
            \\ \hline
            E & 343 & 0.18 & 40
            \\ \hline
            F & 426 & 0.24 & 53
            \\ \hline
            G & 434 & 0.53 & 23
            \\ \hline
            H & 436 & 0.45 & 34
            \\ \hline
            I & 493 & 0.22 & 36
            \\ \hline
            J & 602 & 0.11 & 29
            \\ \hline
        \end{tabular}
    \end{center}
\end{table}

Tables~\ref{tbl:triad_census_sans_anomalies} and \ref{tbl:triad_census_avg_sans_anomalies} show the correlation results with the two anomalous teams excluded. The first significant result is the correlation between the scores teams received for a given Sprint and their relative frequency of triads with one edge, which remains significant, and the coefficient has strengthened from ($r = -0.30$, $p = .033$) to ($r = -0.42$, $p = .006$). Further, the correlation between the percentage of committed story points teams pass and their relative frequency of triads with one edge is now significant with a negative correlation coefficient of ($r = -0.38$, $p = .008$). Finally, the percentage of committed story points passed is positively correlated with the relative frequency of closed triads, where each node is connected to both other nodes in the triad, ($r = 0.29$, $p = .046$).

\begin{table}
\caption{Correlations Between Relative Frequencies of Triad Types and Sprint Performance Excluding Anomalies}
    \begin{center}
        \label{tbl:triad_census_sans_anomalies}
        \begin{tabular}{|p{0.15\columnwidth} | p{0.37\columnwidth} | p{0.2\columnwidth}|} 
            \hline
            & \textbf{\% Passed Story Points} & \textbf{Team Score} 
            \\ \hline
            0 edges & -0.06 & 0.02
            \\ \hline
            1 edge & -0.38$^{*}$ & -0.42$^{**}$
            \\ \hline
            2 edges & 0.10 & 0.00
            \\ \hline
            3 edges & 0.20 & 0.25
            \\ \hline
        \end{tabular}
    \end{center}
\end{table}

\begin{table}
    \caption{Correlations Between Mean Weekly Relative Frequencies of Triad Types and Sprint Performance Excluding Anomalies}
    \begin{center}
        \label{tbl:triad_census_avg_sans_anomalies}
        \begin{tabular}{|p{0.15\columnwidth} | p{0.37\columnwidth} | p{0.2\columnwidth}|} 
            \hline
            & \textbf{\% Passed Story Points} & \textbf{Team Score} 
            \\ \hline
            0 edges & 0.23 & -0.25
            \\ \hline
            1 edge & 0.11 & 0.23
            \\ \hline
            2 edges & 0.11 & 0.09
            \\ \hline
            3 edges & 0.29$^{*}$ & 0.26
            \\ \hline
        \end{tabular}
    \end{center}
\end{table}
