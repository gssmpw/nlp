\section{Discussion and Limitations}
\label{sec:discussion}
\subsection{Implications}
We have established that the percentage of story points teams pass in a given Sprint is significantly affected by their communication, measured as the communication scores team members give each other in Sprint peer-feedback, so we expect that an accurate measure of communication in a Sprint would also be correlated with the percentage of story points passed. 

As there was no significant correlation between the trend in teams' STC score and their percentage story points passed or peer-feedback communication scores, \textbf{we do not consider STC to be adequate as the sole indicator of communication areas of improvement for student teams}. However, it may still provide some insights when paired with other analysis techniques, although we have not explored this yet.  

Triad census is a promising indicator of teams' communication areas of improvement, with a significant, moderate, and positive correlation with the score teams receive for a given Sprint. Identifying and excluding anomalous teams further strengthens this correlation, and creates a significant correlation with the percentage of story points passed. The presence of these anomalous teams indicates that teams' communication is not consistently accurately represented by written communication, although this was adequate for most teams. More work is needed to strengthen our representation of communication to either include the teams that are not captured by our current implementation, or autonomously identify them. However, \textbf{triad census remains suitable for identifying teams' communication areas of improvement in most cases we observed.}

\subsection{Limitations}
There are limitations of our implementations of these analysis techniques. For STC, we expect communication for a merge request to occur in the week it was created in. However, this is inaccurate for commits made in weeks before the MR's creation, as communication for this work may occur when the commit was made, instead of when the MR was made. This could be mitigated by basing the task dependency matrix on files committed during a given time frame around each commit's creation.

For both STC and triad census, we retrieve communication from public channels in Slack, ignoring the content of messages, and assuming Slack is an adequate approximation of all communication mediums. Future work could combine this written communication with in-person communication like pair programming. We rely on the team using threads properly, as we did not include replies to messages outside threads. %Some teams in the 2024 cohort rarely or never use threads, so applying interventions based on communication threads would be an ineffective solution for improving communication in these teams. 
A potential mitigation of this is conversation disentanglement~\cite{elsnerYouTalkingMe2008}, which recreates threads where replies to a message are not in a thread. Finally, we only access communication in public channels, so communication in private channels and elsewhere is not represented.

% , and investigate whether congruence scores vary by the day of the week, which may reflect in-person communication occurring during co-location sessions and meetings.