 %%%%%%%% ICML 2021 EXAMPLE LATEX SUBMISSION FILE %%%%%%%%%%%%%%%%%

\documentclass{article}

% Recommended, but optional, packages for figures and better typesetting:
\usepackage{microtype}
\usepackage{graphicx}
\usepackage{subfigure}
\usepackage{booktabs} % for professional tables
\usepackage{amsmath}
\usepackage{amsthm}
\usepackage{amsfonts}
\usepackage{thmtools} 
\usepackage{thm-restate}

% hyperref makes hyperlinks in the resulting PDF.
% If your build breaks (sometimes temporarily if a hyperlink spans a page)
% please comment out the following usepackage line and replace
% \usepackage{icml2021} with \usepackage[nohyperref]{icml2021} above.
\usepackage{hyperref}

% Attempt to make hyperref and algorithmic work together better:
\newcommand{\theHalgorithm}{\arabic{algorithm}}

\newcommand{\jmei}[1]{\textcolor{magenta}{#1}}
\newcommand{\amnote}[1]{\textcolor{red}{#1}}
% Use the following line for the initial blind version submitted for review:
\newtheorem{claim}{Claim}
\newtheorem{lemma}{Lemma}
\newtheorem{definition}{Definition}
\newtheorem{corollary}{Corollary}
\newtheorem{proposition}{Proposition}
\newtheorem{assumption}{Assumption}
\newtheorem{theorem}{Theorem}
\usepackage{icml2021}


% If accepted, instead use the following line for the camera-ready submission:
%\usepackage[accepted]{icml2021}

% The \icmltitle you define below is probably too long as a header.
% Therefore, a short form for the running title is supplied here:
\icmltitlerunning{Error Bounds for Structured Kernel Interpolation}% via Multivariate Polynomial Interpolation Analysis}

\begin{document}

\twocolumn[
\icmltitle{Error Bounds for Structured Kernel Interpolation}% via Multivariate Polynomial Interpolation Analysis}

% It is OKAY to include author information, even for blind
% submissions: the style file will automatically remove it for you
% unless you've provided the [accepted] option to the icml2021
% package.

% List of affiliations: The first argument should be a (short)
% identifier you will use later to specify author affiliations
% Academic affiliations should list Department, University, City, Region, Country
% Industry affiliations should list Company, City, Region, Country

% You can specify symbols, otherwise they are numbered in order.
% Ideally, you should not use this facility. Affiliations will be numbered
% in order of appearance and this is the preferred way.
\icmlsetsymbol{equal}{*}

\begin{icmlauthorlist}
\icmlauthor{Aeiau Zzzz}{equal,to}
\icmlauthor{Bauiu C.~Yyyy}{equal,to,goo}
\icmlauthor{Cieua Vvvvv}{goo}
\icmlauthor{Iaesut Saoeu}{ed}
\icmlauthor{Fiuea Rrrr}{to}
\icmlauthor{Tateu H.~Yasehe}{ed,to,goo}
\icmlauthor{Aaoeu Iasoh}{goo}
\icmlauthor{Buiui Eueu}{ed}
\icmlauthor{Aeuia Zzzz}{ed}
\icmlauthor{Bieea C.~Yyyy}{to,goo}
\icmlauthor{Teoau Xxxx}{ed}
\icmlauthor{Eee Pppp}{ed}
\end{icmlauthorlist}

\icmlaffiliation{to}{Department of Computation, University of Torontoland, Torontoland, Canada}
\icmlaffiliation{goo}{Googol ShallowMind, New London, Michigan, USA}
\icmlaffiliation{ed}{School of Computation, University of Edenborrow, Edenborrow, United Kingdom}

\icmlcorrespondingauthor{Cieua Vvvvv}{c.vvvvv@googol.com}
\icmlcorrespondingauthor{Eee Pppp}{ep@eden.co.uk}

% You may provide any keywords that you
% find helpful for describing your paper; these are used to populate
% the "keywords" metadata in the PDF but will not be shown in the document
\icmlkeywords{Machine Learning, ICML}
\onecolumn

\vskip 0.3in
]

% this must go after the closing bracket ] following \twocolumn[ ...

% This command actually creates the footnote in the first column
% listing the affiliations and the copyright notice.
% The command takes one argument, which is text to display at the start of the footnote.
% The \icmlEqualContribution command is standard text for equal contribution.
% Remove it (just {}) if you do not need this facility.

%\printAffiliationsAndNotice{}  % leave blank if no need to mention equal contribution
\printAffiliationsAndNotice{\icmlEqualContribution} % otherwise use the standard text.

\begin{abstract}
Structured Kernel Interpolation (SKI) \cite{wilson2015kernel} helps scale Gaussian Processes (GPs) to large datasets by approximating the kernel matrix via interpolation at inducing points, achieving linear computational complexity.  However, SKI lacks rigorous theoretical error analysis. This paper addresses this gap: we establish error bounds for the SKI kernel and its Gram matrix, examine their effect on hyperparameter estimation, and evaluate the resulting error in posterior means and variances. Crucially, using convolutional cubic interpolation, we identify two dimensionality regimes governing the trade-off between SKI Gram matrix spectral norm error and computational complexity. For $d \leq 3$, linear time is achievable for \textit{any} error tolerance, provided the sample size is sufficiently large. For $d > 3$, the error must \textit{increase} with sample size to maintain linear time. Our analysis provides key insights into SKI's scalability-accuracy trade-offs, establishing precise conditions for achieving linear-time GP inference with controlled approximation error for the Gram matrix, optimization and posterior inference.

% Structured Kernel Interpolation (SKI) \cite{wilson2015kernel} helps scale Gaussian Processes (GPs) to large datasets by approximating the kernel matrix via interpolation at inducing points, leading to a computational complexity of $O(n+m\log m)$, where $n$ is the sample size and $m$ is the number of inducing points. Despite its widespread use, SKI lacks rigorous theoretical error analysis. In particular, for a given error tolerance for the approximate kernel matrix, how many inducing points $m$ do we need, and will $m\log m=O(n)$ hold so that the computational complexity is linear as claimed? This paper addresses this gap by deriving error bounds for the SKI kernel approximation and analyzing the impact on GP training and inference. We: (1) establish error bounds for the SKI kernel and its gram matrix; (2) examine how these errors affect kernel hyperparameter estimation; (3) evaluate the error of SKI GP posterior means and  variances. Interestingly, when using convolutional cubic interpolation, we identify two non-obvious dimensionality regimes relating SKI gram matrix spectral norm error to computational complexity. For $d\leq 3$, as long as the sample size is sufficiently large, we maintain linear time for \textit{any} error tolerance. For $d>3$, the error must \textit{increase} with the sample size to maintain our guarantee of linear time. Our theoretical analysis thus provides crucial insights into the scalability-accuracy trade-offs of SKI and how they change with the feature dimensionality, establishing precise conditions under which linear-time GP inference can be achieved while maintaining controlled approximation error.
% \jmei{It looks like the regime boundary in general would end up being $d\leq L-1$, so maybe we could further explicitly link polynomial degree to the dimensionality, error, and complexity. Not sure if there's a generalization of Keys' result that leys us easily do this though.}
\end{abstract}

\section{Introduction}\label{sec:introduction}
\section{Introduction}

Multi-modal models, such as CLIP~\citep{radford2021}, have demonstrated strong performance in representation learning.
By aligning visual and textual representations, these models achieve state-of-the-art results in tasks like image retrieval~\citep{baldrati2022conditioned,baldrati2022effective}, visual question answering~\citep{pan2023retrieving,song2022clip}, and zero-shot classification~\citep{radford2021,ali2023clip,wang2023improving,zhang2022tip}. 
Despite these successes, the mechanisms by which multi-modal models leverage their training data to achieve good generalization remain underexplored. 

In uni-modal setups, both supervised~\citep{feldman2020does,feldman2020neural} and self-supervised~\citep{wang2024memorization}, machine learning models have shown that their ability to \textit{memorize} their training data is essential for generalization. 
It was indicated that, in supervised learning, memorization typically occurs for mislabeled samples, outliers~\citep{bartlett2020benign,feldman2020does,feldman2020neural}, or data points that were seen towards the end of training~\citep{jagielski2022measuring}, while in self-supervised learning, high memorization is experienced particularly for atypical data points~\citep{wang2024memorization}. 
However, it is unclear how these findings extend to models like CLIP which entail elements from both supervised learning (through captions as supervisory signals) and self-supervised learning (through contrastive loss functions).

Existing definitions of memorization offer limited applicability to CLIP and therefore cannot fully address the gap in understanding.
% can, hence, not close the gap in understanding:
The standard definition from supervised learning~\citep{feldman2020does} relies on one-dimensional labels and the model's ability to produce confidence scores for these labels, whereas CLIP outputs high-dimensional representations. While the SSLMem metric~\citep{wang2024memorization}, developed for self-supervised vision models, could, in principle, be applied to CLIP's vision encoder outputs, it neglects the text modality, which is a critical component of CLIP. Additionally, measuring memorization in only one modality, or treating the modalities separately, risks diluting the signal and under-reporting memorization. Our experimental results, as shown in \Cref{sub:sslmem_not_for_clip}, confirm this concern. Therefore, new definitions of memorization tailored to CLIP's multi-modal nature are necessary.
\begin{figure}[t]
    \centering
    \begin{subfigure}[b]{0.475\textwidth}
        \centering
        \includegraphics[width=\textwidth]{image/10_most_1_caption.pdf}
        \caption[]{{\small Most Memorized: CLIPMem $>$ 0.89}}
    \end{subfigure}
    \hfill
    \begin{subfigure}[b]{0.475\textwidth}  
        \centering 
        \includegraphics[width=\textwidth]{image/10_least_1_caption.pdf}
        \caption[]%
        {{\small Least Memorized: CLIPMem $\approx$ 0.0}}    
    \end{subfigure}
    % \vskip\baselineskip
    % \begin{subfigure}[b]{0.475\textwidth}   
    %     \centering 
    %     \includegraphics[width=\textwidth]{Example-Image}
    %     \caption[]%
    %     {{\small Network 3}}    
    %     \label{fig:mean and std of net34}
    % \end{subfigure}
    % \hfill
    % \begin{subfigure}[b]{0.475\textwidth}   
    %     \centering 
    %     \includegraphics[width=\textwidth]{Example-Image}
    %     \caption[]%
    %     {{\small Network 4}}    
    %     \label{fig:mean and std of net44}
    % \end{subfigure}
    \caption{\textbf{Examples of data with different levels of memorization.} Higher memorization scores indicate stronger memorization. 
    We observe that atypical or distorted images, as well as those with incorrect or imprecise captions, experience higher memorization compared to standard samples and easy-to-label images with accurate captions.
    % We observe that atypical or distorted images and images with incorrect or imprecise captions experience higher memorization compared to more standard samples and easy-to-label samples with precise captions. 
    Results are obtained on OpenCLIP~\citep{ilharco_gabriel_2021_5143773}, with encoders based on the ViT-Base architecture trained on the COCO dataset.} 
        \label{fig:examples}
        %\vspace{-0.8cm}
\end{figure}

The only existing empirical work on quantifying memorization in CLIP models~\citep{jayaraman2024} focuses on Déjà Vu memorization~\citep{meehan2023ssl}, a specific type of memorization.
The success of their method relies on the accuracy of the integrated object detection method and on the availability of an additional public dataset from the same distribution as CLIP's training data, limiting practical applicability.
To overcome this limitation, we propose \textit{\ours} that measures memorization directly on CLIP's output representations.
Specifically, it compares the alignment---\ie the similarity between representations---of a given image-text pair in a CLIP model trained with the pair, to the alignment in a CLIP model trained on the same data but without the pair.

% Additionally, we focus on \textit{understanding} memorization rather than quantifying it. 
% %which, in CLIP, can be measured only with respect to an additional public dataset from the same distribution as CLIP's training data and fine-grained object detection methods. Moreover, the work is limited to \textit{quantifying} memorization.
% %---limiting its practical applicability. Moreover, while the work \textit{quantifies} Déjà Vu memorization, it does not offer detailed insights into which specific data points are memorized, why they are memorized, and how this relates to generalization. 
% %In contrast to their work, our focus is on \textit{understanding} memorization in CLIP by 
% We use this to identify which properties of the data and the two modalities contribute to CLIP memorization and on leveraging these insights to achieve \textit{better model utility while mitigating memorization}. To this end, we propose \textit{\ours} that directly measures memorization on the representations produced by CLIP's vision and text encoders.
% Specifically, \ours measures memorization by comparing the alignment, \ie the similarity between representations, of a given image-text pair in a CLIP model trained with this pair to the alignment in a CLIP model trained without this pair but on the same data otherwise.


In our empirical study of memorization in CLIP using \ours, we uncover several key findings. First, examples with incorrect or imprecise captions ("mis-captioned" examples) exhibit the highest levels of memorization, followed by atypical examples, as illustrated in \Cref{fig:examples}.
Second, removing these samples from training yields significant improvements in CLIP's generalization abilities.
These findings are particularly noteworthy, given that state-of-the-art CLIP models are usually trained on large, uncurated datasets sourced from the internet with no guarantees regarding the correctness of the text-image pairs.
Our results highlight that this practice not only exposes imprecise or incorrect data pairs to more memorization, often recognized as a cause for increased privacy leakage~\citep{carlini2019secret, carlini2021extracting, carlini2022privacy,song2017machine,liu2021encodermi}, but that it also negatively affects model performance. 
%By identifying highly memorized samples, our \ours can, hence, support a more private and performant deployment of CLIP.\todo{@Adam, is that last sentence too strong?}
Furthermore, by disentangling CLIP's two modalities, we are able to dissect how memorization manifests within each.
Surprisingly, we find that memorization does not affect both modalities alike, with memorization occurring more in the text modality than in the vision modality.
% even though the training objective is symmetric.\todo{@Adam, is that correct?}
% In fact, our results highlight that memorization occurs more in the text modality than in the vision modality. 
Building on these insights, we propose several strategies to reduce memorization while simultaneously improving generalization---a result that has not been observed in traditional supervised or self-supervised learning, where any reduction of memorization causes decreases in performance.
% which, at the same time, improve generalization.
% Such a result has not been observed in traditional supervised or self-supervised learning, where any reduction of memorization causes decreases in performance. 
Finally, at a deeper level, our analysis of the model internals, following~\citet{wang2024localizing}, shows that CLIP's memorization behavior sits between that of supervised and self-supervised learning. Specifically, neurons in early layers are responsible for groups of data points (\eg classes), similar to models trained using supervised learning, while neurons in later layers memorize individual data points, as seen in self-supervised learning.%\todo{cite our localization paper.}
% Performing an empirical evaluatin of memorization in CLIP according to our \ours, we find that
% -  examples with incorrect ("mis-captioned") or imprecise captions experience highest memorization, and then atypical examples . We show this effect in \Cref{fig:examples}.
% - memorization happens more in the text than in the vision modality
% - by including more captions into training, when only a few of them are mislabeled and the rest is correct, we can reduce memorization and at the same time improve generalization, something that has not been possible for supervised or self-supervised learning.
% - looking at the model internals, we see that the memorization behavior of CLIP is exactly in between supervised and self-supervised learning: in particular, neurons in early layers are responsible for groups (classes) of data points, same like for supervised learning, while neurons in later layers are responsible for individual data points

% \franzi{@Adam, do you think, we need an additional paragraph here on the mitigations we have? We actually only have multi-caption, so far, so probably not so important?}\adam{We also can mitigate the memorization if we remove the most memorized (probably mislabeled) samples.}

In summary, we make the following contributions:
\begin{itemize}
    \item We propose \ours, a metric to measure memorization in multi-modal vision language models.
    \item Through extensive evaluation, we identify that "mis-captioned" and "atypical" data points experience the highest memorization, and that the text encoder is more responsible for memorization than the image encoder.
    \item Based on our insights, we propose and evaluate multiple strategies to mitigate memorization in CLIP. We show that in CLIP, contrary to traditional supervised and self-supervised learning, a reduction of memorization does not need to imply a decrease in performance.
\end{itemize}



\section{Related Work}\label{sec:related}
\subsection{Large Language Models for Chemistry}
%化学agent,大语言模型(LLMs)在化学领域展现出巨大的潜力,被广泛应用于分子生成、性质预测、反应建模、逆合成分析等任务。例如,ChemDFM模型通过在化学文献和教科书上进行预训练,并使用大量指令进行微调,提升了多种化学任务上的表现。论文2介绍了一个名为ChemCrow的新型大型语言模型(LLM)驱动的化学助手,它通过整合多种专家设计的化学工具来增强LLM在化学领域的表现。
%尽管 LLM 已展现出强大的潜力,但它们在应对复杂的化学计算任务时仍存在诸多局限性。即使通过微调(fine-tuning)或指令调控,LLMs 仍难以适应高度复杂或需要精确计算的化学任务,且其泛化能力在面对不同化学任务时表现出不足。并且LLM 无法高效调用现有的化学计算工具,如SMILES 解析工具、分子动力学模拟工具等,更无法有效探索和利用工具之间的组合和堆叠关系。

Large language models (LLMs) have demonstrated significant potential in chemistry, with applications spanning molecular generation, property prediction, reaction modeling, and retrosynthetic analysis~\cite{fang2024molinstructionslargescalebiomolecularinstruction,tang2024prioritizingsafeguardingautonomyrisks,liao2024wordsmoleculessurveylarge}. For instance, the ChemDFM ~\cite{zhao2024chemdfmlargelanguagefoundation} pretrained on chemical literature and textbooks and further refined through extensive instruction tuning, has exhibited enhanced performance across various chemical tasks. Similarly, ChemCrow~\cite{bran2023chemcrowaugmentinglargelanguagemodels}, an LLM-powered chemistry assistant, integrates multiple expert-designed chemical tools to improve LLM performance in chemistry-related applications.Despite these advancements, LLMs continue to face challenges in handling complex chemical computations and generalizing across diverse chemical problems~\cite{ouyang2024structuredchemistryreasoninglarge,han2024generalistspecialistsurveylarge}. Moreover, they remain inefficient in utilizing existing computational chemistry tools~\cite{shi2023relmleveraginglanguagemodels}, and struggle to navigate the combinatorial and hierarchical relationships between these tools.

\subsection{Tool-augmented LLMs}
LLMs~\cite{anil2023palm,achiam2023gpt,touvron2023llama} have demonstrated strong reasoning capabilities in natural language processing and scientific computing. However, they face limitations in specialized tasks in fields such as chemistry and physics~\cite{yang2024moosechemlargelanguagemodels}, including constrained computational accuracy, insufficient numerical reasoning abilities, and a lack of collaboration with external tools. To address these shortcomings, researchers have recently proposed the tool-augmented LLMs approach~\cite{qin2023toolllmfacilitatinglargelanguage,wang2024gtabenchmarkgeneraltool,yang2023autogptonlinedecisionmaking}, enabling LLMs to dynamically call external tools and thereby enhance their task execution capabilities.
Representative methods include ReAct~\cite{yao2023reactsynergizingreasoningacting}, which combines chain-of-thought reasoning (CoT)~\cite{wei2023chainofthoughtpromptingelicitsreasoning} with tool invocation to allow LLMs to dynamically acquire external information during decision-making, and Toolformer~\cite{schick2023toolformerlanguagemodelsteach}, which enables LLMs to autonomously decide when to call tools, improving the accuracy of computational tasks. 
Despite these advancements, existing research primarily focuses on single-tool invocation and has yet to explore hierarchical combinations of tools. A single tool is often insufficient to solve complex scientific problems, whereas the collaborative invocation of multiple tools holds promise for enhancing the reasoning capabilities of LLMs in chemical tasks.

\section{Gaussian Processes, Structured Kernel Interpolation and Convolutional Cubic Interpolation}\label{sec:ski-background}
In this section, we introduce the details of our evaluation framework. We primarily evaluate one representative RAG system and two representative GraphRAG systems, as illustrated in Figure~\ref{fig:framework}.

% \jt{briefly intro figure 1 here}

\subsection{RAG}
\vspace{-0.1in}
We adopt a representative semantic similarity-based retrieval approach as our RAG method~\cite{karpukhin2020dense}. Specifically, we first split the text into chunks, each containing approximately 256 tokens. For indexing, we use OpenAI’s text-embedding-ada-002 model, which has demonstrated effectiveness across various tasks~\cite{nussbaum2024nomic}. For each query, we retrieve chunks with Top-10 similarity scores. To generate responses, we employ two open-source models of different sizes: Llama-3.1-8B-Instruct and Llama-3.1-70B-Instruct~\cite{dubey2024llama}.

For single-document tasks, we generate a separate RAG system for each document, ensuring that queries corresponding to a specific document are processed within its respective indexed chunk pool. For multi-document tasks, we use a shared RAG system by indexing all documents together.

\vspace{-0.1in}
\subsection{GraphRAG}

We select two representative GraphRAG methods for a comprehensive evaluation, as shown in Figure~\ref{fig:framework}, namely KG-based GraphRAG and Community-based GraphRAG.

In the KG-based GraphRAG (KG-GraphRAG)~\cite{Liu_LlamaIndex_2022}, a knowledge graph is first constructed from text chunks using LLMs through triplet extraction. When a query is received, its entities are extracted and matched to those in the constructed KG using LLMs. The retrieval process then traverses the graph from the matched entities and gathers triplets \textit{(head, relation, tail)} from their multi-hop neighbors as the retrieved content. Additionally, for each triplet, we can retrieve the corresponding text associated with it. We define two variants of KG-GraphRAG: {\bf (1)} {\it KG-GraphRAG (Triplets)}, which retrieves only the triplets, and {\bf (2)} {\it KG-GraphRAG (Triplets+Text)}, which retrieves both the triplets and their associated source text. We implement the KG-GraphRAG methods using LlamaIndex~\cite{Liu_LlamaIndex_2022}~\footnote{https://www.llamaindex.ai/}.


For the Community-based GraphRAG~\cite{edge2024local}, in addition to generating KGs using LLMs, hierarchical communities are constructed using graph community detection algorithms, as shown in Figure~\ref{fig:framework}. Each community is associated with a corresponding text summary or report, where lower-level communities contain detailed information from the original text. The higher-level communities further provide summaries of the lower-level communities. Due to the hierarchical community structure, there are two primary retrieval methods for retrieving relevant information given a query: {\bf Local Search and Global Search}.  In Local Search, entities, relations, their descriptions, and lower-level community reports are retrieved based on entity matching between the query's extracted entities and the constructed graph. We refer to this method as {\it Community-GraphRAG (Local)}. In Global Search, only high-level community summaries are retrieved based on semantic similarity to the query. We refer to this method as {\it Community-GraphRAG (Global)}. The Community-GraphRAG methods are implemented using Microsoft GraphRAG~\cite{edge2024local}\footnote{https://microsoft.github.io/graphrag}. 
% \yu{Would it be more clear if we can also visualize such second-level ablation in Figure 1, e.g., KG-GraphRAG(Triplets)/KG-GraphRAG(Triplets+Text)/Community-GraphRAG(Global)/Community-GraphRAG.}

To ensure a fair comparison, we adopt the same settings for both RAG and GraphRAG methods. This includes the chunking strategy, embedding model, and LLMs. We select two representative RAG tasks, i.e., Question Answering and Query-based Summarization, to evaluate RAG and GraphRAG simultaneously.

% \yu{missing something?}




\section{Important Quantities}\label{sec:important-quantities}
\subsection{Proofs Related to Ski Kernel Error Bounds}\label{subsec:proofs-ski-kernel-error}
\subsubsection{Proof of Lemma \ref{lemma:tensor-product-interpolation-error}}\label{sec:proof-tensor-product-interpolation}
\tensorproductinterpolationerror*
\begin{proof}
We define a sequence of intermediate interpolation functions. Let $g_0(\mathbf{x}) \equiv f(\mathbf{x})$ be the original function. For $i = 1, \ldots, d$, we recursively define $g_i(\mathbf{x})$ as the function obtained by interpolating $g_{i-1}$ along the $i$-th dimension using the cubic convolution kernel $u$:

$$
g_i(\mathbf{x}) \equiv \sum_{k=-1}^2 g_{i-1}\left(\mathbf{x} + \left( (\mathbf{c_x})_i - x_i + kh\right)\mathbf{e}_i \right) u\left(\frac{x_i - (\mathbf{c_x})_i - kh}{h}\right).
$$

Here, $\mathbf{c_x}$ is the grid point closest to $\mathbf{x}$, and $\mathbf{e}_i$ is the $i$-th standard basis vector. Thus, $g_1(\mathbf{x})$ interpolates $f$ along the first dimension, $g_2(\mathbf{x})$ interpolates $g_1$ along the second dimension, and so on, until $g_d(\mathbf{x}) = g(\mathbf{x})$ is the final tensor-product interpolated function.

We analyze the error accumulation across multiple dimensions using induction. Using \cite{keys1981cubic}, the error introduced by interpolating a thrice continuous differentiable function along a single dimension with the cubic convolution kernel is uniformly bounded over the interval domain by $Kh^3$ for some constant $K > 0$, provided the grid spacing $h$ is sufficiently small. This gives us the base case:
$$
|g_1(\mathbf{x}) - g_0(\mathbf{x})| \leq Kh^3.
$$

For the inductive step, assume that for some $i=k$ the error is uniformly bounded by
$$
|g_k(\mathbf{x}) - g_{k-1}(\mathbf{x})| \leq c^{k-1}Kh^3.
$$

We want to show that this bound also holds for $i=k+1$. We can express the difference $g_{k+1}(\mathbf{x}) - g_k(\mathbf{x})$ as follows:

\begin{align*}
g_{k+1}(\mathbf{x}) - g_k(\mathbf{x}) &= \sum_{k_{k+1}=-1}^2 g_k\left(\mathbf{x} + ((\mathbf{c_x})_{k+1} - x_{k+1} + k_{k+1}h) \mathbf{e}_{k+1} \right) u\left(\frac{x_{k+1} - (\mathbf{c_x})_{k+1} - k_{k+1}h}{h}\right) \\
&\quad - g_k(\mathbf{x}) \\
&= \sum_{k_{k+1}=-1}^2 \left[\sum_{k_k=-1}^2 g_{k-1}\left(\mathbf{x} + ((\mathbf{c_x})_k - x_k + k_k h) \mathbf{e}_k + ((\mathbf{c_x})_{k+1} - x_{k+1} + k_{k+1}h) \mathbf{e}_{k+1} \right) \right. \\
&\quad \left. u\left(\frac{x_k - (\mathbf{c_x})_k - k_k h}{h}\right)\right] u\left(\frac{x_{k+1} - (\mathbf{c_x})_{k+1} - k_{k+1}h}{h}\right) \\
&\quad - \sum_{k_k=-1}^2 g_{k-1}\left(\mathbf{x} + ((\mathbf{c_x})_k - x_k + k_k h) \mathbf{e}_k \right) u\left(\frac{x_k - (\mathbf{c_x})_k - k_k h}{h}\right) \\
&= \sum_{k_k=-1}^2 u\left(\frac{x_k - (\mathbf{c_x})_k - k_k h}{h}\right) \left[\sum_{k_{k+1}=-1}^2 g_{k-1}\left(\mathbf{x} + ((\mathbf{c_x})_k - x_k + k_k h) \mathbf{e}_k + ((\mathbf{c_x})_{k+1} - x_{k+1} + k_{k+1}h) \mathbf{e}_{k+1} \right) \right. \\
&\quad \left. u\left(\frac{x_{k+1} - (\mathbf{c_x})_{k+1} - k_{k+1}h}{h}\right) - g_{k-1}\left(\mathbf{x} + ((\mathbf{c_x})_k - x_k + k_k h) \mathbf{e}_k\right)\right].
\end{align*}

The inner term in the last expression represents the difference between interpolating $g_{k-1}$ along the $(k+1)$-th dimension and $g_{k-1}$ itself, evaluated at $\mathbf{x} + ((\mathbf{c_x})_k - x_k + k_k h) \mathbf{e}_k$. This can be written as:

\begin{align*}
&\sum_{k_{k+1}=-1}^2 g_{k-1}\left(\mathbf{x} + ((\mathbf{c_x})_k - x_k + k_k h) \mathbf{e}_k + ((\mathbf{c_x})_{k+1} - x_{k+1} + k_{k+1}h) \mathbf{e}_{k+1} \right) u\left(\frac{x_{k+1} - (\mathbf{c_x})_{k+1} - k_{k+1}h}{h}\right) \\
&\quad - g_{k-1}\left(\mathbf{x} + ((\mathbf{c_x})_k - x_k + k_k h) \mathbf{e}_k\right) \\
&= g_k\left(\mathbf{x} + ((\mathbf{c_x})_k - x_k + k_k h) \mathbf{e}_k \right) - g_{k-1}\left(\mathbf{x} + ((\mathbf{c_x})_k - x_k + k_k h) \mathbf{e}_k \right).
\end{align*}

Therefore, we can bound the error as follows:

\begin{align*}
|g_{k+1}(\mathbf{x}) - g_k(\mathbf{x})| &\leq \left|\sum_{k_k=-1}^2 u\left(\frac{x_k - (\mathbf{c_x})_k - k_k h}{h}\right)\right| \cdot \left|g_k\left(\mathbf{x} + ((\mathbf{c_x})_k - x_k + k_k h) \mathbf{e}_k \right) - g_{k-1}\left(\mathbf{x} + ((\mathbf{c_x})_k - x_k + k_k h) \mathbf{e}_k \right)\right|.
\end{align*}

Let $c>0$ be a uniform upper bound for $\sum_{k_k=-1}^2 \left|u\left(\frac{x_k - (\mathbf{c_x})_k - k_k h}{h}\right)\right|$, which exists because $u$ is bounded. By the inductive hypothesis, we have $\left|g_k\left(\mathbf{x} + ((\mathbf{c_x})_k - x_k + k_k h) \mathbf{e}_k \right) - g_{k-1}\left(\mathbf{x} + ((\mathbf{c_x})_k - x_k + k_k h) \mathbf{e}_k \right)\right| \leq c^{k-1}Kh^3$. Thus,

$$
|g_{k+1}(\mathbf{x}) - g_k(\mathbf{x})| \leq c \cdot c^{k-1}Kh^3 = c^k Kh^3.
$$

This completes the inductive step.

Finally, we bound the total error $|g(\mathbf{x}) - f(\mathbf{x})| = |g_d(\mathbf{x}) - g_0(\mathbf{x})|$ by summing the errors introduced at each interpolation step:

$$
|g(\mathbf{x}) - f(\mathbf{x})| \leq \sum_{i=1}^d |g_i(\mathbf{x}) - g_{i-1}(\mathbf{x})| \leq \sum_{i=1}^d c^{i-1}Kh^3 = Kh^3 \sum_{i=0}^{d-1} c^i.
$$

The last sum is a geometric series, which evaluates to $Kh^3 \frac{1 - c^d}{1 - c}$. For a fixed $c>1$ (independent of $d$), this expression is $O(c^{d})$ when $d$ is large. Therefore, tensor-product cubic convolutional interpolation has $O(c^d h^3)$ error. Finally, noticing that $h=O\left(\frac{1}{m^{d/3}}\right)$ gives us the desired result.
\end{proof}

\subsubsection{Curse of Dimensionality for Kernel Regression}

The next lemma shows that when using a product kernel for $d$-dimensional kernel regression (where cubic convolutional interpolation is a special case), the sum of weights suffers from the curse of dimensionality. The proof strategy involves expressing the multi-dimensional sum as a product of sums over each individual dimension, leveraging the initial condition on the one-dimensional bound for each dimension, and taking advantage of the structure of the Cartesian grid.


\begin{restatable}{lemma}{cubicInterpolationWeights}
\label{lemma:cubic-interpolation-weights-curse-dimensionality}
Let $u: \mathbb{R} \rightarrow \mathbb{R}$ be a one-dimensional kernel function with constant $c>0$ defined as in \ref{def:sum-weight-upper-bound}. Let $u_d: \mathbb{R}^d \rightarrow \mathbb{R}$ be a d-dimensional product kernel defined as:
$$
u_d\left(\frac{x - x_i}{h}\right) = \prod_{j=1}^d u\left(\frac{x^{(j)} - x_i^{(j)}}{h}\right),
$$
where $x = (x^{(1)}, x^{(2)}, ..., x^{(d)}) \in \mathbb{R}^d$ and $x_i = (x_i^{(1)}, x_i^{(2)}, ..., x_i^{(d)}) \in \mathbb{R}^d$ are d-dimensional points. Assume the data points $\{x_i\}_{i=1}^n$ ($n$ may differ from the univariate case) lie on a fixed d-dimensional grid $G = G^{(1)} \times G^{(2)} \times ... \times G^{(d)}$, where each $G^{(j)} = \{p_1^{(j)}, p_2^{(j)}, ..., p_{n_j}^{(j)}\}$ is a finite set of $n_j$ grid points along the j-th dimension for $j = 1, 2, ..., d$. Then, for any $x \in \mathbb{R}^d$, the sum of weights in the d-dimensional kernel regression is bounded by $c^d$:
$$
\sum_{i=1}^n \left|u_d\left(\frac{x - x_i}{h}\right) \right|\leq c^d.
$$
\end{restatable}

\begin{proof}
Let the fixed d-dimensional grid be defined by the Cartesian product of d sets of 1-dimensional grid points: $G = G^{(1)} \times G^{(2)} \times ... \times G^{(d)}$, where $G^{(j)} = \{p_1^{(j)}, p_2^{(j)}, ..., p_{n_j}^{(j)}\}$ is the set of grid points along the j-th dimension.

We start with the sum of weights in the d-dimensional case:

$$
\sum_{i=1}^n u_d\left(\frac{x - x_i}{h}\right) = \sum_{i=1}^n \prod_{j=1}^d u\left(\frac{x^{(j)} - x_i^{(j)}}{h}\right)
$$

Since the data points lie on the fixed grid $G$, we can rewrite the outer sum as a nested sum over the grid points in each dimension:

$$
\sum_{i=1}^n \prod_{j=1}^d u\left(\frac{x^{(j)} - x_i^{(j)}}{h}\right) = \sum_{k_1=1}^{n_1} \sum_{k_2=1}^{n_2} ... \sum_{k_d=1}^{n_d} \prod_{j=1}^d u\left(\frac{x^{(j)} - p_{k_j}^{(j)}}{h}\right)
$$

Now we can change the order of summation and product, as proven in Lemma \ref{lemma:switch-sum-product}:

$$
\sum_{k_1=1}^{n_1} \sum_{k_2=1}^{n_2} ... \sum_{k_d=1}^{n_d} \prod_{j=1}^d u\left(\frac{x^{(j)} - p_{k_j}^{(j)}}{h}\right) = \prod_{j=1}^d \left( \sum_{k_j=1}^{n_j} u\left(\frac{x^{(j)} - p_{k_j}^{(j)}}{h}\right) \right)
$$

By the assumption of the lemma, we know that for each dimension $j$, the sum of weights is bounded by $c$. Note that $\{p_{k_j}^{(j)}\}_{k_j=1}^{n_j}$ is simply a set of points in $\mathbb{R}$, thus:

$$
\sum_{k_j=1}^{n_j} \left|u\left(\frac{x^{(j)} - p_{k_j}^{(j)}}{h}\right)\right| \leq c
$$

Therefore, we have:

$$
\prod_{j=1}^d \left(\left| \sum_{k_j=1}^{n_j} u\left(\frac{x^{(j)} - p_{k_j}^{(j)}}{h}\right)\right| \right) \leq \prod_{j=1}^d c = c^d
$$

Thus, we have shown that:

$$
\sum_{i=1}^n \left|u_d\left(\frac{x - x_i}{h}\right) \right|\leq c^d
$$

\end{proof}





% \subsubsection{Elementwise}\label{subsubsec:proofs-elementwise}
\subsubsection{Proof of Lemma \ref{lemma:ski-kernel-elementwise-error}}\label{sec:proof-ski-kernel-elementwise-error}
\skikernelelementwiseerror*
\begin{proof}
Recall that SKI approximates the kernel as
\begin{align*}
k(\mathbf{x}, \mathbf{x}') &\approx \tilde{k}(\mathbf{x}, \mathbf{x}')\\
&= \boldsymbol{w}(\mathbf{x})^\top \mathbf{K}_{\textbf{U}} \boldsymbol{w}(\mathbf{x}'),
\end{align*}

Let $\textbf{K}_{\textbf{U},\textbf{x}'}\in \mathbb{R}^m$ be the vector of kernels between the inducing points and the vector $\textbf{x}'$
\begin{align}
\vert k(\mathbf{x}, \mathbf{x}') -\tilde{k}(\mathbf{x}, \mathbf{x}')\vert &= \vert k(\mathbf{x}, \mathbf{x}')-\boldsymbol{w}(\mathbf{x})^\top \textbf{K}_{\textbf{U},\textbf{x}'} +\boldsymbol{w}(\mathbf{x})^\top \textbf{K}_{\textbf{U},\textbf{x}'}-\boldsymbol{w}(\mathbf{x})^\top \mathbf{K}_{\textbf{U}} \boldsymbol{w}(\mathbf{x}')\vert\nonumber\\
&\leq  \vert k(\mathbf{x}, \mathbf{x}')-\boldsymbol{w}(\mathbf{x})^\top \textbf{K}_{\textbf{U},\textbf{x}'} \vert+\vert \boldsymbol{w}(\mathbf{x})^\top \textbf{K}_{\textbf{U},\textbf{x}'}-\boldsymbol{w}(\mathbf{x})^\top \mathbf{K}_{\textbf{U}} \boldsymbol{w}(\mathbf{x}')\vert\nonumber\\
&\leq \delta_{m,L}+\vert \boldsymbol{w}(\mathbf{x})^\top \textbf{K}_{\textbf{U},\textbf{x}'}-\boldsymbol{w}(\mathbf{x})^\top \mathbf{K}_{\textbf{U}} \boldsymbol{w}(\mathbf{x}')\vert \textrm{ since $\vert k(\mathbf{x}, \mathbf{x}')-\boldsymbol{w}(\mathbf{x})^\top \textbf{K}_{\textbf{U},\textbf{x}'} \vert$ is a single polynomial interpolation}\label{eqn:applying-single-poly-interp}
\end{align}
Now note that $\textbf{w}(x)\in \mathbb{R}^m$ is a sparse matrix with at most $L$ non-zero entries. Thus, letting $\tilde{\textbf{w}}(x)\in \mathbb{R}^L$ be the non-zero entries of $\textbf{w}(x)$ and similarly $\tilde{\textbf{K}}_{\textbf{U},\textbf{x}'}\in \mathbb{R}^L$ be the entries of $\textbf{K}_{\textbf{U},\textbf{x}'}$ in the dimensions corresponding to non-zero entries of $\textbf{w}(x)\in \mathbb{R}^m$, while $\tilde{\textbf{K}}_{\textbf{U}}\in \mathbb{R}^{L\times m}$ is the analogous matrix for $\textbf{K}_{\textbf{U}}$, we have
\begin{align}
    \vert \boldsymbol{w}(\mathbf{x})^\top \textbf{K}_{\textbf{U},\textbf{x}'}-\boldsymbol{w}(\mathbf{x})^\top \mathbf{K}_{\textbf{U}} \boldsymbol{w}(\mathbf{x}')\vert&=\vert \tilde{\textbf{w}}(\textbf{x})^\top \tilde{\textbf{K}}_{\textbf{U},\textbf{x}'}-\tilde{\textbf{w}}(\textbf{x})^\top \tilde{\textbf{K}}_\textbf{U}\textbf{w}(\textbf{x}')\vert\nonumber\\
    &\leq \Vert \tilde{\textbf{w}}(\textbf{x})\Vert_2\Vert \tilde{\textbf{K}}_{\textbf{U},\textbf{x}'}-\tilde{\mathbf{K}}_{\textbf{U}} \boldsymbol{w}(\mathbf{x}')\Vert_2\nonumber\\
    &\leq c^d\sqrt{L}\Vert \tilde{\textbf{K}}_{\textbf{U},\textbf{x}'}-\tilde{\mathbf{K}}_{\textbf{U}} \boldsymbol{w}(\mathbf{x}')\Vert_\infty\nonumber\text{ Lemma \ref{lemma:cubic-interpolation-weights-curse-dimensionality}}\\
    &\leq \sqrt{L}c^d\delta_{m,L}\label{eqn:applying-sparsity-single-kernel-evaluation}
\end{align}
where the last line follows as each element of $\mathbf{K}_{\textbf{U}} \boldsymbol{w}(\mathbf{x}')$ is a polynomial interpolation approximating each element of $\textbf{K}_{\textbf{U},\textbf{x}'}$. Plugging Eqn. \ref{eqn:applying-sparsity-single-kernel-evaluation} into Eqn. \ref{eqn:applying-single-poly-interp} gives us the desired initial result of
\begin{align*}
    \vert k(\textbf{x},\textbf{x}')-\tilde{k}(\textbf{x},\textbf{x}')\vert&\leq \delta_{m,L}+\sqrt{L}c^d\delta_{m,L}
\end{align*}
and Lemma \ref{lemma:tensor-product-interpolation-error} gives us the result when the convolutional kernel is cubic.
\end{proof}

% \subsubsection{Spectral Norm}\label{subsubsec:spectral-norm}

\subsubsection{Proof of Proposition \ref{prop:spectral-norm}}\label{sec:proof-spectral-norm}

\spectralnorm*
\begin{proof}
    Recall that for any matrix $\textbf{A}$, $\Vert \textbf{A} \Vert_2 \leq \sqrt{\Vert \textbf{A} \Vert_1 \Vert \textbf{A} \Vert_\infty}$. Since $\textbf{K}-\tilde{\textbf{K}}$ is symmetric, we have
    \begin{align*}
        \Vert \textbf{K}-\tilde{\textbf{K}}\Vert_2&\leq \sqrt{\Vert \textbf{K}-\tilde{\textbf{K}}\Vert_1\Vert \textbf{K}-\tilde{\textbf{K}}\Vert_\infty} = \Vert \textbf{K}-\tilde{\textbf{K}}\Vert_\infty
    \end{align*}
    Furthermore, $\Vert \textbf{K}-\tilde{\textbf{K}}\Vert_\infty$ is the maximum absolute row sum of $\textbf{K}-\tilde{\textbf{K}}$. Since there are $n$ rows and, by Lemma \ref{lemma:ski-kernel-elementwise-error}, each element of $\textbf{K} - \tilde{\textbf{K}}$ is bounded by $\delta_{m,L}+\sqrt{L}c^d\delta_{m,L}$ in absolute value, we have
    \begin{align*}
        \Vert \textbf{K}-\tilde{\textbf{K}}\Vert_\infty &\leq n \left(\delta_{m,L}+\sqrt{L} c^d\delta_{m,L}\right) = \gamma_{n,m,L}.
    \end{align*}
    Therefore, $\Vert \textbf{K}-\tilde{\textbf{K}}\Vert_2 \leq \gamma_{n,m,L}$.
\end{proof}

\subsubsection{Proof of Lemma \ref{lemma:test-train-kernel-matrix-error}}\label{sec:proof-test-train-kernel-matrix-error}
\testtrainkernelmatrixerror*
\begin{proof}
    Using the same reasoning as in Proposition \ref{prop:spectral-norm}, we have
    \begin{align*}
        \Vert \textbf{K}_{\cdot,\textbf{X}}-\tilde{\textbf{K}}_{\cdot,\textbf{X}}\Vert_2&\leq \sqrt{\Vert \textbf{K}_{\cdot,\textbf{X}}-\tilde{\textbf{K}}_{\cdot,\textbf{X}}\Vert_1\Vert \textbf{K}_{\cdot,\textbf{X}}-\tilde{\textbf{K}}_{\cdot,\textbf{X}}\Vert_\infty} \\
        &\leq \max \left(\Vert \textbf{K}_{\cdot,\textbf{X}}-\tilde{\textbf{K}}_{\cdot,\textbf{X}}\Vert_1, \Vert \textbf{K}_{\cdot,\textbf{X}}-\tilde{\textbf{K}}_{\cdot,\textbf{X}}\Vert_\infty\right).
    \end{align*}
    Now, $\Vert \textbf{K}_{\cdot,\textbf{X}}-\tilde{\textbf{K}}_{\cdot,\textbf{X}}\Vert_1$ is the maximum absolute column sum, which is less than or equal to $T(\delta_{m,L} + \sqrt{L}c^d\delta_{m,L}) = \gamma_{T,m,L}$. Similarly, $\Vert \textbf{K}_{\cdot,\textbf{X}}-\tilde{\textbf{K}}_{\cdot,\textbf{X}}\Vert_\infty$ is the maximum absolute row sum, which is upper bounded by $n(\delta_{m,L} + \sqrt{L}c^d\delta_{m,L}) = \gamma_{n,m,L}$. Therefore,
    $$
    \Vert \textbf{K}_{\cdot,\textbf{X}}-\tilde{\textbf{K}}_{\cdot,\textbf{X}}\Vert_2 \leq \max(\gamma_{T,m,L},\gamma_{n,m,L}).
    $$
\end{proof}
\subsubsection{Additional Spectral Norm Bounds}
\begin{restatable}{lemma}{traintestbound}\label{lemma:test-train-bound}
    Let $\textbf{K}_{\cdot,\textbf{X}} \in \mathbb{R}^{T \times n}$ be cross kernel matrix between $T$ test points and $n$ training points, where the SKI approximation uses $m$ inducing points. If the kernel function $k$ is bounded such that $|k(\textbf{x}, \textbf{x}')| \leq M$ for all $\textbf{x}, \textbf{x}'\in \mathcal{X}$, then:
    \begin{align*}
        \Vert \textbf{K}_{\cdot,\textbf{X}}\Vert_2&\leq \sqrt{Tn}M
    \end{align*}
\end{restatable}
\begin{proof}
\begin{align*}
    \Vert \textbf{K}_{\cdot,\textbf{X}}\Vert_2&\leq \sqrt{\Vert \textbf{K}_{\cdot,\textbf{X}}\Vert_1\Vert \textbf{K}_{\cdot,\textbf{X}}\Vert_\infty}\\
    &\leq \sqrt{Tn}M
\end{align*}
\end{proof}
\begin{restatable}{lemma}{skitesttrainbound}\label{lemma:ski-test-train-bound}
    Let $\tilde{\textbf{K}}_{\cdot,\textbf{X}} \in \mathbb{R}^{T \times n}$ be the matrix of SKI kernel evaluations between $T$ test points and $n$ training points, where the SKI approximation uses $m$ inducing points. Let $\textbf{W}(\cdot) \in \mathbb{R}^{T \times m}$ and $\textbf{W}(\textbf{X}) \in \mathbb{R}^{n \times m}$ be the matrices of interpolation weights for the test points and training points, respectively. Assume that the interpolation scheme is such that the sum of the absolute values of the interpolation weights for any point is bounded by $c^d$, where $c>0$ is a constant. Let $\textbf{K}_{\textbf{U}} \in \mathbb{R}^{m \times m}$ be the kernel matrix evaluated at the inducing points. If the kernel function $k$ is bounded such that $|k(\textbf{x}, \textbf{x}')| \leq M$ for all $\textbf{x}, \textbf{x}'\in \mathcal{X}$, then:
    $$
    \Vert \tilde{\textbf{K}}_{\cdot,\textbf{X}}\Vert_2 \leq \sqrt{Tn} m c^{2d} M
    $$
\end{restatable}
\begin{proof}
    By the definition of the SKI approximation and the submultiplicativity of the spectral norm, we have:
    $$
    \Vert \tilde{\textbf{K}}_{\cdot,\textbf{X}}\Vert_2 = \Vert \textbf{W}(\cdot)\textbf{K}_{\textbf{U}}(\textbf{W}(\textbf{X}))^\top\Vert_2 \leq \Vert \textbf{W}(\cdot)\Vert_2 \Vert\textbf{K}_{\textbf{U}}\Vert_2 \Vert\textbf{W}(\textbf{X})\Vert_2
    $$

    We now bound each term.

    1.  **Bounding $\Vert \textbf{W}(\cdot)\Vert_2$ and $\Vert \textbf{W}(\textbf{X})\Vert_2$:**
        Since the spectral norm is induced by the Euclidean norm, and using the assumption that the sum of absolute values of interpolation weights for any point is bounded by $c^d$, we have
        $$\Vert \textbf{W}(\cdot)\Vert_2 \leq \sqrt{\Vert \textbf{W}(\cdot)\Vert_1 \Vert \textbf{W}(\cdot)\Vert_\infty} \leq \sqrt{T c^d \cdot c^d} = \sqrt{T} c^d.$$
        Similarly, $\Vert\textbf{W}(\textbf{X})\Vert_2 \leq \sqrt{n}c^d$.

    2.  **Bounding $\Vert\textbf{K}_{\textbf{U}}\Vert_2$:**
        Since $\textbf{K}_{\textbf{U}}$ is symmetric, $\Vert \textbf{K}_{\textbf{U}} \Vert_2 \leq \Vert \textbf{K}_{\textbf{U}} \Vert_\infty$. Each entry of $\textbf{K}_{\textbf{U}}$ is bounded by $M$ (by the boundedness of $k$), and each row has $m$ entries, so $\Vert \textbf{K}_{\textbf{U}} \Vert_\infty \leq mM$. Thus, $\Vert\textbf{K}_{\textbf{U}}\Vert_2 \leq mM$.

    Combining these bounds, we get:
    $$
    \Vert \tilde{\textbf{K}}_{\cdot,\textbf{X}}\Vert_2 \leq (\sqrt{T} c^d) (mM) (\sqrt{n} c^d) = \sqrt{Tn} m c^{2d} M
    $$
    as required.
\end{proof}


% \amnote{might get rid of this}
\begin{lemma}\label{lemma:action-inverse-error}
    Let \(\mathbf{\tilde{K}}\) be the SKI approximation of the kernel matrix \(\mathbf{K}\), and let \(\sigma^2\) be the regularization parameter. The spectral error of the regularized inverse can be bounded as follows:
\begin{align*}
\left\|\left(\mathbf{\tilde{K}} + \sigma^2 \mathbf{I}\right)^{-1} - \left(\mathbf{K} + \sigma^2 \mathbf{I}\right)^{-1} \right\|_2 &\leq \frac{\gamma_{n, m, L}}{\sigma^4} 
% \\
% &=\frac{O(n\sqrt{L}\rho^{-L})}{\sigma^4}
\end{align*}
\end{lemma} 

\begin{proof}
    Note that
\[
\left(\mathbf{\tilde{K}} + \sigma^2 \mathbf{I}\right)^{-1} - \left(\mathbf{K} + \sigma^2 \mathbf{I}\right)^{-1} = \left(\mathbf{\tilde{K}} + \sigma^2 \mathbf{I}\right)^{-1} (\mathbf{K} - \mathbf{\tilde{K}}) \left(\mathbf{K} + \sigma^2 \mathbf{I}\right)^{-1}
\]

Taking the spectral norm, we have
\[
\begin{aligned}
\left\|\left(\mathbf{\tilde{K}} + \sigma^2 \mathbf{I}\right)^{-1} - \left(\mathbf{K} + \sigma^2 \mathbf{I}\right)^{-1}\right\|_2 &\leq \left\|\left(\mathbf{\tilde{K}} + \sigma^2 \mathbf{I}\right)^{-1}\right\|_2 \|\mathbf{K} - \mathbf{\tilde{K}}\|_2 \left\|\left(\mathbf{K} + \sigma^2 \mathbf{I}\right)^{-1}\right\|_2 \\
&\leq \gamma_{n, m, L}\left\|\left(\mathbf{\tilde{K}} + \sigma^2 \mathbf{I}\right)^{-1}\right\|_2 \left\|\left(\mathbf{K} + \sigma^2 \mathbf{I}\right)^{-1}\right\|_2\textrm{ by Proposition \ref{prop:spectral-norm}}\\
&\leq   \frac{\gamma_{n, m, L}}{\sigma^4}
\end{aligned}
\]

\end{proof}

\subsection{Proofs Related to Linear Time Analysis}
\subsubsection{Proof of Theorem \ref{thm:inducing-points-count-alt}}\label{sec:proof-inducing-points-count-alt}
\inducingpointscountalt*
\begin{proof}
    We want to choose $m$ such that the spectral norm error $\Vert \textbf{K} - \tilde{\textbf{K}} \Vert_2 \leq \epsilon$. From Proposition \ref{prop:spectral-norm}, we have:
    $$
    \Vert \textbf{K} - \tilde{\textbf{K}} \Vert_2 \leq n(1 + \sqrt{L}c^d) \delta_{m,L}
    $$
    For cubic interpolation ($L=4$), Lemma \ref{lemma:tensor-product-interpolation-error}, combined with the analysis in Lemma \ref{lemma:tensor-product-interpolation-error}, gives us:
    $$
    \delta_{m,L} \leq K' c^{2d} h^3
    $$
    where $K'$ is a constant that depends only on the kernel function (through its derivatives) and the interpolation scheme, but not on $n$, $m$, $h$, or $d$.

    Therefore, a sufficient condition to ensure $\Vert \textbf{K} - \tilde{\textbf{K}} \Vert_2 \leq \epsilon$ is:
    \begin{equation} \label{eq:sufficient_condition_final}
    n(1 + 2c^d) K' c^{2d} h^3 \leq \epsilon
    \end{equation}

    Since the inducing points are placed on a regular grid with spacing $h$ in each dimension, and the domain is $[-D,D]^d$ and assuming that $2D\mod h\equiv 0$, the number of inducing points $m$ satisfies:

    $$
    m = \left(\frac{2D}{h}\right)^d
    $$

    We can rearrange this to get:

    $$
    h = \frac{2D}{m^{1/d}}
    $$
    Substituting this into the sufficient condition \eqref{eq:sufficient_condition_final}, we get:

    $$
    n (1 + 2c^d) K' c^{2d} \left(\frac{2D}{m^{1/d}}\right)^3 \leq \epsilon
    $$

    Rearranging to isolate $m$, we obtain:

    $$
    m^{3/d} \geq \frac{n}{\epsilon} (1 + 2c^d) K' c^{2d} (8D^3)
    $$

    $$
    m \geq \left( \frac{n}{\epsilon} (1 + 2c^d) K' (8 c^{2d} D^3) \right)^{d/3}
    $$
\end{proof}
\subsubsection{Proof of Corollary \ref{cor:linear-time}}\label{proof:cor-linear-time}
\corlineartime*
\begin{proof}
% The idea is that we want to have an error sufficiently large so that when we choose $m$ based on Corollary \ref{cor:inducing-points-count-alt}, we have $m=O\left(\frac{n}{\log n}\right)$. 

Assume that
$$
\epsilon \geq \frac{(1 + 2c^d) K' 8 c^{2d} D^3}{C^{3/d}} \cdot \frac{n (\log n)^{3/d}}{n^{3/d}}.
$$
Rearranging this we obtain
\begin{align*}
    \left( \frac{n}{\epsilon} (1 + 2c^d) K' (8 c^{2d} D^3) \right)^{d/3}&\leq C\frac{n}{\log n}.\\
    &=O\left(\frac{n}{\log n}\right).
\end{align*}
Now taking 
\begin{align*}
    m &= \left( \frac{n}{\epsilon} (1 + 2c^d) K' (8 c^{2d} D^3) \right)^{d/3}
\end{align*}
we have that $m=O\left(\frac{n}{\log n}\right)$ and by Theorem \ref{thm:inducing-points-count-alt}, $\Vert \textbf{K}-\tilde{\textbf{K}}\Vert_2\leq \epsilon$. Now plugging in $\frac{n}{\log n}$ into $m\log m$ we obtain
\begin{align*}
    O\left(m\log m\right)&=O\left(\frac{n}{\log n}\log \frac{n}{\log n}\right)\\
    &=O\left(\frac{n}{\log n}\log n-\frac{n}{\log n}\log \log n\right)\\
    &=O(n)
\end{align*}
as desired.
\end{proof}
% \subsubsection{Proof of Corollary \ref{cor:linear-time-low-dimensional}}
% \begin{proof}
%     We need that $m=O\left(\frac{n}{\log n}\right)$ and $m=\left( \frac{n}{\epsilon} (1 + 2c^d) K' (8 c^{2d} D^3) \right)^{d/3}$ can jointly hold. That is, that
%     \begin{align*}
%         \left( \frac{n}{\epsilon} (1 + 2c^d) K' (8 c^{2d} D^3) \right)^{d/3}&=O\left(\frac{n}{\log n}\right)
%     \end{align*}
%     Now let $C\equiv  (1 + 2c^d) K' (8 c^{2d} D^3)$. Then a sufficient condition is that the following holds for $n$ sufficiently large:
%     \begin{align*}
%         \left( \frac{n}{\epsilon}C\right)^{d/3}&\leq \frac{n}{\log n}\\
%         C^{d/3} \frac{n^{d/3-1}}{\log n}&\leq \epsilon^{d/3}\\
%         \frac{n^{d/3-1}}{\log n}&\leq \epsilon^{d/3}C^{-d/3}.
%     \end{align*}
%     For $d\leq 3$, the lhs goes to $0$ as $n\rightarrow \infty$ and thus for sufficiently large $n$, the inequality will hold.
% \end{proof}

% \subsection{Additional Quantities}\label{subsec:additional-quantities}
% \subsubsection{Action of Regularized Inverse}\label{subsubsec:action-regularized-inverse}



\section{Gaussian Processes Applications}\label{sec:gp-applications}
In this section, we address how SKI affects Gaussian Processes Applications. In Section \ref{sec:kernel-hyperparameter-estimation} we address how using the SKI kernel and log-likelihood affect hyperparameter estimation, showing that gradient ascent on the SKI log-likelihood approaches a ball around a stationary point of the true log-likelihood. In section \ref{sec:posterior-inference} we describe how using SKI affects the accuracy of posterior inference.

\subsection{Kernel Hyperparameter Estimation}\label{sec:kernel-hyperparameter-estimation}

Here we show that, for a $\mu$-smooth log-likelihood, an iterate of gradient ascent on the SKI log-likelihood approaches a neighborhood of a stationary point of the true log-likelihood at an $O\left(\frac{1}{K}\right)$ rate, with the neighborhood size determined by the SKI score function's error. To show this, we leverage a recent result for non-convex inexact gradient ascent \cite{stonyakin2023stopping}, which requires an upper bound on the SKI score function's error. This requires bounding the spectral norm error of the SKI Gram matrix's partial derivatives. In order to obtain this, we note that for many SPD kernels, under weak assumptions, the partial derivatives are \textit{also} SPD kernels, and thus we can reuse the previous results directly on the partial derivatives.

Note that \cite{stonyakin2023stopping} does not actually imply \textit{convergence} to a neighborhood of a critical point, only that at least one iterate will approach it. Given the challenges of non-concave optimization and the fact that we leverage a fairly recent result, we leave stronger results to future work.

%In order to obtain this, we need to bound the errors of the partial derivatives of the SKI gram matrix. To do so, we note that in many cases, the partial derivatives of an SPD kernel are themselves SPD kernels. This allows us to express the gram matrix's gradient error as a different gram matrix error and thus apply our previous results. 

Let $\mathcal{D}\subseteq \Theta$ be a \textit{compact} subset that we wish to optimize over. In the most precise setting we would analyze projected gradient ascent, but for simplicity we analyze gradient ascent. Let let $\tilde{k}_{\theta}: \mathcal{X} \times \mathcal{X} \rightarrow \mathbb{R}$ be the SKI approximation of $k_{\theta}: \mathcal{X} \times \mathcal{X} \rightarrow \mathbb{R}$ using $m$ inducing points and interpolation degree $L-1$. We are interested in the convergence properties of \textit{inexact gradient ascent} using the SKI log-likelihood, e.g.
\begin{align*}
    \boldsymbol{\theta}_{k+1}&=\boldsymbol{\theta}_k+\eta \nabla \tilde{\mathcal{L}}(\boldsymbol{\theta}_k),
\end{align*}
where $\eta\in \mathbb{R}$ is the learning rate and $\nabla \tilde{\mathcal{L}}(\boldsymbol{\theta}_k)$ is the SKI score function (gradient of its log-likelihood). We assume: 1) a $\mu$-smooth log-likelihood. If we optimize on a bounded domain, then for infinitely differentiable kernels (e.g. RBF) this will immediately hold. 3) That the kernel's partial derivatives are themselves SPD kernels (this can be easily shown for the RBF kernel's lengthscale by noting that the product of SPD kernels are themselves SPD kernels). 
%We then mention an equivalence between strong concavity and the Hessian having an upper bound on its smallest eigenvalue.
% 2) That the kernel is continuously differentiable. 3) That the response vectors squared $l^2$ norm grows linearly with the sample size.
\begin{assumption}[$\mu$-smooth-log-likelihood]\label{assumption:mu-smoothness}
The true log-likelihood is $\mu$-smooth over $\mathcal{D}$. That is, for all $\boldsymbol{\theta},\boldsymbol{\theta}'\in \mathcal{D}$,
\begin{align*}
    \Vert \nabla \mathcal{L}(\boldsymbol{\theta})-\nabla \mathcal{L}(\boldsymbol{\theta}')\Vert &\leq \mu \Vert \boldsymbol{\theta}-\boldsymbol{\theta}'\Vert
\end{align*}
\end{assumption}

\begin{assumption}
    (Kernel Smoothness) $k_\theta(x,x')$ is $C^1$ in $\boldsymbol{\theta}$ over $\mathcal{D}$. That is, for each $l \in \{1, ..., p\}$, $k'_{\theta_l}(x, x') = \frac{\partial k_{\theta}(x, x')}{\partial \theta_l}$ exists and is continuous for $\boldsymbol{\theta}\in \mathcal{D}$.
\end{assumption}
\begin{assumption}
    (SPD Kernel Partials) For each $l \in \{1, ..., p\}$, the partial derivative of $k_{\theta}$ with respect to a hyperparameter $\theta_l\in \mathbb{R}$, denoted as $k'_{\theta_l}(x, x') = \frac{\partial k_{\theta}(x, x')}{\partial \theta_l}$, is also a valid SPD kernel.
\end{assumption}
% \begin{assumption}
%     (Linear Squared $l^2$ Response) $\Vert \textbf{y}\Vert_2^2=O(n)$.
% \end{assumption}

We next state several results leading up to our bound on the SKI score function's error. Here we argue that we can apply the same elementwise error we derived previously to the SKI partial derivatives.

\begin{restatable}{lemma}{skikernelderivativeerrorkernel}[Bound on Derivative of SKI Kernel Error using Kernel Property of Derivative]
\label{lemma:ski_kernel_derivative_error_kernel}
 Let $\tilde{k}'_{\theta_l}(x,x')$ be the SKI approximation of $k'_{\theta_l}(x,x')$, using the same inducing points and interpolation scheme as $\tilde{k}_{\theta}$. Then, for all $x, x' \in \mathcal{X}$ and all $\boldsymbol{\theta} \in \Theta$, the following inequality holds:

\begin{align*}
\left\vert \frac{\partial k_{\theta}(x,x')}{\partial \theta_l}-\frac{\partial \tilde{k}_{\theta}(x,x')}{\partial \theta_l}\right\vert &= \left\vert k'_{\theta_l}(x, x') - \tilde{k}'_{\theta}(x, x') \right\vert \\
&\leq\delta_{m,L}'+\sqrt{L}c^d\delta_{m,L}'\\
&=O\left(\frac{c^{2d}}{m^{3/d}}\right)
\end{align*}

where $\delta_{m,L}'$ is an upper bound on the error of the SKI approximation of the kernel $k'_{\theta_l}(x,x')$ with $m$ inducing points and interpolation degree $L-1$, as defined in Lemma \ref{lemma:ski-kernel-elementwise-error}.
\end{restatable}
\begin{proof}
    See Appendix \ref{sec:proofski_kernel_derivative_error_kernelz}
\end{proof}

We then use the elementwise bound to bound the spectral norm of the SKI gram matrix's partial derivative error. This again leverages Proposition \ref{prop:spectral-norm}, noting that these partial derivatives of the Gram matrices are themselves Gram matrices.

\begin{restatable}{lemma}{partialgradientspectralnormbound}[Partial Derivative Gram Matrix Difference Bound]
\label{lemma:partial_gradient_spectral_norm_bound}
For any $l \in \{1, \dots, p\}$,

\begin{align*}
\left\| \frac{\partial \mathbf{K}}{\partial \theta_l} - \frac{\partial \tilde{\mathbf{K}}}{\partial \theta_l} \right\|_2 &\leq \gamma'_{n,m,L,l} \\
&= O\left(\frac{nc^{2d}}{m^{3/d}}\right)
\end{align*}

where $\gamma'_{n,m,L,l}$ is the bound on the spectral norm difference between the kernel matrices corresponding to $k'_{\theta_l}$ and its SKI approximation $\tilde{k}'_{\theta_l}$ (analogous to Proposition \ref{prop:spectral-norm}, but for the kernel $k'_{\theta_l}$).
\end{restatable}
\begin{proof}
See Section \ref{section:proof_partial_gradient_spectral_norm_bound}.    
\end{proof}

% \begin{restatable}{lemma}{gradientspectralnormbound}[Bound on Spectral Norm of Gradient Difference]
% \label{lemma:gradient_spectral_norm_bound}
% Assume that $k'_{\theta,i}(x, x')$, is a valid SPD kernel. Let $\tilde{k}'_{\theta,i}(x,x')$ be the SKI approximation of $k'_{\theta,i}(x,x')$, using the same inducing points and interpolation scheme as $\tilde{k}_{\theta}$. Then, the spectral norm of the difference between the gradient of the true kernel matrix and the gradient of the SKI kernel matrix is bounded by:

% \begin{align*}
% \| \nabla_\theta K - \nabla_\theta \tilde{K} \|_2 &\leq \sqrt{p} \max_{i \in \{1,\ldots,p\}} \gamma'_{n,m,L,i}\\
% &=O\left(\sqrt{p}nc^{2d}h^3\right)\text{, convolutional cubic interpolation,}
% \end{align*}

% where $\nabla_\theta K$ and $\nabla_\theta \tilde{K}$ are represented as $p \times n^2$ matrices using the vec-notation (denominator layout), $p$ is the number of hyperparameters, and $\gamma'_{n,m,L,i}$ is the bound on the spectral norm difference between the kernel matrices corresponding to $k'_{\theta,i}$ and its SKI approximation $\tilde{k}'_{\theta,i}$ (analogous to Proposition \ref{prop:spectral-norm}, but for the kernel $k'_{\theta,i}$).
% \end{restatable}

% \begin{proof}
% See Section \ref{sec:proof_gradient_spectral_norm_bound}
% \end{proof}

We now bound the SKI score function. The key insight to the proof is that the partial derivatives of the difference between regularized gram matrix inverses is in fact a difference between two quadratic forms. We can then use standard techniques \citep{horn2012matrix} for bounding the difference between quadratic forms to obtain our result. The result says that, aside from the response vector's norm, the error grows quadratically in the sample size, at a square root rate in the number of hyperparameters and exponentially in the dimensionality. It further decays at an $m^{\frac{3}{d}}$ rate in the number of inducing points. Noting that to maintain linear time, $m$ should grow at an $n^{d/3}$ rate, we have that aside from the response vector, the error in fact grows linearly with the sample size when choosing the number of inducing points based on Theorem \ref{thm:inducing-points-count-alt}.
%by an expression that is \textit{quadratic} in the sample size. The reason for the quadratic dependency is that the difference in gradients of the quadratic forms in the log-likelihoods can itself be expressed as a gradient of a quadratic form involving the response variables. The response variables contribute a linear dependence and the gradient of the difference between the regularized inverses multiplicatively contribute a linear dependence: combined this is quadratic.
%\amnote{In an attempt to improve the bound, I got something that doesn't decrease with the number of inducing points, which is bad.}
\begin{restatable}{lemma}{scorefunctionbound}[Score Function Bound]\label{lemma:score-function-bound}
Let $\mathcal{L}(\boldsymbol{\theta})$ be the true log-likelihood and $\tilde{\mathcal{L}}(\boldsymbol{\theta})$ be the SKI approximation of the log-likelihood at $\boldsymbol{\theta}$. Let $\nabla \mathcal{L}(\boldsymbol{\theta})$ and $\nabla \tilde{\mathcal{L}}(\boldsymbol{\theta})$ denote their respective gradients with respect to $\boldsymbol{\theta}$. Then, for any $\boldsymbol{\theta}\in \mathcal{D}$,

\begin{align*}
&\| \nabla \mathcal{L}(\boldsymbol{\theta}) - \nabla \tilde{\mathcal{L}}(\boldsymbol{\theta}) \|_2 \\
&\leq \frac{1}{2\sigma^4}\Vert \textbf{y}\Vert\sqrt{p}\max_{1\leq l\leq p} \left( \gamma'_{n,m,L,l}+Cn\gamma_{n,m,L}\right.\\
&\qquad\left.+\gamma_{n,m,L}\gamma'_{n,m,L,l} \right)+\frac{\gamma_{n,m,L}}{2\sigma^4}\\
&=\Vert \textbf{y}\Vert_2 O\left(\frac{\sqrt{p}n^2c^{4d}}{m^{3/d}}\right)\\
&\equiv \epsilon_G
\end{align*}
where $C$ is a constants depending on the upper bound of the derivatives of the kernel function over $\mathcal{D}$.
\end{restatable}

% \begin{restatable}{lemma}{scorefunctionbound}[Score Function Bound]\label{lemma:score-function-bound}
% Let $\mathcal{L}(\theta)$ be the true log-likelihood and $\tilde{\mathcal{L}}(\theta)$ be the SKI approximation of the log-likelihood at $\theta$. Let $\nabla \mathcal{L}(\theta)$ and $\nabla \tilde{\mathcal{L}}(\theta)$ denote their respective gradients with respect to $\theta$. Then, for any $\theta\in \mathcal{D}$,
% \begin{align*}
% \| \nabla \mathcal{L}(\theta) - \nabla \tilde{\mathcal{L}}(\theta) \|_2 &\leq \frac{1}{\sigma^4}\left[\Vert y\Vert_2^2\sqrt{p}\max_{1\leq l\leq p}\left( \gamma'_{n,m,L,l}+Cn\gamma_{n,m,L}+\frac{\gamma_{n,m,L}}{\sigma^4} \gamma'_{n,m,L,l}\right)+\frac{1}{2}\gamma_{n,m,L}\right]\\
% &=O(\sqrt{p}n^3c^{2d}h^3)\\
% &\equiv \epsilon_G
% \end{align*}
% where $\gamma_{n,m,L}$ bounds the elementwise difference between $\mathbf{K}$ and $\tilde{\mathbf{K}}$, and $\gamma'_{n,m,L,i}$ is the bound on the spectral norm difference between the kernel matrices corresponding to $k'_{\theta,i}$ and its SKI approximation $\tilde{k}'_{\theta,i}$ and $C_n$ is a constant depending on the upper bound of the derivatives of the kernel function over $\mathcal{D}$ and the sample size.
% \end{restatable}

\begin{proof}
See Section \ref{sec:proof-score-function-bound}.
\end{proof}

We apply \cite{stonyakin2023stopping} below: the result is the same as in their paper (and assumes $\mu$-smoothness as we did on $\mathcal{L}$), but using gradient ascent instead of descent and using the score function error above. It says that at an $O\left(\frac{1}{K}\right)$ rate, at least one iterate of gradient ascent has its squared gradient norm approach a neighborhood proportional to the squared SKI score function's spectral norm error.

\begin{theorem} \citep{stonyakin2023stopping}
    For inexact gradient ascent on $\mathcal{L}$ with additively inexact gradients satisfying $\|\nabla \mathcal{L}(\boldsymbol{\theta}) - \nabla \tilde{\mathcal{L}}(\boldsymbol{\theta})\| \leq \epsilon_g$, we have:

\begin{equation}
    \max_{k=0,...,N-1} \|\nabla \mathcal{L}(\theta_k)\|^2 \leq \frac{2\mu(\mathcal{L}^* - \mathcal{L}(\boldsymbol{\theta}_0))}{K} + \frac{\epsilon_g^2}{2\mu}
\end{equation}

where $\mathcal{L}^*$ is the value at a stationary point, $\mathcal{L}(\boldsymbol{\theta}_0)$ is the initial, function value, $K$ is the number of iterations and $\epsilon_g$ is the gradient error bound in the previous Lemma.

\end{theorem}

\subsection{Posterior Inference}\label{sec:posterior-inference}
Finally, we treat posterior inference. As the current hyperparameter optimization results only say that \textit{some} iterate approaches a stationary point, we will focus on the error when the SKI and true kernel hyperparameter match.  We first add an assumption
\begin{assumption}
    (Bounded Kernel) Assume that the true kernel satisfies the condition that $|k(\mathbf{x}, \mathbf{x}')| \leq M$ for all $\mathbf{x}, \mathbf{x}'\in \mathcal{X}$.
\end{assumption}

% Consider on a compact domain,
% \begin{align*}
%     \vert k_\theta(x,x')-k_{\theta'}(x,x')\vert &\leq \rho \Vert \theta-\theta'\Vert
% \end{align*}

% which implies that
% \begin{align*}
%     \Vert \textbf{K}^{(\theta)}-\textbf{K}^{(\theta')}\Vert_2&\leq n\rho \Vert \theta-\theta'\Vert
% \end{align*}
% and we can then 
Now we bound the spectral error for the SKI mean function evaluated at a set of test points. The proof follows a standard strategy commonly used for approximate kernel ridge regression. See \cite{bach2013sharp,musco2017recursive} for examples. The result says that the $l^2$ error (aside from the response vector) grows exponentially in the dimensionality, super-linearly but sub-quadratically in the training sample size and at worst linearly in the test sample size. It decays at an $m^{\frac{3}{d}}$ rate in the number of inducing points. Similarly to for the score function error, if we follow Theorem \ref{thm:inducing-points-count-alt} for selecting the number of inducing points, the error in fact grows \textit{sublinearly} with the training sample size.
\begin{restatable}{lemma}{meaninference}\label{lemma:mean-inference} (SKI Posterior Mean Error)
    Let $\boldsymbol{\mu}(\cdot)$ be the GP posterior mean at a set of test points $\cdot\in \mathbb{R}^{T\times d}$ and $\tilde{\boldsymbol{\mu}}(\cdot)$ be the SKI posterior mean at those points. Then the SKI posterior mean $l^2$ error is bounded by:
{\footnotesize
\begin{align*}
    &\Vert \tilde{\boldsymbol{\mu}}(\cdot)- \boldsymbol{\mu}(\cdot)\Vert_2\\
    &\leq\left(\frac{\max(\gamma_{T,m,L},\gamma_{n,m,L})}{\sigma^2}+\frac{\sqrt{Tn}Mc^{2d}}{\sigma^4}\gamma_{n,m,L}\right)\Vert \textbf{y}\Vert_2\\
    &=\Vert \textbf{y}\Vert_2O\left(c^{2d}\frac{\max(T,n)+\sqrt{Tn}n}{m^{3/d}}\right)
    %\left(\frac{\max(\gamma_{T,m,L},\gamma_{n,m,L})}{\sigma^2}+\frac{\Vert \tilde{\mathbf{K}}_{\cdot, \mathbf{X}} \Vert_2}{\sigma^4}\gamma_{n,m,L}\right)\Vert \textbf{y}\Vert_2
\end{align*}
}
\end{restatable}


\begin{proof}
See Appendix \ref{sec:proof-mean-inference}.
\end{proof}

We now derive the spectral error bound for the test SKI covariance matrix. The proof involves noticing that a key term is a difference between two quadratic forms, and using standard techniques for bounding such a difference. The result shows that the error grows at worst super-linearly but subquadratically in the number of test points, quadratically in the training sample size and exponentially in the dimension. Interestingly, due to the use of standard techniques for bounding the difference between quadratic forms, the error is only guaranteed to decay with the number of inducing points at an $m^{3/d-1}$ rate, so that it is only guaranteed to decay at all if $d<3$. If we select the number of inducing points to be proportional to $n^{d/3}$, then the error grows at rate $n^{1+d/3}$ for $d<3$. An interesting question is whether alternate techniques can improve the result for higher dimensional settings e.g. $d\geq 3$.

\begin{restatable}{lemma}{skiposteriorcovarianceerror}[SKI Posterior Covariance Error]\label{lemma:ski-posterior-covariance-error}
Let $\boldsymbol{\Sigma}(\cdot)$ be the GP posterior covariance matrix at a set of test points $\cdot\in \mathbb{R}^{T\times d}$ and $\tilde{\boldsymbol{\Sigma}}(\cdot)$ be its SKI approximation. Then
\begin{align*}
    &\Vert \boldsymbol{\Sigma}(\cdot)-\tilde{\boldsymbol{\Sigma}}(\cdot)\Vert_2\\ &\leq \gamma_{T,m,L} + \frac{\sqrt{Tn}M}{\sigma^2} \max(\gamma_{T,m,L},\gamma_{n,m,L})\\
    &\quad+ \frac{\gamma_{n,m,L}}{\sigma^4}Tn m c^{2d} M^2 \\
    &\quad+ \frac{\sqrt{Tn} m c^{2d} M}{\sigma^2} \max(\gamma_{T,m,L},\gamma_{n,m,L}).\\
    &=O\left(\frac{Tn^2mc^{4d}+\sqrt{Tn}mc^{4d}\max(T,n)}{m^{3/d}}\right).
\end{align*}
where $\gamma_{T,m,L}$ is defined as in Proposition \ref{prop:spectral-norm}.
\end{restatable}

\begin{proof}
See Appendix \ref{sec:proof-ski-posterior-covariance-error}

\end{proof}




\section{Discussion}\label{sec:discussion}


In this paper, we provided theoretical analysis for structured kernel interpolation. In particular, we analyzed the error of tensor-product cubic convolutional interpolation, showed the elementwise SKI kernel error and how that propagates to spectral norm error of the SKI gram and cross-kernel matrices, and showed how this impacts achieving a specific error in linear time. We then analyzed kernel hyperparameter estimation in Gaussian processes, showing that gradient ascent has an iterate approach a ball around a stationary point with size quadratic in the sample size.


\bibliography{main}
\bibliographystyle{icml2021}


\appendix


\section{Auxiliary Technical Results}

\begin{lemma}\label{lemma:switch-sum-product}
Given a function $f: \mathbb{R}^d \rightarrow \mathbb{R}$ of the form $f(x_1, x_2, ..., x_d) = \prod_{j=1}^d f_j(x_j)$, where each $f_j: \mathbb{R} \rightarrow \mathbb{R}$. Let $G = G^{(1)} \times G^{(2)} \times ... \times G^{(d)}$ be a fixed d-dimensional grid, where each $G^{(j)} = \{p_1^{(j)}, p_2^{(j)}, ..., p_{n_j}^{(j)}\}$ is a finite set of $n_j$ grid points along the j-th dimension for $j = 1, 2, ..., d$. Then the following equality holds:

$$
\sum_{k_1=1}^{n_1} \sum_{k_2=1}^{n_2} ... \sum_{k_d=1}^{n_d} \prod_{j=1}^d f_j(p_{k_j}^{(j)}) = \prod_{j=1}^d \left( \sum_{k_j=1}^{n_j} f_j(p_{k_j}^{(j)}) \right)
$$
\end{lemma}
\begin{proof}
\textbf{By Induction on d (the number of dimensions):}

\textbf{Base Case (d = 1):}

When $d=1$, the statement becomes:

$$
\sum_{k_1=1}^{n_1} f_1(p_{k_1}^{(1)}) = \sum_{k_1=1}^{n_1} f_1(p_{k_1}^{(1)})
$$

This is trivially true.

\textbf{Inductive Hypothesis:}

Assume the statement holds for $d = m$, i.e.,

$$
\sum_{k_1=1}^{n_1} \sum_{k_2=1}^{n_2} ... \sum_{k_m=1}^{n_m} \prod_{j=1}^m f_j(p_{k_j}^{(j)}) = \prod_{j=1}^m \left( \sum_{k_j=1}^{n_j} f_j(p_{k_j}^{(j)}) \right)
$$

\textbf{Inductive Step:}

We need to show that the statement holds for $d = m+1$. Consider the left-hand side for $d = m+1$:

$$
\sum_{k_1=1}^{n_1} \sum_{k_2=1}^{n_2} ... \sum_{k_{m+1}=1}^{n_{m+1}} \prod_{j=1}^{m+1} f_j(p_{k_j}^{(j)})
$$

We can rewrite this as:

$$
\sum_{k_1=1}^{n_1} \sum_{k_2=1}^{n_2} ... \sum_{k_m=1}^{n_m} \left( \sum_{k_{m+1}=1}^{n_{m+1}} \left( \prod_{j=1}^m f_j(p_{k_j}^{(j)}) \right) f_{m+1}(p_{k_{m+1}}^{(m+1)}) \right)
$$

Notice that the inner sum (over $k_{m+1}$) does not depend on $k_1, k_2, ..., k_m$. Thus, for any fixed values of $k_1, k_2, ..., k_m$, we can treat $\prod_{j=1}^m f_j(p_{k_j}^{(j)})$ as a constant. Let $C(k_1, ..., k_m) = \prod_{j=1}^m f_j(p_{k_j}^{(j)})$. Then we have:

$$
\sum_{k_1=1}^{n_1} \sum_{k_2=1}^{n_2} ... \sum_{k_m=1}^{n_m}  \left( C(k_1, ..., k_m) \sum_{k_{m+1}=1}^{n_{m+1}} f_{m+1}(p_{k_{m+1}}^{(m+1)}) \right)
$$

Now, the inner sum $\sum_{k_{m+1}=1}^{n_{m+1}} f_{m+1}(p_{k_{m+1}}^{(m+1)})$ is a constant with respect to $k_1, ..., k_m$. Let's call this constant $S_{m+1}$. So we have:

$$
\sum_{k_1=1}^{n_1} \sum_{k_2=1}^{n_2} ... \sum_{k_m=1}^{n_m} C(k_1, ..., k_m) S_{m+1} = S_{m+1} \sum_{k_1=1}^{n_1} \sum_{k_2=1}^{n_2} ... \sum_{k_m=1}^{n_m}  \prod_{j=1}^m f_j(p_{k_j}^{(j)})
$$

By the inductive hypothesis, we can replace the nested sums with a product:

$$
S_{m+1} \prod_{j=1}^m \left( \sum_{k_j=1}^{n_j} f_j(p_{k_j}^{(j)}) \right) = \left( \sum_{k_{m+1}=1}^{n_{m+1}} f_{m+1}(p_{k_{m+1}}^{(m+1)}) \right) \prod_{j=1}^m \left( \sum_{k_j=1}^{n_j} f_j(p_{k_j}^{(j)}) \right)
$$

Rearranging the terms, we get:

$$
\prod_{j=1}^m \left( \sum_{k_j=1}^{n_j} f_j(p_{k_j}^{(j)}) \right) \left( \sum_{k_{m+1}=1}^{n_{m+1}} f_{m+1}(p_{k_{m+1}}^{(m+1)}) \right) = \prod_{j=1}^{m+1} \left( \sum_{k_j=1}^{n_j} f_j(p_{k_j}^{(j)}) \right)
$$

This is the right-hand side of the statement for $d = m+1$. Thus, the statement holds for $d = m+1$.

\textbf{Conclusion:}

By induction, the statement holds for all $d \geq 1$. Therefore,

$$
\sum_{k_1=1}^{n_1} \sum_{k_2=1}^{n_2} ... \sum_{k_d=1}^{n_d} \prod_{j=1}^d f_j(p_{k_j}^{(j)}) = \prod_{j=1}^d \left( \sum_{k_j=1}^{n_j} f_j(p_{k_j}^{(j)}) \right)
$$

\end{proof}


\begin{claim}\label{claim:convex-combo-eigenvalues}
Given a convex combination \(\mathbf{C} = \alpha \mathbf{A} + (1-\alpha) \mathbf{B}\), where \(\alpha \in [0,1]\), and \(\mathbf{A}\) and \(\mathbf{B}\) are symmetric matrices, the eigenvalues of \(\mathbf{C}\) lie in the interval \(\left[\min \left(\lambda_n(\mathbf{A}), \lambda_n(\mathbf{B})\right), \max \left(\lambda_1(\mathbf{A}), \lambda_1(\mathbf{B})\right)\right]\).
    % Given a convex combination $C=\alpha A+(1-\alpha)B,\alpha \in [0,1]$, the eigenvalues of $C$ lie in the interval $[\min(\lambda(A)_n, \lambda(B)_n), \max(\lambda(A)_1,\lambda(B)_1)]$
\end{claim}
\begin{proof}


First, recall that for a symmetric matrix \(\mathbf{A}\), the Rayleigh quotient \(R(\mathbf{A}, \mathbf{x}) = \frac{\mathbf{x}^{\top} \mathbf{A} \mathbf{x}}{\mathbf{x}^{\top} \mathbf{x}}\) is bounded by the smallest and largest eigenvalues of \(\mathbf{A}\):
\[
\lambda_n(\mathbf{A}) \leq R(\mathbf{A}, \mathbf{x}) \leq \lambda_1(\mathbf{A})
\]

Consider the Rayleigh quotient for the matrix \(\mathbf{C}\):
\[
R(\mathbf{C}, \mathbf{x}) = \frac{\mathbf{x}^{\top} (\alpha \mathbf{A} + (1-\alpha) \mathbf{B}) \mathbf{x}}{\mathbf{x}^{\top} \mathbf{x}} = \alpha R(\mathbf{A}, \mathbf{x}) + (1-\alpha) R(\mathbf{B}, \mathbf{x})
\]

Since \(R(\mathbf{A}, \mathbf{x})\) and \(R(\mathbf{B}, \mathbf{x})\) are bounded by their respective eigenvalues, we have:
\[
R(\mathbf{C}, \mathbf{x}) \leq \alpha \lambda_1(\mathbf{A}) + (1-\alpha) \lambda_1(\mathbf{B})
\]
which implies:
\[ 
R(\mathbf{C}, x)  \leq   \max(\lambda_1(\textbf{A}), \lambda_1(\textbf{B}))
\]

Similarly,
\[ 
R(\textbf{C}, \textbf{x})  \geq  \min(\lambda_n(\textbf{A}), \lambda_n(\textbf{B}))
\]

Thus, the eigenvalues of \(\textbf{C} = \alpha \textbf{A} + (1-\alpha)\textbf{B}\) are bounded by:
\[ 
\min(\lambda_n(\textbf{A}), \lambda_n(\textbf{B})) \leq  \lambda(\textbf{C}) \leq \max(\lambda_1(\textbf{A}), \lambda_1(\textbf{B}))
\]
\end{proof}

\section{Proofs Related to Important Quantities}
\subsection{Proofs Related to Ski Kernel Error Bounds}\label{subsec:proofs-ski-kernel-error}
\subsubsection{Proof of Lemma \ref{lemma:tensor-product-interpolation-error}}\label{sec:proof-tensor-product-interpolation}
\tensorproductinterpolationerror*
\begin{proof}
We define a sequence of intermediate interpolation functions. Let $g_0(\mathbf{x}) \equiv f(\mathbf{x})$ be the original function. For $i = 1, \ldots, d$, we recursively define $g_i(\mathbf{x})$ as the function obtained by interpolating $g_{i-1}$ along the $i$-th dimension using the cubic convolution kernel $u$:

$$
g_i(\mathbf{x}) \equiv \sum_{k=-1}^2 g_{i-1}\left(\mathbf{x} + \left( (\mathbf{c_x})_i - x_i + kh\right)\mathbf{e}_i \right) u\left(\frac{x_i - (\mathbf{c_x})_i - kh}{h}\right).
$$

Here, $\mathbf{c_x}$ is the grid point closest to $\mathbf{x}$, and $\mathbf{e}_i$ is the $i$-th standard basis vector. Thus, $g_1(\mathbf{x})$ interpolates $f$ along the first dimension, $g_2(\mathbf{x})$ interpolates $g_1$ along the second dimension, and so on, until $g_d(\mathbf{x}) = g(\mathbf{x})$ is the final tensor-product interpolated function.

We analyze the error accumulation across multiple dimensions using induction. Using \cite{keys1981cubic}, the error introduced by interpolating a thrice continuous differentiable function along a single dimension with the cubic convolution kernel is uniformly bounded over the interval domain by $Kh^3$ for some constant $K > 0$, provided the grid spacing $h$ is sufficiently small. This gives us the base case:
$$
|g_1(\mathbf{x}) - g_0(\mathbf{x})| \leq Kh^3.
$$

For the inductive step, assume that for some $i=k$ the error is uniformly bounded by
$$
|g_k(\mathbf{x}) - g_{k-1}(\mathbf{x})| \leq c^{k-1}Kh^3.
$$

We want to show that this bound also holds for $i=k+1$. We can express the difference $g_{k+1}(\mathbf{x}) - g_k(\mathbf{x})$ as follows:

\begin{align*}
g_{k+1}(\mathbf{x}) - g_k(\mathbf{x}) &= \sum_{k_{k+1}=-1}^2 g_k\left(\mathbf{x} + ((\mathbf{c_x})_{k+1} - x_{k+1} + k_{k+1}h) \mathbf{e}_{k+1} \right) u\left(\frac{x_{k+1} - (\mathbf{c_x})_{k+1} - k_{k+1}h}{h}\right) \\
&\quad - g_k(\mathbf{x}) \\
&= \sum_{k_{k+1}=-1}^2 \left[\sum_{k_k=-1}^2 g_{k-1}\left(\mathbf{x} + ((\mathbf{c_x})_k - x_k + k_k h) \mathbf{e}_k + ((\mathbf{c_x})_{k+1} - x_{k+1} + k_{k+1}h) \mathbf{e}_{k+1} \right) \right. \\
&\quad \left. u\left(\frac{x_k - (\mathbf{c_x})_k - k_k h}{h}\right)\right] u\left(\frac{x_{k+1} - (\mathbf{c_x})_{k+1} - k_{k+1}h}{h}\right) \\
&\quad - \sum_{k_k=-1}^2 g_{k-1}\left(\mathbf{x} + ((\mathbf{c_x})_k - x_k + k_k h) \mathbf{e}_k \right) u\left(\frac{x_k - (\mathbf{c_x})_k - k_k h}{h}\right) \\
&= \sum_{k_k=-1}^2 u\left(\frac{x_k - (\mathbf{c_x})_k - k_k h}{h}\right) \left[\sum_{k_{k+1}=-1}^2 g_{k-1}\left(\mathbf{x} + ((\mathbf{c_x})_k - x_k + k_k h) \mathbf{e}_k + ((\mathbf{c_x})_{k+1} - x_{k+1} + k_{k+1}h) \mathbf{e}_{k+1} \right) \right. \\
&\quad \left. u\left(\frac{x_{k+1} - (\mathbf{c_x})_{k+1} - k_{k+1}h}{h}\right) - g_{k-1}\left(\mathbf{x} + ((\mathbf{c_x})_k - x_k + k_k h) \mathbf{e}_k\right)\right].
\end{align*}

The inner term in the last expression represents the difference between interpolating $g_{k-1}$ along the $(k+1)$-th dimension and $g_{k-1}$ itself, evaluated at $\mathbf{x} + ((\mathbf{c_x})_k - x_k + k_k h) \mathbf{e}_k$. This can be written as:

\begin{align*}
&\sum_{k_{k+1}=-1}^2 g_{k-1}\left(\mathbf{x} + ((\mathbf{c_x})_k - x_k + k_k h) \mathbf{e}_k + ((\mathbf{c_x})_{k+1} - x_{k+1} + k_{k+1}h) \mathbf{e}_{k+1} \right) u\left(\frac{x_{k+1} - (\mathbf{c_x})_{k+1} - k_{k+1}h}{h}\right) \\
&\quad - g_{k-1}\left(\mathbf{x} + ((\mathbf{c_x})_k - x_k + k_k h) \mathbf{e}_k\right) \\
&= g_k\left(\mathbf{x} + ((\mathbf{c_x})_k - x_k + k_k h) \mathbf{e}_k \right) - g_{k-1}\left(\mathbf{x} + ((\mathbf{c_x})_k - x_k + k_k h) \mathbf{e}_k \right).
\end{align*}

Therefore, we can bound the error as follows:

\begin{align*}
|g_{k+1}(\mathbf{x}) - g_k(\mathbf{x})| &\leq \left|\sum_{k_k=-1}^2 u\left(\frac{x_k - (\mathbf{c_x})_k - k_k h}{h}\right)\right| \cdot \left|g_k\left(\mathbf{x} + ((\mathbf{c_x})_k - x_k + k_k h) \mathbf{e}_k \right) - g_{k-1}\left(\mathbf{x} + ((\mathbf{c_x})_k - x_k + k_k h) \mathbf{e}_k \right)\right|.
\end{align*}

Let $c>0$ be a uniform upper bound for $\sum_{k_k=-1}^2 \left|u\left(\frac{x_k - (\mathbf{c_x})_k - k_k h}{h}\right)\right|$, which exists because $u$ is bounded. By the inductive hypothesis, we have $\left|g_k\left(\mathbf{x} + ((\mathbf{c_x})_k - x_k + k_k h) \mathbf{e}_k \right) - g_{k-1}\left(\mathbf{x} + ((\mathbf{c_x})_k - x_k + k_k h) \mathbf{e}_k \right)\right| \leq c^{k-1}Kh^3$. Thus,

$$
|g_{k+1}(\mathbf{x}) - g_k(\mathbf{x})| \leq c \cdot c^{k-1}Kh^3 = c^k Kh^3.
$$

This completes the inductive step.

Finally, we bound the total error $|g(\mathbf{x}) - f(\mathbf{x})| = |g_d(\mathbf{x}) - g_0(\mathbf{x})|$ by summing the errors introduced at each interpolation step:

$$
|g(\mathbf{x}) - f(\mathbf{x})| \leq \sum_{i=1}^d |g_i(\mathbf{x}) - g_{i-1}(\mathbf{x})| \leq \sum_{i=1}^d c^{i-1}Kh^3 = Kh^3 \sum_{i=0}^{d-1} c^i.
$$

The last sum is a geometric series, which evaluates to $Kh^3 \frac{1 - c^d}{1 - c}$. For a fixed $c>1$ (independent of $d$), this expression is $O(c^{d})$ when $d$ is large. Therefore, tensor-product cubic convolutional interpolation has $O(c^d h^3)$ error. Finally, noticing that $h=O\left(\frac{1}{m^{d/3}}\right)$ gives us the desired result.
\end{proof}

\subsubsection{Curse of Dimensionality for Kernel Regression}

The next lemma shows that when using a product kernel for $d$-dimensional kernel regression (where cubic convolutional interpolation is a special case), the sum of weights suffers from the curse of dimensionality. The proof strategy involves expressing the multi-dimensional sum as a product of sums over each individual dimension, leveraging the initial condition on the one-dimensional bound for each dimension, and taking advantage of the structure of the Cartesian grid.


\begin{restatable}{lemma}{cubicInterpolationWeights}
\label{lemma:cubic-interpolation-weights-curse-dimensionality}
Let $u: \mathbb{R} \rightarrow \mathbb{R}$ be a one-dimensional kernel function with constant $c>0$ defined as in \ref{def:sum-weight-upper-bound}. Let $u_d: \mathbb{R}^d \rightarrow \mathbb{R}$ be a d-dimensional product kernel defined as:
$$
u_d\left(\frac{x - x_i}{h}\right) = \prod_{j=1}^d u\left(\frac{x^{(j)} - x_i^{(j)}}{h}\right),
$$
where $x = (x^{(1)}, x^{(2)}, ..., x^{(d)}) \in \mathbb{R}^d$ and $x_i = (x_i^{(1)}, x_i^{(2)}, ..., x_i^{(d)}) \in \mathbb{R}^d$ are d-dimensional points. Assume the data points $\{x_i\}_{i=1}^n$ ($n$ may differ from the univariate case) lie on a fixed d-dimensional grid $G = G^{(1)} \times G^{(2)} \times ... \times G^{(d)}$, where each $G^{(j)} = \{p_1^{(j)}, p_2^{(j)}, ..., p_{n_j}^{(j)}\}$ is a finite set of $n_j$ grid points along the j-th dimension for $j = 1, 2, ..., d$. Then, for any $x \in \mathbb{R}^d$, the sum of weights in the d-dimensional kernel regression is bounded by $c^d$:
$$
\sum_{i=1}^n \left|u_d\left(\frac{x - x_i}{h}\right) \right|\leq c^d.
$$
\end{restatable}

\begin{proof}
Let the fixed d-dimensional grid be defined by the Cartesian product of d sets of 1-dimensional grid points: $G = G^{(1)} \times G^{(2)} \times ... \times G^{(d)}$, where $G^{(j)} = \{p_1^{(j)}, p_2^{(j)}, ..., p_{n_j}^{(j)}\}$ is the set of grid points along the j-th dimension.

We start with the sum of weights in the d-dimensional case:

$$
\sum_{i=1}^n u_d\left(\frac{x - x_i}{h}\right) = \sum_{i=1}^n \prod_{j=1}^d u\left(\frac{x^{(j)} - x_i^{(j)}}{h}\right)
$$

Since the data points lie on the fixed grid $G$, we can rewrite the outer sum as a nested sum over the grid points in each dimension:

$$
\sum_{i=1}^n \prod_{j=1}^d u\left(\frac{x^{(j)} - x_i^{(j)}}{h}\right) = \sum_{k_1=1}^{n_1} \sum_{k_2=1}^{n_2} ... \sum_{k_d=1}^{n_d} \prod_{j=1}^d u\left(\frac{x^{(j)} - p_{k_j}^{(j)}}{h}\right)
$$

Now we can change the order of summation and product, as proven in Lemma \ref{lemma:switch-sum-product}:

$$
\sum_{k_1=1}^{n_1} \sum_{k_2=1}^{n_2} ... \sum_{k_d=1}^{n_d} \prod_{j=1}^d u\left(\frac{x^{(j)} - p_{k_j}^{(j)}}{h}\right) = \prod_{j=1}^d \left( \sum_{k_j=1}^{n_j} u\left(\frac{x^{(j)} - p_{k_j}^{(j)}}{h}\right) \right)
$$

By the assumption of the lemma, we know that for each dimension $j$, the sum of weights is bounded by $c$. Note that $\{p_{k_j}^{(j)}\}_{k_j=1}^{n_j}$ is simply a set of points in $\mathbb{R}$, thus:

$$
\sum_{k_j=1}^{n_j} \left|u\left(\frac{x^{(j)} - p_{k_j}^{(j)}}{h}\right)\right| \leq c
$$

Therefore, we have:

$$
\prod_{j=1}^d \left(\left| \sum_{k_j=1}^{n_j} u\left(\frac{x^{(j)} - p_{k_j}^{(j)}}{h}\right)\right| \right) \leq \prod_{j=1}^d c = c^d
$$

Thus, we have shown that:

$$
\sum_{i=1}^n \left|u_d\left(\frac{x - x_i}{h}\right) \right|\leq c^d
$$

\end{proof}





% \subsubsection{Elementwise}\label{subsubsec:proofs-elementwise}
\subsubsection{Proof of Lemma \ref{lemma:ski-kernel-elementwise-error}}\label{sec:proof-ski-kernel-elementwise-error}
\skikernelelementwiseerror*
\begin{proof}
Recall that SKI approximates the kernel as
\begin{align*}
k(\mathbf{x}, \mathbf{x}') &\approx \tilde{k}(\mathbf{x}, \mathbf{x}')\\
&= \boldsymbol{w}(\mathbf{x})^\top \mathbf{K}_{\textbf{U}} \boldsymbol{w}(\mathbf{x}'),
\end{align*}

Let $\textbf{K}_{\textbf{U},\textbf{x}'}\in \mathbb{R}^m$ be the vector of kernels between the inducing points and the vector $\textbf{x}'$
\begin{align}
\vert k(\mathbf{x}, \mathbf{x}') -\tilde{k}(\mathbf{x}, \mathbf{x}')\vert &= \vert k(\mathbf{x}, \mathbf{x}')-\boldsymbol{w}(\mathbf{x})^\top \textbf{K}_{\textbf{U},\textbf{x}'} +\boldsymbol{w}(\mathbf{x})^\top \textbf{K}_{\textbf{U},\textbf{x}'}-\boldsymbol{w}(\mathbf{x})^\top \mathbf{K}_{\textbf{U}} \boldsymbol{w}(\mathbf{x}')\vert\nonumber\\
&\leq  \vert k(\mathbf{x}, \mathbf{x}')-\boldsymbol{w}(\mathbf{x})^\top \textbf{K}_{\textbf{U},\textbf{x}'} \vert+\vert \boldsymbol{w}(\mathbf{x})^\top \textbf{K}_{\textbf{U},\textbf{x}'}-\boldsymbol{w}(\mathbf{x})^\top \mathbf{K}_{\textbf{U}} \boldsymbol{w}(\mathbf{x}')\vert\nonumber\\
&\leq \delta_{m,L}+\vert \boldsymbol{w}(\mathbf{x})^\top \textbf{K}_{\textbf{U},\textbf{x}'}-\boldsymbol{w}(\mathbf{x})^\top \mathbf{K}_{\textbf{U}} \boldsymbol{w}(\mathbf{x}')\vert \textrm{ since $\vert k(\mathbf{x}, \mathbf{x}')-\boldsymbol{w}(\mathbf{x})^\top \textbf{K}_{\textbf{U},\textbf{x}'} \vert$ is a single polynomial interpolation}\label{eqn:applying-single-poly-interp}
\end{align}
Now note that $\textbf{w}(x)\in \mathbb{R}^m$ is a sparse matrix with at most $L$ non-zero entries. Thus, letting $\tilde{\textbf{w}}(x)\in \mathbb{R}^L$ be the non-zero entries of $\textbf{w}(x)$ and similarly $\tilde{\textbf{K}}_{\textbf{U},\textbf{x}'}\in \mathbb{R}^L$ be the entries of $\textbf{K}_{\textbf{U},\textbf{x}'}$ in the dimensions corresponding to non-zero entries of $\textbf{w}(x)\in \mathbb{R}^m$, while $\tilde{\textbf{K}}_{\textbf{U}}\in \mathbb{R}^{L\times m}$ is the analogous matrix for $\textbf{K}_{\textbf{U}}$, we have
\begin{align}
    \vert \boldsymbol{w}(\mathbf{x})^\top \textbf{K}_{\textbf{U},\textbf{x}'}-\boldsymbol{w}(\mathbf{x})^\top \mathbf{K}_{\textbf{U}} \boldsymbol{w}(\mathbf{x}')\vert&=\vert \tilde{\textbf{w}}(\textbf{x})^\top \tilde{\textbf{K}}_{\textbf{U},\textbf{x}'}-\tilde{\textbf{w}}(\textbf{x})^\top \tilde{\textbf{K}}_\textbf{U}\textbf{w}(\textbf{x}')\vert\nonumber\\
    &\leq \Vert \tilde{\textbf{w}}(\textbf{x})\Vert_2\Vert \tilde{\textbf{K}}_{\textbf{U},\textbf{x}'}-\tilde{\mathbf{K}}_{\textbf{U}} \boldsymbol{w}(\mathbf{x}')\Vert_2\nonumber\\
    &\leq c^d\sqrt{L}\Vert \tilde{\textbf{K}}_{\textbf{U},\textbf{x}'}-\tilde{\mathbf{K}}_{\textbf{U}} \boldsymbol{w}(\mathbf{x}')\Vert_\infty\nonumber\text{ Lemma \ref{lemma:cubic-interpolation-weights-curse-dimensionality}}\\
    &\leq \sqrt{L}c^d\delta_{m,L}\label{eqn:applying-sparsity-single-kernel-evaluation}
\end{align}
where the last line follows as each element of $\mathbf{K}_{\textbf{U}} \boldsymbol{w}(\mathbf{x}')$ is a polynomial interpolation approximating each element of $\textbf{K}_{\textbf{U},\textbf{x}'}$. Plugging Eqn. \ref{eqn:applying-sparsity-single-kernel-evaluation} into Eqn. \ref{eqn:applying-single-poly-interp} gives us the desired initial result of
\begin{align*}
    \vert k(\textbf{x},\textbf{x}')-\tilde{k}(\textbf{x},\textbf{x}')\vert&\leq \delta_{m,L}+\sqrt{L}c^d\delta_{m,L}
\end{align*}
and Lemma \ref{lemma:tensor-product-interpolation-error} gives us the result when the convolutional kernel is cubic.
\end{proof}

% \subsubsection{Spectral Norm}\label{subsubsec:spectral-norm}

\subsubsection{Proof of Proposition \ref{prop:spectral-norm}}\label{sec:proof-spectral-norm}

\spectralnorm*
\begin{proof}
    Recall that for any matrix $\textbf{A}$, $\Vert \textbf{A} \Vert_2 \leq \sqrt{\Vert \textbf{A} \Vert_1 \Vert \textbf{A} \Vert_\infty}$. Since $\textbf{K}-\tilde{\textbf{K}}$ is symmetric, we have
    \begin{align*}
        \Vert \textbf{K}-\tilde{\textbf{K}}\Vert_2&\leq \sqrt{\Vert \textbf{K}-\tilde{\textbf{K}}\Vert_1\Vert \textbf{K}-\tilde{\textbf{K}}\Vert_\infty} = \Vert \textbf{K}-\tilde{\textbf{K}}\Vert_\infty
    \end{align*}
    Furthermore, $\Vert \textbf{K}-\tilde{\textbf{K}}\Vert_\infty$ is the maximum absolute row sum of $\textbf{K}-\tilde{\textbf{K}}$. Since there are $n$ rows and, by Lemma \ref{lemma:ski-kernel-elementwise-error}, each element of $\textbf{K} - \tilde{\textbf{K}}$ is bounded by $\delta_{m,L}+\sqrt{L}c^d\delta_{m,L}$ in absolute value, we have
    \begin{align*}
        \Vert \textbf{K}-\tilde{\textbf{K}}\Vert_\infty &\leq n \left(\delta_{m,L}+\sqrt{L} c^d\delta_{m,L}\right) = \gamma_{n,m,L}.
    \end{align*}
    Therefore, $\Vert \textbf{K}-\tilde{\textbf{K}}\Vert_2 \leq \gamma_{n,m,L}$.
\end{proof}

\subsubsection{Proof of Lemma \ref{lemma:test-train-kernel-matrix-error}}\label{sec:proof-test-train-kernel-matrix-error}
\testtrainkernelmatrixerror*
\begin{proof}
    Using the same reasoning as in Proposition \ref{prop:spectral-norm}, we have
    \begin{align*}
        \Vert \textbf{K}_{\cdot,\textbf{X}}-\tilde{\textbf{K}}_{\cdot,\textbf{X}}\Vert_2&\leq \sqrt{\Vert \textbf{K}_{\cdot,\textbf{X}}-\tilde{\textbf{K}}_{\cdot,\textbf{X}}\Vert_1\Vert \textbf{K}_{\cdot,\textbf{X}}-\tilde{\textbf{K}}_{\cdot,\textbf{X}}\Vert_\infty} \\
        &\leq \max \left(\Vert \textbf{K}_{\cdot,\textbf{X}}-\tilde{\textbf{K}}_{\cdot,\textbf{X}}\Vert_1, \Vert \textbf{K}_{\cdot,\textbf{X}}-\tilde{\textbf{K}}_{\cdot,\textbf{X}}\Vert_\infty\right).
    \end{align*}
    Now, $\Vert \textbf{K}_{\cdot,\textbf{X}}-\tilde{\textbf{K}}_{\cdot,\textbf{X}}\Vert_1$ is the maximum absolute column sum, which is less than or equal to $T(\delta_{m,L} + \sqrt{L}c^d\delta_{m,L}) = \gamma_{T,m,L}$. Similarly, $\Vert \textbf{K}_{\cdot,\textbf{X}}-\tilde{\textbf{K}}_{\cdot,\textbf{X}}\Vert_\infty$ is the maximum absolute row sum, which is upper bounded by $n(\delta_{m,L} + \sqrt{L}c^d\delta_{m,L}) = \gamma_{n,m,L}$. Therefore,
    $$
    \Vert \textbf{K}_{\cdot,\textbf{X}}-\tilde{\textbf{K}}_{\cdot,\textbf{X}}\Vert_2 \leq \max(\gamma_{T,m,L},\gamma_{n,m,L}).
    $$
\end{proof}
\subsubsection{Additional Spectral Norm Bounds}
\begin{restatable}{lemma}{traintestbound}\label{lemma:test-train-bound}
    Let $\textbf{K}_{\cdot,\textbf{X}} \in \mathbb{R}^{T \times n}$ be cross kernel matrix between $T$ test points and $n$ training points, where the SKI approximation uses $m$ inducing points. If the kernel function $k$ is bounded such that $|k(\textbf{x}, \textbf{x}')| \leq M$ for all $\textbf{x}, \textbf{x}'\in \mathcal{X}$, then:
    \begin{align*}
        \Vert \textbf{K}_{\cdot,\textbf{X}}\Vert_2&\leq \sqrt{Tn}M
    \end{align*}
\end{restatable}
\begin{proof}
\begin{align*}
    \Vert \textbf{K}_{\cdot,\textbf{X}}\Vert_2&\leq \sqrt{\Vert \textbf{K}_{\cdot,\textbf{X}}\Vert_1\Vert \textbf{K}_{\cdot,\textbf{X}}\Vert_\infty}\\
    &\leq \sqrt{Tn}M
\end{align*}
\end{proof}
\begin{restatable}{lemma}{skitesttrainbound}\label{lemma:ski-test-train-bound}
    Let $\tilde{\textbf{K}}_{\cdot,\textbf{X}} \in \mathbb{R}^{T \times n}$ be the matrix of SKI kernel evaluations between $T$ test points and $n$ training points, where the SKI approximation uses $m$ inducing points. Let $\textbf{W}(\cdot) \in \mathbb{R}^{T \times m}$ and $\textbf{W}(\textbf{X}) \in \mathbb{R}^{n \times m}$ be the matrices of interpolation weights for the test points and training points, respectively. Assume that the interpolation scheme is such that the sum of the absolute values of the interpolation weights for any point is bounded by $c^d$, where $c>0$ is a constant. Let $\textbf{K}_{\textbf{U}} \in \mathbb{R}^{m \times m}$ be the kernel matrix evaluated at the inducing points. If the kernel function $k$ is bounded such that $|k(\textbf{x}, \textbf{x}')| \leq M$ for all $\textbf{x}, \textbf{x}'\in \mathcal{X}$, then:
    $$
    \Vert \tilde{\textbf{K}}_{\cdot,\textbf{X}}\Vert_2 \leq \sqrt{Tn} m c^{2d} M
    $$
\end{restatable}
\begin{proof}
    By the definition of the SKI approximation and the submultiplicativity of the spectral norm, we have:
    $$
    \Vert \tilde{\textbf{K}}_{\cdot,\textbf{X}}\Vert_2 = \Vert \textbf{W}(\cdot)\textbf{K}_{\textbf{U}}(\textbf{W}(\textbf{X}))^\top\Vert_2 \leq \Vert \textbf{W}(\cdot)\Vert_2 \Vert\textbf{K}_{\textbf{U}}\Vert_2 \Vert\textbf{W}(\textbf{X})\Vert_2
    $$

    We now bound each term.

    1.  **Bounding $\Vert \textbf{W}(\cdot)\Vert_2$ and $\Vert \textbf{W}(\textbf{X})\Vert_2$:**
        Since the spectral norm is induced by the Euclidean norm, and using the assumption that the sum of absolute values of interpolation weights for any point is bounded by $c^d$, we have
        $$\Vert \textbf{W}(\cdot)\Vert_2 \leq \sqrt{\Vert \textbf{W}(\cdot)\Vert_1 \Vert \textbf{W}(\cdot)\Vert_\infty} \leq \sqrt{T c^d \cdot c^d} = \sqrt{T} c^d.$$
        Similarly, $\Vert\textbf{W}(\textbf{X})\Vert_2 \leq \sqrt{n}c^d$.

    2.  **Bounding $\Vert\textbf{K}_{\textbf{U}}\Vert_2$:**
        Since $\textbf{K}_{\textbf{U}}$ is symmetric, $\Vert \textbf{K}_{\textbf{U}} \Vert_2 \leq \Vert \textbf{K}_{\textbf{U}} \Vert_\infty$. Each entry of $\textbf{K}_{\textbf{U}}$ is bounded by $M$ (by the boundedness of $k$), and each row has $m$ entries, so $\Vert \textbf{K}_{\textbf{U}} \Vert_\infty \leq mM$. Thus, $\Vert\textbf{K}_{\textbf{U}}\Vert_2 \leq mM$.

    Combining these bounds, we get:
    $$
    \Vert \tilde{\textbf{K}}_{\cdot,\textbf{X}}\Vert_2 \leq (\sqrt{T} c^d) (mM) (\sqrt{n} c^d) = \sqrt{Tn} m c^{2d} M
    $$
    as required.
\end{proof}


% \amnote{might get rid of this}
\begin{lemma}\label{lemma:action-inverse-error}
    Let \(\mathbf{\tilde{K}}\) be the SKI approximation of the kernel matrix \(\mathbf{K}\), and let \(\sigma^2\) be the regularization parameter. The spectral error of the regularized inverse can be bounded as follows:
\begin{align*}
\left\|\left(\mathbf{\tilde{K}} + \sigma^2 \mathbf{I}\right)^{-1} - \left(\mathbf{K} + \sigma^2 \mathbf{I}\right)^{-1} \right\|_2 &\leq \frac{\gamma_{n, m, L}}{\sigma^4} 
% \\
% &=\frac{O(n\sqrt{L}\rho^{-L})}{\sigma^4}
\end{align*}
\end{lemma} 

\begin{proof}
    Note that
\[
\left(\mathbf{\tilde{K}} + \sigma^2 \mathbf{I}\right)^{-1} - \left(\mathbf{K} + \sigma^2 \mathbf{I}\right)^{-1} = \left(\mathbf{\tilde{K}} + \sigma^2 \mathbf{I}\right)^{-1} (\mathbf{K} - \mathbf{\tilde{K}}) \left(\mathbf{K} + \sigma^2 \mathbf{I}\right)^{-1}
\]

Taking the spectral norm, we have
\[
\begin{aligned}
\left\|\left(\mathbf{\tilde{K}} + \sigma^2 \mathbf{I}\right)^{-1} - \left(\mathbf{K} + \sigma^2 \mathbf{I}\right)^{-1}\right\|_2 &\leq \left\|\left(\mathbf{\tilde{K}} + \sigma^2 \mathbf{I}\right)^{-1}\right\|_2 \|\mathbf{K} - \mathbf{\tilde{K}}\|_2 \left\|\left(\mathbf{K} + \sigma^2 \mathbf{I}\right)^{-1}\right\|_2 \\
&\leq \gamma_{n, m, L}\left\|\left(\mathbf{\tilde{K}} + \sigma^2 \mathbf{I}\right)^{-1}\right\|_2 \left\|\left(\mathbf{K} + \sigma^2 \mathbf{I}\right)^{-1}\right\|_2\textrm{ by Proposition \ref{prop:spectral-norm}}\\
&\leq   \frac{\gamma_{n, m, L}}{\sigma^4}
\end{aligned}
\]

\end{proof}

\subsection{Proofs Related to Linear Time Analysis}
\subsubsection{Proof of Theorem \ref{thm:inducing-points-count-alt}}\label{sec:proof-inducing-points-count-alt}
\inducingpointscountalt*
\begin{proof}
    We want to choose $m$ such that the spectral norm error $\Vert \textbf{K} - \tilde{\textbf{K}} \Vert_2 \leq \epsilon$. From Proposition \ref{prop:spectral-norm}, we have:
    $$
    \Vert \textbf{K} - \tilde{\textbf{K}} \Vert_2 \leq n(1 + \sqrt{L}c^d) \delta_{m,L}
    $$
    For cubic interpolation ($L=4$), Lemma \ref{lemma:tensor-product-interpolation-error}, combined with the analysis in Lemma \ref{lemma:tensor-product-interpolation-error}, gives us:
    $$
    \delta_{m,L} \leq K' c^{2d} h^3
    $$
    where $K'$ is a constant that depends only on the kernel function (through its derivatives) and the interpolation scheme, but not on $n$, $m$, $h$, or $d$.

    Therefore, a sufficient condition to ensure $\Vert \textbf{K} - \tilde{\textbf{K}} \Vert_2 \leq \epsilon$ is:
    \begin{equation} \label{eq:sufficient_condition_final}
    n(1 + 2c^d) K' c^{2d} h^3 \leq \epsilon
    \end{equation}

    Since the inducing points are placed on a regular grid with spacing $h$ in each dimension, and the domain is $[-D,D]^d$ and assuming that $2D\mod h\equiv 0$, the number of inducing points $m$ satisfies:

    $$
    m = \left(\frac{2D}{h}\right)^d
    $$

    We can rearrange this to get:

    $$
    h = \frac{2D}{m^{1/d}}
    $$
    Substituting this into the sufficient condition \eqref{eq:sufficient_condition_final}, we get:

    $$
    n (1 + 2c^d) K' c^{2d} \left(\frac{2D}{m^{1/d}}\right)^3 \leq \epsilon
    $$

    Rearranging to isolate $m$, we obtain:

    $$
    m^{3/d} \geq \frac{n}{\epsilon} (1 + 2c^d) K' c^{2d} (8D^3)
    $$

    $$
    m \geq \left( \frac{n}{\epsilon} (1 + 2c^d) K' (8 c^{2d} D^3) \right)^{d/3}
    $$
\end{proof}
\subsubsection{Proof of Corollary \ref{cor:linear-time}}\label{proof:cor-linear-time}
\corlineartime*
\begin{proof}
% The idea is that we want to have an error sufficiently large so that when we choose $m$ based on Corollary \ref{cor:inducing-points-count-alt}, we have $m=O\left(\frac{n}{\log n}\right)$. 

Assume that
$$
\epsilon \geq \frac{(1 + 2c^d) K' 8 c^{2d} D^3}{C^{3/d}} \cdot \frac{n (\log n)^{3/d}}{n^{3/d}}.
$$
Rearranging this we obtain
\begin{align*}
    \left( \frac{n}{\epsilon} (1 + 2c^d) K' (8 c^{2d} D^3) \right)^{d/3}&\leq C\frac{n}{\log n}.\\
    &=O\left(\frac{n}{\log n}\right).
\end{align*}
Now taking 
\begin{align*}
    m &= \left( \frac{n}{\epsilon} (1 + 2c^d) K' (8 c^{2d} D^3) \right)^{d/3}
\end{align*}
we have that $m=O\left(\frac{n}{\log n}\right)$ and by Theorem \ref{thm:inducing-points-count-alt}, $\Vert \textbf{K}-\tilde{\textbf{K}}\Vert_2\leq \epsilon$. Now plugging in $\frac{n}{\log n}$ into $m\log m$ we obtain
\begin{align*}
    O\left(m\log m\right)&=O\left(\frac{n}{\log n}\log \frac{n}{\log n}\right)\\
    &=O\left(\frac{n}{\log n}\log n-\frac{n}{\log n}\log \log n\right)\\
    &=O(n)
\end{align*}
as desired.
\end{proof}
% \subsubsection{Proof of Corollary \ref{cor:linear-time-low-dimensional}}
% \begin{proof}
%     We need that $m=O\left(\frac{n}{\log n}\right)$ and $m=\left( \frac{n}{\epsilon} (1 + 2c^d) K' (8 c^{2d} D^3) \right)^{d/3}$ can jointly hold. That is, that
%     \begin{align*}
%         \left( \frac{n}{\epsilon} (1 + 2c^d) K' (8 c^{2d} D^3) \right)^{d/3}&=O\left(\frac{n}{\log n}\right)
%     \end{align*}
%     Now let $C\equiv  (1 + 2c^d) K' (8 c^{2d} D^3)$. Then a sufficient condition is that the following holds for $n$ sufficiently large:
%     \begin{align*}
%         \left( \frac{n}{\epsilon}C\right)^{d/3}&\leq \frac{n}{\log n}\\
%         C^{d/3} \frac{n^{d/3-1}}{\log n}&\leq \epsilon^{d/3}\\
%         \frac{n^{d/3-1}}{\log n}&\leq \epsilon^{d/3}C^{-d/3}.
%     \end{align*}
%     For $d\leq 3$, the lhs goes to $0$ as $n\rightarrow \infty$ and thus for sufficiently large $n$, the inequality will hold.
% \end{proof}

% \subsection{Additional Quantities}\label{subsec:additional-quantities}
% \subsubsection{Action of Regularized Inverse}\label{subsubsec:action-regularized-inverse}



\section{Proofs Related to Gaussian Process Applications}
In this section, we address how SKI affects Gaussian Processes Applications. In Section \ref{sec:kernel-hyperparameter-estimation} we address how using the SKI kernel and log-likelihood affect hyperparameter estimation, showing that gradient ascent on the SKI log-likelihood approaches a ball around a stationary point of the true log-likelihood. In section \ref{sec:posterior-inference} we describe how using SKI affects the accuracy of posterior inference.

\subsection{Kernel Hyperparameter Estimation}\label{sec:kernel-hyperparameter-estimation}

Here we show that, for a $\mu$-smooth log-likelihood, an iterate of gradient ascent on the SKI log-likelihood approaches a neighborhood of a stationary point of the true log-likelihood at an $O\left(\frac{1}{K}\right)$ rate, with the neighborhood size determined by the SKI score function's error. To show this, we leverage a recent result for non-convex inexact gradient ascent \cite{stonyakin2023stopping}, which requires an upper bound on the SKI score function's error. This requires bounding the spectral norm error of the SKI Gram matrix's partial derivatives. In order to obtain this, we note that for many SPD kernels, under weak assumptions, the partial derivatives are \textit{also} SPD kernels, and thus we can reuse the previous results directly on the partial derivatives.

Note that \cite{stonyakin2023stopping} does not actually imply \textit{convergence} to a neighborhood of a critical point, only that at least one iterate will approach it. Given the challenges of non-concave optimization and the fact that we leverage a fairly recent result, we leave stronger results to future work.

%In order to obtain this, we need to bound the errors of the partial derivatives of the SKI gram matrix. To do so, we note that in many cases, the partial derivatives of an SPD kernel are themselves SPD kernels. This allows us to express the gram matrix's gradient error as a different gram matrix error and thus apply our previous results. 

Let $\mathcal{D}\subseteq \Theta$ be a \textit{compact} subset that we wish to optimize over. In the most precise setting we would analyze projected gradient ascent, but for simplicity we analyze gradient ascent. Let let $\tilde{k}_{\theta}: \mathcal{X} \times \mathcal{X} \rightarrow \mathbb{R}$ be the SKI approximation of $k_{\theta}: \mathcal{X} \times \mathcal{X} \rightarrow \mathbb{R}$ using $m$ inducing points and interpolation degree $L-1$. We are interested in the convergence properties of \textit{inexact gradient ascent} using the SKI log-likelihood, e.g.
\begin{align*}
    \boldsymbol{\theta}_{k+1}&=\boldsymbol{\theta}_k+\eta \nabla \tilde{\mathcal{L}}(\boldsymbol{\theta}_k),
\end{align*}
where $\eta\in \mathbb{R}$ is the learning rate and $\nabla \tilde{\mathcal{L}}(\boldsymbol{\theta}_k)$ is the SKI score function (gradient of its log-likelihood). We assume: 1) a $\mu$-smooth log-likelihood. If we optimize on a bounded domain, then for infinitely differentiable kernels (e.g. RBF) this will immediately hold. 3) That the kernel's partial derivatives are themselves SPD kernels (this can be easily shown for the RBF kernel's lengthscale by noting that the product of SPD kernels are themselves SPD kernels). 
%We then mention an equivalence between strong concavity and the Hessian having an upper bound on its smallest eigenvalue.
% 2) That the kernel is continuously differentiable. 3) That the response vectors squared $l^2$ norm grows linearly with the sample size.
\begin{assumption}[$\mu$-smooth-log-likelihood]\label{assumption:mu-smoothness}
The true log-likelihood is $\mu$-smooth over $\mathcal{D}$. That is, for all $\boldsymbol{\theta},\boldsymbol{\theta}'\in \mathcal{D}$,
\begin{align*}
    \Vert \nabla \mathcal{L}(\boldsymbol{\theta})-\nabla \mathcal{L}(\boldsymbol{\theta}')\Vert &\leq \mu \Vert \boldsymbol{\theta}-\boldsymbol{\theta}'\Vert
\end{align*}
\end{assumption}

\begin{assumption}
    (Kernel Smoothness) $k_\theta(x,x')$ is $C^1$ in $\boldsymbol{\theta}$ over $\mathcal{D}$. That is, for each $l \in \{1, ..., p\}$, $k'_{\theta_l}(x, x') = \frac{\partial k_{\theta}(x, x')}{\partial \theta_l}$ exists and is continuous for $\boldsymbol{\theta}\in \mathcal{D}$.
\end{assumption}
\begin{assumption}
    (SPD Kernel Partials) For each $l \in \{1, ..., p\}$, the partial derivative of $k_{\theta}$ with respect to a hyperparameter $\theta_l\in \mathbb{R}$, denoted as $k'_{\theta_l}(x, x') = \frac{\partial k_{\theta}(x, x')}{\partial \theta_l}$, is also a valid SPD kernel.
\end{assumption}
% \begin{assumption}
%     (Linear Squared $l^2$ Response) $\Vert \textbf{y}\Vert_2^2=O(n)$.
% \end{assumption}

We next state several results leading up to our bound on the SKI score function's error. Here we argue that we can apply the same elementwise error we derived previously to the SKI partial derivatives.

\begin{restatable}{lemma}{skikernelderivativeerrorkernel}[Bound on Derivative of SKI Kernel Error using Kernel Property of Derivative]
\label{lemma:ski_kernel_derivative_error_kernel}
 Let $\tilde{k}'_{\theta_l}(x,x')$ be the SKI approximation of $k'_{\theta_l}(x,x')$, using the same inducing points and interpolation scheme as $\tilde{k}_{\theta}$. Then, for all $x, x' \in \mathcal{X}$ and all $\boldsymbol{\theta} \in \Theta$, the following inequality holds:

\begin{align*}
\left\vert \frac{\partial k_{\theta}(x,x')}{\partial \theta_l}-\frac{\partial \tilde{k}_{\theta}(x,x')}{\partial \theta_l}\right\vert &= \left\vert k'_{\theta_l}(x, x') - \tilde{k}'_{\theta}(x, x') \right\vert \\
&\leq\delta_{m,L}'+\sqrt{L}c^d\delta_{m,L}'\\
&=O\left(\frac{c^{2d}}{m^{3/d}}\right)
\end{align*}

where $\delta_{m,L}'$ is an upper bound on the error of the SKI approximation of the kernel $k'_{\theta_l}(x,x')$ with $m$ inducing points and interpolation degree $L-1$, as defined in Lemma \ref{lemma:ski-kernel-elementwise-error}.
\end{restatable}
\begin{proof}
    See Appendix \ref{sec:proofski_kernel_derivative_error_kernelz}
\end{proof}

We then use the elementwise bound to bound the spectral norm of the SKI gram matrix's partial derivative error. This again leverages Proposition \ref{prop:spectral-norm}, noting that these partial derivatives of the Gram matrices are themselves Gram matrices.

\begin{restatable}{lemma}{partialgradientspectralnormbound}[Partial Derivative Gram Matrix Difference Bound]
\label{lemma:partial_gradient_spectral_norm_bound}
For any $l \in \{1, \dots, p\}$,

\begin{align*}
\left\| \frac{\partial \mathbf{K}}{\partial \theta_l} - \frac{\partial \tilde{\mathbf{K}}}{\partial \theta_l} \right\|_2 &\leq \gamma'_{n,m,L,l} \\
&= O\left(\frac{nc^{2d}}{m^{3/d}}\right)
\end{align*}

where $\gamma'_{n,m,L,l}$ is the bound on the spectral norm difference between the kernel matrices corresponding to $k'_{\theta_l}$ and its SKI approximation $\tilde{k}'_{\theta_l}$ (analogous to Proposition \ref{prop:spectral-norm}, but for the kernel $k'_{\theta_l}$).
\end{restatable}
\begin{proof}
See Section \ref{section:proof_partial_gradient_spectral_norm_bound}.    
\end{proof}

% \begin{restatable}{lemma}{gradientspectralnormbound}[Bound on Spectral Norm of Gradient Difference]
% \label{lemma:gradient_spectral_norm_bound}
% Assume that $k'_{\theta,i}(x, x')$, is a valid SPD kernel. Let $\tilde{k}'_{\theta,i}(x,x')$ be the SKI approximation of $k'_{\theta,i}(x,x')$, using the same inducing points and interpolation scheme as $\tilde{k}_{\theta}$. Then, the spectral norm of the difference between the gradient of the true kernel matrix and the gradient of the SKI kernel matrix is bounded by:

% \begin{align*}
% \| \nabla_\theta K - \nabla_\theta \tilde{K} \|_2 &\leq \sqrt{p} \max_{i \in \{1,\ldots,p\}} \gamma'_{n,m,L,i}\\
% &=O\left(\sqrt{p}nc^{2d}h^3\right)\text{, convolutional cubic interpolation,}
% \end{align*}

% where $\nabla_\theta K$ and $\nabla_\theta \tilde{K}$ are represented as $p \times n^2$ matrices using the vec-notation (denominator layout), $p$ is the number of hyperparameters, and $\gamma'_{n,m,L,i}$ is the bound on the spectral norm difference between the kernel matrices corresponding to $k'_{\theta,i}$ and its SKI approximation $\tilde{k}'_{\theta,i}$ (analogous to Proposition \ref{prop:spectral-norm}, but for the kernel $k'_{\theta,i}$).
% \end{restatable}

% \begin{proof}
% See Section \ref{sec:proof_gradient_spectral_norm_bound}
% \end{proof}

We now bound the SKI score function. The key insight to the proof is that the partial derivatives of the difference between regularized gram matrix inverses is in fact a difference between two quadratic forms. We can then use standard techniques \citep{horn2012matrix} for bounding the difference between quadratic forms to obtain our result. The result says that, aside from the response vector's norm, the error grows quadratically in the sample size, at a square root rate in the number of hyperparameters and exponentially in the dimensionality. It further decays at an $m^{\frac{3}{d}}$ rate in the number of inducing points. Noting that to maintain linear time, $m$ should grow at an $n^{d/3}$ rate, we have that aside from the response vector, the error in fact grows linearly with the sample size when choosing the number of inducing points based on Theorem \ref{thm:inducing-points-count-alt}.
%by an expression that is \textit{quadratic} in the sample size. The reason for the quadratic dependency is that the difference in gradients of the quadratic forms in the log-likelihoods can itself be expressed as a gradient of a quadratic form involving the response variables. The response variables contribute a linear dependence and the gradient of the difference between the regularized inverses multiplicatively contribute a linear dependence: combined this is quadratic.
%\amnote{In an attempt to improve the bound, I got something that doesn't decrease with the number of inducing points, which is bad.}
\begin{restatable}{lemma}{scorefunctionbound}[Score Function Bound]\label{lemma:score-function-bound}
Let $\mathcal{L}(\boldsymbol{\theta})$ be the true log-likelihood and $\tilde{\mathcal{L}}(\boldsymbol{\theta})$ be the SKI approximation of the log-likelihood at $\boldsymbol{\theta}$. Let $\nabla \mathcal{L}(\boldsymbol{\theta})$ and $\nabla \tilde{\mathcal{L}}(\boldsymbol{\theta})$ denote their respective gradients with respect to $\boldsymbol{\theta}$. Then, for any $\boldsymbol{\theta}\in \mathcal{D}$,

\begin{align*}
&\| \nabla \mathcal{L}(\boldsymbol{\theta}) - \nabla \tilde{\mathcal{L}}(\boldsymbol{\theta}) \|_2 \\
&\leq \frac{1}{2\sigma^4}\Vert \textbf{y}\Vert\sqrt{p}\max_{1\leq l\leq p} \left( \gamma'_{n,m,L,l}+Cn\gamma_{n,m,L}\right.\\
&\qquad\left.+\gamma_{n,m,L}\gamma'_{n,m,L,l} \right)+\frac{\gamma_{n,m,L}}{2\sigma^4}\\
&=\Vert \textbf{y}\Vert_2 O\left(\frac{\sqrt{p}n^2c^{4d}}{m^{3/d}}\right)\\
&\equiv \epsilon_G
\end{align*}
where $C$ is a constants depending on the upper bound of the derivatives of the kernel function over $\mathcal{D}$.
\end{restatable}

% \begin{restatable}{lemma}{scorefunctionbound}[Score Function Bound]\label{lemma:score-function-bound}
% Let $\mathcal{L}(\theta)$ be the true log-likelihood and $\tilde{\mathcal{L}}(\theta)$ be the SKI approximation of the log-likelihood at $\theta$. Let $\nabla \mathcal{L}(\theta)$ and $\nabla \tilde{\mathcal{L}}(\theta)$ denote their respective gradients with respect to $\theta$. Then, for any $\theta\in \mathcal{D}$,
% \begin{align*}
% \| \nabla \mathcal{L}(\theta) - \nabla \tilde{\mathcal{L}}(\theta) \|_2 &\leq \frac{1}{\sigma^4}\left[\Vert y\Vert_2^2\sqrt{p}\max_{1\leq l\leq p}\left( \gamma'_{n,m,L,l}+Cn\gamma_{n,m,L}+\frac{\gamma_{n,m,L}}{\sigma^4} \gamma'_{n,m,L,l}\right)+\frac{1}{2}\gamma_{n,m,L}\right]\\
% &=O(\sqrt{p}n^3c^{2d}h^3)\\
% &\equiv \epsilon_G
% \end{align*}
% where $\gamma_{n,m,L}$ bounds the elementwise difference between $\mathbf{K}$ and $\tilde{\mathbf{K}}$, and $\gamma'_{n,m,L,i}$ is the bound on the spectral norm difference between the kernel matrices corresponding to $k'_{\theta,i}$ and its SKI approximation $\tilde{k}'_{\theta,i}$ and $C_n$ is a constant depending on the upper bound of the derivatives of the kernel function over $\mathcal{D}$ and the sample size.
% \end{restatable}

\begin{proof}
See Section \ref{sec:proof-score-function-bound}.
\end{proof}

We apply \cite{stonyakin2023stopping} below: the result is the same as in their paper (and assumes $\mu$-smoothness as we did on $\mathcal{L}$), but using gradient ascent instead of descent and using the score function error above. It says that at an $O\left(\frac{1}{K}\right)$ rate, at least one iterate of gradient ascent has its squared gradient norm approach a neighborhood proportional to the squared SKI score function's spectral norm error.

\begin{theorem} \citep{stonyakin2023stopping}
    For inexact gradient ascent on $\mathcal{L}$ with additively inexact gradients satisfying $\|\nabla \mathcal{L}(\boldsymbol{\theta}) - \nabla \tilde{\mathcal{L}}(\boldsymbol{\theta})\| \leq \epsilon_g$, we have:

\begin{equation}
    \max_{k=0,...,N-1} \|\nabla \mathcal{L}(\theta_k)\|^2 \leq \frac{2\mu(\mathcal{L}^* - \mathcal{L}(\boldsymbol{\theta}_0))}{K} + \frac{\epsilon_g^2}{2\mu}
\end{equation}

where $\mathcal{L}^*$ is the value at a stationary point, $\mathcal{L}(\boldsymbol{\theta}_0)$ is the initial, function value, $K$ is the number of iterations and $\epsilon_g$ is the gradient error bound in the previous Lemma.

\end{theorem}

\subsection{Posterior Inference}\label{sec:posterior-inference}
Finally, we treat posterior inference. As the current hyperparameter optimization results only say that \textit{some} iterate approaches a stationary point, we will focus on the error when the SKI and true kernel hyperparameter match.  We first add an assumption
\begin{assumption}
    (Bounded Kernel) Assume that the true kernel satisfies the condition that $|k(\mathbf{x}, \mathbf{x}')| \leq M$ for all $\mathbf{x}, \mathbf{x}'\in \mathcal{X}$.
\end{assumption}

% Consider on a compact domain,
% \begin{align*}
%     \vert k_\theta(x,x')-k_{\theta'}(x,x')\vert &\leq \rho \Vert \theta-\theta'\Vert
% \end{align*}

% which implies that
% \begin{align*}
%     \Vert \textbf{K}^{(\theta)}-\textbf{K}^{(\theta')}\Vert_2&\leq n\rho \Vert \theta-\theta'\Vert
% \end{align*}
% and we can then 
Now we bound the spectral error for the SKI mean function evaluated at a set of test points. The proof follows a standard strategy commonly used for approximate kernel ridge regression. See \cite{bach2013sharp,musco2017recursive} for examples. The result says that the $l^2$ error (aside from the response vector) grows exponentially in the dimensionality, super-linearly but sub-quadratically in the training sample size and at worst linearly in the test sample size. It decays at an $m^{\frac{3}{d}}$ rate in the number of inducing points. Similarly to for the score function error, if we follow Theorem \ref{thm:inducing-points-count-alt} for selecting the number of inducing points, the error in fact grows \textit{sublinearly} with the training sample size.
\begin{restatable}{lemma}{meaninference}\label{lemma:mean-inference} (SKI Posterior Mean Error)
    Let $\boldsymbol{\mu}(\cdot)$ be the GP posterior mean at a set of test points $\cdot\in \mathbb{R}^{T\times d}$ and $\tilde{\boldsymbol{\mu}}(\cdot)$ be the SKI posterior mean at those points. Then the SKI posterior mean $l^2$ error is bounded by:
{\footnotesize
\begin{align*}
    &\Vert \tilde{\boldsymbol{\mu}}(\cdot)- \boldsymbol{\mu}(\cdot)\Vert_2\\
    &\leq\left(\frac{\max(\gamma_{T,m,L},\gamma_{n,m,L})}{\sigma^2}+\frac{\sqrt{Tn}Mc^{2d}}{\sigma^4}\gamma_{n,m,L}\right)\Vert \textbf{y}\Vert_2\\
    &=\Vert \textbf{y}\Vert_2O\left(c^{2d}\frac{\max(T,n)+\sqrt{Tn}n}{m^{3/d}}\right)
    %\left(\frac{\max(\gamma_{T,m,L},\gamma_{n,m,L})}{\sigma^2}+\frac{\Vert \tilde{\mathbf{K}}_{\cdot, \mathbf{X}} \Vert_2}{\sigma^4}\gamma_{n,m,L}\right)\Vert \textbf{y}\Vert_2
\end{align*}
}
\end{restatable}


\begin{proof}
See Appendix \ref{sec:proof-mean-inference}.
\end{proof}

We now derive the spectral error bound for the test SKI covariance matrix. The proof involves noticing that a key term is a difference between two quadratic forms, and using standard techniques for bounding such a difference. The result shows that the error grows at worst super-linearly but subquadratically in the number of test points, quadratically in the training sample size and exponentially in the dimension. Interestingly, due to the use of standard techniques for bounding the difference between quadratic forms, the error is only guaranteed to decay with the number of inducing points at an $m^{3/d-1}$ rate, so that it is only guaranteed to decay at all if $d<3$. If we select the number of inducing points to be proportional to $n^{d/3}$, then the error grows at rate $n^{1+d/3}$ for $d<3$. An interesting question is whether alternate techniques can improve the result for higher dimensional settings e.g. $d\geq 3$.

\begin{restatable}{lemma}{skiposteriorcovarianceerror}[SKI Posterior Covariance Error]\label{lemma:ski-posterior-covariance-error}
Let $\boldsymbol{\Sigma}(\cdot)$ be the GP posterior covariance matrix at a set of test points $\cdot\in \mathbb{R}^{T\times d}$ and $\tilde{\boldsymbol{\Sigma}}(\cdot)$ be its SKI approximation. Then
\begin{align*}
    &\Vert \boldsymbol{\Sigma}(\cdot)-\tilde{\boldsymbol{\Sigma}}(\cdot)\Vert_2\\ &\leq \gamma_{T,m,L} + \frac{\sqrt{Tn}M}{\sigma^2} \max(\gamma_{T,m,L},\gamma_{n,m,L})\\
    &\quad+ \frac{\gamma_{n,m,L}}{\sigma^4}Tn m c^{2d} M^2 \\
    &\quad+ \frac{\sqrt{Tn} m c^{2d} M}{\sigma^2} \max(\gamma_{T,m,L},\gamma_{n,m,L}).\\
    &=O\left(\frac{Tn^2mc^{4d}+\sqrt{Tn}mc^{4d}\max(T,n)}{m^{3/d}}\right).
\end{align*}
where $\gamma_{T,m,L}$ is defined as in Proposition \ref{prop:spectral-norm}.
\end{restatable}

\begin{proof}
See Appendix \ref{sec:proof-ski-posterior-covariance-error}

\end{proof}



\end{document}
