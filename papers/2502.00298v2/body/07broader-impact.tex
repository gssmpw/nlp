This work contributes to a deeper theoretical understanding Structured Kernel Interpolation (SKI) \citep{wilson2015kernel} for Gaussian Processes (GPs). By establishing error bounds and analyzing the impact of SKI on hyperparameter estimation and posterior inference, this research can lead to more confident use of approximate Gaussian Processes. These models have broad applications in various domains, including those mentioned in the introduction as well as robotics \citep{deisenroth2015gaussian}, environmental modeling \citep{desai2023deep}, and healthcar \citep{alaa2017bayesian}. Improved Gaussian Process models can enhance prediction accuracy and decision-making, potentially leading to advancements in robotics, more accurate environmental predictions, and better healthcare outcomes. It is important to acknowledge that the application of Gaussian Process models also carries potential risks. For instance, in healthcare, inaccurate predictions or biased models can lead to misdiagnosis or inappropriate treatment \citep{morley2020ethics}. Therefore, understanding potential sources of error when using approximations can be crucial to understanding how reliable we can expect them to be.