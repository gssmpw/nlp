\subsubsection{Proof of Theorem \ref{thm:inducing-points-count-alt}}\label{sec:proof-inducing-points-count-alt}
\inducingpointscountalt*
\begin{proof}
    We want to choose $m$ such that the spectral norm error $\Vert \textbf{K} - \tilde{\textbf{K}} \Vert_2 \leq \epsilon$. From Proposition \ref{prop:spectral-norm}, we have:
    $$
    \Vert \textbf{K} - \tilde{\textbf{K}} \Vert_2 \leq n(1 + \sqrt{L}c^d) \delta_{m,L}
    $$
    For cubic interpolation ($L=4$), Lemma \ref{lemma:tensor-product-interpolation-error}, combined with the analysis in Lemma \ref{lemma:tensor-product-interpolation-error}, gives us:
    $$
    \delta_{m,L} \leq K' c^{2d} h^3
    $$
    where $K'$ is a constant that depends only on the kernel function (through its derivatives) and the interpolation scheme, but not on $n$, $m$, $h$, or $d$.

    Therefore, a sufficient condition to ensure $\Vert \textbf{K} - \tilde{\textbf{K}} \Vert_2 \leq \epsilon$ is:
    \begin{equation} \label{eq:sufficient_condition_final}
    n(1 + 2c^d) K' c^{2d} h^3 \leq \epsilon
    \end{equation}

    Since the inducing points are placed on a regular grid with spacing $h$ in each dimension, and the domain is $[-D,D]^d$ and assuming that $2D\mod h\equiv 0$, the number of inducing points $m$ satisfies:

    $$
    m = \left(\frac{2D}{h}\right)^d
    $$

    We can rearrange this to get:

    $$
    h = \frac{2D}{m^{1/d}}
    $$
    Substituting this into the sufficient condition \eqref{eq:sufficient_condition_final}, we get:

    $$
    n (1 + 2c^d) K' c^{2d} \left(\frac{2D}{m^{1/d}}\right)^3 \leq \epsilon
    $$

    Rearranging to isolate $m$, we obtain:

    $$
    m^{3/d} \geq \frac{n}{\epsilon} (1 + 2c^d) K' c^{2d} (8D^3)
    $$

    $$
    m \geq \left( \frac{n}{\epsilon} (1 + 2c^d) K' (8 c^{2d} D^3) \right)^{d/3}
    $$
\end{proof}
\subsubsection{Proof of Corollary \ref{cor:linear-time}}\label{proof:cor-linear-time}
\corlineartime*
\begin{proof}
% The idea is that we want to have an error sufficiently large so that when we choose $m$ based on Corollary \ref{cor:inducing-points-count-alt}, we have $m=O\left(\frac{n}{\log n}\right)$. 

Assume that
$$
\epsilon \geq \frac{(1 + 2c^d) K' 8 c^{2d} D^3}{C^{3/d}} \cdot \frac{n (\log n)^{3/d}}{n^{3/d}}.
$$
Rearranging this we obtain
\begin{align*}
    \left( \frac{n}{\epsilon} (1 + 2c^d) K' (8 c^{2d} D^3) \right)^{d/3}&\leq C\frac{n}{\log n}.\\
    &=O\left(\frac{n}{\log n}\right).
\end{align*}
Now taking 
\begin{align*}
    m &= \left( \frac{n}{\epsilon} (1 + 2c^d) K' (8 c^{2d} D^3) \right)^{d/3}
\end{align*}
we have that $m=O\left(\frac{n}{\log n}\right)$ and by Theorem \ref{thm:inducing-points-count-alt}, $\Vert \textbf{K}-\tilde{\textbf{K}}\Vert_2\leq \epsilon$. Now plugging in $\frac{n}{\log n}$ into $m\log m$ we obtain
\begin{align*}
    O\left(m\log m\right)&=O\left(\frac{n}{\log n}\log \frac{n}{\log n}\right)\\
    &=O\left(\frac{n}{\log n}\log n-\frac{n}{\log n}\log \log n\right)\\
    &=O(n)
\end{align*}
as desired.
\end{proof}
% \subsubsection{Proof of Corollary \ref{cor:linear-time-low-dimensional}}
% \begin{proof}
%     We need that $m=O\left(\frac{n}{\log n}\right)$ and $m=\left( \frac{n}{\epsilon} (1 + 2c^d) K' (8 c^{2d} D^3) \right)^{d/3}$ can jointly hold. That is, that
%     \begin{align*}
%         \left( \frac{n}{\epsilon} (1 + 2c^d) K' (8 c^{2d} D^3) \right)^{d/3}&=O\left(\frac{n}{\log n}\right)
%     \end{align*}
%     Now let $C\equiv  (1 + 2c^d) K' (8 c^{2d} D^3)$. Then a sufficient condition is that the following holds for $n$ sufficiently large:
%     \begin{align*}
%         \left( \frac{n}{\epsilon}C\right)^{d/3}&\leq \frac{n}{\log n}\\
%         C^{d/3} \frac{n^{d/3-1}}{\log n}&\leq \epsilon^{d/3}\\
%         \frac{n^{d/3-1}}{\log n}&\leq \epsilon^{d/3}C^{-d/3}.
%     \end{align*}
%     For $d\leq 3$, the lhs goes to $0$ as $n\rightarrow \infty$ and thus for sufficiently large $n$, the inequality will hold.
% \end{proof}

% \subsection{Additional Quantities}\label{subsec:additional-quantities}
% \subsubsection{Action of Regularized Inverse}\label{subsubsec:action-regularized-inverse}
