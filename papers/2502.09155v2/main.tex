%%
%% This is file `sample-sigconf.tex',
%% generated with the docstrip utility.
%%
%% The original source files were:
%%
%% samples.dtx  (with options: `all,proceedings,bibtex,sigconf')
%% 
%% IMPORTANT NOTICE:
%% 
%% For the copyright see the source file.
%% 
%% Any modified versions of this file must be renamed
%% with new filenames distinct from sample-sigconf.tex.
%% 
%% For distribution of the original source see the terms
%% for copying and modification in the file samples.dtx.
%% 
%% This generated file may be distributed as long as the
%% original source files, as listed above, are part of the
%% same distribution. (The sources need not necessarily be
%% in the same archive or directory.)
%%
%%
%% Commands for TeXCount
%TC:macro \cite [option:text,text]
%TC:macro \citep [option:text,text]
%TC:macro \citet [option:text,text]
%TC:envir table 0 1
%TC:envir table* 0 1
%TC:envir tabular [ignore] word
%TC:envir displaymath 0 word
%TC:envir math 0 word
%TC:envir comment 0 0
%%
%% The first command in your LaTeX source must be the \documentclass
%% command.
%%
%% For submission and review of your manuscript please change the
%% command to \documentclass[manuscript, screen, review]{acmart}.
%%
%% When submitting camera ready or to TAPS, please change the command
%% to \documentclass[sigconf]{acmart} or whichever template is required
%% for your publication.
%%
%%


\documentclass[sigconf]{acmart}

\usepackage{balance}
\usepackage{xcolor} % Include this in the preamble
\usepackage{hyperref} % Include this in the preamble
\usepackage{graphicx}
\usepackage{subcaption}
\hypersetup{
    colorlinks=true,
    linkcolor=blue, % Color for internal links
    urlcolor=blue,  % Color for URLs
    citecolor=blue  % Color for citations
}
%% \BibTeX command to typeset BibTeX logo in the docs
\AtBeginDocument{%
  \providecommand\BibTeX{{%
    Bib\TeX}}}

%% Rights management information.  This information is sent to you
%% when you complete the rights form.  These commands have SAMPLE
%% values in them; it is your responsibility as an author to replace
%% the commands and values with those provided to you when you
%% complete the rights form.
\setcopyright{acmlicensed}
\copyrightyear{2018}
\acmYear{2018}
\acmDOI{XXXXXXX.XXXXXXX}
%% These commands are for a PROCEEDINGS abstract or paper.
\acmConference[Conference acronym 'XX]{Make sure to enter the correct
  conference title from your rights confirmation emai}{June 03--05,
  2018}{Woodstock, NY}
%%
%%  Uncomment \acmBooktitle if the title of the proceedings is different
%%  from ``Proceedings of ...''!
%%
%%\acmBooktitle{Woodstock '18: ACM Symposium on Neural Gaze Detection,
%%  June 03--05, 2018, Woodstock, NY}
\acmISBN{978-1-4503-XXXX-X/18/06}


%%
%% Submission ID.
%% Use this when submitting an article to a sponsored event. You'll
%% receive a unique submission ID from the organizers
%% of the event, and this ID should be used as the parameter to this command.
%%\acmSubmissionID{123-A56-BU3}

%%
%% For managing citations, it is recommended to use bibliography
%% files in BibTeX format.
%%
%% You can then either use BibTeX with the ACM-Reference-Format style,
%% or BibLaTeX with the acmnumeric or acmauthoryear sytles, that include
%% support for advanced citation of software artefact from the
%% biblatex-software package, also separately available on CTAN.
%%
%% Look at the sample-*-biblatex.tex files for templates showcasing
%% the biblatex styles.
%%

%%
%% The majority of ACM publications use numbered citations and
%% references.  The command \citestyle{authoryear} switches to the
%% "author year" style.
%%
%% If you are preparing content for an event
%% sponsored by ACM SIGGRAPH, you must use the "author year" style of
%% citations and references.
%% Uncommenting
%% the next command will enable that style.
%%\citestyle{acmauthoryear}

%%%%%%%%%%%%%%%%%%%%%%%%%%%%%%%%%%%%%%%%%%%%%%%%%% 
\copyrightyear{2025}
\acmYear{2025}
\setcopyright{cc}
\setcctype{by}
\acmConference[WWW Companion '25]{Companion Proceedings of the ACM Web Conference 2025}{April 28-May 2, 2025}{Sydney, NSW, Australia}
\acmBooktitle{Companion Proceedings of the ACM Web Conference 2025 (WWW Companion '25), April 28-May 2, 2025, Sydney, NSW, Australia}
\acmDOI{10.1145/3701716.3715203}
\acmISBN{979-8-4007-1331-6/2025/04}
%%%%%%%%%%%%%%%%%%%%%%%%%%%%%%%%%%%%%%%%%%%%%%%%%%

%%
%% end of the preamble, start of the body of the document source.

\begin{document}

%%
%% The "title" command has an optional parameter,
%% allowing the author to define a "short title" to be used in page headers.
\title{Use of Air Quality Sensor Network Data for Real-time Pollution-Aware POI Suggestion}
%%
%% The "author" command and its associated commands are used to define
%% the authors and their affiliations.
%% Of note is the shared affiliation of the first two authors, and the
%% "authornote" and "authornotemark" commands
%% used to denote shared contribution to the research.
%\author{Ben Trovato}
%\authornote{Both authors contributed equally to this research.}
%\email{trovato@corporation.com}
%\orcid{1234-5678-9012}
%\author{G.K.M. Tobin}
%\authornotemark[1]
%\email{webmaster@marysville-ohio.com}
%\affiliation{%
%  \institution{Institute for Clarity in Documentation}
%  \city{Dublin}
%  \state{Ohio}
%  \country{USA}
%}

%%% POLIBA

\author{Giuseppe Fasano}
\email{giuseppe.fasano@poliba.it}
\affiliation{
  \institution{Polytechnic University of Bari}
  \city{Bari}
  \country{Italy}
}

\author{Yashar Deldjoo}
\email{yashar.deldjoo@poliba.it}
\affiliation{
  \institution{Polytechnic University of Bari}
  \city{Bari}
  \country{Italy}
}

\author{Tommaso di Noia}
\email{tommaso.dinoia@poliba.it}
\affiliation{
  \institution{Polytechnic University of Bari}
  \city{Bari}
  \country{Italy}
}




%%% AUG

\author{Bianca Lau}
\affiliation{
  \institution{A.U.G. Signals Ltd.}
  \city{Toronto}
  \country{Canada}
}

\author{Sina Adham-Khiabani}
\affiliation{
  \institution{A.U.G. Signals Ltd.}
  \city{Toronto}
  \country{Canada}
}

\author{Eric Morris}
\email{eric.morris@augsignals.com}
\affiliation{
  \institution{A.U.G. Signals Ltd.}
  \city{Toronto}
  \country{Canada}
}

\author{Xia Liu}
\affiliation{
  \institution{A.U.G. Signals Ltd.}
  \city{Toronto}
  \country{Canada}
}



%%% MTU

\author{Ganga Chinna Rao Devarapu}
\email{chinna.devarapu@mtu.ie}
\affiliation{
  \institution{Munster Technological University}
  \city{Cork}
  \country{Ireland}
}

\author{Liam O'Faolain}
\affiliation{
  \institution{Munster Technological University}
  \city{Cork}
  \country{Ireland}
}



\begin{comment}
    
\author{Giuseppe Fasano, Yashar Deldjoo, Tommaso di Noia, Bianca Lau, Sina Adham-Khiabani, Eric A. Morris, Xia Liu}
\email{bianca@augsignals.com, sina@augsignals.com, eric.morris@augsignals.com, xia@augsignals.com}
\affiliation{
  \institution{A.U.G. Signals Ltd.}
  \city{Toronto}
  \state{Ontario}
  \country{Canada}
}

\end{comment}

%%
%% By default, the full list of authors will be used in the page
%% headers. Often, this list is too long, and will overlap
%% other information printed in the page headers. This command allows
%% the author to define a more concise list
%% of authors' names for this purpose.
\renewcommand{\shortauthors}{Giuseppe Fasano et al.}

%%
%% The abstract is a short summary of the work to be presented in the
%% article.
\begin{abstract}

This demo paper introduces \texttt{AirSense-R}, a privacy-preserving mobile application that delivers real-time, pollution-aware recommendations for urban points of interest (POIs). By merging live air quality data from AirSENCE sensor networks in Bari (Italy) and Cork (Ireland) with user preferences, the system enables health-conscious decision-making. It employs collaborative filtering for personalization, federated learning for privacy, and a prediction engine to detect anomalies and interpolate sparse sensor data. The proposed solution adapts dynamically to urban air quality while safeguarding user privacy. The \textbf{code} and demonstration \textbf{video} are available at \href{https://github.com/AirtownApp/Airtown-Application.git}{\textcolor{blue}{https://github.com/AirtownApp/Airtown-Application.git}}.
\end{abstract}


\begin{comment}
    

\begin{abstract}
This demo paper presents \texttt{AirSense-R}, a privacy-preserving mobile application that provides real-time, pollution-aware recommendations for points of interest (POIs) in urban environments. By combining real-time air quality monitoring data with user preferences, the proposed system aims to help users make health-conscious decisions about the locations they visit. The application utilizes collaborative filtering for personalized suggestions, and federated learning for privacy protection, and integrates air pollutant readings from AirSENCE sensor networks in cities such as Bari, Italy, and Cork, Ireland. Additionally, the AirSENCE prediction engine can be employed to detect anomaly readings and interpolate for air quality readings in areas with sparse sensor coverage. This system offers a promising, health-oriented POI recommendation solution that adapts dynamically to current urban air quality conditions while safeguarding user privacy. The \textbf{code} of \texttt{AirTOWN} and a demonstration \textbf{video} is made available at the following repo: \href{https://github.com/AirtownApp/Airtown-Application.git}{\textcolor{blue}{https://github.com/AirtownApp/Airtown-Application.git}}.
\end{abstract}
\end{comment}

%%
%% The code below is generated by the tool at http://dl.acm.org/ccs.cfm.
%% Please copy and paste the code instead of the example below.
%%

    
\begin{CCSXML}
<ccs2012>
   <concept>
      <concept_id>10002951.10003317.10003347.10003350</concept_id>
       <concept_desc>Information systems~Recommender systems</concept_desc>
       <concept_significance>500</concept_significance>
       </concept>
 </ccs2012>
\end{CCSXML}

\ccsdesc[500]{Information systems~Recommender systems}

\keywords{Recommender systems, POI, pollution, privacy, App, mobile, federated learning}
  \begin{comment}

\end{comment}
%% A "teaser" image appears between the author and affiliation
%% information and the body of the document, and typically spans the


%\received{20 February 2007}
%\received[revised]{12 March 2009}
%\received[accepted]{5 June 2009}

%%
%% This command processes the author and affiliation and title
%% information and builds the first part of the formatted document.


\maketitle

\section{Introduction}

\noindent \textbf{Background.} Urban air pollution is a major global concern. The World Health Organization (WHO) reports that 99\% of the world’s population breathes air that exceeds its recommended limits \cite{who_airpollution}. This issue is particularly acute in metropolitan areas, where vehicular and industrial emissions prevail. In this setting, point-of-interest (POI) recommendation systems can enhance urban experiences by offering health-conscious options.

\begin{figure}[!t]
    \centering
    \includegraphics[width=0.880\linewidth]{poi_final.png}
  \caption{\texttt{AirTown} integrates real-time data from sensors installed in Bari and Cork, with user preferences to provide personalized, pollution-aware POI recommendations.}

%    \caption{\texttt{AirTown}, a mobile app providing personalized, pollution-aware POI recommendations using real-time sensor data from Bari (Italy) and Cork (Ireland). The app integrates personalization, privacy, health, and walking distance by merging collaborative filtering, user preferences, and air quality data.}
    \label{POI_recsys_scheme}
\end{figure}

\noindent \textbf{Air Sensor Networks in Bari and Cork.} Over the past decade, several European citizen science initiatives—such as EveryAware \cite{everyaware}, Citi-Sense-MOB \cite{Castell2016,SCHNEIDER2017234}, and OpenSense \cite{MUELLER2016171}—have employed portable sensors to deliver real-time air quality data via web and mobile apps. Additionally, stationary monitoring efforts have been made, including the deployment of 9 KOALA sensors in Gold Coast during the 2018 Commonwealth Games \cite{KUHN2021105707} and Chicago’s Array of Things (AoT) Project \cite{2019AGUFM.A24G..04P}. AirSENCE\footnote{\href{https://airsence.com}{AirSENCE.com}} has installed 10 sensors in Bari and 8 in Cork since July 2022, covering areas like schools, residential neighborhoods, and city centers to capture diverse urban air quality data.

\noindent \textbf{Contribution.} This paper introduces \texttt{AirSense-R}, a system that provides real-time, personalized POI recommendations based on current air quality. By integrating predictive analytics with user preferences, it supports healthier choices in urban environments. The system comprises: (i) the \texttt{AirSense-R} Prediction Engine (see Section \ref{sec:poll_sys}) and (ii) the \texttt{AirSense-R} Recommendation Engine (\texttt{AirTown}) (see Section \ref{sec:poll_rec}).

\begin{itemize}
    \item The \texttt{AirSense-R} \textbf{Prediction} system forecasts pollutant levels using historical data from the AirSENCE networks. Methods such as FBProphet are used to capture patterns, detect anomalies, and interpolate data in areas with sparse sensor coverage.
    \item The \texttt{AirSense-R} \textbf{Recommendation} Engine, \texttt{AirTown}, is a mobile application that combines real-time pollution data with user preferences to deliver health-conscious POI recommendations. Figure \ref{POI_recsys_scheme} illustrates how the app integrates personalization, real-time data, privacy preservation, and distance considerations.
\end{itemize}

\begin{comment}
    

\noindent \textbf{Background.} Air pollution is a pervasive problem affecting nearly all urban populations worldwide. According to the World Health Organization (WHO), 99\% of the global population is breathing air that exceeds the guideline limits by the WHO \cite{who_airpollution}. This problem is particularly severe in metropolitan areas, where emissions from vehicles and industrial processes contribute to high pollution levels. In this context, point-of-interest (POI) recommendation systems can enhance user experiences and contribute to public health by providing health-conscious suggestions. 


\begin{figure}[!t]
    \centering
    \includegraphics[width=0.880\linewidth]{poi_final.png}
    \caption{Developed as a mobile app, \texttt{AirTown} uses real-time sensor data from Bari (Italy) and Cork (Ireland) to offer personalized, pollution-aware POI recommendations. It balances four key factors—personalization, privacy, health, and walking distance—by combining collaborative filtering, user preferences, and air quality data to deliver tailored, health-conscious suggestions.}
    \label{POI_recsys_scheme}
\end{figure}

\noindent \textbf{Air Sensor Network in Bari and Cork.}
Several European citizen science projects were initiated in the last ten years that used compact sensors to communicate real-time air quality data via web and mobile applications, including EveryAware \cite{everyaware}, Citi-Sense-MOB \cite{Castell2016,SCHNEIDER2017234}, and OpenSense \cite{MUELLER2016171}. These projects emphasized portability in that sensors were either carried by participants or mounted to bicycles, automobiles, and public transit vehicles. Stationary monitoring strategies have also been utilized for similar purposes, one example of which being the use of 9 KOALA sensors installed around the city of Gold Coast, Australia to monitor changes in carbon monoxide (CO) and fine particulate (PM\textsubscript{2.5}) levels due to increased traffic during the 2018 Commonwealth Games \cite{KUHN2021105707}. A similar ongoing initiative in Chicago called the Array of Things (AoT) Project utilizes more than 140 data nodes comprised of compact meteorological sensors and air quality monitors measuring CO, nitrogen dioxide (NO\textsubscript{2}), ozone (O\textsubscript{3}), sulfur dioxide (SO\textsubscript{2}), and hydrogen sulfide (H\textsubscript{2}S) \cite{2019AGUFM.A24G..04P}. AirSENCE\footnote{\href{https://airsence.com}{AirSENCE.com}} has deployed two sensor networks in Bari, Italy and Cork, Ireland to monitor local air quality. Ten and eight sensors have been strategically placed across Bari and Cork respectively, covering urban areas such as schools, residential neighborhoods, and city centers. These sensors have been operational since July 2022 and continue to provide valuable data. The primary objective of this deployment was to understand local air quality in diverse urban environments and across locations within each city.  \\


\noindent \textbf{Contribution.} This demo paper introduces \texttt{AirSense-R}, a system offering real-time, personalized POI recommendations based on current air quality. By merging predictive analytics with user preferences, it helps users make healthier choices in urban areas. The system comprises: (i) the \texttt{AirSense-R} Prediction Engine (see Section \ref{sec:poll_sys}) and (ii) the \texttt{AirSense-R} Recommendation Engine (\texttt{AirTown}) (see Section \ref{sec:poll_rec}).

\begin{itemize}
    \item The \texttt{AirSense-R} \textbf{Prediction} system forecasts pollutant concentrations using historical data from AirSENCE networks. Leveraging methods like FBProphet, it captures air quality patterns, detects anomalies, and interpolates data for regions with sparse sensor coverage.
    \item The \texttt{AirSense-R} \textbf{Recommendation} Engine---\texttt{\textbf{AirTown}}---is a mobile app that combines real-time pollution data with user preferences to deliver POI recommendations for health-conscious urban navigation. Figure \ref{POI_recsys_scheme} shows that the app integrates personalization, real-time data, privacy preservation, and distance considerations.
\end{itemize}



%\noindent \textbf{Contribution.} This demo paper introduces \texttt{AirSense-R}, a system that provides real-time, personalized recommendations for points of interest (POIs) based on current air quality conditions. By combining predictive analytics with user preferences, it helps users make healthier and more informed choices when exploring urban areas. The system consists of two main components: (i) the \texttt{AirSense-R} Prediction Engine (see Section \ref{sec:poll_sys}) and (ii) the \texttt{AirSense-R} Recommendation Engine (also called \texttt{AirTown}) (see Section \ref{sec:poll_rec}).


%\begin{itemize}
 %   \item The \texttt{AirSense-R} \textbf{Prediction} system is designed to analyze and forecast air pollutant concentrations using historical data collected from AirSENCE networks. Leveraging advanced prediction techniques such as FBProphet (a time series forecasting tool developed by Facebook that models seasonal effects and trends), the engine captures complex air quality patterns, detects anomalies, and interpolates data for regions with sparse sensor coverage. 
    %This integration of temporal forecasting and spatial data interpolation aim to capture highly localized and accurate air quality information tailored to the user’s specific location.
  %  \item The \texttt{AirSense-R} \textbf{Recommendation} Engine ---
   % \texttt{\textbf{AirTown}} -- is a mobile application that combines real-time pollution data from AirSENCE quality sensor networks with user preferences to deliver POI recommendations tailored for health-conscious urban navigation. Figure \ref{POI_recsys_scheme} shows the application leverages four essential features in its recommendation process: personalization, real-time sensory data integration, privacy preservation, while also considering the distance between users and recommended venues. %By balancing these factors, AirTOWN aims to help users make informed, healthier choices about the places they visit.

%\end{itemize}

\end{comment}








% This paper presents \texttt{AirTOWN}, a mobile application that combines real-time pollution data from an air quality sensor network with user preferences to deliver POI recommendations tailored for health-conscious urban navigation. 
%\texttt{AirTown} is a mobile application that combines real-time pollution data from AirSENCE quality sensors networks with user preferences to deliver POI recommendations tailored for health-conscious urban navigation. Figure \ref{POI_recsys_scheme} shows the application leverages four essential features in its recommendation process: \textit{personalization}, \textit{real-time sensory data integration}, and \textit{privacy preservation}, while also considering the \textbf{distance} between users and recommended venues. By balancing these factors, \texttt{AirTOWN} aims to help users make informed, healthier choices about the places they visit. 
% Figure~\ref{POI_recsys_scheme} illustrates this approach: when a user requests recommendations for a place to visit (e.g., a restaurant) the system first identifies nearby POIs, then personalizes suggestions using collaborative filtering (CF) based on individual preferences. Finally, the recommendations are re-ranked, integrating proximity information with real-time Air Quality Index (AQI) measurements generated by AirSENCE\footnote{\href{https://airsence.com}{AirSENCE.com}}, a compact real-time ambient air quality monitor. \texttt{AirTOWN} relies on AirSENCE sensor networks in Bari, Italy, and Cork, Ireland, which offer contrasting urban settings with distinct air quality challenges.  This data allows \texttt{AirTOWN} to assess local pollution levels dynamically, striving to provide recommendations tailored to the user's needs. Additionally, the AirSENCE prediction engine can be used to analyze and forecast air pollutant concentrations based on the past collected local data. Complex air quality patterns can be captured by integrating prediction techniques such as FBProphet, enabling the detection of anomalies and interpolation of data for areas with limited sensor coverage. As a result, by utilizing both the temporal predictions and spatial interpolation of air quality data we can provide highly localized information, ensuring precise accuracy specific to the user's location. 




%A few notable works focus on monitoring exposure \cite{LIN2023114186} or predicting pollution along routes without suggesting healthier POIs \cite{IoT-app}. Health-based recommendation systems typically merge user preferences with health metrics, e.g., in food-related personalized applications \cite{health-aware,10.1145/3503252.3531312}, however often rely on static data rather than real-time inputs. Privacy-centric systems \cite{POIRSDemo,chen2020practical} also lack real-time environmental integration. In contrast, \texttt{AirTOWN} ~combines real-time pollutant data, personalization, and privacy to deliver health-conscious POI recommendations that adapt dynamically to urban air quality. Overall these features make \texttt{AirTOWN}~a promising approach toward a holistic, health-conscious recommendation system, balancing user satisfaction, privacy, and well-being. Despite some limitations, \texttt{AirTOWN}~demonstrates the potential for real-time pollution data integration, personalization, and privacy to support healthier urban navigation, a rarity among current systems.


%The main contributions of this paper are multi-fold: \textit{(i)} personalization via collaborative filtering, \textit{(ii)} pollution awareness through real-time AQI data, \textit{(iii)} sensory data integration from urban sensors in Bari and Cork, \textit{(iv)} privacy preservation through federated learning, and \textit{(v)} distance awareness. 



%The main contributions of this paper include the integration of five essential features: \textit{(i)} personalization using collaborative filtering to align recommendations with user preferences, \textit{(ii}) pollution awareness through real-time Air Quality Index (AQI) data to help users avoid high-pollution areas, \textit{(iii)} real-time sensory data integration from urban sensors in Bari, Italy, and Cork, UK, ensuring up-to-date environmental insights, \textit{(iv)} privacy preservation via federated learning that keeps personal data on the device, and \textit{(v)} distance awareness to prioritize nearby, accessible POIs. Together, these features make \airtown~a promising start toward a comprehensive, health-conscious recommendation system that balances user satisfaction, privacy, and well-being. Despite some limitations, \airtown demonstrates the potential of integrating real-time pollution data, personalization, and privacy considerations to support healthier urban navigation.

%In the following sections, we describe the technical architecture of \airtown and provide experimental results demonstrating its effectiveness in supporting healthier POI choices in diverse user scenarios across Bari and Cork.

%\airtown ~leverages four essential features in its recommendation process: \textit{personalization}, \textit{pollution awareness}, \textit{real-time sensory data integration}, and \textit{privacy preservati}on, all while considering the distance between users and recommended venues. By balancing these factors, \airtown~ aims to create a system that helps users make healthier choices regarding the places they visit. Figure \ref{POI_recsys_scheme} illustrates this approach: when a user initiates a recommendation request, \airtown identifies nearby POIs and personalizes the suggestions using collaborative filtering (CF) based on user preferences. The recommendations are then re-ranked by incorporating real-time Air Quality Index (AQI) data and proximity considerations, ensuring that users are presented with options that are both personalized and health-conscious.


%An effective POI recommendation system (RS) must balance three key factors: \textit{personalization}, \textit{pollution levels,} and \textit{distance} of the target venue (POI). Personalization ensures that recommendations are tailored to individuals' tastes and preferences, enhancing user satisfaction. Incorporating pollution levels into the recommendation process helps users make health-conscious decisions by avoiding highly polluted areas. Distance plays a crucial role in the practicality of recommendations, as users are more likely to visit nearby locations. Balancing these factors is essential for delivering useful and health-preserving POI recommendations.

% Yet another critical consideration in modern RS is \textit{privacy}, especially in POI recommendations due to the sensitive nature of user location and preference data. Trustworthy recommender systems must balance utility with privacy, not only for legal reasons (regulations such as GDPR) but also from social and ethical perspectives, so as not to compromise users' private and sensitive data. Naturally, there is an inherent trade-off between utility and privacy: the more information you reveal to a POI recommender (e.g., your last visited place), the better it can personalize recommendations, but at the price of higher privacy leakage. Many users may not want their mobile devices to simply track their information. Limiting the information shared reduces personalization, so the question is how to balance these factors. Our answer to this is making use of privacy-by-design techniques, using federated Learning (FL), which offers a solution by enabling models to be trained across devices without centralizing sensitive user data.


% Our proposed demo is a \textit{real-time} mobile application named \airtown which combines four key factors—personalization, pollution levels, distance, and privacy into a unified system—to provide effective and health-conscious POI recommendations. The proposed solution is developed for Android \yashar{check} and is an application that features a user-friendly interface, simplifying user interaction while ensuring privacy through FL. A notable aspect of \airtown is its integration of real-time sensory data, such as Air Quality Index (AQI), temperature, humidity, and pressure, received from AirSENCE sensors. This real-time data ensures that recommendations are accurate and up-to-date, allowing users to make informed decisions about the places they visit.





% Figure  \ref{POI_recsys_scheme} illustrates how \airtown integrates personalization, pollution levels, and distance into its recommendation process. When a user requests recommendations, the system first identifies POIs within a certain radius (e.g., 1 km) of the user's current location, addressing the distance factor. Next, a collaborative filtering (CF) model predicts user preferences to personalize the recommendations. Finally, the system applies a weighted re-ranking function that balances the predicted preferences, the Air Quality Index (AQI) of the POIs, and the proximity to the user. This multi-faceted approach ensures that the recommended places are not only aligned with the user's interests but also promote healthier choices by avoiding areas with poor air quality, all within a convenient distance. On the right, you see the interface of \airtown, which displays personalized recommendations based on a user’s preferences, the proximity of POIs, and current AQI levels. It provides a clear, real-time overview of nearby locations, highlighting those with healthier air quality to help users avoid polluted areas. 


%The main features of \airtown are the following:
% \yashar{Here refer to the overall picture, and write an example}
%Figure \ref{POI_recsys_scheme} shows on an high-level perspective how the recommendation is computed in \airtown. We highlight two aspect: the recommendation is personalized for each user and it is both based on users' preferences and places air quality. In the example shown, two user, an elderly person and a young one, ask for recommended places. The list of places recommended at the end of the process are different per user, because of personal preferences and air quality of the locations. 

%\begin{itemize}
    %\item \textbf{Pollution-aware Recommendations.} AirTOWN provides personalized Points of Interest (POI) suggestions based on real-time air pollution data, helping users make health-conscious decisions about their activities and locations.
  
%\item \textbf{Real-time Sensor Data Integration} The system utilizes real-time data from AirSENCE sensors, which monitor various environmental parameters including AQI, temperature, humidity, and pressure, to ensure accurate and up-to-date recommendations.
  
%\item \textbf{Federated Learning for Privacy.} AirTOWN employs privacy-by-design federated learning algorithms, enabling the training of recommendation models directly on user devices. This approach protects user privacy by ensuring that sensitive data does not need to be shared or stored centrally.

%\item \textbf{Modular and User-friendly Design:} The system is organized in a modular architecture with several servers, enhancing scalability and maintainability. The user-friendly mobile application interface ensures easy access and interaction for users, making the system practical and accessible for everyday use.
%\end{itemize}




%\section{Related Systems}

%Despite the increased interest in health-conscious recommendation systems in  recent years, %\cite{healtcare_review},not much attention was paid to the integration of pollutant data in recommendations. Some works focus on monitoring user exposure to pollutants, like in \cite{LIN2023114186}. \cite{IoT-app} used pollutant data and AQI to predict the pollution level of a route from a location to another. However, no recommendations are provided by the system. Instead, our system uses this data in a re-ranking function to suggest healthier Points of Interest (POIs). 

%Data fusion techniques are already used in health-conscious recommendation systems, like in mobile food applications. For instance, preferences and health-conscious recommendations can be integrated to provide recommended recipes\cite{health-aware,10.1145/3503252.3531312}. While in these applications integration data is in form of label or a calories function, in \airtown real-time sensor data is involved.

%Finally, to achieve data privacy, our approach is inspired by developments such as \cite{POIRSDemo}, and \cite{chen2020practical}. However, we extend their capabilities by incorporating user feedback, system usability, and a health-oriented re-ranker, as seen in \cite{Elahi2015InteractionDI}.

%In conclusion, our work builds upon existing health-oriented technologies by synthesizing the best aspects of each while incorporating novel features for more effective results. By leveraging user feedback, system usability data, and customizability, \airtown seeks to enhance how we manage our exposure to air pollution.

\begin{figure} [!t]
    \centering
    \includegraphics[width=1\linewidth]{FL_final.png}
    \caption{The architecture of the proposed system.}
    \label{FL_architecture}
\end{figure}
\section{Proposed System}
As shown in Figure \ref{FL_architecture}, the proposed system is designed to leverage a client-server model featuring four layers: Application, Service, Interface, and Data Resources Layer.

% The architecture of the \airtown app is designed to leverage a multi-layered client-server model to enhance user experience by providing personalized point-of-interest (POI) recommendations based on air quality and user preferences. This section describes the layers involved in the system's architecture, as represented in Figure \ref{FL_architecture}.


\textbf{Application Layer.} This layer serves as the user interface on mobile smartphones for \texttt{AirTown} and performs POI suggestions that integrate real-time air quality information. The suggestions are computed by the Recommendation Engine (cf. Section \ref{sec:poll_rec}). The Application Layer collects data about user preferences, but it never shares user data since the recommendation model is trained through the Federated Learning (FL) approach, allowing for user privacy.

%To obtain an effective POI recommendation system (RS), we focused on balancing three key factors: personalization, pollution levels, and distance of the target venue (POI). When a user requests suggestions, the system first identifies POIs within a certain radius (e.g., 1 km) of the user’s current location, addressing the distance factor. Next, a collaborative filtering (CF) model predicts user preferences to personalize the recommendations. We chose the Matrix Factorization (MF) solution over more sophisticated alternatives as Deep Learning, because of its better transparency and lower complexity, allowing for distributed training on devices with limited resources. Finally, the system applies a weighted re-ranking function that balances the predicted preferences and the AQI of the POIs. The parameter $\alpha \in [0,1]$ modulates the strength of AQI influence in the re-ranking process $S = \alpha \cdot S_{MF} + (1 - \alpha) \cdot S_{AQI}$. To enhance MF performances, the model is trained on user preferences, which are locally collected via surveys on already visited POIs. The Application Layer never shares user data and the update of the local model is performed in a client-server implementation of FL paradigm, allowing for user privacy.


% \textbf{Service Layer.} Hosted on a server, the Service Layer manages back-end processes such as login, registration, and data retrieval for the Application Layer. \textcolor{black}{It approximates AQI data for unsensed locations using radial basis function interpolation and} performs federated learning rounds for global model training via the Federated Averaging protocol. Only item embedding updates are shared, while user embeddings remain local.

\textbf{Service Layer.} Hosted on a server, the Service Layer manages back-end processes such as login, registration, and data retrieval for the Application Layer. It orchestrates federated learning rounds.


\textbf{Interface and Data Resources Layer.} Interface Layer comprises APIs linking the Service Layer with Data Resources. The AirSENCE API provides localized air quality data, while Directions and Places APIs enable navigation and location information. The Google Database manages Google-related data (Photos, place information, routing data), and the AirSENCE Database stores air quality measurements by sensor networks in Bari and Cork.

%\textbf{Service Layer.} The Service Layer is hosted on a server and is responsible for various back-end processes. This layer handles login and registration tasks. Furthermore, the Service Layer retrieves all the data requested by the Application Layer and collected in the Data Resources Layer (routes, pollutant data, photos). The layer uses radial basis function interpolation to approximate unknown AQI data on places not covered by sensors. Finally, this layer orchestrates the federated learning rounds to achieve the distributed training of a global model using Federated Averaging protocol. Only updates of items embedding are shared between clients and server, while users embeddings are stored locally on devices.

%\textbf{Interface and Data Resources Layer.} The Interface Layer includes several APIs that facilitate communication between the Service Layer and the Data Resources Layer. The AirSENCE API provides access to localized air quality data from the AirSENCE database. Directions API and Places API enable the app to integrate navigational and place information into the service offerings. The Data Resources Layer collects data used by the Application Layer. The Google Database stores and retrieves information related to Google services (Photos, places information, routing data). The AirSENCE Database collects and provides access to air quality measurements. This layer stores the parameters of the federated model too.

%\subsection*{Interface Layer}
%The Interface Layer includes several APIs that facilitate communication between the client applications and the server:
%\begin{itemize}
    %\item \textbf{AirSENCE API}: Provides access to localized air quality data from the AirSENCE database.
    %\item \textbf{Directions API and Places API (Google)}: Enable the app to integrate navigational and place information into the service offerings.
%\end{itemize}


%\subsection*{Data Resources}
%The system utilizes a variety of data resources to support its functions:
%\begin{itemize}
    %\item \textbf{Google Database}: Stores and retrieves information related to Google services (Photos, places information, routing data).
    %\item \textbf{AirSENCE Database}: Collects and provides access to air quality measurements.
    %\item \textbf{User Data Server}: Manages sensitive user information securely to maintain privacy.
    %\item \textbf{Model Parameters}: Stores the parameters of the federated learning model, facilitating continuous learning and system enhancement.
%\end{itemize}

%\section{Experiments and Demonstrations.}
%To show the functioning of \airtown, we provide intra-user and inter-user investigations in Bari (Italy). To overcome the current limited sensor coverage provided by AirSENCE over the city, we considered synthetic AQI data.\footnote{\url{https://anonymous.4open.science/r/Airtown-Application/}}
\subsection*{AirSENCE Sensor Network.}
A key feature of \texttt{AirTOWN} is the integration of real-time air quality data. After sensor data collection, the \textcolor{black}{Air Quality Index} (AQI) is calculated to assess air quality in real time. Effective integration relies on a robust sensor network. AirSENCE utilizes a multi-parameter sensor array with ML-based signal processing and data fusion, forming an industrial Internet of Things network. Ambient air is actively sampled, with measurements averaged every minute using on-board processing and storage before transmission to a host server. Pollutants measured include CO, NO, NO\textsubscript{2}, O\textsubscript{3}, SO\textsubscript{2}, PM\textsubscript{1}, PM\textsubscript{2.5}, PM\textsubscript{10}, temperature, humidity, and pressure. Data from two AirSENCE networks in Bari (Italy) and Cork (Ireland), over two years demonstrate the effectiveness of distributing multiple monitoring devices across diverse environments such as schools, residential areas, ferry docks, and city centers to capture urban air quality.


%One of the key features in \texttt{AirTOWN} is the integration of real-time air quality data. After collecting data from sensors, the AQI can be calculated to assess the air quality in real-time. However, real-time sensor data integration is only effective when it is supported by a robust and reliable sensor network. AirSENCE incorporates a multi-parameter sensor array with machine learning-based signal processing and data fusion to provide an industrial Internet of Things sensor network. Ambient air is actively sampled and measurements are averaged every minute with on-board processing and storage, after which the results are transmitted to a host server. Pollutants measured include CO, NO, NO\textsubscript{2}, O\textsubscript{3}, SO\textsubscript{2}, PM\textsubscript{1}, PM\textsubscript{2.5}, and PM\textsubscript{10}, along with temperature, humidity, and pressure. Data obtained from two AirSENCE networks deployed in Bari, Italy and Cork, Ireland over the course of two years has demonstrated the value of having multiple air quality monitoring devices distributed throughout a city. These devices have been placed in diverse environments, including schools, residential neighbourhoods, ferry docks and city centers to accurately capture the air quality across the cities. 


\begin{figure}[!t]
    \centering
    \includegraphics[width=0.75\linewidth]{weeklydiurnalNO2_2.png}
    \caption{Weekly diurnal trends of NO\textsubscript{2} for six AirSENCE devices in Bari, Italy 2023.}
    \label{simulations2}
\end{figure}

\begin{figure}[h]
    \centering
    \includegraphics[width=0.750\linewidth]{potentialevent_2.png}
    \caption{NO readings for six AirSENCE devices in Cork, Ireland on January 29, 2024.}
    \label{simulations3}
\end{figure}

For instance, Figure \ref{simulations2} displays the weekly diurnal patterns of NO\textsubscript{2} from six AirSENCE devices in Bari, Italy, in 2023. These devices were located near a daycare, an industrial parking lot, a residential cul-de-sac, and a ferry dock port. While all devices showed similar weekly trends, NO\textsubscript{2} concentrations varied significantly by location, primarily due to vehicle emissions. Higher concentrations occurred during rush hours around 6 a.m. and 6:30 p.m., with the highest levels at device 158 in a residential area, likely reflecting increased traffic. This example underscores the advantage of multiple monitoring devices in providing a more accurate local air quality representation compared to single monitoring stations.

Additionally, multiple real-time monitoring devices can detect sudden, localized pollution events. Figure  \ref{simulations3} illustrates a notable spike in NO levels by device 164 around 6 p.m., which was not observed by nearby devices. This highlights how real-time data can identify abrupt air quality changes that static data might miss, enhancing the ability to offer accurate and timely recommendations based on current air conditions.


\vspace{-2mm}





%For example, Figure \ref{simulations2} shows the NO\textsubscript{2} weekly diurnal patterns for six AirSENCE devices located in Bari, Italy in 2023. These devices were placed at various sites including near a daycare, in a parking lot of an industrial facility, in a residential cul-de-sac and at the port of the ferry dock. Although all devices showed similar trends throughout the week, the concentrations of NO\textsubscript{2} varies significantly between locations. This variation is expected, as NO\textsubscript{2} emissions are primarily driven by vehicle-exhaust. The higher concentrations typically occur during rush hours - around 6 a.m. and 6:30 p.m., when people are commuting. Notably, the highest emissions were recorded at AirSENCE device 158, located in a residential area, which likely experiences higher traffic volume. This example highlights the benefits of deploying multiple monitoring devices throughout a city, as they provide a more accurate representation of air quality at a certain location compared to a conventional monitoring station. 



%Another advantage of using multiple real-time air quality monitoring devices is their ability to detect sudden, localized air pollution events. Figure 3 highlights a significant spike in NO readings by AirSENCE device 164 around 6 p.m., while other nearby devices did not show a similar increase. This demonstrates how real-time data can capture sudden fluctuations in air quality that may be missed by static data. Without real-time data, localized events could go undetected, limiting the ability to provide accurate and responsive recommendations that reflect current air quality changes for users.   


\subsection*{AirSENCE Prediction Engine}
\label{sec:poll_sys}

Predicting and modeling air pollutant concentrations enhance air quality analysis by identifying patterns and anomalies in real-time data. Air quality data shows long-term trends, daily/weekly cycles from human activities, and seasonal variations influenced by climate factors. Techniques such as Seasonal-Trend Decomposition using Loess (STL) and Fourier Transforms separate time series into trend, seasonality, and residuals. Advanced models like ARIMA, Long Short-Term Memory (LSTM)\cite{ref2}, and FBProphet capture complex interactions. FBProphet is particularly effective due to its flexibility with irregular data, non-linear trend modeling, and support for seasonality and holidays, as demonstrated by its accurate NO forecasts in Bari, Italy.

Pollution events—like wildfire-induced particulate spikes or elevated NO\textsubscript{2} during rush hours—are detected through residual analysis that highlights deviations from expected patterns. This anomaly detection method, which considers factors like time, season, and location, offers more precise assessments than traditional AQI methods. Furthermore, prediction models enable spatial interpolation in areas with sparse sensor coverage. Integrating the AirSENCE prediction engine with a real-time monitoring network facilitates early detection of pollution events and provides localized air quality data for pollution-aware POI suggestions.
%Predicting and modeling air pollutant concentrations enhances air quality analysis by identifying periodic patterns and anomalies from real-time data. Air quality data exhibits long-term trends, daily and weekly cycles from human activities, and seasonal variations influenced by climate factors like temperature and wind. Decomposition techniques such as Seasonal-Trend Decomposition using Loess (STL) and Fourier Transforms separate the time series into trend, seasonality, and residuals. Advanced machine learning models like ARIMA, Long Short-Term Memory (LSTM)\cite{ref2}, and FBProphet are increasingly used for their ability to capture complex interactions. FBProphet is especially effective for air quality prediction due to its flexibility with irregular data, modeling of non-linear trends, and support for seasonality and holidays. It accurately forecasts long-term air quality, as demonstrated by its alignment with actual NO readings in Bari, Italy.

%Air pollution events, such as spikes in particulate matter from wildfires or elevated NO\textsubscript{2}  during rush hours, are identified through residual analysis, which detects deviations from expected patterns. Unlike traditional AQI methods, anomaly detection accounts for contextual factors like time, season, and location, providing more accurate pollution assessments.

%Additionally, prediction models enable spatial interpolation of air quality data, allowing estimates in areas with sparse sensor coverage. Integrating the AirSENCE prediction engine with a real-time monitoring network facilitates early detection of pollution events and data interpolation, offering more accurate and localized air quality information for pollution-aware POI suggestions.

%Predicting and modeling air pollutant concentrations can add significant depth to the air quality analysis after the collection of real-time data. Air quality data often exhibits distinct periodic patterns, including long-term trends, daily and weekly cycles, in addition to seasonal variations. Daily patterns arise from human activities, such as traffic and industrial operations, while weekly cycles reflect workweek versus weekend behaviors. Seasonal patterns, on the other hand, are influenced by climatic factors like temperature, humidity, and wind. After capturing and understanding these periodic patterns, anomalies or pollution events can be isolated from the regular trends. 

%To effectively analyze these patterns, decomposition techniques are used to separate the time series into components: trend (long-term changes), seasonality (recurring cycles), and residuals (unexplained variations). Classical statistical methods like Seasonal-Trend Decomposition using Loess (STL) or algorithms like Fourier Transforms have been widely employed for this purpose \cite{ref1}. However, advanced machine learning models, such as ARIMA, Long Short-Term Memory (LSTM) \cite{ref2}\cite{ref3}, and Meta Prophet (FBProphet), have gained prominence in recent years for their ability to capture complex interactions between components. FBProphet, in particular, is a powerful algorithm for air quality prediction and trend decomposition due to its flexibility in handling irregularly spaced data, robust modeling of non-linear trends, and built-in support for seasonality and holiday effects \cite{ref4}. It is especially effective in automatically identifying patterns and projecting them into the future, making it ideal for long-term air quality forecasting. Figure 4 presents the actual and predicted NO readings, utilizing FBProphet and AirSENCE data from device 161 in Bari, Italy after being trained on approximately one and half years of historical data. The predicted values for 2024 demonstrate strong alignment with the actual measured readings, highlighting the model's promising accuracy.   

\begin{figure}[h]
    \centering
    \includegraphics[width=0.9\linewidth]{predict.png}
    \caption{NO pattern prediction using FBProphet and AirSENCE device 161 in Bari, Italy.}
    \label{simulations}
\end{figure}

%Air pollution events, such as sudden spikes in particulate matter due to wildfires, increased O\textsubscript{3} levels from photochemical reactions, or elevated NO\textsubscript{2} concentrations during rush hours, can then be identified. The residuals from decomposition play a key role here, as they reveal deviations from expected trends and seasonal patterns. By analyzing these residuals, we can then identify and attribute pollution peaks or other atypical atmospheric phenomena to their respective causes. As a result, we are able to detect anomaly readings which arise due to air pollution events at its early stages. Unlike traditional methods that rely solely on AQI and threshold limits, anomaly detection accounts for contextual factors like the time of day, season and location. By comparing real-time readings against the general trends identified earlier, we can determine whether the observed values are typical or unusual and then provide a more accurate assessment of current air pollution levels. 

%Lastly, prediction models can be used to spatially interpolate air quality data across a region. Given the challenges of deploying an extensive sensor network, using fewer sensors combined with prediction models can help estimate the air quality in areas with sparse coverage.  

%By integrating the AirSENCE prediction engine alongside a real-time air quality monitoring sensor network, air pollution events can be detected at their early stages and data can be interpolated for areas with limited sensor coverage. This approach enables a more accurate and localized air quality information to achieve a more meaningful pollution-aware POI suggestions for users. 

\subsection*{Recommendation Engine}
We developed a health-oriented POI recommendation system that balances personalization, pollution levels, distance, and privacy. Personalization helps users discover new items aligned with their tastes, while integrating pollution awareness encourages healthier choices. To promote walking exploration, only POIs within a given radius (e.g., 1\,km) are considered. Finally, user privacy is preserved by keeping preferences and mobility data local, in compliance with GDPR~\cite{GDPR}.

As illustrated in Figure~\ref{POI_recsys_scheme}, the system first obtains nearby POIs via the Places API. A Matrix Factorization (MF) model~\cite{koren2009matrix} then predicts user preferences, chosen for its transparency and suitability for resource-limited devices. Real-time AQI values are retrieved from the AirSENCE Database; missing areas are approximated with radial basis function interpolation. The final recommendation score for a POI is:
\[
S = \alpha \cdot S_{\mathrm{MF}} + (1 - \alpha) \cdot S_{\mathrm{AQI}},
\]
where $\alpha \in [0,1]$ modulates the weight of air quality in the re-ranking step.

To train the MF model, user preferences are gathered locally (e.g., through surveys) and remain on the device. We adopt a client-server Federated Learning (FL) scheme, where each client holds a private user embedding and a shared POI embedding matrix. During training, clients update both embeddings but only send the POI embedding updates to the server, which then aggregates them via Federated Averaging~\cite{mcmahan2017communication} and redistributes the aggregated updates to all clients. This process can be scheduled periodically (e.g., biweekly) to preserve privacy while continuously improving recommendations.

\begin{comment}
    
\subsection*{Recommendation engine}
To obtain an effective health-oriented POI recommendation system (RS), we focused on balancing four key factors: personalization, pollution levels, distance of the target venue (POI), and privacy. \textcolor{black}{Personalization is a fundamental aspect of recommendation systems to allow the user to explore new items according to his preferences. Combining personalization and pollution awareness
leads to healthier suggestions. To stimulate afoot exploration, we chose to take into account the distance of the target venues. Finally, the engine works with sensitive data (user preferences and movements) and user privacy should be preserved, as requested by the General Data Protection Regulation \cite{GDPR}}.

As shown by Figure \ref{POI_recsys_scheme}, when a user requests suggestions, the system first identifies POIs within a certain radius (e.g., 1 km) of the user’s current location, addressing the distance factor. \textcolor{black}{This step is accomplished by contacting Places API.} Next, a collaborative filtering (CF) model predicts user preferences to personalize the recommendations. We chose the Matrix Factorization (MF) \cite{koren2009matrix} solution over more sophisticated alternatives such as Deep Learning, because of its better transparency and lower complexity, allowing for distributed training on devices with limited resources. \textcolor{black}{Then, real-time AQI values are retrieved from the AirSENCE Database, and values for areas uncovered by the sensor network are approximated using radial basis function interpolation.} Finally, the system applies a weighted re-ranking function that balances the predicted preferences and the AQI of the POIs.  The parameter $\alpha \in [0,1]$ modulates the strength of AQI influence in the re-ranking process: $S = \alpha \cdot S_{MF} + (1 - \alpha) \cdot S_{AQI}$.

To enhance MF performances, the model is trained on user preferences, which are locally collected via surveys on already visited POIs. User data is kept local and the update of the MF model is performed in a client-server implementation of the Federated Learning paradigm, allowing for user privacy. \textcolor{black}{Each client model is composed of a private user embedding and a public matrix collecting POI embeddings. In the FL process, each client locally updates the embeddings, training the model with user data. However, only the updates of POI embeddings are shared with the server, which aggregates them with Federated Averaging \cite{mcmahan2017communication} and sends them back to the client. When AirTown is opened to the public, this operation should be held on a biweekly basis.}
\end{comment}

%In the FL process, the server asks clients to train their local version of MF using their private data. Then, the server collects only the updates of the items embeddings from the clients. The updates are merged with the Federated Averaging protocol\cite{mcmahan2017communication} and the updated item embedding are sent back to the model.

%In FL rounds, only item embedding updates are shared, while user embeddings remain local. 
% Federated Averaging protocol\cite{mcmahan2017communication} is adopted as aggregation protocol because of its discrete communication cost and the partial parallelization it achieves as compared to other solution, like Federated Stochastic Gradient Descent. 

\begin{comment}
\begin{figure}[t]
\begin{subfigure}{.5\textwidth}
  \centering
  % include first image
  \includegraphics[width=.7\linewidth]{user-study condensed top4.png}  
  \caption{Recommendation lists for two users in the simulated scenario.}
  \label{user-study}
\end{subfigure}
\begin{subfigure}{.5\textwidth}
  \centering
  % include second image
  \includegraphics[width=.5\linewidth]{boxplots.png}  
  \caption{Box-plots of Absolute Error distribution per client for each setting.}
  \label{boxplots}
\end{subfigure}
\caption{Demonstration results.}
\label{results}
\end{figure}
\end{comment}

\section{Demonstration and Results}
\label{sec:poll_rec}

We used a dataset of 11{,}606 ratings from 8{,}982 users on 2{,}594 POIs in Bari, with each rating ranging from 1 to 5. Due to limited sensor coverage, we simulated eight virtual sensors in a $1\times1$\,km area around each user, assigning random AQI values (20--70). Two users in Aldo Moro Square were considered: 
\begin{itemize}
    \item \textbf{User 1:} A generally healthy individual. 
    \item \textbf{User 2:} An elderly person with higher sensitivity to air quality.
\end{itemize}
Each user’s preferences were locally collected to compute embeddings for personalized recommendations.

\subsubsection*{Intra-User Demonstration}
Figure~\ref{results} (red-boxed lists) shows how varying $\alpha$ shifts the recommendation focus:
\begin{itemize}
    \item $\alpha=0$ (AQI-driven): Only air quality matters.
    \item $\alpha=1$ (preference-driven): Only personal preferences matter.
    \item $\alpha=0.5$ (balanced): A mix of preferences and AQI.
\end{itemize}

\subsubsection*{Inter-User Demonstration}
Figure~\ref{results} (blue-boxed list) shows that User 2, with $\alpha=0.3$, is more cautious about AQI, receiving different recommendations than User 1. This illustrates \texttt{AirTOWN}'s flexibility in addressing individual health needs.

\subsubsection*{Federated Learning Demonstration}
A baseline matrix-factorization model was first trained on the entire dataset except for three users with the highest number of ratings. Those users were then involved in a second training step under three scenarios: centralized, distributed, and federated. The boxplots in Figure~\ref{results} confirm that the federated model consistently achieves the lowest median absolute error per user, highlighting its effectiveness for personalized and privacy-preserving learning.


\begin{comment}
    

\section{Demonstration and Results}
\label{sec:poll_rec}

\texttt{AirTOWN} can be evaluated depending on different aspects. In this work, we provide intra-user and inter-user studies to show the effectiveness of \texttt{AirTOWN} in providing personalized, health-conscious recommendations. Furthermore, we proved the feasibility of the FL application. A dataset of 11 606 ratings by 8 982 users about 2594 POIs in Bari was used in all cases. \textcolor{black}{Each rating is between 1 and 5.}


The intra-user and inter-user demonstrations were set in Bari, Italy, using synthetic AQI data due to limited sensor coverage. Two users were simulated in Aldo Moro Square, Bari, each using \texttt{AirTOWN} on mobile devices. User preferences were collected to compute individual embeddings through local models, allowing for personalized recommendations. 
\textcolor{black}{A grid of 8 virtual sensors in a $1 \times 1$ km layout surrounding each user was simulated. Random AQI values within a range of 20 to 70 were assigned to sensors, enabling realistic air quality variation.} Both users requested restaurant recommendations, with User 1 representing a generally healthy individual and User 2 representing an elderly individual with increased sensitivity to air quality.

For the FL investigation, a baseline model was trained on the whole dataset except for three users \textcolor{black}{with the highest number of ratings}. The selected user data was involved in a second step of learning in a centralized scenario, a distributed setting, and a federated one. \textcolor{black} {Differently from the federated scenario, in the distributed setting updates are not shared with the server, and each client has a version of the model tuned only on its data. In all scenarios, the MF model was trained to predict ratings of items by users.} Then, the performance of the trained models was compared through the evaluation of Absolute Error on the \textcolor{black}{predicted rating.}

\textbf{Intra-user demonstration.} As illustrated by the red-boxed results in Figure~\ref{results}, varying $\alpha$ shows different recommendation lists:
\begin{itemize}
    \item \textbf{$\alpha = 0$ (AQI-driven):} Recommendations are ordered solely by AQI, disregarding preferences.
    \item \textbf{$\alpha = 1$ (preference-driven):} Recommendations prioritize user preferences, sometimes listing POIs with higher AQI.
    \item \textbf{$\alpha = 0.5$ (balanced):} Recommendations are adjusted to reflect both AQI levels and personal tastes, achieving a compromise between health considerations and preferences.
\end{itemize}

\textbf{Inter-user demonstration.} User 2, representing an elderly individual, applied $\alpha = 0.3$ to favor lower AQI levels due to increased health sensitivity, as shown in the blue-boxed list in Figure~\ref{results}. This setting provided a recommendation list distinct from that of User 1, who prioritized preferences over AQI, demonstrating \texttt{AirTOWN}'s adaptability to individual health needs.

\textbf{Federated Learning demonstration.}
The boxplots in Figure~\ref{results} show how the performance of the federated model per client (user) is always better than the other models, with an error that is always the lowest in terms of the median.



%The boxplots in Figure~\ref{results} present Absolute Error values (i.e., the absolute difference between each predicted rating and the actual rating, aiming to quantify prediction accuracy). For example, while the centralized model may show median Absolute Errors above 1.0 for certain users, and the distributed model’s median errors typically hover around 0.5–0.7, the federated model consistently achieves median Absolute Errors below approximately 0.3. These lower Absolute Errors indicate that the federated approach yields more accurate personalized predictions, thus confirming its effectiveness in improving recommendation performance for individual clients.



\end{comment}


%\subsection{Conclusion}
%These demonstrations validate \texttt{AirTOWN}'s capability to balance personal preferences and air quality data for health-conscious, tailored POI recommendations. Although limited sensor coverage necessitated synthetic data, results indicate the system's potential in urban navigation with a focus on user well-being. Future enhancements will focus on expanding real-time sensor networks and refining AQI interpolation for improved accuracy in diverse urban settings.


%\textbf{Model and Setting Preparation.} For the demonstration, two users were considered. At the moment of the experimentation, each user was located in Aldo Moro Square, Bari, and had \airtown~installed on his own mobile. After collecting their preferences, each local model computed its user embedding and the recommendation system was ready. The user was located at the center of a virtual distribution of eight sensors, in a $1 \times 1$ km grid, and each sensor was initialized with a random AQI value $\in [20,70]$. Each user asked for a restaurant recommendation. We assume that the first user is a normal healthy user, while the second one is an elderly user.



%\textbf{Intra-user demonstration.} As the red box in Figure \ref{simulations} shows, the order of the first five suggested POIs changes according to $\alpha$ among the screenshots (The interface was slightly modified to enhance readability). When $\alpha = 0$, the recommendation is completely AQI-driven; indeed, the suggestions are sorted in descending AQI order. When $\alpha = 1$, the recommendation only rely on user preferences, and POIs with higher AQI can appear in higher positions, like "Al Sorso Preferito" (AQI = 41.59). When $\alpha = 0.5$, the same importance is given to AQI and user preferences, leading to meeting point of the previous ones.

%\textbf{Inter-user demonstration.} An elderly user should pay more attention to the air quality of the POIs, choosing $\alpha = 0.3$ for the suggestions. The first five suggestions for the second user are displayed in the blu box in Figure \ref{simulations}. His sequence of POIs differs from the respective one of the healthy user, because of the preferences differences among them.


\begin{figure}[H]
    \centering
    \includegraphics[width=0.95\linewidth]{airtown_results.png}
    \caption{Demonstration results for AirTown.}
    \label{results}
\end{figure}





\section{Conclusion}

%In conclusion, this work presented a two-part system combining the AirSENCE prediction engine and the AirTOWN recommendation engine to deliver personalized, health-oriented POI suggestions informed by real-time and forecasted air quality data. Together, these components enable a unique approach to urban navigation, marrying user preferences with environmental considerations for more health-conscious decision-making. By leveraging federated learning, the system aims to maintain user privacy while refining recommendation accuracy and adaptability.

%However, fully evaluating the system’s effectiveness remains a challenge. While our preliminary results offer initial insights into its usability and potential benefits, further research and real-world deployments are necessary to confirm its scalability and reliability. As sensor networks expand and predictive models improve, we anticipate more comprehensive studies to better understand the true impact and guide future enhancements.

We presented \texttt{AirSense-R}, a mobile application that combines personalization, real-time pollution awareness, privacy, and proximity considerations to deliver health-conscious point-of-interest (POI) recommendations. By integrating collaborative filtering with federated learning, \texttt{AirSense-R}~provides users with tailored suggestions while protecting the privacy of personal data. Initial experiments highlight the effectiveness of the Mobile App. in balancing user preferences with real-time air quality data, making it a valuable tool for urban navigation that promotes healthier choices. 

\textcolor{black}{\texttt{AirSense-R} reserves room for improvement. Currently, the recommendation engine employs only AQI as a source of context for pollution, but data such as temperature or humidity could enhance the usage of the system.
%e.g. for all users sensible to high temperatures.
Furthermore, the prediction engine could be used to predict the best time in the day to reach a suggested POI, improving \texttt{AirTOWNS}'s applicability. Further investigation on the performance of the application in realistic conditions will be held, alongside user studies to gauge users' perception of the application.}



This paper positions \textcolor{black}{\texttt{AirSense-R}} within the broader landscape of trustworthy recommender systems \cite{deldjoo2022survey2,deldjoo2024understanding,deldjoo2025cfairllm,nazary2025poison}, highlighting its emphasis on privacy preservation, transparency, and health-conscious decision-making \cite{10.1145/2792838.2796554,deldjoo2018content}. Additionally, we aim to incorporate insights from recent advancements in generative recommender models \cite{biancofiore2024interactive,deldjoo2024review,deldjoo2024recommendation}, which underscore the potential for dynamic, user-adaptive recommendations.


\subsubsection*{Acknowledgments}
This work was partially supported by the following projects: PASSEPARTOUT, LUTECH DIGITALE 4.0, P+ARTS, 2022LKJWHC - TRex-SE: Trustworthy Recommenders for Software Engineers, 2022ZLL7MW - Conversational Agents: Mastering, Evaluating, Optimizing.

% Looking ahead, the AirSENCE prediction engine will help overcome the challenges of having limited sensor coverage by spatially interpolating in regions with sparse environmental monitoring. Additionally, the models will not only interpolate air quality across areas but can also detect early signs of potential environmental events based on pollutant readings and patterns. By leveraging both spatial and temporal patterns, this approach will improve \texttt{AirTOWNS}'s applicability and enhance recommendation accuracy. 

%While current limitations, such as limited sensor coverage and reliance on simpler interpolation techniques, offer areas for improvement, future work will focus on expanding sensor networks and enhancing recommendation accuracy. Ultimately, \airtown~demonstrates a promising step toward urban health-focused applications that support user well-being and environmental awareness.
%\airtown is a user-friendly mobile application which combines personalization, pollution levels, distance, and privacy to provide effective and health-conscious POI recommendations. \airtown allows the user to explore new nearby places both involving personal preferences and real-time sensors data. 

%\section{Acknowledgements}

%This research was funded by the European Union’s Horizon 2020 RIA project "Photonic Accurate and %Portable Sensor Systems Exploiting Photoacoustic and Photo-Thermal Based Spectroscopy for Real-%Time Outdoor Air Pollution Monitoring" (PASSEPARTOUT, grant No. 101016956).

% LIMITATIONS:
%- Simulated data
%- Airsence small coverage
%- A better model than RBF interpolation can be adopted 
%- A better re-rank function can be considered

% PROS: 
%- MF more thrustworthy than more complex methods?
%- MF lighter than ANN, better for smartphone computational power
%- MF more scalable


\bibliographystyle{ACM-Reference-Format}
\balance
\bibliography{main}


\end{document}
\endinput
%%
%% End of file `sample-sigconf.tex'.
