\section{Introduction}  \label{sec:intro}
% 未来的智能机器人需要在人类复杂自然语言指令下在动态环境中执行各种长期复杂任务。这就要求智能体拥有动态环境感知与理解的能力,过去的工作可以从传感器数据中构建开放词汇的静态场景图表征某一时刻的环境拓扑结构,利用其增强智能体对复杂指令的理解能力并促进任务的完成。但是静态的场景图在真实世界中的实用性很差,因为机器人的工作环境布局通常会因为人类或其他智能体的活动发生改变,由于其内存中指导任务规划与执行的场景图与真实环境存在延迟,机器人实际上就很难成功完成相应的任务。

Future intelligent robots are supposed to execute diverse long-term complex tasks in dynamic environments based on intricate natural language instructions from humans. To achieve this, agents must possess dynamic environment perception and comprehension capabilities. Prior studies \cite{gu2023conceptgraphsopenvocabulary3dscene, werby23hovsg, yan2025dynamicopenvocabulary3dscene, linok2024barequeriesopenvocabularyobject} construct static open-vocabulary scene graphs from sensor data to capture the topological structures of environments at specific moments, which enhances robots' understanding of complex instructions and facilitates task completion. However, static scene graphs are of limited use in real-world scenarios, as robot workspaces typically change due to human activities or other agents' operations. Additionally, the inherent latency between the scene graph stored in memory (which guides task planning and execution) and the actual environment state significantly hinders successful task completion.

% 最近在可微分渲染领域取得显著进步的3D高斯被机器人社区广泛关注,并将其应用在高质量重建,机器人操作,语义嵌入,环境理解等领域中。我们发现3D 高斯的显式基于点的表征方式能够有效的支持我们将先进视觉基础模型的语义信息嵌入其中来构建场景的拓扑表征,并且我们能够基于其快速可微分渲染的能力来探究对进行重建后场景快速更新的方法。同时,3D 高斯带来的高质量环境重建,天然使得我们可以构建一个动态的高质量场景图构建系统
Recent advancements in 3D Gaussian splatting~\cite{kerbl20233d, Huang2DGS2024,yu2024mip,cheng2024gaussianpro} have attracted significant attention within the robotics community. These developments have applications in various domains, including high-quality reconstruction~\cite{kerbl20233d, Huang2DGS2024, keetha2024splatam, yan2024gs} robotic manipulation~\cite{zheng2024gaussiangrasper3dlanguagegaussian}, semantic embedding~\cite{ye2024gaussiangroupingsegmentedit, Li_2024}, and 3D environment understanding~\cite{qin2024langsplat3dlanguagegaussian,wu2024opengaussianpointlevel3dgaussianbased}. The explicit point-based representation of 3D Gaussians effectively facilitates semantic information integration from advanced vision language models, enabling the construction of topological scene representations. Based on fast differentiable rendering and explicit representation, we find 3D Gaussians are particularly suitable for locally rapid updates of reconstructed scenes. Moreover, the high-fidelity environment reconstruction provided by 3D Gaussians naturally supports the development of dynamic, high-quality scene graph construction systems.

\begin{figure}
  \centering
  \includegraphics[width=1.0\linewidth]{figures/1_introduction/1.pdf}
  \caption{The dynamic high-fidelity multi-layer Gaussian scene graphs we constructed can adapt to environment changes, represent the spatial and semantic relationships of the objects, and support various forms of language-guided object retrieval.}
  \label{fig:teaser}
\end{figure}

Most previous works on scene graph construction primarily rely on point clouds~\cite{gu2023conceptgraphsopenvocabulary3dscene, werby23hovsg, linok2024barequeriesopenvocabularyobject, hughes2022hydra, hughes2024foundations} These methods often struggle to promptly respond to dynamic environment changes and fail to capture fine-grained details of 3D scenes due to the inherent limitations of traditional representation techniques. 

% 我们使用先进的视觉模型来检测 RGB-D 序列中的实例物体并提取他们的语义特征。然后借助 GPT-4o 进行多层次长镜头的构建
In this paper, we propose \textbf{DynamicGSG}, a framework that utilizes 3D Gaussian Splatting to construct dynamic, high-fidelity, open-vocabulary scene graphs. Advanced vision models such as Yolo-World \cite{cheng2024yoloworldrealtimeopenvocabularyobject}, Segment Anything \cite{kirillov2023segment}, CLIP \cite{radford2021learningtransferablevisualmodels} are employed to detect objects and extract their semantic features. Subsequently, we analyze the spatial and semantic relationships of objects to build hierarchical scene graphs using Large Vision Language Model (LVLM). Through incorporating additional semantic supervision, we improve the accuracy of instance-level Gaussian grouping and the overall reconstruction quality. Moreover, the rapid training and differentiable rendering of 3D Gaussians facilitate efficient scene updates to accommodate environment changes.  

In summary, our contributions are as follows:
\begin{itemize}
    \item We propose \textbf{Dynamic} 3D \textbf{G}aussian \textbf{S}cene \textbf{G}raphs, combining instance-level rendering with VLM semantic information to achieve 3D-2D object association and building multi-layer scene graphs with LVLM.
    \item We design a joint loss function that ensures accurate intra-instance Gaussian grouping and high-fidelity scene reconstruction.
    \item We utilize the fast differentiable rendering of Gaussians to update the scene graphs, enabling our system to adapt to dynamic environment changes.
    \item We deployed DynamicGSG in real-lab environments, demonstrating its capability to construct 3D Gaussian scene graphs and perform dynamic updates for effective environment adaptation.
\end{itemize}

