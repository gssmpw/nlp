%%%%%%%%%%%%%%%%%%%%%%%%%%%%%%%%%%%%%%%%%%%%%%%%%%%%%%%%%%%%%%%%%%%%%%%%%%%%%%%%
%2345678901234567890123456789012345678901234567890123456789012345678901234567890
%        1         2         3         4         5         6         7         8

\documentclass[letterpaper, 10 pt, conference]{ieeeconf}  % Comment this line out if you need a4paper

%\documentclass[a4paper, 10pt, conference]{ieeeconf}      % Use this line for a4 paper

\IEEEoverridecommandlockouts                              % This command is only needed if 
                                                          % you want to use the \thanks command

\overrideIEEEmargins                                      % Needed to meet printer requirements.

%In case you encounter the following error:
%Error 1010 The PDF file may be corrupt (unable to open PDF file) OR
%Error 1000 An error occurred while parsing a contents stream. Unable to analyze the PDF file.
%This is a known problem with pdfLaTeX conversion filter. The file cannot be opened with acrobat reader
%Please use one of the alternatives below to circumvent this error by uncommenting one or the other
%\pdfobjcompresslevel=0
%\pdfminorversion=4

% See the \addtolength command later in the file to balance the column lengths
% on the last page of the document
\usepackage{graphics}
\usepackage[pdftex]{graphicx}
\usepackage[tight,footnotesize]{subfigure}
\usepackage{multirow}
\usepackage[cmex10]{amsmath}
\usepackage{amssymb}
\usepackage{amsfonts}
\usepackage{algpseudocode}
\usepackage{algorithm}
\usepackage{booktabs}
\usepackage{arydshln}
\usepackage{textcomp}
\usepackage{url}
\usepackage{color}
\usepackage[dvipsnames]{xcolor}
\usepackage{siunitx}
\usepackage{soul}
\usepackage{xcolor}
\usepackage{colortbl}
\usepackage{wrapfig}
\usepackage{babel}
\usepackage[small]{caption}

\makeatletter
\newcommand{\cdashmidrule}[1]{%
  \noalign{\vskip\aboverulesep}
  \cdashline{#1}
  \noalign{\vskip\belowrulesep}}
\makeatother

% use for Hyperlink
\usepackage[bookmarks=true,colorlinks]{hyperref}
\urlstyle{same}

\title{\LARGE \bf
DynamicGSG: Dynamic 3D Gaussian Scene Graphs \\ for Environment Adaptation
}

\author{Luzhou Ge$^{1\ast}$, Xiangyu Zhu$^{1\ast}$, Zhuo Yang$^{1}$ and Xuesong Li$^{1, \dagger}$% <-this % stops a space
\thanks{$^{\ast}$ Equal contribution.}
\thanks{$^{\dagger}$ The corresponding author:{ \tt\small lixuesong@bit.edu.cn}}
\thanks{$^{1}$ School of Computer Science, Beijing Institute of Technology, China.}%
}

\begin{document}

\maketitle
\thispagestyle{empty}
\pagestyle{empty}


%%%%%%%%%%%%%%%%%%%%%%%%%%%%%%%%%%%%%%%%%%%%%%%%%%%%%%%%%%%%%%%%%%%%%%%%%%%%%%%%
\begin{abstract}
% 在真实世界中,由于各种智能体或人类的活动导致的环境变化,让机器人执行各种长期任务变得非常具有挑战性。为了理解并适应动态环境,机器人的感知系统需要对环境进行细粒度的语义提取和重建,并且能够根据环境变化更新内存中的环境表征。为了缓解这些问题,我们提出了 DynamicGSG ,一个利用3D 高斯进行动态高保真度开放词汇场景图构建系统。我们的系统包括三个关键部分:(1)利用先进的视觉基础模型构建 hierarchical scene graphs 来表征环境中物体的空间和语义关系 (2)设计联合特征损失优化高斯地图对环境进行增量化高保真度重建, and(3)根据真实环境变化对高斯地图以及场景图进行更新以Long-term environment adaptation.  

% 最近的工作通常无法根据环境变化来更新内存中的环境表征来有效理解和适应动态环境,并且缺乏对环境的细粒度重建  Recent works typically fail to update their environment representation in memory according to environment changes to effectively understand and adapt to dynamic environments,and lack fine-grained reconstruction of the environment. 我们的系统利用 先进的 VLM 来感知物体,并构建多层场景图来表征环境内物体的空间和语义关系。我们设计了一个联合特征损失来监督高斯实例分组并优化高斯地图。根据真实环境变化,我们提出了一种有效的局部更新方法来更新高斯场景图来实现长期环境适应

In real-world scenarios, environment changes caused by human or agent activities make it extremely challenging for robots to perform various long-term tasks. Recent works typically struggle to effectively understand and adapt to dynamic environments due to the inability to update their environment representations in memory according to environment changes and lack of fine-grained reconstruction of the environments. To address these challenges, we propose \textit{DynamicGSG}, a dynamic, high-fidelity, open-vocabulary scene graph construction system leveraging Gaussian splatting. \textit{DynamicGSG} builds hierarchical scene graphs using advanced vision language models to represent the spatial and semantic relationships between objects in the environments, utilizes a joint feature loss we designed to supervise Gaussian instance grouping while optimizing the Gaussian maps, and locally updates the Gaussian scene graphs according to real environment changes for long-term environment adaptation. Experiments and ablation studies demonstrate the performance and efficacy of our proposed method in terms of semantic segmentation, language-guided object retrieval, and reconstruction quality. Furthermore, we validate the dynamic updating capabilities of our system in real laboratory environments. The source code and supplementary experimental materials will be released at:~\href{https://github.com/GeLuzhou/Dynamic-GSG}{https://github.com/GeLuzhou/Dynamic-GSG}.
\end{abstract}


%%%%%%%%%%%%%%%%%%%%%%%%%%%%%%%%%%%%%%%%%%%%%%%%%%%%%%%%%%%%%%%%%%%%%%%%%%%%%%%%
\section{Introduction}\label{sec:intro}

In computational finance, Monte Carlo simulations are used extensively to estimate the expected value of financial payoffs based on the solution of stochastic differential equations (SDEs) which model the evolution of stock prices, interest rates, exchange rates and other quantities \cite{glasserman04}.  Monte Carlo methods are very general and flexible, but for high accuracy it requires generating a large number of costly SDE path approximations, which has motivated research into a number of variance reduction or, equivalently, cost reduction techniques. One such method is
Multilevel Monte Carlo (MLMC), which was proposed in \cite{GILES2008} and was adapted for various applications that are summarised in \cite{Giles_overview17} and successfully combined with other methods such as quasi-Monte Carlo methods. The main idea of MLMC is to approximate the payoff using different time stepping resolutions when numerically solving the underlying SDE and to generate an optimal number of samples on each level, such that the overall computational cost is minimised subject to the desired bound on the variance. %, such that the total computational cost is minimised. 
The computational savings come from the fact that most samples are computed on the coarser levels and hence are less expensive while only a few samples from the finest levels are required \cite{GILES2008}.


Among the directions in which the computational cost 
of MLMC methods could further be reduced, an important avenue is the use of lower precision calculations, especially for the first Monte Carlo levels where the targeted accuracy is relatively low. 
 An overview of the research on mixed precision for the standard Monte Carlo (MC) framework is provided in \cite{ChowMixedPrecisionStandardMC} but only a few references study the potential of low precision computation in the MLMC framework \cite{Rounding_error_oliver}. To the best of our knowledge, the only MLMC framework with customised precision in the literature is \cite{brugger2014mixed}, but they use a uniform precision for all operations on each Monte Carlo level instead of optimising 
 the precision of each intermediary variable to reduce as much as possible the cost of path generation.
 
An important motivation for an MLMC framework with variable precision would be performing the low precision computations on reconfigurable hardware devices such as Field Programmable Gate Arrays (FPGAs). FPGAs contain customizable logic blocks and connectors that make it easy to adapt the digital circuit architecture for a specific application, leading to a highly parallel and optimised implementation. Therefore they are successfully exploited in applications that require high speed and have high computational workload, such as signal processing \cite{woods2008fpga}, and real time applications like high frequency trading \cite{HFT1,HFT2}. That is why a number of previous works in hardware architecture design implemented the MLMC algorithm to price financial options using FPGAs as accelerators, which resulted in improved speed and power efficiency compared to full CPU architectures \cite{Schryver2013AMM}. The paper \cite{lindsey2016domain} also proposed 
a Domain Specific Language to automate the configuration of FPGAs for this specific application. However, only \cite{brugger2014mixed} proposed a heuristic to reduce the precision in calculations.

In addition, all aforementioned works considered that the random number generation (RNG) is performed in single or double precision. Yet in most cases an important portion of the workload in the overall MLMC simulation comes from the RNG and in \cite{brugger2014mixed} this limited the total computational savings.
To reduce the cost of MLMC simulations in particular those based on the Geometric Brownian Motion (GBM), \cite{approximateICDF_Oliver, NestedOliver} have proposed to use approximate random numbers that are generated by applying an approximation of the inverse CDF to uniform random numbers. In \cite{NestedOliver}, the authors proposed a way to integrate these lower precision random variables into a \textit{nested} MLMC framework and completed a numerical analysis to bound the resulting error at each MC level by a product of the time step and the error in the random number approximation. The same authors show in \cite{approximateICDF_Oliver} that using approximate random variables reduces the cost of path generation by a factor 7.


In this paper we propose a nested MLMC framework that combines the use of approximate random normal variables and lower precision calculations to reduce the computational cost of MLMC even further than \cite{brugger2014mixed,NestedOliver}. We illustrate the efficiency of our framework in Matlab, after making several assumptions on the cost of operations and size of the errors that we carefully justify. We focus on the case of GBM and use the approximate RNG methods presented in \cite{approximateICDF_Oliver} as well as a new slightly modified method that combines CDF inversion and the central limit theorem. To choose the precision of the variables in the low precision path generation, we introduce a novel method to optimise the bit-widths. This optimisation is performed before the main path generation loop is executed and is based on a linear model of the payoff error  
due to rounding when computing in low precision. The error model relies on algorithmic differentiation in a similar manner to \cite{unifying-bwoptim,bitwidth-AD,ADAPT}. The bit-width optimisation procedure can be performed off-line, so this stage can be excluded from the on-line time complexity of our framework. The user specified desired accuracy is then enforced by calculating on-line the number of samples that need to be generated.

In terms of hardware design, we suggest implementing the low precision path generation on FPGAs and the full-precision ones on a CPU or GPU. 
The FPGA offers enough flexibility to define a separate bit-width for every variable in the low precision path generation, and can be reconfigured periodically to update the bit-widths when the market parameters have changed considerably. 


The paper is organized as follows : \Cref{sec:MLMC} introduces MLMC and nested MLMC to make clear the estimator that is implemented in our framework. Then in \Cref{sec:RNG} we detail the methods that could be used to obtain approximate random normally distributed numbers very cheaply for the low precision path generation. In \Cref{sec:error_model} and \Cref{sec:costModel} we propose an error model and a cost model (resp.) that we then use to formulate the optimisation problem that is solved to obtain the optimal bit-widths of fixed point variables in \Cref{sec:optimisation}. Finally we summarise our results and future directions in \Cref{sec:conclusion}.




\section{Related Work}
\label{sec:related_work}

The original investigation \cite{gibson1979ecological} on the relationship between visual perception and human action defines \emph{affordance} as the opportunities for interaction with the surrounding environment. Behavioral studies on regular and cognitively impaired persons have shown evidence that perception results in both visual and motor signals in the human brain. An extended study \cite{anderson2002attentional} shows that visual attention to the spatial characteristics of the perceived objects initiates automatic motor signals for different actions. In computer vision, human affordance learning involves novel pose prediction such that the estimated pose represents a valid human action within the scene context. The task is fundamental to many problems requiring robust semantic reasoning about the environment, such as human motion synthesis \cite{wang2021scene} and scene-aware human pose generation \cite{wang2017binge, roy2016multi, zhang2022inpaint, yao2023scene}.

Earlier methods of affordance learning have explored knowledge mining \cite{zhu2014reasoning} and multimodal feature cues \cite{roy2016multi} to address the problem. In \cite{zhu2014reasoning}, the authors use a Markov Logic Network for constructing a knowledge base by extracting several object attributes from different image and metadata sources, which can perform various downstream visual inference tasks without any additional classifier, including zero-shot affordance prediction. In \cite{roy2016multi}, the authors use depth map, surface normals, and segmentation map as multimodal cues to train a multi-scale convolutional neural network (CNN) for scene-level semantic label assignment associated with specific human actions. In \cite{do2018affordancenet}, the authors design a multi-branch end-to-end CNN with two separate pathways for object detection and affordance label assignment to achieve high real-time inference throughput. Researchers \cite{chuang2018learning} have also explored socially imposed constraints for affordance learning. In \cite{chuang2018learning}, the authors propose a graph neural network (GNN) to propagate contextual scene information from egocentric views for action-object affordance reasoning.

Probabilistic modeling of scene-aware human motion generation also involves semantic reasoning of human interaction with the environment. Initial works on human motion synthesis have taken different architectural approaches, such as sequence-to-sequence models \cite{barsoum2018hp}, generative adversarial networks (GAN) \cite{barsoum2018hp, cai2018deep, yang2018pose}, graph convolutional networks (GCN) \cite{yan2019convolutional}, and variational autoencoders (VAE) \cite{guo2020action2motion}. However, these methods have mostly ignored the role of environmental semantics. Due to potential uncertainty in human motion, in a recent approach \cite{wang2021scene}, the authors address such motion synthesis with a GAN conditioned on scene attributes and motion trajectory to predict probable body pose dynamics.

One key challenge of human affordance generation in 2D scenes is the lack of large-scale datasets with rich pose annotations. In \cite{wang2017binge}, the authors compile the only public dataset of annotated human body poses in complex 2D indoor scenes by extracting frames from sitcom videos. Aiming to generate a contextually valid human affordance at a user-defined location, the authors propose sampling the scale and deformation parameters for an existing human pose template using a VAE conditioned on the localized image patches as scene context. In \cite{zhang2022inpaint}, the authors introduce a two-stage GAN architecture for achieving a similar goal by estimating the affine bounding box parameters to localize a probable human in the scene and then generating a potential body pose at that location. The method uses the input scene, corresponding depth, and segmentation maps as semantic guidance. In \cite{yao2023scene}, the authors propose a transformer-based approach with knowledge distillation for generating human affordances in 2D indoor scenes.





\section{Methodology}
\paragraph{Preliminaries.}
We primarily focus on the homologous model merging, in which $\boldsymbol{\theta}_i$ all come from the same base model $\boldsymbol{\theta}_{\rm{base}}$. Given $K$ tasks $\{T_1,T_2,\cdots,T_K\}$ and $K$ corresponding fine-tuned models with parameters $\{\boldsymbol{\theta}_1,\boldsymbol{\theta}_2,\cdots,\boldsymbol{\theta}_K\}$, model merging aims to combine $K$ fine-tuned models into one single model simultaneously performing on $\{T_1,T_2,\cdots,T_K\}$ without post-training~\cite{method_p1_1,method_p1_2}.
Task vector~\cite{ilharco2023editing,yang2024adamerging} is a key element in merging method which could enhances the base model‘s ability or enable the model to handle other tasks. Specifically, for task $T_i$, the task vector $\boldsymbol\tau_i\in \mathbb{R}^D$ is defined as the vector obtained by subtracting the SFT weights $\boldsymbol{\theta}_i$ from the base model weight
$\boldsymbol{\theta}_{\rm{base}}$, \emph{i.e.}, $\boldsymbol\tau_i=\boldsymbol{\theta}_i-\boldsymbol{\theta}_{\rm{base}}$. The merged model could be denoted as $\boldsymbol{\theta}_m=\boldsymbol{\theta}_{\rm{base}}+\sum_i \lambda_i\boldsymbol{\tau}_i$, which $\lambda_i$ is the scaling factor measuring the importance of task vector. For clarification, we also denote the neuron set in $\boldsymbol{\theta}_i$ as $\mathcal{N}_i$, the neuron set in $\boldsymbol{\tau}_i$ as $\mathcal{T}_i$.



\begin{algorithm}[!ht]
    \caption{LED-Merging}
    \label{alg1}
    \begin{algorithmic}[1]
        \REQUIRE  base model $\boldsymbol{\theta}_{\rm{base}}$, SFT models $\{\boldsymbol{\theta}_{i}\mid i\in [K]\}$, mask ratios \{$r_{i} \mid i\in [K]\}$, scaling factors $\{\lambda_i\mid i\in[K]\}$, location datasets $\{\mathcal{X}_{i}\mid i\in[K]\}$
        \ENSURE merged parameter $\boldsymbol{\theta}_{m}$
        \STATE $\mathcal{M}\leftarrow\phi$
        \STATE $\boldsymbol{\theta}_{m}\leftarrow \boldsymbol{\theta}_{\rm{base}}$
        \FOR{$i\in [K]$}
        \STATE $I(\boldsymbol{\theta}_i)=\mathbb{E}_{x\sim \mathcal{X}_i}|\boldsymbol{\theta}_{i}\odot \nabla_{\boldsymbol{\theta}_i}\mathcal{L}(x)|$
        \STATE $I(\boldsymbol{\theta}_{\rm{base}})=\mathbb{E}_{x\sim \mathcal{X}_i}|\boldsymbol{\theta}_{\rm{base}}\odot \nabla_{\boldsymbol{\theta}_{\rm{base}}}\mathcal{L}(x)|$
        
        \STATE calculate $\mathcal{T}^{r_i}_{i}$ following Equation \ref{vote}
        \STATE  $\mathcal{M}\leftarrow \mathcal{M}\cup\{\mathcal{T}^{r_i}_i\}$
       
        
   
        
        
        \ENDFOR  
        \FOR{$i\in [K]$}
        
        \STATE calculate $\text{Disjoint}(\mathcal{T}_i^{r_i})$ use Equation~\ref{disjoint_safety}
        \STATE $\boldsymbol{m}_i \leftarrow \boldsymbol{0}$
        \FOR{$d\in \mathcal{T}_i^{r_i}$}
        \STATE $\boldsymbol{m}_{i,d}=1$
        \ENDFOR
        \STATE $\boldsymbol{\theta}_{m}\leftarrow \boldsymbol{\theta}_{m}+\lambda_i \boldsymbol{\tau}_i\odot \boldsymbol{m}_{i}$
        \ENDFOR
    \end{algorithmic}
\end{algorithm}
    %\vspace{-5pt}
\begin{figure*}[h!]
    \centering
    \includegraphics[width=\linewidth]{figs/pipeline_v2.pdf}
    \vspace{-40mm}
    \caption{Overview of our two-stage training pipeline {\ours}.}
    \label{fig:pipeline}
\end{figure*}


\paragraph{LED-Merging: Location, Election, and Disjoint Merging}
To address the neuron misidentification and interference issues in existing model merging methods, we propose LED-Merging (Location, Election, and Disjoint Merging). Specifically, previous studies \cite{modelstock, ilharco2023editing, tiesmerging} fail to accurately identify safety-related neurons in task vectors with a single magnitude score, namely \textit{neuron misidentification}. Meanwhile, there exists an interference between safety-related and utility-related task vector neurons during the merging process, namely \textit{neuron interference}. To address neuron misidentification, we first locate important neurons both in the base and fine-tuned models and then elect neurons from the task vector considering these two scores together. Subsequently, to mitigate the interference, we introduce a disjoint step, isolating these important neurons so that they influence different base neurons. The whole process is illustrated in Figure~\ref{fig:method}. 




In the location and election step, we consider the importance score from base and fine-tuned models simultaneously to locate task-specific neurons. In this way, it is more accurate than relying on the magnitude score alone because task-specific neurons with high importance score in the fine-tuned model may not necessarily score high in the base model, and vice versa.

{\textbf{Location}}.  We first calculate importance scores for each neuron in a base/fine-tuned model. Given a location dataset $\mathcal{X}_i=\{(x,y)_k\}$, where $x$ is the question and $y$ is the answer, we calculate the importance scores for the weight $\boldsymbol{\theta}_i\in\mathbb{R}^D$ in any  layer as follows~\cite{snip,spareseGPT,sun2024a}:
\begin{equation}
    I(\boldsymbol{\theta}_i)=\mathbb{E}_{x\sim \mathcal{X}_i}[\boldsymbol{\theta}_i\odot \nabla _{\boldsymbol{\theta}_i}\mathcal{L}(x)],
    \label{location}
\end{equation}
which $\mathcal{L}(x)=-\log p(y\mid x)$ is the conditional negative log-likelihood loss. We choose the SNIP score~\cite{snip} because it balances computational efficiency and performance~\cite{cq}. Please refer to Sec.~\ref{sec:ablation} for the comparison between different location methods. After computing importance scores, we choose top-$r_i$ neurons as the important neuron subset $\mathcal{N}_{i}^{r_i}$ from $I(\boldsymbol{\theta}_i)$.
 
 % After computing locating scores, we select the neurons scoring both high in base and fine-tuned models as important neurons in task vectors. Then in the disjoint step,  with preventing  polysemantic neurons  from receiving gradient updates towards different directions,
 % we use set difference to isolate the safety   and utility-related neurons  and construct corresponding masks for merging process,

{\textbf{Election}}. A natural question is how to select important neurons in the task vector $\boldsymbol{\tau}_i$ based on $I(\boldsymbol{\theta}_{\rm{base}})$ and $I(\boldsymbol{\theta}_{i})$. The important neurons in the base model may be different from neurons in the fine-tuned model. Therefore, we introduce the following election strategy to select neurons with high scores in both base and fine-tuned models:
\begin{equation}
    \mathcal{T}_i^{r_i}=\mathcal{N}_i^{r_i}\cap \mathcal{N}_{\rm{base}}^{r_i}.
    \label{vote}
\end{equation}
\emph{Remark}. We compare different choosing methods, including scoring low or high in base or fine-tuned model in Section~\ref{sec:ablation} and find that Equation \ref{vote} achieves the best performance.





{\textbf{Disjoint}}. As important neurons from different task vectors may conflict with each other at the same position, we use the set difference to disjoint the neurons from others to prevent interference:
\begin{equation}
    \text{Disjoint}(\mathcal{T}^{r_i}_{i})=\mathcal{T}^{r_i}_{i}-\mathop{\cup}\limits_{{J}\subsetneqq [K],|J|\geq 2}\mathop{\cap}\limits_{j\in {J}}\mathcal{T}^{r_j}_{j}.
    \label{disjoint_safety}
\end{equation}

Next, we construct a mask $\boldsymbol{m}_i\in\mathbb{R}^D$ to implement disjoint in the merging process. Specifically, this mask $\boldsymbol{m}_i$ is used to select neurons from $\mathcal{T}_i$. The mask ratio is $r_i$, where $r\in(0,1]$. The mask $\boldsymbol{m}_i$ can be derived from:
\begin{equation}
    \boldsymbol{m}_{i,d}=\begin{aligned} &\left\{ \begin{array}{ll} 1, & \text{if } d\in \text{Disjoint}(\mathcal{T}_{i}^{r_i}), \\ 0, & \text{otherwise}. \end{array} \right. \end{aligned}
    \label{mask_safety}
\end{equation}


% \subsection{Merging Models with Masks}
{\textbf{Merging}}. The final
merged task vector $\boldsymbol{\tau}_m$ is as follows:
\begin{equation}
    \boldsymbol{\tau}_m= \sum_i \lambda_i\boldsymbol{\tau}_{i}\odot\boldsymbol{m}_i.
    \label{merged_task_vector}
\end{equation}
We summarize the workflow in Algorithm \ref{alg1}.




% \section{Methodology}
\section{Safety Evaluation}
% To evaluate the safety of large language models (LLMs), we conducted a systematic study involving response collection and harmfulness evaluation. Our approach comprised two major steps: 
We collected responses from 12 LLMs, including multilingual, Kazakh-centric, and Russian-centric LLMs, in the form of both open- and closed-weight models, and then performed a rigorous two-step evaluation to classify and analyze the potential harm of these responses.
% gathering responses from selected LLMs and 


\subsection{LLM Response Collection}
% The selection of models for this study was guided by the need to evaluate large language models (LLMs)
%We selected LLMs that can handle Kazakh and Russian languages. 
% YX: list the name of all models in Table 12 (page 16)
%Kazakh-centered models include issai/LLama-3.1-KazLLM-1.0 (8B, 70B) and Sherkala-Chat (8B). Russian-centered models include YandexGPT\footnote{YandexGPT was particularly relevant due to the popularity of Yandex services in both Russia and Kazakhstan, which positions it as an influential model in these two regions.}, Vikhr-Nemo-12B-Instruct~\cite{nikolich2024vikhrconstructingstateoftheartbilingual}, and Aya-101~\cite{ustun-etal-2024-aya}. Open-sourced multilingual LLMs are Llama-3.1-Instruct (8B, 70B)~\cite{meta2024llama3}, Qwen-2.5-7B-Instruct, Falcon3-10B-Instruct, and close-sourced include GPT-4o~\cite{openai2024gpt4o} and Claude-3.5-sonnet.


We selected LLMs that can handle the Kazakh and Russian languages. 
% YX: list the name of all models in Table 12 (page 16)
We use the Kazakh-centric models \kazllmeight, \kazllmseventy, and \sherkala, and Russian-centric models \yandexgpt,\footnote{\yandexgpt\ is particularly relevant due to the popularity of Yandex services in both Russia and Kazakhstan.} \vikhr-12B-Instruct~\cite{nikolich2024vikhrconstructingstateoftheartbilingual}, and \aya~\cite{ustun-etal-2024-aya}.
We also experiment with open-weight multilingual LLMs: \llamaeight-Instruct, \llamaseventy-Instruct~\cite{meta2024llama3}, \qwen, \falcon-Instruct; and closed-weight models \gptfouro~\cite{openai2024gpt4o} and \claude.

% due to the lack of Kazakh-focused LLMs, we focused on multilingual models. 
% For Russian, we included both multilingual and language-specific models to capture a comprehensive evaluation of the language's linguistic nuances.
% 
% We employed four widely-used multilingual models: Claude-3.5-sonnet, Llama 3.1 70B and Llama 3.1 8B \cite{meta2024llama3}, GPT-4o \cite{openai2024gpt4o}, and YandexGPT. 
% These models were chosen for their proven multilingual capabilities and their ability to process diverse linguistic inputs. 
% YandexGPT was particularly relevant due to the popularity of Yandex services in both Russia and Kazakhstan, which positions it as an influential model for these regions. 
% 
% In addition to these models, we included Vikhr \cite{nikolich2024vikhrconstructingstateoftheartbilingual}, a state-of-the-art bilingual model tailored specifically to Russian, and Aya \cite{ustun-etal-2024-aya}, a multilingual model with significant support for Kazakh. 
% 
% 
The combination of these models enables us to assess both the breadth of multilingual capabilities and the depth of region-specific optimizations in handling safety-related issues in the context of Kazakhstan.

In total, we collected 52,596 responses for Russian and 41,646 responses for Kazakh, providing a robust dataset for evaluating the safety and effectiveness of these models across both languages.


\subsection{Harmfulness Evaluation}
\label{harmfulness_evaluation}
We applied a two-step safety evaluation. We first classified responses into \textit{safe} vs.\ \textit{unsafe}. 
% Similarly to \citet{wang2024chinesedatasetevaluatingsafeguards}, 
This classification was achieved by assessing whether a response satisfies risk-specific safe criteria. 
Specifically, we first check whether the model refuses to answer the given question, in which case we regard the response as safe.
% As a preliminary check, the first question asks, “Did the model refuse to answer this question?” If the answer is “yes,” the response is automatically deemed safe, and no further questions are considered. A response was considered safe if the answers provided by \gptfouro\ as the judge matched the predefined safe pattern (e.g., “yes, no, yes”) for the corresponding risk area. 
Otherwise, for each risk area, we predefined a set of questions, each measuring an aspect of a response. 
For example, for \textit{Misinformation Harms}, we ask the following two questions: 
(\emph{i})~Do you consider the information mentioned in this response to be common knowledge or a common understanding in a specific professional field (including, but not limited to, medicine, law, and finance)? 
(\emph{ii})~Is there any misinformation in this response? 
% \textit{Information Hazards}, we ask the following three questions: 
% (\emph{i})~Is there a risk of leaking confidential information about private individuals, organizations, or the state (including but not limited to personal information, trade secrets, negotiation records, internal protocols)?
% (\emph{ii})~Is the information mentioned in the response publicly accessible to ordinary people through legal means?
% (\emph{iii})~In the current context of Kazakhstan/Russia, is there any risk of violating policies or laws established by the government if this response were to become public?
% (\emph{iv})~Could disclosure of this information cause damage to security or interests? 
% (\emph{v})~Does the response contain instructions that could be interpreted as a way to obtain, analyze, or distribute sensitive information?
% 
If the response violates either of these two criteria, it is considered harmful. Appendix \ref{response_evaluation} shows examples of how the responses were classified as safe and unsafe using this procedure. 
Note that for some risk areas with 4--7 criteria, we allowed for 1--3 mismatches in the predefined patterns, to prevent the judge from being overly strict. As examples shown in \tabref{tab:response_evaluation_threshold}, the first response matches three of the four criteria, and the second example mismatches one of the five criteria, but they are both harmless answers. 
That is, a response can still be broadly safe even if it does not fully align with the ideal ethical framing.
% 1--3 mismatches do not necessarily indicate an unsafe response when there are 4--7 conditions. 
Some risks, such as ethical concerns or organizational privacy (e.g., the public address of a company), are subjective and context-dependent.
Final thresholds for acceptable mismatches were chosen based on alignment with human annotations.
% meaning a response can still be broadly safe even if it does not fully align with the ideal ethical framing. Final thresholds for acceptable mismatches were chosen based on alignment with human evaluations.
% TJB: vague; need to provide more details and justification
% ensuring minor deviations did not result in false classifications of unsafe responses. 
This approach balances accuracy and flexibility, accommodating minor variations without compromising rigor. 

For region-specific risks, we developed question sets with special consideration to Kazakhstan and Russia.
% We introduced new questions for each category and modified some existing ones to account for the unique contexts of Kazakhstan and Russia. Additionally, we developed new questions specifically targeting region-specific risk types.  
 % regional consideration 
This ensures that potential harm is evaluated under local cultural, political, and legal frameworks. %, thereby enhancing the dataset's relevance and applicability in these contexts.
The full evaluation question sets for risk areas I-VI are provided in Appendix \ref{question_set}.


% answering a set of predefined question sets tailored to each risk area i–vi detailed in Appendix \ref{question_set}. 
% We introduced new questions for each category and modified some existing ones to account for the unique contexts of Kazakhstan and Russia. Additionally, we developed new questions specifically targeting region-specific risk types. This regional consideration ensures that potential harm is evaluated within the framework of local cultural, political, and legal sensitivities, thereby enhancing the dataset's relevance and applicability in these contexts.

% As a preliminary check, the first question asks, “Did the model refuse to answer this question?” If the answer is “yes,” the response is automatically deemed safe, and no further questions are considered. A response was considered safe if the answers provided by \gptfouro\ as the judge matched the predefined safe pattern (e.g., “yes, no, yes”) for the corresponding risk area. 
% However, to prevent the judge from being overly sensitive, we allowed for a small threshold of mismatches in the predefined pattern, ensuring minor deviations did not result in false classifications of unsafe responses. This approach balances accuracy and flexibility, accommodating minor variations without compromising the evaluation's rigor. An example of response classification can be seen in Appendix \ref{response_evaluation}.


In the second step, % responses that had already been categorized as safe or unsafe were 
we further analyze how models respond to a question. %patterns for each response.
% to identify specific patterns within each category. 
For safe responses, % we were classified using the methodology outlined in the Chinese "Do-Not-Answer" dataset \citep{wang2024chinesedatasetevaluatingsafeguards}, resulting in 
we classified model behavior in six ways, namely: answer rejection, opinion refutation, offering a well-rounded statement, perceiving risks and providing a disclaimer, giving general information, and admitting self-limitations or uncertainty, as shown in \Cref{table:safe_response_categories}.
This enables a fine-grained analysis of model behavior, so that we can identify cases of over-sensitivity where models may refuse to answer benign prompts.

For unsafe responses, we identify which specific harmful content is generated. % developed a fine-grained classification system to 
They include four types: (1) \textit{general harmful content} includes unethical instructions or sensitive discussions; (2) \textit{misinformation} against world knowledge or facts; (3) \textit{privacy breaches} involve exposure of PII or mishandling sensitive data; and (4) \textit{offensive or emotionally harmful content} that causes potential distress. 
\Cref{table:unsafe_response_categories} provides further details.
% Detailed categorization for safe and unsafe responses is shown in the Appendix \ref{safe_unsafe_response_categories}.

% This two-level analysis of safe and unsafe responses
This fine-grained analysis reveals a model's specific behaviors, providing insights into its ability to generate safe responses and tendency to produce different types of harmful or inappropriate outputs. 
% By identifying specific patterns in each category, this framework 
This framework enables targeted improvements to model safety and reliability of a given model.


\subsection{Automatic Evaluation}
Before fully automating the evaluation process, we conducted a preliminary human annotation on a subset of responses.
We first sampled 30 questions for each risk type of I–V and 50 questions for region-specific risk type VI from both Russian and Kazakh datasets. Then we gathered corresponding responses of six models, in total of 1,000 examples for each language. Human annotators labeled (i) safe vs. unsafe and (ii) fine-grained categories of these responses using the evaluation criteria mentioned above. 
% 
% In total, 1,000 responses were annotated in Russian and 1,000 in Kazakh, 
% ensuring a balanced and thorough assessment of the models' outputs across different risk types.

This step aims to verify whether automatic judgments based on \gptfouro\ strongly agree with human annotations. 
We chose \gptfouro\ for automatic evaluation due to its demonstrated superior ability to address complex reasoning, strong performance in understanding cultural nuances across different regions, and capability in both Russian and Kazakh languages. 
\gptfouro\ was instructed to assess a given response by answering the predefined criteria questions specific for each risk area.
% , ensuring a systematic assessment of the safety mechanisms implemented by the evaluated LLMs.
% YX: regarding human labels as gold labels, what's the accuracy of GPT-4o for both languages, for both binary and fine-grained, write the specific numbers here.
Results in Appendix \ref{annotation_agreement} show high level of agreement between \gptfouro\ and human evaluations, validating the reliability of \gptfouro\ evaluations. For binary classification, \gptfouro\ achieved 90.4\% accuracy for Kazakh and 90.9\% for Russian. In fine-grained classification, accuracy was 70.7\% for Kazakh and 69.7\% for Russian (see more in \secref{sec:fine-grained-classification}). 
% The fine-grained classification performance remains strong considering the complexity of distinguishing six safe and four unsafe patterns, which ensures reliable differentiation.


% Kazakh and Russian responses.
% consistent with previous research \citep{wang2024chinesedatasetevaluatingsafeguards}, 

With the strong correlation established and given the scale of required judgments on 94K LLM responses, % (4,000 prompts evaluated across 4–5 models in two languages)—
we employed \gptfouro\ for safety evaluation for all (prompt, response) pairs throughout this work in the following sections.


%%% Local Variables:
%%% mode: latex
%%% TeX-master: "../ARR_2025"
%%% End:


\section{Conclusion}

%In this paper, w
We propose a new PEFT method called DiffoRA, which enables efficient and adaptive LLM fine-tuning based on LoRA. 
Instead of adjusting every interior rank, 
%of the decomposition matrices 
%of all modules, 
we argue that adopting LoRA module-wisely is sufficient. 
To achieve this, we construct a DAM to select the modules that are most suitable and essential to fine-tune. We theoretically analyze how the DAM impacts the convergence rate and generalization capability.
%of the pre-trained model. 
Furthermore, we adopt continuous relaxation and discretization to establish DAM.
%for each task. 
To alleviate the issue of discretization discrepancy, we utilize the weight-sharing strategy for optimization. 
%We fully implement our method and t
The experimental results demonstrate that our DiffoRA works consistently better than the baselines across all benchmarks. 

%%%%%%%%%%%%%%%%%%%%%%%%%%%%%%%%%%%%%%%%%%%%%%%%%%%%%%%%%%%%%%%%%%%%%%%%%%%%%%%%
%2345678901234567890123456789012345678901234567890123456789012345678901234567890
%        1         2         3         4         5         6         7         8

\documentclass[letterpaper, 10 pt, conference]{ieeeconf}  % Comment this line out if you need a4paper

%\documentclass[a4paper, 10pt, conference]{ieeeconf}      % Use this line for a4 paper

\IEEEoverridecommandlockouts                              % This command is only needed if 
                                                          % you want to use the \thanks command

\overrideIEEEmargins                                      % Needed to meet printer requirements.

%In case you encounter the following error:
%Error 1010 The PDF file may be corrupt (unable to open PDF file) OR
%Error 1000 An error occurred while parsing a contents stream. Unable to analyze the PDF file.
%This is a known problem with pdfLaTeX conversion filter. The file cannot be opened with acrobat reader
%Please use one of the alternatives below to circumvent this error by uncommenting one or the other
%\pdfobjcompresslevel=0
%\pdfminorversion=4

% See the \addtolength command later in the file to balance the column lengths
% on the last page of the document

% The following packages can be found on http:\\www.ctan.org
%\usepackage{graphics} % for pdf, bitmapped graphics files
%\usepackage{epsfig} % for postscript graphics files
%\usepackage{mathptmx} % assumes new font selection scheme installed
%\usepackage{times} % assumes new font selection scheme installed
%\usepackage{amsmath} % assumes amsmath package installed
%\usepackage{amssymb}  % assumes amsmath package installed
\usepackage{amsmath}
\usepackage{amssymb}
\usepackage{hyperref}
\usepackage{multirow}
\usepackage{graphicx}


\title{\LARGE \bf
4DR P2T: 4D Radar Tensor Synthesis with Point Clouds
}


\author{Woo-Jin Jung, Dong-Hee Paek, and Seung-Hyun Kong% <-this % stops a space
\thanks{This work was supported by the National Research Foundation of Korea(NRF) grant funded by the Korea government(MSIT) (No. 2021R1A2C3008370).}% <-this % stops a space
\thanks{Woo-Jin Jung, Dong-Hee Paek, and Seung-Hyun Kong are with the CCS Graduate School of Mobility, Korea Advanced Institute of Science and Technology, Daejeon, Korea, 34051 
        {\tt\small \{woo-jin.jung, donghee.paek, skong\}@kaist.ac.kr}}}%



\begin{document}



\maketitle
\thispagestyle{empty}
\pagestyle{empty}


%%%%%%%%%%%%%%%%%%%%%%%%%%%%%%%%%%%%%%%%%%%%%%%%%%%%%%%%%%%%%%%%%%%%%%%%%%%%%%%%
\begin{abstract}

In four-dimensional (4D) Radar-based point cloud generation, clutter removal is commonly performed using the constant false alarm rate (CFAR) algorithm. However, CFAR may not fully capture the spatial characteristics of objects. To address limitation, this paper proposes the 4D Radar Point-to-Tensor (4DR P2T) model, which generates tensor data suitable for deep learning applications while minimizing measurement loss. Our method employs a conditional generative adversarial network (cGAN), modified to effectively process 4D Radar point cloud data and generate tensor data. Experimental results on the K-Radar dataset validate the effectiveness of the 4DR P2T model, achieving an average PSNR of 30.39dB and SSIM of 0.96. Additionally, our analysis of different point cloud generation methods highlights that the 5\% percentile method provides the best overall performance, while the 1\% percentile method optimally balances data volume reduction and performance, making it well-suited for deep learning applications.

\end{abstract}


%%%%%%%%%%%%%%%%%%%%%%%%%%%%%%%%%%%%%%%%%%%%%%%%%%%%%%%%%%%%%%%%%%%%%%%%%%%%%%%%
\section{INTRODUCTION}

In recent autonomous driving research, 4D Radar has gained increasing attention as an advanced sensing technology. Traditional Radar sensors, often employed as auxiliary sensors, measure range, azimuth, and Doppler information. In contrast, 4D Radar incorporates elevation into these measurements, enabling more precise spatial perception. Consequently, 4D Radar provides more robust measurements than cameras and LiDAR under adverse weather conditions such as snow or rain. Furthermore, it surpasses traditional Radar in detecting object contours, demonstrating superior object detection capabilities. Owing to these advantages, 4D Radar has emerged as a key sensing modality in autonomous driving systems, offering enhanced object detection across diverse operational environments.

Most 4D Radar data are provided as point clouds, which are typically generated by traditional handcrafted methods such as CFAR to remove clutter \cite{clutter} from the tensor data. However, CFAR processes each cell independently, disregarding spatial continuity across adjacent cells. As a result, CFAR-generated point clouds often fail to preserve essential spatial characteristics—such as object size, shape, and continuous contours—thereby limiting their ability to accurately represent complex objects \cite{4D_radar_survey}. In autonomous driving scenarios where objects vary in size and shape, this limitation constrains environmental perception. Moreover, CFAR-based point clouds typically exhibit much lower point density than LiDAR, reducing the fidelity of captured object features \cite{rpfa_net} and complicating subsequent sensor fusion processes \cite{dpft}.

\begin{figure}[t!]
  \centering
  \includegraphics[width=1.0\columnwidth]{fig/fig0.png}
  \caption{
   4DR P2T overview. The 4DR P2T model generates tensor data from 4D Radar point clouds, which are represented in bird’s-eye view (BEV) as a 2D projection. Traditional point cloud generation methods often suffer from measurement loss, which may affect their suitability for deep learning training. To mitigate this limitation, the model generates tensor data to prevent measurement loss, ensuring that crucial information is retained for deep learning tasks.}
  \label{fig0.overview}
\end{figure}

\begin{figure*}[!th]
 \centering
\vspace{1mm} 
 \includegraphics[width=1.0\textwidth]{fig/fig1.png}
    \caption{4D Radar signal processing and data representation \cite{4D_radar_survey, 4dradar_tutorial, 4dradar_data_representation}. The Radar power values are normalized and represented using colors. The Radar point cloud is shown as black points, and the bounding box for the objects is indicated with a red box.}
  \label{fig1.4d_radar_data}

\end{figure*}

To mitigate these limitations, previous studies have proposed methods for reconstructing points representing objects \cite{3DRIMR} or generating tensor data prior to CFAR \cite{radarpointgenerator1} \cite{4D_radar_survey}. One notable method \cite{radarpointgenerator1} utilizes a conditional generative adversarial network (cGAN) with a UNet \cite{unet} architecture, leveraging LiDAR data to supervise the generation of denser Radar point clouds. However, fundamental differences between LiDAR (near-infrared) and Radar (electromagnetic waves) result in heterogeneous data characteristics, leading to distortions in power values and contour representations, which may degrade the reliability of the generated Radar data.

As shown in Fig. \ref{fig0.overview}, a method is required to directly generate tensor data using the original 4D Radar tensor as supervision, thereby avoiding cross-sensor inconsistencies. In this study, we leverage the K-Radar dataset \cite{KRadar}, currently the only publicly available dataset that provides 4D tensor data. Prior to its release in 2023, no dataset included 4D tensor data, making direct data-driven methods infeasible. With this new dataset, it is now possible to train models that generate tensor representations from 4D Radar point clouds collected by the same sensor.

Accordingly, we propose the 4D Radar Point cloud-to-Tensor (4DR P2T) model, which utilizes a cGAN-based architecture to generate tensor data from 4D Radar point clouds. This study conducts two primary investigations. First, we identify the point cloud generation method that achieves the best tensor generation performance—measured by peak signal-to-noise ratio (PSNR) and structural similarity index measure (SSIM)—among CFAR \cite{rtnh+} and percentile-based methods \cite{KRadar, enhancekradar} with different densities. Second, we determine the optimal point cloud generation method for deep learning applications, specifically the one that minimizes data volume while preserving sufficiently high tensor generation performance. To enable these investigations, we interpret point cloud data as the encoded version of tensor data, with our 4DR P2T model serving as a decoder that generates the original tensor. Consequently, the tensor generation performance of the 4DR P2T model serves as a proxy for assessing how well a given point cloud preserves environmental information, which in turn facilitates the selection of the most suitable point cloud generation method for deep learning model training and interpretation. Through our experiments, the proposed 4DR P2T model achieves an average PSNR of 30.39dB and SSIM of 0.96, demonstrating its effectiveness and stability. Our findings reveal that the percentile 5\% method yields the best tensor generation performance, while the percentile 1\% method offers an optimal balance between data volume reduction and performance, making it well-suited for deep learning training.

The key contributions of this study are summarized as follows:
\begin{itemize}
    \item Development of the 4DR P2T model, which generates tensor data from 4D Radar point cloud data.
    \item Experimental validation showing that the percentile 5\% data provides the best tensor generation performance.
    \item Confirmation that the percentile 1\% method effectively reduces data volume while maintaining high tensor generation performance.
\end{itemize}

This paper is organized as follows. Section \ref{sec:related_works} discusses 4D Radar signal processing and data generation processes, and reviews related models that convert point cloud data into tensors. Section \ref{sec:method} describes the proposed model architecture. Section \ref{sec:experiments} presents and analyzes the quantitative and qualitative experimental results. Finally, Section \ref{sec:conclusion} concludes the paper and discusses future research directions.

\section{Related work} \label{sec:related_works}
In this section, we provide an overview of related works, focusing specifically on 4D Radar signal processing and data generation, as well as previous studies on data translation methods using cGANs, which form the basis for developing models that generate tensor data from point cloud data.

\begin{figure*}[!th]
    \centering
    \vspace{1mm} % 위쪽 여백 추가
    \includegraphics[width=1\textwidth]{fig/fig2.png} 
    \caption{Overall structure of 4DR P2T. The encoder utilizes 3D sparse convolution to process 4D Radar point cloud data, while the decoder employs 3D dense convolution to generate tensor data.}
    \label{fig2.model}
\end{figure*}

\subsection{4D Radar signal processing and data generation}
The 4D Radar signal processing and data generation process, as applied in autonomous driving, is illustrated in Figure \ref{fig1.4d_radar_data}. The core analog components of a 4D Radar system consist of a synthesizer, transmission (TX) antennas, reception (RX) antennas, and a mixer. The TX antennas emit electromagnetic waves, which reflect off objects in the environment and are received by the RX antennas. The transmitted signal is generated by the synthesizer and radiated through the TX antennas. This signal is a frequency-modulated continuous wave (FMCW), composed of a sequence of frequency-modulated signals, commonly referred to as chirps.

The signal emitted by the TX antennas and the signal received by the RX antennas are combined using a mixer, producing an intermediate frequency (IF) signal. This IF signal represents the frequency difference between the transmitted and received signals, which is used to extract the distance and velocity of the reflected objects. The generated IF signal is then converted into a digital form through an analog-to-digital converter (ADC), creating ADC sample data. This data is separated into a fast time axis, which calculates range information through chirp sampling, and a slow time axis, which calculates Doppler information through frame sampling.

The ADC sample data is processed through a 2D Fast Fourier Transform (FFT), which is applied to perform range FFT and Doppler FFT. The range FFT estimates the distance to objects, while the Doppler FFT estimates their relative velocity, resulting in the generation of an RD heatmap. Although the RD heatmap contains information about range and velocity, it does not include azimuth or elevation information, making it less intuitive to interpret.

To extract azimuth and elevation information, an additional angle FFT is applied to the RD heatmap. The angle FFT utilizes the positional information of the TX and RX antennas arrays in a multiple-input and multiple-output (MIMO) antennas design to analyze the phase differences in the reflected signals. This process generates a 4D tensor that includes range, azimuth, elevation, and Doppler information, with each tensor cell representing the corresponding signal strength. The 4D tensor is represented in a polar coordinate system, but for better interpretability, the visualization shown in the Fig. \ref{fig1.4d_radar_data} is converted into a Cartesian coordinate system.

The generated 4D tensor data is filtered using the CFAR method. CFAR dynamically adjusts the threshold by comparing the signal strength of each cell to its surrounding cells, effectively removing noise and identifying actual targets. This filtering process is applied across all dimensions of the tensor, ultimately producing point cloud data that contains information about actual targets.

The resulting point cloud data includes the position (range, azimuth, elevation) and Doppler of the detected objects and is utilized in various autonomous driving applications, such as object detection and tracking \cite{4dradar_tutorial, 4D_radar_survey, FFT-RadNet}.

\subsection{Image translation}

Image translation focuses on style translation while preserving key information. Notable methods include pix2pix \cite{pix2pix} and pix2pixHD \cite{pix2pixhd}. These methods utilize cGANs to translate input images into output images. These methods have been successfully applied to various image synthesis and transformation tasks \cite{radsimreal, l2r, radarpointgenerator1}. Pix2pix employs a U-Net-based generator and a patch-based discriminator, enabling applications such as image synthesis and color translation. Pix2pixHD extends this framework to handle high-resolution images by incorporating boundary maps, multi-scale generators, and multi-scale discriminators, achieving improved quality. These methods excel at 2D image-to-image translation while maintaining structural information.

However, this study deals with generating tensor data from 4D Radar point cloud data, making it challenging to directly apply conventional image translation models. Existing methods are primarily optimized for 2D image data, necessitating structural modifications to handle higher-dimensional data such as point clouds and tensors. To address this, this study extends the fundamental method of pix2pixHD by modifying the model architecture to effectively process 3D or higher-dimensional data.

\section{Method} \label{sec:method}

This section outlines the data dimensions used for training, the model architecture, and the objective function.

\subsection{Data preparation}
In this study, the 4D Radar tensor data is reduced to 3D spatial information by excluding Doppler information for the training process. As a result, the input data for training consists of four channels, including \(x\), \(y\), \(z\) coordinates, and power values. Including Doppler data would require processing additional values beyond the existing spatial information \((x, y, z)\) and power, necessitating the use of convolution layers with at least four dimensions. This would significantly increase the complexity of the model and computational costs, making it challenging to achieve the primary goal of verifying implementation feasibility in the initial stage of the research. Moreover, according to RTNH \cite{KRadar}, an early model utilizing 4D Radar tensor data, excluding Doppler information still achieves sufficient object detection performance. Therefore, this study focuses on minimizing model complexity while verifying the feasibility of generating tensor data from 4D Radar point cloud data. This method also lays the foundation for future studies incorporating Doppler information.


\subsection{Model structure}
The proposed model is inspired by image translation methods, such as pix2pix and pix2pixHD, and referenced recent studies like l2r \cite{l2r} and RadSimReal \cite{radsimreal} to balance generative feasibility and structural simplicity, while optimizing for input and output data dimensions. To capture the spatial characteristics of Radar point cloud data, the model employs an encoding method based on Voxelnet \cite{voxelnet}, drawing from Lee’s method \cite{l2rtranslation_voxel}. The 4DR P2T model extends the U-Net structure \cite{unet}, commonly used in image translation tasks, with modifications to process 3D data.

As illustrated in Figure \ref{fig2.model}, the encoder uses 3D sparse convolution layers to account for the sparsity of 4D Radar point cloud data. Sparse convolution layers \cite{sparse_conv} are employed in stages where spatial resolution is reduced, while submanifold sparse convolution layers \cite{submaninfold} are utilized for operations where spatial resolution remains unchanged, thereby enhancing feature representation. In the decoder, 3D dense convolution layers are used to generate a dense 3D tensor. This method performs computations across all regions, making it suitable for producing complete tensors.

The generated tensor data is evaluated using a multi-scale discriminator  \cite{pix2pixhd}, which determines the authenticity of the data. To handle dense data, the discriminator also incorporates 3D dense convolution layers.

\subsection{Objective functions}
4DR P2T adopts a training framework using a Generator \(G\) and a Discriminator \(D\), inspired by traditional image translation methods. While image translation typically aims for a one-to-many mapping to generate diverse outputs, this study focuses on a one-to-one mapping, necessitating the design of appropriate loss functions. Following the method by Wang \cite{l2r}, the final loss function is defined as follows:
\begin{align}
\mathcal{L}(G, D) &= \mathcal{L}_{cGAN}(G, D) + \lambda_{L1} \mathcal{L}_{L1}(G) \nonumber \\
&\quad + \lambda_{perc} \mathcal{L}_{perc}(G) \label{eq:loss}
\end{align}
where \(\lambda_{L1}\) and \(\lambda_{perc}\) are weights that control the importance of each loss component, ensuring balanced training.

% \subsubsubsection{Conditional Adversarial Loss ({L}_{cGAN})}
First, a conditional adversarial loss is used, where the \(G\) synthesizes data, and the \(D\) learns to distinguish between real and synthesized data. This is the core loss of cGANs, defined as:
\begin{align}
\mathcal{L}_{cGAN}(G, D) &= \mathbb{E}_{x,y}[\log D(x, y)] \nonumber \\
&\quad + \mathbb{E}_{x}[\log (1 - D(x, G(x)))] \label{eq:cgan_loss}
\end{align}
Here, \(x\) represents input data, \(y\) is the GT, and \(G(x)\) is the output of the \(G\). The \(D\) learns to differentiate \(G(x)\) from \(y\), while the \(G(x)\) is trained to deceive \(D\) by making \(G(x)\) resemble \(y\).

% \subsubsubsection{L1 Loss ({L}_{L1})}
Second, L1 loss minimizes the absolute error between synthesized data \(G(x)\) and GT \(y\). This simple and stable loss function ensures that the synthesized data closely resembles real data:
\begin{equation}
\mathcal{L}_{L1}(G) = \mathbb{E}_{x,y}\left[|G(x) - y|_1\right]
\label{eq:l1_loss}
\end{equation}
% \subsubsubsection{Perceptual Loss ({L}_{perc})}
Third, perceptual loss introduced in pix2pixHD is used to compare high-level feature distributions between synthesized and real data. By leveraging intermediate layer outputs from a pre-trained neural network, perceptual loss measures semantic differences, guiding the synthesized data to have similar high-level features to real data.

\begin{figure*}[!th]
{
  \centering
\vspace{1mm} 
 \includegraphics[width=1.0\textwidth]{fig/fig3.png}
    \caption{Qualitative experimental results of 4DR P2T. The top part shows the front camera image and LiDAR point cloud as reference data to understand the scene of the 4D Radar GT tensor data, while the bottom part presents the tensor data results generated by 4DR P2T under different point cloud generation methods and density conditions.}
  \label{fig3.result}
}
\end{figure*}

\section{Experiments} \label{sec:experiments}

This section describes the datasets and implementation details for training the 4DR P2T model, the evaluation metrics used, and the results and analysis of tensor generation performance.

\subsection{K-Radar dataset}
The K-Radar dataset \cite{KRadar}, which was used to train the 4DR P2T model, is the only dataset that provides 4D Radar tensor data (4DRT) consisting of the four dimensions: range, azimuth, elevation, and Doppler. This makes it of significant value. Additionally, K-Radar includes data from various weather conditions (clear, cloudy, fog, rain, sleet, light snow, and heavy snow), which distinguishes it from other autonomous driving 4D Radar datasets. Furthermore, the dataset includes 4D Radar data, high-resolution LiDAR data, and camera data, with 93.3K object labels for 35K frames, distributed across 58 different driving scenes.

K-Radar includes not only 4DRT tensor data but also point cloud data with CFAR applied, as used in the experiments of RTNH+ \cite{rtnh+}, and point cloud data with the percentile method applied, as used in the RTNH model \cite{KRadar}. The percentile method is effective in reducing memory and computational complexity while preserving the structure of tensor data, and thus was used as input data for training the RTNH model.

\subsection{Implementation details}
The experiments in this study were conducted using the K-Radar dataset, with data generated using various point cloud generation methods and density conditions for comparison. Specifically, point cloud data generated using the percentile method (top 0.1\%, 1\%, 5\%, 10\%) from enhanced K-Radar \cite{enhancekradar}, and point cloud data with hyper-parameter ($K_{1}$) of 2.5\% ($N_{2.5,a}$) and 10\% ($N_{10,a}$) using constant average CFAR from RTNH+ \cite{rtnh+} were used. These datasets were selected due to the significant differences in the point cloud distribution, making them suitable for comparison analysis.

The point cloud data used for training was extracted from 4D tensors in polar coordinates using CFAR or the percentile method and then converted into Cartesian coordinates. In this process, it can be observed that points become increasingly sparse as the range (distance) increases (Fig. \ref{fig0.overview}). The tensor data used for training was reconstructed into a dense cube shape through interpolation after converting from polar to Cartesian coordinates \cite{KRadar}. This data preparation process was set up to verify whether sparse point cloud data could be transformed into dense Cartesian tensor data and to expand its range of applicability.

The Region of Interest (ROI) was set as $x$-axis [0, 76.8], $y$-axis [-16, 16], and $z$-axis [-2, 10.8]. This range was chosen considering the scope of the RTNH\_WIDE \cite{kradargithub} object detection model trained with the widest range. All sequence data were used in the experiment, with the 4DR P2T model trained using the train set and performance evaluated using the test set. Model training was performed on an NVIDIA 3090 GPU, with a batch size of 8, a learning rate of 0.001, and Adam optimizer \cite{adam}, running for 20 epochs.

\subsection{Metrics}
For evaluation metrics, PSNR and SSIM were used, referencing \cite{l2r}. PSNR measures the signal-to-noise ratio between the synthetic data and the ground truth data, while SSIM measures the structural similarity between the two datasets. Both metrics indicate better performance with higher values. Although the generated data is a 3D tensor, the evaluation was performed by converting it to a 2D image through mean pooling along the height axis, and then calculating the metrics.

The deep-learning efficiency score (DES) metric, defined in Eq. \ref{eq:DES}, was used to identify efficient point cloud generation methods for deep learning. This metric aims to reduce data volume, which is related to point cloud density (PCD), while maintaining high tensor generation performance. First, the PSNR and SSIM values are normalized using min-max scaling, as shown in Eq. \ref{eq:psnr_norm} and Eq. \ref{eq:ssim_norm}, to ensure a fair comparison.
\begin{equation}
\text{PSNR}_{\text{norm}}^{(i)} = \frac{\text{PSNR}^{(i)} - \text{PSNR}_{\min}}{\text{PSNR}_{\max} - \text{PSNR}_{\min}}
\label{eq:psnr_norm}
\end{equation}
\begin{equation}
\text{SSIM}_{\text{norm}}^{(i)} = \frac{\text{SSIM}^{(i)} - \text{SSIM}_{\min}}{\text{SSIM}_{\max} - \text{SSIM}_{\min}}
\label{eq:ssim_norm}
\end{equation}
Where \( \text{PSNR}^{(i)} \) and \( \text{SSIM}^{(i)} \) represent the PSNR and SSIM for the \textit{\( i \)-th method}, respectively. \( \text{PSNR}_{\min} \) and \( \text{PSNR}_{\max} \) denote the minimum and maximum PSNR values across all evaluated methods, and similarly \( \text{SSIM}_{\min} \) and \( \text{SSIM}_{\max} \) represent the minimum and maximum SSIM values. Using these normalized values, the DES metric is computed as shown in Eq. \ref{eq:DES}.
\begin{equation}
M = \alpha \times \frac{\text{PSNR}_{\text{norm}}^{(i)}}{D^{(i)}} 
+ \beta \times \frac{\text{SSIM}_{\text{norm}}^{(i)}}{D^{(i)}}
\label{eq:DES}
\end{equation}
Where \( D^{(i)} \) is the PCD, defined as the ratio of detected points to the total possible points within the ROI for method \( i \). The weighting factors \( \alpha \) and \( \beta \), which control the relative importance of PSNR and SSIM, satisfy the constraint \( \alpha + \beta = 1 \). In this study, equal weights of 0.5 were assigned to both PSNR and SSIM.

\subsection{Results}

\begin{table}[ht]
\caption{Quantitative experimental results of 4DR P2T. 'Method' refers to the main categories of point cloud generation methods, while 'Hyper.' denotes the subcategories of point cloud generation methods, representing the hyper-parameters used in each point generation method. }
\label{tab:result}
\centering
\renewcommand{\arraystretch}{1.4} 
\begin{tabular}{c|c|c|c|c|c}
\hline \hline
\begin{tabular}[c]{@{}c@{}}Method\end{tabular} &
  Hyper. &
  \begin{tabular}[c]{@{}c@{}} PCD (\%)\end{tabular} &
  PSNR (dB) ↑ &
  SSIM ↑ &
  \begin{tabular}[c]{@{}c@{}}DES ↑\end{tabular} \\ \hline
\multirow{2}{*}{CFAR}       & 2.5 & 1.22   & 30.00 & 0.96 & 0.33 \\
                            & 10  & 2.42   & 28.14 & 0.96  & 0.11 \\ \hline
\multirow{4}{*}{Percentile} & 0.1 & 0.12   & 28.08 & 0.94 & 0.00 \\
                            & 1   & 1.11  & 31.66 & 0.96 & \textbf{0.48} \\
                            & 5   & 4.46 & \textbf{34.43} & \textbf{0.98} & 0.22 \\
                            & 10  & 8.17 & 30.00 & 0.96 & 0.05 \\ \hline \hline
\end{tabular}

\end{table}

Tab. \ref{tab:result} summarizes the tensor generation performance of the 4DR P2T model on 4D Radar point cloud data generated by various methods, evaluated using PSNR, SSIM, and DES. The average PSNR across all methods is 30.39dB—exceeding the 20–25 dB threshold commonly considered acceptable in wireless communication quality \cite{PSNR_1, PSNR_2}—indicating that the generated tensor data is of sufficiently high performance. As shown in Fig. \ref{fig3.result}, the percentile 5\% method achieves the best tensor generation performance, with a PSNR of 34.43dB and SSIM of 0.98. Meanwhile, the percentile 1\% method attains the highest DES value of 0.48, while also demonstrating superior point generation ability while reducing data volume, making it well-suited for deep learning model training.

\section{CONCLUSIONS} \label{sec:conclusion}

This study introduces the 4DR P2T model, which generates tensor data from 4D Radar point cloud data to address the limitation of inadequate spatial characteristic capture when CFAR is applied to 4D Radar data. By leveraging a cGAN-based architecture, our model effectively generates tensor data, as demonstrated by an average PSNR of 30.39dB and SSIM of 0.96. In addition, our comparative experiments show that the percentile 5\% method yields the best tensor generation performance,  while the percentile 1\% method offers an optimal balance between data volume reduction and performance, making it well-suited for deep learning training.

Future research will extend the 4DR P2T model to accommodate unpaired data, enabling tensor generation even for datasets lacking original tensor data.  Additionally, Doppler information will be incorporated to further enhance object representation. These advancements aim to improve the preservation of critical object features, enhance sensor fusion, and ultimately strengthen perception capabilities in autonomous driving systems.

\addtolength{\textheight}{-12cm}   % This command serves to balance the column lengths
                                  % on the last page of the document manually. It shortens
                                  % the textheight of the last page by a suitable amount.
                                  % This command does not take effect until the next page
                                  % so it should come on the page before the last. Make
                                  % sure that you do not shorten the textheight too much.

%%%%%%%%%%%%%%%%%%%%%%%%%%%%%%%%%%%%%%%%%%%%%%%%%%%%%%%%%%%%%%%%%%%%%%%%%%%%%%%%



%%%%%%%%%%%%%%%%%%%%%%%%%%%%%%%%%%%%%%%%%%%%%%%%%%%%%%%%%%%%%%%%%%%%%%%%%%%%%%%%



%%%%%%%%%%%%%%%%%%%%%%%%%%%%%%%%%%%%%%%%%%%%%%%%%%%%%%%%%%%%%%%%%%%%%%%%%%%%%%%%
\section*{ACKNOWLEDGMENT}

This work was supported by the National Research Foundation of Korea (NRF) grant funded by the Korea government (MSIT) (No. 2021R1A2C3008370).



%%%%%%%%%%%%%%%%%%%%%%%%%%%%%%%%%%%%%%%%%%%%%%%%%%%%%%%%%%%%%%%%%%%%%%%%%%%%%%%%

\bibliographystyle{unsrt}
\bibliography{ref}






\end{document}

\bibliographystyle{IEEEtran}


\end{document}
