\begin{proof}
    We prove each property sequentially.
\begin{enumerate}
    \item For $r_t(p)$, we have:
    \begin{equation}
        \label{equ:r_p_calculate}
        \begin{aligned}
            r_t(p)&= \E[p_t\cdot D_t|p_t=p]\\
            &= p\cdot\E[\min\{\gamma_t, a-bp_t+N_t\}|p_t=p]\\
            &=p\cdot\E[\ind[a-bp+N_t\leq\gamma_t]\cdot(a-bp+N_t)+\ind[a-bp+N_t>\gamma_t]\cdot \gamma_t]\\
            &=p\cdot\E[\ind[N_t\leq \gamma_t-a+bp]\cdot(a-bp+N_t) + \ind[N_t >\gamma_t-a+bp]\cdot\gamma_t]\\
            &=p\left(\int_{-c}^{\gamma_t-a+bp}(a-bp+x)f(x)dx + \int_{\gamma_t-a+bp}^c \gamma_t f(x)dx\right)\\
            &=p\left(\int_{-c}^c(a-bp+x)f(x)dx + \int_{\gamma_t-a+bp}^c(\gamma_t-(a-bp+x))f(x)dx\right)\\
            &=p\left((a-bp)\cdot\int_{-c}^cf(x)dx + \int_{-c}^cxf(x)dx+(\gamma_t-a+bp)\int_{\gamma_t-a+bp}^cf(x)dx-\int_{\gamma_t-a+bp}^cxf(x)dx\right)\\
            &=p(a-bp)+0+p(\gamma_t-a+bp)(1-F(\gamma_t-a+bp))-p\cdot(xF(x)-G(x))|_{\gamma_t-a+bp}^c\\
            &=p\gamma_t-p(\gamma_t-a+bp)F(\gamma_t-a+bp)-p(c-G(c)-F(\gamma_t-a+bp)\cdot(\gamma_t-a+bp)+G(\gamma_t-a+bp))\\
            &=p(\gamma_t-c+G(c)-G(\gamma_t-a+bp)).
        \end{aligned}
    \end{equation}
    Here we adopt the notation $f(x)$ as \emph{proximal derivatives} of $F(x)$. According to Rademacher's Theorem \citep[see][Section 3.5]{folland1999real}, given that $F(x)$ is Lipschitz, the measure of $x$ such that $f(x)$ does not exist is zero, hence the intergral holds. Here the eighth line comes from $\int_{-c}^c f(x)dx = F(c) - F(-c) = 1$ and $\int_{-c}^{c}xf(x)dx=\E[x]=0$. Given the close form of $r_t(p)$, we derive the form of $r'_t(p)$.
    \\
    \item As we have assumed, $F(x)$ is $L_F$-Lipschitz, and therefore $F(\gamma_t-a+bp)$ is $b_{\max}L_F$-Lipschitz, and $bpF(\gamma_t-a+bp)$ is $(b_{\max} + b_{\max}^2p_{\max}L_F)$-Lipschitz. Also, we have $\frac{d G(\gamma_t-a+bp)}{dp}=b\cdot F(\gamma_t-a+bp)\in[0, b_{\max}]$. Let $L_r:=(2b_{\max} + b_{\max}^2p_{\max}L_F)$, and we know that $r'_t(p)$ is $L_r$-Lipschitz.
    \\
    \item On the one hand, we have
    \begin{equation}
        \label{eq:r'_first_part_non_increasing}
        \begin{aligned}
            \frac{d(\gamma_t - c + G(c)-G(\gamma_t-a+bp))}{d p}=-bF(\gamma_t - a + bp)\leq 0.
        \end{aligned}
    \end{equation}
    On the other hand, for any $\Delta_p>0$, we have
    \begin{equation}
        \label{eq:r'_second_part_non_increasing}
        \begin{aligned}
            &-[b(p+\Delta_p)\cdot F(\gamma_t-a+b(p+\Delta_p))]- [-(bpF(\gamma_t-a+bp))]\\
            =&-b\Delta_pF(\gamma_t-a+b(p+\Delta_p)) + bp(F(\gamma_t-a+bp) - F(\gamma_t-a+b(p+\Delta_p)))\\
            \leq& 0 + 0 = 0.
        \end{aligned}
    \end{equation}
    Since $r'_t(p)=\gamma_t - c + G(c)-G(\gamma_t-a+bp) - bpF(\gamma_t-a+bp)$, we know that both components are monotonically non-increasing.
    \\
    \item We first show the \emph{existence} of $p_t^*\in[0,\frac{a}{b}]$ such that $r'_t(p_t^*)=0$. Recall that $G(c + x) = G(c) + x$ for $\forall x>0$, and $G(c) - G(-c) = \int_{-c}^{c}F(\omega)d\omega\geq0$, and that $\gamma_t>2c>c$ as we assumed. Given those, we have:
    \begin{equation}
        \label{eq:r'_t_0_and_r'_t_a/b}
        \begin{aligned}
            r'_t(0)&=\gamma_t-c+G(c)-G(\gamma_t-a)\\
            &>\gamma_t-c+G(c)-G(-c)\\
            &>0.\\
            r'_t(\frac{a}{b})&=\gamma_t - C + G(c) - G(\gamma_t)-b\cdot\frac{a}{b}\cdot F(\gamma_t)\\
            &=\gamma_t - c + G(c) - G(c+(\gamma_t-c)) - a\cdot 1\\
            &=\gamma_t - c + G(c) - G(c) - (G(c) + (\gamma_t - c)) - a\\
            &=\gamma_t - c + 0 - (\gamma_t - c) - a\\
            &=-a<0.
        \end{aligned}
    \end{equation}
    Also, $r'_t(p)$ is Lipschitz as we proved above. Therefore, $\exists p_t^*\in(0, \frac{a}{b})$ such that $r'_t(p_t^*)=0$.
    \\
    Now we show the uniqueness of $p_t^*$. If there exists $0<p_t^*<q_t^*<\frac{a}{b}$ such that $r'_t(p_t^*) = r'_t(q_t^*)=0$, then it leads to.
    \begin{equation}
        \label{eq:uniqueness_p_t^*}
        \begin{aligned}
            &r'_t(p)\equiv 0, \forall p\in[p_t^*, q_t^*] \text{ due to the monotonicity of }r'_t(p)\\
            \Rightarrow\quad&r'_t(p)\text{ is differentiable in} (p_t^*, q_t^*)\\
            \Rightarrow\quad&r''_t(p)\equiv 0 , \forall p\in (p_t^*, q_t^*)\\
            \Rightarrow\quad&F(\gamma_t-a+bp)\text{ is differentiable with respect to} p \text{ on }(p_t^*, q_t^*)\\
            \Rightarrow\quad&f(\gamma_t-a+bp)\text{ exists on }(p_t^*, q_t^*)\\
            \Rightarrow\quad&-2bF(\gamma_t-a+bp)-b^2pf(\gamma_t-a+bp)\equiv0, \forall p\in(p_t^*, q_t^*)\\
            \Rightarrow\quad&\left\{
            \begin{aligned}
            &F(\gamma_t-a+bp)=0,\text{ and}\\
            &f(\gamma_t-a+bp)=0, \forall p\in(p_t^*, q_t^*)\\
            \end{aligned}
            \right.\\
            \Rightarrow\quad&0\leq F(\gamma_t-a+bp_t^*)\leq \lim_{p\rightarrow q_t^*-}F(\gamma_t-a+bp)=0\\
            \Rightarrow\quad& F(\omega)\equiv 0, \forall\omega<\gamma_t-a+bq_t^*\\
            \Rightarrow\quad&r'_t(p_t^*)=\gamma_t-c+G(c)-G(\gamma_t-a+bp_t^*)-bp_t^*\cdot F(\gamma_t-a+bp_t^*)\\
            &\qquad\quad=\gamma_t-c+G(c)-\int_{-\infty}^{\gamma_t-a+bp_t^*}F(\omega)d\omega-0\\
            &\qquad\quad=\gamma_t-c+G(c)\\
            &\qquad\quad\geq\gamma_t-c\\
            &\qquad\quad>0.
        \end{aligned}
    \end{equation}
    This leads to contradictions that $r'_t(p_t^*)=0$. Therefore, $p_t^*$ is unique. Given this, we know that $r_t(p)$ is unimodal, which increases on $(0, p_t^*)$ and decreases on $(p_t^*, \frac{a}{b})$.
    \\
    \item Since $p_t^*$ is unique, and $r'_t(p)$ is $L_r$-Lipschitz, we have:
    \begin{equation}
        r_t(p_t^*)-r_t(p)\leq\frac{L_r}2\cdot(p_t^*-p)^2, \forall p\in[0, \frac{a}{b}].
    \end{equation}
    \item From the proof of part (4), we know that $F(\gamma_t-a+bp_t^*)>0$ (or otherwise $r'_t(p_t^*)>0$ leading to contradiction). Denote $\epsilon_t:=\frac{F(\gamma_t-a+bp_t^*)}{2L_Fb_{\max}}$, and we have:
    \begin{equation}
        \label{eq:epsilon_t}
        \begin{aligned}
            F(\gamma_t-a+b(p_t^*-\epsilon_t))\geq&F(\gamma_t-a+bp_t^*)-L_F\cdot b\cdot\epsilon_t\\
            \geq&F(\gamma_t-a+bp_t^*)-L_F\cdot b \cdot\frac{F(\gamma_t-a+bp_t^*)}{2L_F\cdot b_{\max}}\\
            =&\frac{F(\gamma_t-a+bp_t^*)}2.
        \end{aligned}
    \end{equation}
    Let $C_{\epsilon}:=\frac{b_{\min}}2\cdot\inf_{\gamma_t\in[\gamma_{\min}, \gamma_{\max}]} F(\gamma_t-a+bp_t^*)$. As $[\gamma_{\min}, \gamma_{\max}]$ is a close set and $F(\gamma_t-a+bp_t^*)$ holds for any $\gamma_t\in[\gamma_{\min}, \gamma_{\max}]$, we know that $C_{\epsilon}>0$ is a universal constant. Given this coefficient, for any $p_1, p_2\in[p_t^*-\epsilon_t, p_t^*+\epsilon_t], p_1<p_2$, we have
    \begin{equation}
        \label{eq:locally_strongly_concave}
        \begin{aligned}
            &r'_t(p_1) - r'_t(p_2)\\
            =&-(G(\gamma_t-a+bp_1)-G(\gamma_t-a+bp_2))-(bp_1F(\gamma_t-a+bp_1)-bp_2F(\gamma_t-a+bp_2))\\
            \geq&-(G(\gamma_t-a+bp_1)-G(\gamma_t-a+bp_2))\\
            =&\int_{\gamma_t-a+bp_1}^{\gamma_t-a+bp_2}F(\omega)d\omega\\
            \geq&\min_{\omega\in[p_t^*-\epsilon_t, p_t^*+\epsilon_t]}F(\omega)\cdot b(p_2-p_1)\\
            \geq&F(\gamma_t-a+b(p_t^*-\epsilon_t))\cdot b(p_2-p_1)\\
            >&\frac{F(\gamma_t-a+bp_t^*)}2\cdot b(p_2-p_1)\\
            \geq&C_{\epsilon}\cdot(p_2-p_1).
        \end{aligned}
    \end{equation}
    Here the third line is because $0<p_1<p_2$ and therefore $0< F(\gamma_t-a+bp_1)\leq F(\gamma_t-a+bp_2)$.
    \\
    \item According to part (6), for any $p\in(p_t^*-\epsilon_t, p_t^*+\epsilon_t)$, we have $|r'_t(p)|=|r'_t(p)-r'_t(p_t^*)|\geq C_{\epsilon}\cdot|p-p_t^*|$. Therefore, we have
    \begin{equation}
        \label{eq:loss_leq_derivative_square}
        \begin{aligned}
            (r'_t(p))^2\geq C_{\epsilon}^2\cdot(p-p_t^*)^2\geq\frac{C_{\epsilon}^2}{\frac{L_r}2}\cdot(r_t(p_t^*)-r_t(p))=\frac{2C_{\epsilon}^2}{L_r}\cdot(r_t(p_t^*)-r_t(p)).
        \end{aligned}
    \end{equation}
    Let $C_v:=\frac{L_r}{2C_{\epsilon}^2}$ and the property is proven.
\end{enumerate}
\end{proof}