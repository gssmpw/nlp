The problem of dynamic pricing, where the seller proposes and adjusts their prices over time, has been studied since the seminal work of \citet{cournot1897researches}. The crux to pricing is to balance the profit of sales per unit with the quantity of sales. Therefore, it is imperative for the seller to learn customers' demand as a function of price (commonly known as the \emph{demand curve}) on the fly. However, the demand can often be obfuscated by the observed quantity of sales, especially when \emph{censored} by \emph{inventory} stockouts. Such instances severely impede the seller from learning the underlying demand distributions, thereby hindering our pursuit of the optimal price.

Existing literature has devoted considerable effort to the intersection of pricing and inventory decisions. Such works often consider scenarios with indirectly observable lost demands \citep{keskin2022data}, recoverable leftover demands \citet{chen2019coordinating}, or controllable inventory level \citep{chen2023optimal}. However,these assumptions do not always align with the realities faced in various common business environments. To illustrate, we present two pertinent examples:
\vspace{1em}

\begin{example}[Performance Tickets]\label{example_performance}
    Imagine that we manage a touring company that arranges a series of performances featuring a renowned artist across various cities. Each venue has a different seating capacity, which substantially affects how we set ticket prices. If the price is too high, it might deter attendance, leading to lower revenue. On the other hand, setting it too low could mean that tickets sell out quickly, leaving many potential attendees unable to purchase them. We do not know exactly how many people attempt to buy tickets and fail. Moreover, because the performances are unique, there is no assurance that those who miss out on one show will choose or be able to attend another. This variability in venue size across different locations requires us to continually adapt our pricing strategy, ensuring that we maximize both attendance and revenue while accommodating unpredictable changes in seat availability.
\end{example}
\vspace{1em}
\begin{example}[Fruit Retails]\label{example_fruit}
    Sweetsop (\emph{Annona squamosa}, or so-called "sugar apple") is a  particularly-perishable tropical fruit, typically lasting only 2 to 4 days \citep{crane2005sugar}. Suppose we manage a local fruit shop and have partnered with a nearby farm for the supply of sweetsops during the harvest season. Due to the perishability nature, we receive sweetsops as soon as they are ripen and picked from the farm every day. This irregular supply means that some days we might receive a large quantity while getting very few on other days. We must quickly sell these fruits before they spoil, yet managing the price becomes challenging. If we exhaust our inventory ahead of time, customers will turn to other fruit shops to make their purchases rather than waiting for our next restock.  %  Otherwise, the excessing fruits will deteriorate, leading to waste. % Consequently, we are compelled to sell the entire stock from the previous delivery before the arrival of the next supply. Otherwise, the remaining sweetsops will blacken and rot. However, if we exhaust our inventory ahead of time, customers will turn to other fruit shops to make their purchases rather than waiting for our next restock.
\end{example}
Products in the two instances above have the following properties,
\begin{enumerate}
    \item Inventory level is determined by natural factors, and is arbitrarily given for every individual time period.
    \item Products are perishable and salable only within a single time period.
\end{enumerate}

In this work, we study a dynamic pricing problem where the products possess these properties. The problem model is defined as follows. At each time $t=1,2,\ldots, T$, we firstly propose a price $p_t$, and then a price-dependent \emph{potential demand} occurs as $d_t$. However, we might have no access to $d_t$ as it is censored by an \emph{adversarial} inventory level $\gamma_t$. Instead, we observe a censored demand $D_t = \min\{\gamma_t, d_t\}$ and receive the revenue $r_t$ as a reward at $t$. Our goal is to approach the optimal price $p_t^*$ at every time $t$, thereby maximizing the cumulative revenue.

\fbox{\parbox{0.95\textwidth}{Dynamic pricing with adversarial inventory constraint. For $t=1,2,...,T:$
		\small
		\noindent
		\begin{enumerate}[leftmargin=*,align=left]
			\setlength{\itemsep}{0pt}
                \item The seller (we) receives $\gamma_t$ identical products.
			\item The seller proposes a price $p_t\geq0$.
                \item The customers generate an invisible potential demand $d_t\geq 0$, dependent on $p_t$.
			\item The market reveals an inventory-censored demand $D_t=\min\{\gamma_t, d_t\}$.
			\item The seller gets a reward $r_t = p_t\cdot D_t$.
                \item All unsold products perish before $t+1$.
		\end{enumerate}
	}
}

\subsection{Summary of Contributions}
\label{subsec:summary_contribution}
We consider the problem setting displayed above and assume the potential demand $d_t=a-bp_t+N_t$ is \emph{linear} and \emph{noisy}. Here $a, b\in\R^+$ are fixed unknown parameters and $N_t$ is an \emph{unknown} i.i.d.\footnote{Independently and identically distributed} noise with zero mean. Under this premise, the key to deriving the optimal price is to accurately learn the expected reward function $r(p)$, which is equivalent to learning the linear parameters $[a,b]$ and the noise distribution. We are confronted by three principal challenges:

\begin{enumerate}
    \item The absence of unbiased observations of the potential demand or its derivatives with respect to $p$, which prevents us from estimating $[a,b]$ directly.
    \item The dependence of the optimal prices on the noise distribution, which is assumed to be unknown and partially censored. 
    \item The arbitrariness of the inventory levels, leading to non-stationary and highly-differentiated optimal prices $\{p_t^*\}$ over time.
\end{enumerate}

In this paper, we introduce an algorithm that employs innovative techniques to resolve the aforementioned challenges. First, we devise a pure-exploration phase that bypasses the censoring effect and obtains an unbiased estimator of $\frac1{b}$ (which leads to $\hat b$ and $\hat a$ as a consequence). Secondly, we maintain estimates of the noise CDF $F(x)$ and $\int F(x)dx$ over a series of discrete $x$'s, as well as the confidence bounds of each estimate. Thirdly, we design an \emph{optimistic} strategy, \textbf{C20CB} as ``Closest-To-Zero Confidence Bound'', that proposes the price $p_t$ whose reward derivative $r'_t(p_t)$ is probably $0$ or closest to $0$ among a set of discretized prices. As we keep updating the estimates of $r'_t(\cdot)$ with shrinking error bar, we asymptotically approach the optimal price $p^*_t$ since $r'_t(p_t^*)=0$ for any $t=1,2,\ldots, T$.


\paragraph{Novelty.} To the best of our knowledge, we are the first to study the online dynamic pricing problem under \emph{adversarial} inventory levels. Our C20CB algorithm attains an \emph{optimal} $\tilde{O}(\sqrt{T})$ regret guarantee with high probability. The methodologies we develop are not only pivotal to our design and analysis but also potential for broad application in a variety of online decision-making scenarios with censored feedback.


\subsection{Paper Structure}
\label{subsec:paper_structure}
The rest of this paper is organized as follows. We discuss and compare with related works in \Cref{sec:related_works}, and then describe the problem setting in \Cref{sec:preliminaries}. We propose our main algorithm C20CB in \Cref{sec:algorithm} and analyze its regret guarantee in \Cref{sec:regret_analysis}. We further discuss the limitations and potential extensions of our methodologies in \Cref{sec:discussion}, followed by a brief conclusion in \Cref{sec:conclusion}.