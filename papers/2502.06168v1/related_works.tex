There exists a large volume of literature related to the problem we study in this work. Here we discuss them in the following categories.

\paragraph{Data-driven dynamic pricing}
% non-contextual
Dynamic pricing for identical products is a well-established research area, starting with \citet{kleinberg2003value} and continuing through seminal works by \citet{besbes2009dynamic, broder2012dynamic, wang2014close, wang2021multimodal}. The standard approach involves learning a demand curve from price-sensitive demand arriving in real-time, aiming to approximate the optimal price. \citet{kleinberg2003value} provided algorithms with regret bounds of $O(T^{\frac23})$ and $O(\sqrt{T})$ for arbitrary and infinitely smooth demand curves, respectively. \citet{wang2021multimodal} refined this further, offering an $O(T^{\frac{k+1}{2k+1}})$ regret for $k$-times continuously differentiable demand curves. This line of inquiry is also intricately linked to the multi-armed bandit problems \citep{lai1985asymptotically, auer2002nonstochastic} and continuum-armed bandits \citep{kleinberg2004nearly}, where each action taken reveals a reward without insight into the foregone rewards of other actions.

\paragraph{Contextual pricing: Linear valuation and binary-censored demand}
A surge of research has delved into \emph{feature-based} dynamic pricing \citep{cohen2020feature_journal} or \emph{pricing with contexts/covariates} \citep{amin2014repeated, miao2019context, liu2021optimal}.  These works considered situations where each pricing period was preceded by a context, influencing both the demand curve and noise distribution. Specifically, \citet{cohen2020feature_journal, javanmard2019dynamic, xu2021logarithmic} explored a linear valuation framework with known distribution noise, leading to binary customer demand outcomes based on price comparisons to their valuations. Expanding on this, \citet{golrezaei2019incentive, fan2021policy, luo2021distribution, xu2022towards} examined similar models but with unknown noise distributions. In another vein, \citet{ban2021personalized, wang2021dynamic, xu2024pricing} investigated personalized pricing where demand was modeled as a generalized linear function sensitive to contextual price elasticity. Many of these works on valuation-based contextual pricing also assume a censored demand: The seller only observes a binary feedback determined by a comparison of price with valuation, instead of observing the valuation directly. However, it was important to differentiate between the linear (potential) demand model we assumed and their linear valuation models, and there exists no inclusive relationship from each other. 

\paragraph{Network Revenue Management (NRM)} NRM \citep{talluri2006theory} studies the pricing and allocation problem on shared resources in a network. In the problem settings of \citet{besbes2012blind} and the following works \citet{simchi2019blind, miao2024demand}, if we take a marginal observation from each different product, the occupation of resources by other products may result in adversarial supply. In this work, we also consider adversarial inventories. However, our paper focuses on resolving the censoring effect on demand functions and reducing regret through online learning methodologies. \citet{perakis2010robust} also adopts regret as a metric and considers censoring effect on the demand data as we do. However, their definition of minimax and maximin regret are different from ours. Also, instead of providing theoretical solutions, the censoring effect is only considered in empirical validation to test the robustness of the control system. Other representative works on NRM including \citet{gallego1994optimal} that proposes a classic model for dynamic pricing with stochastic demand, highlighting structural monotonicity and showing that simple policies can be asymptotically optimal; \citet{talluri1998analysis}, which shows bid‐price controls are near‐optimal in large capacities but are not strictly optimal; and \citet{meissner2012network} that addresses overbooking with product‐specific no‐shows via a randomized linear program.

\paragraph{Pricing with inventory concerns}
Dynamic pricing problems began to incorporate inventory constraints with the work of \citet{besbes2009dynamic}, which assumed a fixed initial stock available at the start of the selling period. They introduced near-optimal algorithms for both parametric and non-parametric demand distributions, operating under the assumption that the inventory was non-replenishable and non-perishable. \citet{wang2014close} adopted a comparable framework but allowed for customer arrivals to follow a Poisson process. In these earlier works, the actual demand is fully disclosed until the inventory is depleted. Subsequent research allowed inventory replenishment, with the seller's decisions encompassing both pricing and restocking at each time interval. \citet{chen2019coordinating} proposed a demand model subject to additive/multiplicative noise and developed a policy that achieved $O(\sqrt{T})$ regret. More recent studies, such as those by \citet{chen2020data, keskin2022data, xu2025joint} explored the dynamic pricing of perishable goods where unsold inventory would expire. However, the uncensored demand is observable as assumed in both works. Specifically, \citet{chen2020data} allowed recouping backlogged demand, albeit at a cost, and introduced an algorithm with optimal regret. \citet{keskin2022data} focused on cases where both fulfilled demands and lost sales were observable.

\citet{chenbx2021nonparametric} and their subsequent work, \citet{chen2023optimal}, are the closest works to ours as they adopt similar problem settings: In their works, the demand is \emph{censored} by the inventory level and any leftover inventory or lost sales disappear at the end of each period. With the assumption of concave reward functions and the restriction of at most $m$ price changes, \citet{chenbx2021nonparametric} proposed MLE-based algorithms that attain a regret of $\tilde{O}(T^{\frac1{m+1}})$ in the well-separated case and $\tilde{O}(T^{\frac12+\epsilon})$ for some  $\epsilon=o(1)$ as $T\rightarrow\infty$ in the general case. Under similar assumptions (except infinite-order smoothness), \citet{chen2023optimal} developed a reward-difference estimator, with which they not only enhanced the prior result for concave reward functions to $\tilde{O}(\sqrt{T})$ but also obtained a general $\tilde{O}(T^{2/3})$ regret for non-concave reward functions. Our problem model mirrors their difficulty in lacking access to both the uncensored demand and its gradient. However, they allowed the sellers to determine inventory levels with sufficient flexibility, hence better balancing the information revealed by the censored demand and the reward from (price, inventory) decisions. On the other hand, we assume the inventory level at each time period is provided \emph{adversarially} by nature, which could impede us from learning the optimal price in the worst-case scenarios. Furthermore, due to the non-stationarity of inventory levels in our setting, the optimal price $p_t^*$ deviates over time. Given this, the searching-based methods adopted in \citet{chen2023optimal} is no longer applicable to our problem.