In this work, we study the following online non-contextual dynamic pricing problem. At each time step $t=1,2,\ldots, T$, the seller (we) first receives an inventory quantity $\gamma_t$ determined by \emph{adversarial} nature. Then we propose a price $p_t$ for these items and receive an \emph{inventory-censored} demand $D_t := \min\{d_t, \gamma_t\}$. Here $d_t=a-b\cdot p_t + N_t$ is a potential linear demand, where $a, b$ are unknown parameters and $N_t$ is the demand noise.

In the following subsections, we will adopt definitions and assumptions to further clarify the scope of our methodologies.

\subsection{Definitions}
\label{subsec:definitions}
Here we define some key quantities that are involved in the algorithm design and analysis. Firstly, there are two different types of demand functions.

\begin{definition}[Demand functions]
    \label{def:demand}
    Denote $d_t(p):=a-bp+N_t$ as the \emph{potential demand function}, and $d(p):=a-bp$ as the \emph{expected potential demand function}. Denote $D_t(p):=\min\{\gamma_t, d_t(p)\}$ as the \emph{censored demand} function.
\end{definition}

Then we define distributional functions of the noise $N_t$.
\begin{definition}[Distributional functions]
    \label{def:cdf_pdf}
    For $N_t$ as the demand noise, denote $F(x)$ as its \emph{cumulative distribution function} (CDF), $x\in\R$. Also, denote
    \begin{equation}
        \label{equ:gdf}
        G(x):=\int_{-\infty}^x F(\omega)d\omega, x\in\R
    \end{equation}
    as the \emph{integrated CDF}.
\end{definition}

We will make more assumptions on the noise distribution later. Notice that we do not assume the existence of PDF for $N_t$. However, if there exists its PDF in specific cases, we will adopt $f(x)$ as a notation.

Finally, we may define the revenue function and the regret.

\begin{definition}[Revenue function]
    \label{def:revenue}
    Denote $r_t(p)$ as the expected revenue function of price $p$, satisfying
    \begin{equation}
        \label{equ:revenue_function}
        r_t(p) := p\cdot\E_{N_t}[D_t(p)|\gamma_t], p\geq 0.
    \end{equation}
    Also, denote $p_t^*:=\argmax_{p}r_t(p)$ as the optimal price at time $t$.
\end{definition}

\begin{definition}[Regret]
    \label{def:regret}
    Denote
    \begin{equation}
        \label{equ:regret_def}
        Regret:= \sum_{t=1}^T r_t(p_t^*)-r_t(p_t)
    \end{equation}
    as the \emph{cumulative regret} (or \emph{regret}) of the price sequence $\{p_t\}_{t=1}^T$.
\end{definition}

\subsection{Assumptions}
\label{subsec:assumptions}
We make reasonable and mild assumptions as follows. Firstly, we assume boundaries for parameters and price.

\begin{assumption}[Boundedness]
    \label{assumption:boundedness}
    There exist \emph{known} \emph{finite} constants $a_{\max}, b_{\min}, b_{ \max}, \gamma_{\min}, c >0$ such that $0<a\leq a_{\max}$, $0<b_{\min}\leq b\leq b_{\max}$, $\gamma_t\geq\gamma_{\min}$, $N_t\in[-c, c]$. Also, we restrict the proposed price $p_t$ at any $t=1,2,\ldots, T$ satisfies $0\leq p_t\leq p_{\max}$ with a \emph{known} \emph{finite} constant $p_{\max}>0$.
\end{assumption}

Secondly, we make assumptions on the noise distribution.
\begin{assumption}[Noise Distribution]
    \label{assumption:noise}
    Each $N_t$ is drawn from an \emph{unknown} independent and identical distribution (i.i.d.) satisfying $\E[N_t] = 0$. The CDF $F(x)$ is $L_F$-\emph{Lipschitz}. Also, according to \Cref{assumption:boundedness}, we have $F(-c)=0, F(c)=1$.
\end{assumption}

Thirdly, we make assumptions on the inequality relationships among parameters:

\begin{assumption}[Inequalities of Parameters]
    \label{assumption:inequality}
    The parameters and constants satisfy the following inequalities:
    \begin{enumerate}
        \item $a-c > \gamma_t, \forall t\in[T]$. Demands at $p_t=0$ must be censored.
        \item $\gamma_t > 2c, \forall t\in[T]$. Inventory level exceeds noise support.
        \item $a-b p_{\max} - c > 0$ and therefore $p_{\max}\leq\frac{a}{b}$. Demands must be positive.
        \item $\gamma_{\min} > a_{\max} -b_{\min} p_{\max} + c$. Demands at $p_t=p_{\max}$ must be uncensored. Meanwhile, we denote $\gamma_0:=a_{\max} -b_{\min} p_{\max} + c$ for further use.
        \item Without loss of generality,  let $p_{\max}\geq \frac{a}{2b}+1$. Optimal price must be included in $[0, p_{\max}]$.
    \end{enumerate}
\end{assumption}

Finally, we assume that the time horizon $T$ is sufficiently large, such that it will not confound with any constant or coefficient.

\begin{assumption}[Large $T$]
    \label{assumption:large_T}
    For any given polynomial of parameters $poly(a_{\max}, b_{\max}, 1/b_{\min}, c, 1/\gamma_{\min}, p_{\max})$, we have $T>poly$.
\end{assumption}

% % description
% \begin{description}
% 	\item[Inventory]
% 	inventory
% 	\item[Censoring Effect]
% 	censoring effect
% \end{description}