Here we discuss the limitations, potential extensions and impacts of our work.

\paragraph{Generalization to Unbounded Noises}
We assume the noise is bounded in a constant-width range. This assumption streamlines the pure-exploration phase and facilitates the estimation of the parameters $b$ and $a$. While our methods and results can be extended to unbounded $O(\frac1{\log T})$-subGaussian noises by simple truncation, challenges remain for handling generic unbounded noises. Moreover, the problem can be more sophisticated with \emph{dual-censoring}, both from above by inventory--as we have discussed-- and from below by $0$, especially when considering unbounded noises.

\paragraph{Extensions to Contextual Pricing}
In this work, we assume $a$ and $b$ are static, which may not hold in many real scenarios. \Cref{example_performance} serves as a good instance, showcasing significant fluctuations in popularity across different performances. A reasonable extension of our work would be modeling $a$ and $b$ as \emph{contextual} parameters. Similar modelings have been adopt by \citet{wang2021dynamic} and \citet{ban2021personalized} in the realm of personalized pricing research.

\paragraph{Extensions to Non-Lipschitz Noise CDF}
In this work, we assume the noise CDF as a Lipschitz function as many pricing-related works did \citep{fan2021policy, tullii2024improved}. This assumption enables the local smoothness at $p_t^*$ and leads to a quadratic loss. However, this prevent us from applying our algorithm to non-Lipschitz settings, which even includes the noise-free setting. In fact, although we believe that a better regret rate exists for the noise-free setting, we have to state that the hardness of problems are completely different with Lipschitz noises versus without it. Although a Lipschitz noise makes the observation ``more blur'', it also makes the revenue curve ``more smooth''. We would like to present an analog example from the feature-based dynamic pricing problem: When the Gaussian noise $\cN(0, \sigma^2)$ is either negligible (with $\sigma<\frac1T$, see \citet{cohen2020feature_journal}) or super significant (with $\sigma>1$, see \citet{xu2021logarithmic}), the minimax regret is $O(\log T)$. However, existing works can only achieve $O(\sqrt{T})$ regret when $\sigma\in[\frac1T, 1]$. We look forward to future research on our problem setting once getting rid of the Lipschitzness assumption.

\paragraph{Extensions to Non-linear Demand Curve}
In this work, we adopt a linear-and-noisy model for the potential demands, which is standard in dynamic pricing literature. Also, we utilize the unimodal property brought by this linear demand model, even after the censoring effect is imposed. If we would like to generalize our methodologies to non-linear demand functions, especially where the unimodality does not hold any longer, we have to carefully distinguish between local optimals and saddle-points that may also cause $r'_t(p)=0$ for some sub-optimal $p$. We conjecture an $\Omega(T^{\frac{m+1}{2m+1}})$ lower bound in that case, where $m$ is the time of smoothness. However, it is still unclear whether the censoring effect will introduce new local optimals or swipe off existing ones in multimodal settings.

\paragraph{Societal Impacts}
Our research primarily addresses a non-contextual pricing model that does not incorporate personal or group-specific data, thereby adhering to conventional fairness standards relating to temporal, group, demand, and utility discrepancies as outlined by \citet{cohen2022price, chen2023utility, xu2023doubly}. However, the non-stationarity of inventory levels could result in varying \emph{fulfillment rate} over time, i.e. the proportions of satisfied demands at $\{p_t^*\}$'s might be different for $t=1,2,\ldots, T$. This raises concern regarding unfairness in fulfillment rate \citep{spiliotopoulou2022fairness} particularly on product of significant social and individual importance.