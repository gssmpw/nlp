%%
%% This is file `sample-sigconf.tex',
%% generated with the docstrip utility.
%%
%% The original source files were:
%%
%% samples.dtx  (with options: `sigconf')
%% 
%% IMPORTANT NOTICE:
%% 
%% For the copyright see the source file.
%% 
%% Any modified versions of this file must be renamed
%% with new filenames distinct from sample-sigconf.tex.
%% 
%% For distribution of the original source see the terms
%% for copying and modification in the file samples.dtx.
%% 
%% This generated file may be distributed as long as the
%% original source files, as listed above, are part of the
%% same distribution. (The sources need not necessarily be
%% in the same archive or directory.)
%%
%%
%% Commands for TeXCount
%TC:macro \cite [option:text,text]
%TC:macro \citep [option:text,text]
%TC:macro \citet [option:text,text]
%TC:envir table 0 1
%TC:envir table* 0 1
%TC:envir tabular [ignore] word
%TC:envir displaymath 0 word
%TC:envir math 0 word
%TC:envir comment 0 0
%%
%%
%% The first command in your LaTeX source must be the \documentclass
%% command.
%%
%% For submission and review of your manuscript please change the
%% command to \documentclass[manuscript, screen, review]{acmart}.
%%
%% When submitting camera ready or to TAPS, please change the command
%% to \documentclass[sigconf]{acmart} or whichever template is required
%% for your publication.
%%
%%
\documentclass[sigconf]{acmart}

\usepackage{amsmath}
\usepackage{algorithm}
\usepackage{enumitem}
\usepackage{picinpar}

\usepackage{graphicx}
% \usepackage{caption}
% \usepackage{subcaption}
% \usepackage{multirow}

\usepackage{algorithmicx}
\usepackage[noend]{algpseudocode}
\usepackage{subfigure}
\usepackage{multirow}
\usepackage{color}
\usepackage{balance}
\usepackage{enumitem}
\usepackage{hhline}
\usepackage[normalem]{ulem}
\usepackage{booktabs}
\usepackage{wrapfig}
\usepackage{cancel}
\usepackage{hyperref}
\usepackage{makecell}


\newtheorem{theorem}{Theorem}
\newtheorem{lemma}{Lemma}
\newtheorem{corollary}{Corollary}
\newtheorem{assumption}{Assumption}
\newtheorem{definition}{Definition}

\newcommand{\bL}{\ensuremath{\mathcal{L}}}
\newcommand{\bU}{\ensuremath{\mathcal{U}}}
\newcommand{\bS}{\ensuremath{\mathcal{S}}}
\newcommand{\bH}{\ensuremath{\mathcal{H}}}
\newcommand{\bW}{\ensuremath{\mathcal{W}}}
\newcommand{\bV}{\ensuremath{\mathcal{V}}}
\newcommand{\bG}{\ensuremath{\mathcal{G}}}
\newcommand{\bN}{\ensuremath{\mathcal{N}}}
\newcommand{\bE}{\ensuremath{\mathcal{E}}}
\newcommand{\bT}{\ensuremath{\mathcal{T}}}
\newcommand{\bY}{\ensuremath{\mathcal{Y}}}
\newcommand{\bM}{\ensuremath{\mathcal{M}}}
\newcommand{\bC}{\ensuremath{\mathcal{C}}}
\newcommand{\bP}{\ensuremath{\mathcal{P}}}
\newcommand{\bD}{\ensuremath{\mathcal{D}}}
\newcommand{\bO}{\ensuremath{\mathcal{O}}}

\newcommand{\cls}{\text{cls}}
\newcommand{\str}{\text{str}}
\newcommand{\con}{\text{con}}

\newcommand{\fang}[1]{\textcolor{red}{[FANG: #1]}}
\newcommand{\kien}[1]{\textcolor{purple}{[Kien: #1]}}

\renewcommand{\vec}[1]{\ensuremath{\mathbf{#1}}}

\newcommand{\stitle}[1]{\vspace{0.8mm} \noindent {\bf #1}}
\newcommand{\cp}{specific prompts{}}
\newcommand{\op}{holistic prompts{}}


\newcommand{\eg}{{\it e.g.}}
\newcommand{\etal}{{\it et al.}}
\newcommand{\ie}{{\it i.e.}}
\newcommand{\etc}{{\it etc.}}
\newcommand{\wrt}{w.r.t. }
\newcommand{\vs}{{\it vs.}}
\newcommand{\tabincell}[2]{\begin{tabular}{@{}#1@{}}#2\end{tabular}}

\newcommand{\method}[1]{\textsc{#1}}
\newcommand{\model}{\method{SAMGPT}{}}
\newcommand{\modelS}[1]{\method{SOTA-{#1}}}

\newcommand{\eat}[1]{}

\renewcommand{\algorithmicrequire}{\textbf{Input:}}
\renewcommand{\algorithmicensure}{\textbf{Output:}}

\newcommand{\stkout}[1]{\ifmmode\text{\sout{\ensuremath{#1}}}\else\sout{#1}\fi}


\copyrightyear{2025}
\acmYear{2025}
\setcopyright{acmlicensed}
\acmConference[WWW '25] {Proceedings of the ACM Web Conference 2025}{April 28--May 2, 2025}{Sydney, NSW, Australia.}
\acmBooktitle{Proceedings of the ACM Web Conference 2025 (WWW '25), April 28--May 2, 2025, Sydney, NSW, Australia}
\acmISBN{979-8-4007-1274-6/25/04}
\acmDOI{10.1145/3696410.3714828}%%
%% \BibTeX command to typeset BibTeX logo in the docs
\AtBeginDocument{%
  \providecommand\BibTeX{{%
    Bib\TeX}}}

%% Rights management information.  This information is sent to you
%% when you complete the rights form.  These commands have SAMPLE
%% values in them; it is your responsibility as an author to replace
%% the commands and values with those provided to you when you
%% complete the rights form.
% \setcopyright{acmcopyright}
% \copyrightyear{2018}
% \acmYear{2018}
% \acmDOI{XXXXXXX.XXXXXXX}

%% These commands are for a PROCEEDINGS abstract or paper.
% \acmConference[Conference acronym 'XX]{Make sure to enter the correct
%   conference title from your rights confirmation emai}{June 03--05,
%   2018}{Woodstock, NY}
%%
%%  Uncomment \acmBooktitle if the title of the proceedings is different
%%  from ``Proceedings of ...''!
%%
%%\acmBooktitle{Woodstock '18: ACM Symposium on Neural Gaze Detection,
%%  June 03--05, 2018, Woodstock, NY}
% \acmPrice{15.00}
% \acmISBN{978-1-4503-XXXX-X/18/06}


%%
%% Submission ID.
%% Use this when submitting an article to a sponsored event. You'll
%% receive a unique submission ID from the organizers
%% of the event, and this ID should be used as the parameter to this command.
%%\acmSubmissionID{123-A56-BU3}

%%
%% For managing citations, it is recommended to use bibliography
%% files in BibTeX format.
%%
%% You can then either use BibTeX with the ACM-Reference-Format style,
%% or BibLaTeX with the acmnumeric or acmauthoryear sytles, that include
%% support for advanced citation of software artefact from the
%% biblatex-software package, also separately available on CTAN.
%%
%% Look at the sample-*-biblatex.tex files for templates showcasing
%% the biblatex styles.
%%

%%
%% The majority of ACM publications use numbered citations and
%% references.  The command \citestyle{authoryear} switches to the
%% "author year" style.
%%
%% If you are preparing content for an event
%% sponsored by ACM SIGGRAPH, you must use the "author year" style of
%% citations and references.
%% Uncommenting
%% the next command will enable that style.
%%\citestyle{acmauthoryear}


%%
%% end of the preamble, start of the body of the document source.
\begin{document}


%%
%% The "title" command has an optional parameter,
%% allowing the author to define a "short title" to be used in page headers.
\title{SAMGPT: Text-free Graph Foundation Model for Multi-domain Pre-training and Cross-domain Adaptation}


\author{Xingtong Yu}
\affiliation{%
 \institution{Singapore Management University}
 \city{Singapore}
  \country{Singapore}}
\email{xingtongyu@smu.edu.sg}

\author{Zechuan Gong}
\affiliation{%
 \institution{University of Science and Technology of China}
 \city{Hefei}
  \country{China}}
\email{gongzechuan@mail.ustc.edu.cn}

\author{Chang Zhou}
\affiliation{%
 \institution{University of Science and Technology of China}
 \city{Hefei}
  \country{China}}
\email{zhouchang21sy@mail.ustc.edu.cn}

\author{Yuan Fang$^{*}$}
\affiliation{%
  \institution{Singapore Management University}
  \city{Singapore}
  \country{Singapore}}
\email{yfang@smu.edu.sg}

\author{Hui Zhang$^*$}
\affiliation{%
  \institution{University of Science and Technology of China}
  \city{Hefei}
  \country{China}}
\email{fzhh@ustc.edu.cn}

\thanks{
    $^*$ Corresponding authors.
}

%%
%% By default, the full list of authors will be used in the page
%% headers. Often, this list is too long, and will overlap
%% other information printed in the page headers. This command allows
%% the author to define a more concise list
%% of authors' names for this purpose.
\renewcommand{\shortauthors}{Xingtong Yu, Zechuan Gong, Chang Zhou, Yuan Fang, and Hui Zhang}


\begin{abstract}
Graphs are able to model interconnected entities in many online services, supporting a wide range of applications on the Web. This raises an important question: How can we train a graph foundational model on multiple source domains and adapt to an unseen target domain? A major obstacle is that graphs from different domains often exhibit divergent characteristics. Some studies leverage large language models to align multiple domains based on textual descriptions associated with the graphs, limiting their applicability to text-attributed graphs. For text-free graphs, a few recent works attempt to align different feature distributions across domains, while generally  neglecting structural differences. In this work, we propose a novel Structure Alignment framework for text-free Multi-domain Graph Pre-Training and cross-domain adaptation (\model). It is designed to learn multi-domain knowledge from graphs originating in multiple source domains, which can then be adapted to address applications in an unseen target domain. Specifically, we introduce a set of \emph{structure tokens} to harmonize structure-based aggregation across source domains during the pre-training phase. Next, for cross-domain adaptation, we design dual prompts, namely, \emph{\op} and \emph{\cp}, which adapt unified multi-domain structural knowledge and fine-grained, domain-specific information, respectively, to a target domain. Finally, we conduct comprehensive experiments on seven public datasets to evaluate and analyze the effectiveness of \model.
\end{abstract}

%%
%% The code below is generated by the tool at http://dl.acm.org/ccs.cfm.
%% Please copy and paste the code instead of the example below.
%%
\begin{CCSXML}
<ccs2012>
   <concept>
       <concept_id>10002951.10003260.10003277</concept_id>
       <concept_desc>Information systems~Web mining</concept_desc>
       <concept_significance>500</concept_significance>
       </concept>
   <concept>
       <concept_id>10002951.10003227.10003351</concept_id>
       <concept_desc>Information systems~Data mining</concept_desc>
       <concept_significance>500</concept_significance>
       </concept>
   <concept>
       <concept_id>10010147.10010257.10010293.10010319</concept_id>
       <concept_desc>Computing methodologies~Learning latent representations</concept_desc>
       <concept_significance>500</concept_significance>
       </concept>
 </ccs2012>
\end{CCSXML}

\ccsdesc[500]{Information systems~Web mining}
\ccsdesc[500]{Information systems~Data mining}
\ccsdesc[500]{Computing methodologies~Learning latent representations}

%%
%% Keywords. The author(s) should pick words that accurately describe
%% the work being presented. Separate the keywords with commas.
\keywords{Graph learning, foundation models, multi-domain pre-training, prompt learning, few-shot learning.}

% \received{20 February 2007}
% \received[revised]{12 March 2009}
% \received[accepted]{5 June 2009}

%%
%% This command processes the author and affiliation and title
%% information and builds the first part of the formatted document.
\maketitle

% !TEX root = ../main.tex

Large Language Models (LLMs) have shown remarkable capabilities on numerous tasks in Natural Language Processing (NLP), 
ranging from language understanding to generation \cite{bubeck2023sparks, achiam2023gpt,team2023gemini, dubey2024llama}. The huge success of LLMs comes with important challenges to deploy them due to their massive size and computational costs. For instance,  Llama-3-405B \cite{dubey2024llama} requires 780GB of storage in half precision (FP16) and hence multiple high-end GPUs are needed just for inference. \textit{Model compression} has emerged as an important line of research to reduce the costs associated with deploying these foundation models. In particular, neural network pruning \cite{obd, hassibi1992second, benbaki2023fast}, where model weights are made to be sparse after training, has garnered significant attention. Different sparsity structures (Structured, Semi-Structured and Unstructured) obtained after neural network pruning result in different acceleration schemes. \textit{Structured pruning} removes entire structures such as channels, filters, or attention heads \cite{lebedev2016fast,wen2016learning,voita2019analyzing,el2022data} and readily results in acceleration as model weights dimensions are reduced. \textit{Semi-Structured pruning}, also known as, N:M sparsity \cite{zhou2021learning} requires that at most $N$ out of $M$ consecutive elements are non-zero elements. Modern NVIDIA GPUs provide support for 2:4 sparsity acceleration. \textit{Unstructured pruning} removes individual weights \cite{han2015learning, guo2016dynamic} from the model's weights and requires specialized hardware for acceleration. For instance, DeepSparse \cite{kurtic2022optimal, pmlr-v119-kurtz20a, DBLP:journals/corr/abs-2111-13445} provide CPU inference acceleration for unstructured sparsity.\\
Specializing to LLMs, one-shot pruning~\cite{meng2024alps, frantar2023sparsegpt, sun2023simple, zhang2023dynamic}, where one does a single forward pass on a small amount of calibration data, and prunes the model without expensive fine-tuning/retraining, is of particular interest. This setup requires less hardware requirements. For instance, \citet{meng2024alps} show how to prune an OPT-30B \cite{opt} using a single consumer-level V100 GPU with 32GB of CUDA memory, whereas full fine-tuning of such model using Adam \cite{kingma2014adam} at half-precision requires more than 220GB of CUDA memory.

Although one-shot pruning has desirable computational properties, it can degrade models' predictive and generative performance. To this end, recent work has studied extensions of model pruning to achieve smaller utility drop of model performance from compression. 
% Multiple one-shot methods have been developed in quantization \cite{frantar2022gptq, frantar2023sparsegpt, lin2024awq, behdin2023quantease, dettmers2023spqr} and neural network pruning \cite{frantar2023sparsegpt, meng2024alps, zhang2024oats}, which is closer to this paper's line of research. These one-shot methods do not require retraining--which is extremely expensive for models of the size of Llama-3-405B-- and work as resource-saving techniques that retain the model's performance. 

An interesting compression mechanism in the field of \textit{model compression} is the Sparse plus Low-Rank Matrix-Decomposition problem which aims to approximate model's weights by a sparse component plus a low-rank component~\cite{hintermuller2015robust, candes2011robust, lin2011linearized, 5394889, zhou2011godec, JMLR:v24:21-1130, NIPS2014_443cb001, yu2017compressing, li2023losparse}. Specializing to LLMs,~\citet{zhang2024oats} propose OATS 
%that addresses this type of %compression and 
that outperforms pruning methods for the same compression ratio (number of non-zero elements) on a wide range of LLM evaluation benchmarks (e.g. perplexity in Language generation). 

OATS \cite{zhang2024oats} is however a matrix decomposition algorithm inspired from a pruning algorithm Wanda \cite{sun2023simple}. Wanda has been designed as a relaxation/approximation of another state-of-the-art pruning algorithm SparseGPT \cite{frantar2023sparsegpt}. While Wanda has been found to be extremely useful and efficient in practice, recent work \cite{meng2024alps} show results where Wanda fails for high-sparsity regimes. In this paper, we provide a unified optimization framework to decompose pre-trained model weights into sparse plus low-rank components based on a layer-wise loss function. Our framework is modular and can incorporate different pruning and matrix-decomposition algorithms (developed independently in different contexts).
%under the umbrella of the local %layer-wise reconstruction error; 
Similar to~\cite{meng2024alps} we observe that our optimization-based framework results in models with better model utility-compression tradeoffs. The difference is particularly pronounced for higher compression regimes. 
%especially for higher compression %budgets, where SOTA methods 
% Our numerical results also show similar findings to \citet{meng2024alps} where high-sparsity significantly degrades the performance of approximation-based optimization methods like OATS.

Concurrently, in a different and complementary line of work,~\citet{mozaffari2024slope} have open-sourced highly-specialized CUDA kernels designed for N:M sparse \cite{zhou2021learning} plus low-rank matrix decompositions that result in significant acceleration and memory reduction for the pre-training of LLMs.
We note that our focus here is on improved algorithms for one-shot sparse plus low-rank matrix decompositions for foundation models with billions of parameters which is different from the work of \citet{mozaffari2024slope} that focuses on accelerating the pre-training of LLMs. The designed CUDA kernels \cite{mozaffari2024slope} can be exploited in our setting for faster acceleration and reduced memory footprint during inference.





% \textbf{Summary of approach and contributions:} We propose \ourmethod: an accurate and scalable framework for Sparse plus Low-Rank Matrix Decomposition for LLMs. Following the previous work on one-shot pruning and model compression, we pursue a layerwise approach. In particular, the reconstruction error resulting from compression in the output of each layer is minimized, under the compression constraints (i.e., sparsity and low-rank constraints).

\textbf{Summary of approach.\,\,\,\,} Our framework is coined \ourframework: \underline{H}ardware-\underline{A}ware (Semi-\underline{S}tructured) \underline{S}parse plus \underline{L}ow-rank \underline{E}fficient \& approximation-\underline{free} matrix decomposition for foundation models.

Hardware-aware refers to the fact that we mostly focus on a N:M sparse \cite{zhou2021learning} plus low-rank decomposition, for which acceleration on GPUs is possible, although \ourframework supports any type of sparsity pattern (unstructured, semi-structured, structured) in the sparsity constraint. Approximation-free refers to the fact that we directly minimize the local layer-wise reconstruction error introduced in \cref{eq:matrix-decomposition}, whereas we show prior work minimizes an approximation of this objective.

%Our unified framework introduces a well-posed 
%%As a part of our proposed framework, we consider an 
%%optimization form
We formulate the compression/decomposition task as a clear optimization problem; we minimize a local layer-wise reconstruction objective where the weights are given by the sum of a sparse and low-rank component. 
%%%of dense model weights under the  
%This optimization problem is decoupled into a sparse minimization subproblem and a low-rank minimization subproblem. 
We propose an efficient Alternating-Minimization approach that scales to models with billions of parameters relying on 
two key components: one involving sparse minimization (weight sparsity) and the other involving a low-rank optimization. 
For each of these subproblems 
we discuss how approximations to the optimization task can retrieve prior algorithms.
%the introduced subproblems, 
%we consider approximations to the minimization objective and retrieve different algorithms from related works given different %approximations.

% We provide an efficient and scalable algorithm based on Alternating-Minimization that does not rely on any approximation at the objective minimization level. 
% While \ourframework supports any sparsity pattern (unstructured, semi-structured, structured) in the sparsity constraint, we mostly focus on N:M sparsity \cite{zhou2021learning}, to make the decomposition Hardware-aware, as \citet{mozaffari2024slope} show how to get acceleration on modern GPUs for N:M sparse plus low-rank decomposition.

We note that \ourframework~differs from prior one-shot (sparse) pruning methods~\cite{frantar2023sparsegpt, meng2024alps, benbaki2023fast} as we seek a sparse plus low-rank decompositon of weights.
%%%%%introducing the low-rank component. 
Additionally, it differs from prior one-shot sparse plus low-rank matrix decomposition methods~\cite{zhang2024oats}
%by considering an approximation-free minimization approach of the 
as we directly minimize the local layer-wise reconstruction objective introduced in \cref{eq:matrix-decomposition}.

Our main \textbf{contributions} can be summarized as follows.
\begin{compactitem}
    \item We introduce \ourframework a unified one-shot LLM compression framework that scales to models with billions of parameters where we directly minimize the local layer-wise reconstruction error subject to  a sparse plus low-rank matrix decomposition of the pre-trained dense weights. 
    %    formulates a sparse plus low-rank matrix decomposition as an optimization problem with a local layer-wise reconstruction objective. We discuss approximations of this objective and show that OATS a popular method is recovered in a particular approximation.

    
    \item \ourframework uses an Alternating-Minimization approach that iteratively minimizes a Sparse and a Low-Rank component. \ourframework uses a given pruning method as a plug-in for the subproblem pertaining to the sparse component. Additionally, it uses Gradient-Descent type methods for the subproblem pertaining to the Low-Rank component.
    
    % \item In the subproblem pertaining to the sparse component, a rewrite of the optimization formulation shows that one can use any pruning algorithm, that minimizes the layer-wise reconstruction error, as a plug-in to sparsify the weights. We choose to show results for the algorithm SparseGPT.
    
    % In this pruning subproblem, we also enhance the performance of \ourmethod by exploiting the invariance of the Hessian--of the layer-wise reconstruction error--in each subproblem of the Alternating Minimization procedure, for a given layer. In particular, we use a pre-processing step that computes and stores the Hessian inverse--of the objective--, which is then passed to the deployed pruning algorithm (e.g. SparseGPT). 
    % \item In the subproblem  pertaining to the Low-Rank component, we give a theoretical closed form solution to the subproblem.
    % which does not scale to problems with billions of parameters. 
    % We also present a more tractable first-order optimization method for a reparametrization of the the low-rank problem, which is scalable to models with billions of parameters.
    
    % as $\bfUVt$ and use first-order optimization methods to minimize the layer-wise reconstruction objective.

    \item We discuss how special cases of our framework relying on specific approximations of the objective retrieve popular methods such as OATS, Wanda and MP --- \cite{zhang2024oats, sun2023simple,han2015learning, sze2020efficient}. This provides valuable insights into the underlying connections across different methods. 

    \item \ourframework improves upon state-of-the-art methods for one-shot sparse plus low-rank matrix decomposition. 
    For the Llama3-8B model with a 2:4 sparsity component plus a 64-rank component decomposition, \ourframework reduces the test perplexity by $12\%$ for the WikiText-2 dataset and reduces the gap (compared to the dense model) of the average of eight popular zero-shot tasks by $15\%$ compared to existing methods.
\end{compactitem}




\section{Related Work}
\label{sec:related-work}

We give a brief review of data compression for volumetric data. 
We then discuss the use of contour trees in topological data analysis, followed by related work for topology-preserving compression techniques.   

\para{Lossy compression.} 
Lossless compression techniques allow the original data to be perfectly reconstructed, 
but they usually suffer from limited compression ratios (less than $2\times$ according to~\cite{son2014data}) in scientific data and thus are not practical. 
Lossy compression is an alternative way to reduce the unprecedented  size of scientific data. 
Traditional lossy techniques such as JPEG/JPEG2000 leverage wavelet theories and bit plane encoding to compress image data, but they are not adept at dealing with multidimensional scientific data in floating-point format. 
Recently, there has been an increasing trend to leverage deep learning techniques, such as the autoencoder~\cite{le2023hierarchical} and implicit neural representation (INR)~\cite{lu2021compressive}, for data compression.
An autoencoder is a neural network composed of two components: an encoder and a decoder. 
The encoder is trained to produce low-dimensional representations of the input data, whereas the decoder is trained to reconstruct the original input data from the output of the encoder. 
An INR model trains a small neural network that can be used to recreate the ground truth. 
The neural network itself is shipped as a compressed file, and to decompress it, one must simply evaluate the network on an appropriate input. 
One notable INR model for volumetric scalar fields is Neurcomp~\cite{lu2021compressive}.
Recently, spatial super-resolution (SSR) models have employed neural networks to accurately upscale low-resolution representation of data as a form of interpolation. 
Several volumetric scalar field compressors incorporate SSR models, such as SSR-TVD~\cite{han2020ssr} and the deep hierarchical model~\cite{wurster2022deep}.
Unfortunately, these general lossy techniques lack precise error control on the data, which limits their use on scientific data.


Error-controlled lossy compressors~\cite{lindstrom2014fixed,ballester2019tthresh,zhao2021optimizing,lakshminarasimhan2013isabela} have been proposed and leveraged by the scientific computing community to reduce the data size while controlling the distortion in the decompressed data. 
In general, such compressors can be categorized into transform-based and prediction-based. 
Transform-based lossy compressors rely on domain transforms for data decorrelation. 
For instance, ZFP~\cite{lindstrom2014fixed} divides data into small blocks and then compresses each block independently. The compression procedure inside each block includes exponent alignment for fixed point conversion, a near-orthogonal domain transform, and embedded encoding. 
TTHRESH~\cite{ballester2019tthresh} is another transform-based compressor that leverages singular value decomposition (SVD) to improve the decorrelation efficiency for high-dimensional data.

Prediction-based compressors employ prediction methods such as interpolation to approximate the ground truth. The differences between original and predicted data are quantized and then encoded using entropy encoding and lossless techniques. 
ISABELA~\cite{lakshminarasimhan2013isabela}, as one of the pioneering error-controlled prediction-based compressors, uses B-splines to predict data. 
SZ3~\cite{liang2022sz3,zhao2021optimizing,liang2018error}, the most recent general release in the SZ compressor family, uses a combination of a Lorenzo predictor~\cite{ibarria2003out}, cubic spline interpolation, and linear interpolation. 
In addition, AE-SZ~\cite{liu2021exploring} is proposed as a variation of SZ that incorporates autoencoders in the prediction pipeline.

\para{Contour trees.} Our augmented compressors aim to preserve the contour tree of an input scalar field. 
Contour trees capture the relationships among contours of scalar fields. 
They have been used to support data analysis and visualization tasks across diverse disciplines, such as astronomy \cite{rosen2021using}, fluid dynamics \cite{aydogan2014characterization}, and medicine \cite{aydogan2013analysis, wang2018fast, szymczak2010categorical}. 
They have also been incorporated into algorithms in computer vision \cite{mizuta2005description} and visualization \cite{zhou2009automatic, kopp2022temporal} for interactive exploration of contours.  

\para{Topology-preserving compression.} 
To the best of our knowledge, only three compressors have been developed for scalar field compression with topological guarantees. 
The first compressor was developed by Soler et al. \cite{soler2018topologically}. We shall refer to it as TopoQZ. 
TopoQZ allows the user to specify a single parameter $\varepsilon$. 
It preserves all critical point pairs with finite persistence greater than $\varepsilon$ and eliminates all critical points with persistence less than $\varepsilon$. 
TopoQZ is not designed to perfectly preserve the contour tree. Therefore, the locations of preserved critical points may shift slightly during compression, and the connectivity of the critical points in the contour tree may be altered. 
TopoQZ can also guarantee that the reconstructed values differ from the ground truth at most by a user-specified error bound $\xi$. It is required that $\xi > \varepsilon$.
TopoQZ is currently implemented in the Topology Toolkit \cite{TiernyFavelierLevine2017, MasoodBudinFalk2021, leguillou_tvcg24}. That implementation couples TopoQZ with ZFP \cite{lindstrom2014fixed}, which improves the smoothness of the data but introduces additional pointwise error.

Another topology-preserving compressor is TopoSZ \cite{yan2023toposz}. 
TopoSZ modifies the classic SZ pipeline to perfectly preserve the contour tree of the ground truth data up to the persistence threshold of $\varepsilon$. 
That is, the contour tree of the output of TopoSZ will be equal to that of the ground truth after both datasets have been topologically simplified with a persistence threshold of $\varepsilon$. 
TopoSZ also allows the user to impose a strict error-bound $\xi$ (and allows $\xi \leq \varepsilon$). When compared with TopoQZ, TopoSZ yields generally higher compression ratios and reconstruction quality, although the algorithm takes longer to execute. 
Our general framework borrows some elements from the TopoSZ pipeline. However, our framework differs significantly from TopoSZ due to two technical innovations: progressive bound tightening and logarithmic-scaling quantization (see \cref{sec:method} for details).  

Most recently, Li et al. developed mSZ~\cite{li2024msz} that augments an existing lossy compressor to compress a 2D/3D scalar field while preserving its piecewise-linear (PL) Morse--Smale segmentation \cite{edelsbrunner2001hierarchical,edelsbrunner2003morse},~i.e.,~a partition of the data domain based on the Morse--Smale complex. 
In comparison to the contour tree that is based on the level sets of a scalar field, a Morse--Smale complex is a different topological descriptor based on the gradient behavior of a scalar field.   
Because our framework instead preserves contour trees and does not consider the gradients in its pipeline, mSZ is not directly comparable to our work. 

Finally, even though it does not preserve any common topological descriptor, cpSZ \cite{liang2022toward}---a variation of SZ---preserves the critical points of a vector field. cpSZ also introduces a log-scale quantization technique to store different error bounds for individual points.

\section{Preliminaries}\label{sec.preliminaries}
\stitle{Graph.}
A graph is defined as \( G = (V, E) \), where \( V \) is the set of nodes and \( E \) is the set of edges. The nodes are associated with a feature matrix $\mathbf{X} \in \mathbb{R}^{|V| \times d}$, where \( \vec{x}_v \in \mathbb{R}^d \) is a row of $\mathbf{X}$ representing the feature vector for node \( v \in V \). For a collection of multiple graphs, we define it as \( \mathcal{G} = \{ G_1, G_2, \dots, G_N \} \).

\begin{figure*}[t]
\centering
\includegraphics[width=1\linewidth]{figures/framework.pdf}
\caption{Overall framework of \model.}
\label{fig.framework}
\end{figure*}


\stitle{Graph encoder.}
Towards graph representation learning, one of the most widely used families of graph encoders is graph neural networks (GNNs), which generally rely on message passing to capture structural knowledge \cite{wu2020comprehensive,zhou2020graph}. Each node updates its representation by aggregating information from its neighbors, and stacking multiple GNN layers enables iterative message propagation across the graph.
Formally, let $\vec{H}^l$ denote the embedding matrix at the $l$-th layer, where each row $\vec{h}_i^l$ represents the embedding of node $v_i$. This matrix is iteratively computed using the embeddings from the preceding layer:
\begin{equation}\label{eq.gnn}
\vec{H}^l = \textsc{MP}(\vec{H}^{l-1},G;\theta^l),
\end{equation}
where $\textsc{MP}(\cdot)$ is the message passing function, and $\theta^l$ represents the learnable parameters of the graph encoder at layer $l$. The initial embedding matrix, $\vec{H}^0$, is the input feature matrix, i.e., $\vec{H}^0=\vec{X}$. The output after a total of $L$ layers is then $\vec{H}^L$; for brevity we simply write $\vec{H}$. We abstract the multi-layer encoding process as 
\begin{align}
    \{\vec{H}^1, \vec{H}^2, \cdots, \vec{H}^L\} = \textsc{GraphEncoder}(\vec{X},G;\Theta),
\end{align}
where $\{\vec{H}^1, \vec{H}^2, \cdots, \vec{H}^L\}$ denotes the embedding matrix of the each layer of the graph encoder, respectively. $\Theta=(\theta^1,\ldots,\theta^L)$ is the collection of weights across the layers.

\stitle{Pre-training.}
As stated by prior studies \cite{yu2024generalized,yu2024non}, all contrastive pre-training task on graphs \cite{liu2023graphprompt,velickovic2019deep,you2020graph} can be unified under the task template of similarity calculation.
Formally, the unified pre-training objective is defined as follows:
\begin{equation}\label{eq:generalized_loss}
     \bL(\Theta)= -\sum_{o\in \bT_\text{pre}}\ln\frac{\sum_{a\in Pos_o}\exp(\text{sim}(\vec{h}_{a}, \vec{h}_{o})/\tau)}{\sum_{b\in Neg_o}\exp(\text{sim}(\vec{h}_{b}, \vec{h}_{o})/\tau)},
\end{equation}
where \( Pos_o \) and \( Neg_o \) denote the sets of positive and negative samples for a target instance \( o \), respectively. \( \vec{h}_o \) represents the embedding of the target instance, while \( \vec{h}_a \) and \( \vec{h}_b \) correspond to the embeddings of positive and negative samples. The hyperparameter \( \tau \) controls the temperature scaling in the similarity computation. In our framework, we follow previous work \cite{liu2023graphprompt,yu2024generalized} by employing similarity calculation as the task template and using link prediction as the pre-training task.

\stitle{Problem definition.}
In this work, we explore chain-of-thought (CoT) prompt learning framework for text-free graphs. We focus on two widely used tasks in graph learning: node classification and graph classification, in few-shot scenarios.
For node classification, given a graph \( G = (V, E) \) with a set of node classes \( Y \), each node \( v_i \in V \) is associated with a label \( y_i \in Y \). In contrast, graph classification considers a collection of graphs \( \mathcal{G} \), where each graph \( G_i \in \mathcal{G} \) is assigned a class label \( Y_i \in Y \).
In the few-shot setting, only \( m \) labeled examples per class are available (e.g., \( m \leq 10 \)), a paradigm defined as \( m \)-shot classification \cite{liu2023graphprompt,yu2024generalized}.

\section{Proposed Approach: \model}

In this section, we present \model, beginning with an overview and then delving into the details of multi-domain pre-training and cross-domain adaptation.

\subsection{Overall Framework}
\model\ consists of two phases: multi-domain pre-training, and cross-domain adaptation, as shown in Fig.~\ref{fig.framework}.

In the pre-training phase, as depicted in Fig.~\ref{fig.framework}(a), we first align the feature distributions from multiple source domains following previous work \cite{zhao2024all,yu2024text}. 
Next, we introduce a set of \textit{structure tokens} designed to align the structural distributions across diverse domains. These tokens are domain-specific and are integrated into each layer of the graph encoder, modifying the structure-based aggregation at each layer. 
Finally, the structure token-enhanced graph encoder is pre-trained using a self-supervised loss  \cite{liu2023graphprompt}.
%based on a universal task template \cite{liu2023graphprompt}.

In the adaptation phase, as shown in Fig.~\ref{fig.framework}(b), we first align the feature dimension of the target domain with that of the source domains. Then, we introduce \textit{dual prompts}. The first type, \emph{\op}, are learnable vectors that integrate the target domain with the holistic structural knowledge from all source domains. The second type, \emph{\cp}, comprise learnable mixtures of pre-trained structure tokens that incorporate domain-specific topological information tailored to the target domain. These prompts are applied to each layer of the graph encoder to adjust the structure-based aggregation, while keeping the pre-trained weights of the graph encoder frozen. 
%facilitating cross-domain adaptation with lightweight tuning.

\subsection{Multi-domain Graph Pre-training with Structure Alignment}

As defined in Sect.~\ref{sec.pre.multi-domain}, we are given a set of pre-training graphs from multiple source domains, $\bG_S$. As both the features and structures of these domains can exhibit divergent distributions, effective integration of these multi-domain graphs requires aligning both.
As our work focuses on structure alignment, we follow previous feature alignment methods  \cite{zhao2024all,yu2024text}, as outlined in the preliminaries.  





\stitle{Structure alignment.}
Recall that in the graph encoder, node representations are updated layer-wise through a structure-based aggregation. Each layer captures different levels of structural information. For example, the first layer aggregates one-hop neighborhood information, while the second layer incorporates a broader two-hop neighborhoods. These layer-wise structural patterns may vary significantly across domains. 

Therefore, to unify the structural characteristics in multiple source domains, we introduce learnable \emph{structure tokens}. For each domain $D_{S_i}$, we inject a series of structure tokens $\bT_{S_i}=\{\Vec{t}^l_{S_i}:  l\in \{1,\ldots,L\}\}$ into the graph encoder, where $L$ denotes the number of layers. Specifically, when encoding the graph $G_i=(V_i,E_i,\tilde{\mathbf{X}}_i$) in $D_{S_i}$, we assign structure token $\Vec{t}^l_{S_i}$ to the $l$-th layer, guiding structure-based aggregation:
\begin{align}\label{eq:structure_aggregation}
    \vec{h}^l_{v} = \mathtt{Aggr}(\vec{h}^{l-1}_{v}, \{\Vec{t}^l_{S_i} \odot \vec{h}^{l-1}_{u} : u\in\bN_v\}; \theta^l),~\forall v\in V_i,
\end{align}
%\begin{align}\label{eq:structure_aggregation}
%    \vec{h}^l_{S_i,v} = \mathtt{Aggr}(\vec{h}^{l-1}_{S_i,v}, \{\Vec{t}^l_{S_i} \odot \vec{h}^{l-1}_{S_i,u} : u\in\bN_v\}; \theta^l),
%\end{align}
where $\odot$ represents element-wise multiplication. Note that the graph encoders for feature alignment and structure alignment on all graphs share the same parameters $\Theta$.
Let $\vec{H}^{\mathtt{SAL}}_{i}$ denote the structure-aligned output node embedding matrix for $G_i$ in $D_{S_i}$, following the aggregation in Eq.~\eqref{eq:structure_aggregation}.
In general, each source domain is attached with its own set of structure tokens,
which are applied to modify the aggregation on the graph in the corresponding domain. By stacking the structure-aligned output matrix across graphs in all domains, we obtain the overall structure-aligned embedding matrix,  
$\Vec{H}^\mathtt{SAL}= \mathtt{Stack}(\vec{H}^{\mathtt{SAL}}_{1},\ldots,\vec{H}^{\mathtt{SAL}}_{K})$.
%, defined as $\bT=\{\bT_{S_i}: \forall S_i \text{ s.t. } D_{S_i} \in \bD_{S}\}$. By applying these structure tokens to all source domains, we obtain the structure aligned embedding matrix  $\Vec{H}^\mathtt{SAL}_{S}$. These structure tokens modify the message passing of various domains in a structurally synergistic manner, allowing the pre-trained model to extract both a holistic and domain-specific structural knowledge from a wide range of source domains.

Finally, we fuse $\Vec{H}^\mathtt{SAL}$ with $\Vec{H}^{\mathtt{FAL}}$ in Eq.~\eqref{eq.pre-train.feature-align}  to obtain the multi-domain node embedding matrix $\vec{H}$, incorporating both feature and structure alignment, as shown below.
\begin{align}\label{eq.source-fusion}
    \vec{H}^{\mathtt{AL}} =\Vec{H}^{\mathtt{FAL}}+\alpha\Vec{H}^\mathtt{SAL},
\end{align}
where $\alpha>0$ is a hyperparameter.

\stitle{Pre-training loss.}\label{sec.pre-training}
We leverage a universal task template based on subgraph similarity calculation \cite{liu2023graphprompt,yu2023generalized}, which ensures compatibility across different tasks such as node classification and graph classification. %Note that our framework is flexible with other pre-training methods. 
As demonstrated in GraphPrompt+ \cite{yu2023generalized}, prevailing contrastive pre-training objectives can be unified under this template, making them suitable choices for the pre-training loss in \model. 
%This inherent compatibility of the pre-training task could enhance the model's ability to adapt seamlessly to diverse downstream tasks. 
In general, we can adopt the following form of contrastive loss in pre-training.
\begin{align}\label{eq:generalized_loss}
     \textstyle     \bL_\text{pre}(\bO;\Theta,\bT,\Psi)=-\sum_{o\in \bO}\ln\frac{\sum_{a\in \text{Pos}_o}\exp(\mathtt{sim}(\vec{h}_{a}, \vec{h}_{o})/\tau)}{\sum_{b\in \text{Neg}_o}\exp(\mathtt{sim}(\vec{h}_{b}, \vec{h}_{o})/\tau)},
\end{align}
where $\bO$ denotes the set of observed graph element in pre-training, $a\in\text{pos}_o,b\in\text{neg}_o$ represent the positive or negative instance of $o$, respectively, and $\vec{h}_o,\vec{h}_a,\vec{h}_b$ are their corresponding embeddings. 
Furthermore, $\mathtt{sim}(\cdot,\cdot)$ is a similarity function, such as cosine similarity \cite{rahutomo2012semantic} in our implementation, and $\tau>0$ is a temperature hyperparameter. 
Note that \model\ is flexible in the materialization of $o,a,b$ to realize different contrastive losses \cite{yu2023generalized}. Our experiments adopt GraphCL \cite{you2020graph}, where \( o \) is the original graph \( G \), and \( a,b \) represent two different augmentations of  \( G \).
Hence, $\vec{h}_o,\vec{h}_a,\vec{h}_b$ are the corresponding graph embeddings, which can be obtained through a readout operation \cite{liu2023graphprompt} on the aligned node embeddings in $\mathbf{H}^{\mathtt{AL}}$.
%\( o \) represents an augmented graph \( G' \), \( a \) is the original graph \( G \), and \( b \) is a another augmented graph of \( G'' \).

The pre-training loss is optimized by updating the weights of graph encoder $\Theta$, structure tokens across all source domains %$\bT=\{\bT_{S_i}: \forall S_i \text{ s.t. } D_{S_i} \in \bD_{S}\}$, 
$\bT=\{\bT_{S_1},\ldots,\bT_{S_K}\}$, and feature alignment parameters $\Psi$.
%, namely, $(\Theta_\text{pre},\bT_\text{pre},\Psi_\text{pre})=\arg\min_{\Theta,\bT,\Psi} \mathcal{L}_{\text{pre}}(\Phi_\text{pre};\Theta,\bT,\Psi)$, which are further employed in downstream adaptation.

% Let $\vec{h}_{S,v}$ denote the output embedding of node $v$ from the graph encoder, and $Z_v$ be a subgraph of node $v$ and $Z_G$ be the graph $G$. 
% We define $z_{S,x}$ as the embedding of $Z_x$ by readout $\{\vec{h}_{S,v}:v~\text{in}~Z_x\}$. %Note that a node's smallest subgraph is itself, where $z_{S,x}=\vec{h}_{S,x}$.
% The readout function can be sum pooling, mean pooling and more complex strategies. Then, we define the generalized pre-training loss as follows.
% \begin{align}\label{eq:generalized_loss}
%      \textstyle
%      \bL_\text{pre}(\Phi_\text{pre};\Theta,\bT,\Psi)=-\sum_{o\in \Phi_\text{pre}}\ln\frac{\sum_{a\in Pos_o}\exp(\mathtt{sim}(\vec{z}_{a}, \vec{z}_{o})/\tau)}{\sum_{b\in Neg_o}\exp(\mathtt{sim}(\vec{z}_{b}, \vec{z}_{o})/\tau)},
% \end{align}
% where $\Phi_\text{pre}$ denotes the pre-training datasets. $o$ is the target node/graph, $a,b$ represent the positive or negative samples of $o$. $\mathtt{sim}(\cdot,\cdot)$ is a similarity function, such as cosine similarity \cite{rahutomo2012semantic} in our experiments. $\tau$ is a temperature hyperparameter. In our experiment, we utilize GraphCL \cite{you2020graph} as the pre-training task, where \( o \) represents an augmented graph \( G \), \( a \) is the original graph \( G \), and \( b \) is a another augmented graph of \( G \) by a different augmentation.
% The pre-training loss is optimized by updating the weights of graph encoder $\Theta$, domain tokens $\bT$, and learnable parameters $\Psi$ in feature alignment method, namely, $(\Theta_\text{pre},\bT_\text{pre},\Psi_\text{pre})=\arg\min_{\Theta,\bT,\Psi} \mathcal{L}_{\text{pre}}(\Phi_\text{pre};\Theta,\bT,\Psi)$, which are further employed in downstream adaptation.


\subsection{Cross-domain Structure Adaptation}
Beyond multi-domain pre-training, another challenge lies in cross-domain adaptation. Given a model pre-trained on graphs $\bG_S$ from source domains $\bD_S$, we aim to adapt it to a downstream task on graphs $\bG_T$ from a target domain $D_T \notin \bD_S$. 
As this work focuses on structure adaptation, we directly apply previous work  \cite{yu2024text} for feature adaptation, as outlined in Sect.~\ref{sec:problem_definition}. 

For structure adaptation, we propose \emph{dual prompts}, consisting of \emph{\op} and \emph{\cp}. On one hand, the \op\ are designed to holistically utilize the pre-trained structural knowledge from all source domains. On the other hand, the \cp\ combine multi-domain structure tokens through a learnable mixture, adapting fine-grained, domain-specific structural knowledge to the target domain.


% As both the features and structures of these domains can exhibit divergent distributions, effective integration of these multi-domain graphs requires aligning both.
% As our work focuses on structure alignment, we simply follow previous works \cite{zhao2024all,yu2024few} for feature alignment. 

% , our objective extends to bridge the gap between source and target domains. Previous works solely leverage feature adaptation, failing to narrow the structural gap \cite{zhao2024all}. Therefore, we propose a novel structure adaptation strategy and integrate feature adaptation technique.

% For feature adaptation, we first employ the same dimension alignment method as used in the pre-training phase. Given a downstream graph $G=(V,E,\Vec{X}) \in \bG_T$ from target domain $D_T$, we transform its feature matrix to $\tilde{\Vec{X}}=\mathtt{DA}_T(\Vec{X})$. We then cooperate a feature adaptation technique to align the target domain feature semantics with the source domains as follow.
% \begin{align}\label{eq.feature-emb}
%     \Vec{H}_\text{feat} = \mathtt{GE}(\mathtt{SAG}(\bG,\tilde{\Vec{X}};\Gamma);\Theta_\text{pre}),
% \end{align}
% where $\mathtt{SAG}$ represents the feature adaptation method, and $\Gamma$ denotes the learnable parameters in $\mathtt{SAG}$.


\stitle{Holistic prompts.} 
To transfer the holistic multi-domain structural knowledge to a downstream task, we propose a set of \op\ designed to align the target domain \( D_T \) with the model pre-trained on the source domains \( \bD_S \). %Given a graph \( G_T = (V_T, E_T, \Vec{X}_T) \) from \( D_T \), 
Like any pre-training framework, we encode a downstream graph $G=(V,E,\tilde{\vec{X}})$ using the pre-trained graph encoder with frozen layer-wise weights $\Theta_\text{pre}=\{\theta^1_\text{pre},\ldots,\theta^L_\text{pre}\}$. 
However, the key difference is that we inject a series of learnable vectors \( \bP_\text{hol} = \{ \Vec{p}^1_\text{hol},  \ldots, \Vec{p}^L_\text{hol} \} \) as \op\ into the downstream structure-based aggregation:
\begin{align}\label{eq.open-prompt}
    \vec{h}^l_{v} = \mathtt{Aggr}(\vec{h}^{l-1}_{v}, \{\Vec{p}^l_\text{hol} \odot \vec{h}^{l-1}_{u} : u\in\bN_v\}; \theta^l_\text{pre}),~\forall v\in V.
\end{align}
The final layer outputs a holistic node embedding matrix for the downstream graph $G$, denoted as \( \vec{H}^\text{hol}\).
%\( \vec{H}^\text{hol}=\mathtt{STACK}(\vec{\tilde{h}}_{v}:~\forall v\in G_T) \).

\stitle{Specific prompts.}
In contrast to the \op, \cp\ are designed to adapt structural knowledge specific to each source domain. Since knowledge from related source domains is likely to be more applicable, it is essential to align the target domain with different source domains to varying extents, prioritizing the most relevant ones. Consequently, we define \cp\ as \( \bP_\text{spe} = \{ \Vec{p}^1_\text{spe}, \ldots, \Vec{p}^L_\text{spe} \} \), which will also be injected into different layers of the pre-trained graph encoder. Specifically, in the $l$-th layer, \( \Vec{p}^l_\text{spe} \) is a combination of \( \{ \vec{t}^l_{S_1},\ldots,\vec{t}^l_{S_K}\} \), the pre-trained structure tokens in the corresponding layer 
%\( \bT^l_\text{pre} = \{ t^l_{S_i} \mid \forall S_i, \text{s.t.}~D_{S_i} \in \bD_{S} \} \), 
 across all source domains $D_{S_i} \in \bD_{S}$.
%which are integrated into the \( l \)-th layer of the pre-trained graph encoder. %The \cp\ serve as intermediate prompts, balancing between hard and soft prompts \cite{liu2023pre}, by utilizing pre-trained tokens alongside learnable fusion coefficients in downstream tasks. 
Formally, we define
\begin{align}\label{eq.specific-prompt-generation}\textstyle
\Vec{p}^l_\text{spe} = \sum_{i=1}^K \lambda^l_i \Vec{t}^l_{S_i},
\end{align}
where \( \Lambda^l = \{ \lambda^l_1, \ldots, \lambda^l_K \} \) are learnable coefficients. Thus, the full set of learnable parameters for the \cp\ is \( \Lambda = \{ \Lambda^1, \ldots, \Lambda^L \} \). Subsequently, \cp\ modify the structure-based aggregation in the same way as in Eq.~\eqref{eq.open-prompt}, while freezing the pre-trained weights of the graph encoder. Similarly, we denote the output node embedding matrix based on the specific prompts as \( \vec{H}^\text{spe} \).

\stitle{Prompt tuning.}
To leverage both holistic multi-domain and domain-specific structural knowledge from the pre-trained model, we fuse the output embedding matrices obtained via \op\ and \cp\ as follows.
\begin{align}\label{eq.structure-emb}
    \vec{H}^\mathtt{SAD} =\vec{H}^\text{hol}+\beta\vec{H}^\text{spe},
\end{align}
where $\beta>0$ is a hyperparameter. 
Further incorporating feature adaptation in Eq.~\eqref{eq.feature-emb}, 
we obtain the overall node embedding matrix with both feature and structure adaptations, given by
%and structure-level adapted embeddings $\vec{H}^\text{SAD}$ from Eq.~(\ref{eq.structure-emb}) as follow.
\begin{align}\label{eq.down-fusion}
    \vec{H}^\mathtt{AD} =\vec{H}^\mathtt{FAD}+\alpha\vec{H}^\mathtt{SAD}.
\end{align}
Here, $\alpha$ is the same hyperparameter used in Eq.~\eqref{eq.source-fusion}, as both share the objective of integrating the feature and structure counterparts.

For downstream node and graph classification tasks, the loss function \( \mathcal{L}_\text{down} \) is formulated based on the same task template with subgraph similarity \cite{liu2023graphprompt}, akin to the pre-training loss \( \mathcal{L}_\text{pre} \). Let \( \Omega = \{(x_1, y_1), (x_2, y_2), \ldots\} \) represent the labeled training set, where each \( x_i \) is either a node or graph instance, and \( y_i \in Y \) is its respective class from the set \( Y \). Subsequently, we optimize the following cross-domain adaptation loss:
\begin{align}\label{eq.downstream_loss}\textstyle
    \bL_\text{down}(\Omega;\bP_\text{hol},\Lambda,\Gamma)=-\sum_{(x_i,y_i)\in \Omega}\ln\frac{\exp(\text{sim}(\vec{h}_{x_i},{\vec{h}}_{y_i})/\tau )}{\sum_{y\in Y}\exp(\text{sim}(\vec{h}_{x_i},{\vec{h}}_{y})/\tau )}.
\end{align}
Here, $\vec{h}_{x_i}$ represents the adapted embedding of the node or graph $x_i$ based on $\vec{H}^\mathtt{AD}$, 
where a readout operation on $\vec{H}^\mathtt{AD}$ is required if $x_i$ is a graph. 
Additionally, ${\vec{h}}_y$ denotes the prototype embedding for class $y$, which is calculated as the average embeddings of all training instances of class $y$. %For a node $v$, $\vec{\hat{h}}_{v}$ is a row of $\Vec{\hat{H}}$. 

We outline the key steps for prompt tuning in Algorithm~\ref{alg.prompt}, Appendix~\ref{app.alg} and assess its complexity in Appendix~\ref{complexity}.

\section{Experiments}

\subsection{Experiment Setting}

\noindent\textbf{Benchmarks and Evaluated Protocols.}
The statistic of  NumericBench is provided in Table~\ref{tab:data_stat}.
Also,
we set the exact answer for mixed-number-string dataset, 
set
the computed answer to two decimal places for arithmetic datasets, and  set the answer of each question as a single choice (e.g., A, B, or C) for other datasets to reliable evaluate LLMs~\citep{bai2024longbench}.
The evaluation metric is accuracy.

\begin{table*}[t]
	\centering
		\vspace{-2em}
	\setlength\tabcolsep{2pt}
	\footnotesize
	\caption{Evaluation of LLMs on numerical contextual retrieval, comparison, and summary tasks across number list, stock, and weather datasets. 
		Also, * indicates that scores are calculated based on a short subset of outputs, as these models cannot handle  long contexts and exhibit disruption when tested on longer instances.}
	\begin{tabular}{c|ccc|ccc|ccc|c}
		\toprule
		\multirow{2}{*}{\textbf{Model}} & \multicolumn{3}{c|}{\textbf{Retrieval}}                                                         & \multicolumn{3}{c|}{\textbf{Comparison}}                                                           & \multicolumn{3}{c|}{\textbf{Summary}}                                                           & \textbf{Logic}    \\ \cmidrule{2-11} 
		
		& \multicolumn{1}{c}{\textbf{Number}} & \multicolumn{1}{c}{\textbf{Stock}} & \textbf{Weather} & \multicolumn{1}{c}{\textbf{Number}} & \multicolumn{1}{c}{\textbf{Stock}} & \textbf{Weather} & \multicolumn{1}{c}{\textbf{Number}} & \multicolumn{1}{c}{\textbf{Stock}} & \textbf{Weather} & \textbf{Sequence} \\ \midrule
		
		\textbf{\texttt{Random}} & \multicolumn{1}{c}{12.5}                  & \multicolumn{1}{c}{12.5}               &          12.5        & \multicolumn{1}{c}{12.5}                  & \multicolumn{1}{c}{12.5}               &                 12.5 & \multicolumn{1}{c}{12.5}                  & \multicolumn{1}{c}{12.5}               &      12.5            &         12.5          \\ \midrule
		
		
		\textbf{\texttt{Llama-3.1-8B-Inst}}& \multicolumn{1}{c}{22.8}                  & \multicolumn{1}{c}{14.4}               &      13.5          & \multicolumn{1}{c}{19.5}                  & \multicolumn{1}{c}{11.7}               &     13.7            & \multicolumn{1}{c}{18.1}                  & \multicolumn{1}{c}{13.8}               &       13.9*          &       18.2            \\  
		
		\textbf{\texttt{Llama-3.1-70B-Inst}}& \multicolumn{1}{c}{37.3}                  & \multicolumn{1}{c}{17.4}               &     23.0             & \multicolumn{1}{c}{28.3}                  & \multicolumn{1}{c}{15.0}               &     28.7             & \multicolumn{1}{c}{24.7}                  & \multicolumn{1}{c}{16.4}               &       15.2           &     17.8              \\  
		
		\textbf{\texttt{Llama-3.3-70B-Inst}}& \multicolumn{1}{c}{44.4}                  & \multicolumn{1}{c}{19.4}               &      23.1            & \multicolumn{1}{c}{31.5}                  & \multicolumn{1}{c}{13.8}               &       35.8           & \multicolumn{1}{c}{26.3}                  & \multicolumn{1}{c}{16.8}               &   18.0               &     18.6              \\  
		
		\textbf{\texttt{Llama-3.1-405B-Inst}}& \multicolumn{1}{c}{44.6}                  & \multicolumn{1}{c}{26.8}               &          19.8        & \multicolumn{1}{c}{25.1}                  & \multicolumn{1}{c}{14.8}               &     29.8             & \multicolumn{1}{c}{32.9}                  & \multicolumn{1}{c}{17.0}               &    16.1              &     16.6              \\  
		
		\textbf{\texttt{Llama-3.1-Nemotron-70B-Inst}}& \multicolumn{1}{c}{41.6}                  & \multicolumn{1}{c}{19.3}               &        24.9          & \multicolumn{1}{c}{26.6}                  & \multicolumn{1}{c}{13.7}               &      33.6            & \multicolumn{1}{c}{29.4}                  & \multicolumn{1}{c}{16.5}               &     17.0             &     16.4              \\  
		
		\textbf{\texttt{Qwen2.5-7B-Inst}}& \multicolumn{1}{c}{20.2}                  & \multicolumn{1}{c}{17.3}               &    19.6              & \multicolumn{1}{c}{24.8}                  & \multicolumn{1}{c}{17.8}               &      18.8            & \multicolumn{1}{c}{18.5}                  & \multicolumn{1}{c}{11.7}               &     13.8             &    14.4               \\  
		\textbf{\texttt{Qwen2.5-72B-Inst}}& \multicolumn{1}{c}{28.8}                  & \multicolumn{1}{c}{41.4*}               &       12.4*           & \multicolumn{1}{c}{28.0}                  & \multicolumn{1}{c}{26.0*}               &       31.0*           & \multicolumn{1}{c}{31.9}                  & \multicolumn{1}{c}{18.8*}               &        16.4*          &      19.0             \\  
		\textbf{\texttt{GLM-4-Long}}& \multicolumn{1}{c}{26.5}                  & \multicolumn{1}{c}{19.5}               &       8.4           & \multicolumn{1}{c}{18.9}                  & \multicolumn{1}{c}{14.8}               &      21.6            & \multicolumn{1}{c}{20.8}                  & \multicolumn{1}{c}{10.8 }               &      10.5            &        17.6           \\  
		
				\textbf{\texttt{Deepseek-V3}}& \multicolumn{1}{c}{47.2}                  & \multicolumn{1}{c}{47.5}               &       10.9          & \multicolumn{1}{c}{27.0}                  & \multicolumn{1}{c}{22.5}               &       35.8          & \multicolumn{1}{c}{21.8}                  & \multicolumn{1}{c}{13.0}               &       15.1          &   15.8                \\  
		
		\textbf{\texttt{GPT-4o}}& \multicolumn{1}{c}{41.7}                  & \multicolumn{1}{c}{37.5}               &        15.4          & \multicolumn{1}{c}{30.6}                  & \multicolumn{1}{c}{33.0}               &       64.2           & \multicolumn{1}{c}{11.6}                  & \multicolumn{1}{c}{17.4}               &      16.5            &        14.6           \\ 
		

		
		\midrule
		
		\textbf{\texttt{Human Evaluation}}& \multicolumn{1}{c}{\textbf{100}}                  &  \multicolumn{1}{c}{\textbf{100}}               &      \textbf{100}            & \multicolumn{1}{c}{\textbf{100}}                  & \multicolumn{1}{c}{\textbf{100}}               &        \textbf{100}          & \multicolumn{1}{c}{\textbf{100}}                  & \multicolumn{1}{c}{\textbf{100}}               &          \textbf{100}        &               \textbf{52.6}   \\ \bottomrule
	\end{tabular}
	\label{tab:main_experiments}
\end{table*}
\begin{figure*}[t]
		\vspace{-1em}
	\centering 	
	\subfloat[Contextual  Retrieval]	
	{\centering\includegraphics[width=0.33\linewidth]{image/main_fig/retrieval-num-list.pdf}}
	\hfill
	\subfloat[Comparison]
	{\centering\includegraphics[width=0.33\linewidth]{image/main_fig/compare-num-list.pdf}}
	\subfloat[Summary]	
	{\centering\includegraphics[width=0.33\linewidth]{image/main_fig/summary-num-list.pdf}}
	\hfill
	\caption{Evaluation on short and long context on number list.}
	\label{fig:length_number}
	
\end{figure*}
\noindent\textbf{Evaluated Models.}
To comprehensively evaluate the retrieval and reasoning abilities of state-of-the-art and widely-used LLMs on numeric data, 
we benchmark over 10 popular LLMs with our constructed NumericBench, as follows.
\begin{itemize}[leftmargin=*]
	\item \textbf{The Llama Series~\citep{grattafiori2024llama3herdmodels}.} include Llama-3.1-8B Instruct, Llama-3.1-70B Instruct, Llama-3.1-405B Instruct, 
	Llama-3.3-70B-Instruct and Llama-3.1-Nemotron-70B-Instruct.
	%Deepseek-R1~\citep{deepseekai2025deepseekr1incentivizingreasoningcapability}, Deepseek R1-Zero, Deepseek-V3~\citep{liu2024deepseek}, GLM-4-Plus~\citep{glm2024chatglm}, GLM-4-Long~\citep{glm2024chatglm}, Claude Sonnet 3.5, Claude 3.5 Haiku, GPT-4o, GPT-4o-mini, GPT-o3 mini, Gemini 2.0 Pro, Llama-3.1-8B/70B/405B Instruct~\citep{grattafiori2024llama3herdmodels}, Llama-3.3-8B/70B Instruct, Llama-3.1-Nemotron-70B-Instruct, Qwen2.5-7B/72B Instruct, InternLM2.5-20B-Chat
	\item \textbf{The Qwen Series~\citep{qwen2025qwen25technicalreport}.} include the effective Qwen2.5-7B-Instruct and Qwen2.5-72B-Instruct. 
	%	\item \textbf{Math-oriented Models} include DeepSeek-Math-Instruct 7B~\citep{deepseek-math}, MetaMath-Llemma-7B~\citep{azerbayev2023llemma}~\citep{yu2023metamath} Mammoth-7B/14B~\citep{yue2023mammoth}	
	\item \textbf{The GLM Series~\citep{glm2024chatglm}.} We use GLM4-Long to run the benchmark, since it is the commonly used in GLM series.
	% due to the overly high price of GLM4-Plus. 
	
	\item \textbf{The Deepseek Series~\citep{liu2024deepseek}~\citep{deepseekai2025deepseekr1incentivizingreasoningcapability}.} We currently use Deepseek V3 to run the benchmark. 
	Deepseek R1 will be evaluated in the future, since its API is down and unavailable now . 
	\item  \textbf{The GPT Series~\cite{achiam2023gpt}.} We use GPT-4o to run the benchmark. 
\end{itemize}

 
	We attempted to conduct experiments   on various math-oriented LLMs, such as Metamath-Llemma-7B~\citep{yu2023metamath}, Deepseek-Math-7B-instruct~\citep{deepseek-math}, InternLM2-Math-7B~\citep{ying2024internlmmathopenmathlarge} and MAmmoTH-7B~\citep{yue2023mammoth}.
	 However, these models fail during experiments for various reasons such as overly long output sequence length and limited input sequence length. Fail cases are demonstrated in the Figure~\ref{fig:fail_internlm},  Figure~\ref{fig:fail_ds_math},  Figure~\ref{fig:fail_llemma}, and  Figure~\ref{fig:fail_mammoth} in Appendix. 



\subsection{Main Experiments}
\noindent \textbf{Evaluation on Contextual Retrieval, Comparison, Summary, and Logic Reasoning Abilities.}
As shown in Table~\ref{tab:main_experiments}, 
current popular and effective LLMs perform poorly on basic numerical tasks, 
including retrieval, comparison, summarization, and logical reasoning. 
The random baseline for each task is 12.5\%, as there are 8 choices, and the probability of randomly selecting the correct answer is 1/8. 
Human evaluation was conducted by three undergraduate students. 

Firstly,
LLMs particularly struggle with accurately retrieving numerical data.
This limitation arises from LLMs treating numbers as discrete tokens rather than continuous ones, coupled with insufficient exposure to structured numerical datasets during training, which restricts their ability to handle simple numeric retrieval tasks. 
Secondly, LLMs demonstrate weaknesses in recognizing numerical relationships, such as greater-than or less-than comparisons, due to a lack of numerical semantics and underdeveloped arithmetic reasoning capabilities. 
Thirdly,
LLMs also perform poorly in summarizing numerical data (e.g., calculating sums or means), reflecting their inability to execute multi-step numerical operations. 
Similarly, logical reasoning tasks, especially those involving patterns or sequences, are particularly challenging, with all models scoring below 20\%. 
These tasks require multi-step reasoning, pattern recognition, and arithmetic operations, which expose the architectural limitations of current LLMs.









\begin{figure*}[t]
	\vspace{-2em}
	\centering 	
	\subfloat[Contextual  Retrieval]	
	{\centering\includegraphics[width=0.33\linewidth]{image/noisy_dataset_fig/retrieval-noisy-stock.pdf}}
	\hfill
	\subfloat[Comparison]
	{\centering\includegraphics[width=0.33\linewidth]{image/noisy_dataset_fig/compare-noisy-stock.pdf}}
	\subfloat[Summary]	
	{\centering\includegraphics[width=0.33\linewidth]{image/noisy_dataset_fig/summary-noisy-stock.pdf}}
	\hfill
 
	\caption{Evaluation on noisy stock dataset. Due to the input sequence length limit of Qwen2.5-72B-Inst on the API platform, the data containing 6 irrelevant attributes cannot be evaluated using this model.}
	\label{fig:noisy_stock}
	
\end{figure*}




\begin{figure*}[t]
	\vspace{-1em}
	\centering 	
	\subfloat[Accuracy on $Q_{oper}$  (i.e., $a+b$)]	
	{\centering\includegraphics[width=0.32\linewidth]{image/arithmetic_fig/arith_bar.pdf}}
	\hfill
	\subfloat[ $Q_{oper}$  of different digits]
	{\centering\includegraphics[width=0.32\linewidth]{image/arithmetic_fig/arith_plot.pdf}}
	\subfloat[Accuracy on $Q_{context}$  (i.e., $a$ plus $b$)]
	{\centering\includegraphics[width=0.32\linewidth]{image/arithmetic_text_fig/arith_bar.pdf}}
 
	\caption{Evaluation on arithmetic operation.}
	\label{fig:arithmetic_fig}
		\vspace{-1em}
\end{figure*}



 



 
\noindent \textbf{Evaluation on  Different Context Length via Stock and Weather Datasets.}
We evaluate LLMs on varying context lengths.
Specifically, we categorize the contexts of number lists, stock data, and weather data into short and long contexts.
The average token numbers for the short and long contexts across the three datasets are listed in Table~\ref{tab:data_stat_short_long}.
As illustrated in Figure~\ref{fig:length_number}, Figure~\ref{fig:length_stock}, and Figure~\ref{fig:length_weather},
LLMs generally achieve lower accuracy on long contexts compared to short contexts. This is because long contexts require the model to have a stronger ability to capture long-range dependencies.
Furthermore, if an LLM fails to perform well on short contexts, it is unlikely to achieve good results on long contexts. 
It highlights the importance of the inherent capabilities of LLMs in understanding numeric data.



\noindent \textbf{Evaluation on Noisy Context  via Stock and Weather Datasets.}
To evaluate the numerical abilities of LLMs in  noisy contexts, we add $k\in\{2,4,6\}$ irrelevant attributes to each instance in the stock and weather. 
These irrelevant attributes are not used in the user queries.
As shown in Figure~\ref{fig:noisy_stock} and Figure~\ref{fig:noisy_weather} in Appendix, 
as $k$ increases, most LLMs exhibit degraded performance. This indicates that irrelevant context can  affect the LLM's numerical retrieval and reasoning abilities.

 

\noindent \textbf{Evaluation on Arithmetic Operations}
Similarly, 
we evaluate five strong LLMs on arithmetic operations.
Specifically, as illustrated in Figure~\ref{fig:arithmetic_fig}~(a), even for simple arithmetic operations involving two numbers, LLMs fail to achieve 100\% accuracy. 
Moreover, as the number of digits increases shown in Figure~\ref{fig:arithmetic_fig}~(b), the accuracy of LLMs decreases, highlighting their limited ability to handle arithmetic tasks effectively, which is also observed in~\citep{qiu2024dissecting}.
This poor performance stems from how LLMs generate responses. LLMs  predict the highest-order digit  before the lower-order digit~\citep{zhang2024reverse}, contradicting the standard arithmetic logic of progressing from lower- to higher-order digits.
In particular, Figure~\ref{fig:arithmetic_fig}~(a) and (c) shows that LLMs perform similarly on addition, subtraction, and division operations but achieve extremely low accuracy on multiplication tasks.






\noindent \textbf{Evaluation on Number Recognition via Mixed-number-string Dataset.}
We evaluate the number recognition ability of effective LLMs by identifying numbers from mixed-number-string sequences. For this evaluation, we select five  effective LLMs based on Table~\ref{tab:main_experiments}, including DeepSeek-v3, GLM-4-Long, LLaMA3.1-405B, and Qwen2.5-72B.
As shown in Table~\ref{tab:number_counting}, all LLMs achieve extremely low accuracy in counting numbers within strings. Moreover, as the length of the string increases from 50 to 100, the accuracy of the LLMs decreases further.
These results highlight that LLMs are significantly weak at distinguishing numbers from strings. The underlying reason is that current LLMs treat numbers as strings during training. 
This training paradigm inherently limits their ability to understand and process numbers effectively.
Also, the tokenizer can split a single number into multiple tokens, which can negatively affect the numeric meaning of each number.








\begin{table}[]
	\centering
	\small
	\caption{Evaluation on mixed-number-string data with lengths ranging from 50 (i.e., 50 L) to 200.}
	
	%	\footnotesize
	\begin{tabular}{c|cccc}
		\toprule
		\textbf{Model}    & \textbf{50 L} & \textbf{100 L} & \textbf{150 L} & \textbf{200 L} \\ \midrule
		
		
		\textbf{\texttt{LLama3.1-405B }}& 10.8      & 9.2        & 3.2        & 2.2        \\  
		
		\textbf{\texttt{Qwen2.5-72B}}   & 3.0         & 1.2        & 0.6        & 0.2        \\  
		
		\textbf{\texttt{GLM4-Long}  }   & 6.6       & 4.8        & 3.0          & 2.4        \\  
		
		\textbf{\texttt{GPT-4o }}       & 18.2      & 6.4        & 4.0          & 4.2        \\ 
		
		\textbf{\texttt{DeepSeek-V3}}   & 13.2      & 4.0          & 3.2        & 2.0          \\  
		\midrule
		\textbf{\texttt{Human Eval } }   & \textbf{100}      & \textbf{100}        & \textbf{100}         & \textbf{100}        \\ \bottomrule
	\end{tabular}
	\label{tab:number_counting}
\end{table}
\subsection{Discussions on Numeracy Gaps of LLMs}
In summary, extensive experimental results show that current state-of-the-art LLMs perform poorly on six fundamental numerical abilities.
% such as number recognition and arithmetic operations. 
Here we discuss five potential reasons behind their poor performance on numerical tasks.

\noindent \textbf{Tokenizer Limitation.}
LLMs use tokenizers to split input text into smaller units (tokens). Thus,
Numbers are split into chunks as strings, based on statistical patterns in the training data.
For example, $10000$ is split into $100$ and $00$ tokens\footnote{\url{https://gptforwork.com/tools/tokenizer}}.
These tokenizers do not considering  the real meaning of numbers and continuous magnitude of numbers.
Thus, LLMs do not perform well on simple number retrieval and comparison tasks.

\noindent \textbf{Training Corpora Limitation.}
LLMs are trained on extensive corpora, which also limits their ability to understand numerical-related symbols, such as $*$.
For example, the multiplication of 246 and 369 can be denoted as $246*369$.
However, $246*369$ may be interpreted as a password or encrypted text, since $*$ in text strings is often associated with encryption.
As a result, enabling LLMs to accurately interpret arithmetic symbols remains an open problem.


\noindent \textbf{Training Paradigm Limitation.}
The training of LLMs relies on the next-token prediction paradigm, which is inherently misaligned with the logic of numerical computation.
For example, when solving $16 + 56$ with the result being $72$, an LLM will first predict the highest-order digit of the answer (i.e., $7$) before predicting the lower-order digit (i.e., $2$). This approach contradicts the fundamental logic of arithmetic computation, which typically proceeds from the lower-order digit to the higher-order digit.
This discrepancy implies that LLMs effectively need to know the entire result upfront to generate digits sequentially in the correct order. As a result, LLMs struggle to perform well even on simple arithmetic operations.

\noindent \textbf{Positional Embedding Limitation.}
Note that LLMs incorporate positional embeddings for  tokens in sequence inputs. In arithmetic operations like $12 + 26$ and $26 + 12$, the order of the numbers does not affect the result. However, LLMs assign different positional embeddings to the number $12$ in each equation, as its position in the sequence differs. 
This lack of invariance in positional embeddings for numbers can influence the results.
Therefore, how to design the positional embedding that improves numerical ability of LLMs without affecting the text understanding  of LLMs is critical~\cite{mcleish2024transformers,golovneva2024contextual}.



\noindent \textbf{Transformer Architecture Limitation.}
LLMs use Transformer to process input sequence, which rely on pattern recognition rather than explicit algorithmic reasoning.
The computational power of transformers has upper bounds~\cite{merrill2023parallelism}. Considering the complexity of arithmetic operations in real-world applications, it still needs to be theoretically investigated whether transformers can perform well on numerical operations.

\section{Conclusions}
In this paper, we propose \model, the first CoT prompting framework for graphs. We define an inference step with three substages: prompt-based inference, thought construction, and thought conditioned prompt learning. Specifically, we first feed the prompt modified query into the pre-trained encoder, and then construct a thought by fusing the hidden embeddings from each layer of the pre-trained graph encoder. To guide the subsequent inference step, we generate a series of prompts conditioned on the thought from the previous step. By repeating the above inference steps, \model\ obtain the answer. Finally, we conduct extensive experiments on eight public datasets, demonstrating that \model\ significantly outperforms a range of state-of-the-art baselines.



%%
%% The acknowledgments section is defined using the "acks" environment
%% (and NOT an unnumbered section). This ensures the proper
%% identification of the section in the article metadata, and the
%% consistent spelling of the heading.
% \begin{acks}
% To Robert, for the bagels and explaining CMYK and color spaces.
% \end{acks}

%%
%% The next two lines define the bibliography style to be used, and
%% the bibliography file.
%\newpage

\section*{Acknowledgments}
This research / project is supported by the Ministry of Education, Singapore under its Academic Research Fund (AcRF) Tier 1 grant (22-SIS-SMU-054) and Tier 2 grant (Proposal ID: T2EP20122-0041). Any opinions, findings and conclusions or recommendations expressed in this material are those of the author(s) and do not reflect the views of the Ministry of Education, Singapore. 

\clearpage
%\newpage
\bibliographystyle{ACM-Reference-Format}
\bibliography{references}
\balance

% \newpage
%%
%% If your work has an appendix, this is the place to put it.
\appendix
\section*{Appendices}
\renewcommand\thesubsection{\Alph{subsection}}
\renewcommand\thesubsubsection{\thesubsection.\arabic{subsection}}
\clearpage
 
\definecolor{exampleblue}{RGB}{0, 114, 188} % Blue for header
\definecolor{exampleborder}{RGB}{0, 114, 188} % Blue for border
\definecolor{redtext}{RGB}{204, 0, 0}         % Red text for emphasis

\section{Appendix}
In this appendix, we provide additional details about the design of \textbf{NumericBench}, along with supplementary experimental results and case studies. The organization of the supplementary materials in this appendix is as follows:

\begin{enumerate}[leftmargin=*]
	
	\item \textbf{Question formats for contextual retrieval, comparison, and summary abilities.}  
As shown in Table~\ref{appx:number_question}, Table~\ref{appx:stock_question}, and Table~\ref{appx:weather_question}, 
we designed diverse question types tailored to each dataset to evaluate the three fundamental numerical abilities of LLMs: contextual retrieval, comparison, and summary. contextual retrieval  assesses the model’s capacity to accurately extract relevant numerical information from complex contexts; comparison tests the ability to analyze and compare numerical values;  Summary evaluates the synthesis of numerical information into concise and meaningful insights for tasks like reporting or trend analysis.


By designing tailored questions for each dataset, we ensure a comprehensive evaluation of LLMs’ numerical reasoning abilities across varying scenarios and complexities.
	\item \textbf{Basic numerical questions answered incorrectly by GPT-4o.}  
	As illustrated in Figure~\ref{fig:number_compare}, Figure~\ref{fig:multiplication}, and Figure~\ref{fig:number_couting}, GPT-4o failed to answer three basic numerical questions correctly. This result is surprising, considering GPT-4o's impressive performance in real-world applications. However, these findings highlight the weak fundamental numerical abilities of LLMs.
	
	\item \textbf{Token counts for short and long contexts.}  
	As shown in Table~\ref{tab:data_stat_short_long}, the token counts of long and short contexts differ significantly. This distinction enables a more thorough evaluation of LLM performance across scenarios involving varying context lengths. Short contexts are designed to test the model's ability to process and understand concise information, focusing on immediate comprehension and reasoning. In contrast, long contexts present a more complex challenge, requiring the model to handle extended sequences of information, maintain coherence over a larger context window, 
	and retrieve relevant details from earlier parts of the input. Such two type length can more comprehensively evaluate LLMs. 
	
	\item \textbf{Additional experimental results on noisy and varying-length contexts.}  
	As shown in Figure~\ref{fig:length_stock} and Figure~\ref{fig:length_weather}, existing LLMs perform poorly on the stock and weather datasets, although they achieve better performance compared to their results on short contexts. 
	Similarly, as shown in Figure~\ref{fig:noisy_weather}, LLMs perform poorly on noisy weather data.
 
 \item \textbf{Real failure cases of math-oriented LLMs.} In this paper, we do not compare existing math-oriented LLMs, such as Metamath-Llemma-7B~\citep{yu2023metamath}, Deepseek-Math-7B-Instruct~\citep{deepseek-math}, InternLM2-Math-7B~\citep{ying2024internlmmathopenmathlarge}, and MAmmoTH-7B~\citep{yue2023mammoth}. 
 This is primarily because these math-oriented LLMs are designed for specialized geometric and structured mathematical problems. They are unable to understand the tasks in NumericBench, fail to follow a correct reasoning process, and directly produce meaningless outputs. These failure cases are illustrated in Figure~\ref{fig:fail_internlm}, Figure~\ref{fig:fail_ds_math}, Figure~\ref{fig:fail_llemma}, and Figure~\ref{fig:fail_mammoth}.
 
\end{enumerate}

\noindent \textbf{The Use of AI Tools.} When writing  this paper, we use Grammarly\footnote{https://www.grammarly.com/} for automated spell checking and use GPT-4o\footnote{https://platform.openai.com/docs/models/gpt-4o} to refine several sentences.


\clearpage
 
 

\begin{table*}[!h]
	\centering
	
	\caption{Question format on number list dataset. R: contextual retrieval, C: comparison, S: summary. In the contextual retrieval task, a number $x$ is randomly selected from the given number list. For the comparison task, the $k$-th largest number is randomly generated within the range of one to the length of the number list. The indices $x$ corresponds to twenty percent of the length of the number list, while $y$ corresponds to eighty percent of the length. The number $z$ is randomly chosen within the range $(\min(\text{list}), \max(\text{list}))$. For the summary task, the top $k$ is set to thirty percent of the length of the number list.}
 
	\renewcommand{\arraystretch}{1.15}  
	\setlength{\tabcolsep}{1.5pt}  
	\begin{tabular}{c|c}
		\toprule
		\textbf{Ability}    & \textbf{Question Format} \\ \midrule
		\textit{\textbf{R}} &  \begin{tabular}[c]{@{}l@{}}
			$Q_0$: What is the index of the first occurrence of the number $x$ in the list?\\
			$Q_1$: What is the index of the last occurrence of the number $x$ in the list?\\
			$Q_2$: What is the number after the first occurrence of the number $x$ in the list?\\
			$Q_3$: What is the number before the last occurrence of the number $x$ in the list?\\
			$Q_4$: What is the index of the first even number in the list?\\
			$Q_5$: What is the index of the first odd number in the list?\\
			$Q_6$: What is the index of the last even number in the list?\\
			$Q_7$: What is the index of the last odd number in the list?
		\end{tabular} \\ \midrule
		\textit{\textbf{C}} &  \begin{tabular}[c]{@{}l@{}}
			$Q_8$: What is the index of the first occurrence of the $k$-th largest number in the given list?\\
			$Q_9$: Which index holds the greatest number in the list between the indices $x$ and $y$?\\
			$Q_{10}$: Which index holds the smallest number in the list between the indices $x$ and $y$?\\
			$Q_{11}$: Which number is closest to $z$ in the list between the indices $x$ and $y$?\\
			$Q_{12}$: Which number is furthest from $z$ in the list between the indices $x$ and $y$?\\
			$Q_{13}$: Which number is the largest among those less than $z$ in the list?\\
			$Q_{14}$: Which number is the smallest among those greater than $z$ in the list?
		\end{tabular} \\ \midrule
		\textit{\textbf{S}} &  \begin{tabular}[c]{@{}l@{}}
			$Q_{15}$: What is the maximum sum of any two consecutive items in the list?\\
			$Q_{16}$: What is the maximum sum of any three consecutive items in the list?\\
			$Q_{17}$: What is the maximum absolute difference between two consecutive items in the list?\\
			$Q_{18}$: What is the sum of the indices of the top $k$ largest numbers in the list?\\
			$Q_{19}$: What is the sum of the indices of the top $k$ smallest numbers in the list?\\
			$Q_{20}$: What is the average of the indices of the top $k$ largest numbers in the list?\\
			$Q_{21}$: What is the average of the indices of the top $k$ smallest numbers in the list?\\
			$Q_{22}$: How many times do numbers consecutively increase for more than five times?\\
			$Q_{23}$: How many times do numbers consecutively decrease for more than five times?\\
			$\cdots \cdots$ \\
		\end{tabular} \\ \bottomrule
	\end{tabular}	
	\label{appx:number_question}
\end{table*}
\clearpage


 

\begin{table*}[]
	
	\caption{Question format on stock dataset. R: contextual retrieval, C: comparison, S: summary. $x$ and $y$ lie within the minimum and maximum ranges of their respective attributes. The top $k$ corresponds to thirty percent of the number list. $date_1$ represents the day at the twentieth percentile of the stock history, while $date_2$ corresponds to the day at the eightieth percentile.}
	\centering
	\renewcommand{\arraystretch}{1.15} % 设置行间距为默认的 1.15 倍
	\setlength{\tabcolsep}{1.5pt} % 将列间距设置为 1pt
	\begin{tabular}{c|c}
		\toprule
		\textbf{Ability}    & \textbf{Question Format} \\ \midrule
		\textit{\textbf{R}} &  \begin{tabular}[c]{@{}l@{}}
			$Q_0$: On which date did the close price of the stock first reach $x$ yuan?\\
			$Q_1$: On which date did the highest price of the stock first reach $x$ yuan?\\
			$Q_2$: On which date did the volume of the stock first reach $x$ lots?\\
			$Q_3$: On which date did the value of the stock first reach $x$ thousand yuan?\\
			$Q_4$: On which date did the price change rate of the stock first reach $x$\%?\\
			$Q_5$: On which date did the price change of the stock first reach $x$ yuan?\\
		\end{tabular} \\ \midrule
		\textit{\textbf{C}} &  \begin{tabular}[c]{@{}l@{}}
			\begin{tabular}[c]{@{}l@{}}
				$Q_6$: On which date did the stock have the highest turnover rate when the close \\price was greater than $x$ yuan?
			\end{tabular}\\
			
			\begin{tabular}[c]{@{}l@{}}
				$Q_7$: On which date did the stock have the highest quantity relative ratio when \\the open price was less than $x$ yuan?
			\end{tabular}\\
			
			\begin{tabular}[c]{@{}l@{}}
				$Q_8$: On which date did the stock have the highest difference between the highest \\and lowest prices when the trading volume exceeded $x$ lots?
			\end{tabular}\\
			
			\begin{tabular}[c]{@{}l@{}}
				$Q_9$: On which date did the stock record the highest daily average price, calculated \\as 'value' divided by 'volume,' when the PE ratio was less than $x$?
			\end{tabular}\\
			
			\begin{tabular}[c]{@{}l@{}}
				$Q_{10}$: Among the top-$k$ trading value days, on which date did the stock have the \\lowest close price?
			\end{tabular}\\
			
			\begin{tabular}[c]{@{}l@{}}
				$Q_{11}$: When the quantity relative ratio exceeded $x$, on which date did the stock \\have the highest sum of the open price and close price?
			\end{tabular}\\
			
			\begin{tabular}[c]{@{}l@{}}
				$Q_{12}$: When the absolute price change rate exceeded $x$\%, on which date did the \\stock have the highest difference between the highest and lowest prices?
			\end{tabular}
		\end{tabular} \\ \midrule
		\textit{\textbf{S}} &  \begin{tabular}[c]{@{}l@{}}
			\begin{tabular}[c]{@{}l@{}}
				$Q_{13}$: How many days had a volume greater than $x$ from $date_1$ to $date_2$?
			\end{tabular}\\
			
			\begin{tabular}[c]{@{}l@{}}
				$Q_{14}$: How many days had the close price higher than the open price from \\$date_1$ to $date_2$?
			\end{tabular}\\
			
			\begin{tabular}[c]{@{}l@{}}
				$Q_{15}$: How many days had a close price higher than the open price, with the \\quantity relative ratio exceeding $x$\%?
			\end{tabular}\\
			
			\begin{tabular}[c]{@{}l@{}}
				$Q_{16}$: How many days had the close price reach $x$ yuan with the absolute price \\change rate exceeding $x$\%?
			\end{tabular}\\
			
			\begin{tabular}[c]{@{}l@{}}
				$Q_{17}$: What was the average trading volume when both the turnover rate \\exceeded $x$\% and the price change rate was greater than $y$\%?
			\end{tabular}\\
			
			\begin{tabular}[c]{@{}l@{}}
				$Q_{18}$: Excluding non-trading days, how many times did the open price of \\the stock rise for three or more consecutive days?
			\end{tabular}\\
			
			\begin{tabular}[c]{@{}l@{}}
				$Q_{19}$: Excluding non-trading days, how many times did the close price of \\the stock rise for three or more consecutive days?
			\end{tabular}\\
			
			\begin{tabular}[c]{@{}l@{}}
				$Q_{20}$: Excluding non-trading days, how many times did the open price and \\close price of the stock both rise for three or more consecutive days?
			\end{tabular}\\
		
			\begin{tabular}[c]{@{}l@{}}
			$\cdots \cdots$
		\end{tabular}
		
		\end{tabular} \\ \bottomrule
	\end{tabular}
\label{appx:stock_question}
\end{table*}
\clearpage
 

\begin{table*}[]
	\centering
	\caption{Question format on weather dataset.  R: contextual retrieval, C: comparison, S: summary. The value of $x$ falls within the minimum and maximum ranges of its respective attribute. $date_1$ represents the day at the twentieth percentile of the stock history, while $date_2$ represents the day at the eightieth percentile.}
	\begin{tabular}{c|c}
		\toprule
		\textbf{Ability}    & \textbf{Question Format} \\ \midrule
		\textit{\textbf{R}} &  \begin{tabular}[c]{@{}l@{}}
			$Q_0$: On which date did the temperature at two meters first reach $x$°C?\\
			$Q_1$: On which date did the relative humidity at two meters first exceed $x$\%?\\
			$Q_2$: On which date did the dew point temperature at two meters first drop below $x$°C?\\
			$Q_3$: On which date did the precipitation first exceed $x$ mm?\\
			$Q_4$: On which date did the sea-level air pressure first exceed $x$ hPa?\\
			$Q_5$: On which date did the cloud cover first reach $x$\%?\\
			$Q_6$: On which date did the wind speed at 10 meters first exceed $x$ m/s?
		\end{tabular} \\ \midrule
		\textit{\textbf{C}} &  \begin{tabular}[c]{@{}l@{}}
			\begin{tabular}[c]{@{}l@{}}
				$Q_7$: On which date did the temperature at two meters reach its highest value \\ 
				when the relative humidity was below $x$\%? 
			\end{tabular} \\
			
			\begin{tabular}[c]{@{}l@{}}
				$Q_8$: On which date did the relative humidity at two meters reach its lowest value \\ 
				when the temperature at two meters was above $x^\circ$C?
			\end{tabular} \\
			
			\begin{tabular}[c]{@{}l@{}}
				$Q_9$: On which date did the difference between the temperature and dew point \\ 
				at two meters reach its maximum when the cloud cover was below $x$\%? 
			\end{tabular} \\
			
			\begin{tabular}[c]{@{}l@{}}
				$Q_{10}$: On which date did the precipitation reach its highest value \\ 
				when the temperature at two meters was below $x^\circ$C? 
			\end{tabular} \\
			
			\begin{tabular}[c]{@{}l@{}}
				$Q_{11}$: On which date did the cloud cover reach its lowest value \\ 
				when the wind speed at 10 meters exceeded $x$ m/s? 
			\end{tabular} \\
			
			\begin{tabular}[c]{@{}l@{}}
				$Q_{12}$: On which date did the wind speed at 10 meters reach its highest value \\ 
				when the precipitation exceeded $x$ mm? 
			\end{tabular} \\
			
			\begin{tabular}[c]{@{}l@{}}
				$Q_{13}$: On which date did the sea-level air pressure reach its highest value \\ 
				when the cloud cover was below $x$\%? 
			\end{tabular}
		\end{tabular} \\ \midrule
		\textit{\textbf{S}} &  \begin{tabular}[c]{@{}l@{}}
			\begin{tabular}[c]{@{}l@{}}
				$Q_{14}$: How many days had a temperature at two meters greater than $x^\circ$C \\from $date_1$ to $date_2$? 
			\end{tabular} \\
			
			\begin{tabular}[c]{@{}l@{}}
				$Q_{15}$: How many days had a relative humidity at two meters exceeding $x$\% \\from $date_1$ to $date_2$? 
			\end{tabular} \\
			
			\begin{tabular}[c]{@{}l@{}}
				$Q_{16}$: How many days had a precipitation greater than $x$ mm from $date_1$ \\to $date_2$? 
			\end{tabular} \\
			
			\begin{tabular}[c]{@{}l@{}}
				$Q_{17}$: What was the average temperature at two meters when the relative \\humidity exceeded $x$\%? 
			\end{tabular} \\
			
			\begin{tabular}[c]{@{}l@{}}
				$Q_{18}$: What was the average wind speed at 10 meters when the precipitation \\exceeded $x$ mm? 
			\end{tabular} \\
			
			\begin{tabular}[c]{@{}l@{}}
				$Q_{19}$: How many times did the temperature at two meters rise for three or more \\consecutive days? 
			\end{tabular} \\
			
			\begin{tabular}[c]{@{}l@{}}
				$Q_{20}$: How many times did the relative humidity at two meters drop for \\three or more consecutive days? 
			\end{tabular} \\
		
					
		\begin{tabular}[c]{@{}l@{}}
			$\cdots \cdots$
		\end{tabular} \\
	
		\end{tabular} \\ \bottomrule
	\end{tabular}
\label{appx:weather_question}
\end{table*}

\clearpage
\begin{figure*}[t]
	\centering	
	\vspace{-1em}
	\frame{
		\includegraphics[width = 0.9\textwidth]{image/intro_example/number_compare.png}
	}
	%	\captionsetup{labelformat=empty}
	%	\addtocounter{figure}{-1}
	\caption{Number comparisons on GPT-4o. The correct answer is -9.11. }
	\label{fig:number_compare}
\end{figure*}

\begin{figure*}[t]
	\centering	
	\vspace{-1em}
	\frame{
		\includegraphics[width = 0.9\textwidth]{image/intro_example/multiplication.png}
	}
	%	\captionsetup{labelformat=empty}
	%	\addtocounter{figure}{-1}
	\caption{Number multiplication on GPT-4o. The correct answer is 102244.12. }
	\label{fig:multiplication}
\end{figure*}
\begin{figure*}[t]
	\centering	
	\vspace{-1em}
	\frame{
		\includegraphics[width = 0.9\textwidth]{image/intro_example/number_counting.jpg}
	}
	%	\captionsetup{labelformat=empty}
	%	\addtocounter{figure}{-1}
	\caption{Number counting on GPT-4o, which is required directly give answer. The correct answer is 4. }
	\label{fig:number_couting}
\end{figure*}

\clearpage


%\subsection{Additional Experiment Results}
%\subsubsection{Additional results on context length evaluation for stock and weather data}\label{appx:sssec:length}
		\begin{figure*}[t]
		\centering 	
		\subfloat[Contextual Retrieval]	
		{\centering\includegraphics[width=0.33\linewidth]{image/main_fig/retrieval-stock.pdf}}
		\hfill
		\subfloat[Comparison]
		{\centering\includegraphics[width=0.33\linewidth]{image/main_fig/compare-stock.pdf}}
		\subfloat[Summary]	
		{\centering\includegraphics[width=0.33\linewidth]{image/main_fig/summary-stock.pdf}}
		\hfill
		%	\subfloat[MUTAG]
		%	{\centering\includegraphics[width=0.25\linewidth, height=3.05cm]{image/g1-4.pdf}}	
		%	\hfill
		%	
		\caption{Evaluation on short and long context on stock dataset. Due to the input sequence length limit of Qwen2.5-72B-Inst on the API platform, the long dataset of all three abilities cannot be evaluated using this model.}
		\label{fig:length_stock}
	\end{figure*}
	
	
	
	\begin{figure*}[t]
		\centering 	
		\subfloat[Contextual Retrieval]	
		{\centering\includegraphics[width=0.33\linewidth]{image/main_fig/retrieval-weather.pdf}}
		\hfill
		\subfloat[Comparison]
		{\centering\includegraphics[width=0.33\linewidth]{image/main_fig/compare-weather.pdf}}
		\subfloat[Summary]	
		{\centering\includegraphics[width=0.33\linewidth]{image/main_fig/summary-weather.pdf}}
		\hfill
		%	\subfloat[MUTAG]
		%	{\centering\includegraphics[width=0.25\linewidth, height=3.05cm]{image/g1-4.pdf}}	
		%	\hfill
		%	
		\caption{Evaluation on short and long context on weather dataset. Due to the input sequence length limit of Qwen2.5-72B-Inst on the API platform, the long dataset of all three abilities cannot be evaluated using this model.}
		\label{fig:length_weather}
		
	\end{figure*}

		\begin{figure*}[t]
		
		\centering 	
		\subfloat[Contextual  Retrieval]	
		{\centering\includegraphics[width=0.33\linewidth]{image/noisy_dataset_fig/retrieval-noisy-weather.pdf}}
		\hfill
		\subfloat[Comparison]
		{\centering\includegraphics[width=0.33\linewidth]{image/noisy_dataset_fig/compare-noisy-weather.pdf}}
		\subfloat[Summary]	
		{\centering\includegraphics[width=0.33\linewidth]{image/noisy_dataset_fig/summary-noisy-weather.pdf}}
		\hfill
		%	\subfloat[MUTAG]
		%	{\centering\includegraphics[width=0.25\linewidth, height=3.05cm]{image/g1-4.pdf}}	
		%	\hfill
		%	
		\caption{Evaluation on  noisy weather dataset. Due to the input sequence length limit of Qwen2.5-72B-Inst on the API platform, the data containing 4 and 6 irrelevant attributes cannot be evaluated using this model.}
		\label{fig:noisy_weather}
		
	\end{figure*}
\clearpage
	
	
	
	
%	\begin{figure*}[t]
%		
%		\centering 	
%		\subfloat[Context Retrieval]	
%		{\centering\includegraphics[width=0.33\linewidth]{image/multi_dataset_fig/retrieval-multi-stock.pdf}}
%		\hfill
%		\subfloat[Comparison]
%		{\centering\includegraphics[width=0.33\linewidth]{image/multi_dataset_fig/compare-multi-stock.pdf}}
%		\subfloat[Summary]	
%		{\centering\includegraphics[width=0.33\linewidth]{image/multi_dataset_fig/summary-multi-stock.pdf}}
%		\hfill
%		%	\subfloat[MUTAG]
%		%	{\centering\includegraphics[width=0.25\linewidth, height=3.05cm]{image/g1-4.pdf}}	
%		%	\hfill
%		%	
%		\caption{Evaluation on multi-turn QA on stock dataset. Due to the input sequence length limit of Qwen2.5-72B-Inst on the API platform, the model cannot generate outputs in the third turn of the conversation. }
%		\label{fig:multurn_stock}
%		
%	\end{figure*}
	
	\begin{table*}[]
		\caption{The average token number on short and long instances for each data.}
		\centering
		\begin{tabular}{c|c|cc|cc}
			\toprule
			\multirow{2}{*}{\textbf{Dataset}}                                               & \multirow{2}{*}{\textbf{Ability}} & \multicolumn{2}{c|}{\textbf{Short}}                            & \multicolumn{2}{c}{\textbf{Long}}                             \\ \cmidrule{3-6} 
			
			&                                   & \multicolumn{1}{c|}{\textbf{\# Instance}} & \textbf{Avg Token} & \multicolumn{1}{c|}{\textbf{\# Instance}} & \textbf{Avg Token} \\ \midrule
			
			\multirow{3}{*}{\textbf{\begin{tabular}[c]{@{}c@{}}Number\\ List\end{tabular}}} & \textit{Contextual Retrieval}

                  & \multicolumn{1}{c|}{500}                     &        809.12     & \multicolumn{1}{c|}{500}                     &         6599.34      \\   
			
			& \textit{Comparison}                        & \multicolumn{1}{c|}{500}                     &     804.86     & \multicolumn{1}{c|}{500}                     &        6566.27      \\ 
			
			& \textit{Summary}



                  & \multicolumn{1}{c|}{500}                     &       822.49      & \multicolumn{1}{c|}{500}                     &       6487.07       \\ \midrule
			
			\multirow{3}{*}{\textbf{Stock}}                                                 & \textit{Contextual Retrieval}

                  & \multicolumn{1}{c|}{500}                     &        18529.07      & \multicolumn{1}{c|}{500}                     &      36641.63     \\  
			& \textit{Comparison}                        & \multicolumn{1}{c|}{500}                     &    18539.58     & \multicolumn{1}{c|}{500}                     &      36651.22      \\ 
			& \textit{Summary}

                  & \multicolumn{1}{c|}{500}                     &      18504.51      & \multicolumn{1}{c|}{500}                     &       36618.07      \\ \midrule
			
			\multirow{3}{*}{\textbf{Weather}}                                               & \textit{Contextual Retrieval}

                  & \multicolumn{1}{c|}{500}                     &        18362.38        & \multicolumn{1}{c|}{500}                     &        36356.13    \\  
			& \textit{Comparison}                        & \multicolumn{1}{c|}{500}                     &        18371.11    & \multicolumn{1}{c|}{500}                     &       36365.27     \\ 
			& \textit{Summary}



                  & \multicolumn{1}{c|}{500}                     &        18334.15     & \multicolumn{1}{c|}{500}                     &        36328.27    \\ \bottomrule
		\end{tabular}
		\label{tab:data_stat_short_long}
	\end{table*}
	\clearpage
%	\begin{figure*}[t]
%		
%		\centering 	
%		\subfloat[Context Retrieval]	
%		{\centering\includegraphics[width=0.33\linewidth]{image/multi_dataset_fig/retrieval-multi-weather.pdf}}
%		\hfill
%		\subfloat[Comparison]
%		{\centering\includegraphics[width=0.33\linewidth]{image/multi_dataset_fig/compare-multi-weather.pdf}}
%		\subfloat[Summary]	
%		{\centering\includegraphics[width=0.33\linewidth]{image/multi_dataset_fig/summary-multi-weather.pdf}}
%		\hfill
%		%	\subfloat[MUTAG]
%		%	{\centering\includegraphics[width=0.25\linewidth, height=3.05cm]{image/g1-4.pdf}}	
%		%	\hfill
%		%	
%		\caption{Evaluation on multi-turn QA on weather dataset.}
%		\label{fig:multurn_weather}
%		
%	\end{figure*}
	

	
	
%\noindent\textbf{Case Study}
	\newtcolorbox[auto counter, number within=section]{examplebox}[2][]{%
		colback=white,        % Background color
		colframe=exampleborder, % Border color
		coltitle=white,       % Header text color
		fonttitle=\bfseries,  % Header font
		colbacktitle=exampleblue, % Header background color
		title=Example~\thetcbcounter: #2, % Title format
	}
	\begin{figure*}
		\begin{examplebox}{InternLM2-Math-7B}
			\textbf{Failure:} \\Model repeatedly output the same number until the output sequence ends.
			
			\vspace{0.2cm}
			\textbf{User:} \\ Question: What is the maximum sum of any three consecutive items in the list? Options: A: 21501.9737231253, B: 21502.8144848529, C: 21501.9642201794, D: 21502.7354388437, E: 21502.0327435643, F: 21501.7526575621, G: 21501.7057960155, H: 21502 Data: [-5279, ....(100 numbers)]
			
			\vspace{0.3cm}
			\textbf{InternLM2-Math-7B Model:} \\
			Step 1: To find the maximum sum of any three consecutive items in the list, we need to identify the three largest numbers in the list and add them together. <PRM +> \\Step 2: We can start by sorting the list in ascending order to make it easier to identify the largest numbers. <PRM +> \\Step 3: After sorting, the list becomes: [-8509, -8509, -8509, -8509, -8509, -8509, -8509, -8509, -8509, -8509, -8509, -8509, -8509, -8509, -8509, -8509, -8509, -8509, -8509, -8509, -8509, -8509, -8509, -8509, -8509, -8509, -8509, -8509...(output too long)
		\end{examplebox}
		\caption{Failure type and conversation record of InternLM2-Math-7B. The model constantly generate the same "-8509" text until the output sequence was too long and truncated, leaving the problem unsolved.}
		\label{fig:fail_internlm}
	\end{figure*}
	
	\begin{figure*}
		\begin{examplebox}{DeepSeek-Math-7B}
				\textbf{Failure:} \\ Model outputs sequence without meaning, probably due to overly long input sequences. 
				
				\vspace{0.2cm}
				\textbf{User:} \\Question: What is the sum of the index of top 300 smallest numbers in the list? Options: A: 146734, B: 146738, C: 146733, D: 146724, E: 146728, F: 146727, G: 146721, H: 146729 Data: [3640, 1560, ...(100 numbers)]
				
				\vspace{0.3cm}
				\textbf{Deepseek-Math-7B Model:} \\
				31,3,9,3,1,1,3,3,3,3,3,3,3,3, 3,3,1,3,3,3,1,3,3,3, 3, 3,3,3, 1, 1,3,3, 3,1,3, 3,3,3, 1, 3,3,3, 3,3,1,3,1,3,91, 3,3,3, 3,31,3,3,3, 3, 339, 3, 3, 311, 1, 3, 1,3,3, 3,3,6,3,3, 3,31,3,3,1, 1, 1, 1, 3,1,1, 3,3, 1,1, 3,1,3,1, 1, 1, 1, 1, 19,1, 1, 1,191,1, 1,1, 19, 31,1,1, 1, 3,1, 1,3, 1133,3, 1, 1,1, 11, 11,3, 31, 3, 3,1,1,3,1, 1,33, 11,3,1911,3, 1,13,3131311119191,1, 3,1,11,1,19,11,1,1,1....(keeps outputting numbers)
		\end{examplebox}
		\caption{Failure type and conversation record of DeepSeek-Math-7B. The model only generates a series of numbers throughout its output, and we're unable to acquire any useful information from the output.}
		\label{fig:fail_ds_math}
	\end{figure*}	

	\begin{figure*}
		\begin{examplebox}{Metamath-Llemma-7B}
			\textbf{Failure:}\\ Model outputs ineffective chain of thoughts, and giving the wrong answer in the wrong format(ground truth is B))
			
			\vspace{0.2cm}
			\textbf{User:}\\ Question: What is the sum of the index of top 30 smallest numbers in the list? Options: A: 1566, B: 1572, C: 1575, D: 1573, E: 1577, F: 1564, G: 1569, H: 1567 Data: [-2568.1542995196, 1035.3645340299...(100 numbers)]
			
			\vspace{0.3cm}
			\textbf{Metamath-Llemma-7B Model:} \\ The data you receive is a list of numbers. To find the sum of the index of the top 30 smallest numbers, we need to find the 30 smallest numbers in the list and then calculate their sum. To find the 30 smallest numbers, we can use the min function in Python. The min function takes a list as input and returns the smallest number in the list. We can use the min function to find the 30 smallest numbers in the list. Now, we need to calculate the sum of these 30 smallest numbers. To do this, we can use the sum function in Python. The sum function takes a list as input and returns the sum of all the numbers in the list. We can use the sum function to calculate the sum of the 30 smallest numbers. Finally, we need to find the index of the sum. To do this, we can use the index function in Python. The index function takes a number as input and returns the index of that number in the list. We can use the index function to find the index of the sum. Therefore, the answer is (correct option). \#\#\#\# 1566 The answer is: 1566
		\end{examplebox}
		\caption{Failure type and conversation record of Metamath-Llemma-7B. The model attempts to use programming language to solve the problem, but it does not provide executable lines of code, only providing vague description of certain functions instead. After outputting the vague solution, the model outputs an answer, which is different from the ground truth and not in the expected format(We expect the model to provide the answer with a capital letter corresponding to the correct option).}
		\label{fig:fail_llemma}
	\end{figure*}

	\begin{figure*}
	\begin{examplebox}{MAmmoTH-13B}
		\textbf{Failure:}\\ Model outputs ineffective chain of thoughts, and giving the wrong answer in the wrong format(ground truth is B))
		
		\vspace{0.2cm}
		\textbf{User:}\\ Question: What is the sum of the index of top 30 smallest numbers in the list? Options: A: 1566, B: 1572, C: 1575, D: 1573, E: 1577, F: 1564, G: 1569, H: 1567 Data: [-2568.1542995196, 1035.3645340299...(100 numbers)]
		
		\vspace{0.3cm}
		\textbf{MAmmoTH-13B Model:}\\  Notice that there are two numbers in your list that have the smallest index: 4346.5174909575 and 8169.6394989672. They have the smallest index of 4346 and 8169 respectively. Since in the options the first number is the index of the smallest number, then the correct option is D.\\             Option: D
	\end{examplebox}
	\caption{Failure type and conversation record of Mammoth-13B. The model fail to understand the problem's instruction, and does not provide reasonable output. Moreover, it provides the wrong answer in incorrect format, just like Metamath-Llemma-7B(We expect the model to provide the answer with a capital letter corresponding to the correct option).}
	\label{fig:fail_mammoth}
	\end{figure*}

\end{document}
\endinput
%%
%% End of file `sample-sigconf.tex'.
