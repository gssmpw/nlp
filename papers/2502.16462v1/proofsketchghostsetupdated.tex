\paragraph{Ghost Set.}
Thus we focus on showing the claim in~\cref{eq:proofsketch2}. As in many previous proofs of generalization bounds, we first seek to discretize the infinite hypothesis class $\dlh$ and then apply a union bound over a finite set of events/hypotheses. In our proof, we will construct a $(\gamma_0/2)$ $\ell_\infty$-covering $N$ of $\dlh$. Ideally, such a covering would contain for every $f \in \dlh$, a function $f' \in N$ such that $|f(x)-f'(x)| \leq \gamma_0/2$ all $x \in \cX$. Unfortunately, there might not be a finite such $N$ when requiring $|f(x)-f'(x)| \leq \gamma_0/2$ for all $x$ in the full input domain $\cX$. We thus start by introducing a \emph{ghost set} $\rS' \sim \cD^m$ to replace all references to $\ls_{\cD}(f)$ (which depends on the full domain $\cX$) by $\ls_{\rS'}(f)$ (which depends only on $\rS'$). Using standard arguments relating $\ls_{\cD}(f)$ to $\ls_{\rS'}(f)$, we show that~\cref{eq:proofsketch2} follows if we can show that with probability $1-\delta$ over the pair $(\rS,\rS')$, it holds for every $f \in \dlh$ and $\gamma \in [\gamma_0,\gamma_1]$ that either $\ls_{\rS}^\gamma(f) \notin [\tau_0, \tau_1]$ or
\begin{align}\label{eq:goalWithghost}
 \ls_{\rS'}(f)\leq \tau_{1}+O\left(\sqrt{\tau_{1}\left(\frac{d\func{\left(\frac{m \gamma_{0}^{2}}{d} \right)}}{\gamma_{0}^{2}m}+\frac{\ln{\left(\frac{1}{\delta}\right)}}{m} \right)}+\left(\frac{d\func{\left(\frac{m\gamma_{0}^{2}}{d} \right)}}{\gamma_{0}^{2}m}+\frac{\ln{\left(\frac{1}{\delta}\right)}}{m} \right) \right).
\end{align}
Observe that we substituted $\ls_{\cD}(f)$ by $\ls_{\rS'}(f)$ compared to~\cref{eq:proofsketch2}.

\paragraph{Covering.}
To establish~\cref{eq:goalWithghost}, consider first a fixed $ f\in \dlh $. Since $\rS$ and $\rS'$ are i.i.d.\ samples from $\cD^m$, we have that $\ls^\gamma_{\rS}(f)$ and $\ls_{\rS'}(f)$ are strongly concentrated around their means $\ls^\gamma_\cD(f)$ and $\ls_\cD(f)$. Moreover, since $\ls^\gamma_\cD(f) \geq \ls_\cD(f)$, it is highly unlikely that $\ls_{\rS'}(f)$ is significantly larger than $\ls_\rS(f)$. This is precisely what we need to establish~\cref{eq:goalWithghost}. Concretely, we need to show that there is no $f$ with $\ls^\gamma_\rS(f) \in [\tau_0, \tau_1]$ such that $\ls_{\rS'}(f)$ is much larger than $\tau_1$. Since this event is unlikely for a fixed $f$, we now introduce a discretization of $\dlh$ that would preserve any large gap between $\ls^\gamma_\rS(f)$ and $\ls_{\rS'}(f)$.

To this end, we discretize $\dlh$ on $ \rX=\rS\cup\rS' $  via a $\gamma_0/2$ $\ell_\infty$-covering $ \Net $, i.e., for any $f \in \dlh$, there is an $f' \in \Net$ with $|f(x)-f'(x)| \leq \gamma_0/2$ for all $x\in \rX$. Now observe that whenever $yf(x) > \gamma_0$, we also have $yf'(x) > \gamma_0/2$. Thus, for any $\gamma \in [\gamma_0,\gamma_1]$, we have $\ls_{\rS}^\gamma(f) \geq \ls_{\rS}^{\gamma_0/2}(f')$. Similarly, we have for $yf(x) \leq 0$ that $yf'(x) \leq \gamma_0/2$, and thus $\ls_{\rS'}(f) \leq \ls_{\rS'}^{\gamma_0/2}(f')$. Therefore, a $\gamma_0/2$ $\ell_\infty$-covering $ \Net $ preserves the imbalance between $ \ls_{\rS}^{\gamma}(f) $ and $ \ls_{\rS'}(f)$ via $ \ls_{\rS}^{\gamma_0/2}(f') $ and $ \ls_{\rS'}^{\gamma_0/2}(f').$   
 
To construct a $\gamma_0/2$ $\ell_\infty$-covering $ \Net $ of $ \rX $ and union bound over it, we need the point set $ \rX $ to be fixed - however we still want to be able to show that an imbalance between $ \ls_{\rS}^{\gamma_0/2}(f') $ and $ \ls_{\rS'}^{\gamma_0/2}(f')$ for some $ f'\in \Net$ is highly unlikely. As in previous works, we employ the following way of viewing the sampling of $ \rS $ and $ \rS'$. First, we draw $ \rX\sim \cD^{2m} $, consisting of $ 2m $ i.i.d.\ training examples from $ \cD $, and then let $ \rS $ be $ m $ points drawn without replacement from $ \rX,$ and $ \rS' $ be the remaining $ m $  points of $ \rX,$ i.e., $ \rS'=\rX\backslash \rS'.$ Taking this viewpoint of drawing $ \rS $ and $ \rS $ allows us to fix the realization $ X $ of the points in $ \rX,$ while still having which training examples ending up in $ \rS $ and $ \rS' $ being random. This still allows us to argue that an imbalance between $ \ls_{\rS}^{\gamma_0/2}(f') $ and $ \ls_{\rS'}^{\gamma_0/2}(f')$ for some $ f'\in \Net$ is unlikely.
              
Thus, we now consider an arbitrary but fixed realization $ X $ of $ \rX, $ and let $ \Net $ be a $\gamma_0/2$ $\ell_\infty$-covering of $ X. $ By the above arguments above, if we can show for any $0 < \delta < 1$ and an arbitrary $f\in \Net $, it holds with probability at least $1-\delta$ over the random partitioning of $X$ into $\rS, \rS'$ that either $\ls_{\rS}^{\gamma_0/2}(f) > \tau_1$ or
 \begin{align}\label{eq:proofsketch6}
  \ls^{\gamma_{0}/2}_{\rS'}(f) = \tau_{1}+O\left(\sqrt{\tau_{1}\frac{\ln{\left(\frac{1}{\delta}\right)}}{m} }+\frac{\ln{\left(\frac{1}{\delta}\right)}}{m} \right),
 \end{align}
 then we can union bound over all $f \in \Net$, with $\delta$ rescaled to $\delta/|\Net|$, to conclude that with probability $1-\delta$ it holds for all $f \in \Net$ that either $\ls_{\rS}^{\gamma_0/2}(f) > \tau_1$ or
 \begin{align*}
  \ls^{\gamma_{0}/2}_{\rS'}(f) = \tau_{1}+O\left(\sqrt{\tau_{1}\frac{\ln{\left(\frac{|\Net|}{\delta}\right)}}{m} }+\frac{\ln{\left(\frac{|\Net|}{\delta}\right)}}{m} \right).
 \end{align*}
Giving an appropriate upper bound on $|\Net|$ will then imply~\cref{eq:goalWithghost}
 
Now, to argue~\cref{eq:proofsketch6} for a fixed $ f \in \Net$, we want to show that the event $ \ls_{\rS}^{\gamma_{0}/2}(f)\leq \tau_{1} $ and $ \ls_{\rS'}^{\gamma_{0}/2}(f) = \tau_{1}+\Omega(\sqrt{\tau_{1}\ln{\left(1/\delta \right)}/m}+\ln{\left(1/\delta \right)}/m) $ happens with probability at most $ \delta$. Let $\mu$ denote the fraction of mistakes $ f $ makes on $ X $ and observe that $\mu= (\ls_{\rS}^{\gamma_{0}/2}(f)+\ls_{\rS'}^{\gamma_{0}/2}(f))/2$. We notice that $ \mu $ has to be at least $ \ls_{\rS'}^{\gamma_{0}/2}(f)/2 = (\tau_{1}+\Omega(\sqrt{\tau_{1}\ln{\left(1/\delta \right)}/m}+\ln{\left(1/\delta \right)}/m))/2 $ for the event to occur. Since $ \mu $ is $ \Omega(\ln{\left( 1/\delta\right)}/m) $ and $ \ls_{\rS}^{\gamma_{0}/2}(f) $ has expectation equal to $ \mu,$ it follows by an invocation of a Chernoff bound (without replacement) that with probability at least $ 1-\delta $ over $ \rS $ (drawn from $ X $)  that
 \begin{align*}
   \ls_{\rS}^{\gamma_{0}/2}(f)&\geq \left(1-\sqrt{\frac{2\ln{\left(1/\delta \right)}}{\mu m}}\right)\mu\\
   &=(\ls_{\rS}^{\gamma_{0}/2}(f)+\ls_{\rS'}^{\gamma_{0}/2}(f))/2-\sqrt{\frac{2\ln{\left(1/\delta \right)}(\ls_{\rS}^{\gamma_{0}/2}(f)+\ls_{\rS'}^{\gamma_{0}/2}(f))/2}{m}}, 
 \end{align*}    
where doing some rearrangements implies the following inequality
 \begin{align*}
  \frac{\ls_{\rS}^{\gamma_{0}/2}(f)}{2}+\sqrt{\frac{\ls_{\rS}^{\gamma_{0}/2}(f)\ln{\left(1/\delta \right)}}{m}}\geq \frac{\ls_{\rS'}^{\gamma_{0}/2}(f)}{2}-\sqrt{\frac{\ls_{\rS'}^{\gamma_{0}/2}(f)\ln{\left(1/\delta \right)}}{m}}.
 \end{align*}
We notice that the above inequality is implying that $ \ls_{\rS'}^{\gamma_{0}/2}(f) $ cannot be too large compared to $ \ls_{\rS}^{\gamma_{0}/2}(f).$ Specifically the inequality implies that for $ \ls_{\rS}(f)\leq \tau_{1} $, we must have $ \ls_{\rS'}^{\gamma_{0}/2}(f)= \tau_{1}+O(\sqrt{(\tau_{1}\ln{\left(1/\delta \right)})/m}+\ln{\left(1/\delta \right)}/m)$ as desired. Let us finally remark that applying Hoeffding's inequality would be insufficient to obtain our bounds in that we crucially exploit that Chernoff (or Bernstein's) gives bounds relative to the mean $\mu$.