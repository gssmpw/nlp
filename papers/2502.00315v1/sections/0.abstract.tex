This paper proposes novel methods to enhance the performance of monocular 3D object detection models by leveraging the generalized feature extraction capabilities of a vision foundation model. Unlike traditional CNN-based approaches, which often suffer from inaccurate depth estimation and rely on multi-stage object detection pipelines, this study employs a Vision Transformer (ViT)-based foundation model as the backbone, which excels at capturing global features for depth estimation. It integrates a detection transformer (DETR) architecture to improve both depth estimation and object detection performance in a one-stage manner. Specifically, a hierarchical feature fusion block is introduced to extract richer visual features from the foundation model, further enhancing feature extraction capabilities. Depth estimation accuracy is further improved by incorporating a relative depth estimation model trained on large-scale data and fine-tuning it through transfer learning. Additionally, the use of queries in the transformer's decoder, which consider reference points and the dimensions of 2D bounding boxes, enhances recognition performance. The proposed model outperforms recent state-of-the-art methods, as demonstrated through quantitative and qualitative evaluations on the KITTI 3D benchmark and a custom dataset collected from high-elevation racing environments. Code is
available at \url{https://github.com/JihyeokKim/MonoDINO-DETR}.
%Moreover, ablation studies on the core modules provide evidence of their individual contributions to object detection performance, further substantiating the model's effectiveness.