%%%%%%%%%%%%%%%%%%%%%%%%%%%%%%%%%%%%%%%%%%%%%%%%%%%%%%%%%%%%%%%%%%%%%%%%%%%%%%%%
%2345678901234567890123456789012345678901234567890123456789012345678901234567890
%        1         2         3         4         5         6         7         8

\documentclass[letterpaper, 10 pt, conference]{ieeeconf}  % Comment this line out if you need a4paper
\let\labelindent\relax
%\documentclass[a4paper, 10pt, conference]{ieeeconf}      % Use this line for a4 paper

\IEEEoverridecommandlockouts                              % This command is only needed if 
                                                          % you want to use the \thanks command

\overrideIEEEmargins                                      % Needed to meet printer requirements.

%In case you encounter the following error:
%Error 1010 The PDF file may be corrupt (unable to open PDF file) OR
%Error 1000 An error occurred while parsing a contents stream. Unable to analyze the PDF file.
%This is a known problem with pdfLaTeX conversion filter. The file cannot be opened with acrobat reader
%Please use one of the alternatives below to circumvent this error by uncommenting one or the other
%\pdfobjcompresslevel=0
%\pdfminorversion=4

% See the \addtolength command later in the file to balance the column lengths
% on the last page of the document

% The following packages can be found on http:\\www.ctan.org
\usepackage{cite}
\usepackage{amsmath,amssymb,amsfonts}
\usepackage{graphicx}
\usepackage{textcomp}
\usepackage{xcolor}
\usepackage{gensymb}
\usepackage{multirow}
\usepackage{subcaption}
\usepackage{commath} % for \norm
\usepackage{ushort} % for below bar (minimum value)
\usepackage{url} % for citing website url

\usepackage{epsfig} % for postscript graphics files
\usepackage{balance} % for balancing reference columns

% for quantitative result table
\usepackage{adjustbox}
\usepackage{booktabs, makecell, tabularx}
\usepackage{siunitx}
% \usepackage{SIunits} % for degree per second
\setcellgapes{3.5pt}
\newcommand\mcc[1]{\multicolumn{2}{c}{#1}}% shortcut, added

% for pseudo-code
% \usepackage{algorithm, algorithmicx, algpseudocode}
% \usepackage{algorithm} % for algorithm (no algorithmic!)
\usepackage[ruled, vlined]{algorithm2e}
\usepackage{amssymb}
% \usepackage[ruled]{algorithm2e}
\usepackage[noend]{algpseudocode} % for algorithm
\usepackage{footnote} % for footnote, savenote
\usepackage{float} % for H option in Algorithm, force to place it exactly where it is.
\usepackage{kotex}
\usepackage{wrapfig}
\usepackage{setspace}
\usepackage{hyperref}
\usepackage{url}
\usepackage{setspace}
\usepackage{caption}
\usepackage{tabularx}
\usepackage{tabularray}
\usepackage{enumitem}
\usepackage{eucal}
\usepackage{bbm}
% \usepackage{setspace}
% \linespread{0.8}

%---------

\title{\LARGE \bf
MonoDINO-DETR: Depth-Enhanced Monocular 3D Object Detection Using a Vision Foundation Model
}


\author{Jihyeok Kim$^{1}$, Seongwoo Moon$^{1}$, Sungwon Nah$^{1}$ and David Hyunchul Shim$^{1}$% <-this % stops a space
\thanks{\textsuperscript{1}School of Electrical Engineering, Korea Advanced Institute of Science and Technology, Daejeon, South Korea.
        {\texttt{\{jihyeokkim, seongwoo.moon, sw.nah, hcshim\}@kaist.ac.kr}}}
%\thanks{$^*$Corresponding Author}
\thanks{This work was partially supported by College of Engineering at KAIST, South Korea.}
}

\begin{document}

\maketitle
\thispagestyle{empty}
\pagestyle{empty}

%%%%%%%%%%%%%%%%%%%%%%%%%%%%%%%%%%%%%%%%%%%%%%%%%%%%%%%%%%%%%%%%%%%%%%%%%%%%%%%%

\begin{abstract}
% \begin{abstract}
% Adversarial attacks pose significant threats to deploying Graph Neural Networks (GNNs) in real-world applications. Lines of studies have made progress in minimizing the influence of adversarial perturbations. However, existing methods often rely on fixed priors about the dataset or attacker, limiting their ability to generalize across diverse scenarios. These approaches cannot adaptively learn the intrinsic properties of the dataset.
% In this paper, we propose a novel framework, \ModelName (Graph \textbf{P}urification through t\textbf{R}ansfer \textbf{EN}tropy-guided \textbf{N}on-i\textbf{S}otropic Diffu\textbf{S}ion), which leverages a graph diffusion generative model to learn intrinsic properties and recover the clean structure of adversarial graphs. However, two key challenges arise: (1) The graph diffusion model’s uniform noise injection to all nodes during the forward process can over-perturb the graph, erasing valuable information and making recovery difficult, and (2) the diversity of the diffusion model in the reverse denoising process may cause the generated graph to deviate from the target clean structure.
% To address these challenges, we introduce a LID-Driven Non-Isotropic Diffusion process, which injects noise selectively, focusing on adversarial nodes while preserving the clean structure. Additionally, we propose a Graph Transfer Entropy-Guided Reverse Denoising process that maximizes transfer entropy to reduce uncertainty in the reverse process, ensuring that the generated graph remains aligned with the clean structure.
% Extensive experiments on both graph and node classification demonstrate our proposed \ModelName framework's robustness and superior generalization.
% Our code is available at \textcolor{mytablecolor}{\url{https:///}}
% \end{abstract}


% \begin{abstract}
% % priors and no priors-free 继续总结凝练
% % 鲁棒和各向同性有关
% Adversarial attacks pose significant threats to deploying Graph Neural Networks (GNNs) in real-world applications. Lines of studies have made progress in minimizing the influence of adversarial perturbations. 
% They often rely on priors such as neighbor similarity in clean graphs to restore the correct structure. However, this approach is less effective on datasets where these priors do not hold.
% % Their robustness methods often rely on priors of clean graphs or attacks.
% To achieve more generalized robustness, we need methods that can learn clean graph properties and recover the correct structure based on those learned properties, rather than depending on prior assumptions.
% Driven by this goal, in this work, we approach adversarial attacks from a distribution perspective: these attacks cause the graph distribution to deviate from the original clean distribution.
% % From this perspective, we propose using a graph generative model to learn the clean graph distribution without relying on priors and to purify adversarial graphs through distribution mapping.
% % 前面不要,直接提diffusion
% % Among various graph generative models, the diffusion model’s reverse denoising process naturally aligns with the removal of adversarial perturbations,
% % making it an ideal choice for mapping between adversarial and clean distributions. 
% From this perspective, we propose using the graph diffusion model to learn the clean graph distribution and purify adversarial graphs through distribution mapping.
% % The diffusion model’s reverse denoising process naturally aligns with the removal of adversarial perturbations, making it an ideal choice for mapping between adversarial and clean distributions.
% However, in graph diffusion models, 1) the indiscriminate noise injection across all nodes during the forward process can remove useful information still present in adversarial samples, making it difficult to recover the clean structure during reverse purification. 2) the diversity of the reverse denoising process may cause the generated graph to deviate from the target clean structure.
% To address these challenges, we propose a novel framework, \ModelName, to enhance gra\textbf{P}h rob\textbf{U}stness through t\textbf{R}ansfer \textbf{EN}tropy guid\textbf{E}d non-i\textbf{S}otropic diffu\textbf{S}ion purification.
% Our method introduces a LID-based Non-Isotropic Diffusion process, where we use local intrinsic dimensionality (LID) to estimate the adversarial degree of each node, enabling selective noise injection to focus on adversarial nodes while preserving the clean structure. Additionally, we propose a Graph Transfer Entropy-Guided Denoising process, which maximizes transfer entropy at each step to reduce uncertainty during the reverse process, 
% % ensuring the generated graph stays aligned with the clean structure without deviation.
% ensuring the generated graph matches the target clean graph without deviation.
% Extensive experiments on both graph and node classification tasks demonstrate the robustness of our \ModelName framework. Our code is available at \textcolor{mytablecolor}{\url{https:///}}.
% \end{abstract}

\begin{abstract}
Adversarial evasion attacks pose significant threats to graph learning, with lines of studies that have improved the robustness of Graph Neural Networks (GNNs).
However, existing works rely on priors about clean graphs or attacking strategies, which are often heuristic and inconsistent.
To achieve robust graph learning over different types of evasion attacks and diverse datasets, we investigate this problem from a prior-free structure purification perspective.
Specifically, we propose a novel \underline{\textbf{Diff}}usion-based \underline{\textbf{S}}tructure \underline{\textbf{P}}urification framework named \textbf{\ModelName}, which creatively incorporates the graph diffusion model to learn intrinsic distributions of clean graphs and purify the perturbed structures by removing adversaries under the direction of the captured predictive patterns without relying on priors.
\ModelName~is divided into the forward diffusion process and the reverse denoising process, during which structure purification is achieved.
To avoid valuable information loss during the forward process, we propose an LID-driven non-isotropic diffusion mechanism to selectively inject noise anisotropically.
To promote semantic alignment between the clean graph and the purified graph generated during the reverse process, we reduce the generation uncertainty by the proposed graph transfer entropy guided denoising mechanism.
Extensive experiments demonstrate the superior robustness of \ModelName~against evasion attacks.
% The reverse denoising process of diffusion models naturally aligns with removing graph adversarial perturbations, making them suitable for learning clean graph distribution and removing adversarial perturbations based on the learned distributional patterns without relying on priors.
% purifying adversarial graphs through distribution mapping.
% However, the indiscriminate noise injection in graph diffusion models can erase useful information, while the diversity of the reverse process may cause generated graphs to deviate from the target clean graph, making it difficult to directly apply them for purifying adversarial graph data.
% To address these challenges, 
% In this work, we propose a novel framework \ModelName, which introduces a LID-driven non-isotropic forward diffusion process and a transfer entropy-guided reverse denoising process to precisely remove adversarial perturbations and guide the generation toward the target clean graph.
% Our code is available at \textcolor{mytablecolor}{\url{https:///}}.
\end{abstract}


\keywords{robust graph learning, adversarial evasion attack, graph structure purification, graph diffuison}
\end{abstract}



\section{Introduction}
\IEEEPARstart{I}{n} recent years, flourishing of Artificial Intelligence Generated Content (AIGC) has sparked significant advancements in modalities such as text, image, audio, and even video. 
Among these, AI-Generated Image (AGI) has garnered considerable interest from both researchers and the public.
Plenty of remarkable AGI models and online services, such as StableDiffusion\footnote{\url{https://stability.ai/}}, Midjourney\footnote{\url{https://www.midjourney.com/}}, and FLUX\footnote{\url{https://blackforestlabs.ai/}}, offer users an excellent creative experience.
However, users often remain critical of the quality of the AGI due to image distortions or mismatches with user intentions.
Consequently, methods for assessing the quality of AGI are becoming increasingly crucial to help improve the generative capabilities of these models.

Unlike Natural Scene Image (NSI) quality assessment, which focuses primarily on perception aspects such as sharpness, color, and brightness, AI-Generated Image Quality Assessment (AGIQA) encompasses additional aspects like correspondence and authenticity. 
Since AGI is generated on the basis of user text prompts, it may fail to capture key user intentions, resulting in misalignment with the prompt.
Furthermore, authenticity refers to how closely the generated image resembles real-world artworks, as AGI can sometimes exhibit logical inconsistencies.
While traditional IQA models may effectively evaluate perceptual quality, they are often less capable of adequately assessing aspects such as correspondence and authenticity.

\begin{figure}\label{fig:radar}
    \centering
    \includegraphics[width=1.0\linewidth]{figures/radar_plot.pdf}
    \caption{A comparison on quality, correspondence, and authenticity aspects of AIGCIQA2023~\cite{wang2023aigciqa2023} dataset illustrates the superior performance of our method.}
\end{figure}

Several methods have been proposed specifically for the AGIQA task, including metrics designed to evaluate the authenticity and diversity of generated images~\cite{gulrajani2017improved,heusel2017gans}. 
Nevertheless, these methods tend to compare and evaluate grouped images rather than single instances, which limits their utility for single image assessment.
Beginning with AGIQA-1k~\cite{zhang2023perceptual}, a series of AGIQA databases have been introduced, including AGIQA-3k~\cite{li2023agiqa}, AIGCIQA-20k~\cite{li2024aigiqa}, etc.
Concurrently, there has been a surge in research utilizing deep learning methods~\cite{zhou2024adaptive,peng2024aigc,yu2024sf}, which have significantly benefited from pre-trained models such as CLIP~\cite{radford2021learning}. 
These approaches enhance the analysis by leveraging the correlations between images and their descriptive texts.
While these models are effective in capturing general text-image alignments, they may not effectively detect subtle inconsistencies or mismatches between the generated image content and the detailed nuances of the textual description.
Moreover, as these models are pre-trained on large-scale datasets for broad tasks, they might not fully exploit the textual information pertinent to the specific context of AGIQA without task-specific fine-tuning.
To overcome these limitations, methods that leverage Multimodal Large Language Models (MLLMs)~\cite{wang2024large,wang2024understanding} have been proposed.
These methods aim to fully exploit the synergies of image captioning and textual analysis for AGIQA.
Although they benefit from advanced prompt understanding, instruction following, and generation capabilities, they often do not utilize MLLMs as encoders capable of producing a sequence of logits that integrate both image and text context.

In conclusion, the field of AI-Generated Image Quality Assessment (AGIQA) continues to face significant challenges: 
(1) Developing comprehensive methods to assess AGIs from multiple dimensions, including quality, correspondence, and authenticity; 
(2) Enhancing assessment techniques to more accurately reflect human perception and the nuanced intentions embedded within prompts; 
(3) Optimizing the use of Multimodal Large Language Models (MLLMs) to fully exploit their multimodal encoding capabilities.

To address these challenges, we propose a novel method M3-AGIQA (\textbf{M}ultimodal, \textbf{M}ulti-Round, \textbf{M}ulti-Aspect AI-Generated Image Quality Assessment) which leverages MLLMs as both image and text encoders. 
This approach incorporates an additional network to align human perception and intentions, aiming to enhance assessment accuracy. 
Specially, we distill the rich image captioning capability from online MLLMs into a local MLLM through Low-Rank Adaption (LoRA) fine-tuning, and train this model with human-labeled data. The key contributions of this paper are as follows:
\begin{itemize}
    \item We propose a novel AGIQA method that distills multi-aspect image captioning capabilities to enable comprehensive evaluation. Specifically, we use an online MLLM service to generate aspect-specific image descriptions and fine-tune a local MLLM with these descriptions in a structured two-round conversational format.
    \item We investigate the encoding potential of MLLMs to better align with human perceptual judgments and intentions, uncovering previously underestimated capabilities of MLLMs in the AGIQA domain. To leverage sequential information, we append an xLSTM feature extractor and a regression head to the encoding output.
    \item Extensive experiments across multiple datasets demonstrate that our method achieves superior performance, setting a new state-of-the-art (SOTA) benchmark in AGIQA.
\end{itemize}

In this work, we present related works in Sec.~\ref{sec:related}, followed by the details of our M3-AGIQA method in Sec.~\ref{sec:method}. Sec.~\ref{sec:exp} outlines our experimental design and presents the results. Sec.~\ref{sec:limit},~\ref{sec:ethics} and~\ref{sec:conclusion} discuss the limitations, ethical concerns, future directions and conclusions of our study.
\section{Related Works}
\label{sec:relatedworks}

\subsection{Monocular 3D Object Detection}
Monocular 3D Object Detection models aim to detect the 3D bounding boxes of target objects from a single input image. These models can be divided into three categories: 2D-detector-based, depth-image-based, and transformer-based.

\textbf{2D-Detector-Based M3OD} typically begins by localizing 2D bounding boxes and subsequently estimating 3D bounding boxes using geometric relationships or predefined 2D-3D box constraints, as demonstrated in M3D-RPN \cite{brazil2019m3d}.
%Alternatively, SMOKE \cite{liu2020smoke} formulates 3D object detection as a keypoint-based regression task in a single-stage pipeline.
MonoGround \cite{qin2022monoground} incorporates the ground plane beneath objects as a prior, transforming the ill-posed 2D-to-3D mapping problem into a more constrained and solvable task. However, these methods show poor performance due to inaccurate depth estimation and lack generalizability in diverse circumstances, as constraints like flat ground may not apply in high-elevation environments.

\textbf{Depth-Image-Based M3OD} such as D\textsuperscript{4}LCN \cite{ding2020learning} and DDMP-3D \cite{wang2021depth} first estimate depth maps from RGB images using a pre-trained depth generator while also utilizing the RGB image in a visual backbone network to extract visual features. Both types of features are extracted using Convolutional Neural Network (CNN)-based modules and fused to estimate 3D bounding boxes. 
However, as the entire network is CNN-based, it struggles to capture the global context of the image, leading to suboptimal performance. Additionally, these models often require ground-truth depth map data, which further limits their applicability.

\textbf{Transformer-Based M3OD} methods have been recently proposed, showing promising performance. MonoDTR \cite{huang2022monodtr} exploits a Transformer \cite{vaswani2017attention} encoder-decoder architecture to globally integrate context and depth-aware features, requiring LiDAR data for auxiliary supervision. In contrast, MonoDETR \cite{zhang2023monodetr} uses the DETR \cite{carion2020end} architecture to predict 3D bounding boxes and estimates foreground depth maps for supervision without relying on additional data. Both models utilize depth distributions for each pixel, following the approach introduced in CaDDN \cite{reading2021categorical} for supervision. Although these transformer-based models improve global feature extraction for depth estimation, they still face limitations due to their reliance on CNN backbones, which struggle to effectively capture global features.

\subsection{Detection Transformer}
DETR\cite{carion2020end} is an end-to-end object detector that eliminates the need for hand-crafted anchor boxes by combining a CNN backbone with a transformer architecture to model global relationships for object detection. Despite its strong performance, DETR faces challenges such as slower training convergence due to its computational complexity and inefficiency in matching object queries.
%Deformable-DETR\cite{zhu2020deformable} addresses these issues by introducing deformable attention, which reduces the computational burden and enhances performance by focusing attention on a small set of sampling locations and leveraging multi-scale feature maps.
DAB-DETR\cite{liu2022dab} improves DETR's efficiency and accuracy by introducing 4D anchor-based queries that are dynamically updated during decoding, enabling effective bounding box refinement. The proposed model extends this idea by incorporating 6D dynamic anchor boxes to better handle asymmetric shapes which achieves improved performance in M3OD task.

\subsection{Vision Foundation Model}
Vision Foundation Models (VFMs) are large-scale pre-trained models designed for versatile vision tasks, such as object detection, semantic segmentation, and depth estimation, leveraging extensive datasets like ImageNet-1k \cite{russakovsky2015imagenet}. Unlike traditional CNN-based backbones, such as ResNet \cite{he2016deep} and DenseNet \cite{huang2017densely}, VFMs like CLIP \cite{radford2021learning}, DINO \cite{caron2021emerging}, SAM \cite{kirillov2023segment}, and DINOv2 \cite{oquab2023dinov2} are based on ViT \cite{dosovitskiy2020image} backbones, which provide richer contextual features for a wide range of applications. In this paper, DINOv2 is selected as the backbone for the M3OD task to enhance depth estimation and 3D object detection performance.
\section{Methodology}

\method consists of three key components.
(1) A hierarchical linguistic structure with supporting corpora for linguistic mechanism analysis;
(2) Linguistic feature analysis for interpreting SAE extracted features; and
(3) Linguistic feature intervention for causal analysis and LLM steering.


\begin{figure*}[tp]
    \centering
    \includegraphics[width=0.97\textwidth]{figure/methology.pdf}
    \vspace{-0.01in}
    \caption{
    The overall framework of \method.
    We propose a large-model linguistic mechanism framework encompassing six dimensions and select classical features from these dimensions for experimentation. 
    The experimental workflow is as follows: 
    (1) Construct minimal contrast and counterfactual datasets; 
    (2) Extract features and evaluate their relevance by analyzing the activation values of base vectors on the datasets; 
    (3) Intervene in the model output by modifying activation values and assess causality using an LLM as a judge.
    }\label{fig:method}
\end{figure*}


\subsection{Linguistic Structure}

\paragraph{Hierarchical Linguistic Structure.}
To systematically interpret the language capabilities of large models, we adopt a six level structure based on theoretical linguistics~\cite{fromkin2017introduction}: phonetics, phonology, morphology, syntax, semantics, and pragmatics.
The structure follows a logical progression from the external, physical realization of sound to the internal, contextual understanding of meaning. 
Each linguistic capability contains several concrete linguistic features, \textit{e.g.,} semantics level includes metaphor, simile, \textit{etc}.
We provide the exact definition for each linguistic capability in Appendix~\ref{app:ling}

% Our structure provides a comprehensive and modular way to explain how large language models achieve different levels of language ability. 
% By finding linguistic features at different levels in the SAE latent space of large models, we can more accurately reveal how these models represent and process natural language, thereby revealing the underlying mechanism of the models' language ability.
% This mechanism can also bring linguists a clearer understanding of how language knowledge is organized.

\paragraph{Dataset Construction.}
The sparse feature activation distribution of SAE is closely related to the conditions under which their corresponding linguistic features hold in linguistic knowledge.
To find the linguistic features and evaluate its dominance, we propose a method to construct the dataset and analyze feature activation frequencies.

For each linguistic feature, we first construct a set of sentences that significantly align with the desired feature. 
The feature activation representing this linguistic feature in SAE’s hidden space will be significantly activated on these sentences. 
However, this is not enough to accurately identify them, as there are some background noise vectors that are activated on all sentences in the dataset and interfere with our judgment. 
We need to include a control group without the feature in the constructed sentences. 

We introduce two types of control groups: minimal pairs and counterfactual sentences. Minimal pairs are constructed by changing only the part of a sentence that corresponds to a particular linguistic feature, while keeping all other parts unchanged. However, this approach often results in syntactically incorrect sentences.

To overcome this limitation, we also construct fully grammatically correct control groups, called counterfactual sentences, which differ from the original sentence only in terms of its linguistic features. Detailed dataset construction procedures are provided in Appendix~\ref{app:data_construction}.

\subsection{Feature Analysis}
We propose a causal probability approach to evaluate the relationship between extracted linguistic features and their activation on sentences containing those features. 

For a given feature \(x\), we define two key probabilities. The \emph{Probability of Necessity} (PN) quantifies how necessary the feature is for the activation of a corresponding base vector, while the \emph{Probability of Sufficiency} (PS) measures the likelihood that introducing the feature triggers activation. These probabilities are then combined into a \emph{Feature Representation Confidence} (FRC) score, which assesses both the representational capacity of the SAE latent space and the discriminative ability of the feature to identify the corresponding linguistic phenomenon. 

During feature analysis, we calculate the FRS on both the minimal contrast dataset and the counterfactual dataset, then average the results. This average more accurately reflects the ability of the base vectors to represent the linguistic features. Detailed definitions and calculation methods are provided in Appendix~\ref{app:frc}.



\subsection{Feature Intervention}
When we modify the values of SAE’s activation during forward propagation, we expect that such targeted interventions will influence the model’s behavior. 
However, our experiments show that altering only a small subset of features may not significantly impact the output—likely because linguistic phenomena are represented by multiple features across various layers. 
To assess the true impact of these interventions, we use a large language model as a judge. For each linguistic feature, we conduct both ablation and enhancement experiments. 
In the ablation experiment, we set the target feature’s activation to $0$, and in the enhancement experiment, we set it to $10$. 
In both cases, we also perform baseline experiments by randomly selecting 25 base vectors from the same layer.

For brevity, we denote the interventions as follows: let \(I_{abl}^{T}\) denote the targeted ablation intervention, \(I_{abl}^{B}\) the baseline ablation intervention, \(I_{enh}^{T}\) the targeted enhancement intervention, and \(I_{enh}^{B}\) the baseline enhancement intervention.

Let \(P_{abl}^{T}\) and \(P_{abl}^{B}\) denote the success probabilities (\textit{i.e.,} the probability that the intended change in the linguistic phenomenon is observed) for the targeted and baseline ablation experiments. The normalized ablation effect is then defined as
\[
\begin{aligned}
E_{abl} &= P_{abl}^{T} - P_{abl}^{B} \\
        &= \frac{P(Y=0 \mid I_{abl}^{T}) - P(Y=0 \mid I_{abl}^{B})}{P(Y=0 \mid I_{abl}^{T})}.
\end{aligned}
\]
Similarly, let \(P_{enh}^{T}\) and \(P_{enh}^{B}\) be the success probabilities for the targeted and baseline enhancement experiments, with \(Y=1\) indicating the presence of the phenomenon. The normalized enhancement effect is given by
\[
\begin{aligned}
E_{enh} &= P_{enh}^{T} - P_{enh}^{B} \\
        &= \frac{P(Y=1 \mid I_{enh}^{T}) - P(Y=1 \mid I_{enh}^{B})}{1 - P(Y=1 \mid I_{enh}^{B})}.
\end{aligned}
\]

Finally, we define the Feature Intervention Confidence (FIC) score as the harmonic mean of the normalized ablation and enhancement effects:
\[
\text{FIC} = \frac{2\, E_{abl}\, E_{enh}}{E_{abl} + E_{enh}}.
\]
When calculating FIC, if one or both of the $E$ values are negative, we incorporate a penalty coefficient $w$ to reflect the weakened or lost causality in such cases. 
This FIC score provides a balanced measure of how effectively targeted interventions, as opposed to random ones, influence the model’s output with respect to specific linguistic features.
The details for FIC are shown in Appendix~\ref{app:fic}.
% The detailed computation can be found in Appendix~\ref{app:fic}.
\vspace{-1mm}
\section{Experiments}
In this section, we evaluate our \model framework on three distinct research problems: 1) Self-Supervised Representation Learning, 2) Few-Shot Transfer, and 3) Multimodal Generative Tasks. 
Table~\ref{tab:dataset} lists all 14 datasets used in the experiments.
\vspace{-1mm}
\subsection{Self-Supervised Representation Learning}
\label{sec:lp}
\vpara{Setup.}
We adopt the widely used linear probing protocol to evaluate the representation learning capability of self-supervised pre-trained models on unseen datasets. Specifically, we train a linear classifier on top of the embeddings generated by a frozen pre-trained model. Our model, along with all self-supervised learning baselines, is first jointly pre-trained on ogbn-Product, ogbn-Papers100M, Goodreads-LP, and Amazon-Cloth. We then evaluate the pre-trained models on each individual dataset. Detailed settings and hyperparameters are provided in Appendix~\ref{appendix:imple}.

For the baselines, we compare \model with state-of-the-art generative graph self-supervised learning methods, GraphMAE2~\cite{hou2023graphmae2}, and contrastive methods, BGRL~\cite{thakoor2021bootstrapped}. As these methods are not inherently designed for cross-domain tasks, we leverage CLIP~\cite{radford2021learning} to unify the input node features across different graphs. We also include a comparison with a multi-graph pre-training method, GCOPE~\cite{zhao2024all}. \model and all baseline methods utilize GAT~\cite{velivckovic2018graph} as the backbone GNN. 
For baselines that use TAGs as input, we select GIANT-XRT~\cite{zhaolearning} and UniGraph~\cite{he2024unigraphlearningunifiedcrossdomain}. Since these methods cannot process image data, they rely solely on text from MMG as node features, ignoring image inputs. For baseline approaches that accept multimodal data, we choose widely used multimodal models, CLIP~\cite{radford2021learning} and ImageBind~\cite{girdhar2023imagebind}. To maintain consistency with the baselines, \model also uses CLIP's pre-trained vision and text encoders as Modality-Specific Encoders.


Our objective is to develop a general embedding model capable of generating high-quality representations for any MMG. To assess this, we evaluate the performance of \model and the baselines in three different settings: (1) \textit{In-distribution}, where models are pre-trained on multiple datasets and evaluated on each corresponding dataset individually; (2) \textit{In-domain Generalization}, which tests pre-trained models on target datasets from the same domain as one of the pre-training datasets; and (3) \textit{Out-of-domain Generalization}, where models are evaluated on datasets from domains unseen during pre-training.

\vpara{Research Questions.} In this subsection, we aim to answer the following research questions: 
\begin{itemize}[leftmargin=*,itemsep=0pt,parsep=0.2em,topsep=0.3em,partopsep=0.3em]
    \item \textbf{RQ1: Negative Transfer in Multi-Graph Pre-Training.} How do existing graph pre-training methods, which are primarily designed for single-graph pre-training, perform when applied to multi-graph pre-training, and how do they compare to our proposed \model?
    \item \textbf{RQ2: Comparison to Other Foundation Models.} How does \model, which takes both multimodal data and graph structures as input, perform compared to methods that consider only multimodal data (CLIP, ImageBind) or only TAGs (UniGraph)?
    \item \textbf{RQ3: Generalization Capability.} How does \model, designed as a foundation model, perform in terms of generalizing to unseen graphs, and how does it compare to methods trained directly on the target graphs?
\end{itemize}

\begin{table*}[t]\footnotesize
    \centering
    \renewcommand\tabcolsep{3.5pt}
    \caption{\textbf{Experiment results in few-shot transfer.} We report accuracy (\%) for node/edge classification tasks. \model and other self-supervised baselines (rows in white) are jointly pre-trained on Product, Papers100M, Goodreads-NC and Amazon-Cloth, and then evaluated on the individual target dataset. \textit{"In-domain Generalization"} tests on target datasets from the same domain as one of the pre-training datasets. \textit{"Out-of-domain Generalization"} evaluates on datasets from domains not seen during pre-training. The performance of methods that are direcly pre-trained on the individual target dataset, is marked in \colorbox{Gray}{gray}. 
    }
    \vskip -0.10in
    \label{tab:fwt}
    \begin{tabular}{lcccccccccccccccccc}
    \toprule[1.1pt]
    & \multicolumn{12}{c}{\textbf{In-domain Generalization}}& \multicolumn{6}{c}{\textbf{Out-of-domain Generalization}}\\
   \cmidrule(lr){2-13}\cmidrule(lr){14-19}
        & \multicolumn{2}{c}{Cora-5-way} & \multicolumn{2}{c}{PubMed-2-way} & \multicolumn{2}{c}{Arxiv-5-way} & \multicolumn{3}{c}{Goodreads-NC-5-way} & \multicolumn{3}{c}{Ele-fashion-5-way} & \multicolumn{2}{c}{Wiki-CS-5-way} & \multicolumn{2}{c}{FB15K237-20-way} & \multicolumn{2}{c}{WN18RR-5-way} \\
    \cmidrule(lr){2-3}\cmidrule(lr){4-5}\cmidrule(lr){6-7}\cmidrule(lr){8-10}\cmidrule(lr){11-13}\cmidrule(lr){14-15}\cmidrule(lr){12-13}\cmidrule(lr){14-15}\cmidrule(lr){16-17}\cmidrule(lr){18-19}
    &5-shot & 1-shot  & 5-shot & 1-shot  &5-shot & 1-shot  &5-shot& 3-shot & 1-shot  &5-shot & 3-shot& 1-shot  & 5-shot & 1-shot  &5-shot & 1-shot  & 5-shot & 1-shot \\
    \midrule
    \multicolumn{10}{l}{\textbf{Use CLIP to encode raw multimodal data as input features.}} \\ 
    NoPretrain & 41.09 & 27.05 & 59.81 & 55.28 & 63.78 & 41.10 & 41.64 & 40.01 & 31.04 & 63.96 & 58.32 & 47.48 & 52.29 & 32.94 & 72.97 & 47.01 & 50.75 & 30.11  \\
    BGRL & 52.01 & 35.18 & 66.04 & 59.04 & 60.12 & 46.67 & 47.01 & 44.22 & 30.35 & 64.72 & 60.16 & 46.49 & 52.10 & 32.85 & 75.39 & 45.15 & 47.42 & 34.57 \\
    % \rowcolor{Gray} BGRL \\
    GraphMAE2 & 52.89 & 36.25 & 66.89 & 59.95 & 60.91 & 47.29 & 47.84 & 44.80 & 30.93 & 65.52 & 60.92 & 47.24 & 52.83 & 33.41 & 75.95 & 45.81 & 48.14 & 35.21 \\
    Prodigy & 53.01 & 39.59 & 69.11 & 60.42 & 63.53 & \underline{51.33} & \underline{50.01} & \underline{46.39} & 34.98 & 67.35 & 63.87 & 50.79 & 55.94 & 36.35 & 78.01 & 51.39 & 54.94 & 38.73 \\
    \rowcolor{Gray} OFA & 53.11 & 40.04 & 69.45 & \underline{60.38} & 63.11 & 50.25 & 49.61 & 46.24 & \underline{35.14} & \underline{67.94} & \underline{64.18} & \underline{51.35} & \underline{56.01} & \underline{37.02} & \underline{78.33} & 52.02 & 55.05 & 39.11 \\
    % \rowcolor{Gray} GraphMAE2 \\
    GCOPE & 51.98 & 36.14 & 66.25 & 59.16 & 60.29 & 47.19 & 48.52 & 44.89 & 31.20 & 65.10 & 61.33 & 48.51 & 53.74 & 34.19 & 76.10 & 48.93 & 50.19 & 35.05 \\
    \midrule
    \multicolumn{10}{l}{\textbf{Use raw text as input features.}} \\
    GIANT-XRT   & 50.11 & 37.85 & 68.19 & 58.78 & 62.01 & 49.01 & 46.01 & 43.86 & 30.01 & 62.97 & 61.21 & 47.76 & 54.01 & 35.04 & 76.09 & 50.25 & 53.01 & 35.19\\
    % +GraphMAE2 &  \\
    UniGraph & \underline{54.23} & \underline{40.45} & \underline{70.21} & 60.19 & \underline{64.76} & 50.63 & 46.19 & 44.01 & 33.53 & 66.21 & 62.04 & 50.17 & 56.16 & 37.19 & 78.21 & \underline{52.19} & \underline{55.18} & \underline{39.18}\\
    % \rowcolor{Gray} UniGraph  &  \\
    \midrule
    \multicolumn{10}{l}{\textbf{Use raw multimodal data as input features.}} \\
    CLIP & 41.23 & 28.41 & 61.67 & 55.71 & 63.46 & 40.14 & 41.24 & 40.11 & 30.97 & 62.51 & 58.23 & 46.15 & 51.69 & 31.61 & 72.31 & 47.14 & 50.83 & 31.35 \\
    ImageBind & 32.19 & 23.90 & 58.20 & 54.24 & 62.48 & 38.17 & 29.10 & 28.14 & 21.42 & 51.25 & 48.05 & 44.93 & 48.14 & 30.28 & 69.12 & 41.80 & 41.24 & 26.91 \\
    \hdashline
    NoPretrain & 42.41 & 28.39 & 60.78 & 55.90 & 64.29 & 41.98 & 42.21 & 41.20 & 31.14 & 64.15 & 58.91 & 47.90 & 52.90 & 33.14 & 74.10 & 48.11 & 51.92 & 31.84  \\
    \model & \textbf{56.01} & \textbf{42.98} & \textbf{72.19} & \textbf{61.24} & \textbf{66.24} & \textbf{51.98} & \textbf{51.73} & \textbf{47.42} & \textbf{37.01} & \textbf{69.29} & \textbf{65.29} & \textbf{53.85} & \textbf{57.28} & \textbf{38.47} & \textbf{79.34} & \textbf{52.19} & \textbf{55.59} & \textbf{39.93}\\
    % \rowcolor{Gray} \model  &  \\
    \bottomrule[1.1pt]
    \end{tabular}
    \vspace{-4.6mm}
\end{table*}




\vpara{Results.}
Table~\ref{tab:ssrl} presents the results.
We interpret these results by answering three research questions:
\begin{itemize}[leftmargin=*,itemsep=0pt,parsep=0.2em,topsep=0.3em,partopsep=0.3em]
    \item \textbf{RQ1: Negative Transfer in Multi-Graph Pre-Training.} Existing graph pre-training methods exhibit negative transfer when applied to multi-graph pre-training, whereas \model shows improvements in this context. The results in the \textit{In-distribution} setting demonstrate that both BGRL and GraphMAE2 experience a significant performance drop when pre-trained on multi-graphs (rows in white), compared to pre-training on single graph only (rows in gray). This suggests that pre-training on other datasets negatively affects performance on the target dataset. However, UniGraph2 shows improvement under multi-graph pre-training, indicating that it successfully addresses the shortcomings of existing graph pre-training algorithms struggling with multi-graphs.
    \item \textbf{RQ2: Comparison to Other Foundation Models.} UniGraph2 outperforms methods that consider only multimodal data (CLIP, ImageBind) or only TAGs (UniGraph). We observe that without considering the graph structure, the performance of the acknowledged powerful multimodal foundation models like CLIP is not comparable to UniGraph2. Meanwhile, UniGraph, which cannot process image data, also shows less ideal results due to the lack of information. This further highlights the necessity of designing foundation models specifically for multimodal graphs.
    \item \textbf{RQ3: Generalization Capability.} Compared to baseline methods, UniGraph2 demonstrates strong generalization capabilities. The results in the \textit{In-domain Generalization} and \textit{Out-of-domain Generalization} settings show that UniGraph2 effectively transfers knowledge from pre-training to unseen graphs. Compared to the NoPretrain method, UniGraph2 shows significant improvements. The consistent performance gains indicate that UniGraph2 can extract meaningful patterns during pre-training, which are beneficial for tackling graph learning tasks. Furthermore, UniGraph2 is comparable to methods trained directly on the target datasets, achieving similar accuracy while benefiting from greater efficiency without requiring exhaustive task-specific training.
\end{itemize}










\vspace{-2.8mm}
\subsection{Few-Shot Transfer}
\vpara{Setup.}
In this part, we evaluate the ability of the pre-trained models to perform few-shot in-context transfer without updating the model parameters. 
For baseline methods, in addition to the pre-trained models mentioned in Section~\ref{sec:lp}, we also compare two recent graph in-context learning methods: the self-supervised pre-training method Prodigy~\cite{huang2024prodigy} and the supervised pre-training method OFA~\cite{liuone}.


For evaluation, we strictly follow the setting of Prodigy~\cite{huang2024prodigy}. 
For an N-way K-shot task, we adopt the original train/validation/test splits in each downstream classification dataset, and construct a $K$-shot prompt for test nodes (or edges) from the test split by randomly selecting $K$ examples per way from the train split. By default in all experiments, we sample 500 test tasks.

We adopt the few-shot classification strategy in UniGraph~\cite{he2024unigraphlearningunifiedcrossdomain} for \model. The model computes average embeddings for each class and assigns a query sample to the class with the highest similarity to its embedding.

% \vpara{Research Questions.}
% In this subsection, we aim to answer the following research questions: 
% \begin{itemize}[leftmargin=*,itemsep=0pt,parsep=0.2em,topsep=0.3em,partopsep=0.3em]
%     \item \textbf{RQ1:} How does \model, which takes both multimodal data and graph structures as input, perform in terms of few-shot transfer capabilities compared to foundation models that consider only multimodal data (CLIP, ImageBind) or only TAGs (UniGraph)?
%     \item \textbf{RQ2:} How does \model perform compared to other graph few-shot learning methods?
% \end{itemize}
\vpara{Results.}
In Table~\ref{tab:fwt}, our \model model consistently outperforms all the baselines. This further demonstrates the powerful generalization capabilities of UniGraph2 as a foundation model.
In particular, compared to other graph few-shot learning methods such as Prodigy, OFA, and GCOPE, UniGraph2 does not rely on complex prompt graph designs, and its simple few-shot strategy is both efficient and effective.


\begin{table*}[t]
\centering
 \renewcommand\tabcolsep{4.3pt}
\caption{Experiment results in multimodal generative tasks. We strictly follow the setting in MMGL~\cite{yoon2023multimodal}. The task is to generate a single sentence that summarizing the content of a particular section. The summary is generated based on all images and (non-summary) text present in the target and context sections. We provide different information of MMGs to the base LM: (1) section all (text + image), (2) page text, and (3) page all (all texts and images). We encode multiple multimodal neighbor information using three different neighbor encodings methods: \textit{Self-Attention with Text+Embeddings (SA-TE)}, \textit{Self-Attention with Embeddings (SA-E)}, and \textit{Cross-Attention with Embeddings (CA-E)}.}
\vskip -0.10in
\label{tab:gen}
\begin{tabular}{llcccccccccccc}
\toprule[1.1pt]
& & \multicolumn{4}{c}{BLEU-4} & \multicolumn{4}{c}{ROUGE-L} & \multicolumn{4}{c}{CIDEr} \\
\cmidrule(lr){3-6}\cmidrule(lr){7-10}\cmidrule(lr){11-14}
Input Type & Method & SA-TE & SA-E & CA-E  & Avg. gain & SA-TE & SA-E & CA-E  & Avg. gain & SA-TE & SA-E & CA-E & Avg. gain\\
\midrule
\multirow{2}{*}{Section all} & MMGL & 8.03 & 7.56 & 8.35 & - & 40.41 & 39.89 & 39.98 & - & 77.45 & 74.33 & 75.12 & - \\
& +\model & \textbf{9.24} & \textbf{9.01} & \textbf{9.39} & 15.57\% & \textbf{43.01} & \textbf{43.24} & \textbf{42.98} & 7.44\% & \textbf{81.15} & \textbf{80.39} & \textbf{81.91} & 7.32\% \\
\midrule
\multirow{2}{*}{Page text} & MMGL & 9.81 & 8.37 & 8.47 & - & 42.94 & 40.92 & 41.00 & & 92.71 & 80.14 & 80.72 & - \\
& +\model & \textbf{10.31} & \textbf{10.10} & \textbf{9.98} & 14.53\% & \textbf{43.19} & \textbf{43.08} & \textbf{42.75} &3.38\% & \textbf{93.19} & \textbf{90.41} & \textbf{93.11} & 9.56\% \\
\midrule
\multirow{2}{*}{Page all} & MMGL & 9.96 & 8.58 & 8.51 & - & 43.32 & 41.01 & 41.55 & - & 96.01 & 82.28 & 80.31 & - \\
& +\model & \textbf{10.12} & \textbf{10.05} & \textbf{10.33} & 13.38\% & \textbf{44.10} & \textbf{42.08} & \textbf{42.44} & 2.18\% & \textbf{96.32} & \textbf{91.24} & \textbf{94.15} & 9.49\% \\
% \midrule
% Max input length &  \\
    \bottomrule[1.1pt]
\end{tabular}
\vspace{-3mm}
\end{table*}


\vspace{-5mm}
\subsection{Multimodal Generative Tasks}
\vpara{Setup.}
\model is designed as a general representation learning model. The embeddings it generates can be utilized by various generative foundation models, such as LLMs, to empower downstream generative tasks. 
% \model is a general embedding model designed to generate embeddings that can be used by various generative foundation models, such as LLMs, to enhance downstream generative tasks. 
To further demonstrate this, we select the section summarization task on the WikiWeb2M dataset for our experiments.
The WikiWeb2M dataset~\cite{burns2023suite} is designed for multimodal content understanding, using many-to-many text and image relationships from Wikipedia. It includes page titles, section titles, section text, images, and indices for each section.
In this work, we focus on section summarization, where the task is to generate a summary sentence from section content using both text and images.

% \todo{how mmgl do}
For the experiments, we follow the MMGL~\cite{yoon2023multimodal} setup, using four types of information: section text, section images, context text, and page-level text/images. 
Consistent with MMGL, we fine-tune Open Pre-trained Transformer (OPT-125m)~\cite{zhang2022opt} to read the input section text/images and generate a summary. Multimodal neighbors are first encoded using frozen vision/text encoders and then aligned to the text-only LM space using 1-layer MLP mapper.
In MMGL, CLIP~\cite{radford2021learning} encoders are used for text and image encoding, remaining frozen during fine-tuning. In our experiments, we replace CLIP embeddings with our \model embeddings.

% \vpara{Research Questions.}
% In this subsection, we aim to answer the following research question: 
% \begin{itemize}[leftmargin=*,itemsep=0pt,parsep=0.2em,topsep=0.3em,partopsep=0.3em]
%     \item \textbf{RQ1:} How do the embeddings generated by \model perform on generative tasks compared to multimodal foundation models like CLIP?
% \end{itemize}


\vpara{Results.}
Table~\ref{tab:gen} shows that under different input types and different neighbor encoding strategies, the embeddings generated by UniGraph2 bring significant improvements compared to MMGL's default CLIP embeddings. 
We also observe that UniGraph2's embeddings are more robust to different neighbor encoding strategies compared to CLIP and do not rely on a specific strategy.



\begin{table}[t]%\small
\centering
\renewcommand\tabcolsep{1.6pt}
\caption{\textbf{Ablation studies on \model key components.}}
\vskip -0.1in
\label{tab:kc}
\begin{tabular}{lcccc}
\toprule[1.1pt]
    & Products & Amazon-Cloth & Goodreads-NC  & WN18RR \\
\midrule
    \model & \textbf{82.79{\tiny$\pm$0.02}} & \textbf{24.64{\tiny$\pm$0.09}} & \textbf{81.15{\tiny$\pm$0.12}} & \textbf{85.47{\tiny$\pm$0.11}}\\
    w/o MoE & 81.01{\tiny$\pm$0.10} & 21.33{\tiny$\pm$0.04} & 80.10{\tiny$\pm$0.04} & 83.99{\tiny$\pm$0.21}\\
    w/o feat loss& 69.12{\tiny$\pm$0.09} & 18.43{\tiny$\pm$0.24} & 68.12{\tiny$\pm$0.01} & 74.11{\tiny$\pm$0.03}\\
    w/o SPD loss& 82.42{\tiny$\pm$0.11} & 23.39{\tiny$\pm$0.05} & 80.24{\tiny$\pm$0.02} & 85.24{\tiny$\pm$0.11}\\
\bottomrule[1.1pt]
\end{tabular}
\vspace{-3.3mm}
\end{table}

\begin{table}[t]%\small
\centering
\renewcommand\tabcolsep{2.4pt}
\caption{\textbf{Ablation studies on Modality-Specific Encoders.}}
\vskip -0.1in
\label{tab:enc}
\begin{tabular}{lcccc}
\toprule[1.1pt]
    & Products & Amazon-Cloth & Goodreads-NC  & WN18RR \\
\midrule
    CLIP & 82.79{\tiny$\pm$0.02} & 24.64{\tiny$\pm$0.09} & 81.15{\tiny$\pm$0.12} & \textbf{85.47{\tiny$\pm$0.11}}\\
    ImageBind & 82.32{\tiny$\pm$0.05} & \textbf{25.01{\tiny$\pm$0.11}} & 80.33{\tiny$\pm$0.22} & 84.29{\tiny$\pm$0.07}\\
    T5+ViT& \textbf{82.99{\tiny$\pm$0.04}} & 24.38{\tiny$\pm$0.28} & \textbf{81.28{\tiny$\pm$0.11}} & 84.16{\tiny$\pm$0.04}\\
\bottomrule[1.1pt]
\end{tabular}
\vspace{-4.8mm}
\end{table}

\subsection{Model Analysis}
We select four datasets from different domains to conduct more in-depth studies. We adopt self-supervised representation learning for evaluation.

\vpara{Ablation on Key Components.}
Table~\ref{tab:kc} shows the performance of the \model framework after removing some key designs. "W/o MoE" represents that we use simple MLP instead MoE to align node features. 
"W/o feat loss" represents that we only use the SPD loss for pre-training, while "w/o SPD loss" refers to the opposite.
The overall results confirm that all key designs contribute positively to the performance of \model.

\vpara{Ablation on Modality-Specific Encoders}
In Table~\ref{tab:enc}, we study the influence of different Modality-Specific Encoders on the performance of encoding raw multimodal data. CLIP and ImageBind are feature encoders that map features from various modalities to a shared embedding space, whereas T5+ViT employs SOTA embedding methods for each modality independently, without specific alignment. The results show that all methods achieve comparable performance, indicating that \model effectively aligns features regardless of whether they have been pre-aligned or not.

\begin{table}[t] \scriptsize
\centering
\renewcommand\tabcolsep{3.5pt}
\caption{\textbf{Comparison of GPU hours and performance on ogbn-Arxiv and ogbn-Papers100M.}}
\vskip -0.1in
\label{tab:ccp}
\begin{tabular}{ccccc}
\toprule[1.1pt]
Method & Pre-training & Downstream Training & Downstream Inference & Test Accuracy \\
\midrule
\multicolumn{5}{l}{\textbf{ogbn-Arxiv (169,343 nodes)}} \\ 
% \multirow{3}{*}{\shortstack{ogbn-Arxiv \\ (169,343 nodes)}} 
  GAT        & -    & 0.39 h & 5.5 mins  & 70.89 $\pm$ 0.43 \\
  GraphMAE2  & -    & 5.1 h     & 5.4 mins  & 70.46 $\pm$ 0.07 \\
  UniGraph   & 28.1 h & -      & 9.8 mins & 72.15 $\pm$ 0.18 \\
  UniGraph2  & 5.2 h & - & 5.7 mins &    \textbf{72.56 $\pm$ 0.15}  \\
\midrule
\multicolumn{5}{l}{\textbf{ogbn-Papers100M (111,059,956 nodes)}} \\
  GAT        & -    & 6.8 h     & 23.1 mins & 65.98 $\pm$ 0.23 \\
  GraphMAE2  & -    & 23.2 h    & 23.0 mins & 61.97 $\pm$ 0.24 \\
  UniGraph   & 28.1 h & -      & 40.1 mins & 67.89 $\pm$ 0.21 \\
  UniGraph2 & 5.2 h & - & 24.8 mins &  \textbf{67.95 $\pm$ 0.11} \\
\bottomrule[1.1pt]
\end{tabular}
\vspace{-4.5mm}
\end{table}

\vpara{Efficiency Analysis.}
\model, designed as a foundation model, incurs significant computational costs primarily during the pre-training phase. 
However, it offers the advantage of applicability to new datasets in the inference phase without requiring retraining. 
We compare of the training and inference costs of our model with other models. GAT~\cite{velivckovic2018graph} is a supervised trained GNN. 
GraphMAE2~\cite{hou2023graphmae2} is a self-supervised learning method with GAT as the backbone network. 
UniGraph~\cite{he2024unigraphlearningunifiedcrossdomain} is a graph foundation model for TAGs.
We select ogbn-Arxiv and ogbn-Papers100M, two datasets of different scales for experiments. 
From the results in the Table~\ref{tab:ccp}, we observe that although UniGraph2 has a long pre-training time, its inference time on downstream datasets is comparable or shorter than the combined training and inference time of GNN-based methods. This advantage further increases with the size and potential quantity of downstream datasets.
% The same conclusion also applies to space complexity. Although LM has a larger number of parameters, since we only need to perform inference on the downstream dataset, we avoid the additional space occupation in the backward propagation during training. 
\section{Conclusion}\label{sec:conclusion}

In this study, we propose M2-omni, a highly competitive omni-MLLM model to GPT-4o, characterized by its comprehensive modality and task support, as well as its exceptional performance. M2-omni demonstrates competitive performance across a diverse range of tasks, including image understanding, video understanding, interleaved image-text understanding, audio understanding and generation, as well as free-form image generation. We employ a multi-stage training approach to train M2-omni, which enables progressive modality alignment. To address the challenge of maintaining consistent performance across all modalities, we propose a step-wise balance strategy for pretraining and a dynamically adaptive balance strategy for instruction tuning, which can effectively mitigate the impact of significant variations in data volume and convergence rates across heterogeneous multimodal tasks. We publicly release M2-omni, along with its comprehensive training details, including data configurations and training procedures, to facilitate future research in this domain.

% \clearpage

% \addtolength{\textheight}{-12cm}   % This command serves to balance the column lengths
%                                   % on the last page of the document manually. It shortens
%                                   % the textheight of the last page by a suitable amount.
%                                   % This command does not take effect until the next page
%                                   % so it should come on the page before the last. Make
%                                   % sure that you do not shorten the textheight too much.

% %%%%%%%%%%%%%%%%%%%%%%%%%%%%%%%%%%%%%%%%%%%%%%%%%%%%%%%%%%%%%%%%%%%%%%%%%%%%%%%%

%\section{Acknowledgement}
%We thank Daegyu Lee, Hyunwooo Nam, Seongwoo Moon, Dokyeong Kim, Juhyeong Roh at Korea Advanced Institute of Science and Technology for their help in the real-world experiment.


% %%%%%%%%%%% References %%%%%%%%%%%%%%%%
% \balance
\bibliographystyle{IEEEtran}
\bibliography{root}

\end{document}
