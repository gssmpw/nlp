% This must be in the first 5 lines to tell arXiv to use pdfLaTeX, which is strongly recommended.
\pdfoutput=1
% In particular, the hyperref package requires pdfLaTeX in order to break URLs across lines.

\documentclass[11pt]{article}

% Change "review" to "final" to generate the final (sometimes called camera-ready) version.
% Change to "preprint" to generate a non-anonymous version with page numbers.
\usepackage[preprint]{acl}

% Standard package includes
\usepackage{times}
\usepackage{latexsym}
\usepackage{algorithm}
\usepackage{algpseudocode}
\usepackage{amsmath}
\usepackage{multirow}
\usepackage{array}
\usepackage{multirow}
\usepackage{graphicx}
\usepackage{colortbl}
\usepackage{amssymb}
\usepackage{pifont} 
\usepackage{booktabs}
\usepackage{subcaption}  
\usepackage{caption}  

% For proper rendering and hyphenation of words containing Latin characters (including in bib files)
\usepackage[T1]{fontenc}
% For Vietnamese characters
% \usepackage[T5]{fontenc}
% See www.latex-project.org/help/documentation/encguide.pdf for other character sets

% This assumes your files are encoded as UTF8
\usepackage[utf8]{inputenc}

% This is not strictly necessary, and may be commented out,
% but it will improve the layout of the manuscript,
% and will typically save some space.
\usepackage{microtype}

% This is also not strictly necessary, and may be commented out.
% However, it will improve the aesthetics of text in
% the typewriter font.
\usepackage{inconsolata}

%Including images in your LaTeX document requires adding
%additional package(s)
\usepackage{graphicx}

% If the title and author information does not fit in the area allocated, uncomment the following
%
%\setlength\titlebox{<dim>}
%
% and set <dim> to something 5cm or larger.

% \title{Sens-Merging: Sensitivity-Guided Parameter Balancing for Large Language Model Merging}

\title{Sens-Merging: Sensitivity-Guided Parameter Balancing for Merging \\ Large Language Models}


% Author information can be set in various styles:
% For several authors from the same institution:
% \author{Author 1 \and ... \and Author n \\
%         Address line \\ ... \\ Address line}
% if the names do not fit well on one line use
%         Author 1 \\ {\bf Author 2} \\ ... \\ {\bf Author n} \\
% For authors from different institutions:
% \author{Author 1 \\ Address line \\  ... \\ Address line
%         \And  ... \And
%         Author n \\ Address line \\ ... \\ Address line}
% To start a separate ``row'' of authors use \AND, as in
% \author{Author 1 \\ Address line \\  ... \\ Address line
%         \AND
%         Author 2 \\ Address line \\ ... \\ Address line \And
%         Author 3 \\ Address line \\ ... \\ Address line}

\author{Shuqi Liu$^{1,2}$, Han Wu$^{2,}$$^\dagger$, Bowei He$^{1}$, Xiongwei Han$^2$, Mingxuan Yuan$^2$, Linqi Song$^{1,}$$^\dagger$\\
$^{1}$ Department of Computer Science, City University of Hong Kong\\
$^{2}$ Huawei Noah's Ark Lab\\
\texttt{\{shuqiliu4-c, boweihe2-c\}@my.cityu.edu.hk}\\
\texttt{wu.han1@huawei.com}\\
\texttt{linqi.song@cityu.edu.hk}
}

%\author{
%  \textbf{First Author\textsuperscript{1}},
%  \textbf{Second Author\textsuperscript{1,2}},
%  \textbf{Third T. Author\textsuperscript{1}},
%  \textbf{Fourth Author\textsuperscript{1}},
%\\
%  \textbf{Fifth Author\textsuperscript{1,2}},
%  \textbf{Sixth Author\textsuperscript{1}},
%  \textbf{Seventh Author\textsuperscript{1}},
%  \textbf{Eighth Author \textsuperscript{1,2,3,4}},
%\\
%  \textbf{Ninth Author\textsuperscript{1}},
%  \textbf{Tenth Author\textsuperscript{1}},
%  \textbf{Eleventh E. Author\textsuperscript{1,2,3,4,5}},
%  \textbf{Twelfth Author\textsuperscript{1}},
%\\
%  \textbf{Thirteenth Author\textsuperscript{3}},
%  \textbf{Fourteenth F. Author\textsuperscript{2,4}},
%  \textbf{Fifteenth Author\textsuperscript{1}},
%  \textbf{Sixteenth Author\textsuperscript{1}},
%\\
%  \textbf{Seventeenth S. Author\textsuperscript{4,5}},
%  \textbf{Eighteenth Author\textsuperscript{3,4}},
%  \textbf{Nineteenth N. Author\textsuperscript{2,5}},
%  \textbf{Twentieth Author\textsuperscript{1}}
%\\
%\\
%  \textsuperscript{1}Affiliation 1,
%  \textsuperscript{2}Affiliation 2,
%  \textsuperscript{3}Affiliation 3,
%  \textsuperscript{4}Affiliation 4,
%  \textsuperscript{5}Affiliation 5
%\\
%  \small{
%    \textbf{Correspondence:} \href{mailto:email@domain}{email@domain}
%  }
%}

\begin{document}
\maketitle
\begin{abstract}
Recent advances in large language models have led to numerous task-specialized fine-tuned variants, creating a need for efficient model merging techniques that preserve specialized capabilities while avoiding costly retraining. While existing task vector-based merging methods show promise, they typically apply uniform coefficients across all parameters, overlooking varying parameter importance both within and across tasks. We present Sens-Merging, a sensitivity-guided coefficient adjustment method that enhances existing model merging techniques by operating at both task-specific and cross-task levels. Our method analyzes parameter sensitivity within individual tasks and evaluates cross-task transferability to determine optimal merging coefficients. Extensive experiments on Mistral 7B and LLaMA2-7B/13B models demonstrate that Sens-Merging significantly improves performance across general knowledge, mathematical reasoning, and code generation tasks. Notably, when combined with existing merging techniques, our method enables merged models to outperform specialized fine-tuned models, particularly in code generation tasks. Our findings reveal important trade-offs between task-specific and cross-task scalings, providing insights for future model merging strategies.
\end{abstract}

{
\let\thefootnote\relax\footnotetext{
$^\dagger$Corresponding author.}
}

% \begin{figure}[t]
%     \centering
%     \begin{subfigure}[b]{0.49\linewidth}
%         \centering
%         \includegraphics[width=\linewidth]{latex/sens_radar_1.png}
%         \label{fig:radar1}
%     \end{subfigure}
%     \hfill
%     \begin{subfigure}[b]{0.49\linewidth}
%         \centering
%         \includegraphics[width=\linewidth]{latex/sens_radar_2.png}
%         \label{fig:radar2}
%     \end{subfigure}
%     \caption{Sens-Merging functions as a plug-and-play enhancement to existing task vector-based merging techniques. Notably, when integrated with DARE, it surpasses even specialized code models in code generation.}
%     \label{fig:radar_combined}
%     \vspace{-0.3cm}
% \end{figure}

\begin{figure}
    \centering
    \includegraphics[width=1.0\linewidth]{latex/sens_cover.pdf}
    \caption{Sens-Merging functions as a plug-and-play enhancement to existing task vector-based merging techniques. Notably, when integrated with DARE, it surpasses even specialized code models in code generation.}
    \label{fig:radar_combined}
    \vspace{-0.3cm}
\end{figure}

\section{Introduction}

The rapid advancement of large language models has significantly enhanced performance across a diverse range of tasks \cite{touvron2023llama, zhao2023survey}. As these models continue to be fine-tuned for specialized domains, the necessity to merge these specialized models into a unified framework becomes increasingly critical \citep{merge_survey, mergeKit}. While multi-task learning has been proposed as a solution, it incurs substantial training costs and requires simultaneous access to data and labels for all tasks \cite{sanh2022multitask, fifty2021efficiently}. Recently, researchers have developed parameter-level model merging methods that not only comply with data privacy regulations but also improve efficiency by eliminating the need for retraining \cite{ties_merging, dare}.

In the context of model merging, task vectors \cite{ilharco2023editingmodelstaskarithmetic} have emerged as a powerful component for representing task-specific capabilities. These vectors, defined as the differences between parameter values before and after fine-tuning, enable effective integration of specialized knowledge from different models. 
While task vector-based merging methods \cite{ties_merging, dare} have shown promising results, their reliance on uniform coefficients for each task and parameter limits their potential effectiveness. 
This uniformity implies that every task and every parameter is treated with equal importance during the merging process. Consequently, it overlooks the fact that parameters within each layer demonstrate varying levels of importance for specific tasks, and parameters from different tasks contribute distinctly during the merging process.


To address these challenges, we propose Sens-Merging, a sensitivity-guided merging coefficient adjustment method that functions as a plug-and-play enhancement to existing task vector-based merging techniques. Our method operates at two levels: within individual tasks and across different tasks, allowing for fine-grained control over parameter importance. 
Within each task-specific model, we perform parameter sensitivity analysis to highlight critical layers that significantly impact performance. Concurrently, across different tasks, we conduct task sensitivity analysis to prioritize models that enhance the performance of others. By combining these two factors, we derive the final merging coefficients, which are then applied to merge the corresponding layers.
% When using Mistral 7B as the foundation model, 
Figure \ref{fig:radar_combined} highlights how Sens-Merging enhances existing task-vector techniques like Task Arithmetic \cite{task_arithmetic} and DARE \cite{dare}. Notably, when combined with DARE method, Sens-Merging enables merged models to outperform specialized fine-tuned models, particularly in code generation tasks.
% improves performance across task-specialized fine-tuned models from three key domains: general knowledge tasks (MMLU, HellaSwag, TruthfulQA), mathematical reasoning (GSM8K, MATH), and code generation (MBPP, HumanEval). 
% Sens-Merging not only enhances baseline task-vector techniques like Task Arithmetic and DARE but achieves a notable breakthrough when integrated with DARE - remarkably, this combination proves more effective at code generation than models explicitly trained for that purpose. 

To empirically demonstrate the effectiveness of Sens-Merging, we conduct extensive experiments by combining it with existing model merging approaches. We merged three widely adopted fine-tuned models—specializing in general knowledge (Chat), mathematical reasoning (Math), and code generation (Code)—derived from the LLaMA2-7B/13B and Mistral 7B families. 
% The integration of our Sens-Merging method consistently enhanced the performance of baseline approaches. Furthermore, incorporating Sens-Merging even allowed the merged models to outperform the original individually fine-tuned models. 
The integration of our Sens-Merging not only improves baseline merging performance but enables merged models to surpass individual fine-tuned models. 
Notably, when merging Code model with Math and Chat models using Sens-Merging, it achieves superior performance on coding tasks compared to code-specific fine-tuning alone. These results indicate that model merging can effectively address the challenges of training a single model for complex tasks by integrating the specialized capabilities of multiple fine-tuned models.

To sum up, our contributions include:
(1) We propose a novel model merging coefficient determination method based on both task-specific and cross-task sensitivity analysis.
(2) Through comprehensive evaluations, 
we validate that our proposed method enhances model merging performance across various domains.
(3) We empirically demonstrate that different task-specific models contribute unequally to model merging, and parameter importance varies across different layers within each model.
(4) We validate that each scaling approach presents distinct trade-offs: task-specific scaling excels in specialized domains like mathematics but offers limited general benefits, while cross-task scaling achieves broader performance gains at the cost of peak task-specialized performance.

\begin{figure*}[tbp]
    \centering
    \includegraphics[width=0.95\textwidth]{latex/sens-arch.pdf}
    \caption{Overall framework of our Sens-Merging method. Sens-Merging adjusts layer-wise scaling coefficients for task-specialized fine-tuned models through two mechanisms: task-specific scaling and cross-task scaling.
    }

    \label{fig: framework}
    \vspace{-0.3cm}
\end{figure*}

\section{RELATED WORK}
\label{sec:relatedwork}
In this section, we describe the previous works related to our proposal, which are divided into two parts. In Section~\ref{sec:relatedwork_exoplanet}, we present a review of approaches based on machine learning techniques for the detection of planetary transit signals. Section~\ref{sec:relatedwork_attention} provides an account of the approaches based on attention mechanisms applied in Astronomy.\par

\subsection{Exoplanet detection}
\label{sec:relatedwork_exoplanet}
Machine learning methods have achieved great performance for the automatic selection of exoplanet transit signals. One of the earliest applications of machine learning is a model named Autovetter \citep{MCcauliff}, which is a random forest (RF) model based on characteristics derived from Kepler pipeline statistics to classify exoplanet and false positive signals. Then, other studies emerged that also used supervised learning. \cite{mislis2016sidra} also used a RF, but unlike the work by \citet{MCcauliff}, they used simulated light curves and a box least square \citep[BLS;][]{kovacs2002box}-based periodogram to search for transiting exoplanets. \citet{thompson2015machine} proposed a k-nearest neighbors model for Kepler data to determine if a given signal has similarity to known transits. Unsupervised learning techniques were also applied, such as self-organizing maps (SOM), proposed \citet{armstrong2016transit}; which implements an architecture to segment similar light curves. In the same way, \citet{armstrong2018automatic} developed a combination of supervised and unsupervised learning, including RF and SOM models. In general, these approaches require a previous phase of feature engineering for each light curve. \par

%DL is a modern data-driven technology that automatically extracts characteristics, and that has been successful in classification problems from a variety of application domains. The architecture relies on several layers of NNs of simple interconnected units and uses layers to build increasingly complex and useful features by means of linear and non-linear transformation. This family of models is capable of generating increasingly high-level representations \citep{lecun2015deep}.

The application of DL for exoplanetary signal detection has evolved rapidly in recent years and has become very popular in planetary science.  \citet{pearson2018} and \citet{zucker2018shallow} developed CNN-based algorithms that learn from synthetic data to search for exoplanets. Perhaps one of the most successful applications of the DL models in transit detection was that of \citet{Shallue_2018}; who, in collaboration with Google, proposed a CNN named AstroNet that recognizes exoplanet signals in real data from Kepler. AstroNet uses the training set of labelled TCEs from the Autovetter planet candidate catalog of Q1–Q17 data release 24 (DR24) of the Kepler mission \citep{catanzarite2015autovetter}. AstroNet analyses the data in two views: a ``global view'', and ``local view'' \citep{Shallue_2018}. \par


% The global view shows the characteristics of the light curve over an orbital period, and a local view shows the moment at occurring the transit in detail

%different = space-based

Based on AstroNet, researchers have modified the original AstroNet model to rank candidates from different surveys, specifically for Kepler and TESS missions. \citet{ansdell2018scientific} developed a CNN trained on Kepler data, and included for the first time the information on the centroids, showing that the model improves performance considerably. Then, \citet{osborn2020rapid} and \citet{yu2019identifying} also included the centroids information, but in addition, \citet{osborn2020rapid} included information of the stellar and transit parameters. Finally, \citet{rao2021nigraha} proposed a pipeline that includes a new ``half-phase'' view of the transit signal. This half-phase view represents a transit view with a different time and phase. The purpose of this view is to recover any possible secondary eclipse (the object hiding behind the disk of the primary star).


%last pipeline applies a procedure after the prediction of the model to obtain new candidates, this process is carried out through a series of steps that include the evaluation with Discovery and Validation of Exoplanets (DAVE) \citet{kostov2019discovery} that was adapted for the TESS telescope.\par
%



\subsection{Attention mechanisms in astronomy}
\label{sec:relatedwork_attention}
Despite the remarkable success of attention mechanisms in sequential data, few papers have exploited their advantages in astronomy. In particular, there are no models based on attention mechanisms for detecting planets. Below we present a summary of the main applications of this modeling approach to astronomy, based on two points of view; performance and interpretability of the model.\par
%Attention mechanisms have not yet been explored in all sub-areas of astronomy. However, recent works show a successful application of the mechanism.
%performance

The application of attention mechanisms has shown improvements in the performance of some regression and classification tasks compared to previous approaches. One of the first implementations of the attention mechanism was to find gravitational lenses proposed by \citet{thuruthipilly2021finding}. They designed 21 self-attention-based encoder models, where each model was trained separately with 18,000 simulated images, demonstrating that the model based on the Transformer has a better performance and uses fewer trainable parameters compared to CNN. A novel application was proposed by \citet{lin2021galaxy} for the morphological classification of galaxies, who used an architecture derived from the Transformer, named Vision Transformer (VIT) \citep{dosovitskiy2020image}. \citet{lin2021galaxy} demonstrated competitive results compared to CNNs. Another application with successful results was proposed by \citet{zerveas2021transformer}; which first proposed a transformer-based framework for learning unsupervised representations of multivariate time series. Their methodology takes advantage of unlabeled data to train an encoder and extract dense vector representations of time series. Subsequently, they evaluate the model for regression and classification tasks, demonstrating better performance than other state-of-the-art supervised methods, even with data sets with limited samples.

%interpretation
Regarding the interpretability of the model, a recent contribution that analyses the attention maps was presented by \citet{bowles20212}, which explored the use of group-equivariant self-attention for radio astronomy classification. Compared to other approaches, this model analysed the attention maps of the predictions and showed that the mechanism extracts the brightest spots and jets of the radio source more clearly. This indicates that attention maps for prediction interpretation could help experts see patterns that the human eye often misses. \par

In the field of variable stars, \citet{allam2021paying} employed the mechanism for classifying multivariate time series in variable stars. And additionally, \citet{allam2021paying} showed that the activation weights are accommodated according to the variation in brightness of the star, achieving a more interpretable model. And finally, related to the TESS telescope, \citet{morvan2022don} proposed a model that removes the noise from the light curves through the distribution of attention weights. \citet{morvan2022don} showed that the use of the attention mechanism is excellent for removing noise and outliers in time series datasets compared with other approaches. In addition, the use of attention maps allowed them to show the representations learned from the model. \par

Recent attention mechanism approaches in astronomy demonstrate comparable results with earlier approaches, such as CNNs. At the same time, they offer interpretability of their results, which allows a post-prediction analysis. \par




\section{Methodology}
Our Sens-Merging method combines two levels of sensitivity analysis: layer-wise analysis within individual models and cross-task analysis across different models to achieve a balanced parameter distribution. For layer-wise analysis, we compute sensitivity scores using gradient information from calibration datasets. For cross-task analysis, we evaluate model alignment through logit comparison. These two components determine the final merging coefficients used to merge corresponding layers into a unified model, as shown in Figure XX.

\subsection{Preliminary}
Considering \( K \) task-specialized fine-tuned models \( \{\theta^{t_1}_{\text{SFT}}, \ldots, \theta^{t_K}_{\text{SFT}}\} \) derived from a common pre-trained backbone \( \theta_{\text{PRE}} \), model merging aims to merge them into a single model $\theta_M$ that can effectively handle all tasks simultaneously. 
The task-specific capabilities of each fine-tuned model are captured by task vectors, defined as the difference between the fine-tuned parameters and the pre-trained backbone:
\[
\delta_{t_k} = \theta^{t_k}_{\text{SFT}} - \theta_{\text{PRE}}, \quad \text{for } k \in \{1, \ldots, K\}.
\]
Task vector-based merging aggregates these task vectors to construct a single, static merged model:
\[
\theta_{\text{M}} = \theta_{\text{PRE}} + \sum_{k=1}^{K} \lambda * \delta_{t_k}.
\]
where the coefficient $\lambda$ represents the importance of each merged task vector.

% \subsection{Overview}
% : we first compute layer-wise sensitivity scores for each model using gradient information from their respective calibration datasets, and then calculate cross-task sensitivity scaling factors by measuring how well each model aligns with other expert models. The final merging coefficients are determined by combining these two factors and merges the corresponding layers using these coefficients to produce the final model.
% Specifically, we perform sensitivity analysis both within individual tasks and across different tasks to adjust the scaling of each task-specific model $\theta^{t_i}_{\text{SFT}}$ and each layer within these models $\theta^{t_i, l}_{\text{SFT}}$, as illustrated in Figure 3.




% \begin{algorithm}[H]
% \caption{Sensitivity-Guided Model Merging}
% \label{alg:sensitivity-guided-merging}
% \begin{algorithmic}[1]
% \Require Fine-tuned models $\{\theta^{t_i}_{\text{SFT}}\}_{i=1}^K$, 
% % derived from the same pre-trained backbone $\theta_{\text{base}}$, 
% calibration datasets $\{\mathcal{D}_i\}_{i=1}^K$.
% \Ensure Merged model $\theta_{\text{M}}$

% \For{each model $M_i$}
%     \State Compute layer-wise sensitivity scores: \\
%         $\qquad \quad \alpha_i^l \gets \text{gradient norms on } \mathcal{D}_i.$
%     \State Calculate generalization factor: \\
%         $\qquad \quad \tau_i \gets \text{logits alignment with experts.}$
% \EndFor

% \For{each layer position $l$}
%     \State Compute scaling factors:\\
%         $\qquad \quad \sigma_i^l \gets \text{softmax}(\tau_i \cdot \alpha_i^l)$
%     \State Merge parameters:\\
%         $\quad \theta_M^l \gets \theta_{\text{base}}^l + \sum_{i=1}^{K} \sigma_i^l \cdot K \cdot \left(\theta^{t_i,l}_{\text{SFT}} - \theta_{\text{base}}^l\right)$
% \EndFor

% \Return $\theta_{\text{M}}$
% \end{algorithmic}
% \end{algorithm}


% Algorithm 1 presents our complete model merging procedure. 
% % The algorithm takes as input a set of fine-tuned models, and their corresponding calibration datasets. It outputs the scaling factors that leverage both layer-wise sensitivity and cross-task generalization ability.
% The algorithm first computes layer-wise sensitivity scores for each model using gradient information from their respective calibration datasets. It then calculates the generalization scaling factors by measuring how well each model aligns with other expert models. The final merging coefficients are determined by combining these two factors and applying softmax normalization. Finally, the algorithm merges the corresponding layers using these coefficients to produce the final model.

\subsection{Task-Specific Scaling}

To accurately balance the parameters within individual task models, we conduct layer-wise sensitivity analysis by measuring each layer's contribution to model performance through aggregating parameter sensitivities within that layer.

\paragraph{Parameter Sensitivity.}
We define parameter sensitivity as the change in loss when setting that parameter to zero. A parameter is considered highly sensitive if zeroing it results in a significant loss increase.
For a fine-tuned model with parameters $\theta^{t_i}_{\text{SFT}} = [\theta_1, ..., \theta_{N}]$, where $N$ represents the total number of parameters, the $j$-th parameter can be expressed as $\theta^{t_i}_{j} = [0, ..., \theta_j, ..., 0]$. With gradients of the loss relative to $\theta^{t_i}_{\text{SFT}}$ represented as $\nabla_{\theta^{t_i}_{\text{SFT}}} L$, the sensitivity of the $j$-th parameter for a specific sample $x_k$ from task $t_i$ is determined as:
\begin{equation}
S_{j,k}^{t_i} = |(\theta^{t_i}_{j})^\top \nabla_{\theta^{t_i}_{\text{SFT}}} L(x_k)|
\end{equation}
% This approximates the first-order Taylor expansion of loss change when removing parameter $\theta_j$: 
The rationale behind this sensitivity definition stems from the first-order Taylor expansion of $L(x_k)$ relative to $\theta_j$. In essence, $S_{j,k}^{t_i}$ provides an approximation for how the loss might change in the absence of $\theta_j$:
\begin{equation}
(\theta^{t_i}_{j})^\top \nabla_{\theta^{t_i}_{\text{SFT}}} L(x_k) \approx L(\theta^{t_i}_{\text{SFT}}) - L(\theta^{t_i}_{\text{SFT}} - \theta^{t_i}_{j})
\end{equation}
% The final sensitivity score $S_j$ aggregates the sensitivities across $m$ sampled calibration samples: $S_j = \sum_{k=1}^m S_{j,k}$. 
To estimate the parameter sensitivity $S_{j}^{t_i}$ for task $t_i$,
we randomly sample $m$ instances from the task training set as calibration samples. 
The final sensitivity score $S_j^{t_i}$ aggregates the individual sensitivities across all sampled instances:
$S_{j}^{t_i} = \sum_{k=1}^m S_{j,k}^{t_i}$.

% We aggregate sensitivities across $m$ calibration samples to obtain $S_j = \sum_{k=1}^m S_{j,k}$. 


\paragraph{Layer-Wise Sensitivity and Normalization.}

The layer-wise sensitivity \( s_i^l \) is then calculated by summing the sensitivities of all parameters within each layer, thereby reflecting each layer's overall contribution to the model's performance. To allow for meaningful comparisons of these importance scores across different models, we apply \( L_2 \) normalization to the sensitivities of all layers.
Consequently, the task-specific sensitivity scaling factors \( \alpha_i^l \) are defined as:
\begin{equation}
    s_i^l = \sum_{j \in \mathcal{P}_l} S_{j}^{t_i}, \quad \alpha_i^l = \frac{s_i^l}{\|\mathbf{s}_i\|_2}
\end{equation}
where \( \mathcal{P}_l \) denotes the set of parameters in layer \( l \), and \( L \) is the total number of layers in the model. 

\subsection{Cross-Task Scaling}

% While task-specific sensitivity effectively captures the importance of layers within individual tasks, it is equally crucial to consider how each task-specific model contributes to other tasks during the merging process. To evaluate cross-task sensitivity, we assess the alignment of logits across different tasks. A high alignment score indicates that a task model's outputs closely match those of other tasks, suggesting the presence of shared features and decision-making processes. Consequently, a high cross-task sensitivity score reflects the model's significant influence on the performance of other tasks.

While task-specific sensitivity focuses on the importance of layers within individual tasks, it is equally essential to evaluate how each task-specific model influences other tasks during the merging process. Cross-task sensitivity captures the interdependencies and shared representations between different tasks, ensuring that the merged model benefits from common features and decision-making processes.

The measurement of cross-task influence begins with evaluating logits alignment between different task-specific models. Specifically, for calibration samples from task $t_j$, we compute the alignment score between model $\theta^{t_i}_{\text{SFT}}$ and the expert model for task $t_j$, $\theta^{t_j}_{\text{SFT}}$, using the $L_2$ distance between their output logits:
\begin{equation}
g_{i,j} = ||f_{\theta^{t_i}_{\text{SFT}}}(x_k^j) - f_{\theta^{t_j}_{\text{SFT}}}(x_k^j)||_{2}
\end{equation}
where $f_{\theta}(x)$ denotes the output logits of model $\theta$ for input $x$, and $||\cdot||_{2}$ represents $L_{2}$ distance. This alignment score quantifies how closely the predictions of model $\theta^{t_i}_{\text{SFT}}$ match those of the expert model for task $\theta^{t_j}_{\text{SFT}}$, providing insight into the degree of shared knowledge and representational similarity between tasks.
To obtain a comprehensive measure of cross-task sensitivity for a specific task model $\theta^{t_i}_{\text{SFT}}$, we aggregate the alignment scores across all other tasks. This aggregation process involves computing the normalized alignment:
% The overall cross-task sensitivity of model $\theta^{t_i}_{\text{SFT}}$ is then computed as the normalized average alignment across $K$ tasks:
\begin{equation}
    \tau_i = \sum_{i=1, i\neq j}^{K} g_{i,j}, \quad \tau_i = \frac{\tau_i}{\|\boldsymbol{\tau}\|_1}
\end{equation}
% \begin{equation}
% \tau_i = \frac{1}{K} \sum_{j=1}^K g_{i,j}, \quad \tau_i = \frac{\tau_i}{\|\boldsymbol{\tau}\|_2}
% \end{equation}
% Thus, the cross-task scaling factor $\tau_i$ measures the ability of model $i$ to transfer knowledge across tasks. Higher values of $\tau_i$ indicate better cross-task generalization, while lower values reflect greater task-specific specialization.
The resulting cross-task scaling factor $\tau_i$ serves as a crucial metric that quantifies model $\theta^{t_i}_{\text{SFT}}$'s ability to transfer knowledge across tasks. Higher values of $\tau_i$ indicate superior cross-task generalization capabilities, suggesting that the model has learned robust representations that are valuable across multiple tasks. Conversely, lower values of $\tau_i$ reflect greater task-specific specialization, indicating that the model's features are more narrowly focused on its primary task. 
% While specialized models may excel at their specific tasks, their limited cross-task influence suggests they might contribute less to the overall generalization capabilities of the merged model. 


\subsection{Integration with Merging Methods}

Our Sens-Merging method combines task-specific scaling factor $\alpha_i^l$ and the cross-task scaling factor $\tau_i$ into a plug-and-play module, which can be seamlessly integrated with existing task vector-based model merging methods. 
To effectively combine these sensitivity factors, we employ a two-step process. First, we multiply the task-specific scaling factor $\alpha_i^l$ with the cross-task scaling factor $\tau_i$ to capture both task-specific and cross-task importance. Then, we apply a softmax function with temperature T to normalize these products and obtain the final scaling coefficients:
\begin{equation}
    \sigma_i^l = \text{Softmax} (\tau_i \cdot \alpha_i^l, T)
\end{equation}
% The temperature parameter T controls the sharpness of the distribution - lower values create more concentrated weights on high-sensitivity components, while higher values produce more uniform scaling.

The final step involves computing the merged model parameters $\theta_M^l$ for each layer $l$. We start with the base model parameters $\theta_{\text{base}}^l$ and incorporate weighted contributions from all $K$ fine-tuned models. The contribution of each task-specific model is scaled by its normalized coefficient $\sigma_i^l$ and multiplied by $K$ to preserve the magnitude of updates:
\begin{equation}
\theta_M^l = \theta_{\text{base}}^l + \sum_{i=1}^K K \cdot \sigma_i^l \cdot (\theta_{\text{SFT}}^{t_i,l} - \theta_{\text{base}}^l)
\end{equation}


% \multirow{6}{*}{\begin{tabular}[c]{@{}c@{}}Weight\\ Average\end{tabular}} &
%   \multirow{2}{*}{Chat\&Math} &
%   \ding{55} &
%   - &
%   - &
%   - &
%    &
%   45.49 &
%   - &
%    &
%   25.8 &
%   - \\
%  &
%    &
%   $\checkmark$ &
%   \cellcolor[HTML]{EFEFEF}- &
%   \cellcolor[HTML]{EFEFEF}- &
%   \cellcolor[HTML]{EFEFEF}- &
%   \cellcolor[HTML]{EFEFEF} &
%   \cellcolor[HTML]{EFEFEF}45.94 &
%   \cellcolor[HTML]{EFEFEF}- &
%   \cellcolor[HTML]{EFEFEF} &
%   \cellcolor[HTML]{EFEFEF}26.6 &
%   \cellcolor[HTML]{EFEFEF}- \\
%  &
%   \multirow{2}{*}{Math\&Code} &
%   \ding{55} &
%   - &
%   - &
%   - &
%    &
%   38.51 &
%   - &
%    &
%   23.3 &
%   - \\
%  &
%    &
%   $\checkmark$ &
%   \cellcolor[HTML]{EFEFEF}- &
%   \cellcolor[HTML]{EFEFEF}- &
%   \cellcolor[HTML]{EFEFEF}- &
%   \cellcolor[HTML]{EFEFEF} &
%   \cellcolor[HTML]{EFEFEF}41.09 &
%   \cellcolor[HTML]{EFEFEF}- &
%   \cellcolor[HTML]{EFEFEF} &
%   \cellcolor[HTML]{EFEFEF}24.6 &
%   \cellcolor[HTML]{EFEFEF}- \\
%  &
%   \multirow{2}{*}{\begin{tabular}[c]{@{}c@{}}Chat\&Math\\ \&Code\end{tabular}} &
%   \ding{55} &
%   - &
%   - &
%   - &
%    &
%   34.19 &
%    &
%    &
%   27.6 &
%   - \\
%  &
%    &
%   $\checkmark$ &
%   \cellcolor[HTML]{EFEFEF}- &
%   \cellcolor[HTML]{EFEFEF}- &
%   \cellcolor[HTML]{EFEFEF}- &
%   \cellcolor[HTML]{EFEFEF} &
%   \cellcolor[HTML]{EFEFEF}35.86 &
%   \cellcolor[HTML]{EFEFEF}- &
%   \cellcolor[HTML]{EFEFEF} &
%   \cellcolor[HTML]{EFEFEF}28.3 &
%   \cellcolor[HTML]{EFEFEF}- \\
%   \hline




\section{Experiments}

% \subsection{Experimental Setup}
% - Training and evaluation datasets
% - Model architectures and configurations
% - Implementation details and hyperparameters
% - Baseline methods for comparison

\paragraph{Baselines.} 
We evaluate the effectiveness of our Sens-Merging method by comparing it against both individual task-specific models and several established model-merging techniques, including Task Arithmetic, Ties-Merging, and DARE-Merging. Task Arithmetic \citep{task_arithmetic} enhances the merging process by introducing task vectors, suggesting that simple arithmetic operations on these vectors can effectively edit models and produce a merged model. Building on the concept of task vectors, both DARE \citep{dare} and Ties \citep{ties_merging} employ pruning-then-scaling methods to merge task vectors, based on the assumption that not all parameters contribute equally to the final performance.  

\paragraph{Hyperparameters.}
Both baselines and our Sens-Merging enhanced baselines use the same hyperparameters for fair comparison. 
For Task Arithmetic, we use a default scaling coefficient of $\lambda = 1$, which maintains the original magnitude of task vector when adding the pretrained backbone. 
However, the DARE method has been observed to be more sensitive to variations in both the scaling coefficient \(\lambda\) and the drop rate parameter \(r\). 
% Changes in these hyperparameters can disproportionately impact DARE's performance. 
To achieve a balanced performance, we set the scaling coefficient to \(\lambda = 0.5\) and establish a default drop rate of \(r = 0.5\) for DARE.
Similarly, for Ties-Merging, which requires the specification of a masking ratio, we set the default mask ratio to \(r = 0.7\) across all experiments. 
% This value is chosen based on preliminary evaluations to strike an optimal balance between preserving critical task-specific information and promoting the sharing of common features across tasks.

% When no additional scaling is applied, we use a default value of $\lambda=1$ for all task-vector based methods except additional descrption. For Ties-Merging, which requires a masking ratio, we set the default mask ratio to $r=0.7$ across all experiments. Similarly, for DARE, which involves a drop rate, we establish a default drop rate of $r=0.5$.


\paragraph{Benchmark.} 
Our experimental evaluation encompasses three families of models: LLaMA-2 7B series \cite{touvron2023llama}, Mistral 7B series \cite{jiang2023mistral}, and LLaMA-2 13B series \cite{touvron2023llama}, each specialized in distinct domains: general knowledge, mathematical reasoning, and code generation. For comprehensive evaluation, we utilize seven benchmark datasets spanning three key domains: MMLU \cite{hendrycks2020measuring}, HellaSwag \cite{zellers2019hellaswag} and TruthfulQA \cite{lin2022truthfulqa} for assessing general knowledge and reasoning capabilities; GSM8K \cite{cobbe2021training} and MATH \cite{hendrycks2021measuring} for testing mathematical reasoning proficiency; and HumanEval \cite{chen2021evaluating} and MBPP \cite{austin2021program} for evaluating code generation abilities. To ensure consistent and unbiased assessment, model performance is evaluated using zero-shot accuracy, with pass@1 rate specifically measuring code generation correctness.
% Additionally, we randomly select 10 samples and 128 samples from the training datasets of GSM8K, MBPP, and Alpaca to serve as calibration data for task-specific scaling and cross-task scaling.

\begin{table*}[]
% \small
\caption{Performance evaluation of merged LLaMA2-7B Models (Chat, Math, Code) across 7 task-specific datasets}
\resizebox{\textwidth}{!}{%
\begin{tabular}{lcccclcclccc} 
\toprule
& & \multicolumn{3}{c}{General Knowledge} &
   &
  \multicolumn{2}{c}{Mathemetical Reasoning} &
   &
  \multicolumn{2}{c}{Code Generation} &
  \multicolumn{1}{c}{} \\ \cline{3-5} \cline{7-8} \cline{10-11} 
\multirow{-2}{*}{Method} &
  \multirow{-2}{*}{\begin{tabular}[c]{@{}c@{}}Use\\ Sens\end{tabular}} &
  \multicolumn{1}{c}{MMLU} &
  \multicolumn{1}{c}{HellaSwag} &
  \multicolumn{1}{c}{TruthfulQA} &
   &
  \multicolumn{1}{c}{GSM8K} &
  \multicolumn{1}{c}{MATH} &
   &
  \multicolumn{1}{c}{MBPP} &
  \multicolumn{1}{c}{HumanEval} &
  \multicolumn{1}{c}{\multirow{-2}{*}{Average}} \\
  \hline
Chat & & 46.38 & 57.79 & 45.17 &  & 23.43 & 4.86 & & 0.3 & 0.6 & 25.50 \\
Math & & 40.05 & 56.30 & 32.56 &  & 48.60 & 8.50 &  & 21.8 & 12.8 & 31.52\\
Code & \multirow{-3}{*}{\textbackslash{}} & 40.76 & 57.87 & 33.17 &  & 7.13 & 3.62 &  & 26.8 & 5.5 & 24.98\\
\hline
 & \ding{55} & 41.50 & 49.63 & 37.45 &  & \underline{47.34} & 6.46 &  & 13.5 & 7.3 & 29.03\\
\multirow{-2}{*}{Task Arithmetic} &
  \checkmark &
  \cellcolor[HTML]{EFEFEF}46.12 &
  \cellcolor[HTML]{EFEFEF}\textbf{59.10} &
  \cellcolor[HTML]{EFEFEF}36.84 &
  \cellcolor[HTML]{EFEFEF} &
  \cellcolor[HTML]{EFEFEF}42.29 &
  \cellcolor[HTML]{EFEFEF}7.12 &
  \cellcolor[HTML]{EFEFEF}  &
  \cellcolor[HTML]{EFEFEF}\textbf{33.1} &
  \cellcolor[HTML]{EFEFEF}\underline{18.9} & 
  \cellcolor[HTML]{EFEFEF}34.78 \\
 & \ding{55} & 45.75 & 56.63 & \underline{39.89} &  & 46.93 & \underline{7.74} &  & 29.1 & 17.1 & 34.73 \\
\multirow{-2}{*}{Ties-Merging} &
  \checkmark &
  \cellcolor[HTML]{EFEFEF}46.03 &
  \cellcolor[HTML]{EFEFEF}56.87 &
  \cellcolor[HTML]{EFEFEF}\textbf{40.02} &
  \cellcolor[HTML]{EFEFEF} &
  \cellcolor[HTML]{EFEFEF}\textbf{47.69} &
  \cellcolor[HTML]{EFEFEF}\textbf{7.80} &
  \cellcolor[HTML]{EFEFEF} &
  \cellcolor[HTML]{EFEFEF}29.8 &
  \cellcolor[HTML]{EFEFEF}17.7 &
  \cellcolor[HTML]{EFEFEF}\textbf{35.13}\\
     & \ding{55}  & \underline{46.78} & 57.57 & 38.19 &  & 44.05 & 6.96 &  & 31.6 & \underline{18.9} & \underline{34.86}\\
\multirow{-2}{*}{DARE} & 
  \checkmark &
  \cellcolor[HTML]{EFEFEF}\textbf{46.81} &
  \cellcolor[HTML]{EFEFEF}\underline{58.24} &
  \cellcolor[HTML]{EFEFEF}37.33 &
  \cellcolor[HTML]{EFEFEF} &
  \cellcolor[HTML]{EFEFEF}44.73 &
  \cellcolor[HTML]{EFEFEF}6.98 &
  \cellcolor[HTML]{EFEFEF} &
  \cellcolor[HTML]{EFEFEF}\underline{32.3} &
  \cellcolor[HTML]{EFEFEF}\textbf{19.5} & 
  \cellcolor[HTML]{EFEFEF}\textbf{35.13} \\
  \bottomrule
\end{tabular}%
}
\label{tab:llama2-7b}
\end{table*}

\begin{table*}[]
% \small
\caption{Performance evaluation of merged Mistral 7B Models (Chat, Math, Code) across 7 task-specific datasets}
\resizebox{\textwidth}{!}{%
\begin{tabular}{lcccclcclccc}
\toprule
& & \multicolumn{3}{c}{General Knowledge} &
   &
  \multicolumn{2}{c}{Mathemetical Reasoning} &
   &
  \multicolumn{2}{c}{Code Generation} &
  \multicolumn{1}{c}{} \\ \cline{3-5} \cline{7-8} \cline{10-11} 
\multirow{-2}{*}{Method} &
  \multirow{-2}{*}{\begin{tabular}[c]{@{}c@{}}Use\\ Sens\end{tabular}} &
  \multicolumn{1}{c}{MMLU} &
  \multicolumn{1}{c}{HellaSwag} &
  \multicolumn{1}{c}{TruthfulQA} &
   &
  \multicolumn{1}{c}{GSM8K} &
  \multicolumn{1}{c}{MATH} &
   &
  \multicolumn{1}{c}{MBPP} &
  \multicolumn{1}{c}{HumanEval} &
  \multicolumn{1}{c}{\multirow{-2}{*}{Average}} \\
  \hline
Chat & & 59.05 & 65.97 & 55.69 &  & 42.53 & 9.16 & & 49.6 & 42.7 & 46.37 \\
Math & & 60.77 & 58.68 & 44.68 &  & 63.38 & 22.74 &  & 38.1 & 23.8 & 44.59\\
Code & \multirow{-3}{*}{\textbackslash{}} & 50.58 & 53.19 & 45.29 &  & 31.69 & 4.84 &  & 50.9 & 40.9 & 39.63 \\
\hline
 & \ding{55} & 47.34 & 46.80 & 41.00 &  & 52.16 & 13.26 &  & 32.1 & 29.9 & 37.51 \\
\multirow{-2}{*}{Task Arithmetic} &
  \checkmark &
  % \cellcolor[HTML]{EFEFEF}60.75 &
  % \cellcolor[HTML]{EFEFEF}60.05 &
  % \cellcolor[HTML]{EFEFEF}49.08 &
  % \cellcolor[HTML]{EFEFEF} &
  % \cellcolor[HTML]{EFEFEF}58.07 &
  % \cellcolor[HTML]{EFEFEF}17.06 &
  % \cellcolor[HTML]{EFEFEF}  &
  % \cellcolor[HTML]{EFEFEF}52.4 &
  % \cellcolor[HTML]{EFEFEF}37.8 \\
  \cellcolor[HTML]{EFEFEF}\textbf{62.43} &
  \cellcolor[HTML]{EFEFEF}\textbf{61.94} &
  \cellcolor[HTML]{EFEFEF}45.29 &
  \cellcolor[HTML]{EFEFEF} &
  \cellcolor[HTML]{EFEFEF}\textbf{59.74} &
  \cellcolor[HTML]{EFEFEF}\textbf{17.06} &
  \cellcolor[HTML]{EFEFEF}  &
  \cellcolor[HTML]{EFEFEF}\underline{54.4} &
  \cellcolor[HTML]{EFEFEF}34.1 & 
  \cellcolor[HTML]{EFEFEF}\underline{47.85} \\
 & \ding{55} & 57.20 & 57.59 & \textbf{48.71} &  & 55.50 & 15.00 &  & 48.4 & 40.2 & 46.09 \\
\multirow{-2}{*}{Ties-Merging} &
  \checkmark &
  \cellcolor[HTML]{EFEFEF}57.36 &
  \cellcolor[HTML]{EFEFEF}57.94 &
  \cellcolor[HTML]{EFEFEF}\underline{48.12} &
  \cellcolor[HTML]{EFEFEF} &
  \cellcolor[HTML]{EFEFEF}56.25 &
  \cellcolor[HTML]{EFEFEF}15.56 &
  \cellcolor[HTML]{EFEFEF} &
  \cellcolor[HTML]{EFEFEF}49.9 &
  \cellcolor[HTML]{EFEFEF}\underline{41.5} & 
  \cellcolor[HTML]{EFEFEF}46.66 \\
     & \ding{55}  & 55.36 & 55.77 & 42.84 &  & 57.39 & 15.00 &  & 49.4 & 39.0 & 44.97\\
\multirow{-2}{*}{DARE} & 
  \checkmark &
  \cellcolor[HTML]{EFEFEF}\underline{58.22} &
  \cellcolor[HTML]{EFEFEF}\underline{58.92} &
  \cellcolor[HTML]{EFEFEF}46.88 &
  \cellcolor[HTML]{EFEFEF} &
  \cellcolor[HTML]{EFEFEF}\underline{58.45} &
  \cellcolor[HTML]{EFEFEF}\underline{16.46} &
  \cellcolor[HTML]{EFEFEF} &
  \cellcolor[HTML]{EFEFEF}\textbf{55.1} &
  \cellcolor[HTML]{EFEFEF}\textbf{43.3} & 
  \cellcolor[HTML]{EFEFEF}\textbf{48.19} \\
  \bottomrule
\end{tabular}%
}
\label{tab:mistral-7b}
\vspace{-0.2cm}
\end{table*}


\subsection{Main Results}


\paragraph{Merging Models with Sense-Merging.} 
We first evaluate the effectiveness of our Sens-Merging method by utilizing it as a plug-and-play module to enhance existing task-vector-based baselines. Table~\ref{tab:llama2-7b} presents the performance of the baseline methods alongside their Sens-Merging enhanced counterparts across seven datasets. Specifically, when merging fine-tuned models specialized in general knowledge (Chat\footnote{huggingface.co/meta-llama/Llama-2-7b-chat-hf}), mathematical reasoning (Math\footnote{huggingface.co/TIGER-Lab/MAmmoTH-7B}), and code generation (Code\footnote{huggingface.co/mrm8488/llama-2-coder-7b}), all derived from LLaMA2-7B\footnote{huggingface.co/meta-llama/Llama-2-7b-hf}, Sens-Merging demonstrates a consistent improvement in the average performance across all domains. Specifically, when comparing the average scores of each method with and without Sens-Merging,
we find that:

\begin{table*}[t]
% \small
\caption{Performance evaluation of merged LLaMA2-13B Models (Chat, Math, Code) across 7 task-specific datasets}
\resizebox{\textwidth}{!}{%
\begin{tabular}{lcccclcclccc}
\toprule
& & \multicolumn{3}{c}{General Knowledge} &
   &
  \multicolumn{2}{c}{Mathemetical Reasoning} &
   &
  \multicolumn{2}{c}{Code Generation} &
  \multicolumn{1}{c}{} \\ \cline{3-5} \cline{7-8} \cline{10-11} 
\multirow{-2}{*}{Method} &
  \multirow{-2}{*}{\begin{tabular}[c]{@{}c@{}}Use\\ Sens\end{tabular}} &
  \multicolumn{1}{c}{MMLU} &
  \multicolumn{1}{c}{HellaSwag} &
  \multicolumn{1}{c}{TruthfulQA} &
   &
  \multicolumn{1}{c}{GSM8K} &
  \multicolumn{1}{c}{MATH} &
   &
  \multicolumn{1}{c}{MBPP} &
  \multicolumn{1}{c}{HumanEval} &
  \multicolumn{1}{c}{\multirow{-2}{*}{Average}} \\
  \hline
Chat & & 53.17 & 60.73 & 40.88 &  & 32.37 & 6.70 &  & 16.5  & 7.9 & 31.18 \\
Math & & 52.73 & 61.10 & 37.09 &  & 55.50 & 10.84 &  & 28.8 & 15.9 & 37.42 \\
Code & \multirow{-3}{*}{\textbackslash{}} & 52.65 & 60.42 & 40.64 &  & 27.29 & 5.74 &  & 21.3 & 10.4 & 31.21\\
\hline
 & \ding{55} & 52.22 & 57.52 & \textbf{41.49} &  & 49.89 & 7.32 &  & 24.1 & 9.1 & 34.52\\
\multirow{-2}{*}{Task Arithmetic} &
  \checkmark &
  \cellcolor[HTML]{EFEFEF}\textbf{55.88} &
  \cellcolor[HTML]{EFEFEF}\textbf{61.84} &
  \cellcolor[HTML]{EFEFEF}39.05 &
  \cellcolor[HTML]{EFEFEF} &
  \cellcolor[HTML]{EFEFEF}53.07 &
  \cellcolor[HTML]{EFEFEF}8.84 &
  \cellcolor[HTML]{EFEFEF}  &
  \cellcolor[HTML]{EFEFEF}\textbf{42.6} &
  \cellcolor[HTML]{EFEFEF}20.1 & 
  \cellcolor[HTML]{EFEFEF}40.20 \\
 & \ding{55} & 55.48 & 60.65 & 39.05 &  & 52.46 & \underline{9.90} &  & 40.4 & \textbf{21.3} & 39.89 \\
\multirow{-2}{*}{Ties-Merging} &
  \checkmark &
  \cellcolor[HTML]{EFEFEF}55.20 &
  \cellcolor[HTML]{EFEFEF}60.64 &
  \cellcolor[HTML]{EFEFEF}39.17 &
  \cellcolor[HTML]{EFEFEF} &
  \cellcolor[HTML]{EFEFEF}54.44 &
  \cellcolor[HTML]{EFEFEF}\textbf{10.20} &
  \cellcolor[HTML]{EFEFEF} &
  \cellcolor[HTML]{EFEFEF}\underline{41.3} &
  \cellcolor[HTML]{EFEFEF}20.6 & 
  \cellcolor[HTML]{EFEFEF}\underline{40.22} \\
     & \ding{55}  & 55.43 & 61.51 & 40.51 &  & \underline{55.19} & 9.08 &  & 39.1 & 20.1 & 40.13\\
\multirow{-2}{*}{DARE} & 
  \checkmark &
  \cellcolor[HTML]{EFEFEF}\underline{55.65} &
  \cellcolor[HTML]{EFEFEF}\underline{61.66}  &
  \cellcolor[HTML]{EFEFEF}\underline{40.64} &
  \cellcolor[HTML]{EFEFEF} &
  \cellcolor[HTML]{EFEFEF}\textbf{55.42} &
  \cellcolor[HTML]{EFEFEF}9.08 &
  \cellcolor[HTML]{EFEFEF} &
  \cellcolor[HTML]{EFEFEF}39.3 &
  \cellcolor[HTML]{EFEFEF}\underline{20.7} & 
  \cellcolor[HTML]{EFEFEF}\textbf{40.35} \\
  \bottomrule
\end{tabular}%
}
\label{tab:llama2-13b}
\end{table*}

\begin{table*}[]
% \small
\caption{Ablation studies on task-specific scaling and cross-task scaling for Task Arithmetic in LLaMA2 7B models.}
\resizebox{\textwidth}{!}{%
\begin{tabular}{lccclcclccc}
\toprule
\multirow{2}{*}{Method} &
  \multicolumn{3}{c}{General Knowledge} &
   &
  \multicolumn{2}{c}{Mathemetical Reasoning} &
   &
  \multicolumn{2}{c}{Code Generation} &
  \multirow{2}{*}{Average} \\ \cline{2-4} \cline{6-7} \cline{9-10}
& MMLU & HellaSwag & TruthfulQA &  & GSM8K & MATH &  & MBPP & HumanEval &  \\
\hline
Task Arithmetic & 41.50 & 49.63 & 37.45 & & 47.34 & 6.46 & & 13.5 & 7.3 & 29.03 \\ 
\quad + task-specific & 41.57 & 49.60 & \textbf{37.94} &  & \textbf{48.29} & \textbf{7.84} &  & 13.3 & 7.3 & 29.41 \textcolor{blue}{(+0.38)} \\
\quad + cross-task & 45.99 & 59.07 & 36.35 & & 42.00 & 7.00      & & 32.1 & 18.3 & 33.40 \textcolor{blue}{(+5.37)} \\
\quad + Sens-Merging & \textbf{46.12} & \textbf{59.10} & 36.84 &  & 42.29 & 7.12 &  & \textbf{33.1} & \textbf{18.9} & 34.78 \textcolor{blue}{(+5.75)}\\
% \hline
% Ties-Merging & 45.75 & 56.63 & 39.89 & & 46.93 & 7.74 & & 29.1 & 17.1 & 34.73 \\ 
% \quad + task-specific &  &  &  &  & \textbf{48.29} &  &  & 13.3 & 7.3 &  \\
% \quad + cross-task                                                    &      &           &            &  &       &      &  &      &           &  \\
% \quad + Sens-Merging & 46.03 & 56.87 & 40.02 &  & 47.69 & 7.80 &  & 29.8 & 17.7 & 35.13 \textcolor{blue}{(+5.75)}\\
\bottomrule
\end{tabular}%
}
\label{tab:ablation}
\vspace{-0.3cm}
\end{table*}


% \textbf{(1) Enhanced Performance on Challenging Datasets:} 
\textbf{(1) Superior Improvement in Task Arithmetic:} 
% Sens-Merging consistently enhances the performance of Task Arithmetic, Ties-Merging, and DARE when combining Chat, Math and Code fine-tuned LLMs. 
Task Arithmetic exhibits a particularly notable increase from an average score of 29.03 without Sens-Merging to 34.78 with Sens-Merging, achieving a 19.22\% relative improvement of 5.58 points.
As both Ties-Merging and DARE have implemented drop strategies to mitigate parameter interference, the integration of scaling coefficient adjustments through Sens-Merging does not achieve as substantial an enhancement as seen with Task Arithmetic. Nevertheless, Sens-Merging still contributes to performance improvements in these methods, with Ties-Merging increasing from an average score of 34.73 to 35.13, and DARE improving from 34.86 to 35.13. 
% Specifically, Sens-Merging achieves a 19.22\% relative improvement (5.58 points) in Task Arithmetic. 
\textbf{(2) Domain-Specific Improvement:}
Within the general knowledge domain, Sens-Merging significantly enhances performance on both the MMLU and HellaSwag datasets across all merging methods. In mathematical reasoning, combining Sens-Merging with the Ties-Merging baseline achieves the highest scores on both GSM8K (47.69) and MATH (7.80), surpassing their respective baselines.
In code generation, Task Arithmetic shows substantial improvements, increasing from 13.5 to 33.1 on MBPP and from 7.3 to 18.9 on HumanEval. 
% Additionally, the DARE method benefits from Sens-Merging, with relative improvements of 3.7\% (0.7 points) on MBPP and 3.2\% (0.6 points) on HumanEval.
\textbf{(3) Enhanced Performance than Individual Fine-tuned Models:} Sens-Merging enables the combined models to achieve higher performance on general knowledge and code generation tasks, even surpassing the original code fine-tuned model.
For example, when integrating the Chat, Math, and Code models using Sens-Merging, performance on the MBPP and HumanEval datasets increases significantly. Specifically, accuracy improves from 26.8 to 32.3 on the MBPP dataset and from 12.8 to 19.5 on the HumanEval dataset.
This demonstrates that model merging can overcome the challenges associated with training a single model for complex tasks by effectively integrating capabilities from other specialized fine-tuned models. Notably, when a Code model is merged with Math and Chat models, it achieves superior performance on coding tasks compared to code-specific fine-tuning alone.
% For instance, on the GSM8K dataset, merging fine-tuned models using Sens-Merging with Ties-Merging resulted in an 11.53\% relative improvement, increasing scores from 46.93 to 47.69. 
% \textbf{(2) Superior Improvement in Task Arithmetic:}
% \textbf{(3) Greater Benefits from Merging Three Models Compared to Two:} 



\paragraph{Using Different Model Architecture.} To verify the generalizability of our method across architectures, we conduct experiments using Mistral-7B models. 
Using task-specific models derived from the base Mistral-7B model\footnote{huggingface.co/mistralai/Mistral-7B-v0.1} - specifically Chat\footnote{huggingface.co/mistralai/Mistral-7B-Instruct-v0.1}, Math\footnote{huggingface.co/TIGER-Lab/MAmmoTH2-7B}, and Code\footnote{huggingface.co/Nondzu/Mistral-7B-codealpaca-lora} - our method demonstrates consistent performance improvements despite the architectural differences from LLaMA-based models.
As shown in Table \ref{tab:mistral-7b}, when combined with Task Arithmetic and DARE, Sens-Merging demonstrated remarkable performance gains, surpassing the original baselines by 10.34 and 3.22 points respectively across all evaluated datasets. With Task Arithmetic, our method shows impressive gains across domains: 11.58 points in general knowledge, 4.86 points in mathematical reasoning, and 8.45 points in code generation. When combined with DARE, Sens-Merging particularly excelled in code generation, achieving a 5-point improvement over the original DARE and even outperforming task-specialized fine-tuned models. This superiority is evidenced by higher scores on coding benchmarks: 55.1 versus 50.9 on MBPP and 43.3 versus 40.0 on HumanEval.







\paragraph{Scaling to Larger Model Size.} We further evaluate the scalability of our method using the LLaMA-2 13B\footnote{huggingface.co/meta-llama/Llama-2-13b-hf} models by merging Chat\footnote{huggingface.co/meta-llama/Llama-2-13b-chat-hf}, Math\footnote{huggingface.co/TIGER-Lab/MAmmoTH-13B}, and Code\footnote{huggingface.co/emre/llama-2-13b-code-chat} fine-tuned models. As presented in Table \ref{tab:llama2-13b}, our approach maintains consistent performance gains at larger scales. Sens-Merging with Task Arithmetic demonstrates particularly strong improvements, outperforming the baseline by 5.68 points across all datasets, with notably impressive gains in code generation (14.75 points). When combined with Ties-Merging, Sens-Merging excels in mathematical reasoning tasks. Specifically, it achieves a 3.77\% relative improvement (1.98 points) on the GSM8K dataset and a 3.03\% relative improvement on the MATH dataset.

% The consistent performance across model sizes suggests that our sensitivity-guided merging effectively captures and preserves capabilities regardless of model scale, while the relative improvements indicate that larger models may benefit even more from our approach.

\begin{table*}[]
\caption{Performance evaluation of two merged fine-tuned LLaMA2-7B models (Chat\&Math and Math\&Code) across seven task-specific datasets.}
\resizebox{\textwidth}{!}{%
\begin{tabular}{ccccccccccccc}    
\toprule
\multirow{2}{*}{\begin{tabular}[c]{@{}c@{}}Merging\\ Methods\end{tabular}} &
  \multirow{2}{*}{Models} &
  \multirow{2}{*}{\begin{tabular}[c]{@{}c@{}}Use\\ Sens\end{tabular}} &
  \multicolumn{3}{c}{General Knowledge} &
  &
  \multicolumn{2}{c}{Mathematical Reasoning} &
  &
  \multicolumn{2}{c}{Code Generation} &
  \multirow{2}{*}{Average} \\ \cline{4-6} \cline{8-9} \cline{11-12} 
 & & & MMLU & HellaSwag & TQA & & GSM8K & MATH & & MBPP & HumanEval & \\ 
  \hline
\multirow{3}{*}{/} & Chat &
  \multicolumn{1}{c}{\multirow{3}{*}{/}} & 46.38 & 57.79 & 45.17 & & 23.43 & 4.86 & & 0.3 & 0.6 & 25.5 \\ 
 & Math &
  \multicolumn{1}{c}{} & 40.05 & 56.30 & 32.56 && 48.60 & 8.50 & & 21.8 & 12.8 & 31.52\\ 
 &
  Code &
  \multicolumn{1}{c}{} & 40.76 & 57.87 & 33.17 && 7.13 & 3.62 & & 26.8 & 5.5 & 24.98\\ 
  \hline
\multirow{4}{*}{\begin{tabular}[c]{@{}c@{}}Task\\ Arithmetic\end{tabular}} &
  \multirow{2}{*}{\begin{tabular}[c]{@{}c@{}}Chat \&Math\end{tabular}} &
  \ding{55} &
  {41.36} & 49.77 & \underline{36.96} & & 45.34 & 6.96 & & 13.8 & 7.3 & 28.78\\ 
 &
  &
  $\checkmark$ &
  \cellcolor[HTML]{EFEFEF}\textbf{45.67} &
  \cellcolor[HTML]{EFEFEF}\underline{58.02} &
  \cellcolor[HTML]{EFEFEF}\textbf{37.21} &
  \cellcolor[HTML]{EFEFEF} &
  \cellcolor[HTML]{EFEFEF}\underline{45.49} &
  \cellcolor[HTML]{EFEFEF}\underline{7.14} &
  \cellcolor[HTML]{EFEFEF} &
  \cellcolor[HTML]{EFEFEF}\textbf{30.1} &
  \cellcolor[HTML]{EFEFEF}\textbf{17.7} &
  \cellcolor[HTML]{EFEFEF}\textbf{34.48} \\ 
 &
  \multirow{2}{*}{\begin{tabular}[c]{@{}c@{}}Math \&Code\end{tabular}} &
  \ding{55} & 40.54 & 56.63 & 32.80 & & \textbf{50.42} & \textbf{9.38} & &
  {22.6} & {10.4} & {31.82} \\ 
 &
  &
  $\checkmark$ &
  \cellcolor[HTML]{EFEFEF}\underline{43.67} &
  \cellcolor[HTML]{EFEFEF}\textbf{59.15} &
  \cellcolor[HTML]{EFEFEF}33.90 &
  \cellcolor[HTML]{EFEFEF} &
  \cellcolor[HTML]{EFEFEF}42.08 &
  \cellcolor[HTML]{EFEFEF}7.12 &
  \cellcolor[HTML]{EFEFEF} &
  \cellcolor[HTML]{EFEFEF}\underline{27.8} &
  \cellcolor[HTML]{EFEFEF}\underline{16.5} &
  \cellcolor[HTML]{EFEFEF}\underline{32.89} \\ 
  \hline
\multirow{4}{*}{\begin{tabular}[c]{@{}c@{}}Ties-\\ Merging\end{tabular}} &
  \multirow{2}{*}{\begin{tabular}[c]{@{}c@{}}Chat \&Math\end{tabular}} &
  \ding{55} & \underline{45.84} & 56.48 &\underline{39.17} && \underline{48.29} &7.86 && 30.1 &\underline{17.1} & \underline{34.98}\\ 
 &
  &
  $\checkmark$ &
  \cellcolor[HTML]{EFEFEF}\underline{45.86} &
  \cellcolor[HTML]{EFEFEF}56.63 &
  \cellcolor[HTML]{EFEFEF}\textbf{40.02} &
  \cellcolor[HTML]{EFEFEF} &
  \cellcolor[HTML]{EFEFEF}\textbf{48.98} &
  \cellcolor[HTML]{EFEFEF}7.92 &
  \cellcolor[HTML]{EFEFEF} &
  \cellcolor[HTML]{EFEFEF}\textbf{31.3} &
  \cellcolor[HTML]{EFEFEF}\textbf{17.7} &
  \cellcolor[HTML]{EFEFEF}\textbf{35.49}\\ 
 &
  \multirow{2}{*}{Math\&Code} &
  \ding{55} &42.52 &\underline{58.30} &36.72 &&45.11 &
 \underline{8.44} &&29.1 &14.6 & 33.54 \\ 
 &
  &
  $\checkmark$ &
  \cellcolor[HTML]{EFEFEF}42.74 &
  \cellcolor[HTML]{EFEFEF}\textbf{58.55} &
  \cellcolor[HTML]{EFEFEF}36.96 &
  \cellcolor[HTML]{EFEFEF} &
  \cellcolor[HTML]{EFEFEF}45.49 &
  \cellcolor[HTML]{EFEFEF}\textbf{8.54} &
  \cellcolor[HTML]{EFEFEF} &
  \cellcolor[HTML]{EFEFEF}\underline{30.3} &
  \cellcolor[HTML]{EFEFEF}14.8 &
  \cellcolor[HTML]{EFEFEF}33.91 \\ 
  \hline
\multirow{4}{*}{\begin{tabular}[c]{@{}c@{}}DARE\end{tabular}} &
  \multirow{2}{*}{Chat\&Math} &
  \ding{55} &\underline{46.72} &58.04 &\textbf{39.53} &&\underline{44.88} & 6.58 && \underline{30.3} &\underline{20.1} & \underline{35.16}\\ 
 &
  &
  $\checkmark$ &
  \cellcolor[HTML]{EFEFEF}\textbf{46.78} &
  \cellcolor[HTML]{EFEFEF}58.12 &
  \cellcolor[HTML]{EFEFEF}\textbf{39.53} &
  \cellcolor[HTML]{EFEFEF} &
  \cellcolor[HTML]{EFEFEF}\textbf{45.03} &
  \cellcolor[HTML]{EFEFEF}\textbf{7.06} &
  \cellcolor[HTML]{EFEFEF} &
  \cellcolor[HTML]{EFEFEF}\underline{31.6} &
  \cellcolor[HTML]{EFEFEF}\textbf{20.7} &
  \cellcolor[HTML]{EFEFEF}\textbf{35.55}\\ 
 &
  \multirow{2}{*}{Math\&Code} &
  \ding{55} &43.95 &\underline{59.21} &33.78 &&39.80 &6.34 &&29.6 &17.1 & 32.83\\ 
 &
  &
  $\checkmark$ & \cellcolor[HTML]{EFEFEF}43.98 &
  \cellcolor[HTML]{EFEFEF}\textbf{59.25} &
  \cellcolor[HTML]{EFEFEF}33.90 &
  \cellcolor[HTML]{EFEFEF} &
  \cellcolor[HTML]{EFEFEF}40.41 &
  \cellcolor[HTML]{EFEFEF}\underline{6.82} &
  \cellcolor[HTML]{EFEFEF} &
  \cellcolor[HTML]{EFEFEF}29.8 &
  \cellcolor[HTML]{EFEFEF}17.1 &
  \cellcolor[HTML]{EFEFEF}33.04 \\ 
  \bottomrule
\end{tabular}%
}
\label{tab:2-models}
% \vspace{-0.2cm}
\end{table*}

\subsection{Ablation Studies}
To understand the contribution of each component in our framework, we conduct ablation studies by incorporating either task-specific scaling factors or cross-task scaling factors into the Task Arithmetic method. As shown in Table~\ref{tab:ablation}, different scaling approaches exhibit task-dependent effectiveness. For mathematical reasoning, task-specific sensitivity scaling yields notable gains (a 21.36\% relative improvement and an increase of 1.38 points on the MATH dataset) while having limited impact on other tasks. Conversely, cross-task scaling delivers significant improvements in general knowledge and code generation tasks (4.28 and 14.8 points respectively) but decreases mathematical reasoning performance by 4.8 points. Overall, cross-task scaling provides stronger aggregate performance enhancements, achieving a total gain of 5.37 points. 
Therefore, each scaling method involves a trade-off: task-specific scaling excels at enhancing specialized capabilities (particularly mathematical reasoning) but with limited broader impact, while cross-task scaling offers stronger overall performance improvements at the cost of sacrificing some task-specific excellence.



\section{In-depth Analysis}
\subsection{Scaling Factors Analysis}
\paragraph{Layerwise Sensitivity Distribution.}
% - Visualization of sensitivity distributions across layers
% - Correlation between sensitivity scores and model performance
% - Impact of different calibration dataset sizes
% - Stability analysis of sensitivity measurements

\begin{figure}[t]
    \centering
    \includegraphics[width=1.0\linewidth]{latex/sensitivity.png}
    \caption{Layer-wise sensitivity scores across different task-specific models, with the Top-5 most sensitive layers highlighted in red.}
    \label{fig:layer_sensitivity}
    \vspace{-0.5cm}
\end{figure}

Figure~\ref{fig:layer_sensitivity} reveals distinct layer-wise sensitivity patterns across model specializations: the chat model peaks at layer 10, leveraging lower layers for language processing; the math model shows maximum sensitivity around layer 15, emphasizing middle layers for mathematical reasoning; and the code model exhibits a unique dual-peak pattern, reflecting its need for both linguistic processing in lower layers and logical reasoning in middle layers. 
Thus, by leveraging layer-wise sensitivity, we enhance the weights of the layers that are most critical to performance, thereby ensuring that specialized functions are optimally preserved.

\begin{table}[htbp]
    \centering
    \small
    \caption{Cross-task scaling coefficients across Chat, Math, and Code models.}
    \resizebox{\linewidth}{!}{%
    \begin{tabular}{lccc}
        \toprule
        Models & LLaMA2 7B & Mistral 7B & LLaMA2 13B \\ \hline
        Chat    & 0.3095   & 0.2454   & 0.2958   \\ 
        \rowcolor{gray!10}
        Math    & \textbf{0.5062}   & \textbf{0.6379}   & \textbf{0.5284}   \\ 
        Code    & 0.1843   & 0.1167   & 0.1758   \\ \bottomrule
    \end{tabular}
    }
    \label{tab:scaling}
\end{table}





\paragraph{Cross-Task Sensitivity Scaling.}

In Table \ref{tab:scaling}, we observe consistent variations in cross-task scaling factors across task vectors. 
Math models show the highest scaling coefficients (0.51-0.64), followed by Chat models (0.25-0.31), and Code models (0.12-0.18). These coefficients reveal that: mathematical reasoning provides strong transferable skills across tasks, chat abilities facilitate general language understanding, while coding skills are more specialized and less transferable.


\subsection{Merge Two Fine-tuned Models}

In addition to merging three models, we also evaluate the performance of combining two models: Chat \& Math and Math \& Code. We exclude the Chat \& Code combination as its performance is significantly lower than that of the three-model merging. 
As shown in Table \ref{tab:2-models}, when applied to two-model combinations, our Sens-merging method also outperforms baseline approaches, showing substantial improvements in Task Arithmetic method (3 points). In code generation, Sens-merging significantly improved performance over existing methods, achieving relative gains of 3.99\% (1.2 points) over Ties-Merging and 4.29\% (1.3 points) over DARE on the MBPP dataset.  
Notably, for both Ties-Merging and DARE methods, combining Chat and Math models yields better performance than merging all three models across all tasks, with Ties-Merging showing scores of 34.98 versus 34.73, and DARE showing 35.16 versus 34.86.
This indicates that simply adding more models does not guarantee better performance in merging methods. 
% The superior performance of the Chat-Math combination suggests that these two models may have more complementary capabilities and less redundant or conflicting information, allowing for more effective knowledge integration. The slight performance degradation when adding the Code model hints at potential interference during the merging process.

% Moreover, we find that merging three models and two dissimilar task models often leads to more significant performance gains than merging two similar tasks. In Task Arithmetic, merging three models results in a 4.58$\times$ improvement over merging two similar models (Math\&Code) and a 1.36$\times$ improvement over merging two dissimilar tasks (Math\&Chat). 
% This verifies our scaling method effectively balances parameter interference, thus enhancing performance.

% Sens-Merging enables the merged models to achieve higher performance on more challenging tasks, such as mathematical reasoning and code generation. 
% even surpassing the original fine-tuned models. For instance, on the MATH dataset, merging the Math and Code models using Sens-Merging with Task Arithmetic resulted in an 11.53\% relative improvement, increasing scores from 8.50 to 9.48.



% \subsection{Impact of Calibration Data}

% We investigate how the size and composition of calibration data affects our sensitivity-guided merging performance. For each task, we vary the calibration set size from 100 to 1000 samples and analyze its impact on both sensitivity measurement and final model performance.

% Results show that sensitivity patterns stabilize relatively quickly: using 500 calibration samples yields nearly identical layer-wise sensitivity distributions (±0.015 variance) compared to using 1000 samples. However, very small calibration sets (100 samples) lead to noisy sensitivity estimates, particularly in specialized tasks like mathematical reasoning where sensitivity scores show up to 0.045 variance. For cross-task generalization measurement, we find a minimum threshold of 300 samples is needed to obtain reliable alignment scores between models.

% We also examine the impact of calibration data diversity. When using domain-specific calibration data (e.g., only mathematical problems), the computed sensitivities over-emphasize domain-specific layers, leading to suboptimal merging. A balanced mixture of samples across domains (approximately 30\% from each specialized task plus 10\% general text) produces the most robust sensitivity measurements. This balanced approach helps capture both task-specific importance and cross-task interactions, resulting in better preservation of specialized capabilities after merging.

\section{Conclusion}

We introduce Sens-Merging, a novel method that determines model merging coefficients by analyzing parameter sensitivity both within specific tasks and across multiple tasks. Through extensive evaluation on Mistral 7B and LLaMA-2 7B/13B model families, we demonstrate that Sens-Merging enhances model merging performance across multiple domains, consistently outperforming both existing merging techniques and individually fine-tuned models. This improvement is particularly pronounced in code generation tasks, where merged models achieve superior results compared to specialized fine-tuning.


% \subsection{Cross-task Generalization}
% - Evaluation of generalization ability across different tasks
% - Analysis of expert model selection impact
% - Comparison of different distance metrics for generalization measurement
% - Validation of generalization scaling factors

% \subsection{Ablation Studies}
% - Impact of using only sensitivity scores vs. only generalization factors
% - Effect of different softmax temperature values
% - Influence of calibration dataset selection
% - Analysis of different layer merging strategies

% \subsection{Real-world Applications}
% - Performance on downstream tasks
% - Resource efficiency and computational cost
% - Model size vs. performance trade-offs
% - Practical deployment considerations

\section*{Limitations}
% While Sens-Merging demonstrates promising results in model merging, several limitations need to be addressed in future work.  First, the computational overhead of our method increases significantly with model size due to the need to compute sensitivity scores for each parameter. Second, our current approach primarily focuses on homogeneous model merging where base models share the same architecture. The effectiveness of Sens-Merging on heterogeneous model architectures remains to be explored. Third, the method's scalability is limited when merging multiple models simultaneously. As shown in our experiments with two-model versus three-model merging, simply adding more models does not guarantee better performance. The complexity of cross-task sensitivity analysis grows substantially with the number of merged models. Finally, while our method shows superior performance in code generation tasks, its effectiveness may vary across different specialized domains. A more comprehensive evaluation across diverse domain-specific tasks would be valuable for understanding the method's generalizability.


While Sens-Merging demonstrates remarkable performance in model merging, achieving consistent improvements across various benchmarks, it shares fundamental limitations with existing task arithmetic-based methods. For example, our current implementation primarily addresses homogeneous model merging where base models share identical architectures. While this focus allows us to achieve state-of-the-art results in such scenarios, extending Sens-Merging to heterogeneous architectures remains an exciting direction for future research. Moreover, our method is specifically designed for large language models and has been primarily validated with LoRA fine-tuned models, where weight differences between specialized models are relatively constrained. For smaller-scale models or fully fine-tuned models with larger weight divergences, our approach may require adaptations. 

% the computational complexity increases with model size, as calculating sensitivity scores for each parameter requires significant resources. However, this overhead is a one-time cost that enables more effective merging and better final performance compared to existing methods.
% Besides, our current implementation primarily addresses homogeneous model merging where base models share identical architectures. While this focus allows us to achieve state-of-the-art results in such scenarios, extending Sens-Merging to heterogeneous architectures remains an exciting direction for future research.

%Third, the method's scalability faces challenges when merging multiple models simultaneously. Our experiments with two-model versus three-model merging reveal that simply increasing the number of merged models does not guarantee proportional improvements. The complexity of cross-task sensitivity analysis grows significantly with each additional model, though our method still outperforms existing approaches in multi-model scenarios.

%Finally, while Sens-Merging demonstrates superior performance in code generation tasks, its effectiveness across different specialized domains requires further investigation. Although our current results show promising generalization capabilities, a more comprehensive evaluation across diverse domain-specific tasks would provide valuable insights into the method's broader applicability.

\section*{Ethics Statement}
This study utilizes publicly available datasets for our models. Prior research endeavors have generally taken ethical considerations into account. We have manually inspected a subset of samples and found no explicit ethical concerns, including violent or offensive content. Nonetheless, it is crucial to highlight that the output generated by large language models lacks the degree of control we might assume. Consequently, we are prepared to implement measures to mitigate any unforeseen outputs.


% Bibliography entries for the entire Anthology, followed by custom entries
%\bibliography{anthology,custom}
% Custom bibliography entries only
\bibliography{custom}

% \appendix

% \section{Example Appendix}
% \label{sec:appendix}

% This is an appendix.

\end{document}
