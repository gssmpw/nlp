\DeclareMathOperator*{\argmin}{argmin}
\DeclareMathOperator*{\argmax}{argmax}
\newcommand{\alg}{\texttt{Alg}\xspace}
\newcommand{\binomial}[2]{\ensuremath{\texttt{\textup{Binom}}\paren{#1, #2}}}
\newcommand{\bernoulli}[1]{\texttt{Bernoulli}(#1)}

% Definitions for shorthand notations
\def\ddefloop#1{\ifx\ddefloop#1\else\ddef{#1}\expandafter\ddefloop\fi}
\def\ddef#1{\expandafter\def\csname bb#1\endcsname{\ensuremath{\mathbb{#1}}}} % \bbA \mathbb
\ddefloop ABCDEFGHIJKLMNOPQRSTUVWXYZ\ddefloop
\def\ddef#1{\expandafter\def\csname b#1\endcsname{\ensuremath{\mathbf{#1}}}} % \bA \mathbf
\ddefloop ABCDEFGHIJKLMNOPQRSTUVWXYZ\ddefloop
\def\ddef#1{\expandafter\def\csname c#1\endcsname{\ensuremath{\mathcal{#1}}}} %\cA \mathcal
\ddefloop ABCDEFGHIJKLMNOPQRSTUVWXYZ\ddefloop
\def\ddef#1{\expandafter\def\csname h#1\endcsname{\ensuremath{\widehat{#1}}}} %\hA \widehat
\ddefloop ABCDEFGHIJKLMNOPQRSTUVWXYZ\ddefloop
\def\ddef#1{\expandafter\def\csname t#1\endcsname{\ensuremath{\widetilde{#1}}}} %\tA \widetilde
\ddefloop ABCDEFGHIJKLMNOPQRSTUVWXYZ\ddefloop


\newcommand{\abs}[1]{\left\lvert #1 \right\rvert} % absolute value
\newcommand{\cbrk}[1]{\left\{ #1 \right\}} % curly brackets
\newcommand{\sbrk}[1]{\left[ #1 \right]} % square brackets
\newcommand{\paren}[1]{\left( #1 \right)} % parenthesis
\NewDocumentCommand{\E}{o m o}{
    \bbE%
    \IfValueT{#1}{_{#1}}%
    \sbrk{#2%
    \IfValueT{#3}{\mid #3}%
    }
}
\NewDocumentCommand{\V}{o m o}{
    \bbV%
    \IfValueT{#1}{_{#1}}%
    \sbrk{
        #2
        \IfValueT{#3}{\mid #3}
    }
}
\NewDocumentCommand{\Prob}{o m o}{
    \bbP%
    \IfValueT{#1}{_{#1}}%
    \paren{
        #2
        \IfValueT{#3}{\mid #3}
    }
}
\newcommand{\I}[1]{\bbI\sbrk{#1}}

\NewDocumentCommand{\bigO}{o}{
    \cO%
    \IfValueT{#1}{\paren{#1}}
}
\NewDocumentCommand{\bigTildeO}{o}{
    \widetilde{\cO}%
    \IfValueT{#1}{\paren{#1}}
}
\NewDocumentCommand{\bigTheta}{o}{
    \ensuremath{\Theta}%
    \IfValueT{#1}{\paren{#1}}
}
\NewDocumentCommand{\bigOmega}{o}{
    \ensuremath{\Omega}%
    \IfValueT{#1}{\paren{#1}}
}


\newcommand{\nterm}[2]{\underbrace{#2}_{\text{Term #1}}} % numbered term term
\newcommand{\lterm}[2]{\underbrace{#2}_{#1}} % labelled term


\NewDocumentCommand{\filtration}{o}{
    \ensuremath{\cF}%
    \IfValueT{#1}{_{#1}}
}
\newcommand{\simplex}[1]{\cS(#1)}

\NewDocumentCommand{\history}{o}{
    \ensuremath{\cH}%
    \IfValueT{#1}{_{#1}}
}

\NewDocumentCommand{\CS}{o o}{\cC\cS_{#1}({#2})} % CS_{time}(\estimate)
\NewDocumentCommand{\CI}{o o}{\mathcal{CI}_{#1}({#2})}

\NewDocumentCommand{\UCB}{o o}{\mathcal{U}_{#1}({#2})}
\NewDocumentCommand{\UCS}{o o}{\mathcal{U}_{#1}({#2})}
\NewDocumentCommand{\LCB}{o o}{\mathcal{L}_{#1}({#2})}
\NewDocumentCommand{\LCS}{o o}{\mathcal{L}_{#1}({#2})}

\newcommand{\loss}{\ell}


\newcommand{\goodEvent}{\cE}
