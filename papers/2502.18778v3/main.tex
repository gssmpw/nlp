\documentclass{article}


% if you need to pass options to natbib, use, e.g.:
%     \PassOptionsToPackage{numbers, compress}{natbib}
% before loading neurips_2024


% ready for submission
% \usepackage{neurips_2024}
\usepackage[preprint]{neurips_2024}
% to compile a preprint version, e.g., for submission to arXiv, add add the
% [preprint] option:
%     \usepackage[preprint]{neurips_2024}


% to compile a camera-ready version, add the [final] option, e.g.:
%     \usepackage[final]{neurips_2024}


% to avoid loading the natbib package, add option nonatbib:
%    \usepackage[nonatbib]{neurips_2024}

% \usepackage{cite}

\usepackage{subcaption}
\usepackage{diagbox}
\usepackage{supertabular}
\usepackage{colortbl}
\usepackage{bbding}
\usepackage[utf8]{inputenc} % allow utf-8 input
\usepackage[T1]{fontenc}    % use 8-bit T1 fonts
\usepackage{microtype}      % microtypography
\usepackage{xcolor}         % colors
\usepackage{xspace}
\usepackage{amsmath,amssymb,amsfonts,dsfont,pifont,bm,bbm,mathrsfs,mathtools,nicefrac}
\usepackage{algorithm,algpseudocode,listings}
\usepackage{wrapfig}
\usepackage{booktabs,multirow,adjustbox,diagbox,threeparttable,makecell}
\definecolor{citeblue}{HTML}{0071bc}
\definecolor{textpurple}{RGB}{135,89,201}
\definecolor{Gray}{gray}{0.90}
\usepackage[pagebackref=true,breaklinks=true,colorlinks=true,citecolor=citeblue,bookmarks=false]{hyperref}
\usepackage{url}            % simple URL typesetting
\usepackage{cleveref}  % Should be loaded after 'hyperref', and works perfectly with 'subfigure'.
\usepackage{lipsum}
\usepackage{graphicx}
% \usepackage{breakurl}
% \usepackage[numbers]{natbib}
\usepackage{natbib}
\setcitestyle{numbers,square}

\crefname{section}{Sec.}{Secs.}
\Crefname{section}{Section}{Sections}
\crefname{appendix}{Appendix}{Appendixes}
\crefname{table}{Tab.}{Tabs.}
\Crefname{table}{Table}{Tables}
\crefname{figure}{Fig.}{Figs.}
\Crefname{figure}{Figure}{Figures}
\crefname{equation}{Eq.}{Eqs.}
\Crefname{equation}{Equation}{Equations}
\hyphenpenalty=1200

\newcommand{\method}{M2-omni\xspace}

\title{\method: Advancing Omni-MLLM for Comprehensive Modality Support with Competitive Performance}

% The \author macro works with any number of authors. There are two commands
% used to separate the names and addresses of multiple authors: \And and \AND.
%
% Using \And between authors leaves it to LaTeX to determine where to break the
% lines. Using \AND forces a line break at that point. So, if LaTeX puts 3 of 4
% authors names on the first line, and the last on the second line, try using
% \AND instead of \And before the third author name.


\author{%
  Qingpei~Guo\thanks{Corresponding author.}\quad
  Kaiyou~Song\thanks{Equal Contributors.}\quad
  Zipeng~Feng\footnotemark[2]\quad
  Ziping~Ma{\footnotemark[2]} \quad
  Qinglong~Zhang \quad
  Sirui~Gao \\
  \textbf{Xuzheng~Yu} \quad
  \textbf{Yunxiao~Sun}\quad
  \textbf{Tai-Wei~Chang}\quad
  \textbf{Jingdong~Chen}\quad
  \textbf{Ming~Yang} \quad
  \textbf{Jun~Zhou}\\[1pt]
  %
  Ant Group\\
  %
  \texttt{\{qingpei.gqp,jingdongchen.cjd\}@antgroup.com}\\
  \enskip \\
}


\begin{document}


\maketitle
\begin{abstract} 
The integration of Large Language Models (LLMs) into software development has revolutionized the field, particularly through the use of Retrieval-Augmented Code Generation (RACG) systems that enhance code generation with information from external knowledge bases. However, the security implications of RACG systems, particularly the risks posed by vulnerable code examples in the knowledge base, remain largely unexplored. This risk is notably concerning given that public code repositories, which often serve as the sources for knowledge base collection in RACG systems, are usually accessible to anyone in the community. Malicious attackers can exploit this accessibility to inject vulnerable code into the knowledge base, making it toxic. 
Once these poisoned samples are retrieved and incorporated into the generated code, they can propagate security vulnerabilities into the final product. This paper presents the first comprehensive study on the security risks associated with RACG systems, focusing on how vulnerable code in the knowledge base compromises the security of generated code. We investigate the LLM-generated code security across different settings through extensive experiments using four major LLMs, two retrievers, and two poisoning scenarios. Our findings highlight the significant threat of knowledge base poisoning, where even a single poisoned code example can compromise up to 48\% of the generated code. 
Our findings provide crucial insights into vulnerability introduction in RACG systems and offer practical mitigation recommendations, thereby helping improve the security of LLM-generated code in future works.
\end{abstract}


\section{Introduction}

\begin{figure}[h]
    \centering
    \begin{overpic}[trim=0cm 0cm 0cm 0cm,clip,angle=0,origin=c,width=.4\linewidth]{images/teaser_absolute.png}
        %  trim={<left> <lower> <right> <upper>}
        %  \put(horiz, vert)
        %  \put(horiz, vert){\rotatebox{90}{Text}}
        %
        \put(107, 32){$\mathbf{\to}$}
    \end{overpic}\hspace{1cm}
    \begin{overpic}[trim=0cm 0cm 0cm 0cm,clip,angle=0,origin=c,width=.4\linewidth]{images/teaser_translated_yellow.png}
        %  trim={<left> <lower> <right> <upper>}
        %  \put(horiz, vert)
        %  \put(horiz, vert){\rotatebox{90}{Text}}
        %
    \end{overpic}
    \caption{Using translation methods, a controller trained on an environment with a given visual variation \textit{(left)} can be reused without any training or fine-tuning on a different environment (\textit{right}) with comparable performance. In red we see the trajectory of a car driven by the same controller when connected to two different encoders, one for each visual variation.
    }
    \label{fig:teaser}
\end{figure}

Deep Reinforcement Learning (RL) has enabled agents to achieve remarkable performance in complex decision-making tasks, from robotic manipulation to high-dimensional games (Mnih et al., 2015; Silver et al., 2017). 
Although recent RL techniques achieved strong improvements over sample efficiency \citep{yarats2021drqv2, kostrikov2020image}, training new agents remains a costly process, both in computational and temporal terms.
Despite these advances, most methods still require at least partial retraining when dealing with domain shifts such as visual appearance, reward functions, or action spaces \citep{pmlr-v97-cobbe19a, zhang2020learning}. These domain changes typically require expensive retraining, which can be prohibitive for real-world settings that require millions of interactions.

A variety of approaches have been proposed to address these shifting conditions. Domain randomization \citep{tobin2017domain, sadeghi2016cad2rl} trains agents across diverse visual styles or physics settings, promoting invariant features but demanding broader coverage of possible variations. Multi-task RL \citep{parisotto2015actor, teh2017distral} attempts to learn shared representations across multiple tasks.

In the supervised setting, recent representation learning techniques \citep{Moschella2022-yf,maiorca2023latent, norelli2022b, cannistraci2023bricks}, show that it is possible to zero-shot recombine encoders and decoders to perform new tasks across different modalities (images, text..) and tasks (classification, reconstruction) and even architectures.
In RL, methods adopting the relative representation framework \citep{Moschella2022-yf} have shown promising results in adapting encoders to different controllers with zero or few-shots adaptation, for robotic control from proprioceptive states \citep{jian2021adversarial} or for playing games in the Gymnasium suite \citep{towers2024gymnasium} from pixels \citep{ricciardi2025r3lrelativerepresentationsreinforcement}.
These methods, however, still require training models to use the new relative representations.

By contrast, \cite{maiorca2023latent} suggest that modules from independently trained neural networks can be connected via a simple linear or affine transformation, with no training constraint or fine-tuning required, if such transformations can be reliably estimated from a small set of “anchor” samples, pairs of states or observations deemed semantically equivalent.

Our main contribution is the implementation of a RL method based on semantic alignment to map between latent spaces of different neural models, so that their encoders and controllers can be stitched with the goal of creating new agents that can act on visual-task combinations never seen together in training. This includes the use of the transformations to map modules from different networks, and the collection of anchor samples used to estimate these transformations. We call our method Semantic Alignment for Policy Stitching (\textbf{SAPS}).
We perform analyses and empirical tests on the CarRacing and LunarLander environments to show the performance of new agents created via zero-shot stitching of encoders and controllers trained on different visual-task variations, demonstrating significant gains compared to existing zero-shot methods.
\section{Unified Framework for MultiModal Understanding and Generation}\label{sec:overall_architecture}



\begin{figure}[t]
    \centering
    \includegraphics[width=1.0\linewidth]{figures/architecture.pdf}
    \caption{
    \textbf{Overall architecture of \method.} \method can process arbitrary combinations of text, image, video, and audio modalities as input, generating multimodal sequences interleaving with text, image, or audio outputs.
    }
    \label{fig-architecture}
\end{figure}

% In this section, we discuss the overall architecture of \method in \cref{subsec:app_architecture} first.
% Subsequently, the detailed multimodal multi-task training procedure is introduced in \cref{subsec:app_training}.
% Then, we introduce the comprehensive training data configurations used in \method in \cref{subsec:app_data}.

\subsection{Overall Architecture}\label{subsec:app_architecture}

% It is challenging to build a unified framework for multimodal understanding and generation due to the disparate representational spaces required for understanding and generation tasks.
% In this work, our key insight is to decouple the two representational spaces and design a unified modeling framework that can effectively integrate multimodal understanding and generation tasks.
% To this end, our \method employs modality-specific processing pathways to minimize mutual interference between modalities and tasks.
% The overall architecture is shown in \cref{fig-architecture}, which consists of modality encoders (the vision encoder and the audio encoder), connectors, LLM, and modality generators.
% Taking multimodal information (textual, image, video, and audio) as input, the corresponding modality encoders encode them into the feature spaces.
% The connectors project the features into a unified LLM embedding space, in which the projected tokens are concatenated as the input of the LLM.
% The LLM integrates the multimodal information and output the decodding embecdding for unified multimodal understanding and generation.
% For image generation tasks, we employ textual descriptions as an intermediate representation, thereby circumventing the need for direct alignment of latent image features.
% For speech generation, we leverage a discrete token prediction-based approach to enable real-time, streaming audio synthesis with minimal impact on the performance of other modality branches.
% Below we introduce the details of each module.


We aim to build a unified framework that simultaneously supports multimodal understanding and generation tasks, while minimizing interference between different modality tasks through decoupled architecture design. Our encoding procedure is inspired by the design of UNIFIED-IO2~\cite{unified_io2}, which utilizes a modality-aware encoder to map diverse inputs (such as images, text, audio, and video) into a shared token representation space. Previous studies, such as Janus~\cite{Janus}, have demonstrated that multimodal understanding and generation tasks can interfere with each other, mainly due to the disparate levels of information granularity required for image understanding and generation.  In contrast to Janus~\cite{Janus}, which employs separate pathways for visual encoding, we leverage textual descriptions as an intermediate representation for image generation tasks, effectively bypassing the need for direct alignment of latent image features. For speech generation, we adopt a discrete token prediction-based approach, which enables real-time, streaming audio synthesis while minimizing the impact on the performance of other modality branches. Figure 2 illustrates the overall architecture of the proposed model. We will elaborate on the details of each module below.






% Most vision encoders used in previous MLLMs, e.g., CLIP ViT encoder~\cite{clip_2021}, EVA-CLIP~\cite{eva-clip_2023}, SigLIP~\cite{siglip_2023}, and InternViT~\cite{internvl_2024}, can only process fixed resolution images.
% To process images with arbitrary resolutions, most previous works, e.g., LLaVA-Next~\cite{llava-next_2024} and InternVL~\cite{internvl_2024}, split a large image into several tiles that the vision encoder can process.
% However, this method will disrupts the spatial relationships between patches within an image.
% following recent methods~\cite{qwen2-vl_2024, laurenccon2024matters, points1.5_2024}.
% It can process images of any resolution by dynamically converting them into a variable number of visual tokens, without patch splitting.
% During training, multiple image sequences are packed into a single sequence for batch forwarding.
% For image, we add two special tokens (\texttt{<image>} and \texttt{</image>}) at the beginning and end of the compressed image tokens.
% Consequently, a $448\times448$ image is represented by 258 tokens.
% For each video, we uniformly sample a fixed number of 16 frames during the pre-training stage to accelerate training and ensure stability.
% The two special tokens are replaced with \texttt{<video>} and \texttt{</video>} to differentiate between video and image.

\textbf{Vision Encoder.}
In \method, the vision encoder extracts representations from images or whole videos. We utilize a NaViT~\cite{navit_2024}  as the vision encoder, capable of processing videos and images of arbitrary resolution. To reduce the length of visual tokens, we concatenate adjacent $2\times2$ tokens into a single token and use an MLP to reduce the dimension to the original dimension, thereby downsampling the visual representation.
% In \method, we employ a NaViT~\cite{navit_2024} as the vision encoder, which can process images of any resolution. To reduce the visual tokens of each image, an MLP is used to compress adjacent $2\times2$ tokens into a single token.


\textbf{Audio Encoder.}
We utilize the SAN-M~\cite{gao2020san, gao2022paraformer} encoder to extract audio tokens. Subsequently, we apply 1x3 average pooling to the audio encoder's output, aggregating every three adjacent tokens into a single token, which reduces the overall number of audio tokens. To accommodate the variability in audio token sequence lengths, we pad the compressed audio sequence with special \texttt{<audio\_pad>}  tokens, thereby ensuring that all sequences conform to a uniform length.


% Additionally, we incorporate two special tokens (\texttt{<audio>} and \texttt{<{/audio>}}) at the beginning and end of the audio tokens.
% Consequently, each audio is represented by 258 tokens.

\textbf{M2-omni LLM.} The M2-omni LLM integrates the multimodal information and outputs the decoder embedding for unified multimodal understanding and generation. Our M2-omni LLM is initialized with pre-trained weights from the Llama3~\cite{llama_2023, llama3_2024} series, specifically Llama3.1-8B or Llama3.3-70B. To facilitate unified positional encoding across textual, image, video, and audio modalities, and to enable the model to generalize to longer sequences during inference, we substitute the original 1D-RoPE~\cite{su2024roformer} in Llama with M-RoPE~\cite{qwen2-vl_2024}.

\textbf{Image Generator.} To decouple the representation spaces of generation and understanding, building upon the insights from~\cite{li2023textbind, unifiedmllm, wang2024modaverse}, we utilize textual descriptions as an intermediate representation for image generation. During training, we warp the image captions with two special tokens, i.e. \texttt{<gen\_image>} and \texttt{</gen\_image>},  allowing the model to generate textual descriptions for image generation in a flexible and unconstrained manner. At inference time, the \method LLM generates the textual description, and the generated captions enclosed by the two special tokens are utilized as the textual condition for image generation. We employ an offline Stable Diffusion (SD) model~\cite{sd_2022} as the image generator.
% To decouple the representation spaces of generation and understanding, inspired by~\cite{li-etal-2024-textbind, unifiedmllm, wang2024modaverse}, we employ textual descriptions as an intermediate representation for image generation.
% Specifically, we warp the image captions with two special tokens \texttt{<gen\_image>} and \texttt{</gen\_image>} during training, which enables the model to output the textual description for image generation in a free-form manner.
% During inference, the LLM outputs the textual description and the generated captions between the two special tokens are employed as the textual condition for image generation.
% We employ an offline Stable Diffusion (SD) model as the image Generator.

\textbf{Audio Decoder.}
Inspired by the approaches in ~\cite{MinMo, mini_omni2}, we utilize the M2-omni LLM to predict discrete audio tokens for speech generation in an end-to-end style. The predicted discrete audio tokens are then fed into the pretrained CosyVoice~\cite{du2024cosyvoice} flow matching and vocoder model to generate audio streams. Given the similarity in form between audio discrete tokens and language tokens, we can repurpose the M2-omni LLM's model structure to facilitate audio generation tasks, thereby enabling compatibility with multimodal understanding tasks.

% For speech generation, we leverage the model to predict discrete audio tokens, emulating the style of MinMo~\cite{MinMo} and Janus~\cite{mini_omni2}. Subsequently, the CosyVoice~\cite{du2024cosyvoice} decoder is applied to convert these discrete audio tokens into audio streams. Since audio discrete tokens are similar in form to language tokens, we can leverage the model structure of the LLM backbone, ensuring compatibility of audio generation tasks with multimodal understanding tasks.

\cref{tab-model_config} illustrates the detailed model configuration of \method, and \cref{{fig-template}} demonstrates the data templates for image, video, and audio.

\begin{table}[t]
\centering\footnotesize
\caption{
\textbf{Detailed pre-trained model configuration of \method.}
}
% \vspace{3pt}
\setlength{\tabcolsep}{3pt}
\begin{tabular}{c|c|c|c|c|c}
\toprule
Model Name & \#Param & Vision Encoder & Audio Encoder & LLM & SD \\
\midrule
\method-9B & 8.8B & \multirow{2}{*}{ViT-600M~\cite{qwen2-vl_2024}} & \multirow{2}{*}{paraformer-zh~\cite{gao2022paraformer}} & Llama3.1-8B~\cite{llama3_2024} & \multirow{2}{*}{SD-3-medium~\cite{esser2403scaling}} \\
\method-72B & 71.8B & & & Llama3.3-70B~\cite{llama3_2024} & \\
\bottomrule
\end{tabular}
\label{tab-model_config}
% \vspace{-12pt}
\end{table}

\begin{figure}[t]
    \centering
    \includegraphics[width=0.9\linewidth]{figures/template.pdf}
    \caption{
    \textbf{Illustration of the templates of image, video, and audio.}
    }
    \label{fig-template}
\end{figure}

\subsection{Multi-Stage Training with Progressively Modality Alignment}\label{subsec:app_training}

% Given a multimodal dataset containing discrete token sequences, each of which can be formulated as $x = (x_1, \cdots, x_\ell)$.
% The overall training objective of \method is modeling the multimodal sequence distribution in an autoregressive manner:
% \begin{equation}
% \log p_{\theta} (x) = \sum_{i = s}^{\ell-1} \log p_{\theta}(x_{i+1} | x_{1}, \dots, x_{i}),
% \end{equation}
% where $\theta$ denotes the parameters of the model and $\ell$ is the sequence length, and $s$ denotes the start index of tokens to apply autoregressive loss.
% The goal of \method is to achieve the unification of multimodal understanding and generation.
% However, it is challenging to maintain optimal performance across all modalities when training such a complex omni-MLLMs framework.
% This performance degradation often arises from significant disparities in data quantity and convergence rates across different tasks.
% To overcome this, we designed a multimodal multi-task balanced training procedure.
% Specifically, the pre-training and instruction tuning stages are divided into three stages, designed to progressively integrate more modalities and enhance the model’s capabilities on all modalities.
% Furthermore, we introduce a \textit{step balance strategy during pre-training} and a \textit{dynamic adaptive balance strategy during instruction tuning} to ensure all modalities reach their optimal performance.
% In this section, we will introduce the overall training procedure of \method first, then the step balance strategy during pre-training and the dynamic adaptive balance strategy during instruction tuning will be introduced in details.
% The training procedure of \method includes three main stages: pre-training, instruction tuning, and alignment tuning.

Given a multimodal dataset, we employ modality-aware encoders to project diverse modality inputs, including images, text, audio, and video, into a unified token representation space. Formally, the input multimodal sequences are denoted as  $x = (x_1, \cdots, x_\ell)$, where $\ell$ represents the length of the sequence, and each $x_i$ corresponds to a modality input token (e.g., image, text, audio, or video).  In particular, we model the joint probability distribution of the multimodal sequence in an autoregressive manner, where each token is conditioned on the previous tokens, as shown in the following equation:
\begin{equation}
\log p_{\theta} (x) = \sum_{i = s}^{\ell-1} \log p_{\theta}(x_{i+1} | x_{0}, \dots, x_{i}),
\end{equation}
Notably, $s$ denotes the start index of discrete output tokens and only discrete output tokens $x_{>s}$ are considered as the modeling targets, $\theta$ denotes the parameters of the model.
We introduce a multi-stage training framework that progressively achieves modality alignment by incrementally incorporating knowledge from multiple modalities. As shown in \cref{fig-pretrain_pipeline}, the overall training procedure of our proposed \method consists of three primary stages: pre-training, instruction tuning, and alignment tuning.  Both the pre-training and instruction tuning stages are further divided into three sub-stages, each designed to incrementally incorporate additional modalities. The training hyperparameters and configurations are summarized in \cref{tab:app_train_hyperparameter}.

\subsubsubsection{\textbf{Pre-training}}

\begin{figure}[t]
    \centering
    \includegraphics[width=1.0\linewidth]{figures/training_pipeline.pdf}
    \caption{
    \textbf{Illustration of the training pipeline of \method.}
    Both the pre-training and the instruction tuning contain three stages, designed to progressively absorb knowledge from more modalities and ensure the model's optimal performance on all modalities and tasks. $L_{un}$ and $L_{gen\_a}$ denote understanding and audio generation loss, respectively.
    }
    \label{fig-pretrain_pipeline}
\end{figure}

The pre-training stage primarily focuses on aligning multiple modalities with our \method LLM, thereby enabling it to capture multimodal concept representations and develop cross-modal perception capabilities.
% The pre-training pipeline of \method is illustrated in \cref{fig-pretrain_pipeline}, which includes three stages, designed to progressively absorb knowledge from more modalities.
% During this stage, we employ the step balance strategy to ensure that the model can stably and unbiasedly learn the knowledge of each modality.
% In addition, we prioritize preserving strong performance on pure text tasks to maintain the robustness of \method's language understanding capability throughout the training process.

\textbf{Stage 1. Encoder Alignment.}
This phase leverages image-text pairs, OCR data, and audio-text pairs for training, achieving alignment between the visual/audio encoders and \method LLM. Moreover, by concatenating multiple image-text pairs into a single interleaved sequence, we enhance in-context understanding capabilities and attain a $1.5\times$ acceleration in training efficiency.
% This phase exclusively utilizes image-text pairs, OCR data, and audio-text pairs for training, achieving alignment between the visual/audio encoders and M2-omni LLM.
% During this stage, only the image and audio connectors are trained while the encoders and LLM remain frozen.
%Following \cite{llava-next_2024}, the bridging module employs a two-layer MLP architecture to enhance modal alignment capabilities.
% For image-text pairs, the maximum number of image tokens is maintained at 320, while 960 for OCR data.
%Audio embeddings after the connector are compressed to half length via a 1×2 pooling.
% We pack the single-image pairs into interleaved sequences, which enhance the in-context understanding capabilities while achieving a $1.5\times$ acceleration in training efficiency.
% The learning rate is set to $2e^{-5}$.

\textbf{Stage 2. Image-Text Knowledge Enhancement.}
This stage concentrates on training with high-quality image-text pair data (selected from stage 1) and OCR data, with a specific focus on enhancing image-text fine-grained comprehension capabilities. This, in turn, facilitates improved understanding of interleaved image-text and video understanding tasks in stage 3. Furthermore, language-only pure text data is incorporated to prevent the degradation of \method LLM's language understanding capabilities.

% This stage focuses exclusively on training with high-quality image-text understanding data (selected from stage 1) and OCR data, specifically targeting the enhancement of image-text fine-grained comprehension capabilities to ensure more stable convergence for interleave-image-text and video understanding tasks in stage 3.
% Additionally, language-only pure text data is included to maintain the robustness of \method's language understanding capability.



% During this phase, all modules except the audio encoder and audio connector are unfrozen.
% We maintain the same maximum image resolution as in Stage 1.
% The learning rate is set to $1e^{-5}$ for all modules except the vision encoder which is set to $1e^{-6}$.

\textbf{Stage 3. MultiModal Joint Training.}
This stage integrates omni-modality knowledge in a single stage, thereby facilitating comprehensive modality alignment and unified representation learning. In this stage, we incorporate high-quality image-text pairs, video-text pairs, interleaved image-text sequences, audio-text pairs, and language-only data for end-to-end multimodal pre-training.
To balance convergence rates across different modalities, a step balance strategy is employed in this stage, which will be introduced in \cref{subsubsec-Step Balancing Strategy}.


% This stage aims at learning all-modality knowledge.
% This comprehensive stage incorporates high-quality image-text pairs, video-text pairs, interleaved image-text sequences, audio-text pairs, and language-only data for end-to-end multimodal pre-training.
% To balance convergence rates across different modalities, a step balance strategy is performed in this stage, which will be introduced in \cref{subsubsec-Step Balancing Strategy}.


\subsubsubsection{\textbf{Instruction Tuning}}

Instruction tuning aims to make models better understand the instructions from users and fulfill the demanded tasks.

\textbf{Stage 1. Image-Text Instruction Tuning}.
This stage concentrates on enhancing the model's instruction-following ability for image modality, particularly in specialized image-related tasks, such as science, OCR, documents, and charts, which were not adequately learned during pre-training.

\textbf{Stage 2. Visual Instruction Tuning}.
This stage aims to enhance the modell's comprehensive capability on visual modality, including the capability on image-text, video-text, and interleaved image-text understanding.

\textbf{Stage 3. Omni-Modality Instruction Tuning}.
This stage further integrates the audio modality and generation tasks, enabling the model to follow instructions on mixed multi-modal sequences. Our study reveals that coordinating the training progress of diverse modalities and tasks is challenging, as the model's optimal performance across all modalities is hindered by inconsistent data volumes and convergence speeds among tasks. We propose a dynamic adaptive balance strategy to address this issue, which will be introduced in \cref{subsubsec-Dynamic Adaptive Balance}.

% As shown in \cref{fig-pretrain_pipeline}, the instruction tuning pipeline of \method includes three stages, designed to progressively integrate more modalities and enhance the model’s capabilities on all modalities.
% During this stage, we propose a dynamically adjusted gradient weighting strategy to ensure all modalities reach optimal performance.

% \begin{figure}[t]
%     \centering
%     \includegraphics[width=1.0\linewidth]{figures/sft_pipeline.pdf}
%     \caption{
%     \textbf{Illustration of the instruction tuning pipeline of \method.}
%     The training procedure includes three stages, designed to progressively integrate more modalities and ensure the model’s optimal performance on all modalities and tasks.
%     }
%     \label{fig-sft_pipeline}
% \end{figure}

% \textbf{Stage 1. Image-Text Instruction Tuning}.
% As shown in \cref{fig-pretrain_pipeline}, this stage focuses on instruction tuning for vision modality.
% The aim of this stage is to enchance the model’s instruction following ability and make it capture more vision-related knowledge that is not sufficiently learned during pre-training, such as science, OCR, documents, charts, etc.
% During this stage, only the vision encoder (and its MLP) and LLM are trainable.
% The learning rate is set to $2e^{-5}$ for all modules except the vision encoder which is set to $2e^{-6}$.
% The minium and maxium token number for each image are set to 100 and 1280, respectively.
% Image-text data and pure text data are used in this stage, which is denoted as \textbf{IT-Stage-1-Data} and will be introduced in \cref{subsubsec:app_sft_data}.

% \textbf{Stage 2. Vision Instruction Tuning}.
% The illustration of this stage is shown in \cref{fig-sft_pipeline}(b).
% This stage aims to enhance the model’s comprehensive capability on vision modality, including the capability on image-text, video-text, and interleaved image-text understanding.
% Four types of data are used in this stage, i.e., text, image-text, video-text, interleaved image-text, which is denoted as \textbf{IT-Stage-2-Data} and will be introduced in \cref{subsubsec:app_sft_data}.
% During this stage, the trainable modules and hyperparameters keep the same as stage-1.
% The maxium token number for each image is raised to 2560.
% For video and interleaved image-text data, the minium and maxium token number for each frame (or image) are set to 128 and 768, respectively.
% For video, the maxium frame number is set to 24.

% \textbf{Stage 3. All-Modality Instruction Tuning}.
% This stage further integrates the audio modality and generation tasks, allowing the model to be capable of handling all modalities and empowering it with generation ability.
%The learning rate is set to $1e^{-5}$ for all modules except the vision encoder which is set to $1e^{-6}$.
% The setting of the token number is the same as stage-2, except for video.
% For videos, we uniformly sample two frames per second by default.
% To constrain the total video token length, we sample up to 128 frames and dynamically adjust the frame resolution between 100,352 and 570,752 pixels.
% There are 6 types of data used in this stage, including text, image-text, video-text, interleaved image-text, audio-text, and text-audio, denoted as \textbf{IT-Stage-3-Data}.
% During this stage, all modules are trainable except the vision encoder and audio encoder considering that their feature extraction abilities have been sufficiently enhanced during previous stages.
% In this study, we found that it’s difficult to coordinate the training progress of various modalities and tasks and guarantee the model’s optimal performance on all modalities since the data volume and convergence speed of each task are inconsistent.
% We propose a dynamic adaptive balance strategy to address this issue, which will be introduced in \cref{subsubsec-Dynamic Adaptive Balance}.

% \subsubsection{Post-training}\label{subsubsec:app_post_training}
\subsubsubsection{\textbf{Alignment Tuning}}

This phase focuses on refining the quality and stylistic coherence of chat interactions, ensuring the model's proficiency is maintained across all modalities.  The instruction tuning stage equips the model with general multimodal conversational abilities. However, the model's responses often suffer from limitations, including brevity, lack of fluency, irrelevance, inappropriate formatting, and hallucinations, which can compromise the user experience. To mitigate these limitations and further enhance the chat experience, a preference alignment tuning stage is introduced following the instruction tuning stage. This stage employs a unified training strategy that integrates DPO~\cite{rafailov2024directpreferenceoptimizationlanguage} and instruction tuning, as defined by the equation:
\begin{equation}\label{alignment_tuning}
Loss_{at}(x) = L_{dpo}(x_{chosen}, x_{rejected})+\lambda*L_{it}(x_{chosen}, x_{it}),
\end{equation}
where $L_{dpo}$ and $L_{it}$ denotes the DPO loss and instruction tuning loss respectively. $x_{chosen}$ and $x_{rejected}$ represent the chosen samples and rejected samples from the preference dataset, and $x_{it}$ denotes the samples from the instruction dataset. In practice, we empirically set $\lambda$ to 0.3.  Additionally, we employ Low-Rank Adaptation (LoRA) \cite{hu2021loralowrankadaptationlarge} to update 5.0\% of the LLM's backbone weights, thereby preventing catastrophic forgetting.

% This phase aims to enhance the quality and stylistic consistency of chat interactions while preserving the model’s proficiency across all modalities.
% During the instruction tuning phase, the model acquires general multimodal conversational abilities. Nevertheless, its responses often exhibit shortcomings such as brevity, lack of fluency, irrelevance, inappropriate formatting, or hallucinations, which can detract from the user experience.

% To address these issues and further refine the chat experience, we introduce a preference alignment tuning stage following the instruction tuning. This stage primarily employs Direct Preference Optimization (DPO ~\cite{rafailov2024directpreferenceoptimizationlanguage}) based on our curated datasets focused on experience preference training.

% To preserve the model's initial capabilities and alleviate catastrophic forgetting, we have developed an innovative alignment training approach utilizing two strategies. First, we adopt a unified training strategy that integrates DPO and instruction tuning. This involves adjusting the ratio of instruction tuning and DPO datasets and employing a mixed-loss training approach on a per-batch basis, as defined by the equation:
% \begin{equation}
% L_{at} (x) = w_1L_{dpo}(x_{chosen}, x_{rejected})+w_2L_{it}(x_{chosen}, x_{it}),
% \end{equation}
% where $L_{at}$, $L_{dpo}$ and $L_{it}$ denotes the alignment tuning unified loss, DPO loss, and instruction loss respectively. $x_{chosen}$ and $x_{rejected}$ denotes the chosen samples and rejected samples of preference dataset, respectively, and $x_{it}$ denotes the samples of instruction dataset. Additionally, we set hyperparameters $w_1$ as 1.0 and $w_2$ as 0.3.
% Secondly, we apply Low-Rank Adaptation (LoRA) \cite{hu2021loralowrankadaptationlarge}, updating 5.0\% of the LLM's backbone weights.


% \begin{figure}[t]
%     \centering
%     \includegraphics[width=1.0\linewidth]{figures/posttraining_pipeline.pdf}
%     \caption{
%     \textbf{Illustration of the instruction tuning pipeline of \method.}
%     }
%     \label{fig-post_training_pipeline}
% \end{figure}

\begin{table*}[t]
\centering
\caption{\textbf{Detailed training hyperparameters and configurations for \method.}
The model configurations are meticulously tuned to achieve consistent performance across various modalities and tasks.
Note that T, I, V, and A in the modalities row denote textual, image, video (and interleaved image-text), and audio, respectively.
U and G in the task row denote understanding and generation tasks, respectively.
LR denotes the learning rate, and AT represents the alignment tuning stage.
}
{\fontsize{8}{10}\selectfont
\renewcommand{\arraystretch}{1.0}
{
\setlength\tabcolsep{5pt}
\begin{tabular}{l|ccc|ccc|c}
\toprule
\multirow{2}{*}{Settings}         & \multicolumn{3}{c|}{Pre-training}   & \multicolumn{3}{c|}{Instruction Tuning} & AT \\
                                  & Stage 1    & Stage 2    & Stage 3    & Stage 1          & Stage 2           & Stage 3          &           \\
\hline
Data                           & \makecell{PT-Stage-\\1-Data}  & \makecell{PT-Stage-\\2-Data}    & \makecell{PT-Stage-\\3-Data}  & \makecell{IT-Stage-\\1-Data}        & \makecell{IT-Stage-\\2-Data}         & \makecell{IT-Stage-\\3-Data}        & \makecell{AT\\-Data}        \\
Modalities & I\&A        & T\&I      & All              & T\&I        & T\&I\&V         &  All  &  All      \\
Tasks  & U        & U      & U              & U        & U         &  U\&G  &  U\&G      \\
Trainable                         & Connectors        & \makecell{Vision Encoder\\+Connector}      & Full Model & Full Model              & Full Model        & \makecell{w/o\\Encoders}              & \makecell{w/o\\Encoders}       \\
LR                     & 2e-5       & 1e-5         & 2e-5       & 2e-5             & 2e-5              & 1e-5             & 5e-6             \\
LR of Encoders                     & --       & 1e-6         & 2e-6       & 2e-6             & 2e-6              & --             & --             \\
Weight Decay                      & 0.05       & 0.05         & 0.05       & 0.05             & 0.05              & 0.05             & 0.05             \\
Training Epochs                   & --         & --           & --          & 1               & 2                 & 1               & 2                \\
Max Image Tokens              & 320         & 320           & 320         & 1280               & 2560                & 2560               & 2560               \\
Max Video Frames              & --         & --           & 8         & --               & 16                & 128               & 128               \\
\bottomrule
\end{tabular}
}
}
\label{tab:app_train_hyperparameter}
\end{table*}













\section{Baseline Agent Architectures}

To demonstrate the usefulness of \asyncfw{} as a research platform, we develop and evaluate a reference  
LM-powered agent implementation to perform tasks by coordinating interactions, retrieving relevant information, and posing targeted queries to other organization members. 
We consider an event-based reactive agent, which is triggered by user actions: upon getting a message from any organization member, the agent follows ReAct-style prompting loop \cite{DBLP:conf/iclr/YaoZYDSN023}, taking actions, making observations, and performing reflection, until it decides to pause and wait for a next event, or terminate the session.




\subsection{Actions}

The agent can perform a few types of actions.
\textbf{Document Retrieval}:
agents have access to documents accessible to the initiating user, by invoking a function \texttt{search\_documents(query: str)}. Documents are indexed using a standard BM25 index, and the tool call returns a fixed number (upto 3) of documents with the highest matching score. 
\textbf{People Retrieval}:
agents can search through a repository of employee profiles and knowledge areas, by invoking a function \texttt{search\_relevant\_people(query: str)}. 
However, these expertise profiles may be outdated or imprecise, requiring the agent to navigate uncertainty while coordinating queries. As in document retrieval, descriptions are retrieved using a standard BM25 index. A fixed number (up to 10) of highest-scoring results are returned.
\textbf{Sending Messages:} 
the agent is capable of exchanging messages with any person in the organization. 
%One simplifying design choice we make is that all communications happen between exactly two participants at a time: the agent and one person. 
\textbf{Person Resolution}:
the agent can resolve a person name to get their user ids, to be used to send messages to them.
\textbf{Turn and Session Completion}: agent can mark the current turn or the entire session as completed. 

Signatures of Python functions corresponding to the allowed actions are provided in the prompt. See Appendix \ref{sec:action_descriptions} for the full set of action descriptions. %Appendix \ref{sec:action_descriptions} has detailed function signatures.



\subsection{Observations and Reflection}

After each action is taken, the agent receives a textual observation.
These include retrieved documents or descriptions of collaborators. 
%Send message action does not produce an immediate return value. 
As is typical in LLM-based agent architectures, these observations are simply appended to the agent's prompt. Before invoking additional actions, the agent may perform \emph{reflection} actions, corresponding to text-based (``scratchpad'' or ``chain-of-thought'') reasoning about its future plans. Our agent represents reflection as tool calls that return no value but remain in the agent's prompt at future timesteps.



%\begin{table}[h!]
%\centering
%\small 
%\begin{tabular}{l}
%\toprule
%% \textbf{Prompt Structure} \\
%% \hline
%\texttt{Action Descriptions} \\
%\midrule
%\texttt{Exemplars}  \\
%\midrule
%\texttt{<Current Interaction>} \\
%\verb|# {received-message}| \\
%\verb|>>> {action-1}| \\
%\verb|{observation-1}| \\
%\verb|>>> {reflection-1}| \\
%\verb|>>> {action-2}| \\
%\ldots \\
%\verb|>>> {turn-complete-action}| \\
%\verb|# {received-message}| \\
%\verb|>>> {action-1}| \\
%\ldots \\
%\hline
%\end{tabular}
%\caption{Overview of the prompt structure. Expanded prompts available in Appendix \ref{sec:appendix-approach}.}
%\label{tab:prompt}
%\end{table}


\subsection{Prompt Structure}

The prompt has 3 parts: action descriptions (outlined above); exemplars; and interaction history.

\textbf{Exemplars:} In each domain, we manually annotated four exemplars (See Appendix \ref{sec:exemplars} for a full exemplar) with events, actions, and observations. The exemplars are designed to reflect all relevant phenomena in the domain in question, such as dealing with fragmented information, handling unanswerable questions, and managing redirection. %(where one user suggests contacting another for the relevant information), etc.

\textbf{Interaction History:} An event (receiving a message from an employee) triggers LLM into a loop of action prediction, observation, and reflection, till an end of turn or session is predicted. 
Actions are executed immediately after they are predicted;
%A predicted action is parsed into an allowed Python function and its parameters, and immediately executed.  
events, action, and observation are incrementally appended in the prompt in the order in which they occur (see Appendix~\ref{sec:appendix-approach}).


%(Table \ref{tab:prompt}). %(Event -> Action-1 -> Observation-1 (if not None) -> Reflection-1 -> Action-2 -> ... -> Reflection-n -> Turn-completed). 


\section{Experiments}\label{sec:exp}

\begin{table}[t]
\centering
\caption{\textbf{Quantitative results on OpenCompass~\cite{2023opencompass} multimodal leaderboard.}
$^{\ddag}$ denotes closed-source models. Hall denotes HallusionBench.
}
\label{tab:exp_it_oc}
\setlength{\tabcolsep}{1pt}
\begin{tabular}{l|c|c|cccccccc}
\toprule
Models   & Params & Avg. & MM- & MM- & MM- & Math- & Hall & AI2D  & OCR- & MMVet \\
   &  &  & Bench & Star & MU & Vista &  &  & Bench & \\
\midrule
Step-1o$^{\ddag}$   & N/A   & \textbf{77.7}  & 87.3  & 69.3  & 69.9 & 74.7  & 55.8 & 89.1 & 926 & \textbf{82.8}  \\
SenseNova$^{\ddag}$  & N/A   & 77.4  & 85.7  & \textbf{72.7}  & 69.6 & \textbf{78.4}  & 57.4 & 87.8 & 894 & 78.2  \\
InternVL2.5-78B-MPO~\cite{wang2024mpo}  & 78B  & 77.0   & 87.7  & 72.1  & 68.2  & 76.6  & 58.1  & 89.2 & 909 & 73.5  \\
Qwen2.5-VL-72B~\cite{bai2025qwen25vltechnicalreport}   & 73.4B  & 76.2  & \textbf{87.8}  & 71.1  & 67.9  & 70.8  & 58.8  & 88.2  & 881  & 76.7  \\
TeleMM$^{\ddag}$   & N/A   & 75.9  & 79.9 & 70.8 & 66.6 & 75.7  & \textbf{60.6}  & 88.5 & 891 & 75.7  \\
InternVL2.5-38B-MPO~\cite{wang2024mpo}  & 38B  & 75.3  & 85.4  & 70.1 & 63.8 & 73.6 & 59.7 & 87.9 & 894 & 72.6  \\
InternVL2.5-78B~\cite{chen2024expanding}  & 78B  & 75.2 & 87.5  & 69.5 & 70 & 71.4 & 57.4 & 89.1 & 853 & 71.8   \\
Qwen2-VL-72B~\cite{qwen2-vl_2024}   & 73.4B  & 74.8  & 85.9  & 68.6  & 64.3  & 69.7  & 58.7  & 88.3  & 888  & 73.9  \\
InternVL2.5-38B~\cite{chen2024expanding}  & 38B  & 73.5  & 85.4  & 68.5  & 64.6  & 72.4  & 57.9  & 87.6  & 841  & 67.2  \\
JT-VL-Chat-V3.0$^{\ddag}$  & N/A   & 73.4  & 81.7  & 67.5  & 59.3  & 71.9  & 53.9  & 87.2  & \textbf{967}  & 69.2  \\
Taiyi$^{\ddag}$  & N/A   & 73.0  & 84.8  & 69  & 60.4  & 72.3  & 56.8  & \textbf{90.8}  & 820  & 67.9  \\
Step-1.5V$^{\ddag}$  & N/A   & 72.5 & 82.0  & 65.1  & 61.2  & 69.7  & 54.3  & 87.5  & 886  & 71.3  \\
Gemini-1.5-Pro-002$^{\ddag}$~\cite{geminiteam2024gemini15unlockingmultimodal}   & N/A   & 72.1 & 82.8  & 67.1  & 68.6  & 67.8  & 55.9  & 83.3  & 770  & 74.6  \\
InternVL2.5-26B-MPO~\cite{wang2024mpo}  & 26B  & 72.1  & 84.2  & 67.7  & 56.4  & 71.5  & 52.4  & 86.2  & 905  & 68.1  \\
GPT-4o-20241120$^{\ddag}$~\cite{openai2024gpt4ocard}  & NA   & 72.0   & 84.3  & 65.1  & \textbf{70.7}  & 59.9  & 56.2  & 84.9  & 806  & 74.5  \\
LLaVA-OneVision-72B~\cite{li2024llavaonevision}  & 73B  & 68.0  & 84.5  & 65.8  & 56.6  & 68.4  & 47.9  & 86.2  & 741  & 60.6  \\
NVLM-D-72B~\cite{nvlm2024}   & 79.4B  & 67.6  & 78.5  & 63.7  & 60.8  & 63.9  & 49.7  & 80.1  & 849  & 58.9  \\
Molmo-72B~\cite{deitke2024molmo}  & 73.3B  & 64.1  & 79.5  & 63.3  & 52.8  & 55.8  & 46.6  & 83.4  & 701  & 61.1  \\
\rowcolor{Gray} \textbf{\method-72B}   & 71.8B  & 75.1  & 86.3  & 70.7  & 57.6  & 73.3  & 56.4  & 87.6  & 889   & 79.8  \\
\midrule
\multicolumn{11}{l}{\textit{Models smaller than 20B}} \\
\midrule
Ola-7b~\cite{ola_2025}   & 8.88B   & \textbf{72.6}  & \textbf{84.3}  & \textbf{70.8}  & \textbf{57.0}  & 68.4  & \textbf{53.5}  & \textbf{86.1}  & 822  & \textbf{78.6}  \\
Qwen2.5-VL-7B~\cite{bai2025qwen25vltechnicalreport}   & 8.29B   & 70.4  & 82.6  & 64.1  & 56.2  & 65.8  & 56.3  & 84.1  & 877  & 66.6  \\
InternVL2.5-8B-MPO~\cite{wang2024mpo}   & 8B   & 70.3  & 82  & 65.2  & 54.8  & 67.9  & 51.7  & 84.5  & \textbf{882}  & 68.1  \\
MiniCPM-o-2.6~\cite{yao2024minicpm}   & 8.67B   & 70.2  & 80.6  & 63.3  & 50.9  & \textbf{73.3}  & 51.1  & 86.1  & 889  & 67.2  \\
Ovis1.6-Gemma2-9B~\cite{lu2024ovis}  & 10.2B  & 68.8  & 80.5  & 62.9  & 55.0  & 67.2  & 52.2  & 84.4  & 830  & 65.0  \\
InternVL2.5-8B~\cite{chen2024expanding}   & 8B   & 68.1  & 82.5  & 63.2  & 56.2  & 64.5  & 49.0  & 84.6  & 821  & 62.8  \\
POINTS1.5-Qwen2.5-7B~\cite{points1.5_2024} & 8.3B   & 67.4  & 80.7  & 61.1  & 53.8  & 66.4  & 50.0  & 81.4  & 832  & 62.2  \\
Valley-Eagle$^{\ddag}$   & 8.9B   & 67.4  & 80.7  & 60.9  & \textbf{57.0}  & 64.6  & 48.0  & 82.5  & 842  & 61.3  \\
Qwen2-VL-7B~\cite{qwen2-vl_2024}  & 8B   & 67.0  & 81.0 & 60.7 & 53.7 & 61.4  & 50.4 & 83 & 843 & 61.8 \\
DeepSeek-VL2~\cite{wu2024deepseekvl2}   & 16.1B  & 66.4  & 81.2  & 61.0  & 50.7  & 59.4  & 51.5  & 84.5  & 825  & 60.0  \\
VITA-1.5~\cite{fu2025vita}   & 8.3B   & 63.3  & 76.8  & 60.2  & 52.6  & 66.2  & 44.6  & 79.2  & 741  & 52.7  \\
Baichuan-Omni~\cite{baichuan-omni}   & 7B   & -  & 75.6  & -  & 47.3  & 51.9  & 47.8  & -  & 700  & 65.4  \\
LLaVA-OneVision-7B~\cite{li2024llavaonevision}   & 8B   & 61.2  & 76.8  & 56.7  & 46.8  & 58.5  & 47.5  & 82.8  & 697  & 50.6  \\
Molmo-7B-D~\cite{deitke2024molmo}   & 8B   & 58.9  & 70.9  & 54.4  & 48.7  & 47.3  & 47.7  & 79.6  & 694  & 53.3  \\
% MiniCPM-o 2.6~\cite{yao2024minicpm}   & 8B   & 70.2  & 80.5  & 64.0  & 50.4  & 71.9  & 51.9  & 85.8  & 897  & 67.5  \\
\rowcolor{Gray} \textbf{\method-9B}  & 8.8B   & 69.7  & 80.7  & 60.5  & 51.2  & 68.3  & 51.8  & 84.5  & 883 & 72.3 \\
\bottomrule
\end{tabular}
\end{table}

\begin{table}[t]
  \caption{\textbf{Performance comparison on video and Interleave benchmarks} compared with existing approaches. $^*$ indicates officially released checkpoints evaluated by us. Best performance is marked \textbf{bold}. }
  \label{tab: video_n_interleave}
  \centering
  \setlength{\tabcolsep}{7.5pt}
  \begin{tabular}{lccccc}
    \toprule
       & \multicolumn{2}{c}{\textbf{VideoMME}} & \multicolumn{1}{c}{\textbf{MVBench}} & \multicolumn{2}{c}{\textbf{Llava-Interleave}}\\
    \cmidrule(r){2-3} \cmidrule(r){4-4} \cmidrule(r){5-6}
    Model & w/o subs & w subs & avg & in-domain & out-domain \\
    \midrule
     MiniCPM-V-2.6~\cite{yao2024minicpm} &  60.9 &  63.6 &  - &  - &  - \\
     LLaVA-OneVision-7B~\cite{li2024llavaonevision} &  58.2 &  - &  - &  - &  - \\
     Qwen2-VL-7B~\cite{qwen2-vl_2024} &  63.3 &  69.0 &  67.0 &  49.5$^*$ &  51.0$^*$ \\
     InternVL2-8B~\cite{chen2024far} &  56.3 & 59.3 &  65.8 &  - &  - \\
     VITA-1.5~\cite{fu2025vita} &  56.1 & 58.7 &  55.4 &  - &  - \\
     Baichuan-Omni~\cite{baichuan-omni} &  58.2 & - &  60.9 &  - &  - \\
     MiniCPM-o-2.6~\cite{yao2024minicpm} & 63.0$^*$ & 65.3$^*$ & 58.1$^*$ &  43.5$^*$ &  36.8$^*$ \\
     \rowcolor{Gray} \textbf{\method-9B} &  60.4  & 65.0 &  66.3 &  59.8 &  87.8 \\
    \midrule
    VideoLLaMA2-72B~\cite{cheng2024videollama2} & 61.4 & 63.1 & 62.0 & - & - \\
    LLaVA-OneVision-72B~\cite{li2024llavaonevision} &  66.2 &  69.5 &  59.4 &  - &  - \\
    Qwen2-VL-72B~\cite{qwen2-vl_2024} &  71.2 &  77.8 &  \textbf{73.6} &  - &  - \\
    InternVL2-Llama3-76B~\cite{chen2024far} &  64.7  & 67.8 &  69.6 &  - &  - \\
    \rowcolor{Gray} \textbf{\method-72B} &  65.2  & 67.7 &  69.6 &  \textbf{63.5} &  \textbf{89.9} \\
    \midrule
    GPT-4v~\cite{GPT4VisionSystemCard} & 59.9 & 63.3 & 43.7 & 39.2 & 57.78 \\
    GPT-4o-20240513~\cite{openai2024gpt4ocard} & 71.9 & 77.2 & - & - & - \\
    Gemini-1.5-Pro~\cite{geminiteam2024gemini15unlockingmultimodal} & \textbf{75.0} & \textbf{81.3} & - & - & - \\
    \bottomrule
\end{tabular}
\end{table}


In this section, we present a comprehensive evaluation of our \method model, comprising both quantitative and qualitative analyses of its performance. Furthermore, we conduct ablation studies to analyze the contributions of several key design components to the performance of our \method model, providing insights into their distinct impacts.

% In this section, we first evaluate the model’s performance on a variety of mainstream benchmarks, demonstrating the advantages of \method.
% Then, a series of qualitative results are presented to show the model’s specific capabilities, including multimodal understanding and free-form image generation.
% Finally, we conduct an ablation study to analyze several key components in \method.

\subsection{Quantitative Results}\label{subsec:exp_quantitative_results}

\subsubsection{Image-Text Understanding}
To evaluate the effectiveness of our \method in image-text understanding, we benchmark it against state-of-the-art MLLMs on the OpenCompass~\cite{2023opencompass} multimodal leaderboard, a widely recognized platform for multimodal evaluation. This leaderboard contains 8 different multimodal benchmarks, including complex VQA (MMBench~\cite{liu2025mmbench}, MMStar~\cite{chen2024we}, MMMU~\cite{yue2023mmmu}, AI2D~\cite{kembhavi2016diagram}, and MMVet~\cite{yu2024mm}), multimodal reasoning (MathVista~\cite{lu2024mathvista}), hallucination evaluation (Hallusionbench~\cite{Guan_2024_hallusionbench}), and OCR (OCRBench~\cite{Liu_2024}).
\cref{tab:exp_it_oc} shows the overall results. Our \method-72B model achieves top-tier performance on most benchmarks, surpassing closed-source models like GPT-4o and Gemini-1.5-Pro. Furthermore, our \method-9B model exhibits competitive performance among models of similar size, showcasing its robust capabilities in image-text understanding tasks.

% In this section, we compare our \method with leading MLLMs on the mainstream OpenCompass~\cite{2023opencompass} multimodal leaderboard to demonstrate its advancement on image-text understanding.
% This leaderboard contains 8 different multimodal benchmarks, including complex VQA (MMBench~\cite{liu2025mmbench}, MMStar~\cite{chen2024we}, MMMU~\cite{yue2023mmmu}, AI2D~\cite{kembhavi2016diagram}, and MMVet~\cite{yu2024mm}), multimodal reasoning (MathVista~\cite{lu2024mathvista}), hallucination evaluation (Hallusionbench~\cite{Guan_2024_hallusionbench}), and OCR (OCRBench~\cite{Liu_2024}).
% \cref{tab:exp_it_oc} shows the overall results.
% \method exhibits competitive performance compared with other MLLMs.
% Our \method-71B model achieves top-tier performance on most benchmarks.
% It outperforms closed-source models such as GPT-4o and Gemini-1.5-Pro.
% The \method-9B model also achieves competitive performance among other vision-language specific MLLM models smaller than 20B.
% Notably, it achieves excellent performance on MathVista, AI2D, and MMVet, demonstrating its comprehensive ability on multimodal reasoning and complex VQA.




\subsubsection{Video \& Interleaved Image-Text Understanding}

We evaluate our model's video and interleaved image-text understanding abilities on three mainstream benchmarks.

\textbf{Video-MME}~\cite{fu2024video}: Video-MME is a benchmark designed to evaluate MLLMs in full-spectrum video analysis. It encompasses a wide variety of video types across multiple domains and durations, featuring multimodal inputs such as video, subtitles, and audio. For this benchmark, testing is conducted with under 96 frames, and results are reported for both "with subtitles" and "without subtitles" settings.

\textbf{MVBench}~\cite{li2024mvbench}: MVBench serves as a video understanding benchmark aimed at thoroughly evaluating the temporal awareness of MLLMs in an open-world context. It includes 20 challenging video tasks that range from perception to cognition, which cannot be adequately addressed using a single frame. Testing for this benchmark utilizes dynamic sampling frames.

\textbf{LLaVA-Interleave Bench}~\cite{llava-next_2024}: LLaVA-Interleave Bench comprises a comprehensive suite of multi-image benchmarks collected from public datasets or generated via the GPT-4V API. It is created to assess the interleaved multi-image reasoning capabilities of MLLMs, with reported results for both "in-domain" and "out-domain" subsets.

As shown in Table~\ref{tab: video_n_interleave}, \method-9B achieves the second-best results across VideoMME and MVBench (outperformed only by Qwen2-VL-7B but requiring significantly fewer frames). However, the performance gains do not scale up to \method-72B due to limitations in the quantity of instruction-tuned video data. Moreover, both our \method-9B and \method-72B greatly surpass all other baselines in multi-image benchmarks, both in-domain and out-of-domain, highlighting their potential as strong competitors for complex tasks.


\subsubsection{Audio Understanding}

We evaluate our M2-omni model's audio understanding abilities on four mainstream benchmarks.

\textbf{Multilingual LibriSpeech (MLS)}~\cite{MLS_English}: The Multilingual LibriSpeech dataset is an extensive collection of read audiobooks sourced from Librivox, available in eight different languages. We utilize the English test set from this dataset to assess the model's speech comprehension capabilities. The latest version of this corpus comprises approximately 50,000 hours.

\textbf{Librispeech}~\cite{Librispeech}: The Librispeech corpus comprises approximately 1,000 hours of transcribed speech audio data derived from read English audiobooks. The entire dataset is categorized into three training sets (100 hours of clean, 360 hours of clean, and 500 hours of other), two validation sets (clean and other), and two test sets (clean and other). In this study, we assess our model's audio comprehension capabilities using both the clean and other testsets.

\textbf{Aishell1}~\cite{AISHELL1}:  The Aishell1 dataset comprises 178 hours of speech data, recorded by 400 speakers from various accent regions across China. It is organized into three subsets: a training set consisting of 340 speakers, a validation set with 40 speakers, and a test set featuring 20 speakers.

\textbf{AudioCaps}~\cite{AudioCaps}: AudioCaps is a comprehensive dataset featuring audio event descriptions specifically curated for the purpose of audio captioning. The sounds within this collection are derived from the AudioSet dataset. We utilize this dataset to assess the audio captioning capabilities of our \method.
 % To facilitate accurate captioning, annotators were supplied with audio tracks and corresponding categorical hints, with additional video hints provided as necessary.

The results are presented in Table~\ref{tab:exp_audio_understand}, and our \method-9B demonstrates competitive performance in speech recognition and audio captioning tasks. 
Specifically, our \method-9B is comparable to GPT-4o-Realtime~\cite{openai2024gpt4ocard}.
In addition, \method-9B significantly outperforms all other baselines on AudioCaps benchmarks, while achieving the second-best results for the MLS English, Librispeech other, Librispeech-clean and Aishell1 benchmarks.

\begin{table}[]
\centering
\caption{\textbf{Quantitative results on speech recognition and audio captioning.}
 $^*$ indicates results from \cite{yao2024minicpm}.
}
\label{tab:exp_audio_understand}
\setlength{\tabcolsep}{7pt}
\begin{tabular}{l|cccccccccc}
\toprule
Models   & MLS- & Librispeech- & Librispeech- & Aishell1 & AudioCaps \\
                & English & other & clean &  & \\
                & WER$\downarrow$ & WER$\downarrow$ & WER$\downarrow$ & WER$\downarrow$ & CIDER$\uparrow$ \\
\midrule
UIO2-L-1.1B~\cite{lu2023uio2}   & - & - & - & - & 45.7   \\
UIO2-XL-3.2B~\cite{lu2023uio2}  & - & - & - & - & 45.7   \\
UIO2-XXL-6.8B~\cite{lu2023uio2} & - & - & - & - & 48.9  \\
Whisper-large-v2~\cite{Whisper}  & \textbf{6.83} & \textbf{5.16} & 2.87 & - & - \\
Paraformer-cn~\cite{gao2022paraformer} & - & - & - & 2.12 & - \\
VITA-1.5~\cite{VITA_1.5} & - & 7.5 & 3.4 & 2.2 & - \\
Mini-Omini2~\cite{mini_omni2} & - & 9.8 & 4.8 & - & - \\
Freeze-Omini~\cite{Freeze_Omni} & - & 10.5 & 4.1 & 2.8 & - \\
MiniCPM-o-2.6~\cite{yao2024minicpm} & - & - & \textbf{1.7} & \textbf{1.6} & - \\
GPT-4o-Realtime~\cite{openai2024gpt4ocard} & - & - & 2.6$^*$ & 7.3$^*$ & - \\
\rowcolor{Gray} \textbf{\method-9B}   & 7.19 & 5.29 & 2.07 & 1.99 & \textbf{49.2} \\
\bottomrule
\end{tabular}
\end{table}

\begin{table}[t]
\centering
\caption{\textbf{Quantitative results on language benchmarks.} $^*$ indicates officially released checkpoints evaluated using the tools provided by OpenCompass~\cite{2023opencompass}.
}
\label{tab:exp_language}
\setlength{\tabcolsep}{5pt}
\begin{tabular}{cccccccc}
\hline
Tasks & MMLU & AGIEVAL & ARC-C & GPQA & MATH & HellaSwag & \begin{tabular}[c]{@{}l@{}}Avg.\\ Accuracy\end{tabular} \\ \hline
LLama3.1-8B & 69.4 & 41.2$^*$ & 83.4 & 30.4 & 51.9 & 75.1$^*$ & 58.6  \\
\rowcolor{Gray} \textbf{\method-9B} & 68.5 & 43.7 & 78.7 & 32.3 & 51.8 & 80.1 & 59.2  \\ \hline
\end{tabular}
\end{table}

\subsubsection{Audio Generation}
In this section, we also evaluated our model on the commonly-used test set: SEED-TTS test-zh. \textbf{SEED-TTS}~\cite{SEED_TTS} serves as an out-of-domain evaluation test set, comprising diverse input texts and reference speeches from various domains. We present the experimental results for \method-9B and the baseline models in Table~\ref{tab:exp_audio_generation}. As shown in Table~\ref{tab:exp_audio_generation}, our model outperforms MiniCPM-o-2.6~\cite{yao2024minicpm} in speech generation capability, achieving significant improvements in both evaluation metrics. However, our \method-9B still lags behind traditional vertical speech generation models, highlighting the need for further research and development to bridge this gap.


\subsubsection{Text-only Performance}
In this section, we assess the performance of our proposed \method-9B model and its initial counterpart, Llama3.1-8B~\cite{llama3_2024}. To evaluate the models' knowledge and examination capabilities, we employ a range of benchmarks, including AGIEVAL~\cite{zhong2023agievalhumancentricbenchmarkevaluating} and MMLU~\cite{hendrycks2021measuringmassivemultitasklanguage}. Furthermore, we utilize a diverse set of benchmarks to evaluate the models' multi-step problem-solving capabilities, including MATH~\cite{hendrycks2021measuringmathematicalproblemsolving} for mathematical derivation, HellaSwag~\cite{zellers2019hellaswagmachinereallyfinish} for commonsense reasoning in real-world contexts, ARC-C~\cite{allenai:arc} for scientific logical chains, and GPQA~\cite{rein2023gpqagraduatelevelgoogleproofqa} for critical analysis in expert-level domains. For all evaluation datasets, we adopt a generation-based assessment approach with greedy decoding.

Our experimental results, presented in \cref{tab:exp_language}, demonstrate that the performance of our proposed \method-9B model outperforms its initial counterpart, Llama3.1-8B across most evaluation datasets,   which is attributed to our multi-stage language preservation strategy and the high-quality instruction tuning data used in our training process.

% In this section, we evaluate the performance of our \method-9B and its initial Llama3.1~\cite{llama3_2024} models. To assess the models' knowledge and examination capabilities, we utilize the AGIEVAL~\cite{zhong2023agievalhumancentricbenchmarkevaluating},  MMLU~\cite{hendrycks2021measuringmassivemultitasklanguage} benchmarks. Additionally, we employ  MATH~\cite{hendrycks2021measuringmathematicalproblemsolving}, HellaSwag~\cite{zellers2019hellaswagmachinereallyfinish}, ARC-C~\cite{allenai:arc} and GPQA~\cite{rein2023gpqagraduatelevelgoogleproofqa} to evaluate the models' multi-step problem-solving ability, including mathematical derivation, commonsense reasoning in real-world contexts, scientific logical chains, and critical analysis in expert-level domains. For all evaluation datasets, we adopt a generation-based assessment approach with greedy decoding. The overall results are in \cref{tab:exp_language}.It can be observed that in most of the evaluation datasets, the performance of our \method-9B and Llama3.1~\cite{llama3_2024} models is comparable, maintaining their linguistic capabilities. Furthermore, in some rankings, our models exhibit superior performance in certain aspects compared to their text-only baseline models. This improvement is attributed to our multi-stage language preservation strategy and the high-quality instruction tuning data used in our training process.

\begin{table}[t]
\centering
\caption{
\textbf{Free-form dialogue generation evaluation results.}
}
% \vspace{3pt}
\setlength{\tabcolsep}{8pt}
\begin{tabular}{c|c|c|c}
\toprule
Model & Relevance & Fluency & Informativeness\\
\midrule
TextBind~\cite{li2023textbind} & 3.85 & 4.30 & 3.25\\
\rowcolor{Gray} \textbf{\method-9B} & 4.60 & 4.80 & 3.80\\
\bottomrule
\end{tabular}
\label{tab-model_freeform_results}
% \vspace{-12pt}
\end{table}




% For a evaluation of open-world multi-turn multimodal instruction following, we collect a test set comprising 50 conversations from realistic scenarios and utilize \method-9B to generate arbitrarily interleaved text and images in proper conversation contexts. For quantitative results, we ask GPT-4o~\cite{openai2024gpt4ocard} to rate each conversation ranging from 0 to 5 considering relevance, fluency and informativeness. We carry out our quantitative results against recent work TextBind~\cite{li2023textbind}. As shown in \cref{tab-model_freeform_results}, \method-9B exhibits overall better understanding and generating ability of multi-turn multimodal conversations. More qualitative cases can be found in \cref{fig-IT-Freeform-Result}.



\begin{table}[t]
  \caption{\textbf{Quantitative results on audio generation.} $^*$ indicates officially released checkpoints evaluated by us.}
  \label{tab:exp_audio_generation}
  \centering
  \setlength{\tabcolsep}{14pt}
  \begin{tabular}{lccccc}
    \toprule
       & \multicolumn{2}{c}{\textbf{SEED test-zh}}\\
    \cmidrule(r){2-3}
    Model & CER(\%)$\downarrow$ & SS$\uparrow$  \\
    \midrule

     Human & 1.26 &0.755 \\
     Vocoder Resyn. & 1.27 & 0.720 \\
     \midrule
     Seed-TTS~\cite{SEED_TTS} & 1.12 & 0.796 \\
     FireRedTTS~\cite{FireRedTTS} & 1.51 &0.635 \\
     MaskGCT~\cite{MaskGCT} & 2.27 & 0.774 \\
     E2-TTS(32 NFE)~\cite{E2_TTS} & 1.97 & 0.730 \\
     F5-TTS(32 NFE)~\cite{F5_TTS} & 1.56 & 0.741 \\
     CosyVoice~\cite{CosyVoice} &3.63 &0.723 \\
     CosyVoice2~\cite{CosyVoice2} &1.45 &0.748 \\
     CosyVoice2-S~\cite{CosyVoice2} &1.45 &0.753 \\
     CosyVoice2-S~\cite{CosyVoice2} &1.45 &0.753 \\
     \midrule
     MiniCPM-o-2.6~\cite{yao2024minicpm} &8.03$^*$ &0.474$^*$ \\
     \rowcolor{Gray} \textbf{\method-9B} &  6.36  & 0.604 \\
    \bottomrule
\end{tabular}
\end{table}


\subsubsection{User Experience Evaluation}\label{sec:human_evaluation}
\textbf{Evaluation Metric}:
Current benchmarks such as MMBench~\cite{liu2025mmbench}, MMStar~\cite{chen2024we}, and MMMU~\cite{yue2023mmmu} primarily focus on assessment through judgment-style questions. However, this assessment does not align with the users' actual interactive experience with MLLMs. To address this limitation, drawing inspiration from SuperclueV~\cite{supercluev}, we develop evaluation criteria specifically for assessing the models' performance on user experience, which contains four key dimensions: relevance, fluency, informativeness, and format rationality. \textit{Relevance} assesses the extent to which the model's responses align with both the provided prompts and the multimodal inputs.
\textit{Fluency} evaluates the naturalness, smoothness, clarity, comprehensibility, and anthropomorphic quality of the model's responses.
\textit{Informativeness} measures the extent to which the model's responses provide relevant information, knowledge, and analytical reasoning, enhancing their utility, detail, depth, and innovation.
\textit{Format rationality} examines the model's ability to adaptively generate appropriately structured and clear formats, for presenting results based on varying prompt types.



% Current benchmarks such as MMBench~\cite{liu2025mmbench}, MMStar~\cite{chen2024we}, and MMMU~\cite{yue2023mmmu} primarily focus on assessment through judgment-style questions. However, this assessment does not align with the users' actual interactive experience with MLLMs. Drawing inspiration from SuperclueV~\cite{supercluev}, we develop evaluation criteria specifically for assessing the models' experience performance, which contains four key dimensions: relevance, fluency, content richness, and format rationality. \textbf{Relevance} assesses the extent to which the model's responses align with both the provided prompts and the multi-modal inputs.
% \textbf{Fluency} evaluates the naturalness, smoothness, clarity, comprehensibility, and anthropomorphic quality of the model's responses.
% \textbf{Content richness} gauges the degree to which the model's responses are enriched with supplementary information, knowledge, and analytical reasoning, enhancing their utility, detail, depth, and innovation.
% \textbf{Format rationality} examines the model's ability to adaptively generate appropriately structured and clear formats for presenting results based on varying prompt types.


\begin{table}[t]
\centering
\caption{
\textbf{Detailed model experience evaluation standards.}
}
% \vspace{3pt}
\setlength{\tabcolsep}{4pt}
\begin{tabular}{c|c}
\toprule
Score & Description\\
\midrule
1 & Totally unsatisfied, totally unacceptable \\
2 & Basically not satisfied, with many obvious problems \\
3 & Generally satisfied, with a few obvious problems \\
4 & Basically satisfied, minor flaws allowed \\
5 & Completely satisfied, almost perfect \\
\bottomrule
\end{tabular}
\label{tab-model_expr_standards}
% \vspace{-12pt}
\end{table}

\textbf{Evaluation Dataset}: We collect chat samples from the actual users' multi-turn interaction dialogues, which cover a variety of tasks, including visual question answering (VQA), conversational interactions, chart interpretation, mathematical problem-solving, optical character recognition (OCR), and other related tasks. GPT-4o~\cite{openai2024gpt4ocard} is instructed to follow the evaluation criteria to generate initial reference answers for these collected samples. To ensure accuracy, human annotators refine the initial responses generated by GPT-4o. This process yields an evaluation dataset with nearly 300 samples, each with a corresponding ground truth.

We utilize GPT-4o to evaluate the model's responses against the ground truth, adhering to the standards outlined in  \cref{tab-model_expr_standards}.  As shown in \cref{tab-user_experience},  our M2-omni model, after undergoing  alignment tuning,  demonstrates an average increase of 5.7\%-23.4\% in user experience performance, which is further validated by human annotations on selected cases. Meanwhile, our model's performance on the OC benchmark across other modalities remains relatively consistent, thereby demonstrating the effectiveness of our unified training strategy, which integrates DPO and instruction tuning in the alignment tuning stage.

% We employ GPT-4o to score the models' responses compared with ground truth according to the standards of \cref{tab-model_expr_standards}.  \cref{tab-user_experience} shows the model after alignment tuning demonstrates an average increase of 5.7\% in performance. This enhancement is corroborated by human annotations on selected cases. Simultaneously, the general capabilities on OC benchmark across other modalities remain nearly the same, with a decrease in average evaluation scores of less than 1\%. This demostrates the effectiveness of our unified training strategy that integrates DPO and
% instruction tuning in alignment tuning stage.


\subsubsection{Free-Form Dialogue Generation}
To evaluate the open-world multi-turn multimodal instruction following capabilities of our model, we create a test set consisting of 50 conversations derived from realistic scenarios. We utilize \method-9B to generate arbitrarily interleaved text and images in proper conversation contexts.
For quantitative results, following our user experience evaluation metric, we employ GPT-4o to rate each conversation on a scale of 0 to 5 across three evaluation dimensions: relevance, fluency, and informativeness.
We carry out our quantitative results against recent work TextBind~\cite{li2023textbind}. As shown in \cref{tab-model_freeform_results}, \method-9B exhibits overall better understanding and generating ability of multi-turn multimodal conversations. More qualitative cases can be found in \cref{fig-IT-Freeform-Result}.





\begin{table}[t]
\centering\footnotesize
\caption{
\textbf{Detailed evaluation on user experience benchmark and OC benchmark. OC is short for the OpenCompass image-text understanding benchmark.}
}
% \vspace{3pt}
\setlength{\tabcolsep}{3pt}
\begin{tabular}{c|c|c|c|c|c|c}
\toprule
Model & Relevance & Fluency & Informativeness & Format Rationality & Expr. Avg($\Delta$\%) & OC Avg($\Delta$)\\
\midrule
\method-9B & 4.556 & 4.036 & 2.742 & 3.573 & 3.726 & -\\
\rowcolor{Gray} \method-9B-Align & 4.893 & 4.735 & 4.118 & 4.644 & 4.598(+23.4\%) & -0.3\\
\method-72B & 4.942 & 4.689 & 3.267 & 4.265 & 4.351 & -\\
\rowcolor{Gray} \method-72B-Align & 4.946 & 4.875 & 3.961 & 4.615 & 4.598(+5.7\%) & -0.2\\
InternVL2-26B~\cite{internvl_2024} & 4.886 & 4.76 & 4.15 & 4.52 & 4.577 & -\\
GPT-4o~\cite{openai2024gpt4ocard} & 5 & 4.878 & 3.854 & 4.831 & 4.64 & -\\
\bottomrule
\end{tabular}
\label{tab-user_experience}
% \vspace{-12pt}
\end{table}



\subsection{Qualitative Results}\label{subsec:exp_qualitative_results}

In this section, we qualitatively assess the capabilities of our \method, presenting examples of each modality and different tasks.

We show multimodal understanding abilities of our \method in \cref{fig-exp_case_all}. \method demonstrates promising capabilities in processing cross-modal problems, encompassing image understanding, video understanding, interleaved image-text understanding, and image-audio understanding. More examples can be found in the appendix, provided in \cref{subsec:appendix_cases}.

\cref{fig-IT-Freeform-Result} illustrates the model's ability to generate free-form dialogue, where our \method can create images based on the conversation context without explicit user input, useful for explaining ideas to users.




\begin{figure}[t]
    \centering
    \includegraphics[width=0.9\linewidth]{figures/case_exp.pdf}
    \caption{
    \textbf{Cases for multimodal understanding.}
    \method shows great potential to solve various multimodal problems.
    }
    \label{fig-exp_case_all}
\end{figure}




\begin{figure}[t]
    \centering
    \includegraphics[width=0.9\linewidth]{figures/free_form_gen.pdf}
    \caption{
    \textbf{Cases for Free-Form Dialogue Generation.}
    }
    \label{fig-IT-Freeform-Result}
\end{figure}


\subsection{Ablation Study}\label{subsec:exp_ablation}

\begin{table}[t]
\centering
\caption{\textbf{Ablation studies on step balancing strategy.} The loss weight setting [1,1,1] corresponds to the uniform weighting of the loss functions for image-text pairs, interleaved image-text, and video datasets.  * and \# represent the loss weight settings. * is obtained through experimental trials and parameter tuning. \# is obtained by normalizing the loss weights using the inverse of the loss at convergence, as described in Section \cref{subsubsec-Step Balancing Strategy}. We evaluate the few-shot performance on VQA tasks and the zero-shot performance on the captioning task of our pre-trained model.}
\label{tab:ablation_step_balance_pretrain}
\setlength{\tabcolsep}{4pt}
\begin{tabular}{c|c|ccc}
\toprule
\multicolumn{1}{l|}{Data Sample Balance} & Loss Weight Balance & \multicolumn{1}{l}{OK-VQA(4-shot)} & \multicolumn{1}{l}{VQAv2(4-shot)} & \multicolumn{1}{l}{Flickr30k(0-shot)} \\ \hline
Random Sample                        & {[}1,1,1{]}          & 40.5                             & 54.3                             & 87.0                                 \\
Round-robin                          & {[}1,1,1{]}          & 41.6                             & 54.4                             & 88.1                                 \\
Accumulation                         & {[}1,1,1{]}          & 41.7                             & 54.6                             & 88.2                                 \\
Accumulation                         & ${[}0.2,1.0,0.03{]}^{*}$   & 39.7                             & 52.5                             & 87.1                                 \\
Accumulation                         & ${[}0.45,0.36,1.09{]}^{\#}$ & \textbf{42.1}                             & \textbf{55.4}                             & \textbf{88.2}                                 \\
\bottomrule
\end{tabular}
\end{table}


In this section, we conduct ablation studies to investigate the effectiveness of our step balance strategy and dynamic adaptive balance strategy in our M2-omni model. These experiments aim to provide insights into the impact of these key components on our M2-omni’s performance.

\subsubsection{Step Balance Strategy}\label{subsubsec:step_balance_ablaton}

As described in \cref{subsubsec-Step Balancing Strategy} ,  we investigate the impact of various data sample balancing strategies and loss weight balancing schemes on the multimodal joint training stage of pre-training. We evaluate the performance of candidate strategies on two VQA benchmarks, OK-VQA~\cite{marino2019ok} and VQAv2~\cite{goyal2017making}, and assess its image captioning performance using the Flickr30k~\cite{young2014image} benchmark.

For pretrained models lacking in instruction following ability, to assess the effectiveness of our approach, we evaluate the performance of these models on VQA tasks using a few-shot approach and on image caption tasks using a zero-shot approach. \cref{tab:ablation_step_balance_pretrain} presents the results of our M2-omni pretrained models, which demonstrate the effectiveness of our step balance strategy.

% Besides, three task weighting manner are compared: [1,1,1], which means all data shares the same optimization step size; [0.2,1.0,0.03], which is consistent with that proposed in \cite{alayrac2022flamingo}; [0.45,0.36,1.09], the inverse of the loss at convergence state, as \cref{subsubsec-Step Balancing Strategy} described. Note that the three values in the ratio correspond to image-text pairs, interleaved image-text and video datasets.

%  We directly evaluate the pre-trained model's performance on VQA tasks using a few-shot approach and on image caption tasks using a zero-shot approach. For VQA tasks, we use two benchmarks: OK-VQA~\cite{marino2019ok} and VQAv2~\cite{goyal2017making}, while for image captioning, we use the Flickr30k~\cite{young2014image} benchmark. \cref{tab:ablation_step_balance_pretrain} shows the results of training models on the combined datasets using three different merging regimes. It can be observed that the accumulation strategies and setting the task weights to the inverse of the loss achieve the best performance.







\begin{table}[t]
\centering
\caption{
\textbf{Ablation results of the dynamic adaptive balance strategy}. Results for unimodal baselines are derived from the following single-modal models: \textsuperscript{$\dagger$} Image-Text Model, \textsuperscript{$\ddagger$} Video-Text Model, and \textsuperscript{
$\mathsection$} Audio-Text Model. The best result for each benchmark is \textbf{bolded}, while the best result for each model across all epochs is \underline{underlined}.
}
% \vspace{3pt}
\setlength{\tabcolsep}{3pt}
\begin{tabular}{l|l|ccccc|cc|cc}
\toprule
Models &  & MM- & OK- & VQAv2 & Text- & GQA & MSVD- & MSRVTT & Audio & MLS- \\
& & Bench & VQA &&VQA&& QA & QA & Caps & English($\downarrow$) \\
\midrule
\multirow{3}{*}{\makecell[l]{Single-modal\\Baselines}} & ep1 & 68.0\textsuperscript{$\dagger$} & 56.4\textsuperscript{$\dagger$} & 74.8\textsuperscript{$\dagger$} & \underline{70.4}\textsuperscript{$\dagger$} & 58.4\textsuperscript{$\dagger$} & 72.3\textsuperscript{$\ddagger$} & 59.3\textsuperscript{$\ddagger$} & 29.0\textsuperscript{$\mathsection$} & 11.4\textsuperscript{$\mathsection$} \\
& ep2 & \underline{\textbf{77.8}}\textsuperscript{$\dagger$} & \underline{59.8}\textsuperscript{$\dagger$} & \underline{76.9}\textsuperscript{$\dagger$} & 69.8\textsuperscript{$\dagger$} & 60.6\textsuperscript{$\dagger$} & \underline{\textbf{76.5}}\textsuperscript{$\ddagger$} & \underline{60.1}\textsuperscript{$\ddagger$} & \underline{39.9}\textsuperscript{$\mathsection$} & 9.33\textsuperscript{$\mathsection$} \\
& ep3 & 77.3\textsuperscript{$\dagger$} & 58.0\textsuperscript{$\dagger$} & 76.8\textsuperscript{$\dagger$} & 69.1\textsuperscript{$\dagger$} & \underline{60.8}\textsuperscript{$\dagger$} & 74.4\textsuperscript{$\ddagger$} & 58.6\textsuperscript{$\ddagger$} & 39.5\textsuperscript{$\mathsection$} & \underline{8.96}\textsuperscript{$\mathsection$} \\
\midrule
\multirow{3}{*}{\makecell[l]{Mixture \\w/o MM-Bal.}}
& ep1 & 70.5 & 55.9 & 75.7 & 70.2 & 57.7 & \underline{75.1} & \underline{59.6} & 27.5 & 12.1 \\
& ep2 & \underline{75.8} & \underline{58.8} & \underline{77.0} & \underline{70.5} & \underline{\textbf{61.1}} & 73.4 & 58.5 & 33.5 & 9.45 \\
& ep3 & 75.6 & 58.4 & 76.5 & 69.5 & 60.1 & 70.2 & 56.9 & \underline{39.6} & \underline{8.98} \\
\midrule
\multirow{3}{*}{\makecell[l]{Mixture \\w/ MM-Bal.}}
& ep1 & 74.7 & 59.6 & 76.0 & 71.2 & 59.0 & 73.1 & 58.7 & 35.5 & 9.27 \\
& ep2 & \underline{\textbf{77.8}} & \underline{\textbf{61.7}} & \underline{\textbf{77.2}} & \underline{\textbf{71.8}} & 60.5 & \underline{74.8} & 58.5 & 41.2 & 8.31 \\
& ep3 & 77.1 & 60.5 & 77.0 & 69.8 & \underline{60.7} & 74.6 & \underline{\textbf{60.2}} & \underline{\textbf{44.1}} & \underline{\textbf{8.04}} \\

\bottomrule
\end{tabular}
\label{tab-multi_task_balanced_ablation}
% \vspace{-12pt}
\end{table}

\subsubsection{Dynamic Adaptive Balance Strategy}

We conducted a evaluation of our dynamic adaptive balance strategy across text-image, video, and audio modalities using constrained datasets. The evaluation was conducted on benchmark datasets specific to each modality: for text-image tasks, MMbench~\cite{liu2025mmbench}, OK-VQA~\cite{marino2019ok}, VQAv2~\cite{goyal2017making}, TextVQA~\cite{singh2019towards}, and GQA~\cite{hudson2019gqa} were employed; for video, MSVD-QA~\cite{xu2017video} and MSRVTT-QA~\cite{xu2017video} benchmarks were utilized; and for audio, we assessed performance on the AudioCaps~\cite{kim2019audiocaps} (AAC) and MLS~\cite{Pratap2020MLSAL}-English (ASR) tasks. The experimental outcomes are detailed in Table~\ref{tab-multi_task_balanced_ablation}.

In contrast to actual training pipeline, our evaluation involved instruction tuning starting from pre-trained models. Specifically, for each modality, we initially trained single-modality baseline models (the 'Sinle-modal Baselines' in Table~\ref{tab-multi_task_balanced_ablation}) individually over three epochs to establish the maximum achievable performance per modality. The results indicate that optimal performance was predominantly observed by the second epoch. However, the ASR task, due to its more complex patterns, had not fully converged even by the third epoch. Subsequently, we combined data from all three modalities to train a unified model (the 'Mixture w/o MM-Bal.' in Table~\ref{tab-multi_task_balanced_ablation}). Under this multimodal training regimen, the image-text modality reached its optimal performance at the second epoch, while the video modality achieved peak performance as early as the first epoch and with performance consistently decreasing in subsequent epochs. In contrast, the audio modality demonstrated continuous improvement, attaining its best performance by the third epoch. These observations underscore the imbalance in training progress among different modalities when engaged in multimodal training.

To address this imbalance, we introduced the dynamic adaptive balance strategy within our M2-omni training framework. This strategy dynamically adjusts the loss weights for each modality based on their respective training progress. In the context of this evaluation, it accelerates the training of the audio modality while appropriately reducing the learning weights for the image-text and video modalities to prevent overfitting. The evaluation results for this balanced training approach are denoted as 'Mixture w/ MM-Bal.' in Table~\ref{tab-multi_task_balanced_ablation}. The results demonstrate that, although some degree of imbalance among modalities persists, the balanced training strategy significantly alleviates the issues observed with simple mixed training: optimal performances across benchmarks are now concentrated around the second and third epochs, and performance across all modalities has been markedly enhanced. Moreover, under the balanced training strategy, the model achieved single-modality optimal performance in 7 out of 9 benchmarks. The best-performing model (at epoch 2) surpassed the optimal performance of each single-modality baseline in 6 out of 9 benchmarks (MMBench, OK-VQA, VQAv2, TextVQA, AudioCaps, MLS-English). Additionally, for the audio modality, the model at epoch 3 outperformed the single-modality baselines in 5 out of 9 benchmarks (OK-VQA, VQAv2, MSRVTT-QA, AudioCaps, MLS-English), with significant improvements in audio performance. These experimental results highlight the effectiveness of our dynamic adaptive balance strategy.

\begin{figure*}[h]
    \centering
    \includegraphics[width=14cm]{figures/visualized_kitti5.jpg}
    \caption[Qualitative Results on KITTI \textit{val.} set]{\textbf{Qualitative results on the KITTI \textit{val} set for the car class.} The proposed method (green) and ground truth (red).
    } \label{fig:KITTI visualized}
\end{figure*}

\begin{figure*}[t]
    \centering
     \includegraphics[width=14cm]{figures/custom_result_monodetr_monoground3.jpg}
    \caption{\textbf{Qualitative results on the custom dataset.} Comparison of detection results between the proposed model (blue), the state-of-the-art models (green), and ground truth (red) in ego-view (left) and bird's-eye view (right); MonoDETR (left) and MonoGround (right).}
    \label{fig:custom_result_visualized}
\end{figure*}

\section{Conclusion}
\label{sec:conclusion}
This paper presents a novel approach to monocular 3D object detection by integrating a Vision Foundation Model as the backbone with the DETR architecture, enabling enhanced depth estimation and feature extraction within a single-stage, real-time framework. By incorporating a Hierarchical Feature Fusion Block for multi-scale detection and 6D Dynamic Anchor Boxes for iterative bounding box refinement, the proposed model achieves improved performance without relying on additional data sources, such as LiDAR. Future work will focus on extending the model's capabilities to detect 3D bounding boxes while accounting for rolling and pitching angles.
% \clearpage

{
\bibliographystyle{IEEEtran}
\bibliography{ref}
}

\newpage

%\clearpage

\appendix
\section*{Appendix}
\begin{table*}[h!]
\caption{The basic information of grid-based spatio-temporal data.}
\label{tbl:append_data}
\begin{threeparttable}
\resizebox{1.9\columnwidth}{!}{
\begin{tabular}{cccccccc}
\toprule
Dataset & City & Type & Temporal Period & Spatial partition & Interval & Mean & Std \\
\hline
TaxiBJ & Beijing & Taxi flow&  2014/03/01 - 2014/06/30 & $32 \times 32$ & Half an hour & 111.5 & 139.3 \\
BikeDC & Washington, D.C. & Bike flow&  2010/09/20 - 2010/10/20 & $20 \times 20$ & Half an hour & 0.924 & 4.88 \\



CellularSH & Shanghai & Cellular traffic &  2014/08/01 - 2014/08/21 & $32\times28$ & One hour & 0.175 & 0.212 \\
CellularNJ & Nanjing & Cellular traffic &  2021/02/02 - 2021/02/22 & $20\times28$ & One hour & 0.842 & 1.30 \\
CrowdBJ & Beijing & Crowd flow &  2018/01/01 - 2018/01/31 & $1010$ & One hour & 7.07 & 11.1 \\
CrowdBM & Baltimore & Crowd flow &  2019/01/01 - 2019/05/31 & $403$ & One hour & 14.4 & 29.3 \\
Los-Speed & Los Angeles & Traffic speed&  2012/03/01 - 2012/03/07 & $207$ & Five minutes & 59.0 & 12.5 \\

\bottomrule
\end{tabular}}
\end{threeparttable}
\end{table*}

% \begin{table*}[t!]
% \caption{The basic information of Graph-based spatio-temporal data.}
% \label{tbl:append_data_graph}
% \begin{threeparttable}
% \resizebox{1.8\columnwidth}{!}{
% \begin{tabular}{ccccccccc}
% \toprule
% Dataset & City & Type & Temporal Period & Interval & \#Nodes & \#Edges & Mean & Std \\
% \hline

% TrafficBJ & Beijing & Traffic speed & 2022/03/05 - 2022/04/05 & 15min& 13675& 24444& 6.837&  3.412\\
% TrafficSH & Shanghai & Traffic speed & 2022/01/27 - 2022/02/27 & 15min & 21099& 39065& 7.815&  4.044\\
% TrafficNJ & Nanjing & Traffic speed  & 2022/03/05 - 2022/04/05 & 15min & 13419& 25100& 6.699&  4.253\\

% \bottomrule
% \end{tabular}}
% \end{threeparttable}
% \end{table*}
\begin{table*}[h!]
\caption{Short-term prediction results on two additional datasets in terms of both deterministic and probabilistic metrics. \textbf{Bold} indicates the best performance, while \underline{underlining} denotes the second-best.}
\label{tbl:short1-app}
\begin{threeparttable}
% \resizebox{2.0\columnwidth}

\resizebox{1.5\columnwidth}{!}{
\begin{tabular}{ccccccccccc}
\toprule
\multirow{2}{*}{\textbf{Model}}
& \multicolumn{5}{c}{\textbf{CellularNJ}} & \multicolumn{5}{c}{\textbf{CrowdBM}}   \\
\cmidrule(lr){2-6} \cmidrule(lr){7-11} 
 &\textbf{MAE} & \textbf{RMSE}  &\textbf{CRPS} & \textbf{QICE} & \textbf{IS} & 
\textbf{MAE} & \textbf{RMSE} & \textbf{CRPS} & \textbf{QICE} & \textbf{IS} \\


\midrule
D3VAE& 0.580 & 1.135   &	0.565&	0.096&6.03	&	11.0&24.7	&	0.593& 0.110	&136.4\\


DiffSTG& 0.317& 0.649&	0.291&	0.071&3.11	&	8.88&21.3	&0.453&0.047	&68.5\\

TimeGrad& 0.340&  0.357  &	0.432&	0.162&5.87	&10.1	& \underline{12.4}	&\textbf{0.240}&\underline{0.085}	&\underline{46.9}\\


CSDI&0.129 & 0.237   &	\underline{0.111}&	 \underline{0.039}&	\underline{0.80}&	7.31& 19.3&	0.390& 0.054&61.1\\

NPDiff& \underline{0.123}&  \underline{0.175}  &0.128	&	0.133&2.22	&	\underline{5.42}& 13.7	&0.331&0.119	&91.2\\

DyDiffusion&0.222&   0.357 &0.196	&0.080	&	1.80&-	&-	&-	&-&-\\




\cmidrule(lr){1-1} \cmidrule(lr){2-6} \cmidrule(lr){7-11}
\textbf{CoST}&\textbf{0.102} &\textbf{0.172}    &\textbf{0.090}	&\textbf{0.037}	&\textbf{0.682}	&\textbf{5.04}	&\textbf{12.1}	&\underline{0.256}	&\textbf{0.027}& \textbf{37.8}\\
%\textbf{Reduction}& &    &	&	&	&	&	&	&\\
\bottomrule
\end{tabular}}
\end{threeparttable}
\end{table*}


% \input{Tables/short-term2-append}


\end{document}
