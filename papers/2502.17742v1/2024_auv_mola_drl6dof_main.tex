\documentclass[conference]{sty/IEEEtran}

\usepackage{amsmath,amsfonts,amssymb}
\usepackage{algorithmic}
\usepackage{algorithm}
\usepackage{array}
\usepackage[caption=false,font=normalsize,labelfont=sf,textfont=sf]{subfig}
\usepackage{textcomp}
\usepackage{stfloats}
\usepackage{url}
\usepackage{verbatim}
\usepackage{booktabs}
\usepackage{multirow}
\usepackage{rotating}
\usepackage{makecell}
\usepackage{cite}
\usepackage{xspace}
\usepackage{bm}
\usepackage{sty/my_acronyms}
\usepackage{siunitx}
\usepackage{tikz}
\usepackage{sty/my_acronyms}
\usetikzlibrary{shapes,arrows,calc}
\hyphenation{op-tical net-works semi-conduc-tor IEEE-Xplore}
\newcommand{\RomanNumeralCaps}[1] {\MakeUppercase{\romannumeral #1}}

% Prevent line-breaking inline formulas
\relpenalty=99999
\binoppenalty=99999

% Required to identify funding agencies
\IEEEoverridecommandlockouts

% Paper title
\title{\LARGE \bf
Toward 6-DOF Autonomous Underwater Vehicle Energy-Aware Position Control based on Deep Reinforcement Learning: Preliminary Results
\thanks{(*) These authors contributed equally. This work was supported by the David and Lucile Packard Foundation and the ANID-Chile grants for doctoral program N°21200358 and master program N°22231977.}
}

\author{\IEEEauthorblockN{Gustavo Boré${}^*$}
\IEEEauthorblockA{\textit{Pontificia Universidad Católica de Chile} \\
Santiago, Chile \\
Email: gibore@uc.cl}
\\ % Line break to align authors
\IEEEauthorblockN{Sebastián Rodríguez-Martínez}
\IEEEauthorblockA{\textit{Monterey Bay Aquarium Research Institute} \\
Moss Landing, California 95039--9644\\
Email: srodriguez@mbari.org}
\and
\IEEEauthorblockN{Vicente Sufán${}^*$}
\IEEEauthorblockA{\textit{Pontificia Universidad Católica de Chile} \\
Santiago, Chile \\
Email: vicente.sufan@uc.cl}
\\ % Line break to align authors
\IEEEauthorblockN{Giancarlo Troni}
\IEEEauthorblockA{\textit{Monterey Bay Aquarium Research Institute} \\
Moss Landing, California 95039--9644 \\
Email: gtroni@mbari.org}
}

\begin{document}

\maketitle
\thispagestyle{empty}
\pagestyle{empty}

\begin{abstract}
The use of \acp{auv} for surveying, mapping, and inspecting unexplored underwater areas plays a crucial role, where maneuverability and power efficiency are key factors for extending the use of these platforms, making \ac{6dof} holonomic platforms essential tools. Although \ac{pid} and \acl{mpc} controllers are widely used in these applications, they often require accurate system knowledge, struggle with repeatability when facing payload or configuration changes, and can be time-consuming to fine-tune. While more advanced methods based on \acl{drl} have been proposed, they are typically limited to operating in fewer degrees of freedom.
%
This paper proposes a novel \acs{drl}-based approach for controlling holonomic \ac{6dof} \acp{auv} using the \ac{tqc} algorithm, which does not require manual tuning and directly feeds commands to the thrusters without prior knowledge of their configuration. Furthermore, it incorporates power consumption directly into the reward function. Simulation results show that the \acl{tqc-hp} method achieves better performance to a fine-tuned \ac{pid} controller when reaching a goal point, while the \acl{tqc-ea} method demonstrates slightly lower performance but consumes \SI{30}{\percent} less power on average.
\end{abstract}

% Reset acronyms
\acresetall

%%%%%%%%%%%%%%%%%%%%%%%%%%%%%%%%%%%%%%%%%%%%%%%%%%%%%%%%%%%%%%%%%%%%%%%%%%%%%%%%
% Sections

\section{Introduction}
\label{section:introduction}

% redirection is unique and important in VR
Virtual Reality (VR) systems enable users to embody virtual avatars by mirroring their physical movements and aligning their perspective with virtual avatars' in real time. 
As the head-mounted displays (HMDs) block direct visual access to the physical world, users primarily rely on visual feedback from the virtual environment and integrate it with proprioceptive cues to control the avatar’s movements and interact within the VR space.
Since human perception is heavily influenced by visual input~\cite{gibson1933adaptation}, 
VR systems have the unique capability to control users' perception of the virtual environment and avatars by manipulating the visual information presented to them.
Leveraging this, various redirection techniques have been proposed to enable novel VR interactions, 
such as redirecting users' walking paths~\cite{razzaque2005redirected, suma2012impossible, steinicke2009estimation},
modifying reaching movements~\cite{gonzalez2022model, azmandian2016haptic, cheng2017sparse, feick2021visuo},
and conveying haptic information through visual feedback to create pseudo-haptic effects~\cite{samad2019pseudo, dominjon2005influence, lecuyer2009simulating}.
Such redirection techniques enable these interactions by manipulating the alignment between users' physical movements and their virtual avatar's actions.

% % what is hand/arm redirection, motivation of study arm-offset
% \change{\yj{i don't understand the purpose of this paragraph}
% These illusion-based techniques provide users with unique experiences in virtual environments that differ from the physical world yet maintain an immersive experience. 
% A key example is hand redirection, which shifts the virtual hand’s position away from the real hand as the user moves to enhance ergonomics during interaction~\cite{feuchtner2018ownershift, wentzel2020improving} and improve interaction performance~\cite{montano2017erg, poupyrev1996go}. 
% To increase the realism of virtual movements and strengthen the user’s sense of embodiment, hand redirection techniques often incorporate a complete virtual arm or full body alongside the redirected virtual hand, using inverse kinematics~\cite{hartfill2021analysis, ponton2024stretch} or adjustments to the virtual arm's movement as well~\cite{li2022modeling, feick2024impact}.
% }

% noticeability, motivation of predicting a probability, not a classification
However, these redirection techniques are most effective when the manipulation remains undetected~\cite{gonzalez2017model, li2022modeling}. 
If the redirection becomes too large, the user may not mitigate the conflict between the visual sensory input (redirected virtual movement) and their proprioception (actual physical movement), potentially leading to a loss of embodiment with the virtual avatar and making it difficult for the user to accurately control virtual movements to complete interaction tasks~\cite{li2022modeling, wentzel2020improving, feuchtner2018ownershift}. 
While proprioception is not absolute, users only have a general sense of their physical movements and the likelihood that they notice the redirection is probabilistic. 
This probability of detecting the redirection is referred to as \textbf{noticeability}~\cite{li2022modeling, zenner2024beyond, zenner2023detectability} and is typically estimated based on the frequency with which users detect the manipulation across multiple trials.

% version B
% Prior research has explored factors influencing the noticeability of redirected motion, including the redirection's magnitude~\cite{wentzel2020improving, poupyrev1996go}, direction~\cite{li2022modeling, feuchtner2018ownershift}, and the visual characteristics of the virtual avatar~\cite{ogawa2020effect, feick2024impact}.
% While these factors focus on the avatars, the surrounding virtual environment can also influence the users' behavior and in turn affect the noticeability of redirection.
% One such prominent external influence is through the visual channel - the users' visual attention is constantly distracted by complex visual effects and events in practical VR scenarios.
% Although some prior studies have explored how to leverage user blindness caused by visual distractions to redirect users' virtual hand~\cite{zenner2023detectability}, there remains a gap in understanding how to quantify the noticeability of redirection under visual distractions.

% visual stimuli and gaze behavior
Prior research has explored factors influencing the noticeability of redirected motion, including the redirection's magnitude~\cite{wentzel2020improving, poupyrev1996go}, direction~\cite{li2022modeling, feuchtner2018ownershift}, and the visual characteristics of the virtual avatar~\cite{ogawa2020effect, feick2024impact}.
While these factors focus on the avatars, the surrounding virtual environment can also influence the users' behavior and in turn affect the noticeability of redirection.
This, however, remains underexplored.
One such prominent external influence is through the visual channel - the users' visual attention is constantly distracted by complex visual effects and events in practical VR scenarios.
We thus want to investigate how \textbf{visual stimuli in the virtual environment} affect the noticeability of redirection.
With this, we hope to complement existing works that focus on avatars by incorporating environmental visual influences to enable more accurate control over the noticeability of redirected motions in practical VR scenarios.
% However, in realistic VR applications, the virtual environment often contains complex visual effects beyond the virtual avatar itself. 
% We argue that these visual effects can \textbf{distract users’ visual attention and thus affect the noticeability of redirection offsets}, while current research has yet taken into account.
% For instance, in a VR boxing scenario, a user’s visual attention is likely focused on their opponent rather than on their virtual body, leading to a lower noticeability of redirection offsets on their virtual movements. 
% Conversely, when reaching for an object in the center of their field of view, the user’s attention is more concentrated on the virtual hand’s movement and position to ensure successful interaction, resulting in a higher noticeability of offsets.

Since each visual event is a complex choreography of many underlying factors (type of visual effect, location, duration, etc.), it is extremely difficult to quantify or parameterize visual stimuli.
Furthermore, individuals respond differently to even the same visual events.
Prior neuroscience studies revealed that factors like age, gender, and personality can influence how quickly someone reacts to visual events~\cite{gillon2024responses, gale1997human}. 
Therefore, aiming to model visual stimuli in a way that is generalizable and applicable to different stimuli and users, we propose to use users' \textbf{gaze behavior} as an indicator of how they respond to visual stimuli.
In this paper, we used various gaze behaviors, including gaze location, saccades~\cite{krejtz2018eye}, fixations~\cite{perkhofer2019using}, and the Index of Pupil Activity (IPA)~\cite{duchowski2018index}.
These behaviors indicate both where users are looking and their cognitive activity, as looking at something does not necessarily mean they are attending to it.
Our goal is to investigate how these gaze behaviors stimulated by various visual stimuli relate to the noticeability of redirection.
With this, we contribute a model that allows designers and content creators to adjust the redirection in real-time responding to dynamic visual events in VR.

To achieve this, we conducted user studies to collect users' noticeability of redirection under various visual stimuli.
To simulate realistic VR scenarios, we adopted a dual-task design in which the participants performed redirected movements while monitoring the visual stimuli.
Specifically, participants' primary task was to report if they noticed an offset between the avatar's movement and their own, while their secondary task was to monitor and report the visual stimuli.
As realistic virtual environments often contain complex visual effects, we started with simple and controlled visual stimulus to manage the influencing factors.

% first user study, confirmation study
% collect data under no visual stimuli, different basic visual stimuli
We first conducted a confirmation study (N=16) to test whether applying visual stimuli (opacity-based) actually affects their noticeability of redirection. 
The results showed that participants were significantly less likely to detect the redirection when visual stimuli was presented $(F_{(1,15)}=5.90,~p=0.03)$.
Furthermore, by analyzing the collected gaze data, results revealed a correlation between the proposed gaze behaviors and the noticeability results $(r=-0.43)$, confirming that the gaze behaviors could be leveraged to compute the noticeability.

% data collection study
We then conducted a data collection study to obtain more accurate noticeability results through repeated measurements to better model the relationship between visual stimuli-triggered gaze behaviors and noticeability of redirection.
With the collected data, we analyzed various numerical features from the gaze behaviors to identify the most effective ones. 
We tested combinations of these features to determine the most effective one for predicting noticeability under visual stimuli.
Using the selected features, our regression model achieved a mean squared error (MSE) of 0.011 through leave-one-user-out cross-validation. 
Furthermore, we developed both a binary and a three-class classification model to categorize noticeability, which achieved an accuracy of 91.74\% and 85.62\%, respectively.

% evaluation study
To evaluate the generalizability of the regression model, we conducted an evaluation study (N=24) to test whether the model could accurately predict noticeability with new visual stimuli (color- and scale-based animations).
Specifically, we evaluated whether the model's predictions aligned with participants' responses under these unseen stimuli.
The results showed that our model accurately estimated the noticeability, achieving mean squared errors (MSE) of 0.014 and 0.012 for the color- and scale-based visual stimili, respectively, compared to participants' responses.
Since the tested visual stimuli data were not included in the training, the results suggested that the extracted gaze behavior features capture a generalizable pattern and can effectively indicate the corresponding impact on the noticeability of redirection.

% application
Based on our model, we implemented an adaptive redirection technique and demonstrated it through two applications: adaptive VR action game and opportunistic rendering.
We conducted a proof-of-concept user study (N=8) to compare our adaptive redirection technique with a static redirection, evaluating the usability and benefits of our adaptive redirection technique.
The results indicated that participants experienced less physical demand and stronger sense of embodiment and agency when using the adaptive redirection technique. 
These results demonstrated the effectiveness and usability of our model.

In summary, we make the following contributions.
% 
\begin{itemize}
    \item 
    We propose to use users' gaze behavior as a medium to quantify how visual stimuli influences the noticebility of redirection. 
    Through two user studies, we confirm that visual stimuli significantly influences noticeability and identify key gaze behavior features that are closely related to this impact.
    \item 
    We build a regression model that takes the user's gaze behavioral data as input, then computes the noticeability of redirection.
    Through an evaluation study, we verify that our model can estimate the noticeability with new participants under unseen visual stimuli.
    These findings suggest that the extracted gaze behavior features effectively capture the influence of visual stimuli on noticeability and can generalize across different users and visual stimuli.
    \item 
    We develop an adaptive redirection technique based on our regression model and implement two applications with it.
    With a proof-of-concept study, we demonstrate the effectiveness and potential usability of our regression model on real-world use cases.

\end{itemize}

% \delete{
% Virtual Reality (VR) allows the user to embody a virtual avatar by mirroring their physical movements through the avatar.
% As the user's visual access to the physical world is blocked in tasks involving motion control, they heavily rely on the visual representation of the avatar's motions to guide their proprioception.
% Similar to real-world experiences, the user is able to resolve conflicts between different sensory inputs (e.g., vision and motor control) through multisensory integration, which is essential for mitigating the sensory noise that commonly arises.
% However, it also enables unique manipulations in VR, as the system can intentionally modify the avatar's movements in relation to the user's motions to achieve specific functional outcomes,
% for example, 
% % the manipulations on the avatar's movements can 
% enabling novel interaction techniques of redirected walking~\cite{razzaque2005redirected}, redirected reaching~\cite{gonzalez2022model}, and pseudo haptics~\cite{samad2019pseudo}.
% With small adjustments to the avatar's movements, the user can maintain their sense of embodiment, due to their ability to resolve the perceptual differences.
% % However, a large mismatch between the user and avatar's movements can result in the user losing their sense of embodiment, due to an inability to resolve the perceptual differences.
% }

% \delete{
% However, multisensory integration can break when the manipulation is so intense that the user is aware of the existence of the motion offset and no longer maintains the sense of embodiment.
% Prior research studied the intensity threshold of the offset applied on the avatar's hand, beyond which the embodiment will break~\cite{li2022modeling}. 
% Studies also investigated the user's sensitivity to the offsets over time~\cite{kohm2022sensitivity}.
% Based on the findings, we argue that one crucial factor that affects to what extent the user notices the offset (i.e., \textit{noticeability}) that remains under-explored is whether the user directs their visual attention towards or away from the virtual avatar.
% Related work (e.g., Mise-unseen~\cite{marwecki2019mise}) has showcased applications where adjustments in the environment can be made in an unnoticeable manner when they happen in the area out of the user's visual field.
% We hypothesize that directing the user's visual attention away from the avatar's body, while still partially keeping the avatar within the user's field-of-view, can reduce the noticeability of the offset.
% Therefore, we conduct two user studies and implement a regression model to systematically investigate this effect.
% }

% \delete{
% In the first user study (N = 16), we test whether drawing the user's visual attention away from their body impacts the possibility of them noticing an offset that we apply to their arm motion in VR.
% We adopt a dual-task design to enable the alteration of the user's visual attention and a yes/no paradigm to measure the noticeability of motion offset. 
% The primary task for the user is to perform an arm motion and report when they perceive an offset between the avatar's virtual arm and their real arm.
% In the secondary task, we randomly render a visual animation of a ball turning from transparent to red and becoming transparent again and ask them to monitor and report when it appears.
% We control the strength of the visual stimuli by changing the duration and location of the animation.
% % By changing the time duration and location of the visual animation, we control the strengths of attraction to the users.
% As a result, we found significant differences in the noticeability of the offsets $(F_{(1,15)}=5.90,~p=0.03)$ between conditions with and without visual stimuli.
% Based on further analysis, we also identified the behavioral patterns of the user's gaze (including pupil dilation, fixations, and saccades) to be correlated with the noticeability results $(r=-0.43)$ and they may potentially serve as indicators of noticeability.
% }

% \delete{
% To further investigate how visual attention influences the noticeability, we conduct a data collection study (N = 12) and build a regression model based on the data.
% The regression model is able to calculate the noticeability of the offset applied on the user's arm under various visual stimuli based on their gaze behaviors.
% Our leave-one-out cross-validation results show that the proposed method was able to achieve a mean-squared error (MSE) of 0.012 in the probability regression task.
% }

% \delete{
% To verify the feasibility and extendability of the regression model, we conduct an evaluation study where we test new visual animations based on adjustments on scale and color and invite 24 new participants to attend the study.
% Results show that the proposed method can accurately estimate the noticeability with an MSE of 0.014 and 0.012 in the conditions of the color- and scale-based visual effects.
% Since these animations were not included in the dataset that the regression model was built on, the study demonstrates that the gaze behavioral features we extracted from the data capture a generalizable pattern of the user's visual attention and can indicate the corresponding impact on the noticeability of the offset.
% }

% \delete{
% Finally, we demonstrate applications that can benefit from the noticeability prediction model, including adaptive motion offsets and opportunistic rendering, considering the user's visual attention. 
% We conclude with discussions of our work's limitations and future research directions.
% }

% \delete{
% In summary, we make the following contributions.
% }
% % 
% \begin{itemize}
%     \item 
%     \delete{
%     We quantify the effects of the user's visual attention directed away by stimuli on their noticeability of an offset applied to the avatar's arm motion with respect to the user's physical arm. 
%     Through two user studies, we identified gaze behavioral features that are indicative of the changes in noticeability.
%     }
%     \item 
%     \delete{We build a regression model that takes the user's gaze behavioral data and the offset applied to the arm motion as input, then computes the probability of the user noticing the offset.
%     Through an evaluation study, we verified that the model needs no information about the source attracting the user's visual attention and can be generalizable in different scenarios.
%     }
%     \item 
%     \delete{We demonstrate two applications that potentially benefit from the regression model, including adaptive motion offsets and opportunistic rendering.
%     }

% \end{itemize}

\begin{comment}
However, users will lose the sense of embodiment to the virtual avatars if they notice the offset between the virtual and physical movements.
To address this, researchers have been exploring the noticing threshold of offsets with various magnitudes and proposing various redirection techniques that maintain the sense of embodiment~\cite{}.

However, when users embody virtual avatars to explore virtual environments, they encounter various visual effects and content that can attract their attention~\cite{}.
During this, the user may notice an offset when he observes the virtual movement carefully while ignoring it when the virtual contents attract his attention from the movements.
Therefore, static offset thresholds are not appropriate in dynamic scenarios.

Past research has proposed dynamic mapping techniques that adapted to users' state, such as hand moving speed~\cite{frees2007prism} or ergonomically comfortable poses~\cite{montano2017erg}, but not considering the influence of virtual content.
More specifically, PRISM~\cite{frees2007prism} proposed adjusting the C/D ratio with a non-linear mapping according to users' hand moving speed, but it might not be optimal for various virtual scenarios.
While Erg-O~\cite{montano2017erg} redirected users' virtual hands according to the virtual target's relative position to reduce physical fatigue, neglecting the change of virtual environments. 

Therefore, how to design redirection techniques in various scenarios with different visual attractions remains unknown.
To address this, we investigate how visual attention affects the noticing probability of movement offsets.
Based on our experiments, we implement a computational model that automatically computes the noticing probability of offsets under certain visual attractions.
VR application designers and developers can easily leverage our model to design redirection techniques maintaining the sense of embodiment adapt to the user's visual attention.
We implement a dynamic redirection technique with our model and demonstrate that it effectively reduces the target reaching time without reducing the sense of embodiment compared to static redirection techniques.

% Need to be refined
This paper offers the following contributions.
\begin{itemize}
    \item We investigate how visual attractions affect the noticing probability of redirection offsets.
    \item We construct a computational model to predict the noticing probability of an offset with a given visual background.
    \item We implement a dynamic redirection technique adapting to the visual background. We evaluate the technique and develop three applications to demonstrate the benefits. 
\end{itemize}



First, we conducted a controlled experiment to understand how users perceived the movement offset while subjected to various distractions.
Since hand redirection is one of the most frequently used redirections in VR interactions, we focused on the dynamic arm movements and manually added angular offsets to the' elbow joint~\cite{li2022modeling, gonzalez2022model, zenner2019estimating}. 
We employed flashing spheres in the user's field of view as distractions to attract users' visual attention.
Participants were instructed to report the appearing location of the spheres while simultaneously performing the arm movements and reporting if they perceived an offset during the movement. 
(\zhipeng{Add the results of data collection. Analyze the influence of the distance between the gaze map and the offset.}
We measured the visual attraction's magnitude with the gaze distribution on it.
Results showed that stronger distractions made it harder for users to notice the offset.)
\zhipeng{Need to rewrite. Not sure to use gaze distribution or a metric obtained from the visual content.}
Secondly, we constructed a computational model to predict the noticing probability of offsets with given visual content.
We analyzed the data from the user studies to measure the influence of visual attractions on the noticing probability of offsets.
We built a statistical model to predict the offset's noticing probability with a given visual content.
Based on the model, we implement a dynamic redirection technique to adjust the redirection offset adapted to the user's current field of view.
We evaluated the technique in a target selection task compared to no hand redirection and static hand redirection.
\zhipeng{Add the results of the evaluation.}
Results showed that the dynamic hand redirection technique significantly reduced the target selection time with similar accuracy and a comparable sense of embodiment.
Finally, we implemented three applications to demonstrate the potential benefits of the visual attention adapted dynamic redirection technique.
\end{comment}

% This one modifies arm length, not redirection
% \citeauthor{mcintosh2020iteratively} proposed an adaptation method to iteratively change the virtual avatar arm's length based on the primary tasks' performance~\cite{mcintosh2020iteratively}.



% \zhipeng{TO ADD: what is redirection}
% Redirection enables novel interactions in Virtual Reality, including redirected walking, haptic redirection, and pseudo haptics by introducing an offset to users' movement.
% \zhipeng{TO ADD: extend this sentence}
% The price of this is that users' immersiveness and embodiment in VR can be compromised when they notice the offset and perceive the virtual movement not as theirs~\cite{}.
% \zhipeng{TO ADD: extend this sentence, elaborate how the virtual environment attracts users' attention}
% Meanwhile, the visual content in the virtual environment is abundant and consistently captures users' attention, making it harder to notice the offset~\cite{}.
% While previous studies explored the noticing threshold of the offsets and optimized the redirection techniques to maintain the sense of embodiment~\cite{}, the influence of visual content on the probability of perceiving offsets remains unknown.  
% Therefore, we propose to investigate how users perceive the redirection offset when they are facing various visual attractions.


% We conducted a user study to understand how users notice the shift with visual attractions.
% We used a color-changing ball to attract the user's attention while instructing users to perform different poses with their arms and observe it meanwhile.
% \zhipeng{(Which one should be the primary task? Observe the ball should be the primary one, but if the primary task is too simple, users might allocate more attention on the secondary task and this makes the secondary task primary.)}
% \zhipeng{(We need a good and reasonable dual-task design in which users care about both their pose and the visual content, at least in the evaluation study. And we need to be able to control the visual content's magnitude and saliency maybe?)}
% We controlled the shift magnitude and direction, the user's pose, the ball's size, and the color range.
% We set the ball's color-changing interval as the independent factor.
% We collect the user's response to each shift and the color-changing times.
% Based on the collected data, we constructed a statistical model to describe the influence of visual attraction on the noticing probability.
% \zhipeng{(Are we actually controlling the attention allocation? How do we measure the attracting effect? We need uniform metrics, otherwise it is also hard for others to use our knowledge.)}
% \zhipeng{(Try to use eye gaze? The eye gaze distribution in the last five seconds to decide the attention allocation? Basically constructing a model with eye gaze distribution and noticing probability. But the user's head is moving, so the eye gaze distribution is not aligned well with the current view.)}

% \zhipeng{Saliency and EMD}
% \zhipeng{Gaze is more than just a point: Rethinking visual attention
% analysis using peripheral vision-based gaze mapping}

% Evaluation study(ideal case): based on the visual content, adjusting the redirection magnitude dynamically.

% \zhipeng{(The risk is our model's effect is trivial.)}

% Applications:
% Playing Lego while watching demo videos, we can accelerate the reaching process of bricks, and forbid the redirection during the manipulation.

% Beat saber again: but not make a lot of sense? Difficult game has complicated visual effects, while allows larger shift, but do not need large shift with high difficulty



%!TEX root = 2024_auv_mola_drl6dof_main.tex
%%%%%%%%%%%%%%%%%%%%%%%%%%%%%%%%%%%%%%%%%%%%%%%%%%%%%%%%%%%%%%%%%%%%%%%
\section{Background}
\label{sec:background}

%%%%%%%%%%%%%%%%%%%%%%%%%%%%%%%%%%%%%%%%%%%%%%%%%%
%%% Reinforcement learning
\subsection{Deep Reinforcement learning}

\ac{drl} \cite{RichardSutton20} is a method that aims to train an agent's policy $\pi$ to map states into actions by interacting with the environment. This is achieved by maximizing a numerical reward signal and using a \ac{mdp} framework to regulate the interaction between the \ac{rl} agent’s policy and the environment. At each time step, the agent observes a state $\bm{s}$, takes an action $\bm{a}$, and upon transitioning to the next state, receives a reward $r$. Once the episode (i.e., process) is complete, the accumulated reward is calculated as the sum of all time steps rewards in that episode.

\ac{drl} methods can be model-based or model-free. Model-based methods use a model to predict the next state and reward, while model-free methods learn solely from experiencing the unmodeled and unknown consequences of an action. While learning from trial and error may result in less efficient learning, model-free methods have the advantage when a model is unavailable or inaccurate.

\begin{figure}[t!]
\centering
\includegraphics[width=0.45\textwidth]{figures/reward_vs_step.pdf}%
\caption{Average reward per episode over a moving window of 100 episodes obtained by the TQC, SAC, and TD3 algorithms during a $2.5\times10^6$ step training, equivalent to 3125 episodes.}
\label{fig:rewards}
\end{figure}

%%%%%%%%%%%%%%%%%%%%%%%%%%%%%%%%%%%%%%%%%%%%%%%%%%
%% 6DOF Error Computation
\subsection{\ac{6dof} Error Computation}

The position errors are determined by the difference between the current position $(x, y, z)$ and the goal position $(x_d, y_d, z_d)$ following the North-East-Down (NED) convention, computed as
%%%
% Keep to remove space between equations and paragraph
%%%
\begin{equation}
    e_x(t) = x^t - x_d^t,\; e_y(t) = y^t - y_d^t,\; e_z(t) = z^t - z_d^t.
\label{eq:errors}
\end{equation}

To compute the error in attitude, we will evaluate the difference between the current orientation and the goal attitude, both with respect to the fixed world frame. This involves representing both poses as rotation matrices ($\bm{R}\in SO(3)$) and converting their difference to exponential coordinates $[\bm{{e_\theta}}]\in so(3)$ through the matrix logarithm:
%%%
% Keep to remove space between equations and paragraph
%%%
\begin{equation}
     [\bm{{e_\theta}}(t)] = \log(\bm{R}(t)^T \cdot \bm{R}_d)
\end{equation}

Then, the skew-symmetric matrix $[\bm{{e_\theta}}(t)]$ is converted into its vector representation $\bm{{e_\theta}}(t) \in \mathbb{R}^3$, where its entries correspond to the element-wise error for the attitude, defined as
%%%
% Keep to remove space between equations and paragraph
%%%
\begin{equation}
    \begin{bmatrix} \theta_{x}^t & \theta_{y}^t & \theta_{z}^t \end{bmatrix} = \bm{{e_\theta}}(t).
    \label{eq:attitude_error}
\end{equation}

Furthermore, to provide a single metric for attitude error evaluation, we compute $\theta^t$ based on the axis-angle representation for $\bm{{e_\theta}}(t)$, as described in \eqref{eq:theta_error}. By using this metric, we obtain a global evaluation of orientation, which aligns the controller's performance with practical manual navigation comparisons.
%%%
% Keep to remove space between equations and paragraph
%%%
\begin{equation}
    \theta^t = ||\bm{{e_\theta}}(t)||
    \label{eq:theta_error}
\end{equation}

%!TEX root = 2024_auv_mola_drl6dof_main.tex
%%%%%%%%%%%%%%%%%%%%%%%%%%%%%%%%%%%%%%%%%%%%%%%%%%%%%%%%%%%%%%%%%%%%%%%
\section{Proposed Approach}
\label{sec:proposed_approach}

In this work, we introduce \acf{tqc-hp} and \acf{tqc-ea}: two end-to-end \ac{drl}-based approaches for the low-level control of a holonomic \ac{6dof} \ac{auv} using the \ac{tqc} algorithm. These methods require neither manual tuning nor prior knowledge of the thruster configuration.

%%%%%%%%%%%%%%%%%%%%%%%%%%%%%%%%%%%%%%%%%%%%%%%%%%
%% Action Space
\subsection{Action space}

The action space $\bm{a}(t) \in \mathbb{R}^{8}$, normalized between $[-1, 1]$, is designed to enable the agent to generate precise control commands for the vehicle’s thrusters, ensuring the desired movement and orientation based on the given state inputs. The action space is defined as:
%%%
% Keep to remove space between equations and paragraph
%%%
\begin{equation}
\bm{a}(t) = \begin{bmatrix}
T_{1}^t & T_{2}^t & T_{3}^t & T_{4}^t & T_{5}^t & T_{6}^t & T_{7}^t & T_{8}^t
\end{bmatrix}^T,
\label{eq:action_space}
\end{equation}

\noindent where $\bm{a}(t)$ represents the normalized \ac{pwm} signals sent to the eight thrusters at time $t$.

%%%%%%%%%%%%%%%%%%%%%%%%%%%%%%%%%%%%%%%%%%%%%%%%%%
%% Observation Space
\subsection{Observation space}

The observation space $\bm{s}(t) \in \mathbb{R}^{20}$, normalized between $[-1, 1]$, is designed to capture all relevant environmental information necessary for the agent’s decision-making, enabling an understanding of the system’s current state relative to the desired goal. The observation space is defined as:
%%%
% Keep to remove space between equations and paragraph
%%%
\begin{equation}
\bm{s}(t) = \begin{bmatrix}
\bm{e}(t) & \bm{v}(t) & \bm{a}(t-1)
\end{bmatrix}^T,
\label{eq:observation_space}
\end{equation}

\noindent where $\bm{e}(t)=[e_{x}^t, e_{y}^t, e_{z}^t, \theta_{x}^t, \theta_{y}^t, \theta_{z}^t]^T$ represents the error vector at time $t$ between the current pose and the goal pose for each \ac{dof}, $ \bm{v}(t)=[v_{x}^t, v_{y}^t, v_{z}^t, \omega_{x}^t, \omega_{y}^t, \omega_{z}^t]^T$ corresponds to the spatial twist of the vehicle at time $t$, composed by the linear and angular velocities, and $\boldsymbol{a}(t-1)$ the action in the previous time step.

\begin{table}[b!]
\centering
\caption{Weights $\alpha_i$ for reward functions for \ac{tqc-hp} and \ac{tqc-ea}}
\label{tab:weights}
\begin{tabular}{ccccccccc}
\toprule
 & $\bm{\alpha_1}$ & $\bm{\alpha_2}$ & $\bm{\alpha_3}$ & $\bm{\alpha_4}$ & $\bm{\alpha_5}$ & $\bm{\alpha_6}$  \\ \midrule
\textbf{\ac{tqc-hp}} & $-4$ & $-4$ & $-3$ & $-1.8$ & $-1$ & $0$ \\
\textbf{\ac{tqc-ea}} & $-4$ & $-4$ & $-3$ & $-1.7$ & $-0.8$ & $-0.3$  \\ \bottomrule
\end{tabular}
\end{table}

%%%%%%%%%%%%%%%%%%%%%%%%%%%%%%%%%%%%%%%%%%%%%%%%%%
%% Reward Function
\subsection{Reward function}

To evaluate policy performance and provide feedback during the training stage, an optimal reward function must be defined to enable the model to learn the desired behavior. In this work, we define the reward function \eqref{eq:reward_fx} that aims to bring the \ac{auv} closer to the target position in each of the \ac{6dof}, linearly penalizing the agent for positional \eqref{eq:reward_position} and angular \eqref{eq:reward_attitude} errors over time. Additionally, the reward function aims to generate smoother commands for the thrusters by penalizing signal fluctuations \eqref{eq:reward_smoothness} and to minimize the utilization of each thruster, thereby reducing the energy consumption of the \ac{auv} \eqref{eq:reward_power}. Each term is weighted with the parameter  $\alpha_i \leq 0$ to denote the importance of each term within \eqref{eq:reward_fx}.
%%%
% Keep to remove space between equations and paragraph
%%%
\begin{subequations}
\begin{align}
r(t) &= \sum\nolimits_{i=1}^{4} r_i(t) \label{eq:reward_fx} \\
r_1(t) &= \alpha_1 \cdot{|e_x^t|} + \alpha_2 \cdot{|e_y^t|} + \alpha_3 \cdot{|e_z^t|}\label{eq:reward_position} \\
r_2(t) &= \alpha_4 \cdot{|\theta^t|}
\label{eq:reward_attitude} \\
r_3(t) &= \sum\nolimits_{i=1}^{8} \alpha_5 \cdot \left|{T_{i}^t - T_{i}^{t-1}} \right| \label{eq:reward_smoothness} \\
r_4(t) &= \sum\nolimits_{i=1}^{8} \alpha_6 \cdot \left|{T_i^t} \right| \label{eq:reward_power}
\end{align}
\end{subequations}

%%%%%%%%%%%%%%%%%%%%%%%%%%%%%%%%%%%%%%%%%%%%%%%%%%
%% DRL Algorithm
\subsection{Proposed \ac{drl} algorithms}

The proposed approaches are based on a \ac{drl} framework using a \ac{tqc} algorithm \cite{kuznetsov2020}. The selection of this algorithm was guided by an experiment conducted to compare the performance of the \ac{tqc}, \ac{sac}, and \ac{td3} algorithms in a \ac{6dof} \ac{auv} position control problem. The methodology described in Section \ref{sec:metodology} was used, and the three methods were trained using the same reward function to compare which of the aforementioned models obtained the highest average reward during the training episodes, indicating superior performance in the specific control task.

Fig. \ref{fig:rewards} shows the average reward of an episode over a moving window of 100 episodes obtained by the three \ac{drl} algorithms during training. It is evident that the \ac{tqc} algorithm converges in a smaller number of episodes compared to \ac{sac} and \ac{td3}, as well as reaching higher reward values than the other two algorithms. This experiment correlates with the results obtained by Lidtke et al. \cite{lidtke2024} for a 3-\acs{dof} \ac{auv} and confirms our decision to use the \ac{tqc} algorithm for the task of controlling an \ac{auv} in \ac{6dof}.

In this work, two different approaches of \ac{tqc} models are proposed: \ac{tqc-hp} aims to reach the goal position in the \ac{6dof} using smooth commands to the thrusters, i.e., its reward function includes the components $r_1$, $r_2$, and $r_3$, while \ac{tqc-ea} has the same objectives as \ac{tqc-hp} but adds energy-awareness, minimizing the energy consumption of the \ac{auv} by incorporating the $r_4$ component in its reward function. Table \ref{tab:weights} presents the $\alpha_i$ weights used in the reward function for both approaches. Both approaches are implemented in \textit{Python3} using the \textit{Stable Baselines3} library, which is based on principles used in both \ac{sac} and \ac{td3} algorithms but utilizes the distributed representation of a critic, truncation of critic predictions, and a set of multiple critics.


This section introduces \algosmall\ (\algoabb), a novel approach for learning cooperative policies in MARL. The primary motivation for developing \algoabb\ is to capture the richness of interactions in complex environments. \edits{By modeling diverse multi-agent dependencies, higher-order relationships, and indirect interactions, \algoabb\ enables more expressive representations, leading to improved coordination among agents. The workflow of \algoabb, as illustrated in Figure~\ref{fig:methodology}, consists of several key steps: starting from input observations and predefined interaction types represented by adjacency matrices, the approach dynamically models interactions, generates meta coordination graphs, and uses graph convolutions to learn cooperative policies.} 

\begin{figure*}[!ht]
    \centering
    \includegraphics[width=0.95\linewidth]{figures/diagrams/dmcg.png} 
    \caption{\edits{Illustration} of \algotitle\ (\algoabb) for learning cooperative policies in MARL. The diagram shows how the approach captures and adapts to multiple types of interactions among agents, including higher-order and indirect relationships, and generates new graph structures by exploring dynamic interactions, even among initially unconnected agents.} 
    \label{fig:methodology}
\end{figure*} 

\paragraph{Modeling \edits{higher-order} interactions.} Multi-agent interactions often involve diverse and complex relationships. To address this, \edits{we model interactions using a set of adjacency matrices \( \{A_k\}_{k=1}^K \), where each \( A_k \) represents a specific interaction type. Each adjacency matrix encodes the dependencies between agents, where \( A_k[i, j] \neq 0 \) indicates the presence of a \( k \)-th type edge from agent \( j \) to agent \( i \). Notably, an agent can have different types of interactions with different agents, resulting in multiple edges going out of and coming into a single node. This is captured by having \( K \) individual adjacency matrices, each of size \( n \times n \), corresponding to the same \( n \) agents but encoding different interaction types. The graph representation can be expressed as a tensor \( \bm{A} \in \mathbb{R}^{n \times n \times K} \), where \( n \) is the number of agents and \( K \) is the total number of interaction types. This representation allows the model to incorporate multiple interaction types simultaneously, enabling it to capture higher-order relationships.} Additionally, each agent's local observations are represented in a feature matrix \( X \in \mathbb{R}^{n \times d} \), where \( d \) is the dimensionality of the features. \edits{Figure~\ref{fig:methodology} (left block) depicts this initial setup, where multiple adjacency matrices \( \{A_k\}_{k=1}^K \) represent diverse interaction types, forming the input for downstream MCG generation.} 

\paragraph{\edits{Modeling} indirect interactions.} Indirect interactions \edits{arise when the effects of one agent's actions propagate through intermediary agents, creating cascading influences. For example, in a search-and-rescue mission, the actions of one drone may influence others operating far away by altering paths, redistributing tasks, or sharing critical information. Such relationships can be modeled as sequences of interactions \( M \)}, defined as:
\[
v_1 \xrightarrow{e_1} v_2 \xrightarrow{e_2} \dots \xrightarrow{e_l} v_{l+1},
\]
where \( l \) is the length of \( M \). \edits{The composite relationship \( c = e_1 \circ e_2 \circ \dots \circ e_l \) can then represent the combined effect of interactions. This formulation also inherently captures multi-hop connections.} 

\paragraph{Generating meta coordination graphs.} We now \edits{introduce meta coordination graphs (MCGs) which extend traditional coordination graphs by incorporating multiple types of edges, enabling the representation of diverse and dynamic interactions among agents. Formally, an MCG can be defined as \( \mathcal{G}_M = (\mathcal{V}, \mathcal{E}) \), where \( \mathcal{V} \) represents agents as vertices, and \( \mathcal{E} \) represents edges of \( K \) types. Each edge type \( e_k \in \mathcal{E} \) captures a specific kind of interaction or dependency between two agents. Note here that, when \( |K| = 1 \), it becomes a standard coordination graph. For dynamic tasks, edges evolve over time and can be represented as \( e^t_k \in \mathcal{E}_t \), where \( \mathcal{E}_t \) is the set of edges at timestep \( t \). This temporal flexibility allows MCGs to adapt to the changing dynamics of multi-agent environments. As illustrated in Figure~\ref{fig:methodology} (middle block), to dynamically generate MCGs, \algoabb\ starts with the input adjacency matrices \( \{A_k\}_{k=1}^K \), where each \( A_k \) corresponds to a specific interaction type, as discussed previously. Using a softmax-based weighting mechanism, \algoabb\ combines the adjacency matrices to prioritize the most relevant interactions for the task. Specifically, it softly selects \( l \) graph structures from the set of \( K \) candidate adjacency matrices in \( \bm{A} \) by applying a \( 1 \times 1 \) convolution operation. This can be represented as \( \phi(\bm{A}; \text{softmax}(W_\phi)) \), where \( \phi \) is the convolution layer and \( W_\phi \in \mathbb{R}^{1 \times 1 \times K} \) is its parameter. This approach is inspired by channel attention pooling techniques from research in computer vision and pattern recognition \cite{chen20182}. Meta coordination graphs are then generated by multiplying selected adjacency matrices, capturing composite and higher-order dependencies. For instance, given a sequence of interactions \( M \), the resulting adjacency matrix \( A_M \) (corresponding to the MCG \( \mathcal{G}_M\)) is computed as \[ A_M = A_{e_l} \cdots A_{e_2} A_{e_1}, \] where \( l \) is the sequence length. This process allows \algoabb\ to dynamically refine and adapt the graph structure as learning progresses. The generated meta coordination graphs provide a flexible and expressive representation of agent interactions, capturing both direct and indirect relationships. This enables \algoabb\ to model emergent multi-agent behaviors and discover non-obvious interaction patterns, leading to improved coordination in complex MARL environments (discussed further with experiments and results in Section \ref{sec:results}).} \algoabb\ can be regarded as the MARL counterpart to methodologies like Graph Transformer Networks \cite{yun2019graph} for graph neural network-based tasks and Spatial Transformer Networks \cite{jaderberg2015spatial} for computer vision. Both aim to dynamically learn and adapt the underlying structure (graph-based relationships and spatial configurations, respectively) just as \algoabb\ learns and encodes emergent interactions among agents in the MARL context. 

\begin{figure*}[!htp]
    \centering
    \captionsetup[subfigure]{justification=centering}
    \subfloat[Gather]{\includegraphics[width=0.12\linewidth]{figures/envs/gather.png}} \hfill 
    \subfloat[Disperse]{\includegraphics[width=0.25\linewidth]{figures/envs/disperse.png}} \hfill  
    \subfloat[Pursuit]{\includegraphics[width=0.12\linewidth]{figures/envs/pursuit.png}} \hfill 
    \subfloat[Hallway]{\includegraphics[width=0.25\linewidth]{figures/envs/hallway.png}} \hfill
    \subfloat[SMACv2]{\includegraphics[width=0.2\linewidth]{figures/envs/smacv2.jpeg}} 
    \caption{Evaluation environments. The figure shows an example of (a) Gather with $g_1$ as the optimal goal and \{$g_2$, $g_3$\} as suboptimal goals, (b) Disperse illustrating a need for 6 agents at Hospital 1 at time $t$, (c) Pursuit where at least 2 predator agents (purple) must coordinate to capture a prey (yellow), (d) Hallway with 2 groups of agents, and (e) a SMACv2 scenario.} 
    \label{fig:envs}
\end{figure*}

\paragraph{Modeling arbitrary lengths of indirect interactions.} To encode \edits{relationships of arbitrary lengths $l$,} the adjacency matrix \( A_M \) \edits{can be generalized} as  \[ A_M = \prod_{i=1}^{l} \left( \sum_{e_i \in \mathcal{E}} \alpha^{(i)} A_{e_i} \right) \] where \( \alpha^{(i)} \) is the weight for the adjacency matrix \( A_{e_i} \) associated with the \( e_i \)-th interaction type. When \( \alpha \) is not a one-hot vector, \( A_M \) can be seen as the weighted sum of adjacency matrices for all length-\( l \) connections. We also append the identity matrix \( I \) in \( \bm{A} \), i.e., \( A_0 = I \). 

\paragraph{Generating multiple meta coordination graphs.} We can capture and represent a variety of interaction patterns and dependencies among agents in a flexible and scalable manner by generating multiple meta coordination graphs simultaneously in \algoabb. To do so, the output channels of \(1 \times 1\) convolution can be set to a desired value \(o\). Then, the soft selection will render adjacency tensors of size $n\times n\times o$, and their multiplication will give us an adjacency tensor for $\mathbb{A}_M \in \mathbb{R}^{n\times n\times o}$. 

\paragraph{\edits{Graph Convolution.}} We then compute a concatenation of node representations from each output channel using a graph convolutional layer \cite{kipf2017semisupervised} as follows \[ Z = \bigg\|_{i=1}^{o} \sigma(\tilde{D}_i^{-1} \tilde{\mathbb{A}}_M^{(i)} XW), \] where \(\| \) is the concatenation operator, \(o\) denotes the number of channels, \( \tilde{\mathbb{A}}_M^{(i)} = \mathbb{A}_M^{(i)} + I\) is the adjacency matrix from the \(i\)-th channel of \( \mathbb{A}_M \), \(\tilde{D}_i\) is the degree matrix of \(\tilde{\mathbb{A}}_M^{(i)}\), \(W \in \mathbb{R}^{d \times d}\) is a trainable weight matrix shared across channels and \(X \in \mathbb{R}^{n \times d}\) is a feature matrix. \(Z\) contains the node representations from \(o\) different meta coordination graphs. 

\paragraph{Learning cooperative policies.} Then, on the newly generated meta coordination graphs, our approach models coordination relations upon the graph-based value factorization specified by deep coordination graphs \cite{bohmer2020deep}, in which the global value function is factorized into the summation of individual utility functions \( Q_i \) and payoff functions \( Q_{ij} \) \edits{(Figure~\ref{fig:methodology}; right block)} as follows: 
\begin{equation*}
    \begin{split}
        Q&(\tau^t, a; \mathcal{G}_M)= \frac{1}{|\mathcal{V}|} \sum_{i \in I} Q_i(a_i \mid \tau_i) \quad + \\
        & \frac{1}{2|\mathcal{E}|} \sum_{(i,j) \in \mathcal{E}}\left[Q_{ij}(a_i, a_j \mid \tau^t_i, \tau^t_j) + Q_{ji}(a_j, a_i \mid \tau^t_j, \tau^t_i) \right].
    \end{split}
\end{equation*} 

\noindent The primary advantage of \algoabb\ lies in its capability to discover, learn, and leverage complex multi-agent interactions for improving coordination. Higher-order and indirect relationships are often far more intricate and emergent in nature, involving subtle dependencies and influences that cannot be explicitly specified during the initial setup of a task or environment. These interactions must be uncovered and modeled through careful analysis of the agents’ behaviors and the underlying structure of the coordination graph. \edits{\algoabb\ requires an initial set of coordination graphs (CGs) and edge types ( K ) as input. If the environment does not provide such input, the initialization of CGs and edge types becomes a tunable hyperparameter. In such cases, we investigate the performance of various topologies defined in DCG, such as fully connected, cyclic, linear, and star structures, to determine their suitability for different tasks. Through its attention mechanism, \algoabb\ dynamically learns and refines these edge types, prioritizing the most relevant interactions during training. This adaptive mechanism ensures the model evolves beyond the initial CG structure to align with task-specific requirements. By dynamically combining and refining initial edge types, \algoabb\ captures both higher-order and indirect relationships,} enabling the generation of novel graph structures that reflect nuanced, dynamic interactions, potentially among initially unconnected agents. As we will see in the following sections, this design results in significant performance enhancements across a range of challenging multi-agent scenarios. 


\section{Results}
We first describe communication patterns within the full chronological context of the game in \textit{League of Legends (LoL)}, separated into four sections based on changing coordination dynamics. Based on this context, we identify core factors players assess to decide when to participate in communication with other teammates. Afterward, we discuss how communication shapes player perceptions toward their teammates, showing player's wariness towards players actively engaging in communication. 

\subsection{Communication Patterns in Context}

We discuss the communication patterns among teammates within the game. We organize the data into chronological phases of the game for a structured analysis of how the context shapes communication patterns. 

\subsubsection{Pre-game stage}
Before gameplay begins, team communication opens with \textit{team drafting}, where players are assigned roles (Top, Mid, Bot, Support, or Jungle) and take turns picking or banning champions. In Solo Ranked mode, roles are pre-assigned based on player preferences selected before queueing. Once teams are set, all players enter \textit{champion select} stage, alternating champion picks and banning up to five champions per team. During this stage, communication is limited to text chat. The usernames are anonymized (i.e., replacing the name with aliases) to prevent queue dodging by checking third-party stats sites such as OP.GG\footnote{https://www.op.gg/}, leaving the chat as the only option to inform individual strengths and preferences. 

Team composition in \textit{LoL} is crucial to the strategy and outcome of the game~\cite{ong2015player}, setting the basis for future interactions. Most participants acknowledged the importance of balanced and synergistic team composition, especially as players move into higher ranks where team coordination outweighs individual excellence. Yet, we observed a distinct lack of verbal communication between the members during this period across all ranks. Participants attributed the lack of willingness to initiate a conversation on the dangers of starting the game on a bad footing. They prioritized ``not creating friction'' during this stage as negative impressions can propagate throughout the game. Some participants attempted communication to reduce such friction, such as P14, who stated,``\textit{If I had the time, I wanted to say that I will be banning [this Champion], just in case a player on my team wanted to play them.}'' However, several participants viewed any communication during the pre-game phase with wariness, as dissatisfaction or conflict at this step portended negative interactions between players in the game (P3, P9, P15). Thus, even when participants expressed doubt about other teammates' unconventional or non-meta champion picks, they refrained from entering into discourse. This contrasts with findings by Kou and Gui~\cite{kou2014}, which showed players attempt to maintain a harmonious and constructive atmosphere through greetings and introductions.

Another emergent code of the reason for not engaging in communication in the pre-game stage stems from different purposes of playing the game (P1, P5, P13, P16, P17). Despite being in ranked mode, which is more prone to increased competitiveness and effort, participants showed differing goals and levels of interest in winning the game. Several players stated that they had previously exerted great mental load in coordinating synergistic plays, but stopped as they gave less importance to winning at all costs (``\textit{I don't really play to win. I play \textit{LoL} to relieve stress, so I don't engage in chat.}'', P5). These players saw verbal communication with the goal of coordination as an unnecessary or even cumbersome component of the pre-game stage.


\subsubsection{Structured phase}
In many MOBAs, including \textit{LoL}, the early stages of the game play out in a formulaic manner: players join their lanes (Top, Mid, and Bot/Support), defeat minions to gain gold, buy items towards certain ``builds'', kill or assist in early objectives (Jungle), and battle counterparts in their respective lanes. Participants at this stage expressed that most players possessed tacit knowledge of what must be done, such as knowing when to aid their Jungle to capture a jungle monster, choosing the opportune moments to leave their lanes, or positioning wards (i.e., a deployable unit which provides a vision of the surrounding area) at the ideal placements. The participants assumed each player knew their ``role'' to fulfill, often comparing it to ``doing their share'' (P1, P3, P7, P19). In line with this belief, players rarely initiated preemptive or proactive verbal communication for strategic or social purposes at the early stage. 

Pings, on the other hand, constantly permeated the game. At this stage, players used ping to provide information relevant to others from their position, such as letting others know if an enemy went missing from their lane. As the players are largely separated and independent from one another, pings (coupled with the minimap and scoreboard) served as the primary channel for maintaining context over the game rather than as warnings or direct guidance to the players. For other non-verbal gestures, while objective votes would occasionally appear, they were rarely answered. Instead, relevant players near the objective would place pings or move toward it to help out their teammates.

Participants viewed the structured phase as a routine, but uncertain period of the game where the pendulum could swing in either team's favor. Players --- especially Jungles who roam the board looking for opportunities to ambush the enemy team in lanes (``gank'') --- sometimes felt hesitant to make calls and demands at this stage since ``\textit{[they] could make a call, but if I fail, they'll start blaming my decisions down the line.}'' (P7) But at this stage, participants believed that they held personal agency over the final game outcome. P1 and P6 stated that they entered the game with the mindset that only they had to succeed regardless of the performance of their teammates. This belief was reflected in their chatting behavior, where players prioritized focusing on their circumstances over the team's (``\textit{I mute the chat so that I don't get swayed by the team, as I can win the game if I do well.}'', P9).


\subsubsection{Group engagement phase}
As the game enters its middle phase, it provides opportunities for more diverse decision-making. Players may swap lanes, seize or trade crucial objectives, and fight in large battles involving multiple champions. At this point, teams typically have a clear outlook on which players and team have the advantage, requiring more team-driven decisions to maintain or overcome their current standing. Thus, players used verbal communication to discuss more complicated tactics that could not be effectively conveyed through pings.

But more often than not, chat messages became judgment-based. As enemy engagement with larger groups occurred more frequently, the availability for chatting would come after death, which led to comments on past actions rather than future choices. Additionally, the respawn timer for deaths becomes longer as the game progresses, providing more time to observe other players than in earlier phases. This gave players more opportunities to express dissatisfaction specifically towards certain plays, such as placing Enemy Missing pings on the map where other teammates are located to bring attention to their questionable play.

This stage also gave much more exposure of each other to the allies as the team would gather at a single point, giving way to greater scrutiny by their teammates. Repeated or critical mistakes put participants on edge, as they braced for criticism from their teammates. They expressed relief or surprise when the chat remained silent or civil, with P8 stating ``\textit{I messed up there. No one is saying anything, thankfully.}''


\subsubsection{Point of no return}
Meanwhile, verbal communication flowed out when the game had a clear trajectory to the end. Previous research has shown that both toxic and non-toxic communication skyrockets near the end of the game~\cite{kwak2015linguistic} when the players have determined the game outcome with certainty. We saw that this phase opened up both positive and negative sides of communication for guaranteed win and loss, respectively. The winning team would compliment and cheer each other through chat messages and emotes, while the losing side devolved into arguments and calling out. The communication at this stage was driven by emotion, showing excitement or venting frustration.


\subsection{Communication Assessment Process}

We describe the factors that users mainly focus on to assess when or when not to involve themselves in communication with their teammates. 

\subsubsection{Calculating communication cost}
One of the most proximate factors behind when communication is performed is the limited action economy of the game. In \textit{LoL} and other MOBAs, players can rarely afford time to type out messages due to the fast-paced nature of the game. In time-sensitive scenarios, the time pressure makes communication particularly costly. It is therefore unsurprising that much of the communication occurs after major events (e.g., battles and objective hunting), as players are given more downtime while waiting for teammates or enemies to respawn or regroup.

For periods where players were still actively involved in gameplay, the players made conscious decisions on choosing which communication media to use based on the perceived action availability and the importance of communicating the message. Players relied on pings for non-critical indications, believing that the mutual understanding of the game would get the message across. However, many players recognized that pings were prone to be missed, ignored, or misinterpreted by their allies (P2, P9, P16, P17, P20). Subsequently, participants typed out information considered to be too important to the situation to be misunderstood or missed by other players even if it caused delays in their gameplay (P10, P11, P14). Simultaneously, the priority of importance constantly shifted --- we observed multiple times participants start to type, but stop to react to an ongoing play, only to never send out their message.

\subsubsection{Evaluating relevance and responsiveness}
When the brief window of communication opportunity is missed, players are unlikely to ever send out that information. In \textit{LoL}, situations can change within seconds and certain communication media cannot keep up with the changing state of the game. For example, almost all study participants did not participate in votes for objectives. Among the tens of objective votes initiated among all the games in this study, no objective vote saw more than three votes, frequently being left with no vote beyond the player who initiated the vote. Some players, when asked why they did not participate, stated that the votes they made often became irrelevant as the game state had changed during the time it took to vote (P2, P11). Other players also discussed how information conveyed through communication can get outdated fast (P1, P8, P9). 

\begin{quote}
I can't always follow through with what I say [in the chat] since the game is really dynamic. My teammates don't understand such situations, so I tend to not chat proactively. - P9
\end{quote}

Thus, some players instead preferred to react through direct action (P8, P10, P11, P16, P20). P10 stated, ``\textit{I think it's enough to show through action rather than [using objective voting]. I can look out for how the player reacts when I request something from them.}''

On the other hand, such action-based responses left the player to assess whether and how the communication was received. P10 stated that they tried to predict whether a player understood their ping direction by how they moved, but it was hard to interpret their intent: ``\textit{members sometimes seem to move towards me but then turn around, and sometimes they even ping back but don't come.}''. P16 discussed how they weren't sure whether the ping was received, but performed it anyway since it felt helpful.

Similarly, participation in surrender votes (or lack thereof) carried different intent by the player. During most of the games that ended in a loss, one or more surrender votes were called by the participant's team. However, only two surrender votes achieved four or more players' participation. However, the reasons why a player chose to not participate varied. Some had decided to wait and see how other teammates voted, which may have paradoxically led many members to not participate in the vote (P4, P9). Meanwhile, others didn't reply as they didn't think the vote was actually calling for a response: P13 stated, ``\textit{I didn't vote because they were just showing their anger. It's just a member venting through a surrender vote that they're not doing well.}''

\subsubsection{Balancing information access and psychological safety}
While recognizing that communication would be useful or even necessary in certain situations, participants also put their psychological safety first over information access. Some players, worn down by the normalization of toxic communication such as flaming, muted the chat (P1, P9).

Many participants expressed the sentiment of ``protecting [their] mentality'', describing how certain communication harmed their psychological well-being. This communication did not always refer to negative communication; P9 often muted players who gave commands as they did not want to be ``swept up'' by others' play-related judgments. This separation even extended to other more widely considered essential communication forms, such as pings. Even after acknowledging that pings were vital and useful to the game, P9 went as far as muting the ping of the support player in the same lane after they sent a barrage of Enemy Missing pings that signified aggression and criticism. 

Additionally, the abundance and high frequency of communication also strained the limited mental capacity of the players. Many players, when asked why they had not replied to an objective vote or other chat messages, stated that they simply did not notice them among other events happening (P1, P2, P3, P9, P12, P15, P18, P20, P21). The information overload caused stress and became distracting to players.

\subsubsection{Reducing potential friction}
As demonstrated in the pre-game stage of the game, players sometimes used communication to minimize friction between their teammates. Some participants sacrificed time to apologize to other players when they believed themselves to be at fault. When asked why, P12 replied, ``\textit{There are too many people who don't come to help gank if I don't apologize.}''. Similarly, P5 sacrificed time typing in an apology after a teammate had died despite still being in the middle of a fight as they didn't wish to give the other player a reason to start an attack.

However, some noted that silence is sometimes the best answer to a negative situation. P4, after dying to the enemy, put into chat ``Fighting!'' (roughly meaning, ``We can do it!''). They stated ``\textit{I don't know why I do it... it probably angers [my teammates] more.
}'' They also stated that ``\textit{for certain people, talking in the chat only spurs them more. You just have to let them be.}'' Other players shared similar sentiments that being quiet and dedicating focus to the game was a better choice (P1, P11, P14).

For female players, the fear of gender-based harassment shaped their communication patterns. While \textit{LoL} does not provide any demographic information of a player to other players, almost all female participants noted experiences of receiving derogatory remarks or doubts about their abilities based on other players' assumptions of their gender, a trend frequently seen in male-dominated online gaming cultures~\cite{fox2016women, norris2004gender, mclean2019female}. They noted that the players were able to correctly guess their gender when the participant's role and champion fit into the preconceived notions of what women ``tended to play'' (i.e., female-identifying support champions, such as Lulu) or their username ``seemed feminine'' (P18, P19, P20, P21, P22). This led to certain players adopting tactics that signaled male-like behavior, such as changing their speech style to be more gender-neutral or male-like (P19, P21) and changing their username to sound more gender-neutral. Cote describes similar instances of ``camouflaging gender'' as one of five main strategies for women coping with harassment~\cite{cote2017coping}. However, some players opted to keep playing their preferred character or maintaining their username even if it signaled their gender, such as P21 who expressed, ``\textit{I cherish and feel attached to my username, so I don’t want to change it just because of [harassment and inappropriate comments].}'' These players valued self-expression and identity even at the risk of increased risk to unpleasant communication experiences.


\subsubsection{Forming performance-based hierachy}
Naturally formed leadership has often been observed in other works on \textit{LoL} teams~\cite{kou2014}. Kim et al. showed that more hierarchy in in-game decision-making led to higher collective intelligence~\cite{kim2017}. While they used ``hierarchy'' to mean varying amounts of communication throughout the game, we observed that the hierarchy extends further to performance-based hierarchy, where teammates in more advantageous positions are given greater weight when communicating with other players. Players actively chose to refrain from suggesting strategic plans when they were ``holding down the team'', recognizing that they held less power and trust among the team members (P8, P10, P12, P14, P22). The player who was losing against the enemy team was viewed as having no ``right'' to lead the team, which was reserved for well-performing players.


\subsubsection{Enforcing norms and habits}
One of the most common answers to why players performed certain communication actions, especially non-verbal actions such as pings and emotes, was ``a force of habit'' (P6, P7, P8, P9, P10, P12, P17). Players formed learned practices of using communication channels at certain points by observing other players exhibit the same behaviors. This promoted, for example, replying to an emote sent by the teammate with their own or pinging readied skills and items to emphasize relevant information for other players throughout the game. 

On the other hand, this meant that players were averse to communication patterns outside of the norm --- participants stated that they had a hard time adapting to new forms of communication, seeing no immediate benefit or impact from using them (P1, P8, P14, P13, P15, P17). Most egregiously, the recently introduced objective pings were largely viewed to be awkward to use and unnecessary (P1, P4, P8, P12).


\subsection{Impact of Communication Assessment}
We describe how the communication patterns and assessment of the players impact the individual players' perspectives on team dynamics.

\subsubsection{Relationship between trust and communication frequency}
Most participants saw value in constant and well-informed communication but with an important distinction: verbal communication with strangers rarely ended well. Players largely recognized frequent verbal communication to burgeon conflict, regardless of the message within. Even when players understood the helpful intent behind positive messages from the players, they compared actively talking players to be possible bad actors who were likely to exhibit toxic behaviors when the game turned against them. (P1, P4, P8, P12, P14)

\begin{quote}
I need to make sure to not disturb Twisted Fate. I saw him start to flame. It's not because I don't want to hear more criticism. I know these types. The more I react and chat with them, the more deviant they will become. - P4  
\end{quote}

Similarly, P19 lamented that players used to socialize more in the chat during the pre-game phase to build a fun and prosocial environment, noting a memorable example of encouraging each other to do well on their academic exams, but noted that such prosocial behavior has become much rarer during the recent seasons. They noted that there are inevitably players ``who take it negatively'' and thus stopped proactively typing non-game related messages in the chat.

Ultimately, players desired assurance and trust of player commitment. The participants trusted actions more than words to prove that the player remained dedicated to the game. Both P10 and P17 pointed out that it was easy to tell who was still ``in the game'' and motivated to try their best and that ``staying on the keyboard'' likely meant that they weren't invested or focused on the game. Players viewed such commitment to be the most important aspect of a ``good'' teammate, sometimes even more than their skill or performance (P9, P14). It is interesting to note that unlike what previous literature may suggest~\cite{marlow2018}, players' averseness to talkative teammates had less to do with the cognitive overload or distraction caused by the frequent communication, but rather due to the threats of future team breakdown. This view in turn also affected how players decided to communicate or not, as they believed that players would not take their suggestions or comments in a positive light. 


\subsubsection{Perception of player commitment and fortitude}

Communication also acted as a mirror of their teammates' mental fortitude. A number of players mentioned how they valued a resilient mindset in their teammates playing the game, referring to players who remained committed to the game until the very end. They saw players who provoked or complained to teammates as ``having a weak mentality'' who had been altered by the bad outcomes of the game to act in an unhelpful manner towards the team through their communication. The communication actions of the teammate informed the participants of how steadfast their teammate remained in disadvantageous situations.  

\begin{quote}
It's not like I constantly reply in the chat or anything, but I pay attention [to the chat] to grasp the overall atmosphere of the team. If the team doesn't collaborate well then we lose, so I try to have a rough understanding of the mentality of the other players. - P13
\end{quote}

There were also instances of communication that helped players maintain a positive view of their teammates. For example, P11 mentioned near the beginning of the game, ``\textit{Looking at the chat, Varus player has strong mentality [for being so positive]. There were lots of points [in his support's] plays that he could have criticized.}'' Unfortunately, this view quickly soured when the Varus player devolved into criticism later in the late game phase where the Varus player started criticizing the support and other players. P11 then noted that the Varus player seemed to merely be ``bearing through the game''.
\section{Conclusions}
\label{sec:conclusions}

In this work, we proposed \plangen{}, an easily scalable multi-agent approach incorporating three key components: constraint, verification, and selection agents. We leveraged these agents to improve the verification process of existing inference algorithms and proposed three frameworks: Multi-Agent Best of $\mathcal{N}$, ToT, and REBASE. Further, we introduced a Mixture of Algorithms, an iterative framework that integrates the selection agent (Figure \ref{fig:teaser}) to dynamically choose the best algorithm. We evaluated our frameworks on NATURAL PLAN, OlympiadBench, GPQA, and DocFinQA. Experimental results demonstrate that \plangen{} outperforms strong baselines, achieving SOTA results across datasets. Furthermore, our findings suggest that the proposed frameworks are scalable and generalizable to different LLMs, improving their natural language planning ability.


\section*{Limitations}

Despite the strong performance of our frameworks, an area of improvement is the reliance on predefined heuristics for selecting inference-time algorithms, which may not always generalize optimally across all tasks and domains. Additionally, while our frameworks demonstrate strong performance, their computational overhead could be further optimized for efficiency in real-world applications. We believe that our frameworks can be useful in further boosting the planning and reasoning capabilities of existing models such as o1 and Gemini-thinking. In addition, the use of reinforcement learning or meta-learning techniques to dynamically adapt agent strategies based on task complexity could be an interesting area to explore. Moreover, broadening the scope to multi-modal and multi-lingual reasoning would significantly expand the applicability of our approach, and exploring the use of generated planning trajectories for model training offers valuable direction.

\section*{Ethics Statement}

The use of proprietary LLMs such as GPT-4, Gemini, and Claude-3 in this study adheres to their policies of usage. We have used AI assistants (Grammarly and Gemini) to address the grammatical errors and rephrase the sentences.

% \section*{Acknowledgments}

%%%%%%%%%%%%%%%%%%%%%%%%%%%%%%%%%%%%%%%%%%%%%%%%%%%%%%%%%%%%%%%%%%%%%%%%%%%%%%%%
% Acknowledgement
\section*{Acknowledgments}

The authors gratefully acknowledge the \acs{compas} laboratory at \acs{mbari} engineering and technical team.

%%%%%%%%%%%%%%%%%%%%%%%%%%%%%%%%%%%%%%%%%%%%%%%%%%%%%%%%%%%%%%%%%%%%%%%%%%%%%%%%
% This command balances the column lengths on the last page of the document manually. It shortens the text height of the last page by a suitable amount.
% This command does not take effect until the next page, so it should come on the page before the last. Make sure that you do not shorten the text height too much.

\addtolength{\textheight}{-8.75cm}   

%%%%%%%%%%%%%%%%%%%%%%%%%%%%%%%%%%%%%%%%%%%%%%%%%%%%%%%%%%%%%%%%%%%%%%%%%%%%%%%%
% References
\bibliography{refs/IEEEabrv, refs/DRL}
\bibliographystyle{IEEEtran}

\end{document}


