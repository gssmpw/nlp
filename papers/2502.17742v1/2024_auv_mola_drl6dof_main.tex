\documentclass[conference]{sty/IEEEtran}

\usepackage{amsmath,amsfonts,amssymb}
\usepackage{algorithmic}
\usepackage{algorithm}
\usepackage{array}
\usepackage[caption=false,font=normalsize,labelfont=sf,textfont=sf]{subfig}
\usepackage{textcomp}
\usepackage{stfloats}
\usepackage{url}
\usepackage{verbatim}
\usepackage{booktabs}
\usepackage{multirow}
\usepackage{rotating}
\usepackage{makecell}
\usepackage{cite}
\usepackage{xspace}
\usepackage{bm}
\usepackage{sty/my_acronyms}
\usepackage{siunitx}
\usepackage{tikz}
\usepackage{sty/my_acronyms}
\usetikzlibrary{shapes,arrows,calc}
\hyphenation{op-tical net-works semi-conduc-tor IEEE-Xplore}
\newcommand{\RomanNumeralCaps}[1] {\MakeUppercase{\romannumeral #1}}

% Prevent line-breaking inline formulas
\relpenalty=99999
\binoppenalty=99999

% Required to identify funding agencies
\IEEEoverridecommandlockouts

% Paper title
\title{\LARGE \bf
Toward 6-DOF Autonomous Underwater Vehicle Energy-Aware Position Control based on Deep Reinforcement Learning: Preliminary Results
\thanks{(*) These authors contributed equally. This work was supported by the David and Lucile Packard Foundation and the ANID-Chile grants for doctoral program N°21200358 and master program N°22231977.}
}

\author{\IEEEauthorblockN{Gustavo Boré${}^*$}
\IEEEauthorblockA{\textit{Pontificia Universidad Católica de Chile} \\
Santiago, Chile \\
Email: gibore@uc.cl}
\\ % Line break to align authors
\IEEEauthorblockN{Sebastián Rodríguez-Martínez}
\IEEEauthorblockA{\textit{Monterey Bay Aquarium Research Institute} \\
Moss Landing, California 95039--9644\\
Email: srodriguez@mbari.org}
\and
\IEEEauthorblockN{Vicente Sufán${}^*$}
\IEEEauthorblockA{\textit{Pontificia Universidad Católica de Chile} \\
Santiago, Chile \\
Email: vicente.sufan@uc.cl}
\\ % Line break to align authors
\IEEEauthorblockN{Giancarlo Troni}
\IEEEauthorblockA{\textit{Monterey Bay Aquarium Research Institute} \\
Moss Landing, California 95039--9644 \\
Email: gtroni@mbari.org}
}

\begin{document}

\maketitle
\thispagestyle{empty}
\pagestyle{empty}

\begin{abstract}
The use of \acp{auv} for surveying, mapping, and inspecting unexplored underwater areas plays a crucial role, where maneuverability and power efficiency are key factors for extending the use of these platforms, making \ac{6dof} holonomic platforms essential tools. Although \ac{pid} and \acl{mpc} controllers are widely used in these applications, they often require accurate system knowledge, struggle with repeatability when facing payload or configuration changes, and can be time-consuming to fine-tune. While more advanced methods based on \acl{drl} have been proposed, they are typically limited to operating in fewer degrees of freedom.
%
This paper proposes a novel \acs{drl}-based approach for controlling holonomic \ac{6dof} \acp{auv} using the \ac{tqc} algorithm, which does not require manual tuning and directly feeds commands to the thrusters without prior knowledge of their configuration. Furthermore, it incorporates power consumption directly into the reward function. Simulation results show that the \acl{tqc-hp} method achieves better performance to a fine-tuned \ac{pid} controller when reaching a goal point, while the \acl{tqc-ea} method demonstrates slightly lower performance but consumes \SI{30}{\percent} less power on average.
\end{abstract}

% Reset acronyms
\acresetall

%%%%%%%%%%%%%%%%%%%%%%%%%%%%%%%%%%%%%%%%%%%%%%%%%%%%%%%%%%%%%%%%%%%%%%%%%%%%%%%%
% Sections

\begin{figure}[ht]
    \centering
    \includegraphics[width=0.8\linewidth]{graphs/greater_than_naive.pdf}
    \vspace{0.5cm}
    \includegraphics[width=0.8\linewidth]{graphs/p1_bottom.png}
    \vspace{-5pt}
    \caption{\textcolor{positional}{Positional} vs.\ \textcolor{nonpositional}{non-positional} circuits. In a \textcolor{nonpositional}{non-positional} circuit, the same edges must be included at all positions. A \textcolor{positional}{positional} circuit can distinguish between the same edge at different positions. This specificity yields better trade-offs between circuit size and faithfulness. It can also increase both precision and recall.}
    \label{fig:p1}
    \vspace{-5pt}
\end{figure}

\section{Introduction}

\looseness=-1
A primary goal of interpretability research is to characterize the internal mechanisms in language models (LMs) and other NLP models. 
A core approach in this area is \textbf{circuit discovery}---identifying the minimal subgraph within the model's computation graph that performs a specific task \citep{olah2021framework,olah-mech}.
Typically, the nodes of a circuit represent model components (e.g., attention heads, neurons, or layers).
While manual circuit discovery methods can yield position-specific insights \citep{wanginterpretability,goldowskydill2023localizingmodelbehaviorpath}, \emph{automatic methods often overlook positional information}, treating components as uniformly relevant across all input token positions \citep{conmytowards,syed2023attribution}. 
For instance, if an attention head is included in a circuit, it is assumed to contribute equally to the computation for every position in the input sequence.
The assumption that circuits are position-invariant ignores the fact that different positions often require distinct computations.
By ignoring positions, current methods limit their ability to capture mechanisms that operate across positions, such as interactions between attention heads across positions.

In this study, we start by demonstrating that positional agnosticism is a significant limitation (\S\ref{sec:motivating}). Then, to address these limitations, we introduce a new approach: position-aware edge attribution patching (PEAP; \S\ref{sec:full_circ_discovery}; Figure~\ref{fig:p1}). Current approaches  assume that if an edge is in a circuit, then the same edge will be in the circuit at all positions, thus leading to low precision. It is also assumed that an edge's importance should be aggregated across positions before deciding whether it should be included in the circuit; this can lead to cancellation effects, and thus low recall. PEAP instead allows us to compute the importance of cross-positional edges, and separately evaluates edge importance at each position. We show that this leads to smaller and more accurate circuits; see Figure~\ref{fig:p1}.

Incorporating positional information into circuit discovery is straightforward when inputs have the same length and structure across examples.

However, realistic datasets are not nearly this templatic.
How, then, can we incorporate positional information into automatic circuit discovery?
To address this challenge, we propose \textbf{schemas} (\S\ref{sec:schema}). 
Schemas assign semantic labels to spans of tokens, enabling information aggregation across examples even when the spans differ in length.

For example, in the input ``The \textcolor{positional}{war} lasted from 1453 to 14\underline{\hspace{1em}},'' the span ``\textcolor{positional}{war}'' could be labeled as ``\emph{Subject}''.
This enables handling spans with varying lengths: the phrase ``\textcolor{positional}{Black Plague}'' in another example can be treated as a single positional span with the same role as ``\textcolor{positional}{war}''.
In experiments with two LMs and three tasks, we find that circuits discovered using schemas achieve a better trade-off between circuit size and faithfulness to the model's behavior than position-agnostic circuits.
Importantly, position-aware circuits offer a more precise representation of the underlying mechanisms, providing a more concise foundation for mechanistic explanations.

We also present a fully automated pipeline for schema generation and application (\S\ref{sec:schema-generation}) using large language models (LLMs). 
We evaluate the quality of the generated schemas and their utility in discovering position-aware circuits (\S\ref{sec:schema-eval}).
Notably, circuits derived using automatically generated and applied schemas achieve comparable faithfulness scores to circuits discovered with human-designed and manually applied schemas.

We summarize our contributions as follows:
\begin{itemize}[noitemsep,leftmargin=*,topsep=1pt,parsep=1pt]
    \item Introduce a position-aware circuit discovery method, which obtains better faithfulness than position-agnostic discovery.  
    \item Introduce dataset schemas,  facilitating positional circuit discovery in more naturalistic settings. 
    \item Develop an automated schema generation and application pipeline with LLMs, yielding schemas that are comparable to manually-annotated ones.
\end{itemize}

\section{Background of Cost Estimation} \label{sec:background}
This section first gives a brief overview of classical and learned cost estimation. 
Afterwards, we describe the learning procedure of \lcms and provide a taxonomy that guides our selection of recent \lcms for this study in \Cref{sec:methodology}.

\subsection{Traditional \& Learned Cost Estimation}
\textbf{Traditional Cost Estimation.} 
Precise cost estimates for different plan candidates in a database are crucial for the query optimizer to select optimal plans from a large search space.
Thus, a lot of engineering effort has been spent since the beginning of database development to estimate the execution costs of a query plan.
Most database systems such as MySQL \cite{widenius2002}, Oracle, PostgreSQL, or System R \cite{astrahan1976} use hand-crafted cost models to reason about the execution costs of a query plan.
These models typically provide a cost function for each physical operator in a query plan that estimates its runtime costs according to CPU usage, I/O operations, memory consumption, expected tuples, and random or sequential page accesses.
However, due to the wide variety of data, queries, and data layouts, traditional cost models need to make simplifying assumptions (e.g., independence of attributes).
These often lead to incorrect predictions of the execution cost. 
Consequently, the query optimizer makes sub-optimal decisions that degrade the query performance by increasing its runtime \cite{leis_how_2015}.
%-------------------------------------------------------

\noindent\textbf{Learned Cost Estimation.}
The need to improve prediction accuracy and the rise of machine learning motivated the idea of \lcms. 
The main idea is to approximate the complex cost functions with a learned model.
Generally, a typical model learns from previous query executions to predict execution costs like runtime.
In contrast to traditional cost models, the promise of \lcms is that they can better learn arbitrarily complex functions.
Thus, improved prediction accuracy can be expected in contrast to traditional approaches based on simplifying assumptions.
Overall, the higher accuracy is expected to lead to a selection of query plans with improved query performance.

%-----------------------------------------
\subsection{Learning Procedure of \lcms}
For our study, we look at effects that also result from the learning procedure of \lcms.
As such, we briefly review the traditional procedure as depicted in \Cref{fig:learning_procedure} to provide the necessary background:
\circles{A}~At first, a workload generator is used to create a large set of randomized, synthetic SQL-Strings that involve a variety of representative query properties such as filter predicates, joins, or aggregation types.
\circles{B}~These queries are executedses (e.g., an airline or movie database) to collect the actual costs of queries.
An important aspect here is that training procedures of many \lcms leads to biases in the dataset due to timeouts and pre-optimized queries, as discussed later.
\circles{C}~Next, various information is extracted from the workload execution.
Most importantly, the physical query plans are extracted, which serve as input to cost models.
In addition to physical plans, \lcms require different information, such as data characteristics like histograms or sample bitmaps.
\circles{D}+\circles{E}~Finally, the workload (i.e., plans and runtime) is then split for training and testing the \lcms. 

\begin{figure*}
    \centering
    \includegraphics[width=\linewidth]{./figures/training_procedure.pdf}
    \caption{
    Learning procedure of \lcms. 
    \circles{\textsc{A}} Generation of synthetic training queries. \circles{\textsc{B}} Query execution on training databases. 
    \circles{\textsc{C}} Feature (query plans, data characteristics, and sample bitmaps) and label (query runtimes) extraction to generate the training and test dataset. 
    \circles{\textsc{D}} Training of the \lcm with supervised learning. 
    \circles{\textsc{E}} Evaluation of the \lcm against unseen test data.}
    \label{fig:learning_procedure}
\end{figure*}

%------------------------------------------------------
\subsection{Taxonomy of \lcms} \label{subsec:taxonomy}
\lcms developed in the last years differ in various dimensions.
This section provides a brief taxonomy of recent \lcms to structure the different methodological approaches. 
This taxonomy will guide the selection of \lcms that we use in this study and ensure that we cover the different methodologies to analyze how they affect the ability of \lcms to support query optimization.

\noindent\textbf{Input Features.}
The first crucial dimension is the input features that a \lcm learns from.
The input features are extracted from the executed workloads (cf. \Cref{fig:learning_procedure}\circles{C}).
The query plan and the underlying data distribution need to be modeled so that a \lcm is informed to make reasonable predictions about the execution costs, which in turn affects query optimization, as we will show.
However, \lcms make use of different information for cost estimation.

\begin{enumerate}[leftmargin=*, nosep]
\item \textbf{SQL-String vs. Query Plans}: 
Some of the first models rely on the SQL string to describe a query, as it gives insights about the tables, predicates, and joins. 
However, details of the execution plan, such as physical operators or the order of joins, are not described there. 
Thus, most \lcms utilize the physical query plan, which includes the operators (e.g., scans, joins) and physical operator types  (e.g., nested loop vs. hash join).
As we will see later, this is fundamental for query optimization.

\item \textbf{Cardinalities}:
Intermediate cardinalities are an important input signal for the overall cost of a plan as they denote the number of tuples an operator needs to process \cite{leis_how_2015}.
Thus, many \lcms leverage intermediate cardinalities as input features, which are either annotated by the databases' cardinality estimator or obtained through an additional learned estimator from related work \cite{hilprecht2020deepdb, kipf2019, yang2020}.
While some \lcms also ignore cardinalities as input for cost prediction, we show in our study that they, in fact, improve the usefulness of cost estimates from \lcms for various query optimization tasks.

\item\textbf{Data Distribution}:
Another helpful factor in estimating cost is understanding the data distribution in the base tables, especially if no cardinalities are used.
For instance, the fact of how many distinct values exist in a column might influence the efficiency of physical operators (e.g., hash join). 
As such, some \lcms use data distribution represented as database statistics and histograms or sample bitmaps (which we explain later) from the base tables as inputs.
However, as we will show in our study, their effect on query optimization tasks remains unclear.

\item \textbf{Cost Estimates}: 
Finally, some of the most recent \lcms even leverage the cost estimates provided by a classical cost estimator as an input feature, which serves as a strong input signal.
This idea renders these \lcms to \textit{hybrid} as they combine a traditional cost model with a learned approach.
The study shows that this provides significant benefits.
\end{enumerate}

%-----------------------------------------------------
\noindent \textbf{Query Representation.}
Many \lcms use model architectures use a graph-based representation to encode query plans as input to the models\footnote{The graph-based representation of the queries refers to the fact whether a model leverages the query graph structure and not to the model learning architecture itself.}.
These approaches thus explicitly leverage information about the order (parent-child relationships) of operators in plans.
However, other \lcms \cite{kipf2019, akdere2012} represent a query plan (or the SQL string) as a flat vector of fixed size without modeling the operator dependencies, which we refer to as \textit{flat} representation in this paper.
While intuitively, capturing the structure and not using a flat representation should be beneficial for \lcms, the results of using graph structure in this study are not that clear. 

%-------------------------------------------------------
\noindent \textbf{Database Dependency.}
Furthermore, an important aspect is whether \lcms can generalize to unseen databases (i.e., a new set of tables) or not.
\textit{Database-agnostic} \lcms were designed \cite{hilprecht2022, zibo_liang_dace_2024} to enable cost predictions for unseen databases that were not part of the training data.
This approach has the advantage of directly providing results without requiring database-specific training data. 
In contrast, \textit{database-specific} \cite{sun2019, zhao2022, marcus2019} models cannot generalize for unknown databases.
For this study, an interesting question is if one of these classes is better suited to support query optimization tasks as database-specific can better adapt to one single database while database-agnostic models can generalize better.

%----------------------------------------
\noindent \textbf{Model Architecture.}
Finally, the presented \lcms differ largely in their learning approach.
Various learning architectures were proposed, including decision trees, tree-structured neural networks, neural units, graph neural networks, and transformer architectures.
While different architectures show different results on the cost estimation tasks, it is still open to see which architecture provides the best results for query optimization.
%----------------------------------------
%!TEX root = 2024_auv_mola_drl6dof_main.tex
%%%%%%%%%%%%%%%%%%%%%%%%%%%%%%%%%%%%%%%%%%%%%%%%%%%%%%%%%%%%%%%%%%%%%%%
\section{Proposed Approach}
\label{sec:proposed_approach}

In this work, we introduce \acf{tqc-hp} and \acf{tqc-ea}: two end-to-end \ac{drl}-based approaches for the low-level control of a holonomic \ac{6dof} \ac{auv} using the \ac{tqc} algorithm. These methods require neither manual tuning nor prior knowledge of the thruster configuration.

%%%%%%%%%%%%%%%%%%%%%%%%%%%%%%%%%%%%%%%%%%%%%%%%%%
%% Action Space
\subsection{Action space}

The action space $\bm{a}(t) \in \mathbb{R}^{8}$, normalized between $[-1, 1]$, is designed to enable the agent to generate precise control commands for the vehicle’s thrusters, ensuring the desired movement and orientation based on the given state inputs. The action space is defined as:
%%%
% Keep to remove space between equations and paragraph
%%%
\begin{equation}
\bm{a}(t) = \begin{bmatrix}
T_{1}^t & T_{2}^t & T_{3}^t & T_{4}^t & T_{5}^t & T_{6}^t & T_{7}^t & T_{8}^t
\end{bmatrix}^T,
\label{eq:action_space}
\end{equation}

\noindent where $\bm{a}(t)$ represents the normalized \ac{pwm} signals sent to the eight thrusters at time $t$.

%%%%%%%%%%%%%%%%%%%%%%%%%%%%%%%%%%%%%%%%%%%%%%%%%%
%% Observation Space
\subsection{Observation space}

The observation space $\bm{s}(t) \in \mathbb{R}^{20}$, normalized between $[-1, 1]$, is designed to capture all relevant environmental information necessary for the agent’s decision-making, enabling an understanding of the system’s current state relative to the desired goal. The observation space is defined as:
%%%
% Keep to remove space between equations and paragraph
%%%
\begin{equation}
\bm{s}(t) = \begin{bmatrix}
\bm{e}(t) & \bm{v}(t) & \bm{a}(t-1)
\end{bmatrix}^T,
\label{eq:observation_space}
\end{equation}

\noindent where $\bm{e}(t)=[e_{x}^t, e_{y}^t, e_{z}^t, \theta_{x}^t, \theta_{y}^t, \theta_{z}^t]^T$ represents the error vector at time $t$ between the current pose and the goal pose for each \ac{dof}, $ \bm{v}(t)=[v_{x}^t, v_{y}^t, v_{z}^t, \omega_{x}^t, \omega_{y}^t, \omega_{z}^t]^T$ corresponds to the spatial twist of the vehicle at time $t$, composed by the linear and angular velocities, and $\boldsymbol{a}(t-1)$ the action in the previous time step.

\begin{table}[b!]
\centering
\caption{Weights $\alpha_i$ for reward functions for \ac{tqc-hp} and \ac{tqc-ea}}
\label{tab:weights}
\begin{tabular}{ccccccccc}
\toprule
 & $\bm{\alpha_1}$ & $\bm{\alpha_2}$ & $\bm{\alpha_3}$ & $\bm{\alpha_4}$ & $\bm{\alpha_5}$ & $\bm{\alpha_6}$  \\ \midrule
\textbf{\ac{tqc-hp}} & $-4$ & $-4$ & $-3$ & $-1.8$ & $-1$ & $0$ \\
\textbf{\ac{tqc-ea}} & $-4$ & $-4$ & $-3$ & $-1.7$ & $-0.8$ & $-0.3$  \\ \bottomrule
\end{tabular}
\end{table}

%%%%%%%%%%%%%%%%%%%%%%%%%%%%%%%%%%%%%%%%%%%%%%%%%%
%% Reward Function
\subsection{Reward function}

To evaluate policy performance and provide feedback during the training stage, an optimal reward function must be defined to enable the model to learn the desired behavior. In this work, we define the reward function \eqref{eq:reward_fx} that aims to bring the \ac{auv} closer to the target position in each of the \ac{6dof}, linearly penalizing the agent for positional \eqref{eq:reward_position} and angular \eqref{eq:reward_attitude} errors over time. Additionally, the reward function aims to generate smoother commands for the thrusters by penalizing signal fluctuations \eqref{eq:reward_smoothness} and to minimize the utilization of each thruster, thereby reducing the energy consumption of the \ac{auv} \eqref{eq:reward_power}. Each term is weighted with the parameter  $\alpha_i \leq 0$ to denote the importance of each term within \eqref{eq:reward_fx}.
%%%
% Keep to remove space between equations and paragraph
%%%
\begin{subequations}
\begin{align}
r(t) &= \sum\nolimits_{i=1}^{4} r_i(t) \label{eq:reward_fx} \\
r_1(t) &= \alpha_1 \cdot{|e_x^t|} + \alpha_2 \cdot{|e_y^t|} + \alpha_3 \cdot{|e_z^t|}\label{eq:reward_position} \\
r_2(t) &= \alpha_4 \cdot{|\theta^t|}
\label{eq:reward_attitude} \\
r_3(t) &= \sum\nolimits_{i=1}^{8} \alpha_5 \cdot \left|{T_{i}^t - T_{i}^{t-1}} \right| \label{eq:reward_smoothness} \\
r_4(t) &= \sum\nolimits_{i=1}^{8} \alpha_6 \cdot \left|{T_i^t} \right| \label{eq:reward_power}
\end{align}
\end{subequations}

%%%%%%%%%%%%%%%%%%%%%%%%%%%%%%%%%%%%%%%%%%%%%%%%%%
%% DRL Algorithm
\subsection{Proposed \ac{drl} algorithms}

The proposed approaches are based on a \ac{drl} framework using a \ac{tqc} algorithm \cite{kuznetsov2020}. The selection of this algorithm was guided by an experiment conducted to compare the performance of the \ac{tqc}, \ac{sac}, and \ac{td3} algorithms in a \ac{6dof} \ac{auv} position control problem. The methodology described in Section \ref{sec:metodology} was used, and the three methods were trained using the same reward function to compare which of the aforementioned models obtained the highest average reward during the training episodes, indicating superior performance in the specific control task.

Fig. \ref{fig:rewards} shows the average reward of an episode over a moving window of 100 episodes obtained by the three \ac{drl} algorithms during training. It is evident that the \ac{tqc} algorithm converges in a smaller number of episodes compared to \ac{sac} and \ac{td3}, as well as reaching higher reward values than the other two algorithms. This experiment correlates with the results obtained by Lidtke et al. \cite{lidtke2024} for a 3-\acs{dof} \ac{auv} and confirms our decision to use the \ac{tqc} algorithm for the task of controlling an \ac{auv} in \ac{6dof}.

In this work, two different approaches of \ac{tqc} models are proposed: \ac{tqc-hp} aims to reach the goal position in the \ac{6dof} using smooth commands to the thrusters, i.e., its reward function includes the components $r_1$, $r_2$, and $r_3$, while \ac{tqc-ea} has the same objectives as \ac{tqc-hp} but adds energy-awareness, minimizing the energy consumption of the \ac{auv} by incorporating the $r_4$ component in its reward function. Table \ref{tab:weights} presents the $\alpha_i$ weights used in the reward function for both approaches. Both approaches are implemented in \textit{Python3} using the \textit{Stable Baselines3} library, which is based on principles used in both \ac{sac} and \ac{td3} algorithms but utilizes the distributed representation of a critic, truncation of critic predictions, and a set of multiple critics.

%!TEX root = 2024_auv_mola_drl6dof_main.tex
%%%%%%%%%%%%%%%%%%%%%%%%%%%%%%%%%%%%%%%%%%%%%%%%%%%%%%%%%%%%%%%%%%%%%%%
\section{Evaluation Methodology}
\label{sec:metodology}

The proposed methods have been developed in the numerical simulation environment Stonefish  \cite{cieslak2019}, which allows the introduction of underwater physics and environmental conditions together with the vehicle geometry, accounting for the complete vehicle dynamics, including drag effects and added masses. The middleware used to communicate between the vehicle's sensors and the control algorithm is established through the \ac{lcm} library \cite{LCM}, which is the current middleware used by the field platform that can be used for future field testing; therefore, easing the transition from simulation to reality.

\begin{figure}[t!]
\centering
\includegraphics[width=0.48\textwidth]{figures/n_position_vs_time_v.pdf}%
\caption{Position in the $x$, $y$, and $z$ axes and angular distance to the target, $\theta$, over time during one of the evaluation episodes.}
\label{fig:position_vs_time}
\end{figure}

The field platform modeled within the simulation environment (depicted in Fig. \ref{fig:mola}) is \ac{mola} \ac{6dof} \ac{auv}, a research platform used for developing and testing novel research methods to enable the exploration of deep and rugged terrains. This platform is owned and operated by the \acs{compas} Lab at the \acf{mbari}. This system is equipped with an \ac{imu}, \ac{dvl}, depth sensor, cameras, sonar, and eight thrusters, and can perform complete holonomic movements in \ac{6dof}. Currently, \ac{mola} has a fine-tuned double-loop 36-gain \ac{pid} controller, with good performance in the same simulation environment. It has been extensively validated in the physical platform during tank tests and is used as the proper performance comparison for the \ac{drl} controller.

Both \ac{tqc-hp} and \ac{tqc-ea} are trained within the simulation environment in over $2.5\times10^6$ steps, starting from a random pose, without restriction in orientation but constrained to the region defined by an interior box with sides of $\SI{3}{\meter}$ and an exterior box of sides $\SI{6}{\meter}$, both centered on the goal point. The goal point is located at $x = \SI{0}{\meter}$, $y = \SI{0}{\meter}$, $z = \SI{4}{\meter}$, $r = \SI{0}{\radian}$, $p = \SI{0}{\radian}$, and $h = \SI{0}{\radian}$. The \ac{auv} is required to reach the goal point within $\SI{40}{\second}$, i.e., 800 steps, with a sampling time of $\SI{50}{\milli\second}$. 

In particular, for the energy consumption evaluation, addressing the relation between thruster power and action commands performed by the \ac{drl} controller is relevant. Based on the Blue Robotics T200 thrusters datasheet \cite{T200Thruster}, we can associate the power draw for a given \ac{pwm}, being \SI{1500}{} equivalent to $0 \; RPM$ and $\left[\SI{1100}{}, \SI{1900}{}\right]$ the bounds for forward and backward thrust, respectively, which can be mapped to the normalized actions space; therefore we can infer the energy consumption directly from the PWM signal, which can be approximated to a 4th order polynomial.

\begin{figure*}[t!]
\centering
\includegraphics[width=0.95\textwidth]{figures/3d_trayectory_h.pdf}%
\caption{3D trajectory followed by the \ac{auv} in the same episode as depicted in Fig. 3, using the PID, \ac{tqc} HP, and \ac{tqc} EA controllers}
\label{fig:3d_trayectory}
\end{figure*}

For the final evaluation of the trained policy, both proposed methods using the \ac{pid} controller as a benchmark were tested, combining three starting points for each \ac{dof}, resulting in $3^6 = 729$ combinations, all targeting $x = \SI{0}{\meter}$, $y = \SI{0}{\meter}$, $z = \SI{4}{\meter}$, $r = \SI{0}{\radian}$, $p = \SI{0}{\radian}$, and $h = \SI{0}{\radian}$. The starting positions for each axis are: $x = \SI[parse-numbers=false]{\{-5, 0, 5\}}{\meter}$, $y = \SI[parse-numbers=false]{\{-5, 0, 5\}}{\meter}$, and $z = \SI[parse-numbers=false]{\{2, 4, 6\}}{\meter}$; the initial attitudes are: $r = \SI[parse-numbers=false]{\{0, \pi/2, -\pi/2\}}{\radian}$, $p = \SI[parse-numbers=false]{\{0, \pi/2, -\pi/2\}}{\radian}$, and $h = \SI[parse-numbers=false]{\{0, \pi/2, -\pi/2\}}{\radian}$.

Both training and testing were conducted on an Intel i7 12th gen processor, 16 GB of RAM, and an NVIDIA RTX 3050 4 GB graphics card. To assess the performance of the methods, we calculated the average power consumption of the episodes, the \ac{rmse} error in position and orientation, and the approximated settling time ($t_s$).


\begin{table*}[t]
    \centering
    \resizebox{\textwidth}{!}{
\begin{tabular}{l|rrllrrll}
\toprule
\textbf{Dataset} & \multicolumn{4}{c}{\textbf{GSM8K}} & \multicolumn{4}{c}{\textbf{MATH}} \\
\cmidrule(lr){1-1} \cmidrule(lr){2-5} \cmidrule(lr){6-9}
\textbf{Method} & Acc & Len & Rel. Acc & Rel. Len & Acc & Len & Rel. Acc & Rel. Len \\
\midrule
\multicolumn{9}{l}{\textit{Zero-Shot Prompting}} \\
\midrule
\hspace{12pt}Baseline & 78.06 & 241.87 & 100.00 \small{(0.00)} & 100.00 \small{(0.00)} & 46.40 & 480.37 & 100.00 \small{(0.00)} & 100.00 \small{(0.00)} \\
\hspace{12pt}Be Concise & 77.98 & 214.87 & 99.85 \small{(1.18)} & 88.46 \small{(10.37)} & 47.76 & 446.09 & 102.71 \small{(7.59)} & 92.66 \small{(7.46)} \\
\hspace{12pt}Hand Crafted 2 (ours) & 76.72 & 184.13 & 98.27 \small{(3.67)} & 77.10 \small{(22.27)} & 46.84 & 404.85 & 101.62 \small{(4.79)} & 85.26 \small{(15.97)} \\
\midrule
\multicolumn{9}{l}{\textit{FT - External Data}} \\
\midrule
\hspace{12pt}Direct Answer & 19.70 & 3.17 & 24.88 \small{(5.03)} & 1.36 \small{(0.40)} & 15.08 & 6.98 & 35.16 \small{(10.34)} & 1.44 \small{(0.73)} \\
\hspace{12pt}Human CoT & 65.73 & 127.85 & 83.82 \small{(7.28)} & 54.95 \small{(13.17)} & 33.88 & 243.54 & 75.61 \small{(13.56)} & 53.14 \small{(13.87)} \\
\hspace{12pt}GPT4o CoT & 76.36 & 156.24 & 97.65 \small{(3.63)} & 67.60 \small{(16.70)} & 40.44 & 399.80 & 90.52 \small{(15.07)} & 87.21 \small{(22.22)} \\
\midrule
\multicolumn{9}{l}{\textit{FT - Best-of-N Self-Generation}} \\
\midrule
\hspace{12pt}Naive BoN & 77.12 & 214.22 & 98.79 \small{(1.64)} & 87.17 \small{(8.79)} & 47.64 & 433.26 & 101.74 \small{(7.04)} & 89.89 \small{(3.99)} \\
\hspace{12pt}Rational Metareasoning & 76.15 & 207.49 & 97.21 \small{(5.74)} & 84.93 \small{(5.09)} & 47.56 & 432.56 & 103.02 \small{(6.56)} & 90.56 \small{(5.25)} \\
\midrule
\multicolumn{9}{l}{\textit{FT - Few-Shot Conditioned Self-Generation (ours)}} \\
\midrule
\hspace{12pt}FS-Human & 76.66 & 161.72 & 98.06 \small{(3.28)} & 67.96 \small{(16.62)} & 46.44 & 421.54 & 99.69 \small{(6.97)} & 87.78 \small{(5.98)} \\
\hspace{12pt}FS-GPT4o & 78.07 & 175.54 & 99.94 \small{(1.69)} & 73.15 \small{(13.49)} & 47.36 & 421.21 & 101.87 \small{(5.33)} & 87.58 \small{(6.60)} \\
\hspace{12pt}FS-Self & 77.27 & 190.03 & 98.86 \small{(2.51)} & 77.51 \small{(9.18)} & 48.00 & 426.67 & 102.67 \small{(5.24)} & 88.50 \small{(4.49)} \\
\midrule
\multicolumn{9}{l}{\textit{FT - Few-Shot Conditioned Best-of-N Self-Generation (ours)}} \\
\midrule
% GPT4o Best-of-16 (Naive) & 75.48 & 153.51 & 96.56 \small{(3.79)} & 64.12 \small{(16.35)} & 47.28 & 367.49 & 101.50 \small{(9.81)} & 76.96 \small{(11.42)} \\
\hspace{12pt}FS-GPT4o-BoN & 75.88 & 153.38 & 97.00 \small{(4.11)} & 64.25 \small{(16.66)} & 47.36 & 364.33 & 102.56 \small{(6.24)} & 76.30 \small{(10.56)} \\
\hspace{24pt}\raisebox{0.5ex}{$\llcorner$}\hspace{4pt}\textit{Budget-Matched} & 76.24 & 160.59 & 97.44 \small{(3.67)} & 67.15 \small{(16.41)} & 47.52 & 384.43 & 101.58 \small{(9.53)} & 80.43 \small{(9.04)} \\

\bottomrule
\end{tabular}
    }
    \caption{
        Evaluation of zero-shot prompting, fine-tuning (FT), and our proposed methods on GSM8K and MATH datasets, averaged across five main models.
        We report absolute accuracy (\%) and length (tokens), along with relative accuracy (\%) and length (\%) compared to baseline prompting. Values in parentheses show standard deviations.
    }
    \label{tab:main_results}
\end{table*}

\section{Results}
\label{results}

% \begin{table*}[t]
% \resizebox{\textwidth}{!}{
% \centering
% \small
% \begin{tabular}{m{2.5cm}|m{15cm}}
% \toprule
% \textbf{Method} & \textbf{Content} \\ %& \# \textbf{Tokens}\\
% \midrule
% \midrule[0.05em]
% \multicolumn{2}{c}{Question} \\ 
% \midrule
% - & A robe takes 2 bolts of blue fiber and half that much white fiber.  How many bolts in total does it take?  \\
% \midrule
% \midrule[0.05em]
% \multicolumn{2}{c}{Reasoning Paths} \\
% \midrule
% Zero-Shot & To find the total number of bolts needed, we need to calculate the amount of white fiber first, since it's half the amount of blue fiber. Step 1: Determine the amount of blue fiber needed. The robe takes 2 bolts of blue fiber. Step 2: Calculate the amount of white fiber needed. The robe takes half the amount of white fiber as blue fiber. Half of 2 bolts is $2/2$ = 1 bolt. Step 3: Add the amount of blue fiber and white fiber to find the total. Total bolts = blue fiber + white fiber. Total bolts = 2 bolts + 1 bolt. Total bolts = 3 bolts. The answer is \textbf{\textit{3}}. \\ %& 213 \\
% \midrule
% FS-GPT4o-BoN & The robe takes 2 bolts of blue fiber. It takes half that much white fiber, which is $2 / 2$ = 1 bolt. Add the blue and white fiber together: 2 + 1 = 3 bolts. The answer is \textbf{\textit{3}}. \\ %57 \\
% \bottomrule
% \end{tabular}
% }
% \caption{Example rationales generated from the original Llama-3.1-8B model and the model fine-tuned using our method. The question is randomly sampled from GSM8K. For the same question, one can observe that our rationale is much more concise, removing redundant parts that explain the question.}
% \label{tab:samples}
% \vspace{-0.1in}
% \end{table*}


\begin{table*}[t]
\resizebox{\textwidth}{!}{
\centering
\normalsize
\begin{tabular}{m{17.5cm}}
\toprule
\multicolumn{1}{c}{\textbf{Question}} \\ 
\midrule
A robe takes 2 bolts of blue fiber and half that much white fiber.  How many bolts in total does it take? \\
\midrule
\multicolumn{1}{c}{\textbf{Reasoning Paths}} \\
\midrule
\textbf{[Zero-Shot]} 
To find the total number of bolts needed, we need to calculate the amount of white fiber first, since it's half the amount of blue fiber. Step 1: Determine the amount of blue fiber needed. The robe takes 2 bolts of blue fiber. Step 2: Calculate the amount of white fiber needed. The robe takes half the amount of white fiber as blue fiber. Half of 2 bolts is $2/2$ = 1 bolt. Step 3: Add the amount of blue fiber and white fiber to find the total. Total bolts = blue fiber + white fiber. Total bolts = 2 bolts + 1 bolt. Total bolts = 3 bolts. The answer is 3. \\
\midrule
\textbf{[FS-GPT4o-BoN]} 
The robe takes 2 bolts of blue fiber. It takes half that much white fiber, which is $2 / 2$ = 1 bolt. Add the blue and white fiber together: 2 + 1 = 3 bolts. The answer is 3. \\
\bottomrule
\end{tabular}
}
\caption{Example rationales generated from the original Llama-3.1-8B model (\textbf{Zero-Shot}) and the model fine-tuned using our method (\textbf{FS-GPT4o-BoN}). The question is randomly sampled from GSM8K. For the same question, one can observe that our rationale is much more concise, removing redundant parts that explain the question.}
\label{tab:samples}
\vspace{-0.1in}
\end{table*}


\subsection{Main results}

Our main results, presented in \autoref{tab:main_results} and \autoref{fig:main_methods_comparison}, demonstrate the performance of our self-training methods against baseline approaches.
% We highlight key observations from these results below.

\paragraph{Naive BoN fine-tuning is effective but sample inefficient.}
Naive BoN fine-tuning effectively reduces output length without significantly degrading model performance. 
This also holds true for Qwen2.5-Math-1.5B and DeepSeekMath-7B (\autoref{tab:main_results_full_gsm8k} and \autoref{tab:main_results_full_math}), which failed to achieve length reduction through zero-shot prompting.
% However, while naive BoN does reduce output length, the reduction is limited to 12\%.
However, the length reduction from naive BoN with $N=16$ is limited to 12\% on average.
Furthermore, as illustrated in Figure~\ref{fig:bon_sample_efficiency}, achieving more compression with BoN becomes progressively less efficient.

\paragraph{Iterative baseline yields similar results as naive BoN fine-tuning.}
% We compare our single-step naive BoN approach with Rational Metareasoning \cite{de2024rational}, an iterative approach using expert iteration \cite{zelikman2022star}  which incorporates an additional \textit{utility reward} to balance efficiency and accuracy in BoN sampling.
Rational Metareasoning, an iterative baseline, yields similar relative length reduction and relative accuracy to BoN fine-tuning. 
This suggests that the utility reward proposed by \citet{de2024rational} may not effectively achieve both accuracy gains and token length reduction.

\begin{figure}[t] % "h" places the figure roughly here
    \centering
    \includegraphics[width=\columnwidth]{figures/main_methods_comparison.pdf} % Adjust width as needed
    \caption{Tradeoff between relative accuracy and length reduction for main methods. Results are averaged over GSM8K and MATH across five main models. Matching colors and shapes indicate the same FS prompt. FS conditioning without augmentation (†) are marked with lighter colors. 
    Relative length reduction refers to 100 - relative length (\%).}
    \label{fig:main_methods_comparison} % Label for referencing in text
\end{figure}
% \red{TODO - shorten this}

\paragraph{Few-shot conditioning outperforms BoN in length reduction.}
The results demonstrate that few-shot conditioning achieves a greater relative length reduction compared to naive BoN, including math-specialized models (\autoref{tab:main_results_full_gsm8k} and \autoref{tab:main_results_full_math}).
% This reduction is attributed to the fact that the fine-tuning datasets generated through few-shot conditioning contain shorter reasoning paths compared to those generated by naive BoN, as illustrated in \autoref{fig:bon_sample_efficiency}.  % too long
This is in line with the superior length reduction of few-shot conditioning, compared to naive BoN as shown in \autoref{fig:bon_sample_efficiency}.
Notably, self-training on generations conditioned on human-annotated examples (FS-Human) achieves an average relative length of 67.96\% on GSM8K, compared to 87.17\% with naive BoN.  % good to have some specific numbers in the text
% We further analyze the effect of fine-tuning on length reduction in \autoref{analysis}.



\paragraph{Self-training better preserves accuracy than training with external data.} 
\autoref{tab:main_results} shows fine-tuning with external data (\textit{FT-External Data}) leads to a significant reduction in relative length but causes a severe drop in relative accuracy. 
% \autoref{fig:main_methods_comparison} further highlights that while fine-tuning with GPT-4o CoT (FT-GPT4o) achieves slightly greater reduction in relative length than fine-tuning with self-generated data using few-shots from GPT-4o (FS-GPT4o), it results in substantially lower relative accuracy.  % a bit complicated / not concrete (conrete evidence = one where we beat external FT in both accuracy and reduction)
\autoref{fig:main_methods_comparison} further highlights the accuracy preservation of self-training: fine-tuning with external concise reasoning supervision from GPT-4o (FT-GPT4o) lies below the Pareto-curve of relative accuracy and relative length reduction, established by our self-training methods.
% NAMGYU - TODO add some commentary

\paragraph{Few-shot conditioned BoN achieves best length reduction while maintaining accuracy.}
% Few-shot conditioned BoN enables substantial length reduction compared to all other BoN and few-shot methods while maintaining relative accuracy.
FS-BoN elicits the largest length reduction among our self-training methods, while maintaining relative accuracy, on average.
Notably, for math-specialized models, FS-GPT4o-BoN achieves the greatest reduction among all methods, except those fine-tuned on external data which greatly sacrifice the accuracy (\autoref{tab:main_results_full_gsm8k} and \autoref{tab:main_results_full_math}). 
% This result reflects how applying BoN to few-shot conditioning further reduces the relative length of the training data while also increasing the proportion of correct samples.  % unnecessary

\paragraph{Augmentation boosts accuracy for few-shot conditioning.}
\autoref{fig:main_methods_comparison} compares few-shot conditioning, i.e., FS and FS-BoN, with and without augmentation (†). 
Augmentation improves accuracy by providing solutions for previously unsolvable hard questions as discussed in \autoref{sample_augmentation}. 
While augmentation may slightly affect reduction rates, they remain superior to naive BoN and RM.
% Similar effect is observed for augmentation in FS-BoN.
% Even when matching the budget (\textit{Budget-Matched}) with other fine-tuning methods using self-generated data in \autoref{tab:main_results}, it achieves the greatest length reduction among them with minimal accuracy degradation.
Even when matching the budget (\textit{Budget-Matched}) with other self-training methods in \autoref{tab:main_results}, it achieves the greatest length reduction among them with minimal accuracy degradation.
The effect of augmentation on training data length is analyzed in \autoref{appx_augmentation_length}.
% Furthermore, as shown in Figure \ref{fig:main_methods_comparison}, augmentation on few-shot conditioned BoN enhances accuracy similar to naive BoN and Meta-Reasoning while achieving greater length reduction.

\begin{figure}[t]
    \centering
    \includegraphics[width=\columnwidth]{figures/length_by_difficulty.pdf} % Adjust width as needed
    \caption{\textbf{Tokens are reduced adaptively according to question difficulty.} 
    Token reduction rate for each difficulty level on MATH, for 4 models individually and averaged.
    % Higher difficulty levels show lower reduction rates.
    Relative length reduction refers to 100 - relative length (\%).
    }
    \label{fig:length_difficulty} % Label for referencing in text
\end{figure}

\subsection{Analysis}
\label{analysis}
% This section analyzes length reduction: transfer from generation to fine-tuning, reduction by question difficulty, qualitative analysis, and consistency across model sizes. DeepSeekMath-7B is excluded from quantitative analysis due to cost.
% let's keep this short
In this section, we analyze the length reduction effects in depth.
We exclude DeepSeekMath-7B from quantiative analysis due to cost.


% \paragraph{Analysis on sample efficiency}
% As shown in \autoref{fig:bon_sample_efficiency}, best-of-n (BoN) sampling requires a substantial number of samples to be generated to achieve a level of reasoning length reduction comparable to that achievable through few-shot conditioning.
% In other words, it is infeasible to reach the reasoning length reduction performance of few-shot conditioning using BoN alone, without generating a prohibitively large number of samples.
% However, our experiments consistently demonstrate that combining few-shot conditioning with BoN sampling is more effective in reducing reasoning length than using either technique in isolation.
% Specifically, few-shot conditioning helps to guide the model towards generating more concise reasoning paths, while BoN sampling allows us to select the shortest and most accurate path from a diverse set of candidates.
% This synergistic effect results in a more efficient and effective approach to concise reasoning.


% \paragraph{FT can reduce generation length effectively.}
% As shown in \autoref{fig:ft_length_scatter}, after fine-tuning, the models tend to follow the length of the training data, suggesting that reasoning length reduction can be achieved through simple supervised fine-tuning on short reasoning samples.
% Note that test generation length is relatively longer than the training data length, as the models can generate lengthy incorrect answers, while the training data consists of correct answers.
% Correctly generated answers align more closely with training data length as shown in (Appendix~\ref{appx_length_scatter_correct}).

% \paragraph{Length reduction through generation and fine-tuning}
% Our method reduces reasoning length in two stages: generation and fine-tuning.
% First, as shown in \autoref{fig:ft_length_scatter}, 
% % generation length for training data varies depending on the method. 
% few-shot conditioning methods produce shorter outputs than naive BoN, with few-shot conditioned BoN achieving the shortest. 
% Second, fine-tuning with shorter rationales results in shorter model outputs, showing a strong correlation between test and training lengths\footnote{Test generation lengths are generally longer than training data lengths due to the possibility of lengthy incorrect answers during testing. Test outputs that are correct align more closely with training data lengths, as shown in Appendix~\ref{appx_length_scatter_correct}.}.
% Overall, FS-GPT4o-BoN effectively generates and trains for shorter reasoning paths.
% Additionally, unlike zero-shot methods, our approach significantly reduces token length in math-tuned models like Qwen2.5-Math-1.5B with FS-GPT4o-BoN, achieving 54.7\% relative length after fine-tuning. (See \autoref{tab:main_results_full_gsm8k} and \autoref{tab:main_results_full_math}).

\paragraph{Tokens are reduced adaptively according to question complexity.} 
The MATH dataset's difficulty levels range from 1 (basic algebra) to 5 (advanced calculus and complex mathematical reasoning).
As shown in \autoref{fig:length_difficulty}, our method adaptively reduces tokens based on question difficulty, with higher difficulty leading to less reduction.
% Most models achieve their peak reduction (around 20\%--40\%) at difficulty levels 1-2, where simple concepts allow for more concise explanations.
% The reduction rate gradually declines at levels 3-5, indicating our method's ability to preserve necessary details for complex problems automatically.
%  -> not precise. simple concepts allow for more concise explanations *in absolute terms*, but this does not necessarily mean that length reduction *relative to the default* should be high. E.g., if the model already uses very few tokens for easy questions, then relative reduction would be low.
The higher reduction (20\%--40\%) at easier difficulty levels (1--2) suggests that the original model outputs for these easier questions contained unnecessary tokens.
This reveals a gap in current models' ability to tailor their inference budget to problem complexity.
Our method effectively closes this gap by reducing redundancy, allowing for more precise token allocation based on question difficulty.

\begin{figure}[t] % "h" places the figure roughly here
    \centering
    \includegraphics[width=\columnwidth]{figures/scaling_methods_comparison.pdf} % Adjust width as needed
    \caption{Scaling study on baseline and few-shot conditioned self-training methods. Results are averaged over GSM8K and MATH for Llama 1B, 3B, and 8B.
    % Accuracy tends to be maintained, with greater length reduction using our FS-GPT4o(-BoN) method.
    Relative length reduction refers to 100 - relative length (\%).
    }
    \label{fig:scaling_methods_comparison} % Label for referencing in text
\end{figure}

\paragraph{Self-training maintains consistency across model scales.}
We conduct a scaling study on Llama-3.2-1B, 3B, and Llama-3.1-8B to examine consistency across different model sizes (\autoref{fig:scaling_methods_comparison}). 
Overall, token reduction increases as the model size increases, while the maintenance of accuracy does not show a strong correlation with model size. 
RM exhibits lower reduction rates compared to our few-shot conditioned self-training methods across all models and shows a decrease in accuracy with increasing model size. 
% The few-shot method also shows a similar trend in length reduction, but it achieves the best relative accuracy in the 3B model.
Our standalone few-shot conditioning method (FS-GPT4o) also shows a similar trend in length reduction, but better preserves accuracy.
Our joint FS-GPT4o-BoN method achieves the greatest reduction across all models, maintaining relative accuracy across different model sizes, especially in the largest 8B model.



\paragraph{Sample study}
\autoref{tab:samples} presents qualitative examples of reasoning paths generated by the model before and after fine-tuning with our method. 
The original reasoning exhibits verbosity, containing redundant processes such as question confirmation and repeated instructions. 
In contrast, the reasoning generated by our method includes only the necessary steps, significantly reducing the number of tokens while still arriving at the correct answer. 
% These examples demonstrate the effectiveness of our method in reducing token count. 
More examples are provided in the \autoref{appx_sample_studies}.

\begin{figure}[t]
    \centering
    \includegraphics[width=\columnwidth]{figures/both_length_scatter.pdf} % Adjust width as needed
    \caption{\textbf{Fine-tuning effectively transfers the length reduction to the model.} Correlation between the relative length of train data and test output averaged over GSM8K and MATH across 4 models. Training length includes only correct solutions. Solid points represent test lengths including all generated outputs, while lighter points indicate test lengths of correct solutions only.}
    \label{fig:ft_length_scatter} % Label for referencing in text
\end{figure}

\paragraph{Length reduction is transferred through fine-tuning.}
As shown in \autoref{fig:ft_length_scatter}, fine-tuning with shorter rationales results in shorter model outputs, showing a strong correlation between test and training lengths.
% Test generation lengths (solid datapoints) are generally longer than training data lengths due to the possibility of lengthy incorrect answers during testing.
% However, when comparing with test generation lengths that are correct (lighter datapoints), they align more closely with training data lengths.
We note that the length of test outputs (incorrect and correct) are longer than the length of training samples (only correct) on average.
This is mainly because incorrect paths are generally longer than correct ones.
We find a closer correspondence between train length and test length of correct samples only, indicated by the lighter datapoints.
This discrepancy suggests the need to terminate incorrect paths early to minimize redundant inference overhead.
We consider this for future work.

% !TEX root = template.tex

\section{Conclusion}
\label{sec:conclusion}
This work focuses on MAS coordination and synchronization under recurring LTL. We extended the bottom-up scheme for distributed motion and task coordination of MAS in \cite{meng_paper}, reducing computational complexity to enhance scalability and enable deployment on robotic hardware. The package was developed in ROS2, with a synchronization mechanism to handle action delays in experiments. Future work will focus on developing additional actions and incorporating human-in-the-loop scenarios.


%%%%%%%%%%%%%%%%%%%%%%%%%%%%%%%%%%%%%%%%%%%%%%%%%%%%%%%%%%%%%%%%%%%%%%%%%%%%%%%%
% Acknowledgement
\section*{Acknowledgments}

The authors gratefully acknowledge the \acs{compas} laboratory at \acs{mbari} engineering and technical team.

%%%%%%%%%%%%%%%%%%%%%%%%%%%%%%%%%%%%%%%%%%%%%%%%%%%%%%%%%%%%%%%%%%%%%%%%%%%%%%%%
% This command balances the column lengths on the last page of the document manually. It shortens the text height of the last page by a suitable amount.
% This command does not take effect until the next page, so it should come on the page before the last. Make sure that you do not shorten the text height too much.

\addtolength{\textheight}{-8.75cm}   

%%%%%%%%%%%%%%%%%%%%%%%%%%%%%%%%%%%%%%%%%%%%%%%%%%%%%%%%%%%%%%%%%%%%%%%%%%%%%%%%
% References
\bibliography{refs/IEEEabrv, refs/DRL}
\bibliographystyle{IEEEtran}

\end{document}


