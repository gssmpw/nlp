%!TEX root = 2024_auv_mola_drl6dof_main.tex
%%%%%%%%%%%%%%%%%%%%%%%%%%%%%%%%%%%%%%%%%%%%%%%%%%%%%%%%%%%%%%%%%%%%%%%
\section{Conclusions}
\label{sec:conclusions}

%%%%%%%%%%%%%%%%%%%%%%%%%%%%%%%%%%%%%%%%%%%%%%%%%%
%% Genereal
The use of \acp{auv} for surveying, mapping, and inspecting unexplored underwater areas plays a crucial role, where maneuverability and power efficiency are key factors for extending the use of these platforms. Advances in \ac{drl} have established it as a reliable methodology for analyzing complex systems and developing data-driven control algorithms.

%%%%%%%%%%%%%%%%%%%%%%%%%%%%%%%%%%%%%%%%%%%%%%%%%%
%% Proposed Methods
In this paper, we introduce \ac{tqc-hp} and \ac{tqc-ea}, two end-to-end \ac{drl}-based approaches for the low-level control of a holonomic \ac{6dof} \ac{auv} using the \ac{tqc} algorithm. These methods require no manual tuning or prior knowledge of the thruster configuration and incorporate energy awareness (via \ac{tqc-ea}). Using a simulation environment and the model of \ac{mola}, we trained and demonstrated that the proposed methods show promise compared to the widely used \ac{pid} controller. In particular, the energy-aware method, utilizing a suitable reward function, demonstrates well-balanced performance between behavior and power consumption, which can be further fine-tuned. In future work, we plan to evaluate and deploy the algorithm with the physical platform in the field at \ac{mbari} and perform specific tasks for science operations. 
