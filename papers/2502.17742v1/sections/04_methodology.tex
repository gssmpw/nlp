%!TEX root = 2024_auv_mola_drl6dof_main.tex
%%%%%%%%%%%%%%%%%%%%%%%%%%%%%%%%%%%%%%%%%%%%%%%%%%%%%%%%%%%%%%%%%%%%%%%
\section{Evaluation Methodology}
\label{sec:metodology}

The proposed methods have been developed in the numerical simulation environment Stonefish  \cite{cieslak2019}, which allows the introduction of underwater physics and environmental conditions together with the vehicle geometry, accounting for the complete vehicle dynamics, including drag effects and added masses. The middleware used to communicate between the vehicle's sensors and the control algorithm is established through the \ac{lcm} library \cite{LCM}, which is the current middleware used by the field platform that can be used for future field testing; therefore, easing the transition from simulation to reality.

\begin{figure}[t!]
\centering
\includegraphics[width=0.48\textwidth]{figures/n_position_vs_time_v.pdf}%
\caption{Position in the $x$, $y$, and $z$ axes and angular distance to the target, $\theta$, over time during one of the evaluation episodes.}
\label{fig:position_vs_time}
\end{figure}

The field platform modeled within the simulation environment (depicted in Fig. \ref{fig:mola}) is \ac{mola} \ac{6dof} \ac{auv}, a research platform used for developing and testing novel research methods to enable the exploration of deep and rugged terrains. This platform is owned and operated by the \acs{compas} Lab at the \acf{mbari}. This system is equipped with an \ac{imu}, \ac{dvl}, depth sensor, cameras, sonar, and eight thrusters, and can perform complete holonomic movements in \ac{6dof}. Currently, \ac{mola} has a fine-tuned double-loop 36-gain \ac{pid} controller, with good performance in the same simulation environment. It has been extensively validated in the physical platform during tank tests and is used as the proper performance comparison for the \ac{drl} controller.

Both \ac{tqc-hp} and \ac{tqc-ea} are trained within the simulation environment in over $2.5\times10^6$ steps, starting from a random pose, without restriction in orientation but constrained to the region defined by an interior box with sides of $\SI{3}{\meter}$ and an exterior box of sides $\SI{6}{\meter}$, both centered on the goal point. The goal point is located at $x = \SI{0}{\meter}$, $y = \SI{0}{\meter}$, $z = \SI{4}{\meter}$, $r = \SI{0}{\radian}$, $p = \SI{0}{\radian}$, and $h = \SI{0}{\radian}$. The \ac{auv} is required to reach the goal point within $\SI{40}{\second}$, i.e., 800 steps, with a sampling time of $\SI{50}{\milli\second}$. 

In particular, for the energy consumption evaluation, addressing the relation between thruster power and action commands performed by the \ac{drl} controller is relevant. Based on the Blue Robotics T200 thrusters datasheet \cite{T200Thruster}, we can associate the power draw for a given \ac{pwm}, being \SI{1500}{} equivalent to $0 \; RPM$ and $\left[\SI{1100}{}, \SI{1900}{}\right]$ the bounds for forward and backward thrust, respectively, which can be mapped to the normalized actions space; therefore we can infer the energy consumption directly from the PWM signal, which can be approximated to a 4th order polynomial.

\begin{figure*}[t!]
\centering
\includegraphics[width=0.95\textwidth]{figures/3d_trayectory_h.pdf}%
\caption{3D trajectory followed by the \ac{auv} in the same episode as depicted in Fig. 3, using the PID, \ac{tqc} HP, and \ac{tqc} EA controllers}
\label{fig:3d_trayectory}
\end{figure*}

For the final evaluation of the trained policy, both proposed methods using the \ac{pid} controller as a benchmark were tested, combining three starting points for each \ac{dof}, resulting in $3^6 = 729$ combinations, all targeting $x = \SI{0}{\meter}$, $y = \SI{0}{\meter}$, $z = \SI{4}{\meter}$, $r = \SI{0}{\radian}$, $p = \SI{0}{\radian}$, and $h = \SI{0}{\radian}$. The starting positions for each axis are: $x = \SI[parse-numbers=false]{\{-5, 0, 5\}}{\meter}$, $y = \SI[parse-numbers=false]{\{-5, 0, 5\}}{\meter}$, and $z = \SI[parse-numbers=false]{\{2, 4, 6\}}{\meter}$; the initial attitudes are: $r = \SI[parse-numbers=false]{\{0, \pi/2, -\pi/2\}}{\radian}$, $p = \SI[parse-numbers=false]{\{0, \pi/2, -\pi/2\}}{\radian}$, and $h = \SI[parse-numbers=false]{\{0, \pi/2, -\pi/2\}}{\radian}$.

Both training and testing were conducted on an Intel i7 12th gen processor, 16 GB of RAM, and an NVIDIA RTX 3050 4 GB graphics card. To assess the performance of the methods, we calculated the average power consumption of the episodes, the \ac{rmse} error in position and orientation, and the approximated settling time ($t_s$).

