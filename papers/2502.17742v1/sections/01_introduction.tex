%!TEX root = 2024_auv_mola_drl6dof_main.tex
%%%%%%%%%%%%%%%%%%%%%%%%%%%%%%%%%%%%%%%%%%%%%%%%%%%%%%%%%%%%%%%%%%%%%%%
\section{Introduction}
\label{sec:intro}
%%%%%%%%%%%%%%%%%%%%%%%%%%%%%%%%%%%%%%%%%%%%%%%%%%
%% General Introduction

\noindent The exploration of our oceans has traditionally relied on expensive, large, and energy-consuming underwater vehicles for tasks such as surveying or mapping unexplored areas. These vehicles have the ability to execute different maneuvers depending on the thruster configuration and design. However, various factors have limited these applications, primarily a lack of precise underwater navigation and maneuverability. Specifically, the maneuverability of these platforms is constrained by either inadequate actuation, which restricts movement to fewer than \ac{6dof}, or by the limitations of the controllers, which involve a complex trade-off between tuning efforts and extensibility to more \acp{dof}. Numerous research vehicles are limited from their nominal \ac{6dof}, being designed to be rolling or pitching stable via buoyancy design and mechanically simpler with fewer thrusters, but at the cost of limiting maneuverability. Furthermore, in some cases, the thruster configuration limits the ability of the vehicle to move independently in all six axes, forcing the system to become nonholonomic.

Even though the control of the \ac{auv} is critical for trajectory tracking and the capabilities of the vehicle to operate in complex terrains, the energy used for this task is also a critical consideration, as \acp{auv} operates under battery power and optimal use leads to better autonomy. For research purposes, deploying a holonomic and power-efficient \ac{auv} can improve efficiency in data collection, seabed mapping, and maneuverability, providing significant advantages over nonholonomic vehicles.

%%%%%%%%%%%%%%%%%%%%%%%%%%%%%%%%%%%%%%%%%%%%%%%%%%
%% Figure: MOLA in Stonefish
\begin{figure}[t!]
\centering
\includegraphics[width=\linewidth]{figures/mola_sim_wide.png}%
\caption{Stonefish model of \acs{mola} \acs{6dof} \acs{auv}, \acs{mbari}'s autonomous research platform for complex terrain exploration.}
\label{fig:mola}
\end{figure}

%%%%%%%%%%%%%%%%%%%%%%%%%%%%%%%%%%%%%%%%%%%%%%%%%%
%% Literature Review

%%%%%%%%%%%%%%%%%%%%%%%%%%%%%%%%%%%%%%%%%%%%%%%%%%
%%%% PID
\acp{auv} exhibit complex behavior in underwater environments and can be mathematically modeled as coupled \ac{6dof} dynamic systems. As such, they are highly nonlinear, and their characterization requires the identification of more than 200 parameters \cite{mcfarland2021} to represent the hydrodynamics of the vehicle, which can be solved either numerically or experimentally. A widely used approach to control an \ac{auv} is the classical \ac{pid} control because of its linear nature; however, it is unable to fulfill optimality for energy-efficient controllers, and its performance is highly affected by the nonlinearities, its tight link to the configuration of the thrusters, the payload, and the speed range. Any change in the above requires adjusting the gains, which can involve up to 36 terms for the widely used velocity and position integrated controller. Even though \acp{nn} have been used to tune the \ac{pid} gains in \cite{hernandez2016}, they were not integrated experimentally.

%%%%%%%%%%%%%%%%%%%%%%%%%%%%%%%%%%%%%%%%%%%%%%%%%%
%%%% MPC
Similarly, model-based approaches, such as \ac{mpc}, are widely used because they incorporate system dynamics to frame control as an optimization problem, enabling control actions to minimize a cost function and achieve more optimal outcomes. Anderlini et al. \cite{anderlini2019} propose an adaptive \ac{mpc} framework as a complement to a \ac{pid} controller when the \ac{auv} carries a payload, correcting the difference in dynamics caused by an unmodeled mass, while Fernandez and Hollinger \cite{fernandez2016} present a similar approach for controlling an \ac{auv} in ocean waves. Although \ac{mpc} can achieve high performance, it also relies on system identification, which can be highly complex or unavailable, as it requires accounting for all inertial and drag coefficients. Errors in estimating these parameters or linearizing the system for simplification and convexity in cost optimization can lead to a model mismatch with the real system. Consequently, the optimal solution is calculated iteratively, increasing exponentially with additional \acp{dof}, leading to noticeable time delays. To overcome these problems, Martinsen et al. \cite{Martinsen2020} propose a \ac{nn}-based system identification algorithm to create a data-driven \ac{mpc}, solving numerically a structured known equation of motion to generate data and train a \ac{nn} that can successfully approximate the system's dynamics. However, it has been only tested in simple linear simulated systems.

%%%%%%%%%%%%%%%%%%%%%%%%%%%%%%%%%%%%%%%%%%%%%%%%%%
%%% DRL - Task Efficiency Focus
The advances in \ac{drl} have established it as a reliable methodology for analyzing complex systems and building data-driven control algorithms, based purely on the relation of \ac{mimo} systems and the ability to handle large amounts of data, benefiting from the lack of a rigid model structure. The use of \ac{drl} extends to path planning, attitude control, and navigation for \acp{auv} \cite{Singh2023}, focusing on the high-level control of the vehicle. Carlucho et al. \cite{carlucho2018} propose an end-to-end Actor-Critic approach to control a 5-\acs{dof} \ac{auv} in both a simulation environment and a test tank for experimental validation. Their work focuses on speed control for linear velocities without addressing rotational movements. Furthermore, the study lacks a direct comparison of the proposed controller against a \ac{pid} controller regarding position error and power requirements across all scenarios. Notably, the evaluation does not include testing with a \ac{6dof} trajectory that incorporates rotations. Lagattu et al. \cite{lagattu2024} developed a \ac{drl}-based thruster recovery controller, merging a classical \ac{pid} controller and a \ac{drl} backup controller when an undiagnosed fault is detected, based on the \ac{auv} behavior using a \ac{sac} algorithm. Unlike a \ac{pid} controller, the proposed controller can successfully reach a way-point with some overshooting after a failure. Lidtke et al. \cite{lidtke2024} based their work on the surveying task over a vertical cylinder, including dynamic forces computed with \ac{cfd} in a \ac{2d} representation and computing the movement as the solution to a 3-\acs{dof} differential equation. Several algorithms were tested, where \ac{tqc} outperformed other algorithms such as \ac{sac}, \ac{ppo}, \ac{ddpg}, and \ac{td3}. Even though \cite{carlucho2018, lagattu2024, lidtke2024} successfully enable the vehicle to track trajectories, they use the whole power range of the thrusters to achieve a task-efficient controller. Therefore, energy consumption is not addressed as a constraint.

%%%%%%%%%%%%%%%%%%%%%%%%%%%%%%%%%%%%%%%%%%%%%%%%%%
%%% DRL - Power Efficiency Focus
Only a few reported works have covered end-to-end controllers and their energy-efficiency analysis. Huang et al. \cite{Huang2022A} designed a specific reward function with an energy consumption term that penalizes high-value outputs of the control actions, leading to an energy-efficient \ac{sac} policy for an analytically-simulated torpedo-shaped underactuated \ac{auv} in 5-\acs{dof} within a \ac{3d} action space; however, the capability of this algorithm to operate in \ac{6dof} \acp{auv} is not stated. Sola et al. \cite{Sola2022} propose the use of \ac{sac} for the control and guidance of a \ac{6dof} \ac{auv}, which outperforms the \ac{pid} in terms of energy efficiency; however, without including an energy efficiency term explicitly in the reward function. Nevertheless, \cite{Huang2022A} and \cite{Sola2022} achieve power efficiency through trajectory planning instead of the control itself; thus, the energy metrics are a consequence of proper mission planning rather than the controller itself.

%%%%%%%%%%%%%%%%%%%%%%%%%%%%%%%%%%%%%%%%%%%%%%%%%%
%%% DRL - Aerial Field
% \ac{drl} has also found success in autonomous drone racing. Song et al. \cite{Song2023} demonstrated its superiority over \ac{mpc} in achieving higher success rates and faster lap times, even in the presence of unmodeled dynamics. These advancements may similarly benefit \acp{auv} control.

%%%%%%%%%%%%%%%%%%%%%%%%%%%%%%%%%%%%%%%%%%%%%%%%%%
%%% Contribution
To address the open challenges and limitations of previously proposed methods, this paper introduces \ac{tqc-hp} and \ac{tqc-ea}: two end-to-end \ac{drl}-based approaches for the low-level control of a holonomic \ac{6dof} \ac{auv} using the \ac{tqc} algorithm. These methods require neither manual tuning nor prior knowledge of the thruster configuration. To the best of our knowledge, this is the first report of a \ac{drl} algorithm successfully controlling a holonomic \ac{6dof} \ac{auv} in a simulation environment, directly mapping the eight thrusters without manual configuration (\ac{tqc-hp} and \ac{tqc-ea}). Furthermore, it is the first to incorporate energy-awareness to optimize energy consumption within the controller (\ac{tqc-ea}).

%%%%%%%%%%%%%%%%%%%%%%%%%%%%%%%%%%%%%%%%%%%%%%%%%%
%%% Paper Structure
The paper is structured as follows. Section \RomanNumeralCaps{2} overviews required concepts. Section \RomanNumeralCaps{3} details the proposed method. Section \RomanNumeralCaps{4} describes the evaluation methodology, while Section \RomanNumeralCaps{5} reports the evaluation and analysis of the results. The results and conclusions are summarized in Section \RomanNumeralCaps{6}.