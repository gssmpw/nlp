%!TEX root = 2024_auv_mola_drl6dof_main.tex
%%%%%%%%%%%%%%%%%%%%%%%%%%%%%%%%%%%%%%%%%%%%%%%%%%%%%%%%%%%%%%%%%%%%%%%
\section{Results and Discussion}
\label{sec:results}

The trained agents \ac{tqc-hp} and \ac{tqc-ea} are tested against the tuned \ac{pid} controller in 729 episodes, using the \ac{rmse} to assess performance in pose and energy consumption. Results are summarized in Table \ref{tab:rmse}.

\begin{table}[b!]
\centering
\caption{RMSE and standard deviation results for $x$, $y$, $z$, $\theta$, and $t_s$ in 729 episodes. The best result in each column is bolded}
\label{tab:rmse}
\begin{tabular}{l@{\hspace{3.5mm}}c@{\hspace{3.5mm}}c@{\hspace{3.5mm}}c@{\hspace{3.5mm}}}
\toprule
 & \textbf{\ac{pid}} & \textbf{\ac{tqc-hp}} & \textbf{\ac{tqc-ea}} \\
\midrule
$\bm{x}$\textbf{-axis} (\SI{}{\meter}) & $0.99 \pm 0.73$ & $\mathbf{0.96 \pm 0.64}$ & $1.14 \pm 0.74$ \\
$\bm{y}$\textbf{-axis} (\SI{}{\meter}) & $1.01 \pm 0.74$ & $\mathbf{1.00 \pm 0.66}$ & $1.18 \pm 0.76$ \\
$\bm{z}$\textbf{-axis} (\SI{}{\meter}) & $0.50 \pm 0.32$ & $\mathbf{0.36 \pm 0.19}$ & $0.42 \pm 0.26$ \\
$\bm{\theta}$ (\SI{}{\radian}) & $0.36 \pm 0.15$ & $\mathbf{0.32 \pm 0.08}$ & $0.41 \pm 0.14$ \\
$\bm{t_s}$ (\SI{}{\second}) & $15.67 \pm 9.52$ &  $\mathbf{8.28 \pm 1.95}$ & $21.95 \pm 11.15$ \\
\bottomrule
\end{tabular}
\end{table}

%%%%%%%%%%%%%%%%%%%%%%%%%%%%%%%%%%%%%%%%%%%%%%%%%%
%% TQC-HP vs PID
The \ac{tqc-hp} controller exhibited superior performance across all evaluated metrics. In comparison to the \ac{pid} controller, \ac{tqc-hp} achieved a reduction in position \ac{rmse} by \SI{3.03}{\percent} (\SI{0.03}{\meter}) on the x-axis, \SI{0.1}{\percent} (\SI{0.01}{\meter}) on the y-axis  and \SI{28}{\percent} (\SI{0.14}{\meter}) on the z-axis. Additionally, in terms of attitude \ac{rmse}, \ac{tqc-hp} demonstrated a notable improvement of \SI{11.1}{\percent} (\SI{0.04}{\radian}).

%%%%%%%%%%%%%%%%%%%%%%%%%%%%%%%%%%%%%%%%%%%%%%%%%%
%% TQC-EA vs PID
Conversely, the \ac{tqc-ea} controller, which incorporates energy consumption constraints, showed slightly lower performance relative to both the \ac{pid} and \ac{tqc-hp} controllers. Specifically, when compared to the \ac{pid} controller, \ac{tqc-ea} increased the position \ac{rmse} by \SI{15.2}{\percent} (\SI{0.15}{\meter}) on the x-axis and \SI{16.8}{\percent} (\SI{0.17}{\meter}) on the y-axis, although it achieved a \SI{16.0}{\percent} (\SI{0.08}{\meter}) improvement on the z-axis. The attitude \ac{rmse} also deteriorated by \SI{13.89}{\percent} (\SI{0.05}{\radian}).

%%%%%%%%%%%%%%%%%%%%%%%%%%%%%%%%%%%%%%%%%%%%%%%%%%
%% TQC-EA vs TQC-HP
Finally, comparing the two proposed methods, \ac{tqc-hp} and \ac{tqc-ea}, reveals further insights. The \ac{tqc-ea} controller underperformed against \ac{tqc-hp}, with position \ac{rmse} increasing by \SI{13.27}{\percent} (\SI{0.13}{\meter}) in the x-axis, \SI{9.90}{\percent} (\SI{0.1}{\meter}) in the y-axis, and \SI{29.4}{\percent} (\SI{0.1}{\meter}) in the z-axis. In terms of attitude \ac{rmse}, the decline was \SI{21.85}{\percent} (\SI{0.08}{\radian}).

%%%%%%%%%%%%%%%%%%%%%%%%%%%%%%%%%%%%%%%%%%%%%%%%%%
%% Episode Analysis
From Fig. \ref{fig:position_vs_time}, we can analyze the \ac{auv}'s pose over time for one of the 729 evaluated episodes. In this episode, the \ac{auv} starts at a position of $x = \SI{-5}{\meter}$, $y = \SI{-5}{\meter}$, $z = \SI{6}{\meter}$, $r = \SI[parse-numbers=false]{-\pi/2}{\radian}$, $p = \SI[parse-numbers=false]{\pi/2}{\radian}$, and $h = \SI[parse-numbers=false]{-\pi/2}{\radian}$, which corresponds to an angle $\theta = \SI[parse-numbers=false]{\pi/2}{\radian}$. The goal position is $x = \SI{0}{\meter}$, $y = \SI{0}{\meter}$, $z = \SI{4}{\meter}$, $r = \SI{0}{\radian}$, $p = \SI{0}{\radian}$, and $h = \SI{0}{\radian}$, with a corresponding angle $\theta = \SI{0}{rad}$.

We observe that the \ac{tqc-hp} controller reaches the references in less time (\SI{8.28}{\second}) compared to the \ac{pid} (\SI{15.67}{\second}) and \ac{tqc-ea} (\SI{21.95}{\second}) controllers. Additionally, the \ac{tqc-ea} controller exhibits a smoother response over time due to the energy and smoothness constraints applied. Despite these constraints, it manages to reach the goal position in a time comparable to that of the \ac{pid} controller.

%%%%%%%%%%%%%%%%%%%%%%%%%%%%%%%%%%%%%%%%%%%%%%%%%%
%% Power consumption
Regarding power consumption, Fig. \ref{fig:power} shows the average power used by the \ac{auv} during the evaluation episodes. The \ac{tqc-hp} controller, which has no restrictions on thruster usage, consumed \SI{10}{\percent} more power than the \ac{pid} controller. In contrast, the \ac{tqc-ea} controller, which incorporates restrictions on thruster usage, used \SI{30}{\percent} less power than the \ac{pid} controller during the evaluation episodes.

%%%%%%%%%%%%%%%%%%%%%%%%%%%%%%%%%%%%%%%%%%%%%%%%%%
%% 3D trajectory
We can also visualize the \ac{3d} trajectory followed by the \ac{auv} in the same episode for the three controllers in Fig. \ref{fig:3d_trayectory}. The \ac{pid} controller begins its trajectory by rotating while moving towards the goal position, deviating from the angular reference during the initial part of the trajectory before ultimately reaching the goal. The \ac{tqc-hp} controller demonstrates smoother angular alignment in the initial meters, consistently approaching the angular reference; however, the path taken is not the most intuitively direct. The \ac{tqc-ea} controller's trajectory is similar to that of the \ac{pid} controller but is achieved with \SI{30}{\percent} less power.

Furthermore, the tuning of the weight for the reward function in the \ac{tqc-ea} controller can be further optimized depending on the application's priorities. By adjusting these weights, the power consumption can be reduced further, potentially at the cost of increased pose error, and vice versa.

%%%%%%%%%%%%%%%%%%%%%%%%%%%%%%%%%%%%%%%%%%%%%%%%%%
%% Overall analysis
The results indicate that while the \ac{tqc-hp} controller provides the best positional accuracy and alignment with the goal angular position, the \ac{tqc-ea} model offers a balance between accuracy and energy efficiency. This balance makes the \ac{tqc-ea} model particularly suitable for applications where power consumption is critical, such as long-range cruising, which is typical for \acp{auv}. Furthermore, the energy savings are achieved through the controller itself rather than through path planning, as seen in previous methods.

\begin{figure}[t!]
\centering
\includegraphics[width=0.41\textwidth]{figures/average_power.pdf}%
\caption{Average power consumption and standard deviation of the \ac{auv} during the evaluation using \ac{pid}, \ac{tqc-hp} and \ac{tqc-ea} controllers.}
\label{fig:power}
\end{figure}