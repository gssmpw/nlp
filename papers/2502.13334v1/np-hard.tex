\section{Complexity of computing a profit-maximizing menu} \label{complexity}

This section is devoted to analyzing the computational complexity of computing a profit-maximizing menu in the service provider problem. We present both a strong negative result and a strong positive result. We first show that computing an exact profit-maximizing menu is \tbf{NP}-hard even in a seemingly simple setting of two buyer types and a single seller action. We then derive a fully-polynomial time approximation scheme (FPTAS) for maximizing seller profit: for any constant $\eps>0$, we can output a menu that achieves at least $1-\eps$ fraction of the maximum possible profit. The FPTAS works when the number of buyer types is constant and runs in  time polynomial in $\fr{1}{\eps}$, the number of seller actions, and the number of outcomes.

\subsection{\tbf{NP}-hardness for two types and a single action} \label{hardness}

\citet{bernasconi2024agent} show that in their \emph{mandatory usage} model, a profit-maximizing menu  for any number of buyer types can be computed efficiently using a linear program if the seller is allowed to randomize over actions. In stark contrast, we prove that the \emph{voluntary usage} assumption in our service provider problem makes maximizing seller profit computationally hard. We are able to show \tbf{NP}-hardness of computing the exact maximum seller profit even when there are just two buyer types and just a single seller action.\footnote{Maximizing seller profit for one buyer type is trivial because for each seller action we can extract maximum profit by setting an upfront price equal to exactly the buyer's value so that there is no buyer surplus.}

\begin{theorem} \label{np-hardness-two-types}
Computing the exact maximum seller profit in the service provider problem is \tbf{NP}-hard even when there are only two buyer types and a single seller action.
\end{theorem}

The rest of this subsection is devoted to proving \cref{np-hardness-two-types}. To show hardness, it suffices to consider a uniform distribution $\mu^1 = \mu^2 = \fr12$ over two buyer types, as well as a single seller action $a$. To outline the proof, we first derive formulas for the profit-maximizing upfront prices for two types given usage prices. We then use these formulas to show that a multiset of integers summing to $M$ can be partitioned into two subsets of equal sum if and only if the maximum profit in a specific service provider instance is exactly $\fr{9M}{4}$.

\paragraph{Profit-maximizing upfront prices for two types.}

The single action $a$ will be used in both contracts, and by \cref{highest-type-no-usage-prices} we can assume that $x^2_q = 0,\fl q$. Hence the menu search space consists of choosing $x^1_q\in \bc{0,\infty},\fl q$ as well as setting the upfront prices $w^1, w^2$. Fixing the usage prices $x^1_q$, we derive formulas for the profit-maximizing upfront prices assuming that type 2 is a highest type, noting that if type 1 is the highest type our formulas will hold simply by switching the two types:

\begin{lemma} \label{upfront-price-formula-two-types}
    Let type 2 be a highest type in the service provider problem for two types. In a profit-maximizing menu, denote by $S \coloneq \bc{q:x^1_q = 0}$ the set of outcomes that type 1 accepts. Then
    \begin{align*}
        w^1 &= \sum_{q\in S} p^a_q v^1_q \\
        w^2 &= \min \bc{\sum_{q\notin S} p^a_q v^2_q + \sum_{q\in S} p^a_q v^1_q, \sum_q p^a_q v^2_q}.
    \end{align*}
\end{lemma}

    %, say type 1. At this point we continue increasing the upfront price $w^2$ until one of the following occurs:
    % \begin{itemize}
    %     \item Type 2 has no buyer surplus. In this case both types now have no buyer surplus, so we can rename the type with the higher upfront price to type 2, in which case \cref{highest-type-no-usage-prices} implies that we can set type 2 usage prices to 0. Type 2 could now have nonzero buyer surplus, so we continue increasing $w^2$ until either type 2 has no buyer surplus or type 2 IC is violated. Note that the type 2 IC constraint is $$\sum_q p^a_q v^2_q - w^2 \ge \sum_{q\in S} p^a_q v^2_q - w^1 \imp w^2 \le w^1 + \sum_{q\notin S} p^a_q v^2_q.$$ 
        
    %     Type 2 will have no buyer surplus first if $$\sum_q p^a_q v^2_q \le w^1 + \sum_{q\notin S} p^a_q v^2_q,$$ in which case the result of the lemma holds.

    %     Type 2 IC will be violated first if $$w^1 + \sum_{q\notin S} p^a_q v^2_q \le \sum_q p^a_q v^2_q,$$ in which case the result of the lemma again holds.
        
    %     \item Type 2 IC is violated. At the point in which this happens, we consider two cases. 
        
    %     \textbf{Case 1:} If $w^2 \ge w^1$ then type 2 is the higher type, in which case \cref{highest-type-no-usage-prices} implies that we can set type 2 usage prices to 0. The same analysis as in the previous item applies to show that the lemma holds.

    %     \textbf{Case 2:} If $w^2 < w^1$ then we can increase profit by eliminating contract $\cC^2$ entirely and noting that profit will increase since type 2 is paying a greater upfront price $w^1 > w^2$.
    % \end{itemize}

    
    
%     Since type 2 is assumed to be a highest type, type 1
% \end{proof}
% The IC constraint for $t_2$ states 
% \[\sum_q p_q v^2_q - w^2 \ge \sum_{j\in S} p_q v^2_q - w^1,\]
% so we must have 
% \[w^2 \le \sum_{j\notin S} p_q v^2_q + w^1.\] 
% Because $t_1$'s IR condition binds at an optimal menu, \hfcomment{why? it is not easy for me to see this} we get 
% % \kicomment{should  we state this point as well somewhere?}
% \[w^1 = \sum_{j\in S} p_q v^1_q,\] and hence
% \[w^2 \leq \sum_{j \notin S} p_q v^2_q + \sum_{j\in S} p_q v^1_q.\] 
% In an optimal menu, either the preceding inequality or type 2's IR condition $w^2 \leq \sum_q p_q v_q^2$ binds, and hence, we get
% \[w^2 = \min \bc{\sum_{j\notin S} p_q v^2_q + \sum_{j\in S} p_q v^1_q, \sum_q p_q v^2_q}.\]

% We summarize these useful formulas for $w^1$ and $w^2$ as a lemma: 


% Note that we do not care about the other IC constraint because, as the service provider extracts greater revenue from the contract for $t_2$, the revenue only increases if type $t_1$ chooses type $t_2$'s contract. 

\cref{upfront-price-formula-two-types} is proven in \cref{upfront-price-formula-two-types-proof}. Recalling that we assumed $\mu^1 = \mu^2 = \fr12$, for a given subset $S$ of outcomes seller revenue is

\begin{equation} \label{revenue-formula-two-types}
\fr12 \bp{w^1 + w^2} = \frac{1}{2} \left( \sum_{q \in S} p^a_q v^1_q + \min \bc{\sum_{q\notin S} p^a_q v^2_q + \sum_{q\in S} p^a_q v^1_q, \sum_q p^a_q v^2_q} \right),
\end{equation}

From \cref{revenue-formula-two-types}, we observe the following:

\begin{claim*} \label{q-in-s}
     For outcomes $q$ such that $v^1_q \ge v^2_q$, seller profit is larger when $q \in S$ than when $q\notin S$.
\end{claim*}

\begin{proof}
    Setting $x^1_q = 0$ for any $q$ satisfying $v^1_q \ge v^2_q$ contributes $p^a_q v^1_q$ to the sum $\sum_{q \in S} p_q^a v_q^1$ in \cref{revenue-formula-two-types} and $p^a_q v_q^1$ to the sum $$\sum_{q\notin S} p^a_q v^2_q + \sum_{q\in S} p^a_q v^1_q$$ in the minimum in \cref{revenue-formula-two-types}. On the other hand, setting $x^1_q = \infty$ contributes 0 and $p^a_q v^2_q$ to these respective sums. Since $v^1_q \ge v^2_q$, both contributions are weakly greater when $x^1_q = 0$, so it is optimal to set $x^1_q = 0\iff q\in S$.
\end{proof}

\paragraph{Reducing from \tsf{Partition}.}

To prove \tbf{NP}-hardness of computing the optimal menu for two types, we reduce from the well-known \tsf{Partition} problem, as described below:
\begin{problem*}[\tsf{Partition}]
    Given a multiset of integers $\bc{n_1,n_2,\lds, n_k}$ with sum $M = n_1 + n_2 + \cds + n_k$, determine if there exists a subset that sums to $\fr{M}{2}$.
\end{problem*}

It is well-known that \tsf{Partition} is \tsf{NP}-hard, for example see \citet{hayes2002easiest}. Given an instance $\bc{n_1,n_2,\lds,n_k}$ of the partition problem, we construct an instance of the service provider problem as follows:
\begin{itemize}
    \item Let $Q = 0\cup [k]$ and recall that $T = [2]$ with $\mu^1 = \mu^2 = \fr12$ and $A = \bc{a}$.
    \item Outcome $0$ has valuations $v^1_0 = M(k+1)$ and $v^2_0 = 0$.
    \item For $q\in [k]$, outcome $q$ has valuations $v^1_q = n_q(k+1)$ and $v^2_q = 3n_q(k+1)$.
    \item The single action $a$ has cost $c(a) = 0$ and the transition probabilities to the $k+1$ outcomes are uniform, so $p^a_q = \fr{1}{k+1},\fl q$.
\end{itemize}

The maximum possible type 1 revenue is $\sum_q p^a_q v^1_q = 2M$ and the maximum possible type 2 revenue is $\sum_q p^a_q v^2_q = 3M.$ The key claim in the reduction is the following:

\begin{claim*} \label{partition-revenue}
    There exists a subset of $\bc{n_1,n_2,\lds,n_k}$ that sums to $\fr{M}{2}$ in the \tsf{Partition} problem if and only if the maximum seller profit in the constructed service provider instance is $\fr{9M}{4}$.
\end{claim*}

\begin{proof}
If type 1, whose revenue is at most $2M$, is a highest type, then the maximum seller profit is $2M < \fr{9M}{4}$. Hence if a menu's expected profit is $\fr{9M}{4}$ then type 1 cannot be a highest type. Also, if there exists a subset of $\bc{n_1,n_2,\lds,n_k}$ that sums to $\fr{M}{2}$ then we will show that there exists an menu in which $t_2$ is a highest type such that the seller profit is $\fr{9M}{4}$. For both directions we can thus assume that type 2 is a highest type.

% \kicomment{Also, currently the definition of $S$ is somewhat confused -- $S$ earlier was defined as all qualities that are purchased by $t_1$, but here you sometimes use it as ``all qualities other than quality $0$ that type $1$ purchases''. I think you should stick with the original definition of $S$ (because you are using expressions from before), and maybe define $T = S \setminus \{0\}$. Or you can just define $S$ as before, and define $M_S = \sum_{j \in S, j \neq 0} n_q$. } 

\newcommand{\Szero}{S\sm \bc{0}}

As before, let $S$ denote the outcomes that type 1 accepts, which includes outcome 0, and define $M_S = \sum_{q\in \Szero} n_q$. By the claim in \cref{upfront-price-formula-two-types}, the profit-maximizing upfront prices are
\begin{align*}
    w^1 &= \sum_{q\in S} p^a_q v^1_q = M + M_S \\
    w^2 &=  \min\bc{\sum_{q\notin S} p_q v^2_q + w^1, \sum_q p_q v^2_q}
    = \min\bc{4M - 2M_S, 3M}.
\end{align*}
If $M_S > \fr{M}{2}$ then seller profit is $$\fr12 \bp{(M + M_S) + (4M - 2M_S)} = \fr{5M - M_S}{2} < \fr{9M}{4},$$ and if $M_S < \fr{M}{2}$ then the seller profit is $$\fr12\bp{(M + M_S) + (3M)} = \fr{4M + M_S}{2} < \fr{9M}{4}.$$ Finally, if $M_S = \fr{M}{2}$ then seller profit is exactly $\fr{9M}{4}$, and we can verify that type 2 is indeed a highest type as $$w^1 = \fr{3M}{2} \le 3M = w^2.$$ We conclude that there is a subset $S$ with $M_S = \fr{M}{2}$ if and only if the maximum seller profit is $\fr{9M}{4}$.
\end{proof}

The claim shows how to reduce the \tsf{Partition} problem to computing the maximum seller profit of an instance of the service provider problem with two buyer types and a single action. Since \tsf{Partition} is \tbf{NP}-hard, computing the maximum seller profit in the service provider problem is also \tbf{NP}-hard. Certainty this also proves that computing a profit-maximizing menu that achieves this maximum profit is also \tbf{NP}-hard since computing the numerical profit that a menu achieves can be done in polynomial time.



% \yzedit{
% \begin{lemma}\label{lem:opt-usage-payment-two-types}
%     Given a single action and two agent types, any IC (incentive-compatible) menu can be transformed into an IC menu where the usage payment at each quality level is either $0$ or equal to one of the agent's values at that quality level.
% \end{lemma}

% \begin{proof}
%     Consider an agent of type $t$ with a contract $(w^t, \mathbf{x}^t)$. We replace their contract with $(\Tilde{w}^t, \Tilde{\mathbf{x}}^t)$ depending on the utility derived from the usage payments at any quality level $j$. Define the utility of an agent of type $t'$ choosing the contract designed for type $t$ as:
%     \[
%     U^{w^t, \mathbf{x}^t}(t';t) = \sum_q p_q^{t'}\max \{v_q^{t'} - x_q^t \} - w^t.
%     \]
%    The transformation of the principal's revenue between the usage payment and training payment depending on whether the agents of type $t'$ get less utility in choosing the new contract than the old contract of type $t$. For an agent of type $t$ and a quality level $j$, we consider four scenarios:

%     \begin{enumerate}
%         \item If $x_q^t \leq \min\{v_q^t, v_q^{t'} \}$, both types accept quality level $j$ under the usage payment for type $t$. We transform the usage payment to either all into training payments or split between training and usage payments based on which configuration generates less utility for agent $t'$ in the new contract by choosing contract for type $t$. Specifically:
%         \[
%         (\Tilde{x}_q^t, \Tilde{w}^t) = 
%         \begin{cases}
%             (0, w^t + p_q^t x_q^t), & \text{if } p_q^t \geq p_q^{t'}, \\
%             (\min\{v_q^t, v_q^{t'}\}, w^t - p_q^t(v_q^t - x_q^t)), & \text{if } p_q^t < p_q^{t'}.
%         \end{cases}
%         \]

% In both scenarios, the agent of type \( t \) retains the same utility as under the old contract, and the principal continues to extract the same revenue. Specifically, for quality levels \( j \) where \( p_q^t \geq p_q^{t'} \), the utility difference for an agent of type \( t' \) choosing the new contract over the old is given by
% \[
% U^{\tilde{w}^t, \tilde{\mathbf{x}}^t}(t', t) - U^{w^t, \mathbf{x}^t}(t'; t) = - (p_q^t - p_q^{t'}) x_q^t < 0,
% \]
% indicating reduced utility. Similarly, for quality levels \( j \) where \( p_q^t < p_q^{t'} \), the utility difference is
% \[
% U^{\tilde{w}^t, \tilde{\mathbf{x}}^t}(t', t) - U^{w^t, \mathbf{x}^t}(t'; t) = - (p_q^{t'} - p_q^{t}) (v_q^t - x_q^t) < 0,
% \]
% also less than zero. Consequently, the agent of type \( t' \) derives less utility from opting for the new contract compared to the old one, ensuring incentive compatibility.



%         \item If $v_q^t \leq x_q^t \leq v_q^{t'}$, only agents of type $t'$ accept quality level $j$. The principal can maintain revenue by setting $(\Tilde{x}_q^t, \Tilde{w}^t) = (v_q^{t'}, w^t)$, effectively increasing the usage payment for agents of type $t'$, thus reducing their utility.

%         \item If $v_q^{t'} \leq x_q^t \leq v_q^t$, only agents of type $t$ accept quality level $j$. Here, the principal can reduce the usage payment to $v_q^{t'}$ and adjust the training payment accordingly, $\Tilde{w}^t = w^t + p_q^t(x_q^t - v_q^{t'})$, ensuring the utility for type $t$ is preserved and that for type $t'$ is not increased.

%         \item If $x_q^t > \max \{v_q^t, v_q^{t'}\}$, implying that no agents accept the quality level $j$, setting $x_q^t = \max \{v_q^t, v_q^{t'}\}$ can lead to increased revenue as at least one agent type would then accept the quality level.
%     \end{enumerate}
% \end{proof}
% }

% \begin{lemma}
%     Given a single action and two agent types, the optimal menu can be found by comparing exponential values. 
% \end{lemma}
% \begin{proof}
%     From Lemma~\ref{lem:opt-usage-payment-two-types}, we have that the optimal usage payments $x_q^{t_1}, x_q^{t_2}$ can be transformed into the set $\{0, v_q^{t_1}, v_q^{t_2} \}$ for each $j$. Let $\bar{x}^{t_1}, \bar{x}^{t_2}$ denote a combination of the values. And $U^{t, t'} = \sum_q p_q^t \max\{ v_q^t - \bar{x}_q^{t'} \}$ denote the performance utility for type $t$ under the performance usage payment $\mathbf{\bar{x}}$ . The optimal training payment $w^{t_1}, w^{t_2}$ can be computed by solving the following linear program. 
%     \begin{align*}
%         \max_{w} \quad & \mu^{t_1}  w^{t_1} + \mu^{t_2} w^{t_2} \\
%         & U^{t_1, t_1} - w^{t_1} \geq U^{t_1, t_2}- w^{t_2} \\
%         & U^{t_2, t_2}- w^{t_2} \geq U^{t_2, t_1} - w^{t_1} \\
%          & U^{t_1, t_1}- w^{t_1} \geq 0 \\
%           & U^{t_2, t_2}- w^{t_2} \geq 0
%     \end{align*}


% The optimal solution is 
% \[
% (w^{t_1}, w^{t_2}) = 
% \begin{cases}
%      (U^{t_1,t_1}, U^{t_2, t_2}) \quad &\text{if $U^{t_1,t_1} \geq U^{t_2,t_1}$ and $ U^{t_2, t_2} \geq  U^{t_1, t_2}$}\\
%      (U^{t_1,t_1},  U^{t_2,t_2} +U^{t_1,t_1} -U^{t_2,t_1}) \quad &\text{if $U^{t_2,t_1} > U^{t_1,t_1}$} \\
%     (U^{t_1,t_1}+U^{t_2,t_2}-U^{t_1,t_2}, U^{t_2,t_2}) \quad &\text{if $U^{t_1,t_2} > U^{t_2,t_2}$} 
% \end{cases}
% \]
% Let $r(\bar{x}^t) =  \sum_q p_q^{t} v^{t}_q \ind \{v_q^{t} \geq \bar{x}_q^{t} \}$ . And the optimal revenue is 
% \[
%  \text{Revenue} (\mathbf{\bar{x}}) = 
% \begin{cases}
%     \mu^{t_1} r(\bar{x}^{t_1}) + \mu^{t_2} r(\bar{x}^{t_2})\quad &\text{if $U^{t_1,t_1} \geq U^{t_2,t_1}$ and $ U^{t_2, t_2} \geq  U^{t_1, t_2}$}\\
%     \mu^{t_1} r(\bar{x}^{t_1}) + \mu^{t_2} r(\bar{x}^{t_2}) + (U^{t_1,t_1}-U^{t_2,t_1}) \quad &\text{if $U^{t_2,t_1} > U^{t_1,t_1}$} \\
%    \mu^{t_1} r(\bar{x}^{t_1}) + \mu^{t_2} r(\bar{x}^{t_2}) + (U^{t_2,t_2}-U^{t_1,t_2}) \quad &\text{if $U^{t_1,t_2} > U^{t_2,t_2}$} 
% \end{cases}
% \]


% \end{proof}


% \begin{lemma}
%     For two types agents, the optimal menu can be found by comparing exponential values. 
% \end{lemma}
% \begin{proof}
%     From Lemma~\ref{}, we have that the optimal usage payments $x_q^{t_1}, x_q^{t_2}$ can be transformed into the set $\{0, \max\{v_q^{t_1}, v_q^{t_2} \} \}$ for each $j$. Let $\bar{x}^{t_1}, \bar{x}^{t_2}$ denote a combination of the values. And $U^{t, t'} = \sum_{i,j}\xi_i^{t'} p_q^i \max\{v_q^t - \bar{x}_q^{t'} \}$ denote the performance utility for type $t$ under the performance usage payment $\mathbf{\bar{x}}$ and effort distribution $\xi$. The optimal training payment $w^{t_1}, w^{t_2}$ can be computed by solving the following linear program. 
%     \begin{align*}
%         \max_{w} \quad & \mu^{t_1}  w^{t_1} + \mu^{t_2} w^{t_2} \\
%         & U^{t_1, t_1} - w^{t_1} \geq U^{t_1, t_2}- w^{t_2} \\
%         & U^{t_2, t_2}- w^{t_2} \geq U^{t_2, t_1} - w^{t_1} \\
%          & U^{t_1, t_1}- w^{t_1} \geq 0 \\
%           & U^{t_2, t_2}- w^{t_2} \geq 0
%     \end{align*}


% The optimal solution is 
% \[
% (w^{t_1}, w^{t_2}) = 
% \begin{cases}
%      (U^{t_1,t_1}, U^{t_2, t_2}) \quad &\text{if $U^{t_1,t_1} \geq U^{t_2,t_1}$ and $ U^{t_2, t_2} \geq  U^{t_1, t_2}$}\\
%      (U^{t_1,t_1},  U^{t_2,t_2} +U^{t_1,t_1} -U^{t_2,t_1}) \quad &\text{if $U^{t_2,t_1} > U^{t_1,t_1}$} \\
%     (U^{t_1,t_1}+U^{t_2,t_2}-U^{t_1,t_2}, U^{t_2,t_2}) \quad &\text{if $U^{t_1,t_2} > U^{t_2,t_2}$} 
% \end{cases}
% \]
% Let $r(\bar{x}^t) =  \sum_q p_q^{t} v^{t}_q \ind \{v_q^{t} \geq \bar{x}_q^{t} \}$ . And the optimal revenue is 
% \[
%  \text{Revenue} (\mathbf{\bar{x}}) = 
% \begin{cases}
%     \mu^{t_1} r(\bar{x}^{t_1}) + \mu^{t_2} r(\bar{x}^{t_2})\quad &\text{if $U^{t_1,t_1} \geq U^{t_2,t_1}$ and $ U^{t_2, t_2} \geq  U^{t_1, t_2}$}\\
%     \mu^{t_1} r(\bar{x}^{t_1}) + \mu^{t_2} r(\bar{x}^{t_2}) + (U^{t_1,t_1}-U^{t_2,t_1}) \quad &\text{if $U^{t_2,t_1} > U^{t_1,t_1}$} \\
%    \mu^{t_1} r(\bar{x}^{t_1}) + \mu^{t_2} r(\bar{x}^{t_2}) + (U^{t_2,t_2}-U^{t_1,t_2}) \quad &\text{if $U^{t_1,t_2} > U^{t_2,t_2}$} 
% \end{cases}
% \]


% \end{proof}

% \hfcomment{
% Assuming type $t_1$ is the one with no usage payments, we get the following program:
% \begin{align*}
%     \max_{w, \xi} \quad & \expec \left[ \sum_{j = 1}^n \left( \sum_{i = 1}^m \xi_i^{t} p_q^{i, t} \right) \bar{x}_q^t \ind \{v_q^t - \bar{x}_q^{t} \geq 0 \} + w^{t} - \sum_{i = 1}^m \xi_i^{t} c_i \right]\\
%      &\sum_{j=1}^n \left(\sum_{i = 1}^m \xi^{t_1}_i p_q^{i, t_1}\right) v_q^{t_1}  - w^{t_1} \geq \sum_{j=1}^n \left(\sum_{i = 1}^m \xi^{t_2}_i p_q^{i, t_1}\right)\max \{v_q^{t_1} - \bar{x}_q^{t_2}, 0\}- w^{t_2},\\
%      &\sum_{j=1}^n \left(\sum_{i = 1}^m \xi^{t_2}_i p_q^{i, t_2}\right)\max \{v_q^{t_2} - \bar{x}_q^{t_2}, 0\}- w^{t_2} \geq \sum_{j=1}^n \left(\sum_{i = 1}^m \xi^{t_1}_i p_q^{i, t_2}\right) v_q^{t_2}  - w^{t_1},\\
%     &\sum_{j=1}^n \left(\sum_{i = 1}^m \xi^{t}_i p_q^{i, t}\right)\max \{v_q^{t} - \bar{x}_q^{t}, 0\}- w^{t} \geq 0, \text{for $t \in \{t_1, t_2\}$}
% \end{align*}

% }


% \textbf{Optimal menus $(w, x)$ for upfront $\xi$}. 

% \hfcomment{

% \textbf{An Alternative View of IC,IR Mechanisms} 

% We start with a useful lemma that characterizes IC and IR menus, and offers an alternative way to view the mechanism design problem.  
% \begin{lemma}\label{lem:pay-attribution}[Usage Payment Attribution Lemma] For any IC and IR menu $\{(w^{t}, \xi^t, x^{t}) \}_{t}$, there exists $y^t \in \mathbb{R}^n_+$ for each $t$ such that the usage payment $w^t$ can be ``attributed'' to a payment $y^t_q$ to each quality level $j$ in the following sense (where $p^{t' \leftarrow t} = \sum_i \xi^{t'}_i p^{i,t}_q$ is the quality distribution for type $t$ under action distribution $\xi^{t'}$, and $p^t = p^{t \leftarrow t}$):
% \begin{enumerate}
%     \item \textbf{Preserving Revenue:} $\sum_q y^t_q p^t_q = w^t$;
%     \item \textbf{Preserving IR: } $ y^t_q  + \min \{ x^t_q, v^t_q \} \leq v^t_q $; 
%     \item \textbf{Preserving IC: } Let $U(t';t) = \sum_q \big[  p^{t' \leftarrow t}_q \max\{0, v^t_q - x^{t'}_q \} - p^{t'}_q y^{t'}_q   \big] $ denote type $t$'s expected utility when misreporting $t'$, then we have  $U(t';t) \leq U(t; t)$. 
% \end{enumerate}
% Conversely, for any $\{(y^{t}, \xi^t, x^{t}) \}_{t}$ satisfying the three conditions above, $\{(w^{t}, \xi^t, x^{t}) \}_{t}$ with $w^t = \sum_q y^t_q p^t_q$ is a an IC and IR menu for the problem. 
% \end{lemma}
% \begin{proof}
% [Proof Sketch.] The proof is actually quite simple. Condition (3) is trivial and follows from IC definition. The only non-trivial part is that Condition (2) is fixed-wise for $j$ whereas original IR holds as $\sum_q  p^t_q [\max \{ 0, v^t_q - x^t_q\} - y^t_q ] \geq 0 $, which is weaker. However, it is easy to see that there exists $y^t_q$ that can make the IR constraint satisfied quality-wise. 
% \end{proof}
% \begin{corollary}
%     $y^t_q = 0$ if $x^t_q \geq v^t_q$. 
% \end{corollary}

% Lemma \ref{lem:pay-attribution} offers an alternative characterization of IC and IR menus. Despite a simple fact, this lemma allows us to  ``disentangle'' the payment design for each quality $j$ as an \emph{ex-ante} payment $\frac{y^t_q}{p^t_q}$ and a \emph{ex-post} payment $x^t_q$, defined below
% \begin{definition}
% Due to Lemma \ref{lem:pay-attribution}, henceforth, we denote an IC, IR mechanism as  $\{(y^{t}, \xi^t, x^{t}) \}_{t}$. Moreover, we will call  $ y^t_q $  the \emph{ex-ante} payment  for quality $j$ and   $x^t_q$ the \emph{ex-post} payment.
% \end{definition}
% Notably, from any type $t$'s own perspective, ex-ante and ex-post payments do not make a difference to $t$'s own payment. This distinction is primarily introduced to allow price discrimination and enforce IC constraints. The only  constraint that entangles these payments across qualities are the IC constraints.  Condition (3) in Lemma \ref{lem:pay-attribution} offers a clean distinction among the two payments, in terms of their effects to IC constraints. In particular, the ex-ante payment $y^{t'}_q$ enters the total payment with a multiplier $p^{t'}_q$ \emph{without} a choice of not paying, whereas the ex-post payment $x^{t'}_q$  enters the payment with a multiplier  $p^{t'\leftarrow t}_q$ \emph{with} a choice of not paying. These differences give rise to different power for enforcing IC constraints. As one would naturally observe, ex-post payment $x^{t'}_q$ is preferred, i.e., making $U(t';t)$ small,  when $p^{t'\leftarrow t}_q \leq p^{t'}_q$ and $v_q^t > x_q^{t'}$, whereas ex-ante payment $y^{t'}_q$ is preferred when $p^{t'}_q \leq  p^{t'\leftarrow t}_q$ or $v_q^t \leq x_q^{t'}$. 

% \textbf{Solving the Two-Type Case} 

% Next, we will   use the above lemma to solve  the binary-type case with \emph{deterministic} efforts. That is, I will assume for now $\xi^{t} \in \{ e_1, \cdots, e_m \}$ is deterministic in my arguments below (I think randomized ones is probably similarly solvable). The fundamental reason that this is doable is because there will only be a single IC constraint in this case, hence correspond to a Knapsack problem. 



% Without loss of generality, let me assume $t_1$ is the type who only has usage payment but no performance payment. The argument for the $t_2$ case is symmetric, and the optimal solution can be obtained by comparing the two situations to obtain the better one. 

% My main algorithmic step is to solve the following "promised version" of the problem. 
% \begin{problem}\label{prob:promised}[The promised problem] The promised version of the menu design problem is the following simplified variant: suppose some oracle promised  that the optimal solution is a contract $(y^1, a^1, 0)$ for type $t_1$ and $(y^{?}, a^2, x^{?})$ for type $t_2$, and moreover the revenue from $t_1$ is at least that from $t_2$. The promised problem is to  find the missing $y^{?}, x^{?}$ values.  
% \end{problem}

% The following is a simple observation about the above promised problem. It may not be useful for designing the algorithm, but could serve as a sanity check for our algorithm design as we shall see that type $t_2$ indeed has $0$ surplus in our construction.   
% \begin{observation}
% Type $t_2$ will have $0$ surplus at optimal solution of the promised problem above.
% \end{observation}
% \begin{proof}
% This simply because, if $t_2$ has any positive surplus, increasing his usage payment will simultaneously increase revenue and strengthen the (only) IC constraint since the incentive for $t_1$   to misreport $t_2$ is even further reduced.  
% \end{proof}

% Note that, we know the optimal solution satisfies the structure of the above promised problem for some $y^1, a^1, a^2$. Hence, if we can solve the above problem, we can exhaustively search for $a^1, a^2, y^1$, which is easily doable since $y^1$ comes to the problem only through a single value $w^1 = \sum_q p^{a^1,1}_q y^1_q$ --- indeed, this will be the value we use in the following program \eqref{op:binary-program}.

% \begin{lemma}[Solving the Promised Problem]
% The promised problem can be solved by the following optimization program with only $y, x \in \mathbb{R}^n_+$ as variables ($a^1, a^2, w^1$ are assumed to be given)
% \begin{lp}\label{op:binary-program}
%     \maxi{ \sum_{j} p^{a^2, 2}_q \bigg( y_q + x_q \mathbb{I} ( v_q^2 \geq x_q ) \bigg)}
%     \st 
%     \qcon{y_q + \min \{ x_q, v_q^2 \} \leq v^2_q}{j = 1, \cdots, n}
%     \con{  w^1 \geq  \sum_q p^{a^2, 2}_q  \big(y_q + x_q \mathbb{I} ( v_q^2 \geq x_q )  \big)  }
%     \con{ \sum_q \big[  p^{a^2, 1}_q \max \{0, \,  v^1_q - x_q \} - p^{a^2, 2}_q y_q   \big] \leq u^1 }
% \end{lp}
% where $u^1 =  \sum_q p^{a^1, 1}_q v^1_q -w^1 = \sum_q p^{a^1, 1}_q [v^1_q -y^1_q]  \geq 0$ is type $t_1$'s surplus under truthful report of $t_1$. 
 
% \end{lemma}
% \begin{proof} This is almost by definition. Just note that there is only one IC constraint $U(t_2; t_1) \leq U(t_1, t_1) = u^1$ (the last constraint) since we do not worry about $t_2$ switches to $t_1$ (due to the middle constraint) 
% \end{proof}

% Next lemma characterizes the major properties of the OP \eqref{op:binary-program}, which makes it a knapsack problem. 
% \begin{lemma}\label{lem:binary-opt-property}
% Suppose the optimal objective of OP \eqref{op:binary-program} is strictly smaller than $w^1 = \sum_q p^{a^1, 1}_q  y^1_q $, then there always exists an optimal solution $(y^*, x^*)$ to OP \eqref{op:binary-program} that satisfies the following properties. 

% For any  $j$,
% \begin{enumerate}
%     \item if $p_q^{a^2, 2} \geq p_q^{a^2,1}$ \emph{AND} $  v_q^1 \leq  v^2_q$, then $y^*_q = v^2_q, \, x^*_q = 0$. 
%     \item if $p_q^{a^2, 2} \geq p_q^{a^2, 1}$ \emph{AND} $v^j_1 > v^j_2$, then either $x^*_q = \infty, y^*_q = 0$ (which gives up quality $j$ at $t_2$ and contributes $0$ to  IC constraint) or $x^*_q = 0, y^*_q = v^2_q$ (which achieves maximum possible charge at $j$ and also is the best for IC due to $p_q^{a^2, 2} \geq p_q^{a^2, 1}$); 
%     \item if $p_q^{a^2, 2} < p_q^{a^2,1}$ \emph{AND} $v_q^1 \leq v_q^2$, then $  x^*_q = v_q^1 $ and $y^*_q = v^2_q - v_q^1$ (because this obtains maximum possible charge $v^2_q$ from $t_2$ on quality $j$ and contributes the most for IC by making the $ p^{a^2, 1}_q \max \{0, \,  v^1_q - x_q \} - p^{a^2, 2}_q y_q$ the smallest).   
%     \item if $p_q^{a^2, 2} < p_q^{a^2,1}$ \emph{AND} $v_q^1 > v_q^2$, the either either $x^*_q = \infty, y^*_q = 0$ (which gives up quality $j$ at $t_2$ and contributes $0$ to  IC constraint) or $y^*_q = 0, x^*_q = v_q^2$ (which achieves maximum possible charge at $j$ and also is the best for IC due to $p_q^{a^2, 2} < p_q^{a^2, 1}$) 
% \end{enumerate}

% \end{lemma}
% \begin{proof}
% If  the optimal objective of OP \eqref{op:binary-program} is strictly smaller than $w^1 = \sum_q p^{a^1, 1}_q  y^1_q $, then the second constraint is redundant, so the problem only has the first and third constraint. Under these two constraints, we prove the properties described above.

% \textbf{Proof of the   claim 1:} for any originally optimal solution with $\bar{y}_q, \bar{x}_q$ for some $j$ satisfying  $ v_q^1 \leq   v^2_q$ and $p_q^{a^2, 2} \geq p_q^{a^2,1}$, we observe that switching to $y^*_q = v^2_q, \, x^*_q = 0$ (without changing any other $y,x$ values)    will not violate the first constraint in OP \eqref{op:binary-program} (by definition of $y^*_q, x^*_q$), will not decrease the objective value (which is trivial because the maximum possible value $v_q^2$ is achieved for term $y_q + x_q \mathbb{I} ( v_q^2 \geq x_q )$ in the objective),   and will only strengthen the last IC constraint, which is argued below
%  \begin{itemize}
%      \item If $ \bar{x}_q > v^1_q$, then the first constraint   $y_q + \min \{ x_q, v_q^2 \} \leq v^2_q$ implies $\bar{y}_q \leq  v^2_q -  \min \{ \bar{x}_q, v_q^2 \} \leq  v^2_q - v^1_q$, hence the $j$'th term in IC constraint is at least $-(v^2_q - v^1_q) p_q^{a^2, a}$, but this term under $y^*_q = v^2_q, \, x^*_q = 0$ is $p^{a^2, 1}_q v_q^1 - p^{a^2, 2}_q v_q^2$ which is strictly smaller since $p^{a^2, 1}_q \leq p^{a^2, 2}_q$. 
%      \item If $ \bar{x}_q \leq v^1_q (\leq v_q^2)$, then the first constraint   $y_q + \min \{ x_q, v_q^2 \} \leq v^2_q$ implies $\bar{y}_q + \bar{x}_q  \leq v^2_q$, we thus have   
%      \begin{eqnarray*}
%          p^{a^2, 1}_q \max \{0, \,  v^1_q - \bar{x}_q \} - p^{a^2, 2}_q \bar{y}_q &=&  p^{a^2, 1}_q   (v^1_q - \bar{x}_q ) - p^{a^2, 2}_q \bar{y}_q  \\
%          &\geq &  p^{a^2, 1}_q    v^1_q -  p^{a^2, 2}_q 
%  \bar{x}_q  - p^{a^2, 2}_q \bar{y}_q  \\ 
%           &\geq &  p^{a^2, 1}_q    v^1_q -   - p^{a^2, 2}_q v_q^2  \\  
%           & = &   p^{a^2, 1}_q \max \{0, \,  v^1_q - x^*_q \} - p^{a^2, 2}_q y^*_q 
%      \end{eqnarray*} 
%  \end{itemize}

% \textbf{Proof of the   claim 2:}   I think this is true, with similar arguments above, but please verify it.

% \textbf{Proof of the   claim 3:}  I think this is true, with similar arguments above, but please verify it.

% \textbf{Proof of the   claim 4:}  I think this is true, with similar arguments above, but please verify it.

% \end{proof}

%  The four conditions in Lemma \ref{lem:binary-opt-property} leads to a partition of the qualities into 4 categories, defined below:
% \begin{itemize}
%     \item $A = \{j: p_q^{a^2, 2} \geq p_q^{a^2,1} \text{ and } v_q^1 \leq v_^2 \}$; 
%     \item $B = \{j: p_q^{a^2, 2} \geq p_q^{a^2,1} \text{ and } v_q^1 > v_^2 \}$;  
%     \item $A = \{j: p_q^{a^2, 2} < p_q^{a^2,1} \text{ and } v_q^1 \leq v_^2 \}$; 
%     \item $A = \{j: p_q^{a^2, 2} < p_q^{a^2,1} \text{ and } v_q^1 > v_^2 \}$.
% \end{itemize}
% Lemma \ref{lem:binary-opt-property} shows that for any $j \in A \cup C$, the correspond $x^*_q, y^*_q$ values are already determined. Hence all we need to finalize are these values for $j \in B \cup D$. This turns out to be a fractional knapsack problem, since the tradeoff here is about the violation to the only IC constraint and the increase to the objective value. 

% \textbf{Formulating the  Knapsack problem. } First of all, the   $j \in A \cup C$ contributed to the budget of knapsack with cost defined below
% \begin{equation}\label{eq:knapsack-budget}
%     C = \sum_q p^{a^1, 1}_q v^1_q -w^1  + \sum_{j \in A} [-p_q^{a^2,1} v_q^1 + p_q^{a^2, s} v^2_q]  + \sum_{j \in C} p_q^{a^2, 2} (v^2_q - v_q^1) 
% \end{equation}
% where $ \sum_q p^{a^1, 1}_q v^1_q -w^1 = u_1$ is type $t_1$'s surplus. 
% The $j$'s in set $B, D$ are our items. By Lemma \ref{lem:binary-opt-property}, for any $j \in B$, either $x_q^* = \infty$ meaning we will not pick quality/item $j$, or $x_q^* = 0, y^*_q = v_q^2$ meaning we will pick this item which has value $p_q^{a^2, 2}v_q^2$  and cost $c_q = p_q^{a^2,1} v_q^1 - p_q^{s^2, 2} v_q^2$ (note that this cost may be negative, which will even contribute to our cost and we should for sure pick it). Similarly, for any $j \in D$, either $x_q^* = \infty$ meaning we will not pick quality/item $j$, or $y_q^* = 0, x^*_q = v_q^2$ meaning we will pick this item which has value $p_q^{a^2, 2} v_q^2$ and cost $c_q = p_q^{a^2,1} (v_q^1 - v_q^2)$ (this cost is for sure positive). Finding the optimal $(x^*, y^*)$ boils down to find a subset   $J \subset B \cup D$ that maximizes the total value subject to their total cost sum up to at most $C$. Note that fractional item is not allowed here by Lemma \ref{lem:binary-opt-property}. 

% \textbf{Solving the Two-Type Problem Efficiently. } The Knapsack formulation may make the problem appear NP-hard at the first glance, but fortunately it actually turns out to have an efficient algorithm. The main reason is that  the designer can trade her (continuous) revenue $w_1$ from type $t_1$ for more cost budget for solving the above Knapsack problem. This effectively smoothed the Knapsack problem to make it admit an efficient algorithm. This tradeoff has economic interpretation ---- we can reduce $t_1$'s payment $w_1$ to strengthen $t_1$'s incentive of being truthful, which will then increase the cost budget defined in Equation \eqref{eq:knapsack-budget} so that we can pack more items within the cost $C$. Recall that the designer's objective is $$\mu(t_1) \times w_1 + \mu(t_2) \times \sum_{j} p^{a^2, 2}_q \bigg( y_q + x_q \mathbb{I} ( v_q^2 \geq x_q ) \bigg)$$
% where  $\mu(t)$ is the probability of type $t$. 
% Therefore, the  cost for reducing each unit of $w_1$ (i.e., increasing a unit value of $C$) is $\mu(t_1)$ whereas the value for having one unit of item $j$ is $\mu(t_2) p_q^{a^2, 2} v_q^2/c_q$ where $c_q$ is defined as in the Knapsack problem above. Since $w_1$ is continuous, this problem can be solved by sorting $\mu(t_2) p_q^{a^2, 2} v_q^2/c_q$ decreasingly, and pick all the $j$'th such that $\mu(t_2) p_q^{a^2, 2} v_q^2/c_q \geq \mu(t_1)$, and picking $w_1$ to be precisely the value so that  the cost $C$ equals the sum of all these picked items.  

% TODO: there is a slight issue here since we also need to guarantee $w_1  \geq    \sum_{j} p^{a^2, 2}_q \bigg( y_q + x_q \mathbb{I} ( v_q^2 \geq x_q ) \bigg) $, but I think this should be fixable by making more careful arguments above. 
% }

% \paragraph{Quality decomposition for two types}

% \begin{definition}
%     Define a \emph{sub-menu} to be a menu restricted to a single quality level $j$. A menu $(w^t, \xi^t, x^t)_{t\in \Theta}$ has $\ab{Q}$ sub-menus of the form $(w^t_q, \xi^t, x^t_q)_{t\in \Theta}$ where $(w^t_q)_{j\in Q}$ are chosen arbitrarily such that $\sum_{j\in Q} w^t_q = w^t,\fl t$. For a sub-menu associated with a quality level $j$, the transition probabilities $p^{t,t'}_q$ are the same as in the original menu, and with probability $1-p^{t,t'}_q$ the action $\xi^{t'}$ results the trivial quality level for type $t$.
% \end{definition}

% \begin{definition}
%     Define the \emph{concatenation} of sub-menus $(w^t_q, \xi^t, x^t_q)_{t\in \Theta, j\in Q}$ to be the menu $$\bp{\sum_{j\in Q} w^t_q, \xi^t, (x^t_q)_{j\in Q}}_{t\in \Theta}.$$ Note that the effort distribution $(\xi^t)_{t\in \Theta}$ must be the same for all quality levels for concatenation to make sense.
% \end{definition}

% \begin{proposition}
%     The concatenation of sub-menus that are IC and IR returns a menu that is also IC and IR. Furthermore the total revenue of the menu is the sum of the revenues of the sub-menus.
% \end{proposition}

% \begin{proof}
%     This follows from the fact that the global IC and IR constraints can be written as the sum of the IC and IR constraints for the sub-menus.
% \end{proof}

% \begin{definition} \label{performance-prices-ic}
%     For a sub-menu associated to quality level $j$, call a tuple of performance payments $(x^1_q, x^2_q)$ \emph{incentive-compatible} (IC) if there exist training payments $(w^1_q, w^2_q)$ that make the menu of contracts for the sub-menu incentive-compatible.
% \end{definition}

% We write down the IC constraints for a sub-menu explicitly as
% \begin{align*}
%     \begin{split}
%         -w^1_q + p^{1,1}_q \max\bc{0, v^1_q - x^1_q} &\ge -w^2_q + p^{1,2}_q \max\bc{0, v^1_q - x^2_q} \\
%         -w^2_q + p^{2,2}_q \max\bc{0, v^2_q - x^2_q} &\ge -w^1_q + p^{2,1}_q \max\bc{0, v^2_q - x^1_q}
%     \end{split}
% \end{align*}
% and rearrange to
% \begin{align}\label{ic-submenu}
%     \begin{split}
%         w^2_q - w^1_q \in \bb{p^{1,2}_q \max\bc{0, v^1_q - x^2_q} - p^{1,1}_q \max\bc{0, v^1_q - x^1_q}, p^{2,2}_q \max\bc{0, v^2_q - x^2_q} - p^{2,1}_q \max\bc{0, v^2_q - x^1_q}}.
%     \end{split}
% \end{align}
% We interpret $$U^{t',t}_q \coloneq p^{t',t}_q \max\bc{0, v^{t'}_q - x^t_q}$$ as the \emph{performance utility} for type $t'$ under contract $t$ in the sub-menu for quality $j$ and also define $U^{t',t} = \sum_{j\in Q} U^{t',t}_q$ as the global performance utility. The IC condition for performance prices $(x^1_q, x^2_q)$ can be written as $$U^{1,2}_q - U^{1,1}_q \le U^{2,2}_q - U^{2,1}_q \iff U^{1,1}_q + U^{2,2}_q \ge U^{1,2}_q + U^{2,1}_q.$$ Define the \emph{performance utility gap} as $$\Delta(x^1_q, x^2_q) \coloneq U^{1,1}_q + U^{2,2}_q - U^{1,2}_q + U^{2,1}_q$$ so that the IC condition is $\Delta(x^1_q, x^2_q) \ge 0$. We use the shorthand $\Delta_i \coloneq \Delta(x^1_i, x^2_i) < 0$ for the performance utility gap for quality $i$. Returning now to the original menu, which is the concatenation of $\ab{Q}$ sub-menus corresponding to different quality levels, we see that the sum of the performance utility gaps across all quality levels must be nonnegative, that is, $$\sum_{j=1}^\ab{Q} \Delta(x^1_q, x^2_q) \ge 0,$$ since there exist global training payments, namely $(w^1, w^2)$, that make the menu IC. Conversely, if the sum of the performance utility gaps is nonnegative, then there exist global training payments for which the menu is IC.

% \begin{lemma} \label{performance-prices-ic-transformation}
%     Any IC and IR menu can be transformed into another IC and IR menu while weakly increasing revenue such that the performance prices $(x^1_q, x^2_q)$ of the new menu are all IC.
% \end{lemma}

% \begin{proof}
%     We claim that we can change the performance payments to make all of the performance utility gaps nonnegative while ensuring that the total revenue does not decrease. Our technique will be to decrease the performance payment of qualities with a negative performance utility gap in a direction that makes the gap less negative while simultaneously increasing the training payments to compensate for the performance revenue loss.

%     % By the global IC constraint, it must be the case that $$\sum_{\text{$i$ not IC}} \ab{\Delta_i} \le \sum_{\text{$j$ IC}} \Delta_q.$$ 

%     % We first consider a sub-menu $(x^1_q, x^2_q)$ that has positive performance utility gap $\Delta_q > 0$. Note that $$x^1_q = x^2_q\imp U_q^{1,1} = U^{1,2}_q, U^{2,1}_q = U^{2,2}_q \imp \Delta_q = 0,$$ so $$\Delta_q\neq 0\imp x^1_q\neq x^2_q.$$ Without loss of generality assume $$x^1_q < x^2_q\imp U^{1,1}_q \ge U^{1,2}_q, U^{2,1}_q \ge U^{2,2}_q \xRightarrow{\Delta_q > 0} U^{1,1}_q > U^{1,2}_q \ge 0 \imp x^1_q < v^1_q.$$ Consider increasing the performance price $x^1_q$ by $\eps$ until $x^1_q = x^2_q$ or $x^1_q = v^1_q$, whichever happens earlier. Note that type 1 will still purchase quality $j$, hence the performance revenue increases by $p^{1,1}_q \eps$ while the performance utility $U^{1,1}_q$ decreases by $p^{1,1}_q \eps$. The performance utility $U^{2,1}_q$ weakly decreases while $U^{1,2}, U^{2,2}$ are unchanged, hence the performance utility gap $\Delta_q$ decreases by at most $p^{1,1}_q \eps$. (We have not technically shown that $\Delta_q$ actually decreases as opposed to increase, but below we see that the choice of $\eps$ makes the performance utility gap now 0, so $\Delta_q$ must have decreased.) The crucial fact is that the increase in revenue is at least the decrease in performance utility gap. Note that when $\eps = \min\bc{x^2_q, v^1_q} - x^1_q$ then at this point either $x^1_q = x^2_q$ or $x^1_q = v^1_q$. In the former case the performance utility gap is now 0 and in the latter case the performance utility gap is nonpositive, which proves that the performance utility gap actually does decrease as $\eps$ increases. Furthermore, the performance utility gap decreases continuously on $\eps\in \bb{0, \min\bc{x^2_q, v^1_q} - x^1_q}$, so for every desired decrease in performance utility gap in $[0, \Delta_q]$ we can find an $\eps$ that achieves it, noting that the increase in performance revenue is at least the decrease in performance utility gap.

%     % Consider first increasing $x^1_i$ by $\eps$ until $x^1_i = x^2_i$ or $x^1_i \ge v^1_i$, whichever happens earlier. We split into cases:
%     % \begin{itemize}
%     %     \item If $x^1_i \ge v^1_i$ then nothing happens and we move on.
%     %     \item If $x^1_i < x^2_i \le v^1_i$ then after increasing $x^1_i$ we have $x^1_i = x^2_i$. At this point we stop, noting that the performance utility gap is now 0 and that the performance revenue weakly increases.
%     %     \item If at this point $x^1_i = v^1_i$ then note that $U^{2,1}_i$ decreased, $U^{1,1}_i$ decreased by $p^{1,1}_i \eps$, and the performance revenue increased by $p^{1,1}_i \eps$. Hence the increase in performance revenue is at least the decrease in performance utility gap.
%     %     \item Otherwise $x^1_i \ge $
%     % \end{itemize}
    
%     Consider a quality $j$ for which $\Delta_q < 0$. Without loss of generality assume $$x^1_q < x^2_q\imp U^{1,1}_q \ge U^{1,2}_q, U^{2,1}_q \ge U^{2,2}_q \xRightarrow{\Delta_q < 0} U^{2,1}_q > U^{2,2}_q \ge 0 \imp x^1_q < v^2_q.$$ Consider decreasing $x^2_q$ until it reaches $x^1_q$. At this point we see that $U^{1,1}_q, U^{2,1}_q$ are unchanged and $U^{1,2}_q, U^{2,2}_q$ weakly increased by $\eps^{1,2}_q, \eps^{2,2}_q>0$ respectively. Since the performance utility gap when $x^2_q = x^1_q$ is now zero, it must have been the case that $U^{2,2}_q$ increased more than $U^{1,2}_q$, that is, $\eps^{1,2}_q < \eps^{2,2}_q$. Note that the loss in type 2's performance revenue after decreasing $x^2_q$ is at most $\eps^{1,2}_q$ since the sum of type 2's performance utility and performance revenue equals their value $v^2_q \cd \one[v^2_q \ge x^2_q]$ for quality $i$ when faced with price $x^2_q$, and this value weakly increases when $x^2_q$ decreases. To compensate for the loss in performance revenue, we increase type 2's training payment $w^2$ by $\eps^{1,2}_q$. Note that the global utility $U^{2,2}$ is unchanged since $w^2$ increased by exactly the same amount as $U^{2,2}_q$ decreased and the global utility $U^{1,2}$ decreased since $\eps^{1,2}_q < \eps^{2,2}_q$, so the global IC constraints still hold. The global utilities $U^{1,1}, U^{2,2}$ are unchanged, so the global IR constraints still hold.
    
%     % Note that type 2 will still purchase quality $i$, hence the performance revenue increases by $p^{1,1}_i \eps$ while the performance utility $U^{2,1}_i$ decreases by $p^1_q \eps$. The performance utility $U^{2,1}_q$ weakly decreases while $U^{1,2}, U^{2,2}$ are unchanged, hence the performance utility gap $\Delta_q$ decreases by at most $p^1_q \eps$. (We have not technically shown that $\Delta_q$ actually decreases as opposed to increase, but below we see that the choice of $\eps$ makes the performance utility gap now 0, so $\Delta_q$ must have decreased.) The crucial fact is that the increase in revenue is at least the decrease in performance utility gap. Note that when $\eps = \min\bc{x^2_q, v^1_q} - x^1_q$ then at this point either $x^1_q = x^2_q$ or $x^1_q = v^1_q$. In the former case the performance utility gap is now 0 and in the latter case the performance utility gap is nonpositive, which proves that the performance utility gap actually does decrease as $\eps$ increases. Furthermore, the performance utility gap decreases continuously on $\eps\in \bb{0, \min\bc{x^2_q, v^1_q} - x^1_q}$, so for every desired decrease in performance utility gap we can find an $\eps$ that achieves it, noting that the increase in performance revenue is at least the decrease in performance utility gap. Performing the above process for all such qualities $j$ for which $\Delta_q$ yields the desired result.

%     % The performance price increase might break type 1's IR constraint. Hence we should decrease type 1's training payment $w^1$ by $p^{1,1}_i \eps$ in order to maintain the IR constraint, noting that type 1's revenue is unchanged overall. (We allow for negative training payments in intermediate stages.)
    
%     % [Maybe we don't need this] We first note that without loss of generality at least one $x^1_i, x^2\in \bc{0, v^1, v^2}$. This is because the IC constraints are linear in $(x^1, x^2)$ away from the boundary values, so for any interior $(x^1, x^2)$ one of the two perturbation directions will weakly increase the performance utility gap, meaning that the global IC constraint is maintained, while leaving all 
% \end{proof}

% \begin{definition}
%     Let a decomposition $(w^t_q, \xi^t, x^t_q)_{j\in Q, t\in \Theta}$ of a menu $(w^t, \xi^t, x^t)_{t\in \Theta}$ into $\ab{Q}$ sub-menus be \emph{valid} if $(w^t_q)_{j\in Q}$ are chosen such that $\sum_{j\in Q} w^t_q = w^t,\fl t$ and every sub-menu is both IR and IC.
% \end{definition}

% For a decomposition to be valid, it certainly must be the case the performance prices $(x^1_q, x^2_q)$ are all IC according to \cref{performance-prices-ic} for training payments $(w^1_q, w^2_q)$ to even exist. Fortunately, by \cref{performance-prices-ic-transformation}, any IR and IC menu can be transformed such a menu while weakly increasing revenue.

% \begin{lemma}
%     Any IC and IR menu for which the performance prices $(x^1_q, x^2_q)$ are all IC has a valid decomposition into sub-menus. (This lemma is not true, we need extra conditions.)
% \end{lemma}

% \begin{proof}
%     Recall that the IC constraint \cref{ic-submenu} for sub-menus is $$w^2_q - w^1_q \in \bb{U^{1,2}_q - U^{1,1}_q, U^{2,2}_q - U^{2,1}_q}.$$ Since the performance prices $(x^1_q, x^2_q)$ are all IC, these intervals for all $j$ are nonempty. Note that if $(w^t, \xi^t, x^t)_{t\in \Theta}$ has a valid decomposition then so does $(w^t-\eps \cd \one, \xi^t, x^t)_{t\in \Theta}$ since we can choose an arbitrary quality level and decrease the sub-menu training payments $(w^t_q, w^t_q)_{t\in \Theta}$ by $\eps$ each. Note that the global and sub-menu IC constraints still hold and the global and sub-menu IR constraints are weakly relaxed. Hence for convenience, we can first increase $w^t$ by $\eps\cd \one$ so that at least one global IR constraint is binding, without loss of generality type 1, noting that this transformation weakly increases revenue. The binding global IR constraint means that $$\sum_{j\in Q} w^1_q = w^1 = U^{1,1} = \sum_{j\in Q} U^{1,1}_q,$$ which combined with the sub-menu IR constraints $w^1_q \le U^{1,1}_q,\fl j$ imply that the only valid sub-menu training payments for type 1 are $w^1_q = U^{1,1}_q,\fl j$. It remains to choose $(w^2_q)_{j\in Q}$ to satisfy
%     \begin{align*}
%         \sum_{j\in Q} w^2_q &= w^2 \\
%         w^2_q &\le U^{2,2}_q && \te{(IR)} \\
%         w^2_q &\in \bb{U^{1,2}_q, U^{2,2}_q - U^{2,1}_q + U^{1,1}_q} && \te{(IC)} 
%     \end{align*}
%     assuming the global constraints
%     \begin{align*}
%         w^2 &\le \sum_{j\in Q} U^{2,2}_q && \te{(IR)} \\
%         w^2 &\in \bb{U^{1,2}, U^{2,2} - U^{2,1} + U^{1,1}} && \te{(IC)} 
%     \end{align*}
% \end{proof}

% \paragraph{Two types and fixed action}

% From Lemma~\ref{highest-revenue-no-performance}, there exists an optimal menu with no usage payments for one type. Without loss of generality, we assume that type $t_1$, there exists an optimal menu such that $x_q^{t_1} = 0$ for all $j$.

% To derive the optimal menu, we divide the solution space into four spaces. Let $p_q^{t, t'} =  \sum_{i=1}^m \xi^{t'} p_q^{i, t}$  denote the probability of inducing quality level $j \in Q$ under the menu $t'$. For simplicity, $p_q^t = p_q^{t, t}$. And let $IR(t)$ denote the utility of type $t$ agent choosing the menu $(w^t, x^t)$ for $t = t_1, t_2$ under the fixed effort distribution $\xi$, i.e., $U(t) = \sum_{j=1}^n p_q^t \max \{v_q^{t} - x_q^{t}, 0\}- w^{t}$. From the individual rationality constraints, we have that $U(t) \geq 0$ for $t = t_1, t_2$. We denote the space $\mathcal{S}_1 = \{w, x: U(t_2) = 0 \}, \mathcal{S}_2 = \{w, x: U(t_1)  = 0 \}, \mathcal{S}_3 = \{w, x: U(t_1)  > 0, U(t_2) > 0 \}$. Note that the set $\mathcal{S}_1, \mathcal{S}_2, \mathcal{S}_3$  depends on the effort distribution $\xi$, we omit the dependence of $\xi$ in the notation. We derive the optimal menu in the set $\mathcal{S}_1, \mathcal{S}_2$ and prove that any solution in the $\mathcal{S}_3$ could not achieve the strictly better revenue than the optimal solution in $\mathcal{S}_1$ or $\mathcal{S}_2$. 
% \begin{enumerate}
%   \item $(w, x) \in \mathcal{S}_3$. For any solution $(w, x) \in \mathcal{S}_3$, because $U(t_1) > 0, U(t_2) > 0$, the agents with type $t_1$ and type $t_2$ both have positive utility. In this case, by increasing $w^{t_1}, w^{t_2}$ with the same amount, until at least one type of agent achieves $0$ utility, it would achieve strictly better revenue with a guarantee that the IC constraints hold.
  
%    \item $(w, x) \in \mathcal{S}_2$. Within this solution space, the agent $t_1$ always has $0$ utility. We assume that the menu for $t_1$ does not have usage payment, i.e., $x_q^{t_1} = 0$ for all $j \in Q$. Thus, the principal extracts the largest revenue from agent $t_1$ by charging the training payment $w^{t_1} = \sum_q p_q^{t_1} v_q^{t_1}$. Because the menu $(w^{t_1}, x^{t_1})$ is fully determined in this space, the remaining is to design the optimal menu for $t_2$ to achieve the optimal revenue. We show that no usage payments for any quality levels and transforming all payments to training payments can weakly increase the principal's revenue. To see this, for any usage payment $x_q^{t_2} > 0$, 
%    The optimal menu can be found by solving the following program by maximizing the objective over $w^{t_2}, x^{t_2}$. 
%     \begin{align*}
%         \max_{w^{t_2}, x^{t_2}} &\expec \left[ \sum_{j} p_q^{t_1} v_q^{t_1} + \sum_{j} p_q^{t_2} x_q^{t_2}\ind \{v_q^{t_2} - x_q^{t_2} \geq 0 \} + w^{t_2}\right] \\
%         & 0 \geq \sum_{j}p_q^{t_1, t_2}\max \{v_q^{t_1} - x_q^{t_2}, 0\}- w^{t_2}, \\
%         & \sum_{j} p_q^{t_2}\max \{v_q^{t_2} - x_q^{t_2}, 0\}- w^{t_2} \geq \sum_{j} p_q^{t_2, t_1}v_q^{t_2}- \sum_{j}p_q^{t_1} v_q^{t_1} \\
%         & \sum_{j} p_q^{t_2}\max \{v_q^{t_2} - x_q^{t_2}, 0\}- w^{t_2} \geq 0, \\
%         & x_q^{t_2} \in \{0, v_q^{t_1}, v_q^{t_2} \}, \forall j.
%     \end{align*}
%     For any $x^{t_2}$ such that this program is feasible, we have that 
%     \begin{align*}
%         w^{t_2} \leq \sum_{j}  p_q^{t_2}\max \{v_q^{t_2} - x_q^{t_2}, 0\} + \min \left\{\sum_{j}(p_q^{t_1} v_q^{t_1} -p_q^{t_2, t_1}v_q^{t_2}), 0\right\}. 
%     \end{align*}
%     % \begin{align*}
%     %     \sum_{i,j}\xi^{t_2}_i p_q^{i, t_1}\max \{v_q^{t_1} - x_q^{t_2}, 0\} \geq  \sum_{i,j} \xi^{t_2}_i p_q^{i, t_2}\max \{v_q^{t_2} - x_q^{t_2}, 0\} + \min \{\sum_{i, j}\xi_i^{t_1}p_q^{i, t_1} v_q^{t_1} - \sum_{i, j}\xi^{t_1}_i p_q^{i, t_2}v_q^{t_2}, 0\}
%     % \end{align*}
%     \yzcomment{$x_q^{t_2} = 0$ for all $j$ may not be feasible. Need fixed}
%     For type $t_2$ agent, the optimal $x^{t_2}$ is such that the right-hand side of the inequality is the largest among all $\bar{x}_q^{t_2} \in \{0, v_q^{t_1}, v_q^{t_2} \}$. Note that by setting  $x^{t_2} = 0$, the right-hand side can achieve the maximum and the principal achieves the optimal revenue  
%     \begin{align*}
%         \expec \left[ \sum_{j}p_q^{t_1} v_q^{t_1} + \sum_{j}  p_q^{t_2} v_q^{t_2} + \min \left\{\sum_{j}(p_q^{t_1} v_q^{t_1} -p_q^{t_2, t_1}v_q^{t_2}), 0\right\}\right]. 
%     \end{align*}
  
%     \item $(w, x) \in \mathcal{S}_1$. For any menu $(w, x) \in \mathcal{S}_1$, agent $t_2$ receives $0$ utility. Therefore, the training payment $w^{t_2}$ in the optimal menu must be set such that agent $t_2$ has $0$ utility. Consequently, the principal's revenue from agent $t_2$ is given by $\sum_{j} p_q^{t_2} v_q^{t_2}\ind \{v_q^{t_2} - x_q^{t_2} \geq 0 \}$. This implies that if agent $t_2$ accepts quality level $j$, the principal can extract the full revenue from agent $t_2$ at quality level $j$, and $0$ otherwise. 
    
%     Finding the optimal menu thus reduces to determining the optimal training payment for agent $t_1$ and the usage payment for agent $t_2$. Increasing the training payment for agent $t_1$ and the usage payment can lead to greater revenue from agent $t_1$. However, agent $t_1$ may opt to switch to option $t_2$. To prevent this scenario, the principal may consider increasing the usage payment $x^{t_2}$ as well, ensuring that agent $t_1$ does not benefit from switching to option $t_2$. However, this adjustment may result in a loss in the principal's revenue from agent $t_2$. Therefore, the tradeoff lies in determining the optimal usage payment $x^{t_2}$ for agent $t_2$ and training payment for agent $t_1$ to maximize the difference between the revenue gain from agent $t_1$ and the revenue loss from agent $t_2$.

%     As the usage payment $x^{t_1}$ is zero and $w^{t_2}$ is determined by $x^{t_2}$, the optimal menu can be found by maximizing the revenue over $w^{t_1}$ and $x^{t_2}$. This problem can be formally solved using the following Mixed Integer Linear Program (MILP).
    
    
%     \begin{align}
%         \max_{w^{t_1}, x^{t_2}} &\expec \left[ w^{t_1} + \sum_{j} \xi_i^{t_2} p_q^{t_2} v_q^{t_2}\ind \{v_q^{t_2} - x_q^{t_2} \geq 0 \} \right]\notag \\
%         &\sum_{j}p_q^{t_1} v_q^{t_1} - w^{t_1} \geq \sum_{j} \left( p_q^{t_1, t_2}\max \{v_q^{t_1} -x_q^{t_2}, 0\}-  p_q^{t_2} \max \{v_q^{t_2} - x_q^{t_2} , 0 \}\right)\notag \\
%         & \sum_{j}p_q^{t_1} v_q^{t_1} - w^{t_1} \geq 0, \label{MILP-two-type-case3}\\
%         & 0 \geq \sum_{j}p_q^{t_2, t_1} v_q^{t_2}- w^{t_1}\notag\\
%         &x_q^{t_2} \in \{0, v_q^{t_1}, v_q^{t_2}\}, \forall j \notag.
%     \end{align}
%     We denote the $U_{12}^j(x) = p^{t_1, t_2}_q\max \{v_q^{t_1} -x_q^{t_2}, 0\}-  p_q^{t_2} \max \{v_q^{t_2} - x_q^{t_2} , 0 \}$ to be the utility of agent $t_1$ at quality level $j\in Q$ if she selects menu $t_2$. Hence the total utility of agent $t_1$ from the menu $t_2$ can be written as the sum of utility at each quality level $j$, i.e., $\sum_q U_{12}^j(x)$. For any usage payment $x^{t_2}$, if $ \sum_{j} p_q^{t_2, t_1} v_q^{t_2} > \sum_{j}p_q^{t_1} v_q^{t_1}  - \max\{\sum_q U_{12}^j(x), 0\}$, the MILP~\eqref{MILP-two-type-case3} is infeasible. Otherwise, the optimal $w^{t_1} $ is equal to $\sum_{j}\sum_{j}p_q^{t_1} v_q^{t_1}  - \max\{\sum_q U_{12}^j(x), 0\}$. Therefore, the MILP~\eqref{MILP-two-type-case3} can be rewritten as the following discrete optimization problem. 
%      \begin{align}
%         \max_{x^{t_2}} &\quad \expec \left[ \sum_{j}p_q^{ t_1} v_q^{t_1}  - \max\{\sum_q U_{12}^j(x), 0\} + \sum_{j} p_q^{t_2} v_q^{t_2}\ind \{v_q^{t_2} - x_q^{t_2} \geq 0 \} \right]\notag \\
%         &\sum_{j}p_q^{t_2, t_1} v_q^{t_2} \leq \sum_{j}p_q^{t_1} v_q^{t_1}  - \max\{\sum_q U_{12}^j(x), 0\}\label{IP-two-type-case3}
%     \end{align}
    
%    Let $Q_+ = \{ j \in Q \mid v_q^{t_2} \geq v_q^{t_1} \}$ and $Q_- = Q \setminus Q_+$. Also, we denote $a_q = p_q^{t_2} - p_q^{t_1, t_2}$ to be the difference in agent $t_1$'s utility between $x_q^{t_2} = v_q^{t_1}$ and $0$.\kiedit{We will consider the two cases depending on whether $j \in Q_+$ and not separately.}
   
%    For $j \in Q_+$, setting $x_q^{t_2} = v_q^{t_1}$ can yield weakly better revenue compared to the menu $x_q^{t_2} = v_q^{t_2}$ with the optimal $w^{t_1}$ under $x_q^{t_2}$. This is because for any quality level $j \in Q_+$, $U_{12}^j(v_q^{t_2}) = 0$ and $U_{12}^j(v_q^{t_1}) = -p_q^{t_2} (v_q^{t_2} - v_q^{t_1}) \leq 0$. For any quality level $j$ with $x_q^{t_2} = v_q^{t_2}$, one can decrease it to $v_q^{t_1}$ and increase the training payment $w^{t_1}$. The principal would achieve weakly better revenue from agent $t_1$ and the same revenue from agent $t_2$ agent $t_2$ will still accept quality level $j$. Thus, for $j \in Q_+$, the principal's decision revolves around whether to set $x_q^{t_2} = v_q^{t_1}$ or $0$. For $j \in Q_{+}$, the agent $t_2$ would accept all quality levels because the usage payment $x_q^{t_2}$ would not exceed the value $v_q^{t_2}$ in any case. Therefore, $x_q^{t_2} = v_q^{t_1}$ or $0$ would only affect the utility $U_{12}^j$ for agent $t_1$ at level $j$. The less $U_{12}^j$, the principal has more room to increase $w^{t_1}$ while preserving IC constraint. Specifically, for $j \in \{j \in Q_+: a_q > 0\}$, $U_{12}^j(0) \leq U_{12}^j(v_q^{t_1})$ thus, $x_q^{t_2} = 0$ leads to strictly better revenue while for 
%     $j \in \{j \in Q_+: a_q \leq 0 \}$, $x_q^{t_2} = v_q^{t_1}$ has weakly better revenue. Until now, we have derived the explicit optimal solution for $j \in Q_+$. And the total utility of agent $t_1$ in menu $t_2$ is give by 
%     \begin{align*}
%          \sum_{j \in Q_+} U^j_{12} &= \sum_{j \in Q_+\cap\{j:a_q > 0\}} p_q^{t_1, t_2}v_q^{t_1} + \sum_{ij \in Q_+ \cap\{j:a_q \leq 0\}}p_q^{t_2} v_q^{t_1} - \sum_{j \in Q_+}p_q^{t_2}v_q^{t_2}  < 0
%     \end{align*}
    
%     For $j \in Q_{-}$, observe that $U_{12}^j(v_q^{t_2}) - U_{12}^j(0) = a_q v_q^{t_2}$. Hence, for $j \in \{ j \in Q_{-}: a_q > 0\} $, the principal can have more room for increasing $w^{t_1}$ by setting $x_q^{t_2} = 0$ compared with $v_q^{t_2}$. Thus it can achieve strictly better revenue from agent $t_1$ while keeping the same revenue from agent $t_2$. For any $j \in \{ j \in Q_{-}: a_q \leq 0\} $, $x_q^{t_2} = v_q^{t_2}$ gives weakly better revenue.  Given this, the principal only needs to decide whether $x_q^{t_2} = v_q^{t_1}$ or $0$ for $j \in Q_{-} \cap \{j: a_q > 0\}$ and decide whether $x_q^{t_2} = v_q^{t_1}$ or $v_q^{t_2}$ for $j \in Q_{-} \cap \{j: a_q \leq 0\}$. 


    
%     Since we derive the optimal solution for $j \in Q_{+}$, it remains to solve the program~\eqref{IP-two-type-case3} of choosing the optimal $x_q^{t_2}$ for $j \in Q_{-}$. Let $u_q = \ind \{x_q^{t_2} \neq v_q^{t_1} \}$ denote the binary variable. First, the principal's revenue from $Q_{-}$ can be written as 
%     \begin{align*}
%         - \max\{\sum_{j \in Q_{+}} U_{12}^j + \sum_{j \in Q_{-}} U_{12}^j(u), 0\}  + \sum_{j \in Q_{-}} p_q^{t_2} v_q^{t_2} u_q,
%     \end{align*}
%     where 
%     % \begin{align*}
%     %     \sum_{i,j \in Q_+} \xi^{t_2}_i \left( p_q^{i, t_1}\max \{v_q^{t_1} -x_q^{t_2}, 0\}-  p_q^{i, t_2} \max \{v_q^{t_2} - x_q^{t_2} , 0 \}\right) &= \sum_{i, j \in Q_+\cap\{j:a_q > 0\}} \xi^{t_2}_i p_q^{i, t_1}v_q^{t_1} - \sum_{i,j \in Q_+} \xi^{t_2}_i p_q^{i, t_2}v_q^{t_2} \\
%     %     \sum_{i, j \in Q_{-}} \xi^{t_2}_i \left( p_q^{i, t_1}\max \{v_q^{t_1} -x_q^{t_2}, 0\}-  p_q^{i, t_2} \max \{v_q^{t_2} - x_q^{t_2} , 0 \} \right) &= \sum_{i, j \in Q_{-} \cap \{j: a_q > 0\}}(1-u_q^t) \xi^{t_2}_i \left(p_q^{i, t_1}v_q^{t_1}-p_q^{i, t_2}v_q^{t_2}\right) + \sum_{i, j \in Q_{-} \cap \{j: a_q \leq 0\}}(1-u_q^t) \xi^{t_2}_i p_q^{i, t_1} \left(v_q^{t_1}-v_q^{t_2}\right).
%     % \end{align*}
%     % And
%     %  \begin{align*}
%     %     &\sum_{i, j \in Q_{-}} \xi^{t_2}_i \left( p_q^{i, t_1}\max \{v_q^{t_1} -x_q^{t_2}, 0\}-  p_q^{i, t_2} \max \{v_q^{t_2} - x_q^{t_2} , 0 \} \right)\\
%     %     =&  \sum_{i, j \in Q_{-}} (1-u_q^t) \xi^{t_2}_ip_q^{i, t_1}v_q^{t_1} -\sum_{i, j \in Q_{-} \cap \{j: a_q > 0\}}(1-u_q^t)  \xi^{t_2}_i p_q^{i, t_2}v_q^{t_2} - \sum_{i, j \in Q_{-} \cap \{j: a_q \leq 0\}}(1-u_q^t)  \xi^{t_2}_i p_q^{i, t_1}v_q^{t_2} \\
%     %     =& \sum_{i, j \in Q_{-} \cap \{j: a_q > 0\}}(1-u_q^t) \xi^{t_2}_i \left(p_q^{i, t_1}v_q^{t_1}-p_q^{i, t_2}v_q^{t_2}\right) + \sum_{i, j \in Q_{-} \cap \{j: a_q \leq 0\}}(1-u_q^t) \xi^{t_2}_i p_q^{i, t_1} \left(v_q^{t_1}-v_q^{t_2}\right).
%     % \end{align*}
%     \begin{align*}
%          \sum_{j \in Q_-} U^j_{12}(u) &= 
%         \sum_{j \in Q_{-} \cap \{j: a_q > 0\}}u_q \left(p_q^{t_1, t_2}v_q^{t_1}-p_q^{t_2}v_q^{t_2}\right) + \sum_{j \in Q_{-} \cap \{j: a_q \leq 0\}}u_q p_q^{t_1, t_2} \left(v_q^{t_1}-v_q^{t_2}\right)
%     \end{align*}

    
 
%     % \begin{align*}
%     %      &\sum_{i,j} \xi^{t_2}_i p_q^{i, t_1}\max \{v_q^{t_1} -x_q^{t_2}, 0\}-  \sum_{i, j}\xi_i^{t_2}p_q^{i, t_2} \max \{v_q^{t_2} - x_q^{t_2} , 0 \} \\
%           % =& \sum_{i,j} \xi_i^{t_2}p_q^{i,t_1} v_q^{t_1} - \sum_{i,j} \xi_i^{t_2}p_q^{i,t_2} v_q^{t_2} + \sum_{i, j \in Q_+} \left(p_q^{i,t_2} - p_q^{i,t_1}\right) \xi_i^{t_2} v_q^{t_1} u_q^t   \\
%           %  & +\sum_{i, j \in Q_- \cap \{j: c_q > 0\} }\left(p_q^{i,t_2}v_q^{t_2} - p_q^{i,t_1}v_q^{t_1}\right)\xi_i^{t_2} u_q^t + \left(p_q^{i,t_2} - p_q^{i,t_1}\right) \xi_i^{t_2} v_q^{t_2}(1-u_q^t) \\
%            % & =  \sum_{i,j} \xi_i^{t_2}p_q^{i,t_1} v_q^{t_1} - \sum_{i,j} \xi_i^{t_2}p_q^{i,t_2} v_q^{t_2} + \sum_{i, j \in Q_+ \cap \{j: a_q \leq 0\}} a_q v_q^{t_1} + \sum_{j \in Q_- \cap \{j: c_q > 0\} }a_q v_q^{t_2} \\
%            % & +\sum_{i, j \in Q_- \cap \{j: c_q > 0\}} p_q^{t_1}\left(v_q^{t_2} - v_q^{t_1}\right) \xi_i^{t_2}u_q^t
%          % =& \sum_{i,j} \xi_i^{t_2}p_q^{i,t_1} v_q^{t_1} - \sum_{i,j} \xi_i^{t_2}p_q^{i,t_2} v_q^{t_2} + \sum_{i, j \in Q_+} \left(p_q^{i,t_2} - p_q^{i,t_1}\right) \xi_i^{t_2} v_q^{t_1} u_q^{t_1} + \left( p_q^{i, t_2} v_q^{t_2} -  p_q^{i, t_1} v_q^{t_1}\right) \xi_i^{t_2} u_q^{t_2}  \\
%          % & +\sum_{i, j \in Q_-}\left(p_q^{i,t_2}v_q^{t_2} - p_q^{i,t_1}v_q^{t_1}\right)\xi_i^{t_2} u_q^{t_1} + \left(p_q^{i,t_2} - p_q^{i,t_1}\right) \xi_i^{t_2} v_q^{t_2}u_q^{t_2} 
%     % \end{align*}
%      We divide the Integer program~\eqref{IP-two-type-case3} into two subproblems conditional on whether $U_{12}(u) \leq 0$ or not. In the first subproblem, we restrict the menu such that $\sum_{j \in Q_{+}} U_{12}^j + \sum_{j \in Q_{-}} U_{12}^j(u) \leq 0$. Hence, it is equivalent to solving the following binary decision problem. 

%      \begin{align*}
%         \max_{u} &\quad \sum_{j \in Q_{-}}p_q^{t_2} v_q^{t_2} u_q^{t} \\
%         & \sum_{j \in Q_{-} \cap \{j: a_q > 0\}}u_q \left(p_q^{t_1, t_2}v_q^{t_1}-p_q^{t_2}v_q^{t_2}\right) + \sum_{j \in Q_{-} \cap \{j: a_q \leq 0\}}u_q p_q^{t_1, t_2} \left(v_q^{t_1}-v_q^{t_2}\right) \leq -\sum_{j \in Q_{+}} U_{12}^j,
%     \end{align*}
%     Hence, the problem reduces a $0-1$ Knapsack problem.  
    
 
%     The second subproblem is the menu such that $\sum_{j \in Q_{+}} U_{12}^j + \sum_{j \in Q_{-}} U_{12}^j(u) \geq 0$
    
%     \begin{align*}
%         \max_{u} & -\sum_{j \in Q_{-} \cap \{j: a_q > 0\}} \left(p_q^{t_1, t_2}v_q^{t_1}-p_q^{t_2}v_q^{t_2}\right) u_q - \sum_{j \in Q_{-} \cap \{j: a_q \leq 0\}}  p_q^{t_1, t_2} \left(v_q^{t_1}-v_q^{t_2}\right) u_q+ \sum_{j \in Q_{-}} p_q^{t_2} v_q^{t_2} u_q \\
%         &  \sum_{j \in Q_{-} \cap \{j: a_q > 0\}}\left(p_q^{t_1, t_2}v_q^{t_1}-p_q^{t_2}v_q^{t_2}\right) u_q + \sum_{j \in Q_{-} \cap \{j: a_q \leq 0\}}  p_q^{t_1, t_2} \left(v_q^{t_1}-v_q^{t_2}\right) u_q\geq - \sum_{j \in Q_{+}} U_{12}^j,   \\
%         & \sum_{j \in Q_{-} \cap \{j: a_q > 0\}}\left(p_q^{t_1, t_2}v_q^{t_1}-p_q^{t_2}v_q^{t_2}\right) u_q + \sum_{j \in Q_{-} \cap \{j: a_q \leq 0\}}  p_q^{t_1, t_2} \left(v_q^{t_1}-v_q^{t_2}\right) u_q \leq \\
%         & \sum_{j} \left(p_q^{t_1} v_q^{t_1} - p_q^{t_2, t_1} v_q^{t_2}\right) - \sum_{j \in Q_{+}} U_{12}^j 
%     \end{align*}

% We assume that $ \sum_{j} \left(p_q^{t_1} v_q^{t_1} - p_q^{t_2, t_1} v_q^{t_2}\right) \geq 0$ otherwise, the problem is infeasible. 






% % \textbf{Observation 1: } The solution $u_q^t = 1$ for all $j \in Q_{-}$ minimizes the function $\sum_{i, j \in Q_- \cap \{j: c_q > 0\}} p_q^{t_1}\left(v_q^{t_2} - v_q^{t_1}\right) \xi_i^{t_2}u_q^t$. Further, the minimum value $g(\xi) + \sum_{i, j \in Q_- \cap \{j: c_q > 0\}} p_q^{t_1}\left(v_q^{t_2} - v_q^{t_1}\right) \xi_i^{t_2} \leq 0$. 









% %     \begin{align*}
% %         & w^{t_1} = \sum_{i, j}\xi_i^{t_1} p_q^{i, t_1} v_q^{t_1} + \min \{\sum_{i, j}\xi_i^{t_2}p_q^{i, t_2} \max \{v_q^{t_2} - \bar{x}_q^{t_2} , 0 \}- \sum_{i,j} \xi^{t_2}_i p_q^{i, t_1}\max \{v_q^{t_1} - \bar{x}_q^{t_2}, 0\}, 0\}\\
% %          & \bar{x}_q^{t_2} \in \arg\max_x \sum_{i, j} \xi_i^{t_2} p_q^{i, t_2} v_q^{t_2}\ind \{v_q^{t_2} - x_q^{t_2} \geq 0 \} + \min \Bigl\{\sum_{i, j}\xi_i^{t_2}\left(p_q^{i, t_2} \max \{v_q^{t_2} - x_q^{t_2} , 0 \}- p_q^{i, t_1}\max \{v_q^{t_1} - x_q^{t_2}, 0\}\right), 0\Bigr\}
% %     \end{align*}
% %     We denote the set $Q_+ = \{ j: \sum_i \xi_i^{t_2} (p_i^{j, t_2} v_q^{t_2}- p_i^{j, t_1}v_q^{t_1}) > 0\}$ and the set 
% %     $ Q_- = \{ j: \sum_i \xi_i^{t_2} (p_i^{j, t_2} v_q^{t_2}- p_i^{j, t_1}v_q^{t_1}) \leq 0\}$. For all $j \in Q_{-}$, if $v_q^{t_1} \leq v_q^{t_2}$, $\bar{x}^{t_2}_q = v_q^{t_1} $. 
    
% %     If $v_q^{t_1} \geq v_q^{t_2}$, for $j \in Q_{-} \cap  \{j:\sum_i \xi_i^{t_2} (p_i^{j, t_2} - p_i^{j, t_1}) \geq 0 \} $
% %      \[
% %     \bar{x}^{t_2}_q= 
% % \begin{cases}
% %     0,& \text{if $\sum_i \xi_i^{t_2}p_q^{i,t_2} v_q^{t_2} + \sum_i \xi_i^{t_2}(p_q^{i,t_2} v_q^{t_2} - p_q^{i,t_1} v_q^{t_1})\geq 0$}\\
% %     v_q^{t_1}, & \text{otherwise}
% % \end{cases}
% % \]
% % For $j \in Q_{-} \cap  \{j:\sum_i \xi_i^{t_2} (p_i^{j, t_2} - p_i^{j, t_1}) < 0 \} $, 
% %      \[
% %     \bar{x}^{t_2}_q= 
% % \begin{cases}
% %     0,& \text{if $\sum_i \xi_i^{t_2}p_q^{i,t_2} v_q^{t_2} + \sum_i \xi_i^{t_2}p_q^{i,t_1} (v_q^{t_2} -v_q^{t_1})\geq 0$}\\
% %     v_q^{t_1}, & \text{otherwise}
% % \end{cases}
% % \]

% %     For all $j \in Q_{+} \cap \{j:\sum_i \xi_i^{t_2} (p_i^{j, t_2} - p_i^{j, t_1}) \geq 0 \} $, 
% %     \[
% %     \bar{x}^{t_2}_q= 
% % \begin{cases}
% %     v_q^{t_2},& \text{if } v_q^{t_1} \geq v_q^{t_2}\\
% %     0,              & \text{otherwise}
% % \end{cases}
% % \]
% % For all $j \in Q_{+} \cap \{j:\sum_i \xi_i^{t_2} (p_i^{j, t_2}-  p_i^{j, t_1}) < 0 \} $, 
% %     \[
% %     \bar{x}^{t_2}_q= 
% % \begin{cases}
% %     0,& \text{if } v_q^{t_1} \geq v_q^{t_2}\\
% %     v_q^{t_1},              & \text{otherwise}
% % \end{cases}
% % \]
  
  
%  \end{enumerate}
 

% \newpage 
% \hfcomment{HOpefully, we do not need these, but use the binary type hardness IN SECTION 4.2. } 

% \subsection{\textsf{NP}-hardness with pure effort distributions}

% % \subsection{Attempt 1}

% % Let $G$ be a graph with $n$ vertices and let $E$ be the edge set of $G$. Consider $2n$ types indexed by $t_{i1}, t_{i2},i\in [n]$ and the uniform distribution over these types. Labeling the vertices of $G$ by $[n]$, let $t_{i1}, t_{i2}$ correspond to vertex $i\in [n]$. Similarly, define $2n$ quality levels $q_{i1}, q_{i2},i\in [n]$ in addition to the trivial quality level $q_0$ that has valuation 0 for all types. In addition to the trivial action $e_0$ which maps deterministically to $q_0$, define $n$ nontrivial actions $e_1,\lds,e_n$. For a given type, the transitions from actions to quality levels as well as the type's valuations are given as follows:
% % \begin{itemize}
% %     \item For a type $t_{i1}$, action $e_i$ maps deterministically to quality level $q_{i1}$ and $v^{t_{i1}}_{q_{i1}} = 1$. actions $e_q, \bc{i, j}\in E$ map deterministically to quality level $q_{j2}$ and $v^{t_{i1}}_{q_{j2}} = 1$. All other actions deterministically map to $q_0$.

% %     \item For a type $t_{i2}$, action $e_i$ maps deterministically to quality level $q_{i2}$ and $v^{t_{i2}}_{q_{i2}} = \fr12$. All other actions deterministically map to $q_0$.
% % \end{itemize}

% % Intuitively, we can think of types $t_{i1}$ as \emph{high-value} types that have value 1 for each quality level that type can achieve with some action, and types $t_{i2}$ are \emph{low-value} types that have value $\fr12$ for each quality level that type can achieve. Note the following ideas:

% % \begin{itemize}
% %     \item In order to extract revenue from type $t_{i2}$, we must utilize action $e_i$.

% %     \item In order to extract revenue from two types $t_{i2}$ and $t_{j2}$ with $\bc{i, j}\in E$, we must utilize actions $e_i$ and $e_q$.

% %     \item However, this presents an obstacle from extracting a full revenue of 1 from types $t_{i1}$ and $t_{j1}$. The seller profit from type $t_{i2}$ can be upper-bounded using the IC constraint $$-w^{t_{i2}} + \xi^{t_{i2}}_{e_i} \cd \bp{\fr{1}{2} - x^{t_{i2}}_{q_{i2}}} \ge 0 \imp w^{t_{i2}} + \xi^{t_{i2}}_{e_i} x^{t_{i2}}_{q_{i2}} \le \fr{1}{2} \xi_{e_i}^{t_{i2}}.$$ If type $t_{j1}$ misreports their type as $t_{i2}$, their expected utility is at least $$-w^{t_{i2}} + \xi^{t_{i2}}_{e_i}\cd \bp{1 - x^{t_{i2}}_{q_{i2}}} \ge \fr12 \xi^{t_{i2}}_{e_i}.$$ Noting that type $t_{j1}$'s expected utility is their expected valuation of the produced quality level, which is at most 1, minus the seller profit, we see that by the IC constraint the seller profit from type $t_{j1}$ is at most $1 - \fr12 \xi^{t_{i2}}_{e_i}$. We conclude that the sum of seller profits from types $t_{i2}$ and $t_{j1}$ for any edge $\bc{i, j}\in E$ is at most $$\fr{1}{2} \xi_{e_i}^{t_{i2}} + \bp{1 - \fr12 \xi^{t_{i2}}_{e_i}} = 1.$$
% % \end{itemize}

% % Noting that for any vertices $i,j\in [n]$ with $\bc{i, j}\in E$, the combined revenue from types $i$ and $j$ is at most 2 by the observation above. The optimal revenue can thus be computed using the linear programming relaxation of independent set. If there is some way to discretize the revenues in the construction above, we can recover the MIP formulation of independent set.

% % \subsection{Attempt 2}

% % Let $G$ be a graph with $n$ vertices and let $E$ be the edge set of $G$. Consider $n$ types indexed by $[n]$ corresponding to vertices of $G$ and a type distribution that puts weight proportional to $C^i$ for type $i\in [n]$ for some large constant $C$. Define $n$ quality levels $q_1,\lds,q_n$ in addition to the trivial quality level $q_0$ that has valuation 0 for all types. In addition to the trivial action $e_0$ which maps deterministically to $0$, define $n$ nontrivial actions $e_1,\lds,e_n$. For a given type, the transitions from actions to quality levels as well as the type's valuations are given as follows:

% % \begin{itemize}
% %     \item For type $i$ and action $e_q$, if $j=i$ or $\bc{i, j}\in E$ then $e_q$ maps deterministically to quality level $q_q$ and $v^i_{q_q} = C^{-i}$. Otherwise, $e_q$ maps deterministically to quality level $q_0$.
% % \end{itemize}

% % Note that the type-weighted maximum revenue is $C^i\cd C^{-i} = 1$ for all types $i\in [n]$. Call an action $e_q$ \emph{heavy} for type $i$ if it is used in some contract for some type $i$ to generate positive revenue and furthermore $i$ is the maximum type index for which $e_q$ is used.

% % %  and furthermore generates the highest performance revenue out of all actions in the support of that contract. Here we define revenue as training payment plus performance payment associated to that action.

% % \begin{claim}
% %     If action $e_q$ is heavy for type $i\in [n]$, then the revenue generated from types $k < i$ for which $\bc{j, k}\in E$ is at most $\fr{2}{C}$.
% % \end{claim}

% % \begin{proof}
% %     Note that the training payment $w^i$ for type $i\in [n]$ satisfies $w^i < C^{-i}$ and the performance payment for quality $q_q$ satisfies $x^i_{q_q} < C^{-i}$ in order for action $e_q$ to generate positive revenue. When any type $k<i$ chooses contract $i$, their expected utility is $-w^i - x^i_{q_q} + C^{-k}  > C^{-k} - 2C^{-i}$, so the principal's revenue from type $k$ is at most $$C^k\bp{C^{-k} - (C^{-k} - 2C^{-i})} = 2C^{k-i} \le \fr{2}{C}.$$
% % \end{proof}

% % In other words, the optimal revenue is characterized by the maximum matching of actions $e_q$ with types $i = i(j)$ such that any action $e_q$ matched with type $i = i(j)$ precludes types $k<i$ neighboring $j$ from being matched. This does introduce a discrete problem on a graph but it does not relate to well-known problems such a independent set. It is also possible that this discrete problem can be solved efficiently using something like dynamic programming.

% % \subsection{Attempt 3}

% We reduce from \tsf{Dominating Set}. Let $G$ be a graph with $n$ vertices and let $E$ be the edge set of $G$. We assume for convenience that all self-edges $\bc{i, i}$ are in $E$. Consider $2n$ types indexed by $t_{i1}, t_{i2},i\in [n]$ and a type distribution that puts weight proportional to 1 for $t_{i1}, i\in [n]$ and $\fr{1}{2Cn}$ for $t_{i2}, i\in [n]$ for some large constant $C$. Labeling the vertices of $G$ by $[n]$, let $t_{i1}, t_{i2}$ correspond to vertex $i\in [n]$. Define $n$ quality levels $q_{i}, i\in [n]$ in addition to the trivial quality level $q_0$ that has valuation 0 for all types. In addition to the trivial action $e_0$ which maps deterministically to $q_0$, define $n$ nontrivial actions $e_1,\lds,e_n$. For a given type, the transitions from actions to quality levels as well as the type's valuations are given as follows:
% \begin{itemize}
%     \item For a type $t_{i1}$, actions $e_q, \bc{i, j}\in E$ map deterministically to quality level $q_{j}$ and $v^{t_{i1}}_{q_{j}} = 1$. All other actions deterministically map to $q_0$.

%     \item For a type $t_{i2}$, action $e_i$ maps deterministically to quality level $q_{i}$ and $v^{t_{i2}}_{q_{i2}} = C$. All other actions deterministically map to $q_0$.
% \end{itemize}

% Intuitively, we can think of types $t_{i1}$ as \emph{low-value} types that have value 1 for each quality level that type can achieve with some action, and types $t_{i2}$ are \emph{high-value} types that have value $C\gg 1$ for each quality level that type can achieve. However, when weighted by the type distribution, the principal cares more about extracting revenue from \emph{low-value} types.

% \begin{note}
%     We want the following property: Any menu that maximizes revenue for the principal must correspond to dominating set of actions, that is, if $\{e_q\}_{j\in S}$ is the set of actions used to extract positive revenue from types of the form $t_{i1}$ then $S$ is a minimum dominating set of $G$.
% \end{note}

% We observe that the total revenue extracted from types of the form $t_{i2}$ is at most $\fr{1}{2Cn} \cd n\cd C = \fr12$, so the revenue from types of the form $t_{i1}$ is more important to the principal. In order to extract positive revenue from type $t_{i1}$, an action $e_q$ with $\bc{i, j}\in E$ must be used. Call such an action $e_q$ a \emph{heavy} action.

% % \begin{proof}
% %     If all nontrivial actions are used with probability less than $\fr{1}{3n^2}$, then the expected value for type $t_{i1}$ is less than $n\cd \fr{1}{3n^2} = \fr{1}{3n}$, which is certainly an upper bound for the revenue extracted from $t_{i1}$.
% % \end{proof}

% If $\{e_q\}_{j\in S}$ is the set of heavy actions, the maximum total revenue extracted from types of the form $t_{i1}$ is bounded by $\ab{\p S}$, where $\p S$ is the set of vertices of $G$ at most distance 1 from a vertex $i\in S$. Now adding in the revenue extracted from types of the form $t_{i2}$, the total revenue extracted from all types is at most $\ab{\p S} + \fr12$. This implies the following key fact:

% \begin{fact}
%     If the principal revenue is at least $n\in \bZ$, the set of heavy actions in the principal's menu must correspond to a dominating set of $G$.
% \end{fact}

% Since a revenue of $n\in\bZ$ is certainly achievable by $n$ contracts that charge a training payment of 1, exert actions $e_i, i\in [n]$, and have no performance payments, in any optimal menu the set of heavy actions forms a dominating set of $G$, which we now assume. Note also that the maximum total revenue we can possibly extract from agents of type $t_{i1}$ is $n$, so the remaining optimization problem is only over revenue of agents of type $t_{i2}$. In order for our reduction from \tsf{Dominating Set} to work, we wish for the optimal menu for the principal must have its heavy actions correspond to a \emph{minimum} dominating set of $G$. To do this, we need to bound the revenue extracted from agents of type $t_{i2}$ in terms of the number of heavy actions.

% % In the case where we restrict action distributions to have support 1, that is, the principal can only choose a single action for each type, we can prove \tsf{NP}-hardness. In this case a heavy action is exerted with probability 1 for some type and actions that are not heavy are never used.

% \begin{claim}
%     Any heavy action $e_q$ extracts at most $\fr{1}{2Cn}$ revenue from type $t_{j2}$. Call such a type \emph{inextractable}. Note that $e_q$ extracts 0 revenue from any type $t_{i2}, i\neq j$.
% \end{claim}

% \begin{proof}
%     Since $e_q$ extracts positive revenue from some type $t_{i1}$, the sum of the training payments and performance payments for $t_{i1}$ is less than their maximum valuation, which is 1. The utility of type $t_{j2}$ when they misreport their type as $t_{i1}$ is thus at least $C-1$. Noting that their maximum possible utility is $C$, we conclude that the revenue extracted can be at most $C - (C-1) = 1$ by the IC constraint, which when weighted becomes $\fr{1}{2Cn}$.
% \end{proof}

% On the other hand, for any action $e_q$ that is not heavy, which recall means that they are not used at all to extract revenue from types of the form $t_{i1}$, we can extract revenue $\fr{1}{2Cn} \cd C = \fr{1}{2n}$ from type $t_{j2}$ by setting the action to $e_q$, the training payment to $C$, and the performance payment to 0. Note that no type $t_{j1}$ is incentivized to misreport their type as $t_{j2}$ since the training payment is too high. Furthermore, no other extractable type $t_{k2}, k\neq j$ is incentived to misreport since the training payments for extractable types are identical.

% Since there is a multiplicative gap of $C$ for the revenue between extractable and inextractable types, as long as $C>n$ then the principal's optimization problem is equivalent to minimizing the number of heavy actions, that is, the size of the dominating set $S$. We conclude that any menu that achieves optimal revenue for the principal corresponds to a minimium size dominating set if we consider the heavy actions.