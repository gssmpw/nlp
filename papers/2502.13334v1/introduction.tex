\section{Introduction} \label{introduction}

Motivated by the recent popularity of machine learning as a service, we study how to price such a service through the lens of \emph{algorithmic contract theory} \citep{dutting2024algorithmic}. We introduce a new contract design problem between a provider (seller) and a customer (buyer). The service consists of the seller choosing one of finitely many possible actions, each of which incurs a cost for seller and leads to a known distribution over possible end products, which we denote by \emph{outcomes}. The buyer has a value for every outcome that depends on their private type. In practice, the service in question can be seen as a machine learning training or fine-tuning service. The action can be interpreted as the effort level that the seller undertakes to train a machine learning model. Each effort level has a different cost for the seller. The outcome can be understood as the model quality, which is uncertain given the stochastic nature of training.

The seller first presents \emph{menu} of contracts to the buyer. Each \emph{contract} specifies an action, an \emph{upfront} price, and an outcome-dependent \emph{usage} price. After selecting a contract, the seller implements a two-part payment mechanism. The seller commits to the contract's action and charges the buyer an upfront price for performing the action. Upon the action's completion, the buyer is able to observe the outcome and decide whether to accept it or reject it. If the buyer accepts the outcome, the seller charges the buyer a further usage price, the amount of which can depend on the outcome. If the buyer rejects the outcome, they cannot use the product but are exempt from further payment. The idea of using two types of payments, also known as the \emph{two-part tariff}, is widely studied in economics as a means to extract more consumer surplus through price discrimination \citep{hayes1987competition, schlereth2010optimization, leland1976monopoly, armstrong2011competitive, murphy_price_1977, danaher_optimal_2002}. The two objectives we focus on maximizing are the seller \emph{revenue}, which is the expected total payment received from the buyer, and the maximum seller \emph{profit}, which is the revenue minus the expected cost of the seller's action.\footnote{The distinction between revenue and profit is only important in \cref{single-parameter}. All other results in the paper hold for both maximizing revenue and maximizing profit. The results will be phrased only in terms of maximizing profit since revenue can be thought of as a special case of profit with zero action costs.}

Our research has connections to recent literature on Bayesian contract design \citep{alon2023bayesian,guruganesh2021contracts, castiglioni2024reduction}. However, unlike standard contract design problems in which the principal issues a contract and an agent is paid to perform actions that benefit the principal, in our model the principal (provider) both issues contracts and performs actions that benefit the agent (customer). Our model of selling actions using contracts is most related to recent work by \citet{bernasconi2024agent} on selling \emph{hidden} actions, however there are key differences between their model and ours which we highlight in in \cref{connections}.

\paragraph{Motivation.}

Our service provider problem is inspired primarily by the surge in companies 
offering machine learning services: examples include  automated machine learning (AutoML) training services offered by Google Vertex AI and Amazon SageMaker and enterprise large language models (LLMs) sold by   \citet{openai-enterprise}. AutoML services help users automatically train their models, and the service price is based on the amount of cloud computing resources users consume. Selling fine-tuned LLMs to businesses, such as the LLM enterprise product of \citet{openai-enterprise}, has also recently become a profitable industry, with custom models costing millions of dollars. The above settings share two characteristics which our service provider problem capture. First, they require the service provider to exert costly effort, whether in the form of cloud computing resources for AutoML or engineering and computing effort for fine-tuning LLMs. Second, the outcome has high uncertainty since the performance of AutoML and fine-tuned LLMs is very problem dependent and difficult to predict in advance.

% In this paper, we introduce a pricing problem that naturally cast both properties and study the optimal  pricing contract in such   high-stakes endeavors.
%fine-tuning training services, Open AI, Vertex AI, Anthropic, Amazon SageMaker, and DeepSeek to name a few. Each service provider has several pre-trained foundation models, and users can create custom LLM models by fine-tuning these foundation  models. Service providers incur large costs to train custom LLM models and yet customers that find custom LLM models valuable are more than willing to pay these high costs. As such, training services represent a huge revenue potential for service providers and thus studying mechanisms to price these services is a high-stakes endeavor. Currently, service providers such as Vertex AI and Amazon Sage Maker charge a uniform price based on number of tokens for training, \hfcomment{token pricing is a good a good motivation. Need to rephrase motivations. TODO for Haifeng.} input, and output usage of their models. Notably, they do not distinguish between possibly different buyer types and also do not account for the quality of the model in the usage prices.

Currently, service providers such as Vertex AI and SageMaker charge a uniform price based on the amount of computing usage. Notably, they do not distinguish between different buyer types and also do not account for the quality of the final model. The goal of this paper is to introduce a contract-based pricing scheme inspired by the two-part tariff in order to maximize seller profit. Our mechanism accounts for \emph{adverse selection} in which the buyer type is unknown to the seller, and it also allows the buyer to pay different amounts according to the outcome quality. Our central research question is the following: how good is our pricing scheme at maximizing seller profit compared to other models, and how hard is it to compute profit-maximizing contracts in our model?

\paragraph{Contributions.} For the problem of selling a service with stochastic outcomes, our paper makes the following main contributions:
\begin{itemize}
    %\item \textbf{Modeling services with stochastic outcomes:}  First, we formalize the seller-buyer interaction protocol for selling a service as well as the provider's profit optimization problem, \cref{profit-maximization-direct-menu}. 
    
    \item \textbf{Necessity of two-part tariff structure.} 
    We show that implementing a two-part tariff in the service provider problem yields significantly greater seller profit compared to using only upfront payments (\cref{upfront-payment-only}) or using only usage payments (\cref{usage-payment-only}), which justifies the inclusion of both types of payments in our mechanism. We provide a tight bound on the worst-case multiplicative gap in the seller profit achievable by our two-part tariff structure versus one that only uses upfront payments.
    
    \item \textbf{Superiority of voluntary usage.} We prove that maximum seller profit is \emph{always} weakly higher when the seller allows the buyer the freedom to accept or reject the outcome than when the seller forces the buyer to accept and pay for every outcome (\cref{voluntary-usage-subsumes}).
    
    \item \textbf{Complexity of computing profit-maximizing menus.}  We show that maximum profit can be achieved by offering just two distinct usage prices: 0 and $\infty$ (\cref{two-usage-prices-suffice}). Using this reduction, we prove that computing the exact maximum seller profit in the service provider problem is \tbf{NP}-hard even when there are only two buyer types and a single seller action (\cref{np-hardness-two-types}). Despite this hardness result, we use a dynamic program framework to derive a fully polynomial time approximation scheme (FPTAS) for \emph{approximating} the maximum seller profit as long as the number of buyer types $T$ is constant (\cref{fptas}). 
    
    \item \tbf{Revenue-optimality in single-parameter settings.} Even though the general service provider problem is \tbf{NP}-hard, when buyers' valuations are parametrized by a single real number we show that not only can we efficiently compute a \emph{revenue}-maximizing menu, but also revenue can be maximized by a menu consisting of a single contract (\cref{single-parameter-revenue}).
\end{itemize}

Taken together, these results provide comprehensive insights into the structure of optimal contracts to sell services with stochastic outcomes.

% \section{Outline}
% \begin{enumerate}
%     \item Model: multiple effort levels (pure effort levels, could mention randomized effort levels in the single effort levels) 
%     \item necessity of two payments (Zoe) 
%     \item NP-hardness + MILP formulation 
%     \item results of special cases: single-parameter settings (Alec), two types (PTAS approximate optimal solution)
% \end{enumerate}

% \section{Introduction}
% \begin{enumerate}
%     \item Motivation
%     \item Contribution
% \end{enumerate}

% \hfcomment{characteristics of the model: adverse selection + uncertainty, but no moral hazard as in contract design. Another way to view our model is that we are selling two correlated items: training service + a performance; this classic paper is very related \citet{mcafee1992correlated}.


% Another very interesting question is the single-parameter setting, in which the buyer's type is assumed to be a single number $v^t$, i.e., $v^t_j = q_j v^t$ for all $j$. This is very related to these two classic papers \citet{mussa1978monopoly,maskin1984monopoly}. 

% \citet{chen2015complexity} studies complexity of selling lottery, but they design the lottery probabilities of each item and charge the expected price, whereas we do not design probability 
% }


% \subsection{ Related Works}
% \hfcomment{This paper is very related "LMaaS: Exploring Pricing Strategy of Large Model as a Service for Communication", yet different. Can discuss about it.}
% \begin{enumerate}
%     \item Simple versus Optimal Contracts study the problem where a single principal designs a contract to incentivize an agent to take a costly and unobservable action. They show that when the distribution of outcomes resulting from each effort is known, the optimal contract can be computed by linear programming, while the optimal contract is complex and unintuitive. They propose a simple linear contract (payments are linear in the outcomes). 

%     \item Contracts with private cost per Unit-of-Effort between a principal and an agent with hidden action and single-dimensional private type. They show the linear programming duality based on the characterization of implementable allocation rules for both discrete and continuous types. The agent's cost is associated with their types. 

%     \item Contracts under Moral Hazard and Adverse Selection study the similar setting where the agent's type affects the outcomes of actions. In this setting, the optimal contracts is APX-hard. They show that the optimal linear contracts achieves an $O(n \log T)$ approximation of the best possible profit, where $n$ is the number of actions (training effort) and $T$ is the number of possible types. 
%     \item Complexity of contracts 
%     \item performance based contract \citep{bui2019exploring, selviaridis2015performance}
% \end{enumerate}

