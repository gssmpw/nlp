\section{Optimality of two-part tariffs and voluntary usage} \label{two-payments}

The seller-buyer interaction protocol described in \cref{interaction} is more complex than the protocols for selling hidden actions \citep{bernasconi2024agent} and selling lotteries \citep{chen2015complexity} because we implement a \emph{two-part tariff}: a pricing scheme with two payment stages. While the two-part tariff is a widely studied mechanism \citep{hayes1987competition}, one must ask whether two stages of payment are necessary to achieve maximum seller profit, as
it may well be the case that only upfront payments or only usage payments suffice. We begin this section by showing that both stages of payment are indeed necessary to maximize profit (\cref{upfront-payment-only} and \cref{usage-payment-only}). Second, we leverage a surprising connection with \cref{upfront-payment-only} to show our first main result (\cref{voluntary-usage-subsumes}): \emph{voluntary usage}, defined in \cref{interaction}, always leads to weakly higher seller profit than \emph{mandatory usage}, defined in \cref{connections}.

\subsection{Usage payments are necessary} \label{usage-payment-necessary}

\newcommand{\Rupfront}{R_\mathtt{upfront}}

To prove that usage payments are needed to achieve maximum seller profit, we consider a restricted version of the service provider problem where all usage prices are set to 0. Let $R$ denote the maximum seller profit achievable through the two-part tariff and let $\Rupfront$ denote the maximum profit achievable through upfront payments only. We have the following definition and result:

%\hfcomment{I think this can be removed. We already said in the model and Footnote 1 that $\mu$ is full support. The order is trivially without loss.}
\begin{definition*}
   For a buyer type distribution $\mu \in \Delta_{[T]}$, define  
    \begin{equation}\label{eq:H-mu}
    H_\mu \coloneq \sum_{t\in [T]} \fr{\mu^{\sigma(t)}}{\sum_{i=1}^t \mu^{\sigma(i)}}
    \end{equation}
    where $\sigma : [T] \to [T]$ is a permutation satisfying $0 < \mu^{\sigma(1)} \le \cds \leq \mu^{\sigma(T)} \leq 1$. 
\end{definition*}

\begin{proposition} \label{upfront-payment-only}
     Let $\mu \in \Delta_{[T]}$ denote the buyer type distribution. Then we have the following:
    \begin{enumerate}
        \item There exists a problem instance for which $\fr{R}{\Rupfront} = H_\mu$.
        \item For all problem instances, $\fr{R}{\Rupfront} \le H_\mu$.
    \end{enumerate}
\end{proposition}
% \kicomment{Possibly, we should define the set $\Delta_{<}([T])$ as the set of distributions $\mu$ over $[T]$ with $0 < \mu^1 \leq \cdots \leq \mu^T \leq 1$. We keep using this frequently, and this way it concise. Also, do we need the first inequality here to be strict? We could define the set and the function $H_\mu$ before Lemma 1. Also, include the discussion about range of $H_\mu$ there. Currently, discussion about $H_\mu$ is interspersed with discussion about Lemma 1. Remove $0 < ... < \mu^T$ assumption entirely by incorporating it in definition of $H_\mu$ before Lemma 1.}
\cref{upfront-payment-only}, proven in \cref{upfront-payment-only-proof}, shows that the worst-case gap in profit between implementing the two-part tariff versus using only upfront payments is characterized by the prior-dependent multiplicative factor $H_\mu$, and furthermore this multiplicative gap is tight. We prove in \cref{H_mu} that $H_\mu$ always lies in the range $[H_T, T)$, where $H_T = \sum_{t\in [T]} \fr{1}{t}$ is the $T$-th harmonic number, and furthermore that this range for $H_\mu$ is tight.

% \begin{remark*} \hfcomment{what is this remark about? Remove it?}
%     When there are zero usage prices, \cref{profit-maximization-direct-menu} is a linear program with variables $(w^t)_{t\in [T]}$. Using only upfront payments, a profit-maximizing menu can thus be computed in polynomial time given the action tuple $(a^t)_{t\in [T]}$ as upfront. Enumerating over all $\ab{A}^T$ possible action tuples and choosing the tuple that yields the highest profit constitutes a polynomial time algorithm to compute the global profit-maximizing menu using only upfront payments as long as the number $T$ of types is constant.
% \end{remark*}

\para{Usage payments are redundant in lottery pricing.}

We have established that outcome-dependent usage payments are crucial for maximizing seller profit in the service provider problem. A natural question is whether they also increase seller profit in other models. Interestingly, the answer turns out to be negative for the lottery pricing problem, which recall can be viewed as an instance of the service provider problem but with infinitely many actions that induce every possible outcome distribution. We show the following result, which implies that maximum seller revenue in lotteries is the same with or without usage payments:

 \begin{proposition} \label{usage-payment-lottery-pricing}
      In the lottery pricing problem, any menu with voluntary or mandatory item-dependent payments can be modified into one with only lottery payments such that the buyer utilities and seller revenue remain unchanged.
 \end{proposition}
 
 At a very high-level, \cref{usage-payment-lottery-pricing} is true because the availability of all outcome distributions implies that to price discriminate between buyer types, the seller can choose actions that lead to different desired outcome distributions for different types. The power of being able to induce any distribution turns out to subsume the power of setting outcome-dependent usage prices. \cref{usage-payment-lottery-pricing} is proven in \cref{proof-usage-payment-lottery-pricing}.

% Interestingly, we show that the optimal service provider profit achievable in the \emph{mandatory usage} model of \citet{bernasconi2024agent} with service profit commitment power is exactly the optimal service provider profit achievable in our \emph{voluntary usage} model \emph{using only training payments}, showing that there is a significant profit increase when we allow voluntary usage as opposed to mandatory usage of the outcome. 


\subsection{Upfront payments are necessary} \label{upfront-payment-necessary}

\newcommand{\Rusage}{R_\mathtt{usage}}

To prove that upfront payments are needed to achieve maximum seller profit, we consider a modified version of the service provider problem where all upfront prices are set to 0. Let $\Rusage$ denote the maximum seller profit achievable through usage payments only.

\begin{proposition} \label{usage-payment-only}
    There exists a service provider problem instance for which $\fr{R}{\Rusage} \ge \fr32$.
\end{proposition}

% In conclusion, voluntary usage is a generalization of mandatory usage that always achieves weakly greater profit for the service provider. To recover any menu under mandatory usage, we can restrict the voluntary usage model to only allow training payments by setting all usage payments to zero. Furthermore, there exist instances in which the service profit provider is significantly greater under voluntary usage than under mandatory usage, which demonstrates the importance of assuming voluntary usage in order to achieve the highest service provider profit.

% \begin{corollary}
%     If $\mu^1 = \cds = \mu^T = \fr{1}{T}$ then there is a polynomial-time approximation algorithm that achieves $\Omega\bp{\fr{1}{\log T}}$ fraction of optimal profit.
% \end{corollary}

% \begin{proof}
%     By \cref{training-payment-only-gap}, we have $$\fr{R}{R_\te{training}} \le \sum_{t\in [T]} \fr{1}{t} = \Omega\bp{\fr{1}{\log T}},$$ so the maximum profit using only training payments is at least $\Omega\bp{\fr{1}{\log T}}$ fraction of the maximum profit using both forms of payment. To finish, we claim that the optimal menu using only training payments can be computed in polynomial time.

%     \todo{}
% \end{proof}

% As in \cref{training-payment-only-gap}, an upper bound of $T$ can be obtained by throwing away the revenue generated by all types except the most profitable. More formally, 

% Note that the hard case in \cref{training-payment-only-gap} was designed by having a spectrum of agent revenues generated by a single effort level with deterministic transition probabilities. To design the hard case here, we also consider a single non-trivial effort level $e_1$ but construct a spectrum of transition probabilities that all lead to the same expected revenue for each agent. Let there be one non-trivial quality level $q_1$ in addition to the trivial quality level $q_0$. For effort level $e_1$, the contract corresponding to agent type $t$ will have transition probabilities $$\xi^{e_1}_{q_0} = 1 - \fr{1}{v^t_{q_1}}, \quad \xi^{e_1}_{q_1} = \fr{1}{v^t_{q_1}}.$$ Note that a single non-trivial contract with training payment $w=0$ and usage payments $x=0$ extracts revenue 1 from all agents, and this is clearly the optimal revenue.

% On the other hand, we show that without training payments, the best guaranteed revenue is only $\fr{1}{T}$. Set $v^t_{q_1} = C^t$ for some large constant $C$.

% \subsection{Training payments are necessary}

% \cref{only-training-payments} showed that obtaining the maximum service provider profit requires usage payments. In this section we prove that training payments are also necessary.

% \begin{lemma} \label{usage-payment-only-gap}
%     Let $R$ denote the highest service provider profit achievable using both training and usage payments and let $R_\te{usage}$ denote the highest service provider profit achievable using only usage payment. Then there exists a problem instance for which $\fr{R}{R_\te{usage}} \ge \fr32$.
% \end{lemma}

% \begin{proof}
%     We construct the following problem instance:
%     \begin{itemize}
%         \item Let $T = 2$ and $Q = \bc{1, 2}$. There is a single action $a$ with cost $c(a) = 0$ and transition probabilities $p^a_q = \fr{1}{2},\fl q\in [2]$.
%         \item Valuations are given by $\mathbf{v}^1 = \bp{1,\fr12}$ and $\mathbf{v}^2 = \bp{\fr12, 1}$.
%     \end{itemize}

%     Note that $V(t;a) = \fr34$, so $R \le \fr34$. The upper bound on $R$ can be achieved using the contract $\cC = \bp{a, \fr34, \mathbf{0}}$ for all types. On the other hand, we show that $R_\te{usage}\le \fr12$. Assume for contradiction that $R_\te{usage} > \fr12$. Then the revenue from some type, without loss of generality type 1, is greater than $\fr12$, implying $x^1_1 + x^1_2 > 1$. This means that type 1 pays greater than $\fr12$ for outcome 1, $\fr12 < x^1_1 \le 1$, and it also means that $0 < x^1_2 \le \fr12$. Note that $$\fr12 \bp{\fr32 - x^1_1 - x^1_2} = U(1;\cC^1) \ge U(1;\cC^2) \ge \fr12 \bp{1 - x^2_1},$$ which combined with $x^1_1 + x^1_2 > 1$ yields $x^2_1 > \fr12$, namely type 2 does not use outcome 1. Then we have $$\fr12 \bp{1 - x^2_2} = U(2;\cC^2) \ge U(2;\cC^1) = \fr12 \bp{1 - x^1_2},$$ which implies $x^2_2 \le x^1_2 \le \fr12$. This yields $$R_\te{usage} \le \fr14\bp{x^1_1 + x^1_2 + 0 + \fr12} \le \fr12,$$ contradiction. We conclude that $R_\te{usage} \le \fr12$.
    
%     % Now consider $U(2;\cC^1)$. Type 2 receives zero utility from outcome 1 in $\cC^1$ since $x^1_1 > \fr12 = v^2_1$ and $p^a_2 \cd (1 - x^1_2) = \fr12 (1 - x^1_2)$ utility 

% \end{proof}

\cref{usage-payment-only} is proven in \cref{proof-usage-payment-only}. Note that the multiplicative gap of $\fr32$ here is not as large as the gap of $H_\mu$ gap in \cref{usage-payment-necessary}. In particular, note that $H_\mu\to \infty$ as $T\to \infty$. This could indicate that charging a buyer to enter into a contract is less important than charging for the buyer's outcome usage, though deriving tight bounds for $\fr{R}{\Rusage}$ remains an open question.

\para{Incomparability of $\Rupfront$ and $\Rusage$.}

\cref{upfront-payment-only} and \cref{usage-payment-only} further show that the quantities $\Rupfront$ and $\Rusage$ are generally incomparable, namely, it is not true that one is always at least the other. For the problem instance constructed in \cref{upfront-payment-only}, there is a profit-maximizing menu with only usage payments whereas any menu with only upfront payments only is suboptimal, so $\Rusage > \Rupfront$ for that instance. For the problem instance constructed in \cref{proof-usage-payment-only}, there is a profit-maximizing menu with only upfront payments whereas any menu with only usage payments is suboptimal, so $\Rupfront > \Rusage$ for that instance.

% Unlike in \cref{training-payment-only-gap}, the gap in \cref{usage-payment-only-gap} does not approach $\infty$ as $t\to \infty$. We show that we can obtain a much larger gap if we consider a restricted version of the service provider problem in which they are only allowed to use a single contract for all types. We note that restricting a menu to contain only a single contract is well-studied in the contract design literature. For simplicity we assume in the next result that the type distribution is uniform, namely $\mu^1 = \mu^2 = \cds = \mu^T = \fr{1}{T}$, noting that the gap can also be computed in the non-uniform case.

% The reason we make the uniform type distribution assumption is that in the usage payments only setting, proving a gap for arbitrary $\mu$ requires more computation but is not more conceptually illuminating. 

% \begin{proposition} \label{usage-gap-single-contract}
%     Let $\mu^1 = \mu^2 = \cds = \mu^T = \fr{1}{T}$ be the agent type distribution. Let $R$ denote the highest service provider profit achievable using both training and usage payments \emph{in a single contract} and let $R_\te{usage}$ denote the highest service provider profit achievable using only usage payment \emph{in a single contract}. Then there exists a problem instance for which $\fr{R}{R_\te{usage}} \ge H_T$.
% \end{proposition}

% \begin{proof}
%     \begin{enumerate}
%         \item We construct the following problem instance:
%         \begin{itemize}
%             \item Let $Q = [T]$. There is a single action $a$ with cost $c(a) = 0$ and transition probabilities $p^a_q = \fr{1}{T},\fl q\in Q$.
%             \item Valuations are given by $v^t_q = \case{\fr{1}{1 + (t + q)\bmod T}& \teif q = t \\ 0 & \teif q \neq t.}$ In other words, the valuations for the $T$ types comprise all cyclic shifts of the sequence $\fr{1}{1}, \fr{1}{2}, \cds \fr{1}{T}$.
%         \end{itemize}

%         Note that $V(t;a) = \fr{H_T}{T},\fl t\in [T]$ so $R \le \fr{H_T}{T}$. The upper bound on $R$ can be achieved using the contract $$\cC = \bp{a, \fr{H_T}{T}, \mathbf{0}}$$ for all types, so $R = \fr{H_T}{T}$. 

%         We show that any single contract $(a, 0, \mathbf{x})$ consisting of only usage payments has expected revenue at most $\fr{1}{T}$. Setting a usage payment in the range $$x_q \in \left( \fr{1}{t+1}, \fr{1}{t}\right]$$ yields $t$ types purchasing quality $q$, for a revenue of at most $$\fr{1}{T} \cd \fr{t}{T} \cd \fr{1}{t} = \fr{1}{T^2}$$ from quality $q$. Summing over all $\ab{Q} = T$ qualities yields a total revenue of at most $\fr{1}{T}$.

%         % On the other hand, if we are restricted to using usage payments, we claim that for every constant $c \ge 0$, there exists sufficiently large $T$ for which $R_\te{usage} \le \fr{R}{c}$.
        
%         % The next step is to show that any menu of contracts can be reduced to the single contract case. Consider any quality $q$ and let the minimum usage payment $\min_{t\in [T]} x^t_q$ for quality $q$ lie in the interval $\left( \fr{1}{u+1}, \fr{1}{u} \right]$.
%     \end{enumerate}
% \end{proof}

% \begin{remark}
%     We conjecture that the problem instance described in \cref{usage-gap-single-contract} actually yields a profit gap of $H_T$ for the uniform type distribution even if we remove the single contract restriction. For example, \cref{usage-payment-only-gap} proves this conjecture for $T=2$. However, the analysis for $T>2$ seems quite tricky so we are unable to prove the general conjecture.
% \end{remark}

% Now suppose the principal is only allowed to use training payments. Since there is only one effort level, incentive compatibility guarantees that all types that do not choose the trivial contract will choose the contract with the least training payment, call it $w$. (Technically there is effort level $e_0$, so types will actually choose the contract with the least training payment multiplied by the probability of $e_1$.) Let $t$ be such that $\frac{1}{t+1} < w \le \frac{1}{t}$. Note that $w$ is too large for any type $u > t$ to purchase quality $q_{j(u)}$, so all types $u>t$ will choose the trivial contract, leading to a maximum total revenue of $tw \le 1$. We thus have a multiplicative gap of $H_T = \Theta(\log T)$ between the optimal revenue achievable using both training payments and usage payments and the optimal revenue achievable using only training payments. Note that $H_T\to \infty$ as $T\to \infty$, so the gap is unbounded.

% \subsubsection{Gap in terms of number of types}

% We also prove an upper bound of $H_T$ on the worst multiplicative gap in terms of the number of types $T$, which shows that the construction above is tight.

% \subsubsection{Proof of upper bound}

% We first note that an upper bound of $T$ can be obtained by throwing away the revenue generated by all types except the most profitable. More formally, consider a revenue-optimal menu allowing usage payments and let $(w^t, \xi^t, x^t)$ be the contract for the type $t$ from which the principal extracts the maximum revenue $$w^t + \sum_{j:v^t_j\le x^t_j} \xi^t(j) x^t_j,$$ where $\xi^t(j)$ denotes the the probability that type $t$ receives quality $j$ under $\xi^t$. Consider a menu that consists of just the trivial option and the option $$\bp{w^t + \sum_{j:v^t_j\le x^t_j} \xi^t(j) x^t_j, \xi^t, 0}.$$ The revenue extracted from type $t$ is the same as in the original menu, and furthermore type $t$ is weakly incentivized to choose the new contract over the trivial one, hence the principal can extract at least $\fr{1}{T}$ the revenue using only training payments.

% We now improve this upper bound to $H_T$ in the case that type distribution is uniform. For simplicity let us first consider the case where there is a single nontrivial effort level. Suppose that a menu with a single nontrivial option extracts revenue $r_t$ from type $t$ and without loss of generality assume $r_1 \ge r_2 \ge \cds \ge r_T$. Note that setting a training payment of $r_t$ extracts $t\cd r_t$ revenue. We claim that $$\max_{t\in T} t\cd r_t \ge \fr{1}{H_T} \sum_{i=1}^T r_i.$$ If not, then $$t\cd r_t < \fr{1}{H_T} \sum_{i=1}^T r_i\imp r_t < \fr{1}{t\cd H_T} \sum_{i=1}^T r_i,\fl t\in [T].$$ Summing over all $t\in [T]$ yields a contradiction. We conclude that the worst multiplicative gap is $H_T$.

% Now we consider the general case. Suppose that the contract $(w^t, \xi^t, x^t)$ for type $t$ extracts revenue $r_t$ and again assume $r_1 \ge r_2 \ge \cds \ge r_T$. Similar to in the simple case, we claim that we can extract $t\cd r_t$ revenue for each $t$, from which the result will follow. To do this, we will replace each contract $(w^u, \xi^u, x^u)$ with $(r_t, \xi^u, 0)$. In particular note that the training payments for all nontrivial contracts are constant. Note that with this menu change, type $u$ may not choose $(r_t, \xi^u, 0)$ and may also choose the trivial contract. Nevertheless, for any type $u\le t$, their utility from contract $(r_t, \xi^u, 0)$ is 
% \[-r_t + \sum_{j=1}^n \xi^u(j) v^u_{j} \ge -r_t + \sum_{j:v^u_j\ge x^u_j} \xi^u(j) x^u_{j} + w^u = -r_t + r_u \ge 0,\]
% noting that 
% \[\sum_{j=1}^n \xi^u(j) v^u_{j} \ge \sum_{j:v^u_j\ge x^u_j} \xi^u(j) x^u_{j} + w^u\] 
% follows from $(w^u, \xi^u, x^u)$ being weakly better than the trivial option for type $u$. We conclude that any type $u\le t$ finds $(r_t, \xi^u, 0)$ weakly better than the trivial option, implying they will choose some nontrivial contract, not necessarily $(r_t, \xi^u, 0)$. However, all nontrivial contracts generate $r_t$ revenue for the principal, implying that modifying the menu in this way to remove usage payments yields at least $t\cd r_t$ utility for the principal. The rest of the analysis is the same as the simple case above. We conclude that the worst multiplicative gap for $T$ types when restricting to training payments only is $H_T$.

% \subsubsection{Non-uniform type distribution}

% The above results, which were proven in the uniform type distribution setting, can be extended to non-uniform type distribution that assigns weights $\omega_1\le \omega_2 \le \cds \le \omega_T, \sum_{i=1}^T \omega_i = 1$ to the $T$ types. In this case we prove that the worst multiplicative gap is $\sum_{i=1}^T \fr{\omega_i}{\sum_{j=1}^i \omega_j}$ using the same analysis above. \todo{Write down full proof.}

% \kicomment{I think the distribution where $\omega_i = \sum_{j < i} w_j = 2^{i-2} \omega_1$ gives a multiplicative gap that is $\Theta(T)$. So, the worst case distribution is not the ``uniform type distribution''. Good to write it down carefully.}

% Furthermore, assume that there exists an effort level that has value 0 to all agent types. The principal's optimization problem is:
% \begin{align*}
%     \max_{w, \xi, x} \quad & \expec \left[\sum_{i = 1}^m \xi_i^t x^t_q + w^t - \sum_{i = 1}^m \xi_i^t c_i \right] \\
%     & \max\{0, v^t_q - x^t_q\} - w^t \geq \max\{0, v^t_q - x_i^{t'}\} - w^{t'}, \quad \forall t, t' \\
%     & \max\{0, v^t_q - x^t_q\} - w^t \geq 0, \quad \forall t.
% \end{align*}
% We show that requiring usage payments, unlike the setting of \citet{chen2015complexity}, it 

% Questions from conference:

% Figure out when we have deterministic transitions, is our two-stage payment mechanism the optimal model?

% How is our problem different than selling an item with a random quality (discuss freedom of choosing transition probabilities in lottery selling vs. no freedom in our paper.)



% commnets

% \begin{theorem} \label{voluntary-usage-better-than-mandatory-usage}
%     Let $\mu^1\le \cds \le \mu^T$ with $\sum_{t\in [T]} \mu^t = 1$ denote the buyer type distribution $\mu\in \Delta_{[T]}$. Let $R$ denote the maximum service provider profit achievable under voluntary usage, let $R_\te{mandatory}$ denote the maximum service provider profit achievable under mandatory usage, and let $R_\te{training}$ denote the maximum service provider profit achievable under voluntary usage using only training payments. Then we have the following:
%     \begin{itemize}
%         \item $R \geq R_{mandatory} = R_{training}$.
%         \item The multiplicative gap between profit under voluntary usage and profit under mandatory usage is upper bounded by $$R \le \bp{\sum_{t\in [T]} \fr{\mu^t}{\sum_{i=1}^t \mu^i}} \cd R_{mandatory}.$$
%         \item There exists a problem instance for which the above gap is tight, namely $$R = \bp{\sum_{t\in [T]} \fr{\mu^t}{\sum_{i=1}^t \mu^i}} \cd R_{mandatory}.$$    
%     \end{itemize}
% \end{theorem}

% \cref{voluntary-usage-better-than-mandatory-usage} implies that the maximum service provider profit is \emph{always at least as high} under voluntary usage than under mandatory usage, and furthermore sometimes significantly so as we show in \cref{F_T-range}. The rest of this subsection is devoted to proving \cref{voluntary-usage-better-than-mandatory-usage} through a series of technical lemmas.

% We first show that $R_\te{mandatory} = R_\te{training}$ using a \emph{payment redistribution} argument similar to \cref{usage-payments-to-training-payment-in-lottery}.

% \begin{corollary} \label{voluntary-usage-multiplicatively-better}
%     Given a type distribution $\mu^1\le \cds \le \mu^T$ where $\sum_{t\in [T]} \mu^t = 1$, there exist problem instances for which the maximum service provider profit is a multiplicative factor of $\sum_{t\in [T]} \fr{\mu^t}{\sum_{i=1}^t \mu^i}$ greater under voluntary usage than under mandatory usage.
% \end{corollary}

% \begin{proof}
%     This follows from \cref{mandatory-usage-profit-is-R_training} and \cref{training-payment-only-gap}.
% \end{proof}

% \begin{lemma} \label{mandatory-usage-implies-only-training-payment}
%     Assuming mandatory usage, any IC and IR menu can be modified into one that has only training payments that is equivalent to the original menu in the following sense:
%     \begin{itemize}
%         \item Every utility $U(u;\cC^t)$ for type $u$ choosing contract $\cC^t$ are the same in the old menu as the new one.
%         \item The action used for each type and the revenue from each type remains the same.
%     \end{itemize}
% \end{lemma}

% \begin{proof}
%     The proof is the same as the mandatory usage case of \cref{usage-payments-to-training-payment-in-lottery}. As long as there exists a usage payment $x^t_q > 0$, we set $x^t_q = 0$ and increase the training payment $w^t$ by $p^{a^t}_q x^t_q$. Because of the mandatory usage condition, the decrease in usage payment increases all utilities $U(u;\cC^t)$ by $p^{a^t}_q x^t_q$, and the increase in $w^t$ decreases all utilities by the same amount. We observe that because all utilities are invariant, the new menu is IC and IR. Furthermore, we observe the revenue from each type remains the same and is only being redistributed from the usage payment to the training payment.
% \end{proof}

% \begin{corollary} \label{mandatory-usage-profit-is-R_training}
%     The optimal service provider menu under mandatory usage is equal to $R_\te{training}$.
% \end{corollary}

% \begin{proof}
    % By \cref{mandatory-usage-implies-only-training-payment}, the optimal service provider menu under mandatory usage without loss of generality uses only training payments. If all usage payments are zero, however, the distinction between mandatory usage and voluntary usage is irrelevant since the agent will always choose to use the outcome since their nonnegative valuation will always be at least the zero usage price. More formally:
    % \begin{itemize}
    %     \item For every service provider menu under mandatory usage there exists a menu under voluntary usage that has only training payments and achieves the same profit.
    %     \item For every menu under voluntary usage that has only training payments, there exists a menu in the under mandatory usage that achieves the same profit, namely the same menu.
    % \end{itemize}
    % We conclude that $R_\te{mandatory} = R_\te{training}$.
% \end{proof}

% Next, we characterize the profit gap between only allowing training payments and allowing both types of payments.

% \begin{lemma} \label{training-payment-only-gap}
%     Let $\mu^1\le \cds \le \mu^T$ with $\sum_{t\in [T]} \mu^t = 1$ be the agent type distribution $\mu\in \Delta_{[T]}$. Let $R$ denote the maximum service provider profit achievable using both training and usage payments and let $R_\te{training}$ denote the maximum service provider profit achievable using only training payments.

%     \begin{enumerate}
%         \item There exists a problem instance for which $\fr{R}{R_\te{training}} = \sum_{t\in [T]}\fr{\mu^t}{\sum_{i=1}^t \mu^i}$.
%         \item For all problem instances, $\fr{R}{R_\te{training}} \le \sum_{t\in [T]} \fr{\mu^t}{\sum_{i=1}^t \mu^i}$.
%     \end{enumerate}
% \end{lemma}

% \begin{proof}
%     \begin{enumerate}
%         \item We construct the following problem instance:
%         \begin{itemize}
%             \item Let $Q = [T]$. There is a single action $a$ with cost $c(a) = 0$ and transition probabilities $p^a_q = \fr{1}{T},\fl q\in Q$.
%             \item Valuations are given by $v^t_q = \case{\fr{T}{\sum_{i=1}^t \mu_i} & \teif q = t \\ 0 & \teif q \neq t.}$
%         \end{itemize}

%         Note that $$V(t;a) = \fr{1}{T} \cd \fr{T}{\sum_{i=1}^t \mu^i} = \fr{1}{\sum_{i=1}^t \mu^i},$$ so $$R \le \sum_{t\in [T]} \mu^t \cd \fr{1}{\sum_{i=1}^t \mu^i}.$$ The upper bound on $R$ can be achieved using the contract $$\cC = \bp{a, 0, \bp{\fr{T}{\sum_{i=1}^t \mu^i}}_{t\in [T]}}$$ for all types, so $$R = \sum_{t\in [T]} \mu_t \cd \fr{1}{\sum_{i=1}^t \mu^i}.$$ On the other hand, if we are restricted to using training payments, note that any IC menu with a single action consists of a single contract since all types will choose the contract with the least training payment which yields the highest utility. Setting a training payment in the range $$w\in \left(\fr{1}{\sum_{i=1}^{t+1} \mu^i}, \fr{1}{\sum_{i=1}^{t} \mu^i}\right]$$ yields $t$ types choosing the contract over the opt-out option, for a revenue of at most $$\sum_{i=1}^t \mu^i \cd \fr{1}{\sum_{i=1}^{t} \mu^i} = 1,$$ with equality if $$w = \fr{1}{\sum_{i=1}^{t} \mu^t}$$ for some $t$. We conclude that $$R_\te{training} = 1\implies \fr{R}{R_\te{training}} = \sum_{t\in [T]} \mu_i \cd \fr{1}{\sum_{i=1}^t \mu^i},$$ from which the result follows.

%         \item Suppose that the contract $\cC^t = (a^t, w^t, \mathbf{x}^t)$ yields profit $r^t$ from type $t$ and assume without loss of generality that $r^1 \ge r^2 \ge \cds \ge r^T$. We claim that for every $t\in [T]$ we can construct a modified menu with only training payments that achieves at least $$\bp{\sum_{i=1}^t \mu_i} \cd r^t$$ profit. Replace every contract $\cC^u = (a^u, w^u, \mathbf{x}^u)$, including $\cC^t$, with $\cC'^u = (a^u, r^t + c^u, \mathbf{0})$. For any type $u\le t$, note that the revenue from type $u$ is equal to $r^u + c(a^u)$. The IR constraint for type $u$ in the original menu says that type $u$'s value for the outcomes induced by $a^u$ is at least the revenue $r^u + c(a^u)$ from type $u$. Combined with $r^u \ge r^t$, this implies $$U(u; \cC'^u) \ge r^u + c(a^u) - \bp{r^t + c(a^u)} \ge 0,$$
%         so type $u$ will choose some nontrivial contract over the opt-out option. Each contract $\cC'^u$ yields profit at least $r^t + c(a^u) - c(a^u) = r^t$ for the seller, thus overall profit is at least $$\bp{\sum_{i=1}^t \mu^i}\cd r^t$$ as desired, proving the claim.

%         The above claim shows that $$R_\te{training} \ge \bp{\sum_{i=1}^t \mu_i} \cd r^t,\quad \fl t \imp \mu^t \cd r^t \le R_\te{training} \cd \fr{\mu^t}{\sum_{i=1}^t \mu^i},\quad \fl t.$$ Summing over $t\in [T]$ yields $$R \le R_\te{training} \cd \sum_{t\in [T]} \fr{\mu^t}{\sum_{i=1}^t \mu^i}.$$
%     \end{enumerate}
    
    % For every type $t\in \Theta$, effort level $e_1$ deterministically yields a quality level $q_{j(t)}$ that is different for every $t$. The quality level $q_{j(t)}$ has value $v^u_{j(t)} = \fr{1}{t}$ for all types $u\in \Theta$, not only for type $t$. First note that with usage payments we can extract revenue $\fr{1}{t}$ from type $t\in \Theta$ for all $t$ by setting $w^t = 0,\forall t$ and $x^u_{j(t)} = \frac{1}{t},\forall u$. Note that all types $t\in \Theta$ are weakly incentivized to purchase quality $q_{j(t)}$ since $v^t_{j(t)}\ge x^t_{j(t)}$, and furthermore no type $t$ is strictly better off by misreporting their type as $u$ because $x^u_{j(t)}\ge v^t_{j(t)}, \fl t,u\in \Theta$. The principal's total revenue is $\sum_{t=1}^T \frac{1}{t} \eqcolon H_T$, and this is clearly the best possible. Here we use total revenue to denote the sum of the revenues generated by all types. To get the expected revenue, we divide the total revenue by $T$.
% \end{proof}