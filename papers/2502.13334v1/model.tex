\section{Model} \label{model}

We now formally describe our model for selling a service. A problem instance consists of a tuple $(A, Q, [T])$, where $A$ is a finite set of actions, $Q$ is a finite set of outcomes, and $[T]$ is a finite set of $T$ buyer types. Each action $a\in A$ incurs cost $c(a)$ for the seller and leads to a distribution $\mathbf{p}^a \in \Delta_Q$ over outcomes\footnote{We use the notation $\Delta_Q= \{ \mathbf{p} \in [0,1]^Q: \sum_{q\in Q} p_q = 1 \}$ for the simplex supported on $Q$.}, with $p^a_q$ denoting the probability with which outcome $q$ is realized. This stochastic outcome distribution is introduced to capture the highly uncertain performance outcomes when providers sell training services to downstream users \citep{sun2023automl}.

The buyer type $t \in [T]$ is drawn from a prior type distribution $\mu = (\mu^1,\lds,\mu^T) \in \Delta_{[T]}$ known to the seller. If $\mu^t = 0$ for some $t$ then the buyer type never shows up and can be excluded, so we assume that $\mu^t > 0$ for all $t\in [T]$. Each type $t$ is associated with a \emph{valuation vector} $\mathbf{v}^t \in \bR^Q$ such that $v^t_q$ is type $t$'s value for outcome $q$. As is common in contract design, we assume that the buyer type does not affect the action-to-outcome transition probabilities \citep{dutting2024algorithmic, guruganesh2021contracts, guruganesh2023menus, zuo2024new}.

%  To the best of our knowledge, this natural mechanism design setup is new yet connects to a few widely studied problems (see detailed discussion in Section \ref{sec:related-model}).

\paragraph{Menus of contracts.}

We define a \emph{menu} $\cM$ as a set of contracts, where each contract $\cC = (a, w, \mathbf{x}) \in \cM$ specifies an action $a \in A$, an upfront price $w \in \bR_{\ge 0}$, and a vector of usage prices $\mathbf{x} \in \bR^Q_{\geq 0}$ where $x_q$ is the usage price for outcome $q \in Q$.

% Given a menu $\cM$, a type $t$ buyer who selects a contract $\cC$ has expected utility given by
% $$U(t; \cC) \coloneq   \sum_{q\in Q}  p^a_q \cdot  \max\{v^t_q - x_q, 0\} - w$$

\subsection{Seller-buyer interaction protocol} \label{interaction}

The interaction between the seller and buyer goes through the following steps:

\begin{enumerate}
    \item The seller commits to a menu $\cM $ of contracts and presents it to the buyer.
    
    \item A buyer with type $t$ drawn from the prior $\mu$ arrives and selects a contract $\cC = (a, w, \mbf{x}) \in \cM$.
    
    \item The buyer pays the seller the upfront price $w$ to enter into the contract, and the seller performs the action $a$ that they committed to.
    
    \item An outcome $q\sim \mathbf{p}^a$ is realized and observed by the buyer.
    
    \item After observing the outcome, the buyer can either choose to accept the outcome at price $x_q$ or reject it and pay nothing. 
\end{enumerate}

% Given the above interaction protocol, we now turn to analyzing the buyer's decision. When facing the given menu, the buyer's expected utility from any menu upfront $(w^{t'}, \xi^{t'}, \mathbf{x}^{t'})$ is analyzed as follows. Under the effort level $e^{t'}$,  the quality level $q_j$ will be realized with probability $p_{e^{t'}}^j$. \hfcomment{Let's try to use such (simpler) dot products in our equations as much as well can, since it is cleaner and also easier for readers to digest.}  

% s mentioned in the introduction, the incorporation of outcome-based usage payments and the voluntary usage assumption sets our pricing protocol apart from baseline industrial pricing schemes employed by Vertex AI and Amazon Sage Maker.

 We refer to the assumption that a buyer is free to accept or reject the outcome as the \emph{voluntary usage} assumption. A rational buyer of type $t$, after observing the outcome $q$, will accept it if and only if their value for the outcome is at least the corresponding usage price: $v_q^t \geq x_q$. We assume, as is common in  principal-agent problems, that the buyer breaks ties in favor of the seller. The expected utility of a buyer with type $t$ that selects a contract $\cC = (a, w, x) \in \cM$ is given by
$$U(t; \cC) \coloneq \sum_{q\in Q}  p^{a}_q \cdot  \max\{v^t_q - x_q, 0\} - w.$$ 



%Note that the seller keeps the buyer's upfront payment regardless of whether the final outcome is accepted.

% We refer to the buyer's freedom to accept or reject the outcome in Step (5) as \emph{voluntary usage}. The buyer accepts the outcome $q$ and pays the usage price $x_q$ if and only if their value for outcome $q$ is at least the cost, namely $v_q \ge x_q$. Otherwise, the buyer rejects outcome $q$ and is exempt from additional payment. As is common in principal-agent problems, we assume that the buyer breaks ties in favor of the seller, which means that they accept the outcome if $v_q = x_q$. 

    % In Step (5), note that after observing the outcome $q$ the buyer can either choose to accept outcome $q$ at price $x^t_q$ or reject it. We refer to this assumption as \emph{voluntary usage}.
    


\paragraph{Incentive compatibility and direct menus.} \label{IC}

We can without loss of generality assume that the size of the menu $\cM$ is at most $T$ because each buyer type will select only the contract that yields the highest expected utility for them. By relabeling the utility-maximizing contract for type $t$ in the menu as $\cC^t$ and allowing for duplicate contracts if multiple types select the same contract, we can equivalently define a menu as a tuple of contracts $\cM = \bp{\cC^t}_{t\in [T]}$, one for every type. This reduction from \emph{indirect} menus to \emph{direct} menus\footnote{The term \emph{direct} refers to indexing a contract by the type that selects it.} using the revelation principle is standard in contract design \citep{castiglioni2021bayesian}. 

% Note that the condition that type $t$ selects $\cC^t$ over any other contract is precisely the IC condition for type $t$.

\begin{definition*}[Incentive compatibility (IC)]
A menu $\cM = \bp{\cC^t}_{t\in [T]}$  is called \emph{incentive compatible (IC)} if $U(t; \cC^t)  \geq U(t;\cC)$ for all buyer types $t$ and for all contracts $\cC\in \cM$.
\end{definition*}

% The menu $\cM $ is incentive compatible, if 
% \begin{align}
%     U(t; \cC^t) \geq U(t ; \cC), \quad \text{for all $\cC \in \cM$}.
% \end{align}

The IC constraints imply that it is optimal for a type $t$ buyer to report their type truthfully and receive contract $\cC^t$ since no other contract $\cC^{t'}$ will yield a higher utility for them. 

% \begin{align*}
%     \sum_{j = 1}^n \left( \sum_{i = 1}^m \xi_i^{t'} p_j^{i,t} \right) \max\{v_j^t - x_j^{t'}, 0\} - w^{t'}.
% \end{align*}


\paragraph{Individual rationality.}

Finally, we assume that the buyer is always allowed to not select any contract and opt-out of the mechanism entirely. Thus, any contract selected by a buyer must yield nonnegative buyer surplus, a constraint commonly known as \emph{individual rationality} (IR).
\begin{definition*}[Individual rationality (IR)]
A menu $\cM = \bp{\cC^t}_{t\in T}$ is called \emph{individually rational (IR)} if $U(t; \cC^t)  \geq 0$ for all buyer types $t$.
\end{definition*}

% Not only is the IR assumption is natural, we also show that IR is \emph{necessary} for obtaining sensible results. We refer the reader to \cref{opt-out-option} for a discussion about how unnatural results can arise if the buyer is not allowed to opt-out.

%Formally, this is equivalent to always including in the menu a \emph{trivial} contract $\cC^0$ that deterministically maps a \emph{trivial} action $a^0$ with zero cost to a \emph{trivial} outcome that has zero value for all types, and that has zero upfront and usage prices. 

\paragraph{The service provider optimization problem.}

% \begin{subequations} \label{full-optimization-problem}
% \begin{empheq}[box=\widefbox]{align}
% \max_{ \{(a^t, w^t, \mathbf{x}^t) \}_{t \in [T]]}  } & \quad \bE_{t \sim \mu} \left[ w^t + \sum_{q\in Q} p^{a^t}_q x^t_q \cdot \one \bb{v^t_q \ge x^t_q} - c(a^t) \right] \notag\\ 
%     & U(t; \cC^t) \geq U(t; \cC^{t'}) & \fl t, t' \in [T]   \notag\\
%     & U(t; \cC^t) \geq 0  & \fl t  \in [T]  \notag 
% \end{empheq}
% \end{subequations}

% \begin{tcolorbox}[title=Maximizing revenue of a direct menu]
% \begin{align}
% 				 \max_{ \{(a^t, w^t, \mathbf{x}^t) \}_{t \in [T]}  } & \quad \bE_{t \sim \mu} \left[ w^t + \sum_{q\in Q} p^{a^t}_q x^t_q \cdot \one\bb{v^t_q \ge x^t_q} \right] \label{revenue-maximization-direct-menu} \\ 
%     & U(t; \cC^t) \geq U(t; \cC^{t'}) && \fl t, t'   \notag \\ 
%     & U(t; \cC^t) \geq 0  && \fl t  \notag
% \end{align}
% \end{tcolorbox}

Given an IC and IR menu $\cM = \bp{\cC^t}_{t \in [T]}$, the seller's expected profit is given by 
\begin{align*}
    \pi(\cM) \coloneq \bE_{t \sim \mu} \left[ w^t  - c(a^t) + \sum_{q\in Q} p^{a^t}_q x^t_q \cdot \one\bb{v^t_q \ge x^t_q} \right]
\end{align*}
The seller's goal is to design a menu that maximizes their expected profit\footnote{Henceforth, for brevity we refer to expected profit as simply \emph{profit}.}, which amounts to solving the following optimization problem:

\begin{tcolorbox}[title=Maximizing profit of a direct menu]
\begin{align}
 %   \max_{ \{(a^t, w^t, \mathbf{x}^t) \}_{t \in [T]}  } & \quad \bE_{t \sim \mu} \left[ w^t  - c(a^t) + \sum_{q\in Q} p^{a^t}_q x^t_q \cdot \one\bb{v^t_q \ge x^t_q} \right] \label{profit-maximization-direct-menu} \\
 \max_{\cM = \bp{\cC^t}_{t \in [T]}}  & \pi(\cM) \label{profit-maximization-direct-menu} \\
         U(t; \cC^t) &\geq U(t; \cC^{t'}) & & \fl t, t' \in [T]  \notag \\ 
         U(t; \cC^t) &\geq 0  && \fl t \in [T] \notag
\end{align}
\end{tcolorbox}

% Henceforth we refer to \cref{profit-maximization-direct-menu} as the \emph{service provider problem}.

% \begin{align*}
%     \expec \left[ \sum_{j = 1}^n \left( \sum_{i = 1}^m \xi_i^{t} p_j^{i,t} \right) x_j^t \ind \{v_j^t - x_j^{t} \geq 0 \} + w^{t} - \sum_{i = 1}^m \xi_i^{t} c_i \right].
% \end{align*}

% \subsection{Revelation Principle}
% % \hfcomment{Open problem 1:  prove the revelation principle and prove that   the absolutely  optimal utility can be achieved by the following formulation with a menu} 
% % \begin{definition}
% %     A \textit{direct menu} is a 
% % \end{definition}
% \begin{lemma}[Revelation Principle]
%     Given any equilibrium of any menu, there
%     exists an equivalent menu in which truthful revelation of type is an equilibrium and
%     in which the service provider and the buyer get the same expected utilities.
% \end{lemma}
% \kicomment{What do we mean by direct menu vs any menu? has it been defined formally? if not, do we need this result? } 
% \begin{proof}
%     Assume that in every equilibrium the buyer tiebreaks in favor of the service provider. Without loss of generality assume the reporting function $r:[T] \to \Delta ([T])$ for every type is deterministic, as all menus in the support of any buyer's strategy must generate the same expected utility for the buyer as well as the service provider by the tiebreaking assumption. For any type $t$ that reports type $r(t)$, we have $$\sum_{j=1}^n \left(\sum_{i = 1}^m \xi^{r(t)}_i p_j^{i, r(t)}\right)\max \{v_j^{r(t)} - x_j^{r(t)}, 0\}- w^{r(t)} \geq \sum_{j=1}^n \left(\sum_{i = 1}^m \xi^{t'}_i p_j^{i, r(t)}\right)\max \{v_j^t - x_j^{t'}, 0\}- w^{t'}, \forall t'.$$ Hence if we replace $(w^t, \xi^t, x^t)$ by $(w^{r(t)}, \xi^{r(t)}, x^{r(t)})$ for all $t$, an equilibrium in the resulting menu has every type reporting honestly while satisfying the IR constraints. Furthermore, the expected utilities of the buyer and service provider remain the same. 
% \end{proof}

% \subsection{Model example}

% % \begin{lemma}[Necessity of training payment]
% %  %   The optimal contract may have the upfront payment $w>0$, i.e., $\exists t$, such that $w^t > 0$.
% % Training payment is necessary for optimality. Specifically, there exists an instance with a single effort level where any contract with $w^t = 0, \forall t$ is strictly sub-optimal.   
% % \end{lemma}
% % \begin{proof}
%  We explain the necessity of incorporating usage payments through the following example. Specifically, we reveal that an optimal menu replying solely on training payments falls short on in achieving optimal revenue, unlike one that also includes usage payments. 
% \hfcomment{This example seems not valid any more, since our model assumes transitions are independent of type, but the following example depends on type. }
% \begin{example}[label = exm:cont]

% To demonstrate the necessity of usage payments, consider a scenario with a single effort level where the fine-tuned model has two quality levels: low quality $j_1$ and high quality $j_2$. Suppose there are two buyer types, $t_1$ and $t_2$, each occurring with equal probability. The buyer of type $t_1$ has a higher quality training dataset, leading to a higher probability of achieving a high-quality model, with values $(v_j^{t_1}) = (1, 2)$. Conversely, the buyer of type $t_2$ has a lower quality dataset, reflected in values $(v_j^{t_2}) = (0.5, 1)$ and a higher probability of resulting in a low-quality model.

% The transition probabilities for each buyer type reaching each quality level are summarized as follows:

% % \begin{table}[!htb]
% % \begin{minipage}[t]{0.45\linewidth}
% %        \caption{The buyer's value $v_j^t$}
% %         \centering
% %         \begin{tabular}{l|cc}
% %         \toprule
% %         \diagbox[width=13em]{Buyer's type}{Quality level}
% %         &$j_1$ & $j_2$ \\
% %          \midrule
% %          $t_1$   & 1 & 2 \\
% %            $t_2$  & 0.5 & 1\\
% %         \bottomrule
% %         \end{tabular}
% % \end{minipage}
% % \hfill
% % \begin{minipage}[t]{0.45\linewidth}
% %         \centering
% %          \caption{The transition probabilities $p_j^t$}
% %         \begin{tabular}{l|cc}
% %         \toprule
% %      \diagbox[width=13em]{Buyer's type}{Quality level}
% %         &$j_1$ & $j_2$ \\
% %          \midrule
% %          $t_1$   & 0.2 & 0.8 \\
% %            $t_2$  & 0.8 & 0.2\\
% %         \bottomrule
% %         \end{tabular}
% % \end{minipage}
% % \end{table}

% \begin{table}[!htb]
% % \begin{minipage}[t]{0.45\linewidth}
% %        \caption{The buyer's value $v_j^t$}
% %         \centering
% %         \begin{tabular}{ccc}
% %         \toprule
% %         &$j_1$ & $j_2$ \\
% %          \midrule
% %          $t_1$   & 1 & 2 \\
% %            $t_2$  & 0.5 & 1\\
% %         \bottomrule
% %         \end{tabular}
% % \end{minipage}
% % \hfill
% \begin{minipage}[t]{0.9\linewidth}
%         \centering
%          \caption{The transition probabilities $p_j^t$}
%         \begin{tabular}{ccc}
%         \toprule
%         &$j_1$ & $j_2$ \\
%          \midrule
%          $t_1$   & 0.2 & 0.8 \\
%            $t_2$  & 0.8 & 0.2\\
%         \bottomrule
%         \end{tabular}
% \end{minipage}
% \end{table}
% We solve the non-linear program using Gurobi \yzedit{add solver} and obtain the optimal menu with training payments and usage payments with the optimal revenue $1.1$.
% % \begin{align*}
% %     \max_{x, w} \quad &  \frac{1}{2}\left(p_{j_1}^{t_1} x_{j_1}^{t_1} \ind \{v_{j_1}^{t_1}-x_{j_1}^{t_1} \geq 0\} + p_{j_2}^{t_1} x_{j_2}^{t_1}\mathbf{I}  \{v_{j_2}^{t_1}-x_{j_2}^{t_1} \geq 0 \} + \\ 
% %     & p_{j_1}^{t_2} x_{j_1}^{t_2} \mathbf{I} \{v_{j_1}^{t_2}-x_{j_1}^{t_2} \geq 0\} + p_{j_2}^{t_2} x_{j_2}^{t_2}\mathbf{I}  \{v_{j_2}^{t_2}-x_{j_2}^{t_2} \geq 0 \}  +w^{t_1} + w^{t_2} \right)- c \notag \\
% %    &p_{j_1}^{t_1} \max \{v_{j_1}^{t_1}-x_{j_1}^{t_1}, 0\}+p_{j_2}^{t_1} \max \{v_{j_2}^{t_1}-x_{j_2}^{t_1}, 0\} - w^{t_1} \geq p_{j_1}^{t_1} \max \{v_{j_1}^{t_1}-x_{j_1}^{t_2}, 0\}+p_{j_2}^{t_1} \max \{v_{j_2}^{t_1}-x_{j_2}^{t_2}, 0\} - w^{t_2}\\
% %    & p_{j_1}^{t_2} \max \{v_{j_1}^{t_2}-x_{j_1}^{t_2}, 0\}+p_{j_2}^{t_2} \max \{v_{j_2}^{t_2}-x_{j_2}^{t_2}, 0\} - w^{t_2} \geq p_{j_1}^{t_2} \max \{v_{j_1}^{t_2}-x_{j_1}^{t_1}, 0\}+p_{j_2}^{t_2} \max \{v_{j_2}^{t_2}-x_{j_2}^{t_1}, 0\} - w^{t_1}\\
% %    & p_{j_1}^{t_1} \max \{v_{j_1}^{t_1}-x_{j_1}^{t_1}, 0\}+p_{j_2}^{t_1} \max \{v_{j_2}^{t_1}-x_{j_2}^{t_1}, 0\} - w^{t_1} \geq 0\\
% %    & p_{j_1}^{t_2} \max \{v_{j_1}^{t_2}-x_{j_1}^{t_2}, 0\}+p_{j_2}^{t_2} \max \{v_{j_2}^{t_2}-x_{j_2}^{t_2}, 0\} - w^{t_2} \geq 0 \\
% %    & x, w \geq 0, 
% % \end{align*}

% \begin{align*}
%     &w^{t_1} = 0.1, w^{t_2} = 0.4, \\
%     & x_{j_1}^{t_1} = 0.5, x_{j_2}^{t_1}  = 2, x_{j_1}^{t_2} = 0, x_{j_2}^{t_2} = 1.75.   
% \end{align*}
% However, it achieves strictly less revenue with the training payment menu only and usage payment only menu. 
% On one hand, if we restrict our menus to only have training payments, the optimal revenue is \yzcomment{add revenue of training payment only}. On the other hand, if we restrict our menus that only has usage payments, it achieves the optimal value $1.05$ with the optimal usage payments

% \begin{align*}
%     x_{j_1}^{t_1} = 1, x_{j_2}^{t_1}  = 1.875, x_{j_1}^{t_2} = 0.5, x_{j_2}^{t_2} = 2
% \end{align*}
% \end{example}

