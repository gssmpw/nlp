\subsection{Connections with other models} \label{connections}

% A recent survey by \citet{duetting2024algorithmic} views principal-agent problems as interactions between an informed and uninformed party and classifies them according to two criteria. The first criterion is whether the private information concerns \emph{who} the agent is, or whether it concerns \emph{what} action the agent takes. The second criterion is whether the uninformed party moves first to design the incentive scheme or whether the informed party moves first.

% In the language of \citet{duetting2024algorithmic}, our problem can be viewed

%\hf{Conjecture. In the setting of \citet{chen2015complexity}, if you are allowed to do performance pricing, it is not more powerful than fixed pricing. The fundamental reason is due to your power of designing the lottery probabilities (an equivalent view, in our setup, is that the effort levels are infitnitely large, so large that you can induce any possible quality distributions. We should be able to prove in this case upfront price is optimal. )}

% \hfcomment{I formulated this as a proposition. Please make sure the statement is more formal and the proof is rigorously tailored towards the statement.}

While our service provider problem is new, it has close connections to a few widely studied problems in the mechanism design literature. 
% Interestingly, some previously studied models can be viewed as special cases of our problem under careful reformulations.

\paragraph{Selling hidden actions.}

\citet{bernasconi2024agent} study a related problem of selling a service modeled by a hidden action. Though both models are variants of principal-agent problems in which the seller is performs the action, there are two fundamental differences between their setting and ours. First, we assume that the seller action is not hidden from the buyer  but rather can be committed to. We argue that this absence of \emph{moral hazard} is a natural assumption in our service provider problem. From a practical perspective, automated machine learning (AutoML) platforms such as Vertex AI and SageMaker are large-scale and backed by highly regulated parent companies and thus can commit to performing the services they offer. From a theoretical perspective, it is well-known that commitment leads to higher leader utility compared to no commitment in leader-follower games \citep{von2010leadership}, hence there is a clear economic incentive for the seller (leader) in our contract design setting to be able to commit to actions. Second, we do not require the buyer to purchase the end product and instead give the buyer the option to \emph{reject} the outcome, in which case they do not receive the product but are also not required to pay for it. We show that under this \emph{voluntary usage} assumption, seller profit is, perhaps surprisingly, always weakly higher than \emph{mandatory usage}, which is when the seller forces the buyer to accept and pay for every outcome.

 % The second fundamental difference is that, in \citep{bernasconi2024agent} the buyer is \emph{required} to pay for whatever the outcome is, even when the usage payment exceeds the buyer valuation for that outcome. We refer to their assumption as \emph{mandatory usage}, in contrast to \emph{voluntary usage} in our model. We show in \cref{two-payments} that mandatory usage can actually be viewed as a special case of voluntary usage in which the seller only charges upfront prices and not usage prices\footnote{Mandatory usage is not a special case of voluntary usage if the seller cannot commit to actions, hence the reduction in \cref{two-payments} is not applicable to the model studied by \citet{bernasconi2024agent}.}. This reduction demonstrates that voluntary usage is more general and hence yields higher seller profit. Voluntary usage is perhaps also more natural in practice, since the seller may not want to force a buyer to pay for an outcome whose quality they do not find satisfactory.

% (2) the actions correspond to training algorithms and computing resources which are observable anyway; 

%Their fundamental difference from ours is that they require the agent to accept whatever the outcome is, even if the usage payment exceeds the agent valuation for that outcome. We refer to their assumption as mandatory usage, which is contrasted with voluntary usage in our model. As we show in Section \cref{two-payments}, mandatory usage can be viewed as a restriction of voluntary usage when only training payments are allowed, thereby demonstrating that voluntary usage is both more general and sometimes yields significantly higher service provider profit than mandatory usage. Voluntary usage is perhaps also more natural in practice, since the service provider may not want to force a customer into paying to use a model whose quality they do not find satisfactory. \hfcomment{cite some paper that says Stackelberg is always better than Nash. }

\paragraph{Selling lotteries.}

Since a key feature of our model is that the service has uncertain outcomes, it is naturally related to the well-studied problem of selling lotteries, for example see \citet{chen2015complexity}. A \emph{lottery} draws an \emph{item}, which is analogous to the \emph{outcome} in our model, from a set $Q$, with different items having different probabilities of being drawn. An buyer's value for a lottery is the expected value of their valuation for the item that the lottery draws. A \emph{menu} in the lottery pricing problem consists of, for each type $t$, a tuple $(\mathbf{p}^t, w^t)$ for each type $t$, where $w^t$ is the lottery price and $p^t_q$ is the probability of receiving item $q$ if the lottery is purchased. The fundamental difference between lottery pricing and service pricing is that a lottery seller has the freedom to design arbitrary lottery distributions $\mathbf{p}^t$ for any type $t$, whereas in our model only those distributions that are achievable by provider actions are available. Nonetheless, the lottery pricing problem can be viewed as a special case of our service provider problem where there are infinitely many actions that induce all of the possible outcome distributions $\mathbf{p}$. The upfront payment in our model corresponds to the lottery price. We prove in \cref{usage-payment-lottery-pricing} that in lotteries, the maximum seller revenue remains the same with or without usage payments, so the two-part tariff structure in lottery pricing is not needed. In contrast, usage payments in our service provider problem are crucial, as we show in \cref{usage-payment-necessary}.

% A type $t$ buyer with valuation vector $\mathbf{v}^t$ for the items, when faced with the decision problem of which lottery to choose, chooses the lottery that yields the highest expected utility by solving the optimization problem $$\max_{t\in [T]} \left(-w^t + \sum_{q\in Q} p^t_q v^t_q \right)$$ where the maximum is taken over all lotteries in the menu. The seller's expected revenue from this menu is simply the expected payment $\bE\bb{w^t}$.

\paragraph{Selling products of differing qualities.}

\citet{mussa1978monopoly} and \citet{maskin1984monopoly} initiated the study of pricing products of differing qualities. Their model corresponds to a special case of the service provider problem where each action deterministically maps to a unique quality and hence these two notions are interchangeable. Our \emph{outcomes} correspond to the \emph{qualities} in \citep{mussa1978monopoly} and our action costs are their production costs. While \citet{mussa1978monopoly} and \citet{maskin1984monopoly} provide structural characterizations of profit-maximizing mechanisms and solve for special cases, the optimal solution to the general quality pricing problem as well as its computational complexity remain open, which also hints on the challenge for solving our even more general setup. While \citet{mussa1978monopoly} and \citet{maskin1984monopoly} assume continuous qualities and cost functions, our focus in this work is on analyzing the computational complexity of the service provider problem with discrete model primitives.
