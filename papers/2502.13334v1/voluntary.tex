\subsection{Voluntary usage subsumes mandatory usage} \label{voluntary-usage}

In this section we show that seller profit is, unintuitively, always weakly higher when the seller allows the buyer to choose whether to use the realized outcome or not than when the seller forces the buyer to use and pay for every outcome. Recall that the former assumption is \emph{voluntary usage} and the latter assumption is \emph{mandatory usage}. We actually show a stronger result: when the seller can commit to an action, mandatory usage is a \emph{special case} of voluntary usage. This is done by establishing a connection between mandatory usage and the zero usage price scenario from \cref{upfront-payment-only}. In particular, we show a surprising equivalence: the maximum seller profit under mandatory usage is exactly equal to the maximum seller profit under voluntary usage with zero usage prices, which is the quantity $\Rupfront$ from \cref{usage-payment-necessary}.

% \footnote{Recall that in \emph{mandatory usage}, the buyer is \emph{required} to accept the outcome and pay the associated usage price. In contrast, our model assumes \emph{voluntary usage} in which the buyer, after observing the outcome, can choose to either accept it and pay the usage price or reject it in which case no further payment is required.}

\paragraph{Quantitative result.}

\newcommand{\Rmandatory}{R_\mathtt{mandatory}}

Consider a modified version of the service provider problem that has the same seller-buyer interaction protocol described in \cref{interaction} but where in the last step the buyer is forced to accept the realized outcome $q$ and pay the usage price $x^t_q$. Denote by $\Rmandatory$ the maximum seller profit in this \emph{mandatory usage} model. Our main result is the following:

\begin{theorem} \label{voluntary-usage-subsumes}
 Let $\mu \in \Delta_{[T]}$ denote the buyer type distribution. Then
 \begin{equation} \label{R-Rmandatory}
     \Rupfront = \Rmandatory \le R \le H_\mu \cd \Rmandatory,
 \end{equation}
 %  where the prior-dependent factor $$H_\mu = \sum_{t\in [T]} \fr{\mu^t}{\sum_{i=1}^t \mu^i}$$ is the same quantity from \cref{usage-payment-necessary}.
where $H_\mu$ was defined in \cref{eq:H-mu}. Moreover, \cref{R-Rmandatory} is tight: there exist problem instances for which $R = \Rmandatory$ and problem instances for which $R =  H_\mu \cd \Rmandatory$.
\end{theorem}

\begin{proof}
    The heavy lifting has already been done for us by \cref{upfront-payment-only}, and to complete the proof it suffices to show that $\Rmandatory = \Rupfront$. We do so using a \emph{payment redistribution} argument:

    \begin{claim*}
    Assuming mandatory usage, any IC menu can be modified into a menu with only upfront prices such that buyer utilities and seller profit remain the same.
    \end{claim*}

    \begin{proof}
           As long as there exists a contract $\cC^u= (a^u, w^u, \mathbf{x}^u)$ with a positive usage price $x^u_q > 0$, we modify $\cC^u$ to $\cC'^u$ by setting $x'^u_q = 0$ and $w'^u = w^u + p^{a^u}_q x^u_q$. Because of the mandatory usage assumption, the decrease in usage price increases all buyer utilities by the same amount that the increase in upfront prices decreases them. Since buyer utilities remain the same, the modified menu is IC. Furthermore, the revenue from each type is redistributed in equal amount from the usage price into the upfront price and hence remains the same.
    \end{proof}

        By the claim, a profit-maximizing menu under mandatory usage without loss of generality has only upfront payments and no usage payments. Under zero usage payments, however, the distinction between mandatory usage and voluntary usage is irrelevant since the buyer will always accept the outcome as their nonnegative valuation will always be at least the zero usage price. More formally, the claim implies the following equivalence:
        \begin{itemize}
            \item For every menu under mandatory usage there exists a menu under voluntary usage that has only upfront payments and achieves the same profit.
            \item Every menu under voluntary usage that has only upfront payments is a menu under mandatory usage that achieves the same profit.
        \end{itemize}
       We conclude that $\Rmandatory = \Rupfront$ as desired. As for tightness of \cref{voluntary-usage-subsumes}, note that any problem instance where every type has the same valuation vector is equivalent to an instance with one buyer type, in which case $H_\mu = 1\imp R = \Rmandatory$. The existence of problem instances for which $R = H_\mu \cd \Rmandatory$ comes from \cref{upfront-payment-only}. 
\end{proof}

\paragraph{Qualitative discussion.}

Previous models to sell contracts either assume forced payments for each outcome regardless of whether the price exceeds the buyer's valuation of the outcome, for example selling hidden actions \citep{bernasconi2024agent}, or charges only a lump sum payment for the action without requiring usage payments, for example selling lotteries \citep{chen2015complexity}. A natural  question is why we should consider a two-part payment scheme that first charges an  upfront price but then allows buyers the freedom to accept or reject the realized outcome.  From a technical perspective, \cref{voluntary-usage-subsumes} proves that the maximum seller profit is \emph{always} weakly higher under voluntary usage than under mandatory usage, and furthermore sometimes significantly so, as characterized by the prior-dependent multiplicative factor $H_\mu\in [H_T, T)$. But also in practice, voluntary usage gives the buyer the freedom of choice and hence could be more attractive from a marketing standpoint.

% \todo{I feel the discussion of Adam smith is somewhat loose. Can we be more detailed about it? Perhaps we can say that while the imposition of mandatory usage requirement is beneficial {\em after} the buyer has chosen to participate, it reduces the likelihood a buyer participates in the mechanism. In particular, under mandatory usage, the seller has to incentivize the buyer more to participate in the mechanism, thereby leading to lower profits.} We may be able to qualitatively understand voluntary usage's advantage for the seller through the lens of Adam Smith and his \emph{invisible hand} argument for the efficiency of free markets. 

Our result may appear counterintuitive since one might think the seller should benefit when they have the power to enforce something, in this case the buyer's acceptance of the outcome. Evidently, this line of thinking is flawed. While the mandatory usage requirement is beneficial {\em after} the buyer has chosen to participate, it reduces the likelihood that a buyer participates in the mechanism. Under mandatory usage the seller has to incentivize the buyer more to participate in the mechanism. Voluntary usage, on the other hand, reduces the buyer's risk that arises from the uncertain nature of outcomes and hence can make buyers value a contract \emph{more}. This makes them more willing to pay a higher upfront price, leading to increased seller profit.

% We also prove that our profit gap in the case of training payments only is tight. Finally, we show how an optimal menu using only training payments can be computed in polynomial time, which implies as a corollary the existence of a polynomial-time algorithm that achieves $\Omega\bp{\fr{1}{\log T}}$ fraction of optimal profit \emph{using only training payments} when the type distribution is uniform.




