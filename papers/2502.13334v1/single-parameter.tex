\section{The single-parameter setting} \label{single-parameter}

This section examines a fundamental special case of the general service provider problem in which each buyer type's valuation vector can be parametrized by single real-valued number. % We assume that every outcome $q$ has a \emph{baseline} valuation $v(q) \in \bR_{\ge 0}$.

\begin{definition*}[Single-parameter setting] \label{single-parameter-definition}
 Buyer types are \emph{single-parameter} if each buyer type $t$'s valuation vector is characterized by a single parameter $\alpha^t>0$  such that $v^t_q = \alpha^t \cd v(q)$ for every $q$, where $v(q) \in \bR_{\ge 0}$ is the \emph{baseline} valuation for outcome $q$. 
 \end{definition*}

 % 
% Given two buyer types, we refer to the type with a higher parameter as the \emph{higher} type and the type with a lower parameter as the \emph{lower} type. 

Since buyers with the same single parameter $\alpha$ have the same valuations and hence can be merged into one type, we can without loss of generality label the buyer types so that $\alpha^1 < \alpha^2 < \cds < \alpha^T$. Single-parameter preferences are well-studied in contract design problems \citep{alon2021contracts,li2022selling,chen2015complexity} as well as in broader economic settings. For example, the linear buyer utility function in \citet{mussa1978monopoly} is equivalent to the above single parameter setting. The main result of this section is that in the single-parameter setting, the maximum seller \emph{revenue} can be achieved by a single contract for all buyer types:

\begin{theorem} \label{single-parameter-revenue}
    The single-parameter setting admits a \emph{revenue}-maximizing menu that consists of a single contract for all buyer types. Moreover, this single contract can be computed in polynomial time in $T$,  $\ab{A}$, and $\ab{Q}$.
\end{theorem}

Our computationally positive result for the single-parameter setting contrasts with the \tbf{NP}-hardness of computing a revenue-maximizing menu in the general service provider problem (\cref{np-hardness-two-types}) and thus shows a computational advantage of the single-parameter setting. This insight extends to maximizing seller profit with zero action costs since in that case seller revenue equals seller profit. However, the analog of \cref{single-parameter-revenue} for profit maximization is not true in general. Specifically, in \cref{counterexample-profit}  we construct a problem
instance with nonzero action costs in which no single contract can maximize profit.  In the remainder of this section we prove \cref{single-parameter-revenue}, relegating the proofs of a few technical lemmas to \cref{appendix}.

\paragraph{Proof outline.} First, we show that any IC direct menu of contracts satisfies conditions on the values $V(t;\cC^t)$ of contract $\cC^t$ for type $t$ that roughly say that a buyer type with a higher single parameter $\alpha$ must have higher value for their contract than a type with lower $\alpha$. Second, we use this characterization to derive exact formulas for the revenue-maximizing upfront prices in terms of the values. Third, we use the fact that these formulas are \emph{linear} in the buyer values to relax the revenue-maximization problem in the single-parameter setting into a linear program in the buyer values. Finally, we show that the buyer values in any extreme point of the feasible region of this relaxed linear program are achievable by a menu consisting of a single contract. Since a linear program is maximized at an extreme point, we conclude that a single contract is revenue-maximizing. We now elaborate on these proof steps, beginning with some notation.
 

\paragraph{Notation.}

Define $S^t \coloneq \{q : x^t_q = 0\}$ and recall that $V(t;\cC^u) = \sum_{q\in S^u} p^{a^u}_q v^t_q$ is the value of contract $\cC^u$ for type $t$. For notational brevity, we define $V(t;u) \coloneq V(t;\cC^u)$, $V^t \coloneq V(t;\cC^t)$, and $U(t;u) \coloneq U(t;\cC^u)$. We can upper-bound $V^t$ by $$V^t \le \max_{a\in A} \sum_q p^a_q \alpha^t \cd v(q) = \alpha^t \cd \bp{\max_{a\in A} \sum_q p^a_q \cd v(q)} \eqqcolon \alpha^t \cd M,$$ with equality if the contract action is $\argmax_{a\in A} \sum_q p^a_q \cd v(q).$ By the single-parameter assumption, we have $V(t;u) = \fr{\alpha^t}{\alpha^u} \cd V^u$. In our new notation we have
\begin{align*}
    U(t;u) = V(t;u) - w^u = \fr{\alpha^t}{\alpha^u} \cd V^u - w^u.
\end{align*}
The IC constraints are $U(t;t) \ge U(t;u)$ and the IR constraints are $U(t;t) \ge 0 \iff w^t \le V^t$.

\paragraph{Types with higher $\alpha$ have higher values.}

% For $t' > t$, we can compute the utility of contract $\cC^t$ for type $t'$ as $\fr{\alpha^{t'}}{\alpha^t} \cd V^t - w^t$ because $$V(t';t) = \fr{\alpha^{t'}}{\alpha^t}\cd V^t.$$ Note that by \cref{two-usage-prices-suffice}, type $t'$ will not use any outcomes $j\notin S^t$ since their usage payments are $\infty$. 


We first show a condition on the $V^t$'s and $w^t$'s that implies that types with higher $\alpha$ have higher values and higher seller revenues:
\begin{lemma} \label{higher-types-higher-values}
    In any IC menu of contracts, we have $\fr{V^t}{\alpha^t} \geq \fr{V^u}{\alpha^u}$ and $w^t \geq w^u$ for all buyer types $t,u$ satisfying $t \ge u$.
\end{lemma}

\cref{higher-types-higher-values} is proven in \cref{higher-types-higher-values-proof}.

\paragraph{Computing revenue-maximizing upfront prices.}

Using \cref{higher-types-higher-values} allows us to derive exact formulas for the revenue-maximizing upfront prices in terms of the values:

\begin{lemma} \label{revenue-maximizing-upfront-prices}
Fixing the value vector $\bp{V^t}_{t\in [T]}$, the revenue-maximizing upfront prices are
\begin{align*}
w^t = V^1 + \sum_{i=2}^t \bp{V^i - \frac{\alpha^i}{\alpha^{i-1}}\cd V^{i-1}} = V^t  - \sum_{i=1}^{t-1} \left( \fr{\alpha^{i+1}}{\alpha^i} - 1\right) V^i.
\end{align*}
\end{lemma}

\cref{revenue-maximizing-upfront-prices} is proven in \cref{revenue-maximizing-upfront-prices-proof}.

\paragraph{A linear program for revenue maximization.}

By \cref{revenue-maximizing-upfront-prices}, the maximum seller revenue in terms of the $V^t$ is
\begin{align}
\sum_{t\in [T]} \mu^t \cd w^t = \sum_{t\in [T]} \left(\mu^t -  \left(\fr{\alpha^{t+1}}{\alpha^t} - 1\right) \cd \sum_{i=t+1}^T  \mu^i  \right) V^t.
\end{align}

We can thus write down a linear programming relaxation for the maximum seller revenue as follows:

\begin{tcolorbox}[title=Linear program for maximum revenue with variables $\bp{\fr{V^t}{\alpha^t}}_{t\in [T]}$]
    \begin{align}
    \max & \sum_{t\in [T]} \left(\mu^t -  \left(\fr{\alpha^{t+1}}{\alpha^t} - 1\right) \cd \sum_{i=t+1}^T  \mu^i  \right) \alpha^t \cd \fr{V^t}{\alpha^t}  \label{relaxation}\\
    &  \frac{V^t}{\alpha^t} \geq \frac{V^{t-1}}{\alpha^{t-1}}, &&  \fl t  \notag\\
    & 0\le \fr{V^t}{\alpha^t} \le M, && \fl t \notag
\end{align}
\end{tcolorbox}

The reason why \cref{relaxation} is a relaxation is that not all vectors $\bp{\fr{V^t}{\alpha^t}}_{t\in [T]}$ are achievable by a seller menu. Nevertheless, the optimal value of \cref{relaxation} serves as an upper bound for the maximum seller revenue. To finish, we prove that the optimal value is actually achievable by an indirect menu consisting of a single contract.

\paragraph{Optimal value of linear program is achievable by a single contract.}

The optimal value of a linear program is achieved at an extreme point of the feasible region. The extreme points \cref{relaxation} are of the following form: there exists $t^*\in [T]$ such that $\fr{V^t}{\alpha^t} = 0$ for all $t < t^*$ and $\fr{V^t}{\alpha^t} = M$ for all $t \ge t^*$. We claim that these buyer values $\bp{V^t}_{t\in [T]}$ can be achieved by offering the single contract $$\cC^{t^*} = \bp{\argmax_{a\in A} \sum_q p^a_q \cd v(q), M \alpha^{t^*}, \mathbf{0}}$$ to all buyer types. Indeed, buyer types $t$ with $t<t^*$ will opt-out of the mechanism and buyer types with $t \ge t^*$ will choose $\cC^{t^*}$. To see that this contract can be computed in polynomial time, we can iterate over all $t^*\in [T]$, compute the revenue of the single-contract menu $\cM^{t^*} = \bc{\cC^{t^*}}$, and output the menu $\cM^{t^*}$ that yields the highest revenue. This completes the proof of \cref{single-parameter-revenue}.

%  These buyer values $\bp{V^t}_{t\in [T]}$ can be achieved in a direct menu by offering the trivial contract $\cC^0 = (a^0, 0, \mathbf{0})$ to all buyer types $t < t^*$ and the contract $$\cC^{t^*} = \bp{\argmax_{a\in A} \sum_q p^a_q \cd v(q), M \alpha^{t^*}, \mathbf{0}}$$ to all buyer types $t \ge t^*$. Since $w^t \le V^t,\fl t$ the IR constraints hold, and since types $t<t^*$ weakly prefer $\cC^0$ to $\cC^{t^*}$ and types $t \ge t^*$ weakly prefer $\cC^{t^*}$ to $\cC^0$, the IC constraints hold. This direct menu can be converted into an indirect menu that consists of the single contract $\cC^{t^*}$ and has equal seller revenue. This finishes the proof of \cref{single-parameter-revenue}.\kicomment{Why do we need to talk about trivial contracts etc here. Can we just a single contract is offered, only types $t \geq t^*$ accept that contract, whereas others choose not to participate.} 



% Any optimal solution to the above LP has the form where there exists a $t^* \in [N]$ with $V^t = \alpha^t K_{\max}$ for all $t \geq t^*$ and $V^t = 0$ for all $t < t^*$. It is straightforward to verify that the single contract with $x_j =0$ for all $j$ and training payment $w = K_{\max} \alpha^{t^*}$ attains the same objective value and hence is optimal.


% We now derive an explicit expression for $w^t$ in terms of the relative ordering of the values $\bc{\fr{V^1}{\alpha^1},\lds,\fr{V^N}{\alpha^N}}$. Let $\pi$ denote an ordering of the types $t\in [N]$ so that type $t$ has the $\pi(t)$-th lowest value of $\fr{V^t}{\alpha^t}$ among the ratios $\bc{\fr{V^1}{\alpha^1},\lds,\fr{V^N}{\alpha^N}}$.

% \begin{proposition} \label{explicit-r}
%     For every $t>1$ we have
%     \begin{align*}
%         w^{t} &= \alpha^{t} \cd \fr{V^{t}}{\alpha^{t}} + (\alpha^{t_1} - \alpha^{t}) \cd \fr{V^{t_1}}{\alpha^{t_1}} + (\alpha^{t_2} - \alpha^{t_1}) \cd \fr{V^{t_2}}{\alpha^{t_2}} + \cds + (\alpha^{t_k} - \alpha^{t_{k-1}}) \cd \fr{V^{t_k}}{\alpha^{t_k}}
%     \end{align*}
%     where $t_i = \argmax_{t < t_{i-1}} \bc{\fr{V^{t}}{\alpha^{t}}},\fl i\ge 1$, $t_0 = t$, and $t_k = 1$. \kicomment{I think the boundary condition here should instead be $t_1 = N$?}
% \end{proposition}

% \begin{proof}
%     We use strong induction with trivial base case $t = 1$. Assume that the claim holds for all types up to $t -1$. By \cref{r-in-terms-of-v}, we have $$w^{t} = V^{t} + \min_{t'<t}\bc{w^{t'} - \fr{\alpha^{t}}{\alpha^{t}} V^{t}} = \alpha^{t} \cd \fr{V^{t}}{\alpha^{t}} + \min_{t'<t}\bc{w^{t'} - \fr{\alpha^{t}}{\alpha^{t'}} V^{t'}}.$$ Defining $f(t') = w^{t'} - \fr{\alpha^{t}}{\alpha^{t'}} V^{t'}$, we claim that the minimum of $f$ is achieved at $t^* = \argmax_{t' < t}  \bc{\fr{V^{t'}}{\alpha^{t'}}}$, from which the inductive step follows. We split into cases depending on whether $t' > t^*$ or $t' < t^*$. In both cases we write
%     \begin{equation} \label{t-star}
%         f(t^*) = w^{t^*} - \fr{\alpha^t}{\alpha^{t^*}} V^{t^*} = (\alpha^t - \alpha^{t^*}) \alpha^{t^*} \cd \fr{V^{t^*}}{\alpha^{t^*}} + \cds
%     \end{equation}
%     according to \cref{explicit-r}.

%     If $t' > t^*$ then we have $$f(t') = w^{t'} - \fr{\alpha^t}{\alpha^{t'}} V^{t'} = (\alpha^t - \alpha^{t'}) \cd \fr{V^{t'}}{\alpha^{t'}} + (\alpha^{t^*} - \alpha^{t'}) \cd \fr{V^{t^*}}{\alpha^{t^*}} + \cds$$ by the induction hypothesis, where the terms in $\cds$ are the same as those in \cref{t-star}. By definition of $t^*$ we have $\fr{V^{t^*}}{\alpha^{t^*}} \ge \fr{V^{t'}}{\alpha^{t'}}$, and since all coefficients are non-positive we conclude that $f(t^*) \le f(t')$ as desired.

%     If $t' < t^*$ then we have $$f(t') = w^{t'} - \fr{\alpha^t}{\alpha^{t'}} V^{t'} = (\alpha^t - \alpha^{t'}) \cd \fr{V^{t'}}{\alpha^{t'}} + (\alpha^{t''} - \alpha^{t'}) \cd \fr{V^{t''}}{\alpha^{t''}} + \cds$$ by the induction hypothesis where $t'' = \argmax_{t_0 < t'} \bc{\fr{V^{t_0}}{\alpha^{t_0}}}$. By definition of $t''$, $\fr{V^{t''}}{\alpha^{t''}}$ will appear as one of the terms in the $\cds$ in \cref{t-star}, after which all terms in the two $\cds$ will be identical. By the inductive hypothesis, all terms of the form $\fr{V^{t_0}}{\alpha^{t_0}}$ in the $\cds$ in \cref{t-star} that appear before $t''$, including $t_0 = t^*$, all satisfy $\fr{V^{t_0}}{\alpha^{t_0}} \ge \fr{V^{t''}}{\alpha^{t''}}$, otherwise the index $t_0$ would have been skipped in favor of $t''$. By the same reasoning, we also have $\fr{V^{t_0}}{\alpha^{t_0}} \ge \fr{V^{t'}}{\alpha^{t'}}$ for all such $t_0$. We conclude that since the coefficients in front of the terms are all non-positive and have equal sum that $f(t^*) \le f(t')$ as desired.
% \end{proof}

% The key takeaway from \cref{explicit-r} is that fixing the relative ordering of $\bc{\fr{V^1}{\alpha^1},\lds,\fr{V^N}{\alpha^N}}$, the expressions for $w^t$ can be written down explicitly without the need for the minimum function. We now complete the proof of \cref{single-parameter-revenue}. The total service provider profit is $\mu^1 w^1 + \mu^2 w^2 + \cds + \mu^N w^N$, where $\mu^t$ is the weight of type $t$. Note that since $w^t$ can be written a linear combination of $\bp{V^t}_{t\in N}$, the total service provider profit is linear in the $V^t$, which will be used to argue that the service provider profit is maximized at boundary values of $V^t$.

% \begin{align*}
%     \mu^1 (w^1 - c_1) + \mu^2 (w^2 - c_2) + \cds + \mu^N (w^N - c_N)
%     % &= \sum_{t=1}^N V^t \cd \bp{\mu^t
%     % - {\sum_{k=t+1}^N \mu^k} \cd \fr{\alpha^{t+1} - \alpha^t}{\alpha^t}} - \sum_{t=1}^N \mu^t c_t,
% \end{align*}
% % so we can increase $w^t$ to $$V^t - \fr{\alpha^t - \alpha^{t-1}}{\alpha^{t-1}} V^{t-1} - \cds - \fr{\alpha^2 - \alpha^1}{\alpha^1} V^1,$$ where the general formula for $w^t$ follows by induction on $t$. 

% Let $t^*$ denote the minimum type for which $V^{t^*} > 0$. The IC constraints $$V^t - w^t \ge \fr{\alpha^t}{\alpha^{t^*}} V^{t^*} - w^{t^*},\quad \fl t > t^*$$ yield upper bounds $$w^t \le w^{t^*} + V^t - \fr{\alpha^t}{\alpha^{t^*}} V^{t^*}$$ on revenue from all types $t > t^*$. Letting $v^{t^*} = \sum_j p^i_j v^{t^*}_j$ denote the maximum possible value of contract $t^*$

% View the service provider's problem $\max \mu^1 w^1 + \mu^2 w^2 + \cds + \mu^N w^N$ as an optimization problem with variables $V^1,\lds,V^N$, noting that $w^1,\lds,w^N$ are completely determined by the $V^t$ according to \cref{r-in-terms-of-v}. The range for each $V^t$ is $V^t\in [0, \alpha^t \cd V]$ for some constant value $V$, which represents the maximum value of a contract for type $t$ achievable when type $t$ purchases all qualities and the service provider chooses the effort level that maximizes value for the agent. The fact that $V$ is a constant comes from the single-parameter valuations assumption. Note that not all values in the range $[0, \alpha^t \cd V]$ are actually achievable since the set of qualities that type $t$ purchases is discrete, but we can certainly relax the problem to this continuous range and show that the optimal $V^t$ is actually achievable.

% Fixing the relative ordering of $\bc{\fr{V^1}{\alpha^1},\lds,\fr{V^N}{\alpha^N}}$, which can be viewed as multiple degree-1 homogeneous constraints in the $V^i$, the service provider's revenue is an explicit linear function in the $V^i$. Finally we have the constraints $w^t\ge 0,\fl t$, which are also degree-1 homogeneous constraints. All in all, we see that conditioned on a particular ordering, the service provider's optimization problem is a linear optimization problem in the domain $V^t\in [0, \alpha^t \cd V],\fl t$ and with homogeneous degree-1 constraints in the $V^t$. By linear programming, the optimal value is achievable at a vertex of the feasible polytope. However, because all constraints are homogenous degree-1, we see that by scaling, any vertex must satisfy $V^t = 0,\fl t$, in which case the revenue is 0, $V^t = \alpha^t \cd V$, namely, at least one of the domain upper bounds is tight. This argument can be repeated because a tight domain upper bounds for type $t$ implies that $\fr{V^t}{\alpha^t} = V$ is the maximum in the relative ordering of $\bc{\fr{V^1}{\alpha^1},\lds,\fr{V^N}{\alpha^N}}$. At every step, if we take the set of types $t$ for which the domain upper bound is not tight, that is, $\bc{t: \fr{V^t}{\alpha^t} < V}$, and scale them by the same multiplicative factor, the relative ordering of $\bc{\fr{V^1}{\alpha^1},\lds,\fr{V^N}{\alpha^N}}$ remains unchanged, so we stay within the feasible polytope. Either all $t$ in this set satisfy $V^t = 0$, or we can scale so that another type meets the domain upper bound. We conclude that any vertex of the feasible polytope for any relative ordering satisfies $\fr{V^t}{\alpha^t}\in \bc{0, V},\fl t\in [N]$, which means that the unconstrained optimal value of the service provider's optimization problem can be achieved by $(V^t)_{t\in [N]}$ satisfying $\fr{V^t}{\alpha^t}\in \bc{0, V},\fl t\in [N]$. In such an optimal solution, let $t^*\in [N]$ be the minimum type for which $V^{t^*} = \alpha^{t^*} \cd V$. Then by \cref{r-in-terms-of-v}, we see that any $t > t^*$ satisfies $$w^t \le V^t + w^{t^*} - \fr{\alpha^t}{\alpha^{t^*}} V^{t^*} = V^t - \alpha^t \cd V + w^{t^*}\le w^{t^*} \le V^{t^*}.$$ Also, any $t < t^*$ satisfies $w^t \le V^t = 0$. Hence the total service provider's revenue is bounded above by
% \begin{equation} \label{single-parameter-revenue-upper-bound}
%     V^{t^*} \cd \sum_{k = t^*}^N \mu^k = \alpha^{t^*} \cd V \cd \sum_{k=t^*}^N \mu^k
% \end{equation}

% The final step is to relate the revenue expression above to the revenue of a single contract. In particular, we claim that all revenues of the form $V^{t} \cd \sum_{k = t}^N \mu^k$ are achievable by a single contract. Note that for a single contract with upfront price $w$, usage prices in $\bc{0,\infty}$ and maximum value $\alpha^t \cd V$ for type $t$, there is a threshold type $t$ such that all types $t'\ge t$ derive nonnegative utility and thus generate revenue $\mu^{t'} \cd w$ and all types $t' < t$ derive negative utility and thus generate no revenue. Hence it is optimal to set $w$ to be the value $\alpha^t \cd V$ of the contract for some type $t$, in which case the revenue generated is $\alpha^t \cd V \cd \sum_{k=t}^N \mu^k$, which matches \cref{single-parameter-revenue-upper-bound} as desired. This concludes the proof of \cref{single-parameter-revenue}.

% \begin{remark}
%     We end with some remarks on the structure of the optimal single contract in the single-parameter setting. The service provider chooses the effort level that provides the greatest value for the types. Due to the single-parameter assumption, this effort level is the same for every type. The training payment should be set to the value of the contract for type $$t^* = \argmax_{t\in [N]} \alpha^t \cd \sum_{k=t}^N \mu^k,$$ which can be accomplished by a linear search over all the types, and there are no usage payments.
% \end{remark}



% Future work

% Two groups, within each group single parameter, is two contracts optimal? (Test on 3 types with 2 valuations being multiples of each other.)

% Example where if you enforce purchases, the revenue is smaller than if you do not force them to buy (and vice versa)

% In Section 1, we need theorem that states that enforced payment optimal = w^training, from which it follows that voluntary usage is better.

        