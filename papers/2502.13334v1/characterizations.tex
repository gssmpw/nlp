\section{Characterizing profit-maximizing menus} \label{characterizations}

The feasible space of seller menus in \cref{profit-maximization-direct-menu} is huge because there is a usage price for each type and outcome. The goal of this section is to simplify the search space by proving structural properties that the profit-maximizing menu without loss of generality satisfies. These characterizations serve as preliminaries to the results in \cref{complexity} and \cref{single-parameter}.

\paragraph{Two usage prices suffice}

% For the lottery pricing problem, we have shown that incorporating usage prices does not generate any extra profit and thus usage prices are not needed for this problem. While \cref{upfront-price-only} shows that usage prices are required to achieve maximum profit in the service provider problem, i
To reduce the large search space over usage prices, it is natural to ask whether some tuples of usage prices are always better than others. Our main result in this section is a general reduction between IC menus that allows the seller to consider only usage prices that are either 0 or $\infty$.

\begin{theorem} \label{two-usage-prices-suffice}
    Any IC menu can be modified so that all usage prices satisfy $x^t_q\in \bc{0, \infty}$ while leaving the seller's profit and the buyers' utilities unchanged.
\end{theorem}

\begin{proof}
    We provide a sketch, deferring the formal proof to \cref{two-usage-prices-suffice-proof}. Define $S^t \coloneq \bc{q: v^t_q \ge x^t_q}$ to be the set of outcomes that type $t$ accepts. Our idea is to \emph{redistribute} all usage prices for outcomes in $S^t$ into the upfront price. For all $q\in S^t$, we set $x^t_q = 0$ and increase the upfront price $w^t$ by $p^{a^t}_q x^t_q$, calling the new contract $\cC'^t$. In this way, the usage price decreases by exactly the same amount that the upfront price increases and hence $U(t;\cC^{t}) = U(t;\cC^t)$. For all other types $u$, we have $U(u;\cC'^t) \le U(u;\cC^t)$ since $u$ now pays an additional upfront price of $p^{a^t}_q x^t_q$ while their utility from the decrease in usage price $x^t_q$ increases by at most $p^{a^t}_q x^t_q$. Hence $u$ will still choose $\cC^u$ so the menu remains IC. After applying the price redistribution for all contracts, the end result is a menu whose usage prices $x^t_q$ satisfy $x^t_q = 0$ or $x^t_q > v^t_q$. In the latter case we can simply increase the usage prices to $\infty$.
\end{proof}

Note that the $\infty$ price in \cref{two-usage-prices-suffice} is only for notational convenience and the proof similarly works when $\infty$ is replaced by any price that excludes all types from accepting the outcome, for example the maximum valuation $\max_{t,q} v^t_q$ of any type for any outcome.

\begin{remark*}
\cref{two-usage-prices-suffice} has the interpretation that without loss of generality we can view the seller's revenue as coming entirely from the upfront price. The $\infty$ price for an outcome is merely a way to exclude buyers from using that outcome. We emphasize that even though no revenue is collected from usage prices in $\bc{0,\infty}$, having $\infty$ usage prices is still necessary by \cref{upfront-payment-only}.
\end{remark*}

To summarize, \cref{two-usage-prices-suffice} shows that any profit-maximizing mechanism can, though is not required to, have a format where the seller collects upfront prices for various actions but limits the buyer's access to certain outcomes. Such pricing mechanisms are similar in spirit to many software subscriptions like ChatGPT for which there are various subscription tiers, each with access to models with differing capabilities.


For the rest of the paper we assume that all usage prices are in $\bc{0,\infty}$, so the revenue from type $t$ is simply the upfront price $w^t$. The service provider problem, \cref{profit-maximization-direct-menu}, simplifies to the following:

% \begin{figure}
\begin{tcolorbox}[title=Maximizing profit of a direct menu with usage prices in $\bc{0,\infty}$]
    \begin{align}
             \max_{ \substack{\{(a^{t'}, w^{t'}, \mathbf{x}^{t'}) \}_{{t'} \in [T]} \\ x^{t'}_q\in \bc{0,\infty}}} & \quad \bE_{t \sim \mu} \left[ w^t  - c(a^t)
    \right] \label{binary-optimization-problem} \\ 
        & U(t; \cC^t) \geq U(t; \cC^{t'}) && \fl t, t' \in [T]   \notag \\ 
        & U(t; \cC^t) \geq 0  && \fl t  \in [T]  \notag
          \end{align}   
\end{tcolorbox}
% \end{figure}

% \para{A mixed-integer linear program for computing a profit-maximizing menu.}

% Lemma \ref{two-usage-prices-suffice} implies a mixed-integer linear program (MILP) for computing a profit-maximizing menu. We refer the reader to \cref{MILP} for the MILP formulation.

\para{Zero usage prices for the types with highest upfront payment}

We show that in any profit-maximizing menu it is not necessary to set $\infty$ usage prices for the type or types that pay the most to the seller.

\begin{definition*}[Highest-revenue types] \label{highest-revenue-definition}
    In the context of menus with usage prices in $\bc{0,\infty}$, a \emph{highest-revenue} type $t$ is a type with the largest upfront price, namely $\argmax_{t} w^t$. Since there can be ties, $\argmax_{t} w^t$ in general is a set consisting of all highest-revenue types.
\end{definition*}

\begin{proposition} \label{highest-type-no-usage-prices}
    Any IC menu with usage payments in $\bc{0,\infty}$ can be modified so that the contracts for the highest-revenue types have zero usage prices while remaining IC and weakly increasing seller profit.
\end{proposition}

\begin{proof}
    Let $t$ be a highest-revenue type, meaning $w^t\ge w^u,\fl u$. For each highest-revenue type $t$, we replace type $t$'s contract $\cC^t = \bp{a^t, w^t, \mathbf{x}^t}$ with $\cC'^t = \bp{a^t, w^t, \mathbf{0}}$. Since usage prices decrease from $\cC^t$ to $\cC'^t$, we have $U\bp{t; \cC'^t} \ge U\bp{t; \cC^t}$, so type $t$ will choose $\cC'^t$ over any other contract $\cC^u$ and thus yield the same seller revenue $w^t$. Any other type $u\neq t$ will choose either $\cC^u$ or $\cC'^t$ in the modified menu, yielding seller revenue $w^u$ or $w^t \ge w^u$, respectively. In the latter case we replace $\cC^u$ by $\cC'^t$ in the modified menu to maintain IC, noting that seller profit weakly increases.
\end{proof}

\begin{remark*}
     \cref{highest-type-no-usage-prices}, which shows that the highest-revenue types do not require usage payments is similar in flavor to a result in \citet{bergemann2018design} that shows in the context of selling information that in any optimal menu of statistical experiments, the highest buyer types purchase the fully informative experiment. A similar phenomenon of allowing higher buyer types to receive more information is reflected in \citet{liu2021optimal}.
\end{remark*}