\section{Conclusion and open problems}

In this paper we describe a contract-based mechanism for selling a service that produces an outcome of uncertain quality. We believe that our model assumptions such as voluntary usage are closely aligned with how machine learning training providers sell their services. We show that the two-part payment scheme is rich enough to significantly increase seller profit compared to other natural payment structures, but not so complex as to make it intractable to solve for an approximately profit-maximizing menu for a constant number of buyer types. There are a few interesting future directions left open by our research.

\paragraph{Optimizing single-contract menus.} It would be interesting to study the computational complexity of computing a revenue-maximizing or profit-maximizing menu that consists of a single contract in the general service provider problem. This problem falls into a classic area of contract design that explores the trade-off between \emph{optimal} contracts and \emph{simple} contracts \citep{dutting2019simple, guruganesh2023menus, dutting2024algorithmic}. In standard principal-agent problems, computing the optimal contract is known to be \tbf{NP}-hard \citep{guruganesh2021contracts, castiglioni2022bayesian}.
In the context of our service provider problem, a natural definition of a simple menu is one that consists of a small number of distinct contracts. Empirical studies have shown that customers may be negatively biased when presented with large sets of options, leading to reduced purchase behavior \citep{thaler2015misbehaving}. \citet{bernasconi2024agent} show that in their hidden action model, computing a profit-maximizing menu consisting of a constant $k$ number of contracts can be done in polynomial time. In our model the complexity of computing a profit-maximizing menu consisting of a single contract is an open question, although we note that our dynamic program framework in \cref{fptas-section} provides an FPTAS to \emph{approximate} the maximum seller profit achievable using a menu with $k$ contracts when both $k$ and the number of buyer types $T$ is constant. Motivated by our result that single-contract menus are revenue-optimal in the single-parameter setting (\cref{single-parameter-revenue}), we can ask in what other settings is a single contract (approximately) revenue- or profit-optimal?

%  On the other hand, in our model it is an open problem to efficiently compute an exact profit-maximizing menu even consisting of a single contract, and this problem is open even in the single-parameter setting.
% \footnote{Note that \cref{single-parameter-revenue} shows that we can efficiently compute a \emph{revenue}-maximizing single-contract menu in the single-parameter setting.}

% We remark that if we are satisfied with \emph{approximating} the maximum profit of a single contract, a simplification of our FPTAS framework in \cref{fptas} provides a way to do so. \todo{Describe}

\paragraph{Optimizing menus when the seller cannot commit.}
%  in part because large-scale machine learning platforms possess such commitment power due to government regulation and monitoring\kicomment{seems weak reason, also needs citation if true}, and in part 
This paper assumes that the service provider can commit to actions, in part because automated machine learning platforms are regulated and in part because committing always leads to weakly higher profit for the provider. In contrast, \citet{bernasconi2024agent} adopt a model in which the seller action is \emph{hidden} and the buyer cannot assume that the seller will always perform the action specified by the contract. It would be interesting to see if some of our results, either on the necessity of a two-part payment scheme or on the computational complexity of computing an optimal menu, extend to a setting where the seller cannot commit to performing their actions.