\section{Appendix} \label{appendix}

% \subsection{On the opt-out assumption} \label{opt-out-option}

% We show an example as to why the opt-out assumption is realistic. Consider a problem instance with two types and type distribution $\mu^1 = \mu^2 = \fr12$, a single nontrivial action $a$ with cost $c(a) = \fr23$, a single nontrivial outcome, and valuations $v^1_1 = 1$ and $v^2_1 = 0$. Without the existence of a trivial action, action $a$ must be used for both type 1 and type 2, incurring a cost of $\fr12$ for the service provider no matter the type. Since the revenue from type 1 and type 2 is bounded above by 1 and 0, respectively, the expected service provider profit is $\fr12 \cd 1 + \fr12 \cd 0 - \fr23 = -\fr16$, which is negative. A negative seller profit is clearly unrealistic, as it would have been better for the seller not to interact with the buyer at all.

% In particular, it is unrealistic to force the seller to interact with every buyer type. In this example, interacting with type 2 using the nontrivial action $a$ is suboptimal because $a$ is too costly compared to type 2's valuations. The seller should instead let the menu consist of the single contract $\cC^1 = (1, a, 0)$ for type 1 and allow type 2 the freedom to not choose any contract in the menu. The act of not choosing any contract is equivalent to choosing the trivial contract $\cC^0$.

% The existence of such an opt-out option allows the seller to extract a revenue of $1$ from type 1 while incurring cost $c(a) = \fr23$ for type 1. Type 2 would would prefer the trivial contract $\cC^0$ to $\cC^1$ because the latter has an upfront payment of 1. The contract $\cC^1$ now yields positive seller profit $\fr12 \cd \bp{1 - \fr23} = \fr16$.

% The opt-out assumption implies that the maximum seller profit is nonnegative since they could have offered the trivial contract to every type. In general, the buyer types that the seller assigns $\cC^0$ in a direct menu are those types that the seller wishes to exclude from interacting with because their valuations are too low to justify incurring the cost of any nontrivial action.

\subsection{Worst-case multiplicative gap between $R$ and $\Rupfront$} \label{upfront-payment-only-proof}

We prove \cref{upfront-payment-only}. By \cref{eq:H-mu}, we can assume that $0<\mu^1\le \cds \le \mu^T \le 1$.
\begin{enumerate}
        \item We construct the following problem instance:
        \begin{itemize}
            \item Let $Q = [T]$. There is a single action $a$ with cost $c(a) = 0$ and transition probabilities $p^a_q = \fr{1}{T},\fl q\in Q$.
            \item Valuations are given by $v^t_q = \case{\fr{T}{\sum_{i=1}^t \mu^i} & \teif q = t \\ 0 & \teif q \neq t.}$
        \end{itemize}

        The utility of buyer type $t$ satisfies $$U(t;\cC^t) \le \fr{1}{T} \cd \fr{T}{\sum_{i=1}^t \mu^i} = \fr{1}{\sum_{i=1}^t \mu^i},$$ so $$R \le \sum_{t\in [T]} \mu^t \cd \fr{1}{\sum_{i=1}^t \mu^i}.$$ The upper bound on $R$ can be achieved using an identical contract $$\cC = \bp{a, 0, \bp{\fr{T}{\sum_{i=1}^t \mu^i}}_{t\in [T]}}$$ for all types, so $$R = \sum_{t\in [T]} \fr{\mu^t}{\sum_{i=1}^t \mu^i}.$$
        
        On the other hand, if the seller can only use upfront payments, note that any IC menu with a single action consists of a single contract since the contract with the lowest upfront payment yields the highest utility for all buyers and hence all buyers will choose the contract with the lowest upfront payment. We do casework on the value of the upfront payment $w$ by partitioning the space of possible upfront payments into disjoint ranges. In the range $$w\in \left(\fr{1}{\sum_{i=1}^{t+1} \mu^i}, \fr{1}{\sum_{i=1}^{t} \mu^i}\right],$$ exactly the first $t$ buyer types will choose the contract over the opt-out option. The seller revenue in this range is upper bounded by $$\sum_{i=1}^t \mu^i \cd \fr{1}{\sum_{i=1}^{t} \mu^i} = 1,$$ with equality if $$w = \fr{1}{\sum_{i=1}^{t} \mu^t}$$ for some $t$. We conclude that $$\Rupfront = 1\implies \fr{R}{\Rupfront} = \sum_{t\in [T]} \fr{\mu^t}{\sum_{i=1}^t \mu^i}.$$

        \item Suppose that the contract $\cC^t = (a^t, w^t, \mathbf{x}^t)$ yields profit $$r^t := w^t - c(a^t) + \sum_{q\in Q} p^{a^t}_q x^t_q \cdot \one\bb{v^t_q \ge x^t_q}  $$ from type $t$ and assume without loss of generality that $r^1 \ge r^2 \ge \cds \ge r^T$. 
        
        \begin{claim*}
            For every $t\in [T]$ we can construct a modified menu with only upfront prices that achieves a profit of at least $$\bp{\sum_{i=1}^t \mu^i} \cd r^t.$$ 
        \end{claim*}
        \begin{proof}
            To construct such a menu, replace every contract $\cC^u = (a^u, w^u, \mathbf{x}^u)$ in the original menu, including $\cC^t$, with $\cC'^u = \bp{a^u, r^t + c(a^u), \mathbf{0}}$. For any type $u\le t$, note that the revenue from type $u$ is equal to $r^u + c(a^u)$. The IR constraint for type $u$ in the original menu says that type $u$'s value for the outcomes induced by $a^u$ is at least the revenue $r^u + c(a^u)$ from type $u$. Combined with $r^u \ge r^t$, this implies $$U(u; \cC'^u) \ge r^u + c(a^u) - \bp{r^t + c(a^u)} \ge 0,$$
        so the modified menu is IR. Each contract $\cC'^u$ yields profit at least $r^t + c(a^u) - c(a^u) = r^t$ for the seller, thus proving the claim.
        \end{proof}

        By the claim, we have $$\Rupfront \ge \bp{\sum_{i=1}^t \mu^i} \cd r^t,\quad \fl t \iff \mu^t \cd r^t \le \Rupfront \cd \fr{\mu^t}{\sum_{i=1}^t \mu^i},\quad \fl t.$$ Summing over $t\in [T]$ yields $$R \le \Rupfront \cd \sum_{t\in [T]} \fr{\mu^t}{\sum_{i=1}^t \mu^i}.$$
    \end{enumerate}

\subsection{Analysis of $H_\mu$} \label{H_mu}

For any buyer type distribution $0<\mu^1\le \cds \le \mu^T \le 1$ satisfying $\sum_{t\in [T]} \mu^t = 1$, we prove that $$H_\mu = \sum_{t\in [T]} \fr{\mu^t}{\sum_{i=1}^t \mu^i}$$ defined in \cref{eq:H-mu} satisfies $H_\mu \in [H_T, T)$ where $H_T = \sum_{t\in [T]} \fr{1}{t}$ is the $T$-th harmonic number. We also show that the lower and upper bounds are both tight.

    We first show that the lower bound is achievable and the upper bound is achievable in the limit. The lower bound is achieved by setting $\mu^1 = \cdots = \mu^T = \frac{1}{T}$. The upper bound can be achieved in the limit by setting $\mu^t = \eps^{T-t} - \eps^{T-t+ 1}$ where $\eps \to 0$. Note that each term $$\frac{\mu^t}{\sum_{i = 1}^t \mu^i} = \frac{ \eps^{T-t} - \eps^{T-t + 1} }{\eps^{T-t} - \eps^T} = \frac{1 - \eps}{1 - \eps^{t}}$$ approaches $1$ for any fixed $t$ as $\eps\to 0$.

    We now show $H_\mu\in [H_T, T)$. The upper bound $T$ is trivial since each term is at most $1$. We prove the lower bound by an induction argument. To begin, we claim that when $H_\mu$ achieves its minimum value, we must have $\mu^T = \mu^{T-1}$. Define $$H_{\mu'} = \sum_{t\in [T-1]} \fr{\mu^t}{\sum_{i=1}^t \mu^i},$$ so that $$H_\mu = H_{\mu'} + \frac{\mu^T}{\sum_{i=1}^T \mu^i},$$ and note that $H_{\mu'}$ does not depend on $\mu^T$. The key observation is that is possible to minimize the value of the term $ \frac{\mu^T}{\sum_{i=1}^T \mu^i}$ without changing the value of $H_{\mu'}$ despite the constraints on $\mu$. This is because scaling the values of $\mu^1,\lds,\mu^{T-1}$ by the same multiplicative factor does not change the value of $H_{\mu'}$. Hence to minimize $H_\mu$, we can first minimize the term $\frac{\mu^T}{\sum_{i=1}^T \mu^i}$ by choosing $\mu^T$ subject to $\mu^T\ge \mu^{T-1}$ while preserving the value of the $H_{\mu'}$ term so long as we scale $\mu^1,\lds,\mu^{T-1}$ to guarantee $\sum_{t\in [T]} \mu^t = 1$. It is easy to see that $\frac{\mu^T}{\sum_{i=1}^T \mu^i}$ is increasing in $\mu^T$ and hence is minimized at $\mu^T = \mu^{T-1}$. Continuing inductively in this manner, we conclude that $H_\mu$ is minimized at $\mu^1 = \cds = \mu^T$, in which case $H_\mu = H_T$.

\subsection{Usage payments are redundant in lottery pricing} \label{proof-usage-payment-lottery-pricing}
 
We prove \cref{usage-payment-lottery-pricing} in both the mandatory usage and voluntary usage models. Because there is an action for each outcome distribution $\mathbf{p}$ in the lottery problem, specifying the action in a contract is equivalent to specifying the outcome distribution. We show that usage prices can be \emph{redistributed} into the lottery price. Let the menu consist of lotteries $(w^t, \mathbf{p}^t, \mathbf{x}^t)$ where $w^t$ is the lottery price, $p^t_q$ is the probability of receiving item $q$, and $x^t_q$ is the price of item $q$.

\paragraph{Mandatory usage.} We first consider mandatory usage in which the buyer, after being presented with item $q$ with probability $p^t_q$, is required to accept it at price $x^t_q$. Buyer type $t$ solves the optimization problem $$\max_{t\in [T]} -w^t+ \sum_{q\in Q} p^t_q \bp{v^t_q - x^t_q}$$ and the seller revenue from type $t$ is $$w^t + \sum_{q\in Q} p^t_q x^t_q.$$ We show how to construct an equivalent menu that does not have usage payments. By setting $x^t_q = 0,\fl q$ and increasing the lottery payment $w^t$ by $\sum_{q\in Q} p^t_q x^t_q$, we observe buyer utility for each contract does not change, hence type $t$ still chooses the same lottery and the seller revenue from type $t$ is unchanged as well.

\paragraph{Voluntary usage.} This case is more complicated. In voluntary usage, the buyer, after being presented with item $q$ with probability $p^t_q$, can decide whether or not to accept the item at price $x^t_q$. Buyer type $t$ solves the optimization problem $$\max_{t\in [T]} -w^t + \sum_{q\in Q} p^t_q \cdot \max\bc{0, v^t_q - x^t_q},$$ and the seller revenue from type $t$ is $$w^t + \sum_{q\in Q} p^t_q x^t_q \cdot \one\bb{v^t_q \ge x^t_q}.$$ We show how to construct an equivalent menu that does not have usage payments. Recall that by the opt-out assumption, there is a trivial item, which we label 0, such that $v^t_0,\fl t\in [T]$. For each $t\in [T]$ replace the lottery $(\mathbf{p}^t, w^t, \mathbf{x}^t)$ with $(\mathbf{p'}^t, {w'}^t, \mathbf{0})$ where
\begin{align*}
        {w'}^t &= w^t + \sum_{q: v^t_q \ge x^t_q} p^t_q x^t_q \\
        {p'}^t_q &= \begin{cases} p^t_q & \teif v^t_q \ge x^t_q \\ 0 & \teif 0 < v^t_q < x^t_q \\ 1 - \sum_{q: v^t_q \ge x^t_q} p^t_q & \teif q = \es.\end{cases}
\end{align*}
In other words, for all items where the usage price is higher than type $t$'s valuation and therefore does not factor into the revenue from type $t$, we instead map the probability mass to the trivial item. Under this mapping, type $t$'s utility in the modified lottery is the same as in the original one. Furthermore, the modified lottery is weakly worse for all other types because probability mass has been moved to the trivial item, which has zero value for all types. Hence the modified menu remains IC and yields the same buyer utilities and seller revenue.

\subsection{Upfront payments are necessary} \label{proof-usage-payment-only}

We prove \cref{usage-payment-only}. To show that $\fr{R}{\Rusage} \ge \fr32$, we construct the following problem instance:
    \begin{itemize}
        \item Let $T = 2$, $\mu^1 = \mu^2 = \fr12$, and $Q = \bc{1, 2}$. There is a single action $a$ with cost $c(a) = 0$ and transition probabilities $p^a_q = \fr{1}{2},\fl q\in [2]$.
        \item Valuations are given by $\mathbf{v}^1 = \bp{1,\fr12}$ and $\mathbf{v}^2 = \bp{\fr12, 1}$.
    \end{itemize}

    Note that $V(t;a) = \fr34$, so $R \le \fr34$. The upper bound on $R$ can be achieved using the contract $\cC = \bp{a, \fr34, \mathbf{0}}$ for all types. On the other hand, we show that $\Rusage\le \fr12$. Assume for contradiction that $\Rusage > \fr12$. Then the revenue from some type, without loss of generality type 1, is greater than $\fr12$, implying $x^1_1 + x^1_2 > 1$. This means that type 1 pays greater than $\fr12$ for outcome 1, $\fr12 < x^1_1 \le 1$, and it also means that $0 < x^1_2 \le \fr12$. Note that $$\fr12 \bp{\fr32 - x^1_1 - x^1_2} = U(1;\cC^1) \ge U(1;\cC^2) \ge \fr12 \bp{1 - x^2_1},$$ which combined with $x^1_1 + x^1_2 > 1$ yields $x^2_1 > \fr12$, namely type 2 does not use outcome 1. Then we have $$\fr12 \bp{1 - x^2_2} = U(2;\cC^2) \ge U(2;\cC^1) = \fr12 \bp{1 - x^1_2},$$ which implies $x^2_2 \le x^1_2 \le \fr12$. This yields $$\Rusage \le \fr14\bp{x^1_1 + x^1_2 + 0 + \fr12} \le \fr12,$$ contradiction. We conclude that $\Rusage \le \fr12$.
    
    % Now consider $U(2;\cC^1)$. Type 2 receives zero utility from outcome 1 in $\cC^1$ since $x^1_1 > \fr12 = v^2_1$ and $p^a_2 \cd (1 - x^1_2) = \fr12 (1 - x^1_2)$ utility 

% Unlike in \cref{training-payment-only-gap}, the gap in \cref{usage-payment-only-gap} does not approach $\infty$ as $t\to \infty$. We show that we can obtain a much larger gap if we consider a restricted version of the service provider problem in which they are only allowed to use a single contract for all types. We note that restricting a menu to contain only a single contract is well-studied in the contract design literature. For simplicity we assume in the next result that the type distribution is uniform, namely $\mu^1 = \mu^2 = \cds = \mu^T = \fr{1}{T}$, noting that the gap can also be computed in the non-uniform case.

% % The reason we make the uniform type distribution assumption is that in the usage payments only setting, proving a gap for arbitrary $\mu$ requires more computation but is not more conceptually illuminating. 

% \begin{proposition} \label{usage-gap-single-contract}
%     Let $\mu^1 = \mu^2 = \cds = \mu^T = \fr{1}{T}$ be the agent type distribution. Let $R$ denote the highest service provider profit achievable using both training and usage payments \emph{in a single contract} and let $\Rusage$ denote the highest service provider profit achievable using only usage payment \emph{in a single contract}. Then there exists a problem instance for which $\fr{R}{\Rusage} \ge H_T$.
% \end{proposition}

% \begin{proof}
%     \begin{enumerate}
%         \item We construct the following problem instance:
%         \begin{itemize}
%             \item Let $Q = [T]$. There is a single action $a$ with cost $c(a) = 0$ and transition probabilities $p^a_q = \fr{1}{T},\fl q\in Q$.
%             \item Valuations are given by $v^t_q = \case{\fr{1}{1 + (t + q)\bmod T}& \teif q = t \\ 0 & \teif q \neq t.}$ In other words, the valuations for the $T$ types comprise all cyclic shifts of the sequence $\fr{1}{1}, \fr{1}{2}, \cds \fr{1}{T}$.
%         \end{itemize}

%         Note that $V(t;a) = \fr{H_T}{T},\fl t\in [T]$ so $R \le \fr{H_T}{T}$. The upper bound on $R$ can be achieved using the contract $$\cC = \bp{a, \fr{H_T}{T}, \mathbf{0}}$$ for all types, so $R = \fr{H_T}{T}$. 

%         We show that any single contract $(a, 0, \mathbf{x})$ consisting of only usage payments has expected revenue at most $\fr{1}{T}$. Setting a usage payment in the range $$x_q \in \left( \fr{1}{t+1}, \fr{1}{t}\right]$$ yields $t$ types purchasing quality $q$, for a revenue of at most $$\fr{1}{T} \cd \fr{t}{T} \cd \fr{1}{t} = \fr{1}{T^2}$$ from quality $q$. Summing over all $\ab{Q} = T$ qualities yields a total revenue of at most $\fr{1}{T}$.

%         % On the other hand, if we are restricted to using usage payments, we claim that for every constant $c \ge 0$, there exists sufficiently large $T$ for which $\Rusage \le \fr{R}{c}$.
        
%         % The next step is to show that any menu of contracts can be reduced to the single contract case. Consider any quality $q$ and let the minimum usage payment $\min_{t\in [T]} x^t_q$ for quality $q$ lie in the interval $\left( \fr{1}{u+1}, \fr{1}{u} \right]$.
%     \end{enumerate}
% \end{proof}

% \begin{remark}
%     We conjecture that the problem instance described in \cref{usage-gap-single-contract} actually yields a profit gap of $H_T$ for the uniform type distribution even if we remove the single contract restriction. For example, \cref{usage-payment-only-gap} proves this conjecture for $T=2$. However, the analysis for $T>2$ seems quite tricky so we are unable to prove the general conjecture.
% \end{remark}


\subsection{Proof that two usage prices suffice} \label{two-usage-prices-suffice-proof}

We provide a formal proof of \cref{two-usage-prices-suffice}. Consider a IC menu of the form $\cC^t = (a^t, w^t, \mathbf{x}^t)$. The utility of contract $\cC^t$ for type $u$ is
    \begin{align*}
        U(u;\cC^t) &:= -w^t + \sum_{q} p^{a^t}_q \cd \max\bc{0, v^u_q - x^t_q}
        \\&= -w^t + \sum_{q: v^u_q \ge x^t_q} p^{a^t}_q \bp{v^u_q - x^t_q}.
    \end{align*}
    The revenue from type $t$ is
    \[w^t + \sum_{q\in S^t} p^{a^t}_q x^t_q,\] recalling that $S^t := \bc{q: v^t_q \ge x^t_q}$ is the set of outcomes that type $t$ accepts.
    Consider replacing $\cC^t$ with the contract
    \begin{align*}
        \cC'^t &= \bp{a'^t, {w'}^t, \mathbf{x'}^t} \\
        a'^t &= a^t \\
        w'^t &= w^t + \sum_{q\in S^t} p^{a^t}_q x^t_q \\
        \mathbf{x'}^t_q &= \case{0 & q\in S^t \\ \infty & \teoth.}
    \end{align*}
    We compute
    \begin{align*}
        U(u;\cC'^t) &= -w'^t + \sum_{q\in S^t} p^{a^t}_q v^u_q
        \\& = -w^t + \sum_{q\in S^t} p^{a^t}_q \bp{v^u_q - x^t_q}.
    \end{align*}
   Note that $U(t; \cC'^t) = U(t; \cC^t)$, so type $t$ prefers $\cC'^t$ to any other contract $\cC^u$ by IC of the original menu. The profit from $\cC'^t$ is exactly the profit from $\cC^t$ since the upfront price is increased by exactly the amount that the type $t$ usage prices are decreased weighted by the outcome probabilities. The profit from all other types is unchanged because only $\cC^t$ is modified to get from the original menu to the modified menu.
   
   We finish by showing IC of the modified menu using IC of the original menu. It suffices to show that $U(u;\cC'^t) \le U(u; \cC^t)$ for all types $u\neq t$, which implies that type $u$ chooses $\cC^u$ in the modified menu. We compare like terms in the sums in our formulas for $U(u;\cC^t)$and $U(u;\cC^{t'})$ above and split into two cases. If $v^u_q < x^t_q$ then the term for outcome $q$ in $U(u;\cC'^t)$ is negative while the term does not exist in $U(u;\cC^t)$. If $v^u_q \ge x^t_q$, the term corresponding to outcome $q$ in $U(u;\cC'^t)$ is either equal to the term in $U(u;\cC^t)$, which is nonnegative, if $q\in S^t$, or does not exist if $q\notin S^t$. In either case, the term for outcome $q$ contributes less to the sum in $U(u;\cC'^t)$ than $U(u;\cC^t)$, so $U(u;\cC'^t) \le U(u; \cC^t)$ as desired.

% \subsection{A mixed-integer linear program for computing a profit-maximizing menu} \label{MILP}

% \begin{lemma}
%    The optimal menu can be obtained by solving the following MILP~\eqref{independent-transition-milp-simplified} with $O(mn\ell)$  variables and $O(\max \{mn \ell, \ell^2 \})$ constraints.
%     \begin{align}
%         \max_{w, z, u, \xi} \quad \bE \quad & w^t - \sum_{i} \xi_i^t c_i \notag\\
%         & \sum_{i,j} p_j^i v_j^t z_j^{i,t} - w^t \geq \sum_{i,j}p_j^i v_j^t z_j^{i,u} - w^{u}, \quad \forall t, u\notag \\
%         & \sum_{i,j} p_j^i v_j^t z_j^{i,t} - w^t \geq 0, \quad  \forall t \notag\\
%         & \label{independent-transition-milp-simplified}\\
%         &\xi_i^t - (1-u_j^t) \leq z_j^{i,t} \leq \xi_i^t, \quad \forall i,j, t \notag\\
%         & z_j^{i,t} \leq u_j^t, \quad \forall i,j, t \notag\\
%         & \sum_i^t \xi_i^t = 1, \quad \forall t \notag\\ 
%         & \sum_i z_j^{i,t} = u_j^t,  \quad \forall j, t\notag \\
%         & \xi_i^t, z_j^{i,t} \geq 0, \quad \forall i,j,t\notag\\
%         & u_j^t \in \{0, 1\}, \quad \forall i, j, t\notag
%     \end{align}
% \end{lemma}
% \begin{proof}
% To prove the equivalence, we need to show that any optimal training payments, usage payments, and effort distribution can be transformed into a feasible solution to the program~\eqref{independent-transition-milp-simplified} and vice versa when the transitions are type-independent.

% We first prove the forward direction.  Given \cref{two-usage-prices-suffice}, we define the binary variable $u_j^t$ with value $1$ if the usage payment $x_j^t$ is $0$, and let $z_j^{i,t} = u_j^t \xi_i^t$. The training payment $w^t$ is set to be the same as in the original optimal solution. Any optimal usage payments $x_j^t$ quality level $j$ and type $t$ such that $x_j^t = 0$, by construction, $u_j^t = 1$ and $z_j^{i,t} = \xi_i^t$. On the other hand, For any optimal usage payments in type $t$ with $x_j^t = \infty$  in a quality level $j$, we have that $z_j^{i,t} = 0$ for all $i$ by the construction. Finally, by noticing that $\sum_{i,j}p_j^iv_j^t z_j^{i,t} = \sum_{i,j}p_j^iv_j^t \xi_i^t u_j^t$, the constructed solution $z, w, \xi$ satisfy all constraints in MILP~\eqref{independent-transition-milp-simplified}, thus feasible. 


% Conversely, given any optimal solution to MILP~\eqref{independent-transition-milp-simplified}, we define the training payments, the effort distribution as the same. For each quality level $j$ and type $t$, we define the usage payment $x_j^t = 0$ if $\sum_i z_j^{i,t} = u_j^t = 1$. From the constraint $\xi_i^t - (1-u_j^t) \leq z_j^{i,t} \leq \xi_i^t$, we have that $z_j^{i,t} = \xi_i^t$. We define $x_j^t = \infty$ if $\sum_i z_j^{i,t} = u_j^t = 0$. This implies that $z_j^{i,t} = 0$ for all $i$. Hence $z_j^{i,t} = \xi_i^t u_j^t = \xi_i^t  \ind \{x_j^t = 0\}$ for all $i,j,t$. It follows that the constructed usage payments is a feasible solution to the original problem. 

% \end{proof}



\subsection{Profit-maximizing upfront prices for two types} \label{upfront-price-formula-two-types-proof}

We prove \cref{upfront-price-formula-two-types}. Note that increasing both upfront prices $w^1$ and $w^2$ at the same rate increases profit while preserving IC since each buyers' utility for each contract decreases at the same rate. This process stops when one buyer's surplus is 0. Hence any profit-maximizing menu must have no buyer surplus for at least one type. To finish the proof of the lemma, we split into three cases.

    \begin{enumerate}
        \item[\tbf{Case 1.}] Type 1 has no buyer surplus, so $$w^1 = \sum_{q\in S} p^a_q v^a_q.$$ By \cref{highest-type-no-usage-prices}, note that type 2's usage prices are 0 in a profit-maximizing menu. The type 2 IC constraint is $$\sum_q p^a_q v^2_q - w^2 \ge \sum_{q\in S} p^a_q v^2_q - w^1 \iff w^2 \le w^1 + \sum_{q\notin S} p^a_q v^2_q$$ and the type 2 IR constraint is $$w^2 \le \sum_q p^a_q v^2_q.$$ Since the menu is assume to be profit-maximizing, one of these two conditions must bind, so $$w^2 = \min \bc{\sum_{q\notin S} p^a_q v^2_q + \sum_{q\in S} p^a_q v^1_q, \sum_q p^a_q v^2},$$ precisely what the lemma states.

        % \item[\tbf{Case 2.}] Both types have no buyer surplus. We treat this case in the same as \tbf{Case 1}, labeling the lower type as type 1.

        \item[\tbf{Case 2.}] Type 1 has positive buyer surplus, so the type with no buyer surplus is the higher type 2. By \cref{highest-type-no-usage-prices}, note that type 2's usage prices are 0 in a profit-maximizing menu. We can assume that the strict inequality $w^2 > w^1$ is true, otherwise if $w^2 = w^1$ we could have treated the type with no buyer surplus as the lower type 1 in which case \tbf{Case 1} applies.

        Since the menu is assumed to be profit-maximizing, increasing $w^1$ slightly to increase profit cannot be possible. Note that type 2 IC is not violated when $w^1$ increases since $\cC^1$ becomes \emph{less} attractive to type 2. Since type 1 has positive buyer surplus, type 1 IR is not violated by increasing $w^1$ slightly, so type 1 IC must bind, meaning type 1 prefers contract $\cC^2$ as much as $\cC^1$. However, if we now replace $\cC^1$ with $\cC^2$, type 1 revenue increases from $w^1$ to $w^2$, contradicting the assumption the menu is profit-maximizing.
        
    \end{enumerate}
    
\subsection{Profit function preserves proximity in the state} \label{value-is-proximity}

    We prove \cref{value-proximity}. It suffices to prove the following:
    
    % \begin{claim}
    %     The objective function $\pi_{\tt{indirect}}(s)$ is 1-Lipschitz in the state $s$ equipped with the $L^1$-norm.
    % \end{claim}

    \begin{claim*} \label{differ}
         For two states $s$ and $s'$ that differ in only one component $i = (t, t')$ by a constant $\eps>0$, we have $\pi_{\tt{indirect}}(s') \ge \pi_{\tt{indirect}}(s) - \eps$. Furthermore, if $s'_i \ge s_i$ then $\pi_{\tt{indirect}}(s') \ge \pi_{\tt{indirect}}(s)$.
    \end{claim*}
    \begin{proof}
        We consider four cases:

    \begin{enumerate}
        
        \item[\tbf{Case 1:}] \emph{$s'$ is derived from $s$ by an increase of $\eps$ in a component of the form $(t,t')$ for $t\in [T]$ and $t' = u(t)$.}

        In this case it is clear $\pi_{\tt{indirect}}(s') \ge \pi_{\tt{indirect}}(s)$ because the only change from $s$ to $s'$ is that type $t$ prefers the contract $u(t)$ that was yielding the greatest utility even more. Hence if we leave the upfront prices unchanged, $u:[T]\to  \bc{0} \cup [T]$ and thus the revenue remains unchanged.

        \item[\tbf{Case 2:}] \emph{$s'$ is derived from $s$ by a decrease of $\eps$ in a component of the form $(t,t')$ for $t\in [T]$ and $t' = u(t)$.}

        This case is significantly harder to analyze then the previous because if we leave the upfront prices unchanged, $u(t)$ might change now that $\cC^{t'}$ is providing less utility for type $t$, leading to an potentially large drop in revenue. In order to prevent this from happening, we must decrease $w^{u(t)}$ by $\eps$ in order to make up for the value difference, and doing so indeed has type $t$ still choosing $\cC^{t'}$. However, decreasing $w^{t'}$ presents its own problems, as now $\cC^{t'}$ becomes more attractive for other types as well, which could lead other types to choose $\cC^{t'}$ and potentially lead to large drops in revenue. In order to solve the problem induced by decreasing $w^{t'}$ by $\eps$ it would seem that we have to decrease $w^{u(\tau)}$ for all other types $\tau$ by $\eps$. However, in doing this we have effectively negated the effect of decreasing $w^{t'}$ by $\eps$ as now type $t$ could now find another contract more attractive again, thus resulting in an infinite recursive loop of revenue decreases.

        To avoid this recursive issue, we choose the contracts whose payments $w$ we decrease by $\eps$. For all $\tau\in [T]$ for which $w^{u(\tau)} - c\bp{a^{u(\tau)}} \ge w^{t'} - c\bp{a^{t'}}$, we decrease $w^{u(\tau)}$ by $\eps$. We leave all other upfront prices unchanged. In particular note that $\tau = t$ satisfies the condition above. We claim that changing the upfront prices based on this condition on $\tau$ leads to a service provider profit of at least $\pi_{\tt{indirect}}(s) - \eps$.

        First observe that all $\tau\neq t$ that meet the condition above will still prefer $\cC^{u(\tau)}$ because $w^{u(\tau)}$ decreases by $\eps$ and no other payment decreases by more than $\eps$. Hence the service provider profit loss in moving from $s$ to $s'$ for type $\tau$ is at most $\mu(\tau) \cd \eps$.

        For $\tau = t$, note that either type $t$ still chooses $\cC^{t'}$ or will choose a contract of the form $\cC^{u(\tau)}$ for $\tau$ satisfying the condition above, as only these contracts have their payments reduced by $\eps$. The profit from the service provider for type $t$ is $$w^{u(\tau)} - \eps - c\bp{a^{u(\tau)}} \ge w^{t'} - c\bp{a^{t'}} - \eps,$$ which represents a profit loss of at most $\mu(t) \cd \eps$.

        Finally, we claim that the service provider profit loss in moving from $s$ to $s'$ for all other types $\tau'$ is at most $\mu(\tau') \cd \eps$. If $\tau'$ switches contracts at all, it must be to a contract of the form $\cC^{u(\tau)}$, as only these contracts have their payments decreased by $\eps$. Since $\tau'$ did not meet the condition above for decreasing payments, it must have been the case that $w^{u(\tau')} - c\bp{a^{u(\tau')}} < w^{t'} - c\bp{a^{t'}}$, so the if $\tau'$ switches contracts to $\cC^{u(\tau)}$, the new profit from type $\tau'$ is $$w^{u(\tau)} - \eps - c\bp{a^{u(\tau)}} \ge w^{t'} - c\bp{a^{t'}} - \eps > w^{u(\tau')} - c\bp{a^{u(\tau')}} - \eps,$$ which represents a service provider profit loss of at most $\mu(\tau') \cd \eps$.

        We conclude that the overall profit loss incurred when moving from $s$ and $s'$ and decreasing payments according to the condition $w^{u(\tau)} - c\bp{a^{u(\tau)}} \ge w^{t'} - c\bp{a^{t'}}$ is at most $\sum_{\tau\in [T]} \mu(\tau) \cd \eps = \eps$ as desired.

        \item[\tbf{Case 3:}] \emph{$s'$ is derived from $s$ by an increase of $\eps$ in a component of the form $(t,t')$ for $t\in [T]$ and $t' \neq u(t)$.}

        This has the same analysis as \tbf{Case 1} since the only change is at type $t$ prefers a contract that was already not utility-maximizing less, so we can leave the upfront prices unchanged.

        \item[\tbf{Case 4:}] \emph{$s'$ is derived from $s$ by a decrease of $\eps$ in a component of the form $(t,t')$ for $t\in [T]$ and $t' \neq u(t)$.}

        This has the same analysis as \tbf{Case 2} since type $t$ preferring a different contract more is effectively the same as type $t$ preferring its currently chosen contract less. Hence we can modify the upfront prices according to the same condition as in \tbf{Case 2} to resolve this case.
    \end{enumerate}
    \end{proof}

    % \newcommand{\pii}{\pi_{\tt{indirect}}}
    
    % Note that the proof of \cref{differ} actually generalizes to the following:

    % \begin{claim} \label{differ-1}
    %      For any two states $s$ and $s'$, $\pii(s') \ge \pii(s) - \norm{s - s'}_{\infty}$.
    % \end{claim} 
    % \begin{proof}
    %     We decompose changes in the state $s$ into changes by component. By the proof of \cref{differ}, no profit loss is incurred in \tbf{Case 1} and \tbf{Case 3}. Since \tbf{Case 4} is equivalent to \tbf{Case 2} in terms of preferences for type $t$, a profit loss of at most $\ab{s_i - s'_i} \le \norm{s - s'}_\infty$ is incurred in \tbf{Case 2} for every tuple of the form $(t, u(t))$, and there are $T$ such tuples.
    % \end{proof}

    % To finish the proof of \cref{value-is-proximity}, we need to show that the claim implies that $\pi_{\tt{indirect}}(s)$ preserves proximity in $s$. Consider two $r$-close states $s$ and $s'$, so $r^{-1} \cd s_i \le s'_i \le r \cd s_i$ for all $i$. By applying the claim to each component of the claim and summing, we have
    % $$\pii(s') \ge \pii(s) - \bp{1 - \fr{1}{r}} \sum_i s_i.$$ Clearly $$\pii(s) \ge \max_{t} s_i$$ since we can extract full consumer surplus from any single buyer type $t$ for any contract $\cC^{t'}$, so we have
    % \begin{align*}
    %     \pii(s) &\ge \max_{i = (t,t')} s_i
    %     \\&\imp \sum_i s_i \le T^2 \cd \max_i s_i \le T^2 \cd \pii(s)
    %     \\&\imp \pii(s') \ge 
    % \end{align*}
    
    % decomposing changes in the state $s$ into changes by component, it suffices to consider one component at a time and prove the following: for two states $s$ and $s'$ that differ in only one component $(t, t')\in [T]\times [T]$ by $\eps>0$, $\pi_{\tt{indirect}}(s') \ge \pi_{\tt{indirect}}(s) - \eps$. Note that we can also show $\pi_{\tt{indirect}}(s) \ge \pi_{\tt{indirect}}(s') - \eps$ by swapping $s$ and $s'$.

\subsection{Profit function is efficiently computable given state} \label{value-is-efficient}

    We prove \cref{value-efficient}. The key idea to enumerate over all possible functions $u:[T]\to \bc{0}\cup [T]$, of which there are $(T+1)^T$, a constant because we assume the number of types $T$ is a constant. Fixing the function $u$, we claim that the service provider's optimization problem can be solved by linear programming. Indeed, fixing $u$ the profit is linear in the upfront prices $\bp{w^t}_{t\in [T]}$ and furthermore the $\argmax$ constraints added by fixing $u$ are precisely the IC and IR constraints:
        \begin{align*}
    U(t; \cC^{u(t)}) \ge U(t; \cC^{t'}), \quad& \fl t,t' \\
        \iff V(t; \cC^{u(t)}) - w^{u(t)} \ge V(t; \cC^{t'}) - w^{t'}, \quad& \fl t,t'\\
        \iff w^{u(t)} \le w^{t'} + V(t; \cC^{u(t)}) - V(t; \cC^{t'}), \quad& \fl t,t' && \te{(IC)} \\
        w^{u(t)} \le V(t;\cC^{u(t)}),\quad& \fl t  && \te{(IR)}
\end{align*}
    
Note that the IC and IR constraints are linear in the upfront prices. We conclude that $\pi_{\tt{indirect}}(s)$ can be computed by solving $(T+1)^T$ linear programs, one for each assignment mapping $u$ from types to contracts, and taking the maximum objective value over all feasible linear programs.

\subsection{Types with higher $\alpha$ have higher values in the single-parameter setting} \label{higher-types-higher-values-proof}

We prove \cref{higher-types-higher-values}. The IC constraints can be rewritten as
    \begin{align*}
        & U(t,t) \geq U(t,u) \\
        \iff & V^t - w^t \ge V(t,u) - w^u  = \fr{\alpha^t}{\alpha^u} \cd V^u - w^u \\
        \iff & w^t  - w^u \le \alpha^t \left( \fr{V^t}{\alpha^t} - \fr{V^u}{\alpha^u} \right)
    \end{align*}
    Combined with the IC constraint
    \begin{align*}
        & U(u;u) \ge U(u,t) 
        \iff  w^t  - w^u \geq \alpha^u \left( \fr{V^t}{\alpha^t} - \fr{V^u}{\alpha^u} \right),
    \end{align*}
    we have
    \begin{align}\label{monotonic-payment}
        \alpha^u \left( \fr{V^t}{\alpha^t} - \fr{V^u}{\alpha^u} \right) \leq w^t - w^u \leq \alpha^t \left( \fr{V^t}{\alpha^t} - \fr{V^u}{\alpha^u} \right),
    \end{align}
    hence
    \begin{align*}
        \bp{\alpha^t - \alpha^u} \left(\fr{V^t}{\alpha^t} - \fr{V^u}{\alpha^u}\right) \geq 0.
    \end{align*}
    Since $t\ge u$, we conclude $\fr{V^t}{\alpha^t} \geq \fr{V^u}{\alpha^u}$ and also $w^t \geq w^u$ from \cref{monotonic-payment}.

\subsection{Revenue-maximizing upfront prices in the single-parameter setting} \label{revenue-maximizing-upfront-prices-proof}

We prove \cref{revenue-maximizing-upfront-prices}. The second equality follows from combining like terms with the same $V^i$. To prove the first equality, by \cref{higher-types-higher-values}, we can consider only IC constraints
\begin{equation} \label{single-parameter-ic}
    V^{t} - w^{t} \ge \fr{\alpha^{t}}{\alpha^{u}} \cd V^{u} - w^{u},\quad \fl t>u.
\end{equation}
for $t>u$. This is because $t>u\imp w^t \ge w^u$, so if an IC constraint of the form \cref{single-parameter-ic} for $t<u$ is violated then we can replace type $t$'s contract $\cC^t$ with $\cC^u$ while weakly increasing seller revenue.

We proceed by induction on $t$. Note that we can increase $w^1$ to $V^1$ while only strengthening the other IC constraints because $w^1$ never appears on the left hand side of an IC constraint. Increasing $w^1$ increases the seller revenue, so any revenue-maximizing menu has $w^1 = V^1$. At this point, note that the only IC constraint with $w^2$ on the left hand side is $$V^2 - w^2 \ge \fr{\alpha^2}{\alpha^1} \cd V^1 - w^1,$$ hence we can increase $w^2$ until $$w^2 = w^1 + V^2 - \fr{\alpha^2}{\alpha^1} \cd V^1 = V^1 + \bp{V^2 - \fr{\alpha^2}{\alpha^1} \cd V^1}.$$

In general, once we have upfront $w^1,\lds,w^{t-1}$, the IC constraints with $w^t$ appearing on the left hand side are
\begin{align} \label{ic-restricted}
    V^t - w^t &\ge \fr{\alpha^t}{\alpha^i} \cd V^i - w^i,\quad \fl i<t
\end{align}
We claim that all IC constraints except the one for $i=t-1$ are redundant. Indeed, by \cref{monotonic-payment} we have $$w^{t-1} - w^i \le \alpha^{t-1}  \bp{\fr{V^{t-1}}{\alpha^{t-1}} - \fr{V^i}{\alpha^i}} \le \alpha^t \bp{\fr{V^{t-1}}{\alpha^{t-1}} - \fr{V^i}{\alpha^i}},$$ which upon rearrangement implies that the largest right hand size among the IC constraints in \cref{ic-restricted} is $$\fr{\alpha^t}{\alpha^{t-1}} \cd V^{t-1} - w^{t-1} \ge \fr{\alpha^t}{\alpha^i} \cd V^i - w^i.$$ We conclude that the revenue-maximizing $w^t$ is $$w^t = w^{t-1} + \bp{V^t - \fr{\alpha^t}{\alpha^{t-1}} \cd V^{t-1}},$$ completing the inductive step.

% Continuing in this manner, note that for every $t>1$ we can assume
% \begin{equation} \label{r-in-terms-of-v}
%     w^t = \min_{t' < t} \bc{w^{t'} + V^t - \fr{\alpha^t}{\alpha^{t'}} V^{t'}} = V^t + \min_{t'<t}\bc{w^{t'} - \fr{\alpha^t}{\alpha^{t'}} V^{t'}}.
% \end{equation}

\subsection{Single-contract menus do not in general maximize profit in the single-parameter setting} \label{counterexample-profit}

We construct a counterexample in the single-parameter setting in which any profit-maximizing menu cannot consist of a single contract. Consider the following single-parameter problem instance:
    \begin{itemize}
        \item Let $T=2$, $\mu = \bp{\fr23, \fr13}$, and $Q = \bc{1,2}$. There are two actions $a^1, a^2$ with costs $c(a^1) = 0, c(a^2) = \fr32$ such that action $a^i$ deterministically maps to outcome $i$.

        \item Valuations are given by $\mathbf{v}^1 = (1,2)$ and $\mathbf{v}^2 = (2,4)$.
    \end{itemize}
We can verify that the direct menu consisting of contracts $\cC^1 = (a^1, 1, \mbf{0}), \cC^2 = (a^2, 3, \mbf{0})$ is IC and yields seller profit $$\fr23 \cd \bp{w^1 - c(a^1)} + \fr13 \cd \bp{w^2 - c(a^2)} = \fr23 \cd (1-0) + \fr13 \cd \bp{3 - \fr32} = \fr76.$$ On the other hand, we claim that no indirect menu consisting of a single contract $\cC = (a, w, \mbf{x})$ can achieve revenue 1. We split into two cases:
\begin{enumerate}
    \item[\tbf{Case 1.}] If $a = a^1$, then since $a^1$ deterministically maps to outcome 1, the usage price $x_2$ does not factor into the profit. To yield profit at all, we must have $x_1 = 0$. If $w \le 1$ then both types prefer $\cC$ to the trivial contract $\cC^0$, for a profit of at most 1. If $w\in (1, 2]$ then only type 2 prefers $\cC$ to $\cC^0$, for a profit of at most $\fr13 \cd 2 = \fr23$.

    \item[\tbf{Case 2.}] If $a = a^2$, then since $a^2$ deterministically maps to outcome 2, the usage price $x_1$ does not factor into the profit. To yield profit at all, we must have $x_2 = 0$. If $w \le 2$ then both types prefer $\cC$ to the trivial contract $\cC^0$, for a profit of at most $2 - \fr32 = \fr12$. If $w\in (2, 4]$ then only type 2 prefers $\cC$ to $\cC^0$, for a profit of at most $\fr13 \cd \bp{4 - \fr32} = \fr56$.
\end{enumerate}
We conclude that the maximum seller profit from a single-contract menu is at most 1 and hence a single contract cannot be profit-maximizing.