%%
%%---------------------------------------------
\section{An E2E Network and Application Slicing Architecture using MEC}
\label{sec:app_slice}
%%
\noindent
In this section, we first introduce the concepts of application services and application slices, and we instantiate these concepts in the context of MEC environments. Then, we discuss application slice from both operational and business perspectives by introducing operation and business roles, respectively, and relationships of the different entities involved in application slicing and service provisioning. We also discuss the different architectural use cases that correspond to the proposed model. Finally, we elaborate on an orchestration/management architecture for 3GPP networks integrating MEC systems that will enable such new use cases, and we propose an extended MEC reference architecture required to support the envisioned management architecture.   
%%
%%---------------------------------------------
\subsection{Preliminary concepts}
\label{sec:pre_concepts}
%%
\noindent
As it could be sensed from both the introduction and the background sections, the 3GPP CSs cannot be straightforwardly used to model the services offered by an edge computing platform. For the sake of disambiguation, in this work, we call the services offered to the customers of an edge computing platform as \textit{Application Services} (ASs). The main reason is that CSs and ASs are designed to support different business purposes, therefore, they are regulated by different SLA frameworks. Specifically, \textit{CSs are dedicated to the transmission of data and network traffic}, while \textit{ASs are designed to support services which are generally not communication oriented in essence, but focus on data processing}, such as Augmented Reality (AR), Virtual Reality (VR), Ultra-High Definition (UHD) video, real-time control, etc. From the management perspective, it would not be sensible to manage ASs with the existing 3GPP CSMF. Some extensions or new management functions should be added to the overall management framework to translate application-related SLAs into application-related SLSs and network-related SLSs. 

In analogy to the 3GPP network slice concept, we define an \textit{Application Slice} (APS) as a \textit{a virtualised data collection system and a set of data processing resources that provides specific processing functions to support an Application Service and the associated SLA. The APS can possibly include Network Functions to ensure correct data communication among the previous processing functions}. A key aspect of the application slice concept is the support of \textit{isolation} not only of the resources that are used by each application slice but also of the management data and functions. Similarly to a network slice, the application slice is an end-to-end concept and can span multiple domains (e.g., device, edge platform, network). Furthermore, an application slice is not a monolithic entity. Following the well-known principles of service-oriented architectures, it is built as an \textit{atomic data-processing function} which has well defined functional behaviours and interfaces, called in this work \emph{Application Component Functions} (ACFs). We can now define an \textit{Application Slice Subnet} (APSS) as a set of at least one ACF and possible NFs supporting an application slice. It is essential to point out that application slices and network slices are not independent concepts, but they are strongly intertwined as application slices deployed at the edge of the network require dedicated network slices to deliver the massive amounts of data and network traffic that application services will typically produce. In the following sections, we will also discuss two potential distinct models to allow an application slice to use a network slice.  

The general concepts that we have introduced so far can be easily instantiated in the context of MEC, NFV and 5G systems. First of all, a MEC platform can be used to offer ASs, which can be deployed as conventional MEC applications and implemented using a collection of ACFs. Furthermore, the MEC system can be considered as one of the domains over which an application slice can span. Thus, the MEC management layer should be responsible for the orchestration and deployment of one of the application slice subnets that compose the E2E APS, hereafter denoted by \textit{MEC application slice subnet}. It is important to point out that a MEC application slice subnet includes one or more ACFs, but it can also include VNFs to support network functionalities. Furthermore, the tenant of a MEC system is not necessarily an end customer, but the MEC system can offer its services following a MEC-Platform-as-a-Service model.  

Finally, it is worth pointing out the differences between VNFs and ACFs which are designed for different purposes - network traffic processing for the former and data processing for the latter. Furthermore, in a multi-tenant MEC environment, like the one in~\cite{Cominardi2020} or~\cite{2020_MNET_MEC_subslice}, it is likely that the MEC applications implementing ACFs will not be orchestrated by the same entity that would orchestrate principal VNFs. Nevertheless, there are a lot of common points in terms of operation and management between a VNF and an ACF as both rely on the same set of virtualisation technologies for their deployment. Thus, we will reuse and extend both 3GPP management architecture for network slice and MEC-in-NFV architecture to support the proposed E2E application slice framework.
%
%\Thai{-- beginning: definitions added --}
%\newline Similarly to the 3GPP network slice framework, we can define successively:
%\begin{itemize}
%    \item An Application Slice Subnet as a set of at least one ACF. This set could include VNFs.
%    \item An Application Slice consists of multiple Application Slice Subnets augmented by SLS parameters (e.g. QoS parameters). The latter are technology-dependent objectives that should be enforced to meet the SLA.
%    \item An Application Service is built from one or multiple Application Slices augmented by a common SLA parameter set (i.e. business commitment)
%\end{itemize}
%\Thai{-- end: definitions added --}
%%
%%---------------------------------------------
\subsection{High-level roles for application slice management}
\label{subsec:roles}
%%
\noindent
From the APS management (and orchestration) perspective, we first need to define high-level roles in order to draw responsibilities boundaries similarly to what was proposed for 5G network slice management (see section~\ref{subsec:mgmt_architecture}). As discussed above, our focus is on Application Services that are offered by a MEC system. The Figure~\ref{fig:AS_roles} shows the different roles identified:
%
\renewcommand\labelitemi{$\bullet$}
\begin{itemize}[noitemsep,topsep=2pt]
    \item \textit{Application Service Consumer} (\textit{ASC}): uses Application Services.
    \item \textit{Application Service Provider} (\textit{ASP}): provides Application Services that are built and operated upon one or multiple application slice subnets, including MEC application slice subnets. Each of those application slice subnets is in turn built from a set of ACFs provided by the \emph{Application Component Function Supplier}.
    \item \textit{MEC Operator} (\textit{MOP}): operates and manages ACFs using a MEC Platform. We assume that the MEC Operator implements the MEC orchestrator and the ETSI MEC standardised interfaces as presented in~\cite{MEC003}. It designs, builds and operates its MEC platform to offer, together with the MEC orchestrator, MEC application slice subnets to ASPs from ACFs that the ASP has provided as an input.
    \item \textit{Application Component Function Supplier}: designs and provides ACFs to ASPs.
\end{itemize}
%
\begin{figure}[t]
\centering
\includegraphics[width=0.4\textwidth]{figures/3_FigureARoles.png}
\caption{High-level functional roles in the MEC application slice framework}
\label{fig:AS_roles}
\end{figure}
%
Interestingly, an ASC can become an ASP while adding new ACFs into consumed ASs and providing new ASs. For instance, let us assume that ASC $C_1$ uses from ASP $P_1$ an AS $S_1$ that provides face recognition capabilities. Then, ASC $C_1$ can integrate into $S_1$ other ACFs that permit to retrieve video stream from customers' cameras and ACFs that take control of customers' door, thus building a new AS $S_2$, which provides automatic door opening based on face recognition. ASC $C_1$ then becomes a new ASP $P_2$ selling the previous new AS to customers such as house/apartment rental platforms. Furthermore, we can easily make a parallel between Figure~\ref{fig:AS_roles} and Figure~\ref{2_figRoles} in Section~\ref{subsec:mgmt_architecture}. Roles responsible for low layers, namely Hardware Supplier, Data Center Service Provider, NFVI Supplier, and Virtualisation Infrastructure Service Provider, remain the same. However, the Application Component Function Supplier has replaced the VNF (or Equipment) Supplier; the MEC Operator has replaced the Network Operator, the Application Service Provider (ASP) the CSP, and the Application Service Customer (ASC) the CSC. 

It is helpful to point out that a business organisation can play one or multiple operational roles simultaneously. Therefore, without trying to be exhaustive in terms of business models, we focus in our work on two categories of organisations whose responsibility boundaries appear to make most of the sense as per our knowledge:
%
\renewcommand\labelitemi{$\bullet$}
\begin{itemize}[noitemsep,topsep=2pt]
    \item \textbf{\textit{MEC Owner}}: it plays both Hardware supplier and NFVI supplier roles. The fact that the MEC Owner also manages the virtualisation infrastructure and the edge servers allows him/her to dynamically split his/her infrastructure into logical partitions or network slices (i.e. greater degree of flexibility). It is noted that the MEC Owner could be the equivalent of what is called the \textit{MEC Host Operator} in \cite{MEC027}, as it offers virtualised MEC hosts and MEC platforms to its tenants. However, we prefer the `MEC Owner' terminology to avoid confusion with the `MEC Operator' role.
    \item \textbf{\textit{MEC Customer}}: plays both the role of the MEC Operator and ASP. It is helpful to point out that the ASP role offers a business interface (with the ASC). In contrast, the MEC Operator role offers the actual enforcement/implementation (i.e. SLS) of the business objectives (i.e. SLA) agreed across the ASP business interface. It is noted that the MEC Operator role alone cannot endorse a business organisation as it only offers Application Slice (including SLS) and not the Application Service (with related SLA). 
\end{itemize}
%
In conclusion, from a business perspective in this study, we advocate a multi-tenancy model in which ASCs are tenants of a MEC Customer, and MEC Customers are tenants of a MEC Owner. As explained in the following section, our proposed multi-tenant MEC system supports data isolation (through separated data planes) and resource orchestration separation (through separate resource orchestrators) between tenants.  
%%
%%---------------------------------------------
\subsection{Deployment models for an E2E application slice}
\label{subsec:AS_deployment}
%%
\noindent
%
We distinguish two distinct deployment models for our proposed application slice concept: $(i)$ the \textit{overlay} model, and $(ii)$ the \textit{stitching} model. 

\begin{figure}[ht]
\centering
\includegraphics[width=0.5\textwidth]{figures/3_DeploymentModels.png}
\caption{\added{Interconnection models between application slice and network slice}}
\label{3_DeploymentModels}
\end{figure}
%
%\begin{figure}[ht]
%\centering
%    \begin{subfigure}[b]{0.5\textwidth}
%         \centering
%         \includegraphics[width=0.8\textwidth]{figures/3_FigureAS.png}
%         \caption{Network slice includes the data plane of the MEC customer}
%         \label{3_figAS}
%     \end{subfigure}
%     \hfill
%    \begin{subfigure}[b]{0.5\textwidth}
%         \centering
%         \includegraphics[width=0.8\textwidth]{figures/3_FigureASoverlay.png}
%         \caption{Network slice does not include the data plane of the MEC customer}
%         \label{3_figASoverlay}
%     \end{subfigure}
%\caption{Overlay deployment model: the application slice is a consumer of an E2E network slice.}
%\label{fig:overlay}
%\end{figure}
%
The first model assumes that the E2E APS is a consumer of a Communication Service offered by the underlying network slice (see Figure~\ref{3_DeploymentModels}\added{a}). In this case, the network slice is responsible for the entire communication latency, and we denote it as E2E \added{network} slice. In addition to the latter, the application slice caters for processing and possibly storage latency at both ends. It is important to point out that the E2E network slice encompasses not only the 5G network slice but also the network slice subnets within the MEC Owner and MEC Customer domains\footnote{Note that we use the term MEC Customer network slice subnet instead of MEC application slice subnet because, in this deployment model, the MEC Customer is only responsible for the management and orchestration of the data plane within its virtualised MEC environment. This is also justified by the need to align the architectural component boundaries with the liability boundaries of the deployment model.}. Indeed, as shown in Figure~\ref{3_DeploymentModels}a our slicing framework leverages recent advantages on virtualisation technologies that allow a virtualisation layer to be composed of multiple nested sub-layers, each using a different virtualisation paradigm. According to the functional role split illustrated in Section~\ref{subsec:roles}, a MEC Owner can use a hypervisor technology to operate its MEC hosts and to deploy multiple virtualised environments to MEC customers (e.g. allocating one or more VMs\footnote{In ETSI NFV terminology, a VM is also designated as `NFV VNF'.} using an NFVI). Each virtualised environment includes a full-fledged MEC system used by the MEC Customer to allocate further the resources assigned to its VMs to the multiple Application Services it deploys to its users by applying internal orchestration and management criteria. In the case of the MEC customer, a container-based virtualisation technology could be used as a top virtualisation sub-layer to manage application deployment into its allocated virtualised MEC system. It is worth noting that in \added{Figure~\ref{3_DeploymentModels}a} the E2E network slice includes the data plane of the MEC customer. However, an alternative overlay model is also possible in which the E2E network slice terminates at the network boundary between the MEC Owner and the MEC Customer as per \added{Figure~\ref{3_DeploymentModels}b}.

In the stitching deployment model \added{(see Figure~\ref{3_DeploymentModels}c)}, we assume that the MEC application slice subnet is a peer of the network slice subnets. Virtual appliances using a subnet border API, such as a virtual gateway or virtual load-balancer, can be used to interconnect MEC application slice subnets to the adjacent network slice subnet, similarly to what was proposed in~\cite{2020_MNET_MEC_subslice} and \cite{2021_TNSM_e22_slice_survey}. Such stitching could be a one-to-one interconnection as well as a multiple-to-one interconnection. The end-to-end application slice could be seen in this case as the composition of different application slice subnets (UE-operated or MEC-operated) together with network slice subnets. Latency-wise, the MEC application slice subnet is responsible, in this case, for a tiny part of the network latency budget in addition to the processing and storage latency induced by the MEC applications and their related ACFs.
%
%\begin{figure}[ht]
%\centering
%\includegraphics[width=0.4\textwidth]{figures/3_FigureASubS.png}
%\caption{Stitching deployment mode: interconnection of the different slice subnets to form the E2E application slice.}
%\label{3_figASstitch}
%\end{figure}
%

We conclude this section by noting that the different deployment models will lead to different approaches to combine our proposed E2E application slice management/orchestration framework with the 3GPP management architecture, as explained in Section~\ref{subsec:AS_Mgmt}.

%Alternatively, the MEC application slice subnet network connectivity could be performed as an overlay network on top of the end-to-end network slice (see Figure \ref{3_figASoverlay}). This architecture is then equivalent to the one illustrated in Figure \ref{3_figAS} except that the end-to-end network slice terminates this time at the network boundary between the MEC Owner and the MEC Customer.
%
%\begin{figure}[ht]
%\centering
%\includegraphics[width=0.4\textwidth]{figures/3_FigureASoverlay.png}
%\caption{Application slice as overlay on top of E2E network slice}
%\label{3_figASoverlay}
%\end{figure}

%%
%%---------------------------------------------
\subsection{Architecture for application slicing in a multi-tenant MEC system}
\label{subsec:new_MEC_arch}
%%
\noindent
%
In this section, we elaborate on the new MEC components and extensions to the current MEC management architecture that are needed to support E2E application slicing and multi-tenancy within multiple MEC customers. Figure~\ref{fig:ext_arch} shows an illustrative example of the proposed extended MEC reference architecture. The primary design rationale of our proposal is that the MEC system should be split into two responsibility domains following a \textit{two-layer hierarchical MEC architecture}, where the bottom layer is managed and orchestrated by the MEC Owner, and the top-layer is independently managed and orchestrated by MEC Customers. Such a hierarchical architecture allows a single MEC deployment to host multiple MEC Customers. Each of them has his own MEC network slice subnet (i.e. his dedicated data plane provided by the MEC Owner) with related management capability. In turn, each MEC Customer manages and orchestrates his own MEC application slices. 
%
\begin{figure*}[ht]
\centering
\includegraphics[clip,trim= 0cm 0cm 0cm 0cm,width=0.8\textwidth]{figures/3_FigureMECCustOwner.png}
\caption{Multi-tenant MEC architecture supporting network and application slicing}
\label{fig:ext_arch}
\end{figure*}
%

Implementation-wise, the proposed two-layer MEC architecture is enabled by the nested virtualisation capability of the MEC infrastructure, as anticipated in Section~\ref{subsec:AS_deployment}. In the system illustrated in Figure~\ref{fig:ext_arch}, the MEC Owner does not deploy individual MEC entities as in the MEC-in-NFV reference architecture, but a collection of 'ETSI VNFs' (or VMs) to provide each MEC Customer with a complete MEC system. Differently from~\cite{Cominardi2020} we do not use 'ETSI VNFs' to deploy MEC applications and MEC platforms but to deploy a virtualised MEC environment encompassing virtualised MEC hosts and a virtualised MEC management system. Furthermore, in our MEC-in-NFV architecture variant, we introduce functional blocks that substitute the \added{MEAO} and \added{MEPM-V} of the original reference architecture. Specifically, we substitute the \added{MEAO} with the \textit{MEC Owner Orchestrator} (MEOO), which is in charge of implementing the policies to select the MEC infrastructures on which to deploy a MEC Owner network slice subnet. As explained in Section~\ref{subsec:AS_Mgmt}, the MEOO receives the commands to create, modify or delete a MEC Owner network slice subnet from a 3GPP management function called MEC NSSMF. Furthermore, the MEOO collaborates with the MEC Owner NFVO to provide a dedicated data plane to each MEC Customer. For the sake of example, we can assume that the MEC Owner offers to each MEC Customer a dedicated Kubernetes cluster, where each Kubernetes node is deployed as an 'ETSI VNF' (or VM) in the NFVI, which is connected to the 5G Core (5GC) via a dedicated data plane (MEC Customer network slice subnet). The second new functional block is the \textit{MEC Network Slice Manager} (MENSM), which delegates the life-cycle management of the 'ETSI VNFs' to a dedicated VNFM, while it is responsible for the management of the network slice subnet (data plane) parameters. For instance, it can reserve network bandwidth between MEC hosts for a given MEC Customer. More over, the MEC Network Slice Manager could behave like an 3GPP Application Function (AF) which interacts with the 5GC to synchronise data plane forwarding rules to realise local breakout traffics to/from MEC applications.

As previously discussed, each MEC Customer manages and orchestrates his own MEC application slices within the assigned virtualised MEC system. To this end, each MEC Customer implements a \textit{MEC Customer Orchestrator} (MECO), which receives the commands to create, modify or delete MEC application slices from a management function called MEC ASSMF (see Section~\ref{subsec:AS_Mgmt} below for more details). Furthermore, the MECO collaborates with the MEC Customer Platform Manager (MECPM) to manage the MEC application slice life-cycle and the MEC Platform instance (e.g. embodied as a containerised application). In order to stitch the application slice subnets to the adjacent network slice subnets, the MENSM creates dedicated VNFs (e.g. gateways) that it communicates to the MECO (or at least the gateway endpoints), similarly to~\cite{2021_TNSM_e22_slice_survey}. A collaboration between the MECO and the MEOO could also be needed in case of MEC application relocation to enforce new 5GC forwarding rules or to tear down old ones.

With regards to the interaction with the 5GC, there are two possible options:
%
\renewcommand\labelitemi{$\bullet$}
\begin{itemize}[noitemsep,topsep=2pt]
    \item The MEC Owner provides a network slice (i.e. a 'big pipe') to the MEC Customer, which directly manages via its MEC Platform (MEP) 5GC forwarding rules for each application slice (e.g. adds new DNS rules to the 5GC local DNS servers). This solution allows for better preserving privacy as the MEC Customer is the only entity that manages the data traffic produced by its own customers' UEs.
    \item The MEC Customer MEP collaborates (e.g. via the MECO and the MEOO) with the MEC Owner Network Slice Manager, which ultimately influences 5GC traffics. This solution allows for the MEC Customer to delegate the interaction with 5GC to the MEC Owner. The latter can aggregate requirements in order to optimise network resources (e.g. bandwidth). Thus, this solution allows for better network optimisation at the MEC Owner infrastructure but does not preserve privacy. Also, it may be less scalable as the number of UEs increases.
\end{itemize}
%
We conclude this section by noting that the standard MEC reference architecture~\cite{MEC003} entails a single MEC orchestrator controlling a single virtualisation infrastructure and managing the instantiation of all MEC applications. Our proposed MEC architecture variant implies a split of the MEC orchestrator responsibilities into a MEC Customer Orchestrator (MECO) and a MEC Owner Orchestrator (MEOO). While the former is responsible for MEC Platform, MEC applications, MEC application slices, and related external interfaces, the latter is responsible for the hardware, the NFVI, MEC NFVI slices (especially MEC network slices) and related external interfaces. 

%%
%%---------------------------------------------
\subsection{Application Slice Management Architecture}
\label{subsec:AS_Mgmt}
%%
\noindent
%
With the aforementioned new roles and architecture in mind, the 3GPP network slice management architecture could also be augmented to manage and orchestrate application slices as illustrated in Figure~\ref{fig:aps_mngmt}.
%
\begin{figure}[ht]
\centering
\includegraphics[clip,trim= 0cm 0cm 0cm 0cm,width=0.5\textwidth]{figures/3_FigureASMgmt.png}
\caption{3GPP-compatible joint network and application slice management architecture.}
\label{fig:aps_mngmt}
\end{figure}
%
We assume that an ASC relies on a web portal to request an application service with a given SLA from a catalogue of offered ASs (see Section~\ref{sec:implementation} for more details on how to implement such service catalogue). The business interaction of the web portal can happen in two manners depending on the AS/APS deployment model that is used in the system (presented in Section~\ref{subsec:AS_deployment}). In both cases, the web portal communicates with a new management function, called \textit{Application Service Management Function} (ASMF), which is responsible for translating the SLA of the requested AS to the SLS of an APS and to trigger the creation of the APS instance by contacting a new management function called \textit{Application Slice Management Function} (APSMF). The APSMF splits the APS into multiple subnets, one for each domain over which the requested APS spans, including possibly the network and the AS's endpoints, namely the UE requesting the AS, and the edge system instantiating the AS. To this end, we introduce the new \textit{Application Slice Subnet Management Functions} (APSSMFs), which apply the APSS life-cycle management commands within the two potential domains that are relevant for an application slice subnets, namely the UE and the MEC.

In the overlay model (label 1 in Figure~\ref{fig:aps_mngmt}), the E2E ASMF is also responsible for translating the E2E AS SLA into E2E CS SLA and for requiring the adapted E2E CS from the CSMF. In the stitching deployment model, two alternative management flows are feasible. In the first case (label 2 in Figure~\ref{fig:aps_mngmt}), the end customer is an ``expert'' user and he is directly responsible for breaking (using the web portal) the E2E AS SLA into an E2E CS SLA and an SLA with a scope restricted to AS's endpoints. Then, the web portal is used for requiring the adapted E2E CS from the CSMF. In the other case (label 3 in Figure~\ref{fig:aps_mngmt}), the end customer is not an expert of the network domain and he does not need to perform the aforementioned E2E AS SLA splitting. On the contrary, the the APSMF is responsible to communicate directly with the NSMF to manage the network slice subnet associated with the APS. Finally, we remind that the management of network slice via the NSMF and the one of per-domain network slice subnets via NSSMFs are well-defined by 3GPP, and they do not need to be extended~\cite{3GPPTS28531,3GPPTS28541}. 

%The life cycle management of the APS can happen in two manners depending on the APS deployment models that is used in the system (see Section~\ref{subsec:AS_deployment}). In the case of the stitching deployment model (label (2) in Figure~\ref{fig:aps_mngmt}), ASMF delegates to a new management function, called \textit{Application Slice Management Function} (APSMF) the deployment and management, in an end-to-end manner, of the APS associated with the requested AS. Specifically, the APSMF splits the APS into multiple subnets, one for each domain over which the APS spans, including both the network and the AS's endpoints, namely the UE requesting the AS, and the edge platform instantiating the AS. To this end, we introduce the new \textit{Application Slice Subnet Management Function} (APSSMF), which applies the APSMF's life-cycle management commands within the two potential domains that are relevant for an application slice subnets, namely the UE and the MEC. Finally, the APSMF communicates directly with the NSMF to manage the network slice subnet associated with the APS. 

%In the case of the overlay deployment model (label (1) in Figure~\ref{fig:aps_mngmt}), the APS is a consumer of the Communication Service offered through the underlying E2E network slice. Thus, the ASMF communicates directly with the CSMF to invoke the desired CS. However, the ASMF also needs to communicate with the ASPMF to manage the application slice subnets in the UE and MEC domains. We remind that the management of per-domain network slice subnets via NSSMFs is well-defined by 3GPP, and it does not need to be extended~\cite{3GPPTS28531,3GPPTS28541}. 

%\Thai{--- beginning : proposed ideas }
%\emph{Application Slice Subnet} vs \emph{Application Sub-Slice}: similarly to the network slice management operations, the E2E ASMF and the Edge ASSMF could pertain or not to the same operator. In the first case, the E2E ASMF splits the E2E application slice into domain-specific \emph{Application Slice Subnets} and provides to the related ASSMF the correspondent Application Slice Subnet template. In the second case, the domain-related \emph{Application Slice Subnet} is wrapped by the E2E ASMF in an \emph{Application Sub-Slice} which is in its turn wrapped in an Application Service. The E2E ASMF communicates, in this case, to the domain ASSMF an Application Service template (as per business-level relationship). 

%Similarly to ETSI NFV Network Service concept (Cf. section \ref{subsec:nw_slice_nw_service}), we define a \emph{ETSI-Like Application Service or ELAS} (not to make confusion with Application Service concept) as a composition of Application Component Functions (ACFs) and possibly of other ELAS (i.e. nested ELASes). 

%ACFs can be composed into ELAS in the form of Application Function Forwarding Graph (ACFFG) similarly to VNFFG described in section \ref{subsec:nw_slice_nw_service}. \Thai{For simplicity, we will call the ACFFG a Service Graph in the rest of the document?}
%\Thai{--- end : proposed ideas }

%
\begin{figure}[ht]
\centering
\includegraphics[clip,trim= 0cm 2.5cm 14cm 7.5cm,width=0.4\textwidth]{figures/fig_aps_mngmt_focus.pdf}
\caption{Illustration of the orchestration entities that are involved in the MEC domain.}
\label{fig:ps_mngmt_focus}
\end{figure}
%
In the remaining of this paper, we will describe our experience in implementing the management architecture described above. As shown in Figure~\ref{fig:ps_mngmt_focus}, several orchestration entities are involved in the management of the various application slice subnets. Our focus will be on implementing the MEC Customer Orchestrator using a popular open-source container orchestration platform. Furthermore, we will detail the interfaces and data models needed to interact with the MEC APSSMF, allowing the deployment of isolated MEC application slice subnets composed of ACFs in the form of Docker containers.  

%\Thai{Operationally, the AS descriptors are received by the MEC customer from a ...\newline
%Sharing of ACFs: concept of Network slice subnet which allows for the management of NFs independently of the network slice - a network slice subnet can be shared between two network slices. similarly, we could define an application slice subnet which is a set of ACFs}
