%%
%%---------------------------------------------
\section{\added{Related Work}}
\label{sec:related}
%%
\noindent
%
\added{Network service orchestration architectures~\cite{2017_JCSI_mano} and network slicing resource allocation~\cite{2020_ACCESS_slicing_survey,2020_ACCESS_slicing_survey2} have been deeply studied. However, network slices built to support Communication Services (i.e. data transmission service from endpoints to endpoints) are not meant to support Applications Services (i.e. data processing services). Consequently, they are not well-adapted to take into account performance requirements related to the application level especially processing and storage latencies related to applications. Finally, how to reconcile the resource partition at both the network and application levels within the MEC is still an open challenge. 

Initial research works on MEC-assisted network slicing in 5G networks, have focused on IoT tenants. For instance, the authors in~\cite{2107_globecom_slaas} propose an architecture where IoT tenants deploy centralised IoT Brokers to manage IoT data collection and to negotiate with a 5G Network Slice Broker the slice resources that are needed to satisfy the QoS requirements coming from their applications. 
%However, how to instantiate such architecture within the MEC standard is not specified. 
A hierarchical two-level management and orchestration architecture for end-to-end slicing is described in~\cite{2107_globecom_slaas}, where a top-level centralised end-to-end orchestrator (EO) controls domain-specific orchestrators of RAN, MEC and CN. A prototype of an IoT Slice Orchestrator that enables the definition and deployment of IoT slices across multiple domains is described in~\cite{2019_SENSORS_iot_slice}, but the focus is mainly on multi-tenancy aspects at the edge and cloud computing domains. The authors in~\cite{2021_JIOT_iot_slice} introduce the concept of virtual IoT slice services (vIoT), namely IoT common service functions that are provided to MEC-enabled IoT applications in the form of VNFs. Then, the focus is on the design of elastic edge computing policies for the autoscaling of vIoT resources. 

The most relevant works to our study are~\cite{2020_COMMAG_cross_slice,Cominardi2020,2020_MNET_MEC_subslice,2022_icc_edge_slice}. The authors in~\cite{2020_MNET_MEC_subslice} propose a MANO-based architecture for optimised inter-slice communication, and extend the MEC application descriptors (AppD) for the description of service attributes. An orchestrator framework is also presented in~\cite{2022_icc_edge_slice} that orchestrates and
manages the deployment of micro-services based on cloud-native MEC Applications as sub-slices at the edge. The focus of this paper is also to extend the MEC AppD to define an Edge Sub-Slice Template (ESST) describing edge-sub-slices. End-to-end multi-slice and multi-tenant support in the current MEC architecture is investigated in~\cite{Cominardi2020}, and two solutions are proposed. The fist one introduces a new slice control function (SCF) to allow interactions between the OSS of the 5G mobile network and the MEC application orchestrator, enabling slice-aware MEC App allocation policies. The second one is a platform registry to enable the self-discovery of independent edge slices installed on the same MEC facilities from different tenants, enabling inter-slices communication. The authors in~\cite{2020_MNET_MEC_subslice} propose a multi-tenant network slicing orchestration/management architecture similar to the one described in this study, with the introduction of a MEC NSSMF. However, the focus is to provide standard-compliant solutions to ensure isolation of traffic and MEC services that are ``private'' to each slice. 
As compared with these previous works, this study proposes a clear distinction between the network slice resources (e.g. VNF) and the application slice resources (e.g. ACF) following the separation of concern principle. 
%Indeed, we are dealing with two ecosystems which coexist within the MEC, namely the network ecosystem which is traditionally very standardised and regulated and the application ecosystem which is more open, but still there is a need for some standardisation as per inter-operability requirements (e.g. MEC apps relocation between different MEC systems). 
Furthermore, we propose a complete two-layer hierarchical slicing of the MEC domain where not only the MEP Platform Manager, but also the MEC Orchestrator is sliced.}
%
\vspace{-0.2cm}