%% 
%% Copyright 2007-2020 Elsevier Ltd
%% 
%% This file is part of the 'Elsarticle Bundle'.
%% ---------------------------------------------
%% 
%% It may be distributed under the conditions of the LaTeX Project Public
%% License, either version 1.2 of this license or (at your option) any
%% later version.  The latest version of this license is in
%%    http://www.latex-project.org/lppl.txt
%% and version 1.2 or later is part of all distributions of LaTeX
%% version 1999/12/01 or later.
%% 
%% The list of all files belonging to the 'Elsarticle Bundle' is
%% given in the file `manifest.txt'.
%% 

%% Template article for Elsevier's document class `elsarticle'
%% with numbered style bibliographic references
%% SP 2008/03/01
%%
%% 
%%
%% $Id: elsarticle-template-num.tex 190 2020-11-23 11:12:32Z rishi $
%%
%%
%%\documentclass[preprint,12pt]{elsarticle}

%% Use the option review to obtain double line spacing
%% \documentclass[authoryear,preprint,review,12pt]{elsarticle}

%% Use the options 1p,twocolumn; 3p; 3p,twocolumn; 5p; or 5p,twocolumn
%% for a journal layout:
%% \documentclass[final,1p,times]{elsarticle}
%% \documentclass[final,1p,times,twocolumn]{elsarticle}
%% \documentclass[final,3p,times]{elsarticle}
%% \documentclass[final,3p,times,twocolumn]{elsarticle}
\documentclass[5p,times]{elsarticle}
%% \documentclass[final,5p,times,twocolumn]{elsarticle}

\usepackage{microtype}

\usepackage{multirow}

%\usepackage{flushend}

\usepackage{caption}
\usepackage{subcaption}

\usepackage{url}

%% For including figures, graphicx.sty has been loaded in
%% elsarticle.cls. If you prefer to use the old commands
%% please give \usepackage{epsfig}

%% The amssymb package provides various useful mathematical symbols
\usepackage{amssymb}
%% The amsthm package provides extended theorem environments
\usepackage{amsthm}

%% The lineno packages adds line numbers. Start line numbering with
%% \begin{linenumbers}, end it with \end{linenumbers}. Or switch it on
%% for the whole article with \linenumbers.
\usepackage[switch]{lineno}

\usepackage{enumitem}
\setlist[itemize]{align=parleft,left=0pt..1em}

%% Set comments per author
\usepackage{soul}
\usepackage[dvipsnames]{xcolor}

%\usepackage[draft]{changes}
\usepackage[final]{changes}

\DeclareRobustCommand{\Thai}[1]{ {\begingroup\sethlcolor{BurntOrange}\hl{(Thai:) #1}\endgroup} }
\DeclareRobustCommand{\Simone}[1]{ {\begingroup\sethlcolor{BurntOrange}\hl{(Simone:) #1}\endgroup} }
\DeclareRobustCommand{\Raffaele}[1]{ {\begingroup\sethlcolor{BurntOrange}\hl{(Raffaele:) #1}\endgroup} }

\journal{Future Generation Computer Systems}

\begin{document}

\fbox{\begin{minipage}[b][1cm][c]{18cm}
\footnotesize This article has been accepted for publication in the Future Generation Computer Systems, Elsevier. This is the author's version which has not been fully edited and content may change prior to final publication. Citation information: \url{https://doi.org/10.1016/j.future.2022.05.027}
\end{minipage}}

\begin{frontmatter}

%% Title, authors and addresses

%% use the tnoteref command within \title for footnotes;
%% use the tnotetext command for theassociated footnote;
%% use the fnref command within \author or \address for footnotes;
%% use the fntext command for theassociated footnote;
%% use the corref command within \author for corresponding author footnotes;
%% use the cortext command for theassociated footnote;
%% use the ead command for the email address,
%% and the form \ead[url] for the home page:
%% \title{Title\tnoteref{label1}}
%% \tnotetext[label1]{}
%% \author{Name\corref{cor1}\fnref{label2}}
%% \ead{email address}
%% \ead[url]{home page}
%% \fntext[label2]{}
%% \cortext[cor1]{}
%% \affiliation{organization={},
%%             addressline={},
%%             city={},
%%             postcode={},
%%             state={},
%%             country={}}
%% \fntext[label3]{}

\title{Towards End-to-End Application Slicing in Multi-access Edge Computing systems: Architecture Discussion and Proof-of-Concept}

%% use optional labels to link authors explicitly to addresses:
%% \author[label1,label2]{}
%% \affiliation[label1]{organization={},
%%             addressline={},
%%             city={},
%%             postcode={},
%%             state={},
%%             country={}}
%%
%% \affiliation[label2]{organization={},
%%             addressline={},
%%             city={},
%%             postcode={},
%%             state={},
%%             country={}}

\author[cnr]{Simone Bolettieri}

\affiliation[cnr]{organization={Institute of Informatics and Telematics (IIT) - Italian National Research Council (CNR)},
            addressline={Via G. Moruzzi 1}, 
            city={Pisa},
            postcode={56124}, 
            country={Italy}}

\author[nokia]{Dinh Thai Bui}
\author[cnr]{Raffaele Bruno}

\affiliation[nokia]{organization={Nokia Bell Labs},
            addressline={1 route de Villejust}, 
            city={Nozay},
            postcode={91620}, 
            country={France}}

\begin{abstract}
%% Text of abstract
Network slicing is one of the most critical 5G pillars. It allows for sharing a 5G infrastructure among different tenants leading to improved service customisation and increased operators' revenues. Concurrently, introducing the Multi-access Edge Computing (MEC) into 5G to support time-critical applications raises the need to integrate this distributed computing infrastructure to the 5G network slicing framework. Indeed, end-to-end latency guarantees require the end-to-end management of slice resources. For this purpose, after discussing the main gaps in the state-of-the-art with regards to such an objective, we propose a novel slicing architecture that enables the management and orchestration of slice segments that span over all the domains of an end-to-end application service, including the MEC. We also show how this general management architecture can be instantiated into a multi-tenant MEC infrastructure. A preliminary implementation of the proposed architecture focusing on the MEC domain is also provided, together with performance tests to validate the feasibility and efficacy of our design approach.
\end{abstract}

%%Graphical abstract
%%\begin{graphicalabstract}
%%\includegraphics{grabs}
%%\end{graphicalabstract}

%%Research highlights
%\begin{highlights}
%\item Research highlight 1
%\item Research highlight 2
%\end{highlights}

\begin{keyword}
%% keywords here, in the form: keyword \sep keyword
edge computing \sep MEC \sep NFV \sep 3GPP network slicing \sep latency-sensitive applications
%% PACS codes here, in the form: \PACS code \sep code
%%\PACS 0000 \sep 1111
%% MSC codes here, in the form: \MSC code \sep code
%% or \MSC[2008] code \sep code (2000 is the default)
%%\MSC 0000 \sep 1111
\end{keyword}

\end{frontmatter}

%\linenumbers

%% main text
%%
\section{Introduction}\label{sec:intro}

In computational finance, Monte Carlo simulations are used extensively to estimate the expected value of financial payoffs based on the solution of stochastic differential equations (SDEs) which model the evolution of stock prices, interest rates, exchange rates and other quantities \cite{glasserman04}.  Monte Carlo methods are very general and flexible, but for high accuracy it requires generating a large number of costly SDE path approximations, which has motivated research into a number of variance reduction or, equivalently, cost reduction techniques. One such method is
Multilevel Monte Carlo (MLMC), which was proposed in \cite{GILES2008} and was adapted for various applications that are summarised in \cite{Giles_overview17} and successfully combined with other methods such as quasi-Monte Carlo methods. The main idea of MLMC is to approximate the payoff using different time stepping resolutions when numerically solving the underlying SDE and to generate an optimal number of samples on each level, such that the overall computational cost is minimised subject to the desired bound on the variance. %, such that the total computational cost is minimised. 
The computational savings come from the fact that most samples are computed on the coarser levels and hence are less expensive while only a few samples from the finest levels are required \cite{GILES2008}.


Among the directions in which the computational cost 
of MLMC methods could further be reduced, an important avenue is the use of lower precision calculations, especially for the first Monte Carlo levels where the targeted accuracy is relatively low. 
 An overview of the research on mixed precision for the standard Monte Carlo (MC) framework is provided in \cite{ChowMixedPrecisionStandardMC} but only a few references study the potential of low precision computation in the MLMC framework \cite{Rounding_error_oliver}. To the best of our knowledge, the only MLMC framework with customised precision in the literature is \cite{brugger2014mixed}, but they use a uniform precision for all operations on each Monte Carlo level instead of optimising 
 the precision of each intermediary variable to reduce as much as possible the cost of path generation.
 
An important motivation for an MLMC framework with variable precision would be performing the low precision computations on reconfigurable hardware devices such as Field Programmable Gate Arrays (FPGAs). FPGAs contain customizable logic blocks and connectors that make it easy to adapt the digital circuit architecture for a specific application, leading to a highly parallel and optimised implementation. Therefore they are successfully exploited in applications that require high speed and have high computational workload, such as signal processing \cite{woods2008fpga}, and real time applications like high frequency trading \cite{HFT1,HFT2}. That is why a number of previous works in hardware architecture design implemented the MLMC algorithm to price financial options using FPGAs as accelerators, which resulted in improved speed and power efficiency compared to full CPU architectures \cite{Schryver2013AMM}. The paper \cite{lindsey2016domain} also proposed 
a Domain Specific Language to automate the configuration of FPGAs for this specific application. However, only \cite{brugger2014mixed} proposed a heuristic to reduce the precision in calculations.

In addition, all aforementioned works considered that the random number generation (RNG) is performed in single or double precision. Yet in most cases an important portion of the workload in the overall MLMC simulation comes from the RNG and in \cite{brugger2014mixed} this limited the total computational savings.
To reduce the cost of MLMC simulations in particular those based on the Geometric Brownian Motion (GBM), \cite{approximateICDF_Oliver, NestedOliver} have proposed to use approximate random numbers that are generated by applying an approximation of the inverse CDF to uniform random numbers. In \cite{NestedOliver}, the authors proposed a way to integrate these lower precision random variables into a \textit{nested} MLMC framework and completed a numerical analysis to bound the resulting error at each MC level by a product of the time step and the error in the random number approximation. The same authors show in \cite{approximateICDF_Oliver} that using approximate random variables reduces the cost of path generation by a factor 7.


In this paper we propose a nested MLMC framework that combines the use of approximate random normal variables and lower precision calculations to reduce the computational cost of MLMC even further than \cite{brugger2014mixed,NestedOliver}. We illustrate the efficiency of our framework in Matlab, after making several assumptions on the cost of operations and size of the errors that we carefully justify. We focus on the case of GBM and use the approximate RNG methods presented in \cite{approximateICDF_Oliver} as well as a new slightly modified method that combines CDF inversion and the central limit theorem. To choose the precision of the variables in the low precision path generation, we introduce a novel method to optimise the bit-widths. This optimisation is performed before the main path generation loop is executed and is based on a linear model of the payoff error  
due to rounding when computing in low precision. The error model relies on algorithmic differentiation in a similar manner to \cite{unifying-bwoptim,bitwidth-AD,ADAPT}. The bit-width optimisation procedure can be performed off-line, so this stage can be excluded from the on-line time complexity of our framework. The user specified desired accuracy is then enforced by calculating on-line the number of samples that need to be generated.

In terms of hardware design, we suggest implementing the low precision path generation on FPGAs and the full-precision ones on a CPU or GPU. 
The FPGA offers enough flexibility to define a separate bit-width for every variable in the low precision path generation, and can be reconfigured periodically to update the bit-widths when the market parameters have changed considerably. 


The paper is organized as follows : \Cref{sec:MLMC} introduces MLMC and nested MLMC to make clear the estimator that is implemented in our framework. Then in \Cref{sec:RNG} we detail the methods that could be used to obtain approximate random normally distributed numbers very cheaply for the low precision path generation. In \Cref{sec:error_model} and \Cref{sec:costModel} we propose an error model and a cost model (resp.) that we then use to formulate the optimisation problem that is solved to obtain the optimal bit-widths of fixed point variables in \Cref{sec:optimisation}. Finally we summarise our results and future directions in \Cref{sec:conclusion}.



\section{Background}
\label{sec:background}


\subsection{Preliminaries}

{\color{red}[TODO: LLMs? in-context learning?]}

\subsection{Problem Definition}

{\color{red}[TODO: define the problem of citation intent]}

%%
%%---------------------------------------------
\section{An E2E Network and Application Slicing Architecture using MEC}
\label{sec:app_slice}
%%
\noindent
In this section, we first introduce the concepts of application services and application slices, and we instantiate these concepts in the context of MEC environments. Then, we discuss application slice from both operational and business perspectives by introducing operation and business roles, respectively, and relationships of the different entities involved in application slicing and service provisioning. We also discuss the different architectural use cases that correspond to the proposed model. Finally, we elaborate on an orchestration/management architecture for 3GPP networks integrating MEC systems that will enable such new use cases, and we propose an extended MEC reference architecture required to support the envisioned management architecture.   
%%
%%---------------------------------------------
\subsection{Preliminary concepts}
\label{sec:pre_concepts}
%%
\noindent
As it could be sensed from both the introduction and the background sections, the 3GPP CSs cannot be straightforwardly used to model the services offered by an edge computing platform. For the sake of disambiguation, in this work, we call the services offered to the customers of an edge computing platform as \textit{Application Services} (ASs). The main reason is that CSs and ASs are designed to support different business purposes, therefore, they are regulated by different SLA frameworks. Specifically, \textit{CSs are dedicated to the transmission of data and network traffic}, while \textit{ASs are designed to support services which are generally not communication oriented in essence, but focus on data processing}, such as Augmented Reality (AR), Virtual Reality (VR), Ultra-High Definition (UHD) video, real-time control, etc. From the management perspective, it would not be sensible to manage ASs with the existing 3GPP CSMF. Some extensions or new management functions should be added to the overall management framework to translate application-related SLAs into application-related SLSs and network-related SLSs. 

In analogy to the 3GPP network slice concept, we define an \textit{Application Slice} (APS) as a \textit{a virtualised data collection system and a set of data processing resources that provides specific processing functions to support an Application Service and the associated SLA. The APS can possibly include Network Functions to ensure correct data communication among the previous processing functions}. A key aspect of the application slice concept is the support of \textit{isolation} not only of the resources that are used by each application slice but also of the management data and functions. Similarly to a network slice, the application slice is an end-to-end concept and can span multiple domains (e.g., device, edge platform, network). Furthermore, an application slice is not a monolithic entity. Following the well-known principles of service-oriented architectures, it is built as an \textit{atomic data-processing function} which has well defined functional behaviours and interfaces, called in this work \emph{Application Component Functions} (ACFs). We can now define an \textit{Application Slice Subnet} (APSS) as a set of at least one ACF and possible NFs supporting an application slice. It is essential to point out that application slices and network slices are not independent concepts, but they are strongly intertwined as application slices deployed at the edge of the network require dedicated network slices to deliver the massive amounts of data and network traffic that application services will typically produce. In the following sections, we will also discuss two potential distinct models to allow an application slice to use a network slice.  

The general concepts that we have introduced so far can be easily instantiated in the context of MEC, NFV and 5G systems. First of all, a MEC platform can be used to offer ASs, which can be deployed as conventional MEC applications and implemented using a collection of ACFs. Furthermore, the MEC system can be considered as one of the domains over which an application slice can span. Thus, the MEC management layer should be responsible for the orchestration and deployment of one of the application slice subnets that compose the E2E APS, hereafter denoted by \textit{MEC application slice subnet}. It is important to point out that a MEC application slice subnet includes one or more ACFs, but it can also include VNFs to support network functionalities. Furthermore, the tenant of a MEC system is not necessarily an end customer, but the MEC system can offer its services following a MEC-Platform-as-a-Service model.  

Finally, it is worth pointing out the differences between VNFs and ACFs which are designed for different purposes - network traffic processing for the former and data processing for the latter. Furthermore, in a multi-tenant MEC environment, like the one in~\cite{Cominardi2020} or~\cite{2020_MNET_MEC_subslice}, it is likely that the MEC applications implementing ACFs will not be orchestrated by the same entity that would orchestrate principal VNFs. Nevertheless, there are a lot of common points in terms of operation and management between a VNF and an ACF as both rely on the same set of virtualisation technologies for their deployment. Thus, we will reuse and extend both 3GPP management architecture for network slice and MEC-in-NFV architecture to support the proposed E2E application slice framework.
%
%\Thai{-- beginning: definitions added --}
%\newline Similarly to the 3GPP network slice framework, we can define successively:
%\begin{itemize}
%    \item An Application Slice Subnet as a set of at least one ACF. This set could include VNFs.
%    \item An Application Slice consists of multiple Application Slice Subnets augmented by SLS parameters (e.g. QoS parameters). The latter are technology-dependent objectives that should be enforced to meet the SLA.
%    \item An Application Service is built from one or multiple Application Slices augmented by a common SLA parameter set (i.e. business commitment)
%\end{itemize}
%\Thai{-- end: definitions added --}
%%
%%---------------------------------------------
\subsection{High-level roles for application slice management}
\label{subsec:roles}
%%
\noindent
From the APS management (and orchestration) perspective, we first need to define high-level roles in order to draw responsibilities boundaries similarly to what was proposed for 5G network slice management (see section~\ref{subsec:mgmt_architecture}). As discussed above, our focus is on Application Services that are offered by a MEC system. The Figure~\ref{fig:AS_roles} shows the different roles identified:
%
\renewcommand\labelitemi{$\bullet$}
\begin{itemize}[noitemsep,topsep=2pt]
    \item \textit{Application Service Consumer} (\textit{ASC}): uses Application Services.
    \item \textit{Application Service Provider} (\textit{ASP}): provides Application Services that are built and operated upon one or multiple application slice subnets, including MEC application slice subnets. Each of those application slice subnets is in turn built from a set of ACFs provided by the \emph{Application Component Function Supplier}.
    \item \textit{MEC Operator} (\textit{MOP}): operates and manages ACFs using a MEC Platform. We assume that the MEC Operator implements the MEC orchestrator and the ETSI MEC standardised interfaces as presented in~\cite{MEC003}. It designs, builds and operates its MEC platform to offer, together with the MEC orchestrator, MEC application slice subnets to ASPs from ACFs that the ASP has provided as an input.
    \item \textit{Application Component Function Supplier}: designs and provides ACFs to ASPs.
\end{itemize}
%
\begin{figure}[t]
\centering
\includegraphics[width=0.4\textwidth]{figures/3_FigureARoles.png}
\caption{High-level functional roles in the MEC application slice framework}
\label{fig:AS_roles}
\end{figure}
%
Interestingly, an ASC can become an ASP while adding new ACFs into consumed ASs and providing new ASs. For instance, let us assume that ASC $C_1$ uses from ASP $P_1$ an AS $S_1$ that provides face recognition capabilities. Then, ASC $C_1$ can integrate into $S_1$ other ACFs that permit to retrieve video stream from customers' cameras and ACFs that take control of customers' door, thus building a new AS $S_2$, which provides automatic door opening based on face recognition. ASC $C_1$ then becomes a new ASP $P_2$ selling the previous new AS to customers such as house/apartment rental platforms. Furthermore, we can easily make a parallel between Figure~\ref{fig:AS_roles} and Figure~\ref{2_figRoles} in Section~\ref{subsec:mgmt_architecture}. Roles responsible for low layers, namely Hardware Supplier, Data Center Service Provider, NFVI Supplier, and Virtualisation Infrastructure Service Provider, remain the same. However, the Application Component Function Supplier has replaced the VNF (or Equipment) Supplier; the MEC Operator has replaced the Network Operator, the Application Service Provider (ASP) the CSP, and the Application Service Customer (ASC) the CSC. 

It is helpful to point out that a business organisation can play one or multiple operational roles simultaneously. Therefore, without trying to be exhaustive in terms of business models, we focus in our work on two categories of organisations whose responsibility boundaries appear to make most of the sense as per our knowledge:
%
\renewcommand\labelitemi{$\bullet$}
\begin{itemize}[noitemsep,topsep=2pt]
    \item \textbf{\textit{MEC Owner}}: it plays both Hardware supplier and NFVI supplier roles. The fact that the MEC Owner also manages the virtualisation infrastructure and the edge servers allows him/her to dynamically split his/her infrastructure into logical partitions or network slices (i.e. greater degree of flexibility). It is noted that the MEC Owner could be the equivalent of what is called the \textit{MEC Host Operator} in \cite{MEC027}, as it offers virtualised MEC hosts and MEC platforms to its tenants. However, we prefer the `MEC Owner' terminology to avoid confusion with the `MEC Operator' role.
    \item \textbf{\textit{MEC Customer}}: plays both the role of the MEC Operator and ASP. It is helpful to point out that the ASP role offers a business interface (with the ASC). In contrast, the MEC Operator role offers the actual enforcement/implementation (i.e. SLS) of the business objectives (i.e. SLA) agreed across the ASP business interface. It is noted that the MEC Operator role alone cannot endorse a business organisation as it only offers Application Slice (including SLS) and not the Application Service (with related SLA). 
\end{itemize}
%
In conclusion, from a business perspective in this study, we advocate a multi-tenancy model in which ASCs are tenants of a MEC Customer, and MEC Customers are tenants of a MEC Owner. As explained in the following section, our proposed multi-tenant MEC system supports data isolation (through separated data planes) and resource orchestration separation (through separate resource orchestrators) between tenants.  
%%
%%---------------------------------------------
\subsection{Deployment models for an E2E application slice}
\label{subsec:AS_deployment}
%%
\noindent
%
We distinguish two distinct deployment models for our proposed application slice concept: $(i)$ the \textit{overlay} model, and $(ii)$ the \textit{stitching} model. 

\begin{figure}[ht]
\centering
\includegraphics[width=0.5\textwidth]{figures/3_DeploymentModels.png}
\caption{\added{Interconnection models between application slice and network slice}}
\label{3_DeploymentModels}
\end{figure}
%
%\begin{figure}[ht]
%\centering
%    \begin{subfigure}[b]{0.5\textwidth}
%         \centering
%         \includegraphics[width=0.8\textwidth]{figures/3_FigureAS.png}
%         \caption{Network slice includes the data plane of the MEC customer}
%         \label{3_figAS}
%     \end{subfigure}
%     \hfill
%    \begin{subfigure}[b]{0.5\textwidth}
%         \centering
%         \includegraphics[width=0.8\textwidth]{figures/3_FigureASoverlay.png}
%         \caption{Network slice does not include the data plane of the MEC customer}
%         \label{3_figASoverlay}
%     \end{subfigure}
%\caption{Overlay deployment model: the application slice is a consumer of an E2E network slice.}
%\label{fig:overlay}
%\end{figure}
%
The first model assumes that the E2E APS is a consumer of a Communication Service offered by the underlying network slice (see Figure~\ref{3_DeploymentModels}\added{a}). In this case, the network slice is responsible for the entire communication latency, and we denote it as E2E \added{network} slice. In addition to the latter, the application slice caters for processing and possibly storage latency at both ends. It is important to point out that the E2E network slice encompasses not only the 5G network slice but also the network slice subnets within the MEC Owner and MEC Customer domains\footnote{Note that we use the term MEC Customer network slice subnet instead of MEC application slice subnet because, in this deployment model, the MEC Customer is only responsible for the management and orchestration of the data plane within its virtualised MEC environment. This is also justified by the need to align the architectural component boundaries with the liability boundaries of the deployment model.}. Indeed, as shown in Figure~\ref{3_DeploymentModels}a our slicing framework leverages recent advantages on virtualisation technologies that allow a virtualisation layer to be composed of multiple nested sub-layers, each using a different virtualisation paradigm. According to the functional role split illustrated in Section~\ref{subsec:roles}, a MEC Owner can use a hypervisor technology to operate its MEC hosts and to deploy multiple virtualised environments to MEC customers (e.g. allocating one or more VMs\footnote{In ETSI NFV terminology, a VM is also designated as `NFV VNF'.} using an NFVI). Each virtualised environment includes a full-fledged MEC system used by the MEC Customer to allocate further the resources assigned to its VMs to the multiple Application Services it deploys to its users by applying internal orchestration and management criteria. In the case of the MEC customer, a container-based virtualisation technology could be used as a top virtualisation sub-layer to manage application deployment into its allocated virtualised MEC system. It is worth noting that in \added{Figure~\ref{3_DeploymentModels}a} the E2E network slice includes the data plane of the MEC customer. However, an alternative overlay model is also possible in which the E2E network slice terminates at the network boundary between the MEC Owner and the MEC Customer as per \added{Figure~\ref{3_DeploymentModels}b}.

In the stitching deployment model \added{(see Figure~\ref{3_DeploymentModels}c)}, we assume that the MEC application slice subnet is a peer of the network slice subnets. Virtual appliances using a subnet border API, such as a virtual gateway or virtual load-balancer, can be used to interconnect MEC application slice subnets to the adjacent network slice subnet, similarly to what was proposed in~\cite{2020_MNET_MEC_subslice} and \cite{2021_TNSM_e22_slice_survey}. Such stitching could be a one-to-one interconnection as well as a multiple-to-one interconnection. The end-to-end application slice could be seen in this case as the composition of different application slice subnets (UE-operated or MEC-operated) together with network slice subnets. Latency-wise, the MEC application slice subnet is responsible, in this case, for a tiny part of the network latency budget in addition to the processing and storage latency induced by the MEC applications and their related ACFs.
%
%\begin{figure}[ht]
%\centering
%\includegraphics[width=0.4\textwidth]{figures/3_FigureASubS.png}
%\caption{Stitching deployment mode: interconnection of the different slice subnets to form the E2E application slice.}
%\label{3_figASstitch}
%\end{figure}
%

We conclude this section by noting that the different deployment models will lead to different approaches to combine our proposed E2E application slice management/orchestration framework with the 3GPP management architecture, as explained in Section~\ref{subsec:AS_Mgmt}.

%Alternatively, the MEC application slice subnet network connectivity could be performed as an overlay network on top of the end-to-end network slice (see Figure \ref{3_figASoverlay}). This architecture is then equivalent to the one illustrated in Figure \ref{3_figAS} except that the end-to-end network slice terminates this time at the network boundary between the MEC Owner and the MEC Customer.
%
%\begin{figure}[ht]
%\centering
%\includegraphics[width=0.4\textwidth]{figures/3_FigureASoverlay.png}
%\caption{Application slice as overlay on top of E2E network slice}
%\label{3_figASoverlay}
%\end{figure}

%%
%%---------------------------------------------
\subsection{Architecture for application slicing in a multi-tenant MEC system}
\label{subsec:new_MEC_arch}
%%
\noindent
%
In this section, we elaborate on the new MEC components and extensions to the current MEC management architecture that are needed to support E2E application slicing and multi-tenancy within multiple MEC customers. Figure~\ref{fig:ext_arch} shows an illustrative example of the proposed extended MEC reference architecture. The primary design rationale of our proposal is that the MEC system should be split into two responsibility domains following a \textit{two-layer hierarchical MEC architecture}, where the bottom layer is managed and orchestrated by the MEC Owner, and the top-layer is independently managed and orchestrated by MEC Customers. Such a hierarchical architecture allows a single MEC deployment to host multiple MEC Customers. Each of them has his own MEC network slice subnet (i.e. his dedicated data plane provided by the MEC Owner) with related management capability. In turn, each MEC Customer manages and orchestrates his own MEC application slices. 
%
\begin{figure*}[ht]
\centering
\includegraphics[clip,trim= 0cm 0cm 0cm 0cm,width=0.8\textwidth]{figures/3_FigureMECCustOwner.png}
\caption{Multi-tenant MEC architecture supporting network and application slicing}
\label{fig:ext_arch}
\end{figure*}
%

Implementation-wise, the proposed two-layer MEC architecture is enabled by the nested virtualisation capability of the MEC infrastructure, as anticipated in Section~\ref{subsec:AS_deployment}. In the system illustrated in Figure~\ref{fig:ext_arch}, the MEC Owner does not deploy individual MEC entities as in the MEC-in-NFV reference architecture, but a collection of 'ETSI VNFs' (or VMs) to provide each MEC Customer with a complete MEC system. Differently from~\cite{Cominardi2020} we do not use 'ETSI VNFs' to deploy MEC applications and MEC platforms but to deploy a virtualised MEC environment encompassing virtualised MEC hosts and a virtualised MEC management system. Furthermore, in our MEC-in-NFV architecture variant, we introduce functional blocks that substitute the \added{MEAO} and \added{MEPM-V} of the original reference architecture. Specifically, we substitute the \added{MEAO} with the \textit{MEC Owner Orchestrator} (MEOO), which is in charge of implementing the policies to select the MEC infrastructures on which to deploy a MEC Owner network slice subnet. As explained in Section~\ref{subsec:AS_Mgmt}, the MEOO receives the commands to create, modify or delete a MEC Owner network slice subnet from a 3GPP management function called MEC NSSMF. Furthermore, the MEOO collaborates with the MEC Owner NFVO to provide a dedicated data plane to each MEC Customer. For the sake of example, we can assume that the MEC Owner offers to each MEC Customer a dedicated Kubernetes cluster, where each Kubernetes node is deployed as an 'ETSI VNF' (or VM) in the NFVI, which is connected to the 5G Core (5GC) via a dedicated data plane (MEC Customer network slice subnet). The second new functional block is the \textit{MEC Network Slice Manager} (MENSM), which delegates the life-cycle management of the 'ETSI VNFs' to a dedicated VNFM, while it is responsible for the management of the network slice subnet (data plane) parameters. For instance, it can reserve network bandwidth between MEC hosts for a given MEC Customer. More over, the MEC Network Slice Manager could behave like an 3GPP Application Function (AF) which interacts with the 5GC to synchronise data plane forwarding rules to realise local breakout traffics to/from MEC applications.

As previously discussed, each MEC Customer manages and orchestrates his own MEC application slices within the assigned virtualised MEC system. To this end, each MEC Customer implements a \textit{MEC Customer Orchestrator} (MECO), which receives the commands to create, modify or delete MEC application slices from a management function called MEC ASSMF (see Section~\ref{subsec:AS_Mgmt} below for more details). Furthermore, the MECO collaborates with the MEC Customer Platform Manager (MECPM) to manage the MEC application slice life-cycle and the MEC Platform instance (e.g. embodied as a containerised application). In order to stitch the application slice subnets to the adjacent network slice subnets, the MENSM creates dedicated VNFs (e.g. gateways) that it communicates to the MECO (or at least the gateway endpoints), similarly to~\cite{2021_TNSM_e22_slice_survey}. A collaboration between the MECO and the MEOO could also be needed in case of MEC application relocation to enforce new 5GC forwarding rules or to tear down old ones.

With regards to the interaction with the 5GC, there are two possible options:
%
\renewcommand\labelitemi{$\bullet$}
\begin{itemize}[noitemsep,topsep=2pt]
    \item The MEC Owner provides a network slice (i.e. a 'big pipe') to the MEC Customer, which directly manages via its MEC Platform (MEP) 5GC forwarding rules for each application slice (e.g. adds new DNS rules to the 5GC local DNS servers). This solution allows for better preserving privacy as the MEC Customer is the only entity that manages the data traffic produced by its own customers' UEs.
    \item The MEC Customer MEP collaborates (e.g. via the MECO and the MEOO) with the MEC Owner Network Slice Manager, which ultimately influences 5GC traffics. This solution allows for the MEC Customer to delegate the interaction with 5GC to the MEC Owner. The latter can aggregate requirements in order to optimise network resources (e.g. bandwidth). Thus, this solution allows for better network optimisation at the MEC Owner infrastructure but does not preserve privacy. Also, it may be less scalable as the number of UEs increases.
\end{itemize}
%
We conclude this section by noting that the standard MEC reference architecture~\cite{MEC003} entails a single MEC orchestrator controlling a single virtualisation infrastructure and managing the instantiation of all MEC applications. Our proposed MEC architecture variant implies a split of the MEC orchestrator responsibilities into a MEC Customer Orchestrator (MECO) and a MEC Owner Orchestrator (MEOO). While the former is responsible for MEC Platform, MEC applications, MEC application slices, and related external interfaces, the latter is responsible for the hardware, the NFVI, MEC NFVI slices (especially MEC network slices) and related external interfaces. 

%%
%%---------------------------------------------
\subsection{Application Slice Management Architecture}
\label{subsec:AS_Mgmt}
%%
\noindent
%
With the aforementioned new roles and architecture in mind, the 3GPP network slice management architecture could also be augmented to manage and orchestrate application slices as illustrated in Figure~\ref{fig:aps_mngmt}.
%
\begin{figure}[ht]
\centering
\includegraphics[clip,trim= 0cm 0cm 0cm 0cm,width=0.5\textwidth]{figures/3_FigureASMgmt.png}
\caption{3GPP-compatible joint network and application slice management architecture.}
\label{fig:aps_mngmt}
\end{figure}
%
We assume that an ASC relies on a web portal to request an application service with a given SLA from a catalogue of offered ASs (see Section~\ref{sec:implementation} for more details on how to implement such service catalogue). The business interaction of the web portal can happen in two manners depending on the AS/APS deployment model that is used in the system (presented in Section~\ref{subsec:AS_deployment}). In both cases, the web portal communicates with a new management function, called \textit{Application Service Management Function} (ASMF), which is responsible for translating the SLA of the requested AS to the SLS of an APS and to trigger the creation of the APS instance by contacting a new management function called \textit{Application Slice Management Function} (APSMF). The APSMF splits the APS into multiple subnets, one for each domain over which the requested APS spans, including possibly the network and the AS's endpoints, namely the UE requesting the AS, and the edge system instantiating the AS. To this end, we introduce the new \textit{Application Slice Subnet Management Functions} (APSSMFs), which apply the APSS life-cycle management commands within the two potential domains that are relevant for an application slice subnets, namely the UE and the MEC.

In the overlay model (label 1 in Figure~\ref{fig:aps_mngmt}), the E2E ASMF is also responsible for translating the E2E AS SLA into E2E CS SLA and for requiring the adapted E2E CS from the CSMF. In the stitching deployment model, two alternative management flows are feasible. In the first case (label 2 in Figure~\ref{fig:aps_mngmt}), the end customer is an ``expert'' user and he is directly responsible for breaking (using the web portal) the E2E AS SLA into an E2E CS SLA and an SLA with a scope restricted to AS's endpoints. Then, the web portal is used for requiring the adapted E2E CS from the CSMF. In the other case (label 3 in Figure~\ref{fig:aps_mngmt}), the end customer is not an expert of the network domain and he does not need to perform the aforementioned E2E AS SLA splitting. On the contrary, the the APSMF is responsible to communicate directly with the NSMF to manage the network slice subnet associated with the APS. Finally, we remind that the management of network slice via the NSMF and the one of per-domain network slice subnets via NSSMFs are well-defined by 3GPP, and they do not need to be extended~\cite{3GPPTS28531,3GPPTS28541}. 

%The life cycle management of the APS can happen in two manners depending on the APS deployment models that is used in the system (see Section~\ref{subsec:AS_deployment}). In the case of the stitching deployment model (label (2) in Figure~\ref{fig:aps_mngmt}), ASMF delegates to a new management function, called \textit{Application Slice Management Function} (APSMF) the deployment and management, in an end-to-end manner, of the APS associated with the requested AS. Specifically, the APSMF splits the APS into multiple subnets, one for each domain over which the APS spans, including both the network and the AS's endpoints, namely the UE requesting the AS, and the edge platform instantiating the AS. To this end, we introduce the new \textit{Application Slice Subnet Management Function} (APSSMF), which applies the APSMF's life-cycle management commands within the two potential domains that are relevant for an application slice subnets, namely the UE and the MEC. Finally, the APSMF communicates directly with the NSMF to manage the network slice subnet associated with the APS. 

%In the case of the overlay deployment model (label (1) in Figure~\ref{fig:aps_mngmt}), the APS is a consumer of the Communication Service offered through the underlying E2E network slice. Thus, the ASMF communicates directly with the CSMF to invoke the desired CS. However, the ASMF also needs to communicate with the ASPMF to manage the application slice subnets in the UE and MEC domains. We remind that the management of per-domain network slice subnets via NSSMFs is well-defined by 3GPP, and it does not need to be extended~\cite{3GPPTS28531,3GPPTS28541}. 

%\Thai{--- beginning : proposed ideas }
%\emph{Application Slice Subnet} vs \emph{Application Sub-Slice}: similarly to the network slice management operations, the E2E ASMF and the Edge ASSMF could pertain or not to the same operator. In the first case, the E2E ASMF splits the E2E application slice into domain-specific \emph{Application Slice Subnets} and provides to the related ASSMF the correspondent Application Slice Subnet template. In the second case, the domain-related \emph{Application Slice Subnet} is wrapped by the E2E ASMF in an \emph{Application Sub-Slice} which is in its turn wrapped in an Application Service. The E2E ASMF communicates, in this case, to the domain ASSMF an Application Service template (as per business-level relationship). 

%Similarly to ETSI NFV Network Service concept (Cf. section \ref{subsec:nw_slice_nw_service}), we define a \emph{ETSI-Like Application Service or ELAS} (not to make confusion with Application Service concept) as a composition of Application Component Functions (ACFs) and possibly of other ELAS (i.e. nested ELASes). 

%ACFs can be composed into ELAS in the form of Application Function Forwarding Graph (ACFFG) similarly to VNFFG described in section \ref{subsec:nw_slice_nw_service}. \Thai{For simplicity, we will call the ACFFG a Service Graph in the rest of the document?}
%\Thai{--- end : proposed ideas }

%
\begin{figure}[ht]
\centering
\includegraphics[clip,trim= 0cm 2.5cm 14cm 7.5cm,width=0.4\textwidth]{figures/fig_aps_mngmt_focus.pdf}
\caption{Illustration of the orchestration entities that are involved in the MEC domain.}
\label{fig:ps_mngmt_focus}
\end{figure}
%
In the remaining of this paper, we will describe our experience in implementing the management architecture described above. As shown in Figure~\ref{fig:ps_mngmt_focus}, several orchestration entities are involved in the management of the various application slice subnets. Our focus will be on implementing the MEC Customer Orchestrator using a popular open-source container orchestration platform. Furthermore, we will detail the interfaces and data models needed to interact with the MEC APSSMF, allowing the deployment of isolated MEC application slice subnets composed of ACFs in the form of Docker containers.  

%\Thai{Operationally, the AS descriptors are received by the MEC customer from a ...\newline
%Sharing of ACFs: concept of Network slice subnet which allows for the management of NFs independently of the network slice - a network slice subnet can be shared between two network slices. similarly, we could define an application slice subnet which is a set of ACFs}

\section{Training details}
\label{sec:HFGD:training_settings}

Unless specified otherwise, the training settings for our proposed VPNeXt are similar to existing works that use ViT mask decoders~\cite{cSETR,cSegViT,cMask2Former}.
This includes the AdamW optimizer, a batch size of 16, and the use of clipnorm along with a mask loss that combines focal and dice losses.

Given that this work focuses exclusively on the plain ViT backbone, all the experiments we conducted are based on the plain ViT without pyramid modifications. 
%
Following common practices, the weights of the ViT are initialized through modern pre-training~\cite{cAugReg,cEVA}.

To accommodate new readers in the field, we utilize the commonly used Mean Intersection over Union (mIOU) metric to evaluate the prediction accuracy of our model.
\section{Experimental Protocol}
\label{sec:evaluation}
\subsection{Model and Dataset}
\begin{figure}[t]
    \centering
\includegraphics[width=\linewidth]{fig/instruction.png}
    \caption{Instructions given to the LLM for the bias detecrtion.}
    \label{fig:instruction}
\end{figure}
We utilized Stable Diffusion 3.5-large~\cite{sd3} as our text-to-image (T2I) model and employed GPT-4o~\cite{gpt4} for bias detection as a blackbox model, and DeepSeek-V3~\cite{liu2024deepseek} as an open-sourced model. The LLM receives prompts as illustrated in Figure \ref{fig:instruction}. Through in-context learning techniques, we enhance model performance by exposing it to an exemplar task~\cite{brown2020language}. To evaluate the debiasing performance for occupations, we used the occupation dataset from Stable Bias~\cite{Luccioni_2023} (hereafter referred to as the stable bias profession dataset), which contains 131 occupations sourced from the U.S. Bureau of Labor Statistics (BLS). The dataset composition is detailed in the Appendix A of~\cite{Luccioni_2023}. All input prompts were formatted as ``A portrait photo of [profession]'' to ensure that the T2I model interprets them specifically as occupations rather than other potential meanings. To assess the performance in removing implicit social biases present in prompts beyond occupations, we used the Parti Prompt dataset~\cite{yu2022scaling}, which consists of over 1,600 diverse English prompts designed to comprehensively evaluate text-to-image generation models and test their limitations. For attribute rebalancing, we employed the uniform distribution, as our primary goal was to verify the debiasing capability of our latent variable guidance.

% For experiments involving bias adjustment using employment statistics log-probabilities, we conducted experiments to mitigate gender bias using BLS2022 statistical data for five occupation prompts mentioned in \cite{naik2023social}: ``CEO'', ``doctor'', ``computer programmer'', ``house keeper'', and ``nurse''.

\subsection{Human Evaluation}
For each prompt, nine images are generated using three methods: a baseline method without debiasing, and two LLM-assisted debiasing methods employing GPT-4o and DeepSeek-V3. These images are arranged in a 3 $\times$ 3 grid, and evaluators assess pairs of images based on image quality, prompt reflection, and diversity of generations. Image quality refers to the aesthetic appeal, high resolution, natural appearance, and detailed refinement of the images. Prompt adherence measures the degree to which the generated images reflect the input text. Diversity of generations evaluates the variety of generated results, particularly whether the images avoid stereotypes and fixed patterns. For each criterion, evaluators rate the results on a 5-point scale, ranging from 1 (very poor) to 5 (very good). To facilitate relative comparisons, images generated by different models for the same input prompt are presented in consecutive questions. This comparative evaluation across the three criteria enables a detailed assessment of the proposed methods' relative strengths and limitations. We randomly selected 50 prompts from Stable Bias profession dataset and Parti Prompt dataset. The subset used for the human evaluation is detailed in Table\ref{tab:sd_subset} and Table\ref{tab:pp_subset} in the supplementary materials. Responses were collected from 20 evaluators, ensuring a diverse range of perspectives. 
\subsection{Non-parametric Evaluation}
Quantitative evaluation of generation diversity presents significant challenges. To address this, we adopt the clustering-based evaluation methodology proposed in Stable Bias~\cite{Luccioni_2023}, implementing a nonparametric diversity assessment using k-Nearest Neighbors (kNN)~\cite{fix1985discriminatory}. Specifically, we generate anchor images based on prompts structured as ``a portrait of a [ethnicity] [gender] at work,'' creating nine images for each combination of ethnicity and gender. This analysis employs 18 ethnic labels from Stable Bias and three gender categories: ``male'', ``female'', and ``non-binary'' (detailed ethnic labels are provided in the Appendix A of~\cite{Luccioni_2023}).

For image embeddings, we utilize Google's VertexAI multimodal embedding model\footnote{https://cloud.google.com/vertex-ai/docs/generative-ai/embeddings/get-multimodal-embeddings}, which converts 512 $\times$ 512 images into 1048-dimensional vector representations. For each prompt in the identity dataset, 30 unique images are generated, yielding a total of 54 $\times$ 30 $=$ 1620 images that serve as anchor points for classification. To examine local trends linked to specific professions, we follow the methodology outlined in \cite{naik2023social}, generating 210 images per method for five professions: ``CEO'', ``computer programmer'', ``doctor'', ``nurse'', and ``housekeeper''. The classification results are visualized to uncover potential biases or distinct patterns specific to each profession.

% In addition, to capture global trends across the entire profession dataset, we generate nine images per profession prompt for each method. These classification results provide an overarching perspective on diversity and potential biases in the generated outputs.

%\section{Related Work}

\paragraph{Commonsense Reasoning Evaluation} 
There are numerous benchmarks and datasets for commonsense reasoning, most of which are in English. 
%Some work focused on evaluating general commonsense knowledge, such as HellaSwag \cite{zellers2019hellaswag}, CommonsenseQA \cite{talmor2019commonsenseqa}, OpenBookQA \cite{OpenBookQA2018}, and WSC \cite{levesque2012winograd}. 
Some studies focus on evaluating general commonsense knowledge \cite{zellers2019hellaswag,talmor2019commonsenseqa,OpenBookQA2018}. 
%Others target specific aspects of commonsense reasoning, including temporal commonsense with MCTACO \cite{zhou2019going}, physical commonsense with PIQA \cite{bisk2020piqa}, social commonsense with SocialIQA \cite{sap2019socialiqa}, numerical commonsense with NumerSense \cite{lin2020birds}, and scientific commonsense with ARC \cite{clark2018think} and QASC \cite{khot2020qasc}. Notably, most of these datasets are in English. 
Others target specific aspects of commonsense reasoning\cite{zhou2019going,bisk2020piqa,sap2019socialiqa,lin2020birds,clark2018think,khot2020qasc}.
There are some Chinese datasets for commonsense reasoning \cite{sun2024benchmarking,shi2024corecode}. 
For instance, CHARM \cite{sun2024benchmarking} distinguishes between global commonsense and Chinese-specific commonsense but includes only a limited number of everyday commonsense cases. 
However, evaluations aimed at assessing the robustness of commonsense reasoning are still understudied. 

\paragraph{Datasets on Different Reasoning Forms}
There are several datasets relevant to our variant design. For reverse reasoning, ART \cite{DBLP:conf/iclr/BhagavatulaBMSH20}, $\delta$-NLI \cite{DBLP:conf/emnlp/RudingerSHBFBSC20}, and CLUTRR \cite{DBLP:conf/emnlp/SinhaSDPH19} explore different reasoning directions. FCR \cite{DBLP:journals/corr/abs-2204-07408} and NatQuest \cite{ceraolo2024analyzinghumanquestioningbehavior} evaluate causal reasoning, while TimeTravel \cite{DBLP:conf/emnlp/QinBHBCC19} focuses on counterfactual scenario refinement. Additionally, PoE \cite{balepur2024s} assesses reasoning involving negation. 
However, not all these datasets focus on commonsense reasoning, nor are they structured by original questions and their variants. Furthermore, they typically target limited reasoning types. Lastly, our dataset is large-scale and covers diverse commonsense knowledge. 

\paragraph{Robustness and Consistency in LLMs} 
Early work focuses on adversarial attacks, with developing evaluation methods for reading comprehension systems \cite{jia2017adversarial}, followed by universal adversarial triggers \cite{wallace2019universal}. The field then expands to examine spurious correlations, with revealing how models often exploit superficial patterns rather than engaging in genuine reasoning \cite{branco2021shortcutted,geirhos2020shortcut}. And \citealp{ross2022does} investigates whether self-explanation can mitigate these spurious correlations. Coherence and consistency evaluation advances through classifier assessment methods \cite{storks2021beyond} and analysis of accuracy-consistency trade-offs \cite{johnson2023much}. While these studies primarily address model robustness against adversarial attacks or spurious correlations, our work takes a novel approach by examining robustness in reasoning forms.
%, specifically focusing on how models maintain consistent reasoning when presented with different reasoning forms of the same commonsense knowledge.
%\paragraph{Dataset Construction by LLM} 
%Research indicates that when LLMs are utilized for dataset generation, the resulting datasets are more accurate and fluent \cite{lu2022fantastically, min-etal-2022-rethinking} than those created by crowd-sourced annotators. Furthermore, generating datasets with LLMs is significantly more cost-effective than using crowd-sourced annotations \cite{liu2022wanli, wiegreffe2022reframing, west2022symbolic}. Hence, we generate our benchmark by LLM in-context learning.

% \paragraph{In-Context Learning} 
% As LLMs become more widely used, in-context learning (\citealp{brown2020language}; \citealp{ouyang2022training}; \citealp{min-etal-2022-rethinking}) has emerged as the primary approach for executing various tasks. This method involves supplying LLMs with textual instructions and examples and removes the necessity for parameter modifications. Research indicates that when LLMs are utilized for dataset generation, the resulting datasets are more accurate and fluent (\citealp{lu2022fantastically}; \citealp{min-etal-2022-rethinking}) than those created by crowd-sourced annotators. Furthermore, generating datasets with LLMs is significantly more cost-effective than using crowd-sourced annotations (\citealp{liu2022wanli}; \citealp{wiegreffe2022reframing}; \citealp{west2022symbolic}). Hence, we have decided to construct our benchmark by over-generating data using in-context learning and employing human annotators for filtering to ensure high efficiency and high quality.
% Moxin: 这部分应该改成用LLM 生成dataset的工作?
% 

\section{Conclusion and future directions} \label{sec:conclusion}

In this paper we proposed a nested MLMC framework that offers important computational savings by performing most calculations in low precision and exploiting approximate random normal variables for the low precision path calculations. The low precision calculations could be performed in fixed precision on an FPGA for greater efficiency, and we suggested a procedure to optimise the bit-widths of every variable at each Monte Carlo level. This is an important improvement over previous mixed precision MLMC frameworks which held the lower precision fixed \cite{Rounding_error_oliver} or defined uniform bit-width at every level heuristically \cite{brugger2014mixed}. Our numerical results suggest that for the first levels our procedure reduces the cost at these levels by a factor 5 or 7. Hence the overall savings are significant since most paths are calculated on the first levels. Our approach would be even more efficient for the Milstein scheme because its higher order strong convergence leads to a greater proportion of the computational costs being on the coarsest levels.

The next stage of the research project will be to implement the RNG methods and the nested framework on FPGAs to determine the hardware requirements and confirm the extent of the computational savings. It would also be good to compare the performance benefits to using half-precision floating point arithmetic on GPUs or CPUs for the low-accuracy computations.




% \begin{abstract}
% This document presents supplementary materials supporting our main submission. In section \ref{sec1} , we provide more details on parameters for watermarking method. In section \ref{sec2}, we analyze the influence of watermark capacity on the bit accuracy and image quality. In section \ref{sec3}, we analyze the impact of the sampling method. In section \ref{sec4}, we present additional evidence to establish the robustness of DistriMark. In section \ref{sec5}, we provide some information on the efficiency of watermark embedding.
% \end{abstract}

% \section{1.Parameters for Watermarking Methods}
% \label{sec1}
% \noindent \textbf{Stable Signature \cite{fernandez2023stable}, FSwatermark\cite{xiong2023flexible}} We use the pretrained watermarked variational autoencoder provided in the paper which embeds 48-bit  and 100-bit messages partially. The watermark-related parameters are consistent with the default settings of the implementation code in the paper. Images are generated using 25 inference steps.

% \noindent \textbf{Tree Ring. \cite{wen2024tree}} We evaluate the Tree-Ring$_{rings}$ method, which the authors state “delivers the best average performance while offering the model owner the flexibility of multiple different random keys”. Using the author’s implementation, we generate and verify watermarked images using 25 inference steps, where we use an empty prompt during verification and keep the remaining default parameters chosen by the authors. The evaluation standard for detection is p-value.

% \noindent \textbf{DwtDct \cite{cox2007digital}, DwtDct \cite{cox2007digital}, RivaGan \cite{zhang2019robust}.} We utilize a 48-bit watermark for DwtDct and DwtDct, and a 32-bit watermark for RivaGan. Default parameters from the Stable Diffusion implementation are maintained.

% \noindent \textbf{Parameters for DistriMark.} The watermark embedded in the robustness evaluation is 48-bit watermark. The watermarked images are generated and verified using 25 inference steps. An empty prompt is used for diffusion inversion. The version of the variational autoencoder used for fine-tuning is Stable-Diffusion-2-1-base \cite{rombach2022high}. The skip connection method is multi-level connections. The parameter settings in the attack layer are as follows: Gauss Blur with a kernel size $7\times7$ and std of 0.01, Gauss Noise(std.) in [0.1,0.8], Brightness in [0.5,1.5], Contrast in [0.5,1.5] and JPEG in [50,90].  The balancing weight $\varepsilon=0.8$, $\delta=0.05$.

% \section{2.Analysis on Watermark Capacity.}
% \label{sec2}
% %对于水印容量,实验分别嵌入了16、32、48、64、96比特水印以探究DistriMark水印容量对于图像质量的影响。
% For embedding capacity, the experiment embedded 16, 32, 48 and 64 bits of DistriMark watermark to investigate its influence on image quality. The main results are shown in Table \ref{tab1:capacity}. When the watermark capacity is within 48 bits, there is no significant difference in image quality parameters such as NIQE, PIQE, and semantic quality parameters such as CLIP between watermarked between watermarked images and non-watermarked images, which is beneficial for practical usage. When the watermark capacity increases to 64, the watermark component experiences convergence difficulties, leading to deviations of the latent variables from the standard Gaussian distribution. This issue is partly attributed that the specific structure of the watermark encoder and decoder poses difficulties in achieving the distribution of a 64-bit watermark. 

% % Please add the following required packages to your document preamble:
% \usepackage{graphicx}
% \usepackage[table,xcdraw]{xcolor}
% Beamer presentation requires \usepackage{colortbl} instead of \usepackage[table,xcdraw]{xcolor}
\begin{table*}[t]
\centering
\resizebox{0.92\textwidth}{!}{%
\begin{tabular}{rcl}
\multicolumn{3}{c}{{\color[HTML]{000000} \Large \textsc{Close-ended}}}                                                                                     \\ \hline 
{\color[HTML]{000000} \textbf{\large \textsc{Tasks}}}                                                                       & {\color[HTML]{000000} \textbf{\large \textsc{Metrics}}}                                                                                  & {\color[HTML]{000000} \textbf{\large \textsc{Datasets}}}                                                                                                                                                                                                      \\ \hline
\rowcolor[HTML]{EFEFD2} 
{\color[HTML]{656565} \textbf{\begin{tabular}[c]{@{}r@{}}Multiple choice\\ questions\end{tabular}}}   & {\color[HTML]{656565} Accuracy}                                                                                          & {\color[HTML]{656565} \begin{tabular}[c]{@{}l@{}}· MedMCQA \cite{pmlr-v174-pal22a} \hspace{30pt} · PubMedQA \cite{jin2019pubmedqa} \\ · MedQA \cite{jin2020disease} \hspace{41pt} · MMLU \cite{hendrycks2020measuring}    \\ · \href{https://huggingface.co/datasets/HPAI-BSC/CareQA}{\careqa{}-Close} \end{tabular}}                                                                                   \\
\rowcolor[HTML]{E3EFD6} 
{\color[HTML]{656565} \textbf{\begin{tabular}[c]{@{}r@{}}Prescriptions \\ writing \end{tabular}}}            & {\color[HTML]{656565} "}                                                                                          & {\color[HTML]{656565} · \href{https://huggingface.co/datasets/devlocalhost/prescription-full}{Prescription} }                                                                                                                                                                                \\

\rowcolor[HTML]{C8E9D9} 
{\color[HTML]{656565} \textbf{\begin{tabular}[c]{@{}r@{}}Medical text\\ classification\end{tabular}}}      & {\color[HTML]{656565} "}                                                                                          & {\color[HTML]{656565} \begin{tabular}[c]{@{}l@{}}· \href{https://www.kaggle.com/datasets/chaitanyakck/medical-text/data}{Medical Text for classification} \cite{10.1145/3582768.3582795}\\ · \href{https://www.kaggle.com/datasets/tboyle10/medicaltranscriptions}{Medical Transcriptions}\end{tabular}} \\



\rowcolor[HTML]{CDEBEB} 
{\color[HTML]{656565} \textbf{\begin{tabular}[c]{@{}r@{}}Relation\\ extraction\end{tabular}}}              & {\color[HTML]{656565} "}                                                                                          & {\color[HTML]{656565} · \href{https://huggingface.co/datasets/YufeiHFUT/BioRED_all_info}{BioRED} \cite{luo2022biored}}         \\

\\ \hline
\multicolumn{3}{l}{}                                                                                                                                                                                                                                           \\ 
\multicolumn{3}{c}{{\color[HTML]{000000} \Large \textsc{Open-ended}}}                                                                                                                                                                                          \\ \hline

\rowcolor[HTML]{C7DBE9} 
        
{\color[HTML]{656565} \textbf{\begin{tabular}[c]{@{}r@{}}Open-ended\\ medical questions\end{tabular}}}     & {\color[HTML]{656565} \normalsize \begin{tabular}[c]{@{}c@{}}BLEU, BLEURT, ROUGE,\\ BERTScore, MoverScore,\\ Prometheus, Perplexity\end{tabular}} & {\color[HTML]{656565} \begin{tabular}[c]{@{}l@{}}· \href{https://huggingface.co/datasets/bigbio/meddialog}{MedDialog Raw} \cite{zeng2020meddialog} \\ · \href{https://huggingface.co/datasets/bigbio/mediqa_qa}{MEDIQA2019} \cite{MEDIQA2019} \\· \href{https://huggingface.co/datasets/HPAI-BSC/CareQA}{\careqa{}-Open} 
\end{tabular}}  


\\


\rowcolor[HTML]{E1DEE9} 
{\color[HTML]{656565} \textbf{\begin{tabular}[c]{@{}r@{}}Making diagnosis \\ and treatment\\ recommendations\end{tabular}}} & {\color[HTML]{656565} " }  & {\color[HTML]{656565} · \href{https://huggingface.co/datasets/BI55/MedText}{MedText} }                                                                                                                                                                                      
\\

\rowcolor[HTML]{E5CFDF} 
{\color[HTML]{656565} \textbf{\begin{tabular}[c]{@{}r@{}}Clinical\\ note-taking\end{tabular}}}             & {\color[HTML]{656565} " } & {\color[HTML]{656565} \begin{tabular}[c]{@{}l@{}}· \href{https://huggingface.co/datasets/har1/MTS_Dialogue-Clinical_Note}{MTS-Dialog} \cite{mts-dialog}\\ · ACI-Bench \cite{aci-bench}\end{tabular} }                                                                            

\\ 

\rowcolor[HTML]{EFD6DC} 
{\color[HTML]{656565} \textbf{\begin{tabular}[c]{@{}r@{}}Medical\\ factuality\end{tabular}}}             & {\color[HTML]{656565} \begin{tabular}[c]{@{}c@{}} " \\ + Relaxed Perplexity \end{tabular}  } & {\color[HTML]{656565} \begin{tabular}[c]{@{}l@{}} · \href{https://huggingface.co/datasets/dmis-lab/MedLFQA}{OLAPH} \cite{jeong2024olaph}\end{tabular} }                                                                                                     \\ 

\rowcolor[HTML]{E9E1DE} 
{\color[HTML]{656565} \textbf{Summarization}}                                                              & {\color[HTML]{656565} \begin{tabular}[c]{@{}c@{}} " \\ + F1-RadGraph \end{tabular}} & {\color[HTML]{656565} · \href{https://huggingface.co/datasets/dmacres/mimiciii-hospitalcourse-meta}{MIMIC-III} \cite{johnson2016mimic}}                                                                                                                                              \\



\rowcolor[HTML]{E4D9D4} 
{\color[HTML]{656565} \textbf{\begin{tabular}[c]{@{}r@{}}Question\\ entailment\end{tabular}}}              & {\color[HTML]{656565} "}  & {\color[HTML]{656565} · \href{https://huggingface.co/datasets/lighteval/med_dialog}{Meddialog Qsumm} \cite{zeng2020meddialog} }                                                               

\\ \hline

\end{tabular}
}\caption{This table presents the tasks implemented in this paper. The first column specifies the different tasks. The second details the metrics used (ROUGE includes ROUGE1, ROUGE2 and ROUGEL, and Perplexity includes Bits per Byte, Byte Perplexity, and Word Perplexity). The third column outlines the benchmarks used for each task.} \label{tab:tasks_bench_metrics}
\end{table*}

% \section{3.Analysis on Sampling Method} 
% \label{sec3}
% Considering that untrustworthy parties might use different sampling patterns to circumvent the watermark embedding process, we conducted experiments with watermark embedding and unauthorized initial latent variable image generation using DDIM Scheduler \cite{song2020denoising}, DPMSolverMultistepScheduler \cite{lu2022dpm}, EulerAncestralDiscreteScheduler \cite{karras2022elucidating}, and HeunDiscreteScheduler \cite{karras2022elucidating}. Figure \ref{fig1} shows the images generated under different sampling methods for watermarked and non-watermarked latent variables. Under different sampling modes, Latent-VAE Skip-Binder remains effective. Random latent variables still fail to generate images correctly and images generated by watermarked initial latent variables remain the same. Some sampling modes such as EulerAncestralDiscreteScheduler do not support diffusion inversion, which can hinder watermark verification. We will continue to investigate this issue in future research.
% \begin{figure}[h]
% \centering
% \includegraphics[width=1\columnwidth]{3.png} % 
% \caption{Images generated from random latents and watermarked latents using different schedulers.}
% \label{fig1}
% \end{figure}

% \section{4.More Analysis on Image Processing Attack}
% \label{sec4}
% As shown in Figure \ref{fig1}, we consider image processing attack of different strength to evaluate the robustness of  the watermark including brightness, Gauss noise , Gauss blur , JPEG compression , image cropping, rotation, BM3D denoising algorithm, two VAE-based attacks and a diffusion based reconstructive attack. 
% The main experimental results are shown in Figure \ref{fig2}. From the figure, DistriMark demonstrates strong robustness against common attacks such as Brightness, Contrast, Resizing and Rescaling and JPEG Compression compared to other watermarking schemes for model distribution scenerios. It shows a clear advantage in dealing with more challenging attacks like VAE-based attacks and Diffusion-based attacks. DistriMark exhibits a certain level of robustness against Image Cropping, but its robustness is weaker for Image Cropping and Rescaling, as well as Image Rotation. 

% \begin{figure}[h]
% \centering
% \includegraphics[width=1\columnwidth]{v1.png} % 
% \caption{ Watermarked Image Subjected to Image Processing. 
%  (a) Original watermarked images. (b) Brightness adjustment of 2. (c) Gaussian noise with a standard deviation of 0.05. (d) Gaussian blur with a kernel size of 7×7. (e) JPEG compression with a quality factor of 50. (f) 20\% area crop. (g) 10\% random crop. (h) 20\% crop and scale. (i) Rotation by 45 degrees. 
%  (j) BM3D algorithm with a std of 0.1}
% \label{fig1}
% \end{figure}

% \begin{figure*}[h]
% \centering
% \includegraphics[width=2\columnwidth]{11.pdf} % 
% \caption{Evaluation on image processing.}
% \label{fig2}
% \end{figure*}

% \section{5.More Analysis on Watermark Embedding Efficiency}
% \label{sec5}
% For the 48-bit pretrained watermark encoder-decoder, it takes approximately 96 GPU hours to train. For fine-tuning the variational autoencoder for a specific model user, generating a single key only requires approximately 1 GPU hour on a single L40 GPU, which is practical for model distribution scenerios. Considering that the total training time for the diffusion model is about 150 to 1000 GPU days \cite{dhariwal2021diffusion}, the time spent on fine-tuning the model components is negligible. 
%% If you have bibdatabase file and want bibtex to generate the
%% bibitems, please use
%%
 \bibliographystyle{elsarticle-num} 
 \bibliography{cas-refs}

%% else use the following coding to input the bibitems directly in the
%% TeX file.

% \begin{thebibliography}{00}

% %% \bibitem{label}
% %% Text of bibliographic item

% \bibitem{}

% \end{thebibliography}
\end{document}
\endinput
%%
%% End of file `elsarticle-template-num.tex'.
