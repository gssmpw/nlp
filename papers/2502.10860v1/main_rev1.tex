%% 
%% Copyright 2007-2020 Elsevier Ltd
%% 
%% This file is part of the 'Elsarticle Bundle'.
%% ---------------------------------------------
%% 
%% It may be distributed under the conditions of the LaTeX Project Public
%% License, either version 1.2 of this license or (at your option) any
%% later version.  The latest version of this license is in
%%    http://www.latex-project.org/lppl.txt
%% and version 1.2 or later is part of all distributions of LaTeX
%% version 1999/12/01 or later.
%% 
%% The list of all files belonging to the 'Elsarticle Bundle' is
%% given in the file `manifest.txt'.
%% 

%% Template article for Elsevier's document class `elsarticle'
%% with numbered style bibliographic references
%% SP 2008/03/01
%%
%% 
%%
%% $Id: elsarticle-template-num.tex 190 2020-11-23 11:12:32Z rishi $
%%
%%
%%\documentclass[preprint,12pt]{elsarticle}

%% Use the option review to obtain double line spacing
%% \documentclass[authoryear,preprint,review,12pt]{elsarticle}

%% Use the options 1p,twocolumn; 3p; 3p,twocolumn; 5p; or 5p,twocolumn
%% for a journal layout:
%% \documentclass[final,1p,times]{elsarticle}
%% \documentclass[final,1p,times,twocolumn]{elsarticle}
%% \documentclass[final,3p,times]{elsarticle}
%% \documentclass[final,3p,times,twocolumn]{elsarticle}
\documentclass[5p,times]{elsarticle}
%% \documentclass[final,5p,times,twocolumn]{elsarticle}

\usepackage{microtype}

\usepackage{multirow}

%\usepackage{flushend}

\usepackage{caption}
\usepackage{subcaption}

\usepackage{url}

%% For including figures, graphicx.sty has been loaded in
%% elsarticle.cls. If you prefer to use the old commands
%% please give \usepackage{epsfig}

%% The amssymb package provides various useful mathematical symbols
\usepackage{amssymb}
%% The amsthm package provides extended theorem environments
\usepackage{amsthm}

%% The lineno packages adds line numbers. Start line numbering with
%% \begin{linenumbers}, end it with \end{linenumbers}. Or switch it on
%% for the whole article with \linenumbers.
\usepackage[switch]{lineno}

\usepackage{enumitem}
\setlist[itemize]{align=parleft,left=0pt..1em}

%% Set comments per author
\usepackage{soul}
\usepackage[dvipsnames]{xcolor}

%\usepackage[draft]{changes}
\usepackage[final]{changes}

\DeclareRobustCommand{\Thai}[1]{ {\begingroup\sethlcolor{BurntOrange}\hl{(Thai:) #1}\endgroup} }
\DeclareRobustCommand{\Simone}[1]{ {\begingroup\sethlcolor{BurntOrange}\hl{(Simone:) #1}\endgroup} }
\DeclareRobustCommand{\Raffaele}[1]{ {\begingroup\sethlcolor{BurntOrange}\hl{(Raffaele:) #1}\endgroup} }

\journal{Future Generation Computer Systems}

\begin{document}

\fbox{\begin{minipage}[b][1cm][c]{18cm}
\footnotesize This article has been accepted for publication in the Future Generation Computer Systems, Elsevier. This is the author's version which has not been fully edited and content may change prior to final publication. Citation information: \url{https://doi.org/10.1016/j.future.2022.05.027}
\end{minipage}}

\begin{frontmatter}

%% Title, authors and addresses

%% use the tnoteref command within \title for footnotes;
%% use the tnotetext command for theassociated footnote;
%% use the fnref command within \author or \address for footnotes;
%% use the fntext command for theassociated footnote;
%% use the corref command within \author for corresponding author footnotes;
%% use the cortext command for theassociated footnote;
%% use the ead command for the email address,
%% and the form \ead[url] for the home page:
%% \title{Title\tnoteref{label1}}
%% \tnotetext[label1]{}
%% \author{Name\corref{cor1}\fnref{label2}}
%% \ead{email address}
%% \ead[url]{home page}
%% \fntext[label2]{}
%% \cortext[cor1]{}
%% \affiliation{organization={},
%%             addressline={},
%%             city={},
%%             postcode={},
%%             state={},
%%             country={}}
%% \fntext[label3]{}

\title{Towards End-to-End Application Slicing in Multi-access Edge Computing systems: Architecture Discussion and Proof-of-Concept}

%% use optional labels to link authors explicitly to addresses:
%% \author[label1,label2]{}
%% \affiliation[label1]{organization={},
%%             addressline={},
%%             city={},
%%             postcode={},
%%             state={},
%%             country={}}
%%
%% \affiliation[label2]{organization={},
%%             addressline={},
%%             city={},
%%             postcode={},
%%             state={},
%%             country={}}

\author[cnr]{Simone Bolettieri}

\affiliation[cnr]{organization={Institute of Informatics and Telematics (IIT) - Italian National Research Council (CNR)},
            addressline={Via G. Moruzzi 1}, 
            city={Pisa},
            postcode={56124}, 
            country={Italy}}

\author[nokia]{Dinh Thai Bui}
\author[cnr]{Raffaele Bruno}

\affiliation[nokia]{organization={Nokia Bell Labs},
            addressline={1 route de Villejust}, 
            city={Nozay},
            postcode={91620}, 
            country={France}}

\begin{abstract}
%% Text of abstract
Network slicing is one of the most critical 5G pillars. It allows for sharing a 5G infrastructure among different tenants leading to improved service customisation and increased operators' revenues. Concurrently, introducing the Multi-access Edge Computing (MEC) into 5G to support time-critical applications raises the need to integrate this distributed computing infrastructure to the 5G network slicing framework. Indeed, end-to-end latency guarantees require the end-to-end management of slice resources. For this purpose, after discussing the main gaps in the state-of-the-art with regards to such an objective, we propose a novel slicing architecture that enables the management and orchestration of slice segments that span over all the domains of an end-to-end application service, including the MEC. We also show how this general management architecture can be instantiated into a multi-tenant MEC infrastructure. A preliminary implementation of the proposed architecture focusing on the MEC domain is also provided, together with performance tests to validate the feasibility and efficacy of our design approach.
\end{abstract}

%%Graphical abstract
%%\begin{graphicalabstract}
%%\includegraphics{grabs}
%%\end{graphicalabstract}

%%Research highlights
%\begin{highlights}
%\item Research highlight 1
%\item Research highlight 2
%\end{highlights}

\begin{keyword}
%% keywords here, in the form: keyword \sep keyword
edge computing \sep MEC \sep NFV \sep 3GPP network slicing \sep latency-sensitive applications
%% PACS codes here, in the form: \PACS code \sep code
%%\PACS 0000 \sep 1111
%% MSC codes here, in the form: \MSC code \sep code
%% or \MSC[2008] code \sep code (2000 is the default)
%%\MSC 0000 \sep 1111
\end{keyword}

\end{frontmatter}

%\linenumbers

%% main text
%%
\section{Introduction}

Video generation has garnered significant attention owing to its transformative potential across a wide range of applications, such media content creation~\citep{polyak2024movie}, advertising~\citep{zhang2024virbo,bacher2021advert}, video games~\citep{yang2024playable,valevski2024diffusion, oasis2024}, and world model simulators~\citep{ha2018world, videoworldsimulators2024, agarwal2025cosmos}. Benefiting from advanced generative algorithms~\citep{goodfellow2014generative, ho2020denoising, liu2023flow, lipman2023flow}, scalable model architectures~\citep{vaswani2017attention, peebles2023scalable}, vast amounts of internet-sourced data~\citep{chen2024panda, nan2024openvid, ju2024miradata}, and ongoing expansion of computing capabilities~\citep{nvidia2022h100, nvidia2023dgxgh200, nvidia2024h200nvl}, remarkable advancements have been achieved in the field of video generation~\citep{ho2022video, ho2022imagen, singer2023makeavideo, blattmann2023align, videoworldsimulators2024, kuaishou2024klingai, yang2024cogvideox, jin2024pyramidal, polyak2024movie, kong2024hunyuanvideo, ji2024prompt}.


In this work, we present \textbf{\ours}, a family of rectified flow~\citep{lipman2023flow, liu2023flow} transformer models designed for joint image and video generation, establishing a pathway toward industry-grade performance. This report centers on four key components: data curation, model architecture design, flow formulation, and training infrastructure optimization—each rigorously refined to meet the demands of high-quality, large-scale video generation.


\begin{figure}[ht]
    \centering
    \begin{subfigure}[b]{0.82\linewidth}
        \centering
        \includegraphics[width=\linewidth]{figures/t2i_1024.pdf}
        \caption{Text-to-Image Samples}\label{fig:main-demo-t2i}
    \end{subfigure}
    \vfill
    \begin{subfigure}[b]{0.82\linewidth}
        \centering
        \includegraphics[width=\linewidth]{figures/t2v_samples.pdf}
        \caption{Text-to-Video Samples}\label{fig:main-demo-t2v}
    \end{subfigure}
\caption{\textbf{Generated samples from \ours.} Key components are highlighted in \textcolor{red}{\textbf{RED}}.}\label{fig:main-demo}
\end{figure}


First, we present a comprehensive data processing pipeline designed to construct large-scale, high-quality image and video-text datasets. The pipeline integrates multiple advanced techniques, including video and image filtering based on aesthetic scores, OCR-driven content analysis, and subjective evaluations, to ensure exceptional visual and contextual quality. Furthermore, we employ multimodal large language models~(MLLMs)~\citep{yuan2025tarsier2} to generate dense and contextually aligned captions, which are subsequently refined using an additional large language model~(LLM)~\citep{yang2024qwen2} to enhance their accuracy, fluency, and descriptive richness. As a result, we have curated a robust training dataset comprising approximately 36M video-text pairs and 160M image-text pairs, which are proven sufficient for training industry-level generative models.

Secondly, we take a pioneering step by applying rectified flow formulation~\citep{lipman2023flow} for joint image and video generation, implemented through the \ours model family, which comprises Transformer architectures with 2B and 8B parameters. At its core, the \ours framework employs a 3D joint image-video variational autoencoder (VAE) to compress image and video inputs into a shared latent space, facilitating unified representation. This shared latent space is coupled with a full-attention~\citep{vaswani2017attention} mechanism, enabling seamless joint training of image and video. This architecture delivers high-quality, coherent outputs across both images and videos, establishing a unified framework for visual generation tasks.


Furthermore, to support the training of \ours at scale, we have developed a robust infrastructure tailored for large-scale model training. Our approach incorporates advanced parallelism strategies~\citep{jacobs2023deepspeed, pytorch_fsdp} to manage memory efficiently during long-context training. Additionally, we employ ByteCheckpoint~\citep{wan2024bytecheckpoint} for high-performance checkpointing and integrate fault-tolerant mechanisms from MegaScale~\citep{jiang2024megascale} to ensure stability and scalability across large GPU clusters. These optimizations enable \ours to handle the computational and data challenges of generative modeling with exceptional efficiency and reliability.


We evaluate \ours on both text-to-image and text-to-video benchmarks to highlight its competitive advantages. For text-to-image generation, \ours-T2I demonstrates strong performance across multiple benchmarks, including T2I-CompBench~\citep{huang2023t2i-compbench}, GenEval~\citep{ghosh2024geneval}, and DPG-Bench~\citep{hu2024ella_dbgbench}, excelling in both visual quality and text-image alignment. In text-to-video benchmarks, \ours-T2V achieves state-of-the-art performance on the UCF-101~\citep{ucf101} zero-shot generation task. Additionally, \ours-T2V attains an impressive score of \textbf{84.85} on VBench~\citep{huang2024vbench}, securing the top position on the leaderboard (as of 2025-01-25) and surpassing several leading commercial text-to-video models. Qualitative results, illustrated in \Cref{fig:main-demo}, further demonstrate the superior quality of the generated media samples. These findings underscore \ours's effectiveness in multi-modal generation and its potential as a high-performing solution for both research and commercial applications.
\section{Background} \label{section:LLM}

% \subsection{Large Language Model (LLM)}   

Figure~\ref{fig:LLaMA_model}(a) shows that a decoder-only LLM initially processes a user prompt in the “prefill” stage and subsequently generates tokens sequentially during the “decoding” stage.
Both stages contain an input embedding layer, multiple decoder transformer blocks, an output embedding layer, and a sampling layer.
Figure~\ref{fig:LLaMA_model}(b) demonstrates that the decoder transformer blocks consist of a self attention and a feed-forward network (FFN) layer, each paired with residual connection and normalization layers. 

% Differentiate between encoder/decoder, explain why operation intensity is low, explain the different parts of a transformer block. Discuss Table II here. 

% Explain the architecture with Llama2-70B.

% \begin{table}[thb]
% \renewcommand\arraystretch{1.05}
% \centering
% % \vspace{-5mm}
%     \caption{ML Model Parameter Size and Operational Intensity}
%     \vspace{-2mm}
%     \small
%     \label{tab:ML Model Parameter Size and Operational Intensity}    
%     \scalebox{0.95}{
%         \begin{tabular}{|c|c|c|c|c|}
%             \hline
%             & Llama2 & BLOOM & BERT & ResNet \\
%             Model & (70B) & (176B) & & 152 \\
%             \hline
%             Parameter Size (GB) & 140 & 352 & 0.17 & 0.16 \\
%             \hline
%             Op Intensity (Ops/Byte) & 1 & 1 & 282 & 346 \\
%             \hline
%           \end{tabular}
%     }
% \vspace{-3mm}
% \end{table}

% {\fontsize{8pt}{11pt}\selectfont 8pt font size test Memory Requirement}

\begin{figure}[t]
    \centering
    \includegraphics[width=8cm]{Figure/LLaMA_model_new_new.pdf}
    \caption{(a) Prefill stage encodes prompt tokens in parallel. Decoding stage generates output tokens sequentially.
    (b) LLM contains N$\times$ decoder transformer blocks. 
    (c) Llama2 model architecture.}
    \label{fig:LLaMA_model}
\end{figure}

Figure~\ref{fig:LLaMA_model}(c) demonstrates the Llama2~\cite{touvron2023llama} model architecture as a representative LLM.
% The self attention layer requires three GEMVs\footnote{GEMVs in multi-head attention~\cite{attention}, narrow GEMMs in grouped-query attention~\cite{gqa}.} to generate query, key and value vectors.
In the self-attention layer, query, key and value vectors are generated by multiplying input vector to corresponding weight matrices.
These matrices are segmented into multiple heads, representing different semantic dimensions.
The query and key vectors go though Rotary Positional Embedding (RoPE) to encode the relative positional information~\cite{rope-paper}.
Within each head, the generated key and value vectors are appended to their caches.
The query vector is multiplied by the key cache to produce a score vector.
After the Softmax operation, the score vector is multiplied by the value cache to yield the output vector.
The output vectors from all heads are concatenated and multiplied by output weight matrix, resulting in a vector that undergoes residual connection and Root Mean Square layer Normalization (RMSNorm)~\cite{rmsnorm-paper}.
The residual connection adds up the input and output vectors of a layer to avoid vanishing gradient~\cite{he2016deep}.
The FFN layer begins with two parallel fully connections, followed by a Sigmoid Linear Unit (SiLU), and ends with another fully connection.
%%
%%---------------------------------------------
\section{An E2E Network and Application Slicing Architecture using MEC}
\label{sec:app_slice}
%%
\noindent
In this section, we first introduce the concepts of application services and application slices, and we instantiate these concepts in the context of MEC environments. Then, we discuss application slice from both operational and business perspectives by introducing operation and business roles, respectively, and relationships of the different entities involved in application slicing and service provisioning. We also discuss the different architectural use cases that correspond to the proposed model. Finally, we elaborate on an orchestration/management architecture for 3GPP networks integrating MEC systems that will enable such new use cases, and we propose an extended MEC reference architecture required to support the envisioned management architecture.   
%%
%%---------------------------------------------
\subsection{Preliminary concepts}
\label{sec:pre_concepts}
%%
\noindent
As it could be sensed from both the introduction and the background sections, the 3GPP CSs cannot be straightforwardly used to model the services offered by an edge computing platform. For the sake of disambiguation, in this work, we call the services offered to the customers of an edge computing platform as \textit{Application Services} (ASs). The main reason is that CSs and ASs are designed to support different business purposes, therefore, they are regulated by different SLA frameworks. Specifically, \textit{CSs are dedicated to the transmission of data and network traffic}, while \textit{ASs are designed to support services which are generally not communication oriented in essence, but focus on data processing}, such as Augmented Reality (AR), Virtual Reality (VR), Ultra-High Definition (UHD) video, real-time control, etc. From the management perspective, it would not be sensible to manage ASs with the existing 3GPP CSMF. Some extensions or new management functions should be added to the overall management framework to translate application-related SLAs into application-related SLSs and network-related SLSs. 

In analogy to the 3GPP network slice concept, we define an \textit{Application Slice} (APS) as a \textit{a virtualised data collection system and a set of data processing resources that provides specific processing functions to support an Application Service and the associated SLA. The APS can possibly include Network Functions to ensure correct data communication among the previous processing functions}. A key aspect of the application slice concept is the support of \textit{isolation} not only of the resources that are used by each application slice but also of the management data and functions. Similarly to a network slice, the application slice is an end-to-end concept and can span multiple domains (e.g., device, edge platform, network). Furthermore, an application slice is not a monolithic entity. Following the well-known principles of service-oriented architectures, it is built as an \textit{atomic data-processing function} which has well defined functional behaviours and interfaces, called in this work \emph{Application Component Functions} (ACFs). We can now define an \textit{Application Slice Subnet} (APSS) as a set of at least one ACF and possible NFs supporting an application slice. It is essential to point out that application slices and network slices are not independent concepts, but they are strongly intertwined as application slices deployed at the edge of the network require dedicated network slices to deliver the massive amounts of data and network traffic that application services will typically produce. In the following sections, we will also discuss two potential distinct models to allow an application slice to use a network slice.  

The general concepts that we have introduced so far can be easily instantiated in the context of MEC, NFV and 5G systems. First of all, a MEC platform can be used to offer ASs, which can be deployed as conventional MEC applications and implemented using a collection of ACFs. Furthermore, the MEC system can be considered as one of the domains over which an application slice can span. Thus, the MEC management layer should be responsible for the orchestration and deployment of one of the application slice subnets that compose the E2E APS, hereafter denoted by \textit{MEC application slice subnet}. It is important to point out that a MEC application slice subnet includes one or more ACFs, but it can also include VNFs to support network functionalities. Furthermore, the tenant of a MEC system is not necessarily an end customer, but the MEC system can offer its services following a MEC-Platform-as-a-Service model.  

Finally, it is worth pointing out the differences between VNFs and ACFs which are designed for different purposes - network traffic processing for the former and data processing for the latter. Furthermore, in a multi-tenant MEC environment, like the one in~\cite{Cominardi2020} or~\cite{2020_MNET_MEC_subslice}, it is likely that the MEC applications implementing ACFs will not be orchestrated by the same entity that would orchestrate principal VNFs. Nevertheless, there are a lot of common points in terms of operation and management between a VNF and an ACF as both rely on the same set of virtualisation technologies for their deployment. Thus, we will reuse and extend both 3GPP management architecture for network slice and MEC-in-NFV architecture to support the proposed E2E application slice framework.
%
%\Thai{-- beginning: definitions added --}
%\newline Similarly to the 3GPP network slice framework, we can define successively:
%\begin{itemize}
%    \item An Application Slice Subnet as a set of at least one ACF. This set could include VNFs.
%    \item An Application Slice consists of multiple Application Slice Subnets augmented by SLS parameters (e.g. QoS parameters). The latter are technology-dependent objectives that should be enforced to meet the SLA.
%    \item An Application Service is built from one or multiple Application Slices augmented by a common SLA parameter set (i.e. business commitment)
%\end{itemize}
%\Thai{-- end: definitions added --}
%%
%%---------------------------------------------
\subsection{High-level roles for application slice management}
\label{subsec:roles}
%%
\noindent
From the APS management (and orchestration) perspective, we first need to define high-level roles in order to draw responsibilities boundaries similarly to what was proposed for 5G network slice management (see section~\ref{subsec:mgmt_architecture}). As discussed above, our focus is on Application Services that are offered by a MEC system. The Figure~\ref{fig:AS_roles} shows the different roles identified:
%
\renewcommand\labelitemi{$\bullet$}
\begin{itemize}[noitemsep,topsep=2pt]
    \item \textit{Application Service Consumer} (\textit{ASC}): uses Application Services.
    \item \textit{Application Service Provider} (\textit{ASP}): provides Application Services that are built and operated upon one or multiple application slice subnets, including MEC application slice subnets. Each of those application slice subnets is in turn built from a set of ACFs provided by the \emph{Application Component Function Supplier}.
    \item \textit{MEC Operator} (\textit{MOP}): operates and manages ACFs using a MEC Platform. We assume that the MEC Operator implements the MEC orchestrator and the ETSI MEC standardised interfaces as presented in~\cite{MEC003}. It designs, builds and operates its MEC platform to offer, together with the MEC orchestrator, MEC application slice subnets to ASPs from ACFs that the ASP has provided as an input.
    \item \textit{Application Component Function Supplier}: designs and provides ACFs to ASPs.
\end{itemize}
%
\begin{figure}[t]
\centering
\includegraphics[width=0.4\textwidth]{figures/3_FigureARoles.png}
\caption{High-level functional roles in the MEC application slice framework}
\label{fig:AS_roles}
\end{figure}
%
Interestingly, an ASC can become an ASP while adding new ACFs into consumed ASs and providing new ASs. For instance, let us assume that ASC $C_1$ uses from ASP $P_1$ an AS $S_1$ that provides face recognition capabilities. Then, ASC $C_1$ can integrate into $S_1$ other ACFs that permit to retrieve video stream from customers' cameras and ACFs that take control of customers' door, thus building a new AS $S_2$, which provides automatic door opening based on face recognition. ASC $C_1$ then becomes a new ASP $P_2$ selling the previous new AS to customers such as house/apartment rental platforms. Furthermore, we can easily make a parallel between Figure~\ref{fig:AS_roles} and Figure~\ref{2_figRoles} in Section~\ref{subsec:mgmt_architecture}. Roles responsible for low layers, namely Hardware Supplier, Data Center Service Provider, NFVI Supplier, and Virtualisation Infrastructure Service Provider, remain the same. However, the Application Component Function Supplier has replaced the VNF (or Equipment) Supplier; the MEC Operator has replaced the Network Operator, the Application Service Provider (ASP) the CSP, and the Application Service Customer (ASC) the CSC. 

It is helpful to point out that a business organisation can play one or multiple operational roles simultaneously. Therefore, without trying to be exhaustive in terms of business models, we focus in our work on two categories of organisations whose responsibility boundaries appear to make most of the sense as per our knowledge:
%
\renewcommand\labelitemi{$\bullet$}
\begin{itemize}[noitemsep,topsep=2pt]
    \item \textbf{\textit{MEC Owner}}: it plays both Hardware supplier and NFVI supplier roles. The fact that the MEC Owner also manages the virtualisation infrastructure and the edge servers allows him/her to dynamically split his/her infrastructure into logical partitions or network slices (i.e. greater degree of flexibility). It is noted that the MEC Owner could be the equivalent of what is called the \textit{MEC Host Operator} in \cite{MEC027}, as it offers virtualised MEC hosts and MEC platforms to its tenants. However, we prefer the `MEC Owner' terminology to avoid confusion with the `MEC Operator' role.
    \item \textbf{\textit{MEC Customer}}: plays both the role of the MEC Operator and ASP. It is helpful to point out that the ASP role offers a business interface (with the ASC). In contrast, the MEC Operator role offers the actual enforcement/implementation (i.e. SLS) of the business objectives (i.e. SLA) agreed across the ASP business interface. It is noted that the MEC Operator role alone cannot endorse a business organisation as it only offers Application Slice (including SLS) and not the Application Service (with related SLA). 
\end{itemize}
%
In conclusion, from a business perspective in this study, we advocate a multi-tenancy model in which ASCs are tenants of a MEC Customer, and MEC Customers are tenants of a MEC Owner. As explained in the following section, our proposed multi-tenant MEC system supports data isolation (through separated data planes) and resource orchestration separation (through separate resource orchestrators) between tenants.  
%%
%%---------------------------------------------
\subsection{Deployment models for an E2E application slice}
\label{subsec:AS_deployment}
%%
\noindent
%
We distinguish two distinct deployment models for our proposed application slice concept: $(i)$ the \textit{overlay} model, and $(ii)$ the \textit{stitching} model. 

\begin{figure}[ht]
\centering
\includegraphics[width=0.5\textwidth]{figures/3_DeploymentModels.png}
\caption{\added{Interconnection models between application slice and network slice}}
\label{3_DeploymentModels}
\end{figure}
%
%\begin{figure}[ht]
%\centering
%    \begin{subfigure}[b]{0.5\textwidth}
%         \centering
%         \includegraphics[width=0.8\textwidth]{figures/3_FigureAS.png}
%         \caption{Network slice includes the data plane of the MEC customer}
%         \label{3_figAS}
%     \end{subfigure}
%     \hfill
%    \begin{subfigure}[b]{0.5\textwidth}
%         \centering
%         \includegraphics[width=0.8\textwidth]{figures/3_FigureASoverlay.png}
%         \caption{Network slice does not include the data plane of the MEC customer}
%         \label{3_figASoverlay}
%     \end{subfigure}
%\caption{Overlay deployment model: the application slice is a consumer of an E2E network slice.}
%\label{fig:overlay}
%\end{figure}
%
The first model assumes that the E2E APS is a consumer of a Communication Service offered by the underlying network slice (see Figure~\ref{3_DeploymentModels}\added{a}). In this case, the network slice is responsible for the entire communication latency, and we denote it as E2E \added{network} slice. In addition to the latter, the application slice caters for processing and possibly storage latency at both ends. It is important to point out that the E2E network slice encompasses not only the 5G network slice but also the network slice subnets within the MEC Owner and MEC Customer domains\footnote{Note that we use the term MEC Customer network slice subnet instead of MEC application slice subnet because, in this deployment model, the MEC Customer is only responsible for the management and orchestration of the data plane within its virtualised MEC environment. This is also justified by the need to align the architectural component boundaries with the liability boundaries of the deployment model.}. Indeed, as shown in Figure~\ref{3_DeploymentModels}a our slicing framework leverages recent advantages on virtualisation technologies that allow a virtualisation layer to be composed of multiple nested sub-layers, each using a different virtualisation paradigm. According to the functional role split illustrated in Section~\ref{subsec:roles}, a MEC Owner can use a hypervisor technology to operate its MEC hosts and to deploy multiple virtualised environments to MEC customers (e.g. allocating one or more VMs\footnote{In ETSI NFV terminology, a VM is also designated as `NFV VNF'.} using an NFVI). Each virtualised environment includes a full-fledged MEC system used by the MEC Customer to allocate further the resources assigned to its VMs to the multiple Application Services it deploys to its users by applying internal orchestration and management criteria. In the case of the MEC customer, a container-based virtualisation technology could be used as a top virtualisation sub-layer to manage application deployment into its allocated virtualised MEC system. It is worth noting that in \added{Figure~\ref{3_DeploymentModels}a} the E2E network slice includes the data plane of the MEC customer. However, an alternative overlay model is also possible in which the E2E network slice terminates at the network boundary between the MEC Owner and the MEC Customer as per \added{Figure~\ref{3_DeploymentModels}b}.

In the stitching deployment model \added{(see Figure~\ref{3_DeploymentModels}c)}, we assume that the MEC application slice subnet is a peer of the network slice subnets. Virtual appliances using a subnet border API, such as a virtual gateway or virtual load-balancer, can be used to interconnect MEC application slice subnets to the adjacent network slice subnet, similarly to what was proposed in~\cite{2020_MNET_MEC_subslice} and \cite{2021_TNSM_e22_slice_survey}. Such stitching could be a one-to-one interconnection as well as a multiple-to-one interconnection. The end-to-end application slice could be seen in this case as the composition of different application slice subnets (UE-operated or MEC-operated) together with network slice subnets. Latency-wise, the MEC application slice subnet is responsible, in this case, for a tiny part of the network latency budget in addition to the processing and storage latency induced by the MEC applications and their related ACFs.
%
%\begin{figure}[ht]
%\centering
%\includegraphics[width=0.4\textwidth]{figures/3_FigureASubS.png}
%\caption{Stitching deployment mode: interconnection of the different slice subnets to form the E2E application slice.}
%\label{3_figASstitch}
%\end{figure}
%

We conclude this section by noting that the different deployment models will lead to different approaches to combine our proposed E2E application slice management/orchestration framework with the 3GPP management architecture, as explained in Section~\ref{subsec:AS_Mgmt}.

%Alternatively, the MEC application slice subnet network connectivity could be performed as an overlay network on top of the end-to-end network slice (see Figure \ref{3_figASoverlay}). This architecture is then equivalent to the one illustrated in Figure \ref{3_figAS} except that the end-to-end network slice terminates this time at the network boundary between the MEC Owner and the MEC Customer.
%
%\begin{figure}[ht]
%\centering
%\includegraphics[width=0.4\textwidth]{figures/3_FigureASoverlay.png}
%\caption{Application slice as overlay on top of E2E network slice}
%\label{3_figASoverlay}
%\end{figure}

%%
%%---------------------------------------------
\subsection{Architecture for application slicing in a multi-tenant MEC system}
\label{subsec:new_MEC_arch}
%%
\noindent
%
In this section, we elaborate on the new MEC components and extensions to the current MEC management architecture that are needed to support E2E application slicing and multi-tenancy within multiple MEC customers. Figure~\ref{fig:ext_arch} shows an illustrative example of the proposed extended MEC reference architecture. The primary design rationale of our proposal is that the MEC system should be split into two responsibility domains following a \textit{two-layer hierarchical MEC architecture}, where the bottom layer is managed and orchestrated by the MEC Owner, and the top-layer is independently managed and orchestrated by MEC Customers. Such a hierarchical architecture allows a single MEC deployment to host multiple MEC Customers. Each of them has his own MEC network slice subnet (i.e. his dedicated data plane provided by the MEC Owner) with related management capability. In turn, each MEC Customer manages and orchestrates his own MEC application slices. 
%
\begin{figure*}[ht]
\centering
\includegraphics[clip,trim= 0cm 0cm 0cm 0cm,width=0.8\textwidth]{figures/3_FigureMECCustOwner.png}
\caption{Multi-tenant MEC architecture supporting network and application slicing}
\label{fig:ext_arch}
\end{figure*}
%

Implementation-wise, the proposed two-layer MEC architecture is enabled by the nested virtualisation capability of the MEC infrastructure, as anticipated in Section~\ref{subsec:AS_deployment}. In the system illustrated in Figure~\ref{fig:ext_arch}, the MEC Owner does not deploy individual MEC entities as in the MEC-in-NFV reference architecture, but a collection of 'ETSI VNFs' (or VMs) to provide each MEC Customer with a complete MEC system. Differently from~\cite{Cominardi2020} we do not use 'ETSI VNFs' to deploy MEC applications and MEC platforms but to deploy a virtualised MEC environment encompassing virtualised MEC hosts and a virtualised MEC management system. Furthermore, in our MEC-in-NFV architecture variant, we introduce functional blocks that substitute the \added{MEAO} and \added{MEPM-V} of the original reference architecture. Specifically, we substitute the \added{MEAO} with the \textit{MEC Owner Orchestrator} (MEOO), which is in charge of implementing the policies to select the MEC infrastructures on which to deploy a MEC Owner network slice subnet. As explained in Section~\ref{subsec:AS_Mgmt}, the MEOO receives the commands to create, modify or delete a MEC Owner network slice subnet from a 3GPP management function called MEC NSSMF. Furthermore, the MEOO collaborates with the MEC Owner NFVO to provide a dedicated data plane to each MEC Customer. For the sake of example, we can assume that the MEC Owner offers to each MEC Customer a dedicated Kubernetes cluster, where each Kubernetes node is deployed as an 'ETSI VNF' (or VM) in the NFVI, which is connected to the 5G Core (5GC) via a dedicated data plane (MEC Customer network slice subnet). The second new functional block is the \textit{MEC Network Slice Manager} (MENSM), which delegates the life-cycle management of the 'ETSI VNFs' to a dedicated VNFM, while it is responsible for the management of the network slice subnet (data plane) parameters. For instance, it can reserve network bandwidth between MEC hosts for a given MEC Customer. More over, the MEC Network Slice Manager could behave like an 3GPP Application Function (AF) which interacts with the 5GC to synchronise data plane forwarding rules to realise local breakout traffics to/from MEC applications.

As previously discussed, each MEC Customer manages and orchestrates his own MEC application slices within the assigned virtualised MEC system. To this end, each MEC Customer implements a \textit{MEC Customer Orchestrator} (MECO), which receives the commands to create, modify or delete MEC application slices from a management function called MEC ASSMF (see Section~\ref{subsec:AS_Mgmt} below for more details). Furthermore, the MECO collaborates with the MEC Customer Platform Manager (MECPM) to manage the MEC application slice life-cycle and the MEC Platform instance (e.g. embodied as a containerised application). In order to stitch the application slice subnets to the adjacent network slice subnets, the MENSM creates dedicated VNFs (e.g. gateways) that it communicates to the MECO (or at least the gateway endpoints), similarly to~\cite{2021_TNSM_e22_slice_survey}. A collaboration between the MECO and the MEOO could also be needed in case of MEC application relocation to enforce new 5GC forwarding rules or to tear down old ones.

With regards to the interaction with the 5GC, there are two possible options:
%
\renewcommand\labelitemi{$\bullet$}
\begin{itemize}[noitemsep,topsep=2pt]
    \item The MEC Owner provides a network slice (i.e. a 'big pipe') to the MEC Customer, which directly manages via its MEC Platform (MEP) 5GC forwarding rules for each application slice (e.g. adds new DNS rules to the 5GC local DNS servers). This solution allows for better preserving privacy as the MEC Customer is the only entity that manages the data traffic produced by its own customers' UEs.
    \item The MEC Customer MEP collaborates (e.g. via the MECO and the MEOO) with the MEC Owner Network Slice Manager, which ultimately influences 5GC traffics. This solution allows for the MEC Customer to delegate the interaction with 5GC to the MEC Owner. The latter can aggregate requirements in order to optimise network resources (e.g. bandwidth). Thus, this solution allows for better network optimisation at the MEC Owner infrastructure but does not preserve privacy. Also, it may be less scalable as the number of UEs increases.
\end{itemize}
%
We conclude this section by noting that the standard MEC reference architecture~\cite{MEC003} entails a single MEC orchestrator controlling a single virtualisation infrastructure and managing the instantiation of all MEC applications. Our proposed MEC architecture variant implies a split of the MEC orchestrator responsibilities into a MEC Customer Orchestrator (MECO) and a MEC Owner Orchestrator (MEOO). While the former is responsible for MEC Platform, MEC applications, MEC application slices, and related external interfaces, the latter is responsible for the hardware, the NFVI, MEC NFVI slices (especially MEC network slices) and related external interfaces. 

%%
%%---------------------------------------------
\subsection{Application Slice Management Architecture}
\label{subsec:AS_Mgmt}
%%
\noindent
%
With the aforementioned new roles and architecture in mind, the 3GPP network slice management architecture could also be augmented to manage and orchestrate application slices as illustrated in Figure~\ref{fig:aps_mngmt}.
%
\begin{figure}[ht]
\centering
\includegraphics[clip,trim= 0cm 0cm 0cm 0cm,width=0.5\textwidth]{figures/3_FigureASMgmt.png}
\caption{3GPP-compatible joint network and application slice management architecture.}
\label{fig:aps_mngmt}
\end{figure}
%
We assume that an ASC relies on a web portal to request an application service with a given SLA from a catalogue of offered ASs (see Section~\ref{sec:implementation} for more details on how to implement such service catalogue). The business interaction of the web portal can happen in two manners depending on the AS/APS deployment model that is used in the system (presented in Section~\ref{subsec:AS_deployment}). In both cases, the web portal communicates with a new management function, called \textit{Application Service Management Function} (ASMF), which is responsible for translating the SLA of the requested AS to the SLS of an APS and to trigger the creation of the APS instance by contacting a new management function called \textit{Application Slice Management Function} (APSMF). The APSMF splits the APS into multiple subnets, one for each domain over which the requested APS spans, including possibly the network and the AS's endpoints, namely the UE requesting the AS, and the edge system instantiating the AS. To this end, we introduce the new \textit{Application Slice Subnet Management Functions} (APSSMFs), which apply the APSS life-cycle management commands within the two potential domains that are relevant for an application slice subnets, namely the UE and the MEC.

In the overlay model (label 1 in Figure~\ref{fig:aps_mngmt}), the E2E ASMF is also responsible for translating the E2E AS SLA into E2E CS SLA and for requiring the adapted E2E CS from the CSMF. In the stitching deployment model, two alternative management flows are feasible. In the first case (label 2 in Figure~\ref{fig:aps_mngmt}), the end customer is an ``expert'' user and he is directly responsible for breaking (using the web portal) the E2E AS SLA into an E2E CS SLA and an SLA with a scope restricted to AS's endpoints. Then, the web portal is used for requiring the adapted E2E CS from the CSMF. In the other case (label 3 in Figure~\ref{fig:aps_mngmt}), the end customer is not an expert of the network domain and he does not need to perform the aforementioned E2E AS SLA splitting. On the contrary, the the APSMF is responsible to communicate directly with the NSMF to manage the network slice subnet associated with the APS. Finally, we remind that the management of network slice via the NSMF and the one of per-domain network slice subnets via NSSMFs are well-defined by 3GPP, and they do not need to be extended~\cite{3GPPTS28531,3GPPTS28541}. 

%The life cycle management of the APS can happen in two manners depending on the APS deployment models that is used in the system (see Section~\ref{subsec:AS_deployment}). In the case of the stitching deployment model (label (2) in Figure~\ref{fig:aps_mngmt}), ASMF delegates to a new management function, called \textit{Application Slice Management Function} (APSMF) the deployment and management, in an end-to-end manner, of the APS associated with the requested AS. Specifically, the APSMF splits the APS into multiple subnets, one for each domain over which the APS spans, including both the network and the AS's endpoints, namely the UE requesting the AS, and the edge platform instantiating the AS. To this end, we introduce the new \textit{Application Slice Subnet Management Function} (APSSMF), which applies the APSMF's life-cycle management commands within the two potential domains that are relevant for an application slice subnets, namely the UE and the MEC. Finally, the APSMF communicates directly with the NSMF to manage the network slice subnet associated with the APS. 

%In the case of the overlay deployment model (label (1) in Figure~\ref{fig:aps_mngmt}), the APS is a consumer of the Communication Service offered through the underlying E2E network slice. Thus, the ASMF communicates directly with the CSMF to invoke the desired CS. However, the ASMF also needs to communicate with the ASPMF to manage the application slice subnets in the UE and MEC domains. We remind that the management of per-domain network slice subnets via NSSMFs is well-defined by 3GPP, and it does not need to be extended~\cite{3GPPTS28531,3GPPTS28541}. 

%\Thai{--- beginning : proposed ideas }
%\emph{Application Slice Subnet} vs \emph{Application Sub-Slice}: similarly to the network slice management operations, the E2E ASMF and the Edge ASSMF could pertain or not to the same operator. In the first case, the E2E ASMF splits the E2E application slice into domain-specific \emph{Application Slice Subnets} and provides to the related ASSMF the correspondent Application Slice Subnet template. In the second case, the domain-related \emph{Application Slice Subnet} is wrapped by the E2E ASMF in an \emph{Application Sub-Slice} which is in its turn wrapped in an Application Service. The E2E ASMF communicates, in this case, to the domain ASSMF an Application Service template (as per business-level relationship). 

%Similarly to ETSI NFV Network Service concept (Cf. section \ref{subsec:nw_slice_nw_service}), we define a \emph{ETSI-Like Application Service or ELAS} (not to make confusion with Application Service concept) as a composition of Application Component Functions (ACFs) and possibly of other ELAS (i.e. nested ELASes). 

%ACFs can be composed into ELAS in the form of Application Function Forwarding Graph (ACFFG) similarly to VNFFG described in section \ref{subsec:nw_slice_nw_service}. \Thai{For simplicity, we will call the ACFFG a Service Graph in the rest of the document?}
%\Thai{--- end : proposed ideas }

%
\begin{figure}[ht]
\centering
\includegraphics[clip,trim= 0cm 2.5cm 14cm 7.5cm,width=0.4\textwidth]{figures/fig_aps_mngmt_focus.pdf}
\caption{Illustration of the orchestration entities that are involved in the MEC domain.}
\label{fig:ps_mngmt_focus}
\end{figure}
%
In the remaining of this paper, we will describe our experience in implementing the management architecture described above. As shown in Figure~\ref{fig:ps_mngmt_focus}, several orchestration entities are involved in the management of the various application slice subnets. Our focus will be on implementing the MEC Customer Orchestrator using a popular open-source container orchestration platform. Furthermore, we will detail the interfaces and data models needed to interact with the MEC APSSMF, allowing the deployment of isolated MEC application slice subnets composed of ACFs in the form of Docker containers.  

%\Thai{Operationally, the AS descriptors are received by the MEC customer from a ...\newline
%Sharing of ACFs: concept of Network slice subnet which allows for the management of NFs independently of the network slice - a network slice subnet can be shared between two network slices. similarly, we could define an application slice subnet which is a set of ACFs}

%%
%%---------------------------------------------
\section{Implementation Experience}
\label{sec:implementation}
%%
\noindent
%
In this section we describe our approach to implement a MEC Customer Orchestrator and to support MEC application slicing using Kubernetes and Helm technologies. Although our implementation is still at a work-in-progress state, our proof-of-concept prototype (hereafter simply PoC), shows the feasibility of our design approach and allowed us to collect a preliminary insight on the efficiency of application slicing using Kubernetes resource management capabilities. In the following, we first briefly overview Kubernetes and Helm features. Then, we detail the implementation of the new APIs and functional components  of our PoC,  namely the MECO and the ACF Image Repository. Finally, we describe our practical approach to support slicing of Kubernetes resources. For the sake of presentation clarity, Figure~\ref{fig:poc} overviews the architecture of the PoC and its internal components.
%
\begin{figure*}[ht]
    \centering
    \includegraphics[clip,trim= 1cm 4.5cm 0cm 2.5cm,width=0.7\textwidth]{figures/fig_poc.pdf}
   \caption{PoC architecture and internal components.}
    \label{fig:poc}
    \vspace{-0.2cm}
\end{figure*}
%
%%
%%---------------------------------------------
\subsection{PoC enabling technologies}
\label{sec:poc_tech}
%%
\noindent
%
Kubernetes is an open-source container life-cycle manager and orchestrator, which is the de-facto standard for running container-based cloud native applications on a cluster of (physical or virtual) machines (called \textit{nodes}). It is out of the scope of this paper to describe the complete Kubernetes architecture and high-level abstractions, but we focus on the components that are most relevant for our PoC. 

A Kubernetes cluster is composed of $(i)$ a set of \textit{worker} nodes that run containerized applications, also called \textit{workloads}; and $(ii)$ (at least) a \textit{master} node that runs the services of the \textit{control plane}, and it is responsible to enforce the desired state of the cluster. Kubernetes provides several built-in workload resources to support various application behaviours (e.g., stateless tasks) and management functions (e.g., creating or deleting application replicas). Each workload must run into a \textit{Pod}, a Kubernetes object that represents a collection of containers running in the same execution environment, which share the same storage, networking, and lifecycle. Pods runs into worker nodes, which host an agent, called \texttt{kubelet}, that is responsible for managing worker's local containers and for synchronising the status with the master node. As better explained later, another main component of the worker node is the \texttt{kube-proxy}, which is responsible for implementing Kubernetes networking services and to enable communication to Pods from inside and outside the cluster. The master node is composed of different components including:$i)$ \textit{etcd}, a distributed key-value store that holds and manages all cluster critical data; $ii)$ \texttt{kube-apiserver}, a component providing a REST-based frontend to the control plane through which all other components interact; $iii)$ \texttt{kube-controller}, a component that monitors the shared state of the cluster using apiserver and runs the controller processes; and $iv)$ \texttt{kube-scheduler}, a component that assigns newly created Pods to nodes.  Note that Kubernetes also support the \textit{namespace} abstraction, namely virtual clusters that share the same IP Address and port space, facilitating the grouping and organisation of objects.    

Networking is a central part of the Kubernetes design and a fundamental capability for application slicing. Specifically, the Kubernetes network model demands certain network features, such as every Pod should have a unique, routable IP inside the cluster, and inter-pod communications should happen without using NATs, regardless of wherever the Pods reside or not on the same worker nodes (i.e. network segmentation is not allowed). Since IP addresses of Pods are ephemeral and change whenever the Pods are restarted or migrated, Kubernetes also defines \textit{Service} resources, namely REST objects that are used to group identical Pods together to provide a consistent means of accessing them, e.g by bounding them to a virtual IP address (called cluster IP) that never changes. It is important to point out that 
Kubernetes does not directly handle the networking aspects, but it rather allows the use of third-party networking plugins that adhere to the Container Network Interface (CNI) specification\footnote{\url{https://github.com/containernetworking/cni}} to manage the containers' data plane. Of particular relevance for our PoC, is Kube-OVN\footnote{\url{https://www.kube-ovn.io/}}, a CNI plugin that integrates network virtualisation into Kubernets  by leveraging the OVN (Open Virtual Network)\footnote{\url{https://www.ovn.org/en/}} technology. Kube-OVN supports advanced features, such as unique subnets per namespace, network policies, namespaced gateways, subnet isolation and dynamcic QoS. We extensively leverage some of those features to support application slicing in our PoC. 

Another technology we use as a basis for our PoC is Helm\footnote{\url{https://helm.sh/}}, an application packaging manager for Kubernetes. Specifically, Helm defines a data format, called \textit{Helm Chart}, to bundle a set of Kubernetes object definition files (i.e. files describing properties of Kubernetes objects) into a single package. This permits to manage the instantiation, upgrade and deletion of the corresponding Kubernetes objects as they were a single entity. The Helm charts are stored into a separate repository, and every time a new instance of the same chart is installed into Kubernetes, a new chart \emph{release} is created. Furthermore, Helm allows to augment Kubernetes object definition files with Helm template commands. By providing Helm a list of arguments for these template commands at chart instantiation time, it is possible to dynamically customise the chart before it is actually deployed.  Examples of these customisation span from overriding object default values with the one passed as command arguments (e.g. a port exposed by a container, the namespace name, etc..) to dynamically enabling disabling chart sections. As explained later, we extensively leveraged Helm features to implement different components of our PoC, such as the KDPManager and the MAPSS Chart Registry (see Figure~\ref{fig:poc}).

We conclude this section by noting that the ETSI MEC standard has recently started analysing how MEC features should be adjusted when deploying a MEC system using a container-based virtualisation technique~\cite{MEC027}. Furthermore, a few initial implementations exist of specific MEC components and interfaces using Kubernets as VIM, such as Akraino\footnote{\url{https://www.lfedge.org/projects/akraino/}} and LightEdge\footnote{\url{https://lightedge.io/}}. A recent work~\cite{2020_broadnets_mec_k8s} analyses how to use Kubernetes not only as a VIM, but also as the core of the MEPM, also leveraging Helm technology for the life-cycle management of MEC applications. 
%
%%
%%---------------------------------------------
\subsection{PoC design and development}
\label{sec:poc_design}
%%
\noindent
%
One of the main objectives of our proof-of-concept implementation is to demonstrate how Kubernetes can natively support multiple, isolated instances of MEC application slice subnets (MAPSSs for brevity) as defined in Section~\ref{sec:pre_concepts}. The key building block of a MAPSS is the ACF. For the sake of simplicity, in our PoC we ignore VNFs and we assume that ACF Suppliers can provide ACFs in the form of Docker container images coupled with an ACF user manual or descriptor (called ACFD). Then, an ASP leverages the ACFDs to select the set of ACFs that are needed to build the AS requested by its end customers, as well as to derive the proper run-time configuration of the graph of ACFs composing the AS. Therefore, a key component of our PoC is the \textit{ACF registry}, where authorised ACFDs are published and stored. The ACF registry is implemented using the open-source Docker Registry 2.0 application\footnote{\url{https://hub.docker.com/_/registry}}, a storage and distribution system for named Docker images. A generic ACFD is structured into two separate sections. The first one details the RESTful APIs exposed by the ACF (OpenAPI\footnote{\url{https://www.openapis.org/}} is used to specify these APIs in a standard, language-agnostic format). The second section details how to properly configure the ACF parameters in order to control how the ACF will behave at run-time. The ACF parameter customisation is a crucial aspect to consider, especially in the context of ACF composition and automated orchestration. An example of ACF customisation is the selection of buffer sizes, which may influence the run-time behaviours and performance of ACFs, directly affecting the fulfilment of SLA requirements. For the sake of simplicity, in our PoC we use \textit{environmental variables} to pass configuration paramters to the running container images of ACF. Finally, it is important to point out that the ACF registry must be accessed also by the \texttt{kube-scheduler} agent to fetch the container images of ACFs to be deployed.

The other key component of our PoC is the MECO that we have implements as an additional component of the master node, using Go as programming language. Internally, the MECO component is composed of three different modules: $i)$ the \textit{MECO API Server} to enable communications with the MEC APSSMF; $ii)$ the \textit{MAPSS Chart Repository} to store the templates of Kubernetes deployment plans of MAPSS; and $iii)$ the \textit{KDPManager}, to manage the Kubernetes deployment plans of run-time instances of activated MAPSS. In the following, we elaborate on the purpose and operations of each module more in detail. 

The main role of the MECO API Server is to act as a lifecycle management proxy for MAPSSs. Specifically, it receives commands for the instantiation, update and termination of MAPSS from the MEC APSSMF, and translates these commands into suitable Kubernetes actions. To this end, the MECO API Server exposes a RESTful management interface, called \texttt{mapss\_mm} API, defined using the OpenAPI description language (see Figure~\ref{fig:poc}). For the sake of the experimentation, the \texttt{mapss\_mm} API currently implements a \texttt{POST} method, which allows the MEC APSSMF to request the instantiation of a new MAPSS instance. The payload of this \texttt{POST} method carries a descriptor, called MAPSSD, which contains all the necessary information to allow the MECO to instantiate at run-time a specific MAPSS instance. The \texttt{mapss\_mm} API also implements a DELETE method, which allows to delete a running instance of a MAPSS. It is clear from this discussion, that the MAPSSD plays a key role in our PoC. We envision a MAPSSD organised into four different sections, as also illustrated in the example in Figure~\ref{fig:poc_massd_example}. The first parts includes the unique identifier of the MAPSS instance (\texttt{mappsiId}), a human readable description of the slice subnet features, and the identifier of an implementation template to be used for the deployment of the MAPSS instance (\texttt{mapssImplTemplateId}) -- see later this section for more details on how to use the \texttt{mapssImplTemplateId}. The second section includes a set of ``\textit{slice-subnet-wise}'' (computational, storage and networking) resource requirements. For example, the MAPSSD shown in Figure~\ref{fig:poc_massd_example} requires two dedicated CPU cores, 8 GB of memory and 100 GB of permanent storage, which will be shared among its constituent ACFs. The third part includes the list of ACFs that compose the MAPSS instance. Each ACF is associated to a unique identifier (\texttt{acfId}), and ``\textit{acf-wise}'' resource requirements can also be specified. For example, the acf1 in Figure \ref{fig:poc_massd_example} requires a dedicated CPU core out of the two dedicated to the whole slice subnet. Moreover, the field \texttt{customParams} can be leveraged to pass arguments (e.g. buffer size in figure) to configure specific ACF behaviours. Finally, the last section includes a list of virtual links among pairs of ACFs and their networking requirements (e.g. maximum usable bandwidth). It is worth noting that the data format of the MAPSSD is agnostic from the underlying virtualisation technology. In other words, the MEC APSSMF could leverage the same data model to interact with a MECO that relies on a different VIM than Kubernetes. Another advantage of the proposed MAPSSD is that it allows the MEC APSSMF to seamlessly integrate SLS requirements that address needs of different architectural levels (i.e. specific to the entire slice subnet, individual ACFs and virtual links between ACFs) in the same data object. 
%
\begin{figure}[th]
    \centering
    \includegraphics[clip,trim= 0cm 16cm 10cm 0cm,width=0.5\textwidth]{figures/fig_poc_mappsd.pdf}
   \caption{Illustrative example of a MEC application slice subnet descriptor (MAPSSD)}
    \label{fig:poc_massd_example}
\end{figure}
%

Clearly, the translation from a VI-agnostic MAPSSD into a Kubernetes deployment plan (\textit{KDP} for short) of the MAPSS instance, namely a package of properly configured Kubernetes objects implementing the requested MAPSS instance, is a critical functionality of the MECO. Following the approach proposed in~\cite{NGMN028} for supporting cost-efficient customisation of network slices, we assume that the MECO hosts a pre-loaded set of MAPSS templates/blueprints that can be used to speed up the creation of a MAPSS instance. Specifically, we implemented each MAPSS KDP template as an Helm chart that includes a set of pre-configured Kubernetes objects. These objects define: $i)$ ACFs Docker containers to run (e.g. via Kubernetes Pods objects), $ii)$ ACFs behavioural parameters (e.g. via environmental variables in Kubernetes ConfigMaps objects), $iii)$ ACFs connection points (i.e. exposed ports), $iv)$ custom scheduling policies (e.g., number of replicas, failure behaviour, etc.), and any other Kubernetes feature that is necessary for the correct deployment of the MAPSS instance. Then, the \textit{mapssImplTemplateId} field of the MAPSSD is used to retrieve the correct MAPSS KDP template. It is important to point out that the MECO should be able to dynamically customise at run-time the MAPSS KDP template using information derived from the MAPSSD (e.g., container resource requirements). To this end we leverage built-in objects and control structures of Helm template that provide access to values passed into an Helm chart, and the ability to include conditions in the template's generation. In the current implementation, we limit such customisation to the selection of: $i)$ the name of the namespace to which objects will belong; $ii)$ the computational, storage and networking requirements for the namespace; $iii)$ the computational and storage requirements per ACF, and the networking requirements per ACF pairs. In our Poc the MAPSS KDP templates are stored in the MAPSS Chart Registry (see Figure~\ref{fig:poc}). According to the operational and management roles defined in Section~\ref{subsec:roles}, the MEC Customer plays the roles of both the MEC operator and the ASP. Thus, the MEC Customer has the necessary expertise not only to properly select, compose and configure ACFs to provide an AS, but also to select and properly configure the subset of Kubernetes objects that allows to implement the MAPSS instance of the designed AS. Finally, the KDPManager module is simply a wrapper of the Helm library, which allows the MECO to embed the Helm functionalities.   

%
\begin{figure*}[ht]
    \centering
    \includegraphics[clip,trim= 0cm 2cm 0cm 2cm,width=0.9\textwidth]{figures/fig_poc_seq_diag.pdf}
   \caption{Sequence of operations to instantiate a new MAPSS in the PoC architecture.}
    \label{fig:poc_seq_diag}
\end{figure*}
%
We can now discuss the sequence of operations and request/response exchanges that are executed to deploy a new MAPPS instance in our Poc, which are also graphically illustrated in Figure~\ref{fig:poc_seq_diag}. First of all, the MEC APSSMF initiates the deployment process by sending a \texttt{POST} request to the MEC API Server over the \texttt{mapss\_mm} interface (see Figure~\ref{fig:poc}), which contains the MAPSSD of the requested MAPSS. In the figure, the requested MAPSS is identified as \textit{demoSlice}, while its Helm chart template is identified as \textit{demoTpl}. In \textit{step2}, the MECO API Server performs a preliminary analysis of the MAPSSD to discover the set of parameters that can be modified to customise the template. Furthermore, the MECO API Server retrieves from the \textit{mapssImplTemplateId} field of the MAPSSD the identified of the associated Helm chart template (\textit{dempoTpl} in this example). Finally, \textit{step2} is concluded with the API Server that instructs the KDPManager to deploy the \textit{demoTpl} Helm chart with the correct set of chart arguments. Subsequently, in \textit{step3}, the KDPManager fetches the \textit{demoTpl} Helm chart from the MAPSS Chart Repository, and it starts the deployment process. First, in \textit{step4}, the KDPManager contacts the \texttt{kube-apiserver} to create a new Kubernetes namespace with name \textit{demoSlice}. Then, the KDPManager applies (\textit{step5}) the customised arguments (e.g., number of cores to be assigned to an ACF container), and starts (\textit{step6}) chart release process, using \textit{dempoSlice} as release name. This process involves the generation of the proper set of Kubernetes objects definition files (i.e. the Kubernetes Deployment Plan). Once the KDP is complete, the KDPManager instructs the \texttt{kube-apiserver} to create the Kubernetes objects in the etcd database (\textit{step7}). Finally, in \textit{step8}, the \texttt{kube-controller} starts performing control actions according to the received Kubernetes objects. The latter includes contacting the ACF Image Repository to fetch ACFs container images for scheduling. For the sake of completeness, we remind that a termination of a run-time instance of a MAPSS is initiated by a \texttt{DELETE} request sent by the MEC APSSMF to the MECO via the \texttt{mapss\_mm} interface. This \texttt{DELETE} contains the \textit{mapssId}) of the MAPSS instance to delete. In this case, the MECO API Server requests the KDPManager to uninstall the chart release associated to \textit{mapssId}. Finally, the KDPManager removes all the Kubernetes objects of the release from Kubernetes cluster, and deletes the \textit{mapssId} namespace.

We complete the presentation of our PoC by explaining how application slice isolation is enforced using the Kubernetes resource control objects. First of all, we create a new Kubernetes namespace for each MAPSS instance, with a name equal to the \textit{mapssId}. All Kubernetes objects within a MAPSS instance are deployed using the same MAPSS namespace. We leverage a combination of Kubernetes \textit{ResourceQuota} objects, \textit{QoS classes} for Pods, and Kubernetes \textit{VolumeClaim} requests to limit the amount of computational and storage resources that could be consumed by both the whole namespace (to enforce requirements for invidual slice subnets) and individual Pods (to enforce requirements for individual ACFs). Network isolation between different instances of MAPSSs is implemented by exploiting Kube-OVN network policies, so that that the traffic arriving from Pods belonging to other namespaces is blocked (except for the system namespace). Finally, we implement per Pod ingress/egress rate limitation via Kube-OVN QoS policies. The implementation of more advanced QoS-aware network control policies (e.g. latency assurance, QoS tagging, etc.), and fine-grained network isolation policies (e.g. tunable network isolation degree with exception handling, etc) is planned as future work.


We conclude this section by  observing that an ASC could discover available ASes and their features by querying a catalogue that is exposed by a web portal (see Figure~\ref{fig:poc}), on which ASPs publish the descriptors of their ASes. We can foresee that AS descriptors include information such as $i)$ high level description of the offered AS; $ii)$ a pointer to the ASP offering the AS; $iii)$ a set of achievable SLAs (e.g. maximum resolution of a video processing service); and, possibly, $iv)$ billing information. The definition of a data model for AS descriptors is out of the scope of our present work. 
%
%%
%%---------------------------------------------
\subsection{Open implementation gaps}
\label{sec:poc_limitations}
%%
\noindent
%
During the implementation of our PoC we also faced several difficulties due to the limitations of the technologies and standards we have used. In the following, we summarise the main technological gaps we have observed to highlight areas of future investigations.

\begin{itemize}[noitemsep,topsep=2pt]
    \item ETSI MEC specification has defined the methods and the data formats for the \texttt{Mm1} reference interface between the OSS and the MEO, which is used to trigger the instantiation and the termination of MEC applications in the MEC system. However, the \texttt{Mm1} implicitly consider a MEC application as a single application package. For instance, in the MEC-in-NFV architecture, the \texttt{Mm1} allows the MEAO to  deploy a single VNF onboarded as a VNF descriptor. In our use case, a MAPSS instance represents a set of ACFs, and it could be conveniently modelled as a graph. To some extent the \texttt{Mm1} interface should be expanded to resemble the capabilities of the \texttt{Os-Ma-nfvo} interface between the NSSMF and the NFVO~\cite{NFV-SOL005} , which allows the NSSMF to request a network service (i.e. a collection of VNFs) to the NFVO.
    %
    \item Kubernetes ResourceQuota objects only permit to limit the amount of CPU and memory resources that Pods in a namespace could use. Pods can get assigned to a ``Guaranteed'' (highest priority) QoS class to receive reserved CPU and memory resources. VolumeClaim requests allow to reserve storage resources to a scheduled Pod. However, Kubernetes does not provide a straightforward mechanism to allocate resources at the namespace level but only at Pod level. This limitation complicates the implementation of resource over-provisioning strategies in dynamic slicing context (e.g. when a slice subnet is assigned more resources than needed to accommodate future demand changes).
    %dynamic slicing (e.g. when an application slice changes the set of deployed ACFs) 
    %
    \item Default scheduling mechanisms available in Kubernetes take into account only CPU and RAM usage rates when scheduling Pods, while network-related metrics (e.g. latency or bandwidth usage rates) are often ignored. However, a network-aware resource allocation and scheduling is crucial for our application slicing model, and initial proposals can be found in~\cite{2019_netsoft_netaware_kube} and~\cite{2020_noms_delayaware_kube}.
    %
    \item Kube-OVN allows to limit the transmission rate on both ingress and egress traffic at the Pod level. This is obtained by using a QoS-aware queue and traffic policing at the vswitch port to which the Pod is connected. However, Kube-OVN does not support to set up rate limits on individual traffic flows, which is an useful feature if a Pod needs to communicate with several other Pods (e.g., for inter-slice communications). A possible workaround is to leverage Multus CNI\footnote{\url{https://github.com/k8snetworkplumbingwg/multus-cni}}, a CNI plugin for Kubernetes that enables attaching multiple network interfaces to pods, to allow a Pod to have a dedicated virtual interface (i.e. network port) for each destination Pod. Then, separated instances of Kube-OVN could be installed on each virtual interface to enforce different QoS policies at the port level. Furthermore, bandwidth reservation mechanisms similar to the ones proposed for SDN-based networks~\cite{2016_bwguar_openflow} should be included in Kube-OVN. 
    %
    \item Service chaining allows to link together VNFs to compose service function chains (SFCs). The implementation of SFCs usually requires support from the network (e.g. via SDN) to route a packet from one VNF to the next in the chain. However, service chaining (which is a crucial feature for integrating VNFs with ACFs in a MAPSS is missing in Kubernetes. Recently, a few projects, such as OVN4NFV K8s Plugin\footnote{\url{https://github.com/opnfv/ovn4nfv-k8s-plugin}} and Service Meshes\footnote{\url{https://istio.io/}} have been initiated to provide support for service chaining in Kubernetes environments . 
    
    %and Network Service Mesh\footnote{\url{https://networkservicemesh.io/}} have been initiated to provide support for service chaining in Kubernetes environments . 
    %
    \item The MAPSSD provides the blueprint for building an application slice subnet within a MEC environments. For the sake of our PoC, we have defined a custom data model for specifying a MAPPSD. On the other hand, standard modelling languages exist, such as TOSCA (Topology and Orchestration Specification for Cloud Applications) for describing the components of a cloud application and their relationships, which facilitate the interoperability, portability and orchestration in a multi-cloud environment~\cite{2018_MCC_TOSCA}. ETIS MANO already advocates the use of TOSCA to specify NFV descriptors~\cite{NFV-SOL001}. In principle, TOSCA could also be leveraged to specify the MAPSS descriptors. However, TOSCA is specifically designed to model classical cloud applications and it needs some adaptations to natively support also contenarised applications. Several approaches have been recently proposed to either $(i)$ extend the TOSCA normative types for support of container-based orchestration platforms (e.g. Cloudify\footnote{\url{https://cloudify.co/}}); or $(ii)$ to decouple the application modelling from the application provisioning by developing ad hoc software connectors between TOSCA workflow and cloud provider’s API (e.g. TORCH~\cite{2021_jgc_torch}). However, no standard specifications have been released yet.  
    %
\end{itemize}



\section{Evaluation}

\subsection{Experimental Setup}\label{subsec:exp_setup}

In our evaluation, we adopt \textit{Code-Llama-7B} as the pre-trained model for fine-tuning, employing the low-rank-adaption (QLoRA)~\cite{hu2021lora, dettmers2024qlora} technique for faster training and lower memory consumption. 
Key configurations include loading the model in 8-bit, a sequence length of 4096, sample packing, and padding to sequence length. We set the warmup steps to 100, with a gradient accumulation of 4 steps, a micro-batch size of 4, and an inference batch size of 2.
For both syntax and functionality checks, we measure pass@3 accuracy as metrics. In the ablation study from~\secref{subsec:exp_finetune} to~\secref{subsec:complexity}, we adopt \textit{MachineGen} for evaluation.

Experiments are conducted on a server with four NVIDIA L20 GPUs (48 GB each), an 80 vCPU Intel® Xeon® Platinum 8457C, and 100GB of RAM. This setup ensures sufficient computational power and memory to handle the intensive demands of fine-tuning and inference efficiently, especially for long data sequences in the feedback loop experiment. 

\subsection{Effect of Supervised Finetuning}\label{subsec:exp_finetune}
Our first ablation study investigates the effect of the model fine-tuning.
We evaluated the performance based on both syntax and functionality checks. 
As shown in~\figref{fig:finetune_cot}(a), the results demonstrate that the finetuning dramatically increases syntax correctness from $54.85$\% to $88.44$\%. 
More importantly, the impact of finetuning is even more pronounced in the functionality evaluation, where the non-finetuned model failed to achieve any correct functionality test, but the accuracy is improved to $53.20$\% in the finetuned model. These enhancements highlight the critical role of finetuning in producing not only syntactically correct but also functionally viable codes, which demonstrates the benefits of finetuning LLMs for hardware design in the HLS code generation task.




\subsection{Effect of Chain-of-Thought Prompting}\label{subsec:exp_cot}
To assess the effect of the chain-of-thought (CoT) technique,
we perform both syntax and functionality evaluation on the fine-tuned model with and without the use of CoT.
As indicated in~\figref{fig:finetune_cot}(b), incorporating CoT leads to a noticeable improvement in both metrics. 
Specifically, syntax correctness increases from $88.44$\% to $94.33$\%, and functionality score rises from $53.20$\% to $61.45$\%. 
The result demonstrates the effectiveness of CoT in enhancing the reasoning capability, thereby improving its overall performance.


\subsection{Effect of Feedback Loops}\label{subsec:exp_feedback}
Our two-step feedback loop provides both syntax and functionality feedback. We evaluate the impact of these feedback loops with different numbers of iterations, ranging from 0 to 2.The results, shown in Figure ~\figref{fig:syntax_feedback} and ~\figref{fig:func_feedback}, indicate that both syntax and functionality feedback loops significantly improve model performance, especially when combined with COT prompting. The initial feedback loop yields substantial accuracy improvements in both syntax correctness and functionality evaluation, though the second loop shows diminishing returns.Syntax feedback loops enhance both syntax correctness and functionality performance, suggesting that iterative refinement is particularly effective for complex tasks. Similarly, functionality feedback loops not only improve functionality checks but also boost syntax accuracy, indicating that enhancements in functional understanding contribute to better syntactic performance.

 \begin{figure}[t]
    \centering
    \includegraphics[width=1\linewidth]{./figures/merged_finetune_cot.pdf}
    \vspace{-5mm}
    \caption{Effect of fine-tuning and chain-of-thought.}
    \label{fig:finetune_cot}
\end{figure}

\begin{figure}[t]
    \centering
    \includegraphics[width=0.95\linewidth]{./figures/Effect_of_Syntax_Feedback_Loop.pdf}
    \vspace{-2mm}
    \caption{Effect of syntax feedback loop.}
    \vspace{-2mm}
    \label{fig:syntax_feedback}
\end{figure}
\begin{figure}[t]
    \centering
    \includegraphics[width=0.95\linewidth]{./figures/Effect_of_Functionality_Feedback_Loop.pdf}
    \caption{Effect of functionality feedback loop.}
    \vspace{-2mm}
    \label{fig:func_feedback}
\end{figure}


\subsection{Time Cost and Hardware Performance}\label{subsec:exp_timecost}

\figref{fig:time_cost} shows the time cost for generating 120 data entries under different conditions, measuring the impact of CoT and feedback loops. Without a feedback loop, CoT significantly reduces the time. Adding a syntax feedback loop increases the time, but CoT continues to notably decrease the duration. The functionality feedback loop is the most time-consuming, though CoT still provides a notable reduction, albeit less dramatic. This demonstrates CoT's effectiveness in reducing operational times across varying complexities.

For the test set,
we evaluate the latency and resource consumption of the generated \textit{HLS} designs using a Xilinx VCU118 as our target FPGA, with a clock frequency of $200$MHz and Xilinx Vivado 2020.1 for synthesis.
As shown in~\tabref{tb:perf_resource}, all \textit{HLS} designs demonstrate reasonable performance, with BRAM usage consistently remained at zero due to the design scale.

\begin{figure}
    \centering
    \includegraphics[width=0.95\linewidth]{./figures/Time_Cost_Analysis.pdf}
    \vspace{-2mm}
    \caption{Time cost of code generation.}
    \label{fig:time_cost}
\end{figure}


\begin{table}[htb]
\centering
\caption{Latency and resource usage of LLM-generated designs synthesized on a VCU118 FPGA.}
\label{tb:perf_resource}
\setlength\tabcolsep{1pt} 
\scalebox{0.8}{
% \begin{tabular}{L{2cm}ccC{1.5cm}C{2.5cm}}
\begin{tabular}{C{2.5cm}|C{1.9cm}|C{1.5cm}|C{1.5cm}|C{1.3cm}|C{1.3cm}}
\toprule
{}& \textbf{Latency} (ms)& \textbf{LUTs} & \textbf{Registers} & \textbf{DSP48s} & \textbf{BRAMs} \\ \midrule
{\textbf{Available}} & - & 1182240 & 2364480 & 6840 & 4320 \\ \midrule
{\textit{ellpack}} & 0.304 & 1011 & 1079 & 11 & 0 \\
{\textit{syrk}} & 21.537 & 1371 & 1621 & 19 & 0 \\
{\textit{syr2k}} & 40.626 & 1572 & 1771 & 19 & 0 \\
{\textit{stencil2d}} & 1.368 & 287 & 123 & 3 & 0 \\
{\textit{trmm-opt}} & 15.889 & 1262 & 1239 & 11 & 0 \\
{\textit{stencil3d}} & 21.537 & 1173 & 1271 & 20 & 0 \\
{\textit{symm}} & 24.601 & 1495 & 1777 & 19 & 0 \\
{\textit{symm-opt}} & 16.153 & 1361 & 1608 & 19 & 0 \\
{\textit{symm-opt-medium}} & 579.0 & 2223 & 2245 & 22 & 0 \\
\bottomrule
\end{tabular}}
\end{table}

\subsection{Effect of Task Complexity}\label{subsec:complexity}
We analyze the effects of code complexity on the performance of fine-tuning our language model with CoT prompting and tested without the use of any feedback loops during inference. 
We categorize \textit{MachineGen} into three classes according to their code complexity: easy, medium, and difficult.
The results shown in the \tabref{tab:model_performance} indicates a clear trend: as the complexity of the generated code increases, both syntax and functionality correctness rates decline. This outcome could be attributed to several factors. First, more complex code inherently presents more challenges in maintaining syntactic integrity and functional accuracy. Second, the absence of feedback loops in the inference phase may have limited the model's ability to self-correct emerging errors in more complicated code generations.

\begin{table}[h]
\centering
\caption{Performance across different complexity levels.}
\scalebox{0.8}{
\begin{tabular}{c|c|c}
\hline
\textbf{Test Set} & \textbf{Syntax Check} & \textbf{Functionality} \\
\hline
Easy & 96.67\% & 63.33\% \\
Medium & 96.67\% & 53.33\% \\
Difficult & 90\% & 53.33\% \\
\hline
\end{tabular}}
\label{tab:model_performance}
\end{table}

\subsection{Analysis of \textit{MachineGen} and \textit{HumanRefine}}
\begin{table}[h]
\centering
\caption{Performance on \textit{MachineGen} and \textit{HumanRefine}.}
\scalebox{0.9}{
\begin{tabular}{c|c|c}
\hline
\textbf{Test Set} & \textbf{Syntax Check} & \textbf{Functionality Check} \\
\hline
\textit{MachineGen} & 93.83\% & 62.24\% \\
\hline
\textit{HumanRefine} & 47.29\% & 21.36\% \\
\hline
\end{tabular}}
\vspace{-3mm}
\label{table:eval_comparison}
\end{table}

As shown in~\tabref{table:eval_comparison}, this section compares the performance of our model on \textit{MachineGen} and \textit{HumanRefine} test sets.
Our findings reveal that the performance on the \textit{HumanRefine} is significantly lower than on the \textit{MachineGen}. This disparity suggests that the model is more adept at handling machine-generated prompts. The primary reasons for this are: the model's training data bias towards machine-generated prompts, the increased complexity and nuanced nature of human-generated prompts, and the conciseness and clarity of human-generated prompts that often omit repetitive or explicit details found in machine-generated prompts, making it harder for the model to generate syntactically and functionally correct code.

\subsection{Thoughts, Insights, and Limitations}

\noindent \textbf{1. \textit{HLS} versus \textit{HDL} for AI-assisted code generation:} The selection of programming language for hardware code generation should mainly depend on two  factors:
\begin{itemize}[leftmargin=*]
    \item \textit{Quality of Generated Hardware Design}: The evaluation of hardware design's quality includes syntax correctness, functionality, and hardware performance.
    Since \textit{HLS} shares similar semantics and syntax with programming languages commonly used during LLM pre-training, this work demonstrates that the LLM-assisted code generation for \textit{HLS} has the potential to achieve high syntax and functional correctness in hardware designs. While this work does not leverage hardware performance as feedback for design generation, it identifies this aspect as a key direction for future research and enhancements.
    \item \textit{Runtime Cost of Hardware Generation}: Although \textit{HLS}-based designs typically require fewer tokens compared to \textit{HDL} during the code generation phase—suggesting potentially lower costs—the overall runtime costs associated with HLS synthesis must also be considered. A more comprehensive quantitative comparison of these runtime costs is planned for our future work. 
\end{itemize}

\noindent \textbf{2. Input instructions and datasets are crucial}: The fine-tuning of pre-trained LLMs on \textit{HLS} dataset can bring a significant improvement in the design quality, echoing findings from previous studies on \textit{Verilog} code generation~\cite{thakur2023verigen}. 
Additionally, during our evaluation, we found that employing simple CoT prompting largely improves hardware design quality. 
This result contrasts with the application of CoT in general-purpose programming languages, where a specialized form of CoT is necessary~\cite{li2023structured}.
Therefore, future efforts for further enhancement can focus on collecting high-quality datasets and exploring better refinement of input prompts.

\noindent \textbf{3. Limitations}: At the time of this research, more advanced reasoning models, such as DeepSeek-R1~\cite{guo2025deepseek}, were not available for evaluation. Additionally, test-time scaling approaches~\cite{welleck2024decoding} could be incorporated to further enhance performance in the future.
Moreover, we observe that the diversity of hardware designs in the benchmark is limited, which may impact the generalizability of our findings.
We intend to address these limitations in our future work.

%\section{Related Work}
\paragraph{LLMs for Visual Program Generation}
Visual programming systems (e.g., LabView~\cite{bitter2006labview}, XG5000~\cite{XG5000Manual}) typically feature node-based interfaces that let users visually write and modify programs. Recently, researchers have begun utilizing LLMs to generate VPLs, as they are known for their powerful text-based code generation capabilities. For example, \citet{cai-etal-2024-low-code} integrates low-code visual programming with LLM-based task execution for direct interaction with LLMs, while \citet{zhang2024benchmarking} studies generation of node-based visual dataflow languages in audio programming. Similarly, \citet{xue2024comfybenchbenchmarkingllmbasedagents,52868} investigates Machine Learning workflow generation from natural language commands and demonstrates that metaprogram-based text formats outperform other formats like JSON. However, these prompting-based methods face limitations for VPLs like Ladder Diagram, where custom I/O mapping and domain-specific syntax are crucial. Thus, we study fine-tuning approaches with domain-specific data to better capture these details.

\paragraph{LLM-based PLC code generation}
Programmable Logic Controllers (PLCs) are essential components in industrial automation and are used to control machinery and processes reliably and efficiently. Among the programming languages defined by the IEC 61131-3 standard~\cite{IEC61131-3}, Structured Text (ST) and Ladder Diagram (LD) are commonly used for programming PLCs. Research in this area has focused on utilizing LLMs to generate ST code from natural language descriptions. Recent studies have demonstrated the potential of LLMs in generating high-quality ST code~\cite{koziolek2023chatgpt, koziolek2024llm}, enhancing safety and accuracy with verification tools and user feedback~\cite{fakih2024llm4plc}, and automating code generation and verification using multi-agent frameworks~\cite{liu2024agents4plc}. Although these advances have improved PLC code generation, they primarily focus on ST, despite LD being widely used in industrial settings due to its similarity to electrical circuits~\cite{ladderlogic}. While \citet{Zhang_2024} attempts to generate LD as an ASCII art based on user instructions in a zero-shot manner, their findings show that even advanced LLMs struggle with basic LD generation. These limitations highlight the necessity of training-based methods for LD generation. In this work, we address this gap by introducing a training-based approach for LLMs to generate LD and thus pave the way for the broader adoption of AI-assisted PLC programming.
\section{Discussion}\label{sec:discussion}



\subsection{From Interactive Prompting to Interactive Multi-modal Prompting}
The rapid advancements of large pre-trained generative models including large language models and text-to-image generation models, have inspired many HCI researchers to develop interactive tools to support users in crafting appropriate prompts.
% Studies on this topic in last two years' HCI conferences are predominantly focused on helping users refine single-modality textual prompts.
Many previous studies are focused on helping users refine single-modality textual prompts.
However, for many real-world applications concerning data beyond text modality, such as multi-modal AI and embodied intelligence, information from other modalities is essential in constructing sophisticated multi-modal prompts that fully convey users' instruction.
This demand inspires some researchers to develop multimodal prompting interactions to facilitate generation tasks ranging from visual modality image generation~\cite{wang2024promptcharm, promptpaint} to textual modality story generation~\cite{chung2022tale}.
% Some previous studies contributed relevant findings on this topic. 
Specifically, for the image generation task, recent studies have contributed some relevant findings on multi-modal prompting.
For example, PromptCharm~\cite{wang2024promptcharm} discovers the importance of multimodal feedback in refining initial text-based prompting in diffusion models.
However, the multi-modal interactions in PromptCharm are mainly focused on the feedback empowered the inpainting function, instead of supporting initial multimodal sketch-prompt control. 

\begin{figure*}[t]
    \centering
    \includegraphics[width=0.9\textwidth]{src/img/novice_expert.pdf}
    \vspace{-2mm}
    \caption{The comparison between novice and expert participants in painting reveals that experts produce more accurate and fine-grained sketches, resulting in closer alignment with reference images in close-ended tasks. Conversely, in open-ended tasks, expert fine-grained strokes fail to generate precise results due to \tool's lack of control at the thin stroke level.}
    \Description{The comparison between novice and expert participants in painting reveals that experts produce more accurate and fine-grained sketches, resulting in closer alignment with reference images in close-ended tasks. Novice users create rougher sketches with less accuracy in shape. Conversely, in open-ended tasks, expert fine-grained strokes fail to generate precise results due to \tool's lack of control at the thin stroke level, while novice users' broader strokes yield results more aligned with their sketches.}
    \label{fig:novice_expert}
    % \vspace{-3mm}
\end{figure*}


% In particular, in the initial control input, users are unable to explicitly specify multi-modal generation intents.
In another example, PromptPaint~\cite{promptpaint} stresses the importance of paint-medium-like interactions and introduces Prompt stencil functions that allow users to perform fine-grained controls with localized image generation. 
However, insufficient spatial control (\eg, PromptPaint only allows for single-object prompt stencil at a time) and unstable models can still leave some users feeling the uncertainty of AI and a varying degree of ownership of the generated artwork~\cite{promptpaint}.
% As a result, the gap between intuitive multi-modal or paint-medium-like control and the current prompting interface still exists, which requires further research on multi-modal prompting interactions.
From this perspective, our work seeks to further enhance multi-object spatial-semantic prompting control by users' natural sketching.
However, there are still some challenges to be resolved, such as consistent multi-object generation in multiple rounds to increase stability and improved understanding of user sketches.   


% \new{
% From this perspective, our work is a step forward in this direction by allowing multi-object spatial-semantic prompting control by users' natural sketching, which considers the interplay between multiple sketch regions.
% % To further advance the multi-modal prompting experience, there are some aspects we identify to be important.
% % One of the important aspects is enhancing the consistency and stability of multiple rounds of generation to reduce the uncertainty and loss of control on users' part.
% % For this purpose, we need to develop techniques to incorporate consistent generation~\cite{tewel2024training} into multi-modal prompting framework.}
% % Another important aspect is improving generative models' understanding of the implicit user intents \new{implied by the paint-medium-like or sketch-based input (\eg, sketch of two people with their hands slightly overlapping indicates holding hand without needing explicit prompt).
% % This can facilitate more natural control and alleviate users' effort in tuning the textual prompt.
% % In addition, it can increase users' sense of ownership as the generated results can be more aligned with their sketching intents.
% }
% For example, when users draw sketches of two people with their hands slightly overlapping, current region-based models cannot automatically infer users' implicit intention that the two people are holding hands.
% Instead, they still require users to explicitly specify in the prompt such relationship.
% \tool addresses this through sketch-aware prompt recommendation to fill in the necessary semantic information, alleviating users' workload.
% However, some users want the generative AI in the future to be able to directly infer this natural implicit intentions from the sketches without additional prompting since prompt recommendation can still be unstable sometimes.


% \new{
% Besides visual generation, 
% }
% For example, one of the important aspect is referring~\cite{he2024multi}, linking specific text semantics with specific spatial object, which is partly what we do in our sketch-aware prompt recommendation.
% Analogously, in natural communication between humans, text or audio alone often cannot suffice in expressing the speakers' intentions, and speakers often need to refer to an existing spatial object or draw out an illustration of her ideas for better explanation.
% Philosophically, we HCI researchers are mostly concerned about the human-end experience in human-AI communications.
% However, studies on prompting is unique in that we should not just care about the human-end interaction, but also make sure that AI can really get what the human means and produce intention-aligned output.
% Such consideration can drastically impact the design of prompting interactions in human-AI collaboration applications.
% On this note, although studies on multi-modal interactions is a well-established topic in HCI community, it remains a challenging problem what kind of multi-modal information is really effective in helping humans convey their ideas to current and next generation large AI models.




\subsection{Novice Performance vs. Expert Performance}\label{sec:nVe}
In this section we discuss the performance difference between novice and expert regarding experience in painting and prompting.
First, regarding painting skills, some participants with experience (4/12) preferred to draw accurate and fine-grained shapes at the beginning. 
All novice users (5/12) draw rough and less accurate shapes, while some participants with basic painting skills (3/12) also favored sketching rough areas of objects, as exemplified in Figure~\ref{fig:novice_expert}.
The experienced participants using fine-grained strokes (4/12, none of whom were experienced in prompting) achieved higher IoU scores (0.557) in the close-ended task (0.535) when using \tool. 
This is because their sketches were closer in shape and location to the reference, making the single object decomposition result more accurate.
Also, experienced participants are better at arranging spatial location and size of objects than novice participants.
However, some experienced participants (3/12) have mentioned that the fine-grained stroke sometimes makes them frustrated.
As P1's comment for his result in open-ended task: "\emph{It seems it cannot understand thin strokes; even if the shape is accurate, it can only generate content roughly around the area, especially when there is overlapping.}" 
This suggests that while \tool\ provides rough control to produce reasonably fine results from less accurate sketches for novice users, it may disappoint experienced users seeking more precise control through finer strokes. 
As shown in the last column in Figure~\ref{fig:novice_expert}, the dragon hovering in the sky was wrongly turned into a standing large dragon by \tool.

Second, regarding prompting skills, 3 out of 12 participants had one or more years of experience in T2I prompting. These participants used more modifiers than others during both T2I and R2I tasks.
Their performance in the T2I (0.335) and R2I (0.469) tasks showed higher scores than the average T2I (0.314) and R2I (0.418), but there was no performance improvement with \tool\ between their results (0.508) and the overall average score (0.528). 
This indicates that \tool\ can assist novice users in prompting, enabling them to produce satisfactory images similar to those created by users with prompting expertise.



\subsection{Applicability of \tool}
The feedback from user study highlighted several potential applications for our system. 
Three participants (P2, P6, P8) mentioned its possible use in commercial advertising design, emphasizing the importance of controllability for such work. 
They noted that the system's flexibility allows designers to quickly experiment with different settings.
Some participants (N = 3) also mentioned its potential for digital asset creation, particularly for game asset design. 
P7, a game mod developer, found the system highly useful for mod development. 
He explained: "\emph{Mods often require a series of images with a consistent theme and specific spatial requirements. 
For example, in a sacrifice scene, how the objects are arranged is closely tied to the mod's background. It would be difficult for a developer without professional skills, but with this system, it is possible to quickly construct such images}."
A few participants expressed similar thoughts regarding its use in scene construction, such as in film production. 
An interesting suggestion came from participant P4, who proposed its application in crime scene description. 
She pointed out that witnesses are often not skilled artists, and typically describe crime scenes verbally while someone else illustrates their account. 
With this system, witnesses could more easily express what they saw themselves, potentially producing depictions closer to the real events. "\emph{Details like object locations and distances from buildings can be easily conveyed using the system}," she added.

% \subsection{Model Understanding of Users' Implicit Intents}
% In region-sketch-based control of generative models, a significant gap between interaction design and actual implementation is the model's failure in understanding users' naturally expressed intentions.
% For example, when users draw sketches of two people with their hands slightly overlapping, current region-based models cannot automatically infer users' implicit intention that the two people are holding hands.
% Instead, they still require users to explicitly specify in the prompt such relationship.
% \tool addresses this through sketch-aware prompt recommendation to fill in the necessary semantic information, alleviating users' workload.
% However, some users want the generative AI in the future to be able to directly infer this natural implicit intentions from the sketches without additional prompting since prompt recommendation can still be unstable sometimes.
% This problem reflects a more general dilemma, which ubiquitously exists in all forms of conditioned control for generative models such as canny or scribble control.
% This is because all the control models are trained on pairs of explicit control signal and target image, which is lacking further interpretation or customization of the user intentions behind the seemingly straightforward input.
% For another example, the generative models cannot understand what abstraction level the user has in mind for her personal scribbles.
% Such problems leave more challenges to be addressed by future human-AI co-creation research.
% One possible direction is fine-tuning the conditioned models on individual user's conditioned control data to provide more customized interpretation. 

% \subsection{Balance between recommendation and autonomy}
% AIGC tools are a typical example of 
\subsection{Progressive Sketching}
Currently \tool is mainly aimed at novice users who are only capable of creating very rough sketches by themselves.
However, more accomplished painters or even professional artists typically have a coarse-to-fine creative process. 
Such a process is most evident in painting styles like traditional oil painting or digital impasto painting, where artists first quickly lay down large color patches to outline the most primitive proportion and structure of visual elements.
After that, the artists will progressively add layers of finer color strokes to the canvas to gradually refine the painting to an exquisite piece of artwork.
One participant in our user study (P1) , as a professional painter, has mentioned a similar point "\emph{
I think it is useful for laying out the big picture, give some inspirations for the initial drawing stage}."
Therefore, rough sketch also plays a part in the professional artists' creation process, yet it is more challenging to integrate AI into this more complex coarse-to-fine procedure.
Particularly, artists would like to preserve some of their finer strokes in later progression, not just the shape of the initial sketch.
In addition, instead of requiring the tool to generate a finished piece of artwork, some artists may prefer a model that can generate another more accurate sketch based on the initial one, and leave the final coloring and refining to the artists themselves.
To accommodate these diverse progressive sketching requirements, a more advanced sketch-based AI-assisted creation tool should be developed that can seamlessly enable artist intervention at any stage of the sketch and maximally preserve their creative intents to the finest level. 

\subsection{Ethical Issues}
Intellectual property and unethical misuse are two potential ethical concerns of AI-assisted creative tools, particularly those targeting novice users.
In terms of intellectual property, \tool hands over to novice users more control, giving them a higher sense of ownership of the creation.
However, the question still remains: how much contribution from the user's part constitutes full authorship of the artwork?
As \tool still relies on backbone generative models which may be trained on uncopyrighted data largely responsible for turning the sketch into finished artwork, we should design some mechanisms to circumvent this risk.
For example, we can allow artists to upload backbone models trained on their own artworks to integrate with our sketch control.
Regarding unethical misuse, \tool makes fine-grained spatial control more accessible to novice users, who may maliciously generate inappropriate content such as more realistic deepfake with specific postures they want or other explicit content.
To address this issue, we plan to incorporate a more sophisticated filtering mechanism that can detect and screen unethical content with more complex spatial-semantic conditions. 
% In the future, we plan to enable artists to upload their own style model

% \subsection{From interactive prompting to interactive spatial prompting}


\subsection{Limitations and Future work}

    \textbf{User Study Design}. Our open-ended task assesses the usability of \tool's system features in general use cases. To further examine aspects such as creativity and controllability across different methods, the open-ended task could be improved by incorporating baselines to provide more insightful comparative analysis. 
    Besides, in close-ended tasks, while the fixing order of tool usage prevents prior knowledge leakage, it might introduce learning effects. In our study, we include practice sessions for the three systems before the formal task to mitigate these effects. In the future, utilizing parallel tests (\textit{e.g.} different content with the same difficulty) or adding a control group could further reduce the learning effects.

    \textbf{Failure Cases}. There are certain failure cases with \tool that can limit its usability. 
    Firstly, when there are three or more objects with similar semantics, objects may still be missing despite prompt recommendations. 
    Secondly, if an object's stroke is thin, \tool may incorrectly interpret it as a full area, as demonstrated in the expert results of the open-ended task in Figure~\ref{fig:novice_expert}. 
    Finally, sometimes inclusion relationships (\textit{e.g.} inside) between objects cannot be generated correctly, partially due to biases in the base model that lack training samples with such relationship. 

    \textbf{More support for single object adjustment}.
    Participants (N=4) suggested that additional control features should be introduced, beyond just adjusting size and location. They noted that when objects overlap, they cannot freely control which object appears on top or which should be covered, and overlapping areas are currently not allowed.
    They proposed adding features such as layer control and depth control within the single-object mask manipulation. Currently, the system assigns layers based on color order, but future versions should allow users to adjust the layer of each object freely, while considering weighted prompts for overlapping areas.

    \textbf{More customized generation ability}.
    Our current system is built around a single model $ColorfulXL-Lightning$, which limits its ability to fully support the diverse creative needs of users. Feedback from participants has indicated a strong desire for more flexibility in style and personalization, such as integrating fine-tuned models that cater to specific artistic styles or individual preferences. 
    This limitation restricts the ability to adapt to varied creative intents across different users and contexts.
    In future iterations, we plan to address this by embedding a model selection feature, allowing users to choose from a variety of pre-trained or custom fine-tuned models that better align with their stylistic preferences. 
    
    \textbf{Integrate other model functions}.
    Our current system is compatible with many existing tools, such as Promptist~\cite{hao2024optimizing} and Magic Prompt, allowing users to iteratively generate prompts for single objects. However, the integration of these functions is somewhat limited in scope, and users may benefit from a broader range of interactive options, especially for more complex generation tasks. Additionally, for multimodal large models, users can currently explore using affordable or open-source models like Qwen2-VL~\cite{qwen} and InternVL2-Llama3~\cite{llama}, which have demonstrated solid inference performance in our tests. While GPT-4o remains a leading choice, alternative models also offer competitive results.
    Moving forward, we aim to integrate more multimodal large models into the system, giving users the flexibility to choose the models that best fit their needs. 
    


\section{Conclusion}\label{sec:conclusion}
In this paper, we present \tool, an interactive system designed to help novice users create high-quality, fine-grained images that align with their intentions based on rough sketches. 
The system first refines the user's initial prompt into a complete and coherent one that matches the rough sketch, ensuring the generated results are both stable, coherent and high quality.
To further support users in achieving fine-grained alignment between the generated image and their creative intent without requiring professional skills, we introduce a decompose-and-recompose strategy. 
This allows users to select desired, refined object shapes for individual decomposed objects and then recombine them, providing flexible mask manipulation for precise spatial control.
The framework operates through a coarse-to-fine process, enabling iterative and fine-grained control that is not possible with traditional end-to-end generation methods. 
Our user study demonstrates that \tool offers novice users enhanced flexibility in control and fine-grained alignment between their intentions and the generated images.


%% The Appendices part is started with the command \appendix;
%% appendix sections are then done as normal sections
\appendix

%%\section{Sample Appendix Section}
%%\label{sec:sample:appendix}


%% If you have bibdatabase file and want bibtex to generate the
%% bibitems, please use
%%
 \bibliographystyle{elsarticle-num} 
 \bibliography{cas-refs}

%% else use the following coding to input the bibitems directly in the
%% TeX file.

% \begin{thebibliography}{00}

% %% \bibitem{label}
% %% Text of bibliographic item

% \bibitem{}

% \end{thebibliography}
\end{document}
\endinput
%%
%% End of file `elsarticle-template-num.tex'.
