%%
%%---------------------------------------------
\section{Conclusions\added{ and Future Work}}
\label{sec:conclusions}
%%
%
\noindent
%
This article addresses the issue of integrating the 3GPP network slicing framework into the MEC infrastructure to facilitate the end-to-end management and orchestration of latency-sensitive resources and time-critical applications. To this end, we have proposed a new end-to-end application-centric slicing framework, which is built upon the new concepts of Application Service (AS), Application Slice (APS), and Application Component Function (ACF). Subsequently, we have designed a comprehensive multi-tenant MEC architecture for application and network slicing that considers different operational and business roles. We have also proposed a 3GPP-like integrated network and application slice management architecture compatible with a multi-tenant management architecture. The latter provides distinct management and orchestration responsibilities within each slice segment. Finally, we have proposed a way to implement a MEC Customer orchestrator within such a MEC slicing architecture, assessed the ability of our implementation approach to support performance isolation between applications, and discussed open implementation gaps. 

This work is the starting point for our future research on sliced, multi-tenant MEC infrastructures. Implementation-wise, there are still several architecture components to design with different possible design choices to benchmark. Architecture-wise, a solution to support MEC application relocation between different MEC Customers (i.e. between different tenants) is necessary. This solution would require a stronger interaction between our proposed MEC orchestration architecture and the 5G Core (5GC) network functions to synchronise traffic forwarding rules between multiple administrative domains (including MEC and 5GC). \added{Furthermore, several research problems are associated to the efficient sharing of resources (e.g., code, data, processing) between different slices. For instance, how to ensure not only traffic isolation (e.g., in terms of bandwidth), but also security isolation (e.g. data cannot be tampered between different slices) when sharing edge resources. Similarly, we plan to investigate techniques to efficiently aggregate and disaggregate shared resources when scaling up or down the different slices which use common shared resources.}
%
\vspace{-0.2cm}
%
%%
%%---------------------------------------------
%\section*{Acknowledgements}
%\label{sec:ack}
%%
%\noindent
%
%\Thai{I am not sure this acknowledgement section is appropriate wrt our patent declaration: you both are enrolled within Invent With Nokia program as free researchers - i.e. researchers not backed/funded by any governmental or corporate program}
%The work of Simone Bolettieri and Raffaele Bruno is partly funded by the EC under the H2020 INFRADEV-2019-3 SLICES-DS (951850) and the H2020-EU.2.1.1 MARVEL (957337) projects, and by the Italian Ministry of Economic Development under the ARTES 4.0 project.