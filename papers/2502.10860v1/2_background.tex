%%---------------------------------------------
\section{Background}
\label{sec:background}
%%
\noindent
%%
Network slicing is not a novel idea. For instance, early network designers proposed to set up virtual LANs to provide network segmentation at the data link layer, addressing issues such as scalability, security, and network management of switched Ethernet networks~\cite{2011_commag_vlan}. However, over the years, the concept of network slicing has significantly evolved. One of the main drivers of this technology has become the need for network service flexibility and programmability to efficiently accommodate diverse business use cases~\cite{2018_COMST_slicing_survey}. Nowadays, a network slice is commonly defined as a \textit{logically isolated network over multi-domain, multi-technology physical networks, which provides specific network capabilities and characteristics} to support certain communication services serving a business purpose while ensuring functional and performance isolation~\cite{2020_ACCESS_slicing_survey}. While there is a consolidated high-level view of network slicing, various network slicing models are currently under definition depending on the virtualisation technology, the network architecture, and the standard development organisation (SDO)~\cite{NGMN028,2017_foukas_5G_slicing,Cominardi2020}. The interested reader is referred to~\cite{MEC024} for a comprehensive comparison of different slicing concepts. This section overviews the 3GPP approach to network slicing as it provides the foundation for our proposed E2E slicing management architecture. Furthermore, we discuss current trends to integrate support for network slicing into the MEC standard, and we identify some gaps to be filled to extend slicing to a multi-tenant edge in a common management architecture. For the sake of presentation clarity, Table~\ref{tab:abbrev} summarises the abbreviations we use in this paper. 
%
\begin{table}[t]
    \centering
    \small
    \begin{tabular}{|p{0.1\textwidth}|p{0.35\textwidth}|}
        \hline
        \bf Notation & \bf Definition \\
        \hline
         ACF    & Application Component Function \\
         APSMF  & Application Slice Management Function \\
         APSS   & Application Slice Subnet \\
         APSSMF & Application Slice Subnet Management Function \\
         APS    & Application Slice \\
         AS     & Application Service \\
         ASMF   & Application Service Management Function \\
          \added{CN} & \added{Core Network} \\
          \added{CNI} & \added{Container Network Interface} \\
         CS     & Communication Service \\
         CSMF   & Communication Service Management Function \\
         E2E    & End-to-End \\
         LCM    & Lifecycle Management \\
         \added{MANO} & \added{Management and Orchestration stack} \\
         MEO    & MEC Orchestrator\\
         MEAO   & MEC Application Orchestrator\\
         MAPSS & MEC Application Slice Subnet \\
         MAPSSD & MEC Application Slice Subnet Descriptor \\
         MECO   & MEC Customer Orchestrator\\
         MECPM  & MEC Customer Platform Manager\\
         MEOO   & MEC Owner Orchestrator\\
         MENSM  & MEC Network Slice Manager\\
         MEPM   & MEC Platform Manager \\
         NF     & Network Function \\
         NSMF   & Network Slice Management Function \\
         NSSMF  & Network Slice Subnet Management Function \\
         NFVI   & NFV Infrastructure \\
         NST    & Network Slice Template \\
         PNF    & Physical Network Function \\
         \added{RAN} & \added{Radio Access Network} \\
         SLA    & Service Level Agreement \\
         SLS    & Service Level Specification \\
         VNF    & Virtual Network Function  \\
         VNFM   & VNF LCM Manager \\
         \hline
    \end{tabular}
    \caption{Summary of notation and abbreviations}
    \label{tab:abbrev}
\end{table}
%%
%%
%%---------------------------------------------
\subsection{3GPP network slice concept and building blocks}
\label{sec:3gpp_slicing}
%%
\noindent
The 3GPP approach to network slicing is inherited from the NGMN network slicing concept~\cite{NGMN028}, and it is based on the distinction between network slices and network slice instances (NSIs). Specifically, a network slice is defined as "\textit{a logical network that provides specific network capabilities and network characteristics, supporting various service properties for network slice customers}" ~\cite{3GPPTS28530}, while  a NSI is an activated network slice, namely "...\textit{a set of network functions and the resources for these network functions which are arranged and configured ... to meet certain network characteristics}..." that are required by a communication service~\cite{3GPPTR28801}. Following the relationship between CS and NS, the 3GPP network slicing architecture is organised into three distinct logical layers: 1) a service instance layer, 2) a network slice instance layer, and 3) a resource layer~\cite{3GPPTR28801}. The first layer encompasses the service instances (a.k.a. Communication services (CS) in 5G networks) that are associated with service-level agreements (SLA), namely business contracts between the service provider and the clients, which specify the service levels to be ensured. The second layer comprises the NSIs that are deployed to serve the CS requirements. Each NSI is associated to a \textit{Network Slice Template} (NST), also referred to as network slice blueprint~\cite{NGMN028}, which describes the general structure and configuration of the network slice, and the \textit{Service Level Specification} (SLS), which lists the set of technical attributes that have to be satisfied by the deployed network slice. In other words, the SLS translates the business objectives of the SLA to network characteristics. Finally, the third layer includes the necessary physical (hardware and software) and logical resources to support the NSIs.

The 3GPP management architecture recognises that a network slice spans different technical domains (or \textit{segment}). Specifically, a network slices spans across the radio access network (RAN), the core network (CN), and the transport network (TN), each segment having separate scope and technologies~\cite{2021_gsma_TR_e2eslicing}. For this reason, the 3GPP network slicing architecture introduces the concept of \textit{Network Slice Subnets} (NSSs), defined as "...\textit{a set of network functions and the associated resources (e.g. compute, storage and networking resources) supporting network slice}." ~\cite{3GPPTS28530}. If a 3GPP CS can be seen as a business wrapper of one or more network slices, in turn, a network slice is a wrapper of one or more network slice subnets with some Service Level Specification (SLS). Figure~\ref{fig:slice_subnet} illustrates the relationship between these different concepts, following an example of~\cite{3GPPTS28530}.
%
\begin{figure}[ht]
\centering
\includegraphics[clip,trim= 2cm 2.5cm 1cm 2cm,width=0.5\textwidth]{figures/fig_slice_subnets.pdf}
\caption{Exemplary illustration of the relationships between communication services, network slices, network slice subnets and resources/network functions}
\label{fig:slice_subnet}
\end{figure}
%
In the figure, there are several points worth noting. Network Functions (NF) refer to processing functions of the 5G network (both access and core networks), which expose APIs to provide one or more services to other NFs, following the producer-consumer concept. NFs include physical network nodes (namely Physical Network Functions or PNF) and Virtual Network Functions (VNF). We remind that a VNF is a software implementation of a network function within an NFV infrastructure (NFVI). In Figure~\ref{fig:slice_subnet}, network slice subnet AN-1 and network slice subnet AN-2 each contain distinct sets of NFs of the RAN, while network slice subnet CN-2 and network slice subnet CN-3 partly share some of the NFs of the core network. The network operator offers network slice subnet 1 as a network slice 1 to CS A. For this purpose, the network operator associates the SLS derived from the SLA of CS A to network slice subnet 1. Note that a network slice can satisfy the service requirements of different communication services (e.g., network slice 1 in Figure~\ref{fig:slice_subnet} is serving both CS A and CS B). Finally, it is helpful to point out that the deployment template of a network slice subnet should contain the descriptors of its constituent NFs and information relevant to the links between these NFs, such as topology of connections and QoS attributes for these links (e.g. bandwidth). The latter information can be represented in the form of a service graph, using the data information model defined by the ETSI NFV standard for describing the Virtualised Network Function Forwarding Graph (VNFFG)~\cite{NFV012}.
%%
%%---------------------------------------------
\subsection{3GPP network slice management architecture}
\label{subsec:mgmt_architecture}
%%
\noindent
In order to lay the ground for the design of an overall network slice management architecture, the 3GPP has defined high-level operational roles, which permit to draw clear boundaries in terms of operational responsibilities. Figure \ref{2_figRoles} illustrates the different roles as identified by the 3GPP \cite{3GPPTS28530}, which are defined as follows:
%
\renewcommand\labelitemi{$\bullet$}
\begin{itemize}[noitemsep,topsep=2pt]
    \item Communication Service Customer (CSC): consumes communication services \added{- e.g. end-users or vertical enterprises pay for their communication services.}
    \item Communication Service Provider (CSP): provides communication services that are designed, built and operated with (or without) a network slice \added{- e.g. Virgin Mobile provides public mobile subscriptions, AT\&T provides private network communication services to enterprises.}
    \item Network Operator (NOP): designs, builds and operates network slices \added{- e.g. AT\&T, Verizon.}
    \item Network Equipment Provider (NEP): supplies network equipment including VNFs to a network operator \added{- e.g. Nokia, Cisco, Ericsson.}
    \item Virtualisation Infrastructure Service Provider (VISP): provides virtualised infrastructure services  \added{- e.g. AWS, MS Azure, GCP provide PaaS.}
    \item Data Centre Service Provider (DCSP): provides data centre services \added{- e.g. AWS, MS Azure, GCP.}
    \item NFVI Supplier: supplies a network function virtualisation infrastructure \added{- e.g. VMWARE supplies Hypervisor and related management functions.}
    \item Hardware Supplier: supplies hardware \added{- e.g. HP, DELL.}
\end{itemize}
%
\begin{figure}[ht]
\centering
\includegraphics[width=0.4\textwidth]{figures/2_FigureRoles.png}
\caption{High Level Roles in Network Slice Framework}
\label{2_figRoles}
\end{figure}
%
An organisation can play one or several roles simultaneously (for example, a company can simultaneously play both the roles of a CSP and a NOP).

The 3GPP has also standardised the orchestration and management functions for the life-cycle management of network slices. Specifically, the 3GPP slicing management architecture boils down to three essential management NFs, called CSMF, NSMF, and NSSMF, as also illustrated by Figure~\ref{fig:ns_mngmt}: 
%
\begin{figure}[ht]
\centering
\includegraphics[clip,trim= 3.5cm 4cm 13cm 8cm,width=0.35\textwidth]{figures/fig_ns_mngmt.pdf}
\caption{3GPP network slice management architecture}
\label{fig:ns_mngmt}
\end{figure}
%
\renewcommand\labelitemi{$\bullet$}
\begin{itemize}[noitemsep,topsep=2pt]
    \item \textit{Communication Service Management Function} (CSMF). The CSMF is the user interface for slice management. It converts the SLAs of the CSs requested by the CSC into SLS and delegates the management of NSIs to NSMFs.
    \item \textit{Network Slice Management Function} (NSMF). The NSMF  manages NSIs and splits them into subnets for the RAN, transport and core domains. Then, the NSMF delegates the management of slice subnets to NSSMFs
    \item \textit{Network Slice Subnet Management Function} (NSSMF). The NSSMF applies the NSMF's life-cycle management commands (e.g., instantiate, scale, terminate, remove) within a particular subnet. 
\end{itemize}
%
Then, a 3GPP tenant can request the creation of a network slice from the NSTs made available by the CSMF, using a dedicated front end (such as the web portal depicted in Figure~\ref{fig:ns_mngmt}). It is also worth pointing out that the NSSMF is where most of the slice intelligence resides. It takes a command from the NSMF, such as "build a slice," and activates it by doing all the behind-the-scenes work of function selection, storage, configuration, and communication. Once each slice subnet is created, the NSMF is in charge of \textit{stitching} them together to build the end-to-end network slice.

\added{It is important to point out that SDN and NFV are the key technology enablers of network slicing in 3GPP networks. There is a plethora of open-source frameworks that offer software-defined implementations of the RAN and CN portions of the 4G/5G cellular architecture, which include Open Networking Foundation (ONF)'s projects}\footnote{\url{https://opennetworking.org}.} \added{such as Aether, SD-CORE, and SD-RAN, and solutions defined by the O-RAN Alliance}\footnote{\url{https://www.o-ran.org/specifications}.} \added{In addition several open-source projects have been established to implement the ETSI NFV Management
and Orchestration (MANO) framework~\cite{2017_JCSI_mano} to support the instantiation, control and configuration of network slices in different portions of the 5G infrastructure, such as ONAP, OSM and OpenBaton~\cite{2020_JSAN_mano}. The interested reader is also referred to~\cite{2020_COMNET_programmable_5G} for a comprehensive survey with extensive details on open virtualisation and management frameworks for 5G networks.} 
%%
%%
%%---------------------------------------------
\subsection{ETSI Multi-access Edge Computing (MEC)}
\label{sec:mec}
%%
\noindent
The ETSI organisation has introduced the Multi-access Edge Computing (MEC) since 2014 to provide a standard framework for the development of inter-operable applications over multi-vendor edge computing platforms. To this end, the MEC technology provides a new distributed software development model containing functional entities, services, and APIs, enabling applications to run on top of a generic virtualisation infrastructure located in or close to the network edge. For the sake of discussion,  Figure~\ref{fig:mec_gen_arch} shows the generic ETSI MEC reference architecture, which consists of three main blocks: $(i)$ the MEC Host, $(ii)$ the MEC Platform Manager (MEPM) and $(iii)$ the MEC Orchestrator (MEO). 
%
\begin{figure}[ht]
\centering
\includegraphics[clip,trim= 0cm 3cm 10cm 4cm,width=0.45\textwidth]{figures/fig_mec.pdf}
\caption{MEC reference architecture (based on~\cite{MEC003}).}
\label{fig:mec_gen_arch}
\vspace{-0.2cm}
\end{figure}

The MEC host is at the core of the MEC architecture as it contains: $(i)$ the generic virtualisation infrastructure (VI), which provides compute, storage, and network resources for the MEC applications; $(ii)$ the MEC applications running on top of the VI\footnote{Existing ETSI MEC specifications assume that MEC applications are deployed as VMs using an hypervisor-based virtualisation platform, but alternative virtualisation technologies and paradigms are under consideration~\cite{MEC027}.}; and $(iii)$ the MEC platform, an environment that hosts MEC services and offers to authorised MEC applications a reference point to discover and consume MEC services, as well as to announce and offer new MEC services. As discussed in detail in the following section, MEC services are an essential component of a MEC system, as they allow MEC applications to be \textit{network-aware}. In addition, the MEC platform is responsible for configuring a local DNS server and instructing the data plane of the VI on how to route traffic among applications, MEC services, DNS servers/proxies, and external networks. 

A management layer is associated with the MEC hosts of a MEC system, including a virtualisation infrastructure manager (VIM) and a MEC platform manager (MEPM). The former is responsible for allocating, maintaining, and releasing the virtual resources of the VI. The latter is responsible for managing the life-cycle of MEC applications and informing individual MEC platforms about application rules, traffic rules, and DNS configuration. Finally, the MEO is the core functionality of the MEC system-level management. Specifically, the MEO is responsible $(i)$ for selecting the MEC host(s) for application instantiation based on application requirements and constraints (e.g. latency), available resources, and available services; and $(ii)$ for maintaining an updated view of the MEC system. Furthermore, the MEO is interfaced with the Operations Support System (OSS) of the network operator, which receives the requests for instantiation or termination of applications from either applications running in the devices (e.g. UEs) or from third-party customers through the CFS portal. It is helpful to point out that the MEC standard provides a complete specification of only a limited set of necessary APIs, and associated data models and formats, but most of the management reference points are voluntarily left open by the standard to foster market differentiation. 

\added{It is important to point out that ETSI has established a dedicated Working Group, called DECODE, to promote the development of open-source MEC solutions that can offer all functional entities of the MEC architecture, or only a subset of them (for instance a MEC Platform, or an API implementation).}\footnote{A partial list of relevant MEC solutions is available at \url{https://mecwiki.etsi.org/index.php?title=MEC_Ecosystem}.}. \added{In addition, both academic projects and industrial initiatives are working towards the implementation of a variety of MEC-compliant edge computing solutions. Open-source implementations that are worthwhile to mention are: $i)$ LightEdge, a microservice-based implementation of the ETSI MEC architecture for 4G and 5G networks}\footnote{\url{https://lightedge.io/}.}\added{; $ii)$ LL-MEC, a platform built upon the OpenFlow and FlexRAN protocols that provides real-time radio network information to MEC applications}\footnote{\url{https://mosaic5g.io/ll-mec/}.}\added{; and Smart Edge Open, a software toolkit developed by Intel to build a variety of edge platforms, including MEC-complaint edge solutions embedded into 5G network}\footnote{\url{https://smart-edge-open.github.io/docs/product-overview/}.}\added{.}  
%%
%%---------------------------------------------
\subsubsection{MEC and network slicing}
%%
\noindent
As pointed out above, the MEC is a distributed computing environment at the edge of the network, on which multiple applications can be served simultaneously while ensuring ultra-low latency and high bandwidth. To achieve this goal, applications have real-time access to network information through APIs exposed by MEC services. According to the MEC standard~\cite{MEC003}, each MEC system is mandated to offer services providing authorised applications with $(i)$ radio network information (such as network load and status), and $(ii)$ location information about UEs served by the radio node(s) associated with a MEC host. Furthermore, the \texttt{Bandwidth Manager} service, when available, permits both the allocation of bandwidth to certain traffic routed to and from MEC applications and the prioritisation of that traffic, also based on traffic rules required by applications. Based on the above, it is straightforward to realise that the fundamental design behind the MEC architecture is to enable a network-aware application design, namely, to allow MEC applications and MEC platforms to leverage network information to satisfy their requirements.

On the contrary, the 3GPP-based network slice concept envisions an architectural shift as it relies on an \textit{communication service-centric} network provisioning. For this reason, the ETSI MEC group has recently started discussing which new MEC functionalities and interfaces, as well as extensions to existing MEC components, are required to support network slicing, e.g. by including network slice ID into different MEC interfaces. The ETSI report~\cite{MEC024} has identified several use cases based on the different network slicing concepts that are advocated in different SDOs. 

%
\begin{figure}[htbp]
\centering
\includegraphics[clip,trim= 2cm 3.5cm 4cm 9cm,width=0.48\textwidth]{figures/fig_e2e_latency.pdf}
\caption{End-to-end network latency of a NSI that include MEC according to~\cite{MEC024}.}
\label{fig:e2e_latency}
\vspace{-0.2cm}
\end{figure}
%
The most advanced proposal for supporting network slicing in MEC systems is the so-called \emph{MEC-in-NFV architecture} which permits to deploy MEC applications and MEC platforms as VNFs. To integrate MEC hosts in a NFV environment, ETSI standard proposes to substitute $(i)$ the MEPM with a MEPM-V entity that delegates the MEC-based VNF life-cycle management to one or more VNF managers (VNFM), and $(ii)$ the MEO with a MEC application orchestrator (MEAO) that delegates to the NFV orchestrator (NFVO) the resource orchestration for MEC VNFs. In~\cite{MEC024} different uses cases are also discussed in the context of the MEC-in-NFV architecture, for instance, to enable the sharing of a MEC host with several NSIs, or to allow MEC applications belonging to multiple tenants to be deployed in a single NSI. ETSI MEC also recognises the importance of the contribution of MEC applications to end-to-end latency. Thus,~\cite{MEC024} also presents a use case in which the MEC platform is included in a 3GPP NSI, and the end-to-end network delay budget takes into account MEC components and in particular, the delay from the data network to the MEC platform as shown in Figure~\ref{fig:e2e_latency}. In this case, the descriptor of the network services to be created in the NFV environment shall include the network latency requirement, which is distributed to the access network, core network, transport network and data network. This would require extending the data model of the NFV service descriptors to include the Application Descriptor, which contains fields to indicate the type of traffic to offload and the MEC service to consume. However, such a proposal has the drawback of requiring the 5G CSMF to translate application-service-related requirements into network-slice-related requirements. Indeed, SLA frameworks regarding application services are typically different from communication services (as per the 3GPP definition). For instance, data backup is often part of an application service SLA but is not part of a communication service SLA. Thus, radical changes in CSMF implementations would be needed to support such use cases.

A recent study~\cite{Cominardi2020} has proposed a harmonised view of the use cases defined in~\cite{MEC024}, to enable MEC frameworks to provide transparent end-to-end multi-slice and multi-tenant support by facilitating the sharing of MEC applications and MEC platforms among multiple NSIs. To this end, an inter-slice communication mechanism is proposed that automatically filters exchanged data between network slices installed on the same MEC facilities. The work in~ \cite{2020_MNET_MEC_subslice} focuses on 5G networks and presents extensions of the slice management architecture to enable the integration of MEC segment into the network slice as a slice subnet by introducing the MEC NSSMF component. Our work takes the architecture in~\cite{Cominardi2020} and~\cite{2020_MNET_MEC_subslice} as starting points and we envision a business model in which multiple tenants exist who are willing to pay to get isolated slices of network and edge computing resources.  However, we go one step further with respect to~\cite{Cominardi2020} and~\cite{2020_MNET_MEC_subslice} as we show how tenants can be provided with complete but isolated MEC stacks so that they can independently orchestrate their assigned resources to the served customers. Furthermore, in our management architecture a tenant is not necessarily an end customer, but, as explained in details in following section, it may also be a service provider that offers its services on top of a virtualised MEC instance (MEC as a Service provisioning model). 
%%
