\section{Preliminaries: Instrumental Variables and Conditional Moment Restrictions}\label{sec:cmr}

We first introduce the concept of Instrumental Variables (IVs) and its connection to Conditional Moment Restrictions (CMRs). Consider a structural model that specifies some outcome $Y$ given treatments $X$: 
\begin{align}
    Y=f(X)+\epsilon(U) \ \text{ with } \  \expectE[\epsilon(U)]=0,\label{eq:iv_reg}
\end{align}
where $U$ is a hidden confounder that affects both $X$ and $Y$ so that $\expectE[\varepsilon(U) \mid X]\neq 0$. Due to the presence of this hidden confounder, standard regressions (e.g., ordinary least squares) generally fail to produce consistent estimates of the causal relationship between $X$ on $Y$, i.e., $\expectE[Y \mid do(X)]=f(X)$, where $do(\cdot)$ is the interventional operator~\citep{Pearl2000}. If we only have observational data, a classic technique for learning $f$ is IV regression~\citep{Newey2003}. An IV $Z$ is an observable variable that satisfies the following conditions: 
\begin{itemize}[leftmargin=10pt, topsep=5pt]
\label{assump:iv}
    \item \textit{Unconfounded Instrument}: $Z\indep U$;
    \item \textit{Relevance}: $\probP(X\lvert Z)$ is not constant in $Z$;
    \item \textit{Exclusion}: $Z$ does not directly affect $Y$: $Z\indep Y \mid (X,U)$.
\end{itemize}

Using IVs, we are able to formulate the problem of learning $f$ into a set of CMRs~\citep{Dikkala2020}, where we aim to solve for $f$ satisfying
\begin{align}
\expectE[Y-f(X)\mid Z]=0.\nonumber
\end{align}

In our work, we show that trajectory histories can be used as instruments to learn the causal relationship between states and expert actions by transforming the problem of causal IL into a set of CMRs.

