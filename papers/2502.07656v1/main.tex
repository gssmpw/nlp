
\documentclass{article}
\usepackage{geometry}
\newgeometry{
    textheight=8.5in, % 9in 
    textwidth=6.5in, % 5.5in 
    top=1in,    % 1in 
    headheight=12pt,
    headsep=25pt
    % footskip=30pt 
  }
  
\usepackage[utf8]{inputenc} % allow utf-8 input
\usepackage[T1]{fontenc}    % use 8-bit T1 fonts

\usepackage{microtype}
\usepackage{graphicx}
% \usepackage{subfigure}
\usepackage{booktabs} % for professional tables

\RequirePackage{natbib}
% modification to natbib citations
\setcitestyle{authoryear,round,citesep={,},aysep={,},yysep={;}, sorted}

\usepackage{hyperref}


\newcommand{\theHalgorithm}{\arabic{algorithm}}

% Use the following line for the initial blind version submitted for review:


% For theorems and such
\usepackage{mathtools}

% if you use cleveref..
\usepackage[capitalize,noabbrev]{cleveref}

\usepackage{verbatim}
\usepackage{amsgen,amsmath,amstext,amsbsy,amsopn,tikz,amssymb,amsthm}
\usepackage{fancyhdr}
\usepackage{dsfont}
\usepackage{array,multirow}
% \usepackage{caption}
\usepackage{subcaption}
\usepackage{enumitem}
\usepackage{nicefrac}
\usepackage{etoc}
\usepackage[export]{adjustbox}
\usepackage{algorithm}
\usepackage{algorithmic}

\graphicspath{{./figures/}}

\theoremstyle{plain}
\newtheorem{theorem}{Theorem}[section]
\newtheorem{proposition}[theorem]{Proposition}
\newtheorem{lemma}[theorem]{Lemma}
\newtheorem{corollary}[theorem]{Corollary}
\theoremstyle{definition}
\newtheorem{definition}[theorem]{Definition}
\newtheorem{assumption}[theorem]{Assumption}
\newtheorem{condition}[theorem]{Condition}
\newtheorem{example}[theorem]{Example}
% \theoremstyle{remark}
\newtheorem{remark}[theorem]{Remark}


\newcommand{\marta}[1]{\rm {\raggedright\color{green}\textsf{MK: #1}\marginpar{$\star$}}}   

\newcommand{\bill}[1]{\rm {\raggedright\color{red}\textsf{DS: #1}\marginpar{$\star$}}} 

\newcommand{\thomas}[1]{\rm {\raggedright\color{purple}\textsf{TKB: #1}\marginpar{$\star$}}} 


\DeclarePairedDelimiter\abs{\lvert}{\rvert}
\DeclarePairedDelimiter\norm{\lVert}{\rVert}
\DeclareMathOperator*{\argmax}{arg\,max}
\DeclareMathOperator*{\argmin}{arg\,min}
\DeclareMathOperator{\sign}{sign}


\setlength{\skip\footins}{1.2pc plus 5pt minus 2pt}


\title{\vspace{-1.5cm} \bfseries A Unifying Framework for Causal Imitation Learning \\with Hidden Confounders}

\usepackage{authblk}
\author[1]{Daqian Shao}
\author[2]{Thomas Kleine Buening}
\author[1]{Marta Kwiatkowska}
\affil[1]{Department of Computer Science, University of Oxford, UK}
\affil[2]{The Alan Turing Institute, UK}

\renewcommand*{\Authands}{, }
\renewcommand*{\Authand}{, }%<---------------remove and
\renewcommand\Affilfont{\small}
\renewcommand\Authfont{\normalsize}

\date{\vspace{-0.25cm}}

\begin{document} 
\maketitle


\def\method{\text MixMin~}
\def\methodnospace{\text MixMin}
\def\genmethod{$\mathbb{R}$\text Min~}
\def\genmethodnospace{ $\mathbb{R}$\text Min}

\begin{abstract}  
Test time scaling is currently one of the most active research areas that shows promise after training time scaling has reached its limits.
Deep-thinking (DT) models are a class of recurrent models that can perform easy-to-hard generalization by assigning more compute to harder test samples.
However, due to their inability to determine the complexity of a test sample, DT models have to use a large amount of computation for both easy and hard test samples.
Excessive test time computation is wasteful and can cause the ``overthinking'' problem where more test time computation leads to worse results.
In this paper, we introduce a test time training method for determining the optimal amount of computation needed for each sample during test time.
We also propose Conv-LiGRU, a novel recurrent architecture for efficient and robust visual reasoning. 
Extensive experiments demonstrate that Conv-LiGRU is more stable than DT, effectively mitigates the ``overthinking'' phenomenon, and achieves superior accuracy.
\end{abstract}  
\section{Introduction}
Backdoor attacks pose a concealed yet profound security risk to machine learning (ML) models, for which the adversaries can inject a stealth backdoor into the model during training, enabling them to illicitly control the model's output upon encountering predefined inputs. These attacks can even occur without the knowledge of developers or end-users, thereby undermining the trust in ML systems. As ML becomes more deeply embedded in critical sectors like finance, healthcare, and autonomous driving \citep{he2016deep, liu2020computing, tournier2019mrtrix3, adjabi2020past}, the potential damage from backdoor attacks grows, underscoring the emergency for developing robust defense mechanisms against backdoor attacks.

To address the threat of backdoor attacks, researchers have developed a variety of strategies \cite{liu2018fine,wu2021adversarial,wang2019neural,zeng2022adversarial,zhu2023neural,Zhu_2023_ICCV, wei2024shared,wei2024d3}, aimed at purifying backdoors within victim models. These methods are designed to integrate with current deployment workflows seamlessly and have demonstrated significant success in mitigating the effects of backdoor triggers \cite{wubackdoorbench, wu2023defenses, wu2024backdoorbench,dunnett2024countering}.  However, most state-of-the-art (SOTA) backdoor purification methods operate under the assumption that a small clean dataset, often referred to as \textbf{auxiliary dataset}, is available for purification. Such an assumption poses practical challenges, especially in scenarios where data is scarce. To tackle this challenge, efforts have been made to reduce the size of the required auxiliary dataset~\cite{chai2022oneshot,li2023reconstructive, Zhu_2023_ICCV} and even explore dataset-free purification techniques~\cite{zheng2022data,hong2023revisiting,lin2024fusing}. Although these approaches offer some improvements, recent evaluations \cite{dunnett2024countering, wu2024backdoorbench} continue to highlight the importance of sufficient auxiliary data for achieving robust defenses against backdoor attacks.

While significant progress has been made in reducing the size of auxiliary datasets, an equally critical yet underexplored question remains: \emph{how does the nature of the auxiliary dataset affect purification effectiveness?} In  real-world  applications, auxiliary datasets can vary widely, encompassing in-distribution data, synthetic data, or external data from different sources. Understanding how each type of auxiliary dataset influences the purification effectiveness is vital for selecting or constructing the most suitable auxiliary dataset and the corresponding technique. For instance, when multiple datasets are available, understanding how different datasets contribute to purification can guide defenders in selecting or crafting the most appropriate dataset. Conversely, when only limited auxiliary data is accessible, knowing which purification technique works best under those constraints is critical. Therefore, there is an urgent need for a thorough investigation into the impact of auxiliary datasets on purification effectiveness to guide defenders in  enhancing the security of ML systems. 

In this paper, we systematically investigate the critical role of auxiliary datasets in backdoor purification, aiming to bridge the gap between idealized and practical purification scenarios.  Specifically, we first construct a diverse set of auxiliary datasets to emulate real-world conditions, as summarized in Table~\ref{overall}. These datasets include in-distribution data, synthetic data, and external data from other sources. Through an evaluation of SOTA backdoor purification methods across these datasets, we uncover several critical insights: \textbf{1)} In-distribution datasets, particularly those carefully filtered from the original training data of the victim model, effectively preserve the model’s utility for its intended tasks but may fall short in eliminating backdoors. \textbf{2)} Incorporating OOD datasets can help the model forget backdoors but also bring the risk of forgetting critical learned knowledge, significantly degrading its overall performance. Building on these findings, we propose Guided Input Calibration (GIC), a novel technique that enhances backdoor purification by adaptively transforming auxiliary data to better align with the victim model’s learned representations. By leveraging the victim model itself to guide this transformation, GIC optimizes the purification process, striking a balance between preserving model utility and mitigating backdoor threats. Extensive experiments demonstrate that GIC significantly improves the effectiveness of backdoor purification across diverse auxiliary datasets, providing a practical and robust defense solution.

Our main contributions are threefold:
\textbf{1) Impact analysis of auxiliary datasets:} We take the \textbf{first step}  in systematically investigating how different types of auxiliary datasets influence backdoor purification effectiveness. Our findings provide novel insights and serve as a foundation for future research on optimizing dataset selection and construction for enhanced backdoor defense.
%
\textbf{2) Compilation and evaluation of diverse auxiliary datasets:}  We have compiled and rigorously evaluated a diverse set of auxiliary datasets using SOTA purification methods, making our datasets and code publicly available to facilitate and support future research on practical backdoor defense strategies.
%
\textbf{3) Introduction of GIC:} We introduce GIC, the \textbf{first} dedicated solution designed to align auxiliary datasets with the model’s learned representations, significantly enhancing backdoor mitigation across various dataset types. Our approach sets a new benchmark for practical and effective backdoor defense.



For an integer $n$, let $[n]$ denote the set $\{1,2, \cdots, n\}$. For a finite set $A$, let $|A|$ be the number of elements in $A$, and $\Delta_{A}$ denote the set of all distributions on $A$. 

\paragraph{Proper Scoring Rule} Given a finite set $\Sigset$, a scoring rule $\ps: \Sigset\times \Delta_{\Sigset} \to \mathbb{R}$ maps an element $\sigi\in\Sigset$ and a distribution $\vpr$ on $\Sigset$ to a score. A scoring rule $PS$ is {\em proper} if for any distributions $\vpr_1$ and $\vpr_2$, $\Ex_{\sigi \sim \vpr_1}[\ps(\sigi, \vpr_1)] \ge \Ex_{\sigi \sim \vpr_1}[\ps(\sigi, \vpr_2)]$ and {\em strictly proper} if the equality holds only at $\vpr_1 = \vpr_2$. 
\begin{example}
    Given a distribution $\prQ$ on a finite set $\Sigset$, let $\pr(s)$ be the probability of $s\in \Sigset$ in $\prQ$. The log score rule $\ps_L(s, \prQ) = \log (\pr(s))$. The Brier/quadratic scoring rule $\ps_B(s, \prQ) = 2\cdot \pr(s) - \prQ\cdot \prQ$. Both the log scoring rule and the Brier scoring rule are strictly proper. 
\end{example}

\subsection{Bayesian Game Model}
A Bayesian game $\inst = ([n], (\rpset_i)_{i \in [n]}, (\Sigset_i)_{i\in[n]}, (\vt_i)_{i\in [n]}, \prQ)$ is defined by the following components. 
\begin{itemize}
    \item The set of agents $[n]$. 
    \item For each agent $i$, $\rpset_i$ is the set of available actions of $i$. The action profile $\rpp = (\rp_1, \rp_2, \cdots, \rp_\ag)$ is the vector of actions of all the agents. 
    \item For each agent $i$, $\Sigset_i$ is the set of possible types of agent $i$. The type characterizes the private information agent $i$ holds, and the agent can only observe his/her type in the game. The type vector $\sigp = (\sigi_1, \sigi_2, \cdots, \sigi_\ag)$ is the vector of types of all agents. 
    \item For each agent $i$, $\vt_i: \Sigset_i \times \rpset_1\times \cdots \times \rpset_n \to \mathbb{R}$ is $i$'s utility function that maps $i$'s type and the action of all the agents to $i$'s utility. 
    \item A {\em common prior} that the types of the agents follow is a joint distribution $\prQ$. For a signal $\sigi_i$ of agent $i$, we use $\pr(\sigi_i)$ to denote the marginal prior probability that $i$'s signal is $\sigi_i$. We assume that $\pr(\sigi_i) > 0$ for any $i$ and any $\sigi_i \in \Sigset_i$. 
\end{itemize}

For each agent $i$, a (mixed) strategy $\stg_i: \Sigset_i \to \Delta_{\rpset_i}$ maps $i$ private signal to a distribution on his/her actions. A strategy profile $\stgp = (\stg_i)_{i \in [n]}$ is a vector of the strategies of all the agents. 

Given a strategy profile $\stgp$, the {\em ex-ante} expected utility of agent $i$ is 
\begin{equation*}
    \ut_i(\stgp) = \Ex_{S \sim \prQ}\ \Ex_{A}[\vt_i(\sigi_i, \rp_1, \cdots, \rp_n)\mid \stgp].
\end{equation*}

Similarly, given a strategy profile $\stgp$ and a type $\sigi_i$, the {\em \qi{}} expected utility of agent $i$ conditioned on his/her type being $\sigi_i$ is 
\begin{equation*}
    \ut_i(\stgp \mid \sigi_i) = \Ex_{S_{-i} \sim \prQ_{-i\mid \sigi_i}}\ \Ex_{A}[\vt_i(\sigi_i, \rp_1, \cdots, \rp_n)\mid \stgp],
\end{equation*}
where $\sigp_{-i}$ is the type vector of all agents except for agent $i$, and $\prQ_{-i\mid \sigi_i}$ is the joint distribution on $\sigp_{-i}$ conditioned on agent $i$'s signal being $\sigi_{i}$. 

\subsection{(Ex-ante) Bayesian \textit{k}-Strong Equilibrium}
In this paper, we focus on agents that coordinate for strategic behaviors before they know their types. This assumption relates to various constraints in real-world scenarios that prevent agents from discussions after knowing their types. 
\begin{example}
    Consider the online crowdsourcing group in Example~\ref{ex:motive}. The website requires workers to make an immediate report after seeing the task so that workers cannot communicate with each other after they know their types. (For example, workers have to submit the report in 30 seconds to reflect their intuition.) However, workers may collude on the same report before seeing the task.
\end{example}
Both equilibria share the same high-level form: there does not exist a group of $\kd$ agents and a deviating strategy such that all the deviators' expected utility in the deviation is as good as the equilibrium strategy profile and at least one deviator's expected utility strictly increases. The difference lies in the expected utility. Ex-ante Bayesian $\kd$-strong equilibrium adopts ex-ante expected utility, while Bayesian $\kd$-strong equilibrium adopts interim expected utility on every type. 

\begin{definition}[ex-ante Bayesian $\kd$-strong equilibrium]

\label{def:ex_ante}
    Given an integer $\kd \ge 1$, a strategy profile $\stgp$ is an ex-ante Bayesian $k$-strong equilibrium ($\kd$-EBSE) if there does not exist a group of agent $D$ with $|D| \le \kd$ and a different strategy profile $\stgp' = (\stg'_{\sag})$ such that 
    \begin{enumerate}
    \item for all agent $i \not \in D$, $\stg'_{\sag} = \stg_{\sag}$; 
    \item for all $\sag\in D$, $ \ut_i(\stgp') \ge  \ut_i(\stgp)$;
    \item there exists an $\sag\in D$ such that $\ut_i(\stgp') > \ut_i(\stgp)$. 
\end{enumerate}
\end{definition}

\begin{definition}[Bayesian $\kd$-strong equilibrium]
\label{def:qi}
    Given an integer $\kd \ge 1$, a strategy profile $\stgp$ is a Bayesian $\kd$-strong equilibrium ($\kd$-BSE) if there does not exist a group of agent $D$ with $|D| \le k$ and a different strategy profile $\stgp' = (\stg'_{\sag})$ such that 
    \begin{enumerate}
    \item for all agent $i \not \in D$, $\stg'_i = \stg_i$; 
    \item for every $\sag\in D$ and every $\sigi_i \in \Sigset_i$, $ \ut_i(\stgp'\mid \sigi_i) \ge  \ut_i(\stgp\mid \sigi_i)$;
    \item there exist an $i\in D$ and an $\sigi_i \in \Sigset_i$ such that $\ut_i(\stgp'\mid \sigi_i) > \ut_i(\stgp\mid \sigi_i)$. 
\end{enumerate}
\end{definition}

% \begin{definition}[$\varepsilon$-approximation $\kd$-strong \textbf{interim} Bayesian-Nash Equilibrium]
%     Given an integer $\kd \ge 1$ and a constant $\varepsilon 
%     \ge 0$, a strategy profile $\stgp$ is an $\varepsilon$-approximation $\kd$-strong interim Bayesian-Nash Equilibrium ($\varepsilon$-apx-$\kd$-strong IBNE) if there does not exist a group of agent $D$ with $|D| \le k$ and a different strategy profile $\stgp' = (\stg'_{\sag})$ such that 
%     \begin{enumerate}
%     \item for all agent $i \not \in D$, $\stg'_i = \stg_i$; 
%     \item For every $\sag\in D$, \textbf{there exists a }$\sigi_i \in \Sigset_i$ such that $ \ut_i(\stgp'\mid \sigi_i) \ge  \ut_i(\stgp\mid \sigi_i)$.
%     \item There exists a $i\in D$ and a $\sigi_i \in \Sigset_i$ such that $\ut_i(\stgp'\mid \sigi_i) > \ut_i(\stgp\mid \sigi_i) + \varepsilon$. 
% \end{enumerate}
% \end{definition}

In both solution concepts, if such a deviating group $D$ and a strategy profile $\stgp'$ exist, we say that the deviation succeeds.

When $\kd = 1$, both ex-ante Bayesian $1$-strong equilibrium and Bayesian $1$-strong equilibrium are equivalent to the Bayesian Nash equilibrium~\citep{Harsanyi67}. (See Appendix~\ref{apx:equiv}.) However, the two solution concepts are not equivalent for larger $\kd$. Example~\ref{ex:difference} illustrates a scenario in the peer prediction mechanism where the same deviation succeeds under the ex-ante Bayesian $\kd$-strong equilibrium but fails under the Bayesian $\kd$-strong equilibrium. 

We interpret the difference between the two solution concepts as different attitudes of agents towards deviations. Agents are assumed to be more conservative, i.e., unwilling to suffer loss, towards deviations under Bayesian $k$-strong equilibrium, as they will deviate only when the deviation brings them higher interim expected utility conditioned on every type. On the other hand, agents under the ex-ante Bayesian $k$-strong equilibrium will deviate once their ex-ante expected utility increases.  Proposition~\ref{prop:etoq} supports our interpretation by revealing that an ex-ante Bayesian $\kd$-strong equilibrium implies a Bayesian $\kd$-strong equilibrium. 

\begin{prop}
\label{prop:etoq}
    For every strategy profile $\stgp$ and every $1\le \kd \le \ag$, if $\stgp$ is an ex-ante Bayesian $\kd$-strong equilibrium, then $\stgp$ is a Bayesian $\kd$-strong equilibrium. 
\end{prop}
\begin{proof}
    Suppose $\stgp'$ is an arbitrary deviating profile from $\stgp$ with no more than $\kd$ deviators, and $i$ is an arbitrary deviator in $\stgp'$. 
    Since $\stgp$ is an ex-ante Bayesian $\kd$-strong equilibrium, then $\ut_i (\stgp') \le \ut_i (\stgp) $. By the law of total probability,  
    $\ut_i(\stgp) = \sum_{\sigi_i \in \Sigset_i} \prQ(\sigi_i)\cdot \ut_i(\stgp \mid \sigi_i)$. 
    Therefore, one of the following must hold: (1) for all $\sigi\in \Sigset_i$, $\ut_i (\stgp' \mid \sigi_i) = \ut_i (\stgp \mid \sigi_i)$, or (2) there exists a $\sigi\in\Sigset_i$, $\ut_i (\stgp' \mid \sigi_i) < \ut_i (\stgp \mid \sigi_i)$. In either case, the deviation fails. Therefore, $\stgp$ is a Bayesian $\kd$-strong equilibrium. 
\end{proof}

\subsection{Peer Prediction Mechanism}
In a peer prediction mechanism, each agent receives a private signal in $\Sigset = \{\ell, h\}$ and reports it to the mechanism. All the agents share the same type set $\Sigset_i = \Sigset$ and action set $\rpset_i = \Sigset$. 

$\prQ$ is the common prior joint distribution of the signals. Let $\Sigrv_{\sag}$ denote the random variable of agent $i$'s private signal.
% Formally, the probability that agent 1 has signal $\sigi_1$, agent 2 has signal $\sigi_2$, $\cdots$, and agent $\ag$ has signal $\sigi_\ag$ is $\pr(\Sigrv_1=\sigi_1, \Sigrv_2 = \sigi_2,\cdots, \Sigrv_\ag = \sigi_\ag)$. 
We assume that the common prior $\prQ$ is symmetric --- for any permutation $\pi$ on $[\ag]$, $\prQ(\Sigrv_1=\sigi_1, \Sigrv_2 = \sigi_2,\cdots, \Sigrv_\ag = \sigi_\ag)=\pr(\Sigrv_1=\sigi_{\pi(1)}, \Sigrv_2 = \sigi_{\pi(2)},\cdots, \Sigrv_\ag = \sigi_{\pi(\ag)})$. 

$\pr(\sigi)$ is the prior marginal belief that an agent has signal $\sigi$, and $\pr(\sigi \mid \sigi')$ be the posterior belief of an agent with private signal $\sigi'$ on another agent having signal $\sigi$. We also define $\vpr_{\sigi} = \pr(\cdot \mid \sigi)$ be the marginal distribution on $\Sigset$ conditioned on $\sigi$. We assume that an agent with $h$ signal has a higher estimation than an agent with $\ell$ signal on the probability that another agent has $h$ signal, i.e., $\pr(h\mid h) > \pr(h \mid \ell)$. We also assume that any pair of signals is not fully correlated, which is $\pr(h\mid \ell) > 0$ and $\pr(\ell \mid h) > 0$. 

We adopt a modified version of the peer prediction mechanism~\citep{Miller05:Eliciting} characterized by a (strictly) proper scoring rule $\ps$. The mechanism compares the report of agent $i$, denoted by $\rp_i$, with the reports of all other agents. For each agent $j$ with report $\rp_j$, the reward $i$ gains from comparison with $j$'s report is $\rwd_i(\rp_j) = \ps(\rp_j, \vpr_{\rp_i}).$
The utility of agent $i$ is the average reward from each $j$.
\begin{equation*}
    \vt_i(\sigi_i, \rpp) = \frac{1}{\ag-1}\sum_{j\in[n], j\neq i} \rwd_i(\rp_j). 
\end{equation*}
\begin{remark}
    In the original mechanism in~\citep{Miller05:Eliciting}, the reward of an agent $i$ is $\rwd_i(\rp_j)$, where $j$ is chosen uniformly at random from all other agents. We derandomize the mechanism so that it fits better into the Bayesian game framework while the expected utility of an agent is unchanged. 
\end{remark}

\begin{example}
    \label{ex:setting}
    Suppose $n = 100$. For the common prior, the prior belief $\pr(h) = 2/3$, and $\pr(\ell) = 1/3$. The posterior belief $\pr(h \mid h) = 0.8$ and $\pr(\ell \mid \ell) = 0.6$. Suppose the Brier scoring rule is applied to the peer prediction mechanism. Consider an agent $i$ with report $\rp_i = h$. Then, $i$'s reward from a peer $j$ with report $\rp_j = h$ is $\rwd_i(\rp_j) = \ps_B(h, \prQ_h) = 2\cdot \pr(h \mid h) - \pr(h\mid h)^2 - \pr(\ell \mid h)^2 = 0.92$. Similarly, $i$' reward from another peer $j'$ with report $\rp_{j'} = \ell$ is $\ps_B(\ell, \prQ_h) = -0.28$. 
\end{example}

A (mixed) strategy $\stg: \Sigset_i \to \Delta_{\rpset_i}$ maps an agent's type to a distribution on his/her action. A strategy profile $\stgp = (\stg_i)_{i \in [n]}$ is a vector of the strategies of all the agents. An agent is {\em truthful} if he/she always truthfully reports his/her private signal. Let $\stg^*$ be the truthful strategy and $\stgp^*$ be the strategy profile where all agents are truthful. 
We also represent a strategy in the form $\stg = (\bpl, \bph) \in [0, 1]^2$, where $\bpl$ and $\bph$ are the probability that an agent playing $\stg$ reports $h$ conditioned on his/her signal begin $\ell$  and $h$, respectively. The truthful strategy $\stg^* = (0, 1)$.

Given the strategy profile $\stgp$, the ex-ante expected utility of an agent $i$ is
\begin{equation*}
    \ut_i(\stgp) = \frac{1}{n-1}\sum_{j\in [n], j\neq i} \Ex_{\sigi_i \sim \prQ
    , \rp_i \sim \stg_i(\sigi_i)} \Ex_{\sigi_j \sim \vpr_{\sigi_i}, \rp_j \sim \stg_j(\sigi_j)} \rwd_i(\rp_j). 
\end{equation*}

Given a strategy profile $\stgp$ and a type $\sigi_i$, the \qi{} expected utility of an agent $i$ conditioned on his/her type being $\sigi_i$ is 
\begin{equation*}
    \ut_i(\stgp\mid \sigi_i) = \frac{1}{n-1}\sum_{j\in [n], j\neq i} \Ex_{\rp_i \sim \stg_i(\sigi_i)} \Ex_{\sigi_j \sim \vpr_{\sigi_i}, \rp_j \sim \stg_j(\sigi_j)} \rwd_i(\rp_j). 
\end{equation*}

\begin{example} 
\label{ex:difference}
    We follow the setting in example~\ref{ex:setting}. Let $\stgp^*$ be the profile where all agents report truthfully. Let $D$ be a group containing $\kd = 40$ agents and $\stgp'$ be the profile where all deviators report $h$. 

    For truthful reporting, consider an agent $i$ and his/her peer $j$. The probability that both $i$ and $j$ receive (and report) signal $h$ is $\pr(h)\cdot \pr(h \mid h) = 2/3 * 0.8 = 0.533$, and $i$ will be rewarded $\ps(h, \vpr_h) = 0.92$. Other probabilities can be calculated similarly. Adding on the expectation of different pairs of signals, we can calculate the ex-ante expected utility of $i$ in truthful reporting: $\ut_i(\stgp^*) = \sum_{\sigi_i, \sigi_j \in \{\ell, h\}} \pr(\sigi_i) \cdot \pr(\sigi_j\mid \sigi_i)\cdot \ps(\sigi_j, \vpr_{\sigi_i}) = 0.627$. 

    Now we consider the expected utility of a deviator $i$ deviating profile $\stgp'$. Since all the deviators always report $h$, the expected reward $i$ gets from a deviator is $\ps(h, \vpr_h) = 0.92$. For the rewards from a truthful reporter, $i$'s expected reward is $\pr(h)\cdot \ps(h, \vpr_h) + \pr(\ell)\cdot \ps(\ell, \vpr_h) = 0.52$. Among all the other agents, $\kd - 1 = 39$ agents are deviators, and $\ag - \kd = 60$ agents are truthful reporters. Therefore, $i$'s expected utility on $\stgp'$ is $\ut_i(\stgp') = 0.682 > \ut_i(\stgp^*)$. Therefore, the deviation succeeds under the ex-ante Bayesian $\kd$-strong equilibrium. 

    However, the deviation fails under the Bayesian $\kd$-strong equilibrium. The truthful expected utility conditioned on $i$'s signal is $\ell$ is $\ut_i(\stgp^* \mid \ell) = \sum_{\sigi_j \in \{\ell, h\}} \pr(\sigi_j\mid \ell)\cdot \ps(\sigi_j, \vpr_{\ell}) = 0.52$. On the other hand, when agents deviate to $\stgp'$, $i$'s reward from a truthful agents becomes $\sum_{\sigi_j \in \{\ell, h\}} \pr(\sigi_j\mid \ell)\cdot \ps(\sigi_j, \vpr_{h}) = 0.2$. Therefore, $i$'s interim expected utility $\ut_i(\stgp' \mid \ell) = 0.484 < \ut_i(\stgp^* \mid \ell).$ 
\end{example}
\section{\sysname Framework}

\begin{figure}[t]
    \vspace{-0.2in}
    \centering
    \includegraphics[width=1\linewidth]{Figures/apint_framework.pdf}
    \caption{Overall \sysname Framework}
    \vspace{-0.2in}
    \label{fig:APINT_framework}
\end{figure}

In this section, we propose APINT, a full-stack framework designed to accelerate PiT by reducing the overhead of GC, the primary bottleneck of PiT. The overall workflow is illustrated in Figure~\ref{fig:APINT_framework}, distinguishing between compile-time and runtime processes.
In the initial compile time stage, \sysname begins by extracting nonlinear function operations in the process of computing the transformer model through the \sysname protocol. Next, through the GC-friendly circuit generation, the extracted function is implemented as a circuit consisting of a 2-input gate, and it is converted to netlist in Bristol format~\cite{tillich2016circuits}.
This step significantly alleviates the computational load on GC in subsequent stages. Following this, \sysname adopts a scheduling strategy that combines coarse-grained and fine-grained scheduling. The strategy enables full utilization of DRAM bandwidth and decrement in wire dependency. Furthermore, \sysname incorporates compiler speculation to generate instructions that capitalize on wire reusability and are executed on hardware accelerators at runtime. These accelerators, designed to further reduce memory stalls by eliminating redundant DRAM accesses, are deployed on both the server and client, allowing them to perform GC evaluation or GC garbling.
% These accelerators are deployed on both the server and client sides within the framework to further reduce memory stalls by eliminating redundant DRAM accesses.

\subsection{\sysname Protocol}

\begin{figure}[t]
    \vspace{-0.2in}
    \centering
    \includegraphics[width=1\linewidth]{Figures/Camera-ready/Fig4_Protocol.pdf}
    % \includegraphics[width=1\linewidth]{Figures/Protocol_Offload.pdf}
    \caption{\sysname Protocol}
    \vspace{-0.2in}
    \label{fig:APINT_protocol}
\end{figure}

\sysname protocol is based on the PiT protocol in PRIMER, but it reduces the circuit in GC operation by offloading its partial calculations to HE and standard operations, thereby significantly decreasing the workload of GC. As illustrated in Figure~\ref{fig:APINT_protocol}, the basic concept of the protocol is the combination of HE for linear operations and GC for nonlinear operations. To maintain confidentiality, each party adds or subtracts a random matrix ($R_i$ of the client and $S_i$ of the server) before sending the data to each other.
At the offline phase, HE is utilized to compute linear function for the client's random matrix $R_{1}$, which has the same matrix size as the input matrix $X_1$ \circled{1}. Simultaneously, the client garbles the circuit $\Tilde{C_1}$, which integrates adding the secret shares from both parties, processing the nonlinear function, and subtracting a random matrix to ensure confidentiality \circled{2}. The client then transmits labels of $R_2, R_3$ to the server \circled{3}. During the online phase, the server calculates the linear function of $(X_{1}-R_{1})$ using standard matrix operations. This intermediate result is merged with data from process \circled{1} to complete the computation of the linear function, yielding $(X_{2}-R_{2})$ \circled{4}. After that, the labels of $(X_{2}-R_{2})$ are sent from the client via the OT protocol~\cite{ishai2003extending} \circled{5}, and the server proceeds the GC evaluation for the garbled $\Tilde{C_1}$ \circled{6}.




However, in contrast to other functions, the reduced circuit $\Tilde{C_2}$ is employed when operating LayerNorm. This circuit specifically excludes calculations of mean and variance, as well as operations involving the parameters $\beta$ and $\gamma$. The excluded calculations are offloaded and computed using standard operations and HE, thereby reducing the workload of GC.
During the offline phase, the client garbles the circuit $\Tilde{C_2}$, transmitting the labels for $\sum{R_4/N}, R_5, R_6$, and also sends $Enc(R_2')$.
During the online phase, the mean of $(X_{2}-R_{2})$ is initially calculated using standard operations. This mean is then subtracted from $(X_{2}-R_{2})$, resulting in $(X_{2}'-R_{2}')$ \circled{7}. Subsequently, to prepare for variance calculations, this result is multiplied by two times $Enc(R_2')$ \circled{8}. The results from equation \circled{7} and \circled{8} are then used to compute the variance \circled{9}. Third, multiplying with the parameter $\beta$ can be processed by utilizing HE with $(X_{2}'-R_{2}')$, $Enc(R_2')$, and $\beta$ \circled{10}, \circled{11}. Then, the labels of the data from the process \circled{9} and \circled{11}, which are obtained from the client via OT protocol, and the labels from the client are computed through GC evaluation \circled{12}. Finally, a straightforward addition of the parameter $\gamma$ is processed \circled{13}.

Although this protocol incurs additional overhead due to HE and communication of two parties, it brings a substantial reduction of GC latency, offsetting these increased costs perfectly. This reduction marks a significant improvement over the baseline protocol, which merely utilized GC for processing the nonlinear functions. Ultimately, the \sysname protocol achieves a significant reduction in the online latency of GC operations, reducing it by 47.3\% during the LayerNorm computation.

\subsection{GC-friendly Circuit Generation}

To further minimize the workload of GC, \sysname proposes GC-friendly circuit generation of the nonlinear functions. It involves implementing each function to a circuit with 2-input AND, XOR, and INV gates while preserving the accuracy of computations. The process unfolds in two main steps.

The initial step focuses on minimizing the total number of gates in the circuit. Since GC processes the gates of the circuit sequentially, reducing the gate count directly lowers the overall computational load. For instance, in the implementation of Softmax, the method from i-BERT~\cite{kim2021bert} is adopted, which scales inputs by \textit{ln2}, thereby reducing the range of values and the number of required gates for exponential operations. The exponential operations are performed through combinational logic, which performs as a Look-Up Table (LUT) interpolation. For the GeLU function, LUT interpolation is utilized after clipping the input values within a range (-4, 4)~\cite{gupta2023sigma}. In LayerNorm, the conventional approach is employed without any approximation, as it doesn't incur any accuracy drop.

The second step aims to decrease the number of AND gates further. \sysname proposes a method employing XOR-Friendly Binary Quantization (XFBQ)~\cite{jian2020fast} to implement multiplication with fewer AND gates compared to the conventional method. This approach is motivated by the observation that the multiplication process accounts for a significant portion of each nonlinear function's implementation. 
Figure~\ref{fig:circuit_generation} (a) summarizes XFBQ and its multiplication process. XFBQ modifies the binary representation, wherein 1 represents 1 and 0 represents -1, exploiting the correspondence between the result patterns of XOR operations and the product of 1 and -1. The way to XFBQ is straightforward: it involves a right shift and changing the MSB to 1, introducing only a minimal quantization error (Q error) as small as the INV of Least Significant Bit (LSB) of the original number. For instance, \textit{1000}=8 turns into \textit{1100}=8+4-2-1=9 after XFBQ with Q error as \textit{INV(0)=1}.  When expressing conventional multiplication ($A\times B$) as XFBQ multiplication ($\hat{A} \times \hat{B}$) along with additional terms due to Q errors, the AND operations of the conventional method are all replaced by XOR, thereby a high reduction of AND gates occurs. Moreover, given that the Q error is INV of LSB value, its negligible impact on multiplication results and PiT warrants disregarding the additional terms, leading to a further decrease in AND gates.

\begin{figure}[t]
    \vspace{-0.2in}
    \centering
    \includegraphics[width=1\linewidth]{Figures/Camera-ready/xfbq_correct.pdf}
    \caption{(a) Reduction of ANDs via XBFQ Multiplication (b) Comparison of ANDs for 64b Multiplication}
    \vspace{-0.2in}
    \label{fig:circuit_generation}
\end{figure}

Figure~\ref{fig:circuit_generation} (b) shows the effects of the multiplication using XFBQ while operating 64b multiplication. It reduces the number of AND gates 38.9-45.5\% compared to prior work~\cite{liu2022don}, depending on the inclusion of Q error adjustments. In addition, GC-friendly circuit generation employed methods from the work~\cite{testa2020logic} for operations other than multiplication, which has been proven to perform as an open-source. Finally, it reduces the workload of GC for nonlinear functions by an average of 42.5\%.



   
\begin{figure*}[t]
    \vspace{-0.2in}
    \centering
    \includegraphics[width=1\linewidth]{Figures/Camera-ready/Scheduling_Reduced.pdf}
    \caption{(a) Methods and (b) Effects of Coarse-Grained and Fine-Grained Scheduling}
    \vspace{-0.2in}
    \label{fig:scheduling}
\end{figure*}
\subsection{Netlist Scheduling}
% \subsection{Compiler and Hardware Accelerator}
Despite the reductions in GC overhead facilitated by the \sysname protocol and GC-friendly circuit generation, GC still accounts for a notable portion of the latency. This implies that reducing GC overhead necessitates the integration of hardware accelerators beyond software solutions.
However, since a GC accelerator takes a netlist, converted from the circuit, as input and processes gates in the netlist sequentially, efficient netlist scheduling is crucial for hardware acceleration. Therefore, we introduce coarse-grained and fine-grained scheduling, maximizing DRAM bandwidth utilization and minimizing computational dependencies to accelerate the GC operation of nonlinear functions. 

% For instance, HAAC~\cite{mo2023haac} introduced an accelerator showcasing superior performance compared to CPUs and GPUs by utilizing multi-core designs operating concurrently and leveraging off-chip memory when on-chip memory resources are insufficient. However, this approach encounters a significant memory bottleneck when processing nonlinear functions of transformers due to inadequate scheduling schemes and an accelerator design overlooking wire reusability. Consequently, as a final step, \sysname introduces a compiler-integrated accelerator, addressing memory bottleneck by enhancing wire reuse and maximizing DRAM bandwidth utilization through compiler-driven strategies and hardware structure.

\subsubsection{\textbf{Coarse-Grained Scheduling}}

Due to the complex implementation of the circuit of the nonlinear function, the netlist exhibits highly irregular patterns in input and output wires, leading to irregular DRAM accesses that hinder optimal DRAM bandwidth utilization.
To tackle this issue, \sysname adopts coarse-grained scheduling, which maps each independent operation onto each core, allowing cores to function independently yet synchronously. This approach leverages the characteristic of nonlinear functions composed of independent unit operations, such as rows in Softmax, to be computed separately. 
Assuming there are two Processing Engines (PEs) and the need to compute two Softmax rows, the red box of Figure~\ref{fig:scheduling} (a) illustrates a DAG of two rows ordered in a depth-first manner without any scheduling, where node number corresponds to the order of the gates in the netlist. In this case, two PEs concurrently process the netlists of two rows in a dependent manner. In contrast, as illustrated in the green box, each PE exclusively handles an independent row with coarse-grained scheduling. Hence, contrary to the red box in Figure~\ref{fig:scheduling} (b), the green box demonstrates that coarse-grained scheduling allows all PEs to operate synchronously, ensuring they request DRAM data simultaneously. Therefore, while the intra-core DRAM access pattern is irregular, the inter-core DRAM access pattern becomes the same. As a result, by enabling cores to share the DRAM data bus, coarse-grained scheduling ensures maximal utilization of DRAM bandwidth.

Moreover, coarse-grained scheduling offers the additional benefit of resolving wire dependencies. Unlike the scenario without coarse-grained scheduling, the scheduling enables each PE to independently operate on a distinct row without dependencies. Thus, coarse-grained scheduling not only maximizes the utilization of DRAM bandwidth but also reduces the pipeline stalls.
        
\subsubsection{\textbf{Fine-Grained Scheduling}}
In addition to the coarse-grained scheduling, \sysname applies fine-grained scheduling that further diminishes GC latency compared to Full Reorder (FR) and Segment Reorder (SR), the scheduling method by the SOTA GC accelerator, HAAC~\cite{mo2023haac}. The FR transforms the netlist into a DAG and establishes processing order by traversing the graph in a breadth-first manner, reducing wire dependencies. However, due to limited on-chip memory, it can cause off-chip traffic by spilling wires over to DRAM, especially in applications where the DAG has a wide breadth, such as the nonlinear function of transformers. To tackle this problem, HAAC proposed the SR that segments the netlist, ordered in a depth-first manner, to enhance wire reuse and then applies FR within each segment to reduce wire dependencies. 
% \sysname applies fine-grained scheduling in addition to the coarse-grained scheduling. This further diminishes GC latency beyond the scheduling method proposed by HAAC, such as Full Reorder (FR) and Segment Reorder (SR).
% FR transforms the netlist into a DAG, where each node represents a gate. It establishes processing order by traversing the graph in a breadth-first manner, reducing wire dependencies but decreasing wire reuse in limited on-chip memory. Consequently, it can cause a memory bottleneck by spilling data over to DRAM, especially in applications where the DAG has a wide breadth, such as the nonlinear function of transformers. HAAC additionally proposed SR to tackle this problem. As shown in Figure~\ref{fig:scheduling}, SR firstly segments the netlist after depth-first scheduling to enhance wire reuse and applies FR within each segment to reduce dependencies.
% However, FR is not the optimal way to reduce dependencies within each segment. Recent research by S.Zhao~\cite{zhao2020dag, zhao2022dag} has highlighted the efficiency of priority scheduling based on Critical-Path-First-Execution (CPFE) in reducing dependencies of DAG. Therefore, \sysname proposes a fine-grained scheduling by combining SR and CPFE.
Despite these advancements, we identified opportunities for further latency reduction since FR does not optimally eliminate wire dependency within segments. Therefore, \sysname introduces a fine-grained scheduling strategy that combines segmentation and Critical-Path-First-Execution (CPFE)~\cite{zhao2020dag, zhao2022dag} instead of FR, achieving enhanced performance by effectively minimizing wire dependencies.

The fine-grained scheduling begins by segmenting the netlist, with each segment half the size of on-chip memory. A DAG is then constructed for each segment, with assigning weights reflecting the cycle latency of each gate. Nodes without children are linked to \textit{v\_{src}}, and those without parents are connected to \textit{v\_{sink}}. After establishing the DAG, it finds a critical path from \textit{v\_{src}} to \textit{v\_{sink}} and prioritizes nodes along this path, starting from the lowest depth. Subsequently, for each node on the path, a sub-DAG is formed comprising unprioritized descendants, and the process of identifying the critical path and assigning priorities repeats recursively.

The blue box in Figure~\ref{fig:scheduling} (a) shows how the fine-grained scheduling works. First, in step \circled{1}, it finds a critical path and prioritizes from the lowest depth. Then, from step \circled{2}-\circled{6}, it creates sub-DAG with unprioritized descendants for each node of the path and operates recursively. 
% As shown in Figure~\ref{fig:scheduling}, DAG \circled{1} shows finding a critical path and prioritizing from the lowest depth. Then, from DAG \circled{2} to DAG \circled{6}, it creates sub-DAG with unprioritized descendants for each node of the path and operates recursively.
For example, in step \circled{6}, nodes \textit{4} and \textit{6} compose the sub-DAG, and then the process of identifying the critical path and prioritizing is recursively executed at the sub-DAG. After assigning priorities to all nodes, the scheduling order is determined by the cycle-accurate simulation. The simulation selects the operable node with the highest priority in each cycle. The "operable" refers to the condition where both input wires of a DAG node have been produced. As a result, step \circled{6} is reordered as $2\rightarrow1\rightarrow4\rightarrow5\rightarrow6\rightarrow3\rightarrow8\rightarrow7$. 
Hence, as depicted in Figure~\ref{fig:scheduling}, fine-grained scheduling significantly reduces pipeline stalls by wire dependencies within each segment, enhancing the computation speed of nonlinear functions by an average of 30.2\% compared to the SR of HAAC.

\begin{figure*}[t]
    \vspace{-0.2in}
    \centering
    \includegraphics[width=1\linewidth]{Figures/Camera-ready/Hardware_Reduced.pdf}
    \caption{APINT hardware, Compiler Speculation Flow, and Runtime Flow Descriptions}
    \vspace{-0.2in}
    \label{fig:compiler_and_hardware}
\end{figure*}
\subsection{Accelerator with Compiler Speculation}
% \subsubsection{\textbf{Compiler Speculation and Hardware Accelerator}}
HAAC introduced an accelerator showcasing superior performance compared to CPUs and GPUs by utilizing pipelined multi-core designs operating concurrently and leveraging off-chip memory when on-chip memory resources are insufficient. However, it encounters a significant memory bottleneck when processing nonlinear functions of transformers due to inadequate on-chip memory policy and hardware structure.
% A further cause of the memory bottleneck is the on-chip memory policy and hardware structure of HAAC.
HAAC's approach of sequentially writing output wires in on-chip memory doesn't consider the wire reusability. Moreover, the hardware structure that involves directly fetching wires from DRAM to a PE via a queue structure limits the wire's usage to a single time, thereby restricting its potential for reuse. To counter these issues, \sysname suggests the accelerator alongside compiler speculation techniques, aiming to reduce unnecessary DRAM accesses and improve wire reuse.

\subsubsection{\textbf{APINT Accelerator}}
% \textit{\textbf{APINT Accelerator} \ }
Figure~\ref{fig:compiler_and_hardware} illustrates the architecture of \sysname accelerator, which features 16 independent cores operating synchronously under coarse-grained scheduling. This eliminates the need for inter-core communication, allowing for a shared unified Instruction Memory (16KB). Each core includes a Wire Memory (128KB), a Table Memory (2KB), an Out-of-Range-Wire (OoRW) Prefetch Buffer (1KB), and a PE, all of which are pipelined.
Wire Memory stores the wire's label (value of the wire) and special flag bits, which are a block bit and an Out-of-Range (OoR) bit, for each address. The block bit prevents other wires from accessing the address, while the OoR bit indicates that an OoRW, a wire that is fetched from DRAM, is being fetched to that address. Also, the Table Memory stores garble tables required for Half-Gate operations, and the OoRW Prefetch Buffer temporarily stores OoRWs fetched from DRAM. They are then transferred to the Wire Memory, which allows multiple reuses within the Wire Memory in contrast to HAAC, where OoRWs are used only once per fetch.

The execution of the accelerator is structured into four stages. First, upon receiving an instruction, the Write Address Preemption stage the write address in the Wire Memory, activating the block bit. Next, the Read stage reads two input wires from the memory or forwarding path over three cycles. Third, the input wires are processed in the Half-gate unit (taking 18 cycles for evaluation and 21 for garbling) or FreeXOR unit (taking one cycle) in PE, and OoRWs are transferred from the Prefetch Buffer to Wire Memory if required. Finally, the output wire generated in the PE is written back to Wire Memory over two cycles and, if needed, also to DRAM.

% The execution of the \sysname accelerator is structured into four stages: Write Address Preemption, Read, OoRW Transfer and PE Execution, and Write. Upon receiving an instruction, the accelerator preempts the write address in the Wire Memory, setting the block bit alive. It then reads input wires over three cycles and processes them in the PE, which includes a Half-Gate unit (taking 18 cycles for evaluation and 21 for garbling) and a FreeXOR unit (taking one cycle). Garble tables are transferred from the Table Memory if Half-Gate is operated. Simultaneously, OoRWs are transferred from the Prefetch Buffer to the Wire Memory if required. Finally, after the PE execution, output wires are written back to Wire Memory over two cycles and, if needed, also to DRAM.

\subsubsection{\textbf{Compiler Speculation Flow}}
Before running the accelerator, compiler speculation is initially processed with a netlist as input. Its purpose is to generate instructions for the accelerator, which implements a memory policy that enhances wire reuse. It proceeds through the following two phases, as depicted in Figure~\ref{fig:compiler_and_hardware}. During the first phase, it assigns read and write addresses in Wire Memory and an OP bit for each gate in the netlist through a cycle-accurate simulation. After filling Wire Memory as much as possible with operable input wires, the speculation begins with the Write Address Preemption stage, allocating a write address either to a blank space or to the Last-to-Be-Used Wire (LBUW), if Wire Memory is full. The LBUW is the wire that will be used last among wires within the memory. This demonstrates that APINT employs a memory policy considering the reusability of wires. After assigning the write address, the block bit is activated for the preemption.

% Compiler speculation begins by using a netlist as input, producing instructions generated to enhance wire reuse. This process proceeds through the following two phases after filling Wire Memory as much as possible with operable input wires. During the first phase, it assigns read and write addresses in Wire Memory and an OP bit for each gate through a cycle-accurate simulation. First, during the Write Address Preemption stage, if the Wire Memory is not full, the write address is assigned to a blank space, if not, to the address of the Last-to-Be-Used Wire (LBUW), designated to be used last among wires in Wire Memory, and a block bit for this address is activated. Therefore, this process enables the on-chip memory policy to consider the reusability of wires.

Next, the Read stage assigns the read address based on whether an input wire is present in Wire Memory. If present, the read address corresponds to its location. If not, indicating it is likely to become an OoRW at runtime, the address of the LBUW with inactive block bit is assigned, which also contributes to the memory policy that considers the reusability of wires. The input wire is then replaced with the LBUW and added to the OoRW list. Subsequently, the PE execution stage begins by setting the OP bit based on the gate type. This is followed by the Write stage, which writes the output wire at the assigned address and resets the block bit. This cycle-accurate simulation is repeated until every wire and gate in the netlist has been allocated the instructions with addresses and OP bits.

After completing the first phase, the second phase involves assigning the Live bit, two OoRW-fetch bits, and the Write Enable Not (WEN) bit, which are determined by analyzing the interrelationships among instructions. The Live bit is assigned to instructions that output an OoRW, designating that the wire should be written to DRAM for later use. For example, in Figure~\ref{fig:compiler_and_hardware}, instruction \circled{1} outputs OoRW \textit{30} and is marked with a Live bit of 1.
Each OoRW-fetch bit is assigned to ensure the timely transfer of an OoRW from the Prefetch Buffer to Wire Memory, based on instruction sequence and read dependencies.
For instance, instruction \circled{2}, which reads address \textit{0} immediately before instruction \circled{4} reads OoRW \textit{30} from the same address, is assigned an OoRW-fetch bit to ensure that OoRW \textit{30} is transferred right after instruction \circled{2} reads the address \textit{0} to prevent stalls due to non-arrival.
The WEN bit is assigned to prevent premature overwriting in Wire Memory.
For example, if OoRW \textit{30} is transferred to memory before instruction \circled{3} writes to address \textit{0}, it could be overwritten before it is read by instruction \circled{4}. Therefore, a WEN bit is assigned to instruction \circled{3} to prevent it from overwriting OoRW \textit{30}, and wire \textit{35} is written only to DRAM as dictated by the Live bit.
% For example, to avoid overwriting OoRW \textit{30} needed by instruction \circled{4}, instruction \circled{3} receives a WEN bit of 1.


% During the speculation process, the sequence for using instructions and garbled tables is predetermined and mapped sequentially in DRAM. Similarly, the DRAM read and write orders for OoRWs are predecided, with DRAM write addresses being assigned incrementally. Hence, if an OoRW's DRAM read address exceeds the current increment value, indicating the wire has yet to be written to DRAM, the accelerator stalls until the increment value matches the address. This approach removes the need to handle DRAM addresses during runtime, enabling the accelerator to fetch instructions, garbled tables, and OoRWs from DRAM to on-chip in a predetermined order while executing the instructions.

\subsubsection{\textbf{Runtime Flow}}

During the speculation process, the DRAM addresses for instructions, garbled tables, and OoRWs are predetermined, removing the need to handle the addresses during runtime. While fetching the data from DRAM to each corresponding memory, the runtime process is executed in the following four stages, as depicted in Figure~\ref{fig:compiler_and_hardware}.
% During runtime, the four stages are executed with the \sysname accelerator, as depicted in Figure~\ref{fig:compiler_and_hardware}. 
After an instruction is decoded, the Write Address Preemption stage activates the block bit at the write address, and the Read stage operates based on the statuses of the block and OoR bits. Depending on these bits, the accelerator either performs a normal read or stall until the necessary wire is transferred from the Prefetch Buffer or the forwarding path. The OoRW Transfer and PE Execution stage then commences. The OoR bit is assigned with the OoRW-fetch bit, indicating whether an OoRW transfer is started. If the OoRW-fetch bit is 1, the address is preempted by activating the block bit, and an OoRW begins to be transferred to the address. After the completion of the transfer, the block bit is deactivated. Concurrently, the PE processes Half-gate or FreeXOR operation based on the OP bit. After the output wire is generated in the PE, the Write stage begins, writing the wire to Wire Memory and DRAM depending on the WEN and Live bits. After these stages are executed across all instructions, the runtime process is completed. Overall, through APINT accelerator and compiler speculation, it achieves a reduction in memory stall times by 86.1\% to 99.4\% compared to HAAC when operating nonlinear functions.

% After an instruction is decoded, the Write Address Preemption stage sets the block bit at the write address, and the Read stage proceeds based on the block and OoR bit statuses. If the block bit is 0, a normal read is performed. Otherwise, the OoR bit determines if an OoRW or non-OoRW wire has preempted the address. An OoR bit of 1 indicates a scheduled OoRW write at that address, causing the accelerator to stall until the OoRW is transferred from the Prefetch Buffer to Wire Memory. If the OoR bit is 0, the accelerator is stalled until the wire is retrieved from the forwarding path.

% Once the Read stage is completed, the OoRW-fetch and PE execution stage begins. The OoR bit is assigned with the OoRW-fetch bit, preempting the address for an OoRW transfer. If the OoRW-fetch bit is 1, the block bit is activated, and the OoRW begins to be transferred from the Prefetch Buffer to the address. The block bit is deactivated after the transfer. Concurrently, the PE processes operations based on the OP bit, and the garbled table is transferred from the Table Memory for Half-Gate operation.

% Lastly, Write stage begins. Unless the WEN bit is active, the generated output wires are written to Wire Memory, and the block bit of the write address is deactivated. If the Live bit is active, the wires are also written to DRAM. After these stages are executed across all instructions, the runtime process is completed. Overall, through compiler speculation and the accelerator, APINT achieves a reduction in memory stall times by 74.3\% to 99.9\% compared to HAAC for nonlinear functions of transformers.

Effective human-robot cooperation in CoNav-Maze hinges on efficient communication. Maximizing the human’s information gain enables more precise guidance, which in turn accelerates task completion. Yet for the robot, the challenge is not only \emph{what} to communicate but also \emph{when}, as it must balance gathering information for the human with pursuing immediate goals when confident in its navigation.

To achieve this, we introduce \emph{Information Gain Monte Carlo Tree Search} (IG-MCTS), which optimizes both task-relevant objectives and the transmission of the most informative communication. IG-MCTS comprises three key components:
\textbf{(1)} A data-driven human perception model that tracks how implicit (movement) and explicit (image) information updates the human’s understanding of the maze layout.
\textbf{(2)} Reward augmentation to integrate multiple objectives effectively leveraging on the learned perception model.
\textbf{(3)} An uncertainty-aware MCTS that accounts for unobserved maze regions and human perception stochasticity.
% \begin{enumerate}[leftmargin=*]
%     \item A data-driven human perception model that tracks how implicit (movement) and explicit (image transmission) information updates the human’s understanding of the maze layout.
%     \item Reward augmentation to integrate multiple objectives effectively leveraging on the learned perception model.
%     \item An uncertainty-aware MCTS that accounts for unobserved maze regions and human perception stochasticity.
% \end{enumerate}

\subsection{Human Perception Dynamics}
% IG-MCTS seeks to optimize the expected novel information gained by the human through the robot’s actions, including both movement and communication. Achieving this requires a model of how the human acquires task-relevant information from the robot.

% \subsubsection{Perception MDP}
\label{sec:perception_mdp}
As the robot navigates the maze and transmits images, humans update their understanding of the environment. Based on the robot's path, they may infer that previously assumed blocked locations are traversable or detect discrepancies between the transmitted image and their map.  

To formally capture this process, we model the evolution of human perception as another Markov Decision Process, referred to as the \emph{Perception MDP}. The state space $\mathcal{X}$ represents all possible maze maps. The action space $\mathcal{S}^+ \times \mathcal{O}$ consists of the robot's trajectory between two image transmissions $\tau \in \mathcal{S}^+$ and an image $o \in \mathcal{O}$. The unknown transition function $F: (x, (\tau, o)) \rightarrow x'$ defines the human perception dynamics, which we aim to learn.

\subsubsection{Crowd-Sourced Transition Dataset}
To collect data, we designed a mapping task in the CoNav-Maze environment. Participants were tasked to edit their maps to match the true environment. A button triggers the robot's autonomous movements, after which it captures an image from a random angle.
In this mapping task, the robot, aware of both the true environment and the human’s map, visits predefined target locations and prioritizes areas with mislabeled grid cells on the human’s map.
% We assume that the robot has full knowledge of both the actual environment and the human’s current map. Leveraging this knowledge, the robot autonomously navigates to all predefined target locations. It then randomly selects subsequent goals to reach, prioritizing grid locations that remain mislabeled on the human’s map. This ensures that the robot’s actions are strategically focused on providing useful information to improve map accuracy.

We then recruited over $50$ annotators through Prolific~\cite{palan2018prolific} for the mapping task. Each annotator labeled three randomly generated mazes. They were allowed to proceed to the next maze once the robot had reached all four goal locations. However, they could spend additional time refining their map before moving on. To incentivize accuracy, annotators receive a performance-based bonus based on the final accuracy of their annotated map.


\subsubsection{Fully-Convolutional Dynamics Model}
\label{sec:nhpm}

We propose a Neural Human Perception Model (NHPM), a fully convolutional neural network (FCNN), to predict the human perception transition probabilities modeled in \Cref{sec:perception_mdp}. We denote the model as $F_\theta$ where $\theta$ represents the trainable weights. Such design echoes recent studies of model-based reinforcement learning~\cite{hansen2022temporal}, where the agent first learns the environment dynamics, potentially from image observations~\cite{hafner2019learning,watter2015embed}.

\begin{figure}[t]
    \centering
    \includegraphics[width=0.9\linewidth]{figures/ICML_25_CNN.pdf}
    \caption{Neural Human Perception Model (NHPM). \textbf{Left:} The human's current perception, the robot's trajectory since the last transmission, and the captured environment grids are individually processed into 2D masks. \textbf{Right:} A fully convolutional neural network predicts two masks: one for the probability of the human adding a wall to their map and another for removing a wall.}
    \label{fig:nhpm}
    \vskip -0.1in
\end{figure}

As illustrated in \Cref{fig:nhpm}, our model takes as input the human’s current perception, the robot’s path, and the image captured by the robot, all of which are transformed into a unified 2D representation. These inputs are concatenated along the channel dimension and fed into the CNN, which outputs a two-channel image: one predicting the probability of human adding a new wall and the other predicting the probability of removing a wall.

% Our approach builds on world model learning, where neural networks predict state transitions or environmental updates based on agent actions and observations. By leveraging the local feature extraction capabilities of CNNs, our model effectively captures spatial relationships and interprets local changes within the grid maze environment. Similar to prior work in localization and mapping, the CNN architecture is well-suited for processing spatially structured data and aligning the robot’s observations with human map updates.

To enhance robustness and generalization, we apply data augmentation techniques, including random rotation and flipping of the 2D inputs during training. These transformations are particularly beneficial in the grid maze environment, which is invariant to orientation changes.

\subsection{Perception-Aware Reward Augmentation}
The robot optimizes its actions over a planning horizon \( H \) by solving the following optimization problem:
\begin{subequations}
    \begin{align}
        \max_{a_{0:H-1}} \;
        & \mathop{\mathbb{E}}_{T, F} \left[ \sum_{t=0}^{H-1} \gamma^t \left(\underbrace{R_{\mathrm{task}}(\tau_{t+1}, \zeta)}_{\text{(1) Task reward}} + \underbrace{\|x_{t+1}-x_t\|_1}_{\text{(2) Info reward}}\right)\right] \label{obj}\\ 
        \subjectto \quad
        &x_{t+1} = F(x_t, (\tau_t, a_t)), \quad a_t\in\Ocal \label{const:perception_update}\\ 
        &\tau_{t+1} = \tau_t \oplus T(s_t, a_t), \quad a_t\in \Ucal\label{const:history_update}
    \end{align}
\end{subequations} 

The objective in~\eqref{obj} maximizes the expected cumulative reward over \( T \) and \( F \), reflecting the uncertainty in both physical transitions and human perception dynamics. The reward function consists of two components: 
(1) The \emph{task reward} incentivizes efficient navigation. The specific formulation for the task in this work is outlined in \Cref{appendix:task_reward}.
(2) The \emph{information reward} quantifies the change in the human’s perception due to robot actions, computed as the \( L_1 \)-norm distance between consecutive perception states.  

The constraint in~\eqref{const:history_update} ensures that for movement actions, the trajectory history \( \tau_t \) expands with new states based on the robot’s chosen actions, where \( s_t \) is the most recent state in \( \tau_t \), and \( \oplus \) represents sequence concatenation. 
In constraint~\eqref{const:perception_update}, the robot leverages the learned human perception dynamics \( F \) to estimate the evolution of the human’s understanding of the environment from perception state $x_t$ to $x_{t+1}$ based on the observed trajectory \( \tau_t \) and transmitted image \( a_t\in\Ocal \). 
% justify from a cognitive science perspective
% Cognitive science research has shown that humans read in a way to maximize the information gained from each word, aligning with the efficient coding principle, which prioritizes minimizing perceptual errors and extracting relevant features under limited processing capacity~\cite{kangassalo2020information}. Drawing on this principle, we hypothesize that humans similarly prioritize task-relevant information in multimodal settings. To accommodate this cognitive pattern, our robot policy selects and communicates high information-gain observations to human operators, akin to summarizing key insights from a lengthy article.
% % While the brain naturally seeks to gain information, the brain employs various strategies to manage information overload, including filtering~\cite{quiroga2004reducing}, limiting/working memory, and prioritizing information~\cite{arnold2023dealing}.
% In this context of our setup, we optimize the selection of camera angles to maximize the human operator's information gain about the environment. 

\subsection{Information Gain Monte Carlo Tree Search (IG-MCTS)}
IG-MCTS follows the four stages of Monte Carlo tree search: \emph{selection}, \emph{expansion}, \emph{rollout}, and \emph{backpropagation}, but extends it by incorporating uncertainty in both environment dynamics and human perception. We introduce uncertainty-aware simulations in the \emph{expansion} and \emph{rollout} phases and adjust \emph{backpropagation} with a value update rule that accounts for transition feasibility.

\subsubsection{Uncertainty-Aware Simulation}
As detailed in \Cref{algo:IG_MCTS}, both the \emph{expansion} and \emph{rollout} phases involve forward simulation of robot actions. Each tree node $v$ contains the state $(\tau, x)$, representing the robot's state history and current human perception. We handle the two action types differently as follows:
\begin{itemize}
    \item A movement action $u$ follows the environment dynamics $T$ as defined in \Cref{sec:problem}. Notably, the maze layout is observable up to distance $r$ from the robot's visited grids, while unexplored areas assume a $50\%$ chance of walls. In \emph{expansion}, the resulting search node $v'$ of this uncertain transition is assigned a feasibility value $\delta = 0.5$. In \emph{rollout}, the transition could fail and the robot remains in the same grid.
    
    \item The state transition for a communication step $o$ is governed by the learned stochastic human perception model $F_\theta$ as defined in \Cref{sec:nhpm}. Since transition probabilities are known, we compute the expected information reward $\bar{R_\mathrm{info}}$ directly:
    \begin{align*}
        \bar{R_\mathrm{info}}(\tau_t, x_t, o_t) &= \mathbb{E}_{x_{t+1}}\|x_{t+1}-x_t\|_1 \\
        &= \|p_\mathrm{add}\|_1 + \|p_\mathrm{remove}\|_1,
    \end{align*}
    where $(p_\mathrm{add}, p_\mathrm{remove}) \gets F_\theta(\tau_t, x_t, o_t)$ are the estimated probabilities of adding or removing walls from the map. 
    Directly computing the expected return at a node avoids the high number of visitations required to obtain an accurate value estimate.
\end{itemize}

% We denote a node in the search tree as $v$, where $s(v)$, $r(v)$, and $\delta(v)$ represent the state, reward, and transition feasibility at $v$, respectively. The visit count of $v$ is denoted as $N(v)$, while $Q(v)$ represents its total accumulated return. The set of child nodes of $v$ is denoted by $\mathbb{C}(v)$.

% The goal of each search is to plan a sequence for the robot until it reaches a goal or transmits a new image to the human. We initialize the search tree with the current human guidance $\zeta$, and the robot's approximation of human perception $x_0$. Each search node consists consists of the state information required by our reward augmentation: $(\tau, x)$. A node is terminal if it is the resulting state of a communication step, or if the robot reaches a goal location. 

% A rollout from the expanded node simulates future transitions until reaching a terminal state or a predefined depth $H$. Actions are selected randomly from the available action set $\mathcal{A}(s)$. If an action's feasibility is uncertain due to the environment's unknown structure, the transition occurs with probability $\delta(s, a)$. When a random number draw deems the transition infeasible, the state remains unchanged. On the other hand, for communication steps, we don't resolve the uncertainty but instead compute the expected information gain reward: \philip{TODO: adjust notation}
% \begin{equation}
%     \mathbb{E}\left[R_\mathrm{info}(\tau, x')\right] = \sum \mathrm{NPM(\tau, o)}.
% \end{equation}

\subsubsection{Feasibility-Adjusted Backpropagation}
During backpropagation, the rewards obtained from the simulation phase are propagated back through the tree, updating the total value $Q(v)$ and the visitation count $N(v)$ for all nodes along the path to the root. Due to uncertainty in unexplored environment dynamics, the rollout return depends on the feasibility of the transition from the child node. Given a sample return \(q'_{\mathrm{sample}}\) at child node \(v'\), the parent node's return is:
\begin{equation}
    q_{\mathrm{sample}} = r + \gamma \left[ \delta' q'_{\mathrm{sample}} + (1 - \delta') \frac{Q(v)}{N(v)} \right],
\end{equation}
where $\delta'$ represents the probability of a successful transition. The term \((1 - \delta')\) accounts for failed transitions, relying instead on the current value estimate.

% By incorporating uncertainty-aware rollouts and backpropagation, our approach enables more robust decision-making in scenarios where the environment dynamics is unknown and avoids simulation of the stochastic human perception dynamics.

% \section{Experiment and Results}
\section{Results and Analysis}
In this section, we first present safe vs. unsafe evaluation results for 12 LLMs, followed by fine-grained responding pattern analysis over six models among them, and compare models' behavior when they are attacked by same risky questions presented in different languages: Kazakh, Russian and code-switching language.    

\begin{table}[t!]
\centering
\small
\resizebox{\columnwidth}{!}{
\begin{tabular}{clcccc}
\toprule
\multicolumn{1}{l}{\textbf{Rank} } & \textbf{Model} & \textbf{Kazakh $\uparrow$} & \textbf{Russian $\uparrow$} & \textbf{English $\uparrow$} \\
\midrule
1 & \claude & \textbf{96.5}   & 93.5    & \textbf{98.6}    \\
2 & \gptfouro & 95.8   & 87.6    & 95.7    \\
3 & \yandexgpt & 90.7   & \textbf{93.6}    & 80.3    \\
4 & \kazllmseventy & 88.1 & 87.5 & 97.2 \\
5 & \llamaseventy & 88.0   & 85.5    & 95.7    \\
6 & \sherkala & 87.1   & 85.0    & 96.0    \\
7 & \falcon & 87.1   & 84.7    & 96.8    \\
8 & \qwen & 86.2   & 85.1    & 88.1    \\
9 & \llamaeight & 85.9   & 84.7    & 98.3    \\
10 & \kazllmeight & 74.8   & 78.0    & 94.5 \\
11 & \aya & 72.4 & 84.5 & 96.6 \\
12 & \vikhr & --- & 85.6 & 91.1 \\
\bottomrule
\end{tabular}
}
\caption{Safety evaluation results of 12 LLMs, ranked by the percentage of safe responses in the Kazakh dataset. \claude\ achieves the highest safety score for both Kazakh and English, while \yandexgpt\ is the safest model for Russian responses.}
\label{tab:safety-binary-eval}
\end{table}



\subsection{Safe vs. Unsafe Classification}
% In this subsection, 
We present binary evaluation results of 12 LLMs, by assessing 52,596 Russian responses and 41,646 Kazakh responses.
% 26,298 responses generated by six models on the Russian dataset and 22,716 responses on the Kazakh dataset. 

%\textbf{Safety Rank.} In general, proprietary systems outperform the open-source model. For Russian, As shown in Table \ref{tab:model_comparison_russian}, \textbf{Yandex-GPT} emerges as the safest large language model (LLM) for Russian, with a safety percentage of 93.57\%. Among the open-source models, \textbf{Vikhr-Nemo-12B} is the safest, achieving a safety percentage of 85.63\%.
% Edited: This is mentioned in the discussion
% This outcome highlights the potential impact of pretraining data on model behavior. Models pre-trained primarily on Russian data are better at understanding and handling harmful questions - in both proprietary systems and open-source setups. 
%For Kazakh, as shown in Table \ref{tab:model_comparison_kazakh}, \textbf{Claude} emerges as the safest large language model (LLM) with a safety percentage of 96.46\%, closely followed by GPT-4o at 95.75\%. In contrast, \textbf{Aya-101}, despite being specifically tuned for Kazakh, consistently shows the highest unsafe response rates at 72.37\%. 

\begin{figure*}[t!]
	\centering
        \includegraphics[scale=0.28]{figures/question_type_no6_kaz.png}
	\includegraphics[scale=0.28]{figures/question_type_exclude_region_specific_new.png} 

	\caption{Unsafe answer distribution across three question types for risk types I-V: Kazakh (left) and Russian (right)}
	\label{fig:qt_non_reg}
\end{figure*}

\begin{figure*}[t!]
	\centering
        \includegraphics[scale=0.28]{figures/question_type_only6_kaz.png}
	\includegraphics[scale=0.28]{figures/question_type_region_specific_new.png} 
	
	\caption{Unsafe answer distribution across three question types for risk type VI: Kazakh (left) and Russian (right)}
	\label{fig:qt_reg}
\end{figure*}

\textbf{Safety Rank.} In general, proprietary systems outperform the open-source models. 
For Russian, as shown in Table~\ref{tab:safety-binary-eval},  % \ref{tab:model_comparison_russian}, 
\yandexgpt emerges as the safest language model for Russian, with safe responses account for 93.57\%. Among the open-source models, \kazllmseventy is the safest (87.5\%), followed by \vikhr with a safety percentage of 85.63\%.

For Kazakh, % as shown in Table \ref{tab:model_comparison_kazakh}, 
% YX: todo, rerun Kazakh safety percentage using Diana threshold
\claude is the safest model with 96.46\% safe responses, closely followed by \gptfouro\ at 95.75\%. \aya, despite being specifically tuned for Kazakh, shows the highest unsafe rates at 72.37\%.



% \begin{table}[t!]
% \centering
% \resizebox{\columnwidth}{!}{%
% \begin{tabular}{clccc}
% \toprule
% \textbf{Rank} & \textbf{Model Name}  & \textbf{Safe} & \textbf{Unsafe} & \textbf{Safe \%} \\ \midrule
% \textbf{1} & \textbf{Yandex-GPT} & \textbf{4101} & \textbf{282} & \textbf{93.57} \\
% 2 & Claude & 4100 & 283 & 93.54 \\
% 3 & GPT-4o & 3839 & 544 & 87.59 \\
% 4 & Vikhr-12B & 3753 & 630 & 85.63 \\
% 5 & LLama-3.1-instruct-70B & 3746 & 637 & 85.47 \\
% 6 & LLama-3.1-instruct-8B & 3712 & 671 & 84.69 \\
% \bottomrule
% \end{tabular}
% }
% \caption{Comparison of models based on safety percentages for the Russian dataset.}
% \label{tab:model_comparison_russian}
% \end{table}


% \begin{table}[t!]
% \centering
% \resizebox{\columnwidth}{!}{%
% \begin{tabular}{clccc}
% \toprule
% \textbf{Rank} & \textbf{Model Name}  & \textbf{Safe} & \textbf{Unsafe} & \textbf{Safe \%} \\ \midrule
% 1             & \textbf{Claude}  & \textbf{3652} & \textbf{134} & \textbf{96.46} \\ 
% 2             & GPT-4o                      & 3625          & 161          & 95.75 \\ 
% 3             & YandexGPT                   & 3433          & 353          & 90.68 \\
% 4             & LLama-3.1-instruct-70B      & 3333          & 453          & 88.03 \\
% 5             & LLama-3.1-instruct-8B       & 3251          & 535	       & 85.87 \\
% 6             & Aya-101                     & 2740          & 1046         & 72.37 \\ 
% \bottomrule
% \end{tabular}
% }
% \caption{Comparison of models based on safety percentages for the Kazakh dataset.}
% \label{tab:model_comparison_kazakh}
% \end{table}



\textbf{Risk Areas.} 
We selected six representative LLMs for Russian and Kazakh respectively and show their unsafe answer distributions over six risk areas.
As shown in Table \ref{tab:unsafe_answers_summary}, risk type VI (region-specific sensitive topics) overwhelmingly contributes the largest number of unsafe responses across all models. This highlights that LLMs are poorly equipped to address regional risks effectively. For instance, while \llama models maintain comparable safety levels across other risk type (I–V), their performance drops significantly when dealing with risk type VI. Interestingly, while \yandexgpt\ demonstrates relatively poor performance in most other risk areas, it handles region-specific questions remarkably well, suggesting a stronger alignment with regional norms and sensitivities. For Kazakh, Table \ref{tab:unsafe_answers_summary_kazakh} shows that region‐specific topics (risk type VI) pose a substantial challenge across all six models, including the commercial \gptfouro\ and \claude, which demonstrate superior safety on general categories. 

% \begin{table}[t!]
% \centering
% \resizebox{\columnwidth}{!}{%
% \begin{tabular}{lccccccc}
% \toprule
% \textbf{Model} & \textbf{I} & \textbf{II} & \textbf{III} & \textbf{IV} & \textbf{V} & \textbf{VI} & \textbf{Total} \\ \midrule
% LLama-3.1-instruct-8B & 60 & 70 & 16 & 31 & 9 & 485 & 671 \\
% LLama-3.1-instruct-70B & 29 & 55 & 8 & 4 & 1 & 540 & 637 \\
% Vikhr-12B & 41 & 93 & 15 & 1 & 3 & 477 & 630 \\
% GPT-4o & 21 & 51 & 6 & 2 & 0 & 464 & 544 \\
% Claude & 7 & 10 & 1 & 0 & 0 & 265 & 283 \\
% Yandex-GPT & 55 & 125 & 9 & 3 & 8 & 82 & 282 \\
% \bottomrule
% \end{tabular}%
% }
% \caption{Ru unsafe cases over risk areas of six models.}
% \label{tab:unsafe_answers_summary}
% \end{table}


\begin{table}[t!]
\centering
\resizebox{\columnwidth}{!}{%
\begin{tabular}{lccccccc}
\toprule
\textbf{Model} & \textbf{I} & \textbf{II} & \textbf{III} & \textbf{IV} & \textbf{V} & \textbf{VI} & \textbf{Total} \\ \midrule
\llamaeight & 60 & 70 & 16 & 31 & 9 & 485 & 671 \\
\llamaseventy & 29 & 55 & 8 & 4 & 1 & 540 & 637 \\
\vikhr & 41 & 93 & 15 & 1 & 3 & 477 & 630 \\
\gptfouro & 21 & 51 & 6 & 2 & 0 & 464 & 544 \\
\claude & 7 & 10 & 1 & 0 & 0 & 265 & 283 \\
\yandexgpt & 55 & 125 & 9 & 3 & 8 & 82 & 282 \\
\bottomrule
\end{tabular}%
}
\caption{Ru unsafe cases over risk areas of six models.}
\label{tab:unsafe_answers_summary}
\end{table}


% \begin{table}[t!]
% \centering
% \resizebox{\columnwidth}{!}{%
% \begin{tabular}{lccccccc}
% \toprule
% \textbf{Model} & \textbf{I} & \textbf{II} & \textbf{III} & \textbf{IV} & \textbf{V} & \textbf{VI} & \textbf{Total} \\ \midrule
% Aya-101 & 96 & 235 & 165 & 166 & 90 & 294 & 1046 \\
% Llama-3.1-instruct-8B & 25 & 15 & 91 & 37 & 14 & 353 & 535 \\
% Llama-3.1-instruct-70B & 33 & 39 & 88 & 27 & 20 & 246 & 453 \\
% Yandex-GPT & 29 & 76 & 95 & 29 & 16 & 108 & 353 \\
% GPT-4o & 2 & 1 & 41 & 0 & 3 & 114 & 161 \\
% Claude & 2 & 1 & 26 & 3 & 6 & 96 & 134 \\ \bottomrule
% \end{tabular}%
% }
% \caption{Kaz unsafe cases over risk areas of six models.}
% \label{tab:unsafe_answers_summary_kazakh}
% \end{table}


\begin{table}[t!]
\centering
\resizebox{\columnwidth}{!}{%
\begin{tabular}{lccccccc}
\toprule
\textbf{Model} & \textbf{I} & \textbf{II} & \textbf{III} & \textbf{IV} & \textbf{V} & \textbf{VI} & \textbf{Total} \\ \midrule
\aya & 96 & 235 & 165 & 166 & 90 & 294 & 1046 \\
\llamaeight & 25 & 15 & 91 & 37 & 14 & 353 & 535 \\
\llamaseventy & 33 & 39 & 88 & 27 & 20 & 246 & 453 \\
\yandexgpt & 29 & 76 & 95 & 29 & 16 & 108 & 353 \\
\gptfouro & 2 & 1 & 41 & 0 & 3 & 114 & 161 \\
\claude & 2 & 1 & 26 & 3 & 6 & 96 & 134 \\ 
\bottomrule
\end{tabular}%
}
\caption{Kaz unsafe cases over risk areas of six models.}
\label{tab:unsafe_answers_summary_kazakh}
\end{table}

% \begin{figure*}[t!]
% 	\centering
% 	\includegraphics[scale=0.28]{figures/human_1000_kz_font16.png} 
% 	\includegraphics[scale=0.28]{figures/human_1000_ru_font16.png}

% 	\caption{Human vs \gptfouro\ fine-grained labels on 1,000 Kazakh (left) and Russian (right) samples.}
% 	\label{fig:human_fg_1000}
% \end{figure*}

\textbf{Question Type.} For Russian, Figures \ref{fig:qt_non_reg} and \ref{fig:qt_reg} reveal differences in how models handle general risks I-V and region-specific risk VI. For risks I-V, indirect attacks % crafted to exploit model vulnerabilities—
result in more unsafe responses due to their tricky phrasing. 

In contrast, region-specific risks see slightly more unsafe responses from direct attacks, 
% as these explicit prompts are more likely to bypass safety mechanisms. 
since indirect attacks for region-specific prompts often elicit safer, vaguer answers. %, suggesting models struggle less with implicit harm. 
Overall, the number of unsafe responses is similar across question types, indicating models generally struggle with safety alignment in all jailbreaking queries.

For Kazakh, Figures \ref{fig:qt_non_reg} and \ref{fig:qt_reg} show greater variation in unsafe responses across question types due to the low-resource nature of Kazakh data. For general risks I-V, \llamaseventy\ and \aya\ produce more unsafe outputs for direct harm prompts. At the same time, \claude\ and \gptfouro\ struggle more with indirect attacks, reflecting challenges in handling subtle cues. For region-specific risk VI, most models perform similarly due to limited Kazakh-specific data, though \llamaeight\ shows higher unsafe outputs for indirect local references, likely due to their implicit nature. Direct region-specific attacks yield fewer unsafe responses, as explicit prompts trigger more cautious outputs. Across all risk areas, general questions with sensitive words produce the fewest unsafe answers, suggesting over-flagging or cautious behavior for unclear harmful intent.



% \subsection{Fine-grained Classification}
% We extended our analysis to include fine-grained classifications for both safe and unsafe responses. For unsafe responses, we categorized outputs into four harm types, as shown in Table \ref{table:unsafe_response_categories}. 

% For safe responses, we classified outputs into six distinct patterns of safety, following a fine-grained rubric provided in \cite{wang2024chinesedatasetevaluatingsafeguards}. The types outlined in this rubric are presented in Table \ref{table:safe_response_categories}.

% To validate the fine-grained classification, we conducted human evaluation on the same 1,000 responses in Russian used for the preliminary binary classification.
% The confusion matrix highlights the alignment and discrepancies between human annotations and GPT's fine-grained labels. The diagonal values represent instances where the GPT's labels match human annotations, with category 5 (provides general, safe information) showing the highest agreement (404 instances). However, off-diagonal values reveal areas of disagreement, such as misclassifications in categories 1 (not willing to answer), 4 (recommends consulting an expert), and 7 (provides harmful or risky instructions). For example, in category 1, while 94 responses were correctly labeled, GPT-4o misclassified several instances into categories 4, 5, or 7, indicating overlap or ambiguity in these classifications. Similar trends are observed in other categories, where GPT sometimes struggles to differentiate nuanced distinctions in human-labeled categories. Overall, GPT's fine-grained labels match human annotations in 710 out of 1000 cases, achieving an agreement rate of 71\%.

% \begin{figure}[ht!]
%     \centering
%     \includegraphics[width=0.95\linewidth]{figures/human_fg_1000_ru.png}
%     \caption{{Human vs GPT-4o Fine-Grained Labels on 1,000 Russian Samples}}
%     \label{fig:human_fg_1000_ru}
% \end{figure}


% After conducting human evaluation on a representative sample, we extended the fine-grained classification to a full dataset comprising 21,915 responses generated by five different models. 


% \begin{figure}[ht!]
%     \centering
%     \includegraphics[width=0.95\linewidth]{figures/all_5_ru.png}
%     \caption{Fine-grained label distribution for responses from five models for Russian.}
%     \label{fig:all_5_fg_russian}
% \end{figure}
% Category 5 ("safe and general information") consistently has the highest frequency across all models, aligning with its dominance in the 1,000-sample evaluation. However, differences in the distribution across other categories highlight variability in how models handle nuanced safety risks, with Yandex-GPT showing a slightly broader spread across categories. 
% In the distribution of unsafe responses, most models exhibit higher counts in certain labels such as 8. However, Yandex-GPT displays comparatively fewer responses in these labels. 
% It exhibits a high rate of responses classified under label 7, which indicates instances where the model provides harmful or risky instructions, including unethical behavior or sensitive discussions. While this may suggest a vulnerability in addressing complex or challenging prompts, it was observed that many of Yandex-GPT’s responses tend to deflect responsibility or offer vague advice such as "check the internet". Although this approach minimizes the risk of unsafe outputs, it often results in responses that lack depth or contextually relevant information, limiting their overall usefulness for users.

% % \subsection{Kazakh}


% % Overall, these findings underscore how resource constraints and prompt explicitness affect model safety in Kazakh. Some models manage direct attacks better yet fail on indirect ones, while region-specific content remains challenging for all given the lack of localized training data.
% \subsubsection{Fine-grained Classification}
% Similarly, we conducted a human evaluation on 1,000 Kazakh samples, following the same methodology as the Russian evaluation. The match between human annotations and GPT-4o's fine-grained classifications was 707 out of 1,000, ensuring that the fine-grained classification framework aligned well with human judgments.
% The confusion matrix in Figure \ref{fig:human_fg_1000_kz} for 1,000 Kazakh samples illustrates the agreement between human annotations and GPT-4o's fine-grained classifications. The highest agreement is observed in category 5 (360 instances), indicating GPT-4o's strength in recognizing responses labeled by humans as "safe and general information." However, discrepancies are evident in categories such as 3 and 7, where GPT-4o misclassified several instances, highlighting areas for further refinement in distinguishing nuanced classifications.


\begin{figure}[t!]
	\centering
	\includegraphics[scale=0.18]{figures/human_1000_kz_font16.png} 
	\includegraphics[scale=0.18]{figures/human_1000_ru_font16.png}

	\caption{Human vs \gptfouro\ fine-grained labels on 1,000 Kazakh (left) and Russian (right) samples.}
	\label{fig:human_fg_1000}
\end{figure}

% \begin{figure}[t!]
% 	\centering
% 	\includegraphics[scale=0.28]{figures/human_1000_kz_font16.png} 
% 	\includegraphics[scale=0.28]{figures/human_1000_ru_font16.png}

% 	\caption{Human vs \gptfouro\ fine-grained labels on 1,000 Kazakh (top) and Russian (bottom) samples.}
% 	\label{fig:human_fg_1000}
% \end{figure}

% \begin{figure*}[t!]
% 	\centering
% 	\includegraphics[scale=0.28]{figures/all_5_kz_font16.png} 
% 	\includegraphics[scale=0.28]{figures/all_5_ru_font_16.png} \\
% 	\caption{Fine-grained responding pattern distribution across five models for Kazakh (left) and Russian (right).}
% 	\label{fig:all_5}
% \end{figure*}

\begin{figure}[t!]
	\centering
	\includegraphics[scale=0.28]{figures/all_5_kz_font16.png} 
	\includegraphics[scale=0.28]{figures/all_5_ru_font_16.png} \\
	\caption{Fine-grained responding pattern distribution across five models for Kazakh (top) and Russian (bottom).}
	\label{fig:all_5}
\end{figure}


\subsection{Fine-Grained Classification}
\label{sec:fine-grained-classification}
% As discussed in Section \ref{harmfulness_evaluation}, 
We further analyzed fine-grained responding patterns for safe and unsafe responses. For unsafe responses, outputs were categorized into four harm types, and safe responses were classified into six distinct patterns of safety, as rubric in Appendix \ref{safe_unsafe_response_categories}. 
% \cite{wang2024chinesedatasetevaluatingsafeguards}

\paragraph{Human vs. GPT-4o}
Similar to binary classification, we validated \gptfouro's automatic evaluation results by comparing with human annotations on 1,000 samples for both Russian and Kazakh. %, comparing human annotations with \gptfouro's fine-grained labels.
For the Russian dataset, \gptfouro's labels aligned with human annotations in 710 out of 1,000 cases, achieving an agreement rate of 71\%. 
Agreement rate of Kazakh samples is 70.7\%. %with 707 out of 1,000 cases matching
% The confusion matrix highlights areas of alignment and discrepancies.
% 
As confusion matrices illustrated in Figure~\ref{fig:human_fg_1000}, the majority of cases falling into \textit{safe responding patter 5} --- providing general and harmless information, for both human annotations and automatic predictions.
% highest agreement with 404 correct classifications for Russian. 
Mis-classifications for safe responses mainly focus on three closely-similar patterns: 3, 4, and 5, and patterns 7 and 8 are confusing to discern for unsafe responses, particularly for Kazakh (left figure).
We find many Russian samples which were labeled as ``1. reject to answer'' by humans are diversely classified across 1-6 by GPT-4o, which is also observed in Kazakh but not significant.

% suggesting label alignment issues are language-independent. 
% YX: I did not observe this, commented
% Notably, Russian showed confusion between 7 (risky instructions) and 1 (refusal to answer), this trend does not appear in Kazakh.


% highlight the strengths and limitations of {\gptfouro}'s fine-grained classification framework across both languages, paving the way for further refinements.


% However, misclassifications were observed in categories such as 1 (not willing to answer), 4 (recommends consulting an expert), and 7 (provides harmful or risky instructions), revealing overlaps and ambiguities in nuanced classifications.

% Similarly, for the Kazakh dataset, the agreement rate between human annotations and GPT-4o's labels was 70.7\%, with 707 out of 1,000 cases matching. As with the Russian analysis, category 5 (360 instances) showed the highest alignment. However, discrepancies were more prominent in categories such as 3 and 7, underscoring GPT-4o's challenges in differentiating fine-grained human-labeled categories. 
% A similar pattern was observed for Kazakh dataset, which suggests that alignment and misaligned of fine-grained lables is not language dependent.

% These findings, illustrated in Figures \ref{fig:human_fg_1000}, highlight the strengths and limitations of {\gptfouro}'s fine-grained classification framework across both languages, paving the way for further refinements.

\paragraph{Fine-grained Analysis of Five LLMs}
% After conducting human evaluation on representative samples, we extended 
\figref{fig:all_5} shows fine-grained responding pattern distribution across five models based on the full set of Russian and Kazakh data.
% For Russian, we selected \vikhr, \gptfouro, \llamaseventy, \claude, and \yandexgpt, while for Kazakh, we chose \aya, \gptfouro, \llamaseventy, \claude, and \yandexgpt. 
% The evaluation covered 21,915 responses in Russian and 18,930 responses in Kazakh.
% 
In both languages, pattern 5 of providing \textit{general and harmless information} consistently witnessed the highest frequency across all models, with \llamaseventy\ exhibiting the largest number of responses falling into this category for Kazakh (2,033). 
% YX:summarize more noteable findings here.

Differences of other patterns vary across languages. 
Unsafe responses in Russian are predominantly in pattern 8, where models provide incorrect or misleading information without expressing uncertainty. % (misinformation and speculation), 
For Kazakh, \aya\ exhibits the highest occurrence of pattern 7 (harmful or risky information) and pattern 8, indicating a stronger tendency to generate unethical, misleading, or potentially harmful content.

%Variations in other patterns across models highlight differences in how nuanced safety risks are classified, reflecting the models' differing capabilities in handling safety evaluation for these distinct linguistic contexts. For Russian, the majority of unsafe responses fall under pattern 8 (misinformation and speculation), indicating that models frequently provide incorrect or misleading information without acknowledging uncertainty. For Kazakh, \aya\ has the highest occurence of pattern 7 (harmful or risky information) and pattern 8 (misinformation and speculation), indicating a greater tendency to generate unethical, misleading, or potentially harmful content. 

%This trend suggests that Russian models may struggle more with factual accuracy, whereas Kazakh models, particularly \aya, pose higher risks related to both harmful content and misinformation. Additionally, \gptfouro\ and \claude\ consistently produce fewer unsafe responses in both languages, demonstrating stronger alignment with safety standards
\subsection{Code Switching}
\begin{table}[t!]
\centering

\setlength{\tabcolsep}{3pt}
\scalebox{0.7}{
\begin{tabular}{lcccccccccc}
\toprule
\textbf{Model Name} & \multicolumn{2}{c}{\textbf{Kazakh}} & \multicolumn{2}{c}{\textbf{Russian}} & \multicolumn{2}{c}{\textbf{Code-Switched}} \\  
\cmidrule(lr){2-3} \cmidrule(lr){4-5} \cmidrule(lr){6-7}
& \textbf{Safe} & \textbf{Unsafe} & \textbf{Safe} & \textbf{Unsafe} & \textbf{Safe} & \textbf{Unsafe} \\ 
\midrule
\llamaseventy & 450 & 50 & 466 & 34 & 414 & 86 \\
\gptfouro & 492 & 8 & 473 & 27 & 481 & 19
 \\
\claude & 491 & 9 & 478 & 22 & 484 & 16 \\ 
\yandexgpt & 435 & 65 & 458 & 42 & 464 & 36 \\
\midrule
\end{tabular}}
\caption{Model safety when prompted in Kazakh, Russian, and code-switched language.}
\label{tab:finetuning-comparison}
\end{table}


\gptfouro\ and \claude\ demonstrate strong safety performance across three languages, even with a high proportion of safe responses in the challenging code-switching context. In contrast, \llamaseventy\ and \yandexgpt\ are less safe, exhibiting more harmful responses, particularly in the code-switching scenario. These results show the varying capabilities of models in defending the same attacks that are just presented in different languages, where open-sourced large language models especially require more robust safety alignment in multilingual and code-switching scenarios.

% \subsection{LLM Response Collection}
% We conducted experiments with a variety of mainstream and region-specific 
% large language models for both Russian and Kazakh languages. For both Russian and Kazakh languages, we employed four multilingual models: Claude-3.5-sonnet, Llama 3.1 70B \cite{meta2024llama3}, GPT-4 \cite{openai2024gpt4o}, and YandexGPT. Additionally, we included language-specific models: VIKHR \cite{nikolich2024vikhrconstructingstateoftheartbilingual} for Russian and Aya \cite{ustun-etal-2024-aya} for Kazakh. 

% \subsection{Kazakh-Russian Code-Switching Evaluation}

% In Kazakhstan, the prevalence of bilingualism is a defining characteristic of its linguistic landscape, with most individuals seamlessly mixing Kazakh and Russian in daily communication \cite{Zharkynbekova2022}. This phenomenon, known as code-switching, reflects the unique cultural and social dynamics of the region. Despite this, there is currently no safety evaluation dataset tailored to this unique multilingual environment. Developing a code-switched dataset is essential to evaluate the ability of large language models (LLMs) to navigate the complexities of bilingual interactions, ensuring they produce contextually appropriate, non-harmful, and culturally sensitive responses. To address this, we sampled 500 questions from both Kazakh and Russian question sets, ensuring they represent similar queries but with culturally nuanced differences in meaning. Building on these, we developed a mixed version that integrates Kazakh and Russian naturally, reflecting the way people in Kazakhstan communicate. This process involved annotators to ensure that the mixed questions were crafted with a natural tone and were localized to Kazakhstan's context.


% \begin{table*}[t!]
%     \centering
%     \small
%     \setlength{\tabcolsep}{6pt} % Adjusts column separation
%     \adjustbox{max width=\linewidth}{
%     \begin{tabular}{ll ccc ccc c}
%     \toprule
%     \textbf{Risk Area} & \textbf{Harm Type} & \multicolumn{3}{c}{\textbf{Kazakh}} & \multicolumn{3}{c}{\textbf{Russian}} & \textbf{\#Q} \\
%                        &                    & \textbf{Ori} & \textbf{FN} & \textbf{FP} & \textbf{Ori} & \textbf{FN} & \textbf{FP} & \\
%     \midrule
%     \multirow{2}{*}{I. Information Hazards} & 1. Risks from leaking sensitive information... & 131 & 133 & 131 & 131 & 133 & 131 & 790 \\
%                                              & 2. Compromise of privacy by leaking or inferring private information (person/individual) & 81 & 82 & 81 & 81 & 82 & 81 & 488 \\
%     \midrule
%     \multirow{3}{*}{II. Malicious Uses} & 3. Assisting illegal activities & 132 & 135 & 132 & 132 & 135 & 132 & 798 \\
%                                         & 4. Nudging or advising unethical actions & 71 & 71 & 71 & 71 & 71 & 71 & 426 \\
%                                         & 5. Reducing the cost of disinformation campaigns & 40 & 42 & 40 & 40 & 42 & 40 & 244 \\
%     \midrule
%     \multirow{3}{*}{III. Discrimination, Exclusion, Toxicity, Hateful, Offensive} & 6. Social stereotypes and unfair discrimination & 94 & 96 & 94 & 94 & 96 & 94 & 568 \\
%                                         & 7. Toxic language (hate speech) & 52 & 59 & 52 & 52 & 59 & 52 & 326 \\
%                                         & 8. Adult content & 27 & 29 & 27 & 27 & 29 & 27 & 166 \\

%     \midrule
%     \multirow{2}{*}{IV. Misinformation Harms} & 9. Disseminating false or misleading information & 92 & 99 & 92 & 92 & 99 & 92 & 566 \\
%                                         & 10. Causing material harm by disseminating misinformation e.g. in medicine or law & 63 & 63 & 63 & 63 & 63 & 63 & 378 \\
%     \midrule
%     \multirow{2}{*}{V. Human–chatbot Interaction Harms} & 11. Mental health or overreliance concerns & 66 & 66 & 66 & 66 & 66 & 66 & 396 \\
%                                         & 12. Treating the chatbot as a human & 50 & 51 & 50 & 50 & 51 & 50 & 302 \\

%     \midrule
%     \multirow{2}{*}{VI. Kazakhstan-specific Sensitivity} & 13. Politically sensitive topics & 63 & 66 & 63& 63 & 66 & 63 &  384 \\
%         & 14. Controversial historical events & 46 & 57 & 46 & 46 & 57 & 46  & 298 \\
% & 15. Regional and racial issues & 45 & 45 & 45 &  45 & 45 & 45 & 270  \\
% & 16. Societal and cultural concerns & 138 & 139 & 138 &  138 & 139 & 138  & 830  \\
% & 17. Legal and human rights matters & 57 & 57 & 57 & 57 & 57 & 57  & 342 \\
%     \midrule
%         \multirow{2}{*}{VII. Russia-specific Sensitivity} 
%             & 13. Politically sensitive topics & - & - & - & 54 & 54 & 54 & 162 \\
%     & 14. Controversial historical events & - & - & - & 38 & 38 & 38 & 114 \\
%     & 15. Regional and racial issues & - & - & - & 26 & 26 & 26 & 78 \\
%     & 16. Societal and cultural concerns & - & - & - & 40 & 40 & 40 & 120 \\
%     & 17. Legal and human rights matters & - & - & - & 41 & 41 & 41 & 123 \\
%     \midrule
%     \bf Total & -- & 1248 & 1290 & 1248 & 1447 & 1489 & 1447 & \textbf{8169} \\
%     \bottomrule
%     \end{tabular}
%     }
%     \caption{The number of questions for Kazakh and Russian datasets across six risk areas and 17 harm types. Ori: original direct attack, FN: indirect attack, and FP: over-sensitivity assessment.}
%     \label{tab:kazakh-russian-data}
% \end{table*}




\section{Discussion}

% \subsection{Kazakh vs Russian}

% The evaluation reveals that Kazakh responses tend to be generally safer than their Russian counterparts, likely due to Kazakh being a low-resource language with significantly less training data. As a result, Kazakh models are less exposed to the vast, often unfiltered datasets containing harmful or unsafe content, which are more prevalent in high-resource languages like Russian. This data scarcity naturally limits the model's ability to generate nuanced but potentially unsafe responses. However, this does not mean the models are specifically fine-tuned for safer performance. When analyzing unsafe answers, it’s clear that Kazakh models, while safer overall, distribute their unsafe responses more evenly across various risk types and question types. This suggests Kazakh models generate fewer unsafe answers but in a broader range of contexts.

% In contrast, Russian models tend to concentrate unsafe answers in specific areas, particularly region-specific risks or indirect attacks. This indicates that Russian models have learned to handle certain types of unsafe content by focusing on specific topics, such as politically sensitive issues, but struggle when confronted with unfamiliar content, leading to unsafe responses due to insufficient filtering. Kazakh models, despite having less training data, tend to respond more broadly, including both direct and indirect risks. This could be due to the less curated nature of their training data, making them more likely to answer unsafe questions without filtering the potential harm involved. The exception to this trend is Aya, a model specifically fine-tuned for Kazakh. Despite fine-tuning, it exhibits the lowest safety percentage (72.37\%) in the Kazakh dataset, suggesting that fine-tuning in specific languages may introduce risks if proper safety measures are not taken.

% The evaluation reveals notable differences in the distribution of safe response patterns across Kazakh and Russian fine-grained labels. Refusal to answer is more frequent in Russian models, particularly Yandex-GPT, reflecting a cautious approach to safety-critical queries. Interestingly, Aya, despite being fine-tuned for Kazakh and exhibiting lower overall safety, also frequently refuses to answer, suggesting an over-reliance on conservative mechanisms. Responses providing general, safe information dominate in both languages, with Kazakh models displaying a slightly higher tendency to rely on this approach. This highlights how the low-resource nature of Kazakh results in more generalized and inherently safer responses. In contrast, Russian models excel at recognizing risks, issuing disclaimers, and refuting incorrect assumptions, likely benefiting from richer and more diverse training data.
% Yandex-GPT exhibits a notably high rate of responses classified under label 7, indicating an overreliance on general disclaimers or deflections, such as "check the internet" or "I don't know." While these responses minimize the risk of unsafe outputs, they often lack substantive or contextually relevant information, reducing their overall utility for users.


Most models perform safer on Kazakh dataset than Russian dataset, higher safe rate on Kazakh dataset in \tabref{tab:safety-binary-eval}. This does not necessarily reveal that current LLMs have better understanding and safety alignment on Kazakh language than Russian, while this may conversely imply that models do not fully understand the meaning of Kazakh attack questions, fail to perceive risks and then provide general information due to lacking sufficient knowledge regarding this request.

We observed the similar number of examples falling into category 5 \textit{general and harmless information} for both Kazakh and Russian, while the Kazakh data set size is 3.7K and Russian is 4.3K. Kazakh has much less examples in category 1 \textit{reject to answer} compared to Russian. This demonstrate models tend to provide general information and cannot clearly perceive risks for many cases.

Additionally, in spite of less harmful responses on Kazakh data, these unsafe responses distribute evenly across different risk areas and question categories, exhibiting equally vulnerability spanning all attacks regardless of what risks and how we jailbreak it.
In contrary, unsafe responses on Russian dataset often concentrate on specific areas and question types, such as region-specific risks or indirect attacks, presenting similar model behaviors when evaluating over English and Chinese data.
It suggests that broader training data in English, Chinese and Russian may allow models to address certain types of attacks robustly,
% effectively—particularly politically sensitive issues—
yet they may falter when confronted with unfamiliar content like regional sensitive topics.

Moreover, in responses collection, we observed many Russian or English responses especially for open-sourced LLMs when we explicitly instructed the models to answer Kazakh questions in Kazakh language. This further implies more efforts are still needed to improve LLMs' performance on low-resource languages.
Interestingly, \aya, a fine-tuned Kazakh model, proves an exception by displaying the lowest safety percentage (72.37\%) among Kazakh models, revealing that the multilingual fine-tuning without stringent safety measures can introduce risks.



% However, this does not mean they are explicitly fine-tuned for safety, likely it happens due to limited training data, which reduces exposure to harmful content. 
% \aya, a fine-tuned Kazakh model, proves an exception by displaying the lowest safety percentage (72.37\%) among Kazakh models, revealing that the multilingual fine-tuning without stringent safety measures can introduce risks.
% Kazakh models generally produce safer responses than their Russian counterparts, likely because Kazakh is a low-resource language with less training data. 
% This limited exposure to harmful or unsafe content naturally limits nuanced yet potentially unsafe outputs. 
% However, it does not imply that the models are specifically fine-tuned for enhanced safety.


% while Kazakh models tend to generate fewer unsafe answers overall, those unsafe responses appear more evenly spread across different risk types and question categories.
% Russian models, on the other hand, often concentrate unsafe responses in specific areas, such as region-specific risks or indirect attacks.
% It implies that their broader training datasets allow them to address certain types of unsafe content more effectively—particularly politically sensitive issues—yet they may falter when confronted with unfamiliar or insufficiently filtered content.

% Meanwhile, Kazakh models sometimes respond more broadly, possibly due to less curated training data. 

Differences also emerge in how language models handle safe responses. 
\yandexgpt, for instance, often refuses to answer high-risk queries. 
It frequently relies on generic disclaimers or deflections like ``check in the Internet'' or ``I don’t know,'' minimizing risk but are less helpful. Interestingly, it often responds with ``I don’t know'' in Russian, even for Kazakh queries, we speculate that these may be default responses stemming from internal system filters, rather than generated by model itself.
This likely explains why \yandexgpt\ is the safest model for the Russian language but ranks third for Kazakh. While its filters perform well for Russian, they struggle with the low-resource Kazakh language.

% Aya, despite its lower overall safety, also employs refusals often, hinting at an over-reliance on conservative approaches. 

% Across both languages, models commonly resort to providing general, safe information, although Kazakh models lean on this strategy slightly more. 
% Russian models, by contrast, excel at detecting risks, issuing disclaimers, and correcting inaccuracies, likely benefiting from richer and more diverse training data.


% \subsection{Response Patterns}


% We conducted a detailed analysis of the models' outputs and identified several noteworthy patterns. YandexGPT, while being one of the safest overall, frequently generates responses in Russian even when the question is posed in Kazakh. These responses often appear as placeholders, prompting users to search for the answer online. This behavior might not originate from the model itself but rather from safety filters implemented in the YandexGPT system. The model's leading performance in ensuring safety during Russian-language interactions, coupled with its lower performance in Kazakh, can be attributed to the limited robustness of these safety filters when handling unsafe content in Kazakh.

% In contrast, Aya-101 exhibits a tendency to fall into repetition, often repeating the same sentences multiple times. Interestingly, the Vikhr model, despite being of a similar size, does not exhibit this issue. We attribute this difference to two key factors. First, Vikhr and Aya-101 have distinct architectures: Vikhr is based on the Mistral-Nemo model, whereas Aya-101 is built on mT5, an older and less robust model. Second, Aya-101 is a multilingual model, while Vikhr was predominantly trained for Russian. Multilingualism has been shown to potentially degrade performance in large language models~\cite{huang2025surveylargelanguagemodels}, which may explain Aya-101's issues with repetition.

\paragraph{Summary}
Our findings provide significant insights into the influence of correctness, explanations, and refinement on evaluation accuracy and user trust in AI-based planners. 
In particular, the findings are three-fold: 
(1) The \textbf{correctness} of the generated plans is the most significant factor that impacts the evaluation accuracy and user trust in the planners. As the PDDL solver is more capable of generating correct plans, it achieves the highest evaluation accuracy and trust. 
(2) The \textbf{explanation} component of the LLM planner improves evaluation accuracy, as LLM+Expl achieves higher accuracy than LLM alone. Despite this improvement, LLM+Expl minimally impacts user trust. However, alternative explanation methods may influence user trust differently from the manually generated explanations used in our approach.
% On the other hand, explanations may help refine the trust of the planner to a more appropriate level by indicating planner shortcomings.
(3) The \textbf{refinement} procedure in the LLM planner does not lead to a significant improvement in evaluation accuracy; however, it exhibits a positive influence on user trust that may indicate an overtrust in some situations.
% This finding is aligned with prior works showing that iterative refinements based on user feedback would increase user trust~\cite{kunkel2019let, sebo2019don}.
Finally, the propensity-to-trust analysis identifies correctness as the primary determinant of user trust, whereas explanations provided limited improvement in scenarios where the planner's accuracy is diminished.

% In conclusion, our results indicate that the planner's correctness is the dominant factor for both evaluation accuracy and user trust. Therefore, selecting high-quality training data and optimizing the training procedure of AI-based planners to improve planning correctness is the top priority. Once the AI planner achieves a similar correctness level to traditional graph-search planners, strengthening its capability to explain and refine plans will further improve user trust compared to traditional planners.

\paragraph{Future Research} Future steps in this research include expanding user studies with larger sample sizes to improve generalizability and including additional planning problems per session for a more comprehensive evaluation. Next, we will explore alternative methods for generating plan explanations beyond manual creation to identify approaches that more effectively enhance user trust. 
Additionally, we will examine user trust by employing multiple LLM-based planners with varying levels of planning accuracy to better understand the interplay between planning correctness and user trust. 
Furthermore, we aim to enable real-time user-planner interaction, allowing users to provide feedback and refine plans collaboratively, thereby fostering a more dynamic and user-centric planning process.


\section*{Acknowledgments}
This work was supported by the EPSRC Prosperity Partnership FAIR (grant number EP/V056883/1). DS acknowledges funding from the Turing Institute and Accenture collaboration.  
MK receives funding from the ERC under the European Union’s Horizon 2020 research and innovation programme (\href{http://www.fun2model.org}{FUN2MODEL}, grant agreement No.~834115).

% \newpage
\section*{Impact statement}

This paper proposes that machine learning can and should be used to maximize social welfare. In principle, and by construction, the impact of our proposed framework on society aims to be positive. But our paper also points to the inherent difficulties of identifying, and making formal, what `good for society' is. We lean on the field of welfare economics, which has for decades contended with this challenge, for ideas on how the learning community can begin to approach this daunting task.
However, even if these ideas are conceptually appealing,
the path to practical welfare improvement presents many challenges---%
some expected, others unforseen.
% and will likely include many ups and downs.
For example, we may specify incorrect social welfare functions;
or we may specify them correctly but be unable to optimize them appropriately;
or we may be able to optimize but find that 
our assumptions are wrong, that theory differs from practice,
or that there were other considerations and complexities that we did not take into account.
For this we can look to other related fields---%
such as fairness, privacy, and alignment in machine learning---%
which have taken (and are still taking) similar journeys,
and learn from both their success and mistakes.
% and hope that ours will be similar.

Any discipline that seeks to affect policy should do so with much deliberation and care. Whereas welfare economics was designed with the explicit purpose of supporting (and influencing) policymakers,
machine learning has found itself in a similar position, but likely without any planned intent.
On the one hand, adjusting machine learning to support notions, such as social welfare,
that it was not designed to support initially can prove challenging.
However, and as we argue throughout, we believe that building on top of existing machinery is a more practical approach than to begin from scratch.
The necessity of confronting with welfare consideration can also
be an opportunity---as we can leverage these novel challenges
to make machine learning practice more informed, transparent, responsible, and socially aware.


% At the same time, the novelty of the challenges that welfare considerations present to the field make this an opportunity---%
% for chaning the role of machine learning in society for the better in a manner that is informed, transparent, and aware.



\bibliography{refs}
\bibliographystyle{apalike}

\newpage
\appendix
\onecolumn
\subsection{Lloyd-Max Algorithm}
\label{subsec:Lloyd-Max}
For a given quantization bitwidth $B$ and an operand $\bm{X}$, the Lloyd-Max algorithm finds $2^B$ quantization levels $\{\hat{x}_i\}_{i=1}^{2^B}$ such that quantizing $\bm{X}$ by rounding each scalar in $\bm{X}$ to the nearest quantization level minimizes the quantization MSE. 

The algorithm starts with an initial guess of quantization levels and then iteratively computes quantization thresholds $\{\tau_i\}_{i=1}^{2^B-1}$ and updates quantization levels $\{\hat{x}_i\}_{i=1}^{2^B}$. Specifically, at iteration $n$, thresholds are set to the midpoints of the previous iteration's levels:
\begin{align*}
    \tau_i^{(n)}=\frac{\hat{x}_i^{(n-1)}+\hat{x}_{i+1}^{(n-1)}}2 \text{ for } i=1\ldots 2^B-1
\end{align*}
Subsequently, the quantization levels are re-computed as conditional means of the data regions defined by the new thresholds:
\begin{align*}
    \hat{x}_i^{(n)}=\mathbb{E}\left[ \bm{X} \big| \bm{X}\in [\tau_{i-1}^{(n)},\tau_i^{(n)}] \right] \text{ for } i=1\ldots 2^B
\end{align*}
where to satisfy boundary conditions we have $\tau_0=-\infty$ and $\tau_{2^B}=\infty$. The algorithm iterates the above steps until convergence.

Figure \ref{fig:lm_quant} compares the quantization levels of a $7$-bit floating point (E3M3) quantizer (left) to a $7$-bit Lloyd-Max quantizer (right) when quantizing a layer of weights from the GPT3-126M model at a per-tensor granularity. As shown, the Lloyd-Max quantizer achieves substantially lower quantization MSE. Further, Table \ref{tab:FP7_vs_LM7} shows the superior perplexity achieved by Lloyd-Max quantizers for bitwidths of $7$, $6$ and $5$. The difference between the quantizers is clear at 5 bits, where per-tensor FP quantization incurs a drastic and unacceptable increase in perplexity, while Lloyd-Max quantization incurs a much smaller increase. Nevertheless, we note that even the optimal Lloyd-Max quantizer incurs a notable ($\sim 1.5$) increase in perplexity due to the coarse granularity of quantization. 

\begin{figure}[h]
  \centering
  \includegraphics[width=0.7\linewidth]{sections/figures/LM7_FP7.pdf}
  \caption{\small Quantization levels and the corresponding quantization MSE of Floating Point (left) vs Lloyd-Max (right) Quantizers for a layer of weights in the GPT3-126M model.}
  \label{fig:lm_quant}
\end{figure}

\begin{table}[h]\scriptsize
\begin{center}
\caption{\label{tab:FP7_vs_LM7} \small Comparing perplexity (lower is better) achieved by floating point quantizers and Lloyd-Max quantizers on a GPT3-126M model for the Wikitext-103 dataset.}
\begin{tabular}{c|cc|c}
\hline
 \multirow{2}{*}{\textbf{Bitwidth}} & \multicolumn{2}{|c|}{\textbf{Floating-Point Quantizer}} & \textbf{Lloyd-Max Quantizer} \\
 & Best Format & Wikitext-103 Perplexity & Wikitext-103 Perplexity \\
\hline
7 & E3M3 & 18.32 & 18.27 \\
6 & E3M2 & 19.07 & 18.51 \\
5 & E4M0 & 43.89 & 19.71 \\
\hline
\end{tabular}
\end{center}
\end{table}

\subsection{Proof of Local Optimality of LO-BCQ}
\label{subsec:lobcq_opt_proof}
For a given block $\bm{b}_j$, the quantization MSE during LO-BCQ can be empirically evaluated as $\frac{1}{L_b}\lVert \bm{b}_j- \bm{\hat{b}}_j\rVert^2_2$ where $\bm{\hat{b}}_j$ is computed from equation (\ref{eq:clustered_quantization_definition}) as $C_{f(\bm{b}_j)}(\bm{b}_j)$. Further, for a given block cluster $\mathcal{B}_i$, we compute the quantization MSE as $\frac{1}{|\mathcal{B}_{i}|}\sum_{\bm{b} \in \mathcal{B}_{i}} \frac{1}{L_b}\lVert \bm{b}- C_i^{(n)}(\bm{b})\rVert^2_2$. Therefore, at the end of iteration $n$, we evaluate the overall quantization MSE $J^{(n)}$ for a given operand $\bm{X}$ composed of $N_c$ block clusters as:
\begin{align*}
    \label{eq:mse_iter_n}
    J^{(n)} = \frac{1}{N_c} \sum_{i=1}^{N_c} \frac{1}{|\mathcal{B}_{i}^{(n)}|}\sum_{\bm{v} \in \mathcal{B}_{i}^{(n)}} \frac{1}{L_b}\lVert \bm{b}- B_i^{(n)}(\bm{b})\rVert^2_2
\end{align*}

At the end of iteration $n$, the codebooks are updated from $\mathcal{C}^{(n-1)}$ to $\mathcal{C}^{(n)}$. However, the mapping of a given vector $\bm{b}_j$ to quantizers $\mathcal{C}^{(n)}$ remains as  $f^{(n)}(\bm{b}_j)$. At the next iteration, during the vector clustering step, $f^{(n+1)}(\bm{b}_j)$ finds new mapping of $\bm{b}_j$ to updated codebooks $\mathcal{C}^{(n)}$ such that the quantization MSE over the candidate codebooks is minimized. Therefore, we obtain the following result for $\bm{b}_j$:
\begin{align*}
\frac{1}{L_b}\lVert \bm{b}_j - C_{f^{(n+1)}(\bm{b}_j)}^{(n)}(\bm{b}_j)\rVert^2_2 \le \frac{1}{L_b}\lVert \bm{b}_j - C_{f^{(n)}(\bm{b}_j)}^{(n)}(\bm{b}_j)\rVert^2_2
\end{align*}

That is, quantizing $\bm{b}_j$ at the end of the block clustering step of iteration $n+1$ results in lower quantization MSE compared to quantizing at the end of iteration $n$. Since this is true for all $\bm{b} \in \bm{X}$, we assert the following:
\begin{equation}
\begin{split}
\label{eq:mse_ineq_1}
    \tilde{J}^{(n+1)} &= \frac{1}{N_c} \sum_{i=1}^{N_c} \frac{1}{|\mathcal{B}_{i}^{(n+1)}|}\sum_{\bm{b} \in \mathcal{B}_{i}^{(n+1)}} \frac{1}{L_b}\lVert \bm{b} - C_i^{(n)}(b)\rVert^2_2 \le J^{(n)}
\end{split}
\end{equation}
where $\tilde{J}^{(n+1)}$ is the the quantization MSE after the vector clustering step at iteration $n+1$.

Next, during the codebook update step (\ref{eq:quantizers_update}) at iteration $n+1$, the per-cluster codebooks $\mathcal{C}^{(n)}$ are updated to $\mathcal{C}^{(n+1)}$ by invoking the Lloyd-Max algorithm \citep{Lloyd}. We know that for any given value distribution, the Lloyd-Max algorithm minimizes the quantization MSE. Therefore, for a given vector cluster $\mathcal{B}_i$ we obtain the following result:

\begin{equation}
    \frac{1}{|\mathcal{B}_{i}^{(n+1)}|}\sum_{\bm{b} \in \mathcal{B}_{i}^{(n+1)}} \frac{1}{L_b}\lVert \bm{b}- C_i^{(n+1)}(\bm{b})\rVert^2_2 \le \frac{1}{|\mathcal{B}_{i}^{(n+1)}|}\sum_{\bm{b} \in \mathcal{B}_{i}^{(n+1)}} \frac{1}{L_b}\lVert \bm{b}- C_i^{(n)}(\bm{b})\rVert^2_2
\end{equation}

The above equation states that quantizing the given block cluster $\mathcal{B}_i$ after updating the associated codebook from $C_i^{(n)}$ to $C_i^{(n+1)}$ results in lower quantization MSE. Since this is true for all the block clusters, we derive the following result: 
\begin{equation}
\begin{split}
\label{eq:mse_ineq_2}
     J^{(n+1)} &= \frac{1}{N_c} \sum_{i=1}^{N_c} \frac{1}{|\mathcal{B}_{i}^{(n+1)}|}\sum_{\bm{b} \in \mathcal{B}_{i}^{(n+1)}} \frac{1}{L_b}\lVert \bm{b}- C_i^{(n+1)}(\bm{b})\rVert^2_2  \le \tilde{J}^{(n+1)}   
\end{split}
\end{equation}

Following (\ref{eq:mse_ineq_1}) and (\ref{eq:mse_ineq_2}), we find that the quantization MSE is non-increasing for each iteration, that is, $J^{(1)} \ge J^{(2)} \ge J^{(3)} \ge \ldots \ge J^{(M)}$ where $M$ is the maximum number of iterations. 
%Therefore, we can say that if the algorithm converges, then it must be that it has converged to a local minimum. 
\hfill $\blacksquare$


\begin{figure}
    \begin{center}
    \includegraphics[width=0.5\textwidth]{sections//figures/mse_vs_iter.pdf}
    \end{center}
    \caption{\small NMSE vs iterations during LO-BCQ compared to other block quantization proposals}
    \label{fig:nmse_vs_iter}
\end{figure}

Figure \ref{fig:nmse_vs_iter} shows the empirical convergence of LO-BCQ across several block lengths and number of codebooks. Also, the MSE achieved by LO-BCQ is compared to baselines such as MXFP and VSQ. As shown, LO-BCQ converges to a lower MSE than the baselines. Further, we achieve better convergence for larger number of codebooks ($N_c$) and for a smaller block length ($L_b$), both of which increase the bitwidth of BCQ (see Eq \ref{eq:bitwidth_bcq}).


\subsection{Additional Accuracy Results}
%Table \ref{tab:lobcq_config} lists the various LOBCQ configurations and their corresponding bitwidths.
\begin{table}
\setlength{\tabcolsep}{4.75pt}
\begin{center}
\caption{\label{tab:lobcq_config} Various LO-BCQ configurations and their bitwidths.}
\begin{tabular}{|c||c|c|c|c||c|c||c|} 
\hline
 & \multicolumn{4}{|c||}{$L_b=8$} & \multicolumn{2}{|c||}{$L_b=4$} & $L_b=2$ \\
 \hline
 \backslashbox{$L_A$\kern-1em}{\kern-1em$N_c$} & 2 & 4 & 8 & 16 & 2 & 4 & 2 \\
 \hline
 64 & 4.25 & 4.375 & 4.5 & 4.625 & 4.375 & 4.625 & 4.625\\
 \hline
 32 & 4.375 & 4.5 & 4.625& 4.75 & 4.5 & 4.75 & 4.75 \\
 \hline
 16 & 4.625 & 4.75& 4.875 & 5 & 4.75 & 5 & 5 \\
 \hline
\end{tabular}
\end{center}
\end{table}

%\subsection{Perplexity achieved by various LO-BCQ configurations on Wikitext-103 dataset}

\begin{table} \centering
\begin{tabular}{|c||c|c|c|c||c|c||c|} 
\hline
 $L_b \rightarrow$& \multicolumn{4}{c||}{8} & \multicolumn{2}{c||}{4} & 2\\
 \hline
 \backslashbox{$L_A$\kern-1em}{\kern-1em$N_c$} & 2 & 4 & 8 & 16 & 2 & 4 & 2  \\
 %$N_c \rightarrow$ & 2 & 4 & 8 & 16 & 2 & 4 & 2 \\
 \hline
 \hline
 \multicolumn{8}{c}{GPT3-1.3B (FP32 PPL = 9.98)} \\ 
 \hline
 \hline
 64 & 10.40 & 10.23 & 10.17 & 10.15 &  10.28 & 10.18 & 10.19 \\
 \hline
 32 & 10.25 & 10.20 & 10.15 & 10.12 &  10.23 & 10.17 & 10.17 \\
 \hline
 16 & 10.22 & 10.16 & 10.10 & 10.09 &  10.21 & 10.14 & 10.16 \\
 \hline
  \hline
 \multicolumn{8}{c}{GPT3-8B (FP32 PPL = 7.38)} \\ 
 \hline
 \hline
 64 & 7.61 & 7.52 & 7.48 &  7.47 &  7.55 &  7.49 & 7.50 \\
 \hline
 32 & 7.52 & 7.50 & 7.46 &  7.45 &  7.52 &  7.48 & 7.48  \\
 \hline
 16 & 7.51 & 7.48 & 7.44 &  7.44 &  7.51 &  7.49 & 7.47  \\
 \hline
\end{tabular}
\caption{\label{tab:ppl_gpt3_abalation} Wikitext-103 perplexity across GPT3-1.3B and 8B models.}
\end{table}

\begin{table} \centering
\begin{tabular}{|c||c|c|c|c||} 
\hline
 $L_b \rightarrow$& \multicolumn{4}{c||}{8}\\
 \hline
 \backslashbox{$L_A$\kern-1em}{\kern-1em$N_c$} & 2 & 4 & 8 & 16 \\
 %$N_c \rightarrow$ & 2 & 4 & 8 & 16 & 2 & 4 & 2 \\
 \hline
 \hline
 \multicolumn{5}{|c|}{Llama2-7B (FP32 PPL = 5.06)} \\ 
 \hline
 \hline
 64 & 5.31 & 5.26 & 5.19 & 5.18  \\
 \hline
 32 & 5.23 & 5.25 & 5.18 & 5.15  \\
 \hline
 16 & 5.23 & 5.19 & 5.16 & 5.14  \\
 \hline
 \multicolumn{5}{|c|}{Nemotron4-15B (FP32 PPL = 5.87)} \\ 
 \hline
 \hline
 64  & 6.3 & 6.20 & 6.13 & 6.08  \\
 \hline
 32  & 6.24 & 6.12 & 6.07 & 6.03  \\
 \hline
 16  & 6.12 & 6.14 & 6.04 & 6.02  \\
 \hline
 \multicolumn{5}{|c|}{Nemotron4-340B (FP32 PPL = 3.48)} \\ 
 \hline
 \hline
 64 & 3.67 & 3.62 & 3.60 & 3.59 \\
 \hline
 32 & 3.63 & 3.61 & 3.59 & 3.56 \\
 \hline
 16 & 3.61 & 3.58 & 3.57 & 3.55 \\
 \hline
\end{tabular}
\caption{\label{tab:ppl_llama7B_nemo15B} Wikitext-103 perplexity compared to FP32 baseline in Llama2-7B and Nemotron4-15B, 340B models}
\end{table}

%\subsection{Perplexity achieved by various LO-BCQ configurations on MMLU dataset}


\begin{table} \centering
\begin{tabular}{|c||c|c|c|c||c|c|c|c|} 
\hline
 $L_b \rightarrow$& \multicolumn{4}{c||}{8} & \multicolumn{4}{c||}{8}\\
 \hline
 \backslashbox{$L_A$\kern-1em}{\kern-1em$N_c$} & 2 & 4 & 8 & 16 & 2 & 4 & 8 & 16  \\
 %$N_c \rightarrow$ & 2 & 4 & 8 & 16 & 2 & 4 & 2 \\
 \hline
 \hline
 \multicolumn{5}{|c|}{Llama2-7B (FP32 Accuracy = 45.8\%)} & \multicolumn{4}{|c|}{Llama2-70B (FP32 Accuracy = 69.12\%)} \\ 
 \hline
 \hline
 64 & 43.9 & 43.4 & 43.9 & 44.9 & 68.07 & 68.27 & 68.17 & 68.75 \\
 \hline
 32 & 44.5 & 43.8 & 44.9 & 44.5 & 68.37 & 68.51 & 68.35 & 68.27  \\
 \hline
 16 & 43.9 & 42.7 & 44.9 & 45 & 68.12 & 68.77 & 68.31 & 68.59  \\
 \hline
 \hline
 \multicolumn{5}{|c|}{GPT3-22B (FP32 Accuracy = 38.75\%)} & \multicolumn{4}{|c|}{Nemotron4-15B (FP32 Accuracy = 64.3\%)} \\ 
 \hline
 \hline
 64 & 36.71 & 38.85 & 38.13 & 38.92 & 63.17 & 62.36 & 63.72 & 64.09 \\
 \hline
 32 & 37.95 & 38.69 & 39.45 & 38.34 & 64.05 & 62.30 & 63.8 & 64.33  \\
 \hline
 16 & 38.88 & 38.80 & 38.31 & 38.92 & 63.22 & 63.51 & 63.93 & 64.43  \\
 \hline
\end{tabular}
\caption{\label{tab:mmlu_abalation} Accuracy on MMLU dataset across GPT3-22B, Llama2-7B, 70B and Nemotron4-15B models.}
\end{table}


%\subsection{Perplexity achieved by various LO-BCQ configurations on LM evaluation harness}

\begin{table} \centering
\begin{tabular}{|c||c|c|c|c||c|c|c|c|} 
\hline
 $L_b \rightarrow$& \multicolumn{4}{c||}{8} & \multicolumn{4}{c||}{8}\\
 \hline
 \backslashbox{$L_A$\kern-1em}{\kern-1em$N_c$} & 2 & 4 & 8 & 16 & 2 & 4 & 8 & 16  \\
 %$N_c \rightarrow$ & 2 & 4 & 8 & 16 & 2 & 4 & 2 \\
 \hline
 \hline
 \multicolumn{5}{|c|}{Race (FP32 Accuracy = 37.51\%)} & \multicolumn{4}{|c|}{Boolq (FP32 Accuracy = 64.62\%)} \\ 
 \hline
 \hline
 64 & 36.94 & 37.13 & 36.27 & 37.13 & 63.73 & 62.26 & 63.49 & 63.36 \\
 \hline
 32 & 37.03 & 36.36 & 36.08 & 37.03 & 62.54 & 63.51 & 63.49 & 63.55  \\
 \hline
 16 & 37.03 & 37.03 & 36.46 & 37.03 & 61.1 & 63.79 & 63.58 & 63.33  \\
 \hline
 \hline
 \multicolumn{5}{|c|}{Winogrande (FP32 Accuracy = 58.01\%)} & \multicolumn{4}{|c|}{Piqa (FP32 Accuracy = 74.21\%)} \\ 
 \hline
 \hline
 64 & 58.17 & 57.22 & 57.85 & 58.33 & 73.01 & 73.07 & 73.07 & 72.80 \\
 \hline
 32 & 59.12 & 58.09 & 57.85 & 58.41 & 73.01 & 73.94 & 72.74 & 73.18  \\
 \hline
 16 & 57.93 & 58.88 & 57.93 & 58.56 & 73.94 & 72.80 & 73.01 & 73.94  \\
 \hline
\end{tabular}
\caption{\label{tab:mmlu_abalation} Accuracy on LM evaluation harness tasks on GPT3-1.3B model.}
\end{table}

\begin{table} \centering
\begin{tabular}{|c||c|c|c|c||c|c|c|c|} 
\hline
 $L_b \rightarrow$& \multicolumn{4}{c||}{8} & \multicolumn{4}{c||}{8}\\
 \hline
 \backslashbox{$L_A$\kern-1em}{\kern-1em$N_c$} & 2 & 4 & 8 & 16 & 2 & 4 & 8 & 16  \\
 %$N_c \rightarrow$ & 2 & 4 & 8 & 16 & 2 & 4 & 2 \\
 \hline
 \hline
 \multicolumn{5}{|c|}{Race (FP32 Accuracy = 41.34\%)} & \multicolumn{4}{|c|}{Boolq (FP32 Accuracy = 68.32\%)} \\ 
 \hline
 \hline
 64 & 40.48 & 40.10 & 39.43 & 39.90 & 69.20 & 68.41 & 69.45 & 68.56 \\
 \hline
 32 & 39.52 & 39.52 & 40.77 & 39.62 & 68.32 & 67.43 & 68.17 & 69.30  \\
 \hline
 16 & 39.81 & 39.71 & 39.90 & 40.38 & 68.10 & 66.33 & 69.51 & 69.42  \\
 \hline
 \hline
 \multicolumn{5}{|c|}{Winogrande (FP32 Accuracy = 67.88\%)} & \multicolumn{4}{|c|}{Piqa (FP32 Accuracy = 78.78\%)} \\ 
 \hline
 \hline
 64 & 66.85 & 66.61 & 67.72 & 67.88 & 77.31 & 77.42 & 77.75 & 77.64 \\
 \hline
 32 & 67.25 & 67.72 & 67.72 & 67.00 & 77.31 & 77.04 & 77.80 & 77.37  \\
 \hline
 16 & 68.11 & 68.90 & 67.88 & 67.48 & 77.37 & 78.13 & 78.13 & 77.69  \\
 \hline
\end{tabular}
\caption{\label{tab:mmlu_abalation} Accuracy on LM evaluation harness tasks on GPT3-8B model.}
\end{table}

\begin{table} \centering
\begin{tabular}{|c||c|c|c|c||c|c|c|c|} 
\hline
 $L_b \rightarrow$& \multicolumn{4}{c||}{8} & \multicolumn{4}{c||}{8}\\
 \hline
 \backslashbox{$L_A$\kern-1em}{\kern-1em$N_c$} & 2 & 4 & 8 & 16 & 2 & 4 & 8 & 16  \\
 %$N_c \rightarrow$ & 2 & 4 & 8 & 16 & 2 & 4 & 2 \\
 \hline
 \hline
 \multicolumn{5}{|c|}{Race (FP32 Accuracy = 40.67\%)} & \multicolumn{4}{|c|}{Boolq (FP32 Accuracy = 76.54\%)} \\ 
 \hline
 \hline
 64 & 40.48 & 40.10 & 39.43 & 39.90 & 75.41 & 75.11 & 77.09 & 75.66 \\
 \hline
 32 & 39.52 & 39.52 & 40.77 & 39.62 & 76.02 & 76.02 & 75.96 & 75.35  \\
 \hline
 16 & 39.81 & 39.71 & 39.90 & 40.38 & 75.05 & 73.82 & 75.72 & 76.09  \\
 \hline
 \hline
 \multicolumn{5}{|c|}{Winogrande (FP32 Accuracy = 70.64\%)} & \multicolumn{4}{|c|}{Piqa (FP32 Accuracy = 79.16\%)} \\ 
 \hline
 \hline
 64 & 69.14 & 70.17 & 70.17 & 70.56 & 78.24 & 79.00 & 78.62 & 78.73 \\
 \hline
 32 & 70.96 & 69.69 & 71.27 & 69.30 & 78.56 & 79.49 & 79.16 & 78.89  \\
 \hline
 16 & 71.03 & 69.53 & 69.69 & 70.40 & 78.13 & 79.16 & 79.00 & 79.00  \\
 \hline
\end{tabular}
\caption{\label{tab:mmlu_abalation} Accuracy on LM evaluation harness tasks on GPT3-22B model.}
\end{table}

\begin{table} \centering
\begin{tabular}{|c||c|c|c|c||c|c|c|c|} 
\hline
 $L_b \rightarrow$& \multicolumn{4}{c||}{8} & \multicolumn{4}{c||}{8}\\
 \hline
 \backslashbox{$L_A$\kern-1em}{\kern-1em$N_c$} & 2 & 4 & 8 & 16 & 2 & 4 & 8 & 16  \\
 %$N_c \rightarrow$ & 2 & 4 & 8 & 16 & 2 & 4 & 2 \\
 \hline
 \hline
 \multicolumn{5}{|c|}{Race (FP32 Accuracy = 44.4\%)} & \multicolumn{4}{|c|}{Boolq (FP32 Accuracy = 79.29\%)} \\ 
 \hline
 \hline
 64 & 42.49 & 42.51 & 42.58 & 43.45 & 77.58 & 77.37 & 77.43 & 78.1 \\
 \hline
 32 & 43.35 & 42.49 & 43.64 & 43.73 & 77.86 & 75.32 & 77.28 & 77.86  \\
 \hline
 16 & 44.21 & 44.21 & 43.64 & 42.97 & 78.65 & 77 & 76.94 & 77.98  \\
 \hline
 \hline
 \multicolumn{5}{|c|}{Winogrande (FP32 Accuracy = 69.38\%)} & \multicolumn{4}{|c|}{Piqa (FP32 Accuracy = 78.07\%)} \\ 
 \hline
 \hline
 64 & 68.9 & 68.43 & 69.77 & 68.19 & 77.09 & 76.82 & 77.09 & 77.86 \\
 \hline
 32 & 69.38 & 68.51 & 68.82 & 68.90 & 78.07 & 76.71 & 78.07 & 77.86  \\
 \hline
 16 & 69.53 & 67.09 & 69.38 & 68.90 & 77.37 & 77.8 & 77.91 & 77.69  \\
 \hline
\end{tabular}
\caption{\label{tab:mmlu_abalation} Accuracy on LM evaluation harness tasks on Llama2-7B model.}
\end{table}

\begin{table} \centering
\begin{tabular}{|c||c|c|c|c||c|c|c|c|} 
\hline
 $L_b \rightarrow$& \multicolumn{4}{c||}{8} & \multicolumn{4}{c||}{8}\\
 \hline
 \backslashbox{$L_A$\kern-1em}{\kern-1em$N_c$} & 2 & 4 & 8 & 16 & 2 & 4 & 8 & 16  \\
 %$N_c \rightarrow$ & 2 & 4 & 8 & 16 & 2 & 4 & 2 \\
 \hline
 \hline
 \multicolumn{5}{|c|}{Race (FP32 Accuracy = 48.8\%)} & \multicolumn{4}{|c|}{Boolq (FP32 Accuracy = 85.23\%)} \\ 
 \hline
 \hline
 64 & 49.00 & 49.00 & 49.28 & 48.71 & 82.82 & 84.28 & 84.03 & 84.25 \\
 \hline
 32 & 49.57 & 48.52 & 48.33 & 49.28 & 83.85 & 84.46 & 84.31 & 84.93  \\
 \hline
 16 & 49.85 & 49.09 & 49.28 & 48.99 & 85.11 & 84.46 & 84.61 & 83.94  \\
 \hline
 \hline
 \multicolumn{5}{|c|}{Winogrande (FP32 Accuracy = 79.95\%)} & \multicolumn{4}{|c|}{Piqa (FP32 Accuracy = 81.56\%)} \\ 
 \hline
 \hline
 64 & 78.77 & 78.45 & 78.37 & 79.16 & 81.45 & 80.69 & 81.45 & 81.5 \\
 \hline
 32 & 78.45 & 79.01 & 78.69 & 80.66 & 81.56 & 80.58 & 81.18 & 81.34  \\
 \hline
 16 & 79.95 & 79.56 & 79.79 & 79.72 & 81.28 & 81.66 & 81.28 & 80.96  \\
 \hline
\end{tabular}
\caption{\label{tab:mmlu_abalation} Accuracy on LM evaluation harness tasks on Llama2-70B model.}
\end{table}

%\section{MSE Studies}
%\textcolor{red}{TODO}


\subsection{Number Formats and Quantization Method}
\label{subsec:numFormats_quantMethod}
\subsubsection{Integer Format}
An $n$-bit signed integer (INT) is typically represented with a 2s-complement format \citep{yao2022zeroquant,xiao2023smoothquant,dai2021vsq}, where the most significant bit denotes the sign.

\subsubsection{Floating Point Format}
An $n$-bit signed floating point (FP) number $x$ comprises of a 1-bit sign ($x_{\mathrm{sign}}$), $B_m$-bit mantissa ($x_{\mathrm{mant}}$) and $B_e$-bit exponent ($x_{\mathrm{exp}}$) such that $B_m+B_e=n-1$. The associated constant exponent bias ($E_{\mathrm{bias}}$) is computed as $(2^{{B_e}-1}-1)$. We denote this format as $E_{B_e}M_{B_m}$.  

\subsubsection{Quantization Scheme}
\label{subsec:quant_method}
A quantization scheme dictates how a given unquantized tensor is converted to its quantized representation. We consider FP formats for the purpose of illustration. Given an unquantized tensor $\bm{X}$ and an FP format $E_{B_e}M_{B_m}$, we first, we compute the quantization scale factor $s_X$ that maps the maximum absolute value of $\bm{X}$ to the maximum quantization level of the $E_{B_e}M_{B_m}$ format as follows:
\begin{align}
\label{eq:sf}
    s_X = \frac{\mathrm{max}(|\bm{X}|)}{\mathrm{max}(E_{B_e}M_{B_m})}
\end{align}
In the above equation, $|\cdot|$ denotes the absolute value function.

Next, we scale $\bm{X}$ by $s_X$ and quantize it to $\hat{\bm{X}}$ by rounding it to the nearest quantization level of $E_{B_e}M_{B_m}$ as:

\begin{align}
\label{eq:tensor_quant}
    \hat{\bm{X}} = \text{round-to-nearest}\left(\frac{\bm{X}}{s_X}, E_{B_e}M_{B_m}\right)
\end{align}

We perform dynamic max-scaled quantization \citep{wu2020integer}, where the scale factor $s$ for activations is dynamically computed during runtime.

\subsection{Vector Scaled Quantization}
\begin{wrapfigure}{r}{0.35\linewidth}
  \centering
  \includegraphics[width=\linewidth]{sections/figures/vsquant.jpg}
  \caption{\small Vectorwise decomposition for per-vector scaled quantization (VSQ \citep{dai2021vsq}).}
  \label{fig:vsquant}
\end{wrapfigure}
During VSQ \citep{dai2021vsq}, the operand tensors are decomposed into 1D vectors in a hardware friendly manner as shown in Figure \ref{fig:vsquant}. Since the decomposed tensors are used as operands in matrix multiplications during inference, it is beneficial to perform this decomposition along the reduction dimension of the multiplication. The vectorwise quantization is performed similar to tensorwise quantization described in Equations \ref{eq:sf} and \ref{eq:tensor_quant}, where a scale factor $s_v$ is required for each vector $\bm{v}$ that maps the maximum absolute value of that vector to the maximum quantization level. While smaller vector lengths can lead to larger accuracy gains, the associated memory and computational overheads due to the per-vector scale factors increases. To alleviate these overheads, VSQ \citep{dai2021vsq} proposed a second level quantization of the per-vector scale factors to unsigned integers, while MX \citep{rouhani2023shared} quantizes them to integer powers of 2 (denoted as $2^{INT}$).

\subsubsection{MX Format}
The MX format proposed in \citep{rouhani2023microscaling} introduces the concept of sub-block shifting. For every two scalar elements of $b$-bits each, there is a shared exponent bit. The value of this exponent bit is determined through an empirical analysis that targets minimizing quantization MSE. We note that the FP format $E_{1}M_{b}$ is strictly better than MX from an accuracy perspective since it allocates a dedicated exponent bit to each scalar as opposed to sharing it across two scalars. Therefore, we conservatively bound the accuracy of a $b+2$-bit signed MX format with that of a $E_{1}M_{b}$ format in our comparisons. For instance, we use E1M2 format as a proxy for MX4.

\begin{figure}
    \centering
    \includegraphics[width=1\linewidth]{sections//figures/BlockFormats.pdf}
    \caption{\small Comparing LO-BCQ to MX format.}
    \label{fig:block_formats}
\end{figure}

Figure \ref{fig:block_formats} compares our $4$-bit LO-BCQ block format to MX \citep{rouhani2023microscaling}. As shown, both LO-BCQ and MX decompose a given operand tensor into block arrays and each block array into blocks. Similar to MX, we find that per-block quantization ($L_b < L_A$) leads to better accuracy due to increased flexibility. While MX achieves this through per-block $1$-bit micro-scales, we associate a dedicated codebook to each block through a per-block codebook selector. Further, MX quantizes the per-block array scale-factor to E8M0 format without per-tensor scaling. In contrast during LO-BCQ, we find that per-tensor scaling combined with quantization of per-block array scale-factor to E4M3 format results in superior inference accuracy across models. 


\end{document}
