Our survey is designed to explore the perceptions and attitudes of Black Americans regarding AAE representation in chat-based AI systems across a variety of settings, ranging from professional to personal. For each setting, we gauge how and when participants want an LLM (or chatbot) to use AAE versus MUSE. We aimed to provide sufficient detail on the settings to make them more relatable and easier to comprehend \cite{lenzner2012effects}. The settings are selected to give a more complete picture of Black American's every-day experiences and preferences \cite{Maedche2019AI-Based}. For each scenario presented, study participants were asked to choose from the answer choices seen in \autoref{table:Vignette Choices} for how they would want such an LLM to interact:

%The specific settings we chose are described below, and cover both professional and personal contexts (as AI is used in both~\cite{Garrotes2021THE}), as well as use-cases the cover both LLM-generated continuations of one's own text as and LLM responses in an assistant-like scenario.

%Since use of these AI technologies is widespread in people' work and private lives \cite{Garrotes2021THE}, we explore both formal (professional) and informal (personal) settings. 


\begin{enumerate}[nolistsep,noitemsep]
\item \textit{AI Assistants} (professional and personal).
These LLM-response settings reflect the use of an AI assistant for helping with either professional or personal tasks, and whether such an assistant should address the user in AAE.


\item\textit{Customer Bot}. This LLM-response setting reflects the use of a text-based chatbot agent for quick assistance, and whether the agent should continue the interaction after greeting the user in AAE.

\item \textit{Email and SMS Autocomplete}. These LLM-continuation settings reflect the use of an LLM to autocomplete a user's own writing for emails or text messages.  % if AAE speakers want these systems to maintain a consistent tone from that with which they started the conversation. 

\item \textit{Educational Avatar}. This LLM-response setting reflects the use of AAE by an avatar in an education platform and whether this could impact learning experience.
% \hal{TODO fill in}.
%As inclusive educational materials enhance learning outcomes,  accessibility and provide overall support for better learning experiences \cite{baker2020linguistic}, we also present this setting.
\end{enumerate}

% \hal{there are a lot of mismatches that we need to clean up here (section 3.1). the above set does not match the set in Figure 1, which also doesn't match the set in 4.1.1. the terminology is also inconsistent (including at least Educational Content Delivery vs Educational Avatar). what's going on?} \kac{I wrote one part and the other was written by Christabel. We used different terminologies. I think the one in 4 is valid. Will check against the data to see where the  mismatch is coming from}

\noindent



%SS: Commented out since we don't discuss in Results.Participants were also asked to rank their preferences for when text-based generative AI should communicate in or understand AAE across each scenario on a scale from 1 (highest priority) to 3 (lowest priority). This segment aimed to compare Black Americans' differing preferences towards AI generation of AAE across distinct contexts, considering them relative to each other, to potentially provide insights into the broader societal expectations and acceptance for AAE in these technologies.
% We also asked participants to rank these settings to provid a clear understanding of where AAE/AAL speakers prioritize the use of AAE.
%The final portion of the survey asked participants to consider the potential benefits and dangers they perceive from the use of AAE by these AI systems, which helps us understand the broader implications of Generative AI's role in linguistic representation and inclusivity. 

%\begin{itemize}
 %   \item Always MUSE - \textit{Choice 0}
  %  \item Option to manually switch between AAE and MUSE - \textit{Choice 1}
   % \item Automatic detection and adaptation to AAE or MUSE - \textit{Choice 2}
    %\item Always AAE - \textit{Choice 3}
    %\item No preference as long as the system is effective - \textit{Choice 4}
%\end{itemize}

\begin{table}[t]
\centering
\footnotesize
%\rowcolors{2}{gray!10}{}
%\resizebox{0.98\linewidth}{!}{
\renewcommand{\arraystretch}{0.85} % Reduce row height
\setlength{\tabcolsep}{4pt}
\begin{tabular}{ @{~}l @{~~~} p{55mm} @{~} }
  \toprule			
%  \textbf{Option} & \textbf{Desired LLM Behavior} \\ 		
%  \midrule			
AlwaysMUSE &	LLM should always use MUSE.	\\
AlwaysAAE &	LLM should always use AAE. \\[0.5em]
AutoDetect &	LLM should automatically detect/adapt to\\&\quad the user's language variety. \\
UserOption &	LLM should provide an option to switch\\&\quad between AAE and MUSE. \\[0.5em]
NoPreference	&	No preference as long as the system is\\&\quad effective. \\
  \bottomrule			
\end{tabular}	
%}
%\end{adjustbox}
\caption{The set of possible choices for the preferences survey, which asked Black Americans about the contexts or scenarios in which they would prefer to have language model-based AI technologies generate AAE vs MUSE.} %Scenarios they considered ranged from email continuations in professional settings to educational avatars.}
%\hal{numbering doesn't match}
\label{table:Vignette Choices}
\end{table}





