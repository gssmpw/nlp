%\\texttt{
%\Linguistic diversity in Large Language Models(LLMs) has been severely understudied and underdeveloped \chris{I don't think I understand this statement. Maybe cite some work that can back this claim. Otherwise, I think there's a lot of work in Language models, but in this specific area of research we are doing, there may not be enough work. }. 
%With the recent widespread adoption of Large Language Models (LLMs) across a diverse set of use cases, the need for these models to be representative of all major social groups, including the most vulnerable, is more important now than ever. The increase in data used in training LLMs, has enabled them to communicate in more traditionally under-served linguistic styles.
We examine the representation of African American English (AAE) in large language models (LLMs), exploring (a) the perceptions Black Americans have of how effective these technologies are at producing authentic AAE, and (b) in what contexts Black Americans find this desirable.
Through both a survey of Black Americans ($n=$ 104) and annotation of LLM-produced AAE by Black Americans ($n=$ 228), we find that Black Americans favor choice and autonomy in determining when AAE is appropriate in LLM output. They tend to prefer that LLMs default to communicating in Mainstream U.S. English in formal settings, with greater interest in AAE production in less formal settings. When LLMs were appropriately prompted and provided in context examples, our participants found their outputs to have a level of AAE authenticity on par with transcripts of Black American speech. Select code and data for our project can be found here: \url{https://github.com/smelliecat/AAEMime.git} %; in most cases the LLMs were judged by our study participants as containing more AAE or being more authentic than our human baseline, with one case demonstrating less. Nevertheless, none of the LLM-generated AAE was judged to be offensive or mocking of African Americans by our study participants. 
% For our human baseline for comparison to LLM generations, we utilized transcribed speech from interviews with Black Americans from the Corpus of Regional African American Language (CORAAL/AAL).
% For our study, a sample of the interview exchanges and LLM generations was manually annotated by AAE speakers to assess the quality and effectiveness of AAE production by three popular LLMs. This helped to ascertain whether Black Americans can distinguish LLM outputs from real human utterances. %/We then surveyed AAE speakers, with a combinations of LLM generated outputs and original CORAAE interactions to ascertain whether they are able to distinguish LLM outputs from real human utterances. We also gather insights into what AAE speakers want from LLMs and how they want to interact with these systems.

%\}
%\nayeem{Add a summary sentence of the main findings}
