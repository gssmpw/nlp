In this study, we explore how Black Americans perceive the appropriate use of AAE pertaining to language-model-based technologies, %\label{rq:1}, 
especially in terms of their ability to authentically represent AAE. We consider both the expectations of the community and the actual output of current systems, exploring the idea that AAE should not merely be an option but a well-integrated feature.

Overall, the LLMs we reviewed were surprisingly capable and comparable in how they were perceived in terms of their ability to produce authentic AAE in comparison to the transcribed speech of Black Americans from the CORAAL corpus. 
We found that the text completions by some LLMs were often perceived as more AAE-heavy, or sounding more like something a Black American would say, than our the language in our human AAE corpus. If our human AAE baseline is assumed to have the ``right amount'' of AAE, then having more linguistic features of AAE than the baseline could be considered to be excessive by some AAE speakers, whereas less may be an insufficient amount of AAE. At the same time, because the human AAE is a transcript, it may not be fully reflective of all ways that AAE is used in practice. LLMs are either slightly under-doing or over-doing AAE, but on average, participants generally disagreed that the machine-generated text by the LLMs we studied was offensive to or mocking of Black Americans.

For the scenarios-based questions of our survey, our findings reveal intricate preferences for AI applications across diverse environments, indicative of broader societal shifts toward technologies that are both personalized and context-sensitive. These insights are pivotal for developers and policymakers tasked with refining AI tech to align more closely with user expectations, thereby facilitating smoother integration of AI into everyday life.

\paragraph{Future Work}
Our study highlights that Black Americans prefer, at a minimum, the option for communication in AAE with popular language-based Generative AI tools and generally deem LLM-generated AAE as credible and similar to spoken AAE. In light of this finding, we encourage the technology community to expand the linguistic diversity in language-based Generative AI tools; in particular, they should consider functionality that provides the autonomy to Black Americans to switch to AAE on demand in circumstances of their choosing, as well as have an array of multimodal options for AAE communications, including but not limited to generation and understanding of AAE text as well as audio communications, including speech recognition and production. More generally, technological support for alternative dialects or sociolects in generative AI systems will make these systems more broadly accepted and equitable for a range of important stakeholder populations. Finally, we encourage future research on the relationship between 1) the diverse attributes that characterize Black Americans (whether they be regional, cultural, socioeconomic or demographic) and 2) whether and how they express AAE personally, or their preferences for its production by Generative AI tools across different contexts.