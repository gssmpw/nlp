\newpage \onecolumn
\section{Appendix} \label{sec:appendix}

\subsection{Preparation of the CORAAL Corpus (Black American transcribed interviews)} \label{apdx:corral_desc}

Our prompt texts or prefixes needed to have authentic AAE to the degree possible and cover a broad range of the different variations of AAE spoken in the wild. \cite{lanehart_1language_2015}. To achieve this goal, we made use of the CORAAL interview transcriptions, choosing 30 interviews from more than 220 transcribed interviews available (interviews were conducted with AAE speakers born between 1888 and 2005). The CORAAL AAE speakers who participated in these interviews are from six (6) cities across the United States with large Black populations, including: Washington, D.C. (from 1968 and 2016 interviews), Detroit, Michigan, Lower East Side New York City, New York, Princeville, North Carolina, Rochester New York and Valdosta, Georgia \cite{KendallFarrington2023}. The thirty (30) interviews mentioned above (\nameref{apdx:siids}) from the CORAAL corpus were chosen to be balanced by sex and randomly sampled by location (but ensuring that we drew from the cities mentioned above). We considered these original interviews of Black Americans from the CORAAL corpus to be our human baseline or the AAE ``ground truth''; in other words, this corpus was considered to represent authentic AAE spoken by Black Americans, to investigate our original research questions. 

We proposed an initial set of criteria for a valid interview in our setting, which was that it must include only two (2) participants and any given interviewer and interviewee statements from the CORAAL corpus must have been greater than five tokens (words) long to be included (since short utterances were typically filler words such as "uh huh" or similar acknowledgments), unless they contained "who, what, where, when and why" types of questions, which contained relevant content. Additionally, any pairs of exchanges between the interviewer and the interviewee, where the interviewee's response was less than 20 tokens (words) long, were excluded since it was important to have sufficiently long prefixes for the LLMs to create coherent and meaningful continuations of the interviewee response. To collect the human judgments on the LLM-generated texts relative to our human baseline, via the second part of our online study we provided our study participants approximately eight conversational exchanges per person, where each exchange consisted of an interviewer statement followed by the interviewee response. 

% \subsection{Selection Criteria} \label{apdx:sc}

\newpage
\subsection{Selected Interviews} \label{apdx:siids}



\setlength{\fboxsep}{1pt}
\noindent\fbox{%
    \parbox{\linewidth}{%
        \texttt{ATL\_textfiles\_2020.05/ATL\_se0\_ag2\_f\_01\_1.txt},\\
        \texttt{DCB\_textfiles\_2018.10.06/DCB\_se1\_ag1\_f\_01\_1.txt},\\
        \texttt{DCB\_textfiles\_2018.10.06/DCB\_se2\_ag1\_m\_01\_1.txt},\\
        \texttt{DTA\_textfiles\_2023.06/DTA\_se1\_ag3\_m\_02\_1.txt},\\
        \texttt{LES\_textfiles\_2021.07/LES\_se0\_ag3\_m\_01\_1.txt},\\
        \texttt{PRV\_textfiles\_2018.10.06/PRV\_se0\_ag2\_m\_02\_1.txt},\\
        \texttt{ROC\_textfiles\_2020.05/ROC\_se0\_ag2\_f\_04\_1.txt},\\
        \texttt{ROC\_textfiles\_2020.05/ROC\_se0\_ag2\_m\_01\_1.txt},\\
        \texttt{ROC\_textfiles\_2020.05/ROC\_se0\_ag3\_f\_02\_1.txt},\\
        \texttt{VLD\_textfiles\_2021.07/VLD\_se0\_ag2\_f\_01\_1.txt},\\
        \texttt{ATL\_textfiles\_2020.05/ATL\_se0\_ag1\_f\_01\_1.txt},\\
        \texttt{DCA\_textfiles\_2018.10.06/DCA\_se1\_ag1\_f\_02\_1.txt},\\
        \texttt{DCB\_textfiles\_2018.10.06/DCB\_se1\_ag2\_f\_01\_1.txt},\\
        \texttt{DTA\_textfiles\_2023.06/DTA\_se1\_ag1\_f\_01\_1.txt},\\
        \texttt{LES\_textfiles\_2021.07/LES\_se0\_ag2\_f\_01\_1.txt},\\
        \texttt{PRV\_textfiles\_2018.10.06/PRV\_se0\_ag1\_f\_01\_2.txt},\\
        \texttt{ROC\_textfiles\_2020.05/ROC\_se0\_ag1\_f\_02\_1.txt},\\
        \texttt{VLD\_textfiles\_2021.07/VLD\_se0\_ag3\_f\_01\_2.txt},\\
        \texttt{DTA\_textfiles\_2023.06/DTA\_se1\_ag1\_f\_02\_1.txt},\\
        \texttt{ROC\_textfiles\_2020.05/ROC\_se0\_ag1\_f\_03\_1.txt},\\
        \texttt{ATL\_textfiles\_2020.05/ATL\_se0\_ag1\_m\_04\_2.txt},\\
        \texttt{DCA\_textfiles\_2018.10.06/DCA\_se1\_ag3\_m\_01\_1.txt},\\
        \texttt{DCA\_textfiles\_2018.10.06/DCA\_se3\_ag4\_m\_01\_1.txt},\\
        \texttt{DCB\_textfiles\_2018.10.06/DCB\_se3\_ag3\_m\_02\_1.txt},\\
        \texttt{DTA\_textfiles\_2023.06/DTA\_se1\_ag1\_m\_01\_1.txt},\\
        \texttt{DTA\_textfiles\_2023.06/DTA\_se2\_ag4\_m\_02\_1.txt},\\
        \texttt{LES\_textfiles\_2021.07/LES\_se0\_ag4\_m\_01\_1.txt},\\
        \texttt{VLD\_textfiles\_2021.07/VLD\_se0\_ag2\_m\_01\_1.txt},\\
        \texttt{VLD\_textfiles\_2021.07/VLD\_se0\_ag3\_m\_02\_1.txt},\\
        \texttt{DCB\_textfiles\_2018.10.06/DCB\_se1\_ag2\_m\_02\_1.txt}
    }%
}

\newpage
\subsection{Prompting LLMs}
The LLMs in our study were prompted to create continuations or text completions of the original interviewee statements. These, along with the responses of the interviewees from our human baseline (CORAAL), were later annotated by our study participants. The process of creating LLM continuations of interviewee statements involved first converting the CORAAL interviewer and interviewee exchanges into a format suitable for LLM input. This included systematic editing to alternate responses between the interviewer and interviewee, to maintain some flow and coherency in the conversation. We also removed non-linguistic features like "<pause>" as outlined in the CORAAL online corpus documentation, so we could focus more on the important linguistic features.

To generate our LLM outputs, we opted to use three of the most popular advanced LLMs. Namely OpenAI GPT 4o-mini, Meta-Llama-3-70B-Instruct and Mixtral-8x7B-Instruct-v0.1 (\citealp{brown2020language}; \citealp{llama3modelcard}; \citealp{jiang2024mixtral}). The choice of these models was based on their leading performance in natural language processing tasks and their widespread adoption \cite{chiang2024chatbot}. We utilized a custom system prompt (\nameref{apdx:fp}) for all 3 models. This system prompt included instructions on the objective of the task and guidelines on how the models were expected to respond to the user prompts. The OpenAI prompt was performed with the OpenAI API, while the open weights models(LLAMA and Mixtral) were prompted with a modified version of the Llama factory code base \cite{zheng2024llamafactory}. Subsequently, we explored 3 prompting strategies. We found that, in the zero-shot setting, all three models provided responses that failed to follow the instruction, refused to answer user prompts, or provided responses that did not fit in the context of the prompt (\nameref{apdx:zsp}). Subsequently, we experimented with providing chat history to the models (i.e. using an ``in context learning'' approach \cite{brown2020language}).  In this setting, we tested two different sources from the chat history. The first was from the model responses, where we kept the models' own responses and added them to the new prompts as chat history. This performed slightly better than the zero-shot setting but still struggled to stay in context of the conversation as once the model deviated, the entire conversation followed in the said deviation. This happened quite often since the first few prompts always had a chat history with few irrelevant chat histories (salutations) (\nameref{apdx:mhp}). The second and best-performing setting was using the CORAAL ground truth data as our chat history. As this had the context of the original intent of the interview passed to the models, though the model responses differed from the ground truth, its flow was a lot more consistent and all 3 models successfully followed to some extent the style requirements of the system prompts. (\nameref{apdx:gtp})

\newpage
\subsection{Prompts} \label{apdx:prompts}
\subsubsection{System Prompt Example 1}
\noindent\fbox{%
    \parbox{\columnwidth}{%
    \texttt{
    You are playing the role of a Black American in an interview. All of the following questions are interview questions or interviewer statements. You should answer as the guest, that is, the person being interviewed. Answer these all in African American English as part of one interview, using no more than 75 words but as few as necessary, and as if you are an African American \{gender\}. It is fine to provide a one-word acknowledgment such as 'ok' in 
    cases where the interviewer is giving background or does not need a response. Your ethnicity is  African American}
 }%
}

\subsubsection{System Prompt Example 2}
\noindent\fbox{%
    \parbox{\columnwidth}{%
    \texttt{
    You are being interviewed. All of the following questions are interview questions or interviewer statements. You should answer as the guest, that is, the person being interviewed. Answer these all in African American English as part of one interview, using no more than 75 words but as few as necessary, and as if you are an African American \{gender\}. It is fine to provide a one-word acknowledgment such as 'ok' in cases where the interviewer is giving background or does not need a response. Your ethnicity is African American}
 }%
}

\subsubsection{System Prompt Example 3}
\noindent\fbox{%
    \parbox{\columnwidth}{%
    \texttt{
    You are playing the role of an American in an interview. All of the following questions are interview questions or interviewer statements. You should answer as the guest, i.e., the interviewee.Answer these all in African American English as part of one interview, using no more than 75 words but as few as necessary, and as if you are an African American \{gender\}. It is fine to provide a one-word acknowledgment such as 'ok' in cases where the interviewer is giving background or does not need a response. Your ethnicity is African American. You will be penalized for your errors}
 }%
}

\newpage
\subsection{Final System Prompts} \label{apdx:fp}

\noindent\fbox{%
    \parbox{\columnwidth}{%
    
        \textbf{Instruction (for Mixtral and Llama) for Continuation:}\\

        \ttfamily
        Objective: Playing the role of an interview guest, extend the last response provided by an interview guest, using African American Vernacular of English (AAVE).\\

        Word Limit: Keep the extension under 125 words.\\

        Response Guidelines: Ensure that the continuation is a seamless extension of the guest's last response, maintaining the conversational tone and context. Do not include anything that serves to explain your continuations.\\

        Exclusion of Labels: Do not include any interview format labels such as "Host:" or "Guest:" in your response.\\

        Output Requirement: The final output should be a direct continuation of the interview guest's last statement, written as if the guest is still speaking.
    }%
}

~

\noindent\fbox{%
    \parbox{\columnwidth}{%
        \textbf{Instruction (for GPT) for Continuation in African American English (AAE):}\\

        \ttfamily
        Provide a continuation of the guest response last given in an interview using African American English in less than 125 words. Only continue and complete the guest response (do not use the strings Host: or Guest: in your completion).
 }%
}
\subsubsection{Zero-Shot Example} \label{apdx:zsp}

\noindent\fbox{%
    \parbox{\columnwidth}{%
        \textbf{Instruction (for GPT) for Continuation in African American English (AAE):}\\

        \ttfamily
        Provide a continuation of the guest response last given in an interview using African American English in less than 125 words. Only continue and complete the guest response (do not use the strings Host: or Guest: in your completion).
 }%
}
\subsubsection{Model History Example} \label{apdx:mhp}
\noindent\fbox{%
    \parbox{\columnwidth}{%
        \textbf{Instruction (for GPT) for Continuation in African American English (AAE):}\\

                \ttfamily
        Provide a continuation of the guest response last given in an interview using African American English in less than 125 words. Only continue and complete the guest response 
        (do not use of the strings Host: or Guest: in your completion).
 }%
}
\subsubsection{Ground Truth Example} \label{apdx:gtp}
\noindent\fbox{%
    \parbox{\columnwidth}{%
        \textbf{Instruction (for GPT) for Continuation in African American English (AAE):}\\

                \ttfamily
        Provide a continuation of the guest response last given in an interview using African American English in less than 125 words. Only continue and complete the guest response 
        (do not use of the strings Host: or Guest: in your completion).
 }%
}
 
\newpage
\subsection{Terms of use for each model}
We adhere to the terms of usage provided by the model authors. 
\begin{itemize}
    \item Llama3: \url{https://huggingface.co/meta-llama/Meta-Llama-3-8B/blob/main/LICENSE}
    \item GPT-3.5-Turbo: \url{https://openai.com/policies/terms-of-use}
    \item Mixtral-Instruct-v0.1: \url{https://mistral.ai/terms-of-service/}
\end{itemize}
\paragraph{Licenses}
The CORAAL dataset is used under the CC-BY \footnote{\textcolor[HTML]{000099}{https://creativecommons.org/licenses/by/4.0/}} license.

\newpage
\subsection{Survey} \label{svy}

\begin{figure}[H]
    \centering
    \includegraphics[width=1\linewidth]{PartI.pdf}
    \caption{Sample question from the survey on participants preference in a realistic scenario.}
    \label{fig:preference_survey}
\end{figure}


\begin{figure}[H]
    \centering
    \includegraphics[width=1\linewidth]{PartII.pdf}
    \caption{Sample question from annotation task where participants are asked to consider the highlighted, underlined part of the interviewee's response, which is
 and mark their level of agreement with the following statements}
    \label{fig:annotation_task}
\end{figure}

\newpage
\subsection{Study Participants}
\label{sec:annotators}
\subsubsection{Study Participant Eligibility and Recruitment}
We recruited participants who were adults aged 18 years or older on the prolific platform. The eligibility criteria ensured that participants' nationality was either the United States or the United States Minor Outlying Islands. Participants self-identified their ethnicity from the following categories: African, Black/African American, Caribbean, Mixed, Other (with an option to specify via email), or Black/British. Additionally, participants reported the place where they spent most of their time before turning 18, limited to the United States or the United States Minor Outlying Islands. 

\subsubsection{Demographics}
We collected detailed demographic information from participants, including gender, age, education, ethnicity, and regional representation. We present detailed demographic plots of our participants for the survey portion of our study below.
These figures illustrate the diversity within our sample and highlight some key observations:

% Gender and Age: The participant pool showed a .....
% Ethnicity: The ethnic group distribution showed representation from African, Black/African American, Caribbean, Mixed, Other, and Black/British participants. 
% Regional Representation: Participants reported the place where they spent most of their time before turning 18, ensuring substantial cultural exposure relevant to the study. The regional distribution showed ..

\textbf{\textit{Gender and Age}} Our survey sample showed a diverse age distribution, with a noticeable peak in the younger age groups, particularly those between 25-34 and 35-44 years old, as shown in the ``Age Group Distribution of Respondents'' graph. Gender distribution varied across different age groups, indicating a broader representation among the younger demographics. The ``Gender Distribution Across Age Groups'' (see ~\autoref{fig:age_gender} and ~\autoref{fig:age_gender_bins}) charts further details this distribution.

\textbf{\textit{Regional Representation}}

Participants reported the region where they spent most of their time before turning 18, ensuring substantial cultural exposure relevant to the study. The regional distribution primarily featured respondents from the South, followed by balanced representation from the Northeast, West, and Midwest. (see ~\autoref{fig:region})

\textbf{\textit{Education Levels}}

Participants’ education levels varied widely, encompassing high school diplomas to doctorate degrees, which is reflective of a broad socio-economic spectrum. This diversity in educational backgrounds helps enrich the insights derived from the study. (see ~\autoref{fig:ed_lvl})

\textbf{\textit{Ethnicity and Language Proficiency}}

The ethnic group distribution showed significant representation from diverse backgrounds, and language proficiency varied widely among participants, which included proficiency in Mainstream U.S. English, African American English, and other specified languages. These factors underscore the multicultural and multilingual composition of our respondents. (see ~\autoref{fig:ling_div})

% \begin{figure*}[t]
%   \centering
%   \includegraphics[width=\columnwidth]{latex/sections/gender_dist2.pdf} 
%   \caption {Add gender captions.}
% \end{figure*}

% \begin{figure*}[t]
%   \centering
%   \includegraphics[width=\columnwidth]{latex/fig/age_group2.pdf} 
%   \caption {Add age captions.}
% \end{figure*}

\begin{figure*}[t]
  \includegraphics[width=0.48\linewidth]{gender_dist2.pdf} \hfill
  \includegraphics[width=0.48\linewidth]{age_group2.pdf}
  \caption {\textbf{Left: } \textit{Bar Plot of Gender Distribution Among Respondents}: This graph displays the count of survey participants according to their gender identification, including Female, Male, Non-Binary, Undisclosed, and Other. The largest groups are Female and Male, with significant representation, while Non-Binary and Other categories show fewer participants. The `Undisclosed' category represents respondents who preferred not to specify their gender. \textbf{Right: } \textit{Bar Plot of Respondent Age Distribution}: This graph quantifies the distribution of survey respondents across various age groups. The largest groups are those aged 25-34 and 35-44, demonstrating strong participation from these demographics. In contrast, the 55-64 age group has the fewest respondents. The category labeled 'Und' represents those who preferred not to disclose their age.}
  \label{fig:age_gender}
\end{figure*}

\begin{figure*}[t]
  \centering
  \includegraphics[width=\linewidth]{gndr_by_age2.pdf} 
  \caption {\textit{Bar Plot of Gender Distribution Across Age Groups}: This graph presents a breakdown of gender identities among survey respondents segmented by age groups ranging from 18 to 64 and over. The categories include Female, Male, and Non-Binary, as well as respondents who prefer not to answer.}
  \label{fig:age_gender_bins}
\end{figure*}

\begin{figure*}[t]
  \centering
  \includegraphics[width=0.48\columnwidth]{respondents_by_region_bar_chart2.pdf} 
  \includegraphics[width=0.48\columnwidth]{lvl_of_und.pdf}
  \caption {\textbf{Left: } \textit{Bar Plot of Survey Respondents by Region}: This graph displays the number of survey respondents categorized by their geographic regions within the United States—South, Northeast, West, and Midwest. The South shows the highest participation with 53 respondents, followed significantly by the Northeast with 21, and the West and Midwest each with 17. This visualization highlights regional engagement in the survey, providing insights into the geographic distribution of participants and potentially reflecting regional differences in perspectives or experiences.  \textbf{Right: } \textit{Bar Plot of Levels of Understanding Among Participants}: This graph categorizes participants' self-rated levels of understanding from `Basic Awareness' to `Expert.' The ratings, scaled from 1 to 4, indicate the depth of knowledge or proficiency individuals feel they possess in a specific context. The plot visually summarizes the distribution, revealing how many participants consider themselves at each understanding level, thereby providing insights into the overall expertise and educational needs within the surveyed group.}
  \label{fig:region}
\end{figure*}

\begin{figure*}[t]
  \centering
  \includegraphics[width=\linewidth]{edu_lvl2.pdf} 
  \caption {\textit{Bar Plot of Education Level Distribution Among Respondents}: This graph shows the diverse educational backgrounds of survey participants, ranging from high school diplomas to doctorate degrees. Each bar represents the count of individuals with specific educational qualifications, such as `Some College, No Degree,' `Undergraduate Degrees,' `Graduate Degrees,' and more. This visualization helps to understand the educational diversity within the surveyed group, highlighting the range of academic achievements.}
  \label{fig:ed_lvl}
\end{figure*}

\begin{figure*}[t]
  \centering
  \includegraphics[width=\columnwidth]{lang_prof_bar.pdf} 
  \caption {\textit{Bar Plot of Language Proficiency Preferences}: This graph quantifies participant preferences for language proficiency in different varieties, focusing on Mainstream U.S. English (MUSE) and African American English (AAE). The bars represent the number of participants proficient in solely MUSE, solely AAE, a combination of both, and those with proficiencies that include other specified languages.}
  \label{fig:ling_div}
\end{figure*}

\begin{figure*}[t]
  \centering
  \includegraphics[width=\linewidth]{personal_benefits.pdf} 
  \caption {\textit{Bar Plot of Perceived Benefits}: This graph illustrates the various benefits identified by participants when African American English (AAE) is incorporated into chatbot interactions. Each bar represents specific advantages such as enhanced cultural representation, personal engagement, and broader acceptance. 
  % The labels simplify complex concepts into clear categories, emphasizing how the inclusion of AAE can improve user experience by fostering a more personal and culturally sensitive communication environment. This visualization underscores the importance of integrating AAE to promote inclusivity and respect within digital communication platforms.
  }
  \label{fig:benefits}
\end{figure*}

\begin{figure*}[t]
  \centering
  \includegraphics[width=\linewidth]{perceived_dangers.pdf} 
  \caption {\textit{Bar Plot of Participant Concerns}: This graph illustrates the range of selected concerns among participants regarding the integration of African American English (AAE) into chatbot technology. Each bar represents a distinct set of issues, from perpetuating stereotypes and biases to potential misunderstandings and fears of cultural appropriation. 
  % The labels categorize the primary concerns raised, highlighting the complex challenges in implementing culturally specific linguistic features in digital communication platforms. This plot serves as a visual summary of the potential hurdles in making chatbot interactions more culturally inclusive and sensitive.
  }
  \label{fig:dangers}
\end{figure*}

% \begin{figure*}[t]
%   \centering
%   \includegraphics[width=\columnwidth]{latex/fig/lvl_of_und.pdf} 
%   \caption {Participants' Level of Understanding with respect to AAE}
%   \label{fig:lvl_und}
% \end{figure*}

\begin{figure*}[t]
  \centering
  \includegraphics[width=\linewidth]{lang_pref_bar.pdf} 
  \caption {\textit{Bar Plot of Terminology Preferences for AAE}: This graph presents the count of participants' preferences for various terms used to describe African American English. Each bar represents the popularity of terms such as `African American English', `African American Vernacular English', `Ebonics', and other variants. The plot underscores the diverse linguistic identities within the African American community and highlights the specific terminology that participants feel most accurately represents their language variety.}
  \label{fig:language variety term}
\end{figure*}

\begin{figure*}[t]
  \centering
  \includegraphics[width=\linewidth]{lang_use_contx.pdf} 
  \caption {\textit{Bar Plot of Contextual Preferences for Using AAE}: This graph displays the frequency of preferences among participants for using African American English (AAE) across various social and professional contexts. Each bar indicates the count of participants who prefer using AAE in settings ranging from personal interactions, such as family and friendship circles, to more formal environments like professional and educational settings. }
  \label{fig:lang_use_contx}
\end{figure*}

\begin{figure*}[t]
  \centering
  \includegraphics[width=\linewidth]{identifying_term.pdf} 
  \caption {\textit{Bar Plot of Preferred Self-Identification Terms}: This graph illustrates the distribution of preferred self-identification terms among respondents, highlighting the diversity within racial and ethnic identities. The terms range from `Black' and `African American' to more specific identities such as `Afro-Latinx' and `Afro-Caribbean.' Each bar represents the count of individuals who prefer each term, with `Black' and `African American' being the most common, followed by `Black American' and `Bi-racial or Multi-racial.' }
  \label{fig:identifying_term}
\end{figure*}

%\newpage
%\subsection{Results}
%\subsubsection{Annotation of Highlighted Texts}

