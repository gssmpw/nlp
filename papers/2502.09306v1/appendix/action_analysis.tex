\begin{proof}
When the schedule satisfies $\max_{t\in[0,T]}\vert \partial_t{\log \lambda_t}\vert \leq C_\lambda$, we consider the reparametrised version of $\mu_t$ in terms of the schedule $\lambda_t$, denoted as $\Tilde{\mu}_\lambda$ and let $X_{\lambda}\sim\Tilde{\mu}_{\lambda}$ and $X_{\lambda + \delta}\sim\Tilde{\mu}_{\lambda + \delta}$.
Recall that 
\begin{equation}\label{appendix:eq_conv_path_random_variables}
    X_{\lambda} = \sqrt{\lambda} X + \sqrt{1-\lambda}\sigma Z 
\end{equation}
where $X\sim\pid$ and $Z\sim \mathcal{N}(0, I)$ are independent from each other. 
We introduce a new random variable $\Tilde{X}$, independent from $Z$, that follows a Gaussian distribution,  $\Tilde{X}\sim N(0, \sigma_\pi^2 I)$, satisfying 
\begin{equation*}
    \sigma_\pi = \argmin_{{\hat{\sigma}}\geq \sigma} W_2\left( \mathcal{L}(X),\mathcal{L}(Y_{\hat{\sigma}})\right), \quad \text{where} \;\; Y_{\hat{\sigma}}\sim \mathcal{N}(0, \hat{\sigma}^2 I), \; Z \independent Y_{\hat{\sigma}}.
\end{equation*}
Furthermore, we select $\tilde{X}$ to be the specific random variable that attains the minimal coupling with $X$, that is, $W_2^2\left( \mathcal{L}(X),\mathcal{L}(\tilde{X})\right) = \mathbb{E}\left[\Vert X-\tilde{X}\Vert^2\right]$.
Using the random variable $\Tilde{X}$, we can rewrite \eqref{appendix:eq_conv_path_random_variables} as
\begin{align*}
    X_\lambda &= \sqrt{\lambda} (X-\Tilde{X}) + \sqrt{\lambda} \Tilde{X} +\sqrt{1-\lambda}\sigma Z\\
    & \overset{d}{=} \sqrt{\lambda} (X-\Tilde{X}) + \sqrt{\lambda\sigma_\pi^2 + (1-\lambda)\sigma^2} Y,
\end{align*}
where $Y\sim \mathcal{N}(0, I)$.  
The Wasserstein-2 distance between $\Tilde{\mu}_\lambda$ and $\Tilde{\mu}_{\lambda+\delta}$ is given by
\begin{align*}
     W_2^2(\Tilde{\mu}_\lambda, &\Tilde{\mu}_{\lambda+\delta}) \leq \mathbb{E}\left[\Vert X_\lambda-X_{\lambda + \delta}\Vert^2\right]\\
&\leq 2\mathbb{E}\left[\left\Vert(\sqrt{\lambda}-\sqrt{\lambda+\delta})(X-\Tilde{X})\right\Vert^2\right] +2 \mathbb{E}\left[\left\Vert\left(\sqrt{\lambda\sigma_{\pi}^2 + (1-\lambda)\sigma^2}-\sqrt{(\lambda + \delta)\sigma_{\pi}^2 +(1-\lambda-\delta)\sigma^2}\right) Y\right\Vert^2\right]\\
&= 2(\sqrt{\lambda}-\sqrt{\lambda+\delta})^2 \mathbb{E}\left[\Vert X-\Tilde{X}\Vert^2\right] + 2\left(\sqrt{\lambda\sigma_{\pi}^2 + (1-\lambda)\sigma^2}-\sqrt{(\lambda + \delta)\sigma_{\pi}^2 +(1-\lambda-\delta)\sigma^2}\right)^2d.
\end{align*}
Using the definition of the metric derivative we have
\begin{equation*}
    \left\vert\Dot{\Tilde{\mu}}\right\vert_\lambda^2 = \lim_{\delta\to 0}\frac{ W_2^2(\Tilde{\mu}_\lambda, \Tilde{\mu}_{\lambda+\delta})}{\delta^2}\leq \frac{\mathbb{E}\left[\Vert X-\Tilde{X}\Vert^2\right]}{2 \lambda} + \frac{(\sigma_\pi^2-\sigma^2)^2}{2(\lambda\sigma_\pi^2 + \sigma^2(1-\lambda))}d = \frac{\mathbb{E}\left[\Vert X-\Tilde{X}\Vert^2\right]}{2 \lambda} + \frac{(\sigma_\pi^2-\sigma^2)^2}{2(\sigma^2 + \lambda(\sigma_{\pi}^2-\sigma^2))}d.
\end{equation*}
Since $\mu_t = \Tilde{\mu}_{\lambda_t}$, we have that $\vert\Dot{\mu}\vert_t = \left\vert\Dot{\Tilde{\mu}}\right\vert_\lambda\left\vert\partial_t{\lambda_t}\right\vert$. Using the assumption on the schedule we have the following expression for the action
\begin{align}
    \mathcal{A}_{\lambda}(\mu) &= \int_0^T \vert\Dot{\mu}\vert_t^2 \md t = \int_0^T \left\vert\Dot{\Tilde{\mu}}\right\vert_\lambda^2\left\vert\partial_t{\lambda_t}\right\vert^2 \md t\nonumber\\
    &\leq \int_0^T \left(\frac{\mathbb{E}\left[\Vert X-\Tilde{X}\Vert^2\right]}{2 \lambda_t} + \frac{(\sigma_\pi^2-\sigma^2)^2}{2(\sigma^2 + \lambda_t(\sigma_{\pi}^2-\sigma^2))}d\right)\left\vert \partial_t{\lambda_t}\right\vert^2\md t \nonumber\\
    &= \int_0^T \left(\frac{\mathbb{E}\left[\Vert X-\Tilde{X}\Vert^2\right]}{2} + \frac{(\sigma_\pi^2-\sigma^2)^2}{2(\sigma^2/\lambda_t + \sigma_{\pi}^2-\sigma^2)}d\right)\left\vert \partial_t{\log\lambda_t}\right\vert \left\vert \partial_t{\lambda_t}\right\vert\md t \nonumber\\    
    &\leq C_\lambda \int_0^T \left(\frac{\mathbb{E}\left[\Vert X-\Tilde{X}\Vert^2\right]}{2} +\frac{(\sigma_\pi^2-\sigma^2)^2}{2(\sigma^2/\lambda_t + \sigma_{\pi}^2-\sigma^2)}d\right) \left\vert \partial_t{\lambda_t}\right\vert\md t \nonumber\\ 
    &= C_\lambda \int_{\lambda_0}^{1} \left(\frac{\mathbb{E}\left[\Vert X-\Tilde{X}\Vert^2\right]}{2} + \frac{(\sigma_\pi^2-\sigma^2)^2}{2(\sigma^2/\lambda+(\sigma_\pi^2-\sigma^2))} d\right) \md \lambda \nonumber\\
    &\leq \frac{C_\lambda}{2} \left(\mathbb{E}\left[\Vert X-\Tilde{X}\Vert^2\right] + d\left(\sigma_\pi^2-\sigma^2 + \sigma^2\log\frac{\sigma^2}{\sigma_\pi^2}\right)\right)\nonumber\\ 
    &\leq \frac{C_\lambda}{2} \left(2\mathbb{E}\left[\Vert X\Vert^2\right] + d(3\sigma_\pi^2-\sigma^2)\right),\nonumber 
\end{align}
where in the last line we have used that $\sigma_\pi\geq\sigma$.
Note that by setting $\sigma=\sigma_\pi$, the second term in the penultimate expression cancels out, resulting in
\begin{equation*}
    \mathcal{A}_\lambda(\mu)\leq \frac{C_\lambda}{2}\mathbb{E}\left[\Vert X-\Tilde{X}\Vert^2\right],
\end{equation*}
where we chose $\Tilde{X}$ such that $\mathbb{E}\left[\Vert X-\Tilde{X}\Vert^2\right] = W_2^2(\mathcal{L}(X), \mathcal{L}(\tilde{X}))$ is minimised.

On the other hand, if the schedule satisfies $\max_{t\in[0,T]}\left\vert \frac{\partial_t{\lambda_t}}{\sqrt{\lambda_t(1-\lambda_t)}}\right\vert \leq C_\lambda$, the Wasserstein-2 distance between $\Tilde{\mu}_\lambda$ and $\Tilde{\mu}_{\lambda+\delta}$ is given by
\begin{align*}
     W_2^2(\Tilde{\mu}_\lambda, &\Tilde{\mu}_{\lambda+\delta}) \leq \mathbb{E}\left[\Vert X_\lambda-X_{\lambda + \delta}\Vert^2\right]\\
&=\mathbb{E}\left[\left\Vert(\sqrt{\lambda}-\sqrt{\lambda+\delta})X\right\Vert^2\right] + \mathbb{E}\left[\left\Vert\left(\sqrt{1-\lambda}-\sqrt{1-\lambda-\delta}\right)\sigma Z\right\Vert^2\right]\\
&= (\sqrt{\lambda}-\sqrt{\lambda+\delta})^2 \mathbb{E}\left[\Vert X\Vert^2\right] + \left(\sqrt{1-\lambda}-\sqrt{1-\lambda-\delta}\right)^2\sigma^2d.
\end{align*}
Using the definition of the metric derivative we have
\begin{align*}
    \left\vert\Dot{\Tilde{\mu}}\right\vert_\lambda^2 = \lim_{\delta\to 0}\frac{ W_2^2(\Tilde{\mu}_\lambda, \Tilde{\mu}_{\lambda+\delta})}{\delta^2} \leq \frac{\mathbb{E}\left[\Vert X\Vert^2\right]}{4\lambda} + \frac{\sigma^2d}{4(1-\lambda)}.
\end{align*}
Therefore, we have the following expression for the action
\begin{align}
    \mathcal{A}_{\lambda}(\mu) &= \int_0^T \vert\Dot{\mu}\vert_t^2 \md t = \int_0^T \left\vert\Dot{\Tilde{\mu}}\right\vert_\lambda^2\left\vert\partial_t{\lambda_t}\right\vert^2 \md t\nonumber\\
    &\lesssim \int_0^T \left(\frac{\mathbb{E}\left[\Vert X\Vert^2\right]}{4 \lambda_t} + \frac{\sigma^2}{4(1-\lambda_t)}d\right)\left\vert \partial_t{\lambda_t}\right\vert^2\md t \nonumber\\
    &= \int_0^T \left(\frac{\mathbb{E}\left[\Vert X\Vert^2\right]\sqrt{1-\lambda_t}}{4\sqrt{\lambda_t}} + \frac{\sigma^2\sqrt{\lambda_t}}{4\sqrt{1-\lambda_t}}d\right)\left\vert \frac{\partial_t{\lambda_t}}{\sqrt{\lambda_t(1-\lambda_t)}}\right\vert \left\vert \partial_t{\lambda_t}\right\vert\md t \nonumber\\    
    &\leq C_\lambda \int_0^T \left(\frac{\mathbb{E}\left[\Vert X\Vert^2\right]\sqrt{1-\lambda_t}}{4\sqrt{\lambda_t}} +\frac{\sigma^2\sqrt{\lambda_t}}{4\sqrt{1-\lambda_t}}d\right) \left\vert \partial_t{\lambda_t}\right\vert\md t \nonumber\\ 
    &\leq C_\lambda \int_{0}^{1} \left(\frac{\mathbb{E}\left[\Vert X\Vert^2\right]\sqrt{1-\lambda}}{4\sqrt{\lambda}} + \frac{\sigma^2\sqrt{\lambda}}{4\sqrt{1-\lambda}} d\right) \md \lambda \nonumber\\
    &\leq \frac{C_\lambda\pi}{8} \left(\mathbb{E}\left[\Vert X\Vert^2\right] + \sigma^2 d\right).\nonumber
\end{align}
\end{proof}