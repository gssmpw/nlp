\begin{proof}
By Lemma \ref{lemma:implications_between_assumptions}, we have that $\nabla V_\pi$ is Lipschitz continuous. Thus, it remains to show that $V_\pi$ is strongly convex outside of a ball.

Recall that $\Tilde{\pi}$ is supported on a compact set $S$, that is, $\Tilde{\pi}(y)=0$ for $y\notin S$ and by assumption $S$ is also convex.
For $x\in\mathbb{R}^d$, define the function
\begin{equation*}
    d(x) = \min_{y\in S} \Vert x-y\Vert,
\end{equation*}
which is well defined by compactness. 
Let $y^*(x)\in S$ be the unique (by convexity of $S$) point where the minimum distance is achieved, i.e., $y^*(x)
$ is the projector of $x$ onto $S$. Then, for every $y\in S$ it holds that
\begin{equation*}
    \Vert x-y\Vert\geq d(x),
\end{equation*}
with equality if and only if $y = y^*(x)$. 
Consider the convolution kernel $G\sim\gamma(x-y)$ defined as
\begin{equation*}
    \gamma(x-y)=\frac{1}{(2\pi\tau^2)^{d/2}}e^{-\frac{\Vert x-y\Vert^2}{2\tau^2}}.
\end{equation*}
Note that the value at $y = y^*(x)$ is given by
\begin{equation*}
    \gamma(x-y^*(x))=\frac{1}{(2\pi\tau^2)^{d/2}}e^{-\frac{d(x)^2}{2\tau^2}}.
\end{equation*}
Besides, for any $y\in S$, we have that
\begin{equation*}
    d(x)\leq\Vert x-y\Vert\leq d(x) + \Vert y^*(x)-y\Vert.
\end{equation*}
Because $S$ is a compact set, the term $\Vert y^*(x)-y\Vert$ for $y\in S$ is bounded independently of $x$, therefore we can write
\begin{equation*}
    \Vert x-y\Vert = d(x) + \delta_x(y),
\end{equation*}
for some $\delta_x(y)\geq 0$, with $\delta_x(y)$ if and only if $y=y^*(x)$ and $\delta_x(y)\leq\Vert y^*(x)-y\Vert$, which implies that $\delta(y)$ remains uniformly bounded for all $x\in\mathbb{R}^d$, in particular as $\Vert x\Vert\to\infty$.
Using this, the convolution kernel can be written as
\begin{equation*}
    \gamma(x-y) = \frac{1}{(2\pi\tau^2)^{d/2}}e^{-\frac{(d(x) + \delta_x(y))^2}{2\tau^2}}.
\end{equation*}
Thus, we obtain the ratio
\begin{equation*}
    \frac{\gamma(x-y)}{\gamma(x-y^*(x))} = e^{-\frac{2d(x)\delta_x(y) + \delta_x(y)^2}{2\tau^2}}.
\end{equation*}
Since $\delta_x(y)$ is uniformly bounded for all $x$, we observe that for $y\neq y^*(x)$, as $\Vert x\Vert\to\infty$ the leading order of the exponent is $d(x)\delta_x(y)$, where $\delta_x(y)>0$ and $d(x)$ grows with order $\Vert x\Vert$. 
Meaning that the ratio becomes arbitrarily small as $\Vert x\Vert \to\infty$ when $y\neq y^*(x)$.
That is, the contribution from $y\neq y^*(x)$ becomes exponentially negligible compared to the contribution from $y^*(x)$.

Given now a bounded test function $f$, we have
\begin{equation*}
    \frac{\int f(y)\gamma(x-y)\Tilde{\pi}(y)\md y}{\int\gamma(x-y)\Tilde{\pi}(y)\md y}  = \frac{\int f(y) e^{-\frac{d(x)\delta_x(y) + \delta_x(y)^2}{2\tau^2}}\gamma(x-y^*(x))\Tilde{\pi}(y)\md y}{\int e^{-\frac{d(x)\delta_x(y) + \delta_x(y)^2}{2\tau^2}}\gamma(x-y^*(x))\Tilde{\pi}(y)\md y}.
\end{equation*}
By the dominated convergence theorem, the contribution in both integrals for $y\neq y^*(x)$ vanishes as $\Vert x\Vert\to \infty$. Therefore, we have that for any fixed $\varepsilon>0$ and any small radius $\delta>0$, there exists $r$ such that for all $\Vert x\Vert>r$, the conditional measure $\rho_x(y)\propto \Tilde{\pi}(y)\gamma(x-y)$ satisfies
\begin{equation*}
    \rho_x\left(S\setminus B(y^*(x), \delta) \right)<\varepsilon,
\end{equation*}
where $B(y^*(x), \delta)$ denotes the ball of radius $\delta$ centred at $y^*(x)$. 
Intuitively, this means that for sufficiently large $\Vert x\Vert$, almost all the mass of $\rho_x$ is concentrated within an arbitrarily small ball around $y^*(x)$. 
It is important to note that, due to the assumption that $S$ is dense in $\mathbb{R}^d$, for any $\delta>0$, there always exist a point $z\in B(y^*(x), \delta)$ such that $\Tilde{\pi}(z)>0$.

Consequently, the mean $\mu_x = \int_S y\rho_x(\md y)$ must be very close to $y^*(x)$, and for any point $y$ in the high-probability region, we have $\Vert y-\mu(x)\Vert\leq 2\delta$ (with the worst-case scenario occurring when $\mu(x)$ lies on the edge of $B(y^*(x), \delta)$). This implies that for sufficiently large $\Vert x\Vert$, the spread of $\rho_x$ becomes arbitrarily small. In particular, the covariance matrix satisfies
\begin{equation*}
    \Vert\text{Cov}_{\rho_x}(Y)\Vert\leq (2\gamma)^2.
\end{equation*}
Taking the limit as $\Vert x\Vert \to\infty$, we have that $\delta\to0$, thus, we obtain that 
\begin{equation*}
    \lim_{\Vert x\Vert\to\infty}\text{Cov}_{\rho_x}(Y) = 0.
\end{equation*}
The final step is to note that 
\begin{equation*}
    \nabla^2 V_\pi(x) = \frac{1}{\tau^2} I -\frac{1}{\tau^4}\text{Cov}_{\rho_x}(Y).
\end{equation*}




\end{proof}