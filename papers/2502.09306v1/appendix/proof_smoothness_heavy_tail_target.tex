It is important to note that, in the proof of Lemma~\ref{lem:regularity_of_gaussian_path} under assumption \Cref{assumption:grad_log_lipschitzness_hessian_decay} below, we only rely on the lower bound of the Hessian from \Cref{assumption:grad_log_lipschitzness_hessian_decay}. If we omit the upper bound on the Hessian,  \Cref{assumption:grad_log_lipschitzness_hessian_decay} becomes a generalisation of \Cref{assumption:grad_log_lipschitzness_convexity_outside_of_a_ball}. The stronger assumption in \Cref{assumption:grad_log_lipschitzness_convexity_outside_of_a_ball} results in tighter bounds for the Lipschitz constants along the diffusion path, thereby improving upon those in \citet{gao2024gaussian}, as our bounds are non-vacuous for all $t\in[0, T]$.

\begin{proof}[Proof of Lemma~\ref{lem:regularity_of_gaussian_path} under assumption \Cref{assumption:grad_log_lipschitzness_hessian_decay}]
Let $\mu_t\propto e^{-U_t}$. 
Note that the expressions for the Hessian of $U_t$ given in \eqref{eq:hessian_conv_1}-\eqref{eq:hessian_conv_3} remain valid in this case. Consequently, the bounds provided in \eqref{eq:hessian_upper_bound_1}-\eqref{eq:hessian_upper_bound_2} also hold.

On the other hand, if $\rho_{t, x}(y)\propto e^{-V_\pi(y/\sqrt{\lambda_t})-\Vert x-y\Vert^2/(2\sigma^2(1-\lambda_t))}$ satisfies a Poincaré inequality with constant $C_{\text{PI}, \rho_t}$ independent of $x$, then the bounds \eqref{eq:hessian_final_bound_1}-\eqref{eq:lipschitzness_constant_moving_target} are valid. Therefore, to conclude that $\nabla\log\mu_t$ is Lipschitz we need to show that $C_{\text{PI}, \rho_t}$ is independent of $x$. Note that under assumption \textbf{A}\ref{assumption:grad_log_lipschitzness_hessian_decay}, we have that
\begin{equation*}
    \left(-\frac{L_\pi}{\lambda_t} + \frac{1}{\sigma^2(1-\lambda_t)}\right) I\preccurlyeq\nabla^2 V_{\rho_{t, x}}(y) \preccurlyeq \left(\frac{L_\pi}{\lambda_t} + \frac{1}{\sigma^2(1-\lambda_t)}\right) I,
\end{equation*}
and for $y$ such that $\Vert y\Vert>\sqrt{\lambda_t }r$
\begin{equation*}
    \left( - \frac{1}{\lambda_t\alpha_1 +\alpha_2 \Vert y\Vert^{2}}+\frac{1}{\sigma^2(1-\lambda_t)}\right) I \preccurlyeq \nabla^2 V_{\rho_{t, x}} (y)\preccurlyeq \left(\frac{1}{\lambda_t\beta_1 +\beta_2 \Vert y\Vert^{2}}+\frac{1}{\sigma^2(1-\lambda_t)}\right) I.
\end{equation*}
Note that if $1\geq\lambda_t>\frac{\sigma^2 L_\pi}{1+\sigma^2L_\pi}= \Tilde{\lambda}$, then $V_{\rho_{t, x}}$ is strongly convex and $\rho_{t, x}$ satisfies a Poincaré inequality with constant $\left(\frac{1}{\sigma^2(1-\lambda_t)}-\frac{L_\pi}{\lambda_t}\right)^{-1}$, which tends to $0$ as $\lambda_t$ tends to $1$. On the other hand, define 
\begin{equation*}
    \tilde{r}_t^2 = \max \{\lambda_t r^2, (2\sigma^2(1-\lambda_t)-\lambda_t\alpha_1)\alpha_2^{-1}\}.
\end{equation*}
We have that $V_{\rho_{t, x}}$ is strongly convex outside of a ball of radius $\Tilde{r}_t$. That is, for $\Vert y\Vert>\Tilde{r}_t$, it follows that $\nabla^2 V_{\rho_{t, x}}\succcurlyeq I/(2\sigma^2(1-\lambda_t))$. Therefore, as in Lemma~\ref{lemma:assumption_implies_LSI}, leveraging the existence of a smooth strongly convex approximation of $V_{\rho_{t, x}}$ and Holley-Stroock perturbation lemma \citep{RefWorks:RefID:85-holley1987logarithmic}, $\rho_{t, x}$ satisfies a Poincaré inequality with constant 
\begin{equation}
    C_{\text{PI}, \rho_t} \leq 2 \left(\frac{1}{2\sigma^2(1-\lambda_t)}\right)^{-1} e^{16\left(\frac{L_\pi}{\lambda_t} + \frac{1}{\sigma^2(1-\lambda_t)}\right)\Tilde{r}_t^2}, \label{eq:bound_C_poincare_inequality}
\end{equation}
independent of $x$. Therefore, for $\lambda_t\in(0, \Tilde{\lambda}]$ $C_{\text{PI}, \rho_t}$ is bounded by \eqref{eq:bound_C_poincare_inequality}, while for $\lambda_t\in(\Tilde{\lambda}, 1]$, $C_{\text{PI}, \rho_t}$ is bounded by
\begin{equation*}
    C_{\text{PI}, \rho_t}\leq \min\left\{\left(\frac{1}{\sigma^2(1-\lambda_t)}-\frac{L_\pi}{\lambda_t}\right)^{-1},  2 \left(\frac{1}{2\sigma^2(1-\lambda_t)}\right)^{-1} e^{16\left(\frac{L_\pi}{\lambda_t} + \frac{1}{\sigma^2(1-\lambda_t)}\right)\Tilde{r}_t^2}\right\}.
\end{equation*}
Finally, observe that if $\lambda_0=0$, then $\mu_0 = \nu = \mathcal{N}(0, \sigma^2 I)$, which implies that $\nabla\log\mu_0$ is Lipschitz continuous. 
This concludes that $\nabla\log\mu_t$ is Lipschitz continuous for all $t\in[0,T]$.
\end{proof}