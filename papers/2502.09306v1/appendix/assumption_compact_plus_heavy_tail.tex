\begin{lemma}\label{lemma:wegihted_pi}
    If $\Tilde{\pi}$ is supported in a closed Euclidean ball $B_d(0, R)$ and $\gamma\sim t(0, \tau^2 I, \alpha)$, then $\pi= \Tilde{\pi}*\gamma$ satisfies a weighted Poincaré inequality.
\end{lemma}
\begin{proof}
Recall that
\begin{equation*}
    \pi = \Tilde{\pi}*\gamma = \int_{\mathbb{R}^d} \gamma_{x}\md \Tilde{\pi}(x),
\end{equation*}
where $\gamma_x\sim t(x, \sigma^2I, \alpha)$.
Following \cite{functional_inequalities_compactly_supported_assumption}, the variance of a function $f\in L^2(\pi)$ can be decomposed as
\begin{equation*}
    \text{Var}_{\Tilde{\pi}*\gamma}(f) = \int_{\mathbb{R}^d} \text{Var}_{\gamma_x}(f)\md \Tilde{\pi}(x) + \text{Var}_{\Tilde{\pi}}\left(x\mapsto\int f \md \gamma_x\right):= A+B.
\end{equation*}
Since $\gamma_x$ satisfies a weighted Poincaré inequality with constant $C_{\text{PI}, \gamma}$ and weight function $\omega = 1 + \Vert x\Vert^2$ \citep[Proposition 2.17]{heavy_tail_functional_inequalities}, the first term $A$ is bounded by
\begin{equation*}
    A\leq C_{\text{PI}, \gamma}\int_{\mathbb{R}^d}\int_{\mathbb{R}^d}\vert\nabla f\vert^2 \omega \md \gamma_x\md\Tilde{\pi}(x) = C_{\text{PI}, \gamma}\int_{\mathbb{R}^d} \vert\nabla f\vert^2 \omega \md(\Tilde{\pi}*\gamma).
\end{equation*}
For the second term $B$, consider $g:x\mapsto\int_{\mathbb{R}^d} f \md \gamma_x$. Using this, $B$ can be rewritten as
\begin{equation*}
    B=\frac{1}{2} \int\int_{\mathbb{R}^d\times\mathbb{R}^d}(g(x)-g(y))^2\md\Tilde{\pi}(x)\md\Tilde{\pi}(y),
\end{equation*}
where using Cauchy-Schwartz inequality
\begin{equation*}
    (g(x)-g(y))^2 \leq \text{Var}_{\gamma_x}(f)\text{Var}_{\gamma_x}\left(1-\frac{\md \gamma
_y}{\md\gamma_x}\right).
\end{equation*}
For the first factor, we reapply the weighted Poincaré inequality for the $t$ distribution $\gamma_x$. The second factor is the $\chi^2$ divergence between the $t$ distributions $\gamma_x$ and $\gamma_y$.
\begin{align*}
    \chi^2(\gamma_x, \gamma_y)=& \int_{\mathbb{R}^d} \left(\frac{\gamma_x(z)}{\gamma_y(z)}-1\right)^2\gamma_y(z)\;\md z=  \int_{\mathbb{R}^d} \frac{\gamma_x(z)^2}{\gamma_y(z)}\;\md z-1
    =\int_{\mathbb{R}^d} \left(\frac{\alpha + (z-x)^{\intercal}\Sigma^{-1}(z-x)}{\alpha + (z-y)^{\intercal}\Sigma^{-1}(z-y)}\right)^{-(\alpha+ d)/2} \gamma_{x}(z)\md z -1 \\
    =& \int_{\mathbb{R}^d}\left(1+\frac{(z-y)^{\intercal}\Sigma^{-1}(z-y)-(z-x)^{\intercal}\Sigma^{-1}(z-x)}{\alpha + (z-x)^{\intercal}\Sigma^{-1}(z-x)}\right)^{(\alpha+d)/2}\gamma_x(z)\md z -1\\
     =& \int_{\mathbb{R}^d}\left(1+\frac{y^{\intercal}\Sigma^{-1}y-x^{\intercal}\Sigma^{-1}x-2z^{\intercal}\Sigma^{-1}(y-x)}{\alpha + (z-x)^{\intercal}\Sigma^{-1}(z-x)}\right)^{(\alpha+d)/2}\gamma_x(z)\md z -1\\
     =&  \int_{A=\{\Vert z-x\Vert^2 > 1\}}\left(1+\frac{y^{\intercal}\Sigma^{-1}y-x^{\intercal}\Sigma^{-1}x-2z^{\intercal}\Sigma^{-1}(y-x)}{\alpha + (z-x)^{\intercal}\Sigma^{-1}(z-x)}\right)^{(\alpha+d)/2}\gamma_x(z)\md z\\
    &+  \int_{\mathbb{R}^d\setminus A}\left(1+\frac{y^{\intercal}\Sigma^{-1}y-x^{\intercal}\Sigma^{-1}x-2z^{\intercal}\Sigma^{-1}(y-x)}{\alpha + (z-x)^{\intercal}\Sigma^{-1}(z-x)}\right)^{(\alpha+d)/2}\gamma_x(z)\md z-1.
\end{align*}
For $z\in \mathbb{R}^d$, the following holds 
\begin{equation*}
    (z-x)^{\intercal}\Sigma^{-1}(z-x) = (z-x)^{\intercal}U^{\intercal}S^{-1/2}S^{-1/2}U(z-x) = \Vert S^{-1/2}U(z-x)\Vert^2 \geq 0.
\end{equation*}
The second integral can be upper bounded as follows
\begin{align*}
    &\int_{\mathbb{R}^d\setminus A}\left(1+\frac{y^{\intercal}\Sigma^{-1}y-x^{\intercal}\Sigma^{-1}x-2z^{\intercal}\Sigma^{-1}(y-x)}{\alpha + (z-x)^{\intercal}\Sigma^{-1}(z-x)}\right)^{(\alpha+d)/2}\gamma_x(z)\md z \\
    & = \int_{\mathbb{R}^d\setminus A}\left(\frac{\alpha + (z-y)^{\intercal}\Sigma^{-1}(z-y)}{\alpha + (z-x)^{\intercal}\Sigma^{-1}(z-x)}\right)^{(\alpha+ d)/2} \gamma_{x}(z)\md z \leq \int_{\mathbb{R}^d\setminus A}\left(\frac{\alpha + \sigma_{\min}^{-1}\Vert z-y\Vert^2}{\alpha}\right)^{(\alpha+ d)/2} \gamma_{x}(z)\md z \\
    &\leq\int_{\mathbb{R}^d\setminus A} \left(\frac{\alpha + \sigma_{\min}^{-1}2(\Vert x-y\Vert^2 + \Vert z-x\Vert^2)}{\alpha}\right)^{(\alpha+ d)/2}  \gamma_{x}(z)\md z \leq \left(\frac{\alpha + \sigma_{\min}^{-1}2(4R^2 + 1)}{\alpha}\right)^{(\alpha+ d)/2} = \kappa_R,
\end{align*}
where we have used that $\Vert x-y\Vert$ is bounded by $2R$.
On the other hand, for $z\in A$ we have
\begin{align*}
    &1+\frac{y^{\intercal}\Sigma^{-1}y-x^{\intercal}\Sigma^{-1}x-2z^{\intercal}\Sigma^{-1}(y-x)}{\alpha + (z-x)^{\intercal}\Sigma^{-1}(z-x)} = 1+\frac{(y-x)^{\intercal}\Sigma^{-1}(y-x)-2(z-x)^{\intercal}\Sigma^{-1}(y-x)}{\alpha + (z-x)^{\intercal}\Sigma^{-1}(z-x)} \\
    & \leq 1 + \frac{(y-x)^{\intercal}\Sigma^{-1}(y-x)}{\alpha + (z-x)^{\intercal}\Sigma^{-1}(z-x)} + \frac{2\Vert (z-x)^{\intercal}\Sigma^{-1}(y-x)\Vert }{ (z-x)^{\intercal}\Sigma^{-1}(z-x)}\\
    & \leq 1 + \frac{(y-x)^{\intercal}\Sigma^{-1}(y-x)}{\alpha + (z-x)^{\intercal}\Sigma^{-1}(z-x)} + \frac{2\Vert S^{-1/2}U(z-x)\Vert\Vert S^{-1/2}U(y-x)\Vert }{ \Vert S^{-1/2}U(z-x)\Vert^2}\\
    & = 1 + \frac{(y-x)^{\intercal}\Sigma^{-1}(y-x)}{\alpha + (z-x)^{\intercal}\Sigma^{-1}(z-x)} + \frac{2\Vert S^{-1/2}U(y-x)\Vert }{ \Vert S^{-1/2}U(z-x)\Vert} \leq 1 + \frac{4R^2 \Vert S^{-1/2}\Vert^2}{\alpha} + \frac{4R\Vert S^{-1/2}\Vert }{\sigma_{\max}^{-1/2}}\leq 1 + \frac{4R^2}{\alpha\sigma_{\min}} + \frac{4R\sigma_{\max}^{1/2}}{\sigma_{\min}^{1/2}}.
\end{align*}
Therefore, we obtain
\begin{equation*}
    \int_{A=\{\Vert z-x\Vert^2 > 1\}}\left(1+\frac{y^{\intercal}\Sigma^{-1}y-x^{\intercal}\Sigma^{-1}x-2z^{\intercal}\Sigma^{-1}(y-x)}{\alpha + (z-x)^{\intercal}\Sigma^{-1}(z-x)}\right)^{(\alpha+d)/2}\gamma_x(z)\md z\leq \left( 1 + \frac{4R^2}{\alpha\sigma_{\min}} + \frac{4R\sigma_{\max}^{1/2}}{\sigma_{\min}^{1/2}}\right)^{(\alpha+d)/2} = \beta_R
\end{equation*}
This shows that $\chi^2(\gamma_x, \gamma_y)$ is upper bounded with dependence on $R$ of the form $R^{(\alpha+d)}$. In particular, $B$ satisfies the following bound
\begin{equation*}
    B\leq  C_{\text{PI}, \gamma} (\kappa_R + \beta_R -1)\int_{\mathbb{R}^d} \vert\nabla f\vert^2 \omega \md(\Tilde{\pi}*\gamma).
\end{equation*}
Therefore, the measure $\Tilde{\pi}*\gamma$ satisfies a weighted Poincaré inequality with constant 
\begin{equation*}
    C_{\text{PI}} \leq C_{\text{PI}, \gamma} (\kappa_R + \beta_R).
\end{equation*}

\end{proof}