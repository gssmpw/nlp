\begin{proof}[Proof of Lemma~\ref{lem:regularity_of_gaussian_path} under assumption \Cref{assumption:grad_log_lipschitzness_convexity_outside_of_a_ball}]
To simplify notation in the proof, we denote $\pi_{\text{data}}$ as $\pi$.
    Let $\mu_t\propto e^{-U_t}$, our aim is to show that $\nabla U_t$ is Lipschitz, to do so we are going to show that the Hessian $\nabla^2 U_t$ is bounded for all $t\in[0, T]$. 
    Using that if $f$ and $g$ are differentiable functions then $\nabla(f*g) = (\nabla f)* g = f * (\nabla g)$, we have the following expressions for $\nabla U_t$ and $\nabla^2 U_t$
\begin{align}
    \nabla U_t(x) &= \frac{1}{\sigma^2(1-\lambda_t)}\left(x-\mathbb{E}_{\rho_{t, x}(y)}\left[Y\right]\right) = \frac{1}{\sqrt{\lambda_t}}\mathbb{E}_{\rho_{t, x}(y)}\left[\nabla V_{\pi}\left(\frac{Y}{\sqrt{\lambda_t}}\right)\right]\label{eq:score_conv_expressions} \\
\nabla^2 U_t(x) &=\frac{1}{\sigma^2(1-\lambda_t)}\left(I - \frac{1}{\sigma^2(1-\lambda_t)}\text{Cov}_{\rho_{t, x}}\left[Y\right]\right)\label{eq:hessian_conv_1}\\
\nabla^2 U_t(x) &=\frac{1}{\lambda_t}\left(\mathbb{E}_{\rho_{t, x}}\left[\nabla^2 V_\pi\left(\frac{Y}{\sqrt{\lambda_t}}\right)\right] - \text{Cov}_{\rho_{t, x}}\left[\nabla V_\pi\left(\frac{Y}{\sqrt{\lambda_t}}\right)\right]\right)\label{eq:hessian_conv_2}\\
    \nabla^2 U_t(x) &=\frac{1}{\sigma^2(1-\lambda_t)\lambda_t}\text{Cov}_{\rho_{t, x}}\left[Y, \nabla V_\pi\left(\frac{Y}{\sqrt{\lambda_t}}\right)\right],\label{eq:hessian_conv_3}
\end{align}
where $\rho_{t, x}(y)\propto e^{-V_\pi(y/\sqrt{\lambda_t})-\Vert x-y\Vert^2/(2\sigma^2(1-\lambda_t))}$.
It is worth mentioning that as $t\to 0$, $\rho_{t, x}$ tends to a Dirac delta centred at $0$, and as $t\to T$, it approaches a Dirac delta centred at $x$, both of which have zero variance.
Note that \eqref{eq:hessian_conv_1} and \eqref{eq:hessian_conv_2} admit the following upper bounds
\begin{align}
    \nabla^2 U_t(x) &\preccurlyeq\frac{1}{\sigma^2(1-\lambda_t)}I \label{eq:hessian_upper_bound_1}\\
\nabla^2 U_t(x) &\preccurlyeq\frac{L_\pi}{\lambda_t} I,\label{eq:hessian_upper_bound_2}
\end{align}
where we have used that the covariance matrix is positive semidefinite.
To find a lower bound for $\nabla^2 U_t$ we need to upper bound $\text{Cov}_{\rho_{t, x}}\left[Y\right]$ and $\text{Cov}_{\rho_{t, x}}\left[\nabla V_\pi(Y/\sqrt{\lambda_t})\right]$. Observe that if $\rho_{t, x}$ satisfies the Poincaré inequality with constant $C_{\text{PI}, \rho_{t}}$ independent of $x$, then using the generalisation of the Poincaré inequality for vector-valued random variables given in Lemma~\ref{lemma:PI_for_vector_valued_functions}, we have
\begin{align*}
    \text{Cov}_{\rho_{t, x}}&\left[Y\right]\preccurlyeq C_{\text{PI}, \rho_{t}} I\\
    \text{Cov}_{\rho_{t, x}}&\left[\nabla V_\pi\left(\frac{Y}{\sqrt{\lambda_t}}\right)\right]\preccurlyeq \frac{C_{\text{PI}, \rho_t}}{\lambda_t} \mathbb{E}_{\rho_{t, x}}\left[\nabla^2 V_\pi\left(\frac{Y}{\sqrt{\lambda_t}}\right)\nabla^2 V_\pi\left(\frac{Y}{\sqrt{\lambda_t}}\right)^{\intercal}\right]\preccurlyeq \frac{C_{\text{PI}, \rho_t} L_\pi^2}{\lambda_t} I.
\end{align*}
This implies that for each $t$ we have that 
\begin{align}
    \nabla^2 U_t(x) &\preccurlyeq \min\left\{\frac{1}{\sigma^2(1-\lambda_t)}, \frac{L_\pi}{\lambda_t}\right\} I =: a_t I \label{eq:hessian_final_bound_1}\\
    \nabla^2 U_t(x) &\succcurlyeq \max\left\{\frac{1}{\sigma^2(1-\lambda_t)}\left(1 - \frac{C_{\text{PI}, \rho_t}}{\sigma^2(1-\lambda_t)}\right), -\frac{L_\pi}{\lambda_t}\left(1 + \frac{C_{\text{PI}, \rho_t} L_\pi}{\lambda_t}\right)\right\} I =: b_t I.\label{eq:hessian_final_bound_2}
\end{align}
Therefore, the Lipschitz constant $L_t$ satisfies the following
\begin{equation}\label{eq:lipschitzness_constant_moving_target}
    L_t \leq \max\{a_t, \vert b_t\vert \}.
\end{equation}
To conclude the proof we need to check that $C_{\text{PI}, \rho_t}$ is independent of $x$, since otherwise the Poincaré constant can get arbitrarily large and we will not have a meaningful bound for the Hessian.
Note that if we denote $\rho_{t, x}\propto e^{-V_{\rho_{t, x}}}$ we have that 
\begin{equation}\label{eq:hessian_bound_joint_1}
    \left(-\frac{L_\pi}{\lambda_t} + \frac{1}{\sigma^2(1-\lambda_t)}\right) I\preccurlyeq\nabla^2 V_{\rho_{t, x}}(y) = \frac{1}{\lambda_t} \nabla^2 V_{\pi}\left(\frac{y}{\sqrt{\lambda_t}}\right) + \frac{I}{\sigma^2(1-\lambda_t)}\preccurlyeq \left(\frac{L_\pi}{\lambda_t} + \frac{1}{\sigma^2(1-\lambda_t)}\right) I.
\end{equation} 

Note that if $\left(-\frac{L_\pi}{\lambda_t}+\frac{1}{\sigma^2(1-\lambda_t)}\right)>0$, then $V_{\rho_{t, x}}$ is strongly convex and using Bakry-Émery criterion \citep{bakry_emery} we have that $\rho_{t, x}$ satisfies a log-Sobolev inequality which implies a Poincaré inequality with the same constant. Thus, for $t$ such that 
\begin{equation*}
    \Tilde{\lambda} = \frac{\sigma^2 L_\pi}{1+\sigma^2L_\pi} < \lambda_t \leq 1,
\end{equation*}
$\rho_{t, x}$ satisfies the Poincaré inequality with constant $\left(\frac{1}{\sigma^2(1-\lambda_t)}-\frac{L_\pi}{\lambda_t}\right)^{-1}$ independent of $x$, which tends to $0$ as $\lambda_t$ tends to $1$. 


On the other hand, using that the potential $V_\pi$ is strongly convex outside of a ball of radius $r$, we have that for $\Vert y\Vert >\sqrt{\lambda_t} r$
\begin{equation}\label{eq:hessian_boudn_joint_2}
   \left( \frac{M_{\pi}}{\lambda_t} + \frac{1}{\sigma^2(1-\lambda_t)} \right) I \preccurlyeq \nabla^2 V_{\rho_{t, x}} (y) \preccurlyeq \left(\frac{L_\pi}{\lambda_t} + \frac{1}{\sigma^2(1-\lambda_t)}\right) I. 
\end{equation}
Equations \eqref{eq:hessian_bound_joint_1}-\eqref{eq:hessian_boudn_joint_2} imply that for $\lambda_t>0$, $\nabla\log \rho_{t, x}$ is Lipschitz continuous and $\rho_{t, x}$ is strongly log-concave outside of a ball of radius $r_t = \sqrt{\lambda_t} r$. 
Similarly to the proof in Lemma \ref{lemma:assumption_implies_LSI}, the existence of a smooth strongly convex approximation of $V_{\rho_{t, x}}$ and the Holley-Stroock perturbation lemma \citep{RefWorks:RefID:85-holley1987logarithmic} imply that $\rho_{t, x}$ satisfies a log-Sobolev inequality and hence a Poincaré inequality with constant
\begin{equation}\label{eq:poincare_inequality_joint}
    C_{\text{PI}, \rho_t} \leq 2\left(\frac{M_{\pi}}{\lambda_t}+ \frac{1}{\sigma^2(1-\lambda_t)} \right)^{-1} e^{16 \left(L_\pi + \frac{\lambda_t}{\sigma^2(1-\lambda_t)}\right) r^2},
\end{equation}
independent of $x$. Observe that when $\lambda_t$ tends to $0$ the upper bound of the Poincaré constant $C_{\text{PI}, \rho_t}$ also tends to $0$. Therefore, for $\lambda_t\in[0, \Tilde{\lambda}]$ $C_{\text{PI}, \rho_t}$ is bounded by \eqref{eq:poincare_inequality_joint}, while for $\lambda_t\in(\Tilde{\lambda}, 1]$, $C_{\text{PI}, \rho_t}$ is bounded by 
\begin{equation*}
    C_{\text{PI}, \rho_t} \leq \min\left\{\left(\frac{1}{\sigma^2(1-\lambda_t)}-\frac{L_\pi}{\lambda_t}\right)^{-1}, 2\left(\frac{M_{\pi}}{\lambda_t}+ \frac{1}{\sigma^2(1-\lambda_t)} \right)^{-1} e^{16 \left(L_\pi + \frac{\lambda_t}{\sigma^2(1-\lambda_t)}\right) r^2}\right\}.
\end{equation*}
\end{proof}