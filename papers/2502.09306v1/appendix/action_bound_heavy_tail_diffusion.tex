\begin{proof}
Consider the reparametrised version of $\mu_t$ in terms of the schedule $\lambda_t$, denoted as $\Tilde{\mu}_\lambda$ and let $X_{\lambda}\sim\Tilde{\mu}_{\lambda}$ and $X_{\lambda + \delta}\sim\Tilde{\mu}_{\lambda + \delta}$.
Recall that 
\begin{equation*}
    X_{\lambda} = \sqrt{\lambda} X + \sqrt{1-\lambda}\sigma Z
\end{equation*}
where $X\sim\pid$ and $Z\sim t(0, I, \alpha)$.  The Wasserstein-2 distance between $\Tilde{\mu}_\lambda$ and $\Tilde{\mu}_{\lambda+\delta}$ is given by
\begin{align*}
     W_2^2(\Tilde{\mu}_\lambda, &\Tilde{\mu}_{\lambda+\delta}) \leq \mathbb{E}\left[\Vert X_\lambda-X_{\lambda + \delta}\Vert^2\right]\\
&=\mathbb{E}\left[\left\Vert(\sqrt{\lambda}-\sqrt{\lambda+\delta})X\right\Vert^2\right] + \mathbb{E}\left[\left\Vert\left(\sqrt{1-\lambda}-\sqrt{1-\lambda-\delta}\right)\sigma Z\right\Vert^2\right]\\
&= (\sqrt{\lambda}-\sqrt{\lambda+\delta})^2 \mathbb{E}\left[\Vert X\Vert^2\right] + \left(\sqrt{1-\lambda}-\sqrt{1-\lambda-\delta}\right)^2 \frac{\sigma^2d\alpha}{\alpha -2}.
\end{align*}
Using the definition of the metric derivative we have
\begin{align*}
    \left\vert\Dot{\Tilde{\mu}}\right\vert_\lambda^2 = \lim_{\delta\to 0}\frac{ W_2^2(\Tilde{\mu}_\lambda, \Tilde{\mu}_{\lambda+\delta})}{\delta^2} \leq \frac{\mathbb{E}\left[\Vert X\Vert^2\right]}{4\lambda} + \frac{1}{4(1-\lambda)}\frac{\sigma^2 d\alpha}{\alpha-2}.
\end{align*}
Since $\mu_t = \Tilde{\mu}_{\lambda_t}$, we have that $\vert\Dot{\mu}\vert_t = \left\vert\Dot{\Tilde{\mu}}\right\vert_\lambda\left\vert\partial_t{\lambda_t}\right\vert$. Using assumption \Cref{assumption:schedule_form_heavy_tail_diffusion} for the schedule, we have the following expression for the action
\begin{align}
    \mathcal{A}_{\lambda}(\mu) &= \int_0^T \vert\Dot{\mu}\vert_t^2 \md t = \int_0^T \left\vert\Dot{\Tilde{\mu}}\right\vert_\lambda^2\left\vert\partial_t{\lambda_t}\right\vert^2 \md t\nonumber\\
    &\lesssim \int_0^T \left(\frac{\mathbb{E}\left[\Vert X\Vert^2\right]}{4 \lambda_t} + \frac{\sigma^2\alpha}{4(1-\lambda_t)(\alpha-2)}d\right)\left\vert \partial_t{\lambda_t}\right\vert^2\md t \nonumber\\
    &= \int_0^T \left(\frac{\mathbb{E}\left[\Vert X\Vert^2\right]\sqrt{1-\lambda_t}}{4\sqrt{\lambda_t}} + \frac{\sigma^2\alpha\sqrt{\lambda_t}}{4\sqrt{1-\lambda_t}(\alpha-2)}d\right)\left\vert \frac{\partial_t{\lambda_t}}{\sqrt{\lambda_t(1-\lambda_t)}}\right\vert \left\vert \partial_t{\lambda_t}\right\vert\md t \nonumber\\    
    &\leq C_\lambda \int_0^T \left(\frac{\mathbb{E}\left[\Vert X\Vert^2\right]\sqrt{1-\lambda_t}}{4\sqrt{\lambda_t}} +\frac{\sigma^2\alpha\sqrt{\lambda_t}}{4\sqrt{1-\lambda_t}(\alpha-2)}d\right) \left\vert \partial_t{\lambda_t}\right\vert\md t \nonumber\\ 
    &\leq C_\lambda \int_{0}^{1} \left(\frac{\mathbb{E}\left[\Vert X\Vert^2\right]\sqrt{1-\lambda}}{4\sqrt{\lambda}} + \frac{\sigma^2\alpha\sqrt{\lambda}}{4\sqrt{1-\lambda}(\alpha-2)} d\right) \md \lambda \nonumber\\
    &\leq \frac{C_\lambda\pi}{8} \left(\mathbb{E}\left[\Vert X\Vert^2\right] + \frac{\sigma^2 d\alpha}{\alpha-2}\right).\nonumber
\end{align}
\end{proof}

