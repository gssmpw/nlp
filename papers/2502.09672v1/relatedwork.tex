\section{Related Work}
\textbf{3D Multi-Object Tracking.} The ongoing advancements in 3D object detection \cite{chenLargeKernel3DScalingKernels2023,yinCenterbased3DObject2021,xiePolyPCPolyhedralNetwork2023} and 2D MOT \cite{wojkeSimpleOnlineRealtime2017,zhang2022bytetrack} technologies have significantly propelled innovation and progress in the field of 3D MOT. AB3DMOT \cite{weng3DMultiObjectTracking2020} first employs a Kalman filter for prediction and uses “Intersection over Union” (IoU) association, while also introducing a new evaluation tool for 3D MOT. SimpleTrack \cite{pangSimpleTrackUnderstandingRethinking2021} modularizes the TBD-based 3D MOT method into four components and proposes a two-stage association strategy. To address the issue of multi-class targets during tracking, Poly-MOT \cite{liPolyMOTPolyhedralFramework2023} introduces the first multi-class tracking framework, incorporating geometric constraints into the motion model. Building on Poly-MOT, Fast-Poly \cite{240313443FastPolyFast} introduces the concept of A-GIoU and a lightweight filter, which improves the FPS. Although single-modal methods rely on a single data source, they tend to have relatively simple systems and low computational costs. Moreover, recent single-modal methods like Poly-MOT and Fast-Poly have outperformed many multi-modal approaches, highlighting their significant potential for further development.

\textbf{Preprocess.} Preprocessing mainly addresses two issues: (1) Duplicate detections for individual objects: Non-Maximum Suppression (NMS) removes duplicate detections based on the overlap between bounding boxes, with methods such as  Soft-NMS \cite{bodlaSoftNMSImprovingObject2017} and Adaptive-NMS \cite{liuAdaptiveNMSRefining2019} being commonly used. (2) Low-quality false detections: Score Filter (SF) filters out low-confidence detections \cite{liPolyMOTPolyhedralFramework2023}, but selecting the appropriate SF threshold is challenging as it is difficult to control the ratio of true positives (TP) to false positives (FP) in the detection process. To address these challenges, a Distance-Based Score Enhancement (DBSE) mechanism is proposed, leveraging distance to enhance the expressiveness of point cloud detection scores and thereby improving the filtering performance of SF.

\textbf{Prediction.} In the prediction phase, the target's position, rotation angle, and other motion characteristics for the next frame are predicted using either  filter-based methods or  learning-based methods \cite{weng3DMultiObjectTracking2020}. Filter-based methods typically employ filters along with corresponding motion models to adapt to the target's motion patterns and predict its future state. There are two main issues: (1) Different types of targets exhibit distinct motion patterns. (2) The motion patterns of the same target may vary at different stages of its movement. To tackle the first issue, Poly-MOT introduces a classification-based multi-model filter, which effectively predicts and tracks targets of different categories \cite{240313443FastPolyFast,liPolyMOTPolyhedralFramework2023}. The existence of issue (2) leads to model mismatch in tracking the motion of a single target.
Fig. \ref{fig1} illustrates the continuous process of a car making a left turn. Throughout this process, the target transitions through three distinct motion models, with two instances of model switching. It is evident that using a single motion model for tracking results in delays in model adaptation when the target's motion pattern changes. To overcome this limitation, we introduce an interacting multiple model (IMM) algorithm for tracking prediction, which can simultaneously address both the variation in motion patterns across different targets and the changes in motion behavior within a single target's trajectory.

\textbf{Trajectory Manager.} The trajectory management module plays a crucial role in determining the creation and termination of trajectories, which directly influences the number of FP and false negatives (FN). In count-based methods \cite{liPolyMOTPolyhedralFramework2023,benbarkaScoreRefinementConfidencebased2021}, the validity of a trajectory is assessed based on the number of consecutive associations or non-associations. However, this approach heavily relies on parameter settings, and the criteria for validity are relatively simplistic.Another method involves iteratively calculating trajectory confidence and comparing it to a threshold to decide whether a trajectory should be initiated or terminated \cite{240313443FastPolyFast,benbarkaScoreRefinementConfidencebased2021}. However, this approach can easily result in the termination of low-confidence true targets, especially those at the detection threshold, due to the natural decay of trajectory scores. To address this issue, we propose a damping window-based trajectory management method, which incorporates the trajectory's association status and uses a damping window score to evaluate the trajectory's validity. This approach helps mitigate the risk of overlooking low-confidence true detections.