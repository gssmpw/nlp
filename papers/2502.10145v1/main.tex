\documentclass[lettersize,journal]{IEEEtran}
\usepackage{amsmath,amsfonts}
\usepackage{algorithmic}
\usepackage{algorithm}
\usepackage{array}
\usepackage[caption=false,font=normalsize,labelfont=sf,textfont=sf]{subfig}
\usepackage{textcomp}
\usepackage{stfloats}
\usepackage{url}
\usepackage{verbatim}
\usepackage{graphicx}
\usepackage{cite}
\usepackage{afterpage}
\hyphenation{op-tical net-works semi-conduc-tor IEEE-Xplore}
% updated with editorial comments 8/9/2021

% MY PACKAGEs
\usepackage{graphicx}
\usepackage{amsmath}
\usepackage{amssymb}
\usepackage{booktabs}
\usepackage{svg}
\usepackage{tabularx}
\usepackage{multirow}
\usepackage{ragged2e}
\usepackage{lipsum}
\usepackage[numbers,sort,compress]{natbib}

% end of MY PACKAGEs



\begin{document}

% \linenumbers
\title{Interpretable Concept-based Deep Learning Framework for Multimodal Human Behavior Modeling}

% \author{Xinyu Li,~\IEEEmembership{Student Member,~IEEE,} Marwa Mahmoud,~\IEEEmembership{Member,~IEEE}
\author{\parbox{16cm}{\centering
   {\large Xinyu Li and Marwa Mahmoud }\\
   {\normalsize
   School of Computing Science, University of Glasgow, United Kingdom\\}}
}
        % <-this % stops a space
% \thanks{This paper was produced by the IEEE Publication Technology Group. They are in Piscataway, NJ.}% <-this % stops a space
% \thanks{Manuscript received April 19, 2021; revised August 16, 2021.}
% }

% The paper headers
% \markboth{Journal of \LaTeX\ Class Files,~Vol.~14, No.~8, August~2021}%
% {Shell \MakeLowercase{\textit{et al.}}: A Sample Article Using IEEEtran.cls for IEEE Journals}

% \IEEEpubid{0000--0000/00\$00.00~\copyright~2021 IEEE}
% Remember, if you use this you must call \IEEEpubidadjcol in the second
% column for its text to clear the IEEEpubid mark.

\maketitle

\begin{abstract}

In the contemporary era of intelligent connectivity, Affective Computing (AC), which enables systems to recognize, interpret, and respond to human behavior states, has become an integrated part of many AI systems. As one of the most critical components of responsible AI and trustworthiness in all human-centered systems, explainability has been a major concern in AC. Particularly, the recently released EU General Data Protection Regulation requires any high-risk AI systems to be sufficiently interpretable, including biometric-based systems and emotion recognition systems widely used in the affective computing field. Existing explainable methods often compromise between interpretability and performance. Most of them focus only on highlighting key network parameters without offering meaningful, domain-specific explanations to the stakeholders. Additionally, they also face challenges in effectively co-learning and explaining insights from multimodal data sources. To address these limitations, we propose a novel and generalizable framework, namely the Attention-Guided Concept Model (AGCM), which provides learnable conceptual explanations by identifying \textit{what} concepts that lead to the predictions and \textit{where} they are observed. AGCM is extendable to any spatial and temporal signals through multimodal concept alignment and co-learning, empowering stakeholders with deeper insights into the model's decision-making process. We validate the efficiency of AGCM on well-established Facial Expression Recognition benchmark datasets while also demonstrating its generalizability on more complex real-world human behavior understanding applications. 
We believe that AGCM’s flexibility and extensibility lay a solid foundation for developing future interpretable and trustworthy models in downstream affective computing applications, including in mental health, psychiatry, education, automotive, and security, offering both competitive performance and domain-specific explanations.


\end{abstract}

\begin{IEEEkeywords}
Explainable AI, multimodal learning, affective computing, facial expression recognition, human-human interaction
\end{IEEEkeywords}

\section{Introduction}
    % \IEEEPARstart{E}{xplainability}
    Affective Computing (AC) aims to develop models and systems that recognize, interpret, and respond to human behavior states. As a human-centered design, explainability and transparency have become critical concerns in AC applications \cite{cortinas2023toward}. The EU AI Act \cite{hupont2022landscape} and the newly proposed General Data Protection Regulation (GDPR) in 2024 \cite{GDPR2024} mandates that high-risk AI systems, including biometric-based systems and emotion recognition systems widely used in the affective computing field, must be sufficiently transparent to allow stakeholders from cross-disciplinary area to comprehend the decision-making process of the framework. Enhancing explainability in AC models not only offers extra insights into AI predictions but also ensures fair, trustworthy, and accountable outcomes in sensitive applications like education, healthcare, and security systems. \cite{yu2024bridging, kumar2023opacity}.

    \begin{figure}[t]
    \centering
    \includegraphics[width=0.99\columnwidth]{fig_1.pdf}
       \caption{Difference between the black-box models, current eXplainable AI (XAI), and our proposed model. (a): Black-box ML models offer no extra insight into the model prediction. (b): Map-based XAI approaches offer explanations by identifying important regions that lead to the prediction, but without any domain-specific knowledge that validates the decision-making process. (c): Our proposed framework explicitly localizes domain-specific indicators, learns their contributions during training, and incorporates multimodal concepts, thereby making predictions based on these intermediate attributes in an inherently interpretable manner.}
    \label{fig_intro}
    \end{figure}

    There is an increasing interest in developing interpretable or eXplainable Artificial Intelligence (XAI) to improve model transparency in AC. As shown in Fig. \ref{fig_intro} (b), approaches such as post-hoc explanations \cite{ribeiro2016should,heimerl2020unraveling,malik2021towards} and map-based methods \cite{gao2021ts,gund2021interpretable,belharbi2024guided} have emerged to address this need. However, these techniques primarily focus on identifying important regions or parameters within deep neural networks, rather than providing an explicit, causal explanation for the predictions. This limitation is especially pronounced in AC, where opposing facial Action Units, like AU12 (Lip Corner Puller) associated with positive emotion and AU15 (Lip Corner Depressor) linked to negative emotion, can occur in the same facial region. Meanwhile, the alignment and co-learning from multimodal sources pose even greater challenges for these approaches due to the inherently different properties of multimodal knowledge. Therefore, they often face a trade-off between performance and interpretability, which, in high-risk XAI, may undermine the system's trustworthiness \cite{rudin2019stop}.

    Consider a common question: How would a human expert explain their prediction of an individual with highly conversational engagement? They would likely point to the activation of specific facial muscles, such as the zygomatic major, indicating engaged smiles, a strong positive indicator of engagement. Meanwhile, the forward gaze direction, proper gesture, body language, and audio indicators can also be used to recognize engagement. Thus, a good explanation from an AC model should address two key aspects: \textit{what} indicators or concepts (e.g., facial muscle activations) contribute to the prediction, and \textit{where} these concepts are observed. Furthermore, the importance of multimodal learning is self-evident in real-world AC applications \cite{abaeikoupaei2020multi, yoon2022d, cafaro2017noxi}. Training and interpreting AC models with multimodal alignment and co-learning is another key challenge in affective XAI \cite{baltruvsaitis2018multimodal}.

    As shown in Fig. \ref{fig_intro} (c), in this paper, we propose an interpretable concept-based framework: the Attention-Guided Concept Model (AGCM), which localizes and learns the key indicators during training and then makes the final prediction according to the contribution of these intermediate concepts. This framework incorporates spatial concept information and multimodal concept fusion within a powerful attention-based architecture, combining the advantage of both domain-specific explanation and state-of-the-art performance. In summary, the main contributions of this paper are as follows:

    \begin{enumerate}
    \item We propose a concept-based interpretable framework for AC applications, namely the Attention-Guided Concept Model (AGCM), which provides both learnable multimodal conceptual explanations and spatial visual concept localization, quantifying the contribution of individual concepts to the predicted affective label.
    \item To address the challenge of multimodal concept alignment and co-learning, AGCM introduces an extendable sequential multimodal concept fusion, which can be easily expanded to any spatial-temporal signal. This approach accounts for temporal and contextual information between input modalities, demonstrating the adaptability to other discrete or continuous signals.
    \item We qualitatively and quantitatively evaluate the proposed framework on three large-scale FER datasets: RAF-DB, AffectNet, and Aff-Wild2, demonstrating that AGCM outperforms previous interpretable models and achieves competitive performance compared to state-of-the-art black-box models. Moreover, the experiment shows that AGCM offers a human-interpretable explanation grounded in domain-specific knowledge.
    \item To demonstrate the generalizability of AGCM on complex real-world AC applications, we conduct extensive experiments on the human-human interaction dataset, validating its ability to provide explainable and accurate prediction in downstream AC applications. We provide a video demonstration in the supplementary material to offer additional insights into the prediction process and its explainability.
    \end{enumerate}

\section{Related Work}

In this section, we examine two primary machine learning approaches commonly used in affective computing: feature-based models and end-to-end models. We then discuss recent advancements in explainable affective computing, emphasizing their contribution and limitation to model transparency and interpretability.

\subsection{Feature and End-to-end Models in Affective Computing}
    Discriminative AC focuses on mapping human-centered data to emotion-related labels, employing two primary approaches: feature-based models and end-to-end models.
    
    Feature-based models \cite{tsalera2022feature, avola2022affective} rely on manually extracted features derived from raw data, which are then used to train machine learning models to establish the relationship between features and labels. The strength of this approach lies in the interpretability of the features, which are often human-understandable and can provide valuable behavioral insights \cite{bento2022comparing}. Additionally, feature-based models typically operate on structured, tabular data, offering a computationally efficient solution \cite{bisogni2023emotion}. However, the reliance on handcrafted features may omit potentially important information embedded in the raw data, causing inevitable information loss \cite{zhao2019affective, cortinas2023toward}. Furthermore, decoupling feature extraction from model training may introduce limitations, such as overfitting, particularly due to the structured nature of the input data \cite{bengio2013representation}.
    
    End-to-end models \cite{li2020deep}, on the other hand, learn directly from raw data, eliminating the need for manual feature engineering. Fully leveraging the representational power of deep neural networks, these models are particularly effective when trained on large datasets. However, their strength is also their weakness: the opacity of their learned representations often leads to what is referred to as the ``black-box'' problem, making these models difficult to interpret as they lack human-understandable intermediate representations \cite{zhao2019affective}. 
    This challenge persists in multi-task learning, where models are designed to predict multiple task labels simultaneously, such as emotion and AUs. Despite their multi-task design, emotion and AU predictions are learned independently, leaving the model as a black box, where the predicted AUs cannot explain the predicted emotions.

    As shown in Fig. \ref{fig_ven}, in this work, we propose a hybrid approach, integrating the strengths of the well-understandable feature-based model and the state-of-the-art black-box models through concept-based learning, where each concept serves as an embedded neural representation of the feature. This approach retains the interpretability inherent in feature-based models while harnessing the robust learning capabilities of end-to-end neural networks. 

    \begin{figure}[t]
    \centering
    \includegraphics[width=0.7\columnwidth]{fig_2.pdf}
       \caption{Feature-based approaches offer inherent interpretability and are easily understood by humans, while end-to-end models deliver state-of-the-art learning capabilities. This work seeks to integrate the strengths of both methods through a concept-based framework, which achieves a balance between high explainability and robust performance. Unlike traditional features, concepts are not static values. They serve as the neural embeddings of features that are trainable within the ML framework, spontaneously quantifying the contribution of individual concepts to the task label. }
    \label{fig_ven}
    \end{figure}
    

\subsection{XAI in Affective Computing} \label{xai_in_ac}
    Recent efforts to enhance the explainability of affective computing models have largely relied on post-hoc, map-based visualizations, and concept-based learning.

    Post-hoc approaches \cite{ribeiro2016should,heimerl2020unraveling,malik2021towards, Wu2019Enhancing} retrospectively analyze the parameter importance of pre-trained black-box models after deployment. These methods attempt to explain the model by manipulating parameters in specific parts of the network to check their impact on the final prediction. Map-based approaches \cite{gund2021interpretable,belharbi2024guided} are another common method used to interpret black-box models, typically highlighting the regions where the model focuses its attention. However, both of these approaches primarily focus on the importance scores within the neural network, without offering additional, domain-relevant information for experts. This limitation is particularly evident in AC, where conflicting indicators, such as AU12 (Lip Corner Puller) signaling positive emotion and AU15 (Lip Corner Depressor) indicating negative emotion, may appear in the same facial region. Therefore, simply presenting the weight importance or model attention provides little insight for domain experts like psychologists to understand the AI decision-making process. Furthermore, the distinct properties of multimodal data make incorporating multimodal alignment and co-learning in post-hoc or map-based XAI methods even more challenging, taking the risk of losing either accuracy or interpretability.

    Recent attempts on concept-based models \cite{xinyuFG24, zarlenga2022concept} try to encapsulate specific, human-understandable features through concept embeddings $C$ that are learned in a fully supervised manner. These models learn the mapping $X \rightarrow C \rightarrow Z$, where $x \in X$ represents the raw image pixels and $z \in Z$ represents the task labels. Specifically, a concept generator $G$ generates concept embeddings, denoted as $\hat{c} = G(x)$, with $\hat{c} \in C$ representing the learned concepts within a bottleneck layer $C$. Subsequently, a facial expression predictor $y$ maps the concept embeddings to task labels $\hat{z} \in Z$, where $\hat{z} = y(\hat{c})$. While concept-based models offer a more interpretable framework than map-based approaches, ongoing research is focused on integrating this explainable architecture with multimodal learning and performance-optimized strategies \cite{xinyuFG24}. Moreover, a key challenge lies in integrating spatial explanations, which reveal \textit{where} the model is focusing, with concept-based explanations, which clarify \textit{what} contributes to the prediction. Achieving this synergy is essential for enhancing both the interpretability and practical utility of models in high-stakes applications.

    Table \ref{tab_previous_work} compares the proposed concept-based framework with previous feature-based, map-based, and black-box FER models in terms of explainability and performance. The proposed framework provides learnable domain-specific insights into the decision-making process for stakeholders while retaining map-based explanations that illustrate the model's areas of attention. A two-stage learning architecture with multimodal concept fusion is introduced, effectively addressing the alignment and co-learning challenges in multimodal interpretable AC. Furthermore, it achieves state-of-the-art performance through deep end-to-end training, successfully balancing the trade-off between interpretability and performance in high-stakes AC applications.

    \begin{table}[t]
    \centering
    \caption{Comparison of our work with previous works on FER in terms of explainability and performance, including feature-based approach, map-based approach, and deep end-to-end approach.}
    \begin{tabular}{ccccc}
    \toprule
                          & Ours & Feature & Map & Black-box \\ \midrule
    Feature-based Insight & +    & +       &     &     \\
    Map-based Explanation & +    &         & +   &     \\
    End-to-end Training   & +    &         & +   & +   \\
    Learnable Explanation & +    &         &     &     \\
    Multimodal Learning   & +    & +       &     & +   \\ 
    \bottomrule
    \end{tabular}
    \label{tab_previous_work}
    \end{table}

\section{Methods}
    
    This section provides a detailed overview of the proposed Attention-Guided Concept Model (AGCM). We begin by detailing the selection and generation of multimodal concepts, a critical step before deploying any concept-based explainable model. Next, we focus on the visual modality, as it is the most widely used and complex modality, uniquely supporting explanations of what concepts contribute to predictions and where they are observed. Finally, we describe the multimodal architecture, addressing the challenges of multimodal alignment and co-learning. Using the audio-visual modality as an example, we demonstrate the framework's functionality and highlight its extendability to other signal-based modalities.
    
\subsection{Multimodal Concept Selection \& Generation} \label{sec_spa_concept}
    The selection of concepts or features plays a pivotal role in producing accurate and explainable results, whether in interpretable concept-based models or traditional feature-based models. In terms of explainability, the concept function - similar to features - acts as a key representation of the underlying data. Moreover, concepts must explicitly capture attributes that are highly relevant and meaningful to the task at hand. For example, in object detection, attributes such as color and shape are critical, while in bird classification, features like wing morphology or bill structure provide significant insights.
    
    Explaining spatial signals, such as those in the visual modality, involves two key aspects: spatial explanations and conceptual insights, which are particularly critical in explainable medical analysis \cite{barnett2021case} and affective XAI \cite{belharbi2024guided}. To address this, AGCM integrates spatial concepts, enabling the model to learn not only \textit{what} to focus on but also \textit{where} to focus.

    For the conceptual explanations (the \textit{what} question), key features such as facial muscle movements, gaze direction, and head pose are important for assessing and interpreting an individual's affective state \cite{adolphs2002recognizing, Bayliss2007Affective, xinyuFG24}. To address spatial explanations (the \textit{where} question), patch-level attention maps are trained alongside each concept in an end-to-end, fully supervised manner. This method allows the model to associate concept contributions with their exact spatial locations, thereby enhancing both interpretability and overall performance.

    Since manually annotated attention maps are not always available for large-scale datasets, the spatial maps are localized based on facial landmarks \cite{belharbi2024guided, ma2021landmark}. In this paper, we utilize an open-source landmark detector \cite{wang2020deep} for automatic landmark detection. According to the landmark locations, Regions of Interest (ROI) maps are generated for all AUs, which are subsequently used to supervise the spatial concept attention map throughout model training.

    To integrate these ROI maps into our transformer-based concept learning framework, they are transferred into patch-level representations, $\text{PatchMaps}[i]$, by performing average interpolation, as described in (\ref{eq_interpolation}). Here, $\text{AUMaps}[i](x_1, y_1)$ denotes the value of the $i$-th input map at position $(x_1, y_1)$. The terms $x'$ and $y'$ correspond to the patch indices in the $x$ and $y$ dimensions, while $S_x$ and $S_y$ denote the respective scaling factors.
%
    \begin{equation}\label{eq_interpolation}
        \text{PatchMaps}[i] = \frac{1}{S_x \cdot S_y} \sum_{x_1 = x' S_x}^{(x' + 1) S_x} \sum_{y_1 = y' S_y}^{(y' + 1) S_y} \text{AUMaps}[i](x_1, y_1)
    \end{equation}

    Fig. \ref{fig_interpolation} presents an example of a patch-level AU map generated using landmark detection and average interpolation. In this map, patches with lighter colors indicate regions of higher importance, effectively highlighting the ROI for each AU. These maps are utilized as part of the ground truth to guide the model’s concept learning process via a concept map loss, ensuring the model's focus aligns with the actual spatial regions of interest during training.

    Other than spatial signals, temporal signals such as audio, Electrocardiogram (ECG), and Electroencephalogram (EEG) are often perceived as less complex in terms of dimensionality since they typically vary along a single axis (time). For these signals, stakeholders often prioritize conceptual insights (the \textit{what} question) over spatial interpretation. Temporal dependencies (the \textit{where} question in time) are naturally addressed by mechanisms like attention models or recurrence in sequential architectures, which excel at capturing temporal relationships.
    
    Using the widely used audio modality as an example, acoustic indicators such as pitch, loudness, and speech rate and their variations provide critical information by capturing subtle vocal variations that reflect emotional or cognitive states directly tied to the affective labels \cite{bachorowski1995vocal, Polzehl2011Anger, Yu2004Detecting, Adami2007Modeling, Grau1988The}. Providing conceptual insights into the decision-making process is essential for explaining predictions derived from these temporal signals.

    \begin{figure}[t]
    \centering
    \includegraphics[width=0.99\columnwidth]{fig_3.pdf}
       \caption{Example of patch-level AU map generated using landmark detection and average interpolation.}
    \label{fig_interpolation}
    \end{figure}

\subsection{Visual Attention-Guided Concept Learning}

     %%%%%%%%%% Start SVG (AGCM) %%%%%%%%%%%%
    \begin{figure*}[th]
    \centering
    \includegraphics[width=1.8\columnwidth]{fig_4.pdf}
       \caption{The architecture of our proposed Attention-Guided Concept Model (AGCM) for the spatial visual modality. The model uses a transformer backbone $\varphi(\cdot)$ to convert the facial image $x$ into a patch-level representation. The Attention-Guided Concept Generator (ACG) applies spatial-channel attention with a Multi-scale Spatial Attention (MSA) block and Channel Attended Concept Mapping (CACM), which together capture attention across both spatial and feature dimensions.
       The MSA block focuses on spatial features at multiple scales, enhancing the model's ability to capture both fine and coarse details. For instance, the concept of the cheek region may benefit from a larger attention area compared to the eye region. Three MSA heads are used to capture diverse spatial patterns within an image, each generating a concept attention map $\hat{a}_{i}$. These maps are weighted and summed to produce the final concept attention map, which is used to update the concept map loss during training.
       CACM further improves the model's focus on the most informative features along the channel dimension, ensuring robust feature selection across multiple channels.
       A concept probability generator $p(\cdot)$ computes the probability of each activated concept, facilitating concept supervision by showing the contribution of individual concepts to the predicted label. Notably, ACG considers both activated and inactivated concept embeddings, as the absence of certain concepts (e.g., AUs) can provide additional information about a subject's facial expression. The predicted activated concepts, $\hat{c}_{i}^{+}$, and inactivated concepts, $\hat{c}_{i}^{-}$, are weighted by their respective probabilities from $p(\cdot)$, then concatenated and passed to the one-layer fully-connected task predictor $y(\cdot)$ to generate the final task label $\hat{{\textit{t}}}$. 
       During loss computation, the model optimizes its performance using the task loss, concept probability loss, and concept map loss associated with the spatial concept attention, ensuring a strong explainability of the model's decision-making process giving not only \textit{what} key concepts contribute the most to the prediction but also \textit{where} these concepts appear. }
    \label{agcm_framwork}
    \end{figure*}
    %%%%%%%%%% END of SVG (CEM-based FER Framework) %%%%%%%%%%%%
  
    Given the complexity and the inherent differences between the spatial visual signal and other temporal signals, AGCM first focuses only on training the visual concept through attention-guided concept learning. This architecture leverages spatial concept supervision and concept attention to interpret the model's decision-making process by determining not only \textit{what} key concepts contribute the most to the prediction but also \textit{where} these concepts appear.

    As illustrated in Fig.~\ref{agcm_framwork}, the proposed Attention-Guided Concept Model (AGCM) is designed to enhance both the accuracy and explainability of the concept-based models. The model begins by processing the input facial image $x$ through a transformer backbone $\varphi(\cdot)$, which converts the image into a patch-level representation. This representation effectively captures local and global features by dividing the image into patches and is essential for subsequent processing.

    The core component of AGCM is the Attention-Guided Concept Generator (ACG), which integrates two attention mechanisms: Multi-scale Spatial Attention (MSA) and Channel Attended Concept Mapping (CACM). The MSA block focuses on spatial features at multiple scales, enabling the model to capture both fine-grained and coarse details within the image. For example, recognizing the concept of the cheek region may require a broader attention area compared to the eye region. To achieve this, three MSA heads are employed to capture diverse spatial patterns, each generating a concept attention map $\hat{a}_{i}$. These maps are then weighted and summed to produce a final concept attention map, which is utilized to update the concept map loss during training. 
    
    Complementing the spatial attention, CACM enhances the model's focus along the channel dimension. By applying attention to the most informative feature channels, CACM ensures robust feature selection across multiple channels, which is crucial for accurately interpreting complex facial expressions.

    The proposed framework also includes a concept probability generator $p(\cdot)$ that computes the probability of each activated concept. This mechanism facilitates concept supervision by quantifying the contribution of individual concepts to the predicted label. Importantly, ACG considers both activated and inactivated concept embeddings because the absence of certain concepts (e.g., deactivation of AUs) can also provide valuable information about one's facial expressions. The $i$-th predicted activated concepts, $\hat{c}_{i}^{+}$, and inactivated concepts, $\hat{c}_{i}^{-}$, are weighted by their respective probabilities from $p(\cdot)$. The probability score $p$ indicates the likelihood that the activated concept contributes to the final prediction. These are then concatenated and passed to the task predictor $y(\cdot)$, which is a one-layer fully connected network, to generate the final task label $\hat{\textbf{\textit{t}}}$. Therefore, it is designed to be adaptable and expandable to any discrete or continuous concepts, given that appropriate concept annotations are available.

    During loss computation, the model optimizes performance through a combination of losses: task loss, $\mathcal{L}_{t}$, concept probability loss, $\mathcal{L}_{c}$, and concept map loss, $\mathcal{L}_{m}$, associated with spatial concept attention. The task loss, $\mathcal{L}_{t}$, is computed using Cross Entropy (CE), while the concept probability loss, $\mathcal{L}_{c}$, is derived from the sum of Binary Cross Entropy (BCE) across all concepts. Instead of relying on Mean Square Error (MSE), the concept map loss, $\mathcal{L}_{m}$, uses Cosine Similarity (sim) to emphasize spatial pattern alignment rather than strict value matching. Therefore, the total loss, $\mathcal{L}$, is formulated as:
    %
   \begin{equation}\label{eq_loss}
    \mathcal{L} = \text{CE}(\hat{t}, t) + \sum_{i=1}^{n} \text{BCE}(p(\hat{c}_i^+), c_i) + \sum_{i=1}^{n} (1 - \text{sim}(\hat{a}_i, a_i)).
    \end{equation}
    %
    Here, $t$ is the ground truth task label, $c_i$ is the label of the $i$-th concept, and $a_i$ is the $i$-th concept attention map, while $n$ denotes the total number of used concepts.
    
    This comprehensive optimization strategy ensures that the AGCM framework achieves high accuracy while maintaining explainability in its predictions.


\subsection{Expandable Multimodal AGCM Concept Fusion}
    
     %%%%%%%%%% Start SVG (AGCM) %%%%%%%%%%%%
    \begin{figure*}[th]
    \centering
    \includegraphics[width=1.3\columnwidth]{fig_5.pdf}
       \caption{
       In the multimodal fusion stage, the pre-learned visual branch functions as a Visual Attention-Guided Concept Generator. The parameters of the Visual Attention-Guided Concept Generator are frozen to ensure reliable visual concept predictions. On the audio side, an Acoustic Concept Generator (ACG) processes the audio input, generating activated ($\hat{c}_{i}^{+}$) and inactivated ($\hat{c}_{i}^{-}$) acoustic concept embeddings via an acoustic feature extractor $G(\cdot)$. The probability of each concept's activation is computed using an acoustic concept probability generator $p(\cdot)$. The acoustic concept embeddings are concatenated with their corresponding visual concept set and passed through a sequential bottleneck layer {$\hat{c}_{0}$, ...$\hat{c}_{k}$}, where $k$ represents the number of samples in the sequence. For a given video clip, it is assumed that acoustic concepts are shared across all frames. A sequence-to-sequence label predictor $y(\cdot)$ is then used to capture the contextual relationships between frames to generate the final by-frame task label. Importantly, the AGCM framework is inherently extendable to other temporal modalities by adding additional branches to accommodate new data inputs, as long as the appropriate data and annotations are available.
       }
    \label{fusion_framwork}
    \end{figure*}
    %%%%%%%%%% END of SVG (CEM-based FER Framework) %%%%%%%%%%%%

    Alignment, fusion, and co-learning are three primary challenges in multimodal learning, involving the ability to identify, combine, and transfer knowledge across different modalities \cite{baltruvsaitis2018multimodal}, particularly in the context of interpretable AC \cite{cortinas2023toward}. After training the visual concept branch in the first stage, AGCM integrates visual information with any other temporal modalities through concept fusion. In this work, we demonstrate AGCM is an expandable multimodal architecture, using the most commonly used audio-visual fusion as an example, which involves identifying audio information using an acoustic concept generator and joining and transferring knowledge via a late fusion concept-label classifier. 

    As shown in Fig.~\ref{fusion_framwork}, the fusion stage builds upon the visual-based branch from the previous stage. During the fusion stage, the task predictor from the visual branch is removed, transforming it into a Visual Attention-Guided Concept Generator. This visual generator is responsible for extracting and predicting key visual concepts, including AUs, gaze direction, and head poses. To ensure stability and reliability in visual concept prediction, the parameters of the visual branch are frozen, preventing further modifications during the audio-visual training phase. This approach allows the model to harness pre-learned visual knowledge without overfitting, facilitating robust integrated learning across diverse input modalities.

    In parallel with the visual concept branch, the fusion stage introduces an audio brunch with an Acoustic Concept Generator (ACG) to process the audio input. This generator identifies relevant audio information using an acoustic feature extractor, denoted as $G(\cdot)$. These features are then mapped into activated ($\hat{c}_{i}^{+}$) and inactivated ($\hat{c}_{i}^{-}$) acoustic concept embeddings. The probability of activation for each concept is computed through an acoustic concept probability generator $p(\cdot)$, which quantifies the likelihood of each acoustic concept being present in the input. 

    For downstream applications, the audio branch can be replaced or expanded to incorporate other temporal modalities, such as Electrocardiogram (ECG), Electroencephalogram (EEG), or Electrodermal Activity (EDA), provided the appropriate data and annotations are available.

    Once the visual and temporal concepts are extracted, they are aligned and concatenated to form a unified multimodal representation. In this architecture, a key assumption is made: for a given video clip, temporal concepts are shared across all frames. This allows the model to maintain temporal coherence in the audio stream while aligning it with frame-specific visual features. The bottleneck layer serves to compress the multimodal information, ensuring that only the most relevant aspects of the fused representation are retained for further processing. The concatenated concepts are then passed through a sequential bottleneck layer, denoted as { $\hat{c}_{0}$, ..., $\hat{c}_{k}$ }, where $k$ represents the number of samples in the sequence. 

    To capture the temporal and contextual relationships between frames, the fusion branch employs a sequence-to-sequence concept-label predictor $y(\cdot)$, using a transformer architecture. This predictor is designed to handle sequential data, leveraging the temporal dependencies between consecutive frames in a video. By utilizing sequential learning, the model effectively integrates and co-learns multimodal information across time, improving the accuracy of by-frame predictions. This is particularly important for tasks where affective signals evolve over time, such as conversational engagement estimation or mental health assessment.

    The final task label is generated on a per-frame basis, with the model predicting the affective state for each frame in the video sequence. The combination of multimodal concept embeddings allows the VA-AGCM to provide robust and accurate predictions, as it captures a wider range of cues that contribute to affective behavior. Notably, the AGCM framework is readily extendable to other temporal modalities by incorporating additional branches for new data inputs.


\section{Experimental Evaluation and Results}

    Given the intricate nature and wide-ranging applications of AC tasks, we initially employed Facial Expression Recognition (FER) in both visual and audio-visual settings to validate the efficacy of our proposed AGCM framework, considering its well-established datasets and baseline models. We quantitatively evaluate the task and concept-level performance of AGCM on three large-scale FER datasets, and provide qualitative visualizations of the visual and multimodal conceptual explanations, demonstrating the framework’s robustness through occlusion experiments and an ablation study.

\subsection{Datasets}

    We employ three popular benchmark datasets, including RAF-DB and AffectNet with visual modality and Aff-Wild2 with audio-visual data. 

    \textbf{RAF-DB} \cite{li2017reliable} is a widely-used static FER dataset sourced from the internet, containing 6 basic emotion labels (Surprise, Disgust, Fear, Happiness, Sadness, Anger), and a Neutral label. The dataset includes 12,271 images in the training set and 3,068 images for testing.

    \textbf{AffectNet} \cite{mollahosseini2017affectnet} is one of the largest FER datasets, comprising 420,000 facial images annotated with categorical emotion labels. We utilize AffectNet-8, which consists of 291,651 manually labeled images with 8-class emotion labels (Neutral, Happy, Angry, Sad, Fear, Surprise, Disgust, and Contempt). In addition, we employ AffectNet-7, which contains 287,401 images annotated with seven emotion labels (excluding Contempt). The test set contains approximately 3,500 images.

    \textbf{Aff-Wild2} \cite{kollias2019expression} is a large-scale in-the-wild dataset specifically designed for FER and AU detection. It includes over 2.7 million frames from 564 videos with 554 subjects. We use the \textbf{by-frame} FER subset which is manually labeled with 8-class discrete emotions (Neutral, Anger, Disgust, Fear, Happiness, Sadness, Surprise, Other). It also provides manual annotation of 12 AUs. 

\subsection{Concept Generation Setup}
    AGCM is designed to be flexible and extendable to all kinds of discrete or continuous concepts, provided suitable concept annotations are available. In this work, we use the most commonly used audio-visual pair as an example. For the audio concepts, pitch, pitch variation, pitch stability (Jitter), loudness, loudness variation, and speech rate are used. For the visual modality, AUs, gaze direction, and head pose are used. 
    
    Unlike AUs, which are binary in nature (activated or inactivated), gaze, head pose and acoustic concepts are continuous and must be mapped into a probability space to fit the concept-based framework. Specifically, gaze concepts are defined as the degree of direct forward gaze in both horizontal and vertical planes, where $1$ represents directly looking forward and $0$ indicates looking elsewhere. Head pose concepts capture deviations in yaw (head shake) and pitch (head nod). These gaze and head pose concepts are scaled to the range $[0, 1]$ to fit within the concept probability generator, and corresponding heatmaps are generated based on facial landmarks, similar to Section \ref{sec_spa_concept}.

    All acoustic concept labels are normalized to the range $[0, 1]$ before AGCM training. To ensure alignment with the visual concepts, the video data is split into one-second clips (FPS=30), with a 33ms stride applied to capture temporal information effectively. 
    For clips containing complete silence, both pitch and loudness are set to $0$, indicating no contribution from the audio modality. Variations in loudness and pitch are calculated using their first-order derivatives, representing the rate of change for these acoustic features, while the Jitter is inherently a percentage. For videos featuring multiple speakers, the audio track for each subject will be individually separated to minimize noise and ensure clarity. 
    
    Furthermore, the AGCM framework is flexible and can incorporate other temporal modalities with continuous or discrete values, provided the appropriate data and annotations are available.

\subsection{Implementation Details}
     Our experimental setup is summarized as follows: AGCM utilized a pre-trained Vision Transformer as the backbone feature extractor \cite{dosovitskiy2020image}. Similar to \cite{xinyuFG24}, the backbone was pre-trained on VGGFace2 \cite{cao2018vggface2} for the facial recognition task. After pre-training, the classification header was removed and replaced with the AGCM workflow. Facial images were cropped from the video dataset using the InsightFace detector \cite{an2022killing}. To prevent overfitting, the preprocessing stage incorporated random data augmentation techniques, including horizontal flipping, random rotation, and random erasing.

     For datasets lacking AU annotations, we utilized OpenFace 2.0 \cite{baltrusaitis2018openface} to automatically extract 18 Action Units (AUs), which served as intermediary concepts in our proposed framework. All models were trained for 100 epochs, with early stopping to avoid overfitting, and optimized using the Adam optimizer (learning rate set to 0.0001). The AGCM generated concepts using a Dropout rate of 0.01 and Leaky-ReLU activation. The concept probability and map loss weights were set to $1$, ensuring a balanced focus on both conceptual explanation and task prediction. 

     AGCM used HuBERT \cite{hsu2021hubert} feature extractor for the audio input. During concept fusion, the learning of the vision branch was frozen, and the Acoustic Concept Generator (ACG) was fine-tuned for 100 epochs, with early stopping (learning rate set to 0.0001). All experiments were conducted on a workstation equipped with dual 48GB Nvidia RTX 6000 Ada GPUs, running a Linux-based PyTorch environment. For quantitative performance evaluation, we report the average performance over four random seeds. 

\subsection{Evaluating Visual-based AGCM}

    \begin{table*}[th]
    \centering
    \caption{Performance comparison of various models in terms of overall accuracy (\%) on RAF-DB, AffectNet-7, and AffectNet-8. The proposed AGCM framework consistently outperforms feature-based, map-based, and concept-based interpretable models. Notably, AGCM also surpasses state-of-the-art black-box models, offering superior performance without sacrificing conceptual interpretability.}
    \begin{tabular}{llllccc}
    \toprule
    Type                               & Model       & Year & Architecture                & RAF-DB         & AffectNet-7    & AffectNet-8    \\ \midrule
    \multirow{6}{*}{Black-box ML}      & AFR  \cite{savchenko2023adr}         & 2023 & EfficientNet                & 90.05          & 66.51          & 63.13          \\
                                       & CL-TransFER \cite{yang2024cl} & 2024 & Transformer                 & 91.33          & \textbf{67.86}          & \textbf{64.69}          \\ 
                                       & HAM \cite{tao2024hierarchical}        & 2024 & Attention                   & 91.92          & 66.97          & 63.82          \\
                                       & Poster++ \cite{mao2024posterpp}   & 2024 & Transformer                 & 92.21          & 67.49          & 63.77          \\
                                       & CEPrompt \cite{zhou2024ceprompt}   & 2024 & Transformer                 & 92.43          & 67.29          & 62.74          \\
                                       & S2D \cite{chen2024static}        & 2024 & Transformer                 & \textbf{92.57}          & 67.62          & 63.76          \\ \midrule
    Feature-based ML                   & FC          & 2024 & 3-layer FC                  & 67.04          & 40.23          & 37.11          \\ \midrule
    \multirow{3}{*}{Map-based XAI}     & TS-CAM \cite{gao2021ts}           & 2021 & Transformer + CAM & 86.70          & 62.28          & 58.99          \\
                                       & \multirow{2}{*}{Att-Map \cite{belharbi2024guided}}  & 2024 & CNN + Map Attention         & 88.88          & \textbf{62.45}          & \textbf{61.30}          \\
                                       &             & 2024 & Transformer + Map Attention & \textbf{91.03}          & 62.28          & 61.19          \\ \midrule
    \multirow{2}{*}{Concept-based XAI} & CEM \cite{xinyuFG24}        & 2024 & Concept Embedding           & 91.05          & 67.60          & 63.70          \\
                                       & AGCM        & 2024 & Spatial Attention Concept   & \textbf{94.40} & \textbf{69.45} & \textbf{65.62} \\ \bottomrule
    \label{tab_acc_rafdb}
    \end{tabular}
    \end{table*}

    Given the complexity and necessity of determining not only \textit{what} key concepts contribute the most to the prediction but also \textit{where} these concepts appear, we begin by evaluating the visual branch on RAF-DB and AffectNet. To assess the efficiency of the proposed AGCM framework against the previous feature-based and explainable models, we compared this work with a feature-based model, end-to-end map-based explainable models (with CNN and ViT backbones), previous concept-based explainable models, and the state-of-the-art black-box model without explicit model explainability. 
    
    The feature-based model uses only handcrafted features (e.g., AUs) as input, and a 3-layer Fully Connected (FC) neural network with ReLU activation, matching the complexity of AGCM's task predictor.

    Table \ref{tab_acc_rafdb} presents the overall accuracy of various models on RAF-DB, AffectNet-7, and AffectNet-8. The proposed AGCM framework achieves the highest accuracy across all datasets, with 94.40\% on RAF-DB, 69.45\% on AffectNet-7, and 65.62\% on AffectNet-8. These results demonstrate AGCM's significant improvement over the classic feature-based methods, particularly on RAF-DB (+27.36\%) and AffectNet-8 (+28.51\%). AGCM also outperforms state-of-the-art black-box transformer models including S2D \cite{chen2024static} and Poster++ \cite{mao2024posterpp}, providing gains of 1.83\% on RAF-DB and 1.86\% on AffectNet-8 compared to S2D. This highlights AGCM's ability to match and exceed black-box model performance while maintaining conceptual explainability. Furthermore, AGCM demonstrates superior results compared to interpretable map-based approaches, with a 3.37\% improvement on RAF-DB and over 4\% on AffectNet. When compared to the previous concept-based model \cite{xinyuFG24}, AGCM shows consistent gains across all datasets, benefiting from its spatial concept and attention learning.

    \begin{table}[t]
    \centering
    \caption{Class-wise performance comparison (\%) of the proposed AGCM and the transformer-based Poster++ \cite{mao2024posterpp} on RAF-DB and AffectNet-8. AGCM gives a more balanced performance along all classes, resulting in higher average accuracy. }
    \begin{tabular}{l|cc|cc}
    \hline
             & \multicolumn{2}{c|}{RAF-DB}     & \multicolumn{2}{c}{AffectNet-8} \\ \cline{2-5} 
             & AGCM           & POST++         & AGCM           & POST++         \\ \hline
    Anger    & \textbf{94.53} & 88.27          & \textbf{66.05} & 60.20          \\
    Disgust  & \textbf{82.43} & 71.88          & \textbf{61.58} & 58.00          \\
    Fear     & \textbf{87.50} & 68.92          & 63.00          & 63.00          \\
    Happy    & \textbf{97.47} & 97.22          & \textbf{79.42} & 76.40          \\ 
    Sad      & \textbf{93.51} & 92.89          & 65.01          & \textbf{66.80} \\
    Surprise & 89.51          & \textbf{90.58} & 62.99          & \textbf{65.60} \\
    Contempt & -              & -              & \textbf{64.08} & 59.52          \\
    Neutral  & \textbf{93.68} & 92.06          & \textbf{62.76} & 60.60          \\ \hline
    Avg.     & \textbf{91.23} & 85.97          & \textbf{65.61} & 63.77          \\ \hline
    \end{tabular}
    \label{tab_classwise_performance}
    \end{table}

    Table \ref{tab_classwise_performance} presents the class-wise performance comparison between the proposed AGCM framework and the black-box transformer-based Poster++ model \cite{mao2024posterpp} on RAF-DB and AffectNet-8. The results clearly demonstrate the effectiveness of AGCM in delivering a more balanced performance across all FER classes than Poster++, resulting in higher average accuracy on both datasets. The result shows the efficiency of considering conceptual prior knowledge, such as AUs and ROI maps, into the training process to quantify the individual concept's contribution towards predicting the label.

    On RAF-DB, AGCM consistently outperforms Poster++ across nearly all emotion classes, particularly in challenging categories such as Anger and Disgust, where AGCM achieves significant improvements of +6.26\% and +10.55\%, respectively. AGCM also demonstrates superior performance in the Fear class (+18.58\%), while maintaining competitive accuracy in easier classes like Happy and Neutral.
    
    Similarly, on AffectNet-8, AGCM provides improved accuracy in most categories, including notable gains in Anger (+5.85\%), Disgust (+3.58\%), and Happy (+3.02\%). Although Poster++ marginally outperforms AGCM in the Sad and Surprise categories, AGCM still delivers a more balanced overall performance, as evidenced by the higher average accuracy (+1.84\%).
    
    The consistent class-wise improvements offered by AGCM highlight its ability to maintain strong performance across both datasets, even in the presence of class imbalance and data variability. More importantly, AGCM not only surpasses Poster++ in terms of average accuracy but also achieves these gains while preserving the model's interpretability, which is essential for applications requiring both performance and transparency.

\subsection{Evaluating Multimodal AGCM}\label{sec_eval_multimodal}

    \begin{table}[t]
    \centering
    \caption{Performance comparison of various models in terms of average F-1 score (\%) on the uni- and multimodal Aff-Wild2 dataset.  }
    \begin{tabular}{lllll}
    \toprule
    Type                       & Model                & Arch.           & Data & F-1            \\ \midrule
    \multirow{5}{*}{Black-box} & DAN \cite{wen2023distract}                 & Attention       & V    & 40.10          \\
                               & AFR \cite{savchenko2023adr}                 & EfficientNet    & V    & 42.10          \\
                               & MAE \cite{ma2023unified}                 & MAE             & V    & 44.60          \\
                               & TCN \cite{zhou2023leveraging}                 & Transformer     & V/A  & 41.38          \\
                               & MMAE \cite{zhang2023multi}                & MAE+Transformer & V/A  & \textbf{48.93} \\ \midrule
    Feature                    & FC                   & FC              & V    & 25.27          \\ \midrule
    \multirow{3}{*}{Map}       & TS-CAM \cite{gao2021ts}              & Transformer     & V    & 37.05          \\
                               & \multirow{2}{*}{Att-Map \cite{belharbi2024guided}} & CNN             & V    & \textbf{41.92} \\
                               &                      & Transformer     & V    & 40.87          \\ \midrule
    \multirow{3}{*}{Concept}   & CEM \cite{xinyuFG24}                 &                 & V    & 42.60          \\
                               & AGCM                 &                 & V    & 44.95          \\
                               & AGCM                & Multimodal Fusion  & V/A  & \textbf{47.52} \\ \bottomrule
    \end{tabular}
    \label{tab_affwild2}
    \end{table}
    
    AGCM framework is designed to be expandable to multimodal inputs and concepts. In this work, we use the most commonly used audio-visual dataset as an example, demonstrating the AGCM's capacity for aligning and co-learning information from spatial and temporal modalities. 
    
    To evaluate the overall performance of the AGCM framework in a multimodal context, we conducted comprehensive experiments using the audio-visual Aff-Wild2 dataset.

    Table \ref{tab_affwild2} presents the performance comparison in terms of the average F-1 score on the Aff-Wild2 dataset. The proposed AGCM framework consistently outperforms feature-based, map-based, and concept-based interpretable models. Notably, AGCM in a multimodal setting achieves competitive results compared to state-of-the-art black-box models that leverage multimodal data, while maintaining conceptual explainability.

    Specifically, AGCM attains an F-1 score of 47.52\% by combining visual and audio inputs, outperforming visual-only AGCM (+2.57\%) and CEM (+4.92\%), showing that generally it works better in the multimodal setting. In comparison to feature-based models, AGCM demonstrates a significant improvement (+22.25\%), emphasizing the effectiveness of concept-level multimodal alignment and co-learning. While CNN-based map models \cite{belharbi2024guided} show stronger performance among map-based approaches, they still lag behind AGCM (by 3.03\%) and AGCM (by 5.6\%).
    
    The black-box MMAE model \cite{zhang2023multi} achieves the highest F-1 score of 48.93\%, largely due to its use of a pre-trained transformer (Masked Autoencoder or MAE), which is computationally expensive, time-consuming, and lacks interpretability. In contrast, the competitive results of AGCM highlight its ability to deliver robust performance while offering interpretability, which is a key advantage over black-box methods, even in the real-world multimodal context.


\subsection{Concept Efficiency}

    \begin{table}[t]
    \centering
    \caption{Concept Alignment Score (CAS) in percentage for all tasks. The score for no concept serves as a comparison. NOXI refers to the engagement estimation task in Section \ref{sec_noxi}.}
    \begin{tabular}{lcccc}
    \toprule
                & No Concept & CEM   & AGCM-V  & AGCM-AV \\ \midrule
    RAF-DB      & 66.10      & 78.62 & 82.36 & -     \\
    AffectNet-7 & 67.51      & 78.43 & 84.33 & -     \\
    AffectNet-8 & 66.29      & 78.01 & 83.52 & -     \\
    Aff-Wild2   & 65.09      & 77.36 & 81.29 & 81.46 \\
    NOXI        & 63.58      & 76.50 & 80.83 & 82.11 \\ \bottomrule
    \end{tabular}
    \label{tab_cas}
    \end{table}

    The efficiency of predicted concepts is a critical metric for both performance as well as explainability. To evaluate the reliability of learned concept representations, we employ the Concept Alignment Score (CAS) \cite{zarlenga2022concept}, which measures how well the predicted concepts align with their corresponding ground truth labels. Unlike traditional accuracy, which struggles with defining thresholds between ``activated'' and ``inactivated'' concepts, CAS uses homogeneity scores and clustering algorithms to assess the proximity of predicted concepts to ground truth, providing a more robust measure of concept alignment.

    As shown in Table \ref{tab_cas}, models without concept supervision (No Concept) serve as a baseline for comparison. The proposed framework in visual (AGCM-V) and audio-visual (AGCM-AV) contexts outperform the previous CEM models \cite{xinyuFG24}, which give higher CAS across all datasets, indicating their superior ability to learn meaningful and aligned concepts for both visual and audio modalities. 

\subsection{Human Interpretable Conceptual Explanation}
    
    In addition to achieving competitive performance compared to black-box deep learning models, a significant advantage of concept-based frameworks lies in their ability to offer clear, human-interpretable conceptual explanations grounded in domain-specific knowledge, making them accessible to even non-AI experts.

\subsubsection{Spatial Conceptual Explanation}
    Compared to the map-based approaches that only give one activation map as an explanation, AGCM combines the advantage of both concept-based and map-based models, which not only identifies \textit{where} the model focuses during inference but also explains \textit{what} specific facial behaviors the model is focusing on. 

    Fig. \ref{fig_heatmap_pred} illustrates the spatial concept explanations generated by the proposed AGCM for a facial image classified as ``Happiness'' from the AffectNet test set. During inference, AGCM produces attention maps for all relevant concepts and assigns probability scores based on the areas of the face highlighted in the maps. Concepts with higher probabilities, such as AU12 (Lip Corner Puller), are identified as making a significant contribution to the final classification, while those with lower probabilities, such as AU28 (Lip Suck), are effectively suppressed by the concept generator, reducing their influence on the predicted label. Compared to the map-based XAI that gives only a single attention map as the explanation, as in Fig. \ref{fig_intro}, the proposed model focuses on every possible expression indicator all over the facial region and then assigns the concept score to further indicate its contribution to a specific affective label, efficiently overcoming the trade-off between explainability and performance. 

    \begin{figure}[t]
    \centering
    \includegraphics[width=0.99\columnwidth]{fig_6.pdf}
       \caption{AGCM offers human-interpretable and intuitive explanations by presenting the contribution of each concept to the prediction alongside its spatial location. The numbers indicate the predicted probability scores for all concepts. During inference, the proposed AGCM generates attention maps for all concepts and assigns probability scores based on the highlighted regions. Concepts with higher probabilities (e.g., AU12) indicate greater contributions to the final label, while concepts with lower probabilities (e.g., AU28) are suppressed by AGCM's concept generator. }
    \label{fig_heatmap_pred}
    \end{figure}

    To simplify the visualization of the overall conceptual explanation, we proposed a weighted concept attention map $\bar{\alpha}$ that combines $i$-th predicted attention heatmaps $\hat{\alpha}_{i}$ with its corresponding concept probability $\hat{p}_{i}$, as given in (\ref{eq_norm_map}). Here, \textit{Norm} represents the min-max normalization, $n$ is the total number of concepts, and $\mathbb{I}(\hat{p}_{i} \geq \rho)$ is an indicator function that includes only concepts with probabilities exceeding the threshold $\rho$. We set $\rho = 0.5$ to visualize all activated concepts.
    %
    \begin{equation}\label{eq_norm_map}
        \bar{\alpha} = \textit{Norm}\left(\sum_{i=1}^{n} \hat{\alpha}_{i}\cdot \hat{p}_{i} \cdot \mathbb{I}(\hat{p}_{i} \geq \rho)\right)
    \end{equation}
    %
    \begin{figure}[t]
    \centering
    \includegraphics[width=0.99\columnwidth]{fig_7.pdf}
       \caption{Example of the facial expression label prediction, top-4 concept probability predictions (\%), and weighted concept attention visualization from the AffectNet and RAF-DB test sets. The proposed AGCM framework offers intuitive interpretability by identifying the most contributing concepts to the prediction (addressing the \textit{what} question) and providing spatial explanations for where these concepts are observed (addressing the \textit{where} question).}
    \label{fig_example_correct}
    \end{figure}

    Fig. \ref{fig_example_correct} shows examples randomly selected from the AffectNet and RAF-DB test sets, illustrating the prediction of emotion labels alongside the top-4 concept probabilities (\%) and corresponding weighted concept attention visualizations. The AGCM framework accurately predicts class labels and provides insightful conceptual explanations through activated concept probabilities and attention heatmaps.

    In the ``Happy'' example, AU6 (Cheek Raiser), AU12 (Lip Corner Puller), and AU14 (Dimpler) are all strong indicators of happiness. AGCM efficiently focuses on the relevant facial areas while highlighting the contributions of these concepts. For the ``Anger'' expression, the model emphasizes AU4 (Brow Lowerer) and AU9 (Nose Wrinkler), which are the primary contributors to this emotion, with the attention maps focusing meaningfully on the brow and nose regions. These examples demonstrate AGCM's ability to combine robust performance with clear, human-interpretable conceptual explanations, making it readily applicable to domain-specific expertise.

\subsubsection{Spatial-temporal Conceptual Explanation}

    Another key advantage of the AGCM framework over previous interpretable approaches \cite{xinyuFG24, gao2021ts, belharbi2024guided}, is its ability to provide multimodal conceptual explanations from spatial and temporal data sources. Using audio-visual fusion as an example, AGCM enables the model to co-learn the information from multimodal data inputs, offering robust performance and better interpretable explanations in real-world multimodal contexts.

    Fig. \ref{fig_example_affwild2} illustrates an example of FER prediction on the Aff-Wild2 test set. We randomly selected this video clip to show approximately 10 seconds of data, which contains an emotional transition and downregulation event. Initially, the subject is in a ``Surprise'' state, where AGCM accurately identifies key visual concepts, such as AU25 (Lips Part) and AU1 (Inner Brow Raiser), which strongly indicate this emotion. 

    As the emotional transition occurs, AU1 decreases while AU12 (Lip Corner Puller) becomes dominant, signaling a shift toward a ``Happy'' state. Additionally, the model detects high intensities in pitch and loudness concepts, which are often associated with happiness because they reflect a sudden increase in physiological arousal, and are a natural reaction to pleasant and positive emotions \cite{kamilouglu2020good}. Toward the end of the clip, all concepts gradually decline, reflecting the downregulation of a high-intensity emotion back to a neutral state. AGCM enables the co-learning and interpretation of multimodal inputs by providing \textit{what}-\textit{where} explanations for the visual modality and identifying \textit{what} key conceptual insights derived from temporal signals. Additionally, temporal dependencies (\textit{where} in time) are handled through attention-based sequential learning during multimodal fusion, ensuring comprehensive interpretability across modalities.

    \begin{figure}[t]
    \centering
    \includegraphics[width=0.99\columnwidth]{fig_8.pdf}
       \caption{AGCM facilitates both the co-learning and interpretation of multimodal inputs. In addition to providing \textit{what}-\textit{where} explanations for the visual modality, AGCM offers \textit{what} the key conceptual insights into temporal signals. Temporal dependencies (\textit{where} in time) are naturally addressed through attention-based sequential learning. This figure shows an example from the Aff-Wild2 test set (~10 seconds), demonstrating this capability by including facial expression label predictions, top-2 AU probability predictions (\%), acoustic concept intensities (\%), and weighted concept attention visualizations. AGCM accurately predicts emotion transitions and downregulation while delivering human-interpretable conceptual explanations for both visual and acoustic modalities.}
    \label{fig_example_affwild2}
    \end{figure}

\subsection{Robustness of the Explanation}
    
    To further evaluate the robustness of the model's explanations, we stress-test AGCM to explore its ability to handle challenging scenarios. Facial occlusion is a common challenge in real-world affective signal processing applications, particularly in in-the-wild datasets, where the subjects may wear VR glasses, causing upper-face occlusion, or masks, leading to lower-face occlusion. These occlusions present difficulties for affective computing, especially when providing conceptual or map-based explanations. The proposed AGCM framework addresses this limitation by generating weighted concept attention maps, which improve both the performance and the interpretability.
    
    To simulate real-world occlusion scenarios, we selected images from the Aff-Wild2 test set and manually occluded either the upper or lower face regions, re-evaluating the performance of the well-trained AGCM framework.

    As shown in Fig. \ref{fig_concept_occ}, we randomly selected samples with varying facial expressions, lighting conditions, and angles, then removed either the upper or lower face regions. Using the same well-trained AGCM model, we re-evaluated the predicted emotion labels, representative top concept probabilities, and the corresponding weighted concept attention maps. After occlusion, AGCM still accurately predicts the emotion by focusing on the unobstructed facial regions. In the ``Happy'' examples, the model shifts attention away from AU6 (Cheek Raiser), which is occluded and focuses more on AU12 (Lip Corner Puller), resulting in a correct prediction despite the occlusion. Similarly, in the ``Surprise'' example, AGCM downweights the contribution of the occluded AU26 (Jaw Drop) and instead focuses on AU2 (Outer Brow Raiser), another strong indicator of surprise. These results demonstrate AGCM’s robustness in handling occluded facial images while maintaining accurate and interpretable predictions.

    Hand-over-face occlusion presents an even more complex challenge than occlusion caused by glasses and masks, as the hand can often be misinterpreted as part of the face during model inference because one's hands often share similar textures with the face. To evaluate AGCM’s performance in such scenarios, we selected additional samples from the Aff-Wild2 dataset, which contains instances of hand-over-face occlusion.

    Fig. \ref{fig_example_occ} shows test images featuring hand-over-face occlusion. Despite these occlusions, AGCM generates accurate emotion predictions by leveraging a few key concepts. For instance, in the ``Surprise'' example, even though the lower-face concepts are occluded, the model identifies high probabilities for upper-face indicators AU1 (Inner Brow Raiser) and AU2, leading to a correct prediction. Similarly, in the ``Happy'' example, AU6 (Cheek Raiser) alone is sufficient for the model to make this accurate prediction. 
    
    These stress-testing results demonstrate that AGCM effectively handles partial face occlusion and hand-over-face occlusion by focusing on unobstructed regions and leveraging spatial concept learning to emphasize visible concepts during training. This capability highlights AGCM's robust, concept-aware spatial explanations, enabling reliable predictions even in challenging scenarios.


    \begin{figure}[t]
    \centering
    \includegraphics[width=0.99\columnwidth]{fig_9.pdf}
       \caption{Example of the face occlusion samples, with the expression label prediction, representative concept probability predictions (\%) from the upper and lower-face region, and weighted concept attention visualization from the Aff-Wild2 test sets. After occlusion, AGCM adapts by shifting attention to the non-occluded areas, ensuring reliable predictions based on the remaining visible concepts. }
    \label{fig_concept_occ}
    \end{figure}
    
    
    \begin{figure}[t]
    \centering
    \includegraphics[width=0.99\columnwidth]{fig_10.pdf}
       \caption{Example of hand-over-face occlusion, with the predicted facial expression label, top-4 concept probability predictions (\%), and weighted concept attention visualization. AGCM accurately focuses on the non-occluded regions and predicts the task label based on the available concepts, demonstrating its robustness in handling facial expressions with hand-over-face occlusion.}
    \label{fig_example_occ}
    \end{figure}

\subsection{Ablation Study}
    Compared to the previous concept-based approaches, the proposed AGCM framework introduces four main components, including Multi-scale Spatial Attention (MSA), Multi-head Attention (MHA), Cannel Attended Concept Mapping (CACM), and Concept Map Loss (CML). As the evaluation of multimodal concept fusion has been given in Section \ref{sec_eval_multimodal}, this section provides an ablation study to show the efficiency of the visual-based AGCM framework. 

    \begin{table}[t]
    \centering
    \caption{Ablation study of the visual-based AGCM framework on RAF-DB and AffectNet-8 test set.}
    \begin{tabular}{cccc|cc}
    \toprule
    MSA & MHA & CACM & CML & RAF-DB   & AffectNet-8 \\ \midrule
    -   & -   & -    & -   & 90.47 & 62.58     \\ 
    +   & -   & -    & -   & 92.84 & 62.99     \\
    +   & +   & -    & -   & 93.26 & 63.10     \\
    +   & +   & +    & -   & 93.31 & 63.46     \\
    +   & +   & +    & +   & \textbf{94.40} & \textbf{65.62}     \\
    \bottomrule
    \end{tabular}
    \label{tab_abl}
    \end{table}

    Table \ref{tab_abl} presents the ablation study for the visual-based AGCM framework on RAF-DB and AffectNet-8. The baseline model without any components achieves 90.47\% on RAF-DB and 62.58\% on AffectNet-8. Adding Multi-scale Spatial Attention (MSA) improves performance significantly, reaching 92.84\% and 62.99\%. Introducing Multi-head Attention (MHA) further boosts accuracy to 93.26\% and 63.10\%, while Channel Attended Concept Mapping (CACM) provides a slight improvement to 93.31\% and 63.46\%. Finally, the full AGCM with Concept Map Loss (CML) achieves the best results, 94.40\% on RAF-DB and 65.62\% on AffectNet-8, demonstrating the cumulative benefit of these components in enhancing accuracy while maintaining explainability.


\section{AGCM for Interpretable Engagement Estimation}\label{sec_noxi}

    The generalizability of the framework to downstream applications is essential for establishing a trustworthy AC system. Real-life affective signal processing is inherently more ambiguous, complex, and diverse compared to the well-defined FER task. One good example is human-human interactions, where the conversational engagement score is designed to measure the level and rate of engagement between participants, illustrating the broader and more nuanced requirements of real-world AC applications.
    
    In this section, we use the NOvice eXpert Interaction (NOXI) dataset, a large-scale,  well-annotated human-human interaction dataset with the engagement label, to illustrate AGCM's generalization capacity in real-world AC contexts. We conduct both qualitative and quantitative evaluations, demonstrating that AGCM achieves robust performance by automatically identifying key indicators and highlighting essential concepts.

    NOXI \cite{cafaro2017noxi} is designed for the analysis of human interaction in real-world, cross-cultural settings. It includes video recordings of novice-expert interactions in eight languages (English, French, German, Spanish, Indonesian, Arabic, Dutch, and Italian), with AU capture via Microsoft \textit{Kinect} \cite{zhang2012microsoft}. The dataset spans over 50 hours of video and is annotated with a \textbf{by-frame} engagement score ranging from 0 to 1. For our experiments, we utilize 76 videos (over 1.5 million frames) for training and 20 videos (over 500,000 frames) for testing.

\subsection{Generalizing AGCM for Engagement Estimation}

    \begin{table}[t]
    \centering
    \caption{Performance comparison of various models in terms of Concordance Correlation Coefficient (CCC) on the uni- and multimodal Noxi dataset.}
    \begin{tabular}{lllcc}
    \toprule
    Type                       & Model   & Arch.          & Data & CCC           \\ \midrule
    \multirow{3}{*}{Black-box} & TCA \cite{he2024tca}    & Attention      & V/A   & 0.73          \\
                               & DCTM \cite{tu2023dctm}   & Transformer    & V/A    & 0.77          \\
                               & S2S \cite{yu2023sliding}     & Transformer    & V/A    & \textbf{0.83} \\ \midrule
    Feature                    & FC      & FC             & V    & 0.23          \\ \midrule
    \multirow{2}{*}{Map}       & TS-CAM \cite{gao2021ts} & Transformer    & V    & 0.36          \\
                               & Att-Map \cite{belharbi2024guided} &                & -    & -             \\ \hline
    Concept                    & CEM     &                & V    & 0.48          \\
                               & AGCM    &                & V    & 0.59          \\
                               & AGCM    & Concept Fusion & V/A  & \textbf{0.80} \\
    \bottomrule
    \end{tabular}
    \label{tab_noxi}
    \end{table}

    AGCM is highly generalizable to downstream AC applications by simply adjusting the configuration of the final task predictor. For example, in FER tasks, a classification header is utilized, whereas in continuous signal prediction tasks, a regression header is employed. This flexibility allows AGCM to adapt a wide range of affective computing applications.

    Table \ref{tab_noxi} shows the performance comparison of various models in terms of the Concordance Correlation Coefficient (CCC) on the Noxi dataset for continuous engagement estimation. CCC is used to evaluate continuous tasks by measuring the agreement between predicted and true values, accounting for both correlation and accuracy, making it ideal for engagement estimation tasks. The proposed multimodal AGCM framework with audio-visual concept fusion again outperforms feature-based and previous concept-based models, showing its outstanding state-of-the-art performance in downstream real-world engagement estimation tasks.

    In the unimodal setting, AGCM with visual concepts achieves a CCC score of 0.59, marking a substantial improvement over the unimodal CEM (+0.11) and feature-based model (+0.23). This result highlights the advantages of spatial concept learning while underscoring the limitations of feature-based models in addressing the complexities of affective signal processing.

    In the multimodal context, AGCM attains a performance of 0.80, demonstrating the significant benefits of co-learning multimodal knowledge. This is particularly valuable in complex real-world AC applications, such as engagement estimation, where multiple modalities are essential for capturing and understanding nuanced human behavior.

    Although the black-box S2S model \cite{yu2023sliding} slightly outperforms AGCM with a CCC of 0.03, AGCM underscores its ability to approximate state-of-the-art results while maintaining interpretability. The attention map-based models \cite{belharbi2024guided} are not well-suited to this task, as they rely on predefined mappings between AUs and labels, which are not available for continuous engagement estimation. Additionally, TS-CAM \cite{gao2021ts}, which is restricted to the visual modality, also performs poorly in engagement estimation.

    Meanwhile, the Concept Alignment Score (CAS), as shown in Table \ref{tab_cas}, illustrates that the AGCM framework with audio-visual co-learning not only maintains competitive performance compared to state-of-the-art black-box deep learning models but also delivers accurate conceptual explanations. 
    
    Therefore, this sophisticated interpretable framework maintains competitive performance without compromise. By simply configuring the AGCM classifier, the performance evaluation on engagement estimation demonstrates the strong generalizability of AGCM to a wide range of downstream applications beyond FER, making it both powerful and accessible for diverse affective computing tasks.

\subsection{AGCM Explainability in Engagement Estimation}
    \begin{figure}[t]
    \centering
    \includegraphics[width=0.99\columnwidth]{fig_11.pdf}
       \caption{Example of the engagement estimation, gaze, head pose direction (mean degree of forward gaze or facing forward in x and y directions), top-1 AU probability predictions (\%), acoustic concept intensities (\%), and weighted concept attention visualization of a Noxi test sample (around 60 seconds). The proposed AGCM framework accurately predicts engagement transition for different states during conversation and provides meaningful visual and acoustic conceptual explanations.}
    \label{fig_example_noxi}
    \end{figure}

    Explainability becomes even more crucial in downstream AC applications compared to FER, given the inherent complexity of human behavior. In tasks such as engagement estimation, delivering domain-specific explanations is vital for non-AI stakeholders to understand and interpret the decision-making process.

\subsubsection{Explaining Engagement Transitions}

    To show AGCM's explanation and prediction capabilities in human-human engagement estimation, Fig. \ref{fig_example_noxi} presents an example from the NOXI dataset. This sample, randomly selected to cover approximately 60 seconds of data, highlights engagement transitions between listening, distraction, and interaction.

    At the beginning of the sequence, the subject actively listens with a direct gaze toward the speaker, as indicated by the high intensity of the conceptual direct gaze. In conversation-based engagement estimation, gaze direction and head pose are critical concepts for predicting and explaining engagement scores. Since the acoustic input is not prominent during the listening phase, the intensities of acoustic concepts remain low, which is expected as the audio track of each subject is recorded separately in this dataset.

    When distraction occurs, the subject shifts attention to a phone call or another person, causing the intensities of direct gaze and forward head pose concepts to decrease, which in turn lowers the engagement score. When the subject looks down with eyes only partly open, as evidenced by the activation of AU45 (Blink), the predicted engagement score reaches its lowest point, signifying that the subject’s attention is fully disengaged from the conversation.

    As the distraction ends and the subject re-engages with the speaker, positive facial expressions, such as AU12, become prominent, associated with increased engagement during interactions \cite{Greipl2021Facial, Savchenko2022Classifying}. The intensities of gaze and head pose concepts increase, and acoustic concepts begin to register, indicating the subject's regained focus and positive emotional feedback. This highlights the subject’s re-engagement in the conversation. 6
    
    AGCM effectively leverages these multimodal concepts to capture subtle changes in engagement states and ensure robust conceptual explainability during inference, demonstrating its strong generalizability to downstream AC applications with complex behavioral labels, extending beyond the scope of FER.

\subsubsection{Explaining Complex Human Behaviours}
    
    Real-world behavioral states are more complex than facial expressions. Higher-level affective states that share similar feature representations usually introduce ambiguity in task predictions, especially in feature-based models. 
    
    The proposed AGCM framework has an inherent advantage in differentiating nuanced states by autonomously learning and capturing both the concept-aware similarities and distinctions between these affective states. Take an example of human-human interaction, real-world applications often involve complex affective states that are more ambiguous and abstract compared to discrete emotions, such as distraction and cognitive load \cite{Krasich2018GazeBased}. 
    
    Fig. \ref{fig_cognitive_load} provides an example of engagement estimation in the presence of distraction and cognitive Load. Distraction occurs when the subject's gaze drifts away, indicating mental disengagement. Conversely, cognitive load happens when the subject looks away while remaining engaged in processing information.  In feature-based affective computing models, these complex behaviors which share similar feature representations, can introduce ambiguity in task predictions. 
    
    
    \begin{figure}[t]
    \centering
    \includegraphics[width=0.95\columnwidth]{fig_12.pdf}
       \caption{Example of the engagement estimation for distraction and cognitive load, with the prediction of gaze, head pose direction (mean degree of forward gaze or facing forward in x and y directions), top-1 AU probability predictions (\%), acoustic concept intensities (\%), and weighted concept attention visualization from Noxi dataset. AGCM differentiates between distraction and cognitive load according to efficient concept learning. }
    \label{fig_cognitive_load}
    \end{figure}


    Thus, the proposed AGCM framework provides robust learning and explainability, even in complex behavioral states such as distraction and cognitive load. This demonstrates its effectiveness in capturing nuanced affective states, providing enhanced generalizability to complex downstream AC applications that are difficult to tackle using conventional methods.


\section{Conclusion \& Future Work}

    In this paper, we introduce the Attention-Guided Concept Model (AGCM), a multimodal concept-based interpretable framework that provides conceptual explanations of \textit{what} concepts contribute to the predictions and \textit{where} they are observed. AGCM is highly extendable to various spatial-temporal modalities, effectively addressing the challenges of multimodal alignment, fusion, and co-learning. The framework demonstrates strong generalizability and flexibility, making it well-suited for diverse real-world AC applications.
    
    We first validate the model's effectiveness in achieving both high performance and robust explanation through qualitative and quantitative evaluations on well-established FER datasets. Then, we demonstrate the generalizability of the AGCM framework to other complex real-world AC applications by extensive experiments on the human-human interaction task. We believe that AGCM establishes a foundation for creating future interpretable systems in downstream AC applications, such as psychology, psychiatry, digital behavior, and Human-Computer Interaction, with competitive performance and human-interpretable explanation.



    AGCM leverages the strengths of both feature-based models and deep black-box models to offer interpretable, high-performance predictions. However, explainability in affective computing remains an evolving area of research. We posit that model explanations should be tailored to end-users, such as psychologists and cognitive scientists. Therefore, we plan to incorporate a human-in-the-loop approach for affective XAI to further enhance model usability. Additionally, while AGCM is trained on large datasets, exploring XAI fairness in terms of gender, cultural, and age biases presents an interesting avenue for further investigation. In this paper, we assess various forms of occlusion using the Aff-Wild2 dataset; future improvements could be achieved by fine-tuning AGCM on occlusion-specific datasets to better handle such challenges. Generating text-based explanations via Large Language Models (LLM) may also give users extra insights. However, given the inherent complexity of LLMs, it is imperative to employ appropriate knowledge distillation techniques, particularly for cross-disciplinary stakeholders.

% {\small
% \bibliographystyle{ieee}
% \bibliography{egbib}
% }

{\small
\documentclass{MITstyle}

%\usepackage[table]{xcolor}
\usepackage{chngcntr}
\usepackage{hyperref}
\usepackage{microtype}

\title{A Lightweight and Extensible Cell Segmentation and Classification Model for Whole Slide Images}

\author{Nikita Shvetsov~$^{1, }$\footnote{Correspondence e-mail: nikita.shvetsov@uit.no}, Thomas K. Kilvaer~$^{2, 3}$, Masoud Tafavvoghi~$^{4}$, Anders Sildnes~$^{1}$, \\ Kajsa Møllersen~$^{4}$, Lill-Tove Rasmussen Busund~$^{5, 6}$, Lars Ailo Bongo~$^{1}$ \\
%
\vspace{1em} % Space between authors and afilliations
%
\normalfont{\small $^{1}$Department of Computer Science, UiT The Arctic University of Norway}\\
\normalfont{\small $^{2}$Department of Oncology, University Hospital of North Norway}\\
\normalfont{\small $^{3}$Department of Clinical Medicine, UiT The Arctic University of Norway}\\
\normalfont{\small $^{4}$Department of Community Medicine, UiT The Arctic University of Norway}\\
\normalfont{\small $^{5}$Department of Medical Biology, UiT The Arctic University of Norway} \\
\normalfont{\small $^{6}$Department of Clinical Pathology, University Hospital of North Norway} %\vspace{2em}
}

\begin{document}
\maketitle

\section*{Abstract}

% \begin{abstract}
% Developing clinically useful cell-level analysis tools in digital pathology remains challenging due to limitations in dataset granularity, inconsistent annotations, computational demands of advanced models, and difficulties in integrating new technologies into clinical workflows. To address these challenges, we propose a multi-faceted solution that enhances data quality, model performance, and usability to create a lightweight and extensible cell segmentation and classification model.

% First, we update data labels by employing a cross-relabeling process that refines the labels of two existing datasets, PanNuke and MoNuSAC, to create a new unified dataset with enhanced granularity, encompassing seven distinct cell types. Second, we leverage the H-Optimus foundation model as a fixed encoder to improve feature representation for simultaneous cell segmentation and classification tasks. Third, to address the computational demands of foundation models, we employ knowledge distillation to reduce model size and complexity while maintaining comparable performance. Finally, to facilitate integration into clinical workflows, we integrate the distilled model into the QuPath software, a widely used open-source platform in digital pathology.

% Our results demonstrate improvements in cell segmentation and classification performance using the H‑Optimus-based model compared to a CNN-based model. Specifically, the average $R^2$ improved from 0.575 to 0.871, and the average $PQ$ score improved from 0.450 to 0.492, indicating better alignment with actual cell counts and enhanced segmentation and classification quality. Furthermore, the distilled student model maintains performance comparable to the larger foundation model while reducing the parameter count by a factor of 48.
% Overall, by reducing computational complexity and integrating it into existing workflows, the proposed approach may significantly impact diagnostic processes, reduce the workload of pathologists, and contribute to improved patient outcomes. Though our approach shows potential enhancements in efficiency and usability of cell segmentation and classification models in digital pathology, extensive validation is needed to deploy these models in clinical practice.
% \end{abstract}

%%% shortened abstract
\begin{abstract}
Developing clinically useful cell-level analysis tools in digital pathology remains challenging due to limitations in dataset granularity, inconsistent annotations, high computational demands, and difficulties integrating new technologies into workflows. To address these issues, we propose a solution that enhances data quality, model performance, and usability by creating a lightweight, extensible cell segmentation and classification model. 

First, we update data labels through cross-relabeling to refine annotations of PanNuke and MoNuSAC, producing a unified dataset with seven distinct cell types. Second, we leverage the H-Optimus foundation model as a fixed encoder to improve feature representation for simultaneous segmentation and classification tasks. Third, to address foundation models' computational demands, we distill knowledge to reduce model size and complexity while maintaining comparable performance. Finally, we integrate the distilled model into QuPath, a widely used open-source digital pathology platform. 

Results demonstrate improved segmentation and classification performance using the H-Optimus-based model compared to a CNN-based model. Specifically, average $R^2$ improved from 0.575 to 0.871, and average $PQ$ score improved from 0.450 to 0.492, indicating better alignment with actual cell counts and enhanced segmentation quality. The distilled model maintains comparable performance while reducing parameter count by a factor of 48. By reducing computational complexity and integrating into workflows, this approach may significantly impact diagnostics, reduce pathologist workload, and improve outcomes. Although the method shows promise, extensive validation is necessary prior to clinical deployment.
\end{abstract}
\clearpage

\section{Introduction}
In digital pathology, accurate segmentation and classification of cells are crucial for many diagnostic, prognostic, and predictive analyses \cite{Jaber_Beziaeva_etal._2019,Lin_Pan_etal._2022,Park_Ock_etal._2022,Shen_Choi_etal._2024}. Nowadays, developments in computational pathology offer multiple solutions \cite{H._Qu_P._Wu_etal._2020,Javed_Mahmood_etal._2020} to utilize cell-level datasets to train machine learning models that solve these problems. The quality and specificity of training datasets are critical for robust and accurate models. Adhering to the principle of "garbage in, garbage out", it is essential to ensure that these datasets are extensively and accurately labeled with distinct classes that reflect the diverse biological characteristics of different cell types. Unfortunately, the number of open-source datasets comprising such high-quality annotations is limited. Existing cell segmentation datasets \cite{Gamper_Koohbanani_etal._2019,Graham_Vu_etal._2019,Verma_Kumar_etal._2021} may offer extensive annotations for certain cell types while providing more general labels for others. For example, in PanNuke, which is one of the largest open-source datasets comprising labeled cells, various types of morphologically and functionally different inflammatory cells like macrophages and lymphocytes are clustered in a broad "inflammatory" class. Consequently, these classes are frequently omitted from analyses or aggregated into broader meta-classes \cite{Gamper_Koohbanani_etal._2020} and likely interfere with other cell classes included in the dataset. This and similar inconsistencies in annotation granularity limit the ability of machine learning models to learn the comprehensive and nuanced features necessary for accurate cell segmentation and classification. To address these challenges, methods for refining and standardizing dataset annotations are essential to enhance the quality of training data.

A complementary approach to mitigate the absence of high-quality training data is the use of foundation models. Foundation models as encoders are defined as large-scale, versatile networks pre-trained on vast, diverse datasets using self-supervised learning, contrasting with convolutional neural network (CNN) pre-trained encoders that rely on supervised learning with labeled data. In practice, foundation models leverage enormous amounts of weakly or unlabeled data from millions of whole slide images (WSIs) and employ self-attention mechanisms to capture long-range dependencies and global context \cite{Chen_Ding_etal._2024,Saillard_Jenatton_etal._2024,Vorontsov_Bozkurt_etal._2024,Xu_Usuyama_etal._2024}. As a consequence, foundation models are able to produce transferable feature representations across different cell types and tissue environments. The feature representations can be leveraged by decoder networks to produce segmentation masks and pixel-level classifications. Because foundation models have comprehensive feature representations, they can be effectively fine-tuned using much smaller amounts of cell-level data compared to the large datasets needed to train models from scratch. Furthermore, foundation models incorporate adversarial training elements or contrastive learning \cite{Chen_Ding_etal._2024,Xu_Usuyama_etal._2024}, enhancing their resilience and adaptability by exposing them to challenging and varied scenarios during training. This may result in more generalizable models, often making them well-suited for diverse and complex tasks in digital pathology.

Despite the inherent advantages of foundation models, their deployment for practical use faces its own obstacles. In particular, they require substantial computational power, financial investments and rigorous testing to ensure reliability and efficacy for a given task \cite{Akkus_Dangott_etal._2022,Dragomir_Cocuz_etal._2022,Go_2022,Jafri_Farooqui_etal._2024}. Moreover, while foundation models enhance feature representation and performance, they depend on the quality of available annotations for decoder fine-tuning and, like any other model, cannot resolve existing inconsistencies or ambiguities in data labels. Therefore, there remains a critical need for solutions that address both data quality and practical deployment considerations.
Further, integrating new technologies into existing clinical workflows often encounters resistance, as it necessitates adjustments to established diagnostic processes. So, there is a need to develop solutions that could be integrated into current practices, minimizing the burden on medical professionals to adopt new tools \cite{King_Williams_etal._2023}.

Existing solutions \cite{Goldsborough_Philps_etal._2024,Hörst_Rempe_etal._2024}, while addressing some aspects of these challenges, fall short in providing a comprehensive approach. To address the data quality and clinical deployment issues, we propose a multi-faceted solution that encompasses data refinement, model optimization, and integration with existing pathology tools (\hyperref[fig:fig1]{Figure 1}). The outcome is a lightweight cell segmentation and classification model that can be integrated into digital pathology workflows for practical clinical use.

\begin{figure}[h!]
    \centering
    \includegraphics[width=\textwidth, height=0.82\textheight, keepaspectratio]{images/Figure_1.pdf}
    \caption{Overview of the proposed solution, including 1) Data refinement using cross-relabeling, 2) Teacher model development and fine tuning, 3) Student model optimization with knowledge distillation and 4) Student model and QuPath integration}
    \label{fig:fig1}
\end{figure}
\clearpage

Our approach begins with preparing the data for the fine-tuning and training of the machine learning models. We create a refined dataset, acquired via cross-relabeling two cell-level datasets, enhancing annotation specificity and consistency of the labeled data. Subsequently, we create a cell segmentation and classification model based on the foundation model. We leverage the foundation model as a fixed encoder and fine-tune a decoder using the refined dataset to improve generalization across diverse tissue- and cell types.
To ensure that the model remains lightweight and deployable in a possibly resource-constrained environment, we employ knowledge distillation to approximate the functionality of the foundation model. Finally, to facilitate the practical application of our model in digital pathology workflows, we integrate it with the QuPath \cite{Bankhead_Loughrey_etal._2017} application. Each methodological component contributes to the overarching goal of enhancing model performance, generalizability, and usability in clinical settings.

The primary contributions of this paper are:
\begin{enumerate}
    \item \textit{Data labels refinement through cross-relabeling:}
    
    We propose a new method for refining labels of cell-level datasets through cross-relabeling. This method employs classification models to re-label broad and ambiguous instances, resulting in a more diverse dataset. Our evaluation demonstrates that these classification models achieve high accuracy on test subsets, indicating the reliability of the method for label refinement.

    \item \textit{Enhanced model performance via foundation models:}
    
    We employ a foundation model as a feature extractor for the cell segmentation and classification task. In comparison with training a CNN model from scratch, the foundation model backbone only needs fine-tuning, which significantly reduces training time, computational resources and data requirements. We show that using a foundation model encoder leads to better performance in cell segmentation and classification networks than using a CNN-based encoder. This improvement may enable the model to generalize more effectively across various tissue types and imaging methods.
    
    \item \textit{Model optimization through knowledge distillation:}
    
    We show that a smaller student model trained using knowledge distillation on the refined dataset obtained via our cross-relabeling approach from a foundation model achieves comparable performance in cell segmentation and quantification tasks. As a result, this model is more suitable for deployment in environments without high-performance computing resources.
    
    \item \textit{Integration with QuPath:}
    
    We integrate the distilled cell segmentation and classification model into QuPath, a widely used open-source digital pathology platform, to accelerate clinical adaptation by enabling pathologists to more easily incorporate advanced computational tools into their existing workflows.
\end{enumerate}

Through these methodological steps, we aim to bridge the gap between advanced machine learning techniques and practical clinical applications, making accurate and efficient digital pathology accessible in a broader range of healthcare settings.

\section{Refining Existing Datasets Using Cross-Relabeling}
To address the limitations of sparse and ambiguous labeling of cell-level datasets, we propose a generalizable cross-relabeling strategy that can be applied to any dataset containing broadly categorized or imprecisely labeled cell types. This approach involves training and subsequently leveraging classification models to refine broad categories into more specific or biologically relevant classes.
When applied to cell-level data, the methodology includes extracting individual cell images from the dataset patches, preprocessing these images to standardize the size and accommodate partial cells, and then training deep learning classifiers capable of distinguishing between the finer cell subtypes within the coarser categories. 
To illustrate our approach, we focus on the PanNuke \cite{Gamper_Koohbanani_etal._2020, Gamper_Koohbanani_etal._2019} and MoNuSAC \cite{Verma_Kumar_etal._2021} datasets that we have used to train models for cell quantification in our previous works \cite{Shvetsov_Grønnesby_etal._2022,Shvetsov_Sildnes_etal._2024}. We find that for better cell differentiation we have to introduce more granular labels. PanNuke includes a broad classification of "inflammatory" cells, encompassing lymphocytes, macrophages, and neutrophils. Each cell type differs significantly in structure, function, and clinical relevance. Conversely, MoNuSAC uses the label "epithelial" for a class that comprises both benign epithelial cells and malignant neoplastic cells. This practice makes it challenging to differentiate between benign and malignant epithelial cells in the dataset, which is a critical distinction when identifying tumor areas within tissue samples. To address these issues, we implement a cross-relabeling strategy as shown in \hyperref[fig:fig2]{Figure 2}. The key components are two classification models: one is trained on singular cell images from PanNuke data to classify the epithelial meta-class into epithelial and neoplastic classes. The other is trained on MoNuSAC to refine the inflammatory class into lymphocytes, neutrophils, and macrophages.

\begin{figure}[h!]
    \centering
    \includegraphics[width=\textwidth]{images/Figure_2.pdf}
    \caption{Refined dataset generation via cross relabeling}
    \label{fig:fig2}
\end{figure}

The refining approach consists of three consecutive steps. The first is the preprocessing step, in which we extract individual cells from both datasets (\hyperref[fig:fig3]{Figure 3}). The specifics of PanNuke and MoNuSAC patch preparation before cell preprocessing are provided in \hyperref[chap:S1]{Appendix S1}.

\begin{figure}[h!]
    \centering
    \includegraphics[width=\textwidth]{images/Figure_3.pdf}
    \caption{Cell instances preprocessing including (1) cell map extraction, (2) bounding box delineation, (3) adjusting cell boxes and (4) cropping and resizing of cell images}
    \label{fig:fig3}
\end{figure}

During preprocessing, we extract cell type maps from the ground truth label mask and calculate bounding boxes around each cell instance. To accommodate partial cells at patch borders, a common issue in cropped patch images, we employ mirror padding and extend the field of view of the cell label by 15 pixels to capture adjacent cells. We then crop and resize the identified regions to $64 \times 64$ pixels using bicubic interpolation.

The preprocessed PanNuke dataset comprises 68,031 neoplastic and 23,207 epithelial cell images, while MoNuSAC comprises  33,104 lymphocytes, 1,252 neutrophils, and 1,695 macrophages, which we subsequently use in training cell classification models and classifying the cell image data \hyperref[fig:S2]{Appendix Figure S2 (1)}. 

The next step is to train two distinct ResNet50-based classifiers tailored to address the specific labeling challenges inherent in each dataset. We use ResNet50 for classification models due to its proven effectiveness for image classification tasks in histopathology \cite{pan2022reviewmachinelearningapproaches}, and its compatibility with small images. For the PanNuke dataset, we design the classifier, trained on MoNuSAC data, to disaggregate the heterogeneous "inflammatory" cell category into distinct subtypes: lymphocytes, macrophages, and neutrophils. Similarly, for the MoNuSAC dataset, the classifier is trained on PanNuke data and distinguishes between benign and malignant epithelial cells within the overarching "epithelial" label. By applying these targeted classifiers to their respective datasets, we assign more specific labels to individual cell instances, thus enabling us to create a unified dataset.
To ensure a balanced representation of classes, we train both models on datasets that had been equalized to match the size of the least represented class. Thus, we obtain datasets comprising 23,207 samples per class for PanNuke and 1,252 samples per class for MoNuSAC data. Next, we partition both of them into training (70\%), validation (20\%), and testing (10\%) subsets. To mitigate the risk of overfitting, we use a single dropout layer with a rate of p=0.5 in both models and data augmentation using randomized color perturbations, rotation, and horizontal and vertical flipping. We employ AdamW optimizer and the cross-entropy loss function for the training criterion.

To evaluate the two trained models, we measure the classification accuracy on the respective test subsets. The accuracies on the test subset for both classifiers are presented in \hyperref[tab:1]{Table 1}. The PanNuke model achieves an average accuracy of 93.57\%, with higher accuracy for neoplastic cells (96.06\%) compared to epithelial cells (86.26\%). The confusion matrix in Figure A3.1 shows that the model predominantly distinguishes accurately between epithelial and neoplastic tissues, with a substantial number of correct classifications and relatively few misclassifications. The MoNuSAC model demonstrates an average accuracy of 98.92\%, excelling in classifying lymphocytes (99.67\%) and macrophages (94.12\%), with lower performance for neutrophils (85.71\%). The confusion matrix in Figure A3.2 shows that the model identifies lymphocytes and performs reasonably well with macrophages and neutrophils.

\begin{table}[h!]
\renewcommand{\arraystretch}{1.5}
  \centering
  \caption{Cell classification results for PanNuke and MoNuSAC trained models (CI 95\%).}
  \label{tab:1}
  \begin{tabular}{|l|c|c|}
   \hline
   %\rowcolor{gray!30}
    Accuracy               & PanNuke model              & MoNuSAC model              \\
    \hline
    Average      & 0.936 (0.931--0.941)         & 0.989 (0.986--0.993)        \\
    \hline
    Neoplastic   & 0.961 (0.956--0.965)         & -                          \\
    \hline
    Epithelial   & 0.863 (0.849--0.877)         & -                          \\
    \hline
    Lymphocytes  & -                          & 0.997 (0.995--0.999)        \\
    \hline
    Neutrophils  & -                          & 0.857 (0.796--0.918)        \\
    \hline
    Macrophages  & -                          & 0.941 (0.906--0.976)        \\
    \hline
  \end{tabular}
\end{table}

Finally, during the last step, we use the model trained on PanNuke data for epithelial cells in MoNuSAC and the model trained on MoNuSAC for the inflammatory cells class in PanNuke. Specifically, we use classifier models to relabel epithelial cells in MoNuSAC and inflammatory cells in PanNuke data. Then we combine cells with refined labels and the rest of the cells in both datasets to create a refined dataset (\hyperref[fig:S2]{Appendix Figure S2 (2)}). The process of relabeling cells and visualizing them on a patch is shown in \hyperref[fig:fig4]{Figure 4}. The cell counts in the refined dataset are provided in \hyperref[tab:S4]{Appendix Table S4}.

\begin{figure}[h!]
    \centering
    \includegraphics[width=\textwidth, height=0.42\textheight, keepaspectratio]{images/Figure_4.pdf}
    \caption{Cell relabeling procedure for epithelial and inflammatory cell classes}
    \label{fig:fig4}
\end{figure}

%\hfill

Relabeling and combining datasets have been explored in a prior study \cite{Parulekar_Kanwat_etal._2023}, where consecutive fine-tuning on multiple datasets was employed to account for hierarchical class label structures. While the method presented in \cite{Parulekar_Kanwat_etal._2023} is intuitive, it often lacks consistency and requires multiple fine-tuning runs, which can be cumbersome and time-consuming. 
In contrast, cross-relabeling simplifies this process by using specialized classification models tailored to each dataset's specific labeling challenges. This approach provides better transparency and produces a unified dataset encompassing seven distinct cell types across multiple tissue samples, enhancing data diversity for further model training or fine-tuning.

Despite these improvements, cross-relabeling does not entirely resolve issues related to poor labeling quality or the amount of labeled data. Specifically, our results show lower accuracies persist for underrepresented classes, such as macrophages, which may stem from a limited sample availability and intrinsic challenges in distinguishing these cells based solely on H\&E staining. Furthermore, while our method enhances label specificity, it relies on the initial quality of the broad labels; thus, any fundamental inaccuracies in the original annotations can propagate through the relabeling process. Addressing the overall problem of limited data labels may require integrating additional data sources or utilizing complementary immunohistochemical staining methods.
Although the reported performance metrics are obtained from evaluations on the native test sets of each dataset, it is important to note that the primary application of these classifiers is to perform cross-relabeling, where a model trained on one dataset (e.g., PanNuke) is applied to another (e.g., MoNuSAC) and vice versa. We acknowledge that a more systematic evaluation of cross-dataset generalization is needed and could be performed in future work.

Overall, the refined dataset produced by our approach can enhance the supervised training or fine-tuning of cell segmentation and classification models, especially those that utilize pre-trained foundation models to improve feature extraction robustness. In addition, these models can detect nuanced classes that enable researchers to conduct more detailed analyses of biological processes in computational pathology.

\section{Foundation models for robust cell segmentation and classification}

Accurate cell segmentation and classification in digital pathology are hindered by limited labeled data and the fact that conventional CNNs are unable to capture global contextual information due to their local receptive field constraints \cite{Gheflati_Rivaz_2022,Yang_Marcus_etal.}. Traditional approaches in cell quantification have predominantly relied on CNN encoders, such as ResNet50, given their proven effectiveness in semantic segmentation tasks \cite{Deshmane_2023,Graham_Vu_etal._2019,Mukasheva_Koishiyeva_etal._2024,Stringer_Wang_etal._2021}. However, approaches that include fine-tuning of pretrained CNNs, data augmentation, and stain normalization to partially increase data variability and address staining differences often fail to achieve the necessary generalization and robustness across diverse tissue types and staining conditions \cite{G._Wang_W._Li_etal._2018,Gao_Bagci_etal._2018,Karim_El_Khoury_Martin_Fockedey_etal._2021}.

To overcome these challenges, we leverage an encoder-decoder network that uses a foundation model as the encoder and a CNN upsampling decoder (\hyperref[fig:fig5]{Figure 5}) for simultaneous cell segmentation and classification in 2D patches extracted from WSIs. Foundation models with transformer-based architectures are viable alternatives to CNN-based encoders \cite{Shamshad_Khan_etal._2023,Sourget_2023}. They enable the creation of more advanced architectures that can decode or transform learned features more effectively \cite{Chen_Duan_etal._2023,Cheng_Misra_etal._2022,Xie_Wang_etal._2021}.

\begin{figure}[h!]
    \centering
    \includegraphics[width=\textwidth]{images/Figure_5.pdf}
    \caption{UNETR-like model with foundational model as backbone}
    \label{fig:fig5}
\end{figure}

By utilizing a transformer-based encoder, we incorporate global contextual information into the feature extraction process, which is a key advantage of such architectures \cite{Chen_Lu_etal._2021}. This foundation model integration facilitates accurate pixel-wise segmentation and classification without the need for extensive encoder training, thereby potentially improving generalization across varied cellular structures and tissue types.
In our implementation, we employ a modified UNETR \cite{Hatamizadeh_Tang_etal._2021} architecture that combines a vision transformer (ViT) \cite{Dosovitskiy_Beyer_etal._2021} encoder with a CNN-based decoder. The encoder utilizes the pretrained H-Optimus foundation model, which contains 1.1 billion parameters and is trained on over 500,000 H\&E stained WSIs \cite{Saillard_Jenatton_etal._2024}. We extract outputs from four evenly spaced transformer blocks $Z_i$, where $i \in [1, 14, 26, 38]$, to serve as residual connections for the CNN decoder. We select these blocks based on our observation that features from non-adjacent levels of the encoder lead to better overall performance on the test subset.

The CNN decoder upsamples the feature representations, acquired from the transformer blocks, to generate an intermediate vector that is handled by two task-specific layers that generate cell segmentation and classification masks. The first task-specific layer is the ‘Cellpose head’,  which is used to delineate cell instances. The layer generates horizontal and vertical gradient maps to form vector fields that are refined through gradient tracking in a post-processing step using the Cellpose algorithm \cite{Stringer_Wang_etal._2021}, known for its efficacy in cell segmentation tasks and generalizability across multiple domains \cite{Pachitariu_Stringer_2022,Stringer_Pachitariu_2024}. The second task-specific layer is the "Cell type head", which assigns labels to individual pixels. In the post-processing step, we determine the output classification label of each segmented cell instance by majority voting over the labeled pixels that comprise the cell in the segmentation map.

To evaluate model performance and measure the impact of adding a foundation model as backbone, we compare it to a ResNet50-based model. ResNet50 is a widely used solution for encoders in segmentation architectures in the medical domain \cite{Deshmane_2023,Graham_Vu_etal._2019,Mukasheva_Koishiyeva_etal._2024,Stringer_Wang_etal._2021}. For the H-Optimus-based model, we utilize frozen weights for the encoder and only fine-tune the decoder to take advantage of the extensive pre-training of the foundation model. For the ResNet50-based model we start with ImageNet \cite{Deng_Dong_etal.} weights and train both encoder and decoder parts. Hyperparameters for the training step are set to be identical, where possible, for comparable evaluation. 
For this evaluation, we deliberately use the PanNuke dataset to provide a standardized and controlled comparison between the H‑Optimus and ResNet50-based models (\hyperref[fig:S2]{Appendix Figure S2 (3)}). Specifically, we use two of the default PanNuke dataset splits (66\%) for training and validation, and reserve the third split (33\%) for testing.

To address the challenge of cell class imbalance in the PanNuke dataset, which is a common characteristic in most cell-level H\&E patch datasets, both models’ training processes employ a weighted loss function comprising cross-entropy and focal loss \cite{Lin_Goyal_etal._2018}. The focal loss component is adjusted with coefficients derived from each cell class' instance frequency, emphasizing learning from underrepresented classes and enhancing the model's sensitivity to rare but significant cellular patterns. The cross-entropy loss is augmented with spectral decoupling regularization \cite{Pezeshki_Kaba_etal._2021,Pohjonen_Stürenberg_etal._2022} and spatially varying label smoothing \cite{Islam_Glocker_2021}, which potentially stabilizes training and improves generalization in case of complex tissue morphologies. For optimization, we employ the \textit{AdamW} \cite{Loshchilov_Hutter_2019} to counter unbalanced class scenarios, with cosine annealing learning rate scheduler.

We utilize the scikit-learn library \cite{Van_der_Walt_Schönberger_etal._2014} and HoVer-Net \cite{Graham_Vu_etal._2019} implementations of $R^2$ (the coefficient of determination) and $PQ$ (panoptic quality) to evaluate our experiments. Complete mathematical formulations and detailed explanations of these metrics are provided in \hyperref[chap:S5]{Appendix S5}. To compute confidence intervals, we use nonparametric bootstrapping, where after calculating the metric on the full sample, we generated 1000 bootstrap replicates by resampling with replacement and then determined the 95\% confidence intervals as the 2.5th and 97.5th percentiles of the resulting empirical distribution.

%\hfill

The model comparisons are summarized in \hyperref[tab:2]{Table 2}. The H‑Optimus-based model achieves higher $R^2$ across all cell classes compared to the ResNet50-based model, which means that its predictions are more closely aligned with the PanNuke cell counts, indicating a stronger correlation with the observed data. Notably, the improvement of $R^2_{dead}$ may be an indicator of better global contextual representations provided by the foundation model backbone. In terms of segmentation and classification quality combined, measured by the PQ score, the H‑Optimus-based model demonstrates notable improvements across most cell classes. Overall, the average $R^2$ improved from 0.575 to 0.871, while the average $PQ$ score improved from 0.450 to 0.492, demonstrating better performance of the H-Optimus-based model.

\begin{table}[h!]
\renewcommand{\arraystretch}{1.5}
  \centering
  \caption{Cell quantification metrics for baseline and proposed models (CI 95\%).}
  \label{tab:2}
  \begin{tabular}{|l|c|c|}
    \hline
    %\rowcolor{gray!30}
    Metric             & Resnet50-based            & H-optimus-based              \\
    \hline
    $R^2_{neoplastic}$    & 0.681 (0.576--0.769)       & \textbf{0.941 (0.917--0.960)} \\
    \hline
    $R^2_{inflammatory}$  & 0.863 (0.778--0.903)       & \textbf{0.949 (0.918--0.966)} \\
    \hline
    $R^2_{connective}$    & 0.600 (0.488--0.698)       & 0.609 (0.436--0.772)          \\
    \hline
    $R^2_{dead}$          & 0.097 (-11.389--0.669)     & 0.925 (0.404--0.982)          \\
    \hline
    $R^2_{epithelial}$    & 0.635 (0.490--0.747)       & \textbf{0.930 (0.886--0.964)} \\
    \hline
    $PQ_{neoplastic}$       & 0.517 (0.499--0.535)       & \textbf{0.589 (0.575--0.604)} \\
    \hline
    $PQ_{inflammatory}$     & 0.455 (0.429--0.482)       & \textbf{0.528 (0.507--0.549)} \\
    \hline
    $PQ_{connective}$       & 0.416 (0.400--0.431)       & \textbf{0.451 (0.436--0.465)} \\
    \hline
    $PQ_{dead}$             & 0.374 (0.342--0.408)       & 0.292 (0.209--0.365)          \\
    \hline
    $PQ_{epithelial}$       & 0.488 (0.460--0.519)       & \textbf{0.599 (0.579--0.618)} \\
    \hline
  \end{tabular}
\end{table}

Our results  show that integrating the H‑Optimus foundation model within the UNETR architecture enhances the model's ability to segment and classify cells across diverse tissues from PanNuke data. The pretrained transformer encoder provides robust feature representations, resulting in higher average $R^2$ and $PQ$ scores compared to the CNN-based model. This leads to more reliable cell quantification and more accurate downstream analysis. Additionally, the streamlined fine-tuning process reduces computational overhead and training time, making the model more adaptable for new data.

Despite these advancements, the foundation model-based approach does not fully resolve all challenges related to cell segmentation and classification. We observe lower metric scores for underrepresented classes in the training data. Furthermore, foundation models typically encompass billions of parameters, resulting in substantial computational and memory requirements. It therefore poses challenges for deployment in resource-constrained environments, limiting their practical applicability in certain clinical settings.

\section{Model optimization via Knowledge Distillation}

To address the limitations posed by the extensive size of foundation models, we implement knowledge distillation — a model compression technique that leverages the teacher-student paradigm \cite{Hinton_Vinyals_etal._2015}. By training a smaller, more efficient student model to replicate the output of a larger, pre-trained teacher model, we retain performance while significantly reducing the model's complexity and resource requirements (\hyperref[fig:fig6]{Figure 6}).

\begin{figure}[h!]
    \centering
    \includegraphics[width=\textwidth, height=0.45\textheight, keepaspectratio]{images/Figure_6.pdf}
    \caption{Knowledge distillation framework for training a student model using a pre-trained teacher}
    \label{fig:fig6}
\end{figure}

We employ knowledge distillation to compress the H‑Optimus-based teacher model into a more efficient student model. The teacher model is the modified UNETR architecture with the H‑Optimus foundation model described in the previous chapter. The student model is based on a UNet architecture augmented with residual connections and incorporates a smaller ViT encoder with 9 million parameters \cite{Steiner_Kolesnikov_etal._2022,Wightman_2019}. 

First, we fine-tune the teacher model using the refined dataset from the cross-relabeling procedure (Section 2). Initially we train the decoder of the teacher model while keeping the encoder weights frozen. We split the refined dataset into train (70\%), validation (20\%) and test (10\%) subsets (\hyperref[fig:S2]{Appendix Figure S2 (4)}). During fine-tuning, we use the train and validation subsets, while leaving the test subset for model evaluation. We set the training procedure and model hyperparameters to be identical to those that were used to demonstrate the utility of foundation models for the simultaneous cell segmentation and classification task.

Next, we perform knowledge distillation from teacher to student using the refined dataset used to fine-tune the teacher model. The student model is trained to replicate the teacher model's outputs. We utilize a specialized loss function that aligns the student's predicted probability distribution with the teacher's, incorporating the teacher's class probability distribution derived from the output. Following the methodology of Hinton et al. \cite{Hinton_Vinyals_etal._2015}, we experiment with various hyperparameter settings for the temperature ($T$) and the balancing coefficients ($\alpha$ and $\beta$) in the loss function. We vary $T$ from 1 to 20 and adjust $\alpha$ and $\beta$ to balance the distillation and student losses. Through iterative tuning and evaluation, we identify that setting $T=14$, $\alpha=0.3$, and $\beta=0.7$ yields a configuration that converges and closely approximates the teacher model's performance during training.

Finally, we assess the performance of both models using the $R^2$ and $PQ$ (defined in \hyperref[chap:S5]{Appendix S5}) on the test set of the refined dataset (\hyperref[tab:3]{Table 3}). We observe that the 95\% confidence intervals overlap for most cell types, so we cannot claim statistically significant performance differences between the teacher and student models. One exception appears in the neoplastic class. The teacher model produces an $R^2$ of 0.919, while the student model shows an $R^2$ of 0.852. In addition, the student model achieves higher $PQ$ values for the neoplastic and connective classes, though the confidence intervals show overlap.

\begin{table}[h!]
\renewcommand{\arraystretch}{1.5}
  \centering
  \caption{Cell quantification metrics for teacher and distilled student models (CI 95\%).}
  \label{tab:3}
  \begin{tabular}{|l|c|c|}
    \hline
    %\rowcolor{gray!30}
    Metric & Teacher & Student \\
    \hline
    $R^2_{neoplastic}$    & \textbf{0.919} (0.898--0.939) & 0.852 (0.800--0.891) \\
    \hline
    $R^2_{lymphocyte}$    & 0.969 (0.956--0.977)         & 0.969 (0.956--0.978) \\
    \hline
    $R^2_{connective}$    & 0.694 (0.548--0.809)         & 0.618 (0.469--0.741) \\
    \hline
    $R^2_{dead}$          & 0.755 (0.400--0.908)         & 0.424 (0.100--0.731) \\
    \hline
    $R^2_{epithelial}$    & 0.922 (0.870--0.958)         & 0.843 (0.738--0.917) \\
    \hline
    $R^2_{macrophage}$    & 0.384 (-0.369--0.724)        & 0.704 (0.352--0.859) \\
    \hline
    $R^2_{neutrofil}$     & 0.854 (0.578--0.929)         & 0.833 (0.502--0.925) \\
    \hline
    $PQ_{neoplastic}$       & 0.581 (0.569--0.593)         & 0.601 (0.588--0.613) \\
    \hline
    $PQ_{lymphocyte}$       & 0.536 (0.520--0.553)         & 0.563 (0.544--0.579) \\
    \hline
    $PQ_{connective}$       & 0.436 (0.421--0.451)         & 0.457 (0.441--0.474) \\
    \hline
    $PQ_{dead}$             & 0.272 (0.235--0.315)         & 0.279 (0.201--0.369) \\
    \hline
    $PQ_{epithelial}$       & 0.522 (0.500--0.545)         & 0.530 (0.506--0.555) \\
    \hline
    $PQ_{macrophage}$       & 0.524 (0.459--0.588)         & 0.474 (0.405--0.543) \\
    \hline
    $PQ_{neutrofil}$        & 0.541 (0.490--0.592)         & 0.565 (0.522--0.607) \\
    \hline
  \end{tabular}
\end{table}


We further decompose the $PQ$ metric into its $SQ$ and $DQ$ components (\hyperref[tab:S6]{Appendix Table S6}). Both models produce nearly identical $SQ$ values, which indicates that they predict instance boundaries with similar precision. Although the student model shows some improvement in $DQ$ scores for certain classes, the confidence intervals overlap and do not confirm a statistically significant difference.

We observe that the student and teacher models yield comparable detection performance despite the student model using a much smaller and simpler architecture. A model with fewer parameters reduces the risk of overfitting when training data are scarce relative to the model’s complexity \cite{Farias_Ludermir_etal._2022}. The knowledge distillation process also encourages the student model to focus on the most generalizable detection features learned from the teacher. These factors enable the student model to achieve similar detection performance across different cell types.

Additionally, considering the model sizes reported in \hyperref[tab:4]{Table 4}, the distilled model achieves a significant reduction compared to the teacher model, with a 48-fold decrease in parameter count and a 5.5-fold reduction in on-disk size. In inference mode, the teacher model requires 16 GB of VRAM for a batch size of 32, while the distilled model only needs 3 GB of VRAM for the same batch size. These reductions make the distilled model significantly more practical for fine-tuning and deployment in resource-constrained environments.

\begin{table}[h!]
\renewcommand{\arraystretch}{1.5}
  \centering
  \caption{Parameter counts and size of teacher and distilled model}
  \label{tab:4}
  \adjustbox{max width=\textwidth}{%
  \begin{tabular}{|l|c|c|c|}
    \hline
    %\rowcolor{gray!30}
    Metric & H-optimus-based (Teacher) & mobileViT-based (Student) & Magnitude of difference \\
    \hline
    Parameters count       & 1,158,917,906   & \textbf{24,093,393}   & \textbf{48x}  \\
    \hline
    Estimated Total Size (MB) & 87,912       & \textbf{15,935}    & \textbf{5.5x} \\
    \hline
  \end{tabular}%
}
\end{table}

%\hfill

With recent advancements in complex network architectures and the use of pretrained encoders to achieve state-of-the-art performance \cite{Baumann_Dislich_etal._2024,Hörst_Rempe_etal._2024} in cell segmentation and classification tasks, model size, computational complexity, and processing times have increased. This limits the scalability and accessibility of these models. As we demonstrate, this may be mitigated using knowledge distillation. Studies in the field of natural language processing have demonstrated the efficacy of knowledge distillation in retaining the capabilities of the teacher model while achieving significant reductions in size and complexity \cite{Huangpu_Gao_2024,Sun_Yu_etal.}. 

We demonstrate the feasibility of knowledge distillation in digital pathology, specifically for cell segmentation and classification tasks. Moreover, we achieve this performance while also significantly reducing the parameter count. In addressing the challenge of knowledge transfer, we found that distillation from a transformer-based model to a smaller transformer is more straightforward than attempting to map transformer features to CNN blocks. In our experiments, using a CNN-based network as a student results in worse cell quantification performance due to the structural constraints of CNN feature space dimensions. 

Although our primary approach relies on a transformer-based student model that performs well, it can be further optimized to incorporate advantages from CNN architectures. For example, employing alternative techniques such as using ViT adapters \cite{Chen_Duan_etal._2023} or $1 \times 1$ convolutions to adjust feature map sizes may be beneficial for harnessing CNN advantages like enhanced local feature extraction. Moreover, if additional performance improvements are desired, the process can be further enhanced by applying supplementary knowledge distillation techniques, such as self-distillation \cite{Zhang_Song_etal._2019} or online distillation \cite{Houyon_Cioppa_etal._2023}.

Despite these promising results, further validation on independent datasets is necessary to fully understand the model's limitations. Underrepresented classes may pose challenges when addressing complex cases. Pathologists need to validate these models to adopt them in clinical settings. While the distilled models are smaller and more deployable, a technological gap persists because pathologists traditionally rely on established methods for inspecting WSIs and diagnosing diseases. Addressing the complexities involved in deploying models for inference and supporting pathologists in adopting new tools is essential for integrating these models into clinical workflows.

\section{Model integration with QuPath}
Digital pathology tools with graphical user interfaces are essential for visualizing and analyzing WSIs. To make our student model useful in clinical pathology workflows, it needs to be integrated into a tool that enables inspecting regions, creating annotations, and providing quantitative analyses of biomarkers. Therefore, we integrate the trained student model from the previous chapter into the QuPath open‑source platform \cite{Bankhead_Loughrey_etal._2017}. QuPath provides the required annotation, visualization, and analysis tools to interpret complex histological data, including workflows for cell segmentation, classification, and quantification (\hyperref[fig:fig7]{Figure 7}). 

\begin{figure}[h!]
    \centering
    \includegraphics[width=\textwidth]{images/Figure_7.pdf}
    \caption{Visualization of model-generated cell quantification annotations (left) and the corresponding unannotated slide (right) in QuPath}
    \label{fig:fig7}
\end{figure}

To identify the regions in a WSI critical for prognosticating tumor development, such as specific tumor areas or border regions without overlapping healthy tissue, the pathologist uses QuPath to outline these regions. Then, the pathologist initiates a cell segmentation and classification script through the QuPath interface for the selected regions. The resulting annotations and quantified cell information are then directly overlaid onto the WSI in the QuPath interface. Additional design and implementation details are in \hyperref[chap:S7]{Appendix S7}. 

Two common approaches for integrating deep learning models into QuPath are Java‑based native QuPath extensions \cite{Goldsborough_Philps_etal._2024} and the execution of RESTful API requests to a model server coupled with handling the response via an extension, as demonstrated in the application of cell segmentation models applied to immunofluorescence images \cite{Sugawara_2023}. While the community is actively working on these integration strategies, there is currently no universal solution that fully addresses all integration and performance requirements.

Extensions may offer better integration with QuPath, allowing slightly improved performance and more widespread usage of the built-in QuPath models, but they lack the flexibility to customize models and modify their behavior. For example, the newest version of QuPath includes models such as StarDist \cite{Weigert_Schmidt} and InstanSeg \cite{Goldsborough_Philps_etal._2024} that can perform cell segmentation. Both models pose limitations when applied to simultaneous cell segmentation and classification. StarDist performs well only on convex, round shapes by design, whereas some neoplastic, inflammatory, and connective cells exhibit complex and non-convex shapes. InstanSeg provides only semantic segmentation without assigning classes to the segmented cells.

%\hfill

In contrast, our approach offers an alternative integration strategy. It utilizes the paquo library to directly interact with QuPath’s internal application programming interface from within Python. This enables data exchange and processing without the need for intermediate conversion steps and provides greater control over model customization, retraining, and the incorporation of custom processing steps.

The integration of our custom model with QuPath underscores its potential to significantly enhance the diagnostic process by reducing the time burden on pathologists and enabling them to focus on more complex interpretative tasks using familiar software. Leveraging a tool that is already well-established among pathologists increases the likelihood of its adoption into daily clinical workflows. The quantitative data generated through the automated workflow is critical for both clinical decision-making and research, facilitating more accurate biomarker analysis, enabling robust statistical evaluations, and supporting hypothesis generation and testing. Additionally, by streamlining cell segmentation and classification, the tool enhances the scalability and reproducibility of pathological assessments, ultimately contributing to improved diagnostic accuracy and patient outcomes.

\section{Conclusion and future work}

In this study, we address critical challenges in digital pathology and tackle the usability and deployment issues of the developed models in standard computing environments without the need for high-performance computing systems. Our multi-faceted approach encompasses data refinement through cross-relabeling, leveraging foundation models for robust cell segmentation and classification, optimizing model performance via knowledge distillation, and integrating the optimized model into the QuPath software for practical application. This approach is used to construct a capable, versatile, and adjustable model for cell segmentation and classification, with enhanced performance and usability.

\begin{sloppypar}
While our approach shows potential in the field of computational pathology, certain limitations persist. 
For example, our implementation currently exhibits lower performance in detecting macrophages. 
This serves as an instance of the broader challenge of accurately identifying complex cell types. In order to address this issue, extending our approach to incorporate additional data sources, exploring alternative modeling approaches, and integrating other imaging modalities such as immunohistochemical staining may help improve detection accuracy. Moreover, although the distilled model reduces computational demands, integrating advanced deep learning models into clinical practice requires addressing technological gaps and potential resistance to adopting new tools within established diagnostic processes.
\end{sloppypar}

Future work could focus on several key areas to refine the proposed approach and facilitate its adoption in clinical environments. Enhancing the cell-relabeling process with additional datasets \cite{Graham_Jahanifar_etal._2021} could improve the representation of underrepresented cell types and enhance overall model performance. Also, incorporating additional data sources, such as multi-modal imaging or complementary staining methods, may address limitations related to cell type differentiation and class imbalance. Exploring other foundation models \cite{Vorontsov_Bozkurt_etal._2024,Zimmermann_Vorontsov_etal._2024} or introducing additional modalities \cite{Ding_Wagner_etal._2024,Vaidya_Zhang_etal._2025} may provide alternative architectures better suited to specific tasks or offer improved efficiency. Implementing more complex knowledge distillation techniques \cite{Houyon_Cioppa_etal._2023,Zhang_Song_etal._2019} could further optimize the model's performance and adaptability. Additionally, deeper integration with QuPath or other digital pathology software could provide pathologists more control over cell quantification analysis directly within the QuPath interface, thereby increasing accessibility and usability. Such enhancements would not only refine model performance but also ensure greater adaptability and scalability within various clinical environments. Finally, extensive validation of the model by pathologists and benchmarking against independent datasets are essential steps toward establishing the model's reliability and fostering confidence in its clinical utility.

\section*{Acknowledgments} 
This work was funded in part by the Research Council of Norway grant no. 309439 SFI Visual Intelligence, and the North Norwegian Health Authority grant no. HNF1521-20.

\bibliographystyle{IEEEtran}
\begin{sloppypar}
\begin{thebibliography}{99}

\bibitem{chaplot2020neural} Chaplot, Devendra Singh, et al. "Neural topological slam for visual navigation." Proceedings of the IEEE/CVF conference on computer vision and pattern recognition. 2020.

\bibitem{maksymets2021thda} Maksymets, Oleksandr, et al. "Thda: Treasure hunt data augmentation for semantic navigation." Proceedings of the IEEE/CVF International Conference on Computer Vision. 2021.

\bibitem{mezghan2022memory} Mezghan, Lina, et al. "Memory-augmented reinforcement learning for image-goal navigation." 2022 IEEE/RSJ International Conference on Intelligent Robots and Systems (IROS). IEEE, 2022.

\bibitem{al2022zero} Al-Halah, Ziad, Santhosh Kumar Ramakrishnan, and Kristen Grauman. "Zero experience required: Plug \& play modular transfer learning for semantic visual navigation." Proceedings of the IEEE/CVF Conference on Computer Vision and Pattern Recognition. 2022.

\bibitem{ye2021auxiliary} Ye, Joel, et al. "Auxiliary tasks and exploration enable objectgoal navigation." Proceedings of the IEEE/CVF international conference on computer vision. 2021.

\bibitem{chaplot2020object} Chaplot, Devendra Singh, et al. "Object goal navigation using goal-oriented semantic exploration." Advances in Neural Information Processing Systems 33 (2020)

\bibitem{ramakrishnan2022poni} Ramakrishnan, Santhosh Kumar, et al. "Poni: Potential functions for objectgoal navigation with interaction-free learning." Proceedings of the IEEE/CVF Conference on Computer Vision and Pattern Recognition. 2022.

\bibitem{ramrakhya2022habitat} Ramrakhya, Ram, et al. "Habitat-web: Learning embodied object-search strategies from human demonstrations at scale." Proceedings of the IEEE/CVF Conference on Computer Vision and Pattern Recognition. 2022.

\bibitem{mousavian2019visual} Mousavian, Arsalan, et al. "Visual representations for semantic target driven navigation." 2019 International Conference on Robotics and Automation (ICRA). IEEE, 2019.

\bibitem{dhariwal2021diffusion} Dhariwal, Prafulla, and Alexander Nichol. "Diffusion models beat gans on image synthesis." Advances in neural information processing systems 34 (2021)

\bibitem{ho2022classifier} Ho, Jonathan, and Tim Salimans. "Classifier-free diffusion guidance." arXiv preprint arXiv:2207.12598 (2022).

\bibitem{nichol2021glide} Nichol, Alex, et al. "Glide: Towards photorealistic image generation and editing with text-guided diffusion models." arXiv preprint arXiv:2112.10741 (2021)

\bibitem{brooks2023instructpix2pix} Brooks, Tim, Aleksander Holynski, and Alexei A. Efros. "Instructpix2pix: Learning to follow image editing instructions." Proceedings of the IEEE/CVF Conference on Computer Vision and Pattern Recognition. 2023.

\bibitem{fu2023guiding} Fu, Tsu-Jui, et al. "Guiding instruction-based image editing via multimodal large language models." arXiv preprint arXiv:2309.17102 (2023).

\bibitem{geng2024instructdiffusion} Geng, Zigang, et al. "Instructdiffusion: A generalist modeling interface for vision tasks." Proceedings of the IEEE/CVF Conference on Computer Vision and Pattern Recognition. 2024.

\bibitem{zhou2024minedreamer} Zhou, Enshen, et al. "Minedreamer: Learning to follow instructions via chain-of-imagination for simulated-world control." arXiv preprint arXiv:2403.12037 (2024).

\bibitem{zhou2023esc} Zhou, Kaiwen, et al. "Esc: Exploration with soft commonsense constraints for zero-shot object navigation." International Conference on Machine Learning. PMLR, 2023.

\bibitem{yu2023l3mvn} Yu, Bangguo, Hamidreza Kasaei, and Ming Cao. "L3mvn: Leveraging large language models for visual target navigation." 2023 IEEE/RSJ International Conference on Intelligent Robots and Systems (IROS). IEEE, 2023.

\bibitem{gadre2023cows} Gadre, Samir Yitzhak, et al. "Cows on pasture: Baselines and benchmarks for language-driven zero-shot object navigation." Proceedings of the IEEE/CVF Conference on Computer Vision and Pattern Recognition. 2023.

\bibitem{shah2023navigation} Shah, Dhruv, et al. "Navigation with large language models: Semantic guesswork as a heuristic for planning." Conference on Robot Learning. PMLR, 2023.

\bibitem{cai2024bridging} Cai, Wenzhe, et al. "Bridging zero-shot object navigation and foundation models through pixel-guided navigation skill." 2024 IEEE International Conference on Robotics and Automation (ICRA). IEEE, 2024.

\bibitem{yu2023co} Yu, Bangguo, Hamidreza Kasaei, and Ming Cao. "Co-NavGPT: Multi-robot cooperative visual semantic navigation using large language models." arXiv preprint arXiv:2310.07937 (2023).

\bibitem{wu2024voronav} Wu, Pengying, et al. "Voronav: Voronoi-based zero-shot object navigation with large language model." arXiv preprint arXiv:2401.02695 (2024).

\bibitem{qin2023mp5} Qin, Yiran, et al. "Mp5: A multi-modal open-ended embodied system in minecraft via active perception." arXiv preprint arXiv:2312.07472 (2023).

\bibitem{du2024learning} Du, Yilun, et al. "Learning universal policies via text-guided video generation." Advances in Neural Information Processing Systems 36 (2024).

\bibitem{ajay2024compositional} Ajay, Anurag, et al. "Compositional foundation models for hierarchical planning." Advances in Neural Information Processing Systems 36 (2024).

\bibitem{liang2024skilldiffuser} Liang, Zhixuan, et al. "Skilldiffuser: Interpretable hierarchical planning via skill abstractions in diffusion-based task execution." Proceedings of the IEEE/CVF Conference on Computer Vision and Pattern Recognition. 2024.

\bibitem{heusel2017gans} Heusel, Martin, et al. "Gans trained by a two time-scale update rule converge to a local nash equilibrium." Advances in neural information processing systems 30 (2017).

\bibitem{zhang2018unreasonable} Zhang, Richard, et al. "The unreasonable effectiveness of deep features as a perceptual metric." Proceedings of the IEEE conference on computer vision and pattern recognition. 2018.

\bibitem{brown2020language} Brown, Tom B. "Language models are few-shot learners." arXiv preprint arXiv:2005.14165 (2020).

\bibitem{podell2023sdxl} Podell, Dustin, et al. "Sdxl: Improving latent diffusion models for high-resolution image synthesis." arXiv preprint arXiv:2307.01952 (2023).

\bibitem{brohan2022rt} Brohan, Anthony, et al. "Rt-1: Robotics transformer for real-world control at scale." arXiv preprint arXiv:2212.06817 (2022).

\bibitem{brohan2023rt} Brohan, Anthony, et al. "Rt-2: Vision-language-action models transfer web knowledge to robotic control." arXiv preprint arXiv:2307.15818 (2023).

\bibitem{li2024manipllm} Li, Xiaoqi, et al. "Manipllm: Embodied multimodal large language model for object-centric robotic manipulation." Proceedings of the IEEE/CVF Conference on Computer Vision and Pattern Recognition. 2024.

\bibitem{shah2023vint} Shah, Dhruv, et al. "ViNT: A foundation model for visual navigation." arXiv preprint arXiv:2306.14846 (2023).

\bibitem{liu2024visual} Liu, Haotian, et al. "Visual instruction tuning." Advances in neural information processing systems 36 (2024).

\bibitem{hu2021lora} Hu, Edward J., et al. "Lora: Low-rank adaptation of large language models." arXiv preprint arXiv:2106.09685 (2021).

\bibitem{qin2023supfusion} Qin, Yiran, et al. "SupFusion: Supervised LiDAR-camera fusion for 3D object detection." Proceedings of the IEEE/CVF International Conference on Computer Vision. 2023.

\bibitem{qin2024worldsimbench} Qin, Yiran, et al. "Worldsimbench: Towards video generation models as world simulators." arXiv preprint arXiv:2410.18072 (2024).

\bibitem{yu2025gamefactory} Yu, Jiwen, et al. "GameFactory: Creating New Games with Generative Interactive Videos." arXiv preprint arXiv:2501.08325 (2025).

\bibitem{zhou2024code} Zhou, Enshen, et al. "Code-as-Monitor: Constraint-aware Visual Programming for Reactive and Proactive Robotic Failure Detection." arXiv preprint arXiv:2412.04455 (2024).

\bibitem{zhang2024ad} Zhang, Zaibin, et al. "AD-H: Autonomous Driving with Hierarchical Agents." arXiv preprint arXiv:2406.03474 (2024).

\bibitem{wang2024toward} Wang, Chaoqun, et al. "Toward Accurate Camera-based 3D Object Detection via Cascade Depth Estimation and Calibration." arXiv preprint arXiv:2402.04883 (2024).

\bibitem{huang2024story3d} Huang, Yuzhou, et al. "Story3d-agent: Exploring 3d storytelling visualization with large language models." arXiv preprint arXiv:2408.11801 (2024).

\bibitem{savinov2018semi} Savinov, Nikolay, Alexey Dosovitskiy, and Vladlen Koltun. "Semi-parametric topological memory for navigation." arXiv preprint arXiv:1803.00653 (2018).

\bibitem{majumdar2022zson} Majumdar, Arjun, et al. "Zson: Zero-shot object-goal navigation using multimodal goal embeddings." Advances in Neural Information Processing Systems 35 (2022): 32340-32352.

\bibitem{yadav2023offline} Yadav, Karmesh, et al. "Offline visual representation learning for embodied navigation." Workshop on Reincarnating Reinforcement Learning at ICLR 2023. 2023.

\bibitem{yadav2023ovrl} Yadav, Karmesh, et al. "Ovrl-v2: A simple state-of-art baseline for imagenav and objectnav." arXiv preprint arXiv:2303.07798 (2023).

\bibitem{sun2024fgprompt} Sun, Xinyu, et al. "FGPrompt: fine-grained goal prompting for image-goal navigation." Advances in Neural Information Processing Systems 36 (2024).

\bibitem{zhu2017target} Zhu, Yuke, et al. "Target-driven visual navigation in indoor scenes using deep reinforcement learning." 2017 IEEE international conference on robotics and automation (ICRA). IEEE, 2017.

\bibitem{koh2024generating} Koh, Jing Yu, Daniel Fried, and Russ R. Salakhutdinov. "Generating images with multimodal language models." Advances in Neural Information Processing Systems 36 (2024).

\bibitem{krantz2022instance} Krantz, Jacob, et al. "Instance-specific image goal navigation: Training embodied agents to find object instances." arXiv preprint arXiv:2211.15876 (2022).

\bibitem{schulman2017proximal} Schulman, John, et al. "Proximal policy optimization algorithms." arXiv preprint arXiv:1707.06347 (2017).

\bibitem{anderson2018evaluation} Anderson, Peter, et al. "On evaluation of embodied navigation agents." arXiv preprint arXiv:1807.06757 (2018).

\bibitem{lin2024navcot} Lin, Bingqian, et al. "NavCoT: Boosting LLM-Based Vision-and-Language Navigation via Learning Disentangled Reasoning." arXiv preprint arXiv:2403.07376 (2024).

\bibitem{NavGPT} Zhou, Gengze, Yicong Hong, and Qi Wu. "Navgpt: Explicit reasoning in vision-and-language navigation with large language models." Proceedings of the AAAI Conference on Artificial Intelligence.

\bibitem{hahn2021no} Hahn, Meera, et al. "No rl, no simulation: Learning to navigate without navigating." Advances in Neural Information Processing Systems 34 (2021): 26661-26673.

\bibitem{li2025t2isafety} Li, Lijun, et al. "T2ISafety: Benchmark for Assessing Fairness, Toxicity, and Privacy in Image Generation." arXiv preprint arXiv:2501.12612 (2025).

\bibitem{an2024agfsync} An, Jingkun, et al. "AGFSync: Leveraging AI-Generated Feedback for Preference Optimization in Text-to-Image Generation." arXiv preprint arXiv:2403.13352 (2024).


\end{thebibliography}
\end{sloppypar}

\clearpage
\beginsupplement
\section*{Appendix}
\renewcommand{\thesubsection}{S\arabic{subsection}}

\subsection{\label{chap:S1}PanNuke and MoNuSAC preprocessing}
The PanNuke dataset comprises a set of 7,901 RGB patches, each with dimensions of $256 \times 256$ pixels, which we set as the standard patch size for our analysis. In contrast, the MoNuSAC dataset encompasses 294 images of heterogeneous dimensions. To standardize the MoNuSAC images with our experiments, we implement a standardization protocol. Specifically, for images exceeding the dimensions of $256 \times 256$ pixels, we segment them into equal-sized patches and apply mirror padding to the remaining portions to avoid information loss at the peripherals. Patches with dimensions less than $128 \times 128$ pixels are excluded from the dataset due to the insufficient resolution to capture relevant cellular details. For patches where either dimension falls between 128 and 256 pixels, we employ upsampling to achieve the standard patch size. As a result, we obtain a total of 2,823 RGB patches derived from the MoNuSAC dataset for subsequent analysis. For additional details on the MoNuSAC data preparation process, refer to the source code \cite{Shvetsov_2025a}.
\clearpage

\subsection{\label{chap:S2}Data usage for the methodology}

\counterwithin{figure}{subsection}
\renewcommand{\thefigure}{S\arabic{subsection}}

\begin{figure}[h!]
    \centering
    \includegraphics[width=\textwidth, height=0.85\textheight, keepaspectratio]{images/A2.pdf}
    \caption{Overview of the methodology for cross-labeling, dataset refinement, and model comparison. (1) Cross-relabeling - training and testing cell classification models, (2) Cross-relabeling - using cell classification models to create refined dataset, (3) Fine-tuning and training models for comparison, (4) Student knowledge distillation with refined dataset}
    \label{fig:S2}
\end{figure}
\clearpage

\subsection{\label{chap:S3}Confusion matrices for classification models}
\counterwithin{figure}{subsection}
\renewcommand{\thefigure}{S\arabic{subsection}.\arabic{figure}}

\begin{figure}[h!]
    \centering
    \includegraphics[width=\textwidth, height=0.4\textheight, keepaspectratio]{images/A3_1.pdf}
    \caption{Confusion matrix for PanNuke trained model}
    \label{fig:S3.1}
\end{figure}

\begin{figure}[h!]
    \centering
    \includegraphics[width=\textwidth, height=0.4\textheight, keepaspectratio]{images/A3_2.pdf}
    \caption{Confusion matrix for MoNuSAC trained model}
    \label{fig:S3.2}
\end{figure}

\clearpage

\subsection{\label{chap:S4}Datasets cell counts}

\counterwithin{table}{subsection}
\renewcommand{\thetable}{S\arabic{subsection}}

\begin{table}[h!]
\renewcommand{\arraystretch}{2.0}
\centering
\caption{\label{tab:S4}Cell counts for PanNuke, MoNuSAC and refined datasets. Numbers in parentheses indicate preprocessed cell counts for cell classifier models training and testing.}
%\adjustbox{max width=\textwidth}{%
\begin{tabular}{|l|c|c|c|}
\hline
%\rowcolor{gray!30}
Cell type & PanNuke & MoNuSAC & Refined \\
\hline
Neoplastic & 77,403 (68,031) & - & 105,451 \\
\hline
Epithelial & 26,572 (23,207) & - & 29,926 \\
\hline
Epithelial (benign and malignant) & - & 31,402 & - \\
\hline
Inflammatory & 32,276 & - & - \\
\hline
Lymphocytes & - & 37,045 (33,104) & 65,275 \\
\hline
Neutrophils & - & 1,355 (1,252) & 3,833 \\
\hline
Macrophage & - & 1,842 (1,695) & 3,410 \\
\hline
Dead & 2,908 & - & 2,908 \\
\hline
Connective & 50,585 & - & 50,585 \\
\hline
\end{tabular}
%
%}
\end{table}



\clearpage

\subsection{\label{chap:S5}Definition of validation metrics}
\counterwithin{equation}{subsection}
\renewcommand{\theequation}{\arabic{equation}}

\subsubsection{\label{chap:S5.1}R\textsuperscript{2}}
The coefficient of determination, denoted as $R^2$, is a statistical measure that represents the proportion of variance in the dependent variable that is predictable from the independent variables. In the context of cell quantification in pathology, $R^2$ is used to assess how well the predicted quantities of different cell types in a patch align with the actual quantities observed in the ground truth data, with higher values representing more accurate quantification. $R^2$ is defined as
\begin{equation*}
R^2 = 1 - \frac{\sum_{i=1}^n (y_i - \hat{y}_i)^2}{\sum_{i=1}^n (y_i - \bar{y})^2},
\end{equation*}
where $y_i$ represents the actual number of cells of a specific type in the $i$-th image, $\hat{y}_i$ represents the predicted number of cells of that type in the $i$-th image, $\bar{y}$ is the mean of the actual numbers across all images, and $n$ is the total number of images in the dataset.

The $R^2$ metric has a range of $(-\infty, 1]$. An $R^2$ of 1 indicates perfect prediction, where all predicted values exactly match the actual values. An $R^2$ of 0 suggests that the model explains none of the variability of the response data around its mean. If $R^2$ is negative, it indicates that the model performs worse than a model that simply predicts the mean of the actual values for all observations.

\subsubsection{\label{chap:S5.2}PQ}
Panoptic Quality ($PQ$) is a comprehensive metric used to evaluate the performance of segmentation models in tasks that require both instance segmentation and classification. $PQ$ provides a single score that encapsulates both the detection accuracy (i.e., how many objects were correctly identified) and the segmentation quality (i.e., how accurately the objects' boundaries were delineated). This metric is particularly useful in multiclass scenarios where each pixel is classified into distinct categories, such as different cell types in pathology images.

$PQ$ is calculated as the product of two terms: Detection Quality ($DQ$) and Segmentation Quality ($SQ$). It can be expressed as
\begin{equation*}
PQ = DQ \cdot SQ,
\end{equation*}
where
\begin{equation*}
DQ = \frac{TP}{TP + 0.5\, FP + 0.5\, FN},
\end{equation*}
\begin{equation*}
SQ = \frac{\sum_{(p, g) \in \mathcal{M}} IoU(p, g)}{TP}.
\end{equation*}
In these formulas, $TP$ denotes the number of correctly matched instances between ground truth and prediction, $FP$ denotes the predicted instances that have no corresponding ground truth, $FN$ denotes the ground truth instances that were not detected, $IoU(p, g)$ is the Intersection over Union for a pair of matched instances $p$ (prediction) and $g$ (ground truth), and $\mathcal{M}$ is the set of matched pairs.

The $PQ$ metric is calculated for each class and is averaged across classes to provide a global performance measure.

The $PQ$ score has a range of $[0, 1.0]$, where a higher score indicates better performance in both detecting and segmenting the instances correctly. A $PQ$ of 1 signifies perfect identification and segmentation of all instances, whereas a $PQ$ of 0 indicates that no instances were correctly identified and segmented.

\clearpage

\subsection{\label{chap:S6}Segmentation and Detection quality metrics for teacher and student models}

\begin{table}[h!]
\renewcommand{\arraystretch}{2.0}
\centering
\caption{Segmentation and detection quality for student and teacher models (CI 95\%)}
\label{tab:S6}
%\adjustbox{max width=\textwidth}{%
\begin{tabular}{|l|c|c|}
\hline
%\rowcolor{gray!30}
Metric & Teacher & Student \\
\hline
$SQ_{neoplastic}$ & 0.819 (0.815--0.823) & 0.824 (0.819--0.828) \\
\hline
$SQ_{lymphocyte}$ & 0.795 (0.788--0.802) & 0.790 (0.783--0.796) \\
\hline
$SQ_{connective}$ & 0.770 (0.762--0.776) & 0.780 (0.772--0.786) \\
\hline
$SQ_{dead}$ & 0.659 (0.623--0.688) & 0.657 (0.624--0.695) \\
\hline
$SQ_{epithelial}$ & 0.780 (0.770--0.790) & 0.788 (0.779--0.797) \\
\hline
$SQ_{macrophage}$ & 0.788 (0.760--0.810) & 0.757 (0.730--0.783) \\
\hline
$SQ_{neutrofil}$ & 0.782 (0.761--0.801) & 0.775 (0.759--0.792) \\
\hline
$DQ_{neoplastic}$ & 0.706 (0.692--0.719) & 0.727 (0.712--0.741) \\
\hline
$DQ_{lymphocyte}$ & 0.675 (0.656--0.698) & 0.713 (0.691--0.734) \\
\hline
$DQ_{connective}$ & 0.566 (0.546--0.584) & 0.583 (0.565--0.602) \\
\hline
$DQ_{dead}$ & 0.410 (0.361--0.465) & 0.435 (0.306--0.561) \\
\hline
$DQ_{epithelial}$ & 0.668 (0.639--0.694) & 0.673 (0.644--0.702) \\
\hline
$DQ_{macrophage}$ & 0.657 (0.583--0.727) & 0.615 (0.531--0.703) \\
\hline
$DQ_{neutrofil}$ & 0.691 (0.625--0.753) & 0.729 (0.679--0.778) \\
\hline
\end{tabular}
%
%}
\end{table}

\clearpage

\subsection{\label{chap:S7}QuPath integration method}
We adopt an integration strategy leveraging the paquo \cite{Bayer_AG} library, a Python package that enables direct interaction with QuPath’s internal API, thereby facilitating seamless data exchange without intermediate conversion steps. The data processing pipeline (\hyperref[fig:S7]{Appendix Figure S7}) begins with the acquisition of WSIs and their associated annotations from QuPath, which are represented as Shapely \cite{Gillies_Wel_etal._2024} polygons. Utilizing paquo, we directly read, create, and modify these annotations and detections within a QuPath project in the Python environment. Images are then cropped using these polygons and processed by cell segmentation and classification models employing standard vision processing toolkits such as OpenCV, pyvips, and PyTorch. Additionally, QuPath employs Groovy scripts to initiate a Python process that starts the entire pipeline from QuPath graphical interface: fetching polygons, extracting images from them, and running deep learning model inference on the cropped images. 
The results are returned to QuPath, leveraging paquo's Python bindings to manipulate QuPath data while minimizing the computational overhead typically associated with cross-environment communication.

\counterwithin{figure}{subsection}
\renewcommand{\thefigure}{S\arabic{subsection}}

\begin{figure}[h!]
    \centering
    \includegraphics[width=\textwidth]{images/A7.pdf}
    \caption{QuPath integration workflow using Python environment}
    \label{fig:S7}
\end{figure}

Compared to traditional workflows that involve exporting annotations as GeoJSON, classifying them in Python, and reimporting them into QuPath, our approach offers several advantages. We eliminate the need to switch between programming languages, providing a cohesive and streamlined development process entirely within QuPath software and removing the necessity to use other tools. Meanwhile, we avoid storing annotations as intermediate JSON files unless required for external use or archiving. By conducting the entire inference and post-processing workflow within the Python environment, we leverage the power and flexibility of Python libraries for image processing and machine learning. This approach also enables adjustments to any set of labels and models, thereby improving its applicability.

%\hfill

The distilled model and QuPath integration code are packaged into a Docker container, enabling streamlined execution with the Docker engine. Detailed integration code and deployment instructions can be found in the GitHub repository \cite{Shvetsov_2025b}.

Despite these benefits, we acknowledge that the paquo library is a proof‑of‑concept project in its early development stage and has not been tested across all versions of QuPath.

\clearpage

\subsection{\label{chap:S8}Data and code availability statement}
All datasets, models, and code used in this study are publicly available and can be obtained from the repositories listed below. 
The PanNuke \cite{Gamper_Koohbanani_etal._2019} and MoNuSAC \cite{Verma_Kumar_etal._2021} datasets are publicly accessible, and download information along with detailed descriptions can be found in their respective articles. Preprocessing scripts for PanNuke and MoNuSAC data, as well as individual cell extraction scripts, are available on GitHub \cite{Shvetsov_2025a}. The H-Optimus foundation model used in our experiments can be downloaded from the HuggingFace repository \cite{hoptimus2024}, and model information is available on GitHub \cite{Saillard_Jenatton_etal._2024}. In addition, the integration code for QuPath and the distilled model packaged in a Docker container are provided in the repository \cite{Shvetsov_2025b}, and paquo Python library is available from the authors GitHub repository \cite{Bayer_AG}.
\clearpage

\end{document}
 % must use main.bbl as the name, following the main.tex file
}
\vfill

\clearpage
% \onecolumn
% \appendix
\section*{Appendix}

\subsection{Expanded Discussion of AGCM}

    %%%%%%%%%% Start SVG (AGCM) %%%%%%%%%%%%
    \begin{figure*}[th]
    \centering
    \includegraphics[width=1.99\columnwidth]{fig_13.pdf}
       \caption{
       (a) Feature-based models rely on manual feature preprocessing using external automatic toolkits, such as OpenFace, which operate outside the model's training loop and are not learnable. These models map preprocessed features to task labels, risking the loss of valuable raw data information that could contribute to more comprehensive predictions.
       (b) Multi-task learning models train multiple tasks independently, with the learning of specific emotional tasks and AUs being uncorrelated and disconnected. As a result, AU predictions in multi-task learning cannot effectively explain the emotional predictions, limiting the interpretability of the model.
       (c): The proposed AGCM framework operates as follows: after feature extraction, the Attention-Guided Concept Generator creates learnable neural representations for both activated and inactivated concepts, along with their respective activation scores. It then computes the emotional concept contribution by combining the activated and inactivated embeddings for each concept. Parameter optimization for concept learning is conducted concurrently with task-label learning in an end-to-end manner, enabling the model to capture emotional concept contributions while effectively overcoming the trade-off between explainability and performance. }
    \label{fig_app_1}
    \end{figure*}
    %%%%%%%%%% END of SVG (CEM-based FER Framework) %%%%%%%%%%%%

    The use of handcrafted features, such as AU detections, has been ongoing for decades. These approaches mainly focus on automatically mapping the facial representation to a single numerical value, without fully accounting for the complexity of one’s affective state. Like in most of the feature-based approaches, relying solely on these numerical values for intricate AC tasks risks overlooking other emotion-related information conveyed by the subject, potentially degrading performance. Similarly, in multi-task learning—for instance, simultaneously predicting AU and expression—each classification head optimizes independently, rather than fostering mutually beneficial learning that emphasizes the relevance of AUs to facial expressions. 

    In contrast, as illustrated in Fig. \ref{fig_app_1}, the proposed AGCM framework enhances both model explainability and performance by bridging this gap. It employs an end-to-end learning strategy that quantifies the contributions of underlying emotion-related indicators to the final task prediction. By design, AGCM naturally advances traditional feature-based and multi-task AC approaches, where feature representations are either static or insufficient as explanations for task predictions.

\subsection{Embedding Size Ablation Study}

    %%%%%%%%%% Start SVG (AGCM) %%%%%%%%%%%%
    \begin{figure}[th]
    \centering
    \includegraphics[width=1\columnwidth]{fig_14.pdf}
       \caption{
       Task performance evaluation (\%) with different embedding sizes. For RAF-DB and AffectNet, the overall accuracy is reported. For Aff-Wild2 and NOXI, the F-1 score and CCC score are reported.}
    \label{fig_emb_size}
    \end{figure}
    %%%%%%%%%% END of SVG (CEM-based FER Framework) %%%%%%%%%%%%
    
    Previous studies have demonstrated that embedding size can impact the task performance of concept-based frameworks \cite{zarlenga2022concept}. The optimal concept size may vary depending on the task. In this work, we use an embedding size of 16 for all FER tasks and 32 for engagement estimation tasks. 

    Fig. \ref{fig_emb_size} shows the task performance across various embedding sizes. For both applications, performance initially improves with increasing embedding size. However, once the embedding size reaches the limitation of the model's learning capacity, further increases do not yield performance gains. Instead, larger embeddings may significantly raise the number of parameters, which can pose challenges for model training and deployment.

\subsection{Comparing End-to-end and By-step AGCM}
    
    To further assess the efficiency of the AGCM framework, we compare the end-to-end and by-step training strategies. In by-step AGCM, the model first optimizes a mapping function from the raw input to all intermediate concept scores. If the concepts include only AUs, this phase operates similarly to an AU detector, generating activation probabilities for all AUs. These AU probabilities are then combined with the embeddings in a subsequent optimization step to predict the final facial expression label separately.
    
    In by-step AGCM, the neural embeddings of intermediate concepts are not trainable during task learning. The parameter optimization treats the concept and task loss separately. This approach contrasts with end-to-end training, where a unified push-pull joint loss is employed to enhance both concept explainability and task performance simultaneously.

    \begin{table}[t]
    \centering
    \caption{Performance comparison (\%) of the end-to-end and by-step AGCM framework. For RAF-DB and AffectNet, the overall accuracy is reported. For Aff-Wild2 and NOXI, the F-1 score and CCC score are reported. }
    \begin{tabular}{cccc}
    \toprule
                & Data & End-to-end AGCM & By-step AGCM \\ \midrule
    RAF-DB      & V    & \textbf{94.40}      & 89.71   \\
    AffectNet-7 & V    & \textbf{69.45}      & 64.08   \\
    AffectNet-8 & V    & \textbf{65.62}      & 61.36   \\
    Aff-Wild2   & V    & \textbf{44.95}      & 39.10   \\
    Aff-Wild2   & V/A  & \textbf{47.52}      & 39.23   \\
    NOXI        & V    & \textbf{59.24}      & 52.01   \\
    NOXI        & V/A  & \textbf{80.39}      & 67.88   \\
    \bottomrule
    \end{tabular}
    \label{tab_by_step}
    \end{table}

    Table \ref{tab_by_step} presents a performance comparison between the end-to-end and by-step AGCM training strategies. Compared to the end-to-end approach, the by-step training strategy results in performance degradation across all datasets, with particularly notable declines in the multimodal AGCM framework, where separately learning concepts can lead to significant information loss from the raw data. Thus, we posit that jointly learning the concept and task label enhances both model explainability and task performance by compelling the model to explicitly supervise human-understandable features derived from domain-specific prior knowledge.

\subsection{Expanded Discussion of AGCM and Map-based XAI}

    Map-based XAI was originally designed for general ML tasks like object localization, where attention heatmaps serve as effective tools to indicate object locations \cite{gao2021ts}. In affective signal processing, however, spatial concept explanations offer significant advantages over map-based XAI by providing domain-specific insights alongside task performance improvements. Simply presenting an attention heatmap over a face region offers minimal value for domain experts in AC applications. For instance, two opposing indicators, AU12 (Lip Corner Puller) and AU15 (Lip Corner Depressor), appear in the same region of the face, making it insufficient to rely solely on attention maps for emotion interpretation. Instead, conceptual explanations that explicitly indicate the activation and contribution of specific AUs provide a more natural and informative approach to AC tasks.

    Recent map-based FER work \cite{belharbi2024guided} uses pre-generated AU maps based on emotion labels to guide model learning, depending on a strict mapping between AUs and facial expressions. For example, for images labeled as ``happiness,'' this approach restricts the model’s focus strictly to the AU6 and AU12 regions, regardless of whether these specific AUs are activated, ignoring other facial information that may contribute to the expression. This rigid mapping not only degrades performance but also proves limiting in downstream AC applications, such as engagement estimation or mental health assessment, where there is no clear mapping between AUs and affective labels.

    Fig. \ref{fig_big_example} compares explanations provided by our proposed AGCM with those from two map-based XAI methods \cite{gao2021ts, belharbi2024guided}. The attention heatmaps from the map-based XAI approaches appear similar across different expression labels, offering insufficient interpretability for high-stakes AC applications. In contrast, AGCM not only localizes each AU but also quantifies its contribution to the final prediction, delivering richer insights into model predictions while achieving state-of-the-art task performance.

    \clearpage
    %%%%%%%%%% Start SVG (AGCM) %%%%%%%%%%%%
    \begin{figure*}[t]
    \centering
    \includegraphics[width=1.99\columnwidth]{fig_15.pdf}
       \caption{Explanation examples of map-based TS-CAM \cite{gao2021ts}, attention map-based FER (Att-Map) \cite{belharbi2024guided}, and the proposed AGCM framework. In addition to all concept locations, AGCM explicitly provides the contribution score of each concept, offering domain-specific insight into the model decision-making process. The images are randomly picked from the AffectNet test set. 
       }
    \label{fig_big_example}
    \end{figure*}
    %%%%%%%%%% END of SVG (CEM-based FER Framework) %%%%%%%%%%%%
\end{document}


