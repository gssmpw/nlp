\documentclass[lettersize,journal]{IEEEtran}
\usepackage{amsmath,amsfonts}
\usepackage{algorithmic}
\usepackage{algorithm}
\usepackage{array}
\usepackage[caption=false,font=normalsize,labelfont=sf,textfont=sf]{subfig}
\usepackage{textcomp}
\usepackage{stfloats}
\usepackage{url}
\usepackage{verbatim}
\usepackage{graphicx}
\usepackage{cite}
\usepackage{afterpage}
\hyphenation{op-tical net-works semi-conduc-tor IEEE-Xplore}
% updated with editorial comments 8/9/2021

% MY PACKAGEs
\usepackage{graphicx}
\usepackage{amsmath}
\usepackage{amssymb}
\usepackage{booktabs}
\usepackage{svg}
\usepackage{tabularx}
\usepackage{multirow}
\usepackage{ragged2e}
\usepackage{lipsum}
\usepackage[numbers,sort,compress]{natbib}

% end of MY PACKAGEs



\begin{document}

% \linenumbers
\title{Interpretable Concept-based Deep Learning Framework for Multimodal Human Behavior Modeling}

% \author{Xinyu Li,~\IEEEmembership{Student Member,~IEEE,} Marwa Mahmoud,~\IEEEmembership{Member,~IEEE}
\author{\parbox{16cm}{\centering
   {\large Xinyu Li and Marwa Mahmoud }\\
   {\normalsize
   School of Computing Science, University of Glasgow, United Kingdom\\}}
}
        % <-this % stops a space
% \thanks{This paper was produced by the IEEE Publication Technology Group. They are in Piscataway, NJ.}% <-this % stops a space
% \thanks{Manuscript received April 19, 2021; revised August 16, 2021.}
% }

% The paper headers
% \markboth{Journal of \LaTeX\ Class Files,~Vol.~14, No.~8, August~2021}%
% {Shell \MakeLowercase{\textit{et al.}}: A Sample Article Using IEEEtran.cls for IEEE Journals}

% \IEEEpubid{0000--0000/00\$00.00~\copyright~2021 IEEE}
% Remember, if you use this you must call \IEEEpubidadjcol in the second
% column for its text to clear the IEEEpubid mark.

\maketitle

\begin{abstract}

In the contemporary era of intelligent connectivity, Affective Computing (AC), which enables systems to recognize, interpret, and respond to human behavior states, has become an integrated part of many AI systems. As one of the most critical components of responsible AI and trustworthiness in all human-centered systems, explainability has been a major concern in AC. Particularly, the recently released EU General Data Protection Regulation requires any high-risk AI systems to be sufficiently interpretable, including biometric-based systems and emotion recognition systems widely used in the affective computing field. Existing explainable methods often compromise between interpretability and performance. Most of them focus only on highlighting key network parameters without offering meaningful, domain-specific explanations to the stakeholders. Additionally, they also face challenges in effectively co-learning and explaining insights from multimodal data sources. To address these limitations, we propose a novel and generalizable framework, namely the Attention-Guided Concept Model (AGCM), which provides learnable conceptual explanations by identifying \textit{what} concepts that lead to the predictions and \textit{where} they are observed. AGCM is extendable to any spatial and temporal signals through multimodal concept alignment and co-learning, empowering stakeholders with deeper insights into the model's decision-making process. We validate the efficiency of AGCM on well-established Facial Expression Recognition benchmark datasets while also demonstrating its generalizability on more complex real-world human behavior understanding applications. 
We believe that AGCM’s flexibility and extensibility lay a solid foundation for developing future interpretable and trustworthy models in downstream affective computing applications, including in mental health, psychiatry, education, automotive, and security, offering both competitive performance and domain-specific explanations.


\end{abstract}

\begin{IEEEkeywords}
Explainable AI, multimodal learning, affective computing, facial expression recognition, human-human interaction
\end{IEEEkeywords}

\section{Introduction}
    % \IEEEPARstart{E}{xplainability}
    Affective Computing (AC) aims to develop models and systems that recognize, interpret, and respond to human behavior states. As a human-centered design, explainability and transparency have become critical concerns in AC applications \cite{cortinas2023toward}. The EU AI Act \cite{hupont2022landscape} and the newly proposed General Data Protection Regulation (GDPR) in 2024 \cite{GDPR2024} mandates that high-risk AI systems, including biometric-based systems and emotion recognition systems widely used in the affective computing field, must be sufficiently transparent to allow stakeholders from cross-disciplinary area to comprehend the decision-making process of the framework. Enhancing explainability in AC models not only offers extra insights into AI predictions but also ensures fair, trustworthy, and accountable outcomes in sensitive applications like education, healthcare, and security systems. \cite{yu2024bridging, kumar2023opacity}.

    \begin{figure}[t]
    \centering
    \includegraphics[width=0.99\columnwidth]{fig_1.pdf}
       \caption{Difference between the black-box models, current eXplainable AI (XAI), and our proposed model. (a): Black-box ML models offer no extra insight into the model prediction. (b): Map-based XAI approaches offer explanations by identifying important regions that lead to the prediction, but without any domain-specific knowledge that validates the decision-making process. (c): Our proposed framework explicitly localizes domain-specific indicators, learns their contributions during training, and incorporates multimodal concepts, thereby making predictions based on these intermediate attributes in an inherently interpretable manner.}
    \label{fig_intro}
    \end{figure}

    There is an increasing interest in developing interpretable or eXplainable Artificial Intelligence (XAI) to improve model transparency in AC. As shown in Fig. \ref{fig_intro} (b), approaches such as post-hoc explanations \cite{ribeiro2016should,heimerl2020unraveling,malik2021towards} and map-based methods \cite{gao2021ts,gund2021interpretable,belharbi2024guided} have emerged to address this need. However, these techniques primarily focus on identifying important regions or parameters within deep neural networks, rather than providing an explicit, causal explanation for the predictions. This limitation is especially pronounced in AC, where opposing facial Action Units, like AU12 (Lip Corner Puller) associated with positive emotion and AU15 (Lip Corner Depressor) linked to negative emotion, can occur in the same facial region. Meanwhile, the alignment and co-learning from multimodal sources pose even greater challenges for these approaches due to the inherently different properties of multimodal knowledge. Therefore, they often face a trade-off between performance and interpretability, which, in high-risk XAI, may undermine the system's trustworthiness \cite{rudin2019stop}.

    Consider a common question: How would a human expert explain their prediction of an individual with highly conversational engagement? They would likely point to the activation of specific facial muscles, such as the zygomatic major, indicating engaged smiles, a strong positive indicator of engagement. Meanwhile, the forward gaze direction, proper gesture, body language, and audio indicators can also be used to recognize engagement. Thus, a good explanation from an AC model should address two key aspects: \textit{what} indicators or concepts (e.g., facial muscle activations) contribute to the prediction, and \textit{where} these concepts are observed. Furthermore, the importance of multimodal learning is self-evident in real-world AC applications \cite{abaeikoupaei2020multi, yoon2022d, cafaro2017noxi}. Training and interpreting AC models with multimodal alignment and co-learning is another key challenge in affective XAI \cite{baltruvsaitis2018multimodal}.

    As shown in Fig. \ref{fig_intro} (c), in this paper, we propose an interpretable concept-based framework: the Attention-Guided Concept Model (AGCM), which localizes and learns the key indicators during training and then makes the final prediction according to the contribution of these intermediate concepts. This framework incorporates spatial concept information and multimodal concept fusion within a powerful attention-based architecture, combining the advantage of both domain-specific explanation and state-of-the-art performance. In summary, the main contributions of this paper are as follows:

    \begin{enumerate}
    \item We propose a concept-based interpretable framework for AC applications, namely the Attention-Guided Concept Model (AGCM), which provides both learnable multimodal conceptual explanations and spatial visual concept localization, quantifying the contribution of individual concepts to the predicted affective label.
    \item To address the challenge of multimodal concept alignment and co-learning, AGCM introduces an extendable sequential multimodal concept fusion, which can be easily expanded to any spatial-temporal signal. This approach accounts for temporal and contextual information between input modalities, demonstrating the adaptability to other discrete or continuous signals.
    \item We qualitatively and quantitatively evaluate the proposed framework on three large-scale FER datasets: RAF-DB, AffectNet, and Aff-Wild2, demonstrating that AGCM outperforms previous interpretable models and achieves competitive performance compared to state-of-the-art black-box models. Moreover, the experiment shows that AGCM offers a human-interpretable explanation grounded in domain-specific knowledge.
    \item To demonstrate the generalizability of AGCM on complex real-world AC applications, we conduct extensive experiments on the human-human interaction dataset, validating its ability to provide explainable and accurate prediction in downstream AC applications. We provide a video demonstration in the supplementary material to offer additional insights into the prediction process and its explainability.
    \end{enumerate}

\section{Related Work}

In this section, we examine two primary machine learning approaches commonly used in affective computing: feature-based models and end-to-end models. We then discuss recent advancements in explainable affective computing, emphasizing their contribution and limitation to model transparency and interpretability.

\subsection{Feature and End-to-end Models in Affective Computing}
    Discriminative AC focuses on mapping human-centered data to emotion-related labels, employing two primary approaches: feature-based models and end-to-end models.
    
    Feature-based models \cite{tsalera2022feature, avola2022affective} rely on manually extracted features derived from raw data, which are then used to train machine learning models to establish the relationship between features and labels. The strength of this approach lies in the interpretability of the features, which are often human-understandable and can provide valuable behavioral insights \cite{bento2022comparing}. Additionally, feature-based models typically operate on structured, tabular data, offering a computationally efficient solution \cite{bisogni2023emotion}. However, the reliance on handcrafted features may omit potentially important information embedded in the raw data, causing inevitable information loss \cite{zhao2019affective, cortinas2023toward}. Furthermore, decoupling feature extraction from model training may introduce limitations, such as overfitting, particularly due to the structured nature of the input data \cite{bengio2013representation}.
    
    End-to-end models \cite{li2020deep}, on the other hand, learn directly from raw data, eliminating the need for manual feature engineering. Fully leveraging the representational power of deep neural networks, these models are particularly effective when trained on large datasets. However, their strength is also their weakness: the opacity of their learned representations often leads to what is referred to as the ``black-box'' problem, making these models difficult to interpret as they lack human-understandable intermediate representations \cite{zhao2019affective}. 
    This challenge persists in multi-task learning, where models are designed to predict multiple task labels simultaneously, such as emotion and AUs. Despite their multi-task design, emotion and AU predictions are learned independently, leaving the model as a black box, where the predicted AUs cannot explain the predicted emotions.

    As shown in Fig. \ref{fig_ven}, in this work, we propose a hybrid approach, integrating the strengths of the well-understandable feature-based model and the state-of-the-art black-box models through concept-based learning, where each concept serves as an embedded neural representation of the feature. This approach retains the interpretability inherent in feature-based models while harnessing the robust learning capabilities of end-to-end neural networks. 

    \begin{figure}[t]
    \centering
    \includegraphics[width=0.7\columnwidth]{fig_2.pdf}
       \caption{Feature-based approaches offer inherent interpretability and are easily understood by humans, while end-to-end models deliver state-of-the-art learning capabilities. This work seeks to integrate the strengths of both methods through a concept-based framework, which achieves a balance between high explainability and robust performance. Unlike traditional features, concepts are not static values. They serve as the neural embeddings of features that are trainable within the ML framework, spontaneously quantifying the contribution of individual concepts to the task label. }
    \label{fig_ven}
    \end{figure}
    

\subsection{XAI in Affective Computing} \label{xai_in_ac}
    Recent efforts to enhance the explainability of affective computing models have largely relied on post-hoc, map-based visualizations, and concept-based learning.

    Post-hoc approaches \cite{ribeiro2016should,heimerl2020unraveling,malik2021towards, Wu2019Enhancing} retrospectively analyze the parameter importance of pre-trained black-box models after deployment. These methods attempt to explain the model by manipulating parameters in specific parts of the network to check their impact on the final prediction. Map-based approaches \cite{gund2021interpretable,belharbi2024guided} are another common method used to interpret black-box models, typically highlighting the regions where the model focuses its attention. However, both of these approaches primarily focus on the importance scores within the neural network, without offering additional, domain-relevant information for experts. This limitation is particularly evident in AC, where conflicting indicators, such as AU12 (Lip Corner Puller) signaling positive emotion and AU15 (Lip Corner Depressor) indicating negative emotion, may appear in the same facial region. Therefore, simply presenting the weight importance or model attention provides little insight for domain experts like psychologists to understand the AI decision-making process. Furthermore, the distinct properties of multimodal data make incorporating multimodal alignment and co-learning in post-hoc or map-based XAI methods even more challenging, taking the risk of losing either accuracy or interpretability.

    Recent attempts on concept-based models \cite{xinyuFG24, zarlenga2022concept} try to encapsulate specific, human-understandable features through concept embeddings $C$ that are learned in a fully supervised manner. These models learn the mapping $X \rightarrow C \rightarrow Z$, where $x \in X$ represents the raw image pixels and $z \in Z$ represents the task labels. Specifically, a concept generator $G$ generates concept embeddings, denoted as $\hat{c} = G(x)$, with $\hat{c} \in C$ representing the learned concepts within a bottleneck layer $C$. Subsequently, a facial expression predictor $y$ maps the concept embeddings to task labels $\hat{z} \in Z$, where $\hat{z} = y(\hat{c})$. While concept-based models offer a more interpretable framework than map-based approaches, ongoing research is focused on integrating this explainable architecture with multimodal learning and performance-optimized strategies \cite{xinyuFG24}. Moreover, a key challenge lies in integrating spatial explanations, which reveal \textit{where} the model is focusing, with concept-based explanations, which clarify \textit{what} contributes to the prediction. Achieving this synergy is essential for enhancing both the interpretability and practical utility of models in high-stakes applications.

    Table \ref{tab_previous_work} compares the proposed concept-based framework with previous feature-based, map-based, and black-box FER models in terms of explainability and performance. The proposed framework provides learnable domain-specific insights into the decision-making process for stakeholders while retaining map-based explanations that illustrate the model's areas of attention. A two-stage learning architecture with multimodal concept fusion is introduced, effectively addressing the alignment and co-learning challenges in multimodal interpretable AC. Furthermore, it achieves state-of-the-art performance through deep end-to-end training, successfully balancing the trade-off between interpretability and performance in high-stakes AC applications.

    \begin{table}[t]
    \centering
    \caption{Comparison of our work with previous works on FER in terms of explainability and performance, including feature-based approach, map-based approach, and deep end-to-end approach.}
    \begin{tabular}{ccccc}
    \toprule
                          & Ours & Feature & Map & Black-box \\ \midrule
    Feature-based Insight & +    & +       &     &     \\
    Map-based Explanation & +    &         & +   &     \\
    End-to-end Training   & +    &         & +   & +   \\
    Learnable Explanation & +    &         &     &     \\
    Multimodal Learning   & +    & +       &     & +   \\ 
    \bottomrule
    \end{tabular}
    \label{tab_previous_work}
    \end{table}

\section{Methods}
    
    This section provides a detailed overview of the proposed Attention-Guided Concept Model (AGCM). We begin by detailing the selection and generation of multimodal concepts, a critical step before deploying any concept-based explainable model. Next, we focus on the visual modality, as it is the most widely used and complex modality, uniquely supporting explanations of what concepts contribute to predictions and where they are observed. Finally, we describe the multimodal architecture, addressing the challenges of multimodal alignment and co-learning. Using the audio-visual modality as an example, we demonstrate the framework's functionality and highlight its extendability to other signal-based modalities.
    
\subsection{Multimodal Concept Selection \& Generation} \label{sec_spa_concept}
    The selection of concepts or features plays a pivotal role in producing accurate and explainable results, whether in interpretable concept-based models or traditional feature-based models. In terms of explainability, the concept function - similar to features - acts as a key representation of the underlying data. Moreover, concepts must explicitly capture attributes that are highly relevant and meaningful to the task at hand. For example, in object detection, attributes such as color and shape are critical, while in bird classification, features like wing morphology or bill structure provide significant insights.
    
    Explaining spatial signals, such as those in the visual modality, involves two key aspects: spatial explanations and conceptual insights, which are particularly critical in explainable medical analysis \cite{barnett2021case} and affective XAI \cite{belharbi2024guided}. To address this, AGCM integrates spatial concepts, enabling the model to learn not only \textit{what} to focus on but also \textit{where} to focus.

    For the conceptual explanations (the \textit{what} question), key features such as facial muscle movements, gaze direction, and head pose are important for assessing and interpreting an individual's affective state \cite{adolphs2002recognizing, Bayliss2007Affective, xinyuFG24}. To address spatial explanations (the \textit{where} question), patch-level attention maps are trained alongside each concept in an end-to-end, fully supervised manner. This method allows the model to associate concept contributions with their exact spatial locations, thereby enhancing both interpretability and overall performance.

    Since manually annotated attention maps are not always available for large-scale datasets, the spatial maps are localized based on facial landmarks \cite{belharbi2024guided, ma2021landmark}. In this paper, we utilize an open-source landmark detector \cite{wang2020deep} for automatic landmark detection. According to the landmark locations, Regions of Interest (ROI) maps are generated for all AUs, which are subsequently used to supervise the spatial concept attention map throughout model training.

    To integrate these ROI maps into our transformer-based concept learning framework, they are transferred into patch-level representations, $\text{PatchMaps}[i]$, by performing average interpolation, as described in (\ref{eq_interpolation}). Here, $\text{AUMaps}[i](x_1, y_1)$ denotes the value of the $i$-th input map at position $(x_1, y_1)$. The terms $x'$ and $y'$ correspond to the patch indices in the $x$ and $y$ dimensions, while $S_x$ and $S_y$ denote the respective scaling factors.
%
    \begin{equation}\label{eq_interpolation}
        \text{PatchMaps}[i] = \frac{1}{S_x \cdot S_y} \sum_{x_1 = x' S_x}^{(x' + 1) S_x} \sum_{y_1 = y' S_y}^{(y' + 1) S_y} \text{AUMaps}[i](x_1, y_1)
    \end{equation}

    Fig. \ref{fig_interpolation} presents an example of a patch-level AU map generated using landmark detection and average interpolation. In this map, patches with lighter colors indicate regions of higher importance, effectively highlighting the ROI for each AU. These maps are utilized as part of the ground truth to guide the model’s concept learning process via a concept map loss, ensuring the model's focus aligns with the actual spatial regions of interest during training.

    Other than spatial signals, temporal signals such as audio, Electrocardiogram (ECG), and Electroencephalogram (EEG) are often perceived as less complex in terms of dimensionality since they typically vary along a single axis (time). For these signals, stakeholders often prioritize conceptual insights (the \textit{what} question) over spatial interpretation. Temporal dependencies (the \textit{where} question in time) are naturally addressed by mechanisms like attention models or recurrence in sequential architectures, which excel at capturing temporal relationships.
    
    Using the widely used audio modality as an example, acoustic indicators such as pitch, loudness, and speech rate and their variations provide critical information by capturing subtle vocal variations that reflect emotional or cognitive states directly tied to the affective labels \cite{bachorowski1995vocal, Polzehl2011Anger, Yu2004Detecting, Adami2007Modeling, Grau1988The}. Providing conceptual insights into the decision-making process is essential for explaining predictions derived from these temporal signals.

    \begin{figure}[t]
    \centering
    \includegraphics[width=0.99\columnwidth]{fig_3.pdf}
       \caption{Example of patch-level AU map generated using landmark detection and average interpolation.}
    \label{fig_interpolation}
    \end{figure}

\subsection{Visual Attention-Guided Concept Learning}

     %%%%%%%%%% Start SVG (AGCM) %%%%%%%%%%%%
    \begin{figure*}[th]
    \centering
    \includegraphics[width=1.8\columnwidth]{fig_4.pdf}
       \caption{The architecture of our proposed Attention-Guided Concept Model (AGCM) for the spatial visual modality. The model uses a transformer backbone $\varphi(\cdot)$ to convert the facial image $x$ into a patch-level representation. The Attention-Guided Concept Generator (ACG) applies spatial-channel attention with a Multi-scale Spatial Attention (MSA) block and Channel Attended Concept Mapping (CACM), which together capture attention across both spatial and feature dimensions.
       The MSA block focuses on spatial features at multiple scales, enhancing the model's ability to capture both fine and coarse details. For instance, the concept of the cheek region may benefit from a larger attention area compared to the eye region. Three MSA heads are used to capture diverse spatial patterns within an image, each generating a concept attention map $\hat{a}_{i}$. These maps are weighted and summed to produce the final concept attention map, which is used to update the concept map loss during training.
       CACM further improves the model's focus on the most informative features along the channel dimension, ensuring robust feature selection across multiple channels.
       A concept probability generator $p(\cdot)$ computes the probability of each activated concept, facilitating concept supervision by showing the contribution of individual concepts to the predicted label. Notably, ACG considers both activated and inactivated concept embeddings, as the absence of certain concepts (e.g., AUs) can provide additional information about a subject's facial expression. The predicted activated concepts, $\hat{c}_{i}^{+}$, and inactivated concepts, $\hat{c}_{i}^{-}$, are weighted by their respective probabilities from $p(\cdot)$, then concatenated and passed to the one-layer fully-connected task predictor $y(\cdot)$ to generate the final task label $\hat{{\textit{t}}}$. 
       During loss computation, the model optimizes its performance using the task loss, concept probability loss, and concept map loss associated with the spatial concept attention, ensuring a strong explainability of the model's decision-making process giving not only \textit{what} key concepts contribute the most to the prediction but also \textit{where} these concepts appear. }
    \label{agcm_framwork}
    \end{figure*}
    %%%%%%%%%% END of SVG (CEM-based FER Framework) %%%%%%%%%%%%
  
    Given the complexity and the inherent differences between the spatial visual signal and other temporal signals, AGCM first focuses only on training the visual concept through attention-guided concept learning. This architecture leverages spatial concept supervision and concept attention to interpret the model's decision-making process by determining not only \textit{what} key concepts contribute the most to the prediction but also \textit{where} these concepts appear.

    As illustrated in Fig.~\ref{agcm_framwork}, the proposed Attention-Guided Concept Model (AGCM) is designed to enhance both the accuracy and explainability of the concept-based models. The model begins by processing the input facial image $x$ through a transformer backbone $\varphi(\cdot)$, which converts the image into a patch-level representation. This representation effectively captures local and global features by dividing the image into patches and is essential for subsequent processing.

    The core component of AGCM is the Attention-Guided Concept Generator (ACG), which integrates two attention mechanisms: Multi-scale Spatial Attention (MSA) and Channel Attended Concept Mapping (CACM). The MSA block focuses on spatial features at multiple scales, enabling the model to capture both fine-grained and coarse details within the image. For example, recognizing the concept of the cheek region may require a broader attention area compared to the eye region. To achieve this, three MSA heads are employed to capture diverse spatial patterns, each generating a concept attention map $\hat{a}_{i}$. These maps are then weighted and summed to produce a final concept attention map, which is utilized to update the concept map loss during training. 
    
    Complementing the spatial attention, CACM enhances the model's focus along the channel dimension. By applying attention to the most informative feature channels, CACM ensures robust feature selection across multiple channels, which is crucial for accurately interpreting complex facial expressions.

    The proposed framework also includes a concept probability generator $p(\cdot)$ that computes the probability of each activated concept. This mechanism facilitates concept supervision by quantifying the contribution of individual concepts to the predicted label. Importantly, ACG considers both activated and inactivated concept embeddings because the absence of certain concepts (e.g., deactivation of AUs) can also provide valuable information about one's facial expressions. The $i$-th predicted activated concepts, $\hat{c}_{i}^{+}$, and inactivated concepts, $\hat{c}_{i}^{-}$, are weighted by their respective probabilities from $p(\cdot)$. The probability score $p$ indicates the likelihood that the activated concept contributes to the final prediction. These are then concatenated and passed to the task predictor $y(\cdot)$, which is a one-layer fully connected network, to generate the final task label $\hat{\textbf{\textit{t}}}$. Therefore, it is designed to be adaptable and expandable to any discrete or continuous concepts, given that appropriate concept annotations are available.

    During loss computation, the model optimizes performance through a combination of losses: task loss, $\mathcal{L}_{t}$, concept probability loss, $\mathcal{L}_{c}$, and concept map loss, $\mathcal{L}_{m}$, associated with spatial concept attention. The task loss, $\mathcal{L}_{t}$, is computed using Cross Entropy (CE), while the concept probability loss, $\mathcal{L}_{c}$, is derived from the sum of Binary Cross Entropy (BCE) across all concepts. Instead of relying on Mean Square Error (MSE), the concept map loss, $\mathcal{L}_{m}$, uses Cosine Similarity (sim) to emphasize spatial pattern alignment rather than strict value matching. Therefore, the total loss, $\mathcal{L}$, is formulated as:
    %
   \begin{equation}\label{eq_loss}
    \mathcal{L} = \text{CE}(\hat{t}, t) + \sum_{i=1}^{n} \text{BCE}(p(\hat{c}_i^+), c_i) + \sum_{i=1}^{n} (1 - \text{sim}(\hat{a}_i, a_i)).
    \end{equation}
    %
    Here, $t$ is the ground truth task label, $c_i$ is the label of the $i$-th concept, and $a_i$ is the $i$-th concept attention map, while $n$ denotes the total number of used concepts.
    
    This comprehensive optimization strategy ensures that the AGCM framework achieves high accuracy while maintaining explainability in its predictions.


\subsection{Expandable Multimodal AGCM Concept Fusion}
    
     %%%%%%%%%% Start SVG (AGCM) %%%%%%%%%%%%
    \begin{figure*}[th]
    \centering
    \includegraphics[width=1.3\columnwidth]{fig_5.pdf}
       \caption{
       In the multimodal fusion stage, the pre-learned visual branch functions as a Visual Attention-Guided Concept Generator. The parameters of the Visual Attention-Guided Concept Generator are frozen to ensure reliable visual concept predictions. On the audio side, an Acoustic Concept Generator (ACG) processes the audio input, generating activated ($\hat{c}_{i}^{+}$) and inactivated ($\hat{c}_{i}^{-}$) acoustic concept embeddings via an acoustic feature extractor $G(\cdot)$. The probability of each concept's activation is computed using an acoustic concept probability generator $p(\cdot)$. The acoustic concept embeddings are concatenated with their corresponding visual concept set and passed through a sequential bottleneck layer {$\hat{c}_{0}$, ...$\hat{c}_{k}$}, where $k$ represents the number of samples in the sequence. For a given video clip, it is assumed that acoustic concepts are shared across all frames. A sequence-to-sequence label predictor $y(\cdot)$ is then used to capture the contextual relationships between frames to generate the final by-frame task label. Importantly, the AGCM framework is inherently extendable to other temporal modalities by adding additional branches to accommodate new data inputs, as long as the appropriate data and annotations are available.
       }
    \label{fusion_framwork}
    \end{figure*}
    %%%%%%%%%% END of SVG (CEM-based FER Framework) %%%%%%%%%%%%

    Alignment, fusion, and co-learning are three primary challenges in multimodal learning, involving the ability to identify, combine, and transfer knowledge across different modalities \cite{baltruvsaitis2018multimodal}, particularly in the context of interpretable AC \cite{cortinas2023toward}. After training the visual concept branch in the first stage, AGCM integrates visual information with any other temporal modalities through concept fusion. In this work, we demonstrate AGCM is an expandable multimodal architecture, using the most commonly used audio-visual fusion as an example, which involves identifying audio information using an acoustic concept generator and joining and transferring knowledge via a late fusion concept-label classifier. 

    As shown in Fig.~\ref{fusion_framwork}, the fusion stage builds upon the visual-based branch from the previous stage. During the fusion stage, the task predictor from the visual branch is removed, transforming it into a Visual Attention-Guided Concept Generator. This visual generator is responsible for extracting and predicting key visual concepts, including AUs, gaze direction, and head poses. To ensure stability and reliability in visual concept prediction, the parameters of the visual branch are frozen, preventing further modifications during the audio-visual training phase. This approach allows the model to harness pre-learned visual knowledge without overfitting, facilitating robust integrated learning across diverse input modalities.

    In parallel with the visual concept branch, the fusion stage introduces an audio brunch with an Acoustic Concept Generator (ACG) to process the audio input. This generator identifies relevant audio information using an acoustic feature extractor, denoted as $G(\cdot)$. These features are then mapped into activated ($\hat{c}_{i}^{+}$) and inactivated ($\hat{c}_{i}^{-}$) acoustic concept embeddings. The probability of activation for each concept is computed through an acoustic concept probability generator $p(\cdot)$, which quantifies the likelihood of each acoustic concept being present in the input. 

    For downstream applications, the audio branch can be replaced or expanded to incorporate other temporal modalities, such as Electrocardiogram (ECG), Electroencephalogram (EEG), or Electrodermal Activity (EDA), provided the appropriate data and annotations are available.

    Once the visual and temporal concepts are extracted, they are aligned and concatenated to form a unified multimodal representation. In this architecture, a key assumption is made: for a given video clip, temporal concepts are shared across all frames. This allows the model to maintain temporal coherence in the audio stream while aligning it with frame-specific visual features. The bottleneck layer serves to compress the multimodal information, ensuring that only the most relevant aspects of the fused representation are retained for further processing. The concatenated concepts are then passed through a sequential bottleneck layer, denoted as { $\hat{c}_{0}$, ..., $\hat{c}_{k}$ }, where $k$ represents the number of samples in the sequence. 

    To capture the temporal and contextual relationships between frames, the fusion branch employs a sequence-to-sequence concept-label predictor $y(\cdot)$, using a transformer architecture. This predictor is designed to handle sequential data, leveraging the temporal dependencies between consecutive frames in a video. By utilizing sequential learning, the model effectively integrates and co-learns multimodal information across time, improving the accuracy of by-frame predictions. This is particularly important for tasks where affective signals evolve over time, such as conversational engagement estimation or mental health assessment.

    The final task label is generated on a per-frame basis, with the model predicting the affective state for each frame in the video sequence. The combination of multimodal concept embeddings allows the VA-AGCM to provide robust and accurate predictions, as it captures a wider range of cues that contribute to affective behavior. Notably, the AGCM framework is readily extendable to other temporal modalities by incorporating additional branches for new data inputs.


\section{Experimental Evaluation and Results}

    Given the intricate nature and wide-ranging applications of AC tasks, we initially employed Facial Expression Recognition (FER) in both visual and audio-visual settings to validate the efficacy of our proposed AGCM framework, considering its well-established datasets and baseline models. We quantitatively evaluate the task and concept-level performance of AGCM on three large-scale FER datasets, and provide qualitative visualizations of the visual and multimodal conceptual explanations, demonstrating the framework’s robustness through occlusion experiments and an ablation study.

\subsection{Datasets}

    We employ three popular benchmark datasets, including RAF-DB and AffectNet with visual modality and Aff-Wild2 with audio-visual data. 

    \textbf{RAF-DB} \cite{li2017reliable} is a widely-used static FER dataset sourced from the internet, containing 6 basic emotion labels (Surprise, Disgust, Fear, Happiness, Sadness, Anger), and a Neutral label. The dataset includes 12,271 images in the training set and 3,068 images for testing.

    \textbf{AffectNet} \cite{mollahosseini2017affectnet} is one of the largest FER datasets, comprising 420,000 facial images annotated with categorical emotion labels. We utilize AffectNet-8, which consists of 291,651 manually labeled images with 8-class emotion labels (Neutral, Happy, Angry, Sad, Fear, Surprise, Disgust, and Contempt). In addition, we employ AffectNet-7, which contains 287,401 images annotated with seven emotion labels (excluding Contempt). The test set contains approximately 3,500 images.

    \textbf{Aff-Wild2} \cite{kollias2019expression} is a large-scale in-the-wild dataset specifically designed for FER and AU detection. It includes over 2.7 million frames from 564 videos with 554 subjects. We use the \textbf{by-frame} FER subset which is manually labeled with 8-class discrete emotions (Neutral, Anger, Disgust, Fear, Happiness, Sadness, Surprise, Other). It also provides manual annotation of 12 AUs. 

\subsection{Concept Generation Setup}
    AGCM is designed to be flexible and extendable to all kinds of discrete or continuous concepts, provided suitable concept annotations are available. In this work, we use the most commonly used audio-visual pair as an example. For the audio concepts, pitch, pitch variation, pitch stability (Jitter), loudness, loudness variation, and speech rate are used. For the visual modality, AUs, gaze direction, and head pose are used. 
    
    Unlike AUs, which are binary in nature (activated or inactivated), gaze, head pose and acoustic concepts are continuous and must be mapped into a probability space to fit the concept-based framework. Specifically, gaze concepts are defined as the degree of direct forward gaze in both horizontal and vertical planes, where $1$ represents directly looking forward and $0$ indicates looking elsewhere. Head pose concepts capture deviations in yaw (head shake) and pitch (head nod). These gaze and head pose concepts are scaled to the range $[0, 1]$ to fit within the concept probability generator, and corresponding heatmaps are generated based on facial landmarks, similar to Section \ref{sec_spa_concept}.

    All acoustic concept labels are normalized to the range $[0, 1]$ before AGCM training. To ensure alignment with the visual concepts, the video data is split into one-second clips (FPS=30), with a 33ms stride applied to capture temporal information effectively. 
    For clips containing complete silence, both pitch and loudness are set to $0$, indicating no contribution from the audio modality. Variations in loudness and pitch are calculated using their first-order derivatives, representing the rate of change for these acoustic features, while the Jitter is inherently a percentage. For videos featuring multiple speakers, the audio track for each subject will be individually separated to minimize noise and ensure clarity. 
    
    Furthermore, the AGCM framework is flexible and can incorporate other temporal modalities with continuous or discrete values, provided the appropriate data and annotations are available.

\subsection{Implementation Details}
     Our experimental setup is summarized as follows: AGCM utilized a pre-trained Vision Transformer as the backbone feature extractor \cite{dosovitskiy2020image}. Similar to \cite{xinyuFG24}, the backbone was pre-trained on VGGFace2 \cite{cao2018vggface2} for the facial recognition task. After pre-training, the classification header was removed and replaced with the AGCM workflow. Facial images were cropped from the video dataset using the InsightFace detector \cite{an2022killing}. To prevent overfitting, the preprocessing stage incorporated random data augmentation techniques, including horizontal flipping, random rotation, and random erasing.

     For datasets lacking AU annotations, we utilized OpenFace 2.0 \cite{baltrusaitis2018openface} to automatically extract 18 Action Units (AUs), which served as intermediary concepts in our proposed framework. All models were trained for 100 epochs, with early stopping to avoid overfitting, and optimized using the Adam optimizer (learning rate set to 0.0001). The AGCM generated concepts using a Dropout rate of 0.01 and Leaky-ReLU activation. The concept probability and map loss weights were set to $1$, ensuring a balanced focus on both conceptual explanation and task prediction. 

     AGCM used HuBERT \cite{hsu2021hubert} feature extractor for the audio input. During concept fusion, the learning of the vision branch was frozen, and the Acoustic Concept Generator (ACG) was fine-tuned for 100 epochs, with early stopping (learning rate set to 0.0001). All experiments were conducted on a workstation equipped with dual 48GB Nvidia RTX 6000 Ada GPUs, running a Linux-based PyTorch environment. For quantitative performance evaluation, we report the average performance over four random seeds. 

\subsection{Evaluating Visual-based AGCM}

    \begin{table*}[th]
    \centering
    \caption{Performance comparison of various models in terms of overall accuracy (\%) on RAF-DB, AffectNet-7, and AffectNet-8. The proposed AGCM framework consistently outperforms feature-based, map-based, and concept-based interpretable models. Notably, AGCM also surpasses state-of-the-art black-box models, offering superior performance without sacrificing conceptual interpretability.}
    \begin{tabular}{llllccc}
    \toprule
    Type                               & Model       & Year & Architecture                & RAF-DB         & AffectNet-7    & AffectNet-8    \\ \midrule
    \multirow{6}{*}{Black-box ML}      & AFR  \cite{savchenko2023adr}         & 2023 & EfficientNet                & 90.05          & 66.51          & 63.13          \\
                                       & CL-TransFER \cite{yang2024cl} & 2024 & Transformer                 & 91.33          & \textbf{67.86}          & \textbf{64.69}          \\ 
                                       & HAM \cite{tao2024hierarchical}        & 2024 & Attention                   & 91.92          & 66.97          & 63.82          \\
                                       & Poster++ \cite{mao2024posterpp}   & 2024 & Transformer                 & 92.21          & 67.49          & 63.77          \\
                                       & CEPrompt \cite{zhou2024ceprompt}   & 2024 & Transformer                 & 92.43          & 67.29          & 62.74          \\
                                       & S2D \cite{chen2024static}        & 2024 & Transformer                 & \textbf{92.57}          & 67.62          & 63.76          \\ \midrule
    Feature-based ML                   & FC          & 2024 & 3-layer FC                  & 67.04          & 40.23          & 37.11          \\ \midrule
    \multirow{3}{*}{Map-based XAI}     & TS-CAM \cite{gao2021ts}           & 2021 & Transformer + CAM & 86.70          & 62.28          & 58.99          \\
                                       & \multirow{2}{*}{Att-Map \cite{belharbi2024guided}}  & 2024 & CNN + Map Attention         & 88.88          & \textbf{62.45}          & \textbf{61.30}          \\
                                       &             & 2024 & Transformer + Map Attention & \textbf{91.03}          & 62.28          & 61.19          \\ \midrule
    \multirow{2}{*}{Concept-based XAI} & CEM \cite{xinyuFG24}        & 2024 & Concept Embedding           & 91.05          & 67.60          & 63.70          \\
                                       & AGCM        & 2024 & Spatial Attention Concept   & \textbf{94.40} & \textbf{69.45} & \textbf{65.62} \\ \bottomrule
    \label{tab_acc_rafdb}
    \end{tabular}
    \end{table*}

    Given the complexity and necessity of determining not only \textit{what} key concepts contribute the most to the prediction but also \textit{where} these concepts appear, we begin by evaluating the visual branch on RAF-DB and AffectNet. To assess the efficiency of the proposed AGCM framework against the previous feature-based and explainable models, we compared this work with a feature-based model, end-to-end map-based explainable models (with CNN and ViT backbones), previous concept-based explainable models, and the state-of-the-art black-box model without explicit model explainability. 
    
    The feature-based model uses only handcrafted features (e.g., AUs) as input, and a 3-layer Fully Connected (FC) neural network with ReLU activation, matching the complexity of AGCM's task predictor.

    Table \ref{tab_acc_rafdb} presents the overall accuracy of various models on RAF-DB, AffectNet-7, and AffectNet-8. The proposed AGCM framework achieves the highest accuracy across all datasets, with 94.40\% on RAF-DB, 69.45\% on AffectNet-7, and 65.62\% on AffectNet-8. These results demonstrate AGCM's significant improvement over the classic feature-based methods, particularly on RAF-DB (+27.36\%) and AffectNet-8 (+28.51\%). AGCM also outperforms state-of-the-art black-box transformer models including S2D \cite{chen2024static} and Poster++ \cite{mao2024posterpp}, providing gains of 1.83\% on RAF-DB and 1.86\% on AffectNet-8 compared to S2D. This highlights AGCM's ability to match and exceed black-box model performance while maintaining conceptual explainability. Furthermore, AGCM demonstrates superior results compared to interpretable map-based approaches, with a 3.37\% improvement on RAF-DB and over 4\% on AffectNet. When compared to the previous concept-based model \cite{xinyuFG24}, AGCM shows consistent gains across all datasets, benefiting from its spatial concept and attention learning.

    \begin{table}[t]
    \centering
    \caption{Class-wise performance comparison (\%) of the proposed AGCM and the transformer-based Poster++ \cite{mao2024posterpp} on RAF-DB and AffectNet-8. AGCM gives a more balanced performance along all classes, resulting in higher average accuracy. }
    \begin{tabular}{l|cc|cc}
    \hline
             & \multicolumn{2}{c|}{RAF-DB}     & \multicolumn{2}{c}{AffectNet-8} \\ \cline{2-5} 
             & AGCM           & POST++         & AGCM           & POST++         \\ \hline
    Anger    & \textbf{94.53} & 88.27          & \textbf{66.05} & 60.20          \\
    Disgust  & \textbf{82.43} & 71.88          & \textbf{61.58} & 58.00          \\
    Fear     & \textbf{87.50} & 68.92          & 63.00          & 63.00          \\
    Happy    & \textbf{97.47} & 97.22          & \textbf{79.42} & 76.40          \\ 
    Sad      & \textbf{93.51} & 92.89          & 65.01          & \textbf{66.80} \\
    Surprise & 89.51          & \textbf{90.58} & 62.99          & \textbf{65.60} \\
    Contempt & -              & -              & \textbf{64.08} & 59.52          \\
    Neutral  & \textbf{93.68} & 92.06          & \textbf{62.76} & 60.60          \\ \hline
    Avg.     & \textbf{91.23} & 85.97          & \textbf{65.61} & 63.77          \\ \hline
    \end{tabular}
    \label{tab_classwise_performance}
    \end{table}

    Table \ref{tab_classwise_performance} presents the class-wise performance comparison between the proposed AGCM framework and the black-box transformer-based Poster++ model \cite{mao2024posterpp} on RAF-DB and AffectNet-8. The results clearly demonstrate the effectiveness of AGCM in delivering a more balanced performance across all FER classes than Poster++, resulting in higher average accuracy on both datasets. The result shows the efficiency of considering conceptual prior knowledge, such as AUs and ROI maps, into the training process to quantify the individual concept's contribution towards predicting the label.

    On RAF-DB, AGCM consistently outperforms Poster++ across nearly all emotion classes, particularly in challenging categories such as Anger and Disgust, where AGCM achieves significant improvements of +6.26\% and +10.55\%, respectively. AGCM also demonstrates superior performance in the Fear class (+18.58\%), while maintaining competitive accuracy in easier classes like Happy and Neutral.
    
    Similarly, on AffectNet-8, AGCM provides improved accuracy in most categories, including notable gains in Anger (+5.85\%), Disgust (+3.58\%), and Happy (+3.02\%). Although Poster++ marginally outperforms AGCM in the Sad and Surprise categories, AGCM still delivers a more balanced overall performance, as evidenced by the higher average accuracy (+1.84\%).
    
    The consistent class-wise improvements offered by AGCM highlight its ability to maintain strong performance across both datasets, even in the presence of class imbalance and data variability. More importantly, AGCM not only surpasses Poster++ in terms of average accuracy but also achieves these gains while preserving the model's interpretability, which is essential for applications requiring both performance and transparency.

\subsection{Evaluating Multimodal AGCM}\label{sec_eval_multimodal}

    \begin{table}[t]
    \centering
    \caption{Performance comparison of various models in terms of average F-1 score (\%) on the uni- and multimodal Aff-Wild2 dataset.  }
    \begin{tabular}{lllll}
    \toprule
    Type                       & Model                & Arch.           & Data & F-1            \\ \midrule
    \multirow{5}{*}{Black-box} & DAN \cite{wen2023distract}                 & Attention       & V    & 40.10          \\
                               & AFR \cite{savchenko2023adr}                 & EfficientNet    & V    & 42.10          \\
                               & MAE \cite{ma2023unified}                 & MAE             & V    & 44.60          \\
                               & TCN \cite{zhou2023leveraging}                 & Transformer     & V/A  & 41.38          \\
                               & MMAE \cite{zhang2023multi}                & MAE+Transformer & V/A  & \textbf{48.93} \\ \midrule
    Feature                    & FC                   & FC              & V    & 25.27          \\ \midrule
    \multirow{3}{*}{Map}       & TS-CAM \cite{gao2021ts}              & Transformer     & V    & 37.05          \\
                               & \multirow{2}{*}{Att-Map \cite{belharbi2024guided}} & CNN             & V    & \textbf{41.92} \\
                               &                      & Transformer     & V    & 40.87          \\ \midrule
    \multirow{3}{*}{Concept}   & CEM \cite{xinyuFG24}                 &                 & V    & 42.60          \\
                               & AGCM                 &                 & V    & 44.95          \\
                               & AGCM                & Multimodal Fusion  & V/A  & \textbf{47.52} \\ \bottomrule
    \end{tabular}
    \label{tab_affwild2}
    \end{table}
    
    AGCM framework is designed to be expandable to multimodal inputs and concepts. In this work, we use the most commonly used audio-visual dataset as an example, demonstrating the AGCM's capacity for aligning and co-learning information from spatial and temporal modalities. 
    
    To evaluate the overall performance of the AGCM framework in a multimodal context, we conducted comprehensive experiments using the audio-visual Aff-Wild2 dataset.

    Table \ref{tab_affwild2} presents the performance comparison in terms of the average F-1 score on the Aff-Wild2 dataset. The proposed AGCM framework consistently outperforms feature-based, map-based, and concept-based interpretable models. Notably, AGCM in a multimodal setting achieves competitive results compared to state-of-the-art black-box models that leverage multimodal data, while maintaining conceptual explainability.

    Specifically, AGCM attains an F-1 score of 47.52\% by combining visual and audio inputs, outperforming visual-only AGCM (+2.57\%) and CEM (+4.92\%), showing that generally it works better in the multimodal setting. In comparison to feature-based models, AGCM demonstrates a significant improvement (+22.25\%), emphasizing the effectiveness of concept-level multimodal alignment and co-learning. While CNN-based map models \cite{belharbi2024guided} show stronger performance among map-based approaches, they still lag behind AGCM (by 3.03\%) and AGCM (by 5.6\%).
    
    The black-box MMAE model \cite{zhang2023multi} achieves the highest F-1 score of 48.93\%, largely due to its use of a pre-trained transformer (Masked Autoencoder or MAE), which is computationally expensive, time-consuming, and lacks interpretability. In contrast, the competitive results of AGCM highlight its ability to deliver robust performance while offering interpretability, which is a key advantage over black-box methods, even in the real-world multimodal context.


\subsection{Concept Efficiency}

    \begin{table}[t]
    \centering
    \caption{Concept Alignment Score (CAS) in percentage for all tasks. The score for no concept serves as a comparison. NOXI refers to the engagement estimation task in Section \ref{sec_noxi}.}
    \begin{tabular}{lcccc}
    \toprule
                & No Concept & CEM   & AGCM-V  & AGCM-AV \\ \midrule
    RAF-DB      & 66.10      & 78.62 & 82.36 & -     \\
    AffectNet-7 & 67.51      & 78.43 & 84.33 & -     \\
    AffectNet-8 & 66.29      & 78.01 & 83.52 & -     \\
    Aff-Wild2   & 65.09      & 77.36 & 81.29 & 81.46 \\
    NOXI        & 63.58      & 76.50 & 80.83 & 82.11 \\ \bottomrule
    \end{tabular}
    \label{tab_cas}
    \end{table}

    The efficiency of predicted concepts is a critical metric for both performance as well as explainability. To evaluate the reliability of learned concept representations, we employ the Concept Alignment Score (CAS) \cite{zarlenga2022concept}, which measures how well the predicted concepts align with their corresponding ground truth labels. Unlike traditional accuracy, which struggles with defining thresholds between ``activated'' and ``inactivated'' concepts, CAS uses homogeneity scores and clustering algorithms to assess the proximity of predicted concepts to ground truth, providing a more robust measure of concept alignment.

    As shown in Table \ref{tab_cas}, models without concept supervision (No Concept) serve as a baseline for comparison. The proposed framework in visual (AGCM-V) and audio-visual (AGCM-AV) contexts outperform the previous CEM models \cite{xinyuFG24}, which give higher CAS across all datasets, indicating their superior ability to learn meaningful and aligned concepts for both visual and audio modalities. 

\subsection{Human Interpretable Conceptual Explanation}
    
    In addition to achieving competitive performance compared to black-box deep learning models, a significant advantage of concept-based frameworks lies in their ability to offer clear, human-interpretable conceptual explanations grounded in domain-specific knowledge, making them accessible to even non-AI experts.

\subsubsection{Spatial Conceptual Explanation}
    Compared to the map-based approaches that only give one activation map as an explanation, AGCM combines the advantage of both concept-based and map-based models, which not only identifies \textit{where} the model focuses during inference but also explains \textit{what} specific facial behaviors the model is focusing on. 

    Fig. \ref{fig_heatmap_pred} illustrates the spatial concept explanations generated by the proposed AGCM for a facial image classified as ``Happiness'' from the AffectNet test set. During inference, AGCM produces attention maps for all relevant concepts and assigns probability scores based on the areas of the face highlighted in the maps. Concepts with higher probabilities, such as AU12 (Lip Corner Puller), are identified as making a significant contribution to the final classification, while those with lower probabilities, such as AU28 (Lip Suck), are effectively suppressed by the concept generator, reducing their influence on the predicted label. Compared to the map-based XAI that gives only a single attention map as the explanation, as in Fig. \ref{fig_intro}, the proposed model focuses on every possible expression indicator all over the facial region and then assigns the concept score to further indicate its contribution to a specific affective label, efficiently overcoming the trade-off between explainability and performance. 

    \begin{figure}[t]
    \centering
    \includegraphics[width=0.99\columnwidth]{fig_6.pdf}
       \caption{AGCM offers human-interpretable and intuitive explanations by presenting the contribution of each concept to the prediction alongside its spatial location. The numbers indicate the predicted probability scores for all concepts. During inference, the proposed AGCM generates attention maps for all concepts and assigns probability scores based on the highlighted regions. Concepts with higher probabilities (e.g., AU12) indicate greater contributions to the final label, while concepts with lower probabilities (e.g., AU28) are suppressed by AGCM's concept generator. }
    \label{fig_heatmap_pred}
    \end{figure}

    To simplify the visualization of the overall conceptual explanation, we proposed a weighted concept attention map $\bar{\alpha}$ that combines $i$-th predicted attention heatmaps $\hat{\alpha}_{i}$ with its corresponding concept probability $\hat{p}_{i}$, as given in (\ref{eq_norm_map}). Here, \textit{Norm} represents the min-max normalization, $n$ is the total number of concepts, and $\mathbb{I}(\hat{p}_{i} \geq \rho)$ is an indicator function that includes only concepts with probabilities exceeding the threshold $\rho$. We set $\rho = 0.5$ to visualize all activated concepts.
    %
    \begin{equation}\label{eq_norm_map}
        \bar{\alpha} = \textit{Norm}\left(\sum_{i=1}^{n} \hat{\alpha}_{i}\cdot \hat{p}_{i} \cdot \mathbb{I}(\hat{p}_{i} \geq \rho)\right)
    \end{equation}
    %
    \begin{figure}[t]
    \centering
    \includegraphics[width=0.99\columnwidth]{fig_7.pdf}
       \caption{Example of the facial expression label prediction, top-4 concept probability predictions (\%), and weighted concept attention visualization from the AffectNet and RAF-DB test sets. The proposed AGCM framework offers intuitive interpretability by identifying the most contributing concepts to the prediction (addressing the \textit{what} question) and providing spatial explanations for where these concepts are observed (addressing the \textit{where} question).}
    \label{fig_example_correct}
    \end{figure}

    Fig. \ref{fig_example_correct} shows examples randomly selected from the AffectNet and RAF-DB test sets, illustrating the prediction of emotion labels alongside the top-4 concept probabilities (\%) and corresponding weighted concept attention visualizations. The AGCM framework accurately predicts class labels and provides insightful conceptual explanations through activated concept probabilities and attention heatmaps.

    In the ``Happy'' example, AU6 (Cheek Raiser), AU12 (Lip Corner Puller), and AU14 (Dimpler) are all strong indicators of happiness. AGCM efficiently focuses on the relevant facial areas while highlighting the contributions of these concepts. For the ``Anger'' expression, the model emphasizes AU4 (Brow Lowerer) and AU9 (Nose Wrinkler), which are the primary contributors to this emotion, with the attention maps focusing meaningfully on the brow and nose regions. These examples demonstrate AGCM's ability to combine robust performance with clear, human-interpretable conceptual explanations, making it readily applicable to domain-specific expertise.

\subsubsection{Spatial-temporal Conceptual Explanation}

    Another key advantage of the AGCM framework over previous interpretable approaches \cite{xinyuFG24, gao2021ts, belharbi2024guided}, is its ability to provide multimodal conceptual explanations from spatial and temporal data sources. Using audio-visual fusion as an example, AGCM enables the model to co-learn the information from multimodal data inputs, offering robust performance and better interpretable explanations in real-world multimodal contexts.

    Fig. \ref{fig_example_affwild2} illustrates an example of FER prediction on the Aff-Wild2 test set. We randomly selected this video clip to show approximately 10 seconds of data, which contains an emotional transition and downregulation event. Initially, the subject is in a ``Surprise'' state, where AGCM accurately identifies key visual concepts, such as AU25 (Lips Part) and AU1 (Inner Brow Raiser), which strongly indicate this emotion. 

    As the emotional transition occurs, AU1 decreases while AU12 (Lip Corner Puller) becomes dominant, signaling a shift toward a ``Happy'' state. Additionally, the model detects high intensities in pitch and loudness concepts, which are often associated with happiness because they reflect a sudden increase in physiological arousal, and are a natural reaction to pleasant and positive emotions \cite{kamilouglu2020good}. Toward the end of the clip, all concepts gradually decline, reflecting the downregulation of a high-intensity emotion back to a neutral state. AGCM enables the co-learning and interpretation of multimodal inputs by providing \textit{what}-\textit{where} explanations for the visual modality and identifying \textit{what} key conceptual insights derived from temporal signals. Additionally, temporal dependencies (\textit{where} in time) are handled through attention-based sequential learning during multimodal fusion, ensuring comprehensive interpretability across modalities.

    \begin{figure}[t]
    \centering
    \includegraphics[width=0.99\columnwidth]{fig_8.pdf}
       \caption{AGCM facilitates both the co-learning and interpretation of multimodal inputs. In addition to providing \textit{what}-\textit{where} explanations for the visual modality, AGCM offers \textit{what} the key conceptual insights into temporal signals. Temporal dependencies (\textit{where} in time) are naturally addressed through attention-based sequential learning. This figure shows an example from the Aff-Wild2 test set (~10 seconds), demonstrating this capability by including facial expression label predictions, top-2 AU probability predictions (\%), acoustic concept intensities (\%), and weighted concept attention visualizations. AGCM accurately predicts emotion transitions and downregulation while delivering human-interpretable conceptual explanations for both visual and acoustic modalities.}
    \label{fig_example_affwild2}
    \end{figure}

\subsection{Robustness of the Explanation}
    
    To further evaluate the robustness of the model's explanations, we stress-test AGCM to explore its ability to handle challenging scenarios. Facial occlusion is a common challenge in real-world affective signal processing applications, particularly in in-the-wild datasets, where the subjects may wear VR glasses, causing upper-face occlusion, or masks, leading to lower-face occlusion. These occlusions present difficulties for affective computing, especially when providing conceptual or map-based explanations. The proposed AGCM framework addresses this limitation by generating weighted concept attention maps, which improve both the performance and the interpretability.
    
    To simulate real-world occlusion scenarios, we selected images from the Aff-Wild2 test set and manually occluded either the upper or lower face regions, re-evaluating the performance of the well-trained AGCM framework.

    As shown in Fig. \ref{fig_concept_occ}, we randomly selected samples with varying facial expressions, lighting conditions, and angles, then removed either the upper or lower face regions. Using the same well-trained AGCM model, we re-evaluated the predicted emotion labels, representative top concept probabilities, and the corresponding weighted concept attention maps. After occlusion, AGCM still accurately predicts the emotion by focusing on the unobstructed facial regions. In the ``Happy'' examples, the model shifts attention away from AU6 (Cheek Raiser), which is occluded and focuses more on AU12 (Lip Corner Puller), resulting in a correct prediction despite the occlusion. Similarly, in the ``Surprise'' example, AGCM downweights the contribution of the occluded AU26 (Jaw Drop) and instead focuses on AU2 (Outer Brow Raiser), another strong indicator of surprise. These results demonstrate AGCM’s robustness in handling occluded facial images while maintaining accurate and interpretable predictions.

    Hand-over-face occlusion presents an even more complex challenge than occlusion caused by glasses and masks, as the hand can often be misinterpreted as part of the face during model inference because one's hands often share similar textures with the face. To evaluate AGCM’s performance in such scenarios, we selected additional samples from the Aff-Wild2 dataset, which contains instances of hand-over-face occlusion.

    Fig. \ref{fig_example_occ} shows test images featuring hand-over-face occlusion. Despite these occlusions, AGCM generates accurate emotion predictions by leveraging a few key concepts. For instance, in the ``Surprise'' example, even though the lower-face concepts are occluded, the model identifies high probabilities for upper-face indicators AU1 (Inner Brow Raiser) and AU2, leading to a correct prediction. Similarly, in the ``Happy'' example, AU6 (Cheek Raiser) alone is sufficient for the model to make this accurate prediction. 
    
    These stress-testing results demonstrate that AGCM effectively handles partial face occlusion and hand-over-face occlusion by focusing on unobstructed regions and leveraging spatial concept learning to emphasize visible concepts during training. This capability highlights AGCM's robust, concept-aware spatial explanations, enabling reliable predictions even in challenging scenarios.


    \begin{figure}[t]
    \centering
    \includegraphics[width=0.99\columnwidth]{fig_9.pdf}
       \caption{Example of the face occlusion samples, with the expression label prediction, representative concept probability predictions (\%) from the upper and lower-face region, and weighted concept attention visualization from the Aff-Wild2 test sets. After occlusion, AGCM adapts by shifting attention to the non-occluded areas, ensuring reliable predictions based on the remaining visible concepts. }
    \label{fig_concept_occ}
    \end{figure}
    
    
    \begin{figure}[t]
    \centering
    \includegraphics[width=0.99\columnwidth]{fig_10.pdf}
       \caption{Example of hand-over-face occlusion, with the predicted facial expression label, top-4 concept probability predictions (\%), and weighted concept attention visualization. AGCM accurately focuses on the non-occluded regions and predicts the task label based on the available concepts, demonstrating its robustness in handling facial expressions with hand-over-face occlusion.}
    \label{fig_example_occ}
    \end{figure}

\subsection{Ablation Study}
    Compared to the previous concept-based approaches, the proposed AGCM framework introduces four main components, including Multi-scale Spatial Attention (MSA), Multi-head Attention (MHA), Cannel Attended Concept Mapping (CACM), and Concept Map Loss (CML). As the evaluation of multimodal concept fusion has been given in Section \ref{sec_eval_multimodal}, this section provides an ablation study to show the efficiency of the visual-based AGCM framework. 

    \begin{table}[t]
    \centering
    \caption{Ablation study of the visual-based AGCM framework on RAF-DB and AffectNet-8 test set.}
    \begin{tabular}{cccc|cc}
    \toprule
    MSA & MHA & CACM & CML & RAF-DB   & AffectNet-8 \\ \midrule
    -   & -   & -    & -   & 90.47 & 62.58     \\ 
    +   & -   & -    & -   & 92.84 & 62.99     \\
    +   & +   & -    & -   & 93.26 & 63.10     \\
    +   & +   & +    & -   & 93.31 & 63.46     \\
    +   & +   & +    & +   & \textbf{94.40} & \textbf{65.62}     \\
    \bottomrule
    \end{tabular}
    \label{tab_abl}
    \end{table}

    Table \ref{tab_abl} presents the ablation study for the visual-based AGCM framework on RAF-DB and AffectNet-8. The baseline model without any components achieves 90.47\% on RAF-DB and 62.58\% on AffectNet-8. Adding Multi-scale Spatial Attention (MSA) improves performance significantly, reaching 92.84\% and 62.99\%. Introducing Multi-head Attention (MHA) further boosts accuracy to 93.26\% and 63.10\%, while Channel Attended Concept Mapping (CACM) provides a slight improvement to 93.31\% and 63.46\%. Finally, the full AGCM with Concept Map Loss (CML) achieves the best results, 94.40\% on RAF-DB and 65.62\% on AffectNet-8, demonstrating the cumulative benefit of these components in enhancing accuracy while maintaining explainability.


\section{AGCM for Interpretable Engagement Estimation}\label{sec_noxi}

    The generalizability of the framework to downstream applications is essential for establishing a trustworthy AC system. Real-life affective signal processing is inherently more ambiguous, complex, and diverse compared to the well-defined FER task. One good example is human-human interactions, where the conversational engagement score is designed to measure the level and rate of engagement between participants, illustrating the broader and more nuanced requirements of real-world AC applications.
    
    In this section, we use the NOvice eXpert Interaction (NOXI) dataset, a large-scale,  well-annotated human-human interaction dataset with the engagement label, to illustrate AGCM's generalization capacity in real-world AC contexts. We conduct both qualitative and quantitative evaluations, demonstrating that AGCM achieves robust performance by automatically identifying key indicators and highlighting essential concepts.

    NOXI \cite{cafaro2017noxi} is designed for the analysis of human interaction in real-world, cross-cultural settings. It includes video recordings of novice-expert interactions in eight languages (English, French, German, Spanish, Indonesian, Arabic, Dutch, and Italian), with AU capture via Microsoft \textit{Kinect} \cite{zhang2012microsoft}. The dataset spans over 50 hours of video and is annotated with a \textbf{by-frame} engagement score ranging from 0 to 1. For our experiments, we utilize 76 videos (over 1.5 million frames) for training and 20 videos (over 500,000 frames) for testing.

\subsection{Generalizing AGCM for Engagement Estimation}

    \begin{table}[t]
    \centering
    \caption{Performance comparison of various models in terms of Concordance Correlation Coefficient (CCC) on the uni- and multimodal Noxi dataset.}
    \begin{tabular}{lllcc}
    \toprule
    Type                       & Model   & Arch.          & Data & CCC           \\ \midrule
    \multirow{3}{*}{Black-box} & TCA \cite{he2024tca}    & Attention      & V/A   & 0.73          \\
                               & DCTM \cite{tu2023dctm}   & Transformer    & V/A    & 0.77          \\
                               & S2S \cite{yu2023sliding}     & Transformer    & V/A    & \textbf{0.83} \\ \midrule
    Feature                    & FC      & FC             & V    & 0.23          \\ \midrule
    \multirow{2}{*}{Map}       & TS-CAM \cite{gao2021ts} & Transformer    & V    & 0.36          \\
                               & Att-Map \cite{belharbi2024guided} &                & -    & -             \\ \hline
    Concept                    & CEM     &                & V    & 0.48          \\
                               & AGCM    &                & V    & 0.59          \\
                               & AGCM    & Concept Fusion & V/A  & \textbf{0.80} \\
    \bottomrule
    \end{tabular}
    \label{tab_noxi}
    \end{table}

    AGCM is highly generalizable to downstream AC applications by simply adjusting the configuration of the final task predictor. For example, in FER tasks, a classification header is utilized, whereas in continuous signal prediction tasks, a regression header is employed. This flexibility allows AGCM to adapt a wide range of affective computing applications.

    Table \ref{tab_noxi} shows the performance comparison of various models in terms of the Concordance Correlation Coefficient (CCC) on the Noxi dataset for continuous engagement estimation. CCC is used to evaluate continuous tasks by measuring the agreement between predicted and true values, accounting for both correlation and accuracy, making it ideal for engagement estimation tasks. The proposed multimodal AGCM framework with audio-visual concept fusion again outperforms feature-based and previous concept-based models, showing its outstanding state-of-the-art performance in downstream real-world engagement estimation tasks.

    In the unimodal setting, AGCM with visual concepts achieves a CCC score of 0.59, marking a substantial improvement over the unimodal CEM (+0.11) and feature-based model (+0.23). This result highlights the advantages of spatial concept learning while underscoring the limitations of feature-based models in addressing the complexities of affective signal processing.

    In the multimodal context, AGCM attains a performance of 0.80, demonstrating the significant benefits of co-learning multimodal knowledge. This is particularly valuable in complex real-world AC applications, such as engagement estimation, where multiple modalities are essential for capturing and understanding nuanced human behavior.

    Although the black-box S2S model \cite{yu2023sliding} slightly outperforms AGCM with a CCC of 0.03, AGCM underscores its ability to approximate state-of-the-art results while maintaining interpretability. The attention map-based models \cite{belharbi2024guided} are not well-suited to this task, as they rely on predefined mappings between AUs and labels, which are not available for continuous engagement estimation. Additionally, TS-CAM \cite{gao2021ts}, which is restricted to the visual modality, also performs poorly in engagement estimation.

    Meanwhile, the Concept Alignment Score (CAS), as shown in Table \ref{tab_cas}, illustrates that the AGCM framework with audio-visual co-learning not only maintains competitive performance compared to state-of-the-art black-box deep learning models but also delivers accurate conceptual explanations. 
    
    Therefore, this sophisticated interpretable framework maintains competitive performance without compromise. By simply configuring the AGCM classifier, the performance evaluation on engagement estimation demonstrates the strong generalizability of AGCM to a wide range of downstream applications beyond FER, making it both powerful and accessible for diverse affective computing tasks.

\subsection{AGCM Explainability in Engagement Estimation}
    \begin{figure}[t]
    \centering
    \includegraphics[width=0.99\columnwidth]{fig_11.pdf}
       \caption{Example of the engagement estimation, gaze, head pose direction (mean degree of forward gaze or facing forward in x and y directions), top-1 AU probability predictions (\%), acoustic concept intensities (\%), and weighted concept attention visualization of a Noxi test sample (around 60 seconds). The proposed AGCM framework accurately predicts engagement transition for different states during conversation and provides meaningful visual and acoustic conceptual explanations.}
    \label{fig_example_noxi}
    \end{figure}

    Explainability becomes even more crucial in downstream AC applications compared to FER, given the inherent complexity of human behavior. In tasks such as engagement estimation, delivering domain-specific explanations is vital for non-AI stakeholders to understand and interpret the decision-making process.

\subsubsection{Explaining Engagement Transitions}

    To show AGCM's explanation and prediction capabilities in human-human engagement estimation, Fig. \ref{fig_example_noxi} presents an example from the NOXI dataset. This sample, randomly selected to cover approximately 60 seconds of data, highlights engagement transitions between listening, distraction, and interaction.

    At the beginning of the sequence, the subject actively listens with a direct gaze toward the speaker, as indicated by the high intensity of the conceptual direct gaze. In conversation-based engagement estimation, gaze direction and head pose are critical concepts for predicting and explaining engagement scores. Since the acoustic input is not prominent during the listening phase, the intensities of acoustic concepts remain low, which is expected as the audio track of each subject is recorded separately in this dataset.

    When distraction occurs, the subject shifts attention to a phone call or another person, causing the intensities of direct gaze and forward head pose concepts to decrease, which in turn lowers the engagement score. When the subject looks down with eyes only partly open, as evidenced by the activation of AU45 (Blink), the predicted engagement score reaches its lowest point, signifying that the subject’s attention is fully disengaged from the conversation.

    As the distraction ends and the subject re-engages with the speaker, positive facial expressions, such as AU12, become prominent, associated with increased engagement during interactions \cite{Greipl2021Facial, Savchenko2022Classifying}. The intensities of gaze and head pose concepts increase, and acoustic concepts begin to register, indicating the subject's regained focus and positive emotional feedback. This highlights the subject’s re-engagement in the conversation. 6
    
    AGCM effectively leverages these multimodal concepts to capture subtle changes in engagement states and ensure robust conceptual explainability during inference, demonstrating its strong generalizability to downstream AC applications with complex behavioral labels, extending beyond the scope of FER.

\subsubsection{Explaining Complex Human Behaviours}
    
    Real-world behavioral states are more complex than facial expressions. Higher-level affective states that share similar feature representations usually introduce ambiguity in task predictions, especially in feature-based models. 
    
    The proposed AGCM framework has an inherent advantage in differentiating nuanced states by autonomously learning and capturing both the concept-aware similarities and distinctions between these affective states. Take an example of human-human interaction, real-world applications often involve complex affective states that are more ambiguous and abstract compared to discrete emotions, such as distraction and cognitive load \cite{Krasich2018GazeBased}. 
    
    Fig. \ref{fig_cognitive_load} provides an example of engagement estimation in the presence of distraction and cognitive Load. Distraction occurs when the subject's gaze drifts away, indicating mental disengagement. Conversely, cognitive load happens when the subject looks away while remaining engaged in processing information.  In feature-based affective computing models, these complex behaviors which share similar feature representations, can introduce ambiguity in task predictions. 
    
    
    \begin{figure}[t]
    \centering
    \includegraphics[width=0.95\columnwidth]{fig_12.pdf}
       \caption{Example of the engagement estimation for distraction and cognitive load, with the prediction of gaze, head pose direction (mean degree of forward gaze or facing forward in x and y directions), top-1 AU probability predictions (\%), acoustic concept intensities (\%), and weighted concept attention visualization from Noxi dataset. AGCM differentiates between distraction and cognitive load according to efficient concept learning. }
    \label{fig_cognitive_load}
    \end{figure}


    Thus, the proposed AGCM framework provides robust learning and explainability, even in complex behavioral states such as distraction and cognitive load. This demonstrates its effectiveness in capturing nuanced affective states, providing enhanced generalizability to complex downstream AC applications that are difficult to tackle using conventional methods.


\section{Conclusion \& Future Work}

    In this paper, we introduce the Attention-Guided Concept Model (AGCM), a multimodal concept-based interpretable framework that provides conceptual explanations of \textit{what} concepts contribute to the predictions and \textit{where} they are observed. AGCM is highly extendable to various spatial-temporal modalities, effectively addressing the challenges of multimodal alignment, fusion, and co-learning. The framework demonstrates strong generalizability and flexibility, making it well-suited for diverse real-world AC applications.
    
    We first validate the model's effectiveness in achieving both high performance and robust explanation through qualitative and quantitative evaluations on well-established FER datasets. Then, we demonstrate the generalizability of the AGCM framework to other complex real-world AC applications by extensive experiments on the human-human interaction task. We believe that AGCM establishes a foundation for creating future interpretable systems in downstream AC applications, such as psychology, psychiatry, digital behavior, and Human-Computer Interaction, with competitive performance and human-interpretable explanation.



    AGCM leverages the strengths of both feature-based models and deep black-box models to offer interpretable, high-performance predictions. However, explainability in affective computing remains an evolving area of research. We posit that model explanations should be tailored to end-users, such as psychologists and cognitive scientists. Therefore, we plan to incorporate a human-in-the-loop approach for affective XAI to further enhance model usability. Additionally, while AGCM is trained on large datasets, exploring XAI fairness in terms of gender, cultural, and age biases presents an interesting avenue for further investigation. In this paper, we assess various forms of occlusion using the Aff-Wild2 dataset; future improvements could be achieved by fine-tuning AGCM on occlusion-specific datasets to better handle such challenges. Generating text-based explanations via Large Language Models (LLM) may also give users extra insights. However, given the inherent complexity of LLMs, it is imperative to employ appropriate knowledge distillation techniques, particularly for cross-disciplinary stakeholders.

% {\small
% \bibliographystyle{ieee}
% \bibliography{egbib}
% }

{\small
% This must be in the first 5 lines to tell arXiv to use pdfLaTeX, which is strongly recommended.
\pdfoutput=1
% In particular, the hyperref package requires pdfLaTeX in order to break URLs across lines.

\documentclass[11pt]{article}

% Change "review" to "final" to generate the final (sometimes called camera-ready) version.
% Change to "preprint" to generate a non-anonymous version with page numbers.
\usepackage{acl}

% Standard package includes
\usepackage{times}
\usepackage{latexsym}

% Draw tables
\usepackage{booktabs}
\usepackage{multirow}
\usepackage{xcolor}
\usepackage{colortbl}
\usepackage{array} 
\usepackage{amsmath}

\newcolumntype{C}{>{\centering\arraybackslash}p{0.07\textwidth}}
% For proper rendering and hyphenation of words containing Latin characters (including in bib files)
\usepackage[T1]{fontenc}
% For Vietnamese characters
% \usepackage[T5]{fontenc}
% See https://www.latex-project.org/help/documentation/encguide.pdf for other character sets
% This assumes your files are encoded as UTF8
\usepackage[utf8]{inputenc}

% This is not strictly necessary, and may be commented out,
% but it will improve the layout of the manuscript,
% and will typically save some space.
\usepackage{microtype}
\DeclareMathOperator*{\argmax}{arg\,max}
% This is also not strictly necessary, and may be commented out.
% However, it will improve the aesthetics of text in
% the typewriter font.
\usepackage{inconsolata}

%Including images in your LaTeX document requires adding
%additional package(s)
\usepackage{graphicx}
% If the title and author information does not fit in the area allocated, uncomment the following
%
%\setlength\titlebox{<dim>}
%
% and set <dim> to something 5cm or larger.

\title{Wi-Chat: Large Language Model Powered Wi-Fi Sensing}

% Author information can be set in various styles:
% For several authors from the same institution:
% \author{Author 1 \and ... \and Author n \\
%         Address line \\ ... \\ Address line}
% if the names do not fit well on one line use
%         Author 1 \\ {\bf Author 2} \\ ... \\ {\bf Author n} \\
% For authors from different institutions:
% \author{Author 1 \\ Address line \\  ... \\ Address line
%         \And  ... \And
%         Author n \\ Address line \\ ... \\ Address line}
% To start a separate ``row'' of authors use \AND, as in
% \author{Author 1 \\ Address line \\  ... \\ Address line
%         \AND
%         Author 2 \\ Address line \\ ... \\ Address line \And
%         Author 3 \\ Address line \\ ... \\ Address line}

% \author{First Author \\
%   Affiliation / Address line 1 \\
%   Affiliation / Address line 2 \\
%   Affiliation / Address line 3 \\
%   \texttt{email@domain} \\\And
%   Second Author \\
%   Affiliation / Address line 1 \\
%   Affiliation / Address line 2 \\
%   Affiliation / Address line 3 \\
%   \texttt{email@domain} \\}
% \author{Haohan Yuan \qquad Haopeng Zhang\thanks{corresponding author} \\ 
%   ALOHA Lab, University of Hawaii at Manoa \\
%   % Affiliation / Address line 2 \\
%   % Affiliation / Address line 3 \\
%   \texttt{\{haohany,haopengz\}@hawaii.edu}}
  
\author{
{Haopeng Zhang$\dag$\thanks{These authors contributed equally to this work.}, Yili Ren$\ddagger$\footnotemark[1], Haohan Yuan$\dag$, Jingzhe Zhang$\ddagger$, Yitong Shen$\ddagger$} \\
ALOHA Lab, University of Hawaii at Manoa$\dag$, University of South Florida$\ddagger$ \\
\{haopengz, haohany\}@hawaii.edu\\
\{yiliren, jingzhe, shen202\}@usf.edu\\}



  
%\author{
%  \textbf{First Author\textsuperscript{1}},
%  \textbf{Second Author\textsuperscript{1,2}},
%  \textbf{Third T. Author\textsuperscript{1}},
%  \textbf{Fourth Author\textsuperscript{1}},
%\\
%  \textbf{Fifth Author\textsuperscript{1,2}},
%  \textbf{Sixth Author\textsuperscript{1}},
%  \textbf{Seventh Author\textsuperscript{1}},
%  \textbf{Eighth Author \textsuperscript{1,2,3,4}},
%\\
%  \textbf{Ninth Author\textsuperscript{1}},
%  \textbf{Tenth Author\textsuperscript{1}},
%  \textbf{Eleventh E. Author\textsuperscript{1,2,3,4,5}},
%  \textbf{Twelfth Author\textsuperscript{1}},
%\\
%  \textbf{Thirteenth Author\textsuperscript{3}},
%  \textbf{Fourteenth F. Author\textsuperscript{2,4}},
%  \textbf{Fifteenth Author\textsuperscript{1}},
%  \textbf{Sixteenth Author\textsuperscript{1}},
%\\
%  \textbf{Seventeenth S. Author\textsuperscript{4,5}},
%  \textbf{Eighteenth Author\textsuperscript{3,4}},
%  \textbf{Nineteenth N. Author\textsuperscript{2,5}},
%  \textbf{Twentieth Author\textsuperscript{1}}
%\\
%\\
%  \textsuperscript{1}Affiliation 1,
%  \textsuperscript{2}Affiliation 2,
%  \textsuperscript{3}Affiliation 3,
%  \textsuperscript{4}Affiliation 4,
%  \textsuperscript{5}Affiliation 5
%\\
%  \small{
%    \textbf{Correspondence:} \href{mailto:email@domain}{email@domain}
%  }
%}

\begin{document}
\maketitle
\begin{abstract}
Recent advancements in Large Language Models (LLMs) have demonstrated remarkable capabilities across diverse tasks. However, their potential to integrate physical model knowledge for real-world signal interpretation remains largely unexplored. In this work, we introduce Wi-Chat, the first LLM-powered Wi-Fi-based human activity recognition system. We demonstrate that LLMs can process raw Wi-Fi signals and infer human activities by incorporating Wi-Fi sensing principles into prompts. Our approach leverages physical model insights to guide LLMs in interpreting Channel State Information (CSI) data without traditional signal processing techniques. Through experiments on real-world Wi-Fi datasets, we show that LLMs exhibit strong reasoning capabilities, achieving zero-shot activity recognition. These findings highlight a new paradigm for Wi-Fi sensing, expanding LLM applications beyond conventional language tasks and enhancing the accessibility of wireless sensing for real-world deployments.
\end{abstract}

\section{Introduction}

In today’s rapidly evolving digital landscape, the transformative power of web technologies has redefined not only how services are delivered but also how complex tasks are approached. Web-based systems have become increasingly prevalent in risk control across various domains. This widespread adoption is due their accessibility, scalability, and ability to remotely connect various types of users. For example, these systems are used for process safety management in industry~\cite{kannan2016web}, safety risk early warning in urban construction~\cite{ding2013development}, and safe monitoring of infrastructural systems~\cite{repetto2018web}. Within these web-based risk management systems, the source search problem presents a huge challenge. Source search refers to the task of identifying the origin of a risky event, such as a gas leak and the emission point of toxic substances. This source search capability is crucial for effective risk management and decision-making.

Traditional approaches to implementing source search capabilities into the web systems often rely on solely algorithmic solutions~\cite{ristic2016study}. These methods, while relatively straightforward to implement, often struggle to achieve acceptable performances due to algorithmic local optima and complex unknown environments~\cite{zhao2020searching}. More recently, web crowdsourcing has emerged as a promising alternative for tackling the source search problem by incorporating human efforts in these web systems on-the-fly~\cite{zhao2024user}. This approach outsources the task of addressing issues encountered during the source search process to human workers, leveraging their capabilities to enhance system performance.

These solutions often employ a human-AI collaborative way~\cite{zhao2023leveraging} where algorithms handle exploration-exploitation and report the encountered problems while human workers resolve complex decision-making bottlenecks to help the algorithms getting rid of local deadlocks~\cite{zhao2022crowd}. Although effective, this paradigm suffers from two inherent limitations: increased operational costs from continuous human intervention, and slow response times of human workers due to sequential decision-making. These challenges motivate our investigation into developing autonomous systems that preserve human-like reasoning capabilities while reducing dependency on massive crowdsourced labor.

Furthermore, recent advancements in large language models (LLMs)~\cite{chang2024survey} and multi-modal LLMs (MLLMs)~\cite{huang2023chatgpt} have unveiled promising avenues for addressing these challenges. One clear opportunity involves the seamless integration of visual understanding and linguistic reasoning for robust decision-making in search tasks. However, whether large models-assisted source search is really effective and efficient for improving the current source search algorithms~\cite{ji2022source} remains unknown. \textit{To address the research gap, we are particularly interested in answering the following two research questions in this work:}

\textbf{\textit{RQ1: }}How can source search capabilities be integrated into web-based systems to support decision-making in time-sensitive risk management scenarios? 
% \sq{I mention ``time-sensitive'' here because I feel like we shall say something about the response time -- LLM has to be faster than humans}

\textbf{\textit{RQ2: }}How can MLLMs and LLMs enhance the effectiveness and efficiency of existing source search algorithms? 

% \textit{\textbf{RQ2:}} To what extent does the performance of large models-assisted search align with or approach the effectiveness of human-AI collaborative search? 

To answer the research questions, we propose a novel framework called Auto-\
S$^2$earch (\textbf{Auto}nomous \textbf{S}ource \textbf{Search}) and implement a prototype system that leverages advanced web technologies to simulate real-world conditions for zero-shot source search. Unlike traditional methods that rely on pre-defined heuristics or extensive human intervention, AutoS$^2$earch employs a carefully designed prompt that encapsulates human rationales, thereby guiding the MLLM to generate coherent and accurate scene descriptions from visual inputs about four directional choices. Based on these language-based descriptions, the LLM is enabled to determine the optimal directional choice through chain-of-thought (CoT) reasoning. Comprehensive empirical validation demonstrates that AutoS$^2$-\ 
earch achieves a success rate of 95–98\%, closely approaching the performance of human-AI collaborative search across 20 benchmark scenarios~\cite{zhao2023leveraging}. 

Our work indicates that the role of humans in future web crowdsourcing tasks may evolve from executors to validators or supervisors. Furthermore, incorporating explanations of LLM decisions into web-based system interfaces has the potential to help humans enhance task performance in risk control.






\section{Related Work}
\label{sec:relatedworks}

% \begin{table*}[t]
% \centering 
% \renewcommand\arraystretch{0.98}
% \fontsize{8}{10}\selectfont \setlength{\tabcolsep}{0.4em}
% \begin{tabular}{@{}lc|cc|cc|cc@{}}
% \toprule
% \textbf{Methods}           & \begin{tabular}[c]{@{}c@{}}\textbf{Training}\\ \textbf{Paradigm}\end{tabular} & \begin{tabular}[c]{@{}c@{}}\textbf{$\#$ PT Data}\\ \textbf{(Tokens)}\end{tabular} & \begin{tabular}[c]{@{}c@{}}\textbf{$\#$ IFT Data}\\ \textbf{(Samples)}\end{tabular} & \textbf{Code}  & \begin{tabular}[c]{@{}c@{}}\textbf{Natural}\\ \textbf{Language}\end{tabular} & \begin{tabular}[c]{@{}c@{}}\textbf{Action}\\ \textbf{Trajectories}\end{tabular} & \begin{tabular}[c]{@{}c@{}}\textbf{API}\\ \textbf{Documentation}\end{tabular}\\ \midrule 
% NexusRaven~\citep{srinivasan2023nexusraven} & IFT & - & - & \textcolor{green}{\CheckmarkBold} & \textcolor{green}{\CheckmarkBold} &\textcolor{red}{\XSolidBrush}&\textcolor{red}{\XSolidBrush}\\
% AgentInstruct~\citep{zeng2023agenttuning} & IFT & - & 2k & \textcolor{green}{\CheckmarkBold} & \textcolor{green}{\CheckmarkBold} &\textcolor{red}{\XSolidBrush}&\textcolor{red}{\XSolidBrush} \\
% AgentEvol~\citep{xi2024agentgym} & IFT & - & 14.5k & \textcolor{green}{\CheckmarkBold} & \textcolor{green}{\CheckmarkBold} &\textcolor{green}{\CheckmarkBold}&\textcolor{red}{\XSolidBrush} \\
% Gorilla~\citep{patil2023gorilla}& IFT & - & 16k & \textcolor{green}{\CheckmarkBold} & \textcolor{green}{\CheckmarkBold} &\textcolor{red}{\XSolidBrush}&\textcolor{green}{\CheckmarkBold}\\
% OpenFunctions-v2~\citep{patil2023gorilla} & IFT & - & 65k & \textcolor{green}{\CheckmarkBold} & \textcolor{green}{\CheckmarkBold} &\textcolor{red}{\XSolidBrush}&\textcolor{green}{\CheckmarkBold}\\
% LAM~\citep{zhang2024agentohana} & IFT & - & 42.6k & \textcolor{green}{\CheckmarkBold} & \textcolor{green}{\CheckmarkBold} &\textcolor{green}{\CheckmarkBold}&\textcolor{red}{\XSolidBrush} \\
% xLAM~\citep{liu2024apigen} & IFT & - & 60k & \textcolor{green}{\CheckmarkBold} & \textcolor{green}{\CheckmarkBold} &\textcolor{green}{\CheckmarkBold}&\textcolor{red}{\XSolidBrush} \\\midrule
% LEMUR~\citep{xu2024lemur} & PT & 90B & 300k & \textcolor{green}{\CheckmarkBold} & \textcolor{green}{\CheckmarkBold} &\textcolor{green}{\CheckmarkBold}&\textcolor{red}{\XSolidBrush}\\
% \rowcolor{teal!12} \method & PT & 103B & 95k & \textcolor{green}{\CheckmarkBold} & \textcolor{green}{\CheckmarkBold} & \textcolor{green}{\CheckmarkBold} & \textcolor{green}{\CheckmarkBold} \\
% \bottomrule
% \end{tabular}
% \caption{Summary of existing tuning- and pretraining-based LLM agents with their training sample sizes. "PT" and "IFT" denote "Pre-Training" and "Instruction Fine-Tuning", respectively. }
% \label{tab:related}
% \end{table*}

\begin{table*}[ht]
\begin{threeparttable}
\centering 
\renewcommand\arraystretch{0.98}
\fontsize{7}{9}\selectfont \setlength{\tabcolsep}{0.2em}
\begin{tabular}{@{}l|c|c|ccc|cc|cc|cccc@{}}
\toprule
\textbf{Methods} & \textbf{Datasets}           & \begin{tabular}[c]{@{}c@{}}\textbf{Training}\\ \textbf{Paradigm}\end{tabular} & \begin{tabular}[c]{@{}c@{}}\textbf{\# PT Data}\\ \textbf{(Tokens)}\end{tabular} & \begin{tabular}[c]{@{}c@{}}\textbf{\# IFT Data}\\ \textbf{(Samples)}\end{tabular} & \textbf{\# APIs} & \textbf{Code}  & \begin{tabular}[c]{@{}c@{}}\textbf{Nat.}\\ \textbf{Lang.}\end{tabular} & \begin{tabular}[c]{@{}c@{}}\textbf{Action}\\ \textbf{Traj.}\end{tabular} & \begin{tabular}[c]{@{}c@{}}\textbf{API}\\ \textbf{Doc.}\end{tabular} & \begin{tabular}[c]{@{}c@{}}\textbf{Func.}\\ \textbf{Call}\end{tabular} & \begin{tabular}[c]{@{}c@{}}\textbf{Multi.}\\ \textbf{Step}\end{tabular}  & \begin{tabular}[c]{@{}c@{}}\textbf{Plan}\\ \textbf{Refine}\end{tabular}  & \begin{tabular}[c]{@{}c@{}}\textbf{Multi.}\\ \textbf{Turn}\end{tabular}\\ \midrule 
\multicolumn{13}{l}{\emph{Instruction Finetuning-based LLM Agents for Intrinsic Reasoning}}  \\ \midrule
FireAct~\cite{chen2023fireact} & FireAct & IFT & - & 2.1K & 10 & \textcolor{red}{\XSolidBrush} &\textcolor{green}{\CheckmarkBold} &\textcolor{green}{\CheckmarkBold}  & \textcolor{red}{\XSolidBrush} &\textcolor{green}{\CheckmarkBold} & \textcolor{red}{\XSolidBrush} &\textcolor{green}{\CheckmarkBold} & \textcolor{red}{\XSolidBrush} \\
ToolAlpaca~\cite{tang2023toolalpaca} & ToolAlpaca & IFT & - & 4.0K & 400 & \textcolor{red}{\XSolidBrush} &\textcolor{green}{\CheckmarkBold} &\textcolor{green}{\CheckmarkBold} & \textcolor{red}{\XSolidBrush} &\textcolor{green}{\CheckmarkBold} & \textcolor{red}{\XSolidBrush}  &\textcolor{green}{\CheckmarkBold} & \textcolor{red}{\XSolidBrush}  \\
ToolLLaMA~\cite{qin2023toolllm} & ToolBench & IFT & - & 12.7K & 16,464 & \textcolor{red}{\XSolidBrush} &\textcolor{green}{\CheckmarkBold} &\textcolor{green}{\CheckmarkBold} &\textcolor{red}{\XSolidBrush} &\textcolor{green}{\CheckmarkBold}&\textcolor{green}{\CheckmarkBold}&\textcolor{green}{\CheckmarkBold} &\textcolor{green}{\CheckmarkBold}\\
AgentEvol~\citep{xi2024agentgym} & AgentTraj-L & IFT & - & 14.5K & 24 &\textcolor{red}{\XSolidBrush} & \textcolor{green}{\CheckmarkBold} &\textcolor{green}{\CheckmarkBold}&\textcolor{red}{\XSolidBrush} &\textcolor{green}{\CheckmarkBold}&\textcolor{red}{\XSolidBrush} &\textcolor{red}{\XSolidBrush} &\textcolor{green}{\CheckmarkBold}\\
Lumos~\cite{yin2024agent} & Lumos & IFT  & - & 20.0K & 16 &\textcolor{red}{\XSolidBrush} & \textcolor{green}{\CheckmarkBold} & \textcolor{green}{\CheckmarkBold} &\textcolor{red}{\XSolidBrush} & \textcolor{green}{\CheckmarkBold} & \textcolor{green}{\CheckmarkBold} &\textcolor{red}{\XSolidBrush} & \textcolor{green}{\CheckmarkBold}\\
Agent-FLAN~\cite{chen2024agent} & Agent-FLAN & IFT & - & 24.7K & 20 &\textcolor{red}{\XSolidBrush} & \textcolor{green}{\CheckmarkBold} & \textcolor{green}{\CheckmarkBold} &\textcolor{red}{\XSolidBrush} & \textcolor{green}{\CheckmarkBold}& \textcolor{green}{\CheckmarkBold}&\textcolor{red}{\XSolidBrush} & \textcolor{green}{\CheckmarkBold}\\
AgentTuning~\citep{zeng2023agenttuning} & AgentInstruct & IFT & - & 35.0K & - &\textcolor{red}{\XSolidBrush} & \textcolor{green}{\CheckmarkBold} & \textcolor{green}{\CheckmarkBold} &\textcolor{red}{\XSolidBrush} & \textcolor{green}{\CheckmarkBold} &\textcolor{red}{\XSolidBrush} &\textcolor{red}{\XSolidBrush} & \textcolor{green}{\CheckmarkBold}\\\midrule
\multicolumn{13}{l}{\emph{Instruction Finetuning-based LLM Agents for Function Calling}} \\\midrule
NexusRaven~\citep{srinivasan2023nexusraven} & NexusRaven & IFT & - & - & 116 & \textcolor{green}{\CheckmarkBold} & \textcolor{green}{\CheckmarkBold}  & \textcolor{green}{\CheckmarkBold} &\textcolor{red}{\XSolidBrush} & \textcolor{green}{\CheckmarkBold} &\textcolor{red}{\XSolidBrush} &\textcolor{red}{\XSolidBrush}&\textcolor{red}{\XSolidBrush}\\
Gorilla~\citep{patil2023gorilla} & Gorilla & IFT & - & 16.0K & 1,645 & \textcolor{green}{\CheckmarkBold} &\textcolor{red}{\XSolidBrush} &\textcolor{red}{\XSolidBrush}&\textcolor{green}{\CheckmarkBold} &\textcolor{green}{\CheckmarkBold} &\textcolor{red}{\XSolidBrush} &\textcolor{red}{\XSolidBrush} &\textcolor{red}{\XSolidBrush}\\
OpenFunctions-v2~\citep{patil2023gorilla} & OpenFunctions-v2 & IFT & - & 65.0K & - & \textcolor{green}{\CheckmarkBold} & \textcolor{green}{\CheckmarkBold} &\textcolor{red}{\XSolidBrush} &\textcolor{green}{\CheckmarkBold} &\textcolor{green}{\CheckmarkBold} &\textcolor{red}{\XSolidBrush} &\textcolor{red}{\XSolidBrush} &\textcolor{red}{\XSolidBrush}\\
API Pack~\cite{guo2024api} & API Pack & IFT & - & 1.1M & 11,213 &\textcolor{green}{\CheckmarkBold} &\textcolor{red}{\XSolidBrush} &\textcolor{green}{\CheckmarkBold} &\textcolor{red}{\XSolidBrush} &\textcolor{green}{\CheckmarkBold} &\textcolor{red}{\XSolidBrush}&\textcolor{red}{\XSolidBrush}&\textcolor{red}{\XSolidBrush}\\ 
LAM~\citep{zhang2024agentohana} & AgentOhana & IFT & - & 42.6K & - & \textcolor{green}{\CheckmarkBold} & \textcolor{green}{\CheckmarkBold} &\textcolor{green}{\CheckmarkBold}&\textcolor{red}{\XSolidBrush} &\textcolor{green}{\CheckmarkBold}&\textcolor{red}{\XSolidBrush}&\textcolor{green}{\CheckmarkBold}&\textcolor{green}{\CheckmarkBold}\\
xLAM~\citep{liu2024apigen} & APIGen & IFT & - & 60.0K & 3,673 & \textcolor{green}{\CheckmarkBold} & \textcolor{green}{\CheckmarkBold} &\textcolor{green}{\CheckmarkBold}&\textcolor{red}{\XSolidBrush} &\textcolor{green}{\CheckmarkBold}&\textcolor{red}{\XSolidBrush}&\textcolor{green}{\CheckmarkBold}&\textcolor{green}{\CheckmarkBold}\\\midrule
\multicolumn{13}{l}{\emph{Pretraining-based LLM Agents}}  \\\midrule
% LEMUR~\citep{xu2024lemur} & PT & 90B & 300.0K & - & \textcolor{green}{\CheckmarkBold} & \textcolor{green}{\CheckmarkBold} &\textcolor{green}{\CheckmarkBold}&\textcolor{red}{\XSolidBrush} & \textcolor{red}{\XSolidBrush} &\textcolor{green}{\CheckmarkBold} &\textcolor{red}{\XSolidBrush}&\textcolor{red}{\XSolidBrush}\\
\rowcolor{teal!12} \method & \dataset & PT & 103B & 95.0K  & 76,537  & \textcolor{green}{\CheckmarkBold} & \textcolor{green}{\CheckmarkBold} & \textcolor{green}{\CheckmarkBold} & \textcolor{green}{\CheckmarkBold} & \textcolor{green}{\CheckmarkBold} & \textcolor{green}{\CheckmarkBold} & \textcolor{green}{\CheckmarkBold} & \textcolor{green}{\CheckmarkBold}\\
\bottomrule
\end{tabular}
% \begin{tablenotes}
%     \item $^*$ In addition, the StarCoder-API can offer 4.77M more APIs.
% \end{tablenotes}
\caption{Summary of existing instruction finetuning-based LLM agents for intrinsic reasoning and function calling, along with their training resources and sample sizes. "PT" and "IFT" denote "Pre-Training" and "Instruction Fine-Tuning", respectively.}
\vspace{-2ex}
\label{tab:related}
\end{threeparttable}
\end{table*}

\noindent \textbf{Prompting-based LLM Agents.} Due to the lack of agent-specific pre-training corpus, existing LLM agents rely on either prompt engineering~\cite{hsieh2023tool,lu2024chameleon,yao2022react,wang2023voyager} or instruction fine-tuning~\cite{chen2023fireact,zeng2023agenttuning} to understand human instructions, decompose high-level tasks, generate grounded plans, and execute multi-step actions. 
However, prompting-based methods mainly depend on the capabilities of backbone LLMs (usually commercial LLMs), failing to introduce new knowledge and struggling to generalize to unseen tasks~\cite{sun2024adaplanner,zhuang2023toolchain}. 

\noindent \textbf{Instruction Finetuning-based LLM Agents.} Considering the extensive diversity of APIs and the complexity of multi-tool instructions, tool learning inherently presents greater challenges than natural language tasks, such as text generation~\cite{qin2023toolllm}.
Post-training techniques focus more on instruction following and aligning output with specific formats~\cite{patil2023gorilla,hao2024toolkengpt,qin2023toolllm,schick2024toolformer}, rather than fundamentally improving model knowledge or capabilities. 
Moreover, heavy fine-tuning can hinder generalization or even degrade performance in non-agent use cases, potentially suppressing the original base model capabilities~\cite{ghosh2024a}.

\noindent \textbf{Pretraining-based LLM Agents.} While pre-training serves as an essential alternative, prior works~\cite{nijkamp2023codegen,roziere2023code,xu2024lemur,patil2023gorilla} have primarily focused on improving task-specific capabilities (\eg, code generation) instead of general-domain LLM agents, due to single-source, uni-type, small-scale, and poor-quality pre-training data. 
Existing tool documentation data for agent training either lacks diverse real-world APIs~\cite{patil2023gorilla, tang2023toolalpaca} or is constrained to single-tool or single-round tool execution. 
Furthermore, trajectory data mostly imitate expert behavior or follow function-calling rules with inferior planning and reasoning, failing to fully elicit LLMs' capabilities and handle complex instructions~\cite{qin2023toolllm}. 
Given a wide range of candidate API functions, each comprising various function names and parameters available at every planning step, identifying globally optimal solutions and generalizing across tasks remains highly challenging.



\section{Preliminaries}
\label{Preliminaries}
\begin{figure*}[t]
    \centering
    \includegraphics[width=0.95\linewidth]{fig/HealthGPT_Framework.png}
    \caption{The \ourmethod{} architecture integrates hierarchical visual perception and H-LoRA, employing a task-specific hard router to select visual features and H-LoRA plugins, ultimately generating outputs with an autoregressive manner.}
    \label{fig:architecture}
\end{figure*}
\noindent\textbf{Large Vision-Language Models.} 
The input to a LVLM typically consists of an image $x^{\text{img}}$ and a discrete text sequence $x^{\text{txt}}$. The visual encoder $\mathcal{E}^{\text{img}}$ converts the input image $x^{\text{img}}$ into a sequence of visual tokens $\mathcal{V} = [v_i]_{i=1}^{N_v}$, while the text sequence $x^{\text{txt}}$ is mapped into a sequence of text tokens $\mathcal{T} = [t_i]_{i=1}^{N_t}$ using an embedding function $\mathcal{E}^{\text{txt}}$. The LLM $\mathcal{M_\text{LLM}}(\cdot|\theta)$ models the joint probability of the token sequence $\mathcal{U} = \{\mathcal{V},\mathcal{T}\}$, which is expressed as:
\begin{equation}
    P_\theta(R | \mathcal{U}) = \prod_{i=1}^{N_r} P_\theta(r_i | \{\mathcal{U}, r_{<i}\}),
\end{equation}
where $R = [r_i]_{i=1}^{N_r}$ is the text response sequence. The LVLM iteratively generates the next token $r_i$ based on $r_{<i}$. The optimization objective is to minimize the cross-entropy loss of the response $\mathcal{R}$.
% \begin{equation}
%     \mathcal{L}_{\text{VLM}} = \mathbb{E}_{R|\mathcal{U}}\left[-\log P_\theta(R | \mathcal{U})\right]
% \end{equation}
It is worth noting that most LVLMs adopt a design paradigm based on ViT, alignment adapters, and pre-trained LLMs\cite{liu2023llava,liu2024improved}, enabling quick adaptation to downstream tasks.


\noindent\textbf{VQGAN.}
VQGAN~\cite{esser2021taming} employs latent space compression and indexing mechanisms to effectively learn a complete discrete representation of images. VQGAN first maps the input image $x^{\text{img}}$ to a latent representation $z = \mathcal{E}(x)$ through a encoder $\mathcal{E}$. Then, the latent representation is quantized using a codebook $\mathcal{Z} = \{z_k\}_{k=1}^K$, generating a discrete index sequence $\mathcal{I} = [i_m]_{m=1}^N$, where $i_m \in \mathcal{Z}$ represents the quantized code index:
\begin{equation}
    \mathcal{I} = \text{Quantize}(z|\mathcal{Z}) = \arg\min_{z_k \in \mathcal{Z}} \| z - z_k \|_2.
\end{equation}
In our approach, the discrete index sequence $\mathcal{I}$ serves as a supervisory signal for the generation task, enabling the model to predict the index sequence $\hat{\mathcal{I}}$ from input conditions such as text or other modality signals.  
Finally, the predicted index sequence $\hat{\mathcal{I}}$ is upsampled by the VQGAN decoder $G$, generating the high-quality image $\hat{x}^\text{img} = G(\hat{\mathcal{I}})$.



\noindent\textbf{Low Rank Adaptation.} 
LoRA\cite{hu2021lora} effectively captures the characteristics of downstream tasks by introducing low-rank adapters. The core idea is to decompose the bypass weight matrix $\Delta W\in\mathbb{R}^{d^{\text{in}} \times d^{\text{out}}}$ into two low-rank matrices $ \{A \in \mathbb{R}^{d^{\text{in}} \times r}, B \in \mathbb{R}^{r \times d^{\text{out}}} \}$, where $ r \ll \min\{d^{\text{in}}, d^{\text{out}}\} $, significantly reducing learnable parameters. The output with the LoRA adapter for the input $x$ is then given by:
\begin{equation}
    h = x W_0 + \alpha x \Delta W/r = x W_0 + \alpha xAB/r,
\end{equation}
where matrix $ A $ is initialized with a Gaussian distribution, while the matrix $ B $ is initialized as a zero matrix. The scaling factor $ \alpha/r $ controls the impact of $ \Delta W $ on the model.

\section{HealthGPT}
\label{Method}


\subsection{Unified Autoregressive Generation.}  
% As shown in Figure~\ref{fig:architecture}, 
\ourmethod{} (Figure~\ref{fig:architecture}) utilizes a discrete token representation that covers both text and visual outputs, unifying visual comprehension and generation as an autoregressive task. 
For comprehension, $\mathcal{M}_\text{llm}$ receives the input joint sequence $\mathcal{U}$ and outputs a series of text token $\mathcal{R} = [r_1, r_2, \dots, r_{N_r}]$, where $r_i \in \mathcal{V}_{\text{txt}}$, and $\mathcal{V}_{\text{txt}}$ represents the LLM's vocabulary:
\begin{equation}
    P_\theta(\mathcal{R} \mid \mathcal{U}) = \prod_{i=1}^{N_r} P_\theta(r_i \mid \mathcal{U}, r_{<i}).
\end{equation}
For generation, $\mathcal{M}_\text{llm}$ first receives a special start token $\langle \text{START\_IMG} \rangle$, then generates a series of tokens corresponding to the VQGAN indices $\mathcal{I} = [i_1, i_2, \dots, i_{N_i}]$, where $i_j \in \mathcal{V}_{\text{vq}}$, and $\mathcal{V}_{\text{vq}}$ represents the index range of VQGAN. Upon completion of generation, the LLM outputs an end token $\langle \text{END\_IMG} \rangle$:
\begin{equation}
    P_\theta(\mathcal{I} \mid \mathcal{U}) = \prod_{j=1}^{N_i} P_\theta(i_j \mid \mathcal{U}, i_{<j}).
\end{equation}
Finally, the generated index sequence $\mathcal{I}$ is fed into the decoder $G$, which reconstructs the target image $\hat{x}^{\text{img}} = G(\mathcal{I})$.

\subsection{Hierarchical Visual Perception}  
Given the differences in visual perception between comprehension and generation tasks—where the former focuses on abstract semantics and the latter emphasizes complete semantics—we employ ViT to compress the image into discrete visual tokens at multiple hierarchical levels.
Specifically, the image is converted into a series of features $\{f_1, f_2, \dots, f_L\}$ as it passes through $L$ ViT blocks.

To address the needs of various tasks, the hidden states are divided into two types: (i) \textit{Concrete-grained features} $\mathcal{F}^{\text{Con}} = \{f_1, f_2, \dots, f_k\}, k < L$, derived from the shallower layers of ViT, containing sufficient global features, suitable for generation tasks; 
(ii) \textit{Abstract-grained features} $\mathcal{F}^{\text{Abs}} = \{f_{k+1}, f_{k+2}, \dots, f_L\}$, derived from the deeper layers of ViT, which contain abstract semantic information closer to the text space, suitable for comprehension tasks.

The task type $T$ (comprehension or generation) determines which set of features is selected as the input for the downstream large language model:
\begin{equation}
    \mathcal{F}^{\text{img}}_T =
    \begin{cases}
        \mathcal{F}^{\text{Con}}, & \text{if } T = \text{generation task} \\
        \mathcal{F}^{\text{Abs}}, & \text{if } T = \text{comprehension task}
    \end{cases}
\end{equation}
We integrate the image features $\mathcal{F}^{\text{img}}_T$ and text features $\mathcal{T}$ into a joint sequence through simple concatenation, which is then fed into the LLM $\mathcal{M}_{\text{llm}}$ for autoregressive generation.
% :
% \begin{equation}
%     \mathcal{R} = \mathcal{M}_{\text{llm}}(\mathcal{U}|\theta), \quad \mathcal{U} = [\mathcal{F}^{\text{img}}_T; \mathcal{T}]
% \end{equation}
\subsection{Heterogeneous Knowledge Adaptation}
We devise H-LoRA, which stores heterogeneous knowledge from comprehension and generation tasks in separate modules and dynamically routes to extract task-relevant knowledge from these modules. 
At the task level, for each task type $ T $, we dynamically assign a dedicated H-LoRA submodule $ \theta^T $, which is expressed as:
\begin{equation}
    \mathcal{R} = \mathcal{M}_\text{LLM}(\mathcal{U}|\theta, \theta^T), \quad \theta^T = \{A^T, B^T, \mathcal{R}^T_\text{outer}\}.
\end{equation}
At the feature level for a single task, H-LoRA integrates the idea of Mixture of Experts (MoE)~\cite{masoudnia2014mixture} and designs an efficient matrix merging and routing weight allocation mechanism, thus avoiding the significant computational delay introduced by matrix splitting in existing MoELoRA~\cite{luo2024moelora}. Specifically, we first merge the low-rank matrices (rank = r) of $ k $ LoRA experts into a unified matrix:
\begin{equation}
    \mathbf{A}^{\text{merged}}, \mathbf{B}^{\text{merged}} = \text{Concat}(\{A_i\}_1^k), \text{Concat}(\{B_i\}_1^k),
\end{equation}
where $ \mathbf{A}^{\text{merged}} \in \mathbb{R}^{d^\text{in} \times rk} $ and $ \mathbf{B}^{\text{merged}} \in \mathbb{R}^{rk \times d^\text{out}} $. The $k$-dimension routing layer generates expert weights $ \mathcal{W} \in \mathbb{R}^{\text{token\_num} \times k} $ based on the input hidden state $ x $, and these are expanded to $ \mathbb{R}^{\text{token\_num} \times rk} $ as follows:
\begin{equation}
    \mathcal{W}^\text{expanded} = \alpha k \mathcal{W} / r \otimes \mathbf{1}_r,
\end{equation}
where $ \otimes $ denotes the replication operation.
The overall output of H-LoRA is computed as:
\begin{equation}
    \mathcal{O}^\text{H-LoRA} = (x \mathbf{A}^{\text{merged}} \odot \mathcal{W}^\text{expanded}) \mathbf{B}^{\text{merged}},
\end{equation}
where $ \odot $ represents element-wise multiplication. Finally, the output of H-LoRA is added to the frozen pre-trained weights to produce the final output:
\begin{equation}
    \mathcal{O} = x W_0 + \mathcal{O}^\text{H-LoRA}.
\end{equation}
% In summary, H-LoRA is a task-based dynamic PEFT method that achieves high efficiency in single-task fine-tuning.

\subsection{Training Pipeline}

\begin{figure}[t]
    \centering
    \hspace{-4mm}
    \includegraphics[width=0.94\linewidth]{fig/data.pdf}
    \caption{Data statistics of \texttt{VL-Health}. }
    \label{fig:data}
\end{figure}
\noindent \textbf{1st Stage: Multi-modal Alignment.} 
In the first stage, we design separate visual adapters and H-LoRA submodules for medical unified tasks. For the medical comprehension task, we train abstract-grained visual adapters using high-quality image-text pairs to align visual embeddings with textual embeddings, thereby enabling the model to accurately describe medical visual content. During this process, the pre-trained LLM and its corresponding H-LoRA submodules remain frozen. In contrast, the medical generation task requires training concrete-grained adapters and H-LoRA submodules while keeping the LLM frozen. Meanwhile, we extend the textual vocabulary to include multimodal tokens, enabling the support of additional VQGAN vector quantization indices. The model trains on image-VQ pairs, endowing the pre-trained LLM with the capability for image reconstruction. This design ensures pixel-level consistency of pre- and post-LVLM. The processes establish the initial alignment between the LLM’s outputs and the visual inputs.

\noindent \textbf{2nd Stage: Heterogeneous H-LoRA Plugin Adaptation.}  
The submodules of H-LoRA share the word embedding layer and output head but may encounter issues such as bias and scale inconsistencies during training across different tasks. To ensure that the multiple H-LoRA plugins seamlessly interface with the LLMs and form a unified base, we fine-tune the word embedding layer and output head using a small amount of mixed data to maintain consistency in the model weights. Specifically, during this stage, all H-LoRA submodules for different tasks are kept frozen, with only the word embedding layer and output head being optimized. Through this stage, the model accumulates foundational knowledge for unified tasks by adapting H-LoRA plugins.

\begin{table*}[!t]
\centering
\caption{Comparison of \ourmethod{} with other LVLMs and unified multi-modal models on medical visual comprehension tasks. \textbf{Bold} and \underline{underlined} text indicates the best performance and second-best performance, respectively.}
\resizebox{\textwidth}{!}{
\begin{tabular}{c|lcc|cccccccc|c}
\toprule
\rowcolor[HTML]{E9F3FE} &  &  &  & \multicolumn{2}{c}{\textbf{VQA-RAD \textuparrow}} & \multicolumn{2}{c}{\textbf{SLAKE \textuparrow}} & \multicolumn{2}{c}{\textbf{PathVQA \textuparrow}} &  &  &  \\ 
\cline{5-10}
\rowcolor[HTML]{E9F3FE}\multirow{-2}{*}{\textbf{Type}} & \multirow{-2}{*}{\textbf{Model}} & \multirow{-2}{*}{\textbf{\# Params}} & \multirow{-2}{*}{\makecell{\textbf{Medical} \\ \textbf{LVLM}}} & \textbf{close} & \textbf{all} & \textbf{close} & \textbf{all} & \textbf{close} & \textbf{all} & \multirow{-2}{*}{\makecell{\textbf{MMMU} \\ \textbf{-Med}}\textuparrow} & \multirow{-2}{*}{\textbf{OMVQA}\textuparrow} & \multirow{-2}{*}{\textbf{Avg. \textuparrow}} \\ 
\midrule \midrule
\multirow{9}{*}{\textbf{Comp. Only}} 
& Med-Flamingo & 8.3B & \Large \ding{51} & 58.6 & 43.0 & 47.0 & 25.5 & 61.9 & 31.3 & 28.7 & 34.9 & 41.4 \\
& LLaVA-Med & 7B & \Large \ding{51} & 60.2 & 48.1 & 58.4 & 44.8 & 62.3 & 35.7 & 30.0 & 41.3 & 47.6 \\
& HuatuoGPT-Vision & 7B & \Large \ding{51} & 66.9 & 53.0 & 59.8 & 49.1 & 52.9 & 32.0 & 42.0 & 50.0 & 50.7 \\
& BLIP-2 & 6.7B & \Large \ding{55} & 43.4 & 36.8 & 41.6 & 35.3 & 48.5 & 28.8 & 27.3 & 26.9 & 36.1 \\
& LLaVA-v1.5 & 7B & \Large \ding{55} & 51.8 & 42.8 & 37.1 & 37.7 & 53.5 & 31.4 & 32.7 & 44.7 & 41.5 \\
& InstructBLIP & 7B & \Large \ding{55} & 61.0 & 44.8 & 66.8 & 43.3 & 56.0 & 32.3 & 25.3 & 29.0 & 44.8 \\
& Yi-VL & 6B & \Large \ding{55} & 52.6 & 42.1 & 52.4 & 38.4 & 54.9 & 30.9 & 38.0 & 50.2 & 44.9 \\
& InternVL2 & 8B & \Large \ding{55} & 64.9 & 49.0 & 66.6 & 50.1 & 60.0 & 31.9 & \underline{43.3} & 54.5 & 52.5\\
& Llama-3.2 & 11B & \Large \ding{55} & 68.9 & 45.5 & 72.4 & 52.1 & 62.8 & 33.6 & 39.3 & 63.2 & 54.7 \\
\midrule
\multirow{5}{*}{\textbf{Comp. \& Gen.}} 
& Show-o & 1.3B & \Large \ding{55} & 50.6 & 33.9 & 31.5 & 17.9 & 52.9 & 28.2 & 22.7 & 45.7 & 42.6 \\
& Unified-IO 2 & 7B & \Large \ding{55} & 46.2 & 32.6 & 35.9 & 21.9 & 52.5 & 27.0 & 25.3 & 33.0 & 33.8 \\
& Janus & 1.3B & \Large \ding{55} & 70.9 & 52.8 & 34.7 & 26.9 & 51.9 & 27.9 & 30.0 & 26.8 & 33.5 \\
& \cellcolor[HTML]{DAE0FB}HealthGPT-M3 & \cellcolor[HTML]{DAE0FB}3.8B & \cellcolor[HTML]{DAE0FB}\Large \ding{51} & \cellcolor[HTML]{DAE0FB}\underline{73.7} & \cellcolor[HTML]{DAE0FB}\underline{55.9} & \cellcolor[HTML]{DAE0FB}\underline{74.6} & \cellcolor[HTML]{DAE0FB}\underline{56.4} & \cellcolor[HTML]{DAE0FB}\underline{78.7} & \cellcolor[HTML]{DAE0FB}\underline{39.7} & \cellcolor[HTML]{DAE0FB}\underline{43.3} & \cellcolor[HTML]{DAE0FB}\underline{68.5} & \cellcolor[HTML]{DAE0FB}\underline{61.3} \\
& \cellcolor[HTML]{DAE0FB}HealthGPT-L14 & \cellcolor[HTML]{DAE0FB}14B & \cellcolor[HTML]{DAE0FB}\Large \ding{51} & \cellcolor[HTML]{DAE0FB}\textbf{77.7} & \cellcolor[HTML]{DAE0FB}\textbf{58.3} & \cellcolor[HTML]{DAE0FB}\textbf{76.4} & \cellcolor[HTML]{DAE0FB}\textbf{64.5} & \cellcolor[HTML]{DAE0FB}\textbf{85.9} & \cellcolor[HTML]{DAE0FB}\textbf{44.4} & \cellcolor[HTML]{DAE0FB}\textbf{49.2} & \cellcolor[HTML]{DAE0FB}\textbf{74.4} & \cellcolor[HTML]{DAE0FB}\textbf{66.4} \\
\bottomrule
\end{tabular}
}
\label{tab:results}
\end{table*}
\begin{table*}[ht]
    \centering
    \caption{The experimental results for the four modality conversion tasks.}
    \resizebox{\textwidth}{!}{
    \begin{tabular}{l|ccc|ccc|ccc|ccc}
        \toprule
        \rowcolor[HTML]{E9F3FE} & \multicolumn{3}{c}{\textbf{CT to MRI (Brain)}} & \multicolumn{3}{c}{\textbf{CT to MRI (Pelvis)}} & \multicolumn{3}{c}{\textbf{MRI to CT (Brain)}} & \multicolumn{3}{c}{\textbf{MRI to CT (Pelvis)}} \\
        \cline{2-13}
        \rowcolor[HTML]{E9F3FE}\multirow{-2}{*}{\textbf{Model}}& \textbf{SSIM $\uparrow$} & \textbf{PSNR $\uparrow$} & \textbf{MSE $\downarrow$} & \textbf{SSIM $\uparrow$} & \textbf{PSNR $\uparrow$} & \textbf{MSE $\downarrow$} & \textbf{SSIM $\uparrow$} & \textbf{PSNR $\uparrow$} & \textbf{MSE $\downarrow$} & \textbf{SSIM $\uparrow$} & \textbf{PSNR $\uparrow$} & \textbf{MSE $\downarrow$} \\
        \midrule \midrule
        pix2pix & 71.09 & 32.65 & 36.85 & 59.17 & 31.02 & 51.91 & 78.79 & 33.85 & 28.33 & 72.31 & 32.98 & 36.19 \\
        CycleGAN & 54.76 & 32.23 & 40.56 & 54.54 & 30.77 & 55.00 & 63.75 & 31.02 & 52.78 & 50.54 & 29.89 & 67.78 \\
        BBDM & {71.69} & {32.91} & {34.44} & 57.37 & 31.37 & 48.06 & \textbf{86.40} & 34.12 & 26.61 & {79.26} & 33.15 & 33.60 \\
        Vmanba & 69.54 & 32.67 & 36.42 & {63.01} & {31.47} & {46.99} & 79.63 & 34.12 & 26.49 & 77.45 & 33.53 & 31.85 \\
        DiffMa & 71.47 & 32.74 & 35.77 & 62.56 & 31.43 & 47.38 & 79.00 & {34.13} & {26.45} & 78.53 & {33.68} & {30.51} \\
        \rowcolor[HTML]{DAE0FB}HealthGPT-M3 & \underline{79.38} & \underline{33.03} & \underline{33.48} & \underline{71.81} & \underline{31.83} & \underline{43.45} & {85.06} & \textbf{34.40} & \textbf{25.49} & \underline{84.23} & \textbf{34.29} & \textbf{27.99} \\
        \rowcolor[HTML]{DAE0FB}HealthGPT-L14 & \textbf{79.73} & \textbf{33.10} & \textbf{32.96} & \textbf{71.92} & \textbf{31.87} & \textbf{43.09} & \underline{85.31} & \underline{34.29} & \underline{26.20} & \textbf{84.96} & \underline{34.14} & \underline{28.13} \\
        \bottomrule
    \end{tabular}
    }
    \label{tab:conversion}
\end{table*}

\noindent \textbf{3rd Stage: Visual Instruction Fine-Tuning.}  
In the third stage, we introduce additional task-specific data to further optimize the model and enhance its adaptability to downstream tasks such as medical visual comprehension (e.g., medical QA, medical dialogues, and report generation) or generation tasks (e.g., super-resolution, denoising, and modality conversion). Notably, by this stage, the word embedding layer and output head have been fine-tuned, only the H-LoRA modules and adapter modules need to be trained. This strategy significantly improves the model's adaptability and flexibility across different tasks.


\section{Experiment}
\label{s:experiment}

\subsection{Data Description}
We evaluate our method on FI~\cite{you2016building}, Twitter\_LDL~\cite{yang2017learning} and Artphoto~\cite{machajdik2010affective}.
FI is a public dataset built from Flickr and Instagram, with 23,308 images and eight emotion categories, namely \textit{amusement}, \textit{anger}, \textit{awe},  \textit{contentment}, \textit{disgust}, \textit{excitement},  \textit{fear}, and \textit{sadness}. 
% Since images in FI are all copyrighted by law, some images are corrupted now, so we remove these samples and retain 21,828 images.
% T4SA contains images from Twitter, which are classified into three categories: \textit{positive}, \textit{neutral}, and \textit{negative}. In this paper, we adopt the base version of B-T4SA, which contains 470,586 images and provides text descriptions of the corresponding tweets.
Twitter\_LDL contains 10,045 images from Twitter, with the same eight categories as the FI dataset.
% 。
For these two datasets, they are randomly split into 80\%
training and 20\% testing set.
Artphoto contains 806 artistic photos from the DeviantArt website, which we use to further evaluate the zero-shot capability of our model.
% on the small-scale dataset.
% We construct and publicly release the first image sentiment analysis dataset containing metadata.
% 。

% Based on these datasets, we are the first to construct and publicly release metadata-enhanced image sentiment analysis datasets. These datasets include scenes, tags, descriptions, and corresponding confidence scores, and are available at this link for future research purposes.


% 
\begin{table}[t]
\centering
% \begin{center}
\caption{Overall performance of different models on FI and Twitter\_LDL datasets.}
\label{tab:cap1}
% \resizebox{\linewidth}{!}
{
\begin{tabular}{l|c|c|c|c}
\hline
\multirow{2}{*}{\textbf{Model}} & \multicolumn{2}{c|}{\textbf{FI}}  & \multicolumn{2}{c}{\textbf{Twitter\_LDL}} \\ \cline{2-5} 
  & \textbf{Accuracy} & \textbf{F1} & \textbf{Accuracy} & \textbf{F1}  \\ \hline
% (\rownumber)~AlexNet~\cite{krizhevsky2017imagenet}  & 58.13\% & 56.35\%  & 56.24\%& 55.02\%  \\ 
% (\rownumber)~VGG16~\cite{simonyan2014very}  & 63.75\%& 63.08\%  & 59.34\%& 59.02\%  \\ 
(\rownumber)~ResNet101~\cite{he2016deep} & 66.16\%& 65.56\%  & 62.02\% & 61.34\%  \\ 
(\rownumber)~CDA~\cite{han2023boosting} & 66.71\%& 65.37\%  & 64.14\% & 62.85\%  \\ 
(\rownumber)~CECCN~\cite{ruan2024color} & 67.96\%& 66.74\%  & 64.59\%& 64.72\% \\ 
(\rownumber)~EmoVIT~\cite{xie2024emovit} & 68.09\%& 67.45\%  & 63.12\% & 61.97\%  \\ 
(\rownumber)~ComLDL~\cite{zhang2022compound} & 68.83\%& 67.28\%  & 65.29\% & 63.12\%  \\ 
(\rownumber)~WSDEN~\cite{li2023weakly} & 69.78\%& 69.61\%  & 67.04\% & 65.49\% \\ 
(\rownumber)~ECWA~\cite{deng2021emotion} & 70.87\%& 69.08\%  & 67.81\% & 66.87\%  \\ 
(\rownumber)~EECon~\cite{yang2023exploiting} & 71.13\%& 68.34\%  & 64.27\%& 63.16\%  \\ 
(\rownumber)~MAM~\cite{zhang2024affective} & 71.44\%  & 70.83\% & 67.18\%  & 65.01\%\\ 
(\rownumber)~TGCA-PVT~\cite{chen2024tgca}   & 73.05\%  & 71.46\% & 69.87\%  & 68.32\% \\ 
(\rownumber)~OEAN~\cite{zhang2024object}   & 73.40\%  & 72.63\% & 70.52\%  & 69.47\% \\ \hline
(\rownumber)~\shortname  & \textbf{79.48\%} & \textbf{79.22\%} & \textbf{74.12\%} & \textbf{73.09\%} \\ \hline
\end{tabular}
}
\vspace{-6mm}
% \end{center}
\end{table}
% 

\subsection{Experiment Setting}
% \subsubsection{Model Setting.}
% 
\textbf{Model Setting:}
For feature representation, we set $k=10$ to select object tags, and adopt clip-vit-base-patch32 as the pre-trained model for unified feature representation.
Moreover, we empirically set $(d_e, d_h, d_k, d_s) = (512, 128, 16, 64)$, and set the classification class $L$ to 8.

% 

\textbf{Training Setting:}
To initialize the model, we set all weights such as $\boldsymbol{W}$ following the truncated normal distribution, and use AdamW optimizer with the learning rate of $1 \times 10^{-4}$.
% warmup scheduler of cosine, warmup steps of 2000.
Furthermore, we set the batch size to 32 and the epoch of the training process to 200.
During the implementation, we utilize \textit{PyTorch} to build our entire model.
% , and our project codes are publicly available at https://github.com/zzmyrep/MESN.
% Our project codes as well as data are all publicly available on GitHub\footnote{https://github.com/zzmyrep/KBCEN}.
% Code is available at \href{https://github.com/zzmyrep/KBCEN}{https://github.com/zzmyrep/KBCEN}.

\textbf{Evaluation Metrics:}
Following~\cite{zhang2024affective, chen2024tgca, zhang2024object}, we adopt \textit{accuracy} and \textit{F1} as our evaluation metrics to measure the performance of different methods for image sentiment analysis. 



\subsection{Experiment Result}
% We compare our model against the following baselines: AlexNet~\cite{krizhevsky2017imagenet}, VGG16~\cite{simonyan2014very}, ResNet101~\cite{he2016deep}, CECCN~\cite{ruan2024color}, EmoVIT~\cite{xie2024emovit}, WSCNet~\cite{yang2018weakly}, ECWA~\cite{deng2021emotion}, EECon~\cite{yang2023exploiting}, MAM~\cite{zhang2024affective} and TGCA-PVT~\cite{chen2024tgca}, and the overall results are summarized in Table~\ref{tab:cap1}.
We compare our model against several baselines, and the overall results are summarized in Table~\ref{tab:cap1}.
We observe that our model achieves the best performance in both accuracy and F1 metrics, significantly outperforming the previous models. 
This superior performance is mainly attributed to our effective utilization of metadata to enhance image sentiment analysis, as well as the exceptional capability of the unified sentiment transformer framework we developed. These results strongly demonstrate that our proposed method can bring encouraging performance for image sentiment analysis.

\setcounter{magicrownumbers}{0} 
\begin{table}[t]
\begin{center}
\caption{Ablation study of~\shortname~on FI dataset.} 
% \vspace{1mm}
\label{tab:cap2}
\resizebox{.9\linewidth}{!}
{
\begin{tabular}{lcc}
  \hline
  \textbf{Model} & \textbf{Accuracy} & \textbf{F1} \\
  \hline
  (\rownumber)~Ours (w/o vision) & 65.72\% & 64.54\% \\
  (\rownumber)~Ours (w/o text description) & 74.05\% & 72.58\% \\
  (\rownumber)~Ours (w/o object tag) & 77.45\% & 76.84\% \\
  (\rownumber)~Ours (w/o scene tag) & 78.47\% & 78.21\% \\
  \hline
  (\rownumber)~Ours (w/o unified embedding) & 76.41\% & 76.23\% \\
  (\rownumber)~Ours (w/o adaptive learning) & 76.83\% & 76.56\% \\
  (\rownumber)~Ours (w/o cross-modal fusion) & 76.85\% & 76.49\% \\
  \hline
  (\rownumber)~Ours  & \textbf{79.48\%} & \textbf{79.22\%} \\
  \hline
\end{tabular}
}
\end{center}
\vspace{-5mm}
\end{table}


\begin{figure}[t]
\centering
% \vspace{-2mm}
\includegraphics[width=0.42\textwidth]{fig/2dvisual-linux4-paper2.pdf}
\caption{Visualization of feature distribution on eight categories before (left) and after (right) model processing.}
% 
\label{fig:visualization}
\vspace{-5mm}
\end{figure}

\subsection{Ablation Performance}
In this subsection, we conduct an ablation study to examine which component is really important for performance improvement. The results are reported in Table~\ref{tab:cap2}.

For information utilization, we observe a significant decline in model performance when visual features are removed. Additionally, the performance of \shortname~decreases when different metadata are removed separately, which means that text description, object tag, and scene tag are all critical for image sentiment analysis.
Recalling the model architecture, we separately remove transformer layers of the unified representation module, the adaptive learning module, and the cross-modal fusion module, replacing them with MLPs of the same parameter scale.
In this way, we can observe varying degrees of decline in model performance, indicating that these modules are indispensable for our model to achieve better performance.

\subsection{Visualization}
% 


% % 开始使用minipage进行左右排列
% \begin{minipage}[t]{0.45\textwidth}  % 子图1宽度为45%
%     \centering
%     \includegraphics[width=\textwidth]{2dvisual.pdf}  % 插入图片
%     \captionof{figure}{Visualization of feature distribution.}  % 使用captionof添加图片标题
%     \label{fig:visualization}
% \end{minipage}


% \begin{figure}[t]
% \centering
% \vspace{-2mm}
% \includegraphics[width=0.45\textwidth]{fig/2dvisual.pdf}
% \caption{Visualization of feature distribution.}
% \label{fig:visualization}
% % \vspace{-4mm}
% \end{figure}

% \begin{figure}[t]
% \centering
% \vspace{-2mm}
% \includegraphics[width=0.45\textwidth]{fig/2dvisual-linux3-paper.pdf}
% \caption{Visualization of feature distribution.}
% \label{fig:visualization}
% % \vspace{-4mm}
% \end{figure}



\begin{figure}[tbp]   
\vspace{-4mm}
  \centering            
  \subfloat[Depth of adaptive learning layers]   
  {
    \label{fig:subfig1}\includegraphics[width=0.22\textwidth]{fig/fig_sensitivity-a5}
  }
  \subfloat[Depth of fusion layers]
  {
    % \label{fig:subfig2}\includegraphics[width=0.22\textwidth]{fig/fig_sensitivity-b2}
    \label{fig:subfig2}\includegraphics[width=0.22\textwidth]{fig/fig_sensitivity-b2-num.pdf}
  }
  \caption{Sensitivity study of \shortname~on different depth. }   
  \label{fig:fig_sensitivity}  
\vspace{-2mm}
\end{figure}

% \begin{figure}[htbp]
% \centerline{\includegraphics{2dvisual.pdf}}
% \caption{Visualization of feature distribution.}
% \label{fig:visualization}
% \end{figure}

% In Fig.~\ref{fig:visualization}, we use t-SNE~\cite{van2008visualizing} to reduce the dimension of data features for visualization, Figure in left represents the metadata features before model processing, the features are obtained by embedding through the CLIP model, and figure in right shows the features of the data after model processing, it can be observed that after the model processing, the data with different label categories fall in different regions in the space, therefore, we can conclude that the Therefore, we can conclude that the model can effectively utilize the information contained in the metadata and use it to guide the model for classification.

In Fig.~\ref{fig:visualization}, we use t-SNE~\cite{van2008visualizing} to reduce the dimension of data features for visualization.
The left figure shows metadata features before being processed by our model (\textit{i.e.}, embedded by CLIP), while the right shows the distribution of features after being processed by our model.
We can observe that after the model processing, data with the same label are closer to each other, while others are farther away.
Therefore, it shows that the model can effectively utilize the information contained in the metadata and use it to guide the classification process.

\subsection{Sensitivity Analysis}
% 
In this subsection, we conduct a sensitivity analysis to figure out the effect of different depth settings of adaptive learning layers and fusion layers. 
% In this subsection, we conduct a sensitivity analysis to figure out the effect of different depth settings on the model. 
% Fig.~\ref{fig:fig_sensitivity} presents the effect of different depth settings of adaptive learning layers and fusion layers. 
Taking Fig.~\ref{fig:fig_sensitivity} (a) as an example, the model performance improves with increasing depth, reaching the best performance at a depth of 4.
% Taking Fig.~\ref{fig:fig_sensitivity} (a) as an example, the performance of \shortname~improves with the increase of depth at first, reaching the best performance at a depth of 4.
When the depth continues to increase, the accuracy decreases to varying degrees.
Similar results can be observed in Fig.~\ref{fig:fig_sensitivity} (b).
Therefore, we set their depths to 4 and 6 respectively to achieve the best results.

% Through our experiments, we can observe that the effect of modifying these hyperparameters on the results of the experiments is very weak, and the surface model is not sensitive to the hyperparameters.


\subsection{Zero-shot Capability}
% 

% (1)~GCH~\cite{2010Analyzing} & 21.78\% & (5)~RA-DLNet~\cite{2020A} & 34.01\% \\ \hline
% (2)~WSCNet~\cite{2019WSCNet}  & 30.25\% & (6)~CECCN~\cite{ruan2024color} & 43.83\% \\ \hline
% (3)~PCNN~\cite{2015Robust} & 31.68\%  & (7)~EmoVIT~\cite{xie2024emovit} & 44.90\% \\ \hline
% (4)~AR~\cite{2018Visual} & 32.67\% & (8)~Ours (Zero-shot) & 47.83\% \\ \hline


\begin{table}[t]
\centering
\caption{Zero-shot capability of \shortname.}
\label{tab:cap3}
\resizebox{1\linewidth}{!}
{
\begin{tabular}{lc|lc}
\hline
\textbf{Model} & \textbf{Accuracy} & \textbf{Model} & \textbf{Accuracy} \\ \hline
(1)~WSCNet~\cite{2019WSCNet}  & 30.25\% & (5)~MAM~\cite{zhang2024affective} & 39.56\%  \\ \hline
(2)~AR~\cite{2018Visual} & 32.67\% & (6)~CECCN~\cite{ruan2024color} & 43.83\% \\ \hline
(3)~RA-DLNet~\cite{2020A} & 34.01\%  & (7)~EmoVIT~\cite{xie2024emovit} & 44.90\% \\ \hline
(4)~CDA~\cite{han2023boosting} & 38.64\% & (8)~Ours (Zero-shot) & 47.83\% \\ \hline
\end{tabular}
}
\vspace{-5mm}
\end{table}

% We use the model trained on the FI dataset to test on the artphoto dataset to verify the model's generalization ability as well as robustness to other distributed datasets.
% We can observe that the MESN model shows strong competitiveness in terms of accuracy when compared to other trained models, which suggests that the model has a good generalization ability in the OOD task.

To validate the model's generalization ability and robustness to other distributed datasets, we directly test the model trained on the FI dataset, without training on Artphoto. 
% As observed in Table 3, compared to other models trained on Artphoto, we achieve highly competitive zero-shot performance, indicating that the model has good generalization ability in out-of-distribution tasks.
From Table~\ref{tab:cap3}, we can observe that compared with other models trained on Artphoto, we achieve competitive zero-shot performance, which shows that the model has good generalization ability in out-of-distribution tasks.


%%%%%%%%%%%%
%  E2E     %
%%%%%%%%%%%%


\section{Conclusion}
In this paper, we introduced Wi-Chat, the first LLM-powered Wi-Fi-based human activity recognition system that integrates the reasoning capabilities of large language models with the sensing potential of wireless signals. Our experimental results on a self-collected Wi-Fi CSI dataset demonstrate the promising potential of LLMs in enabling zero-shot Wi-Fi sensing. These findings suggest a new paradigm for human activity recognition that does not rely on extensive labeled data. We hope future research will build upon this direction, further exploring the applications of LLMs in signal processing domains such as IoT, mobile sensing, and radar-based systems.

\section*{Limitations}
While our work represents the first attempt to leverage LLMs for processing Wi-Fi signals, it is a preliminary study focused on a relatively simple task: Wi-Fi-based human activity recognition. This choice allows us to explore the feasibility of LLMs in wireless sensing but also comes with certain limitations.

Our approach primarily evaluates zero-shot performance, which, while promising, may still lag behind traditional supervised learning methods in highly complex or fine-grained recognition tasks. Besides, our study is limited to a controlled environment with a self-collected dataset, and the generalizability of LLMs to diverse real-world scenarios with varying Wi-Fi conditions, environmental interference, and device heterogeneity remains an open question.

Additionally, we have yet to explore the full potential of LLMs in more advanced Wi-Fi sensing applications, such as fine-grained gesture recognition, occupancy detection, and passive health monitoring. Future work should investigate the scalability of LLM-based approaches, their robustness to domain shifts, and their integration with multimodal sensing techniques in broader IoT applications.


% Bibliography entries for the entire Anthology, followed by custom entries
%\bibliography{anthology,custom}
% Custom bibliography entries only
\bibliography{main}
\newpage
\appendix

\section{Experiment prompts}
\label{sec:prompt}
The prompts used in the LLM experiments are shown in the following Table~\ref{tab:prompts}.

\definecolor{titlecolor}{rgb}{0.9, 0.5, 0.1}
\definecolor{anscolor}{rgb}{0.2, 0.5, 0.8}
\definecolor{labelcolor}{HTML}{48a07e}
\begin{table*}[h]
	\centering
	
 % \vspace{-0.2cm}
	
	\begin{center}
		\begin{tikzpicture}[
				chatbox_inner/.style={rectangle, rounded corners, opacity=0, text opacity=1, font=\sffamily\scriptsize, text width=5in, text height=9pt, inner xsep=6pt, inner ysep=6pt},
				chatbox_prompt_inner/.style={chatbox_inner, align=flush left, xshift=0pt, text height=11pt},
				chatbox_user_inner/.style={chatbox_inner, align=flush left, xshift=0pt},
				chatbox_gpt_inner/.style={chatbox_inner, align=flush left, xshift=0pt},
				chatbox/.style={chatbox_inner, draw=black!25, fill=gray!7, opacity=1, text opacity=0},
				chatbox_prompt/.style={chatbox, align=flush left, fill=gray!1.5, draw=black!30, text height=10pt},
				chatbox_user/.style={chatbox, align=flush left},
				chatbox_gpt/.style={chatbox, align=flush left},
				chatbox2/.style={chatbox_gpt, fill=green!25},
				chatbox3/.style={chatbox_gpt, fill=red!20, draw=black!20},
				chatbox4/.style={chatbox_gpt, fill=yellow!30},
				labelbox/.style={rectangle, rounded corners, draw=black!50, font=\sffamily\scriptsize\bfseries, fill=gray!5, inner sep=3pt},
			]
											
			\node[chatbox_user] (q1) {
				\textbf{System prompt}
				\newline
				\newline
				You are a helpful and precise assistant for segmenting and labeling sentences. We would like to request your help on curating a dataset for entity-level hallucination detection.
				\newline \newline
                We will give you a machine generated biography and a list of checked facts about the biography. Each fact consists of a sentence and a label (True/False). Please do the following process. First, breaking down the biography into words. Second, by referring to the provided list of facts, merging some broken down words in the previous step to form meaningful entities. For example, ``strategic thinking'' should be one entity instead of two. Third, according to the labels in the list of facts, labeling each entity as True or False. Specifically, for facts that share a similar sentence structure (\eg, \textit{``He was born on Mach 9, 1941.''} (\texttt{True}) and \textit{``He was born in Ramos Mejia.''} (\texttt{False})), please first assign labels to entities that differ across atomic facts. For example, first labeling ``Mach 9, 1941'' (\texttt{True}) and ``Ramos Mejia'' (\texttt{False}) in the above case. For those entities that are the same across atomic facts (\eg, ``was born'') or are neutral (\eg, ``he,'' ``in,'' and ``on''), please label them as \texttt{True}. For the cases that there is no atomic fact that shares the same sentence structure, please identify the most informative entities in the sentence and label them with the same label as the atomic fact while treating the rest of the entities as \texttt{True}. In the end, output the entities and labels in the following format:
                \begin{itemize}[nosep]
                    \item Entity 1 (Label 1)
                    \item Entity 2 (Label 2)
                    \item ...
                    \item Entity N (Label N)
                \end{itemize}
                % \newline \newline
                Here are two examples:
                \newline\newline
                \textbf{[Example 1]}
                \newline
                [The start of the biography]
                \newline
                \textcolor{titlecolor}{Marianne McAndrew is an American actress and singer, born on November 21, 1942, in Cleveland, Ohio. She began her acting career in the late 1960s, appearing in various television shows and films.}
                \newline
                [The end of the biography]
                \newline \newline
                [The start of the list of checked facts]
                \newline
                \textcolor{anscolor}{[Marianne McAndrew is an American. (False); Marianne McAndrew is an actress. (True); Marianne McAndrew is a singer. (False); Marianne McAndrew was born on November 21, 1942. (False); Marianne McAndrew was born in Cleveland, Ohio. (False); She began her acting career in the late 1960s. (True); She has appeared in various television shows. (True); She has appeared in various films. (True)]}
                \newline
                [The end of the list of checked facts]
                \newline \newline
                [The start of the ideal output]
                \newline
                \textcolor{labelcolor}{[Marianne McAndrew (True); is (True); an (True); American (False); actress (True); and (True); singer (False); , (True); born (True); on (True); November 21, 1942 (False); , (True); in (True); Cleveland, Ohio (False); . (True); She (True); began (True); her (True); acting career (True); in (True); the late 1960s (True); , (True); appearing (True); in (True); various (True); television shows (True); and (True); films (True); . (True)]}
                \newline
                [The end of the ideal output]
				\newline \newline
                \textbf{[Example 2]}
                \newline
                [The start of the biography]
                \newline
                \textcolor{titlecolor}{Doug Sheehan is an American actor who was born on April 27, 1949, in Santa Monica, California. He is best known for his roles in soap operas, including his portrayal of Joe Kelly on ``General Hospital'' and Ben Gibson on ``Knots Landing.''}
                \newline
                [The end of the biography]
                \newline \newline
                [The start of the list of checked facts]
                \newline
                \textcolor{anscolor}{[Doug Sheehan is an American. (True); Doug Sheehan is an actor. (True); Doug Sheehan was born on April 27, 1949. (True); Doug Sheehan was born in Santa Monica, California. (False); He is best known for his roles in soap operas. (True); He portrayed Joe Kelly. (True); Joe Kelly was in General Hospital. (True); General Hospital is a soap opera. (True); He portrayed Ben Gibson. (True); Ben Gibson was in Knots Landing. (True); Knots Landing is a soap opera. (True)]}
                \newline
                [The end of the list of checked facts]
                \newline \newline
                [The start of the ideal output]
                \newline
                \textcolor{labelcolor}{[Doug Sheehan (True); is (True); an (True); American (True); actor (True); who (True); was born (True); on (True); April 27, 1949 (True); in (True); Santa Monica, California (False); . (True); He (True); is (True); best known (True); for (True); his roles in soap operas (True); , (True); including (True); in (True); his portrayal (True); of (True); Joe Kelly (True); on (True); ``General Hospital'' (True); and (True); Ben Gibson (True); on (True); ``Knots Landing.'' (True)]}
                \newline
                [The end of the ideal output]
				\newline \newline
				\textbf{User prompt}
				\newline
				\newline
				[The start of the biography]
				\newline
				\textcolor{magenta}{\texttt{\{BIOGRAPHY\}}}
				\newline
				[The ebd of the biography]
				\newline \newline
				[The start of the list of checked facts]
				\newline
				\textcolor{magenta}{\texttt{\{LIST OF CHECKED FACTS\}}}
				\newline
				[The end of the list of checked facts]
			};
			\node[chatbox_user_inner] (q1_text) at (q1) {
				\textbf{System prompt}
				\newline
				\newline
				You are a helpful and precise assistant for segmenting and labeling sentences. We would like to request your help on curating a dataset for entity-level hallucination detection.
				\newline \newline
                We will give you a machine generated biography and a list of checked facts about the biography. Each fact consists of a sentence and a label (True/False). Please do the following process. First, breaking down the biography into words. Second, by referring to the provided list of facts, merging some broken down words in the previous step to form meaningful entities. For example, ``strategic thinking'' should be one entity instead of two. Third, according to the labels in the list of facts, labeling each entity as True or False. Specifically, for facts that share a similar sentence structure (\eg, \textit{``He was born on Mach 9, 1941.''} (\texttt{True}) and \textit{``He was born in Ramos Mejia.''} (\texttt{False})), please first assign labels to entities that differ across atomic facts. For example, first labeling ``Mach 9, 1941'' (\texttt{True}) and ``Ramos Mejia'' (\texttt{False}) in the above case. For those entities that are the same across atomic facts (\eg, ``was born'') or are neutral (\eg, ``he,'' ``in,'' and ``on''), please label them as \texttt{True}. For the cases that there is no atomic fact that shares the same sentence structure, please identify the most informative entities in the sentence and label them with the same label as the atomic fact while treating the rest of the entities as \texttt{True}. In the end, output the entities and labels in the following format:
                \begin{itemize}[nosep]
                    \item Entity 1 (Label 1)
                    \item Entity 2 (Label 2)
                    \item ...
                    \item Entity N (Label N)
                \end{itemize}
                % \newline \newline
                Here are two examples:
                \newline\newline
                \textbf{[Example 1]}
                \newline
                [The start of the biography]
                \newline
                \textcolor{titlecolor}{Marianne McAndrew is an American actress and singer, born on November 21, 1942, in Cleveland, Ohio. She began her acting career in the late 1960s, appearing in various television shows and films.}
                \newline
                [The end of the biography]
                \newline \newline
                [The start of the list of checked facts]
                \newline
                \textcolor{anscolor}{[Marianne McAndrew is an American. (False); Marianne McAndrew is an actress. (True); Marianne McAndrew is a singer. (False); Marianne McAndrew was born on November 21, 1942. (False); Marianne McAndrew was born in Cleveland, Ohio. (False); She began her acting career in the late 1960s. (True); She has appeared in various television shows. (True); She has appeared in various films. (True)]}
                \newline
                [The end of the list of checked facts]
                \newline \newline
                [The start of the ideal output]
                \newline
                \textcolor{labelcolor}{[Marianne McAndrew (True); is (True); an (True); American (False); actress (True); and (True); singer (False); , (True); born (True); on (True); November 21, 1942 (False); , (True); in (True); Cleveland, Ohio (False); . (True); She (True); began (True); her (True); acting career (True); in (True); the late 1960s (True); , (True); appearing (True); in (True); various (True); television shows (True); and (True); films (True); . (True)]}
                \newline
                [The end of the ideal output]
				\newline \newline
                \textbf{[Example 2]}
                \newline
                [The start of the biography]
                \newline
                \textcolor{titlecolor}{Doug Sheehan is an American actor who was born on April 27, 1949, in Santa Monica, California. He is best known for his roles in soap operas, including his portrayal of Joe Kelly on ``General Hospital'' and Ben Gibson on ``Knots Landing.''}
                \newline
                [The end of the biography]
                \newline \newline
                [The start of the list of checked facts]
                \newline
                \textcolor{anscolor}{[Doug Sheehan is an American. (True); Doug Sheehan is an actor. (True); Doug Sheehan was born on April 27, 1949. (True); Doug Sheehan was born in Santa Monica, California. (False); He is best known for his roles in soap operas. (True); He portrayed Joe Kelly. (True); Joe Kelly was in General Hospital. (True); General Hospital is a soap opera. (True); He portrayed Ben Gibson. (True); Ben Gibson was in Knots Landing. (True); Knots Landing is a soap opera. (True)]}
                \newline
                [The end of the list of checked facts]
                \newline \newline
                [The start of the ideal output]
                \newline
                \textcolor{labelcolor}{[Doug Sheehan (True); is (True); an (True); American (True); actor (True); who (True); was born (True); on (True); April 27, 1949 (True); in (True); Santa Monica, California (False); . (True); He (True); is (True); best known (True); for (True); his roles in soap operas (True); , (True); including (True); in (True); his portrayal (True); of (True); Joe Kelly (True); on (True); ``General Hospital'' (True); and (True); Ben Gibson (True); on (True); ``Knots Landing.'' (True)]}
                \newline
                [The end of the ideal output]
				\newline \newline
				\textbf{User prompt}
				\newline
				\newline
				[The start of the biography]
				\newline
				\textcolor{magenta}{\texttt{\{BIOGRAPHY\}}}
				\newline
				[The ebd of the biography]
				\newline \newline
				[The start of the list of checked facts]
				\newline
				\textcolor{magenta}{\texttt{\{LIST OF CHECKED FACTS\}}}
				\newline
				[The end of the list of checked facts]
			};
		\end{tikzpicture}
        \caption{GPT-4o prompt for labeling hallucinated entities.}\label{tb:gpt-4-prompt}
	\end{center}
\vspace{-0cm}
\end{table*}
% \section{Full Experiment Results}
% \begin{table*}[th]
    \centering
    \small
    \caption{Classification Results}
    \begin{tabular}{lcccc}
        \toprule
        \textbf{Method} & \textbf{Accuracy} & \textbf{Precision} & \textbf{Recall} & \textbf{F1-score} \\
        \midrule
        \multicolumn{5}{c}{\textbf{Zero Shot}} \\
                Zero-shot E-eyes & 0.26 & 0.26 & 0.27 & 0.26 \\
        Zero-shot CARM & 0.24 & 0.24 & 0.24 & 0.24 \\
                Zero-shot SVM & 0.27 & 0.28 & 0.28 & 0.27 \\
        Zero-shot CNN & 0.23 & 0.24 & 0.23 & 0.23 \\
        Zero-shot RNN & 0.26 & 0.26 & 0.26 & 0.26 \\
DeepSeek-0shot & 0.54 & 0.61 & 0.54 & 0.52 \\
DeepSeek-0shot-COT & 0.33 & 0.24 & 0.33 & 0.23 \\
DeepSeek-0shot-Knowledge & 0.45 & 0.46 & 0.45 & 0.44 \\
Gemma2-0shot & 0.35 & 0.22 & 0.38 & 0.27 \\
Gemma2-0shot-COT & 0.36 & 0.22 & 0.36 & 0.27 \\
Gemma2-0shot-Knowledge & 0.32 & 0.18 & 0.34 & 0.20 \\
GPT-4o-mini-0shot & 0.48 & 0.53 & 0.48 & 0.41 \\
GPT-4o-mini-0shot-COT & 0.33 & 0.50 & 0.33 & 0.38 \\
GPT-4o-mini-0shot-Knowledge & 0.49 & 0.31 & 0.49 & 0.36 \\
GPT-4o-0shot & 0.62 & 0.62 & 0.47 & 0.42 \\
GPT-4o-0shot-COT & 0.29 & 0.45 & 0.29 & 0.21 \\
GPT-4o-0shot-Knowledge & 0.44 & 0.52 & 0.44 & 0.39 \\
LLaMA-0shot & 0.32 & 0.25 & 0.32 & 0.24 \\
LLaMA-0shot-COT & 0.12 & 0.25 & 0.12 & 0.09 \\
LLaMA-0shot-Knowledge & 0.32 & 0.25 & 0.32 & 0.28 \\
Mistral-0shot & 0.19 & 0.23 & 0.19 & 0.10 \\
Mistral-0shot-Knowledge & 0.21 & 0.40 & 0.21 & 0.11 \\
        \midrule
        \multicolumn{5}{c}{\textbf{4 Shot}} \\
GPT-4o-mini-4shot & 0.58 & 0.59 & 0.58 & 0.53 \\
GPT-4o-mini-4shot-COT & 0.57 & 0.53 & 0.57 & 0.50 \\
GPT-4o-mini-4shot-Knowledge & 0.56 & 0.51 & 0.56 & 0.47 \\
GPT-4o-4shot & 0.77 & 0.84 & 0.77 & 0.73 \\
GPT-4o-4shot-COT & 0.63 & 0.76 & 0.63 & 0.53 \\
GPT-4o-4shot-Knowledge & 0.72 & 0.82 & 0.71 & 0.66 \\
LLaMA-4shot & 0.29 & 0.24 & 0.29 & 0.21 \\
LLaMA-4shot-COT & 0.20 & 0.30 & 0.20 & 0.13 \\
LLaMA-4shot-Knowledge & 0.15 & 0.23 & 0.13 & 0.13 \\
Mistral-4shot & 0.02 & 0.02 & 0.02 & 0.02 \\
Mistral-4shot-Knowledge & 0.21 & 0.27 & 0.21 & 0.20 \\
        \midrule
        
        \multicolumn{5}{c}{\textbf{Suprevised}} \\
        SVM & 0.94 & 0.92 & 0.91 & 0.91 \\
        CNN & 0.98 & 0.98 & 0.97 & 0.97 \\
        RNN & 0.99 & 0.99 & 0.99 & 0.99 \\
        % \midrule
        % \multicolumn{5}{c}{\textbf{Conventional Wi-Fi-based Human Activity Recognition Systems}} \\
        E-eyes & 1.00 & 1.00 & 1.00 & 1.00 \\
        CARM & 0.98 & 0.98 & 0.98 & 0.98 \\
\midrule
 \multicolumn{5}{c}{\textbf{Vision Models}} \\
           Zero-shot SVM & 0.26 & 0.25 & 0.25 & 0.25 \\
        Zero-shot CNN & 0.26 & 0.25 & 0.26 & 0.26 \\
        Zero-shot RNN & 0.28 & 0.28 & 0.29 & 0.28 \\
        SVM & 0.99 & 0.99 & 0.99 & 0.99 \\
        CNN & 0.98 & 0.99 & 0.98 & 0.98 \\
        RNN & 0.98 & 0.99 & 0.98 & 0.98 \\
GPT-4o-mini-Vision & 0.84 & 0.85 & 0.84 & 0.84 \\
GPT-4o-mini-Vision-COT & 0.90 & 0.91 & 0.90 & 0.90 \\
GPT-4o-Vision & 0.74 & 0.82 & 0.74 & 0.73 \\
GPT-4o-Vision-COT & 0.70 & 0.83 & 0.70 & 0.68 \\
LLaMA-Vision & 0.20 & 0.23 & 0.20 & 0.09 \\
LLaMA-Vision-Knowledge & 0.22 & 0.05 & 0.22 & 0.08 \\

        \bottomrule
    \end{tabular}
    \label{full}
\end{table*}




\end{document}
 % must use main.bbl as the name, following the main.tex file
}
\vfill

\clearpage
% \onecolumn
% \appendix
\section*{Appendix}

\subsection{Expanded Discussion of AGCM}

    %%%%%%%%%% Start SVG (AGCM) %%%%%%%%%%%%
    \begin{figure*}[th]
    \centering
    \includegraphics[width=1.99\columnwidth]{fig_13.pdf}
       \caption{
       (a) Feature-based models rely on manual feature preprocessing using external automatic toolkits, such as OpenFace, which operate outside the model's training loop and are not learnable. These models map preprocessed features to task labels, risking the loss of valuable raw data information that could contribute to more comprehensive predictions.
       (b) Multi-task learning models train multiple tasks independently, with the learning of specific emotional tasks and AUs being uncorrelated and disconnected. As a result, AU predictions in multi-task learning cannot effectively explain the emotional predictions, limiting the interpretability of the model.
       (c): The proposed AGCM framework operates as follows: after feature extraction, the Attention-Guided Concept Generator creates learnable neural representations for both activated and inactivated concepts, along with their respective activation scores. It then computes the emotional concept contribution by combining the activated and inactivated embeddings for each concept. Parameter optimization for concept learning is conducted concurrently with task-label learning in an end-to-end manner, enabling the model to capture emotional concept contributions while effectively overcoming the trade-off between explainability and performance. }
    \label{fig_app_1}
    \end{figure*}
    %%%%%%%%%% END of SVG (CEM-based FER Framework) %%%%%%%%%%%%

    The use of handcrafted features, such as AU detections, has been ongoing for decades. These approaches mainly focus on automatically mapping the facial representation to a single numerical value, without fully accounting for the complexity of one’s affective state. Like in most of the feature-based approaches, relying solely on these numerical values for intricate AC tasks risks overlooking other emotion-related information conveyed by the subject, potentially degrading performance. Similarly, in multi-task learning—for instance, simultaneously predicting AU and expression—each classification head optimizes independently, rather than fostering mutually beneficial learning that emphasizes the relevance of AUs to facial expressions. 

    In contrast, as illustrated in Fig. \ref{fig_app_1}, the proposed AGCM framework enhances both model explainability and performance by bridging this gap. It employs an end-to-end learning strategy that quantifies the contributions of underlying emotion-related indicators to the final task prediction. By design, AGCM naturally advances traditional feature-based and multi-task AC approaches, where feature representations are either static or insufficient as explanations for task predictions.

\subsection{Embedding Size Ablation Study}

    %%%%%%%%%% Start SVG (AGCM) %%%%%%%%%%%%
    \begin{figure}[th]
    \centering
    \includegraphics[width=1\columnwidth]{fig_14.pdf}
       \caption{
       Task performance evaluation (\%) with different embedding sizes. For RAF-DB and AffectNet, the overall accuracy is reported. For Aff-Wild2 and NOXI, the F-1 score and CCC score are reported.}
    \label{fig_emb_size}
    \end{figure}
    %%%%%%%%%% END of SVG (CEM-based FER Framework) %%%%%%%%%%%%
    
    Previous studies have demonstrated that embedding size can impact the task performance of concept-based frameworks \cite{zarlenga2022concept}. The optimal concept size may vary depending on the task. In this work, we use an embedding size of 16 for all FER tasks and 32 for engagement estimation tasks. 

    Fig. \ref{fig_emb_size} shows the task performance across various embedding sizes. For both applications, performance initially improves with increasing embedding size. However, once the embedding size reaches the limitation of the model's learning capacity, further increases do not yield performance gains. Instead, larger embeddings may significantly raise the number of parameters, which can pose challenges for model training and deployment.

\subsection{Comparing End-to-end and By-step AGCM}
    
    To further assess the efficiency of the AGCM framework, we compare the end-to-end and by-step training strategies. In by-step AGCM, the model first optimizes a mapping function from the raw input to all intermediate concept scores. If the concepts include only AUs, this phase operates similarly to an AU detector, generating activation probabilities for all AUs. These AU probabilities are then combined with the embeddings in a subsequent optimization step to predict the final facial expression label separately.
    
    In by-step AGCM, the neural embeddings of intermediate concepts are not trainable during task learning. The parameter optimization treats the concept and task loss separately. This approach contrasts with end-to-end training, where a unified push-pull joint loss is employed to enhance both concept explainability and task performance simultaneously.

    \begin{table}[t]
    \centering
    \caption{Performance comparison (\%) of the end-to-end and by-step AGCM framework. For RAF-DB and AffectNet, the overall accuracy is reported. For Aff-Wild2 and NOXI, the F-1 score and CCC score are reported. }
    \begin{tabular}{cccc}
    \toprule
                & Data & End-to-end AGCM & By-step AGCM \\ \midrule
    RAF-DB      & V    & \textbf{94.40}      & 89.71   \\
    AffectNet-7 & V    & \textbf{69.45}      & 64.08   \\
    AffectNet-8 & V    & \textbf{65.62}      & 61.36   \\
    Aff-Wild2   & V    & \textbf{44.95}      & 39.10   \\
    Aff-Wild2   & V/A  & \textbf{47.52}      & 39.23   \\
    NOXI        & V    & \textbf{59.24}      & 52.01   \\
    NOXI        & V/A  & \textbf{80.39}      & 67.88   \\
    \bottomrule
    \end{tabular}
    \label{tab_by_step}
    \end{table}

    Table \ref{tab_by_step} presents a performance comparison between the end-to-end and by-step AGCM training strategies. Compared to the end-to-end approach, the by-step training strategy results in performance degradation across all datasets, with particularly notable declines in the multimodal AGCM framework, where separately learning concepts can lead to significant information loss from the raw data. Thus, we posit that jointly learning the concept and task label enhances both model explainability and task performance by compelling the model to explicitly supervise human-understandable features derived from domain-specific prior knowledge.

\subsection{Expanded Discussion of AGCM and Map-based XAI}

    Map-based XAI was originally designed for general ML tasks like object localization, where attention heatmaps serve as effective tools to indicate object locations \cite{gao2021ts}. In affective signal processing, however, spatial concept explanations offer significant advantages over map-based XAI by providing domain-specific insights alongside task performance improvements. Simply presenting an attention heatmap over a face region offers minimal value for domain experts in AC applications. For instance, two opposing indicators, AU12 (Lip Corner Puller) and AU15 (Lip Corner Depressor), appear in the same region of the face, making it insufficient to rely solely on attention maps for emotion interpretation. Instead, conceptual explanations that explicitly indicate the activation and contribution of specific AUs provide a more natural and informative approach to AC tasks.

    Recent map-based FER work \cite{belharbi2024guided} uses pre-generated AU maps based on emotion labels to guide model learning, depending on a strict mapping between AUs and facial expressions. For example, for images labeled as ``happiness,'' this approach restricts the model’s focus strictly to the AU6 and AU12 regions, regardless of whether these specific AUs are activated, ignoring other facial information that may contribute to the expression. This rigid mapping not only degrades performance but also proves limiting in downstream AC applications, such as engagement estimation or mental health assessment, where there is no clear mapping between AUs and affective labels.

    Fig. \ref{fig_big_example} compares explanations provided by our proposed AGCM with those from two map-based XAI methods \cite{gao2021ts, belharbi2024guided}. The attention heatmaps from the map-based XAI approaches appear similar across different expression labels, offering insufficient interpretability for high-stakes AC applications. In contrast, AGCM not only localizes each AU but also quantifies its contribution to the final prediction, delivering richer insights into model predictions while achieving state-of-the-art task performance.

    \clearpage
    %%%%%%%%%% Start SVG (AGCM) %%%%%%%%%%%%
    \begin{figure*}[t]
    \centering
    \includegraphics[width=1.99\columnwidth]{fig_15.pdf}
       \caption{Explanation examples of map-based TS-CAM \cite{gao2021ts}, attention map-based FER (Att-Map) \cite{belharbi2024guided}, and the proposed AGCM framework. In addition to all concept locations, AGCM explicitly provides the contribution score of each concept, offering domain-specific insight into the model decision-making process. The images are randomly picked from the AffectNet test set. 
       }
    \label{fig_big_example}
    \end{figure*}
    %%%%%%%%%% END of SVG (CEM-based FER Framework) %%%%%%%%%%%%
\end{document}


