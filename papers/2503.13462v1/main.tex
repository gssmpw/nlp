\documentclass[conference]{IEEEtran}
\IEEEoverridecommandlockouts
% The preceding line is only needed to identify funding in the first footnote. If that is unneeded, please comment it out.
%Template version as of 6/27/2024

\usepackage{cite}
\usepackage{amsmath,amssymb,amsfonts}
\usepackage{algorithmic}
\usepackage{graphicx}
\usepackage{textcomp}
\usepackage{xcolor}
\usepackage{makecell}

%---------------------------------------------
% Custom packages
%---------------------------------------------
\usepackage{siunitx}
\usepackage{hyperref}
\usepackage[nolist]{acronym}
\usepackage[flushleft]{threeparttable}
\usepackage{placeins}
\usepackage{svg}                                   
\usepackage[percent]{overpic}

\usepackage{mathtools}
%\usepackage{multirow}
\usepackage{booktabs}
%\usepackage{subcaption}
\usepackage[nolist]{acronym}
\usepackage{pbalance} %balances last page
\usepackage[export]{adjustbox}

\usepackage{eso-pic}    % watermark

\def\BibTeX{{\rm B\kern-.05em{\sc i\kern-.025em b}\kern-.08em
    T\kern-.1667em\lower.7ex\hbox{E}\kern-.125emX}}
%---------------------------------------------
% Custom Commands
%---------------------------------------------
\newcommand{\luca}[1]{{\textcolor{red}{#1}}}       % Custom boss comments
\newcommand{\michele}[1]{{\textcolor{orange}{#1}}}  % Custom boss comments
\newcommand{\revise}[1]{{\textcolor{black}{#1}}}

\newcommand{\todo}[1]{\textcolor{red}{#1}}          % Custom command to highlight todos
\newcommand{\done}[1]{\textcolor{green}{#1}}        % Custom command to highlight done
\newcommand{\idea}[1]{\textcolor{violet}{#1}}       % Custom command to highlight ideas
\newcommand{\RNum}[1]{\uppercase\expandafter{\romannumeral #1\relax}}   % To print roman digits: I, II, III, IV etc.
\newcommand{\bfqty}[2]{\text{\bfseries\SI{#1}{#2}}}

% Re-define the command \IEEEauthorrefmark{x} to get numbers instead of symbols
\DeclareRobustCommand{\IEEEauthorrefmark}[1]{\smash{\textsuperscript{\footnotesize #1}}}

% Decalare custom SI units
\DeclareSIUnit\db{dB}                           % Decbel
\DeclareSIUnit\dbi{dBi}                         % Isotropic antenna gain in decibel
\DeclareSIUnit\dbm{dBm}                         % Power in decibel
\DeclareSIUnit\watthour{Wh}                     % Whatthour without space between W and h
\DeclareSIUnit\mbps{Mbps}                       % Mega bit per second
\DeclareSIUnit\kbps{kbps}                       % kilo bit per second
\DeclareSIUnit\bps{bps}                         % bit bit per second
\DeclareSIUnit\msInference{ms/inference}        % inference time 

%---------------------------------------------
% Title
%---------------------------------------------
\begin{document}
\bstctlcite{IEEEexample:BSTcontrol} %used tp shortn author list (needs some text in the bib file too)


\AddToShipoutPictureBG*{
  \AtPageUpperLeft{%
    \put(0,-40){\raisebox{15pt}{\makebox[\paperwidth]{\begin{minipage}{21cm}\centering
      \textcolor{gray}{This work has been submitted to the IEEE for possible publication. Copyright may be transferred without notice, after which this version may no longer be accessible.
      } 
    \end{minipage}}}}%
  }
  \AtPageLowerLeft{%
    \raisebox{25pt}{\makebox[\paperwidth]{\begin{minipage}{21cm}\centering
      \textcolor{gray}{This work has been submitted to the IEEE for possible publication. Copyright may be transferred without notice, after which this version may no longer be accessible.
      }
    \end{minipage}}}%
  }
}

\title{BodySense: An Expandable and Wearable-Sized Wireless Evaluation Platform for Human Body Communication}

%---------------------------------------------
% Author Section
%---------------------------------------------
\author{
\IEEEauthorblockN{
Lukas Schulthess, 
Philipp Mayer, 
Christian Vogt, 
Luca Benini,
Michele Magno
}\\

\IEEEauthorblockA{Dept. of Information Technology and Electrical Engineering, ETH Z{\"u}rich, Switzerland}
%\IEEEauthorblockA{\IEEEauthorrefmark{2} Dept. of Electrical, Electronic and Information Engineering, University of Bologna, Italy}
%\IEEEauthorblockA{\IEEEauthorrefmark{1} Center for Project Based Learning, ETH Z{\"u}rich, Z{\"u}rich, Switzerland}
%\IEEEauthorblockA{\IEEEauthorrefmark{2} Integrated Systems Laboratory, ETH Z{\"u}rich, Z{\"u}rich, Switzerland}
%\IEEEauthorblockA{\IEEEauthorrefmark{3} DEI, University of Bologna, Bologna, Italy}
}

\maketitle
\IEEEpeerreviewmaketitle

% Main Section
%---------------------------------------------
% Manuscripts should consist of a complete description of the proposed technical content and research results, spanning a typical range of 4 to a maximum of 6 pages. Submissions that do not comply with this requirement may be subject to automatic rejection
\begin{abstract}\label{00_Abstract}
Research in the field of automated vehicles, or more generally cognitive cyber-physical systems that operate in the real world, is leading to increasingly complex systems. Among other things, artificial intelligence enables an ever-increasing degree of autonomy. In this context, the V-model, which has served for decades as a process reference model of the system development lifecycle is reaching its limits. To the contrary, innovative processes and frameworks have been developed that take into account the characteristics of emerging autonomous systems. To bridge the gap and merge the different methodologies, we present an extension of the V-model for iterative data-based development processes that harmonizes and formalizes the existing methods towards a generic framework. The iterative approach allows for seamless integration of continuous system refinement. While the data-based approach constitutes the consideration of data-based development processes and formalizes the use of synthetic and real world data. In this way, formalizing the process of development, verification, validation, and continuous integration contributes to ensuring the safety of emerging complex systems that incorporate AI. 
\end{abstract}


\begin{IEEEkeywords}
	Process Reference Model, V-Model, Continuous Integration, AI Systems, Autonomy Technology, Safety Assurance
\end{IEEEkeywords}

\vspace{10pt}
\begin{IEEEkeywords}
Human body communication (HBC), capacitive HBC, wireless data acquisition, energy-efficient communication, body sensor networks
\end{IEEEkeywords}

\section{Introduction}

% \textcolor{red}{Still on working}

% \textcolor{red}{add label for each section}


Robot learning relies on diverse and high-quality data to learn complex behaviors \cite{aldaco2024aloha, wang2024dexcap}.
Recent studies highlight that models trained on datasets with greater complexity and variation in the domain tend to generalize more effectively across broader scenarios \cite{mann2020language, radford2021learning, gao2024efficient}.
% However, creating such diverse datasets in the real world presents significant challenges.
% Modifying physical environments and adjusting robot hardware settings require considerable time, effort, and financial resources.
% In contrast, simulation environments offer a flexible and efficient alternative.
% Simulations allow for the creation and modification of digital environments with a wide range of object shapes, weights, materials, lighting, textures, friction coefficients, and so on to incorporate domain randomization,
% which helps improve the robustness of models when deployed in real-world conditions.
% These environments can be easily adjusted and reset, enabling faster iterations and data collection.
% Additionally, simulations provide the ability to consistently reproduce scenarios, which is essential for benchmarking and model evaluation.
% Another advantage of simulations is their flexibility in sensor integration. Sensors such as cameras, LiDARs, and tactile sensors can be added or repositioned without the physical limitations present in real-world setups. Simulations also eliminate the risk of damaging expensive hardware during edge-case experiments, making them an ideal platform for testing rare or dangerous scenarios that are impractical to explore in real life.
By leveraging immersive perspectives and interactions, Extended Reality\footnote{Extended Reality is an umbrella term to refer to Augmented Reality, Mixed Reality, and Virtual Reality \cite{wikipediaExtendedReality}}
(XR)
is a promising candidate for efficient and intuitive large scale data collection \cite{jiang2024comprehensive, arcade}
% With the demand for collecting data, XR provides a promising approach for humans to teach robots by offering users an immersive experience.
in simulation \cite{jiang2024comprehensive, arcade, dexhub-park} and real-world scenarios \cite{openteach, opentelevision}.
However, reusing and reproducing current XR approaches for robot data collection for new settings and scenarios is complicated and requires significant effort.
% are difficult to reuse and reproduce system makes it hard to reuse and reproduce in another data collection pipeline.
This bottleneck arises from three main limitations of current XR data collection and interaction frameworks: \textit{asset limitation}, \textit{simulator limitation}, and \textit{device limitation}.
% \textcolor{red}{ASSIGN THESE CITATION PROPERLY:}
% \textcolor{red}{list them by time order???}
% of collecting data by using XR have three main limitations.
Current approaches suffering from \textit{asset limitation} \cite{arclfd, jiang2024comprehensive, arcade, george2025openvr, vicarios}
% Firstly, recent works \cite{jiang2024comprehensive, arcade, dexhub-park}
can only use predefined robot models and task scenes. Configuring new tasks requires significant effort, since each new object or model must be specifically integrated into the XR application.
% and it takes too much effort to configure new tasks in their systems since they cannot spawn arbitrary models in the XR application.
The vast majority of application are developed for specific simulators or real-world scenarios. This \textit{simulator limitation} \cite{mosbach2022accelerating, lipton2017baxter, dexhub-park, arcade}
% Secondly, existing systems are limited to a single simulation platform or real-world scenarios.
significantly reduces reusability and makes adaptation to new simulation platforms challenging.
Additionally, most current XR frameworks are designed for a specific version of a single XR headset, leading to a \textit{device limitation} 
\cite{lipton2017baxter, armada, openteach, meng2023virtual}.
% and there is no work working on the extendability of transferring to a new headsets as far as we know.
To the best of our knowledge, no existing work has explored the extensibility or transferability of their framework to different headsets.
These limitations hamper reproducibility and broader contributions of XR based data collection and interaction to the research community.
% as each research group typically has its own data collection pipeline.
% In addition to these main limitations, existing XR systems are not well suited for managing multiple robot systems,
% as they are often designed for single-operator use.

In addition to these main limitations, existing XR systems are often designed for single-operator use, prohibiting collaborative data collection.
At the same time, controlling multiple robots at once can be very difficult for a single operator,
making data collection in multi-robot scenarios particularly challenging \cite{orun2019effect}.
Although there are some works using collaborative data collection in the context of tele-operation \cite{tung2021learning, Qin2023AnyTeleopAG},
there is no XR-based data collection system supporting collaborative data collection.
This limitation highlights the need for more advanced XR solutions that can better support multi-robot and multi-user scenarios.
% \textcolor{red}{more papers about collaborative data collection}

To address all of these issues, we propose \textbf{IRIS},
an \textbf{I}mmersive \textbf{R}obot \textbf{I}nteraction \textbf{S}ystem.
This general system supports various simulators, benchmarks and real-world scenarios.
It is easily extensible to new simulators and XR headsets.
IRIS achieves generalization across six dimensions:
% \begin{itemize}
%     \item \textit{Cross-scene} : diverse object models;
%     \item \textit{Cross-embodiment}: diverse robot models;
%     \item \textit{Cross-simulator}: 
%     \item \textit{Cross-reality}: fd
%     \item \textit{Cross-platform}: fd
%     \item \textit{Cross-users}: fd
% \end{itemize}
\textbf{Cross-Scene}, \textbf{Cross-Embodiment}, \textbf{Cross-Simulator}, \textbf{Cross-Reality}, \textbf{Cross-Platform}, and \textbf{Cross-User}.

\textbf{Cross-Scene} and \textbf{Cross-Embodiment} allow the system to handle arbitrary objects and robots in the simulation,
eliminating restrictions about predefined models in XR applications.
IRIS achieves these generalizations by introducing a unified scene specification, representing all objects,
including robots, as data structures with meshes, materials, and textures.
The unified scene specification is transmitted to the XR application to create and visualize an identical scene.
By treating robots as standard objects, the system simplifies XR integration,
allowing researchers to work with various robots without special robot-specific configurations.
\textbf{Cross-Simulator} ensures compatibility with various simulation engines.
IRIS simplifies adaptation by parsing simulated scenes into the unified scene specification, eliminating the need for XR application modifications when switching simulators.
New simulators can be integrated by creating a parser to convert their scenes into the unified format.
This flexibility is demonstrated by IRIS’ support for Mujoco \cite{todorov2012mujoco}, IsaacSim \cite{mittal2023orbit}, CoppeliaSim \cite{coppeliaSim}, and even the recent Genesis \cite{Genesis} simulator.
\textbf{Cross-Reality} enables the system to function seamlessly in both virtual simulations and real-world applications.
IRIS enables real-world data collection through camera-based point cloud visualization.
\textbf{Cross-Platform} allows for compatibility across various XR devices.
Since XR device APIs differ significantly, making a single codebase impractical, IRIS XR application decouples its modules to maximize code reuse.
This application, developed by Unity \cite{unity3dUnityManual}, separates scene visualization and interaction, allowing developers to integrate new headsets by reusing the visualization code and only implementing input handling for hand, head, and motion controller tracking.
IRIS provides an implementation of the XR application in the Unity framework, allowing for a straightforward deployment to any device that supports Unity. 
So far, IRIS was successfully deployed to the Meta Quest 3 and HoloLens 2.
Finally, the \textbf{Cross-User} ability allows multiple users to interact within a shared scene.
IRIS achieves this ability by introducing a protocol to establish the communication between multiple XR headsets and the simulation or real-world scenarios.
Additionally, IRIS leverages spatial anchors to support the alignment of virtual scenes from all deployed XR headsets.
% To make an seamless user experience for robot learning data collection,
% IRIS also tested in three different robot control interface
% Furthermore, to demonstrate the extensibility of our approach, we have implemented a robot-world pipeline for real robot data collection, ensuring that the system can be used in both simulated and real-world environments.
The Immersive Robot Interaction System makes the following contributions\\
\textbf{(1) A unified scene specification} that is compatible with multiple robot simulators. It enables various XR headsets to visualize and interact with simulated objects and robots, providing an immersive experience while ensuring straightforward reusability and reproducibility.\\
\textbf{(2) A collaborative data collection framework} designed for XR environments. The framework facilitates enhanced robot data acquisition.\\
\textbf{(3) A user study} demonstrating that IRIS significantly improves data collection efficiency and intuitiveness compared to the LIBERO baseline.

% \begin{table*}[t]
%     \centering
%     \begin{tabular}{lccccccc}
%         \toprule
%         & \makecell{Physical\\Interaction}
%         & \makecell{XR\\Enabled}
%         & \makecell{Free\\View}
%         & \makecell{Multiple\\Robots}
%         & \makecell{Robot\\Control}
%         % Force Feedback???
%         & \makecell{Soft Object\\Supported}
%         & \makecell{Collaborative\\Data} \\
%         \midrule
%         ARC-LfD \cite{arclfd}                              & Real        & \cmark & \xmark & \xmark & Joint              & \xmark & \xmark \\
%         DART \cite{dexhub-park}                            & Sim         & \cmark & \cmark & \cmark & Cartesian          & \xmark & \xmark \\
%         \citet{jiang2024comprehensive}                     & Sim         & \cmark & \xmark & \xmark & Joint \& Cartesian & \xmark & \xmark \\
%         \citet{mosbach2022accelerating}                    & Sim         & \cmark & \cmark & \xmark & Cartesian          & \xmark & \xmark \\
%         ARCADE \cite{arcade}                               & Real        & \cmark & \cmark & \xmark & Cartesian          & \xmark & \xmark \\
%         Holo-Dex \cite{holodex}                            & Real        & \cmark & \xmark & \cmark & Cartesian          & \cmark & \xmark \\
%         ARMADA \cite{armada}                               & Real        & \cmark & \xmark & \cmark & Cartesian          & \cmark & \xmark \\
%         Open-TeleVision \cite{opentelevision}              & Real        & \cmark & \cmark & \cmark & Cartesian          & \cmark & \xmark \\
%         OPEN TEACH \cite{openteach}                        & Real        & \cmark & \xmark & \cmark & Cartesian          & \cmark & \cmark \\
%         GELLO \cite{wu2023gello}                           & Real        & \xmark & \cmark & \cmark & Joint              & \cmark & \xmark \\
%         DexCap \cite{wang2024dexcap}                       & Real        & \xmark & \cmark & \xmark & Cartesian          & \cmark & \xmark \\
%         AnyTeleop \cite{Qin2023AnyTeleopAG}                & Real        & \xmark & \xmark & \cmark & Cartesian          & \cmark & \cmark \\
%         Vicarios \cite{vicarios}                           & Real        & \cmark & \xmark & \xmark & Cartesian          & \cmark & \xmark \\     
%         Augmented Visual Cues \cite{augmentedvisualcues}   & Real        & \cmark & \cmark & \xmark & Cartesian          & \xmark & \xmark \\ 
%         \citet{wang2024robotic}                            & Real        & \cmark & \cmark & \xmark & Cartesian          & \cmark & \xmark \\
%         Bunny-VisionPro \cite{bunnyvisionpro}              & Real        & \cmark & \cmark & \cmark & Cartesian          & \cmark & \xmark \\
%         IMMERTWIN \cite{immertwin}                         & Real        & \cmark & \cmark & \cmark & Cartesian          & \xmark & \xmark \\
%         \citet{meng2023virtual}                            & Sim \& Real & \cmark & \cmark & \xmark & Cartesian          & \xmark & \xmark \\
%         Shared Control Framework \cite{sharedctlframework} & Real        & \cmark & \cmark & \cmark & Cartesian          & \xmark & \xmark \\
%         OpenVR \cite{openvr}                               & Real        & \cmark & \cmark & \xmark & Cartesian          & \xmark & \xmark \\
%         \citet{digitaltwinmr}                              & Real        & \cmark & \cmark & \xmark & Cartesian          & \cmark & \xmark \\
        
%         \midrule
%         \textbf{Ours} & Sim \& Real & \cmark & \cmark & \cmark & Joint \& Cartesian  & \cmark & \cmark \\
%         \bottomrule
%     \end{tabular}
%     \caption{This is a cross-column table with automatic line breaking.}
%     \label{tab:cross-column}
% \end{table*}

% \begin{table*}[t]
%     \centering
%     \begin{tabular}{lccccccc}
%         \toprule
%         & \makecell{Cross-Embodiment}
%         & \makecell{Cross-Scene}
%         & \makecell{Cross-Simulator}
%         & \makecell{Cross-Reality}
%         & \makecell{Cross-Platform}
%         & \makecell{Cross-User} \\
%         \midrule
%         ARC-LfD \cite{arclfd}                              & \xmark & \xmark & \xmark & \xmark & \xmark & \xmark \\
%         DART \cite{dexhub-park}                            & \cmark & \cmark & \xmark & \xmark & \xmark & \xmark \\
%         \citet{jiang2024comprehensive}                     & \xmark & \cmark & \xmark & \xmark & \xmark & \xmark \\
%         \citet{mosbach2022accelerating}                    & \xmark & \cmark & \xmark & \xmark & \xmark & \xmark \\
%         ARCADE \cite{arcade}                               & \xmark & \xmark & \xmark & \xmark & \xmark & \xmark \\
%         Holo-Dex \cite{holodex}                            & \cmark & \xmark & \xmark & \xmark & \xmark & \xmark \\
%         ARMADA \cite{armada}                               & \cmark & \xmark & \xmark & \xmark & \xmark & \xmark \\
%         Open-TeleVision \cite{opentelevision}              & \cmark & \xmark & \xmark & \xmark & \cmark & \xmark \\
%         OPEN TEACH \cite{openteach}                        & \cmark & \xmark & \xmark & \xmark & \xmark & \cmark \\
%         GELLO \cite{wu2023gello}                           & \cmark & \xmark & \xmark & \xmark & \xmark & \xmark \\
%         DexCap \cite{wang2024dexcap}                       & \xmark & \xmark & \xmark & \xmark & \xmark & \xmark \\
%         AnyTeleop \cite{Qin2023AnyTeleopAG}                & \cmark & \cmark & \cmark & \cmark & \xmark & \cmark \\
%         Vicarios \cite{vicarios}                           & \xmark & \xmark & \xmark & \xmark & \xmark & \xmark \\     
%         Augmented Visual Cues \cite{augmentedvisualcues}   & \xmark & \xmark & \xmark & \xmark & \xmark & \xmark \\ 
%         \citet{wang2024robotic}                            & \xmark & \xmark & \xmark & \xmark & \xmark & \xmark \\
%         Bunny-VisionPro \cite{bunnyvisionpro}              & \cmark & \xmark & \xmark & \xmark & \xmark & \xmark \\
%         IMMERTWIN \cite{immertwin}                         & \cmark & \xmark & \xmark & \xmark & \xmark & \xmark \\
%         \citet{meng2023virtual}                            & \xmark & \cmark & \xmark & \cmark & \xmark & \xmark \\
%         \citet{sharedctlframework}                         & \cmark & \xmark & \xmark & \xmark & \xmark & \xmark \\
%         OpenVR \cite{george2025openvr}                               & \xmark & \xmark & \xmark & \xmark & \xmark & \xmark \\
%         \citet{digitaltwinmr}                              & \xmark & \xmark & \xmark & \xmark & \xmark & \xmark \\
        
%         \midrule
%         \textbf{Ours} & \cmark & \cmark & \cmark & \cmark & \cmark & \cmark \\
%         \bottomrule
%     \end{tabular}
%     \caption{This is a cross-column table with automatic line breaking.}
% \end{table*}

% \begin{table*}[t]
%     \centering
%     \begin{tabular}{lccccccc}
%         \toprule
%         & \makecell{Cross-Scene}
%         & \makecell{Cross-Embodiment}
%         & \makecell{Cross-Simulator}
%         & \makecell{Cross-Reality}
%         & \makecell{Cross-Platform}
%         & \makecell{Cross-User}
%         & \makecell{Control Space} \\
%         \midrule
%         % Vicarios \cite{vicarios}                           & \xmark & \xmark & \xmark & \xmark & \xmark & \xmark \\     
%         % Augmented Visual Cues \cite{augmentedvisualcues}   & \xmark & \xmark & \xmark & \xmark & \xmark & \xmark \\ 
%         % OpenVR \cite{george2025openvr}                     & \xmark & \xmark & \xmark & \xmark & \xmark & \xmark \\
%         \citet{digitaltwinmr}                              & \xmark & \xmark & \xmark & \xmark & \xmark & \xmark &  \\
%         ARC-LfD \cite{arclfd}                              & \xmark & \xmark & \xmark & \xmark & \xmark & \xmark &  \\
%         \citet{sharedctlframework}                         & \cmark & \xmark & \xmark & \xmark & \xmark & \xmark &  \\
%         \citet{jiang2024comprehensive}                     & \cmark & \xmark & \xmark & \xmark & \xmark & \xmark &  \\
%         \citet{mosbach2022accelerating}                    & \cmark & \xmark & \xmark & \xmark & \xmark & \xmark & \\
%         Holo-Dex \cite{holodex}                            & \cmark & \xmark & \xmark & \xmark & \xmark & \xmark & \\
%         ARCADE \cite{arcade}                               & \cmark & \cmark & \xmark & \xmark & \xmark & \xmark & \\
%         DART \cite{dexhub-park}                            & Limited & Limited & Mujoco & Sim & Vision Pro & \xmark &  Cartesian\\
%         ARMADA \cite{armada}                               & \cmark & \cmark & \xmark & \xmark & \xmark & \xmark & \\
%         \citet{meng2023virtual}                            & \cmark & \cmark & \xmark & \cmark & \xmark & \xmark & \\
%         % GELLO \cite{wu2023gello}                           & \cmark & \xmark & \xmark & \xmark & \xmark & \xmark \\
%         % DexCap \cite{wang2024dexcap}                       & \xmark & \xmark & \xmark & \xmark & \xmark & \xmark \\
%         % AnyTeleop \cite{Qin2023AnyTeleopAG}                & \cmark & \cmark & \cmark & \cmark & \xmark & \cmark \\
%         % \citet{wang2024robotic}                            & \xmark & \xmark & \xmark & \xmark & \xmark & \xmark \\
%         Bunny-VisionPro \cite{bunnyvisionpro}              & \cmark & \cmark & \xmark & \xmark & \xmark & \xmark & \\
%         IMMERTWIN \cite{immertwin}                         & \cmark & \cmark & \xmark & \xmark & \xmark & \xmark & \\
%         Open-TeleVision \cite{opentelevision}              & \cmark & \cmark & \xmark & \xmark & \cmark & \xmark & \\
%         \citet{szczurek2023multimodal}                     & \xmark & \xmark & \xmark & Real & \xmark & \cmark & \\
%         OPEN TEACH \cite{openteach}                        & \cmark & \cmark & \xmark & \xmark & \xmark & \cmark & \\
%         \midrule
%         \textbf{Ours} & \cmark & \cmark & \cmark & \cmark & \cmark & \cmark \\
%         \bottomrule
%     \end{tabular}
%     \caption{TODO, Bruce: this table can be further optimized.}
% \end{table*}

\definecolor{goodgreen}{HTML}{228833}
\definecolor{goodred}{HTML}{EE6677}
\definecolor{goodgray}{HTML}{BBBBBB}

\begin{table*}[t]
    \centering
    \begin{adjustbox}{max width=\textwidth}
    \renewcommand{\arraystretch}{1.2}    
    \begin{tabular}{lccccccc}
        \toprule
        & \makecell{Cross-Scene}
        & \makecell{Cross-Embodiment}
        & \makecell{Cross-Simulator}
        & \makecell{Cross-Reality}
        & \makecell{Cross-Platform}
        & \makecell{Cross-User}
        & \makecell{Control Space} \\
        \midrule
        % Vicarios \cite{vicarios}                           & \xmark & \xmark & \xmark & \xmark & \xmark & \xmark \\     
        % Augmented Visual Cues \cite{augmentedvisualcues}   & \xmark & \xmark & \xmark & \xmark & \xmark & \xmark \\ 
        % OpenVR \cite{george2025openvr}                     & \xmark & \xmark & \xmark & \xmark & \xmark & \xmark \\
        \citet{digitaltwinmr}                              & \textcolor{goodred}{Limited}     & \textcolor{goodred}{Single Robot} & \textcolor{goodred}{Unity}    & \textcolor{goodred}{Real}          & \textcolor{goodred}{Meta Quest 2} & \textcolor{goodgray}{N/A} & \textcolor{goodred}{Cartesian} \\
        ARC-LfD \cite{arclfd}                              & \textcolor{goodgray}{N/A}        & \textcolor{goodred}{Single Robot} & \textcolor{goodgray}{N/A}     & \textcolor{goodred}{Real}          & \textcolor{goodred}{HoloLens}     & \textcolor{goodgray}{N/A} & \textcolor{goodred}{Cartesian} \\
        \citet{sharedctlframework}                         & \textcolor{goodred}{Limited}     & \textcolor{goodred}{Single Robot} & \textcolor{goodgray}{N/A}     & \textcolor{goodred}{Real}          & \textcolor{goodred}{HTC Vive Pro} & \textcolor{goodgray}{N/A} & \textcolor{goodred}{Cartesian} \\
        \citet{jiang2024comprehensive}                     & \textcolor{goodred}{Limited}     & \textcolor{goodred}{Single Robot} & \textcolor{goodgray}{N/A}     & \textcolor{goodred}{Real}          & \textcolor{goodred}{HoloLens 2}   & \textcolor{goodgray}{N/A} & \textcolor{goodgreen}{Joint \& Cartesian} \\
        \citet{mosbach2022accelerating}                    & \textcolor{goodgreen}{Available} & \textcolor{goodred}{Single Robot} & \textcolor{goodred}{IsaacGym} & \textcolor{goodred}{Sim}           & \textcolor{goodred}{Vive}         & \textcolor{goodgray}{N/A} & \textcolor{goodgreen}{Joint \& Cartesian} \\
        Holo-Dex \cite{holodex}                            & \textcolor{goodgray}{N/A}        & \textcolor{goodred}{Single Robot} & \textcolor{goodgray}{N/A}     & \textcolor{goodred}{Real}          & \textcolor{goodred}{Meta Quest 2} & \textcolor{goodgray}{N/A} & \textcolor{goodred}{Joint} \\
        ARCADE \cite{arcade}                               & \textcolor{goodgray}{N/A}        & \textcolor{goodred}{Single Robot} & \textcolor{goodgray}{N/A}     & \textcolor{goodred}{Real}          & \textcolor{goodred}{HoloLens 2}   & \textcolor{goodgray}{N/A} & \textcolor{goodred}{Cartesian} \\
        DART \cite{dexhub-park}                            & \textcolor{goodred}{Limited}     & \textcolor{goodred}{Limited}      & \textcolor{goodred}{Mujoco}   & \textcolor{goodred}{Sim}           & \textcolor{goodred}{Vision Pro}   & \textcolor{goodgray}{N/A} & \textcolor{goodred}{Cartesian} \\
        ARMADA \cite{armada}                               & \textcolor{goodgray}{N/A}        & \textcolor{goodred}{Limited}      & \textcolor{goodgray}{N/A}     & \textcolor{goodred}{Real}          & \textcolor{goodred}{Vision Pro}   & \textcolor{goodgray}{N/A} & \textcolor{goodred}{Cartesian} \\
        \citet{meng2023virtual}                            & \textcolor{goodred}{Limited}     & \textcolor{goodred}{Single Robot} & \textcolor{goodred}{PhysX}   & \textcolor{goodgreen}{Sim \& Real} & \textcolor{goodred}{HoloLens 2}   & \textcolor{goodgray}{N/A} & \textcolor{goodred}{Cartesian} \\
        % GELLO \cite{wu2023gello}                           & \cmark & \xmark & \xmark & \xmark & \xmark & \xmark \\
        % DexCap \cite{wang2024dexcap}                       & \xmark & \xmark & \xmark & \xmark & \xmark & \xmark \\
        % AnyTeleop \cite{Qin2023AnyTeleopAG}                & \cmark & \cmark & \cmark & \cmark & \xmark & \cmark \\
        % \citet{wang2024robotic}                            & \xmark & \xmark & \xmark & \xmark & \xmark & \xmark \\
        Bunny-VisionPro \cite{bunnyvisionpro}              & \textcolor{goodgray}{N/A}        & \textcolor{goodred}{Single Robot} & \textcolor{goodgray}{N/A}     & \textcolor{goodred}{Real}          & \textcolor{goodred}{Vision Pro}   & \textcolor{goodgray}{N/A} & \textcolor{goodred}{Cartesian} \\
        IMMERTWIN \cite{immertwin}                         & \textcolor{goodgray}{N/A}        & \textcolor{goodred}{Limited}      & \textcolor{goodgray}{N/A}     & \textcolor{goodred}{Real}          & \textcolor{goodred}{HTC Vive}     & \textcolor{goodgray}{N/A} & \textcolor{goodred}{Cartesian} \\
        Open-TeleVision \cite{opentelevision}              & \textcolor{goodgray}{N/A}        & \textcolor{goodred}{Limited}      & \textcolor{goodgray}{N/A}     & \textcolor{goodred}{Real}          & \textcolor{goodgreen}{Meta Quest, Vision Pro} & \textcolor{goodgray}{N/A} & \textcolor{goodred}{Cartesian} \\
        \citet{szczurek2023multimodal}                     & \textcolor{goodgray}{N/A}        & \textcolor{goodred}{Limited}      & \textcolor{goodgray}{N/A}     & \textcolor{goodred}{Real}          & \textcolor{goodred}{HoloLens 2}   & \textcolor{goodgreen}{Available} & \textcolor{goodred}{Joint \& Cartesian} \\
        OPEN TEACH \cite{openteach}                        & \textcolor{goodgray}{N/A}        & \textcolor{goodgreen}{Available}  & \textcolor{goodgray}{N/A}     & \textcolor{goodred}{Real}          & \textcolor{goodred}{Meta Quest 3} & \textcolor{goodred}{N/A} & \textcolor{goodgreen}{Joint \& Cartesian} \\
        \midrule
        \textbf{Ours}                                      & \textcolor{goodgreen}{Available} & \textcolor{goodgreen}{Available}  & \textcolor{goodgreen}{Mujoco, CoppeliaSim, IsaacSim} & \textcolor{goodgreen}{Sim \& Real} & \textcolor{goodgreen}{Meta Quest 3, HoloLens 2} & \textcolor{goodgreen}{Available} & \textcolor{goodgreen}{Joint \& Cartesian} \\
        \bottomrule
        \end{tabular}
    \end{adjustbox}
    \caption{Comparison of XR-based system for robots. IRIS is compared with related works in different dimensions.}
\end{table*}


\section{Related Work}

In this section, we review research related to the importance and barriers to parental involvement; parental use of learning technologies; and the use of generative AI and robot in educational and parenting scenarios.

\subsection{Importance and Barriers to Parental Involvement}\label{sec-rw-2.1}

% 79 words
Early childhood is a critical period for predicting future success and well-being, with early education investments resulting in higher returns than later interventions \cite{duncan2007school, doyle2009investing}. Effective parental involvement fosters cognitive and social skills, especially in younger children \cite{blevins2016early, peck1992parent}. Parents are encouraged to prioritize home-based involvement to maximize their influence \cite{ma2016meta}, as their involvement has a greater impact on children's learning outcomes \cite{hoffner2002parents, fehrmann1987home, hill2004parent} within the family setting than partnerships with schools or communities \cite{ma2016meta, harris2008parents, fantuzzo2004multiple, sui1996effects}.

However, parents' involvement in their children's education is often constrained by practical challenges related to parents' \textit{skills}, \textit{time}, and \textit{energy}. The Hoover-Dempsey and Sandler (HDS) framework \cite{green2007parents} and the CAM framework \cite{ho2024s} both highlight these factors-- parents' perceived \textit{skills and knowledge} (capability), \textit{time} (availability), and \textit{motivation} (energy)--influence the extent of their engagement. For instance, a parent confident in math may choose to engage more in math-related tasks, while those facing inflexible schedules may participate less \cite{green2007parents}. Unlike teachers, parents often lack formal pedagogical training and may underestimate their role in supporting children's learning, particularly as young children struggle to articulate their needs \cite{hara1998parent}. The CAM framework similarly suggests parents delegate tasks to a robot when they feel less capable, have limited time, or are unmotivated. These factors reflect parents' life contexts, shaped by demographic backgrounds, occupations, and parenting responsibilities \cite{grolnick1997predictors}, highlighting the need to help parents overcome barriers to effective involvement in early education within their life contexts.

\subsection{Parental Use of Learning Technologies}\label{sec-rw-2.2}
% 207 words
Technology encourages parental involvement by facilitating parent-child engagement in learning activities while introducing risks that require active parental mediation \cite{gonzalez2022parental}. On the positive side, technology offers novel opportunities for parental engagement and enhances children's learning outcomes. For example, e-books promote interactive behaviors between parents and children better than print books \cite{korat2010new}. In addition, having access to computers at home significantly boost academic achievement of young children when parents actively mediate their use \cite{hofferth2010home, espinosa2006technology}. However, the effectiveness of these tools often depends on parents' familiarity with and attitude toward technology. Mobile applications, for instance, can improve learning outcomes but require parents to possess sufficient technology efficacy to guide their use \cite{papadakis2019parental}.

On the negative side, technology introduces risks such as excessive screen time, exposure to inappropriate content, and misinformation, which necessitate parental intervention \cite{oswald2020psychological, howard2021digital}. According to parental mediation theory, parents mitigate these risks through restrictive mediation (e.g., setting limits), active mediation (e.g., discussing content), and co-use (e.g., shared use of technology) \cite{valkenburg1999developing}. Modern technologies like video games, location-based games (\textit{e.g.,} Pokemon Go), and conversational agents (\textit{e.g.,} Alexa) also require parents to adapt their mediation strategies to ensure responsible use \cite{valkenburg1999developing, nikken2006parental, sobel2017wasn, beneteau2020parenting, yu2024parent}. Overall, parents seek to leverage technology to support their children's learning due to ite effectivenss but are also mindful of its risks. Their involvement is therefore driven by both opportunities and concerns, highlighting the need to design tools that effectively involve parents to balance benefits and risks.

\subsection{Generative AI and Companion Robots for Parenting and Education}
Generative AI and companion robots offer human-like affordances, with AI simulating human intelligence and robots providing physical human-like features. Compared to conventional models (\textit{e.g.,} machine learning) and devices (\textit{e.g.,} laptops), these emerging technologies enable natural and social interactions, creating opportunities for novel paradigms to enhance parental involvement and children's learning while introducing their unique challenges.

\subsubsection{Generative AI}
GAI offers promising support for parents by enhancing their ability to educate and engage with their children. Prior work suggested that AI-driven systems can support parenting education \cite{petsolari2024socio} and provide evidence-based advice through applications and chatbots, delivering micro-interventions such as teaching parents how to praise their children effectively \cite{davis2017parent, entenberg2023user} or offering strategies to teach complex concepts \cite{mogavi2024chatgpt, su2023unlocking}. Many parents also prefer using GAI to create educational materials tailored to their children's needs, rather than granting children direct access to these tools \cite{han2023design}. Beyond educational support, AI-based storytelling tools address practical challenges (\textit{e.g.,} time constraints) by alleviating physical labor while fostering parent-child interactions \cite{sun2024exploring}. Furthermore, GAI offers advantages to children's learning directly. It can help create personalized learning experiences by providing timely feedback and tailoring content \cite{su2023unlocking, mogavi2024chatgpt, han2024teachers}, enhancing positive learning experiences \cite{jauhiainen2023generative}. For example, a LLM-driven conversational system can teach children mathematical concepts through co-creative storytelling, achieving learning outcomes similar to human-led instructions \cite{zhang2024mathemyths}.

Despite these benefits, several concerns persist regarding the use of GAI in education. Prior work highlighted the limitations of GAI, such as its limited effectiveness in more complex learning tasks,the limited quality of the training data, and its inability to offer comprehensive educational support \cite{su2023unlocking}. There is also a significant risk of GAI producing inaccurate or biased information, discouraging independent thought among children, and threatening user privacy \cite{su2023unlocking, han2023design, han2024teachers}. Many parents are skeptical about the role of AI in their children's academic processes, concerned about the accuracy of AI-generated content, and worry that over-reliance on AI could stifle independent thinking \cite{han2023design}.

%\todo{might need to add some structural transition here}
\subsubsection{Social companion robots}
Social companion robots have proven potential to assist parents in home education settings through studies in \textit{parent-child-robot} interactions. \citet{gvirsman2020patricc} showed that the robotic system, \texttt{Patricc}, fostered more triadic interaction between parents and toddlers than a tablet, and \citet{gvirsman2024effect} found that, in a parent-toddler-robot interaction, parents tend to decrease their scaffolding affectively when the robot increases its scaffolding behavior. Similarly, \citet{chen2022designing} found that social robots enhanced parent-child co-reading activities, while \citet{chan2017wakey} demonstrated that the WAKEY robot improved morning routines and reduced parental frustration. Beyond educational support, \citet{ho2024s} uncovered that parents envisioned robots as their \textit{collaborators} to support their children's learning at home and that their collaboration patterns can be determined by the parents' capability, availability, and motivation. Although parents generally have positive attitudes toward incorporating robots into their children's learning, they remain concerned about the risk of disrupting school-based learning and potential teacher replacement \cite{tolksdorf2020parents, lee2008elementary, louie2021desire}.

In addition to parental support, social companion robots also support children in education directly through \textit{child-robot interactions}. Physically embodied robots provide adaptive assistance and verbal interaction similar to virtual or conversational agents \cite{ramachandran2019personalized, leyzberg2014personalizing, schodde2017adaptive, brown2014positive}, yet they foster greater engagement with the physical environment and encourage more advanced social behaviors during learning \cite{belpaeme2018social}, leading to improved learning outcomes \cite{leyzberg2012physical}. Prior work demonstrated that companion robots can effectively support both school-based learning (\textit{e.g.,} math \cite{lopez2018robotic}, literacy \cite{kennedy2016social, gordon2016affective}, and science \cite{davison2020working}) and home-based learning activities (\textit{e.g.,} reading \cite{michaelis2018reading, michaelis2019supporting}, number board games \cite{ho2021robomath}, and math-oriented conversations with parents \cite{ho2023designing}). For example, \citet{kennedy2016social} suggested that children can learn elements of a second language from a robot in short-term interactions, and \citet{tanaka2009use} found that children who took on the role of teaching the robot gained confidence and improved learning outcomes.

%\todo{may need to make this a separate section and explain why we propose AI-assisted robots}

% \subsubsection{Research Gap}
Parental involvement in early education is crucial and AI-assisted robots can offer promising support by helping parents overcome practical barriers (\textit{i.e.,} time, energy, and skills) and addressing concerns about technological risks. Yet, limited research has examined how technology design can simultaneously alleviate these barriers and concerns. Though \citet{zhang2022storybuddy} emphasized the importance of flexible parental involvement during reading through a system called \textit{Storybuddy}, yet they focused on a virtual chatbot rather than a physical robot, and how the flexible modes may be used in different scenarios remain unknown. Similarly, \textit{ContextQ} \cite{dietz2024contextq} presented auto-generated dialogic questions to caregivers for dialogic reading, but primarily considered situations where parents are actively involved, not scenarios where parents cannot participate fully.

In this work, we address these gaps by exploring parental involvement contexts, understanding parents' perceptions of AI-generated content, and examining how parents collaborate with AI and robots under different scenarios. In the following sections, we describe our development of  \texttt{SET}, a card-based activity, to understand parental involvement contexts (Section~\ref{sec-card}), the design of the \texttt{PAiREd} system to enable parents to co-create learning activities with an LLM (Section~\ref{sec-system}), and user study aimed to discover use patterns and understand user perceptions of the system (Section~\ref{sec-study}).
\section{System Design Focus}
\label{sec:system_design}
Our goal is to finalize a system that assists blind people in exploring an indoor environment independently.
In this section, we describe the key design elements of the system.

\begin{figure*}
    \centering
    \includegraphics[width=1\linewidth]{figure/device.png}
    \caption{Image of the robot and handle interface used in the study. Panel A-1 shows the robot used in the formative study, while Panel A-2 presents the robot used in the main study. Panels B-1 and B-2 illustrate the mapping of the handle interface buttons' functions, depending on the selected navigation mode.}
    \label{fig:device&ui}
    \Description{The image consists of four panels labeled A-1, A-2, B-1, and B-2. Panels A-1 and A-2 display a suitcase-shaped device used in the robot, while Panels B-1 and B-2 illustrate the button mapping of the handle interface for navigation. Panel A-1 depicts the initial robot used in the formative study. The robot is red and resembles a suitcase. The panel outlines six key components: The handle interface is located where a typical suitcase handle would be and includes five buttons. A smartphone is mounted on the back of the handle using a mounting device. A touch sensor, positioned under the handle, detects when users are touching it. Three RGBD sensors, situated on the front of the robot, are used for depth and color sensing to assist in obstacle detection and navigation. A 360-degree LiDAR sensor is mounted on top of the robot, on top of the three cameras. Motorized wheels at the base provide mobility, allowing the robot to move autonomously or in response to user input. Panel A-2 shows the updated robot used in the main study. The most significant change is the addition of a 1080p-resolution fisheye camera, positioned near the handle to capture images from a higher point of view. Panel C-1 illustrates the button mapping when the robot is in Automated Navigation Mode: The up button increases speed, and the right button decreases it. The left and right buttons adjust the level of detail. A long press on the middle button triggers conversation mode, while a single press switches to manual control mode. Panel B-2 displays the button mapping when the robot is in Manual Control Mode: The up button moves the robot forward, the back button moves it backward, the left button turns it left, and the right button turns it right. A long press on the middle button triggers conversation mode, while a single press returns the robot to automated navigation mode.
    }
\end{figure*}

\subsection{Device}
Assistance systems for blind people have been proposed in various devices, such as smartphones~\cite{presti2019watchout}, \red{handheld haptic devices~\cite{spiers2016outdoor,choiniere2016development,liu2021tactile},} wearable devices~\cite{li2016isana}, cane-like devices~\cite{ranganeni2023exploring} and robots~\cite{liu2024dragon}.
Each type of device offers unique advantages - Smartphones \red{and handheld haptic devices are portable; Smartphones} are also widely used by blind people~\cite{morris2014blind,martiniello2022exploring}; Wearable devices free the user's hands~\cite{lee2014wearable}; Cane-like devices resemble traditional canes~\cite{ranganeni2023exploring}; And robots are able to autonomously guide users~\cite{guerreiro2019cabot}.
\red{While handheld devices~\cite{presti2019watchout,spiers2016outdoor,choiniere2016development,liu2021tactile,ranganeni2023exploring} have often been used due to their portability, in exploration scenarios, they require users to point the devices in their directions of interest while navigating around unfamiliar locations and obstacles, which involve high cognitive load.
Thus}, we chose robots because of their autonomous navigation and obstacle avoidance capabilities. 
This allows users to concentrate on learning the environment~\cite{cai2024navigating,zhang2023follower,jain2023want}. 
In particular, we adopted a wheeled robot~\cite{guerreiro2019cabot,zhang2023follower,wang2022can}.
While wheeled robots are unable to navigate stairs like quadruped robots~\cite{cai2024navigating}, blind users often find wheeled robots more suitable due to their silence and stability~\cite{wang2022can}.
Our assumption is that the devices should ensure the users' safety during navigation and allow users to focus on exploration. 
As a result, the findings in our study can be extended to any similar devices other than wheeled robots. 

\subsection{Describing Scenes}
Previous navigation systems relied on hardcoded information~\cite{sato2019navcog3,Kaniwa2024ChitChatGuide} or simple image captioning models~\cite{saha2019closing} to provide scene descriptions. 
They only convey information related to navigating to destinations. 
In exploratory tasks, any information and details could be relevant. 
Therefore, we decided to use MLLM, a foundational model capable of recognizing a variety of objects and describing them in natural language. 
We injected MLLM into the system to periodically provide comprehensive information about the surroundings to inform blind users during exploration. 
In this paper, we investigate the appropriate presentation format, such as content types and lengths, and the quality of the responses from MLLMs through our user studies.

\subsection{Interaction} % This section is so hard to write... Plz gimme idea if there is any better way
The ability for users to select destinations and routes according to their interests, often referred to as autonomy, is particularly important for exploration~\cite{Kaniwa2024ChitChatGuide,kayukawa2022HowUsers}. 
In our system, to what extent users prefer to take control over the robot (\ie, interaction) remains unknown.
Based on the scene descriptions given by the system~\cite{Kaniwa2024ChitChatGuide}, some blind users may fully embrace letting the robot guide them automatically, while others may prefer to decide which way to go on their own.
Additionally, this preference may also be influenced by the robot's descriptions of the scenes. 
Given that the extent of user preference for autonomy remains unclear, we first conducted the formative study (Sec.~\ref{sec:study1}) to explore the requirement of autonomy based on interaction needs. 
Then, we conducted a full study (Sec.~\ref{sec:study2}) to evaluate the users' opinions on autonomy in our improved system, which integrated the feedback from the formative study.

\section{Experimental Setup}\label{sec:experimental_setup}

\begin{figure}
    \centering
    \begin{overpic}[width=\columnwidth]{./figures/experimental_setup/experimental_setup.pdf}
        \put(28,55){(a)}
        \put(36,38){(b)}
    \end{overpic}
    \caption{Overview of the experimental setup for (a) wireless and (b) classical data readout. Transmitter and receiver placement are identical for both scenarios.}
    \vspace{-5mm}
    \label{fig:experimental_setup}
\end{figure}

To \revise{quantify} the effect of artificially enhanced capacitive coupling for classical \ac{DAQ} and compare it against wireless \ac{DAQ}, three measurement series for each of the two scenarios have been conducted.
One \textit{BodySense} system has been equipped with an Rx carrier placed on the test subject's upper wrist. 
An adhesive Ag/AgCl wet-gel electrode from \textit{TIGA-MED} with a diameter of \qty{48}{\milli\meter} acts as skin-electrode, ensuring a low resistive contact to the body and mechanically holds the system at its position.
The system ground plane and the battery act together as a floating electrode, closing the return path over the environment and earth-ground.
A second \textit{BodySense} system with a Tx carrier extension has been sequentially placed at distances of \qty{10}{\centi\meter}, \qty{30}{\centi\meter}, and \qty{50}{\centi\meter} to the receiver, utilizing the same Ag/AgCl wet-gel electrode as skin-electrode.
\autoref{fig:experimental_setup} provides an overview of the transmitter and receiver positions during the experiments.
Distances below \qty{10}{\centi\meter} have not been investigated as the inter-device coupling significantly strengthens the return path, making it comparable to the forward path loss \cite{yang_2022}.
For each distance, a frequency sweep from \qty{4}{\mega\hertz} up to \qty{64}{\mega\hertz} has been performed.
All measurements have been conducted in a standing position in a laboratory environment while being at least \qty{1}{\meter} away from walls and other equipment.
The arm was outstretched to the side, forming an angle of \qty{90}{\degree} between the arm and the torso's side. 
During the data collection, the subject stood still and kept the arm in a constant position to minimize movement-induced fluctuations in the environmental-coupled return path.

In the classical scenario, the receiver was directly connected over a USB Type-C cable, forwarding the collected data over USB to the plugged-in computer, see \autoref{fig:experimental_setup} (a).
For the wireless scenario, the USB Type-C cable has been removed, and the computer has been unplugged from the grid. The data is wirelessly forwarded via \ac{BLE} to the computer, as depicted in \autoref{fig:experimental_setup} (b).
Finally, the channel gain is calculated to measure the quality of the capacitive \ac{HBC} communication channel, and the received power has been extracted to set capacitive \ac{HBC} into context to conventional \ac{RF} solutions.

It is essential to carefully consider the effects of signal transmission through the human body and respect its safety limits in \ac{HBC} systems. The \ac{ICNIRP} sets limits for non-ionizing radiation exposure to the human body, defining different dosimetric quantities depending on the frequency. In the frequency range between \qty{1}{\hertz} and \qty{100}{\kilo\hertz} the current density is used as limiting metric \cite{icnirp_low} and above \qty{100}{\kilo\hertz} the \ac{SAR}-value \cite{icnirp_high}.
Further, the IEEE standard C95.1-2005 defines safety levels on human exposure to \ac{RF} electromagnetic fields \cite{c95.1-2005}.
By limiting the transmit power to maximal \qty{5}{\dbm} for our experiments, we ensured to meet the above-mentioned safety standards.
\section{Results}\label{Sec:Results}

The system’s performance was evaluated across key metrics, including latency, frame rate, and resolution. Latency was measured for the key components of the virtual human assistant interaction: STT exhibited a latency of $46 \pm 5$ ms, the LLM processing took $552 \pm 187$ ms, and the TTS synthesis had a latency of $1281 \pm 188$ ms.
The visual output was rendered at a resolution of $4128 \times 2208$ on the HMD, with frame rate recorded to assess the visual fluidity of each visualization modality. The \textbf{AR-VG} visualization maintained a consistent average frame rate of 72 FPS. Both \textbf{\revised{AV}-VG} and \textbf{FV-VG} operated at an average frame rate of 36 FPS.


\subsection{Stress Level}

\begin{figure}
    \centering
    \includegraphics[width=\columnwidth]{figures/hrv.png}
    \caption{\textbf{HRV during the Resting and Execution Phases.} In \textbf{RUS}, the RMSSD shows the steepest drop between the two phases, indicating a higher stress level compared to the proposed visualizations. \textbf{AR-VG} and \textbf{\revised{AV}-VG} perform similarly, while \textbf{FV-VG} exhibits highest RMSSD value and the smallest change between phases, suggesting that less stress is induced during the execution.}
    \label{fig:hrv}
\end{figure}

To assess stress levels during the robotic ultrasound procedure, we derived HRV from the ECG sensor data, focusing on the Root Mean Square of the Successive Differences (RMSSD), a commonly used measure of stress~\cite{shaffer2017overview}. Lower RMSSD values generally indicate higher stress levels. The analysis was performed using the HeartPy~\cite{van2019heartpy} Python package.
We analyzed HRV during two phases of the procedure: the resting phase, where the robot remained stationary and participants were free to interact with the virtual agent, and the execution phase, during which the robot performed the ultrasound scan. The HRV data for these phases are shown in Fig.~\ref{fig:hrv}.

Given the non-normal distribution of the data observed by the Shapiro-Wilk test, we used the Wilcoxon Signed-Rank Test for within-condition comparisons, assessing differences between the resting and execution phases for each visualization method. Although we observed a trend of lower RMSSD values during the execution phase compared to the resting phase, in \textbf{RUS} ($z = 25.0, p = 0.846\add{, d = 0.291}$), \textbf{AR-VG} ($z = 13.0, p = 0.547\add{, d = 0.141}$), \textbf{\revised{AV}-VG} ($z = 38.0, p = 0.970\add{, d = 0.180}$), and \textbf{FV-VG} ($z = 28.0, p = 0.700\add{, d = 0.032}$), the results did not indicate significance.
To compare HRV across the different visualization methods during both the resting and execution phases, we employed the Kruskal-Wallis Test. The analysis for the resting phase showed no significant difference in HRV across the visualization methods ($H = 0.485, p = 0.922\add{, \eta^2 = 0.012}$). During the execution phase, the test also yielded no significant difference between methods ($H = 3.430, p = 0.330\add{, \eta^2 = 0.086}$).



\subsection{Subjective Ratings}

\begin{figure*}[t]
    \centering
    \begin{subfigure}[b]{0.325\textwidth}
        \centering
        \includegraphics[width=\textwidth]{figures/hri.png}
        \caption{Trust in Human Robot Interaction}
        \label{fig:hri}
    \end{subfigure}
    \hfill % optional; add some horizontal spacing
    \begin{subfigure}[b]{0.325\textwidth}
        \centering
        \includegraphics[width=\textwidth]{figures/sus.png}
        \caption{System Usability Score}
        \label{fig:sus}
    \end{subfigure}
    \hfill % optional; add some horizontal spacing
    \begin{subfigure}[b]{0.325\textwidth}
        \centering
        \includegraphics[width=\textwidth]{figures/tlx.png}
        \caption{Perceived Workload}
        \label{fig:tlx}
    \end{subfigure}
    \caption{\textbf{Subjective Measurements for Trust Score, Usability, and Workload.} All proposed immersive visualizations with the conversational agent significantly increase the HRI trust score compared to \textbf{RUS}. \textbf{AR-VG} receives the highest trust score, the best usability, and the lowest workload among all methods. Statistical significance is indicated as $\star \left( p<0.05 \right)$, $\star \star \left( p<0.01 \right)$, and $\star \star \star \left( p<0.001 \right)$.}
    \label{fig:subjective}
\end{figure*}

HRI Trust scores under each condition for the robotic ultrasound were as follows: \textbf{RUS} ($M = 3.12, SD = 0.62$), \textbf{AR-VG} ($M = 4.33, SD = 0.42$), \textbf{\revised{AV}-VG} ($M = 4.29, SD = 0.38$), and \textbf{FV-VG} ($M = 4.06, SD = 0.68$). The results are visualized in Fig.~\ref{fig:hri}.
Statistical analysis using the Friedman test revealed a significant difference in trust scores across the visualization methods ($\chi^2(3) = 26.95, p = 6.02 \times 10^{-6}$). Post-hoc Dunn-Sid{\'a}k  pairwise comparisons further emphasized these differences. Significant differences were observed between \textbf{RUS} and \textbf{AR-VG} ($p = 0.000316\add{, d = 2.272}$), \textbf{RUS} and \textbf{\revised{AV}-VG} ($p = 0.00035\add{, d = 2.272}$), and \textbf{RUS} and \textbf{FV-VG} ($p = 0.012\add{, d = 1.428}$).

The SUS scores for each condition\revised{, normalized to a 0-1 scale,} are shown in Fig.~\ref{fig:sus}. A Friedman test revealed a significant difference in usability across the visualization methods ($\chi^2(3) = 16.60, p = 0.000854$). Post-hoc Dunn-Sid{\'a}k  pairwise comparisons indicated that the significant difference lies between \textbf{RUS} and \textbf{AR-VG} ($p = 0.037\add{, d = 1.343}$).

The NASA-TLX scores, \add{normalized to a 0-1 range}, are presented in Fig.~\ref{fig:tlx}. A Friedman test revealed a significant difference in task load across the visualization methods($\chi^2(3) = 9.03, p = 0.02$).
Although there was a tendency for both \textbf{AR-VG} and \textbf{\revised{AV}-VG} to show lower task load scores compared to \textbf{RUS}, Dunn-Sid{\'a}k  pairwise comparisons did not reveal any statistically significant differences between the visualization methods.


\subsection{User Preference and Feedback}

%\begin{figure}
%    \centering
%    \includegraphics[width=\columnwidth]{figures/rank.png}
%    \caption{\textbf{Preference Ranking.} \textbf{AR-VG} was the most preferred, followed by \textbf{\revised{AV}-VG}, \textbf{FV-VG}, and \textbf{RUS}.}
%    \label{fig:rank}
%\end{figure}


The results showed that \textbf{AR-VG} was the most preferred visualization, with 72$\%$ of participants ranking it as their top choice, 14$\%$ ranking it second, and 14$\%$ ranking it third. \textbf{\revised{AV}-VG} followed, with 21$\%$ of participants ranking it as the most preferred, 43$\%$ ranking it second, and 36$\%$ ranking it third. For \textbf{FV-VG}, 36$\%$ of participants ranked it in their top three choices. Finally, no participants ranking \textbf{RUS} as their first choice. However, 22$\%$ ranked it second, 42$\%$ ranked it third, and 36$\%$ ranked it as their least preferred visualization.

The qualitative feedback from participants provided further insight into their preferences. Participants in general appreciated the conversational abilities of the virtual assistant across \textbf{AR-VG}, \textbf{\revised{AV}-VG} and \textbf{FV-VG}. They noted that talking to the avatar felt natural and gave them more control over the procedure. In addition, several participants remarked that the hand animation of the virtual assistant taking control of the probe “made me trust the system more.” However, due to technical limitation, the avatar’s hand was not visible in the \textbf{\revised{AV}-VG} passthrough window, which led to some confusion about the interaction.
Concerns about the accuracy of VR visualizations were also raised. Participants noted that due to tracking error, sometimes misalignment between their real and virtual arms in \textbf{FV-VG} caused uncertainty about the success of the scan. Participants raised concerns about the robot’s actions, particularly when they could not see the real robot.
%, leading to uncertainty about the procedure.

Overall, the feedback indicated that participants favored the visualizations that offered a balance between immersion and real-world visibility and integrating a friendly, responsive avatar can improve patient trust and comfort in robotic ultrasound procedures.

\vspace{-3.5mm}
\section{Conclusion}
\vspace{-2mm}

In summary, we introduce a first-order proximal algorithm to solve the perspective relaxation of cardinality-constrained GLM problems.
By leveraging the problem’s unique mathematical structure, we design a customized PAVA to efficiently evaluate the proximal operator, ensuring scalability to high-dimensional settings.
Further acceleration is achieved through an efficient value-based restart strategy and compatibility with GPUs, which collectively enhance convergence rates and computational speed.
Extensive empirical results demonstrate that our method outperforms state-of-the-art solvers by 1-2 orders of magnitude, establishing it as a practical, high-performance component for integration into next-generation MIP solvers.

\newacronym{rl}{RL}{Reinforcement Learning}
\newacronym{drl}{DRL}{Deep Reinforcement Learning}
\newacronym{mdp}{MDP}{Markov Decision Process}
\newacronym{ppo}{PPO}{Proximal Policy Optimization}
\newacronym{sac}{SAC}{Soft Actor-Critic}
\newacronym{epvf}{EPVF}{Explicit Policy-conditioned Value Function}
\newacronym{unf}{UNF}{Universal Neural Functional}                                   

%---------------------------------------------
% Acknowledgement
%---------------------------------------------
\section*{ACKNOWLEDGMENT}
This work was found by the Swiss National Science Foundation SNSF under the projects “BodyLink: Enabling Battery-free body-worn Sensing and Communication with Energy Transfer” (Grant Nr. 220867) and “Wearable Nano-Opto-electro-mechanic Systems” (Grant Nr. 209675).
% https://data.snf.ch/grants/grant/220867
% https://data.snf.ch/grants/grant/209675
%---------------------------------------------
% Bibliography
%---------------------------------------------
\bibliographystyle{IEEEtranDOI} 
%\bibliographystyle{IEEEtran} 
\input{bib/bodycom.bbl}
% \bibliography{bib/bodycom}

%---------------------------------------------
\end{document}

%% Paper end
