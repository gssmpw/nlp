\documentclass[conference]{IEEEtran}
\IEEEoverridecommandlockouts
% The preceding line is only needed to identify funding in the first footnote. If that is unneeded, please comment it out.
%Template version as of 6/27/2024

\usepackage{cite}
\usepackage{amsmath,amssymb,amsfonts}
\usepackage{algorithmic}
\usepackage{graphicx}
\usepackage{textcomp}
\usepackage{xcolor}
\usepackage{makecell}

%---------------------------------------------
% Custom packages
%---------------------------------------------
\usepackage{siunitx}
\usepackage{hyperref}
\usepackage[nolist]{acronym}
\usepackage[flushleft]{threeparttable}
\usepackage{placeins}
\usepackage{svg}                                   
\usepackage[percent]{overpic}

\usepackage{mathtools}
%\usepackage{multirow}
\usepackage{booktabs}
%\usepackage{subcaption}
\usepackage[nolist]{acronym}
\usepackage{pbalance} %balances last page
\usepackage[export]{adjustbox}

\usepackage{eso-pic}    % watermark

\def\BibTeX{{\rm B\kern-.05em{\sc i\kern-.025em b}\kern-.08em
    T\kern-.1667em\lower.7ex\hbox{E}\kern-.125emX}}
%---------------------------------------------
% Custom Commands
%---------------------------------------------
\newcommand{\luca}[1]{{\textcolor{red}{#1}}}       % Custom boss comments
\newcommand{\michele}[1]{{\textcolor{orange}{#1}}}  % Custom boss comments
\newcommand{\revise}[1]{{\textcolor{black}{#1}}}

\newcommand{\todo}[1]{\textcolor{red}{#1}}          % Custom command to highlight todos
\newcommand{\done}[1]{\textcolor{green}{#1}}        % Custom command to highlight done
\newcommand{\idea}[1]{\textcolor{violet}{#1}}       % Custom command to highlight ideas
\newcommand{\RNum}[1]{\uppercase\expandafter{\romannumeral #1\relax}}   % To print roman digits: I, II, III, IV etc.
\newcommand{\bfqty}[2]{\text{\bfseries\SI{#1}{#2}}}

% Re-define the command \IEEEauthorrefmark{x} to get numbers instead of symbols
\DeclareRobustCommand{\IEEEauthorrefmark}[1]{\smash{\textsuperscript{\footnotesize #1}}}

% Decalare custom SI units
\DeclareSIUnit\db{dB}                           % Decbel
\DeclareSIUnit\dbi{dBi}                         % Isotropic antenna gain in decibel
\DeclareSIUnit\dbm{dBm}                         % Power in decibel
\DeclareSIUnit\watthour{Wh}                     % Whatthour without space between W and h
\DeclareSIUnit\mbps{Mbps}                       % Mega bit per second
\DeclareSIUnit\kbps{kbps}                       % kilo bit per second
\DeclareSIUnit\bps{bps}                         % bit bit per second
\DeclareSIUnit\msInference{ms/inference}        % inference time 

%---------------------------------------------
% Title
%---------------------------------------------
\begin{document}
\bstctlcite{IEEEexample:BSTcontrol} %used tp shortn author list (needs some text in the bib file too)


\AddToShipoutPictureBG*{
  \AtPageUpperLeft{%
    \put(0,-40){\raisebox{15pt}{\makebox[\paperwidth]{\begin{minipage}{21cm}\centering
      \textcolor{gray}{This work has been submitted to the IEEE for possible publication. Copyright may be transferred without notice, after which this version may no longer be accessible.
      } 
    \end{minipage}}}}%
  }
  \AtPageLowerLeft{%
    \raisebox{25pt}{\makebox[\paperwidth]{\begin{minipage}{21cm}\centering
      \textcolor{gray}{This work has been submitted to the IEEE for possible publication. Copyright may be transferred without notice, after which this version may no longer be accessible.
      }
    \end{minipage}}}%
  }
}

\title{BodySense: An Expandable and Wearable-Sized Wireless Evaluation Platform for Human Body Communication}

%---------------------------------------------
% Author Section
%---------------------------------------------
\author{
\IEEEauthorblockN{
Lukas Schulthess, 
Philipp Mayer, 
Christian Vogt, 
Luca Benini,
Michele Magno
}\\

\IEEEauthorblockA{Dept. of Information Technology and Electrical Engineering, ETH Z{\"u}rich, Switzerland}
%\IEEEauthorblockA{\IEEEauthorrefmark{2} Dept. of Electrical, Electronic and Information Engineering, University of Bologna, Italy}
%\IEEEauthorblockA{\IEEEauthorrefmark{1} Center for Project Based Learning, ETH Z{\"u}rich, Z{\"u}rich, Switzerland}
%\IEEEauthorblockA{\IEEEauthorrefmark{2} Integrated Systems Laboratory, ETH Z{\"u}rich, Z{\"u}rich, Switzerland}
%\IEEEauthorblockA{\IEEEauthorrefmark{3} DEI, University of Bologna, Bologna, Italy}
}

\maketitle
\IEEEpeerreviewmaketitle

% Main Section
%---------------------------------------------
% Manuscripts should consist of a complete description of the proposed technical content and research results, spanning a typical range of 4 to a maximum of 6 pages. Submissions that do not comply with this requirement may be subject to automatic rejection
% 150words
% Replying to workplace emails that are typically long and require politeness is time-consuming and cognitively demanding.
% Replying to lengthy and polite workplace emails is often time-consuming and cognitively demanding.
% takes time to understand and reply
\red{Replying to formal emails is time-consuming and cognitively demanding, as it requires crafting polite phrasing and providing an adequate response to the sender's demands.}
% \red{Replying to formal emails, which often takes time to understand and require polite phrasing, is time-consuming and cognitively demanding.}
Although systems with Large Language Models (LLM) were designed to simplify the email replying process, users still need to provide detailed prompts to obtain the expected output.
Therefore, we proposed and evaluated an \red{LLM-powered question-and-answer (QA)-based approach} for users to reply to emails by answering a set of simple and short questions generated from the incoming email.
We developed a prototype system, \textit{ResQ}, and conducted controlled and field experiments with 12 and \red{8} participants.
Our results demonstrated that \red{the QA-based approach} improves the efficiency of replying to emails and reduces workload while maintaining email quality, compared to a conventional prompt-based approach that requires users to craft appropriate prompts to obtain email drafts.
We discuss how \red{the QA-based approach} influences the email reply process and interpersonal relationship dynamics, as well as the opportunities and challenges associated with using a QA-based approach in AI-mediated communication.

% original
% Replying to lengthy and polite workplace emails is often time-consuming and cognitively demanding.
% Although systems with Large Language Models were designed to simplify the email replying process, users still needed to provide detailed prompts to obtain the expected output.
% Therefore, we proposed and evaluated a question-and-answer-based approach for users to reply to emails by answering a set of simple and short questions generated from the incoming email.
% We developed a prototype system, \textit{ResQ}, and conducted both controlled and field experiments with 12 and 9 participants.
% Our results demonstrated that ResQ improves the efficiency of replying to emails and reduces workload while maintaining email quality compared to a conventional prompt-based approach that requires users to craft appropriate prompts to obtain email drafts.
% We discuss how ResQ influences the email reply process and interpersonal relationship dynamics, as well as the opportunities and challenges associated with using a QA-based approach in AI-mediated communication.
\vspace{10pt}
\begin{IEEEkeywords}
Human body communication (HBC), capacitive HBC, wireless data acquisition, energy-efficient communication, body sensor networks
\end{IEEEkeywords}


\section{Introduction}

\begin{figure*}
    \centering
    \includegraphics[width=\textwidth]{figures/Introduction.pdf}
    \caption{Showing the novel problem statement applied to traffic prediction use case. Multiple unstructured observations from the past are used to reconstruct a hidden traffic state from which a full traffic state is forecast with a set of query locations. }
    \label{fig:intro}
\end{figure*}

% Was sagen denn die anderen warum Traffic Prediction gut ist? 
Forecasting the traffic in the near future is an important task for city management.
Data from the near past is used to predict future traffic states with spatio-temporal Graph Neural Networks \cite{bui22}.
Accurate prediction provides the opportunity to optimize traffic flow, reduce traffic jams and increase air quality \cite{Po19}.

% Wieso ist Sparsity in allen Dimensionen wichtig.
While traffic prediction relies on the availability of data from traffic sensors, there exists a plethora of reasons why sensors may stop working temporarily, such as simple errors, energy saving, or overloaded communication systems.
Considering small- or medium-sized cities, the coverage of sensors may be low because the sensors are too expensive or not available.
Also, the sensors are typically static and do not adapt to changes in the traffic flow (e.g. caused by a construction site), which motivates moving sensors that for example could be mounted on cars. 
However, both missing and moving sensors introduce sparsity, since measurements may not be available for all locations at all times.
This sparsity must be explicitly addressed in traffic prediction for a realistic application scenario, which is illustrated in figure \ref{fig:intro}.
From one hour of data on Sunday morning, only few observations of the traffic state are available at each timestep.
The number of observations may differ throughout the observed time and the observation itself can be distributed arbitrarily in the city. 
We assume a relatively low number of sensors to account for resource saving and sensor failure in our proposed framework SUSTeR.
The task is to predict the dense traffic state one timestep after the observations at all possible sensor locations.
We study this problem on the traffic dataset Metr-LA and PEMS-BAY to test our assumption that only a fraction of the sensor values would be enough for good predictions.
By modifying an existing traffic dataset, we are able to compare our results from very sparse observations to the bottom line with all information available.
A successful study will provide insights in how sensors in new cities can be reduced before installing them and further mobile sensors would save more resources and are able to adapt to new traffic situations.
We argue that in order to be adaptable to other cities and changes in traffic flows, prior information like the road network should be neglected and just the sparse observations considered.
This comes with the added benefit of making our solution applicable in regions where no openly available road network is maintained or pathways change frequently (e.g. flood areas, animal observations). 


The aforementioned problem is novel and more challenging than the commonly considered traffic prediction problem, since there exist very few observations in each input sample.
Current works for the traffic prediction problem do not consider any missing values. \cite{Li2021, Shao22}
A common method among state of the art approaches is the usage of Graph Neural Networks on graphs that model the sensor network \cite{bui22}.
The values of a sensor are applied to the same graph node for each timestep which prohibits any non-stationary sensors . 
With fixed sensor locations, the resulting sensor network is highly correlated with the road network.
Streets connecting two intersections with sensors should be also an interesting point for correlations in the sensor network.
However, variable observations and high temporal sparsity rates can not be modeled adequately in a static network.
We show in our experiments that the road network has only a small influence on the traffic predictions.

Besides the traffic prediction for future timesteps, some works explore the field of traffic speed imputation \cite{Cini22, Cuza22} where missing sensor values are predicted.
But the amount of missing values is assumed to be at most 80\%, which on average are still over 40 given sensors in each timestep in the Metr-LA dataset with a total of 207 sensors.
We consider up to 99.9\% missing values which are on average 2.4 observations in each timestep that are used as input.
Such high sparsity rates drastically decrease the chance that multiple values are present in one input sample from the same sensor location, which makes it challenging to recognize and learn temporal correlations for each location on its own.

High sparsity rates (>95\%) result in few sensor values, but if a reconstruction of the traffic state would be possible, we question if spatio-temporal graphs require nodes for each sensor.
In SUSTeR we utilize only a small amount of graph nodes for the encoding of information and do not relate such nodes to the sensor network.
We call this the hidden graph (see figure \ref{fig:intro}), which is still able to reconstruct the complete traffic state.
Due to the reduced number of nodes SUSTeR achieves faster runtimes, as shown in the experiments.
This hidden graph is not embedded directly in the spatial domain, which is why the assignment of observations, as well as the querying of the future traffic, is done with an encoder and a decoder, implemented as neural networks.
The decoding from the hidden graph to future values depends on a set of query locations.
Figure \ref{fig:intro} shows the query locations as given from outside and in combination with the reconstructed traffic state the future values are predicted.

To construct the hidden graph we encode observations from each timestep into from multiple graphs, one for each timestep. 
The graphs are created in a residual style and information is added to the node embeddings from the previous timesteps.
We choose this method to incorporate all timesteps equally into the hidden state because the redundant information along the past is non-existing for high sparsity rates.
From the sequence of graphs where our framework inserted the observations step by step we apply STGCN \cite{Yu18}, an algorithm for traffic prediction to find and learn the spatio-temporal correlations on our small number of graph nodes.
The first future timestep of the STGCN is our hidden graph in which the traffic state is reconstructed. 

% Recent work has an implicit embedding of the graph nodes into the spatial domain as the assignment from the sensor to graph node is fixed one by one.
% Because the graph has the same structure as the road network spatio-temporal correlations can be learned between those sensors.
% We reduce the number of nodes and use a non-linear assignment learned data-driven from the observations.

We find in the experiments that SUSTeR outperforms the plain STGCN and modern traffic prediction frameworks like D2STGNN for high sparsity rates $(\geq 99\%)$.
This is equivalent to only $0.2$ to $2.4$ observation for each timestep on average.
SUSTeR uses fewer parameters than the baselines and can train faster and with less training data.
Our main contributions can be summarized as follows:
\begin{itemize}
    \item We introduce a sparse and unstructured variant of the traffic prediction problem with sparsity in all dimensions. The sensors report only a fraction of their values and are arbitrarily distributed in the spatial domain.
    \item We propose SUSTeR, a framework around the STGCN architecture, which maps sparse observations onto a dense hidden graph to reconstruct the complete traffic state.
    Our code is available at github.\footnote{https://github.com/ywoelker/SUSTeR}
    \item We conducts experiments that show that SUSTeR outperforms the baselines in very sparse situations ($\geq 95\%$) and has a competitive performance in low sparsity rates.
    % \item SUSTeR trains a third faster than the next competitor.
\end{itemize}

\section{Related Work}
\label{sec:related_work}

\subsection{Robustness of Audio-Visual Speech Recognition} 

The robustness of AVSR systems has significantly advanced by integrating auditory and visual cues to improve speech recognition, especially in noisy environments. Conventional ASR methods have evolved from relying solely on audio signals \cite{schneider2019wav2vec, gulati2020conformer, baevski2020wav2vec, hsu2021hubert, chen2022wavlm, chiu2022self, radford2023robust} to incorporating visual data from speech videos \citep{makino2019recurrent}.
The multimodal AVSR methods \citep{pan2022leveraging, shi2022learning, seo2023avformer, ma2023auto} have enhanced robustness under audio-corrupted conditions, leveraging visual details like speaker's face or lip movements as well as acoustic features of speech. These advancements have been driven by various approaches, including end-to-end learning frameworks \citep{dupont2000audio, ma2021end, hong2022visual, burchi2023audio} and self-supervised pretraining \citep{ma2021lira, qu2022lipsound2, seo2023avformer, zhu2023vatlm, kim2025multitask}, which focus on audio-visual alignment and the joint training of modalities~\citep{zhang2023self, lian2023av, haliassos2022jointly, haliassos2024braven}.


Furthermore, recent advancements in AVSR highlight the importance of visual understanding alongside audio \citep{dai2024study, kim2024learning}. While initial research primarily targeted audio disturbances \citep{shi2022robust, hu2023hearing, hu2023cross, chen2023leveraging}, latest studies increasingly focus on the visual robustness to address challenges such as real-world audio-visual corruptions~\citep{hong2023watch, wang2024restoring, kim2025multitask} or modality asynchrony~\citep{zhang2024visual, fu2024boosting, li2024unified}. These efforts remark a shift towards a more balanced use of audio and visual modalities. Yet, there has been limited exploration in scaling model capacity or introducing innovative architectural designs, leaving room for further developments in AVSR system that can meticulously balance audio and visual modalities.



\subsection{MoE for Language, Vision, and Speech Models}

Mixture-of-Experts (MoE), first introduced by \citet{jacobs1991adaptive}, is a hybrid structure incorporating multiple sub-models, \ie experts, within a unified framework. The essence of sparsely-gated MoE \cite{shazeer2017outrageously, lepikhin2021gshard, dai2022stablemoe} lies in its routing mechanism where a learned router activates only a subset of experts for processing each token, significantly enhancing computational efficiency. Initially applied within LLMs using Transformer blocks, this structure has enabled unprecedented scalability \cite{fedus2022switch, zoph2022st, jiang2024mixtral, guo2025deepseek} and has been progressively adopted in multimodal models, especially in large vision-language models (LVLMs) \cite{mustafa2022multimodal, lin2024moellava, mckinzie2025mm1}.
Among these multimodal MoEs, \citet{zhu2022uni, shen2023scaling, li2023pace, li2024uni} and \citet{lee2025moai} share the similar philosophy to ours, assigning specific roles to each expert and decoupling them based on distinct modalities or tasks. These models design an expert to focus on specialized segments of input and enhance the targeted processing.

Beyond its applications in LLMs and LVLMs, the MoE framework has also been applied for speech processing \cite{you2021speechmoe, you2022speechmoe2, hu2023mixture, wang2023language}, where it has shown remarkable effectiveness in multilingual and code-switching ASR tasks. In addition, MoE has been employed in audio-visual models \cite{cheng2024mixtures, wu2024robust}, although they primarily focus on general video processing and not specifically on human speech videos. These approaches leverage MoE to model interactions between audio and visual tokens without directly processing multimodal tokens.
Our research advances the application of the MoE framework to AVSR by designing a modality-aware hierarchical gating mechanism, which categorizes experts into audio and visual groups and effectively dispatches multimodal tokens to each expert group. 
This tailored design enhances the adaptability in managing audio-visual speech inputs, which often vary in complexity due to diverse noise conditions.

\section{CavePI System Design}
The CavePI AUV design includes three major subsystems: sensory bay, computational bay, and locomotion bay. Our proposed system and its components are shown in Fig.~\ref{fig:system_design}.  
% \vspace{-1 mm}


\subsection{Sensor Bay: Acoustic-Optic Perception Subsystem}
The CavePI platform includes visual and acoustic sensors: a front-facing fisheye camera, a downward-facing low-light camera, and a Ping2 active sonar. The fisheye camera, housed within a transparent dome at the \textit{head} of the AUV, captures forward-facing visuals with a $160^\circ$ field-of-view (FOV) and outputs a video feed at $1920\times1080$ resolution. It is worth noting that the cylindrical enclosure is a $6''$ tube while the dome has a $4''$ diameter. A custom interface is built to connect the two; see section \ref{sub:Stability} for more details. The low-light camera, mounted inside the computational enclosure, captures downward-facing visuals with an $80^\circ \times 64^\circ$ FOV, also at the same resolution. Additionally, a Ping2 sonar altimeter-echosounder from Blue Robotics\texttrademark{} is mounted on the underside of the robot; the sonar has a range of $100$ meters, a depth rating of $300$ meters, and a resolution of $0.5\%$ of the range, allowing it to detect obstacles in the surrounding environment beneath CavePI. These sensory components collectively provide robust environmental awareness for autonomous navigation in challenging underwater environments.
%This camera provides critical imagery for tracking cavelines, enabling autonomous operation. 
% \vspace{-1mm}




\subsection{Computational Bay}
\vspace{-1 mm}
As illustrated in Fig.~\ref{fig:system_design}, the computational and electronic components of CavePI are housed within an acrylic cylindrical enclosure. This enclosure, with a thickness of $6.35$\,mm and a depth rating of 65 meters, forms the \textit{main body} of the robot, providing mechanical stability, buoyancy, and waterproof protection for the electronics. The computational elements include a Raspberry Pi-5, a Nvidia\texttrademark{} Jetson Nano, and a Pixhawk\texttrademark{} flight controller. The Jetson Nano is dedicated to processing visual data from the cameras, performing image processing tasks critical for scene perception and state estimation. The Raspberry Pi-5 manages planning and control modules, ensuring real-time underwater navigation. The Pixhawk flight controller acts as a bridge between hardware and software, receiving actuation commands from the Raspberry Pi-5 and transmitting them to the thrusters and lights via the MAVLink communication protocol. Additionally, the Pixhawk integrates a 9-DOF IMU, offering 3-axis gyroscope, accelerometer, and magnetometer measurements, which are used to calculate the attitude of CavePI during underwater operations.

The enclosure also contains the battery compartment, voltage regulators, electronic speed controllers (ESCs), and a Bar-30 pressure sensor. The battery compartment holds a $14.8$\,V ($18$\,Ah) battery pack, regulated to power internal components (\eg, cameras, computers) and external components (\eg, thrusters, sonar). Each thruster is controlled by an ESC, which drives the three-phase brushless motor using PWM signals from the Pixhawk. The Bar-30 sensor provides high-precision pressure readings with a resolution of $0.2$\,mbar and an accuracy of $2$\,mm, with a working depth of up to $300$ meters. This pressure data is processed to determine CavePI’s underwater depth, ensuring reliable and accurate interoceptive perception during operations.

%Furthermore, it houses the battery compartment, voltage regulators, electronic speed controllers (ESCs), and a Bar-30 sensor. The battery compartment features a $14.8$V, $18$Ah pack -- which is regulated to individual components inside (e.g. cameras, computers) and outside (e.g. thrusters, sonar) of the enclosure. The ESCs, one for each thruster, controls a three-phase brushless motor inside a thruster by receiving an input PWM signal from Pixhawk. The Bar-30 sensor provides pressure readings with a resolution of $0.2$\,mbar and $2$\,mm accuracy up to a working depth of $300$\,m.  This pressure data is then processed to compute the underwater depth of the CavePI ensuring robust and accurate interoceptive perception.



% \begin{figure*}[t]
%      \centering
%      \begin{subfigure}[]{0.49\textwidth}
%          \centering
%          \includegraphics[width=\linewidth]{figures/Fig4_Electronics.png}%
%          \vspace{-1.5 mm}
%          \caption{Major electronics and sensor-actuator connections.}
%          \label{fig:electronics}
%      \end{subfigure}~     
%      \begin{subfigure}[]{0.47\textwidth}
%          \centering
%          \includegraphics[width=\linewidth]{figures/Fig4_ROS.png}
%          \vspace{-1.5 mm}
%          \caption{Data flow of major computational modules in the form of \textit{ROS topics}: red and blue arrows represent \textit{subscribed} and \textit{published} topics, respectively.}
%          \label{fig:ROS}
%      \end{subfigure}
%      \vspace{-1 mm}
%         \caption{Simplified outlines of the end-to-end hardware, software, and ROS2 middleware integration of NemoGator are shown.}%
%      \label{fig:hw_mw_sw}
% \vspace{-4 mm}
% \end{figure*}




% \subsection{Locomotion Subsystem}
% \vspace{-1 mm}
% As opposed to traditional thruster-based AUV systems (eg, CUREE~\cite{girdhar2023curee}, ReefGlider~\cite{macauley2024reefglider}, LoCO~\cite{edge2020design}), NemoGator employs a bio-inspired propulsion system~\cite{zhang2010biologically} driven by three servo motors that actuate a \textit{caudal} tail (BCF) and two \textit{pectoral} fins (MPF). This design is inspired by the carangiform design~\cite{macias2024numerical,raj2016fish,costa2018design} (see Sec~\ref{sec:background}), combining the benefits of BCF and MPF for propulsion and maneuverability control, respectively~\cite{zhang2021development,marchese2013towards}. Specifically, the tail generates forward thrust and contributes to yaw motions, powered by a single servo motor; the pectoral fins ensures the stability of the system, regulating pitch within the water column, and assisting in roll and yaw motions~\cite{zhang2021design}. They are powered by respective servo motors, ensuring low-power operation with only three $35$\,Kg-cm torque ($7.4$V) motors. 

% When the fins are angled upward, the resulting hydrodynamic lift combined with tail propulsion, enabling ascent. Conversely, downward angling of the fins induces controlled descent, allowing depth modulation in the water column. We make sure that these fins oscillate synchronously for forward propulsion, while independent actuation of a single fin modulates yaw, enabling precise directional control. 

% The caudal fin of the NemoGator is directly connected to a servo motor to control the oscillations of the caudal fin. Forward movement is achieved through symmetric oscillations of the caudal fin around the NemoGator's longitudinal axis, while the yaw control is managed through asymmetric oscillations.
% cite:Towards a Self-contained Soft Robotic Fish: On-Board Pressure Generation and Embedded Electro-permanent Magnet Valves 

% The two side fins are engineered to replicate the functional dynamics of pectoral fins in fish, contributing to the stability of the system, regulating pitch within the water column, and assisting in roll and yaw motions. The pectoral fins of the NemoGator are directly connected to their respective servo motors, which are mounted on the outer body.%cite:Design and Locomotion Control of a Dactylopteridae-Inspired Biomimetic Underwater Vehicle With Hybrid Propulsion

% Specifically, oscillating only the left fin generates a yaw moment, inducing a left turn, whereas oscillating the right fin similarly facilitates a right turn.
% \vspace{-1mm}


 

% CavePI AUV is intended for low-power autonomous operation with portable ROS support for application-specific perception, planner, and control modules. As shown in Fig.~\ref{fig:ROS}, we incorporate SVIn (sonar-visual-inertial navigation)~\cite{rahman2022svin2} packages for general-purpose state estimation and waypoint-based trajectory planning. NemoGator can also be used as an underwater ROV with an optional tether-based TeleOp module. Depending on \textit{ROV mode} or \textit{autonomous mode}, a unified AutoPilot package is designed to control the servo motor commands for navigation.

\begin{figure}[t]
% \vspace{-1 mm}
    \centering
    \includegraphics[width=0.98\linewidth]{figures/Fig4_Electronics.png}%
    \vspace{-1mm}
    \caption{Major electronics and sensor-actuator connections of CavePI.}
    \label{fig:electronics}
    \vspace{-3mm}
\end{figure}


\subsection{Locomotion Bay: Middleware Integration}
\vspace{-1 mm}
The end-to-end integration of CavePI ensures that each computational component operates in sync, tied to a ROS2 Humble-based middleware backbone. The modular design also allows for future upgrades, ensuring that the CavePI can be tailored to meet evolving research in marine ecosystem exploration and monitoring. The sensor-actuator signal communication graph is illustrated in Fig.~\ref{fig:electronics}.


The CavePI autonomous underwater vehicle (AUV) is designed for low-power operation and integrates a portable ROS2 framework to support application-specific perception, planning, and control modules. As depicted in Fig.~\ref{fig:ROS}, the \textit{detector} node acquires visual data from the two cameras to identify the caveline for navigation. In the absence of GPS underwater, the system employs SVIn (sonar-visual-inertial navigation)~\cite{rahman2022svin2} packages to estimate the AUV’s position relative to the detected caveline. The \textit{mission planner} node then integrates the caveline information with the estimated position data to generate subsequent waypoints for the mission. Finally, the \textit{autopilot controller} node utilizes these waypoints, along with the detected caveline, positional data, and depth readings from the Bar-30 sensor, to generate precise actuation signals for the thrusters, enabling accurate movements and depth control. Additionally, CavePI can function as an ROV through an optional tether-based teleoperation module. This module transmits user commands from a joystick to the onboard Raspberry Pi-5, which processes the inputs and relays them to the thrusters for manual control.

\begin{figure}[t]
    \centering
    \includegraphics[width=\linewidth]{figures/Fig4_ROS.png}%
    \vspace{-1mm}
    \caption{Data flow among major computational modules of CavePI is shown in the form of \texttt{ROS Topics}: red and blue arrows represent \textit{subscribed} and \textit{published} topics in the ROS graph, respectively.}
    \label{fig:ROS}
    \vspace{-4mm}
\end{figure}


\subsection{CavePI Digital Twin}
%\vspace{-1 mm}
% The digital twin of CavePI is created in ROS, contains a similar structure as presented in Fig.~\ref{fig:ROS}. 
We develop a digital twin (DT) model of CavePI by using the Unified Robot Description Format (URDF), with links and joints carefully assigned to represent the various CAD components designed in SolidWorks. To replicate the sensor suite of the physical CavePI, Gazebo plugins are integrated to simulate the front-facing camera, down-facing camera, IMU, pressure sensor, and sonar. Additional plugins are employed to simulate environmental forces, including buoyancy, thrust, and hydrodynamic drag, thereby enhancing the physical realism.

A controlled open-water scenario is created in Gazebo to simulate realistic missions, featuring a thin line arranged in a rectangular loop to mimic a caveline. Since the simulated environment lacks real-world perception challenges such as low light or turbid water conditions, the perception subsystem remains simplified. Instead of deploying computationally intensive deep visual learning models, simpler edge detection and contour extraction techniques~\cite{SUZUKI198532} are used to identify the caveline from the down-facing camera feed. The remaining navigation and control subsystems mirror the real-world implementation and operate via two ROS nodes. The first node processes the extracted contours to make navigation decisions and publishes high-level control commands (\eg, yaw angle). The second node subscribes to these commands, computes the required thrust and hydrodynamic drag forces, and publishes them as ROS topics to control the simulated robot model.

Beyond replicating caveline following experiments, we utilize the DT system for preliminary testing and fine-tuning of new control algorithms. It also enables the simulation of complex cave scenes, such as narrow passages, dead ends, and sharp turns. Conducting repeated real-world experiments in such scenarios to improve the control system can be logistically demanding where the simulation offers an efficient alternative for extensive evaluation and fine-tuning.  

% These force commands are then subscribed to by the robot model in Gazebo, enabling it to execute the desired motion accurately. Finally, the URDF and Gazebo world files, along with the three control nodes, are integrated into a ROS launch file. Executing this launch file initiates a simulation of CavePI's digital twin in Gazebo, demonstrating its ability to follow the caveline within an underwater environment.

% To control the digital twin to follow the cave line in the underwater world, three ROS nodes are created for different applications. In the first ROS node, subscribed frames from the downward-facing camera are utilized by a caveline detection algorithm~\cite{SUZUKI198532} that publishes contours along the caveline. In the second node, control decisions are taken based on the detected contours along the caveline and high-level control signals are published. In the third node, these high-level control signals are subscribed, and thrust and hydrodynamic drag forces on the robot model are calculated. These forces are then published to ROS topics and later subscribed by the robot model in Gazebo to achieve the desired motion.

% \subsubsection{Integration}
% Finally, the URDF and Gazebo world files, along with the three control nodes, are integrated into a ROS launch file. Executing this launch file initiates a simulation of CavePI's digital twin in Gazebo, demonstrating its ability to follow the caveline within an underwater environment.
\section{Experimental Setup}\label{sec:experimental_setup}

\begin{figure}
    \centering
    \begin{overpic}[width=\columnwidth]{./figures/experimental_setup/experimental_setup.pdf}
        \put(28,55){(a)}
        \put(36,38){(b)}
    \end{overpic}
    \caption{Overview of the experimental setup for (a) wireless and (b) classical data readout. Transmitter and receiver placement are identical for both scenarios.}
    \vspace{-5mm}
    \label{fig:experimental_setup}
\end{figure}

To \revise{quantify} the effect of artificially enhanced capacitive coupling for classical \ac{DAQ} and compare it against wireless \ac{DAQ}, three measurement series for each of the two scenarios have been conducted.
One \textit{BodySense} system has been equipped with an Rx carrier placed on the test subject's upper wrist. 
An adhesive Ag/AgCl wet-gel electrode from \textit{TIGA-MED} with a diameter of \qty{48}{\milli\meter} acts as skin-electrode, ensuring a low resistive contact to the body and mechanically holds the system at its position.
The system ground plane and the battery act together as a floating electrode, closing the return path over the environment and earth-ground.
A second \textit{BodySense} system with a Tx carrier extension has been sequentially placed at distances of \qty{10}{\centi\meter}, \qty{30}{\centi\meter}, and \qty{50}{\centi\meter} to the receiver, utilizing the same Ag/AgCl wet-gel electrode as skin-electrode.
\autoref{fig:experimental_setup} provides an overview of the transmitter and receiver positions during the experiments.
Distances below \qty{10}{\centi\meter} have not been investigated as the inter-device coupling significantly strengthens the return path, making it comparable to the forward path loss \cite{yang_2022}.
For each distance, a frequency sweep from \qty{4}{\mega\hertz} up to \qty{64}{\mega\hertz} has been performed.
All measurements have been conducted in a standing position in a laboratory environment while being at least \qty{1}{\meter} away from walls and other equipment.
The arm was outstretched to the side, forming an angle of \qty{90}{\degree} between the arm and the torso's side. 
During the data collection, the subject stood still and kept the arm in a constant position to minimize movement-induced fluctuations in the environmental-coupled return path.

In the classical scenario, the receiver was directly connected over a USB Type-C cable, forwarding the collected data over USB to the plugged-in computer, see \autoref{fig:experimental_setup} (a).
For the wireless scenario, the USB Type-C cable has been removed, and the computer has been unplugged from the grid. The data is wirelessly forwarded via \ac{BLE} to the computer, as depicted in \autoref{fig:experimental_setup} (b).
Finally, the channel gain is calculated to measure the quality of the capacitive \ac{HBC} communication channel, and the received power has been extracted to set capacitive \ac{HBC} into context to conventional \ac{RF} solutions.

It is essential to carefully consider the effects of signal transmission through the human body and respect its safety limits in \ac{HBC} systems. The \ac{ICNIRP} sets limits for non-ionizing radiation exposure to the human body, defining different dosimetric quantities depending on the frequency. In the frequency range between \qty{1}{\hertz} and \qty{100}{\kilo\hertz} the current density is used as limiting metric \cite{icnirp_low} and above \qty{100}{\kilo\hertz} the \ac{SAR}-value \cite{icnirp_high}.
Further, the IEEE standard C95.1-2005 defines safety levels on human exposure to \ac{RF} electromagnetic fields \cite{c95.1-2005}.
By limiting the transmit power to maximal \qty{5}{\dbm} for our experiments, we ensured to meet the above-mentioned safety standards.
\subsection{Effect Game Play on Self Reported Discernment of Misinformation}
\subsubsection {Descriptive Statistics}
Preliminary data analysis reveals significant violations of the normality assumption in pre-tested VOI and NMLS scales. Considering the rather small dataset, we decided to proceed with a non-parametric repeated measures approach (Related-Samples Wilcoxon Signed Rank Test to the results of the scales' pre- and post-evaluations).

\textbf{Effect of the educational game on the verification practicies (VOI-7)}
%To determine if the game practice significantly changed the intention to apply the direct and indirect verification practices, we ran Related-Samples Wilcoxon Signed Rank Test. 
The test revealed significant differences between pre and post-gaming VOI scores (N = 42, Z = 4.361, p < .001). The results suggested that the game positively affected the repertoire of used practices and/or the perceived will to use these practices. 

\textbf{Effect of the educational game on Media Literacy}
To measure the effects of the game on Media Literacy, we first ran the Related-Samples Wilcoxon Signed Rank Test on the full scale. Then, to determine which components of Media Literacy were most affected by the game, we conducted separate subscale tests to analyze changes in each of the four subdomains of Media Literacy. The results demonstrated significant differences in Media Literacy scale results (N = 47, Z = 3.083, p = .002). The analysis revealed the following differences: the game significantly improved both functional consuming   Z = 2.064, p = .039 and critical consuming Z = 3.344, p <.001 ), but not the functional prosuming Z = .435. p = .664 and critical prosuming Z = 1.868, p = .062. Therefore, the results suggest the game improves Media Literacy in the domains connected to understanding the content of the media and being able to critically evaluate the content of the media; however, it has not significantly improved the ability to produce media content which can be influential to others and convey author's ideas \cite{koc2016development}.

\textbf{Effect of the educational game on self-efficacy towards misinformation}
We did not find significant differences in self-efficacy between pre and post-game measurements (Z = .743, p = .458). 

\textbf{Effect of the educational game on the ability to recognise misinformation
}
We took the naive approach to calculate the MIST score, taking it as the sum of the right answers on all 20 questions \cite{maertens2024misinformation}. The results showed that participating in the game significantly improved the participant's ability to discriminate between fake and real news (Z = 2.702, p = .007) 

%A total of 48 participants registered for our game study and completed the informed consent process and pre-survey. These participants were matched into 24 pairs and scheduled for the gameplay experiment. 
%After completing the gameplay session, participants were given a short break before completing a post-survey and participating in a follow-up interview. All 24 pairs successfully completed the gameplay experiment. 


%However, data from one participant was removed due to [reason], leaving 47 participants whose interview responses and gameplay logs (including the outputs of both players and the LLM responses across four rounds) were included in the qualitative analysis for this paper.
\subsection{Qualitative Results}
In this section, we present participants' perceptions and understanding of misinformation, including how they learned to generate it and identify it through gameplay, as well as the strategies they developed for debunking misinformation. %to answer RQ1. 

%We employed a combined inductive-deductive approach to analyze the qualitative interview transcripts and gameplay logs\cite{kuckartz2019analyzing}. This approach ensured a comprehensive understanding of the gameplay experience. Our primary objectives were to understand how participants perceived and understood misinformation through the game, how they learned to distinguish and apply debunking strategies during gameplay, and how interactions with other players influenced their behavior and learning. The analysis process began with inductive coding. Two researchers independently coded a subset of the data, identified themes, and then discussed and reconciled any coding discrepancies, iterating on the coding system as needed. Once the coding system was established, the two researchers independently coded the full dataset. A third researcher then reviewed the coded data, and any differences in interpretation were discussed until a consensus was reached.


\subsubsection{Raise Awareness of Misinformation though gameplay}
Participants reported an increased awareness of misinformation through both the confrontational mechanics of the game and the news narratives presented within it. The game enhanced players' awareness of misinformation in two key ways.

\textbf{In-game News as a Reflection of Real-world Misinformation} 23 out of 47 participants noted that the in-game news mirrored real-world situations, thereby heightening their awareness of the characteristics of misinformation. A common observation was that news is rarely entirely true or false; instead, it often presents a mixture of both. This complexity makes genuine misinformation more challenging to detect. As one participant stated:
\begin{quote}
\textbf{N23}:
Nowadays, news often presents both positive and negative sides of a story, so I believe this game reflects real-life situations quite accurately.
\end{quote}
However, some participants acknowledged that the misinformation in the game appeared more overtly false compared to the more subtle nature of misinformation encountered in real life.

\textbf{The Competition Game Mechanics positively influence learning:} The PvP mechanics enhanced learning by requiring players to identify flaws in each other’s messages and respond effectively to achieve success. This repeated process helped deepen their understanding and sharpen their skills in distinguishing misinformation. As one participant noted:
\begin{quote}
\textbf{N20}:
In the process, I was able to see first-hand some of the flaws in the information (posted by others) and some of the claims made in an attempt to deceive people. And then it's also more accurate for me to judge the misinformation afterwards.
\end{quote}
Participants also learned from observing their opponents. For example, N22, who played the role of a debunker, noticed how the misinformation creator crafted and disseminated false information to persuade others:
\begin{quote}
    \textbf{N22}:
    When I was playing this round, I didn't score as high as my opponent, so I knew what they were saying and how they were letting the false information spread. Next time I come across such information, I will know that it is false.
\end{quote}
%In analyzing the gamelog, we identified several key strategies employed by Player 1 (influencer) to generate misinformation based on unverified evidence and rumors. For example, in Round 4, Player 1 spread rumors about a doctor’s death, which directly triggered public panic (N1). Many participants demonstrated a strong ability for emotional manipulation, frequently inciting fear and sympathy. Across all rounds (N6, N8), Player 1 used emotional appeals and personal stories to enhance the perceived credibility of the misinformation. Exaggerated claims, such as asserting the R drug’s "100\% effectiveness," were also employed to mislead (N36). Additionally, fostering cultural pride and heritage proved to be a powerful emotional tactic to build trust in the misinformation (N4). Player 1 further increased the complexity of the misinformation by incorporating celebrity endorsements(N9). A key example of this strategy was Player 1’s effective use of social media to amplify emotional narratives, which led personas like Emily and Maria to place strong trust in their misinformation (N21).---I took this part to emotion. 

%We observed that Player 2 employed a rational, evidence-based approach to counter misinformation, relying heavily on scientific facts and reasoning. For example, in Round 2, Player 2 (N2) effectively challenged Player 1’s misleading claims by referencing medical expert opinions and providing scientific explanations for patient deaths, underscoring a strong evidence-driven approach. Similarly, Player 2(N3) questioned the lack of clinical data supporting the R medicine and pointed out the potential commercial motivations. As the emotional complexity of the misinformation increased, Player 2's information became more structured and precise, with deeper critiques grounded in both scientific and ethical considerations (N3). In addition, Player 2 (N7) incorporated underlying factual evidence, such as economic motivations, to further strengthen their counterarguments. In some instances, Player 2 (N10) shifted focus from purely scientific critiques to address regulatory frameworks, leveraging an official perspective to challenge the spread of misinformation.

%During the game, some players encountered challenging narrative contexts. In Round 3 where Player 1 was at disadvantage, they responded by adopting a positive storytelling strategy, portraying characters in an optimistic and proactive manner to build trust in the misinformation. Instead of using fake scientific evidence, Player 1 used emotional manipulation, emphasizing values such as "cultural tradition" and "ancestral wisdom" to downplay the role of science and overcome the unfavorable situation.

%Player 2 encountered similar difficulties when anecdotal evidence was used to spread misinformation, particularly in Rounds 2 and 4. In these cases, Player 2 had to maintain logical reasoning despite the strong emotional appeals. Some players countered misinformation by clarifying facts and directly addressing conspiracy theories, while others employed a more cautious approach, appealing public to wait for more evidence to prove. This strategy allowed Player 2 to weaken Player 1’s influence.


\subsubsection{Identifying Misinformation through Source Evaluation:}
After the gameplay sessions, participants reported increased awareness of the varying credibility of different information sources. The game helped them to realize that producers of misinformation often seeks to enhance credibility by deliberately referencing authoritative organizations. One participant reflected on this realization:
\begin{quote}
    \textbf{N40}:
    After playing the game, I found that it was indeed the same as in the experiment. Some news did mention authoritative organizations as references, but I could tell that this was intentional…. (The game) may make my suspicions more valid.
\end{quote}
In addition, participants acknowledged that information from seemingly authoritative sources is not always reliable. It requires information to be cross-checked from multiple sources to verify its authenticity. As one participant noted: 
\begin{quote}
\textbf{N13}:
    I used to trust information from authoritative sources and reputable publications. But the game showed me that even these can be false, as my opponents used fake evidence from supposed authorities.
\end{quote}

\subsubsection{Identifying Misinformation through Emotional Manipulation Tactics}
Participants learned various tactics for both creating and debunking misinformation through the game’s instructions and their in-game experiences. A particularly commonly identified tactic was emotional manipulation, which was noted by 35 out of 47 participants (18 misinformation creators and 17 debunkers). By analyzing the game logs, we identified common emotional manipulation strategies used in the game. Most players crafted messages designed to evoke anxiety and fear, while some also attempted to generate feelings of hope. For example, in Round 4, Player 1 spread rumors about a doctor's death, which directly incited public panic (N1). Across all rounds, Player 1 frequently used emotional appeals and personal stories to enhance the perceived credibility of the misinformation (N6, N8). Additionally, invoking cultural pride and heritage was a powerful tactic used to build trust in misinformation (N4), while celebrity endorsements further increased the complexity and believability of the misinformation (N9). As illustrated in \autoref{fig:emotion}, players significantly increased public trust in their information by using emotionally charged language (N31). These strategies align closely with the characteristics of misinformation, where emotional appeals are commonly used to influence public opinion \cite{chuai2022really}. 
%By analysing gamelog, we identified the common emotional manipulation tactics player used in game. Most players aimed to craft messages that evoked anxiety and fear, while some also attempted to generate feelings of hope. For example, in Round 4, Player 1 spread rumors about a doctor’s death, which directly triggered public panic (N1). Across all rounds, Player 1 used emotional appeals and personal stories to enhance the perceived credibility of the misinformation (N6, N8). Additionally, fostering cultural pride and heritage proved to be a powerful emotional tactic to build trust in the misinformation (N4). Player 1 further increased the complexity of the misinformation by incorporating celebrity endorsements(N9). A example as shown in Figure 7, players were able to significantly increase public trust in their information by using emotionally charged words(N31) . These strategies closely align with the characteristics of misinformation, where emotional appeals are often used to sway public opinion\cite{chuai2022really}.  Through their use of emotional manipulation within the game, participants became more attuned to recognizing these tactics in real life. 

In follow-up interviews, many reported an increased awareness of the emotional undertones embedded in messages, which made them more suspicious of such content. They learned to identify emotionally charged language, such as messages that were \textit{“overly positive,”} \textit{“overly exaggerated,”} or \textit{“overly one-sided about an overly positive point of view.”}

Interestingly, when playing the role of the debunker, participant can reflected on the emotionally inflammatory language used by the misinformation creator and helped them develop a more clear strategy for addressing misinformation. This approach involved separating the factual content of a message from its emotional manipulations and focusing more on the factual aspects, as one participants explained 
\begin{quote}
    \textbf{N2}:
    When reading a story, it is important to put more focus on what is going on at the factual level rather than what is going on above the author's own views and emotions. 
\end{quote}
\begin{figure}
    \centering
    \includegraphics[width=1\linewidth]{figs/results_emotion_RR.jpg}
    \caption{An example of how emotional manipulation tactics from N31 effectively works on public opinion.}
    \label{fig:emotion}
\end{figure}

%Gameplay about emotion
%example of hope:
%“他们真的在尽最大努力。每一次实验、每一组数据都是为了确保我们每个人都能得到更安全、更有效的治疗。他们不仅在做研究,也在为我们的亲人和朋友而战。
%这些科学家也是有家室的人。他们比任何人都清楚,每一天都弥足珍贵。他们的辛勤工作不仅是为了科学的进步,也是为了那些应该享受生活的年轻人,为了我们宝贵的孩子,为了我们的父母。他们理应安享晚年。作为医务工作者,我们要带头给大家信心。我们最清楚,科学需要时间,但每一次进步都是朝着光明的方向。我们都看到了艾米医生团队的专业和坚持。很快,他们就会带来更多的好消息,届时,我们就能更有效地对抗这种病毒了。让我们一起欢呼,支持他们的工作,因为美好的日子就在前方。无论您是年轻人、中年人还是老年人,这项工作都是为了我们所有人。我们的团结是我们最大的力量。让我们共同迈向更健康、更安全的未来!(0819_E组-N19 N20, Pos. 225-228)”
%example of combining emotion and hard science
%Round3 p2:
%[2024-08-20 11:02:49] Player 2 input: R drugs can cure inflammation? Shameful to use data to falsify! Exploiting national cultural pride is even more shameful!
%According to incomplete statistics, nearly more than two-thirds of the European population has died from inflammation in the last 500 years of history, and the R-drug has not produced a definitive positive feedback on historical inflammation cures since its introduction in research. At the same time, taking advantage of the public's recognition of the anecdotal medical culture and its legacy, Dr Amy's medical organisation and its financial consortium have ignored the fact that Eternal Wellness has not conducted any new or further research or testing of the R-drug, and that the R-drug provides only temporary symptomatic relief, is not a complete cure for Zinc, and that the adverse effects of the R-drug are not yet known. What are the intentions of Eternal Wellness? (8.20 L組 N25N26, Pos. 175-177)

%example of fear (N33N34)
%N33 attempted to instill fear in the public regarding Max, using tactics designed to create doubt and anxiety about the medicine. However, N34  countered by directly refuting the claims and offering clear, factual explanations to debunk the misinformation, shifting the focus back to evidence-based reasoning.
%example of exaggeration
%N35 effectively countered N36's exaggerated claims about the R treatment virus and the new medicine MAX by highlighting the lack of evidence, media coverage, and credible reports, while casting doubt on anecdotal accounts like the widow's story to sw
%example of culture (N35N36)
\subsubsection{Critical Thinking about Misinformation Motives}
After the gameplay sessions, many participants (31 out of 47) demonstrated an enhanced understanding of the intricacy of information and the varied perspectives it can convey. This experience increased their awareness of the importance of considering the motivations behind messages. Participants also mentioned that they are now more inclined to think critically about the goals behind the information they encounter, especially in real-life situations where such considerations are common.
One participant explained how the game illustrated the pre-determined nature of many messages:
\begin{quote}
    \textbf{N9}:
    One of the most direct ways is that [the game] lets me know that what I'm reading is very likely to be pre-determined. It's like the rules of the game itself, which is that I'm playing as someone in camp A, and I'm across from someone in camp B, and we are both posting messages for the benefit of our camps. Those messages may take on various styles or appearances, but they are all ultimately very purposeful. This, I think, is a strong point to learn.
\end{quote}
Participants also found that this new perspective would be useful in their future interactions with information. They felt that applying this critical mindset could help them better understand the underlying goals and potential financial motivations behind the messages they encounter:
\begin{quote}
    \textbf{N42}:
    It feels like one of the more educational aspects of the game is that [through this game] it's like I can think about what their ultimate goal is from a reverse mindset, and then look at a lot of information in life with that mindset.
\end{quote}
%N32P2 R2面对不利背景故事情况的时候,未使用提示情况下质疑:“在死亡人群中有“几个人”使用新药无法说明药物是否无效,这与个人体质有很大关系,需要足够多的试验对比,而不是凭借患者家属个人描述。 (N31 N32 0821 Group P, 位置243)”在后续interview时候也提到自己“从我个人角度来说,我认为这种假新闻或者说片面性的新闻,它是故意在舆论上去引导一些信息 (N31 N32 0821 Group P Interview, 位置126)”

\subsubsection{The Impact of Role-Playing as a Misinformation Creator on Learning}
Through the experience of playing the role of the misinformation creator (Player 1), some participants became aware of just how low the barriers are for creating misinformation. This made them more cautious about the influence of certain public figures, particularly online "influencers." As one participant noted: 
\begin{quote}
    \textbf{N45}:
    I'm Player 1, and I realized that the cost of creating rumors is so low. If I were an online celebrity or someone with the ability to influence public opinion, and my job wasn't that of a journalist, I might not need to be very responsible for spreading these kinds of rumors.
\end{quote}
Another participant reflected on how playing as a misinformation creator broadened their perception of misinformation, particularly regarding how easily false information can be fabricated. This experience expanded their understanding of the boundaries of misinformation, making them more aware of how easily those boundaries can be crossed: 
\begin{quote}
    \textbf{N46}:
    I used to think I could recognize information with a stance, and information without a stance. But after playing the game this time, I've realized that it’s something that can be fabricated. The boundaries of awareness of false information have been expanded, and the bottom line has been lowered. That's probably how it feels.
\end{quote}
These findings indicate that when participants took on the role of distributing misinformation, it helped them to better grasp how misinformation is produced and emphasized how easily it can spread. This is further proved in the game. As shown in \autoref{fig:influencer}, Player 1 employed certain strategies when faced with an unfavorable context, such as avoiding or distorting facts and creating a positive image. In the interview, Player N31 also reported that as the game progressed, they felt increasingly confident in their ability to generate misinformation.
\begin{figure}
    \centering
    \includegraphics[width=1\linewidth]{figs/results_influencer_RR.jpg}
    \caption{Strategies of influencers used to deal with unfavorable situations in the game}
    \label{fig:influencer}
\end{figure}
\subsubsection{Tailoring Debunking Strategies to Audience Characteristics}
Many participants found the responses of the LLM-simulated personas to be particularly engaging. They analyzed these responses to understand the reasons behind changes in opinion, how the output of other players influenced these shifts, and what the personas now trusted. Participants noted that the LLM-simulated personas provided clear trust level scores and reactions, which were helpful in organizing their responses. As one participant observed:
\begin{quote}
    \textbf{N23}:
    What I found most interesting was the change in their opinions. They would follow the different points we made and then express their own opinions from various points of view. At first, I didn’t think what they said had any effect on me, but later on, I adjusted my strategy according to their thoughts and used them to control the score (trust level score).
\end{quote}
This insight into the personas' dynamic responses helped players refine their strategies. For example, players noticed that different personas reacted differently to emotional and logical appeals. While three personas were easily swayed by emotional arguments, the other two preferred rational, science-based evidence. Recognizing these tendencies, Player 2 successfully countered emotional tactics through logical analysis and evidence (N35). However, even when Player 2 consistently employed logical reasoning and evidence throughout the rounds, it didn’t always succeed in shifting all five personas' trust levels in their favor.
Participants further realized that using evidence to dispute misinformation wasn’t always effective. For instance, one persona was a traditional-minded person who resisted new scientific findings. As shown in \autoref{fig:debunk}, after several rounds of gameplay, players adapted their strategies to persuade the persona by considering her perspective. In the follow-up interview, a participant reflected:
\begin{quote}
    \textbf{N3}:
    There's a housewife who has always been a supporter of traditional medicine. I felt very confident of being able to persuade her because I observed players' struggles with her, and I saw the issues they sought to have resolved. So, in my final round, I focused specifically on her. I took the view that the best way to address this challenge was to rely on scientific evidence. It think that to have done otherwise would in itself have been a form of misinformation.
\end{quote}


%by tracking the trust level scores, players noticed that different personas reacted variably to emotional and logical appeals. Specifically, three out of five personas demonstrated strong trust in misinformation due to Player 1's emotional manipulation, while the other two like John and Alex did not. In this way, Player 2 successfully countered these emotional tactics through logical analysis and evidence (N35). 
%Furthermore, Player 2’s rational approach (N22) across every round in the game also proved insufficient to fully shift the perspectives of those with entrenched beliefs, as indicated by the trust level scores. Incorporating stronger emotional appeals may improve Player 2's influence on personas who are more receptive to emotional narratives.

%N10 "My main strategy is to persuade those five audiences."
%N5 "I tried to focus more on the reactions of different audiences than on my opponent's, especially when I need to clarify specific points for them. "
%Similarly, N12 (P1) reported that interacting with the personas helped them gain a more precise understanding of certain groups, both in the game and in real life. This understanding enabled them to adjust their strategies more effectively to persuade these groups during the game and in future real-life scenarios.%N21 "The young lady always trust my words. This is very similar to reality as some teenagers are easily influenced by online misinformation."
%Additionally, N10 reflected on the reactions from personas, "For example, the 78-year-old man. Outside his field of expertise, he might react differently. While his strong resistance to misinformation comes from his professional background, in another industry, he could be more easily misled."
\begin{figure}
    \centering
    \includegraphics[width=1\linewidth]{figs/results_debunk_RR.jpg}
    \caption{An example of tailored debunking strategy.}
    \label{fig:debunk}
\end{figure}


\subsubsection{Improving Confidence in Debunking}
Many participants reported an increase in their confidence to debunk misinformation in real-life scenarios. %This boost in confidence stemmed from their success in debunking misinformation within the game. 
The positive outcomes they experienced in the game appeared to strengthen their ability to challenge misinformation outside the game environment. For example, in Round 3, N7 purchased a hint to access supporting data for their argument. However, N8 used the same data to identify flaws and construct a counterargument. By leveraging data-driven reasoning, N8 effectively challenged N7’s claims and strengthened their own position, as shown in \autoref{fig:confidence}. In the post-game interview, N8 reported their increased confidence in using evidence-based rebuttals: 
\begin{quote}
    \textbf{N8:}
    I'd say my confidence went up to around 70\% or 80\%, because I realized that presenting evidence and discuss this is really effective.
\end{quote}
%For example, N11 (P2) noted that the game provided positive feedback after successfully debunking misinformation, which led to higher scores. 
%\begin{quote}

    %\textbf{N11}:
    %I believe my confidence grew as the game moves forward. Each time I counter my opponent's misinformation, I noticed that the game gave me positive feedback and the personas recognize my efforts. This made me more confident in the newly learned debunking methods in the game, as well as my previous approach to distinguish misinformation.
%\end{quote}
%This validation of their debunking strategies through in-game feedback contributed to their increased confidence. Additionally, some players noted that their in-game effective debunking strategies also led to an increase in their confidence. For instance, N8 would compare the data and previous information to debunk the misinformation, as shown in \autoref{fig:confidence}.

However, some participants expressed concern that debunking misinformation in real life would be more challenging. Unlike in the game, where the personas’ cognitive levels and trust levels were explicitly shown, it is much harder to gauge these factors in real-world interactions. This uncertainty could make it more difficult to challenge misinformation effectively in everyday situations.
\begin{figure}
    \centering
    \includegraphics[width=1\linewidth]{figs/results_confidence_RR.jpg}
    \caption{An example of how a debunker identifies logical flaws in the information by comparing different pieces of information.}
    \label{fig:confidence}
\end{figure}

%In round 3, N7 purchased a hint, gaining access to data to support their argument. However, N8 used this data to find flaws and construct a counterargument: "Previously, Eternal Health claimed that 20\% of R drug sales would be used for Zinc virus research and prevention, and that R drug sales were so successful that they sold out. So why does Jack still need to invest 65\% of his funds into building advanced technology and laboratories for the traditional medicine institution producing R drug? It seems that such a large investment shouldn’t be necessary. The reasons behind this are worth considering—perhaps there is no real connection or proper investment." (N8, Round 3) By leveraging data-driven arguments, N8 was able to effectively challenge N7’s claims and strengthen their own position. In the post-game interview, N8 reported their increasing confidence in using evidence-based rebuttals: "I’d say my confidence went up to around 70 or 80, because I realized that presenting evidence and discuss them is really effective." 


\subsubsection{Impact of Gameplay on Future Debunking Actions}
A few participants (4/47) shared that the game increased their likelihood of taking action against misinformation in the future. This change in attitude was driven either by participants' previous negative personal experiences with misinformation or by their realization during gameplay of the serious consequences misinformation can have.  The gameplay experience enhanced their willingness to invest time and effort into distinguishing and debunking false information. As one participant said:
\begin{quote}
\textbf{N41}
    In real life, there is a lot of false information, especially in advertising, media, and even those semi-official accounts, which can lead to changes in public opinion under the influence of these accounts, and in that case, it will definitely have some impact on some ordinary people. The game has strengthened my hatred for this kind of false information, so that I can be more awake and rational in my judgement.
\end{quote}

However, the majority of participants indicated that they might not actively debunk misinformation on social media after playing the game. The primary reasons were a dislike of online debates and the belief that it’s not their responsibility to engage in debunking efforts. These findings align with previous research, which suggests that most users are reluctant to take action to debunk misinformation publicly\cite{tang2024knows}.

\subsubsection{Challenges and Negative Effects of Gameplay}
While some participants gained confidence in their ability to debunk misinformation, others experienced a decrease in confidence (5/47). These participants observed that the game was close to real life, particularly that some individuals held strong pre-existing beliefs that were difficult to challenge. The game reinforced this reality, leading to a reduction in their confidence. As one participant noted: 
\begin{quote}
    \textbf{N11}:
    In the process of debunking, I realized that it is quite difficult to change people's inherent beliefs. Some people do not care much about whether the source of information is true or false, and this is also a social phenomenon that exists.
\end{quote}
Another challenge reported was the overwhelming amount of information presented in the game. In each round, players had to process news stories, the reactions of five personas and their changes, as well as their opponent’s output. After absorbing this information, participants were required to devise strategies and craft tailored responses. The volume of information, particularly by the end of the game, left some participants feeling exhausted, which may have impacted their performance.
In addition, the game’s mechanics required players to have a basic level of media knowledge to effectively take on the roles of “influencer” and “debunker.” Some participants also noted that the competitive nature of the game could lead to an unbalanced experience if one player was significantly stronger than the other.
As one participant noted:
\begin{quote}
    \textbf{N40}:
    The game was very fun, then I felt a bit nervous, and by the time I got to the end, it was a bit exhausting. I felt nervous because there was a lot of information at the beginning, and I was competitive with Player 2. There was much writing involved, and I felt uncertain because I’m not very good at writing, and I knew Player 2 was very skilled. So I felt a little nervous.
\end{quote}

%\subsection{Players' Perception and Understanding Throughout the Gameplays
%}
%Participants reported an increased awareness and understanding of misinformation through various aspects of the gameplay. This included elements such as game instructions, the news presented in each round, role-playing within the game, the reactions from the LLM, and the outputs from other players.

%\subsubsection{Perception of Misinformation Through In-Game News}
%The news presented in the game served as a clear example of misinformation for many players. Due to the confrontational mechanics of the game, both players were tasked with identifying the "false" components of each other's messages and responding accordingly. This experience heightened participants' awareness of misinformation and helped them recognize certain characteristics that signal suspicious information. As one participant noted: “\textit{In the process, I was able to see first-hand some of the flaws in the information (posted by others) and some of the claims made in an attempt to deceive people. And then it's also more accurate for me to judge the dis/misinformation afterwards (out of the game)}.” (N20) Out of 47 participants, X mentioned that the news in the game reflected real-world situations and enhanced their awareness of misinformation's features. One characteristic that emerged was the complexity of news, which is often not entirely true or false but rather a mixture of both. Players observed that this mixture made misinformation more concealed and difficult to recognize.“The information provided in each session, I think it's more in line with the wording of the current news, and then, we know that the news nowadays is more mixed with positive and negative information, so this game, I think, is more in line with a real-life situation”. N23 “The background news in each round feels quite realistic. What makes it feel real? Because I think our source of information should be modified from some real information, right? Just like I said, half-truths are the most misleading” N20.

%\textbf{Awareness of Source Credibility} Participants also became more attuned to the varying credibility of different sources, recognizing that misinformation often intentionally references authorities to bolster its credibility. As one participant reflected:\textit{ “As this round progresses, that news of his gives us more and more of a bit of news... It's the feeling that when I see another story after that, I'll realise consciously. Eh, he's mentioned an authority here. It's possible that it was mentioned on purpose, or there's some (information) that my suspicions about him might be more grounded.” (N40)}
%need more...
%N13 reported that through the game, the player realize that authorative information is not always true, and it it important to compare the authenticity of information through multiple sources. "I used to think that if information came from authoritative sources or reputable publications, it seemed more believable. But after playing the game, I realized that even those might not be true because my opponents in the game used a lot of fake evidence from so-called authority."(N13 N14 0819 Group B Interview, Line 23) "有,就是之前看虚假信息就是有的,感觉没这么讲,就是以前看下西西,感觉带上那种比较权威的那种人或者刊物什么的,感觉会感觉有点真,但是完了之后发现也都是也也有可能都是假的。 (N13 N14 0819 Group B Interview, 位置23)" m(寫的很好)
  

%\textbf{Critical Thinking and Awareness of Motivations} Another significant impact on players was the realization that information is often complicated and can represent different perspectives. This realization encouraged them to think more critically about the motivations behind the messages they encountered, with X participants explicitly mentioning this. For example: \textit{“One of the most direct ways is that (the game) lets me know that what I'm reading is very likely to be pre-determined. It's like the rules of the game itself, which is that I'm playing as someone in camp A, and I'm across from someone in camp B, and we're both posting messages for the benefit of our camps. Those messages may take on various styles or appearances, but they're all ultimately very purposeful. This, I think, is a strong point to learn.” (N9)} \textit{“It feels like one of the more educational aspects of the game is that [through this game] it's like I can think about what their ultimate goal is from a reverse mindset, and then look at a lot of information in life with that mindset.” (N42)}

%\subsubsection{The Role-Playing Experience and Its Influence on Learning}
%The role-playing element of the game also influenced participants' learning. For those who played the role of the misinformation distributor, some participants realized how low the cost of creating misinformation could be, making them more cautious about so-called "influencers." One participant observed:
%\textit{“Just the fact that I'm player 1, I just thought that having played it would show me that the cost of rumour creation is so low. It's that if I'm this kind of online celebrity or something, as this kind of person who can have a certain kind of ability to influence public opinion, and then my job is not a journalist kind of thing, I may not need to be very responsible for spreading this kind of rumour.” (N45)} Others reported that playing the role of a misinformation distributor gave them deeper insights into misinformation itself: \textit{“This information I know to be false, and then my task (in the game) is also to go and turn this false information into a real situation to carry out a persuasion. Under this influence, then, I actually correspond to a deeper understanding of this false information (...) What I'm seeing now is that this disinformation, it may not be that it's completely false, like when I'm writing content, I'm being half-truthful, and that's the hardest thing for people to discern in this situation (.....) And then in this case it's actually improved my ability to judge this, um, disinformation, and in that I've learnt more about the logic of disinformation.” (N20)}
%need more description about P2's role 

%\textbf{Player 2's Role and Its Influence on Learning}Players who took on the role of Player 2 primarily utilized logical reasoning to refute Player 1’s arguments. They identified flaws in the logic of Player 1’s information, pointed out instances of emotional manipulation, questioned the lack of transparency in financial relationships behind the messages, and highlighted the absence of evidence. These strategies effectively revealed the flaws in Player 1's messages and reduced their credibility. To strengthen their arguments, Player 2 often referred to authentic information from reputable sources or authorities, encouraging the public to remain calm and critically assess the information presented.
%N2(P2)通过在游戏中扮演debunker的角色,总结出自己在游戏中遇到P1使用了大量的情绪煽动性词汇,这让他更加明确自己当前角色的策略:即阅读一个信息的时候,将信息的事实和他所表达的情绪分开独立来看:关注更多的精力在事实上面。因为煽动性言论很容易让人迷失在真实事实。(“就是比如说在阅读一篇报道的时候,要放更多的精力在事实层面的事情,而不是作者本身观点和情绪上面的事情。比如说他打的一些感情牌,或者说是一些宣,就是煽动性的言论,这些事跟事实应该是分开的两部分。可能以后我在摄取信息的时候,会将更多精力放在到底哪些是真实发生的事情上面,而不是哪些是这个作者提出的感情方面的事情。 (N1 N2 0818 Group B Interview, 位置101)”)

%In addition to the debunking techniques that Player 2 learned, they also perceived how their opponents generate the misinformation. N22(P2)提到通过观察游戏中P1是怎么传播虚假信息的而学到的(“我在玩儿的同时我错了的话,我就知道了它这个是怎么说?这个虚假信息是怎么传播的?比方说我想发一个虚假信息,诶,我就知道了。嗯,就知道我这个虚假信息是怎么发出来的,下次我遇到这样的信息了,我就知道这个是假的了。 (N21 N22 0819 Group G Interview, 位置152)”)



%This in-game experience also increased participants' confidence in debunking misinformation in real-life scenarios. X out of 47 participants reported that their success in the game bolstered their confidence to challenge misinformation in reality. For example, N11 (P2) noted that the game provided positive feedback after successfully debunking misinformation, which led to higher scores. This validation of their debunking strategies through in-game feedback contributed to their increased confidence.

%\textbf{Adapting Debunking Strategies Based on Audience Characteristics}
%Participants also learned that different audiences require tailored strategies. For instance, in one scenario, a language model (LLM) simulated a young female student who was highly active on social media, trusted social media influencers and celebrities, and easily changed her opinion. After a few rounds, players recognized her personal characteristics and began referencing popular singers and celebrities' endorsements to persuade her. As one participant shared:"Quotation". Similarly, N12 (P1) reported that the personas helped them gain a more precise understanding of certain groups, both in the game and in real life. This understanding allowed them to adjust their strategies to better persuade these groups during the game.

%\textbf{Learning Tactics for Creating and Debunking Misinformation} 
%Participants also learned various tactics for both creating and debunking misinformation through the game’s instructions and their in-game experiences. This learning made them more cautious about information containing these features. Interestingly, many players (X/47) employed emotional manipulation strategies, followed by using hard science/facts as references and finally questioning the motivations behind messages. One explanation for this is that the LLM simulates emotional reactions in the populace, making players more aware of emotional manipulation. Consequently, players were driven to use emotionally manipulative strategies, or they noticed the LLM’s simulation of critical thinking, which encouraged them to seek evidence for support. When using emotional manipulation, most players aimed to stimulate anxiety and fear, with some also attempting to create feelings of hope. These strategies align with the characteristics of health misinformation [REF].  In the follow-up interviews, players mentioned that they became more aware of the emotions embedded in messages, which made them more suspicious of such content: \textit{“After going through this game, the kind of statements that are overly positive, overly exaggerated, or overly one-sided about an overly positive point of view just don't seem so true to me.” (N23)}


%\textbf{Variations in Debunking Confidence Among Participants}
%However, the debunking experience varied among participants. While some participants gained confidence, others reported a decrease in their confidence to debunk misinformation (X/47). These participants found that, similar to real life, some individuals held very strong pre-existing beliefs that were difficult to influence. The game reinforced this challenge, leading to a decrease in their confidence. As one participant explained:“In the process of debunking, I realized that it is quite difficult to change people's inherent beliefs. Some people do not care much about whether the source of information is true or false, and this is also a social phenomenon that exists.” (N11)

%\textbf{Impact of the Game on Future Debunking Actions}
%A few participants shared that the game increased their likelihood of taking action against misinformation in the future. This change was often related to their personal experiences, such as friends encountering disinformation or a heightened awareness of the negative consequences of misinformation on the general population. This increased awareness made them more willing to invest time and effort into distinguishing and debunking misinformation. As one participant expressed:“Quotation”. However, most participants mentioned that they might not actively debunk misinformation on social media after playing the game. They cited a dislike for online debates and the perception that it is a waste of time. These findings are consistent with previous research (REF).

%N2(P2) reflected on Future Debunking Actions


%one reason is that emotion can seem as a tool of misinformation 

%我感覺可能因為LLM的回覆中展現出來的也是emotion, 或者說他是一個critical thinking的人,他就想要尋求證據的支持,這種反饋會進一步驅使讀者使用這兩種策略。玩家實際上也是這樣做的。

%“我会留意那5個人对于对手的信息的reaction,然后我再根据他们的 reaction,比如他们觉得为什么会因为对手发了这个东西,他们会改变这个情绪?然后我再根据这一点去增加一些我下一条写的内容 (8.20 L組 N25N26, Pos. 456)”

%“应该就是那 5 个人的反应吧?就是他们的分数反应啥的,就是在这方面他们就很非常直接直观的告诉你,他们为什么会相信这个信息或不相信这个信息,那从他们的反应里面就可以得到一个下一轮的一个策略。 (8.20 L組 N25N26, Pos. 468)”

%“我觉得最有意思的就是他们的发表言论的变化。就是他们会随着我们说的不同的观点,然后也发表自己的意见,从各个方面。说话人 1 01:15:23所以您觉得这一点比较有趣,是吗?说话人 2 01:15:27对,我觉得这样还挺有意思,就每一次都不太一样啊。说话人 1 01:15:32好的,那为什么你会觉得这个比较有趣呢?这只是一种感受吗?还是有什么特别的。说话人 2 01:15:38原因为一开始我可能没有觉得他们说的对我有什么影响,但是后来的话我会根据他们说的东西,然后就是进行后期的想法。嗯,可以利用他们来控分。N23/N24.

%**LLM的reaction**也會讓玩家意識到,對不同的人有不同的說服策略,有的人是真的很難改變的(所以會導致玩家信心的下降)

%“嗯,比如说在第一轮游戏结束之后,我能看到有几个分数低的是那个退休的护士和一个高中生。然后我看到他们的评价,一个是说需要有更多的证据,然后另一个是可能是医学需要一个证据。嗯,然后所以之后在第二轮还是第三轮,我就有说到那个,我其实就做了一些虚假的新闻,我说他也做一些研究,然后发了一些 paper。 嗯,然后我也还好像还有提到,因为我看到有一个是年龄比较大的老人,嗯,然后因为我发现好像第一轮我用了那个提示,我发现一些网络的言论也会影响到这五个用户。 有一个是高中生,好像是高中生,他好像手机,他看中什么網絡信息,然后所以我也有用到一些像说其他一些死亡或者是老人这种关键词,我要去尝试去影响他们的 (8.22R组-N39N40, p. 25)

%最後,玩家體驗完在遊戲裡贏了之後,比如說服了LLM模擬的民意,看見這個人倒向他們的時候,其實增加了他的信心。

%**在遊戲結束之後**,黨玩家被問到以後的行為時,

%對於distinguish 虛假信息

%比較多的玩家說過:識到信息不是單純的對錯,有些是有立場的,以後可能會更加去關注信息背後的動機。

%從主觀描述來講,玩家對於虛假信息的意識是增強了,

%“因为感觉自己好像可以从逆向思维思考他们最终的目的是什么,然后带着这种思维去看生活当中很多信息”N42

%“嗯,怎么说呢?嗯,我感觉可能会有强一点,因为其实这个是有个人性格的原因,我是一个比较冲动的人,就是我有时候会看到信息会引起我情绪的变化,所以我在现实中就有可能会被一些东西给带着走。但是做了这个实验之后,你就会从一个比较全面的角度去思考,然后这个时候就有可能你在现实生活中之后,也会把这个思维方式带入到现实生活中,就是像去思考一道题一样,思考一下它的整个结构,它整个的框架有可能会有哪些情况,然后可能就可能在情绪上的变动就不会那么大。 (8.20 M組 N27N28, Pos. 208)”

%少數玩家說以後會多去看官方的信息,尋找來源。

%有的玩家會意識到信息中的情緒操控“通過這個遊戲之後,對於过于肯定的观点,过于夸张或过于一边倒的那种说法,在我看来就不是那么真实。”N23

%有的玩家進一步意識到了現實中虛假信息對普通人對負面影響,從而更加願意投入時間和精力去進行判斷

%“我感觉是增强了这个点的话,可能不是我的能力增强,而对于虚假信息的痛恨增强。因为实际上,在现实中虚假信息是特别多的,尤其是营销号、传媒号,甚至是那些半官方性的账号,所以在这种影响下会导致整体的风向或者说舆论变化,然后这样的话肯定是会对一些普通人造成一些影响的。所以说现在我对于这些虚假信息更痛苦,也就是变相的让我更加清醒的去判断。”N41

%**對於Debunking**

%有的人說改變不大,因為他本身就不是會在網上跟人起爭執的人

%有的人說信心會下降,因為他意識到了人是很難改變的,他本身就有這個意識,現在體驗過四輪遊戲之後發現更加是這樣了。

%部分玩家認為遊戲中發生的事情會讓他們聯想到現實中的事情,因此激起他們的反應。

%“然后如果说这个信息这损害到人的一些基本的利益,或者说生存,或者说一些金钱这一块,我可能会直接进行一个反驳,或者说直接进行一个举报等情况。对,这是我在今天,嗯,看到一个最直观的一个感受,因为像什么看到那五个人的一个状态,我尤其想到了什么,就是出高中的时候,就是朋友被骗的那些经历。对,然后看到以后就会有这种想法。对。 (0819_E组-N19 N20, Pos. 507)”

%遊戲難度:

%每個人在每一輪都獲得了很多信息,在思考和閱讀上有難度,但是可能不太適用於所有人,比如小朋友或者老年人沒有辦法使用。

%對以後的建議:

%在遊戲裡讓玩家有更多的主動權去驗證信息,因為目前“也可能是因为这是一个游戏。嗯,我先,所以我也没有别的信息来源去验证。嗯,就对话说的到底是真的还是假的。 (8.20 M組 N27N28, Pos. 323)“ 而且這樣有個問題是,兩個玩家其實全部都是在說自己的觀點,有時候就很難反駁。


%\subsection{Two}

%\subsubsection{Perception of Misinformation}

% To 5.2.1 Confrontational Game Mechanics
%As the game progressed, fluctuations in response times were evident. Player 1’s time to generate misinformation increased, particularly in Rounds 3 and 4, as their strategy shifted from simple emotional appeals to more elaborate narratives incorporating economic details. This shift suggests that Player 1 aimed to enhance the credibility of their misinformation by adding complexity and depth (N4). However, in some cases, Player 1’s response time decreased as they adopted to more inflammatory and emotional manipulation to persuade the LLM personas (N1). The behavioral patterns in N6 and N12 indicate a growing reliance on emotional manipulation over factual engagement when challenged. Additionally, Player 1 demonstrated a deeper understanding of misinformation techniques, increasingly employing multiple emotional manipulation strategies (N12). Furthermore, Player N33’s game content and interview responses suggest that in-game instructions played a significant role in helping them in generating misinformation.
%We observed that Player 2 employed a rational, evidence-based approach to counter misinformation, relying heavily on scientific facts and reasoning. For example, in Round 2, Player 2 (N2) effectively challenged Player 1’s misleading claims by referencing medical expert opinions and providing scientific explanations for patient deaths, underscoring a strong evidence-driven approach. Similarly, Player 2(N3) questioned the lack of clinical data supporting the R medicine and pointed out the potential commercial motivations. As the emotional complexity of the misinformation increased, Player 2's information became more structured and precise, with deeper critiques grounded in both scientific and ethical considerations (N3). In addition, Player 2 (N7) incorporated underlying factual evidence, such as economic motivations, to further strengthen their counterarguments. In some instances, Player 2 (N10) shifted focus from purely scientific critiques to address regulatory frameworks, leveraging an official perspective to challenge the spread of misinformation.

% To 5.2.6 Tailored Debunking Strategies
%Based on the trust level score, we observed that three out of five personas demonstrated strong trust in the misinformation due to emotional manipulation, while the other two did not. Player 2 effectively counters Player 1’s emotional manipulation claims through logical analysis and evidence, especially capturing the attention of more rational personas like John and Alex (N35). Player 2’s rational approach (N22) proved insufficient to fully shift the perspectives of those with entrenched beliefs, as indicated by the trust level scores. Incorporating stronger emotional appeals may improve Player 2's influence on personas who are more receptive to emotionally narratives.

%\subsubsection{Time to Generate Response}
%As the game progressed, we observed the fluctuations in response times. In Round 3 and 4, Player 1 took more time to generate misinformation, where their strategy shifted from emotional appeals to more elaborate narratives that included economic details. This change suggests that Player 1 wanted to make their misinformation more believable by making it more detailed (N4). However, there were instances where Player 1’s response time decreased as they shifted to more inflammatory and emotional manipulation tactics, aiming to convince LLM personas (N1). N6 and N12’s behavioral shift indicates a preference for emotional rather than engaging with factual rebuttals when challenged. Additionally, Player 1 increasingly used multiple emotional manipulative techniques, reflecting a deeper understanding of how to generate misinformation (N12).

%Similarly, the response time of Player 2’s rebuttals increased as Player 1’s information became more intricate. To counter the increasingly sophisticated misinformation, Player 2 needed more time to analyze, verify, and construct detailed rebuttals (N5). Early in the game, Player 2 relied on straightforward scientific evidence, but in Round 4, their responses became more structured and multi-layered. They employed rigorous fact-checking and logical analysis to dismantle Player 1’s conspiracy theories. This demonstrated Player 2’s growing capacity to maintain a calm, rational approach in the face of emotionally charged misinformation (N11).

%\subsubsection{Changes in Behavior Patterns}
% To 5.2.1 Confrontational Game Mechanics
%During the game, some players encountered challenging narrative contexts. In Round 3 where Player 1 was at disadvantage, they responded by adopting a positive storytelling strategy, portraying characters in an optimistic and proactive manner to build trust in the misinformation. Instead of using fake scientific evidence, Player 1 used emotional manipulation, emphasizing values such as "cultural tradition" and "ancestral wisdom" to downplay the role of science and overcome the unfavorable situation.
%Player 2 encountered similar difficulties when anecdotal evidence was used to spread misinformation, particularly in Rounds 2 and 4. In these cases, Player 2 had to maintain logical reasoning despite the strong emotional appeals. Some players countered misinformation by clarifying facts and directly addressing conspiracy theories, while others employed a more cautious approach, appealing public to wait for more evidence to prove. This strategy allowed Player 2 to weaken Player 1’s influence.


%Full width figures.
% \begin{figure*}[htbp]
%   \centering
%   \includegraphics[width=\textwidth]{figs/data.png}
%   \caption{Caption}
%   \label{fig:data}
%   \Description{Caption}
% \end{figure*}

%(18/29) reported increased confidence in identifying misinformation.
%N36(P1), N35(P2), N30(P2), N29(P1), N22(P2), N18(P2), N17(P1), N16(P1), N15(P2), N14(P2), N13(P1), N11(P2), N8(P1), N7(P2), N6(P1), N3(P2), N4(P1), N1(P1).
%\begin{enumerate}
 %\item N11(P2) reported that the game provided positive feedback after successfully debunking misinformation, leading to higher scores, which verify their prior debunking approaches. This process of positive feedback contributed to increased confidence.
 %\item However, N11(P2) reported that it will be more challenging to debunk misinformation in real life because it's difficult to know the cognitive level of the public you're addressing, which is different from the in-game personas, making it hard to effectively convey ideas.
 %\end{enumerate}

%(5/29) reported decreased confidence in identifying misinformation after learning these strategies.
%N47(P2), N31(P1), N33(P1), N10(P2), N9(P1)
%\begin{enumerate}
    %\item N10(P2)发现在游戏里面虚假信息竟然还可以直接这么编,刷新了自己的认知(“我觉得我以前大概认为我自己是能够识别带立场的信息和不带立场的信息的,就是但是我这次游戏玩完之后,我还意识到就是你是可以编的吗哈?就是那个所谓的虚假,对吧?就,就是就边界扩展了,底线降低了。嗯,我觉得大概是这么一种感受,嗯。 (N9 N10 0818 Group T Interview, 位置159)”)
%\end{enumerate}

%(25/29) reported that they learned how to distinguish the misinformation from the game
%N47(P2), N36(P1), N35(P2), N34(P2), N33(P1), N31(P1), N30(P2), N29(P1), N22(P2), N18(P2), N17(P1), N16(P1), N15(P2), N14(P2), N13(P1), N11(P2), N9(P1), N8(P1), N7(P2), N6(P1), N5(P2), N3(P2), N4(P1), N2(P2), N1(P1).
%\b%egin{enumerate}
    %\item N29(P1)提到在游戏中有这种体验之后,那么以后在现实生活中遇到了就会发现自己其实就是这么写的
    %\item N22(P2)提到通过观察游戏中P1是怎么传播虚假信息的而学到的(“我在玩儿的同时我错了的话,我就知道了它这个是怎么说?这个虚假信息是怎么传播的?比方说我想发一个虚假信息,诶,我就知道了。嗯,就知道我这个虚假信息是怎么发出来的,下次我遇到这样的信息了,我就知道这个是假的了。 (N21 N22 0819 Group G Interview, 位置152)”)
    %\item N2(P2)玩完游戏后声称自己会在阅读一个信息的时候,将信息的事实和他所表达的情绪分开独立来看:关注更多的精力在事实上面。因为煽动性言论很容易让人迷失在真实事实。(“就是比如说在阅读一篇报道的时候,要放更多的精力在事实层面的事情,而不是作者本身观点和情绪上面的事情。比如说他打的一些感情牌,或者说是一些宣,就是煽动性的言论,这些事跟事实应该是分开的两部分。可能以后我在摄取信息的时候,会将更多精力放在到底哪些是真实发生的事情上面,而不是哪些是这个作者提出的感情方面的事情。 (N1 N2 0818 Group B Interview, 位置101)”)
    %\item N2(P2) reported that when facing Player 1’s provocative strategies, they sometimes restrained their own statements because their information was truthful, and using provocative language might undermine the credibility of their truthful statements. This is something they will need to consider in real life as well.(“就是我会担心自己的语言过于的具有煽动性,而丧失了那种,对,丧失了本身的一个立场,包括比如说主要是有些修辞,比如说我会用一些排比,或者是一些强烈的情感方面的事情,但是这些又没有什么客观的事实依据,就是会让我感觉到比较矛盾。 (N1 N2 0818 Group B Interview, 位置93)”)
%\end{enumerate}

%(15/29) reported a connection between the game and real-world scenarios
%N34(P2), N33(P1), N31(P1), N21(P1), N18(P2), N8(P1), N14(P2), N12(P1), N11(P2), N10(P2), N7(P2), N6(P1), N5(P2), N2(P2), N1(P1). + N27,
%\begin{enumerate}
    %\item N11(P2) reported that the game deepened their understanding of misinformation.
    %\item N10(P2) reported that by playing the debunker role in the game, they refreshed their understanding of misinformation.
%\end{enumerate}

%(9/29) found the public opinion simulated by AI in the game interesting
%N15(P2), N14(P2), N13(P1), N11(P2), N10(P2), N9(P1), N4(P1), N2(P2), N1(P1).
%\begin{enumerate}
    %\item N9(P1) reported that due to AI model’s strengths, when they fabricated evidence in the game, the personas responded to this evidence, creating an immersive experience through a timely and specific feedback mechanism.(“然后印象比较深刻的就是,嗯,应该是第三轮的时候就是谈那个富豪 Jack 的事情,然后我就瞎写了一下,他有这个,就是把比较多的内容放在了关于这位富豪的个人经历上,然后我可以在市民的反馈中就是读到,嗯,大家对这一点的会就是说,比如说他们关于个人经历的部分不太相信,或者他被个人经历所打动,因为我相信这一点的内容是我纯编造,并且纯主观,就是纯粹是个人的行为,然后他可以在接收到一个特别及时而且特别具体的反馈。我觉得这一点还挺好玩的。对,而且他也造成了我的得就是失分还有失败,所以给我留下了。 (N9 N10 0818 Group T Interview, 位置6)”)
%\end{enumerate}

%(10/29) reported the common strategies they use in the game is to persuade specific persona to get a better score
%N17(P1), N16(P1), N15(P2), N14(P2), N13(P1), N12(P1), N11(P2), N10(P2), N5(P2), N3(P2), N4(P1), N2(P2).

%(5/29) reported the in-game instructions very helpful can learn a lot.
%N35(P2), N34(P2), N33(P1), N15(P2), N9(P1).
%N9(P1) reported that the instruction would serve as a reference in the future real life.

%LLM and persona reactions
%\begin{enumerate}
   %\item N12(P1) reported that the personas helped them gain a precise understanding of certain groups both in game and in the real life, allowing them to adjust their strategies in the game to persuade these groups.
    %\item N11(P2)提到在反驳的过程中会意识到,改变他们的固有观念是挺难的,有些人没有那么在意消息的来源是否真假,这也是存在的一种社会现象。同时不同人物画像的特色设置也增强玩家的认知,了解到这些画像更多,能感受到他们是缺乏很多判断真伪的技能的。
    %\item 对于这一点,N2(P2)玩家也提了建议,如果游戏中把这些persona的丰富特点外露出来更多,将会更有利于玩家们做游戏内的决策。(“在游戏里叫如何能够给玩家一种弱引导,让我知道如何攻克对点人物?比如说这个假如说我就是思维发散的来讲,比如说这个家庭主妇她有一个孩子,那是不是我在这个辩论中提到这个病毒对孩子的侵害,就跟他打感情牌是不是更有可能获得他的认可?但是这种引导我感觉在这个游戏中是比较弱的,就是然后包括那个护士,有一个退休男护士,好像是我觉得这个机制,我猜测背后机制可能就是我提到更多的事实或者医学方面的事情,会更容易获得他的信服。嗯,我不知道这个后面会不会有这样的算法,但我猜测可能是有的。对的,但是可能只有这个角色我感受到了,剩下的角色这方面的引导,我并没有一个很明确的感受,包括像学生中就是年轻学生这种的,我并不知道他喜欢怎样的说法或者材料,这方面的信息我并没有太思考到。 (N1 N2 0818 Group B Interview, 位置149)”)
%\end{enumerate}
\section{Conclusion}\label{sec:conclusion}
%This work explores the impact of grid-connected and wireless measurement setups on capacitive human body communication, revealing significant differences in both channel \revise{gain} and frequency behavior. 
While conventional data acquisition setups are effective for quantifying the forward path loss, which depends on the conductive properties of the human body, they substantially alter the return path behavior by artificially modifying the capacitive coupling to earth ground.
Therefore, a wireless, wearable-sized data acquisition system is essential for quantitatively evaluating the full \ac{HBC} communication channel in a realistic environment with minimal measurement interference. 
To address this challenge, this work introduces \textit{BodySense}, an evaluation platform for human body communication that is fully wireless, compact enough for wearable applications, and designed for extendability.
To validate the proposed system, the measured channel gains of a classical, grid-connected setup and a wireless setup have been determined for distances of \qty{10}{\centi\meter}, \qty{30}{\centi\meter}, and \qty{50}{\centi\meter} between transmitter and receiver for a frequency range between \qty{4}{\mega\hertz} and \qty{64}{\mega\hertz}.
A comparison between the two scenarios yields an average overestimation of \qty{18.15}{\db} over all investigated distances for the classical case, highlighting the importance of evaluating capacitive \ac{HBC} in realistic conditions.
When comparing the energy consumption of capacitive \ac{HBC} with \ac{BLE}, we achieved results comparable to state-of-the-art \ac{BLE} frontends. 
This demonstrates its potential as a promising alternative to conventional \ac{RF} links, offering opportunities to further enhance the overall energy efficiency of wearable devices and move closer to the realization of battery-free, body-worn sensor nodes.



%This paper proposes \textit{Bodysense}, a fully wireless, wearable-sized system designed to accurately evaluate capacitive human body communication. Experimental evaluation has revealed significant differences in both channel loss and frequency behavior. This paper demonstrated that while conventional data acquisition setups are effective for quantifying the forward path loss, which depends on the conductive properties of the human body, they substantially alter the return path behavior by artificially modifying the capacitive coupling to earth ground. Thus, the proposed wearable-sized data acquisition system is essential for quantitatively evaluating the full \ac{HBC} communication channel in a realistic environment with minimal measurement interference. 
%To address this issue, this paper presents \textit{Bodysense}, a fully wireless, wearable-sized, and extendable evaluation platform for human body communication.
%To validate the proposed system, the measured channel gains of a classical, grid-connected setup and a wireless setup have been determined for distances of \qty{10}{\centi\meter}, \qty{30}{\centi\meter}, and \qty{50}{\centi\meter} between transmitter and receiver for a frequency range between \qty{4}{\mega\hertz} and \qty{64}{\mega\hertz}.
%A comparison between the two scenarios yields an average overestimation of \qty{18.15}{\db} over all investigated distances for the classical case, highlighting the importance of evaluating capacitive \ac{HBC} with a measurement setup that is similar or ideally identical to the envisaged use case.


\begin{acronym}
\acro{gan}[GANs]{Generative Adversarial Networks}
\acro{rl}[RL]{Reinforcement Learning}
\acro{pae}[PAE]{Periodic Autoencoder}
\acro{fld}[FLD]{Fourier Latent Dynamics}
\acro{ppo}[PPO]{Proximal Policy Optimization}
\acro{fft}[FFT]{Fast Fourier Transform}
\acro{pca}[PCA]{Principal Component Analysis}
\acro{dfm}[DFM]{Deep Fourier Mimic}
\acro{dof}[DoF]{Degrees of Freedom}
\acro{mlp}[MLPs]{Multi-Layer Perceptrons}
\end{acronym}

                                   

%---------------------------------------------
% Acknowledgement
%---------------------------------------------
\section*{ACKNOWLEDGMENT}
This work was found by the Swiss National Science Foundation SNSF under the projects “BodyLink: Enabling Battery-free body-worn Sensing and Communication with Energy Transfer” (Grant Nr. 220867) and “Wearable Nano-Opto-electro-mechanic Systems” (Grant Nr. 209675).
% https://data.snf.ch/grants/grant/220867
% https://data.snf.ch/grants/grant/209675
%---------------------------------------------
% Bibliography
%---------------------------------------------
\bibliographystyle{IEEEtranDOI} 
%\bibliographystyle{IEEEtran} 
\input{bib/bodycom.bbl}
% \bibliography{bib/bodycom}

%---------------------------------------------
\end{document}

%% Paper end
