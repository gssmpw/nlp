
\section{System Design}\label{sec:methods}
% Introduce the system's overall concept
\textit{BodySense} is an expandable and wearable-sized wireless evaluation platform for human body communication that has been designed to assess \ac{HBC} under realistic conditions.
With an overall size of \qty{33.5}{\milli\meter}~\(\times\)~\qty{22}{\milli\meter}, it is comparable to commercial smartwatches and is thus representable for wearable-to-wearable application scenarios.
%Further, the core system can easily be extended with application-specific circuitry to speed up the evaluation of various front-end designs.
\autoref{fig:system_blockdiagram} shows a simplified block diagram of the main board, detailed schematics of the designed Tx and Rx frontends, and the hardware realization.

\subsection{System Overview}
As the core platform, \textit{BodySense} integrates essential circuits, including the computational unit and power management subsystem, and provides flexible expandability via an M.2 connector. It is built around the \textit{STM32WB5MMG} module, an ultra-low-power, \ac{BLE}-enabled \ac{SiP} that integrates an ARM Cortex-M4 MCU running at up to \qty{64}{\mega\hertz} as the primary core and an ARM Cortex-M0+ hosting the \ac{BLE} stack. Additionally, the module comprises passive components for power supply stabilization, antenna matching, and clock generation. The extensive set of analog and digital peripherals, combined with its compact form factor of \qty{11}{\milli\meter}~\(\times\)~\qty{7.3}{\milli\meter} and integrated wireless connectivity, makes the module an excellent foundation for orchestrating \textit{BodySense}.

The \ac{PMIC} \textit{nPM1300} supplies power to \textit{BodySense} and its extension board while also monitoring the battery's state of charge and managing recharging. Two integrated software-configurable DC-DC converters power the mixed-signal circuitry.
For the presented Rx and Tx circuits, these converters are configured to output \textit{V1} = \qty{2.7}{\volt} and \textit{V2} = \qty{3.3}{\volt}, respectively. Load switches enable power gating of the extension board, thereby extending battery life. Additionally, a \qty{1}{\giga\bit} \textit{W25Q01JV} NOR flash provides onboard data logging capabilities.

An M.2 connector provides electrical and mechanical connectivity between \textit{BodySense} and the carrier board, offering a versatile interface for future circuit designs and expansion.
Dedicated instances of UART, SPI, and I2C peripherals are directly routed to the M.2 connector.
%, enabling seamless communication with the carrier board.
Additionally, \ac{ADC} inputs, timer inputs, conventional \acs{GPIO} pins, and wake-up pins are made available through the M.2 connector, providing flexibility for custom front-end designs and supporting a wide range of potential extensions.

% Interchangeable extension boards
\subsection{Application-specific M.2 carrier boards:}
Two carrier boards, a receiver (Rx) and a transmitter (Tx), were designed to demonstrate the versatility of the system and to evaluate capacitive \ac{HBC} in realistic application scenarios.
Their detailed schematics are shown in \autoref{fig:system_blockdiagram} (b) and (c).
\revise{Development-accompanying SPICE simulations, followed by verification through benchtop measurements, confirmed the correct functionality under the defined conditions.
However, variances in environmental coupling as well as in the skin impedance can vary by a factor of 10 \cite{maity_modelling_2019} and thus have a much higher impact on the measurements than the frontend's performance variation caused by component tolerances.
As a consequence, testing on the human body is the only way to obtain realistic results.}
%The insertable carrier boards, measuring \qty{22}{\milli\meter}~\(\times\)~\qty{22.5}{\milli\meter}, are slightly shorter than the \textit{BodySense} module.

% Rx circuitry used
\textit{Rx Circuit:} Disposable Ag/AgCl electrodes are used to capture the transmitted signal from the body.
\revise{A passive band-pass filter with corner frequencies f\textsubscript{L}~=~\qty{160}{\hertz} and f\textsubscript{H}~=~\qty{70}{\mega\hertz} suppresses the dominant \qty{50}{\hertz} mains noise and other unwanted external signals from the environment that couple to the body. The subsequent logarithmic amplifier of type \textit{AD8307}, with an operating range from DC to \qty{500}{\mega\hertz} and a dynamic range of \qty{92}{\db}, generates an analog output voltage proportional to the received signal level on a logarithmic scale. It is stabilized using a capacitor and then fed to the \ac{ADC} on the main board for quantization. This frontend topology provides the needed high dynamic range to measure the \ac{RSS}.}
% Tx circuitry used
\textit{Tx Circuit:} 
The internal \ac{PLL} of the \textit{STM32WB5MMG} module is used to generate a rectangular carrier frequency f\textsubscript{carrier}, ranging from \qty{4}{\mega\hertz} up to \qty{64}{\mega\hertz}, directly accessible on the modules GPIO.
An AND gate of type \textit{SN74LVC1G08} is used in combination with an operational amplifier of type \textit{LMH6601}, featuring a high slew rate of \qty{260}{\volt/\micro\second} and a \ac{GBP} of \qty{250}{\mega\hertz}, to generate an OOK-modulated transmit signal with a low output impedance.
Finally, the signal is routed over a snap button to a disposable Ag/AgCl electrode to establish good skin contact.


