\section{Background and Related Work}\label{sec:related_work}

\begin{figure}[!t]
    \centering
    \begin{overpic}[width=1\columnwidth]{./figures/background/ground_coupling_circuit.pdf}
    \end{overpic}
    \caption{\revise{Equivalent circuit for classical DAQ (red) and wireless DAQ (blue).
    }}
    \vspace{-5mm}
    \label{fig:hbc_equivalent}
\end{figure}

Based on the underlying coupling principle, \ac{HBC} is divided into three main methods: \textit{Capacitive coupling (CC)}, \textit{galvanic coupling (GC)}, and \textit{magnetic coupling (MC)}.
Capacitive \ac{HBC} uses a signal electrode connected to the body and a floating ground electrode. It achieves low forward path loss throughout the body, but its performance depends on the return path and load capacitances \cite{m_hbc_nath_2022}, limiting its use in implantable devices \cite{9128564}.
\textit{Galvanic} \ac{HBC} injects differential signals using two body-attached electrodes. 
Signal quality predominantly depends on the electrode size, spacing, and parasitic capacitances.
\textit{Magnetic} \ac{HBC} benefits from low tissue permeability, making it effective for \ac{nLoS} communication and intra-bidy communication \cite{m_hbc_wen_2022}. However, its range is limited by the magnetic near-field.

When comparing the coupling principles presented above, capacitive \ac{HBC} shows the best performance when it comes to longer-distance and low-power body-centered communication \cite{m_hbc_nath_2022}. 
%Capacitive \ac{HBC} utilizes two electrodes for operation: A body electrode and a floating ground electrode. 
The transmitter couples a signal over the skin-attached electrode to the conductive layers under the human skin, utilizing a low-impedance forward path between the transmitters and the receivers' skin electrodes \cite{a_bora_2020}. 
The floating ground electrodes form the return path by direct capacitive coupling between each other for short distances or over the environment for longer distances. Whereas the forward path is subject to minor fluctuations caused by the coupling quality of the skin-electrode and the individual physical composition \cite{maity_modelling_2019}, the return path is heavily dependent on the environment, acting as a limiting factor for reception quality.
Inter-body coupling to other people \cite{nath_2020} or coupling to any larger conductive surface such as measurement equipment, laptop, and cables strengthens the return path, leading to over-optimistic results, see \autoref{fig:hbc_highlevel}, left. This makes it difficult to reliably quantify influences on the communication link in a controlled manner for wearable-to-wearable scenarios \cite{avlani_2020}. 
Hence, the analysis in a wearable-to-wearable scenario is necessary to eliminate overestimation in amplitude and power reception of the received signal that arises from artificially enhanced ground coupling over the attached test equipment, see \autoref{fig:hbc_equivalent}.

\begin{figure*}[!t]
    \centering
    \begin{overpic}[width=\textwidth]{./figures/system/high_level_block_diagram_update.pdf}
        \put(1,33){(a)}
        \put(31,33){(b)}
        \put(31,14.5){(c)}
        \put(73,33){(d)}
    \end{overpic}
    \vspace{-6mm}
    \caption{System overview: (a) High-level block diagram of the main platform \textit{BodySense}, 
    (b) Tx carrier circuit, (c) Rx carrier circuit, (d) \textit{BodySense} platform, equipped with M.2 carrier.}
    \vspace{-5mm}
    \label{fig:system_blockdiagram}
\end{figure*}

Several recent works propose chip-integrated front-ends for capacitive \ac{HBC} achieving \unit{\pico\joule\per\bit} communication links with data rates of up to \qty{20}{\kilo\bit\per\second} \cite{wpt_modak_2022, chbc_comm_maity_2019}, demonstrating a significant reduction in energy consumption for data transfer \cite{chatterjee_2023_annual_review}. 
Likewise, recent works exploit capacitive \ac{HBC} for power transmission, demonstrating the possibility of transferring up to tens of \unit{\micro\watt} over the whole body. 
Modak et al. presented a \ac{HBC} \ac{WPT} IC in a \qty{65}{\nano\meter} process, achieving power transfer of \qty{240}{\micro\watt} in a machine-to-machine configuration (close distance, perfect coupling to earth-ground) and \qty{5}{\micro\watt} in a wearable-to-wearable configuration (long distance, bad coupling to earth-ground) \cite{wpt_modak_2022}.
%Cho et al. implemented an intra-body power transfer (IBPT) system in a \qty{180}{\nano\meter} process that \revise{achieves} a maximal power delivery of \qty{136}{\micro\watt} over a distance of \qty{20}{\centi\meter} with a power delivery efficiency of \qty{8.83}{\percent}. For a worst-case scenario of \qty{150}{\centi\meter}, a maximal power transfer of \qty{21.6}{\micro\watt} with a \qty{7.72}{\percent} efficiency has been reported \cite{wpt_cho_2023}.
%Dong et al. present a power transmitter in a \qty{40}{\nano\meter} process. By dynamically adapting transmission power and frequency, they achieve a maximum power transmission of \qty{22}{\micro\watt} at \qty{105}{\centi\meter} \cite{wpt_dong_2021}.

Although these numbers are remarkable, all works clearly highlight the challenges of capacitive \ac{HBC} due to the variability of electrical characteristics within and between humans and its reliance on an earth-ground coupled return path. These factors make direct comparisons between studies particularly difficult, as results are only partially reproducible \cite{datta_2021}.
Furthermore, previous research predominantly evaluates system performance using bulky, well-coupled laboratory equipment, often leading to overly optimistic outcomes.
In contrast, this work focuses on designing a reliable setup capable of accurately measuring \ac{HBC} performance under realistic, application-oriented conditions.


%For communication, quasi-electrostatic capacitive coupling demonstrated particularly effective, leading to proof-of-concept implementations [28], [29], as well as first attempts for commercialization [30]. 
%By reaching a power consumption of 435 nW and 1450 nW for TX and RX, respectively, Maity et al. [29] surpassed the nJ/bit barrier and demonstrated the energy efficiency of the approach even at low data rates of 1 kbps. 
%In a similar manner, recent research works exploit \ac{HBC}-based coupling for power transmission [31]–[34], demonstrating the possibility of transferring hundreds of µW at meters of distance.

%All works clearly point out the challenges of \ac{HBC}. The influence of the return path significantly impacts
%communication range and energy transfer potential, making an explicit comparison of previous works challenging and resulting in limited reproducibility [35]. 

%Hence, a solid characterization inevitably requires an application-specific analysis (machine-to-machine, wearable-to-machine, wearable-to-wearable) combined with an evaluation of \ac{HBC} implementations on multiple subjects.

%Additionally, the analysis in a completely integrated wearable system is necessary to eliminate overestimation in amplitude and power reception of the received signal due to the coupling of external test equipment with the earth-ground. 

%Although defined in the IEEE 802.15.6 standard, only a few works have implemented \ac{HBC} in a full system [24, 53, 54]. 
%IEEE 802.15.6 introduces three different communication band regions, depending on the specific requirements in a WBAN. The first region contains two center frequencies at 16 MHz and 27 MHz, each with a bandwidth of 4 MHz. 

%When comparing all forms of \ac{HBC} presented above, capacitive \ac{HBC} shows the best performance when it comes to longer-distance and low-power body-centered communication [57]. Several recent works propose silicon-based communication front-ends for capacitive \ac{HBC} achieving pJ/bit communication links with data rates of up to 20 kbit/s [58,59], demonstrating a massive reduction in energy consumption for data transfer [18]. 
%Likewise, recent works in research exploit capacitive \ac{HBC} for power transmission, demonstrating the possibility of transferring up to tens of μW over the whole body. Modak et al. presented \ac{HBC} Wireless Power Transfer (WPT) IC in a 65 nm process, achieving power transfer of 240 μW in a machine-to-machine configuration (close distance, very good coupling to earth-ground) and 5 μW in a wearable-to-wearable configuration (long distance, bad coupling to earth-ground) [58]. Cho et al. implemented an intra-body power transfer (IBPT) system in a 180 nm process that archives a maximal power delivery of 136 μW over a distance of 20 cm with a power delivery efficiency of 8.83 \%. 
%For a worst-case scenario of 150 cm, a maximal power transfer of 21.6 μW with a 7.72 \% efficiency has been reported [60]. Dong et al. present a power transmitter in a 40 nm process. By dynamically adapting transmission power and frequency, they achieve a maximum power transmission of 22 μW at 105 cm. [61].

%Unfortunately, most of the above-mentioned works do not clearly specify their measurement setup, leaving room for assumptions and uncertainty. 
%Independent from this, all works point out the challenges of capacitive \ac{HBC} raised by the variability of electrical characteristics within and between human bodies, and the strong dependency on the earth-ground coupled return path. The environmental dependency of the return path determines the communication range and significantly influences energy transfer, making an accurate comparison and replication of previous works challenging [62]. 

%A solid characterization is unavoidable and requires in the first phase an application-specific in-depth analysis and a performance study conducted with multiple subjects in the second phase.

%Further, a completely wearable analysis system is necessary to eliminate any positive effect due to improved coupling of attached external test equipment with the earth-ground.



%However, when electronics directly interact with living beings, special attention must be paid to safety.
%The International Commission on Non-Ionizing Radiation
%Protection (ICNIRP) published two guidelines for limiting exposure to time-varying, electric, magnetic and electromagnetic fields, one for low frequencies (1 Hz to 100 kHz) [65] and one for high frequencies up to 300 GHz [66].
%Further, the IEEE standard C95.1-2005 defines safety levels on human exposure to RF electromagnetic fields [67]. 
%These guidelines and their defined limitations need to be considered, especially when it comes to Wireless Power Transfer wherethe received power can be easily scaled by the transmission power.

% ======================================================
% --- Non-used text snippets 
% ======================================================
% from SNF spark
%Over the last decade, research and industry have seen significant advancements in the design of miniaturized electronic systems. 
%With the emergence of new materials and fabrication techniques, electronic devices have become smaller, faster, and more energy-efficient. 
%The tremendous improvements in computational performance per watt accelerated the embedding of more complex algorithms close to the sensor, making systems increasingly smarter and more autonomous [1], [2], [12].
%Another significant trend has been the development of flexible and stretchable electronics, which allow electronic devices to conform to irregular surfaces, such as the human body [13], [14]. 
%This has led to the development of wearable electronic devices that can monitor health, track fitness, and even administer medication [15]–[17].

%Despite all improvements, the energy supply remains a major bottleneck of today's wearable devices, limiting intelligence, availability, and user experience. 
%The most energy-consuming subsystem is typically the radio frequency (RF) wireless transmission circuitry used for communication between individual wearables or remote hubs [3], [4]. 
%This influence on the power budget is even further exacerbated by an increasing number of smart and wireless sensors distributed over the body,emphasizing the importance of efficient communication.
%A prominent approach to reducing power consumption in the RF subsystem is duty cycling, which involves periodically turning off the communication circuits to conserve energy [4], [5], [18]. 
%While duty cycling can be effective, it inevitably introduces latency and does not eliminate power consumption during idle listening. 
%An alternative approach is the design of always-on wireless receivers that have the
%ability to detect wireless messages of interest while consuming power in the micro- to nano-watt range [10], [19], [20]. 

%Recently, this approach has been further developed with the addition of zero-power
%communication, which utilizes energy harvesting from received power to achieve energy neutrality [21]–[23]. 
%The ultimate goal of zero-power communication is to enable always-on sensors that can operate without the need for a battery.

%A novel and emerging communication approach that has the potential to solve the limitations of stateof-the-art RF-based solutions for Wireless Body Area Networks (WBAN) is Human Body Communication (HBC) [7], [8].
%Unlike conventional wireless transmission techniques, such as Bluetooth (medium short range) or RFID (short range), \ac{HBC} utilizes the electrical conductivity of the human body to transmit signals between devices [9]. 
%This allows by design to achieve secure and localized data transmission combined with low-power consumption, making \ac{HBC} an active and promising research field [10], [11].
%However, due to the challenges of the variability of the human body's electrical characteristics, most works did not pass the conceptual and fundamental phases [24]–[27].

%For communication, quasi-electrostatic capacitive coupling demonstrated particularly effective leading to proof-of-concept implementations [28], [29], as well as first attempts for commercialization [30]. 
%By reaching a power consumption of 435 nW and 1450 nW for TX and RX, respectively, Maity et al. [29] surpassed the nJ/bit barrier and demonstrated the energy efficiency of the approach even at low data rates of 1 kbps. 
%In a similar manner, recent research works exploit \ac{HBC}-based coupling for power transmission [31]–[34], demonstrating the possibility of transferring hundreds of µW at meters of distance.

%All works clearly point out the challenges of \ac{HBC}. The influence of the return path significantly impacts communication range and energy transfer potential, making an explicit comparison of previous works challenging and resulting in limited reproducibility [35]. 

%Hence, a solid characterization inevitably requires an application-specific analysis (machine-to-machine, wearable-to-machine, wearable-to-wearable) combined with an evaluation of \ac{HBC} implementations on multiple subjects.

%Additionally, the analysis in a completely integrated wearable system is necessary to eliminate overestimation in amplitude and power reception of the received signal due to the coupling of external test equipment with the earth-ground. 



%Ultimately power management strategies have to be found to address the variations of power intake on the receiver side.
%This project aims at addressing the energy supply challenges of today's wearable devices by designing battery-free body-worn sensing nodes based on the \ac{HBC} approach.
%Achieving this ambitious objective requires a comprehensive evaluation of \ac{HBC} and the return path variation, considering both the specific applications and the inter-subject variability. 
%Based on that, we aim to leverage the unique characteristics of data and energy transfer to design battery-free sensor nodes.
%Ultimately, the project targets to develop a swarm-like network of body-worn sensors that exchange information and energy, enabling real-time monitoring of human physiology and environmental conditions.


% From research plan
%Body-centered human interfaces and feedback systems represent a rising field of research.
%Bridging the gap between digital and physical experiences promises to redefine how we interact with technologies [31–33] and manage our health and well-being. 
%With the improvement of manufacturing techniques and the implementation of novel processing strategies at the silicon level over the past years, electronic devices have become smaller, faster, and more energy-efficient. 
%The significant improvement in computational performance accelerated the development and deployment of more complex algorithms and ML models directly close to the sensor [34] or even in the sensor [35], making systems smarter and more self-sufficient. 
%Additionally, recent advancements in the development of flexible and stretchable electronics [4], as well as e-textiles [23], allow electronics to have more complex shapes to adapt to a variety of surfaces, for example, the human body [32,36].

%By integrating them into clothes, having them directly attached to the human body, or implanted below the skin, these electronic devices can be used to measure various activity parameters and health metrics like wrist pulses [5] and Electrocardiogram (ECG) [37], galvanic skin response [38], non-invasive glucose monitoring [39] and others.
%These advancements have fueled the development of Wirelss Body Area Networks and led to the rise of the Internet of Bodies. 
%First mentioned in 1996 by T. Zimmerman [40], a WBAN should enable communication, on, around, and near the human body. 
%To give a standard to ensure reliability in communication and guarantee Quality of Service (QoS) to this rising technology, the IEEE 802.15 standard which specifies the Wireless Personal Area Network (WPAN) standards, has been extended in 2012 with the IEEE 802.15.6 standard for body area networks [41].

%IEEE 802.15.6 introduces three different communication band regions, depending on the specific requirements in a WBAN. 
%The first region contains two center frequencies at 16 MHz and 27 MHz, each with a bandwidth of 4 MHz. 
%The second region encompasses several frequency bands with different bandwidths for Narrowband Communication (NB) between 400 MHz and 2.4 GHz. 
%Short-range communication standards such as Wireless Local Area Network (WLAN), BLE, UWB also fall into this category. 
%The third region includes high-frequency bands for Ultra-Wideband located between 3.2 GHz - 4.7 GHz and 6.2 GHz - 10.3 GHz with a bandwidth of 499 MHz.
%Although IEEE 802.15.6 supports many frequency bands, most SoTA implementations use conventional communication methods that are located in the ISM band such as WLAN, BLE, and ZigBee [26].
%They are an appropriate option when the emphasis is on simple integration because of the already existing infrastructure and the vast number of communication submodules available. 
%However, despite all the improvements in several research areas, the energy supply is the major bottleneck of today's wearable systems. 
%While reducing energy consumption by implementing context-aware sensing [28,42], event-triggered sampling [43,44], or even ADC-less data acquisition front-ends [45] is promising, more energy can be saved by first optimizing power-hungry subsystems. 

%The most energy-consuming part is typically the RF wireless transceiver subsystem used for communication between individual wearable devices or the gateway within a WBAN [42]. 
%With an increasing number of smart wireless bioelectronic sensors distributed on and around the body, the energy budget is becoming more and more strained [18]. 
%It highlights the strong need for efficient and effective wireless communication.
%A promising approach to reducing power consumption in the communication subsystem is duty cycling to preserve energy [46]. 

%While duty cycling can be effective for synchronized systems, it unavoidably introduces latency within the network and does not eliminate power consumption during listening in the idle state.
%An alternative concept involves the design of wireless always-on receivers that can detect packets of interest while consuming power in the micro-watt range [47–49].
%Recently, this concept has been extended to achieve zero-power reception for air-based RF communication. 
%By adding energy harvesting capabilities to the communication submodule, energy extracted from the received message can be used for decoding and achieving energy neutrality by simultaneously transmitting information and data wirelessly [50]. 
%The next step in this direction would go beyond energy-neutral communication subsequently enabling always-on sensors that can operate without a battery [51, 52]. 
%An emerging communication approach within the domain of body-centered WBANs is wireless on-body communication. 
%Unlike conventional air-based wireless transmission techniques, \ac{HBC} utilizes the electrical conductivity of the human body to transmit signals between devices [40]. 
%This allows by design to establish secure low-power data transmission [22] on and in very close proximity to the body, consuming orders of magnitude less energy for data transmission than conventional RF communication technologies [43]. 

%Although defined in the IEEE 802.15.6 standard, only a few works have implemented \ac{HBC} in a full system [24, 53, 54]. 
%There are existing several different forms of \ac{HBC}, namely capacitive \ac{HBC} [19], galvanic \ac{HBC} [20], and magnetic \ac{HBC} [21]. 
%Depending on the application and optimization target, another form of \ac{HBC} might be preferred.

%When comparing all forms of \ac{HBC} presented above, capacitive \ac{HBC} shows the best performance when it comes to longer-distance and low-power body-centered communication [57]. 
%Several recent works propose silicon-based communication front-ends for capacitive \ac{HBC} achieving pJ/bit communication links with data rates of up to 20 kbit/s [58,59], demonstrating a massive reduction in energy consumption for data transfer [18]. 
%Likewise, recent works in research exploit capacitive \ac{HBC} for power transmission, demonstrating the possibility of transferring up to tens of μW over the whole body. 
%Modak et al. presented \ac{HBC} Wireless Power Transfer (WPT) IC in a 65 nm process, achieving power transfer of 240 μW in a machine-to-machine configuration (close distance, very good coupling to earth-ground) and 5 μW in a wearable-to-wearable configuration (long distance, bad coupling to earth-ground) [58]. 
%Cho et al. implemented an intra-body power transfer (IBPT) system in a 180 nm process that archives a maximal power delivery of 136 μW over a distance of 20 cm with a power delivery efficiency of 8.83 \%. 
%For a worst-case scenario of 150 cm, a maximal power transfer of 21.6 μW with a 7.72 \% efficiency has been reported [60]. 
%Dong et al. present a power transmitter in a 40 nm process. By dynamically adapting transmission power and frequency, they achieve a maximum power transmission of 22 μW at 105 cm. [61].
%Unfortunately, most of the above-mentioned works do not clearly specify their measurement setup, leaving room for assumptions and uncertainty. 
%Independent from this, all works point out the challenges of capacitive \ac{HBC} raised by the variability of electrical characteristics within and between human bodies, and the strong dependency on the earth-ground coupled return path.
%The environmental dependency of the return path determines the communication range and significantly influences energy transfer, making an accurate comparison and replication of previous works challenging [62]. 
%A solid characterization is unavoidable and requires in the first phase an application-specific in-depth analysis and a performance study conducted with multiple subjects in the second phase.
%Further, a completely wearable analysis system is necessary to eliminate any positive effect due to improved coupling of attached external test equipment with the earth-ground.

%Combining capacitive coupling with conductive clothing [54, 63] or flexible electronics [64] are promising solutions to improve the return path and thus increase the transmitted power.
%However, when electronics directly interact with living beings, special attention must be paid to safety.
%The International Commission on Non-Ionizing Radiation Protection (ICNIRP) published two guidelines for limiting exposure to time-varying, electric, magnetic and electromagnetic fields, one for low frequencies (1 Hz to 100 kHz) [65] and one for high frequencies up to 300 GHz [66].
%Further, the IEEE standard C95.1-2005 defines safety levels on human exposure to RF electromagnetic fields [67]. 
%These guidelines and their defined limitations need to be considered, especially when it comes to Wireless Power Transfer wherethe received power can be easily scaled by the transmission power.