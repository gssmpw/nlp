\section{Introduction}
Wearable, intelligent, and unobtrusive sensor nodes monitoring the human body and its environment have attracted significant attention from researchers and industry \cite{svertoka_2020_industry_wearables, design_wearabe_ates_2022}. These devices, equipped with wireless interfaces and an increasing number of sensors, are already a commercial reality for sports \& fitness, with expanding capabilities aimed at collecting valuable data for human-centric healthcare \cite{flex_electronics_wu_2023}. Advances in system-on-chip integration and flexible electronics have enhanced sensor integration on dynamic, non-planar surfaces like the human body, enabling localized physiological signal acquisition and processing \cite{roadmap_flex_luo_2023, flex_wrist_pulse_lee_2023, jose_2024_e_textile}. However, distributing multiple sensors on body locations with beneficial sensing characteristics inevitably increases the stress on the communication interface due to growing synchronization and connection overhead \cite{iob_survey_celik_2022, eff_sensor_node_adway_2024}. 

\begin{figure}[!t]
    \centering
    \begin{overpic}[width=1\columnwidth]{./figures/background/ground_coupling_problem.pdf}
    \end{overpic}
    \caption{\revise{Comparison between \ac{HBC} \ac{DAQ} setups:
    (left) Classical \ac{DAQ} setup using grid-powered measurement tools, (right) Miniaturized and fully-wireless \ac{DAQ} setup}.
    %The forward path established over conductive skin layers is marked in green, and the capacitive return path is approximated with capacitances.
    }
    \vspace{-5mm}
    \label{fig:hbc_highlevel}
\end{figure}

Today's wearable devices rely predominantly on \ac{RF} wireless transmission, already one of the most expensive subsystems, consuming significant power and limiting device lifetime and usability \cite{ WBAN_energy_basic_2016, tinybird_schulthess_2023}. Due to its widespread adoption and interoperability, \ac{BLE} has emerged as the de facto standard for wearable communications. 
However, \ac{BLE}, like other low-power wireless \ac{RF} transmission technologies, may not be the optimal choice for all wearable applications. It faces challenges such as relatively high power consumption and vulnerability to security risks \cite{wi-R_datta_2023}.


These limitations can make \ac{BLE} less suited for ensuring consistently reliable, energy-efficient, and pervasive operation in wearables.
To reduce the impact of the communication interface on the energy budget, \ac{RF} devices can be periodically deactivated, a method known as duty cycling. However, while duty cycling can be effective, it inevitably introduces latency and does not eliminate power consumption during idle listening. An alternative approach is the design of always-on wireless receivers that have the ability to detect wireless messages of interest while consuming power in the micro to nano-watt range \cite{wur_piyare_2017, wur_shellhammer_2023}.
However, these solutions fail to enable battery-free and, thus, truly pervasive operation. A paradigm shift in communication and power management is essential to realize the vision of a \ac{WBAN} with distributed, battery-free, on-body wearables.

% SoTA wearable and outlook --> towards smaller and self-sustaining systems
% Body-wide distributed sensing
%With the increase of on-body wearable devices for both commercial \cite{svertoka_2020_industry_wearables} and research-based \cite{design_wearabe_ates_2022} applications and the efforts in miniaturization, new challenges arise.
%Advances in flexible sensors \cite{roadmap_flex_luo_2023, flex_wrist_pulse_lee_2023} and e-textiles \cite{jose_2024_e_textile} have enabled the improved integration of sensors onto the non-planar and dynamic surface of the human body. Distributing multiple smart and flexible sensors on body locations forming a locally constraint \ac{WBAN} has the potential to improve sensing and enhance overall situational awareness \cite{flex_electronics_wu_2023}.
%However, powering such small systems remains challenging, and the communication between multiple sensors on body locations inevitably increases the power needs of the communication interface due to growing synchronization and communication overhead.
%Today, wearable devices predominantly rely on \ac{RF} wireless transmission, one of the most power-hungry subsystems, %limiting the device's lifetime and usability \cite{tinybird_schulthess_2023}.
%To reduce the impact of the communication interface on the energy budget, the communication activity can be periodically reduced, or wake-up circuits can be implemented to allow efficient listening \cite{wur_shellhammer_2023}.
%However, these solutions often increase latency, only partly solve the energy limitations, and ultimately do not allow battery-free and truly pervasive operation.
%Thus, to enable the vision of a \ac{WBAN} with distributed and battery-free on-body wearables, a fundamental change in communication and power management scheme is required to reduce power consumption while maintaining low latency and small size. 
% Introduce and explain capacitive HBC
%To realize the vision of a \ac{WBAN} with distributed, battery-free, on-body wearables, a paradigm shift in communication and power management is essential. 
Emerging techniques like \acs{HBC} offer a promising alternative. By using the conductive properties of the human body as a communication medium, HBC can address the limitations of \ac{BLE} and other RF wireless communication by significantly reducing power consumption, mitigating external interference, and enhancing security against external attacks. \cite{das_body_eqs_2019, datta_2021, chatterjee_2023_annual_review}. 
Unlike conventional \ac{RF} transmission techniques, such as \ac{BLE}, \ac{UWB}, or \ac{RFID}, capacitive \ac{HBC} employs the electrical conductivity of the human body to exchange information between devices \cite{hbc_zimmerman_1996}. Thanks to that property, capacitive \ac{HBC} has the potential for secure \cite{nath_2020, yang_2022}, body-constraint, and efficient data and energy transmission \cite{shukla_2019, wpt_dong_2021, wpt_modak_2022}, making it an ideal solution for future wearable devices \cite{chatterjee_2023_annual_review}.

This work presents a wearable-sized evaluation platform for \ac{HBC}. Custom-designed and compact, it enables \ac{HBC} evaluations in wearable-to-wearable application scenarios by minimizing measurement errors caused by artificially strengthened return paths. Comparable in size to commercial smartwatches, the modular evaluation platform, named \textit{BodySense}, is tailored to allow testing under realistic conditions. This platform will be used to evaluate the potential for creating energy-efficient body sensor networks. In particular, this article presents the following contributions.

\begin{enumerate}
\item The design of a versatile, expandable, wearable-sized wireless evaluation platform for \ac{HBC}, enabling practical and reliable measurements. 
\item An analysis of the channel gain across multiple transmission distances at frequencies between \qty{4} {\mega\hertz} and \qty{64}{\mega\hertz}, for both wearable-to-wearable and wearable-to-grid-connected application scenarios.
\item The demonstration of the significance of a fully wearable evaluation setup for capacitive \ac{HBC} by comparing the channel gains between classical and wearable \ac{DAQ}.
\end{enumerate}

The rest of this article is organized as follows: \revise{Section II} positions this work within the context of the \revise{state of the art}. \revise{Section III} details the design and implementation of the \textit{BodySense} hardware platform. The evaluation setup topology is outlined in \autoref{sec:experimental_setup}, with the corresponding measurement results presented in \autoref{sec:results}. Finally, \autoref{sec:conclusion} summarizes the key findings and concludes this work.

% ======================================================
% --- Non-used text snippets 
% ======================================================
%In terms of physical measurements, measurements on phantoms mimicking the human body remove the inter-person dependency, ranging from solid phantoms \cite{4267903} to gel-based ones, which are state-of-the-art solution in \acrfull{mri} research due to their \acrfull{em} wave propagation properties being similar to human tissue. 
%In \cite{khorshid_ibcfap_2019}, the authors developed a gel model with five concentric layers of a gelatinous solution with a good performance from 100 kHz to 100 MHz. 
%The obtained results for the channel path loss showed a good agreement with experiments conducted on three human test subjects for a range of 100 kHz to 100 MHz. 

%Designing a power transmission system, the capacitive coupling is influenced by different factors, such as distance between transmitter and receiver, electrode size and configuration, etc. 
%For varying the device distance between the power transmitter and receiver, varying results exist. 
%Callejon et al. observed a difference of 10dB between 15cm and 125cm \cite{callejon_comprehensive_2013}, as have Zhu et al. in measurements with human test subjects \cite{zhu_investigation_2017}. 
%Higher path losses were found by Callejon et al. with be 5 dB per 20cm of and Datta et al. \cite{datta_advanced_2021} with a loss of 30 to 40 dB between 15cm and 120-180cm.
%Contact of the signal electrode with the skin was shown to increase the received signal \cite{fujii_electric_2007}, especially for distances above 15cm \cite{naranjo-hernandez_past_2018}. 
%However, some studies showed that a small distance up to a couple of mm can also still suffice for capacitive coupling to the skin \cite{shukla_skinnypower_2019}, while others observed a decrease in received signal strength for loose contact \cite{callejon_comprehensive_2013}. 
%Wet electrodes showed a general better performance than dry copper ones by 10 dB for up to 60 MHz. 

%There have been two main implementations of capacitive power transfer systems: SkinnyPower \cite{shukla_skinnypower_2019} implements a battery-powered bracelet that transmits power to a battery-less ring for gesture recognition purposes.
%Over a distance of 10cm and an input power of 6.9 mW the authors achieved a power transfer rate of 14.5\%, with a transmitted power of roughly 1 mW. 
%The best results were obtained for a frequency of 100 MHz with 1003.8 $\pm$ 104.9 $\mu$W \cite{shukla_skinnypower_2019}. 
%In contrast, BoudyCoupled focused on achieving more considerable distances: 2 $\mu$W were transmitted over 160 cm with a 1.2mW custom designed transmitter, also integrating ambient RF harvesting. 
%The receiver is designed on a \acrshort{pcb} with a four-diode bridge rectifier (using Skyworks SMS7621-060).
%The recovered power ranged from 52 $\mu$W at 15cm to 2 $\mu$W at 160cm for the above specified input power and voltage. 
%For 10 $V_{pp}$ the received power went up to 0.8 mW. \textcolor{red}{refs}

%For capacitive transmission, frequencies between 1 and 150MHz \cite{das_enabling_2019, li_body-coupled_2021, shukla_skinnypower_2019} are typically employed, with beyond 150MHz the transmission becoming inefficient \cite{nath_inter-body_2021}.

%Attaining precise models to understand the capacitive coupling behavior in the wanted frequencies is vital to reproducible measurements, usable implementations, and comparable research. 
%To model the complexity of the coupling, there are two main approaches in literature: Electric equivalent circuits from simple models \cite{zimmerman_personal_1996} to including the environment \cite{naranjo-hernandez_past_2018, pereira_characterization_2015} and numerical simulations \cite{celik_internet_2022,datta_advanced_2021}.

% The requirement to recharge batteries prevents continuous medical monitoring \cite{mayer_2021_energy} and a truly unobtrusive usage \cite{yao_2021_unobstrusive_wearable}.

%As a consequence, methods to power these devices are actively researched, ranging from environmental energy harvesting, such as.... to body powered, such as .... and also  \textcolor{red}{refs}.

%This, together with reduced power consumption and novel energy harvesting techniques, paves the way for the next generation of wearable and multisensor networks \cite{wearable_30_2019, design_wearabe_ates_2022}

% ======================================================
% from SNF spark
% ======================================================
%Wearable, intelligent, and unobtrusive sensor nodes that monitor the human body and the surrounding environment gained significant attention from researchers and industry. 
%These devices, equipped with wireless interfaces and an increasing number of sensors, are already a commercial reality for sports & fitness, where industry leaders such as Apple and Samsung are pushing the boundaries. 
%Recent advances in system-on-chip integration and flexible electronics fabrication have enabled the improved integration of sensors onto the non-planar and dynamic surface of the human body opening new ways of localized acquisition and processing of physiological signals [1], [2]. 
%However, distributing multiple smart and flexible sensors on body locations with beneficial sensing characteristics inevitably increases the stress on the communication interface due to growing synchronization and communication overhead.
%Today's wearable devices rely predominantly on radio frequency (RF) wireless transmission, already one of the most expensive subsystems, consuming significant power and limiting device lifetime and usability [3], [4]. 
%To reduce the impact of the communication interface on the energy budget, the subsystem's activity can be periodically reduced, or wake-up circuits can be implemented to allow efficient listening [5], [6]. 
%However, these solutions often increase latency, only partly solve the energy limitations, and ultimately do not allow battery-free and truly pervasive operation.
%Thus, to enable the vision of distributed and battery-free wearable sensors, a fundamental change in communication and power management scheme is required to reduce power consumption while maintaining low latency and small size. 

%A novel and emerging communication approach that has the potential to solve the limitations of state-of-the-art systems is Human Body Communication (HBC) [7], [8]. 
%Unlike conventional RF transmission techniques, such as Bluetooth (medium short range) or RFID (short range), HBC employs the electrical conductivity of the human body to transmit signals between devices [9]. 
%Based on that unique property, HBC has the potential for secure, localized, and efficient data and energy transmission, making it an ideal solution for wearable devices [10], [11].

%This project aims at exploiting HBC for body area networks that combine the unique characteristics of data transfer, energy transfer, and indirect sensing. 
%Ultimately the project shall lead to the design ofheterogeneous body-worn sensing nodes that allow swarm-like data acquisition while exchanging information and energy.
%Battery-free nodes will be implemented in hybrid (rigid/flex) technology by exploiting the latest advances in additive circuit manufacturing. 
%Finally, using embedding lightweight AI algorithms, the aim is to analyze the changes in channel characteristics and power transfer properties for indirect sensing of body movements and power management.