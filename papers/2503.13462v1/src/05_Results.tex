\section{Results}\label{sec:results}
% Describe the channel gain results for both scenarios
The measurement results for the classical as well as for the wireless \ac{DAQ} scenario are discussed. 
%The obtained channel gains show similar behavior over all frequencies for the distances investigated. 
In the classical scenario (\autoref{fig:results_channel_gains}, solid lines), the lower frequencies between \qty{4}{\mega\hertz} and \qty{20}{\mega\hertz}, the channel gains show an almost identical frequency dependency for all distances. 
Whereas the more considerable distances (\qty{30}{\centi\meter} and \qty{50}{\centi\meter}) achieve almost identical results, it shows an improvement of at least \qty{2.6}{\db} for the \qty{10}{\centi\meter}-distance in this frequency region.
For higher frequencies above \qty{20}{\mega\hertz}, the difference between \qty{10}{\centi\meter} and the larger distances start to increase.
Direct coupling between the transmitters and receivers floating ground electrode, illustrated as \(C_{INT}\) in \autoref{fig:hbc_equivalent}, enhances the return path for short distances.
The largest channel gain with a peak value of \qty{-39,3}{\db} has been achieved for \qty{38}{\mega\hertz} at \qty{10}{\centi\meter} distance.
Above \qty{36}{\mega\hertz}, the \qty{50}{\centi\meter}-distance shows a superior channel gain than for \qty{30}{\centi\meter}, indicating an artificially improved return path caused by the wired data transmission.
Overall, all three measurement series of the classical scenario are close to each other, showing a maximum difference of \qty{8}{\db} at \qty{36}{\mega\hertz}.

For the results of the wireless scenario (\autoref{fig:results_channel_gains}, dashed lines),
it is interesting to see that the channel gain curves show a high correlation of 0.88 between the \qty{10}{\centi\meter} and \qty{50}{\centi\meter}, 0.95 between \qty{10}{\centi\meter} and \qty{30}{\centi\meter}, and 0.98 between the \qty{30}{\centi\meter} and \qty{50}{\centi\meter} curves. 
Similar to the results of the classical scenario, the channel gain for \qty{10}{\centi\meter} outperforms the more considerable distances by at least \qty{10}{\db}. 

\autoref{fig:results_rx_power} shows the received power for both the classical and the wireless receiver scenario.
The wired scenario demonstrates superior performance compared to the wireless setup across all distances, achieving a maximum power reception of \qty{-35.53}{\dbm} at \qty{38}{\mega\hertz} and a distance of \qty{10}{\centi\meter}.
The Rx power behaves almost constant for longer distances in the wireless scenario, achieving a minimal power reception \qty{-67.49}{\dbm} with an overall fluctuation of \qty{0.91}{\db} over the whole frequency range.

\begin{figure}
    \centering
    \begin{overpic}[width=\columnwidth]{./figures/results/wired_wireless_channel_gain_update.pdf}
    \end{overpic}
    \vspace{-5mm}
    \caption{Channel gains measured over different frequencies for the classical and the wireless receiver scenario. %as shown in \autoref{fig:experimental_setup}.
    }
    \label{fig:results_channel_gains}
    \vspace{-3mm}
\end{figure}

\begin{figure}[t]
    \centering
    \begin{overpic}[width=\columnwidth]{./figures/results/wired_wireless_rx_power_update.pdf}
    \end{overpic}
    \vspace{-5mm}
    \caption{Received power measured over different frequencies for the classical and the wireless receiver scenario.% as shown in \autoref{fig:experimental_setup}.
    }
    \label{fig:results_rx_power}
        \vspace{-6mm}
\end{figure}

% Conclude and compare
The experimental results visualized in \autoref{fig:results_channel_gains} and \autoref{fig:results_rx_power} show that the classical scenario achieves a significantly higher channel gain than the wearable scenario.
It is essential to recognize that the \qty{10}{\centi\meter} distance outperforms the longer distances in both scenarios. 
Whereas the \qty{10}{\centi\meter}-case is only \qty{2.6}{\db} superior in the classical case, the channel gains curves for \qty{30}{\centi\meter} and \qty{50}{\centi\meter} lie at least \qty{10}{\db} below the \qty{10}{\centi\meter}-curve in the wireless setup.
This is due to the strong inter-device coupling between the transmitter and receiver at close distances.
In contrast to the wireless case, this effect is significantly less pronounced in the wired scenario, as all distances benefit from improved coupling.
When comparing the channel gain curves for the two setups, one can readily observe that the channel gain has increased by an average of \qty{11.17}{\db} for the \qty{10}{\centi\meter}, \qty{20.34}{\db} for \qty{30}{\centi\meter}, and \qty{22.95}{\db} for the \qty{50}{\centi\meter} distance for the wired scenario.
This proofs and quantifies the significant impact of classical \ac{DAQ} on measurements in the realm of capacitive \ac{HBC}.
\revise{The obtained channel gain for wireless \ac{DAQ} shows comparable performance as described in \cite{avlani_2020}. However, a quantitative comparison of this work to the \ac{SoA} remains challenging as differences in the test setup, like device size and environment, can noticeably influence the measured absolute value.
Using the same test setup minimizes the board's effects, as capacitive coupling to the environment dominates the received signal.}
%Using the same hardware for both scenarios 
%Moreover, the relative comparison between classical and wireless \ac{DAQ}, both measured with the same setup to address any system-induced bias, provides reliable results. Uncertainties in the measured quantities can arise from component tolerances

To compare the communication power of capacitive \ac{HBC} with \ac{BLE}, the most prominent \ac{RF} communication method for wearables, we define energy efficiency as a function of the data rate (energy/bit). When taking the measured Tx power of \qty{2.71}{\milli\watt} at \qty{64}{\mega\hertz} and assuming a reasonable data rate of \qty{1}{\mbps} \cite{wi-R_white_paper}, we achieve \qty[per-mode = symbol]{2.71}{\nano\joule\per\bit}. This is comparable to Bluetooth, which achieves energy efficiencies larger than \qty[per-mode = symbol]{1}{\nano\joule\per\bit} up to \qty[per-mode = symbol]{15}{\nano\joule\per\bit} \cite{wi-R_datta_2023}.
With a simple Tx \revise{frontend}, capacitive \ac{HBC} is able to achieve comparable energy efficiency to \ac{BLE}.
\revise{With a Tx power of \qty{2.71}{\milli\watt} at \qty{64}{\mega\hertz}, the maximal received power for distances of \qty{10}{\centi\meter}, \qty{30}{\centi\meter}, and \qty{50}{\centi\meter} lies at \qty{-60}{\dbm}, \qty{-69.9}{\dbm}, and \qty{-71.9}{\dbm}, respectively. Although these power levels are very small, they appear to converge to a fixed value at larger distances. This makes them interesting for energy harvesting applications, especially if the baseline of the received power can be increased.}


% tx power at 64 MHz = 2.71 mW, assuming 1Mbps (reasonable value \cite{wi-R_white_paper}) data rate we get 2.71 nJ/ per bit (2.71mW / 1Mbps) that is comparable to bluetooth which is between  >1nJ/bit up to 15nJ/bit \cite{wi-R_datta_2023}
% even with a simple and unoptimized transmitter system, capacitive \ac{HBC} is comparable to \ac{BLE} in terms of communication efficiency
% very weak argument. what about modulation and encoding?












% ======================================================
% --- Previous version
% ======================================================
% \begin{figure}[t]
%     \centering
%     \begin{overpic}[width=\columnwidth]{./figures/results/wired_channel_gain.pdf}
%     \end{overpic}
%     \caption{Channel gain measured over different frequencies of the grid-connected receiver scenario, as shown in \autoref{fig:experimental_setup} (b).}
%     \label{fig:results_graphs_grid}
% \end{figure}

% % Describe the results of the grid-connected experiment
% The measurement results for the grid-connected scenario are presented in \autoref{fig:results_graphs_grid}. 
% %The obtained channel gains show similar behavior over all frequencies for the distances investigated. 
% For the lower frequencies between \qty{4}{\mega\hertz} and \qty{20}{\mega\hertz}, the channel gains show an almost identical frequency dependency for all distances. 
% Whereas the more considerable distances (\qty{30}{\centi\meter} and \qty{50}{\centi\meter}) achieve almost identical results, it shows an improvement of \qty{2.6}{\db} for the \qty{10}{\centi\meter}-distance for this frequency region.
% For higher frequencies above \qty{20}{\mega\hertz}, the difference between \qty{10}{\centi\meter} and the larger distances start to increase.
% Direct coupling between the transmitters and receivers floating ground electrode, illustrated as \(C_{INT}\) in \autoref{fig:hbc_highlevel}, enhances the return path for short distances.
% The largest channel gain with a peak value of \qty{-39,3}{\db} has been achieved for \qty{38}{\mega\hertz} at \qty{30}{\centi\meter} distance.
% Above \qty{36}{\mega\hertz} the \qty{50}{\centi\meter}-distance shows a superior channel gain than for \qty{30}{\centi\meter}, indicating an artificially improved return path caused by the wired data transmission.
% Overall, all three measurement series are close to each other, showing a maximum difference of \qty{8}{\db} at \qty{36}{\mega\hertz}.
% \begin{figure}
%     \centering
%     \begin{overpic}[width=\columnwidth]{./figures/results/wireless_channel_gain.pdf}
%     \end{overpic}
%     \caption{Channel gain measured over different frequencies with a battery-powered wireless receiver, as shown in \autoref{fig:experimental_setup} (a).}
%     \label{fig:results_graphs_wireless}
% \end{figure}

% % Describe the results of the wireless experiment
% The results for the wireless scenario are shown in \autoref{fig:results_graphs_wireless}.
% It is interesting to see that the channel gain curves show a high correlation of 0.88 between the \qty{10}{\centi\meter} and \qty{50}{\centi\meter}, 0.95 between \qty{10}{\centi\meter} and \qty{30}{\centi\meter}, and 0.98 between the \qty{30}{\centi\meter}- and \qty{50}{\centi\meter} curves. 
% Similar to the results of the grid-connected scenario, the channel gain for \qty{10}{\centi\meter} outperforms the more considerable distances by at least \qty{10}{\db}.

% % Conclude and compare
% From the experimental results shown in \autoref{fig:results_graphs_grid} and \autoref{fig:results_graphs_wireless}, it is evident that the grid-connected scenario achieves a significantly higher channel gain compared to the wearable scenario.
% It is essential to recognize that the \qty{10}{\centi\meter} distance outperforms the longer distances in both scenarios. 
% Whereas the \qty{10}{\centi\meter}-case is only \qty{2.6}{\db} superior in the grid-connected case, the channel gains curves for \qty{30}{\centi\meter} and \qty{50}{\centi\meter} lie at least \qty{10}{\db} below the \qty{10}{\centi\meter}-curve in the wireless setup.
% This is because of the strong inter-device coupling between transmitter and receiver for close distances.
% In contrast to the wireless case, this effect is significantly less pronounced in the wired scenario, as all distances benefit from improved coupling.

% When comparing the channel gain curves for the two setups, one can readily observe that the channel gain has increased by an average of \qty{11.17}{\db} for the \qty{10}{\centi\meter}, \qty{20.34}{\db} for \qty{30}{\centi\meter}, and \qty{22.95}{\db} for the \qty{50}{\centi\meter} distance for the wired scenario. This proofs and quantifies the significant impact of classical \ac{DAQ} on measurements in the realm of capacitive \ac{HBC}.




%\begin{figure*}[!ht]
%   \centering
%    \begin{overpic}[width=\textwidth]{./figures/results/results_compilation.pdf}
%        \put(2,27.5){(a)}
%        \put(36,27.5){(b)}
%        \put(70,27.5){(c)}
%   \end{overpic}
%    \caption{\todo{ad space at "30cm" increase axis text, add wired wireless experiment option}.}
    %\vspace{-5mm}
%    \label{fig:results_graphs}
%\end{figure*}


% The skin impedance is also shown to be in the order of Kohm as found by previous studies \cite{a_wegmuller_2010}, [26]–[28]. The internal fat and conductive tissue have an impedance in the order of 100 s of ohms [26], [29].