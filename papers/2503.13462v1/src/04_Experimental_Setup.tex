\section{Experimental Setup}\label{sec:experimental_setup}

\begin{figure}
    \centering
    \begin{overpic}[width=\columnwidth]{./figures/experimental_setup/experimental_setup.pdf}
        \put(28,55){(a)}
        \put(36,38){(b)}
    \end{overpic}
    \caption{Overview of the experimental setup for (a) wireless and (b) classical data readout. Transmitter and receiver placement are identical for both scenarios.}
    \vspace{-5mm}
    \label{fig:experimental_setup}
\end{figure}

To \revise{quantify} the effect of artificially enhanced capacitive coupling for classical \ac{DAQ} and compare it against wireless \ac{DAQ}, three measurement series for each of the two scenarios have been conducted.
One \textit{BodySense} system has been equipped with an Rx carrier placed on the test subject's upper wrist. 
An adhesive Ag/AgCl wet-gel electrode from \textit{TIGA-MED} with a diameter of \qty{48}{\milli\meter} acts as skin-electrode, ensuring a low resistive contact to the body and mechanically holds the system at its position.
The system ground plane and the battery act together as a floating electrode, closing the return path over the environment and earth-ground.
A second \textit{BodySense} system with a Tx carrier extension has been sequentially placed at distances of \qty{10}{\centi\meter}, \qty{30}{\centi\meter}, and \qty{50}{\centi\meter} to the receiver, utilizing the same Ag/AgCl wet-gel electrode as skin-electrode.
\autoref{fig:experimental_setup} provides an overview of the transmitter and receiver positions during the experiments.
Distances below \qty{10}{\centi\meter} have not been investigated as the inter-device coupling significantly strengthens the return path, making it comparable to the forward path loss \cite{yang_2022}.
For each distance, a frequency sweep from \qty{4}{\mega\hertz} up to \qty{64}{\mega\hertz} has been performed.
All measurements have been conducted in a standing position in a laboratory environment while being at least \qty{1}{\meter} away from walls and other equipment.
The arm was outstretched to the side, forming an angle of \qty{90}{\degree} between the arm and the torso's side. 
During the data collection, the subject stood still and kept the arm in a constant position to minimize movement-induced fluctuations in the environmental-coupled return path.

In the classical scenario, the receiver was directly connected over a USB Type-C cable, forwarding the collected data over USB to the plugged-in computer, see \autoref{fig:experimental_setup} (a).
For the wireless scenario, the USB Type-C cable has been removed, and the computer has been unplugged from the grid. The data is wirelessly forwarded via \ac{BLE} to the computer, as depicted in \autoref{fig:experimental_setup} (b).
Finally, the channel gain is calculated to measure the quality of the capacitive \ac{HBC} communication channel, and the received power has been extracted to set capacitive \ac{HBC} into context to conventional \ac{RF} solutions.

It is essential to carefully consider the effects of signal transmission through the human body and respect its safety limits in \ac{HBC} systems. The \ac{ICNIRP} sets limits for non-ionizing radiation exposure to the human body, defining different dosimetric quantities depending on the frequency. In the frequency range between \qty{1}{\hertz} and \qty{100}{\kilo\hertz} the current density is used as limiting metric \cite{icnirp_low} and above \qty{100}{\kilo\hertz} the \ac{SAR}-value \cite{icnirp_high}.
Further, the IEEE standard C95.1-2005 defines safety levels on human exposure to \ac{RF} electromagnetic fields \cite{c95.1-2005}.
By limiting the transmit power to maximal \qty{5}{\dbm} for our experiments, we ensured to meet the above-mentioned safety standards.