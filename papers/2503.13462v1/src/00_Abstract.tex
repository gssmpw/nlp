\begin{abstract}
%Wearable, wirelessly connected sensors have become a common part of daily life, evolving step by step from their roots in sports and fitness to play a pivotal role in shaping the future of personalized healthcare. 
\revise{Wearable, wirelessly connected sensors have become a common part of daily life and have the potential to play a pivotal role in shaping the future of personalized healthcare. A key challenge in this evolution is designing long-lasting and unobtrusive devices.
%These design requirements inherently demand smaller batteries, which must also support the significant power consumption of wireless communication interfaces.
These design requirements inherently demand smaller batteries, inevitably increasing the need for energy-sensitive wireless communication interfaces.
Capacitive \ac{HBC} is a promising, power-efficient alternative to traditional RF-based communication, enabling point-to-multipoint data and energy exchange.
%However, as this concept relies on the conductive properties of the human body and its surroundings, it is inherently strongly influenced by uncontrollable factors such as environmental conditions and variations in human tissue conductivity, making design and testing particularly challenging.
However, as this concept relies on capacitive coupling to the surrounding area, it is naturally influenced by uncontrollable environmental factors, making testing with classical setups particularly challenging.}

\revise{This work presents a customizable, wearable-sized, wireless evaluation platform for capacitive \ac{HBC}, designed to enable realistic evaluation of wearable-to-wearable applications.
% The experimental setup has been carefully designed to minimize the artificially induced environmental coupling, ensuring precise and \revise{repeatable} measurements.
Comparative measurements of channel gains were conducted using classical grid-connected- and wireless \ac{DAQ} across various transmission distances within the frequency range of $\mathbf{4\mskip3mu}$MHz to $\mathbf{64\mskip3mu}$MHz and
revealed an average overestimation of $\mathbf{18.15\mskip3mu}$dB over all investigated distances in the classical setup.}


%To underscore the importance of realistic evaluation, comparative measurements of channel gain were conducted using classical grid-connected \ac{DAQ} and wireless \ac{DAQ} across various transmission distances within the HBC frequency range of $\mathbf{4\mskip3mu}$MHz to $\mathbf{64\mskip3mu}$MHz. 
%These tests revealed an average overestimation of $\mathbf{18.15\mskip3mu}$dB over all investigated distances in the classical setup, emphasizing the necessity of custom hardware for accurate capacitive \ac{HBC} testing.

%Finally, a comparison of the energy used for communication between capacitive \ac{HBC} and \ac{BLE}, showed comparable results to \ac{SoTA} \ac{BLE} frontends.
%This underscores that capacitive \ac{HBC} is a promising solution for replacing conventional \ac{RF} links, improving the overall energy efficiency of wearables, and making a leap towards battery-free body-worn sensor nodes.


% From previous paper writing
%With new wearable technologies available, many innovative application scenarios emerge in the health, fitness, and consumer branches, inducing high requirements for miniaturization and flexibility. 
%One of the biggest challenges along the way is reducing the size of the battery and consequently increasing the system's overall power efficiency, which is usually bound to wireless communication.
%Further, having the possibility to reliably provide power to multiple on-body wearables allows to reduce or even remove the battery and increases the practicality.
%Capacitive \ac{HBC} is a promising and power-efficient alternative to wirelessly transceive data and power.
%By exploiting the conductive properties of the human body for the forward path and utilizing capacitive coupling to the environment to from the return path.
%However, this concept includes uncontrollable factors, such as the environment, the conductive properties of human tissue, and the measurement setup itself.

%This work presents an expandable and wearable-sized wireless evaluation platform for capacitive \ac{HBC} to evaluate this concept in a realistic wearable-to-wearable application scenario.
%Wireless data transmission minimizes artificially enhanced coupling to the environment, providing pragmatic measurements.
%Comparing the channel gain of classical, grid-connected \ac{DAQ} against wireless \ac{DAQ} for distances of $\mathbf{10\mskip3mu}$cm, $\mathbf{30\mskip3mu}$cm, and $\mathbf{50\mskip3mu}$cm and over a frequency range of $\mathbf{4\mskip3mu}$MHz to $\mathbf{64\mskip3mu}$MHz, yields an average overestimation of $\mathbf{19\mskip3mu}$dB for the classical case, showing the importance of evaluating capacitive \ac{HBC} in a realistic application scenario.

% One option to operate battery-less wearables is capacitive power transfer, which employs the human body as a conductive medium. 

%This work presents clear, quantified system guidelines for capacitive power transfer systems. 

% One option to operate battery-less wearables is capacitive power transfer, which employs the human body as a conductive medium. 
% This work completes the link from theoretical understanding to a guided system implementation via a thorough bottom-up approach: a whole model chain from an intuitive equivalent circuit model to a physical tissue-mimicking phantom is set up and verified. 
% We pinpoint inherent co-dependencies of parameters influencing the path loss and quantify their generalized impact. 
%To validate our findings, we manufactured and evaluated an optimized on-body transmitter, battery-less receiver, and corresponding electrode design. 
% The applicability of the prototype to a real-life scenario as a glucose level tracker is demonstrated. 
% The realized design offers a power transmission of up to 2.5 mW at distances of 15 cm and 5.5 $\mu$W at 125 cm, showing the possibility of battery-less on-body edge computing applications.

\end{abstract}