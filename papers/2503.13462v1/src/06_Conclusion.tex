\section{Conclusion}\label{sec:conclusion}
%This work explores the impact of grid-connected and wireless measurement setups on capacitive human body communication, revealing significant differences in both channel \revise{gain} and frequency behavior. 
While conventional data acquisition setups are effective for quantifying the forward path loss, which depends on the conductive properties of the human body, they substantially alter the return path behavior by artificially modifying the capacitive coupling to earth ground.
Therefore, a wireless, wearable-sized data acquisition system is essential for quantitatively evaluating the full \ac{HBC} communication channel in a realistic environment with minimal measurement interference. 
To address this challenge, this work introduces \textit{BodySense}, an evaluation platform for human body communication that is fully wireless, compact enough for wearable applications, and designed for extendability.
To validate the proposed system, the measured channel gains of a classical, grid-connected setup and a wireless setup have been determined for distances of \qty{10}{\centi\meter}, \qty{30}{\centi\meter}, and \qty{50}{\centi\meter} between transmitter and receiver for a frequency range between \qty{4}{\mega\hertz} and \qty{64}{\mega\hertz}.
A comparison between the two scenarios yields an average overestimation of \qty{18.15}{\db} over all investigated distances for the classical case, highlighting the importance of evaluating capacitive \ac{HBC} in realistic conditions.
When comparing the energy consumption of capacitive \ac{HBC} with \ac{BLE}, we achieved results comparable to state-of-the-art \ac{BLE} frontends. 
This demonstrates its potential as a promising alternative to conventional \ac{RF} links, offering opportunities to further enhance the overall energy efficiency of wearable devices and move closer to the realization of battery-free, body-worn sensor nodes.



%This paper proposes \textit{Bodysense}, a fully wireless, wearable-sized system designed to accurately evaluate capacitive human body communication. Experimental evaluation has revealed significant differences in both channel loss and frequency behavior. This paper demonstrated that while conventional data acquisition setups are effective for quantifying the forward path loss, which depends on the conductive properties of the human body, they substantially alter the return path behavior by artificially modifying the capacitive coupling to earth ground. Thus, the proposed wearable-sized data acquisition system is essential for quantitatively evaluating the full \ac{HBC} communication channel in a realistic environment with minimal measurement interference. 
%To address this issue, this paper presents \textit{Bodysense}, a fully wireless, wearable-sized, and extendable evaluation platform for human body communication.
%To validate the proposed system, the measured channel gains of a classical, grid-connected setup and a wireless setup have been determined for distances of \qty{10}{\centi\meter}, \qty{30}{\centi\meter}, and \qty{50}{\centi\meter} between transmitter and receiver for a frequency range between \qty{4}{\mega\hertz} and \qty{64}{\mega\hertz}.
%A comparison between the two scenarios yields an average overestimation of \qty{18.15}{\db} over all investigated distances for the classical case, highlighting the importance of evaluating capacitive \ac{HBC} with a measurement setup that is similar or ideally identical to the envisaged use case.
