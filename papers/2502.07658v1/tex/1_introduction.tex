\section{Introduction}
% 1 大背景 
% 2 小背景
% 3 小领域研究现状 -> 研究现状不足 -> 本文研究目标
% 4 本课题的重要性 和 独创性
% 5 研究问题 -> 解决办法
% 6 研究贡献
% 7 小结

% (1)Introduction部分要提供作者研究的背景知识、科学问题及该问题的重要性;
% (2)给该问题一个小的总结(综述);
% (3)提出该问题是否存在争议;
% (4)是否完成了或部分解释了提出的科学问题。

% -----------------------------------------
% 大领域背景(RS)+小领域背景(CTR)+研究现状(CTR Prediction)


\begin{figure*}[tbp]
    \includegraphics[width=17cm]{iu4rec_arxiv.png}
    \caption{An overview of the IU4Rec framework, which consists of (a) the construction of interest unit: products are organized into Interest Unit using DSI-based clustering methods on top of the product attributes and textual information~\cite{Tay2022TransformerMA}, and (b) the recommendation of interest unit and the corresponding products. More specifically, the latter component operates in two stages -- stage one recommends products utilizing both item and interest unit attributes, and stage two focuses on identifying products in the given interest unit.}
    \label{fig:iu4rec_overview}
    % \vspace{-0.4cm}
\end{figure*}


The personalized recommendation system plays a vital role in e-commerce platforms ~\cite{cvr1, din,youtubednn,tencent_www}, helping users quickly find the products of their interests, and ultimately improving long-term user engagement~\cite{survey_rec,survey_challenges}. Traditional e-commerce platforms predominantly employ historical behaviors to predict user preferences and, meanwhile, apply diversification strategies to mitigate the risk of filter bubbles, ensuring the trade-off between the exploitation of current interests and the exploration of new interests~\cite{Carbonell1998TheUO}.

% 传统推荐范式的问题
More specifically, industrial recommendation systems generally adopt the Deep Interest Network (DIN) algorithm for content recommendation~\cite{din}, in which historical user-item interactions were employed to locate the most intriguing items that may interest the user (here, an item refers to a selling product). Although such a historical preference-based recommendation mechanism has been widely explored in industrial systems, it still faces several important challenges, especially for C2C (consumer-to-consumer) platforms like Xianyu. In C2C platforms, products of personal sellers often have limited stock availability~\cite{wu2024metasplit}, and once the item is sold, it is no longer available for distribution. This results in most items distributed on Xianyu having relatively few interactions, affecting the effectiveness of DIN-like recommendation algorithms that largely depend on accumulating user-item interactions. Additionally, those items will eventually be materialized as historical user behaviors, and further influence the recommendation effectiveness.

% 常见的方法能
Common solutions to this challenge include (1) offering traffic support through a separate channel~\cite{wu2022adversarial,poso, cold_cross_domain}, (2) leveraging user behaviors from similar items to enhance the representation of the limited-stock product~\cite{wu2024metasplit,cold_graph,meta_emb}, or (3) employing the debias-based methods to ensure that model training is not dominated by popular items~\cite{logq, class_re}, and thereby, allowing less-exposed items to obtain more generalized representations. 
% 能work但不解决本质的问题
While these methods can be effective, they do not address the fundamental issue inherent in the item-based recommendation paradigm. In this paradigm, all user interactions are accumulated on top of unique items. Once the item becomes unavailable, the accumulated interactions are wasted.

% 引出IU-based的推荐
In addition, we often notice that users tend to browse through a set of similar products on our platform, and then examine each item in detail for the final decision. This behavior stems from the nature of C2C platforms, where most listings by individual sellers include both standard product attributes (like brand and class) and second-hand characteristics (such as usage conditions). Therefore, our recommendation algorithm should capture this process, starting from matching the broad-level interest, and further guiding users to find the best option among similar products within the matched interest. To achieve this goal, we redesign our recommendation process into two stages which leads to the development of IU4Rec, an \textbf{I}nterest \textbf{U}nit-based two-stage \textbf{Rec}ommendation system framework consisting of two main components.

% 我们做了啥
% 商品组织,IU构建
As shown in Figure~\ref{fig:iu4rec_overview}, component (a) outlines the construction of the interest unit. We develop, on top of large language models~\cite{Tay2022TransformerMA}, an effective method to identify interest units and the associated queries and products. By reorganizing products around these interest units, we can persist and accumulate user interactions over time, even when the underlying products are sold out. By design, this approach can effectively aggregate user interests and is appropriate for long-term product distribution on C2C platforms.

% 产品界面
Component (b) illustrates the IU-based recommendation process. In addition to upgrading the recommendation algorithm, we extensively redesign the user interface in our system, to make it aligned with the two-stage recommendation process. In stage one, we aim to recommend the interest units rather than individual items. Once the user expresses his/her interest, e.g. by clicking the Interest Unit, the recommendation process turns into the next stage. In stage two, the recommendation system shifts its goal to recommend specific products within the chosen interest unit. 


% 模型升级
The new interest unit based user interaction design makes it possible for the system to aggregate user interactions at the IU level, and across different users. Consequently, we have enhanced our model with the integration of interest unit, and propose an IU-Boosted Network that leverages both item-level and interest-unit-level features, and utilize hierarchical interest unit click sequences to enhance user interests modeling. 

% and changes interface alter user interaction patterns. Users can now engage with multiple items within the same interest unit, allowing for the aggregation of behaviors on interest unit across different users. We proposed IU-Boosted Network to develop interest unit feature to encapsulate the attributes of each interest unit and utilized hierarchical interest unit click sequences to model user interests. These features leverage accumulated user behavior within interest units, enhancing recommendation effectiveness and overall system efficiency.


% 二阶段的优点
Compared to traditional recommendation systems that directly estimate product efficiency, we believe the new two-stage recommendation paradigm based on interest units enhances the precision of user interest modeling and improves decision-making efficiency by presenting similar products at a high density. This approach can simplify purchase decisions for C2C users and offer C2C e-commerce platforms a more effective way to gather user interactions at the  same time.


% \begin{itemize}
%     \item More Comprehensive for Browsing Experience: In stage one, beyond individual products, user interactions with multiple products under the same interest unit can accumulate, allowing for more comprehensive expression of user interests.
%     \item More Efficient for Decision-Making: In stage two, aside from matching efficiency, users on e-commerce platforms often need to browse and compare multiple similar products. The product organization and engagement approach based on interest units  can more conveniently reveal decision factors, enhancing decision efficiency.
%     \item Long-term Adaptability: Organized around collections within interest units, these units can be refined over time, making them suitable for long-term distribution and behavior accumulation on C2C platforms, even as underlying products change. 
%     % 如果想用一段话,而不是 bullet point的话,用下面
%     % \item Additionally, the two-stage engagement method based on interest units allows the platform to present more thematically similar products, thus increasing information density and showcasing more decision-making elements, thereby facilitating purchase decisions for C2C users.
% \end{itemize} 


We evaluated our method by conducting experiments on a production dataset and online A/B testing. The results demonstrate the superiority of our proposed IU4Rec paradigm, which is now fully deployed on Alibaba's Xianyu Platform.

% -----------------------------
% 贡献
To summarize, the main contributions of our work are as follows:
\begin{itemize}
    \item To the best of our knowledge, this is the first solution to optimize recommendation system through item organization, interaction interface design, and recommendation model simultaneously. We propose a novel recommendation paradigm for C2C e-commerce platforms called IU4Rec by grouping similar items into clusters called interest units, redesigning the interaction interface for enhanced platform efficiency and user experience, and propose a novel recommendation model  to enhance CTR prediction ability. We believe this holistic approach opens new possibilities for systematic improvements.
    
    \item The upgraded product format has fundamentally changed user interaction patterns, enabling cross-product behaviors within the same interest unit. We enhance our algorithm through an IU-Boosted Network that aggregates user behaviors under shared interest unit IDs and leverages collective interactions within interest units. This methodology improves recommendation effectiveness within interest domains and normal product domain 
    at the same time.
    
    \item Experiments on industrial datasets and online A/B tests demonstrate the superiority of our proposed IU4Rec paradigm, validating its advantages in both product format design and recommendation modeling. 
\end{itemize}

% The structure of this paper is as follows: Section 2 reviews related work and highlights research gaps. Section 3 covers essential concepts for understanding recommendation systems. Section 4 explains the construction of interest units, the redesigned product interaction interface, and their integration with our CTR prediction model. Section 5 describes the experimental setup, presents results, and discusses their implications. Finally, Section 6 concludes the paper.
