%%%%%%%%%%%%%%%%%%%%

The primary goal of this research is to develop machine learning tools for better processing satellite data for sea ice studies in the polar regions. This inquiry pertains to the profound influence of climate change on GeoScience and society at large, specifically focusing on its impact due to global warming and its consequential effects on ice retreat and melt in the global cryosphere. 


The Ross Sea, situated in Antarctica, is a prominent harbor within the Southern Ocean that is renowned for its exceptional and pristine nature. Although the discussion surrounding the weather's impact on global warming tends to focus on other regions, it is crucial to consider the changes occurring in the Ross Sea. These changes offer valuable insights into the broader implications of climate change and warming trends. The examination of the Ross Sea and its associated meteorological patterns can provide valuable insights into the comprehension of global warming. In general, the documented alterations in the Ross Sea, encompassing its sea ice, temperatures, and ecosystems, offer significant insights into the intricate dynamics between regional and worldwide meteorological patterns and their association with the phenomenon of global warming. The examination of these locations facilitates a better comprehension of the Earth's climate system and its reaction to alterations caused by human activities.


%%%%%%%%%%%%%%%%%55
NASA's ICESat-2 (Ice, Cloud, and Land Elevation Satellite-2) mission is to measure the elevation of Earth's surface, especially its ice sheets, sea ice, and vegetation. One of the key datasets provided by ICESat-2 (IS2) is the ATL03 product that contains precise measurements of the height of Earth's surface, along with geolocation and other information \cite{neumann2019ice}. 
A multitude of Earth science topics, including climate change, polar ice sheet dynamics, and vegetation monitoring, are studied using ATL03 data and other higher-level products. Through NASA's Earthdata Search and other data distribution platforms, the dataset is readily accessible to the public, allowing scientists and researchers worldwide to access and analyze the data for their studies and applications.

%%%%%%%%%%%%%%%%%%%% ATL07 and ATL10 %%%%%%%%%%%%%%%%%%%%%%%
%\textcolor{red}{connection of 03 and 07/10}
The ATL03 data includes the height, latitude, longitude, geolocated photon elevation, and time of individual photons, and it is a large dataset. On the other hand, ATL07 and ATL10 are additional ICESat-2 data products that are derived from ATL03 and measure sea ice height and sea ice freeboard, respectively. 
Freeboard is the thickness of sea ice protruding above the water level. 
These ATL07 and ATL10, level 3 data products are derived from ~150 signal photon aggregation of ATL03, a level 2 data product \cite{kwok2020icesat}.
The ATL07 product comprises along-the-track segments of sea surface and open water leads (at varying length scales) height relative to the WGS84 ellipsoid (ITRF2014 reference frame) after adjustment for geoidal and tidal variations and inverted barometer effects. The along-track length of these segments depends on the distance over which ~150 signal photons (of ATL03) are accumulated; as a result, it can vary depending on the surface type \cite{kwok2020icesat}. The ATL10 product consists of the sea-ice freeboard calculated, each within swath segments that are 10 km (nominally) along the track and 6 km (the distance between the six beams) across the track. For freeboard calculation, the segments of the freeboard swath are utilized to construct a reference sea surface. 
%ATL10 also maintains the ATL07 segment heights used for freeboard calculations for convenience.
%%%%%%%%%%%%%%%%%%%%%%%%%%%%%%%%
The ATL07 and ATL10 products are accumulations of 150 signal photons of ATL03 and have 10m-200m spatial resolution for strong beams and 20m-400m for weak beams \cite{kwok2019surface}. However, the ATL03 data, which has a resolution of 11m footprint with 0.7m spacing, is too big to store and process for domain sea ice scientists due to its huge volume of data. In this study, we adopt a 2m sampling strategy to reprocess the ATL03 data, and we use distributed computing and deep learning technology for processing and classifying the resampled ATL03 data into thick ice, thin ice, and open water. We then derive a higher resolution of local sea level and freeboard products. We aim to get better-resolution products to achieve better scientific sea ice dynamics results from this ATL03 data than the ATL07 and ATL10.

To classify and calculate sea ice surface height and freeboard retrieval, NASA used a decision tree-based approach \cite{kwok2020icesat} on ATL07 data. Nonetheless, this product has the weakness of having a lower resolution than the ATL03.
We propose to use deep learning approaches, namely Multi-layer Perceptron (MLP) and Long Short Term Memory (LSTM), to achieve better sea ice classification accuracy for the ATL03 data. 
Labeled data are required for training to apply deep learning-based approaches for sea ice classification. To label the ATL03 data, we first selected correlated Sentinel-2 (S2) \cite{drusch2012sentinel} images within an 80-minute temporal extent between IS2 and S2. These S2 sea ice images were auto-labeled based on our thin-cloud and shadow-filtered color-based segmentation method\cite{iqrah2023toward}. With the labeled S2 images, we can then map/overlay them to the correlated ALT03 data and automatically transfer the S2 labels to label the thick ice, thin ice and open water in IS2 ATL03 track line data. 
To handle the large amount of ATL03 data labeling, we utilize distributed parallel computing. We use data parallelism and distributed deep learning, utilizing the Horovod framework \cite{sergeev2018horovod} to scale and speedup the deep learning training over multiple GPU machines without sacrificing classification accuracy.

Our parallel workflow includes distributed computing for auto-labeling and freeboard computation, as well as distributed deep-learning training. The distributed scaled auto-labeling of IS2 data achieved around 16.25x speedup, and distributed freeboard computation achieved similar around 15.7x speedup. The distributed LSTM-based sea ice classification model achieved 96.56\% than the MLP model with 91.84\% classification accuracy, achieving a 7.25x speedup on an 8 GPU DGX cluster.

The following are the primary contributions of this paper:
\begin{itemize}
    \item ATL03 sea ice and open water labeled training data using correlated S2 imagery,
    \item Deep learning-based (LSTM and MLP) sea ice classification,
    \item Higher resolution local sea level and freeboard information retrieval using 2m sampling of ATL03 dataset and sea surface estimation based on open water class.
    %at 5km sliding window,
    \item Our distributed computing framework achieved a 16.25x speedup for auto-labeling and 15.70x for freeboard computation, while distributed deep learning training achieved a 7.25x speedup on 8 GPUs.
\end{itemize}


%{To reproduce our experimental results, we made our source code and sample datasets available on GitHub. The link is provided in reference. }%https://github.com/jmiqra/IS2_Classification_Freeboard



The remaining sections of the paper are organized as follows. Section 2 reviews the key related work. Section 3 describes our proposed methodology for sea ice classification and freeboard retrieval. Section 4 contains the evaluation metrics, experimental results, and a discussion of the proposed methodologies. Finally, in Section 5, we provide concluding remarks and suggest future directions for this ongoing work.
%%%%%%

