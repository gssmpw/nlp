%by 12-13th complete this

%\subsection{ICESat-2 Sea Ice Classification - Remote Sensing Point Cloud Data}

NASA's 
%(National Aeronautics and Space Administration) 
ICESat-2 can detect sea ice features due to its high spatial resolution data products. For example, \cite{farrell2020mapping} used ICESat-2 ATL03 geolocated photon data to retrieve six dynamic properties of sea ice, including surface roughness, ridge height, ridge frequency, melt pond depth, floe size distribution, and lead frequency.
Another work on IS2 ATL03 \cite{fredensborg2020estimation} also presented that the degree of ice ridging can be retrieved from this data precisely. 



%(Kwok et al., 2021c) to distinguish thermodynamic and dynamic contributions to sea ice thickness in the central Arctic during the MOSAiC program (Krumpen et al., 2020). %
To examine the surface classification \cite{petty2021assessment}, they utilized ICESat-2 ATL07 and ATL10 sea ice products using near-coincident optical images from S2 over the Western Weddell Sea in March 2019 and the Lincoln Sea in May 2019. However, S2 overlays suggest cloud-induced dark lead misdiagnosis. As a result, they need adjustments to select sea surface reference points \cite{kwok2021refining}. 
Apart from that, in general, this decision-tree-based approach of the ATL07 product shows a good performance on the sea ice and open water classification when compared with other high-resolution satellite images \cite{kwok2019surface}, \cite{kwok2021refining}, \cite{petty2021assessment}.
%
This paper \cite{xu2021deriving}, proposed an improved One-Layer Method (OLMi) for Antarctic sea-ice thickness retrieval with an uncertainty of 0.3 m on ICESat (IS) and IS2. This method examines IS2's monthly sea ice variance and thickness in the Antarctic, demonstrating bi-modal distributions. They also estimate freeboard consistency between IS and IS2.
%
An initial study \cite{kwok2020arctic} to compare satellite lidar (ICESat-2) and radar (CryoSat-2) freeboards to estimate Arctic sea ice snow depth. They determined that the sea ice thickness can be calculated with snow loading from satellite retrievals without resorting to climatology or reconstructions.
%
In \cite{koo2023sea}, for sea ice surface type classification, they utilized coincident S2 to manually label the ATL07 data into different sea ice surface types (thick/snow-covered ice, thin ice, and open water) for building and validating machine learning models. The validated MLP model (99\% accuracy) was used to classify sea ice surface types and then used to derive freeboard. Additionally, in \cite{koo2021weekly}, provided a weekly mapping of freeboard and analysis for the Ross Sea, Antarctic, using the IS2 ATL10 freeboard products.

This \cite{ball2017comprehensive} presented a thorough survey of environmental remote sensing and deep learning research here. They also concentrated on unsolved challenges and opportunities related to (i) inadequate data sets, (ii) human-understandable solutions for modeling physical phenomena, (iii) big data, (iv) nontraditional heterogeneous data sources, (v) DL architectures and learning algorithms for spectral, spatial, and temporal data, (vi) transfer learning, (vii) an improved theoretical understanding of DL systems, (viii) high barriers to entry, and (ix) training and optimizing.
%
This review \cite{yuan2020deep} proceeded into a detailed discussion of the potential for deep learning in the analysis and prediction of environmental remote sensing data. They also assessed deep-learning environmental monitoring for surface temperature, atmosphere, evapotranspiration, hydrology, vegetation, etc.
%This review \cite{yuan2020deep} delved into the discussion of the potential of deep learning in environmental remote sensing, including land cover mapping, environmental parameter retrieval, data fusion and downscaling, information reconstruction and predictions, and typical network structure. They also assessed deep-learning environmental monitoring for atmosphere, vegetation, hydrology, air and land surface temperature, evapotranspiration, solar radiation, and ocean color. 
%%% survey


Therefore, this study represents a novel endeavor to apply machine learning techniques to IS2 ATL03 data to classify sea ice cover types. This research aims to develop innovative machine-learning models for ATL03 surface classification, aiming to achieve better resolution and accuracy in determining sea ice classification and freeboard. Ultimately, these advancements in sea ice classification will contribute to a deeper understanding of sea ice dynamics in polar regions.

%\subsection{topic 2}

%\cite{xie2013summer} in this paper mainly focuses on a visual observation of ice concentration, ice thickness, snow thickness, floe(floating ice) size and melt pond coverage half hourly, higher frequency (1/s) automated EM31 measurements of ice thickness from a ship over the Arctic Pacific region on 21 july to 28 august in 2010. These data results and information are then observed carefully and combined together to get a broad perspective of the sea ice cover with respect to the spatio temporal variations. This dataset can be used for comparison with both previous data and also can be used for future datasets. The both modal thickness of ship based visual observation and EM31 along the cruise track matches well. EM31 also provides systematic datasets that are helpful for quantitative analyses rather than the visual observations as visual observations might have biases because of different cruises and observers. The pack ice zone and the marginal ice zone(MIZ) have differences in ice concentration, ice thickness, ice type, floe size, and pond coverage. The summary of these information in MIZ , 2010 is 
%\begin{enumerate}
%    \item The ice thickness of the dominant ice type was 100 cm in late July (northward leg) and thinned to less than 50 cm in later August (southward leg),
%    \item Ice type was either thick first-year ice or multiyear ice, with a higher fraction of dirty ice for the northward leg, while it seems the same type of ice was observed, but much thinner, in the southward leg,
%    \item Floe size was typically 2–100 m and
%    \item Melt pond coverage was 16\% for the northward leg and 10\% for the southward leg.
%\end{enumerate}
%In the heart of the Central Arctic Ocean, measurements of an 8 profile(4 repeats, 640 points (160x4) ) grid (100m long 10m apart) of ice thickness was conducted and an average 2 cm day-1 melt rate, primarily bottom melt, is found during this 12-day ice station. This grid design enables the sea ice thickness study by making a ice thickness map of the grid cell size area and also studying the changes of these areas over time using repeated measurements. This kind of dataset would be more valuable for calibration and validation purposes when there will be more satellites or other measurements of ice thickness.

%\textcolor{red}{add ref Young's)}