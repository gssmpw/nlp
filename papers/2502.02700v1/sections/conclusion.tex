This research explores the ICESat-2 ATL03 2m sampled data for sea ice classification and freeboard retrieval. We automatically labeled sea ice on ATL03 data using correlated color-based-thin-cloud-shadow-filtered labeled S2 imagery for training and scaled the process.
Our sea ice classification results on ATL03 data indicate that the LSTM model provides more accurate results in classifying thick ice, thin ice, and open water in the polar regions than the MLP model. We successfully scaled and distributed the deep learning training over multiple GPUs using the Horovod framework and achieved a better speedup. 
We also achieved a better resolution of the local sea surface height. We calculated a better resolution of freeboard information along the 2m sampled ALT03 track than those based on the ATL07 data.
%Therefore, it could be used to 
\paragraph{Future Work}
%In this project, we only focused on the Ross Sea region for November 2019. 
Still, downloading the massive ATL03 data to local computers for processing is a big challenge. The future of this work is to directly access the data from the Cloud, while combined with scaled and distributed deep learning to speedup the processing and generate polar-wide scale freeboard and even thickness products. 
%Based on the new freeboard directly from ATL03, similar scaled and distributed spatial and temporal interpolation techniques should be developed to generate polar-wide scale freeboard and even thickness products. 
In the end, better sea ice products will help domain scientists better understand sea ice dynamics and changes in a warming climate.