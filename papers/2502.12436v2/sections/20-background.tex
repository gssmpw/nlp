\section{Deception in the Wild}
% \fgcomment{should we talk about deception in game generally here and then diplomacy specifically in 2.1?}
% \wwcomment{that's true, I rearranged and add real-world deception to connect to ours}


% Detecting \textbf{deceptive intent} in natural language is inherently difficult where true intents are hidden~\citep{granhag2015detecting}.
% % \sscomment{without access to true intent -> where true intents are hidden}.
% Some deceptive statements are easy to identify when they contradict common knowledge, for instance, if there is a claim, ``\textit{Santa Claus bought you that present}''.
% % \sscomment{this isn't a deceptive statement, it's just wrong. You can use `\textit{Urgent: Click here to secure your bank account now!} where we know from how the banking notification system works that it's deceptive. Also, we saw lots of instances that make it easy to detect,' Or you can just say that falsifying some statements are easier, but some are harder, defective ones are of those}. 
% \wwcomment{this is not deceptive as Sadra suggests but Sadra's example is not so common sense for everybody (I suppose) will find better example} \jmcomment{better?} \wwcomment{yes! thank you}
% Given our understanding of religion-based consumerism and culture's common desire to manipulate the world view of its youth~\cite{anderson_encounter_1994}, we (as adults) can confidently reject this statement as false. However, more subtle deception involves manipulating beliefs to steer decisions away from optimal choices, making detection significantly harder~\citep{greenberg1982effect,ettinger2010theory,zhang2019optimal}. 

Real-world deception manifests in various forms, such as \textit{scams} and \textit{phishing} attacks, where perpetrators exploit \textbf{trust} to manipulate victims into believing in the possibility of good fortune, even if it is unlikely~\citep{button2014online,muscanell2014weapons,hanoch2021scams}. These deceptive tactics often rely on persuasive language. If victims fall for these \textbf{too good to be true} claims, they become targets and may comply with the perpetrators' requests---for example, disclosing sensitive information or making financial investments under false pretenses---ultimately resulting in monetary \textit{loss} or data breaches~\citep{burnes2017prevalence,coluccia2020online}. Those scammers would \textit{gain value} through data breaches or simply by acquiring cash.
% If deceptive claims were trivially easy to detect, they would lose their strategic effectiveness, as players would simply counteract them by taking the opposite action \citep{levine2022truth}. 

Detecting deception remains a persistent challenge, especially when it is needed for real-world problems.
% as it pervades human interactions and extends into \abr{ai} systems designed to generate and interpret natural language\sscomment{I feel I am reading the same concepts from section 1}. 
% Users interacting with \abr{ai}-generated content must constantly assess its credibility~\citep{huschens2023trustchatgptperceived}. 
% This challenge is exacerbated by the fact that many natural language tasks involve ambiguous objectives (cite) and rely on subjective human judgments for evaluation (cite)\sscomment{can be presented better-> `this challenge is even more threatening in situations where human judgment is directly required (human is deciding solely)'}. 
We can use \abr{ai}, but it is hard to evaluate and requires quality feedback from humans to train models in detecting real deception. 
%
Deception in a limited space like a strategic game, e.g., Diplomacy, where nuanced persuasion and deception is required for winning, is more tractable to evaluate.
%
A bounded example would allow us to measure the ability of an AI to improve in deception detection.

%
%and is possible to improve \abr{ai} at a current state to be aware of deception. 
% \sscomment{-> For example in strategic games like dip where nuanced persuasion is required for winning}
\jmcomment{Why `Thus'? It doesn't follow from what comes before. I think it would be fine to just describe how it's tricky to deceive and tricky to detect deception. However if we're dealing with detection then we should focus on that difficulty, not the difficulty in creation.} 
% \sscomment{what ingredients? big jump here} \jmcomment{`Cooperating' not correct. Combining? same q as sadra tho}
\wwcomment{I rewrote this section to deliver better}

\subsection{One Gains, One Loses}
% \sscomment{if you're not talking about the game from zero-sum lit. pov I suggest `Deception in Diplomacy' which is consistent with the section title as well}
% \wwcomment{deception in general is already one to lose, one to gain, Deception in Diplomacy will be too bland in my opinion}

%JKK: Cut the Paquette cite since it doesn't really deal with deception in any intentional way
Deception has been studied in games that rely on trust, negotiation, and strategic misrepresentation, such as Werewolf~\citep{chittaranjan2010you,hancock2017towards,girlea2017deception}, Poker~\citep{lee2013deception, palomaki2016machiavelli}, and Diplomacy~\citep{niculae-etal-2015-linguistic,kramar2022negotiation}. Diplomacy is a complex interplay of strategy, high-level cooperation, and subtle betrayal. The game is set on an European map, highlighting key territorial cities known as supply centers. Each of the seven players controls a country and moves units on the map, with the objective of capturing more than half of the supply centers (18 out of 34) to achieve victory. For each turn, players communicate one-to-one and then simultaneously reveal their orders for each units.

Deception plays a crucial role in gaining supply centers and, ultimately, securing a win. Cliques of players agreeing to coordinate to gain advantages over others must operate in secrecy. \jmcomment{I thought more explanation was needed but I am not yet satisfied with my amendment \wwcomment{I will come back to this!}} % However, while minor lies and misdirections occur frequently, strategic deception is relatively rare, as observed in a dataset by \citet{peskov2020takes}. 
Deception must be undetected to be successful. If the player fails to recognize deception, they risk losing supply centers and may lose the game (Figure \ref{fig:intro_example}). If a player's deception succeeds, they may gain supply centers. The challenge lies in quantifying the benefits of deception and the losses of those who are deceived. Given the significance of supply centers as a sparse scoring mechanism, we see an opportunity to integrate reinforcement learning (\abr{rl})~\citep{zinkevich2007regret,brown2019deep} into the analysis.

\abr{rl} has been extensively used to train \abr{ai} agents in optimizing decision-making that maximize a reward. \abr{rl}-based \abr{ai} has been applied to Diplomacy \cite{NEURIPS2019_84b20b1f,anthony2020learning,gray2020human,bakhtin2021nopress}, with a recent model, 
\cicero~\citep{meta2022human}, achieving competitive human-level play. This paper uses a reward model from \cicero to detect action proposals where deception is likely (Figure \ref{fig:overview}).

