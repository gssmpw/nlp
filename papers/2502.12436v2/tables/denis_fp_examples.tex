\begin{table}[t]
    \begin{tabular}{p{1.2cm}p{5.5cm}}
    \hline
    \textbf{Sender} & \textbf{Message}   \\ 
    \hline
     \rowcolor{grayish} Germany (Truth) & \small{Well the feeling is mutual. I wanted to let you know that Austria asked for my help putting pressure on Warsaw. I don't intend to do that, but \textbf{I recommend you use Silesia to support your Rumanian unit into Galicia}. I promise not to interfere with this maneuver if you promise to keep your baltic fleet focused on defending Sweden from England?} \\ 
  \hline
    \end{tabular}
    \caption{An example of False Positive that  CTRL-D detects. Germany has a possible short-term gain had they betrayed Russia, but they nonetheless followed through on their proposal.
    % Germany can receive a better reward if they betray Russia using this proposal, which directly derived from \(U_2\), which is \(U_2 = u_j(\hat{a}_i, a_j) - u_j(a_i, a_j). = 0.0579\), where \(\hat{a}_i\) is ``\textit{Silesia support Romania unit into Galicia}'', \(a_i \in \pi_i\) and \(a_j \in \pi_j\). This makes our classifier misinterpret as a lie.
    }
    \label{tab:fp_cfrl}
\end{table}

\begin{table}[t]
    \begin{tabular}{p{1.2cm}p{5.5cm}}
    \hline
    \textbf{Sender} & \textbf{Message}   \\ 
    \hline
     \rowcolor{grayish} France (Truth) & \small{I am supporting Tys to Wes. Can you use Mar to support spain hold?} \\ 
  \hline
    \end{tabular}
    \caption{An example of False Positive that the LLM baseline detects as deception. It interprets France's message as a promise to support Austria's order. \cicero{}'s predict orders for France with \amr{A MUN H} where LLM baseline misinterprets as France is attacking Germany. It claims that France contradicts.}
    \label{tab:fp_llama3}
\end{table}