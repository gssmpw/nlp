\section{Proofs of the statements in Section~\ref{sec:poly-evaluation}}
\label{appendix:sec-poly-evaluation}

\subparagraph*{Representation for the algebraic numbers involved in the definition of $\cn$.}
Before moving to the proofs of the statements in
Section~\ref{sec:poly-evaluation}, let us come back to the representation of
algebraic numbers. As written in the body of the paper, an algebraic number
$\alpha$ can be represented as a triple $(q,\ell,u)$ where $q$ is a non-zero
integer polynomial and $\ell,u$ are rational numbers such that $\alg$ is the
only root of $q$ that belongs to the interval $[\ell,u]$. Since, in our case, we
are fixing the base~$\cn$ of $\exists\R(\ipow{\cn})$, it is convenient to
improve this representation for \emph{fixed} algebraic numbers (i.e.~those that
do not depend from the input of our procedures, as for instance numbers that may
be involved in the definition of $\cn$). In these cases, we impose the following
restriction on $\ell$ and~$u$: either $\ell = u$, or $\alpha \in
(\ell,u)$ and $(\ell,u) \cap \Z = \emptyset$ (note: this is in addition to the
property that $\alg$ is the only root of $q$ in $[\ell,u]$). This restriction is
without loss of generality. Indeed, given a triple $(q,\ell,u)$ not satisfying
it, we can apply dichotomy search to refine the interval $[\ell,u]$ to an
interval $[\ell',u']$ such that $u'-\ell' < 1$. This refinement is done by tests
of the form $\exists x : q(x) = 0 \land \ell < x \leq \frac{u-\ell}{2}$, which
can be performed (in fact, in polynomial time) by, e.g.,~\Cref{theorem:basu}.
After computing $[\ell',u']$, we reason as follows: 
\begin{itemize}
  \item if $[\ell',u']$ does not contain an integer, $(q,\ell',u')$ is the
  required representation of $\alpha$.
  \item if $[\ell',u']$ contains $k \in \Z$ and $q(k) = 0$, then $(q,k,k)$ the
  required representation of~$\alpha$.
  \item if $[\ell',u']$ contains $k \in \Z$ and $q(k) \neq 0$, then one among
  $(q,\ell,k)$ and $(q,k,u)$ is the required representation of~$\alpha$. It then
  suffices to check where $\alpha$ lies, which can be done by testing $\exists x
  : q(x) = 0 \land \ell < x \leq k$, again with, e.g., the algorithm
  in~\Cref{theorem:basu}.
\end{itemize}
Once more, we stress the fact that this representation is only used for algebraic numbers that are \emph{fixed}, and so the above refinement of $\ell$ and $u$ takes constant time.

\subparagraph*{Proofs of the statements in the paragraph ``The case of~$\cn$
algebraic''.} We need to establish \Cref{lemma:approx-algebraic-body}, which
follows as a simple corollary of the following lemma.

\begin{restatable}{lemma}{LemmaApproxAlgebraic}
  \label{lemma:approx-algebraic}
  Let $\cn$ be a (fixed) algebraic number represented by $(q,\ell,u)$. 
  There is an algorithm that given as input $L \in \N$ written in unary 
  computes in time polynomial in $L$ two rational numbers $\ell'$ and $u'$ 
  such that $(q,\ell',u')$ is a representation of $\alg$ 
  and $0 \leq u'-\ell' \leq 2^{-L}$.  
\end{restatable}
\begin{proof}
  Since $\cn$ is fixed, following the text above, we can assume without loss of generality that either $\ell = u$ or $\cn \in (\ell,u)$ and $(\ell,u) \cap \Z = \emptyset$, 
  which implies $u-\ell < 1$. 
  Once more, we remark that without this assumption, one such interval containing $\cn$ 
  can be computed in constant time. 
  The algorithm refines further refine the interval $[\ell,u]$ to an accuracy that depends on $L$ by performing a dichotomy search: 
  \begin{algorithmic}[1]
    \While{$u-\ell > 2^{-L}$}\label{alg:ref-alg:line1}
      \State $m \gets \frac{u-\ell}{2}$
      \label{alg:ref-alg:line2}
      \If{$q(m) = 0$} 
        \myreturn $(q,m,m)$
        \label{alg:ref-alg:line3} 
      \EndIf
      \If{$\exists x : \ell < x < m \land q(x) = 0$}
        $u \gets m$
        \label{alg:ref-alg:line4}
      \Else \ $\ell \gets m$
        \label{alg:ref-alg:line5}
      \EndIf
    \EndWhile
    \State \myreturn $(q,\ell,u)$
  \end{algorithmic}
  The correctness of the algorithm is immediate: at each iteration of the
  \textbf{while} loop of line~\ref{alg:ref-alg:line1}, after defining $m$ as
  $\frac{u-\ell}{2}$, one of the following three cases holds: $\cn = m$, $\ell <
  \cn < m$, or $m < \cn < u$. Lines~\ref{alg:ref-alg:line3}
  and~\ref{alg:ref-alg:line4} check which of the three cases holds, by relying
  on the fact that $\cn$ is the only root of $q(x)$ in the interval $[\ell,u]$.
  We can implement the test in line~\ref{alg:ref-alg:line4} by relying, e.g., on
  the procedure form~\Cref{theorem:basu}, which runs in polynomial time when 
  the input formula has a fixed number of variables.

  Observe that at each iteration of the \textbf{while} loop the distance between $\ell$ and $u$ is halved.
  Since initially $u-\ell < 1$, this means that the \textbf{while} loop of
  line~\ref{alg:ref-alg:line1} iterates at most $L$ times.
  Therefore, in order to argue that 
  the procedure runs in polynomial time 
  it suffices to track the growth of the numbers $\ell$ and $u$ across $L$ iterations of the while loop.

  Below, we see $\ell$, $u$ and $m$ as standard programming variables, all
  storing pairs of integers representing rational numbers.  
  We also let $(a,b)$ and $(c,d)$ be the content of the variables $\ell$ and
  $u$, respectively, at the beginning of the algorithm. These two pairs encode
  the rationals $\frac{a}{b}$ and $\frac{c}{d}$. We assume $a,c \in \Z$ and $b,d
  \in \N_{\geq 1}$.

  To analyse the growth of the numbers stored in $\ell$ and $u$ throughout the
  execution of the algorithm, let us make a simplifying assumption. Whereas
  throughout the rest of the paper we have encoded rational numbers as pairs of
  \emph{coprime} integers, throughout the run of this algorithm we do not force
  coprimality. In particular, if at the beginning of some iteration of the
  \textbf{while} loop the variables $\ell$ and $u$ store the pairs
  $(\ell_1,\ell_2)$ and $(u_1,u_2)$, respectively, with $\ell_1,u_1
  \in \Z$ and $\ell_2,u_2 \in \N_{\geq 1}$, then in line~\ref{alg:ref-alg:line2}
  the algorithm assigns to the variable $m$ the pair of numbers $(m_1,m_2)$,
  encoding $\frac{m_1}{m_2}$, such that 
  \[ 
    m_1 \coloneqq \frac{\lcm(\ell_2,u_2)}{u_2}u_1 - \frac{\lcm(\ell_2,u_2)}{\ell_2}\ell_1 
    \qquad\text{ and }\qquad
    m_2 \coloneqq 2 \cdot \lcm(\ell_2,u_2),
  \]
  where we remark that $\frac{\lcm(\ell_2,u_2)}{u_2}$ and
  $\frac{\lcm(\ell_2,u_2)}{\ell_2}$ are integers (hence $m_1 \in \Z$), and $m_2
  \in \N_{\geq 1}$.
  Note that this correctly capture the assignment done in line~\ref{alg:ref-alg:line2}:
  \[ 
    m = \frac{m_1}{m_2} = \frac{\frac{\lcm(\ell_2,u_2)}{u_2}u_1 - \frac{\lcm(\ell_2,u_2)}{\ell_2}\ell_1}{2 \cdot \lcm(\ell_2,u_2)} 
    = \frac{\frac{u_1}{u_2}-\frac{\ell_1}{\ell_2}}{2} 
    = \frac{u-\ell}{2}.
  \]
  Coprimality can be restored when the algorithm returns. 
  
  
  We show the following loop invariant: 
  \begin{center}
    \begin{minipage}{0.85\linewidth}
      After the $M$th iteration of the \textbf{while} loop, the program variables $\ell$ and $u$
      store pairs $(\ell_1,\ell_2)$ and $(u_1,u_2)$, respectively, such that $(\ell_1,\ell_2) \in S_M \cup \{(a,b)\}$ 
      and $(u_1,u_2) \in S_M \cup \{(c,d)\}$ 
      where 
      \[ 
        S_M \coloneqq \left\{ (v_1,v_2) \in \Z \times \N_{\geq 1} :\ 
        \begin{aligned} 
          &\abs{v_1} \leq 2^{j}(\abs{a \cdot d} + \abs{c \cdot b}) \text{ and } 
          v_2 = 2^{j} \lcm(b,d)\\[-3pt]
          &\text{ for some } j \in [0..M]
        \end{aligned}
        \right\}.
      \]
    \end{minipage}
  \end{center}
  Observe that the invariant trivially holds after the $0$th iteration of the
  \textbf{while} loop, since at that point $\ell$ stores $(a,b)$ and $u$ stores
  $(c,d)$. Consider now the $(M+1)$th iteration of the \textbf{while} loop, with
  $M \geq 0$. Let $(\ell_1,\ell_2)$ and $(u_1,u_2)$ be the pairs assigned to
  $\ell$ and $u$, respectively, at the beginning of this iteration. If the test
  performed in line~\ref{alg:ref-alg:line3} is successful, then the algorithm
  returns and we do not have anything to prove. Below, assume the test in
  line~\ref{alg:ref-alg:line3} to be unsuccessful, so that the algorithm
  completes the $(M+1)$th iteration of the loop. Let $(m_1,m_2)$ be the pair of
  integers defined as
  \[ 
    m_1 = \frac{\lcm(\ell_2,u_2)}{u_2}u_1 - \frac{\lcm(\ell_2,u_2)}{\ell_2}\ell_1 
    \qquad\text{ and }\qquad
    m_2  = 2 \cdot \lcm(\ell_2,u_2).
  \]
  At the end of the iteration of the loop, one of the following two possibility occur: 
  \begin{itemize}
    \item $\ell$ stores $(\ell_1,\ell_2)$ and $u$ stores $(m_1,m_2)$ (this occurs if the assignment in line~\ref{alg:ref-alg:line4} is executed), 
    \item $\ell$ stores $(m_1,m_2)$ and $u$ stores $(u_1,u_2)$ (this occur if the assignment in line~\ref{alg:ref-alg:line5} is executed). 
  \end{itemize}
  By induction hypothesis $(\ell_1,\ell_2) \in S_M \cup \{(a,b)\}$ 
  and $(u_1,u_2) \in S_M \cup \{(c,d)\}$, 
  and to conclude the proof it suffices to show that $(m_1,m_2) \in S_{M+1}$.
  We split the proof into four cases:
  \begin{description}
    \item[case: $(\ell_1,\ell_2) = (a,b)$ and $(u_1,u_2) = (c,d)$.]
      We have $m_1 = \frac{\lcm(b,d)}{d}c - \frac{\lcm(b,d)}{b}a$ 
      and $m_2 = 2 \cdot \lcm(b,d)$. 
      The first equation yields ${\abs{m_1} 
      \leq \abs{\frac{\lcm(b,d)}{d}c}+\abs{\frac{\lcm(b,d)}{b}a} \leq \abs{c \cdot b}-\abs{a \cdot d}}$. 
      We conclude that $(m_1,m_2) \in S_{M+1}$.
    \item[case: $(\ell_1,\ell_2) \in S_M$ and $(u_1,u_2) = (c,d)$.]
      By definition of $S_M$, there is $j \in [0..M]$ 
      such that $\abs{\ell_1} \leq 2^{j}(\abs{a \cdot d} + \abs{c \cdot b})$ 
      and $\ell_2 = 2^{j} \lcm(b,d)$. By definition of $(m_1,m_2)$, 
      \begin{align*}
        m_2 
          &= 2 \cdot \lcm(2^{j} \lcm(b,d),d)
          = 2^{j+1}\lcm(b,d),
      \end{align*}
      and 
      \begin{align*}
        \abs{m_1} 
           &= \abs{\frac{2^{j}\lcm(b,d)}{d}c - \frac{2^{j}\lcm(b,d)}{2^{j}\lcm(b,d)}\ell_1}
           = \abs{\frac{2^{j}\lcm(b,d)}{d}c - \ell_1}\\
           &\leq \abs{2^{j}c \cdot b} + \abs{\ell_1}\leq \abs{2^{j}c \cdot b} + 2^{j}(\abs{a \cdot d} + \abs{c \cdot b}) 
           \leq 2^{j+1}(\abs{a \cdot d} + \abs{c \cdot b}).
       \end{align*}
       Since $j+1 \in [0..M+1]$, we conclude $(m_1,m_2) \in S_{M+1}$.
    \item[case: $(\ell_1,\ell_2) = (a,b)$ and $(u_1,u_2) \in S_M$.] 
      This case is analogous to the previous one.
    \item[case: $(\ell_1,\ell_2) \in S_M$ and $(u_1,u_2) \in S_M$.]
      There are $j_1,j_2 \in [0..M]$ such that 
      \begin{itemize}
        \item 
          $\abs{\ell_1} \leq 2^{j_1}(\abs{a \cdot d} + \abs{c \cdot b})$ 
          and $\ell_2 = 2^{j_1} \lcm(b,d)$
        \item 
          $\abs{u_1} \leq 2^{j_2}(\abs{a \cdot d} + \abs{c \cdot b})$ 
          and $u_2 = 2^{j_2} \lcm(b,d)$.
      \end{itemize}
      By definition of $(m_1,m_2)$, 
      \begin{align*}
        m_2 
          &= 2 \cdot \lcm(2^{j_1} \lcm(b,d),2^{j_2} \lcm(b,d)) = 2^{\max(j_1,j_2)+1}\lcm(b,d),
      \end{align*}
      and 
      \begin{align*}
       \abs{m_1} 
          &= \abs{\frac{2^{\max(j_1,j_2)}\lcm(b,d)}{2^{j_2} \lcm(b,d)}u_1 - \frac{2^{\max(j_1,j_2)}\lcm(b,d)}{2^{j_1} \lcm(b,d)}\ell_1}\\
          &= \abs{2^{\max(j_1,j_2)-j_2}u_1 - 2^{\max(j_1,j_2)-j_1}\ell_1}\\
          &\leq \abs{2^{\max(j_1,j_2)-j_2}u_1} + \abs{2^{\max(j_1,j_2)-j_1}\ell_1}\\
          &\leq \abs{2^{\max(j_1,j_2)-j_2}2^{j_2}(\abs{a \cdot d} + \abs{c \cdot b})} + \abs{2^{\max(j_1,j_2)-j_1}2^{j_1}(\abs{a \cdot d} + \abs{c \cdot b})}\\
          &\leq 2^{\max(j_1,j_2)+1}(\abs{a \cdot d} + \abs{c \cdot b}).
      \end{align*}
      Since $\max(j_1,j_2)+1 \in [0..M+1]$, we conclude $(m_1,m_2) \in S_{M+1}$.
  \end{description}
  Having established the above loop invariant, 
  it is now clear that, after $L$ executions 
  of the \textbf{while} loop, 
  to the variables $\ell$ and $u$ 
  are assigned pairs of numbers of bit size linear in $L$. Therefore, the algorithm runs in polynomial time.
\end{proof}

\subparagraph*{{Proofs of the lemmas in ``The case of $\cn$ among some classical transcendental numbers''.}}

We work towards a proof of~\Cref{lemma:poly-time-exp-log}. First of all, we
establish two lemmas on approximations of $e^r$ and $\ln(r)$ by truncation of
standard power series.

\begin{restatable}{lemma}{LemmaApproxExp}
  \label{lemma:approx-exp}
  Let $r \in \R$ and $k \geq 1$ with $\abs{r} \leq k$, 
  and let~$t_n(x) \coloneqq \sum_{j=0}^n \frac{x^j}{j!}$. 
  For every $L,M \in \N$ satisfying $M \geq L + 8 k^2$, we have $\abs{e^r - t_M(r)} \leq 2^{-L}$.
\end{restatable}

\begin{proof}
  Following the identity $e^ x = \sum_{j=0}^\infty \frac{x^j}{j!}$, 
  whose right hand side is the Maclaurin series for $e^x$ (see, e.g.,~\cite[Equation~4.2.19]{Olver10}), we have
  \begin{equation*}\label{inequality-exp-taylor}
    \begin{aligned}
      \abs{e^r-t_M(r)}&=\abs{\sum_{j=M+1}^\infty \frac{r^j}{j!}} = 
      \abs{\sum_{j=0}^\infty \frac{r^{M+1+j}}{(M+1+j)!} } = 
      \abs{r^{M+1}\sum^\infty_{j=0}\frac{r^j}{(M+1+j)!}} = \\
      &=\abs{\frac{r^{M+1}}{(M+1)!}\sum_{j=0}^\infty
      \frac{r^j}{(M+1+j)\cdot{\dots}\cdot(M+2)}}\leq
      \frac{k^{M+1}}{(M+1)!}e^k\,,
    \end{aligned}
  \end{equation*}
  where in the last inequalities we used $\abs{r} \leq k$. 
  Let us show that the hypothesis $M \geq L + 8k^2$ in the statement of the lemma implies $\frac{k^{M+1}}{(M+1)!}e^k\leq \frac{1}{2^L}$, 
  concluding the proof. 
  Below, note that $M \geq 2^3k^2$ implies $\log(\frac{M}{2}) \geq 2 + 2 \log(k)$, and so $\frac{\log(\frac{M}{2})}{2} - \log(k) \geq 1$.
  We have the following chain of implications:
  \begin{align*}
    &M\geq L+2^3k^2\\
    \implies{}&M\geq L+\log(k)+k\log(e) \quad\text{and}\quad M\geq 2^3k^2
    & \text{since $2^3k^2 \geq \log(k) + k\log(e)$}\\
    \implies{}&M\Big( \frac{\log(\frac{M}{2})}{2}-\log(k) \Big) \geq
    L+\log(k)+k\log(e)\\
    \implies{}&\frac{M}{2}\log\left( \frac{M}{2} \right)\geq 
    L+M\log(k) + \log(k) + k\log(e)\\
    \implies{}&\log((M+1)!)\geq L+(M+1)\log(k) + k\log(e)
    &\text{since }(M+1)!\geq \frac{M}{2}^{\frac{M}{2}}\\
    \implies{}&(M+1)!\geq 2^Lk^{M+1}e^k\\
    \implies{}& \frac{k^{M+1}}{(M+1)!}e^k\leq \frac{1}{2^L}.
    &&\qedhere
  \end{align*}
\end{proof}

\begin{restatable}{lemma}{LemmaApproxLog}
  \label{lemma:approx-log}
  Let $r > 0$, 
  and let~$t_n(x) \coloneqq 2 \cdot \sum_{j=0}^n \big(\frac{1}{2j+1} \big(\frac{x-1}{x+1}\big)^{2j+1}\big)$. Consider $L,M \in \N$. 
  If~$r = 1$ or $M \geq (L+\log\abs{\ln(r)}) \big({-}2\log\abs{\frac{r-1}{r+1}}\big)^{-1}$, then $\abs{\ln(r) - t_M(r)} \leq 2^{-L}$.
\end{restatable}

\begin{proof}
  If $r = 1$, observe that $\ln(r) = 0$ 
  and $t_n(r) = 0$ for every $n \in \N$, 
  so $\abs{\ln(r)-t_M(r)} = 0$ and the statement trivially follows. 

  Below, assume $r \neq 1$.
  We follow the identity $\ln(x) = 2 \sum_{j=0}^\infty \big(\frac{1}{2j+1} \big(\frac{x-1}{x+1}\big)^{2j+1}\big)$, which holds for every $x > 0$, see~\cite[Equation~4.6.4]{Olver10}.
  We have:
  \begin{equation*}\label{inequality-log-expansion}
    \begin{aligned}
          &\abs{\ln(r)-t_M(r)}\\
      ={} &\abs{\sum_{j=M+1}^\infty\frac{1}{2j+1}
      \left( \frac{r-1}{r+1} \right)^{2j+1}}\\
      ={} &\abs{\sum_{j=0}^\infty\frac{1}{2j+2M+3}
      \left( \frac{r-1}{r+1} \right)^{2j+2M+3}}\\ 
      ={}&\abs{\left( \frac{r-1}{r+1} \right)^{2M+2}
      \sum_{j=0}^\infty\frac{1}{2j+2M+3}\left( \frac{r-1}{r+1} \right)^{2j+1}}\\
      \leq{}&\abs{\left( \frac{r-1}{r+1} \right)^{2M+2}
      \sum_{j=0}^\infty\frac{1}{2j+1}\left( \frac{r-1}{r+1} \right)^{2j+1}}
      &\text{note: $\frac{r-1}{r+1}, \Big( \frac{r-1}{r+1} \Big)^{3},\Big( \frac{r-1}{r+1} \Big)^{5},\dots$}\\[-8pt]
      &&\text{all have the same sign}\\ 
      \leq{}&
      \frac{1}{2}\abs{\frac{r-1}{r+1}}^{2M+2}\abs{\ln(r)}.
    \end{aligned}
  \end{equation*}
  Let us now show that the hypothesis 
  $M \geq (L+\log\abs{\ln(r)}) \big({-}2\log\abs{\frac{r-1}{r+1}}\big)^{-1}$ in the statement of the lemma implies $\frac{1}{2}\abs{\frac{r-1}{r+1}}^{2M+2}\abs{\ln(r)} \leq \frac{1}{2^L}$, concluding the proof. Below, note that $r>0$ and $r \neq 1$ imply 
  that $\abs{\frac{r-1}{r+1}} \in (0,1)$, so $\log\abs{\frac{r-1}{r+1}} < 0$.
  \begin{align*}
    &M\geq(L+\log(\abs{\ln(r)}))\left( -2\log\abs{\frac{r-1}{r+1}} \right)^{-1}\\
    \implies{}&
    2M+2\geq \frac{L+\log\abs{\ln(r)}}{-\log\abs{\frac{r-1}{r+1}}}\\
    \implies{}&(2M+2)\log\abs{\frac{r-1}{r+1}}+\log\abs{\ln(r)}\leq -L
    &\text{since $\textstyle\log\abs{\frac{r-1}{r+1}} < 0$}\\
    \implies{}&\abs{\frac{r-1}{r+1}}^{2M+2}\abs{\ln(r)}\leq 2^{-L}\\
    \implies{}&\frac{1}{2}\abs{\frac{r-1}{r+1}}^{2M+2}\abs{\ln(r)}\leq 2^{-L}.
    &&\qedhere
  \end{align*}
\end{proof}

To prove~\Cref{lemma:poly-time-exp-log} we will also use the following 
technical lemma. 

\begin{lemma}
  \label{lemma:auxiliary-for-approximations}
  Let $\delta(x)$ be an integer polynomial. Consider a function $p \colon \R \to
  \R$ such that $\delta(x) \cdot p(x)$ equals an integer polynomial $q(x)$. Let
  $d,h \in \N$ such that $\max(\deg(\delta),\deg(q)) \leq d$ and
  $\max(\height(\delta),\height(q)) \leq h$. Let $r \in \R$ such that
  $\max(\abs{r},\abs{p(r)}) \leq K$ for some $K \geq 1$. Consider $L,M \in \N$
  satisfying 
  \[
    M \geq L + \log(h+1) + (2 d +1)(\log(K+1)).
  \] 
  For every $r^* \in \R$ satisfying $\delta(r^*) \geq 1$, if $\abs{r-r^*} \leq
  2^{-M}$ then $\abs{p(r)-p(r^*)} \leq 2^{-L}$. 
\end{lemma}

\begin{proof}
  By applying~\Cref{lemma:approx-univ-polynomial} to both $\delta$ and $q$, 
  from $\abs{r-r^*} \leq 2^{-M}$ 
  we derive 
  \begin{align*}
    \abs{\delta(r)-\delta(r^*)} \leq 2^{-L'}
    \text{ \ and \ }
    \abs{q(r) - q(r^*)} \leq 2^{-L'},
  \end{align*}
  where $L' = L + \log(K+1)$. 
  We show that these two inequalities imply $\abs{p(r)-p(r^*)} \leq 2^{-L}$.
  Define $\epsilon \coloneqq \delta(r)-\delta(r^*)$. The following chain of implications holds 
  \begin{align*}
    &\abs{q(r) - q(r^*)} \leq 2^{-L'}\\
    \implies{}&\abs{\delta(r) \cdot p(r) - \delta(r^*) \cdot p(r^*)} \leq 2^{-L'}
    &\text{by hypotheses}\\
    \implies{}&\abs{(\delta(r^*)+\epsilon) \cdot p(r) - \delta(r^*) \cdot p(r^*)} \leq 2^{-L'}\\
    \implies{}&\abs{\delta(r^*)(p(r)-p(r^*)) + \epsilon \cdot p(r)} \leq 2^{-L'}\\
    \implies{}&\abs{\delta(r^*)(p(r)-p(r^*))} \leq 2^{-L'} + \abs{\epsilon \cdot p(r)}\\
    \implies{}&\abs{\delta(r^*)(p(r)-p(r^*))} \leq 2^{-L'} + 2^{-L'}\abs{p(r)}
    &\text{bound on $\epsilon$}\\
    \implies{}&\abs{\delta(r^*)(p(r)-p(r^*))} \leq 2^{-L'}(K+1)
    &\text{bound on $\abs{p(r)}$}\\
    \implies{}&\abs{(p(r)-p(r^*))} \leq \frac{2^{-L'}(K+1)}{\abs{\delta(r^*)}}\\
    \implies{}&\abs{(p(r)-p(r^*))} \leq 2^{-L'}(K+1)
    &\text{since $\delta(r^*) \geq 1$.}
  \end{align*}
  It then suffices to check that $2^{-L'}(K+1) \leq 2^{-L}$, 
  which follows from $L' = L + \log(K+1)$.
\end{proof}

\LemmaPolyTimeTMExpLog*

\begin{proof}[Proof of~\Cref{lemma:poly-time-exp-log}.\ref{lemma:poly-time-exp-log:point1}]
  Let $T$ be the polynomial-time Turing machine computing $r$.
  Following~\Cref{lemma:approx-exp}, for $n \in \N$ we define $t_n(x) \coloneqq
  \sum_{j=0}^{n} \frac{x^j}{j!}$, which we see as a polynomial with rational coefficients encoded in binary (as a pair of integers).
  The pseudocode of the Turing machine for computing $e^r$ is the following:
  \begin{algorithmic}[1]
    \Require \tab A natural number $n$ written in unary
    \Ensure \tab A rational $b$ (given as a pair of integers written in binary) such that $\abs{e^r-b} \leq 2^{-n}$.%
    %
    \State \textbf{let} $J \coloneqq \abs{T_0}+1$ 
    \Comment{recall: $T_0$ computed in constant time, and $\abs{r} \leq \abs{T_0}+1$}
    \label{approxexp:line1}
    \State \textbf{let} $M \coloneqq n + 1 + 8\ceil{J}^2$
    \label{approxexp:line2}
    \State \textbf{let} $N \coloneqq n +1 + 9M^2 (\ceil{\log(J)}+1)$
    \label{approxexp:line3}
    \State \myreturn evaluation of $t_M(T_{N})$
    \label{approxexp:line4}
    \Comment{$T_N$ computed in polynomial-time in $N$}
  \end{algorithmic}
  Below, we prove that this algorithm computes $e^r$ and that it runs in
  polynomial time with respect to the input $n$.

  \noindent
  \textbf{Correctness of the algorithm.}
  From~\Cref{lemma:approx-exp}, we have $\abs{e^r-t_M(r)} \leq \frac{1}{2^{n+1}}$, 
  where $M$ is the value defined in line~\ref{approxexp:line2}.
  Below, we apply~\Cref{lemma:auxiliary-for-approximations} in order to conclude that 
  $\abs{t_M(r)-t_M(T_N)} \leq \frac{1}{2^{n+1}}$.
  Observe that this concludes the proof of correctness, since we get: 
  \[ 
    \abs{e^r-t_M(T_N)} = \abs{e^r-t_M(r)+t_M(r)-t_M(T_N)} 
    \leq \abs{e^r-t_M(r)}+ \abs{t_M(r)-t_M(T_N)} 
    \leq  2^{-n}.%
  \]

  Let $\delta(x) \coloneqq M!$ ($\delta$ is a constant integer polynomial) and
  $q(x) \coloneqq \sum_{j=0}^M ((j+1) \cdot {\dots} \cdot M \cdot x^j)$. Observe
  that $\delta(x) \cdot t_M(x)$ equals $q(x)$, and that $\delta(T_N) \geq 1$. We
  have $\max(\deg(\delta),\deg(q)) \leq~M$ and $\max(\height(\delta),\height(q))
  \leq M!\,$. Lastly, let us define $K \coloneqq 3J^M$, so that
  $\max(\abs{r},\abs{t_M(r)}) \leq K$. (Indeed, observe that $\abs{t_M(r)} =
  \abs{\sum_{j=0}^M \frac{r^j}{j!}} \leq \abs{r}^M \sum_{j=0}^M \frac{1}{j!}
  \leq e \abs{r}^M \leq 3J^M$.) By~\Cref{lemma:auxiliary-for-approximations},
  $\abs{t_M(r)-t_M(T_N)} \leq \frac{1}{2^{n+1}}$ holds as soon as $\abs{r - T_N}
  \leq \frac{1}{2^{L}}$, where $L$ is any integer satisfying $L \geq  n+1 +
  \log(M!+1) + (2 M +1)(\log(K+1))$. Since $\abs{r - T_N} \leq \frac{1}{2^{N}}$,
  it then suffices to show that $N$ defined in line~\ref{approxexp:line3}
  corresponds to such an integer $L$:
  \begin{align*}
    &n+1 + \log(M!+1) + (2 M +1)(\log(K+1))\\
    \leq{}& n+1 + \log(M!+1) + (2 M +1)(\log(3J^M+1))
      &\text{by def.~of $K$}\\
    \leq{}& n+1 + \log(M!+1) + (2 M +1)(\log(4J^M))\\
    \leq{}& n+1 + \log(M!+1) + (2 M +1)(M \log(J)+2)\\
    \leq{}& n+1 + M^2 + (2 M +1)(M \log(J)+2)
      &\hspace{-16pt}\text{since $M \geq 1$, and so $\log(M!+1) \leq M^2$}\\
    \leq{}& n+1 + 9M^2 (\log(J)+1) \\
    \leq{}& n +1 + 9M^2 (\ceil{\log(J)}+1) = N.
  \end{align*}

  \noindent
  \textbf{Running time of the algorithm.}
  Line~\ref{approxexp:line1} does not depend on the input $n$ and thus its
  computation takes constant time. Lines~\ref{approxexp:line2}
  and~\ref{approxexp:line3} compute in polynomial time the numbers $M$ and $N$,
  which are written in unary and have size in~$O(n^2)$. 
  
  To conclude the proof, we show that the computation done in
  line~\ref{approxexp:line4} takes time polynomial in $n$. One of the steps of
  this line is to compute the number $T_N$, which can be done in time $\poly(n)$
  because of the $O(n^2)$ bound on $N$. This also means that $T_N$ is of the
  form $\frac{\ell_1}{\ell_2}$ where $\ell_1,\ell_2$ are integers written in
  binary with bit size polynomial in $n$. The last step of the algorithm is to
  evaluate the expression $\sum_{j=0}^{M} \frac{(\ell_1)^j}{(\ell_2)^j \cdot
  j!}$, which equals the rational number~$\frac{b_1}{b_2}$ 
  where $b_1$ and $b_2$ are the following integers:
  \begin{align*}
    b_1 &\coloneqq \sum_{j=0}^M (\ell_1)^j \cdot
        (\ell_2)^{M-j} \cdot (j+1) \cdot {\dots} \cdot M,\\
    b_2 &\coloneqq (\ell_2)^M \cdot M!\,.
  \end{align*}
  Therefore, we can have the algorithm return~$b = \frac{b_1}{b_2}$. Both $b_1$ and $b_2$ have a bit size polynomial in $n$.
  Indeed, for $b_2$ we have 
  \begin{align*}
          &1+\ceil{\log(b_2+1)} &\text{bit size of $b_2$}\\
    \leq{}&2+\log((\ell_2)^M \cdot M!+1)\\
    \leq{}&2+\log(2(\ell_2)^M \cdot M!) &\text{since $M \geq 1$}\\
    \leq{}&3+M \cdot (\log(\ell_2) + \log(M)) &\text{since $M \geq 1$}\\
    \leq{}&3+ M \cdot (\poly(n) + \log(M)) &\text{the bit size of $\ell_2$ is polynomial in $n$}\\
    \leq{}&3+ O(n^2) \cdot (\poly(n) + \log(O(n^2))) &\text{since $M$ (written in unary) has size in $O(n^2)$}\\
    \leq{}&\poly(n).
  \end{align*}
  The analysis for $b_1$ is similar. Analogously, all intermediate computations
  done to produce $b_1$ and $b_2$ are arithmetic operations on numbers whose bit
  size can be bounded in $\poly(n)$. Since these arithmetic operations require
  polynomial time with respect to the size of their input, we conclude that
  $b_1$ and $b_2$ can be computed in polynomial time in $n$. 
  This shows that also line~\ref{approxexp:line4} takes time polynomial in $n$,
  concluding the proof. 
\end{proof}

For the forthcoming proof of~\Cref{lemma:poly-time-exp-log}.\ref{lemma:poly-time-exp-log:point2}, we need the following simple fact. 

\begin{lemma}
  \label{lemma:logx-to-x}
  For every real number~$x \in (0,1)$ 
  we have $-\log(x) > -x+1 > 0$.
\end{lemma}

\begin{proof}
  The inequality $-x+1 > 0$ is direct from the fact 
  that $x \in (0,1)$. 
  To prove the inequality ${-\log(x) > -x+1}$, 
  consider the identity $\ln(x) = \sum_{j=1}^\infty (-1)^{j+1} \frac{(x-1)^j}{j}$,
  which holds for every $x \in (0,1)$, 
  see e.g.~\cite[Equation~4.6.3]{Olver10}. 
  By truncating the power series in this identity to the first 
  term, we see that $\ln(x) < x-1$; 
  indeed note that $x \in (0,1)$ implies that all terms in 
  this power series are negative.
  Then, we have 
  \[ 
  -\log(x) = -\frac{\ln(x)}{\ln(2)} > -\ln(x) > -x+1.
  \qedhere
  \]
\end{proof}

\begin{proof}[Proof of~\Cref{lemma:poly-time-exp-log}.\ref{lemma:poly-time-exp-log:point2}]
  Let $T$ be the polynomial-time Turing machine computing $r > 0$.
  Following~\Cref{lemma:approx-log}, for $n \in \N$ we define $t_n(x) \coloneqq
  2 \cdot \sum_{j=0}^n \big(\frac{1}{2j+1} \big(\frac{x-1}{x+1}\big)^{2j+1}\big)$, 
  in which we see the rational numbers $\frac{1}{2j+1}$ as encoded in binary (as a pair of integers).
  The pseudocode of the Turing machine for computing $\ln(r)$ is the following:
  \begin{algorithmic}[1]
    \Require \tab A natural number $n$ written in unary
    \Ensure \tab A rational $b$ (given as a pair of integers written in binary) s.t.~$\abs{\ln(r)-b} \leq 2^{-n}$.%
    %
    \State \textbf{let} $k$ be the the smallest natural number 
    such that $\frac{1}{2^k} < T_k$.
    \label{approxln:line1}
    \Statex \Comment{recall: $k$ and $T_k$ are computed in constant time}
    \State \textbf{let} $L \coloneqq T_k-\frac{1}{2^k}$
    \label{approxln:line2}
    \State \textbf{let} $U \coloneqq T_k+\frac{1}{2^k}$ 
    \Comment{note: $0 < L \leq r \leq U$}
    \label{approxln:line3}
    \State \textbf{let} $Z_1 \coloneqq \ceil{\max({\abs{\ln(L)}},{\abs{\ln(U)}})}$ 
    \label{approxln:line4}
      \Comment{$Z_1$ is a positive integer}
    \State \textbf{let} $Z_2 \coloneqq 1+\min(-\abs{\frac{L-1}{L+1}},-\abs{\frac{U-1}{U+1}})$ 
      \Comment{$Z_2$ is a positive rational number}
    \label{approxln:line5}
    \State \textbf{let} $M \coloneqq \ceil{\frac{n+1+Z_1}{2 \cdot Z_2}}$
      \Comment{$M$ is a positive integer written in unary}
    \label{approxln:line6}
    \State \textbf{let} $N \coloneqq n + 2 + 15M \cdot \ceil{\log(U + 4M)}$
      \Comment{$N$ is a positive integer written in unary}
    \label{approxln:line7}
    \State \myreturn evaluation of $t_M(\abs{T_{N}})$
    \label{approxln:line8}
    \Comment{$\abs{T_N}$ computed in polynomial-time in $N$}
  \end{algorithmic}
  Below, we prove that this algorithm computes $\ln(r)$ and that it runs in
  polynomial time with respect to the input $n$.

  \noindent
  \textbf{Correctness of the algorithm.}
  We start with three observations:
  \begin{itemize}
    \item the number $k$ computed in line~\ref{approxln:line1} exists, since
      $\lim_{n \to \infty} T_k = r > 0$ whereas $\lim_{n \to \infty}
      \frac{1}{2^k} = 0$.  
    \item the values $Z_1$ and $Z_2$ are properly defined and positive, because $U > L > 0$, which in turns implies that also $M$ and $N$ are properly defined. 
    To prove that $Z_2 > 0$, observe that for every $y \geq 0$ we have $\abs{\frac{y-1}{y+1}} < 1$, hence $1 - \abs{\frac{y-1}{y+1}} > 0$.
    \item The Turing machine that on input $n$ returns $\abs{T_n}$ is a machine
    running in polynomial time and computing $r$. The latter property follows
    from the fact that $r > 0$ and therefore, for every $n \in \N$, if $T_n < 0$
    we get a better accuracy by considering $\abs{T_n}$ instead. 
    Note that this machine is used in line~\ref{approxln:line8}.
  \end{itemize}
  Below, we show \textbf{(1)} that $\abs{\ln(r)-t_M(r)} \leq \frac{1}{2^{n+1}}$,
  where $M$ is the value defined in line~\ref{approxln:line6}, 
  and \textbf{(2)} that $\abs{t_M(r)-t_M(\abs{T_N})} \leq \frac{1}{2^{n+1}}$, 
  where $N$ is the value defined in line~\ref{approxln:line7}.
  Note that this concludes the proof of correctness, since we get: 
  \begin{align*}
    \abs{\ln(r)-t_M(\abs{T_N})} &= \abs{\ln(r)-t_M(r)+t_M(r)-t_M(\abs{T_N})}\\
    &\leq \abs{\ln(r)-t_M(r)}+ \abs{t_M(r)-t_M(\abs{T_N})} 
    \leq  2^{-n}.%
  \end{align*}
  \begin{enumerate}
    \item \textbf{Proof of $\abs{\ln(r)-t_M(r)} \leq \frac{1}{2^{n+1}}$.} 
      We apply~\Cref{lemma:approx-log}. If $r = 1$, the inequality we want to prove trivially holds. Otherwise, when $r \neq 1$, 
      this inequality holds as soon as $M \geq
      (n+1+\log\abs{\ln(r)}) \big({-}2\log\abs{\frac{r-1}{r+1}}\big)^{-1}$.
      Following the definition of $M$ from line~\ref{approxln:line6}, 
      it suffices then to show that 
      \[ 
        \ceil{\frac{n+1+Z_1}{2 \cdot Z_2}} \geq \frac{n+1+\log\abs{\ln(r)}}{{-}2\log\abs{\frac{r-1}{r+1}}}\,.
      \]
      We do so by establishing that $Z_1 \geq \log\abs{\ln(r)}$ and $Z_2 \leq -\log\abs{\frac{r-1}{r+1}}$. 
      \begin{itemize}
        \item \textbf{Proof of $Z_2 \leq -\log\abs{\frac{r-1}{r+1}}$}. Note that $\abs{\frac{r-1}{r+1}} \in (0,1)$, since $r > 0$ and $r \neq 1$; hence $-\log\abs{\frac{r-1}{r+1}} > 0$. 
        By~\Cref{lemma:logx-to-x}, $-\log\abs{\frac{r-1}{r+1}} >
        -\abs{\frac{r-1}{r+1}} + 1$. By definition of $Z_2$ in
        line~\ref{approxln:line5}, it suffices to prove $-\abs{\frac{r-1}{r+1}}
        \geq \min(-\abs{\frac{L-1}{L+1}},-\abs{\frac{U-1}{U+1}})$, or,
        equivalently, $\abs{\frac{r-1}{r+1}} \leq
        \max(\abs{\frac{L-1}{L+1}},\abs{\frac{U-1}{U+1}})$. Recall that $0 < L
        \leq r \leq U$. The first derivative $f'$ of the function $f(x)
        \coloneqq \abs{\frac{x-1}{x+1}}$ is $f'(x) = \frac{2(x-1)}{(x+1)^3
        \abs{\frac{x-1}{x+1}}}$. Observe that for $x \in (0,1)$, $f'$ is always
        negative, whereas for $x > 1$, $f'$ is always positive. Therefore, if $r
        < 1$ we have $\abs{\frac{r-1}{r+1}} \leq \abs{\frac{L-1}{L+1}}$, whereas
        for $r > 1$ we have $\abs{\frac{r-1}{r+1}} \leq \abs{\frac{U-1}{U+1}}$.
        \item \textbf{Proof of $Z_1 \geq \log\abs{\ln(r)}$}. Recall that ${0 < L \leq r \leq U}$ and that $Z_1$ is define in line~\ref{approxln:line4} 
        as~$\ceil{\max(\abs{\ln(L)},\abs{\ln(U)})}$.
        Since $\log(x) \leq x$ for every $x > 0$, it suffices to show
        $\max(\abs{\ln(L)},\abs{\ln(U)}) \geq \abs{\ln(r)}$. This is immediate.
        If $r < 1$, then $0 < L \leq r$ implies $\abs{\ln(L)} \geq
        \abs{\ln(r)}$. Otherwise, if $r > 1$, then $r \leq U$ implies
        $\abs{\ln(U)} \geq \abs{\ln(r)}$.
      \end{itemize}
    \item \textbf{Proof of $\abs{t_M(r)-t_M(\abs{T_N})} \leq \frac{1}{2^{n+1}}$.}
      Recall that $t_M(x) = 2 \cdot \sum_{j=0}^M \big(\frac{1}{2j+1} \big(\frac{x-1}{x+1}\big)^{2j+1}\big)$.
      With the aim of applying~\Cref{lemma:auxiliary-for-approximations}, let us define:
      \begin{align*}
        \delta(x) &\coloneqq (x+1)^{2M+1} \prod_{j=0}^M (2j+1)\,,\\
        q(x) &\coloneqq 2 \cdot \sum_{j=0}^M \Big( (x-1)^{2j+1} (x+1)^{2(M-j)} \prod_{\substack{k=0\\k \neq j}}^M (2k+1) \Big)\,.
      \end{align*}
      Note that $\delta(x) \cdot t_M(x)$ is equivalent to $q(x)$, and that
      $\delta(\abs{T_N}) \geq 1$ (since $\abs{T_N} \geq 0$), as required by the
      lemma. Moreover, note that $\delta$ and $q$ can be rewritten as integer
      polynomials by simply expanding products such as $(x+1)^{2M+1}$ and
      $(x-1)^{2j+1}$. We analyse the degree and heights of $\delta$ and $q$ in
      this expanded form (as integer polynomials). The computation of the degree
      is straightforward, and yields $\max(\deg(\delta),\deg(q)) \leq d \coloneqq 2M+1$.
      (Note that $(x-1)^{2j+1} (x+1)^{2(M-j)}$ in the definition of $q(x)$
      expands to a polynomial in degree $2j+1+2(M-j) = 2M+1$.)
      For the height, we show that $\max(\height(\delta),\height(q)) \leq h \coloneqq (3M)^{8M}$. 
      Recall that given $m \in \N$ and $a,b \in \R$, 
      we have $(a+b)^m = \sum_{j=0}^m \binom{d}{j}a^{(m-j)}b^j$,
      which as a corollary also shows $\binom{d}{j} \leq 2^m$
      (by setting $a=b=1$). 
      Therefore, 
      \begin{align*}
        \height(\delta) 
        &\leq 2^{2M+1} \textstyle\prod_{j=0}^M (2j+1)
        \leq 2^{2M+1} (2M+1)^M\\
        &\leq 2^{3M} (3M)^M &\text{since $M \geq 1$}\\
        &\leq (3M)^{8M}.
      \end{align*}
      Similarly, for the summand $q_j(x) \coloneqq (x-1)^{2j+1} (x+1)^{2(M-j)} \prod_{\substack{k=0\\k \neq j}}^M (2k+1)$ in the definition of $q(x)$ we have
      \begin{align*}
        \height(q_j) 
        \leq 2^{2M+1} 2^{2M} \textstyle\prod_{j=0}^M (2j+1) \leq 2^{4M+1} (2M+1)^M,
      \end{align*}
      and therefore $\height(q) \leq 2(M+1)2^{4M+1} (2M+1)^M 
      \leq (3M)^{8M}$.

      Lastly, let us define $K \coloneqq \max(U,2(M+1))$. 
      Note that $K \geq 1$ and $\max(\abs{r},\abs{t_M(r)}) \leq K$, 
      since $0 < r < U$ and 
      $\abs{t_M(r)} = \abs{2 \cdot \sum_{j=0}^M \big(\frac{1}{2j+1} \big(\frac{r-1}{r+1}\big)^{2j+1}\big)} \leq 2 \cdot \sum_{j=0}^M \frac{1}{2j+1} \leq 2(M+1)$, 
      because $\frac{r-1}{r+1} \in (-1,1)$.

      Following the fact that $\abs{r-\abs{T_N}} \leq \frac{1}{2^N}$,
      by applying~\Cref{lemma:auxiliary-for-approximations} with respect 
      to the above-defined objects $\delta(x)$, $q(x)$, $d$, $h$ and $K$, 
      and conclude that $\abs{t_M(r)-t_M(\abs{T_N})} \leq \frac{1}{2^{n+1}}$ 
      holds as soon as $N \geq n+1+ \log(h+1) + (2 d +1)(\log(K+1))$.
      From the definition of $N$ in line~\ref{approxln:line7}, 
      it thus suffices to show $\log(h+1) + (2 d +1)(\log(K+1)) \leq 1 + 15M \cdot \ceil{\log(U+4M)}$.
      This inequality indeed holds (recall: $U > 0$ and $M \geq 1$): 
      \begin{align*}
          &\log(h+1) + (2 d +1)(\log(K+1))\\
        \leq{}& \log((3M)^{8M}+1) + (2 (2M+1) +1)(\log(\max(U,2(M+1))+1))\\
        \leq{}& \log(2(3M)^{8M}) + 7M \cdot \log(U+4M)\\ 
        \leq{}& 1 + 8M \cdot \log(3M) + 7M \cdot \log(U+4M)\\ 
        \leq{}& 1 + 15M \cdot \ceil{\log(U+4M)}.
      \end{align*}
  \end{enumerate}

  % Let $\delta(x) \coloneqq M!$ ($\delta$ is a constant integer polynomial) 
  % and $q(x) \coloneqq \sum_{j=0}^M ((j+1) \cdot {\dots} \cdot M \cdot x^j)$.
  % Observe that $\delta(x) \cdot t_M(x)$ equals $q(x)$, 
  % and that $\delta(T_N) \geq 1$.
  % We have $\max(\deg(\delta),\deg(q)) \leq M$ and $\max(\height(\delta),\height(q)) \leq M!$.
  % Lastly, let $K \coloneqq (M+1)J^M$, so that $\max(\abs{r},\abs{t_M(r)}) \leq K$.
  % By~\Cref{lemma:auxiliary-for-approximations}, 
  % $\abs{t_M(r)-t_M(T_N)} \leq \frac{1}{2^{n+1}}$ 
  % holds as soon as $\abs{r - T_N} \leq \frac{1}{2^{L}}$, 
  % where $L$ is any integer satisfying $L \geq  n+1 + \log(M!+1) + (2 M +1)(\log(K+1))$. 
  % It then suffices to show that $N$ defined in line~\ref{approxexp:line3} is such an 
  % integer $L$:

  \noindent
  \textbf{Running time of the algorithm.}
  Lines~\ref{approxln:line1}--\ref{approxln:line5} do not 
  depend on the input $n$, and therefore the computation of $k$, $L$, $U$, $Z_1$ and $Z_2$ 
  takes constant time. Line~\ref{approxln:line6} computes in polynomial time in $n$ 
  the number $M$, which is written in unary and has size $O(n)$.
  Similarly, line~\ref{approxln:line7} computes in polynomial time in $n$ 
  the number $N$, which is written in unary and has size~$O(n \log n)$.  
  
  To conclude the proof, we show that the computation done in
  line~\ref{approxln:line8} takes time polynomial in $n$. The arguments are
  analogous to the one used at the end of the proof
  of~\Cref{lemma:poly-time-exp-log}.\ref{lemma:poly-time-exp-log:point1}. First,
  line~\ref{approxln:line8} compute the number $\abs{T_N}$; this can be done in
  time $\poly(n)$ because of the $O(n \log n)$ bound on $N$. This also means
  that $\abs{T_N}$ is of the form $\frac{\ell_1}{\ell_2}$ where $\ell_1,\ell_2$
  are non-negative integers written in binary with bit size polynomial in $n$,
  and $\ell_2 \geq 1$. The last step of the algorithm is to evaluate the
  expression $2 \cdot \sum_{j=0}^M \big(\frac{1}{2j+1}
  \big(\frac{\frac{\ell_1}{\ell_2}-1}{\frac{\ell_1}{\ell_2}+1}\big)^{2j+1}\big)$,
  which equals the rational number $\frac{b_1}{b_2}$, where $b_1$ and $b_2$ are
  the following integers:
  \begin{align*}
    b_1 &\coloneqq \sum_{j=0}^M \left( (\ell_1+\ell_2)^{2(M-j)}(\ell_1-\ell_2)^{2j+1}\prod_{\substack{k=0\\k\neq j}}^M(2k+1) \right),\\
    b_2 &\coloneqq (\ell_1+\ell_2)^{2M+1} \prod_{j=0}^M (2j+1).
  \end{align*}
  Therefore, we can have the algorithm return~$b = \frac{b_1}{b_2}$. Both $b_1$
  and $b_2$ have a bit size polynomial in $n$. Indeed, for $b_2$ we have 
  \begin{align*}
          &1+\ceil{\log(b_2+1)} &\text{bit size of $b_2$}\\
    \leq{}&2+\log\Big((\ell_1+\ell_2)^{2M+1} \prod_{j=0}^M (2j+1)+1\Big)\\
    \leq{}&2+\log\Big(2(\ell_1+\ell_2)^{2M+1} \prod_{j=0}^M (2j+1)\Big) &\text{since $M \geq 1$}\\
    \leq{}&3+ (2M+1)\log(\ell_1+\ell_2) + M \log(2M+1) &\text{since $\textstyle\prod_{j=0}^M (2j+1) \leq (2M+1)^M$}\\
    \leq{}&3+ (2M+1) \cdot \poly(n) + M \log(2M+1) &\text{the bit sizes of $\ell_1$ and $\ell_2$ are in $\poly(n)$}\\
    \leq{}&3+ O(n) \cdot \poly(n) + O(n) \log( O(n) ) &\hspace{-8pt}\text{as $M$ (written in unary) has size in $O(n)$}\\
    \leq{}&\poly(n).
  \end{align*}
  The analysis for $b_1$ is similar. Moreover, all intermediate computations
  done to produce $b_1$ and $b_2$ are arithmetic operations on numbers whose bit
  size is in $\poly(n)$. As these arithmetic operations require
  polynomial time with respect to the size of their input, we conclude that
  $b_1$ and $b_2$ can be computed in polynomial time in $n$. 
  This concludes the proof.
\end{proof}

\begin{restatable}{lemma}{LemmaCheckRationality}
  \label{lemma:check-rationality}
  There is an algorithm deciding whether an input algebraic number $\beta$ 
  represented by $(q,\ell,u)$ is rational.
  When $\beta$ is rational, the algorithm returns $m,n \in \Q$ such that 
  $\beta = \frac{m}{n}$.
\end{restatable}

\begin{proof}
  By relying on the LLL-based algorithm from~\cite{lenstra1982}, 
  we can compute (in fact, in polynomial time)
  a decomposition of the univariate polynomial $q$ into 
  irreducible polynomials (below, factors) with rational coefficients. 
  Let $E$ be the (finite) set of those factors having degree~$1$. 
  Since $\beta$ is a root of $q$, we have that $\beta$ is rational if and only if it is a root of a polynomial in~$E$.
  Every element of $E$ is a linear polynomial of the form $n \cdot x - m$, where $n,m \in \Q$, having root $\frac{m}{n}$. 
  Recall that $\beta$ is the only root of $q$ in the interval $[\ell,u]$, and therefore, in order to check whether $\beta$ is rational, 
  it suffices to check whether there is $(n \cdot x - m) \in E$
  such that $\ell \leq \frac{m}{n} \leq u$. If the answer is positive, $\beta = \frac{m}{n}$. Otherwise, $\beta$ is irrational. 
\end{proof}

\LemmaRepresentationPowerOfAlgebraic*

\begin{proof}
  Let $r = \frac{m}{n}$ with $m \in \Z$ and $n \geq 1$, and let $q(x) =
  \sum_{i=0}^d a_i \cdot x^i$, with $\deg(q) = d$, and $h \coloneqq \height(q)$.
  Since we are not interested in the runtime of this algorithm, we can apply the
  procedure explained at the beginning of~\Cref{appendix:sec-poly-evaluation} to
  impose that (in addition to $\alg$ being the only root of $q$ in the interval
  $[\ell,u]$) either $\ell = u$ or $\alg \in (\ell,u)$ and $(\ell,u) \cap \Z =
  \emptyset$ holds. Since $\alg > 0$, by
  applying~\Cref{theorem:alg-root-barrier} to the polynomial~$x$ we derive $\alg
  \geq 2^{-d}(h(d+1))^{-1}$, and so we can update $\ell$ and $u$ to be both
  strictly positive.
  
  First, let us reduce the problem to the case $m \geq 1$. If $m = 0$ or then
  $\alg^0 = 1$ and we can simply return $(x-1,1,1)$. To handle the case $m < 0$,
  we remark that $\alg^{-1}$ is a root of the Laurent polynomial $\sum_{i = 0}^d
  a_i \cdot x^{-i}$, and thus also of $x^d \cdot \sum_{i=0}^d a_i \cdot x^{-i}$.
  So, the polynomial $q''(x) \coloneqq \sum_{i=0}^d a_i \cdot x^{d-i}$ is such
  that $(q'',u^{-1},\ell^{-1})$ represents $\alg^{-1}$ (note that no
  root~$\beta$ of $q''$ that is distinct from $\alg^{-1}$ can lie in the
  interval $[u^{-1},\ell^{-1}]$, else $\beta^{-1} \neq \alg$ would lie in
  $[\ell,u]$). We can then compute the representation of $\alg^r$ starting from
  $(q'',u^{-1},\ell^{-1})$, and considering the positive rational $-r$ instead
  of $r$.

  Below, assume $m,n \geq 1$. We start by computing a polynomial $Q(x)$ 
  having $\alg^m$ as a root.
  Since $q(\alg)=0$, for every $j \in \N$, we can express $\alg^j$ as a
  rational linear combination $\mu(j)$ of the terms $1,\alg,\dots,\alg^{d-1}$: 
  \[ 
    \mu(j) \coloneqq  
      \begin{cases}
        \alg^j &\text{if $j \in [0..d-1]$}\\
        \sum_{i=0}^{d-1}\frac{-a_i}{a_d} \alg^{i} &\text{if $j = d$}\\
        b_{d-1}\mu(d) + \sum_{i=0}^{d-2}b_i\alg^{i+1} &\text{if $j > d$, where $\mu(j-1) = \sum_{i=0}^{d-1} b_i \alg^i$}.
      \end{cases}
  \]
  (Note that the last line in the definition of $\mu(j)$ is obtained by
  multiplying $\mu(j-1)$ by $\alg$, to then replace $\alg^d$, which is the only
  monomial with degree above $d-1$, by $\mu(d)$.)

  We can represent the polynomial $\mu(j) = \sum_{i=0}^{d-1} b_i \alg^i$ as the
  vector $(b_0,\dots,b_{d-1}) \in \Q^d$. Consider now the family of polynomials
  $\mu(0),\mu(m),\mu(2m),\dots,\mu(i \cdot m),\dots,\mu(d \cdot m)$. These
  correspond to a set of $d+1$ vectors in $\Q^d$, and therefore they are
  rationally dependent: there is a non-zero vector $(k_0,\dots,k_{d}) \in
  \Q^{d+1}$ such that 
  \[ 
    k_0 \cdot \mu(0) + k_1 \cdot \mu(m) + \dots + k_d \cdot \mu(d \cdot m) = 0.
  \]
  Since $\mu(j) = \alg^j$ for all $j \in \N$, we then conclude that
  $\sum_{j=0}^d k_j \alg^{j \cdot m} = 0$. Let $g$ be the least common multiple
  of the denominators of the rational numbers $k_0,\dots,k_{d}$, and define
  $\hat{k}_j = g \cdot k_j$ for all $j \in [0..d]$. Then, $\alg^m$ is a root of
  the non-zero integer polynomial $Q(x) \coloneqq \sum_{j=0}^d \hat{k}_j \cdot
  x^{j}$.

  We can now take $q'(x) \coloneqq Q(x^n)$ in order to obtain a polynomial
  having $\alg^{\frac{m}{n}}$ as a root.

  %\am{yes it would be good to explain how to refine the interval $(\ell',u')$. I'm thinking that we could look at bounds on the root separation and study $\abs{\ell^m-u^m}$ in order to obtain a bound on $\abs{\ell-u}$ such that $(q'',\ell',u')$ can be taken with $\{\ell',u'\} = \{\ell^m,u^m\}$. Then, it suffices to refine $\ell$ and $u$ accordingly.}

  Now we move on to the problem of isolating $\alg^{\frac{m}{n}}$ from all other
  roots of $q'(x)$ by opportunely defining a separating interval $[\ell',u']$
  where $\ell'$, $u'\in\Q$.

  If $q'$ has degree $1$, then $\alg^{\frac{m}{n}}$ is its only root and it is
  rational. Finding an interval is in this case trivial: given $q'(x) = b \cdot
  x - a$, we have $\alg^{\frac{m}{n}} = \frac{a}{b}$ and so we can take $\ell' =
  u' = \frac{a}{b}$. Hence, below, let us assume $\deg(q') \geq 2$.
  To compute $\ell'$ and $u'$ we need the following result. 

  \begin{claim}
    \label{claim:mean-value}
    Let $0<\ell\leq u$ be rational numbers. Consider a function $f(x)$
    that is both increasing and continuously differentiable in the
    interval~$[\ell,u]$. Let $\delta > 0$ be an upper bound to the maximum of
    its derivative over $[\ell,u]$. If $\abs{u-\ell}\leq \frac{D}{\delta}$, then
    $\abs{f(\ell)-f(u)}\leq D$.
  \end{claim}
  \begin{claimproof}
    Since $f(x)$ is continuously differentiable over $[\ell,u]$, 
    by the mean value theorem we have $\frac{f(u)-f(\ell)}{u-\ell} \leq \delta$. 
    Moreover, since $f(x)$ is increasing inside $[\ell,u]$, then 
    $\frac{f(u)-f(\ell)}{u-\ell} = \frac{\abs{f(u)-f(\ell)}}{\abs{u-\ell}}$.
    We conclude that $\abs{f(u)-f(\ell)}\leq \delta\cdot\abs{u-\ell} \leq \delta \cdot  \frac{D}{\delta} \leq D$.
  \end{claimproof}

  %\begin{claim}
   % Let $l$, $u$ and $r>1$ be rational numbers. Consider $L$, $M\in\N$.
   % If $M\geq r\log(r)+r\abs{\log(\ell)}+L$ , then $\abs{l-u}\leq 2^{-M}$ implies 
   % $\abs{l^r-u^r}\leq 2^{-L}$
  %\end{claim}
  %\begin{claimproof}
  %  Consider the real number $\mu$ such that $u=\ell+\mu$, observe that 
  %  $\abs{\mu}\leq 2^{-M}$. Let us write $r\coloneqq\ceil{r}$.
  %  \begin{align*}
  %    \abs{\ell^r-u^r}&\leq \abs{\ell^r-(\ell+\mu)^r}=&\text{explain steps}\\
  %    &=\abs{\sum_{k=0}^{r-1}\binom{r}{k}\ell^k\mu^{r-k}}\leq\\
  %    &\leq \binom{r}{\floor{\frac{r}{2}}}\abs{\sum_{k=0}^{r-1}\ell^k\mu^{r-k}}\leq\\
  %    &\leq \binom{r}{\floor{\frac{r}{2}}}r\cdot\max\{1,\ell^r\}\cdot \mu
  %  \end{align*}
  %  If we choose $M\coloneqq r\abs{\log(r)}+\max\{1,r\abs{\log(\ell)}\}+L$ then 
  %  \begin{align*}
  %    M&= r\log(r)+\max\{1,r\abs{\log(\ell)}\}+L\implies\\
  %    M&\geq r\log(r)+\max\{1,r\log(\ell)\}+L\implies\\
  %    2^M&\geq r^r\cdot \max\{1,\ell^r\}\cdot 2^L\implies
  %    2^{-M}\cdot r^r\cdot \max\{1,\ell^r\}\leq 2^{-L}\implies\\
  %    &\mu \binom{r}{\floor{\frac{r}{2}}}\max\{1,\ell^r\}\leq 2^{-L}
  %  \end{align*}
  %  hence $\abs{\ell^r-u^r}\leq 2^{-L}$ is satisfied. 
  %\end{claimproof}

  Below, let $h' \coloneqq \height(q')$ and $\deg(q') \coloneqq d'$.
  By applying~\cite[Theorem~A.2]{Bugeaud04}, any two distinct roots $\alg_1$ and
  $\alg_2$ of $q'$ satisfy:
  \begin{equation}
    \label{eq:close-roots}
    \abs{\alg_1 - \alg_2} > D \coloneqq 2^{-d'-1} (d')^{-4d'} (h')^{-2d'}.
  \end{equation}
  Let $\delta \coloneqq \max_{x\in\{\ell,u\}}\{r\cdot x^{r-1}\}$,
  which is maximum of the derivative of $f(x) \coloneqq x^r$ in the interval $[\ell,u]$.
  Let us apply the algorithm in~\Cref{lemma:approx-algebraic} 
  in order to refine the interval $[\ell,u]$ containing $\alg$ 
  so that we achieve 
  \begin{equation*}
    \abs{\ell-u}\leq \frac{D}{2\delta}.
  \end{equation*}

  Note that, since $r > 0$, the function $f$ is increasing and continuously
  differentiable in $[\ell,u]$, from $\alg \in [\ell,u]$ we have $\alg^r\in
  [\ell^r,u^r]$. Moreover, by~\Cref{claim:mean-value}, 
  we have $\abs{u^r-\ell^r} \leq \frac{D}{2}$.
  From~\Cref{eq:close-roots}, 
  we conclude that $\alg^r$ is the only root in the interval $[\ell^r,u^r]$.

  Note that, in general, $\ell^r$ and $u^r$ are not rational numbers, 
  hence we cannot use $(q',\ell^r,u^r)$ in order to represent $\alg^r$. 
  Instead, we now compute two rational numbers $\ell'<\ell^r$ and $u'>u^r$ 
  such that $\alg^r \in [\ell',u']$ and, crucially, $\abs{u'-\ell'}\leq D$. 
  Again, by~\Cref{eq:close-roots}, we will conclude that 
  $\alg^r$ is the only root of $q'$ in $[\ell',u']$, 
  and therefore $(q',\ell',u')$ represents $\alg^r$.

  In order to compute $\ell'$ and $u'$, we rely on two Turing machines $T$ and
  $T'$ computing $\ell^r$ and $u^r$, respectively. To construct these machines,
  we simply apply~\Cref{lemma:turing-machine-products}
  and~\Cref{lemma:poly-time-exp-log}, seeing $\ell^r$ as $e^{r \cdot
  \ln(\ell)}$ and $u^r$ as $e^{r \cdot \ln(u)}$ (note that $\ell,u > 0$, hence
  the two logarithms are well-defined). Since $\ell^r$ and $u^r$ are positive,
  w.l.o.g.~we can assume the outputs of $T$ and $T'$ to be always non-negative.
  Indeed, to force this condition on, e.g., $T$, we can consider a new Turing
  machine that on input $n \in \N$ returns $\abs{T_n}$; this new Turing machine
  still computes $\ell^r$.
  Let $M \coloneqq -\floor{\log(D)}$, and observe that $M \geq 1$, since $D \in (0,1)$.
  
  We are now ready to define the rationals $\ell'$ and $u'$: 
  \[ 
    \ell'\coloneqq T_{M+3}-\frac{1}{2^{M+3}}
    \qquad\text{and}\qquad 
    u'\coloneqq T'_{M+3}+\frac{1}{2^{M+3}}.
  \]
  Recall that $\abs{\ell^r-T_{M+3}} \leq \frac{1}{2^{M+3}}$, and similarly
  $\abs{u^r-T_{M+3}} \leq \frac{1}{2^{M+3}}$. Therefore, $\ell' \leq \ell^r \leq u^r \leq u'$, which in turn implies that $\alg^r \in [\ell',u']$. Moreover,
  we also conclude that $\ell^r-\frac{1}{2^{M+2}} \leq \ell'$ and $u'\leq
  u^r+\frac{1}{2^{M+2}}$.
  At last, let us show that $\abs{u'-\ell'} \leq D$:
  \begin{align*}
    \abs{u'-\ell'}&\leq \abs{u^r+\frac{1}{2^{M+2}}-\Big(\ell^r-\frac{1}{2^{M+2}}\Big)}
    &\quad\text{since $\ell^r-\frac{1}{2^{M+2}} \leq \ell' \leq u'\leq
  u^r+\frac{1}{2^{M+2}}$}\\
    &\leq\abs{u^r-\ell^r}+\frac{1}{2^{M+1}}\\
    &\leq \frac{D}{2} + \frac{1}{2^{-\floor{\log(D)}+1}}
    &\text{by def.~of~$M$ and since~$\abs{u^r-\ell^r} \leq \frac{D}{2}$}\\ 
    &\leq \frac{D}{2} + \frac{D}{2} \leq D.
    &&\qedhere
  \end{align*}
\end{proof}

\begin{restatable}{lemma}{LemmaMultIndependeceRationality}
  \label{lemma:mult-independence-rationality}
  Let $\alg$ and $\beta$ be two algebraic numbers different from $0$ and $1$. Then, $\alpha$ and $\beta$ are multiplicatively dependent 
  if and only if $\frac{\ln(\alg)}{\ln(\beta)}$ is rational.
\end{restatable}

\begin{proof}
  Let $n,m \in \Z$. With either $n$ or $m$ distinct from zero.
  We have 
  \begin{align*}
    \alg^n = \beta^m 
    \iff \ln(\alg^n) = \ln(\beta^m)
    \iff n\ln(\alg) = m\ln(\beta)
    \iff \frac{\ln(\alg)}{\ln(\beta)} = \frac{m}{n},
  \end{align*}
  where we note that one of the two sides of the equality $n\ln(\alg) =
  m\ln(\beta)$ must be non-zero (because $n$ or $m$ are non-zero, and
  $\alg,\beta \neq 1$) which makes non-zero also the other side.
  % Assume $\alg$ and $\beta$ are multiplicatively dependent so there exist
  % two non-zero integers $n$ and $m$ such that $\alg^n=\beta^m$. Then,
  % \begin{equation*}
  %   \frac{\ln(\alg)}{\ln(\beta)}=\frac{n\cdot m\cdot\ln(\alg) }{n\cdot m\cdot\ln(\beta)} = 
  %   \frac{m\cdot\ln(\alg^n) }{n\cdot\ln(\beta^m)}=\frac{m}{n}
  % \end{equation*}
  % hence $\frac{\ln(\alg)}{\ln(\beta)}$ is rational.
  % Assume now that $\frac{\ln(\alg)}{\ln(\beta)}$ is rational for instance 
  % $\frac{\ln(\alg)}{\ln(\beta)}=\frac{m}{n}$. Then,
  % \begin{align*}
  %   &\frac{m}{n}=\frac{\ln(\alg)}{\ln(\beta)}\implies
  %   1=\frac{n\cdot \ln(\alg)}{m\cdot \ln(\beta)}\implies
  %   1=\frac{\ln(\alg^n)}{\ln(\beta^m)}\implies\\
  %   &\ln(\beta^m)=\ln(\alg^n)\implies
  %   \beta^m=\alg^n\implies \beta^m\cdot \alg^{-n}= 1
  % \end{align*}
  % hence $\alg$ and $\beta$ are multiplicatively dependent.
\end{proof}

\TheoremRootBarrier*

\begin{proof}
  The proof of
  \Cref{theorem:result-root-barrier}.\ref{theorem:result-root-barrier:point1} is
  given in~\Cref{sec:poly-evaluation}.
  \Cref{theorem:result-root-barrier}.\ref{theorem:result-root-barrier:point3}
  follows from~\Cref{lemma:correctness-algorithm-1} for bases $\cn > 1$. The case
  for bases~$\cn \in (0,1]$ can be reduced to the case for bases $\cn > 1$, as
  discussed in~\Cref{subsection:small-bases}. 

  Below, let us focus on~\Cref{theorem:result-root-barrier}.\ref{theorem:result-root-barrier:point2}. Following~\Cref{theorem:general-result-root-barrier}.\ref{theorem:general-result-root-barrier:point2}, 
  it suffices to show that all bases considered in this case \textbf{(1)}~are computable by a polynomial-time Turing machine, and \textbf{(2)}~have a polynomial root barrier. 

  \begin{description}
    \item[case: $\cn = \pi$.]~\\
    \textit{Polynomial-time Turing machine:} By~\Cref{lemma:poly-time-pi}.\\
    \textit{Polynomial root barrier:} See~\Cref{table:transcendence-degrees}. 
    \item[case: $\cn = e^\pi$.]~\\
    \textit{Polynomial-time Turing machine:} By~\Cref{lemma:poly-time-pi} and~\Cref{lemma:poly-time-exp-log}.\ref{lemma:poly-time-exp-log:point1}.\\
    \textit{Polynomial root barrier:} See~\Cref{table:transcendence-degrees}.
    \item[case: $e^\eta$.]~\\
    \textit{Polynomial-time Turing machine:} By~\Cref{lemma:approx-algebraic-body} and~\Cref{lemma:poly-time-exp-log}.\ref{lemma:poly-time-exp-log:point1}.\\
    \textit{Polynomial root barrier:} See~\Cref{table:transcendence-degrees}.
    \item[case: $\alg^\eta$ with $\alg > 0$.]~\\
    \textit{Polynomial-time Turing machine:} Consider $e^{\eta \cdot \ln(\alg)}$, and construct the Turing machine by applying~\Cref{lemma:approx-algebraic-body}, \Cref{lemma:turing-machine-products} and~\Cref{lemma:poly-time-exp-log}.\ref{lemma:poly-time-exp-log:point1}.\\
    \textit{Polynomial root barrier:} Use~\Cref{lemma:check-rationality} to check if $\eta$ is rational. If it is, apply~\Cref{lemma:representation-power-of-algebraic} to obtain a representation of the algebraic number $\alg^\eta$, followed by~\Cref{theorem:alg-root-barrier} to obtain a root barrier for it. If instead $\eta$ is irrational, use~\Cref{table:transcendence-degrees}.
    \item[case: $\cn = \ln(\alg)$ with $\alg > 0$.]~\\
    \textit{Polynomial-time Turing machine:} By~\Cref{lemma:approx-algebraic-body} and~\Cref{lemma:poly-time-exp-log}.\ref{lemma:poly-time-exp-log:point2}.\\
    \textit{Polynomial root barrier:} See~\Cref{table:transcendence-degrees}. 
    \item[case: $\cn = \frac{\ln(\alg)}{\ln(\beta)}$ with $\alg,\beta > 0$ (and $\beta \neq 1$).]~\\
    \textit{Polynomial-time Turing machine:} By~\Cref{lemma:approx-algebraic-body} and~\Cref{lemma:poly-time-exp-log}.\ref{lemma:poly-time-exp-log:point2} and~\Cref{lemma:turing-machine-reciprocal}.\\
    \textit{Polynomial root barrier:} From~\Cref{lemma:mult-independence-rationality},
    $\cn > 0$ is rational if and only if $\alg$ and $\beta$ are multiplicatively dependent. Use the procedure from~\cite{CaiLZ00} 
    to compute a basis $K$ of the finitely-generated integer lattice $\{(m,n) \in \Z^2 : \alg^n \beta^{-m} = 1\}$. If $K =
    \{(0,0)\}$ then $\cn$ is irrational and its root barrier is given in~\Cref{table:transcendence-degrees}. Otherwise there is~$(m,n) \in K$ with $n \neq 0$, and
    $\cn = \frac{m}{n}$. We then derive a polynomial
    root barrier of $\cn$ by applying~\Cref{theorem:alg-root-barrier} to the polynomial $n \cdot x - m$.
    \qedhere
  \end{description}
\end{proof}

% \begin{restatable}{lemma}{LemmaApproxPolynomials}
%   \label{lemma:approx-polynomials}
%   Let $p(x,y)=\sum_{i=0}^{d}\sum_{j=0}^{d} a_{ij}x^jy^i$ be an integer
%   polynomial, and let ${r_1,r_2 \in \R}$ with both $\abs{r_1} \leq K$ and $\abs{r_2}
%   \leq K$ for some $K\in \N$.
%   Consider $L,M \in \N$ satisfying 
%   \[
%   M\geq L + 4 + \log(\height(p)) + 5d \cdot \log(K+2).
%   \]
%   For every $r_1^*,r_2^* \in \R$, if $\abs{r_i-r_i^*} \leq 2^{-M}$ for both $i \in \{1,2\}$, 
%   then $\abs{p(r_1,r_2)-p(r_1^*,r_2^*)} \leq 2^{-L}$.
% \end{restatable}

% \begin{proof}
%   This proof is inspired by the reasoning on \cite{newbery1974} with
%   the difference that for us, the source of error is only the approximation to
%   the real value of the variables since we consider operations are computed
%   exactly. It will be useful to rewrite the polynomial as follows:
%   \begin{equation*}
%     \sum_{i,j=0}^d a_{ij}x^jy^i = 
%     \sum_{i=0}^d y^i\left( \sum_{j=0}^d a_{ij} x^j \right)
%   \end{equation*}
%   Let us study the error, $w$, introduced when evaluating a polynomial
%   $A(x)=\sum_{j=0}^d a_{j} x^j$. Horner's scheme gives

%   \begin{equation*}
%     A(x)=b_0, \quad \text{where } b_d=a_d,\text{ and } b_k=a_k+xb_{k+1}\quad
%     k=d-1,d-2,\dots,1,0.
%   \end{equation*}
%   Consider $r_1^*$ where $r_1=r_1^*+\epsilon$, (with 
%   $\abs{\epsilon}\leq 2^{-M}$). Then we can write the following sequence where 
%   $A(r_1)=A(r_1^*+\epsilon)=b_0$.
%   \begin{align*}
%     &b_d=a_d\\
%     &b_k=a_k+r_1^*b_{k+1}+\epsilon b_{k+1},\text{ for }k=d-1,d-2,\dots,1,0
%   \end{align*}
%   we denote $\delta_k=\epsilon\cdot b_{k+1}$ for $k\in[0..d-1]$ 
%   hence $\abs{\delta_k}\leq 2^{-M}\abs{ b_{k+1}}$.
%   We can also write 
%   \begin{align*}
%     &b_d^*=a_d\\
%     &b_k^*=a_k+r_1^*b_{k+1}^*,\text{ for }k=d-1,d-2,\dots,1,0
%   \end{align*}
%   where $A(r_1^*)=b_0^*$.

%   Let us write the previous sequences in matrix form with the help of
%   \begin{equation*}
%     H(x)\coloneqq\begin{bmatrix}
%     1 &  &  &  \\
%     -x & 1 &  &  \\
%     & \dots & \dots &  \\
%     &  & -x & 1 
%     \end{bmatrix}  \quad\text{ and}\quad H^{-1}(x)\coloneqq\begin{bmatrix}
%     1 & & &  &  \\
%     x & 1 & & &  \\
%     x^2& x& 1 & & \\
%     \dots&  &  &\dots & \\
%     x^d& \dots & x^2 & x&1 
%     \end{bmatrix}.  
%   \end{equation*}
%   We define the vectors $\vec{b}$, $\vec{b^*}$, $\vec{a}$ and $\vec{\delta}$
%   as $\vec{b}\coloneqq (b_d, b_{d-1},\dots,b_1, b_0)$ and similarly for 
%   $\vec{b^*}$, $\vec{a}$ and $\vec{\delta}$, in the case of $\vec{\delta}$ 
%   we need to define $\delta_d\coloneqq0$.
%   Then 
%   \begin{equation}\label{sequenceMatrix}
%     H(r_1^*)\vec{b}^*=\vec{a}\quad and \quad H(r_1^*)\vec{b}=\vec{a}+\vec{\delta}
%   \end{equation}
%   express the approximation sequence and the exact sequence respectively.

%   The evaluation error is given by 
  
%   \begin{equation*}
%     \abs{w}=\abs{b_0-b_0^*}=
%     \abs{\sum_{k=0}^d\delta_k(r_1^*)^k}\leq \sum_{k=0}^d\abs{\delta_k(r_1^*)^k}\leq 
%     \sum_{k=0}^d\abs{\epsilon b_{k+1}(r_1^*)^k}=\sum_{k=1}^d\abs{\epsilon b_{k}(r_1^*)^k}.
%   \end{equation*}
%   where the second equality comes from multiplying by $H^{-1}$ on the left in both
%   equations from \ref{sequenceMatrix} and then observing that 
%   $b_0^*=\left((r_1^*)^d,(r_1^*)^{d-1},\dots, (r_1^*),1\right)\cdot 
%   (a_d,a_{d-1},\dots,a_1,a_0)$ and 
%   $b_0=\left((r_1^*)^d,(r_1^*)^{d-1},\dots, (r_1^*),1\right)\cdot 
%   (a_d+\delta_d,\dots,a_1+\delta_1,a_0+\delta_0)$, here the dot denotes the 
%   dot product. For the last equality recall that $\delta_d=0$.

%   To compute a bound for $\abs{w}$ we will use the following statement:
%   \begin{claim}
%     For every $j\in[0..d]$, it holds that $b_j=\sum_{l=j}^da_l\cdot r_1^{l-j}$.
%   \end{claim}
%   \begin{claimproof}
%     The proof is by induction. The case for $j=d$ is trivially true since
%     by definition $b_d=a_d=\sum_{l=d}^d a_l\cdot r_1^{l-d}$.

%     Assuming the statement is true for $j$ let us show it is also true 
%     for $j-1$.
%     \begin{align*}
%       b_{j-1}&=a_{j-1}+r_1b_j=& \text{by the Induction hypotheses}\\
%       &=a_{j-1}+r_1\sum_{l=j}^da_l\cdot r_1^{l-j}=\sum_{l=j-1}^da_l\cdot r_1^{l-(j-1)}
%     \end{align*}
%     and the statement follows.

%   \end{claimproof}

%   With this result at hand
%   \begin{align*}
%     \abs{w}&=\abs{b_0-b^*_0}\leq \sum_{j=1}^{d}\abs{\epsilon b_j\cdot (r_1^*)^j}\leq
%     \abs{\epsilon}\sum_{j=1}^d\abs{\left(\sum_{l=j}^da_l\cdot r_1^{l-j}\right)
%     (r_1^*)^j}\leq\\
%     &\leq \abs{\epsilon} \height(p)\sum_{j=1}^d\sum_{l=j}^d\abs{r_1^{l-j}(r_1^*)^j}
%     \leq \abs{\epsilon} \height(p) \sum_{j=1}^d\sum_{l=j}^d(K+1)^l\leq
%     \abs{\epsilon} \height(p) (K+1)^d\sum_{j=1}^d\sum_{l=j}^d 1\leq \\
%     &\leq \abs{\epsilon} \height(p) (K+1)^d \sum_{j=1}^d (d-j)\leq 
%     \abs{\epsilon} \height(p) (K+1)^d \sum_{j=1}^d j\leq 
%     \abs{\epsilon} \height(p) (K+1)^d (d+1)^2
%   \end{align*}


%   We move to the error estimation of the value of $$p(x,y)=\sum_{i,j=0}^d a_{ij}x^jy^i
%   =\sum_{i=0}^dy^i\left( \sum_{j=0}^d a_{ij}x^j \right)$$ at $x=r_1$ and $y=r_2$. 
%   Now both the approximation to the 
%   variable $y=r_2$ and the approximation to the coefficients 
%   $p_i(r_1)=\left( \sum_{j=0}^d a_{ij}r_1^j \right)$
%   of the polynomial are 
%   sources of error. Since the polynomials $p_i(x)$ have degree and height
%   smaller than the ones from $p(x,y)$, the error commited in evaluating them 
%   at $r_1^*$ is bounded in magnitude by $\abs{w}$.

%   As in the previous step we consider two sequences:
%   \begin{align*}
%     &q_d=p_d+\delta_d\\
%     &q_k=p_k+r_2^*q_{k+1}+\delta_k,\text{ for }k=d-1,d-2,\dots,1,0
%   \end{align*}
%   and 
%   \begin{align*}
%     &q_d^*=p_d\\
%     &q_k^*=p_k+r_2^*q_{k+1}^*,\text{ for }k=d-1,d-2,\dots,1,0
%   \end{align*}
%   Here $\delta_k= w+\epsilon q_{k+1}^*$ for $k\in[0..d-1]$ and 
%   $\delta_d=w$ hence,
%   $\abs{\delta_k}\leq \abs{w}+\abs{\epsilon q_{k+1}^*}$ for $k\in[0..d]$, 
%   we know from the previous step that 
%   $\abs{w}\leq \abs{\epsilon} (d+1)^2\cdot \height(p) (K+1)^d$.
%   Again, we aim to bound the quantity 
%   $$E = \abs{q_0^*-q_0}=\abs{\sum_{k=0}^d\delta_k (r_2^*)^k}
%   \leq \sum_{k=0}^d\abs{\delta_k (r_2^*)^k}\leq 
%   \sum_{k=0}^d \abs{w (r_2^*)^k}+\abs{\epsilon q_{k+1}^* (r_2^*)^k}$$
%   We already know the bound for a sum in the shape of
%   $\sum_{k=0}^d\abs{\epsilon q_{k+1}^* (r_2^*)^k}$ and we also know the magnitude of $w$,
%   hence:
%   \begin{align}\label{boundE}
%     \begin{split}
%     E &\leq \sum_{k=0}^d\abs{w (r_2^*)^k}+\sum_{k=0}^d\abs{\epsilon q_{k+1}^* (r_2^*)^k} 
%     \leq
%     (d+1)\abs{w}K^d+
%     \epsilon (d+1)^2
%     \height(p) K^d\leq\\
%     &\leq\epsilon(d+1)^3 \height(p)  
%     K^{2d} + 
%     \epsilon \cdot (d+1)^2\cdot
%     \height(p)\cdot K^d
%     \leq 2\cdot \epsilon
%     \cdot (d+1)^3\cdot \height(p)\cdot K^{2d}
%     \end{split}
%   \end{align}
%   where $K\coloneq\max\{k_1+1,k_2+1\}$.
%   Finally, assuming that $\abs{\epsilon} d<1$:
%   \begin{align*}
%     &L+4+\log(\height(p))+5d\log\left(K+2\right) \leq M\implies\\
%     &L+\log(2)+\log(\height(p))+3\log(d+1)+2d\log\left(K+2\right) \leq M\implies\\
%     &L+\log\left(2\cdot(d+1)^3\cdot \height(p)\cdot K^{2d}\right) \leq M\implies\\
%     &2^L\cdot\left(2\cdot(d+1)^3\cdot \height(p)\cdot
%     K^{2d}\right) \leq 2^M\implies\\
%     &2^{-M}\cdot\left(2\cdot(d+1)^3\cdot \height(p)\cdot
%     K^{2d}\right) \leq2^{-L}\implies
%     &\text{since }\abs{\epsilon}\leq 2^{-M}\\
%     &\abs{\epsilon} \cdot 2\cdot(d+1)^3\cdot \height(p)\cdot
%     K^{2d}
%     \leq2^{-L}\implies&\text{by the bound in \ref{boundE}} \\
%     &E\leq \frac{1}{2^L}
%   \end{align*}
%   and the statement follows.

% \end{proof} 