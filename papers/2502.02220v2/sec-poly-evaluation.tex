\section{Proof of Theorem~\ref{theorem:result-root-barrier}: classical numbers with polynomial root barriers}\label{sec:poly-evaluation}

In this section, we complete the proof of~\Cref{theorem:result-root-barrier} by
establishing~\Cref{theorem:result-root-barrier}.\ref{theorem:result-root-barrier:point1}
and~\Cref{theorem:result-root-barrier}.\ref{theorem:result-root-barrier:point2}.
Following~\Cref{theorem:general-result-root-barrier}, we discuss natural choices for the base $\cn > 0$ that \textbf{(i)} can be computed with
polynomial-time Turing machines and \textbf{(ii)} have polynomial root barriers. 


\subparagraph*{The case of~$\cn$ algebraic.} 
Let $\cn$ be a fixed algebraic number represented by $(q,\ell,u)$.
The following two results (the first one based on performing a dichotomy search 
to refine the interval~$[\ell,u]$) 
show that one can construct a polynomial-time Turing machine for $\cn$, 
and that~$\cn$ has a polynomial root barrier where the integer~$k$ from~\Cref{theorem:general-result-root-barrier} equals $1$.

\begin{restatable}{lemma}{LemmaApproxAlgebraicBody}
  \label{lemma:approx-algebraic-body}
  Given an algebraic number~$\alg$ represented by $(q,\ell,u)$, 
  one can construct a polynomial-time Turing machine computing $\alg$.
\end{restatable}

\vspace{-7pt}

\begin{restatable}[{\cite[Theorem~A.1]{Bugeaud04}}]{theorem}{TheoremAlgRootBarrier}\label{theorem:alg-root-barrier}
  Let $\alg \in \R$ be a zero
  of a non-zero integer polynomial~$q(x)$,
  and consider a non-constant integer polynomial $p(x)$.
  Then, either ${p(\alg) = 0}$ or 
  ${\ln \abs{p(\alg)} \geq - \deg(q) \cdot \big(\ln(\deg(p)+1) + \ln \height(p)\big)
  - \deg(p) \cdot \big( \ln(\deg(q)+1) + \ln \height(q) \big)}$.
\end{restatable}

By applying~\Cref{theorem:general-result-root-barrier}.\ref{theorem:general-result-root-barrier:point1}, 
\Cref{lemma:approx-algebraic-body} and~\Cref{theorem:alg-root-barrier}, we deduce that the satisfiability problem for $\exists\R(\ipow{\cn})$ is in \twoexptime. However, for algebraic numbers it is possible to obtain a better complexity result (\expspace) by slightly modifying~Steps~II and~III of~\Cref{algo:main-procedure}.

\begin{proof}[Proof of~\Cref{theorem:result-root-barrier}.\ref{theorem:result-root-barrier:point1}]
Let $\phi$ be a formula in input of~\Cref{algo:main-procedure}, and
$\psi(u_1,\dots,u_n)$ to be the formula obtained from $\phi$ after executing
lines~\ref{algo:line1}--\ref{algo:line6}. In lines~\ref{algo:line7}
and~\ref{algo:line8}, guess the integers~$g_1,\dots,g_n$ in binary, instead of
unary. These numbers have at most $m$ bits where,
by~\Cref{theorem:basu,theorem:small-model-property}, $m$ is exponential
in~$\size(\phi)$. Let $g_i = \pm_{i} \sum_{j=0}^{m-1} d_{i,j} 2^j$, with
$d_{i,j} \in \{0,1\}$ and $\pm_i \in \{+1,-1\}$, so that $\cn^{g_i} =
\prod_{j=0}^{m-1}\cn^{\pm_i d_{ij}2^j}$. Note that the formula 
\[ 
  \gamma(x_0,\dots,x_{m-1}) \coloneqq q(x_0) = 0 \land \ell \leq x_0 \leq u \land \textstyle\bigwedge_{i=1}^{m-1} x_i = (x_{i-1})^2
\]
has a unique solution: for every $j \in [0..m-1]$, $x_j$ must be equal to
$\cn^{2^j}$. The formula $\psi$ is therefore equisatisfiable with the formula
$\psi' \coloneqq \psi\sub{x_0}{\cn} \land \gamma \land \bigwedge_{i=1}^n u_i =
\prod_{j=0}^{m-1}x_j^{\pm_id_{ij}}$, which (after rewriting $u_i =
\prod_{j=0}^{m-1}x_j^{\pm_id_{ij}}$ into $u_i \prod_{j=0}^{m-1}x_j^{d_{ij}} = 1$
when $\pm_i = -1$) is a formula from the existential theory of the reals of size
exponential in $\size(\phi)$. Since the
satisfiability problem for the existential theory of the reals is in
\pspace~\cite{Canny88}, we conclude that checking whether $\psi'$ is satisfiable
can be done in~\expspace. Accounting for Steps I and II, we thus obtain a
procedure running in non-deterministic exponential space (because of the guesses
in lines~\ref{algo:line7} and~\ref{algo:line8}), which can be determinised by
Savitch's theorem~\cite{Savitch70}.
%This concludes the proof of~\Cref{theorem:result-root-barrier}.\ref{theorem:result-root-barrier:point1}.
\end{proof}

\subparagraph*{The case of $\cn$ among some classical transcendental numbers (proof sketch of~\Cref{theorem:result-root-barrier}.\ref{theorem:result-root-barrier:point2}).}%
In the context of transcendental numbers, root barriers are usually called
\emph{transcendence measures}. Several fundamental results in number theory
concern deriving a transcendence measure for ``illustrious'' numbers, such as
Euler's $e$, $\pi$, or logarithms of algebraic
numbers~\cite{Popken29,Mahler32,Waldschmidt78}. A few of these results are
summarised in~\Cref{table:transcendence-degrees}, which is taken almost verbatim
from~\cite[Fig.~1 and Corollary~4.2]{Waldschmidt78}. All transcendence
measures in the table are \emph{polynomial} root barriers. Note that in the cases of
$\alg^\eta$ and $\frac{\ln \alg}{\ln \beta}$, the transcendence measures hold
under further assumptions, which are given in the caption of the table.
\begin{table} 
  \begin{center}
  \def\arraystretch{1.15}
    \begin{tabular}{c|l|l}
      Number & \hfill Transcendence measure from~\cite{Waldschmidt78} & \hfill Simplified bound ($\alg,\beta,\eta$ fixed)\\[2pt]
      \hline
      \rule{0pt}{1.1\normalbaselineskip}
      $\pi$ & $2^{40} d (\ln h + d \ln d)(1 + \ln d)$
      & $O(d^2 (\ln d)^2 \ln h)$\\
      $e^\pi$ & $2^{60} d^2 (\ln h + \ln d)(\ln \ln h + \ln d)(1 + \ln d)$
      & $O(d^2 (\ln d)^3 (\ln h) (\ln \ln h))$\\
      $e^\eta$ & $c_\eta \cdot d^2(\ln h + \ln d)\big(\frac{\ln \ln h + \ln d}{\ln \ln h + \ln \max(1,\ln d)}\big)^2$
      & $O(d^2 (\ln d)^3 (\ln h)(\ln \ln h)^2)$\\ 
      $\alg^\eta$ 
      & $c_{\alg,\eta} \cdot d^3 (\ln h + \ln d)\frac{\ln \ln h + \ln d}{(1+\ln d)^2}$
      & $O(d^3 (\ln d)^2(\ln h)(\ln \ln h))$\\
      $\ln \alg$ & $c_\alg \cdot d^2 \frac{\ln h + d \ln d}{1+\ln d}$ & $O(d^3 (\ln d) \ln h)$\\ 
      $\frac{\ln \alg}{\ln \beta}$ 
      & $c_{\alg,\beta} \cdot d^3 \frac{\ln h + d \ln d}{(1+\ln d)^2}$
      & $O(d^4 (\ln d) \ln h)$\\
    \end{tabular}
    \vspace{-4pt}
  \end{center}
\caption{Transcendence measures for some classical real numbers. 
For convenience only, the table assumes $h \geq 16$ (so that $\ln \ln h \geq 1$; replace $h$ by $h+15$ to avoid this assumption).
The numbers $\alg > 0$, $\beta > 0$ and $\eta$ are fixed algebraic numbers, with $\beta \neq 1$.
The integers $c_{\eta}$, $c_{\alg,\eta}$, $c_{\alg}$ and $c_{\alg,\beta}$ are constants that depend on, and can be computed from, polynomials representing $\alg$, $\beta$ and $\eta$.
In the case of $\alg^\eta$, $\eta$ is assumed to be irrational.
In the last line of the table, $\frac{\ln \alpha}{\ln \beta}$ is assumed to be irrational.\vspace{-2pt}}%
\label{table:transcendence-degrees}%
\end{table}

Following~\Cref{theorem:general-result-root-barrier}.\ref{theorem:general-result-root-barrier:point2},
to
prove~\Cref{theorem:result-root-barrier}.\ref{theorem:result-root-barrier:point2}
it suffices to show how to construct a polynomial-time Turing machine for every
number in~\Cref{table:transcendence-degrees}, and derive polynomial root
barriers for the cases $\cn = \alg^{\eta}$ and $\cn = \frac{\ln{\alg}}{\ln
\beta}$ without relying on the additional assumptions in the table. The
following two results solve the first of these two issues.

\begin{theorem}[\cite{Bailey1997OnTR}]\label{lemma:poly-time-pi}
  One can construct a polynomial-time Turing machine computing $\pi$.
\end{theorem}

\vspace{-5pt}

\begin{restatable}{lemma}{LemmaPolyTimeTMExpLog}
  \label{lemma:poly-time-exp-log}
  Given a polynomial-time Turing machine computing $r \in \R$,
  \begin{enumerate}
    \item\label{lemma:poly-time-exp-log:point1} one can construct a polynomial-time Turing machine computing $e^r$;
    \item\label{lemma:poly-time-exp-log:point2} if $r > 0$, one can construct a polynomial-time Turing machine computing $\ln(r)$.
  \end{enumerate}

\end{restatable}

\begin{proof}[Proof idea]
  The two Turing machines use the power series in the
  identities $e^x = \sum_{j=0}^\infty \frac{x^j}{j!}$ and $\ln(x) = 2
  \sum_{j=0}^\infty \big(\frac{1}{2j+1} \big(\frac{x-1}{x+1}\big)^{2j+1}\big)$,
  truncated to obtain the required accuracy. 
\end{proof}

As an example, to construct the Turing machine for~$\frac{\ln(\alg)}{\ln(\beta)}$ 
we construct machines for the following sequence of numbers: $\alg$ and $\beta$ (applying~\Cref{lemma:approx-algebraic-body}), $\ln(\alg)$ and $\ln(\beta)$ (\Cref{lemma:poly-time-exp-log}.\ref{lemma:poly-time-exp-log:point2}), $\frac{1}{\ln(\beta)}$ (\Cref{lemma:turing-machine-reciprocal})
and $\frac{1}{\ln(\beta)} \cdot \ln(\alg)$ (\Cref{lemma:turing-machine-products}). 
For $\alg^\eta$, we follow the operations in $e^{\eta \cdot \ln(\alg)}$.


Let us now discuss how to derive polynomial root barriers when $\cn = \alg^\eta$
or $\cn = \frac{\ln(\alg)}{\ln(\beta)}$. In~the~case~$\cn = \alg^\eta$,
\Cref{table:transcendence-degrees} assumes~$\eta$ to be irrational. To check
whether an algebraic number represented by~$(q,\ell,u)$ is rational, it suffices
to factor $q(x)$ into a product of irreducible polynomials with rational
coefficients, and test for any degree $1$ factor $n \cdot x - m$ whether the
rational number $\frac{m}{n}$ belongs to $[\ell,u]$. The factorisation of $q$
can be computed (in fact, in polynomial time) using LLL~\cite{lenstra1982}. If
such a rational number does not exist, then $\eta$ is irrational and
the polynomial root barrier for $\alg^\eta$ is given
in~\Cref{table:transcendence-degrees}. Otherwise, $\eta = \frac{m}{n}$ and the
number $\alg^{\frac{m}{n}}$ is algebraic. In this case, rely on the following
lemma to construct a representation of $\alg^{\frac{m}{n}}$, and then derive a
polynomial root barrier by applying~\Cref{theorem:alg-root-barrier}.

\begin{restatable}{lemma}{LemmaRepresentationPowerOfAlgebraic}
  \label{lemma:representation-power-of-algebraic}
  There is an algorithm that given a rational $r$ and an algebraic number $\alg > 0$ 
  represented by $(q,\ell,u)$, computes a representation $(q',\ell',u')$ 
  of the algebraic number $\alg^r$.
\end{restatable}

We move to the case $\cn = \frac{\ln(\alg)}{\ln(\beta)}$,
which~\Cref{table:transcendence-degrees} assumes to be irrational. Since $\cn$
is positive, $\alg,\beta \not\in \{0,1\}$. We observe that for every
$\frac{m}{n} \in \Q$, we have $\cn = \frac{m}{n}$ if
and only if~${\alg^n \beta^{-m} = 1}$. (In other words,
$\frac{\ln(\alg)}{\ln(\beta)} \in \Q$ if and only if $\alg$ and $\beta$ are
multiplicatively dependent.) From a celebrated result of Masser~\cite{Masser88},
the set $\{(m,n) \in \Z^2 : \alg^n \beta^{-m} = 1\}$ is a
finitely-generated integer lattice for which we can explicitly construct a
basis~$K$ (see~\cite{CaiLZ00} for a polynomial-time procedure). If $K =
\{(0,0)\}$, then $\cn$ is irrational and its polynomial
root barrier is given in~\Cref{table:transcendence-degrees}. Otherwise, since
$\alg,\beta \not \in \{0,1\}$, there is~$(m,n) \in K$ with $n \neq 0$, and
$\cn = \frac{m}{n}$. We can then derive a polynomial
root barrier by applying~\Cref{theorem:alg-root-barrier}.

% \begin{theorem}[\cite{CaiLZ00}]\label{theorem:multiplicative-idependence}
%   There is an algorithm that given as input algebraic numbers $\alg_1,\dots,\alg_n$, 
%   returns a basis for the lattice $\{(d_1,\dots,d_n) \in \Z^n : \alg_1^{d_1} \cdot \ldots \cdot \alg_n^{d_n} = 1\}$.
% \end{theorem}




%%%%%%%%%%%%%%%%%%%%%%%%%%%%%%%%%%%%%%%%%%%
%%%%%% OLD MATERIAL COMMENTED BELOW 
%%%%%%%%%%%%%%%%%%%%%%%%%%%%%%%%%%%%%%%%%%%
% \subparagraph*{The case of $\cn=\alg^\beta$.}
% Let $\cn$ be $\alpha^\beta$ where both $\alpha$ and $\beta$ are algebraic. 
% %Observe that in \Cref{table:transcendence-degrees}, $\alpha\neq 0$, $1$, and 
% %$\beta$ is assumed to be irrational in this case. This is because 
% %$\alpha^\beta$ is algebraic if and only if $\beta$ is rational.
% Excluding the trivial cases where $\alg\in\{0,1\}$ or $\beta=0$,
% there are two subcases depending on the rationality of $\beta$. 
% This can always be checked from the representation of $\beta$ as 
% an algebraic number thanks to the next result.


% %\am{If you want, I have used a trick in~\Cref{sec:application-entropic-risk} that might help here. Briefly, when dealing with a term $x^{\frac{p}{q}}$, you can introduce a variable~$y$, write $y^q = x$ and replace $x^{\frac{p}{q}}$ with $y^p$.}

% %\jg{text below to be moved to the appendix in case this method is used}
% %Consider $\beta=b_1/b_2$ a rational number and
% %$\alpha$ the only root of the polynomial $p(x)=\sum_{i=0}^da_ix^i$ in 
% %the interval $(l,u)$. The following is a way to compute a polynomial 
% %that has $\alpha^\beta$ as a root.
% %The fact that $\alpha$ is a root of a polynomial of degree $d$, implies
% %that every power of $\alpha$ can be written as a linear combination of 
% %terms $r\cdot \alpha^j$, where $r$ is a rational number and $j\in[0..d-1]$.
% %
% %Therefore we can build a matrix of $d$ columns where in each row
% %we have the coefficients of the rational linear combinations equal to 
% %$(\alpha^b)^k$ with $k\in[0..d]$.

% The following result is useful when $\beta$ is irrational.


% If $\beta$ is irrational then $\cn$ has the root barrier from 
% \Cref{table:transcendence-degrees} corresponding to $\alg^\beta$. 
% Let us show that $\cn$ is computable by a polynomial-time Turing machine
% so that we can apply \Cref{theorem:general-result-root-barrier}.

% We write $\alg^\beta$ as $e^{\ln(\alg)\beta}$. By \Cref{lemma:approx-algebraic}
% we can construct two different polynomial-time Turing machines that compute $\alg$
% and $\beta$, by \Cref{lemma:turing-machine-products} we can also construct
% a polynomial-time Turing machine that computes $\ln(\alg)\cdot\beta$ and 
% finally applying \Cref{lemma:poly-time-exp-log} we deduce that we can construct
% a polynomial-time Turing machine that computes $e^{\ln(\alg)\beta}=\alg^\beta$. 

% Therefore considering the polynomial
% root barrier from \Cref{table:transcendence-degrees} related to $\alg^\beta$ 
% we can apply \Cref{theorem:general-result-root-barrier}.\ref{theorem:general-result-root-barrier:point2},
% \Cref{lemma:approx-algebraic}, \Cref{lemma:poly-time-exp-log} and 
% \Cref{lemma:turing-machine-products} in the way described to 
% deduce that the satisfiability problem for $\exists\R(\ipow{\cn})$ is in \threeexptime.


% \subparagraph*{The case of $\cn=\frac{\ln(\alg)}{\ln(\beta)}$.} 
% Let $\cn$ be $\frac{\ln(\alg)}{\ln(\beta)}$ where both $\alg$ and $\beta$
% are algebraic, non-zero and different from 1. In order to apply the root barrier from \Cref{table:transcendence-degrees}
% we have to check rationality of $\cn$. 
% This can be done studying the multiplicative dependence of 
% $\alpha$ and $\beta$, that is, whether they satisfy $\alpha^n=\beta^m$ 
% for some non-zero integers $n$, $m$. The following lemma allows this.


% Therefore applying \Cref{theorem:multiplicative-idependence} and 
% \Cref{lemma:mult-independence-rationality} we can decide the rationality 
% of $\cn$. 

% If $\cn$ is rational,
% algorithm from \Cref{theorem:multiplicative-idependence} 
% returns some integers $n$ and $m$ such that $\alpha^n=\beta^m$,hence 
% $\cn=\frac{m}{n}$
% and we can proceed as in the case 
% of $\cn$ algebraic. A representation 
% of $\cn$ as an algebraic number is given by $q(x)=n\cdot x-m$, 
% $l=\frac{m}{n}-1$ and $u=\frac{m}{n}+1$. 

% If on the other hand $\cn$ is irrational we proceed as follows. Since $\cn>0$, 
% in this case this means that either both $\alpha$, $\beta>1$ or both 
% $\alpha$, $\beta\in(0,1)$. When one of these conditions holds, something
% that can be checked in constant time, 
% $\cn=\frac{\ln(\alg)}{\ln(\beta)}=\frac{\abs{\ln(\alg)}}{\abs{\ln(\beta)}}$.
% From the previous subsections we know that $\ln(\alg)$ and $\ln(\beta)$
% can be computed by polynomial-time Turing machines and 
% since we know their sign, also can 
% $\abs{\ln(\alg)}$ and $\abs{\ln(\beta)}$. 
% From \Cref{lemma:turing-machine-reciprocal} we deduce
% that $\frac{1}{\abs{\ln(\beta)}}$ can also be computed by a polynomial-time
% Turing machine and finally by \Cref{lemma:turing-machine-products} the 
% same holds for $\abs{\ln(\alg)}\cdot \frac{1}{\abs{\ln(\beta)}}$.
% From \Cref{table:transcendence-degrees} we have a polynomial root 
% barrier for $\cn$.

% Therefore we can apply 
% \Cref{theorem:general-result-root-barrier}.\ref{theorem:general-result-root-barrier:point2} 
% to deduce that the satisfiability problem 
% for $\exists\R(\ipow{\cn})$ is in \threeexptime.


% \am{Shall we take the exact bounds from \cite{CaiLZ00}, and give a result for the root barrier of $\frac{\ln(\alg)}{\ln(\beta)}$ that does not ask for irrationality?}