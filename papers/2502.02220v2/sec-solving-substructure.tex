\section{Finding solutions over integer powers of \texorpdfstring{$\xi$}{the base}}
\label{sec:solving-substructure}

In this section we give a sketch of the proof of
\Cref{theorem:small-model-property}, i.e., we show that $\exists
\ipow{\cn}$ has a~\emph{small witness property}. The proof is split into two parts:
\begin{enumerate}
  \item We first give a quantifier-elimination-like procedure for $\exists
  \ipow{\cn}$. Instead of targeting formula equivalence, we only focus on
  equisatisfiability: given a formula $\exists y\, \phi(y,\vec{x})$, with $\phi$
  quantifier-free, the procedure derives an \emph{equisatisfiable}
  quantifier-free formula $\psi(\vec x)$. Preserving
  equisatisfiability, instead of equivalence, is advantageous complexity-wise.
  (Our procedure
  preserves equivalence for sentences, as
  these are equivalent to~$\top$~or~$\bot$.)
  \item By analysing our quantifier elimination procedure, we derive the bounds
  on the set $P_{\psi}$ from~\Cref{theorem:small-model-property} that are required to
  complete the proof. This step is similar to the \emph{quantifier
  relativisation} technique for Presburger arithmetic (see, e.g.,~\cite[Theorem
  2.2]{Weispfenning90}).
\end{enumerate}
Some of the core mechanisms of our quantifier-elimination-like procedure follow
observations done by Avigad and Yin for their (equivalence-preserving) quantifier elimination
procedure~\cite{AvigadY07}. Apart from targeting equisatisfiability, a key property of our procedure is that it does not require
appealing to a quantifier elimination procedure for the theory of the reals. The
procedure in~\cite{AvigadY07} calls such a procedure once for each eliminated
variable instead.

\subsection{Quantifier elimination}
\label{subsection:quantifier-elimination}
Fix a real number $\cn > 1$. In this section, we rely on some auxiliary notation
and definitions:
\begin{itemize}
  \item We often see an integer polynomial $p(\cn, \vec x)$ as a polynomial in
  variables $\vec x = (x_1,\dots,x_m)$ having as coefficients univariate integer
  polynomials on $\cn$, i.e., ${p(\cn, \vec x) = \sum_{i=1}^n q_i(\cn) \cdot
  \vec{x}^{\vec{d}_i}}$, where the notation $\vec{x}^{\vec{d}_i}$ is short for
  the \emph{monomial} $\prod_{j=1}^m x_j^{d_{i,j}}$, with $\vec{d}_{i} =
  (d_{i,1},\dots,d_{i,m})$.
  \item We sometimes write polynomial (in)equalities using Laurent polynomials,
  i.e., polynomials with negative powers. For instance, \Cref{lemma:lambda-close-to-variable-body} below features equalities with monomials
  $\cn^g \cdot {\vec{x}}^{\vec{d}_i}$ where $g$ may be a negative integer. Laurent polynomials are just a
  shortcut for us, as one can opportunely manipulate the (in)equalities to make
  all powers non-negative (as we did
  in~\Cref{subsection:polynomial-sign-evaluation}): a polynomial
  (in)equality $p(\cn,x_1,\dots,x_m) \sim 0$ is rewritten as
  $p(\cn,x_1,\dots,x_m) \cdot \cn^{-d} \cdot x_1^{-d_1} \cdot {\dots} \cdot
  x_m^{-d_m} \sim 0$, where $d_i$ (resp.~$d$) is the smallest negative integer
  occurring as a power of $x_i$ (resp.~$\cn$) in $p$ (or $0$ if such a negative
  integer does not exist). Observe that this transformation does not change the
  number of monomials nor the height of the polynomial $p$, but it may double the
  degree of each variable and of $\cn$.
  \item Given a formula $\phi$, a variable $x$ and a Laurent polynomial
  $q(\vec{y})$, we write $\phi\sub{q(\vec y)}{x}$ for the formula obtained from
  $\phi$ by replacing every occurrence of $x$ by $q(\vec y)$, and then
  updating all polynomial (in)equalities with negative degrees in the way
  described above.
  \item We write $\lambda \colon \R_{> 0} \to \ipow{\cn}$ for the function
  mapping $a \in \R_{> 0}$ to the largest integer power of $\cn$ that is less or
  equal than $a$, i.e., $\lambda(a)$ is the only element of $\ipow{\cn}$
  satisfying $\lambda(a) \leq a < \cn \cdot \lambda(a)$. 
\end{itemize}

The relation $\lambda(p(\cn,\vec x)) = y$, where $p$ is an integer polynomial,
is definable in $\exists \ipow{\cn}$ as $p(\cn,\vec x) > 0 \land y \leq
p(\cn,\vec x) < \cn \cdot y$. To obtain a quantifier elimination procedure, we
must first understand what values can $y$ take given~$p(\cn,\vec x)$. The next
lemma answers this question. 

\begin{restatable}{lemma}{LemmaLambdaCloseToVariableBody}
  \label{lemma:lambda-close-to-variable-body}
  Let $p(\cn,\vec x) \coloneqq \sum_{i = 1}^n (q_i(\cn) \cdot
  {\vec{x}}^{\vec{d}_i})$, where each~$q_i$ is a univariate integer polynomial.
  In the theory~$\exists \ipow{\cn}$, the formula $p(\cn,\vec x) > 0$ entails the formula $\textstyle\bigvee_{i=1}^n \bigvee_{g \in G} \lambda(p(\cn,\vec x)) = \cn^g \cdot {\vec{x}}^{\vec{d}_i}$\,, for some finite set $G \subseteq \Z$.
  Moreover:
  \begin{enumerate}[I.]
    \item\label{lemma:lambda-close-to-variable-body:i1} If $\cn$ is a computable transcendental number, there is an
          algorithm computing $G$ from $p$.
    \item\label{lemma:lambda-close-to-variable-body:i2} If $\cn$ has a root barrier $\sigma(d,h) \coloneqq c \cdot
            (d+\ceil{\ln(h)})^k$, for some $c,k \in \N_{\geq 1}$, then
            \vspace{-3pt}
            \begin{equation*}
              G\coloneqq\left[-L..L\right],
              \qquad 
              \text{where }
              L \coloneqq \left(2^{3c}D\ceil{\ln(H)}\right)^{6nk^{3n}},
            \end{equation*}
            with $H \coloneqq \max\{8,\height(q_i) : i \in [1,n]\}$, and $D \coloneqq \max\{\deg(q_i)+2 : i \in [1,n]\}$.
  \end{enumerate}

\end{restatable}

\begin{proof}[Proof sketch]
  A suitable set $G$ can be found as follows. Let $\mathcal{Q}$ be the set of
  all univariate integer polynomials~$Q(z)$ for which there are $j \leq \ell \in
  [1..n]$, numbers $g_j,\dots,g_{\ell-1} \in \N$, and integer polynomials
  $Q_j(z),\dots,Q_{\ell}(z)$ such that $Q_\ell = Q$ and
    \begin{enumerate}
      \item\label{mcQ-polynomials-body} the polynomials $Q_j,\dots,Q_\ell$
      are recursively defined as 
          \begin{align*}
          Q_j(z) & \coloneqq q_j(z),                                  \\
          Q_r(z) & \coloneqq Q_{r-1}(z) \cdot z^{g_{r-1}} + q_{r}(z),
              & \text{for every } r \in [j+1,\ell],
          \end{align*}
      \item
      the real numbers $Q_j(\cn),\dots,Q_{\ell-1}(\cn)$ are all non-zero, and
      $Q_\ell(\cn)$ is (strictly) positive,
      \item\label{mcQ-finiteness-of-gr-body} for every $r \in [j..\ell-1]$, 
      the number $\cn^{g_r}$ belongs to the interval
      $\big[1\,,\,\frac{\abs{q_{r+1}(\cn)}+\dots+\abs{q_n(\cn)}}{\abs{Q_r(\cn)}}\big]$.
    \end{enumerate}
  Items~\ref{mcQ-polynomials-body}--\ref{mcQ-finiteness-of-gr-body} ensure the
  set $\mathcal{Q}$ to be finite. We define the (finite) set
  \[ 
    B \coloneqq \big\{ \beta \in \Z : \text{there is } Q \in \mathcal{Q} \text{ such that } \cn^{\beta} \in \big\{{\textstyle\lambda(Q(\cn)), \frac{\lambda(Q(\cn) \cdot (\cn-1))}{\cn}, \frac{\lambda(Q(\cn) \cdot (\cn+1))}{\cn}}\big\}
    \big\}.
  \]
  By induction on $n$, one can prove that any finite set $G$ that includes
  $[\min B.. \max B]$ respects the property in the first statement of the lemma.
  To prove the remaining statements of the lemma 
  (Items~\eqref{lemma:lambda-close-to-variable-body:i1} 
  and~\eqref{lemma:lambda-close-to-variable-body:i2}) 
  one shows how to effectively compute an overapproximation
  of the set $B$. In the case of $\cn$ having a polynomial root barrier, 
  this overapproximation is obtained by bounding the values of
  $\textstyle\lambda(Q(\cn))$, $\frac{\lambda(Q(\cn) \cdot (\cn-1))}{\cn}$, and
  $\frac{\lambda(Q(\cn) \cdot (\cn+1))}{\cn}$, for every $Q \in \mathcal{Q}$.
  See~\Cref{appendix:solving-substructure} for the complete proof.
\end{proof}


\begin{figure}
  \begin{tikzpicture}[scale=4,minimum size=2pt]
    \node[label=-90:$\frac{1}{\cn^3}$] (m3) at (0.295,0) {\textbf{|}};
    \node[label=-90:$\frac{1}{\cn^2}$] (m2) at (0.44,0) {\textbf{|}};
    \node[label=-90:$\frac{1}{\cn}$] (m1) at (0.66,0) {\textbf{|}};
    \node[label=-90:{\footnotesize$1$}] (o) at (1,0) {\textbf{|}};
    \node[label=-90:{\footnotesize$\cn$}] (p1) at (1.5,0) {\textbf{|}};
    \node[label=-90:{\footnotesize$\cn^2$}] (p2) at (2.25,0) {\textbf{|}};
    \node[label=-90:{\footnotesize$\cn^3$}] (p3) at (3.37,0) {\textbf{|}};

    \node (l) at (0.13,0) {}; \node (r) at (3.57,0) {}; \draw[->] (l) -- (r);

    \fill[draw=black,thick,pattern=north east lines] (0.82,-0.023) rectangle
    (1.8,0.023); \fill[thick,pattern=north east lines] (3,-0.023) rectangle
    (3.5,0.023); \draw[thick] (3.46,0.023) -- (3,0.023) -- (3,-0.023) --
    (3.46,-0.023);

    \node (z1) at (1.8,0.2) {\scalebox{0.8}{$q(u^*) = 0$}}; 
    \node (z2) at (3,0.2) {\scalebox{0.8}{$p(w^*) = 0$}}; 
    \draw[dotted,thick] (z1) -- (1.8,-0.25); 
    \draw[dotted,thick] (z2) -- (3,-0.25);

    \draw (1.8,-0.14) edge[->] node[below]{\footnotesize{$\lambda(u^*)$}}
    (1.54,-0.14); \draw (3,-0.14) edge[->] node[below]{\footnotesize{$\cn \cdot
    \lambda(w^*)$}} (3.32,-0.14);

    \node at (1.5,-0.32) {};
  \end{tikzpicture}
  \caption{High-level idea of the quantifier elimination procedure. Dashed
  rectangles are intervals corresponding to the set of solutions over $\R$ of a
  (univariate) formula~$\phi$. To search for a solution over~$\ipow{\cn}$, it
  suffices to look for elements of $\ipow{\cn}$ that are close to the endpoints
  of these intervals. At each endpoint, a polynomial in $\phi$ must evaluate to
  zero (since around endpoints the truth of $\phi$ changes), so it suffices to
  look for integer powers of $\cn$ that are close to roots or polynomials in
  $\phi$.}
  \label{figure:qe-idea}

\end{figure}

We now give the high-level idea of the quantifier elimination procedure, which
is also depicted in~\Cref{figure:qe-idea}. Let $\psi(u,\vec y)$ be a
quantifier-free formula of $\exists\ipow{\cn}$, and $u$ be the variable we
want to eliminate. Suppose to evaluate the variables $\vec y$ with elements in
$\ipow{\cn}$, hence obtaining a univariate formula $\phi(u)$. The set of all
solutions \emph{over the reals} of $\phi(u)$ can be decomposed into a finite set
of disjoint intervals. (This follows from the o-minimality of the FO
theory of the reals~\cite[Chapter~3.3]{marker2002model}.) \Cref{figure:qe-idea}
shows these intervals as dashed rectangles. Around the endpoints of these
intervals the truth of $\phi$ changes, and therefore for each such
endpoint~$u^*$ there must be a non-constant polynomial in $\phi$ such that
$q(u^*) = 0$. If an interval with endpoint $u^* \in \R_{>0}$ contains an element
of $\ipow{\cn}$, then it contains one that is ``close'' to $u^*$:
\begin{itemize} 
  \item If $u^* \in \R_{>0}$ is the \emph{right endpoint} of an interval, at
  least one among $\lambda(u^*)$ and $\cn^{-1} \cdot \lambda(u^*)$ belongs to
  the interval. The first case is depicted in \Cref{figure:qe-idea}. The latter
  case occurs when~$u^*$ belongs to $\ipow{\cn}$ but not to the interval.
  \item If $u^*$ is the \emph{left endpoint} of an interval, then $\cn
  \cdot \lambda(u^*)$ of $\lambda(u^*)$ belongs to the interval.
  The latter case occurs when $u^*$ belongs to $\ipow{\cn}$ and also to the interval.
\end{itemize}
Note that we have restricted the endpoint $u^*$ to be
positive, so that $\lambda(u^*)$ is well-defined. The only case were we may not
find such an endpoint is when $\phi(u)$ is true for every $u > 0$.
But finding an element of $\ipow{\cn}$ is in this case simple: we
can just pick~$1 \in \ipow{\cn}$. Since $u^*$ is positive, we can split it into $x^* \cdot v^*$
with $x^* \in \ipow{\cn}$ and $1 \leq v^* < \cn$ (so, $\lambda(u^*) = x^*$). To
obtain quantifier elimination, our goal is then to characterise, symbolically as
a finite set of polynomials~$\tau(\vec{y})$, the set of all possible values for
$x^*$. In this way, we will be able to eliminate the variable~$u$ by considering the
polynomials $\cn^{-1} \cdot \tau(\vec y)$, $\tau(\vec y)$ and $\cn \cdot \tau(\vec{y})$ representing the
integer powers of $\cn$ that are ``close'' to endpoints. The following lemma
provides the required characterisation.

\begin{restatable}{lemma}{LemmaLastLambdaLemma}
  \label{lemma:last-lambda-lemma}
  Let $r(x,v,\vec y) \coloneqq \sum_{i=0}^n p_i(\cn,\vec y) \cdot (x \cdot v)^i$, where 
  each~$p_i$ is an integer polynomial,
  $M$ be the set of monomials $\vec y^{\vec \ell}$ occurring in some $p_i$, 
  and ${N \coloneqq \{\vec y^{\vec \ell_1 - \vec \ell_2} : \vec y^{\vec \ell_1},\vec y^{\vec \ell_2} \in M\}}$.~Then,%
  \vspace{-3pt}
  \begin{align*}
      \hspace{-10pt}
      \ipow{\cn}(x) 
      \land 1 \leq v < \cn 
      \land r(x,v,\vec y)=0 
      \land \big(\!\bigvee_{i=0}^n p_i(\cn,\vec y) \neq 0\big)
      \land \bigwedge_{\mathclap{y \textup{\,from\,} \vec y}} \ipow{\cn}(y)
      \ \models\ 
      \hspace{-2pt}
      \bigvee_{(j,g,\vec y^{\vec \ell}) \in F}
      \hspace{-5pt}
      x^j = \cn^{g} \cdot \vec y^{\vec \ell}
  \end{align*}
  holds (in the theory~$\exists\R(\ipow{\cn})$) for some finite set $F \subseteq [1..n] \times \Z \times N$.
  Moreover: 
  \begin{enumerate}[I.]
    \item\label{lemma:last-lambda-lemma:i1} If $\cn$ is a computable transcendental number, there is an algorithm computing $F$ from $r$. 
    \item\label{lemma:last-lambda-lemma:i2} If $\cn$ has a root barrier $\sigma(d,h) \coloneqq c \cdot
            (d+\ceil{\ln(h)})^k$, for some $c,k \in \N_{\geq 1}$, then,
            \begin{equation*}
              F \coloneqq [1..n] \times \left[-L..L\right] \times N,
              \qquad\text{where }
              L \coloneqq n\left(2^{4c}D\ceil{\ln(H)}\right)^{6\abs{M} \cdot k^{3\abs{M}}},
            \end{equation*}
              with $H \coloneqq \max\{8,\height(p_i) : i \in [1,n]\}$, 
              and
              $D \coloneqq \max\{\deg(\cn,p_i)+2 : i \in [0,n]\}$.
  \end{enumerate}

\end{restatable}

\begin{proof}[Proof sketch]
  By following the arguments in~\cite[Lemma 3.9]{AvigadY07}, one shows that the
  premise of the entailment in the statement entails a disjunction over
  formulae of the form 
  \begin{align*} 
    & x^{k-j} = \textstyle\frac{\cn^s \cdot \lambda(\pm p_j(\cn,\vec y))}
    {\lambda(\mp p_k(\cn,\vec y))} \land \pm p_j(\cn,\vec y) > 0 \land \mp p_k(\cn,\vec y) > 0,
  \end{align*}
  where $0\leq j < k \leq n$, $s \in [-g..g]$ with $g \coloneqq
  1+\ceil{\log_{\cn}(n)}$, and $m\leq n^2\cdot \left(2 \cdot
  \ceil{\log_{\cn}(n)} + 3\right)$. Afterwards, we rely
  on~\Cref{lemma:lambda-close-to-variable-body} to remove the occurrences of
  $\lambda$ from the above formulae, establishing in this way the first
  statement of the lemma. Items~\eqref{lemma:last-lambda-lemma:i1}
  and~\eqref{lemma:last-lambda-lemma:i2} follow from the analogous items
  in~\Cref{lemma:lambda-close-to-variable-body}. To achieve the bounds in
  Item~\eqref{lemma:last-lambda-lemma:i2} we also rely on the fact that
  $\ceil{\log_{\cn}(n)} \leq 2^{2c} \ceil{\ln(n)}$. This follows from a simple
  computation, noticing that since $\cn$ is not a root of the polynomial $x -
  1$, by the definition of root barrier we have $\cn > 1 + \frac{1}{e^c}$.
\end{proof}

By relying on the characterisation, given in~\Cref{lemma:last-lambda-lemma}, of
the values that $\lambda(u^*)$ can take, where $u^* > 0$ is the root of some
polynomial, and by applying our previous observation that satisfiability can be
witnessed by picking elements of $\ipow{\cn}$ that are ``close'' to $u^*$ (i.e.,
the numbers $\cn^{-1} \cdot \lambda(u^*)$, $\lambda(u^*)$ or $\cn \cdot
\lambda(u^*)$), we obtain the following key lemma.

\begin{restatable}{lemma}{LemmaRelativiseQuantifiers}
  \label{lemma:relativise-quantifiers}
  Let  $\phi(u,\vec y)$ be a quantifier-free formula from $\exists \ipow{\cn}$.
  Then, $\exists u\,\phi$ is equivalent to
  \begin{equation}
    \tag{$\dagger$}
    \label{eq:lemma14}
    \bigvee_{\ell \in [-1..1]}\ 
    \bigvee_{q \in Q}\ 
    \bigvee_{(j,g,\vec y^{\vec \ell}) \in F_q}
    \exists u : u^{j} = \cn^{j \cdot \ell + g} \cdot \vec y^{\vec \ell} \land \phi
  \end{equation}
  where $Q$ is the set of all polynomials in $\phi$ featuring $u$, 
  plus the polynomial $u-1$,
  and each~$F_q$ is the set obtained by applying~\Cref{lemma:last-lambda-lemma}
  to $r(x,v,\vec y) \coloneqq q\sub{x \cdot v}{u}$, with $x$ and $v$ 
  fresh~variables.
\end{restatable}

To eliminate the variable $u$, we now 
consider each disjunct $\exists u \left( u^{j} = \cn^{k} \cdot \vec{y}^{\vec \ell}
\land \phi \right)$ from Formula~\eqref{eq:lemma14} and, roughly speaking, 
substitute $u$ with $\sqrt[j]{\cn^{k} \cdot \vec{y}^{\vec{\ell}}}$. 
We do not need however to introduce~$j$th roots, as shown in the following lemma. 


\begin{restatable}{lemma}{LemmaRemoveU}
  \label{lemma:remove-u}
  Let $\phi(u,\vec y)$ be a quantifier-free formula from $\exists \ipow{\cn}$,
  with $\vec y = (y_1,\dots,y_n)$. Let $j \in \N_{\geq 1}$, $k \in \Z$ and
  $\vec{\ell} \coloneqq (\ell_1,\dots,\ell_n) \in \Z$. Then,
  $\exists \vec{y} \exists u : u^{j} = \cn^{k} \cdot \vec y^{\vec \ell} \land
  \phi$ is equivalent to
  \begin{equation*}
    \textstyle\bigvee_{\vec r \coloneqq (r_1,\dots,r_n) \in R}\
    \exists \vec z : \phi\sub{z_i^j \cdot \cn^{r_i}}{y_i : i \in [1..n]}\sub{\cn^{\frac{k + \vec \ell \cdot \vec r}{j}} \cdot \vec z^{\vec \ell}}{u},
  \end{equation*}
  where $R \coloneqq \big\{(r_1,\dots,r_n)\in[0..j-1]^n : j \text{ divides } k +
  \sum_{i=1}^n r_i \cdot \ell_i \big\}$, $\vec \ell \cdot \vec r \coloneqq
  \sum_{i=1}^n r_i \cdot \ell_i$, and $\vec z \coloneqq (z_1,\dots,z_n)$ is a
  vector of fresh variables.
\end{restatable}

\begin{proof}[Proof sketch]
  Consider a solution to the equality $u^{j} = \cn^{k} \cdot
  \vec{y}^{\vec{\ell}}$. Each $y_i$ evaluates to a number of the form $\cn^{q_i
  \cdot j + r_i}$, with $q_i \in \Z$ and $r_i \in [0..j-1]$. Since $u^{j}$ is of
  the form~$\cn^{j \cdot q}$ for some $q \in \Z$, we must have that $k +
  \sum_{i=1}^n r_i \cdot \ell_i$ is divisible by $j$. Observe that the set $R$
  in the statement of the lemma contains all possible vectors $\vec r =
  (r_1,\dots,r_n)$ satisfying this divisibility condition.
  
  At the formula level, consider a vector $\vec r = (r_1,\dots,r_n) \in R$,
  and replace in ${u^{j} = \cn^{k} \cdot \vec{y}^{\vec{\ell}} \land \phi}$
  every variable $y_i$ with the term $z_i^{j} \cdot
  \cn^{r_i}$. After this replacement, the equality $u^{j} = \cn^{k} \cdot \vec{y}^{\vec{\ell}}$ can be rewritten as 
  $u = \cn^{\frac{k + \vec \ell \cdot \vec r}{j}} \cdot \vec z^{\vec \ell}$, 
  where the division is without remainder. 
  We can therefore substitute $u$ with $\cn^{\frac{k + \vec \ell \cdot \vec r}{j}} \cdot \vec z^{\vec \ell}$ in $\phi$, eliminating it.
\end{proof}

By chaining~\Cref{lemma:relativise-quantifiers,lemma:remove-u}, one can eliminate all variables from a quantifier-free formula $\phi(\vec x)$, obtaining an equisatisfiable
formula with no variables.

\subsection{Quantifier relativisation}
\label{subsection:quantifier-relativisation}

Looking closely at how a quantifier-free formula $\phi(u_1,\dots,u_n)$ of $\exists\ipow{\cn}$ evolves as we chain~\Cref{lemma:relativise-quantifiers,lemma:remove-u} to eliminate all variables, we see that the resulting variable-free formula is a finite disjunction~$\bigvee_{i} \psi_i$ of formulae $\psi_i$ that are obtained from $\phi$ via a sequence of substitutions stemming from~\Cref{lemma:remove-u}. As an example, for a formula in three variables~$\phi(u_1,u_2,u_3)$, each $\psi_i$ is obtained by applying a sequence of substitutions of the form:
\begin{center}
  \vspace{-3pt}
  \begin{tikzpicture}
    \node[align=center] (l1) at (0,0) {
        \footnotesize{elimination of $u_1$}%\\ 
        %\footnotesize{(the fresh variables $z_1,z_2$}\\[-1pt]
        %\footnotesize{replace $u_2,u_3$)}
      };
    \node[align=center] (l2) at (4,0) {
        \footnotesize{elimination of $z_1$}%\\
        %\footnotesize{(the fresh variable $z_3$}\\[-1pt]
        %\footnotesize{replaces $z_2$)}
      };
    \node[align=center] (l3) at (8,0) {
        \footnotesize{elimination of $z_3$}%\\ 
        %~\\[-1pt]
        %~ 
      };

    \node 
      (s1) [above = 0.05cm of l1] 
      {$\begin{cases}
        u_1 = \cn^{k_1} \cdot z_1^{\ell_1} \cdot z_2^{\ell_2}\\ 
        u_2 = z_1^{j_1} \cdot \cn^{r_1}\\
        u_3 = z_2^{j_1} \cdot \cn^{r_2}
      \end{cases}$};

    \node 
      (s2) [above = 0.05cm of l2] 
      {$\begin{cases}
        z_1 = \cn^{k_2} \cdot z_3^{\ell_3}\\ 
        z_2 = z_3^{j_2} \cdot \cn^{r_3}
      \end{cases}$};

    \node 
      (s3) [above = 0.05cm of l3] 
      {$\begin{cases} 
        z_3 = \cn^{k_3}
      \end{cases}$};

    \node (t2) [left = 1cm of s2] {};
    \draw[->] (t2) -- (s2);

    \node (t3) [left = 1cm of s3] {};
    \draw[->] (t3) -- (s3);

  \end{tikzpicture}
  \vspace{-5pt}
\end{center}
We can ``backpropagate'' these substitutions to the initial variables
$u_1,\dots,u_n$, associating to each one of them an integer power of $\cn$. In the
above example, we obtain the system
\begin{center}
  $\begin{cases} u_1 = \cn^{k_1} \cdot (\cn^{k_2} \cdot
        (\cn^{k_3})^{\ell_3})^{\ell_1} \cdot ((\cn^{k_3})^{j_2} \cdot
        \cn^{r_3})^{\ell_2}\\ 
        u_2 = (\cn^{k_2} \cdot (\cn^{k_3})^{\ell_3})^{j_1} \cdot \cn^{r_1}\\
        u_3 = ((\cn^{k_3})^{j_2} \cdot \cn^{r_3})^{j_1} \cdot \cn^{r_2}
      \end{cases}$
\end{center}
By~\Cref{lemma:last-lambda-lemma,lemma:relativise-quantifiers,lemma:remove-u}, we can restrict the integers occurring 
as powers of $\cn$ in the resulting system of substitutions to a finite set.
Since the disjunction $\bigvee_i \psi_i$ is finite, 
this implies that,
under the hypothesis that $\cn$ is a computable number that is either transcendental or has a polynomial root barrier, 
it is possible to compute a finite set $P_{\phi} \subseteq \Z$ witnessing the satisfiability of $\phi$. That is, the sentence $\exists u_1\dots \exists u_n\, \phi$ is equivalent to
\[ 
  \exists u_1 \dots \exists u_n \textstyle\bigvee_{(g_1,\dots,g_n) \in (P_{\phi})^n} \textstyle\left(\phi \land \bigwedge_{i=1}^n u_i = \cn^{g_i}\right).
\]
\Cref{theorem:small-model-property} follows (in particular, the bound on $P_{\phi}$ 
for the case of $\cn$ with a polynomial root barrier is derived 
by iteratively applying the bounds in~\Cref{lemma:last-lambda-lemma,lemma:relativise-quantifiers,lemma:remove-u}).


%%% NEXT LEMMA IS NOT NEEDED ANYMORE
% \begin{restatable}{lemma}{LemmaRewritePowerPredicates}\label{lemma:rewrite-power-predicates}
%   Fix $\cn > 1$. 
%   Consider an equality $u^j = \cn^g u_1^{r_1} \cdot \ldots \cdot u_n^{r_n}$, 
%   where $g,r_1,\dots,r_n \in \Z$ and  $j \in \N$, and a pair $(m,r) \in \N$ with $m \geq 1$ and $r < m$.
%   The following formula in variables $u,u_1,\dots,u_n$ is a tautology of $\exists \R(\ipow{\cn})$:
%   \begin{equation*}
%       u^j = \cn^g u_1^{r_1} \cdot \ldots \cdot u_n^{r_n}\implies \Big(\ipowar{\cn}{m}{r}(u)\iff 
%       \bigvee_{(s_1,\dots,s_n)\in I} \ \bigwedge_{i=1}^n  \ipowar{\cn}{j \cdot m}{s_i}(u_i)\Big),
%   \end{equation*}
%   where 
%   $S \coloneqq \Big\{(s_1,\dots,s_n)\in[0..j \cdot m-1]^n : j \cdot m \text{ divides } r \cdot j - g - \sum_{i=1}^n r_i \cdot s_i\Big\}$.
% \end{restatable}

%%% PREVIOUS VERSION START OF THE SECTION
% \subparagraph*{Auxiliary notions for the procedure.}
% Following~\cite{AvigadY07},
% our procedure relies on an auxiliary function $\lambda_{\cn}
%   \colon \R_{\geq 0} \to \ipow{\cn}$ and auxiliary predicates
% $\ipowar{\cn}{m}{r}$, where $r,n \in \N$ with $r <
%   m$ and $m \geq 1$. The function~$\lambda_{\cn}$ is defined as
% $\lambda_{\cn}(0) \coloneqq 0$ and otherwise, for every $a > 0$,
% $\lambda_{\cn}(a)$ is defined as the largest integer power of $\cn$ that
% is smaller or equal than $a$, i.e., $\lambda_{\cn}(a) \leq a < \cn \cdot
%   \lambda_{\cn}(a)$. Equivalently, for $a > 0$, given the unique pair $(b,v)$
% in $\ipow{\cn} \times [1,\cn)$ such that $a = b \cdot v$, we have
% $\lambda_{\cn}(a) = b$.
% Since the number $\cn$ is fixed and not part of the input of the procedure, we
% often write $\lambda$ instead of $\lambda_{\cn}$.
% The predicate $\ipowar{\cn}{m}{r}$ is interpreted as the set
% $\{ \cn^{i \cdot m + r} : i \in \Z \}$. Note that $\ipowar{\cn}{1}{0}$ is
% equivalent
% to $\ipow{\cn}$.
% Clearly, both $\lambda$ and $\ipowar{\cn}{m}{r}$ are expressible in
% $\exists\R(\cn^{\Z})$:
% \begin{align*}
%   \lambda(x) = y  & \iff (x = 0 \land y = 0) \lor (\ipow{\cn}(y) \land y
%   \leq x < \cn \cdot y)\,;
%   \\
%   \ipowar{\cn}{m}{r}(x) & \iff \exists y : x = y^m \cdot \cn^r \land
%   \ipow{\cn}(y)\,.
% \end{align*}

% The pseudocode of our procedure is given
% in~\Cref{algo:main-procedure,algo:elim-var}. For simplicity of the
% exposition,~\Cref{algo:elim-var} is given as a non-deterministic procedure. As
% usual, a deterministic implementation can be obtained by replacing the
% non-deterministic guesses by a depth-first search (with backtracking), so that
% the overall runtime of the deterministic procedure is polynomial in the number
% of non-deterministic brances multiplied by the runtime of the non-deterministic
% procedure (a fact that we use in~\Cref{sec:complexity} for the complexity
% analysis).

% In the next two subsection, we give an overview of~\Cref{algo:main-procedure,algo:elim-var}, 
% which might be seen as a sketch of their correctness. The precise semantics of these procedures 
% is given in the specification on top of their pseudocode. Complete proofs are given in~\Cref{appendix:procedure}.

% \subsection*{Step II: solving $\exists \ipow{\cn}$}

% Let $\Cmap$ be a map from some finite set of variables to $\Z$, 
% and let $\psi$ be a formula from $\exists\R(\ipow{\cn})$, we write 
% $\psi[\cn^{\Cmap}]$ for the formula obtained from $\psi$ by replacing 
% every variable~$x$ in the domain of $\Cmap$ that occurs free in $\psi$ 
% with the constant $\cn^{\Cmap(x)}$. If this update results in polynomial inequalities featuring a negative degree (e.g., if $\Cmap(x)$ is negative)
% \am{to be continued... see text below}

% \am{todo: explain notation $\psi[\cn^{\Cmap}]$. In particular, it's the following: 
% \begin{algorithmic}
%   \For{$\omega$ in the domain of $\Cmap$}
%     \Comment{obtain a formula that is univariate in $\cn$}
%     \State update $\psi$: replace every occurrence of $\omega$ with $\cn^{\Cmap(\omega)}$
%     \EndFor
%     \State update $\psi$: replace every (in)equality $\tau \sim 0$ with $\cn^{\abs{k}} \cdot \tau \sim 0$,  where $k$ is 
%     \Statex \phantom{update $\psi$:} the largest (in absolute value) negative power of $\cn$ occurring in $\tau$
% \end{algorithmic}
% }

% Let $\vec u = (u_1,\dots,u_n)$ be a vector of variables. A formula $\phi(\vec u)$ from $\exists\R(\ipow{\cn})$ is said to be \emph{$\vec u$-typed} whenever it is of the form 
% $\gamma \land \bigwedge_{i=1}^n \ipowar{\cn}{m_i}{r_i}(u_i)$, 
% and $\gamma$ does not feature predicates~$\ipow{\cn}$. 

% \am{Essentially, $\gamma$ is from existential Tarski arithmetic enriched with the constant $\cn$.}

%%% OLD %%% 
% \begin{lemma}
%   Consider $\phi(u,\vec y)$ a quantifier-free formula from $\exists \R(\ipow{\cn})$, 
%   and let $u$ be a variable from $\vec u$. 
%   The following formula is a tautology of $\exists \R(\ipow{\cn})$: 
%   \begin{equation*}
%     (\exists u\, \phi)
%     \iff 
%     \bigvee_{\ell \in [-m,m]}\ 
%     \bigvee_{q \in Q}\ 
%     \bigvee_{(\mu,g,\vec y^{\vec \ell_1}, \vec y^{\vec \ell_2}) \in F_q}
%     \exists u \left( \vec y^{\vec \ell_1} \cdot u^{\mu} = \cn^{\mu \cdot \ell + g} \cdot \vec y^{\vec \ell_2} \land \phi \right),
%   \end{equation*}
%   where
%   \begin{itemize}
%     \item the integer $m \geq 1$ is such that $\ipowar{\cn}{m}{r}(u)$ occurs in $\phi$, for some $r \in \N$, \am{in new version, this might not be required.}
%     \item the set $Q$ contains the polynomial $u-1$ as well as all polynomials in $\phi$ featuring $u$,
%     \item given $q \in Q$, $F_q$ is the finite set obtained by applying~\Cref{lemma:last-lambda-lemma}
%     to $r(x,v,\vec y) \coloneqq q\sub{x \cdot v}{u}$, where $x$ and $v$ 
%     are two fresh variables.
%   \end{itemize}
% \end{lemma}