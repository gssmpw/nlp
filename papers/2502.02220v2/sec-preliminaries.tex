\section{Preliminaries}
\label{sec:preliminaries}

In this section, we fix our notation, introduce background knowledge on
computable, algebraic and transcendental numbers, and define the
existential theory $\exists\R(\ipow{\cn})$.

\subparagraph*{Sets, vectors, and basic functions.}
Given a finite set $S$, we write $\abs{S}$ for its cardinality. Given $a,b \in
\R$, we write $[a,b]$ for the closed interval $\{ c \in \R : a \leq c \leq b
\}$. We use parenthesis \mbox{$($ and $)$} for open intervals, hence writing,
e.g., $[a,b)$ for the set $\{ c \in \R : a \leq c < b\}$. We write $[a..b]$ for
the set of integers $[a,b] \cap \Z$. Given $A \subseteq \R$, $c \in \R$, and a
binary relation $\sim$ (e.g., $\geq$), we define $A_{\sim c} \coloneqq {\{a \in
A : a \sim c\}}$. The \emph{endpoints} of $A$ are its supremum and infimum, if
they exist. For instance, the endpoints of the interval $[a,b)$ are the numbers
$a$ and $b$, while the endpoints of $[a..b]$ are the numbers $\ceil{a}$ and
$\floor{b}$, where $\floor{\cdot}$ stands for the floor function. 

% We sometimes apply standard set operations and predicates, e.g.,~$\in$ and
% $\subseteq$, to vectors $\vec v = (v_1,\dots,v_d)$. Formally, there is an
% implicit conversion of $\vec v$ into the set $V = \{v_1,\dots,v_d\}$. As an
% example, $\vec {v} \subseteq S$ stand for $V \subseteq S$, where $S$ is a set
% (or another vector).
Given a positive real number $b$ with $b \neq 1$, we write $\log_b(\cdot)$ for
the logarithm function of base $b$. We abbreviate $\log_2(\cdot)$ and
$\log_e(\cdot)$ as $\log(\cdot)$ and $\ln(\cdot)$, respectively. 
%(As it should be clear by now, $e$ stands for Euler's number.) 

Unless stated explicitly, all integers encountered by our algorithms are encoded
in binary; note that $n \in \Z$ can be represented using $1+\ceil{\log(n+1)}$
bits. Similarly, each rational is encoded as a ratio  $\frac{n}{d}$ of two
coprime integers $n$ and $d$ encoded in binary, with $d \geq 1$.

\subparagraph*{Integer polynomials.}
An \emph{integer polynomial} in variables $\vec x = (x_1,\dots,x_n)$ is an
expression $p(\vec x) \coloneqq \sum_{j = 1}^m (a_j \cdot \prod_{i=1}^n
x_i^{d_{j,i}})$, where $a_j \in \Z$ and $d_{j,i} \in \N$ for every $j \in
[1..m]$ and $i \in [1..n]$. In the context of algorithms, we assume the
coefficients $a_j$ to be given in binary encoding, and the exponents $d_{i,j}$
to be given in unary encoding. We rely on the following notions:
\begin{itemize}
  \item The \emph{height} of $p$, denoted $\height(p)$, is defined as
  $\max\{|a_j| : j \in [1..m]\}$. 
  \item  The \emph{degree} of $p$, denoted $\deg(p)$, is defined as
  $\max\{\sum_{i=1}^n d_{j,i} : j \in [1..m]\}$.
  \item Given $i \in [1..n]$, the \emph{partial degree of $p$ in} $x_i$, denoted
  $\deg(x_i,p)$, is $\max\{ d_{j,i} : j \in
  [1..m] \}$. 
  \item The \emph{bit size} of $p$, denoted $\size(p)$, is defined as $m \cdot
  (\ceil{\log(\height(p)+1)} + n \cdot \deg(p))$.
\end{itemize}

\subparagraph*{Computable numbers, and algebraic and transcendental numbers.}
% The set $\D \subseteq \Q$ of \emph{dyadic numbers} is the set of all rationals
% that can be written as $\frac{m}{2^n}$ with $m \in \Z$ and $n \in \N$. Every
% dyadic number can be written in \emph{fixed-point representation} as a word from
% the regular language $\FP \coloneqq \{+,-\}\{0,1\}^+ \dotb \{0,1\}^+$ over the
% alphabet $\{+,-,0,1,{\dotb}\}$. A word $\pm a_n \dots a_0 \dotb a_{-1} \dots
% a_{-m}$ from $\FP$ is said to be a representation of $d \in \D$ whenever $d =
% \pm\sum_{j = -m}^n a_j \cdot 2^j$. 
A real number $\cn \in \R$ is said to be \emph{computable} whenever there is a
(deterministic) Turing machine $T \colon \N \to \Q$ that given in input $n \in
\N$ written in unary (e.g., over the alphabet $\{1\}^*$) returns a rational
number~$T_n$ (represented as described above) such that $\abs{\cn-T_n} \leq
2^{-n}$. We thus have $\cn = \lim_{n \to \infty} T_n$, and for this reason $\cn$
is said to be \emph{computed by} $T$ (or $T$ \emph{computes} $\cn$).
The computable numbers form a field~\cite{Rice54}; 
we will later need the following two statements regarding
their closure under product and reciprocal 
(see~\Cref{appendix:preliminaries} for standalone proofs).

\begin{restatable}{lemma}{LemmaTuringMachineProducts}
  \label{lemma:turing-machine-products}
  Given Turing machines $T$ and $T'$ computing reals $a$ and $b$,
  one can construct a Turing machine $T''$ computing $a \cdot b$.
  If $T$ and $T'$ run in polynomial time, then so does~$T''$.
\end{restatable}
% \begin{proof}[Proof sketch.]
%   Let $\ell \coloneqq \ceil{\log(\abs{T_0}+\abs{T_0'}+3)}$.
%   We define $T''$ as the Turing machine that on input $n$ returns the 
%   rational number $T_{n+\ell} \cdot T_{n+\ell}'$. 
%   Clearly, $T''$ runs in time polynomial in $n$.
%   Moreover, a simple computation shows $\abs{a \cdot b - T''} \leq \frac{1}{2^n}$ for every $n \in \N$.
% \end{proof}
\vspace{-8pt}
\begin{restatable}{lemma}{LemmaTuringMachineReciprocal}
  \label{lemma:turing-machine-reciprocal}
  Given a Turing machine $T$ computing a non-zero real number $r$,
  one can construct a Turing machine $T'$ computing $\frac{1}{r}$. 
  If $T$ runs in polynomial time, then so does $T'$.
\end{restatable}

% \begin{proof}[Proof sketch.]
%   We compute the smallest $k \geq 2$ such that
%   $\frac{1}{2^k} < \abs{T_k}$. Its existence follows from the fact
%   that $\lim_{n \to \infty} T_n = r \neq 0$, whereas $\lim_{n \to \infty}
%   \frac{1}{2^n} = 0$. Since $\abs{r - T_k} \leq \frac{1}{2^k}$, 
%   $T_k$ and $r$ have the same sign. 
%   Let $T_k = \frac{p}{q}$ and define
%   ${\ell \coloneqq 2(k + \ceil{\log(q)})}$. We define $T'$ as the machine that on
%   input $n$ returns the non-zero rational
%   $\frac{s}{\max(\abs{T_{n+\ell}},\abs{T_k} - 2^{-k})}$, 
%   where $s = +1$ if $T_k > 0$, and otherwise $s = -1$. Clearly, if $T$ runs in time polynomial in $n$, so does $T'$. Moreover, one can show 
%   that 
%   $|\frac{1}{r}-T_n'| \leq \frac{1}{2^n}$ for every $n \in \N$, 
%   that is, $T_n'$ computes $\frac{1}{r}$
%   (see~\Cref{appendix:preliminaries}).
% \end{proof}

%%% C-computability commented out %%%

% Given a class $\cclass$ of Turing machines, we say that $\cn$ is
% $\cclass$\emph{-computable} whenever $\cn$ is computed by some machine
% in~$\cclass$.

% \begin{definition}[\cclass-computable numbers] Let \cclass be a class of $\N
%   \to \N$ Turing machines. A real number $\cn$ is said to be
%   \cclass-computable whenever there is a function $f \in C$ computing $\cn$,
%   i.e., on a given in input $n \in \N$ written in unary, $f$ outputs the $n$th
%   bit of $\cn$. \end{definition}

% \am{$C$-computability seems to be needed later, when considering sign
% algorithms. We probably want $\cclass = \pspace$ to cover all algebraic
% numbers (or at least CH). Note: the number is not part of the input, but $n$
% written in unary is.}

% \am{Not sure whether next statement and paragraph are needed here.}

% \begin{restatable}[\cite{Rice54}]{proposition}{PropCompField} The set of
%   computable numbers forms a field, and it is closed under exponentiation
%   $(x,y) \mapsto x^y$ and logarithm $(x,y) \mapsto \log_x(y)$ functions.
%   \end{restatable}

% \noindent Rice's paper~\cite{Rice54} only proves that the set of computable
% numbers forms a field. However, given two computable numbers $a$ and $b$,
% since $\log_a(b) = \frac{\ln(b)}{\ln(a)}$ and $a^b = e^{b \cdot \ln(a)}$, one
% can use truncations of the Taylor expansions of $x \mapsto \ln(x)$ and $x
% \mapsto e^x$, which only require the field operations, to give sequences of
% approximations converging to $a^b$ and $\log_a(b)$, which are thus computable
% numbers. Analogously, one can show that the set of computable numbers is
% closed under the trigonometric functions $x \mapsto \cos(x)$ and $x \mapsto
% \sin(x)$.

A real number $\cn$ is \emph{algebraic} if it is a root of some univariate
non-zero integer polynomial. Otherwise, $\cn$ is \emph{transcendental}. We often
denote algebraic numbers by $\alg,\beta,\eta,\dots$\,. Throughout the paper, we
consider the following canonical representation: an algebraic number $\alg$ is
represented by a triple $(q,\ell,u)$ where $q$ is a non-zero integer polynomial
and $\ell,u$ are (representations of) rational numbers such that $\alg$ is the
only root of $q$ belonging to~$[\ell,u]$. 

\subparagraph*{The existential theory~$\exists\R(\ipow{\cn})$.}
Let $\cn > 0$ be a computable real number. We consider the structure $(\R; 0, 1,
\cn, +, \cdot, \ipow{\cn}, <,=)$ extending the signature of the FO theory of the
reals with the constant $\cn$ and the unary \emph{integer power} predicate
$\ipow{\cn}$ interpreted as $\{ \cn^{i} : i \in \Z \}$. Formulae from the
existential theory of this structure, denoted $\exists \R(\ipow{\cn})$, are
built from the grammar
\begin{align*} 
  \phi,\psi &\,\Coloneqq\,  p(\cn,\vec x) \sim 0  \,\mid\, \ipow{\cn}(x) \,\mid\, \top \,\mid\, \bot \,\mid\, \phi \lor \psi \,\mid\, \phi \land \psi \,\mid\, \exists x \, \phi \,,
\end{align*}
where $\sim$ belongs to $\{<,=\}$,  the argument $x$ of the predicate
$\ipow{\cn}(x)$ is a variable, and $p$ is an integer polynomial involving $\cn$
and variables $\vec{x}$. For convenience of notation, $\cn$ is in this context
seen as a variable of the polynomial $p$, so that we can rely on the previously
defined notions of height, degree and bit size. We remark that, then, $\height(p)$
is independent of $\cn$ whereas $\deg(p)$ depends on the integers occurring as
powers of $\cn$. The bit size of a formula $\phi$, denoted as $\size(\phi)$, is
the number of bits required to write down $\phi$ (where~$\cn$ is stored
symbolically, using a constant number of~symbols). Similarly, we write
$\deg(\phi)$ and $\height(\phi)$ for the maximum degree and height of
polynomials occurring in $\phi$, respectively.


The semantics of formulae from $\exists \R(\ipow{\cn})$ is standard; 
it is the one of the FO theory of the reals, plus a rule stating
that $\ipow{\cn}(x)$ is true whenever $x \in \R$ belongs to the set $\ipow{\cn}$.
The grammar above features disjunctions~($\lor$), conjunctions ($\land$), true
($\top$) and false~($\bot$), but it does not feature negation ($\lnot$) on top
of atomic formulae. 
This restriction is w.l.o.g.: $\lnot \ipow{\cn}(x)$ is
equivalent to the formula $x \leq 0 \lor \exists y : \ipow{\cn}(y) \land y < x
\land x < \cn \cdot y$ stating that $x$ is either non-positive or strictly
between two successive integer powers of~$\cn$, whereas $\lnot (p(\cn,\vec x) <
0)$ and $\lnot (p(\cn,\vec x) = 0)$ are equivalent to $p(\cn, \vec x) = 0 \lor
-p(\cn,\vec x) < 0$, and $p(\cn,\vec x) < 0 \lor -p(\cn,\vec x) < 0$,
respectively. We still sometimes write negations in formulae, but these
occurrences should be seen as shortcuts. The grammar also avoids polynomials in
the scope of~$\ipow{\cn}(\cdot)$, since
$\ipow{\cn}(p(\cn,\vec{x}))$ is equivalent to $\exists y : y = p(\cn,\vec{x})
\land \ipow{\cn}(y)$. We write $\phi \models \psi$ whenever $\phi$
\emph{entails} $\psi$.

% \am{Following might be annoying for complexity, e.g., if $\mu$ has a
% polynomial root barrier but $\frac{1}{\mu}$ does not.} Without loss of
% generality, we can assume $\cn > 1$. Indeed, we can handle integer powers with
% bases $\mu \leq 1$ as follows. For $\mu = 1$ the FO theory of $(\R; 0, 1, \mu,
% +, \cdot, \ipow{\mu}, <,=)$ collapses to Tarski arithmetic. For $\mu < 1$, we
% have $\frac{1}{\mu} > 1$ and the FO theories of the structures $(\R; 0, 1,
% \mu, +, \cdot, \ipow{\mu}, <,=)$ and $(\R; 0, 1, \frac{1}{\mu}, +, \cdot,
% \ipow{\frac{1}{\mu}}, <,=)$ are equivalent --- because $\ipow{\frac{1}{\mu}} =
% \ipow{\mu}$ and $x = \frac{1}{\mu} \iff \mu \cdot x = 1$.

