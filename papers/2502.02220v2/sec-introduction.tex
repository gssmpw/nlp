\section{Introduction}
\label{sec:introduction}

Tarski's exponential function problem asks to determine the decidability of the
validity problem from the first-order (FO) theory of the structure $(\R; 0, 1,
+, \cdot, e^x, <, =)$. This theory, hereinafter denoted $\R(e^x)$, extends the
FO theory of the reals (a.k.a.~Tarski arithmetic) with the exponential
function~${x \mapsto e^x}$. A celebrated result by Macintyre and Wilkie
establishes an affirmative answer to Tarski's problem
conditionally to the truth of Schanuel's conjecture, a profound conjecture from
transcendental number theory~\cite{MacWilkie96}. Recent years have seen this
result being used as a black-box to establish conditional decidability results
for numerous problems stemming from dynamical systems~\cite{Dantam21,Almagor22}
automata theory~\cite{Daviaud21,ChistikovKMP22}, neural networks
verification~\cite{HankalaHKV24,IsacZBK23}, the theory of stochastic
games~\cite{BaierCMP23}, and differential privacy~\cite{BartheCKS021}. 
% This is
% only a non-exhaustive list of the many areas within computer science
% where~$\R(e^x)$, and thus Schanuel's conjecture, finds important applications.

As it is often the case when appealing to a result as a black-box, some of the
computational tasks resolved by relying on the work in~\cite{MacWilkie96} do not
require the full power of~$\R(e^x)$. Consequently, it is natural to ask whether
some of these tasks can be tackled without relying on unproven conjectures,
perhaps by reduction to tame fragments or variants of~$\R(e^x)$. A~few~results
align with this question:
\begin{itemize}
  \item In the papers~\cite{AnaiW00,AchatzMW08,McCallumW12}, Achatz,
    Anai, McCallum and Weispfenning introduce a procedure to decide sentences of
    the form $\exists x \exists y : y = \text{trans}(x) \land \phi(x,y)$, where~$\phi$ 
    is a formula from Tarski arithmetic, and $x \mapsto \text{trans}(x)$
    is any analytic and strongly transcendental function
    (see~\cite[Section~2]{McCallumW12} for the precise definition). 
    Since $x \mapsto e^x$ enjoys such
    properties, this result shows a non-trivial fragment of $\R(e^x)$ that is
    unconditionally decidable. The procedure is implemented in the tool
    Redlog~\cite{DolzmannS97}. No complexity bound is known.

  \item In~\cite{Dries1986}, van den Dries proves decidability of the extension
    of Tarski arithmetic with the unary predicate~$\ipow{2}$ interpreted as the
    set $\{2^i : i \in \Z\}$, i.e., the set of all integer powers of $2$. While
    this result is achieved by model-theoretic arguments, an effective
    quantifier elimination procedure was later given by Avigad and
    Yin~\cite{AvigadY07}. Their procedure runs in~\tower, and in fact it
    requires non-elementary time already for the elimination of a single
    quantified variable. The choice of the base $2$ for the integer powers is
    somewhat arbitrary: in~\cite{DriesG06}, the decidability is extended to 
    any fixed algebraic number (i.e., a number that is root of some polynomial equation;
    see~\Cref{sec:preliminaries} for background knowledge on computable, algebraic and transcendental numbers), and in fact
    Avigad and Yin's procedure is also effective for any such number. 
    Considering any two $\alpha,\beta \in \R$
    satisfying~$\ipow{\alpha} \cap \ipow{\beta} = \{1\}$ yields undecidability,
    as shown by Hieronymi in~\cite{Hieronymi10}.
\end{itemize}

When comparing the two lines of work discussed above, it becomes apparent that
there is a balance to be struck between reasoning about transcendental numbers,
the path followed by the first set of works, and developing algorithms that are
well-behaved from a complexity standpoint, the path taken in particular in~\cite{AvigadY07}. Our aim with this paper is to somewhat bridge this gap: we add
to the second line of work by studying predicates for integer powers of
bases that may be transcendental, all the while maintaining complexity upper bounds. 

From now on, we write $\exists\R(\ipow{\cn})$ to denote the existential fragment
of the FO theory of the structure $(\R; 0, 1, \cn, +, \cdot, \ipow{\cn}, <,=)$,
where $\cn > 0$ is a fixed real number. In this paper, we examine the complexity
of deciding the satisfiability problem of $\exists\R(\ipow{\cn})$ for different
choices of the number $\cn$. The following theorem summarises our results.

\begin{restatable}{theorem}{TheoremRootBarrier}
  \label{theorem:result-root-barrier}
    Fix a real number $\cn > 0$. The satisfiability problem for
    $\exists\R(\ipow{\cn})$ is
    \begin{enumerate}
      \item\label{theorem:result-root-barrier:point1} in \expspace whenever
      $\cn$ is an algebraic number;
      \item\label{theorem:result-root-barrier:point2} in \threeexptime if
        $\cn \in \{\pi,\, e^\pi,\, e^\eta,\, \alg^\eta,\, \ln(\alg),\, 
        \frac{\ln(\alg)}{\ln(\beta)} : \alg,\beta,\eta \text{ algebraic with } \alg > 0 \text{ and } 1 \neq \beta > 0\}$;
      \item\label{theorem:result-root-barrier:point3} decidable whenever $\cn$
      is a computable transcendental number.
    \end{enumerate}
    
\end{restatable}


% \begin{restatable}{theorem}{TheoremRootBarrier}
%   \label{theorem:result-root-barrier} Fix two algebraic numbers $\alg,\beta$,
%   and let $\cn > 0$ be one of the following numbers: \begin{center} $\pi$\,, \
%   $\ln(\alg)$\,, \ $e^\beta$\,, \ $\frac{\ln(\alg)}{\ln(\beta)}$\,, \
%   $\alg^\beta$\,. \end{center} There is a \am{\threeexptime} algorithm
%   deciding the satisfiability problem for $\exists\R(\ipow{\cn})$.
%   \end{restatable}

\noindent
\Cref{theorem:result-root-barrier} has a catch, however.
To be effective, the algorithm for deciding~$\exists\R(\ipow{\cn})$
requires: 
\begin{itemize}
  \item For~\Cref{theorem:result-root-barrier}.\ref{theorem:result-root-barrier:point1},
  to have access to a canonical representation (see~\Cref{sec:preliminaries}) of $\cn$.
  \item In the cases covered
  by~\Cref{theorem:result-root-barrier}.\ref{theorem:result-root-barrier:point2},
  to have access to representations of~$\alg$,~$\beta$, and~$\eta$.
  \item In the case of $\cn$ computable transcendental number
  (\Cref{theorem:result-root-barrier}.\ref{theorem:result-root-barrier:point3}),
  to have access to a Turing machine $T$ that computes $\cn$ (that is,~given an
  input~$n \in \N$ written in unary, $T$ returns a rational number $T_n$ such
  that $\abs{\cn-T_n} \leq 2^{-n}$).
\end{itemize}
In summary, \Cref{theorem:result-root-barrier} shows that $\exists\R(\ipow{\cn})$ is
decidable for every fixed computable number $\cn > 0$, as long as it is known
whether $\cn$ is algebraic or transcendental, and in the former case having
access to a canonical representation of~$\cn$.

% \am{Is the second point correct? I've removed the requirement to knowing
% whether $\beta$ is irrational in $\alpha^\beta$; as we have a polynomial for
% $\beta$ we are able to deduce this without further assumptions. For
% $\frac{\ln(\alg)}{\ln(\beta)}$, I wonder whether we can drop the irrationality
% assumption. E.g., when it is rational, can we deduce a boud on its algebraic
% representation by looking at $\alg,\beta$?}.

The results in~\Cref{theorem:result-root-barrier} are obtained by
\textbf{(i)} reducing the satisfiability problem for~$\exists\R(\ipow{\cn})$ to the
problem of solving instances of~$\exists\R(\ipow{\cn})$ where all variables
range over~$\ipow{\cn}$, and \textbf{(ii)}~showing that a solution over $\ipow{\cn}$ can
be found by only looking at a ``small'' set of integer powers of~$\cn$ (a
\emph{small witness property}). In proving Step~(ii), we also obtain a
quantifier elimination procedure for \emph{sentences}
of~$\exists\R(\ipow{\cn})$, that is formulae where no variable occurs free. This
procedure provides a partial answer to the question raised in~\cite{AvigadY07}
regarding the complexity of removing a single existential variable
in Tarski arithmetic extended with $\ipow{2}$:
within sentences of the existential fragment, such an elimination step 
can be performed in elementary time.

% The technique used to prove~\Cref{theorem:result-root-barrier} extends to all
% computable transcendental numbers:%
% \begin{restatable}{theorem}{TheoremTranscendental}\label{theorem:general-result-transcendental}
% Fix a computable transcendental number $\cn > 0$. There is an algorithm for
% deciding the satisfiability problem for $\exists\R(\ipow{\cn})$.
% \end{restatable}

% \noindent We recall that a number is computable whenever it can be
% approximated to any precision by a Turing machine
% (see~\Cref{sec:preliminaries} for the formal definition). To be effective, the
% algorithm in~\Cref{theorem:general-result-transcendental} requires having
% access to one such Turing machine for computing $\cn$. Together
% with~\Cref{theorem:result-root-barrier},~\Cref{theorem:general-result-transcendental}
% shows that $\exists\R(\ipow{\cn})$ is decidable for every fixed computable
% number $\cn > 0$. However, running the algorithm requires knowing whether
% $\cn$ is algebraic or transcendental, and in the former case having access to
% a polynomial representing~$\cn$.

Coming back to our initial question on identifying computational tasks that
might not need the full power of $\R(e^x)$, as a by-product of our results we
show that the entropic risk threshold problem for stochastic games
studied by Baier, Chatterjee, Meggendorfer and Piribauer~\cite{BaierCMP23} is
unconditionally decidable in~\exptime even when the base of the entropic risk is
$e$ (or algebraic) and the aversion factor is any (fixed) algebraic number.

\section{Approaching complexity bounds with root barriers}\label{section:intro-to-root-barriers}
Theorems~\ref{theorem:result-root-barrier}.\ref{theorem:result-root-barrier:point1}
and \ref{theorem:result-root-barrier}.\ref{theorem:result-root-barrier:point2}
are instances of a more general result concerning classes of computable real
numbers. To properly introduce this result, it is beneficial to go back to
Macintyre and Wilkie's work on $\R(e^x)$. The exact statement made
in~\cite{MacWilkie96} is that~$\R(e^x)$ is decidable as soon as the following
computational problem, implied by Schanuel's conjecture, is established:

\begin{conjecture}
  \label{conjecture:WSC}
  There is a procedure that for input $f_1,\dots,f_n,g \in
  \Z[x_1,\dots,x_n,e^{x_1},\dots,e^{x_n}]$, with $n \geq 1$, returns a positive
  integer $t$ with the following property: for every non-singular\footnote{A
  solution~$\vec \alpha$ of $\bigwedge_{i=1}^nf_i(\vec x) = 0$ is said to be
  non-singular whenever the determinant of the $n \times n$ Jacobian matrix
  $\frac{\partial(f_1,\dots,f_n)}{\partial(x_1,\dots,x_n)}$ is, once evaluated
  at $\vec \alpha$, non-zero. We give this definition only for completeness of
  the discussion on~\Cref{conjecture:WSC}. It is not used in this paper.}
  solution $\vec \alpha \in \R^n$ of the system of equalities
  $\bigwedge_{i=1}^nf_i(\vec x) = 0$, either $g(\vec \alpha) = 0$ or
  $\abs{g(\vec \alpha)} > t^{-1}$.
\end{conjecture}

\noindent
Above, $\Z[x_1,\dots,x_n,e^{x_1},\dots,e^{x_n}]$ is the set of all $n$-variate
exponential-polynomials with integer coefficients. As remarked
in~\cite{MacWilkie96}, $t$ is guaranteed to exist by Khovanskii's
theorem~\cite{Khovanskii91}, hence the crux of the problem concerns how
to effectively compute such a number starting from $f_1,\dots,f_n$ and $g$. 
% Crucially,  
% the statement $\Phi \coloneqq $ ``for every non-singular solution $\vec \alpha
% \in \R^n$ of the system of equalities $\bigwedge_{i=1}^nf_i(\vec x) = 0$,
% either $g(\vec \alpha) = 0$ or $\abs{g(\vec \alpha)} > t^{-1}$'' can be
% encoded within the universal fragment of~$\R(e^x)$. This implies that the
% existence of a procedure for the universal (or, alternatively, existential)
% fragment of~$\R(e^x)$ would entail~\Cref{conjecture:WSC}: given
% $f_1,\dots,f_n,g$, an algorithm consists of enumerating all $t \in \N
% \setminus \{0\}$ until finding one making~$\Phi$ valid. Termination is
% guaranteed by Khovanskii's theorem. In summary, $\R(e^x)$ and its universal
% and existential fragments $\forall\R(e^x)$ and  $\exists\R(e^x)$ are all
% equidecidable, and their decidability is equivalent to the truth
% of~\Cref{conjecture:WSC}.
The purpose of the dichotomy ``either $g(\vec \alpha) = 0$ or
${\abs{g(\vec{\alpha})} > t^{-1}}$'' is in part to resolve what is a
fundamental problem when working with computable real numbers. Let
$\vec{\alpha}$ to be a vector of computable numbers. Consider the problem of
establishing, given in input a polynomial $p$ with integer coefficients, whether
$p(\vec{\alpha})$ is positive, negative, or zero. This \emph{polynomial sign
evaluation} task is a well-known undecidable problem. Intuitively, the
undecidability arises from the possibility that any approximation
$\vec{\alpha}^*$ of $\vec \alpha$ might yield $p(\vec{\alpha}^*) \neq 0$, even
though $p(\vec \alpha) = 0$. However, when working under the hypothesis that
either $p(\vec \alpha) = 0$ or $\abs{p(\vec \alpha)} > t^{-1}$, the problem
becomes decidable: it suffices to compute an approximation $\vec \alpha^*$
enjoying $|p(\vec \alpha) - p(\vec{\alpha}^*)| < (2t)^{-1}$, and then check
whether $|p(\vec \alpha^*)| \leq (2t)^{-1}$. If the answer is positive, then
$p(\vec{\alpha}) = 0$, otherwise $p(\vec{\alpha})$ and $p(\vec \alpha^*)$ have
the same sign.

The same issue occurs in~$\exists\R(\ipow{\cn})$: under the sole hypothesis that
$\cn$ is computable, we cannot even check if $\cn = 2$ holds. However, what we
can do is to draw some inspiration from~\Cref{conjecture:WSC}, and introduce as
a further assumption the existence of what we call a \emph{root barrier} of
$\cn$. Below, $\N_{\geq 1} = \{1,2,3,\dots\}$, and given a polynomial $p$ we
write $\deg(p)$ for its \emph{degree} and $\height(p)$ for its \emph{height}
(i.e., the maximum absolute value of a coefficient~of~$p$).

\begin{definition}
  \label{definition:intro:root-barrier}
  A function $\sigma \colon (\N_{\geq 1})^2 \to \N$ is a root barrier of $\cn
  \in \R$ if for every non-constant polynomial $p(x)$ with integer coefficients, $p(\cn) =
  0$ or $\ln \abs{p(\cn)} \geq {-\sigma(\deg(p), \height(p))}$.
\end{definition}

To avoid non-elementary bounds on the runtime of our algorithms, we focus on
computable numbers having root barriers $\sigma(d,h)$ that are polynomial
expressions of the form ${c \cdot (d + \ceil{\ln h})^k}$, where $c,k \in \N$
are some positive constants and $\ceil{\cdot}$ is the ceiling function. We call such
functions~\emph{polynomial root barriers},  
highlighting the fact that then $\sigma(\deg(p),\height(p))$
in~\Cref{definition:intro:root-barrier} is bounded by a polynomial in the bit
size of $p$. The
aforestated~\Cref{theorem:result-root-barrier}.\ref{theorem:result-root-barrier:point2} is obtained by instantiating the following \Cref{theorem:general-result-root-barrier}.\ref{theorem:general-result-root-barrier:point2} to natural choices of $\cn$.

\begin{restatable}{theorem}{TheoremGeneralResultRootBarrier}
  \label{theorem:general-result-root-barrier}
    Let $\cn > 0$ be a real number computable by a polynomial-time Turing machine, 
    and let~$\sigma(d,h) \coloneqq {c
    \cdot (d + \ceil{\ln h})^k}$ be a root barrier of~$\cn$, 
    for some $c,k \in \N_{\geq 1}$.
    % there is an algorithm that given in
    % input $n \in \N$ encoded in unary, returns in time polynomial in $n$
    % (a~representation of) an algebraic number $\cn^*$ such that $\abs{\cn-\cn^*}
    % \leq 2^{-n}$.
    \begin{enumerate}
      \item\label{theorem:general-result-root-barrier:point1} If $k = 1$, then
      the satisfiability problem for $\exists\R(\ipow{\cn})$ is in \twoexptime.
      \item\label{theorem:general-result-root-barrier:point2} If $k > 1$, then
      the satisfiability problem for $\exists\R(\ipow{\cn})$ is in
      \threeexptime.
    \end{enumerate}

\end{restatable}

\noindent
As we will see in~\Cref{sec:poly-evaluation}, whenever algebraic, the base~$\cn$ has a root barrier with exponent~${k =
1}$, and the related satisfiability problem for~$\exists(\ipow{\cn})$ thus lie in~\twoexptime.
However, a small trick will allow us to further improve this result to
\expspace,
establishing~\Cref{theorem:result-root-barrier}.\ref{theorem:result-root-barrier:point1}. 

% \subparagraph*{A look ahead.} 
% After providing necessary definitions and notation (\Cref{sec:preliminaries}), in~\Cref{sec:the-algorithm} we describe the procedure
% for solving $\exists\R(\ipow{\cn})$, and establish
% \Cref{theorem:result-root-barrier}.~\ref{theorem:result-root-barrier:point3}
% and~\Cref{theorem:general-result-root-barrier}. A key component of the procedure
% is given by the guessing over a ``small'' set of integer powers of $\cn$, to
% find solutions over $\ipow{\cn}$. The correctness of this step of the procedure
% is ensured by~\Cref{theorem:small-model-property} from \Cref{sec:the-algorithm},
% of which we sketch the proof in~\Cref{sec:solving-substructure}.
% In~\Cref{sec:poly-evaluation} we discuss instances of numbers with polynomial
% root barriers, ultimately
% establishing~\Cref{theorem:result-root-barrier}.\ref{theorem:result-root-barrier:point1}
% and~\Cref{theorem:result-root-barrier}.\ref{theorem:result-root-barrier:point2}.
% In~\Cref{sec:application-entropic-risk} we apply our results to the entropic risk threshold problem for stocastic games.





%%%% BELOW RANDOM STUFF FOR THE INTRODUCTION

% \paragraph*{Surrounding literature (and some small misconception).}

% Syntactic proof for $2^\Z$, also works for algebraic numbers $\alg > 1$:
% \cite{AvigadY07}. To observe that, for transcendental numbers $\cn> 0$ is
% stronger for complexity results (because we don't know what happens when
% taking $\frac{1}{\cn}$ in terms of root barriers).

% Original proof of decidability: \cite{Dries1986}. Misconception: it states
% that $2^\Z$ is decidable. In the last page it states that $2$ plays no role
% and any real number can be used. This part however refers to enjoying
% quantifier-elimination, and not to decidability. Indeed, even when considering
% just computable real numbers, the theory is undecidable.

% Multiple basis undecidable: \cite{Hieronymi10}. Open for existential fragment.

% ===========================================



% Possible notation for the introduction: \begin{itemize} \item We denote by
% $\R$ the real numbers. \item Euler's number $e$. \item Definition of an
% algebraic number. And transcendental number. \item Definition of a computable
% number. \item Definition $\cn^{\Z} \coloneqq \{ \cn^i : i \in \Z\}$. \am{Are
% we taking this or $\cn^{\Z} \coloneqq \{\pm \cn^i : i \in \Z\}$ ?}
% \end{itemize}

% Possible motivations: \begin{itemize} \item Theory of the reals plus
% exponentiation. Only decidable assuming Schanuel conjecture. Recently, a lot
% of works make use of this result. Sometimes the instances are simpler, and
% Schanuel conjecture is not required.  
%   \item Entropic risk \end{itemize}

% \am{Do we need $\cn \geq 1 + \epsilon$ for some known rational $\epsilon>0$ ?}



% \begin{restatable}{theorem}{TheoremTranscendental}
%   \label{theorem:transcendental-number} Fix a computable transcendental
%   number~$\cn > 0$. There is an algorithm deciding the satisfiability problem
%   for $\exists\R(\ipow{\cn})$. \end{restatable}

% We stress that, in order to be effective, the algorithms (one for each choice
% of $\cn$) in~\Cref{theorem:transcendental-number} require knowing a Turing
% machine computing $\cn$.


% \am{It could be that $\frac{\ln(\alg)}{\ln(\beta)}$ is harder}

  
% Overview/Structure of the paper (synopsis).
