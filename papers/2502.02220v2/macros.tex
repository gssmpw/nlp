%% Defining new math commands
%% (without worries about math mode or spacing)
\newcommand{\newextmathcommand}[2]{%
    \newcommand{#1}{\ensuremath{#2}\xspace}
}

\newcommand{\renewextmathcommand}[2]{%
    \renewcommand{#1}{\ensuremath{#2}\xspace}
}

%% Textual labels 
\makeatletter
\newcommand{\labeltext}[2]{%
  #1%
  \@bsphack%
  \csname phantomsection\endcsname % in case hyperref is used
  \def\@currentlabel{#1}{\label{#2}}%
  \@esphack%
}
\makeatother

%% Descriptions without indent

\makeatletter
\newenvironment{ddescription}%
               {\list{}{\leftmargin=0pt
                        \labelwidth\z@ \itemindent-\leftmargin
                        \let\makelabel\descriptionlabel}}%
               {\endlist}
\makeatother

%% Numerical domains 

\newextmathcommand{\N}{\mathbb{N}}
\newextmathcommand{\Np}{\Nat_+}
\newextmathcommand{\Z}{\mathbb{Z}}
\newextmathcommand{\Q}{\mathbb{Q}}
\newextmathcommand{\R}{\mathbb{R}}
\newextmathcommand{\B}{\mathbb{B}}
\newextmathcommand{\D}{\mathbb{D}}
\newextmathcommand{\FP}{\mathbb{D}_{\textit{fp}}}

%% Complexity Classes 

\newextmathcommand{\fptime}{\textup{\textsc{FP}}}
\newextmathcommand{\ptime}{\textup{\textsc{P}}}
\newextmathcommand{\np}{\textup{\textsc{NP}}}
\newextmathcommand{\pspace}{\textup{\textsc{PSpace}}}
\newextmathcommand{\nexptime}{\textup{\textsc{NExp}}}
\newextmathcommand{\exptime}{\textup{\textsc{Exp}}}
\newextmathcommand{\twoexptime}{\textup{\textsc{2Exp}}}
\newextmathcommand{\threeexptime}{\textup{\textsc{3Exp}}}
\newextmathcommand{\expspace}{\textup{\textsc{ExpSpace}}}
\newextmathcommand{\twoexpspace}{\textup{\textsc{2ExpSpace}}}
\newextmathcommand{\tower}{\textup{\textsc{Tower}}}
\newextmathcommand{\cclass}{\textup{\textsc{C}}}

\newextmathcommand{\poly}{\textup{poly}}

%% Standard notation

\renewextmathcommand{\phi}{\varphi}
\renewcommand{\vec}{\bm}

\newextmathcommand{\lcm}{{\rm lcm}}

\newcommand{\abs}[1]{\ensuremath{\left|#1\right|}\xspace}
\newcommand{\ceil}[1]{\ensuremath{\left\lceil#1\right\rceil}\xspace}
\newcommand{\floor}[1]{\ensuremath{\left\lfloor#1\right\rfloor}\xspace}
\newcommand{\frpart}[1]{\ensuremath{\left\{#1\right\}}\xspace}

\newextmathcommand{\dom}{\textup{dom}}

%%For multiset notation
\DeclareFontEncoding{LS2}{}{\noaccents@}
  \DeclareFontSubstitution{LS2}{stix}{m}{n}
  \DeclareSymbolFont{stix@largesymbols}{LS2}{stixex}{m}{n}
  \SetSymbolFont{stix@largesymbols}{bold}{LS2}{stixex}{b}{n}
  \DeclareMathDelimiter{\lBrace}{\mathopen} {stix@largesymbols}{"E8}%
                                            {stix@largesymbols}{"0E}
  \DeclareMathDelimiter{\rBrace}{\mathclose}{stix@largesymbols}{"E9}%
                                            {stix@largesymbols}{"0F}
\newcommand{\multiset}[1]{\ensuremath{\left\lBrace#1\right\rBrace}\xspace}
\newextmathcommand{\defeq}{\coloneqq}
\newextmathcommand{\eqdef}{\defeq}

%% Substitutions 

\newcommand{\sub}[2]{\ensuremath{[#1\,/\,#2]}\xspace}

%% For parameter tables 

\definecolor{light-gray}{gray}{0.95}
\newcolumntype{g}{>{\columncolor{light-gray}}r}
\newcommand{\ditto}{\raisebox{-2pt}{\scalebox{1.1}{\textit{\texttt{''}}}}}

%% For algorithms

\newcommand{\myif}{\textbf{if}\xspace}
\newcommand{\mythen}{\textbf{then}\xspace}
\newcommand{\myelse}{\textbf{else}\xspace}
\newcommand{\mycontinue}{\textbf{continue}\xspace}
\newcommand{\myguess}{\textbf{guess}\xspace}
\newcommand{\myreturn}{\textbf{return}\xspace}

\algnewcommand\algorithmicndbranchoutput{\textbf{Output of each branch ($\beta$):}}
\algnewcommand\NDBranchOutput{\item[\algorithmicndbranchoutput]}
\algnewcommand\algorithmicglobalspec{\textbf{Ensuring:}}
\algnewcommand\GlobalSpec{\item[\algorithmicglobalspec]}


\newextmathcommand{\Cmap}{\textup{\textsc{C}}}

%% Procedure names 

\newcommand{\ElimVar}{\textsc{ElimVar}\xspace}

%% Problems 

\newextmathcommand{\SIGN}{\textup{\textsc{Sign}}}
\newextmathcommand{\ERisk}{\textup{\textsc{ERisk}}}

%% Numbers 

\newextmathcommand{\cn}{\xi} % computable number
\newextmathcommand{\alg}{\alpha}

\newextmathcommand{\dotb}{\mathbin{\scalebox{0.7}{$\bullet$}}}

%% Polynomials 

\newextmathcommand{\height}{\textup{h}}
\newextmathcommand{\size}{\textup{size}}

%% Powers 

\newcommand{\ipow}[1]{\ensuremath{{#1}^{\Z}}\xspace}
\newcommand{\ipowe}{\ipow{e}}
\newcommand{\ipowar}[3]{\ensuremath{{#1}^{#2 \Z + #3}}\xspace}
