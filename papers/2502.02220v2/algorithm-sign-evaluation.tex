\begin{algorithm}[t]
  \caption{Algorithm for solving $\SIGN_{\cn}$ when $\cn$ has a root barrier.}
  \label{algo:sign-evaluation}
  \begin{algorithmic}[1]
    \Fixed \tab A number $\cn \in \R$ computed by a Turing machine $T$ and having a root barrier $\sigma$.
    \Require \tab A univariate integer polynomial $p(x)$ of degree $d$ and height $h$.
    \Ensure \tab The symbol $\sim$ from $\{<,>,=\}$ such that $p(\cn) \sim 0$.
    \medskip
    %
    \State $n \gets 1 + 2 \sigma(d,h) + 3 d\ceil{\log(h+4)}$ 
      \label{algo:sign-evaluation:bound-on-n}
    \If{$\abs{p(T_n)} \leq 2^{-2 \sigma(d,h)-1}$ and $\abs{T_n} < h+2$}
      \textbf{return} the symbol $=$
      \label{algo:sign-evaluation:small-pTn}
    \Else \
      \textbf{return} the sign of $p(T_n)$
      \label{algo:sign-evaluation:large-pTn}
    \EndIf 
  \end{algorithmic}
\end{algorithm}%

% \begin{algorithm}[t]
%   \caption{Solving $\SIGN_{\cn}$ when $\cn$ has a polynomial root barrier.}
%   \label{algo:sign-evaluation}
%   \begin{algorithmic}[1]
%     \Fixed 
%       \tab \begin{minipage}[t]{0.9\linewidth}
%         A Turing machine $T$ computing $\cn$,\\
%         and a root barrier $\sigma(d,h) \coloneqq \big(c \cdot (d + \ceil{\log(h+1)}))^k$ for $\cn$, with $c,k \in \N$.
%       \end{minipage}
%       \vspace{2pt}
%     \Require \tab A univariate integer polynomial $p(x)$ of degree $d$ and height $h$.
%     \Ensure \tab The symbol $\sim$ from $\{<,>,=\}$ such that $p(\cn) \sim 0$.
%     \medskip
%     %
%     \If{$p$ is a constant polynomial or $\abs{T_0} \geq h+3$} 
%       \State \textbf{return} the sign of $p(T_0)$
%       \Comment{$\cn$ is far from every root of $p$}
%       \label{algo:sign-evaluation:far-from-roots}
%     \EndIf
%     \State $n \gets 1 + 2 \cdot \big(\sigma(d,h) + \ceil{\log(d)}\big) + d\ceil{\log(h+4)}$ 
%       \label{algo:sign-evaluation:bound-on-n}
%       \Comment{$n$ is in $\poly(\size(p))$}
%     \If{$\abs{p(T_n)} \leq 2^{-2 \cdot \sigma(d,h)-1}$}
%       \textbf{return} the symbol $=$
%       \label{algo:sign-evaluation:small-pTn}
%     \EndIf 
%     \State \textbf{return} the sign of $p(T_n)$
%       \label{algo:sign-evaluation:large-pTn}
%   \end{algorithmic}
% \end{algorithm}%
