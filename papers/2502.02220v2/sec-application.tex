\section{An application: the entropic risk threshold problem}
\label{sec:application-entropic-risk}

We now apply some of the machinery developed
for~$\exists\R(\ipow{\cn})$ to remove the appeal to Schanuel's conjecture from
the decidability proof of the entropic risk threshold problem
for stochastic games from~\cite{BaierCMP23}.
Briefly, a \emph{(turn-based) stochastic game} is a tuple $\mathsf{G} = (S_{\max},S_{\min},A,\Delta)$ where $S_{\max}$ and $S_{\min}$ are disjoint finite set of \emph{states} controlled by two players, $A$ is a function from states to a finite set of \emph{actions}, 
and $\Delta$ is a function taking as input a state $s$ and an action from $A(s)$, 
and returning a \emph{probability distribution} on the set of states. Below, we write $\Delta(s,a,s')$ for the probability associated to $s'$ in $\Delta(s,a)$, 
and set $S \coloneqq S_{\max} \cup S_{\min}$.

Starting from an initial state $\hat{s}$, a play of the game produces an
infinite sequence of states~$\rho=s_1s_2s_3\dots$ (a path), to which we
associate the \emph{total reward} $\sum_{i=1}^{\infty} r(s_i)$, where $r \colon
S \to \R_{\geq 0}$ is a given \emph{reward function}. 
A classical problem is to determine the strategy for one of the
players that optimises (minimises or maximises) its expected total reward.
Instead of expectation, the \emph{entropic risk} yields the normalised logarithm
of the average of the function $b^{-\eta X}$, where the \emph{base} $b > 1$
and the \emph{risk aversion factor} $\eta > 0$ are real numbers, and $X$ is a
random variable ranging over total rewards. We refer the reader
to~\cite{BaierCMP23} for motivations behind this notion, as well as all formal
definitions.

Fix a base~$b > 1$ and a risk aversion factor $\eta \in \R$.
The \emph{entropic risk threshold problem} $\ERisk[b^{-\eta}]$ asks to
determine if the entropic risk is above a threshold $t$. The inputs of this
problem are a stochastic game $\mathsf{G}$ having rational probabilities~$\Delta(s,a,s')$,
%(given as a pair of integers in binary), 
an initial state $\hat{s}$, a reward function $r \colon S \to \Q_{\geq 0}$ and a
threshold $t \in \Q$. In~\cite{BaierCMP23}, this problem is proven to be in \pspace
for $b$ and $\eta$ rationals, and decidable subject to Schanuel's conjecture
if~$b = e$ and $\eta \in \Q$ (both results also hold when $b$ and $\eta$
are not fixed). We improve upon the latter result, by establishing the following
theorem (that assumes having representations of $\alpha$ and~$\eta$):

\begin{theorem}
    \label{thoerem:Erisk}
    The problems $\ERisk[e^{-\eta}]$ and $\ERisk[\alpha^{-\eta}]$ are in~\exptime for every fixed algebraic numbers $\alpha,\eta$.
    When $\alpha,\eta$ are not fixed but part of the input, 
    these problems~are~decidable.
\end{theorem}

\begin{proof}[Proof sketch]
Ultimately, in~\cite{BaierCMP23} the authors show that the problem $\ERisk[b^{-\eta}]$ is reducible in polynomial time to the problem of checking the satisfiability of a system of constraints of the following form (see~\cite[Equation~7]{BaierCMP23} for an equivalent formula):
\vspace{-6pt}
\begin{equation}%
    \label{formula:BCMP}%
    v(\hat{s}) \leq (b^{-\eta})^{t}
    \land\!\! \bigwedge_{s \in T} v(s) = d_s
    \land\!\! \bigwedge_{s \in S} v(s) = \oplus_{a \in A(s)} \Big((b^{-\eta})^{r(s)} \sum_{s' \in S} \Delta(s,a,s') \cdot v(s')\Big),
\end{equation} 
where $T$ is some subset of the states $S$ of the game, $d_s \in \{0,1\}$, and in the notation $\oplus_{a \in
A(s)}$ the symbol $\oplus$ stands for the functions $\min$ or
$\max$, depending on which of the two players controls~$s$.
The formula has one variable $v(s)$
for every $s \in S$, ranging over $\R$.
%(where $\min$ is used for states of $S_{\max}$). 

Since $z = \max(x,y)$ is equivalent to $z \geq x
\land z \geq y \land (z = x \lor z = y)$, and $z = \min(x,y)$ is
equivalent to $z \leq x \land z \leq y \land (z = x \lor z = y)$, 
except for the rationality of the exponents $t$ and $r(s)$ (which we handle below),
Formula~\ref{formula:BCMP} belongs to $\exists \R((b^{-\eta})^{\mathbb{Z}})$.

Fix $b > 1$ to be either $e$ or algebraic, 
and $\eta > 0$ to be algebraic. 
Assume to have access to representations for these algebraic numbers, 
so that if $\eta$ is represented by $(q(x),\ell,u)$, then $-\eta$ is represented by $(q(-x),-u,-\ell)$.
Consider the problem of checking whether a formula $\phi$ of the form given by~Formula~\ref{formula:BCMP} is satisfiable. 
Since $\phi$ does not feature predicates~$(b^{-\eta})^{\mathbb{Z}}$, but only the constant~$b^{-\eta}$, 
instead of~\Cref{algo:main-procedure} we can run the following simplified procedure:%
\begin{enumerate}[I.]
    \item \emph{Update all exponents $t$ and $r(s)$ of $\phi$ to be over $\N$
    and written in unary.} \textbf{(1)}~Compute the l.c.m.~$d \geq 1$ of the
    denominators of these exponents. \textbf{(2)}~Rewrite every
    term~$(b^{-\eta})^{\frac{p}{q}}$, where $\frac{p}{q}$ is one such
    exponent, into $(b^{\frac{-\eta}{d}})^{\frac{p \cdot d}{q}}$. Note that
    $\frac{p \cdot d}{q} \in \Z$. \textbf{(3)}~Rewrite $\phi$ into
    $\phi\sub{x}{b^{\frac{-\eta}{d}}} \land  x^d = b^{-\eta} \land x \geq
    0$, with $x$ fresh variable. \textbf{(4)}~Opportunely multiply both sides of
    inequalities by integer powers of~$x$ to make all exponents range over~$\N$.
    \textbf{(5)}~Change to a unary encoding for the exponents by adding further
    variables, as done in the proof
    of~\Cref{theorem:result-root-barrier}.\ref{theorem:result-root-barrier:point1}
    (\Cref{sec:poly-evaluation}). Overall, this step takes polynomial time
    in~$\size(\phi)$.
    \item \emph{Eliminate $x$ and all variables $v(s)$ with~$s \in S$.} This is
    done by appealing to~\Cref{theorem:basu}, treating~$b^{-\eta}$ as a free
    variable. The result is a Boolean combination $\psi$ of polynomial
    inequalities over $b^{-\eta}$. This step runs in time exponential in
    $\size(\phi)$.
    \item \emph{Evaluate $\psi$.} Call~\Cref{algo:sign-evaluation} on each
    inequality, to then return $\top$ or $\bot$ according to the Boolean
    structure of $\psi$. Since we can construct a polynomial-time Turing machine
    for $b^{-\eta}$ (\Cref{sec:poly-evaluation}),
    by~\Cref{lemma:runtime-sign-poly-root-barrier} this step takes polynomial
    time in $\size(\psi)$. 
    \qedhere
\end{enumerate}
% \noindent
% Accounting for the polynomial-time translation from an input stochastic game into Formula~\ref{formula:BCMP}, 
\end{proof}